%%%%%%%%%%%%%%%%%%%%%%%%%%%%%%%%%
% LaTeX model https://hurlus.fr %
%%%%%%%%%%%%%%%%%%%%%%%%%%%%%%%%%

% Needed before document class
\RequirePackage{pdftexcmds} % needed for tests expressions
\RequirePackage{fix-cm} % correct units

% Define mode
\def\mode{a4}

\newif\ifaiv % a4
\newif\ifav % a5
\newif\ifbooklet % booklet
\newif\ifcover % cover for booklet

\ifnum \strcmp{\mode}{cover}=0
  \covertrue
\else\ifnum \strcmp{\mode}{booklet}=0
  \booklettrue
\else\ifnum \strcmp{\mode}{a5}=0
  \avtrue
\else
  \aivtrue
\fi\fi\fi

\ifbooklet % do not enclose with {}
  \documentclass[french,twoside]{book} % ,notitlepage
  \usepackage[%
    papersize={105mm, 297mm},
    inner=12mm,
    outer=12mm,
    top=20mm,
    bottom=15mm,
    marginparsep=0pt,
  ]{geometry}
  \usepackage[fontsize=9.5pt]{scrextend} % for Roboto
\else\ifav
  \documentclass[french,twoside]{book} % ,notitlepage
  \usepackage[%
    a5paper,
    inner=25mm,
    outer=15mm,
    top=15mm,
    bottom=15mm,
    marginparsep=0pt,
  ]{geometry}
  \usepackage[fontsize=12pt]{scrextend}
\else% A4 2 cols
  \documentclass[twocolumn]{report}
  \usepackage[%
    a4paper,
    inner=15mm,
    outer=10mm,
    top=25mm,
    bottom=18mm,
    marginparsep=0pt,
  ]{geometry}
  \setlength{\columnsep}{20mm}
  \usepackage[fontsize=9.5pt]{scrextend}
\fi\fi

%%%%%%%%%%%%%%
% Alignments %
%%%%%%%%%%%%%%
% before teinte macros

\setlength{\arrayrulewidth}{0.2pt}
\setlength{\columnseprule}{\arrayrulewidth} % twocol
\setlength{\parskip}{0pt} % classical para with no margin
\setlength{\parindent}{1.5em}

%%%%%%%%%%
% Colors %
%%%%%%%%%%
% before Teinte macros

\usepackage[dvipsnames]{xcolor}
\definecolor{rubric}{HTML}{800000} % the tonic 0c71c3
\def\columnseprulecolor{\color{rubric}}
\colorlet{borderline}{rubric!30!} % definecolor need exact code
\definecolor{shadecolor}{gray}{0.95}
\definecolor{bghi}{gray}{0.5}

%%%%%%%%%%%%%%%%%
% Teinte macros %
%%%%%%%%%%%%%%%%%
%%%%%%%%%%%%%%%%%%%%%%%%%%%%%%%%%%%%%%%%%%%%%%%%%%%
% <TEI> generic (LaTeX names generated by Teinte) %
%%%%%%%%%%%%%%%%%%%%%%%%%%%%%%%%%%%%%%%%%%%%%%%%%%%
% This template is inserted in a specific design
% It is XeLaTeX and otf fonts

\makeatletter % <@@@


\usepackage{blindtext} % generate text for testing
\usepackage[strict]{changepage} % for modulo 4
\usepackage{contour} % rounding words
\usepackage[nodayofweek]{datetime}
% \usepackage{DejaVuSans} % seems buggy for sffont font for symbols
\usepackage{enumitem} % <list>
\usepackage{etoolbox} % patch commands
\usepackage{fancyvrb}
\usepackage{fancyhdr}
\usepackage{float}
\usepackage{fontspec} % XeLaTeX mandatory for fonts
\usepackage{footnote} % used to capture notes in minipage (ex: quote)
\usepackage{framed} % bordering correct with footnote hack
\usepackage{graphicx}
\usepackage{lettrine} % drop caps
\usepackage{lipsum} % generate text for testing
\usepackage[framemethod=tikz,]{mdframed} % maybe used for frame with footnotes inside
\usepackage{pdftexcmds} % needed for tests expressions
\usepackage{polyglossia} % non-break space french punct, bug Warning: "Failed to patch part"
\usepackage[%
  indentfirst=false,
  vskip=1em,
  noorphanfirst=true,
  noorphanafter=true,
  leftmargin=\parindent,
  rightmargin=0pt,
]{quoting}
\usepackage{ragged2e}
\usepackage{setspace} % \setstretch for <quote>
\usepackage{tabularx} % <table>
\usepackage[explicit]{titlesec} % wear titles, !NO implicit
\usepackage{tikz} % ornaments
\usepackage{tocloft} % styling tocs
\usepackage[fit]{truncate} % used im runing titles
\usepackage{unicode-math}
\usepackage[normalem]{ulem} % breakable \uline, normalem is absolutely necessary to keep \emph
\usepackage{verse} % <l>
\usepackage{xcolor} % named colors
\usepackage{xparse} % @ifundefined
\XeTeXdefaultencoding "iso-8859-1" % bad encoding of xstring
\usepackage{xstring} % string tests
\XeTeXdefaultencoding "utf-8"
\PassOptionsToPackage{hyphens}{url} % before hyperref, which load url package

% TOTEST
% \usepackage{hypcap} % links in caption ?
% \usepackage{marginnote}
% TESTED
% \usepackage{background} % doesn’t work with xetek
% \usepackage{bookmark} % prefers the hyperref hack \phantomsection
% \usepackage[color, leftbars]{changebar} % 2 cols doc, impossible to keep bar left
% \usepackage[utf8x]{inputenc} % inputenc package ignored with utf8 based engines
% \usepackage[sfdefault,medium]{inter} % no small caps
% \usepackage{firamath} % choose firasans instead, firamath unavailable in Ubuntu 21-04
% \usepackage{flushend} % bad for last notes, supposed flush end of columns
% \usepackage[stable]{footmisc} % BAD for complex notes https://texfaq.org/FAQ-ftnsect
% \usepackage{helvet} % not for XeLaTeX
% \usepackage{multicol} % not compatible with too much packages (longtable, framed, memoir…)
% \usepackage[default,oldstyle,scale=0.95]{opensans} % no small caps
% \usepackage{sectsty} % \chapterfont OBSOLETE
% \usepackage{soul} % \ul for underline, OBSOLETE with XeTeX
% \usepackage[breakable]{tcolorbox} % text styling gone, footnote hack not kept with breakable


% Metadata inserted by a program, from the TEI source, for title page and runing heads
\title{\textbf{ Les travailleurs de la mer }}
\date{1866}
\author{Victor Hugo}
\def\elbibl{Victor Hugo. 1866. \emph{Les travailleurs de la mer}}
\def\elsource{ \href{http://gallica.bnf.fr/ark:/12148/bpt6k374998}{\dotuline{http://gallica.bnf.fr/ark:/12148/bpt6k374998}}\footnote{\href{http://gallica.bnf.fr/ark:/12148/bpt6k374998}{\url{http://gallica.bnf.fr/ark:/12148/bpt6k374998}}} \\
 \href{http://gallica.bnf.fr/ark:/12148/bpt6k37500n}{\dotuline{http://gallica.bnf.fr/ark:/12148/bpt6k37500n}}\footnote{\href{http://gallica.bnf.fr/ark:/12148/bpt6k37500n}{\url{http://gallica.bnf.fr/ark:/12148/bpt6k37500n}}}  \href{http://efele.net/ebooks/livres/000220}{\dotuline{http://efele.net/ebooks/livres/000220}}\footnote{\href{http://efele.net/ebooks/livres/000220}{\url{http://efele.net/ebooks/livres/000220}}} }

% Default metas
\newcommand{\colorprovide}[2]{\@ifundefinedcolor{#1}{\colorlet{#1}{#2}}{}}
\colorprovide{rubric}{red}
\colorprovide{silver}{lightgray}
\@ifundefined{syms}{\newfontfamily\syms{DejaVu Sans}}{}
\newif\ifdev
\@ifundefined{elbibl}{% No meta defined, maybe dev mode
  \newcommand{\elbibl}{Titre court ?}
  \newcommand{\elbook}{Titre du livre source ?}
  \newcommand{\elabstract}{Résumé\par}
  \newcommand{\elurl}{http://oeuvres.github.io/elbook/2}
  \author{Éric Lœchien}
  \title{Un titre de test assez long pour vérifier le comportement d’une maquette}
  \date{1566}
  \devtrue
}{}
\let\eltitle\@title
\let\elauthor\@author
\let\eldate\@date


\defaultfontfeatures{
  % Mapping=tex-text, % no effect seen
  Scale=MatchLowercase,
  Ligatures={TeX,Common},
}


% generic typo commands
\newcommand{\astermono}{\medskip\centerline{\color{rubric}\large\selectfont{\syms ✻}}\medskip\par}%
\newcommand{\astertri}{\medskip\par\centerline{\color{rubric}\large\selectfont{\syms ✻\,✻\,✻}}\medskip\par}%
\newcommand{\asterism}{\bigskip\par\noindent\parbox{\linewidth}{\centering\color{rubric}\large{\syms ✻}\\{\syms ✻}\hskip 0.75em{\syms ✻}}\bigskip\par}%

% lists
\newlength{\listmod}
\setlength{\listmod}{\parindent}
\setlist{
  itemindent=!,
  listparindent=\listmod,
  labelsep=0.2\listmod,
  parsep=0pt,
  % topsep=0.2em, % default topsep is best
}
\setlist[itemize]{
  label=—,
  leftmargin=0pt,
  labelindent=1.2em,
  labelwidth=0pt,
}
\setlist[enumerate]{
  label={\bf\color{rubric}\arabic*.},
  labelindent=0.8\listmod,
  leftmargin=\listmod,
  labelwidth=0pt,
}
\newlist{listalpha}{enumerate}{1}
\setlist[listalpha]{
  label={\bf\color{rubric}\alph*.},
  leftmargin=0pt,
  labelindent=0.8\listmod,
  labelwidth=0pt,
}
\newcommand{\listhead}[1]{\hspace{-1\listmod}\emph{#1}}

\renewcommand{\hrulefill}{%
  \leavevmode\leaders\hrule height 0.2pt\hfill\kern\z@}

% General typo
\DeclareTextFontCommand{\textlarge}{\large}
\DeclareTextFontCommand{\textsmall}{\small}

% commands, inlines
\newcommand{\anchor}[1]{\Hy@raisedlink{\hypertarget{#1}{}}} % link to top of an anchor (not baseline)
\newcommand\abbr[1]{#1}
\newcommand{\autour}[1]{\tikz[baseline=(X.base)]\node [draw=rubric,thin,rectangle,inner sep=1.5pt, rounded corners=3pt] (X) {\color{rubric}#1};}
\newcommand\corr[1]{#1}
\newcommand{\ed}[1]{ {\color{silver}\sffamily\footnotesize (#1)} } % <milestone ed="1688"/>
\newcommand\expan[1]{#1}
\newcommand\foreign[1]{\emph{#1}}
\newcommand\gap[1]{#1}
\renewcommand{\LettrineFontHook}{\color{rubric}}
\newcommand{\initial}[2]{\lettrine[lines=2, loversize=0.3, lhang=0.3]{#1}{#2}}
\newcommand{\initialiv}[2]{%
  \let\oldLFH\LettrineFontHook
  % \renewcommand{\LettrineFontHook}{\color{rubric}\ttfamily}
  \IfSubStr{QJ’}{#1}{
    \lettrine[lines=4, lhang=0.2, loversize=-0.1, lraise=0.2]{\smash{#1}}{#2}
  }{\IfSubStr{É}{#1}{
    \lettrine[lines=4, lhang=0.2, loversize=-0, lraise=0]{\smash{#1}}{#2}
  }{\IfSubStr{ÀÂ}{#1}{
    \lettrine[lines=4, lhang=0.2, loversize=-0, lraise=0, slope=0.6em]{\smash{#1}}{#2}
  }{\IfSubStr{A}{#1}{
    \lettrine[lines=4, lhang=0.2, loversize=0.2, slope=0.6em]{\smash{#1}}{#2}
  }{\IfSubStr{V}{#1}{
    \lettrine[lines=4, lhang=0.2, loversize=0.2, slope=-0.5em]{\smash{#1}}{#2}
  }{
    \lettrine[lines=4, lhang=0.2, loversize=0.2]{\smash{#1}}{#2}
  }}}}}
  \let\LettrineFontHook\oldLFH
}
\newcommand{\labelchar}[1]{\textbf{\color{rubric} #1}}
\newcommand{\milestone}[1]{\autour{\footnotesize\color{rubric} #1}} % <milestone n="4"/>
\newcommand\name[1]{#1}
\newcommand\orig[1]{#1}
\newcommand\orgName[1]{#1}
\newcommand\persName[1]{#1}
\newcommand\placeName[1]{#1}
\newcommand{\pn}[1]{\IfSubStr{-—–¶}{#1}% <p n="3"/>
  {\noindent{\bfseries\color{rubric}   ¶  }}
  {{\footnotesize\autour{ #1}  }}}
\newcommand\reg{}
% \newcommand\ref{} % already defined
\newcommand\sic[1]{#1}
\newcommand\surname[1]{\textsc{#1}}
\newcommand\term[1]{\textbf{#1}}

\def\mednobreak{\ifdim\lastskip<\medskipamount
  \removelastskip\nopagebreak\medskip\fi}
\def\bignobreak{\ifdim\lastskip<\bigskipamount
  \removelastskip\nopagebreak\bigskip\fi}

% commands, blocks
\newcommand{\byline}[1]{\bigskip{\RaggedLeft{#1}\par}\bigskip}
\newcommand{\bibl}[1]{{\RaggedLeft{#1}\par\bigskip}}
\newcommand{\biblitem}[1]{{\noindent\hangindent=\parindent   #1\par}}
\newcommand{\dateline}[1]{\medskip{\RaggedLeft{#1}\par}\bigskip}
\newcommand{\labelblock}[1]{\medbreak{\noindent\color{rubric}\bfseries #1}\par\mednobreak}
\newcommand{\salute}[1]{\bigbreak{#1}\par\medbreak}
\newcommand{\signed}[1]{\bigbreak\filbreak{\raggedleft #1\par}\medskip}

% environments for blocks (some may become commands)
\newenvironment{borderbox}{}{} % framing content
\newenvironment{citbibl}{\ifvmode\hfill\fi}{\ifvmode\par\fi }
\newenvironment{docAuthor}{\ifvmode\vskip4pt\fontsize{16pt}{18pt}\selectfont\fi\itshape}{\ifvmode\par\fi }
\newenvironment{docDate}{}{\ifvmode\par\fi }
\newenvironment{docImprint}{\vskip6pt}{\ifvmode\par\fi }
\newenvironment{docTitle}{\vskip6pt\bfseries\fontsize{18pt}{22pt}\selectfont}{\par }
\newenvironment{msHead}{\vskip6pt}{\par}
\newenvironment{msItem}{\vskip6pt}{\par}
\newenvironment{titlePart}{}{\par }


% environments for block containers
\newenvironment{argument}{\itshape\parindent0pt}{\vskip1.5em}
\newenvironment{biblfree}{}{\ifvmode\par\fi }
\newenvironment{bibitemlist}[1]{%
  \list{\@biblabel{\@arabic\c@enumiv}}%
  {%
    \settowidth\labelwidth{\@biblabel{#1}}%
    \leftmargin\labelwidth
    \advance\leftmargin\labelsep
    \@openbib@code
    \usecounter{enumiv}%
    \let\p@enumiv\@empty
    \renewcommand\theenumiv{\@arabic\c@enumiv}%
  }
  \sloppy
  \clubpenalty4000
  \@clubpenalty \clubpenalty
  \widowpenalty4000%
  \sfcode`\.\@m
}%
{\def\@noitemerr
  {\@latex@warning{Empty `bibitemlist' environment}}%
\endlist}
\newenvironment{quoteblock}% may be used for ornaments
  {\begin{quoting}}
  {\end{quoting}}

% table () is preceded and finished by custom command
\newcommand{\tableopen}[1]{%
  \ifnum\strcmp{#1}{wide}=0{%
    \begin{center}
  }
  \else\ifnum\strcmp{#1}{long}=0{%
    \begin{center}
  }
  \else{%
    \begin{center}
  }
  \fi\fi
}
\newcommand{\tableclose}[1]{%
  \ifnum\strcmp{#1}{wide}=0{%
    \end{center}
  }
  \else\ifnum\strcmp{#1}{long}=0{%
    \end{center}
  }
  \else{%
    \end{center}
  }
  \fi\fi
}


% text structure
\newcommand\chapteropen{} % before chapter title
\newcommand\chaptercont{} % after title, argument, epigraph…
\newcommand\chapterclose{} % maybe useful for multicol settings
\setcounter{secnumdepth}{-2} % no counters for hierarchy titles
\setcounter{tocdepth}{5} % deep toc
\markright{\@title} % ???
\markboth{\@title}{\@author} % ???
\renewcommand\tableofcontents{\@starttoc{toc}}
% toclof format
% \renewcommand{\@tocrmarg}{0.1em} % Useless command?
% \renewcommand{\@pnumwidth}{0.5em} % {1.75em}
\renewcommand{\@cftmaketoctitle}{}
\setlength{\cftbeforesecskip}{\z@ \@plus.2\p@}
\renewcommand{\cftchapfont}{}
\renewcommand{\cftchapdotsep}{\cftdotsep}
\renewcommand{\cftchapleader}{\normalfont\cftdotfill{\cftchapdotsep}}
\renewcommand{\cftchappagefont}{\bfseries}
\setlength{\cftbeforechapskip}{0em \@plus\p@}
% \renewcommand{\cftsecfont}{\small\relax}
\renewcommand{\cftsecpagefont}{\normalfont}
% \renewcommand{\cftsubsecfont}{\small\relax}
\renewcommand{\cftsecdotsep}{\cftdotsep}
\renewcommand{\cftsecpagefont}{\normalfont}
\renewcommand{\cftsecleader}{\normalfont\cftdotfill{\cftsecdotsep}}
\setlength{\cftsecindent}{1em}
\setlength{\cftsubsecindent}{2em}
\setlength{\cftsubsubsecindent}{3em}
\setlength{\cftchapnumwidth}{1em}
\setlength{\cftsecnumwidth}{1em}
\setlength{\cftsubsecnumwidth}{1em}
\setlength{\cftsubsubsecnumwidth}{1em}

% footnotes
\newif\ifheading
\newcommand*{\fnmarkscale}{\ifheading 0.70 \else 1 \fi}
\renewcommand\footnoterule{\vspace*{0.3cm}\hrule height \arrayrulewidth width 3cm \vspace*{0.3cm}}
\setlength\footnotesep{1.5\footnotesep} % footnote separator
\renewcommand\@makefntext[1]{\parindent 1.5em \noindent \hb@xt@1.8em{\hss{\normalfont\@thefnmark . }}#1} % no superscipt in foot
\patchcmd{\@footnotetext}{\footnotesize}{\footnotesize\sffamily}{}{} % before scrextend, hyperref


%   see https://tex.stackexchange.com/a/34449/5049
\def\truncdiv#1#2{((#1-(#2-1)/2)/#2)}
\def\moduloop#1#2{(#1-\truncdiv{#1}{#2}*#2)}
\def\modulo#1#2{\number\numexpr\moduloop{#1}{#2}\relax}

% orphans and widows
\clubpenalty=9996
\widowpenalty=9999
\brokenpenalty=4991
\predisplaypenalty=10000
\postdisplaypenalty=1549
\displaywidowpenalty=1602
\hyphenpenalty=400
% Copied from Rahtz but not understood
\def\@pnumwidth{1.55em}
\def\@tocrmarg {2.55em}
\def\@dotsep{4.5}
\emergencystretch 3em
\hbadness=4000
\pretolerance=750
\tolerance=2000
\vbadness=4000
\def\Gin@extensions{.pdf,.png,.jpg,.mps,.tif}
% \renewcommand{\@cite}[1]{#1} % biblio

\usepackage{hyperref} % supposed to be the last one, :o) except for the ones to follow
\urlstyle{same} % after hyperref
\hypersetup{
  % pdftex, % no effect
  pdftitle={\elbibl},
  % pdfauthor={Your name here},
  % pdfsubject={Your subject here},
  % pdfkeywords={keyword1, keyword2},
  bookmarksnumbered=true,
  bookmarksopen=true,
  bookmarksopenlevel=1,
  pdfstartview=Fit,
  breaklinks=true, % avoid long links
  pdfpagemode=UseOutlines,    % pdf toc
  hyperfootnotes=true,
  colorlinks=false,
  pdfborder=0 0 0,
  % pdfpagelayout=TwoPageRight,
  % linktocpage=true, % NO, toc, link only on page no
}

\makeatother % /@@@>
%%%%%%%%%%%%%%
% </TEI> end %
%%%%%%%%%%%%%%


%%%%%%%%%%%%%
% footnotes %
%%%%%%%%%%%%%
\renewcommand{\thefootnote}{\bfseries\textcolor{rubric}{\arabic{footnote}}} % color for footnote marks

%%%%%%%%%
% Fonts %
%%%%%%%%%
\usepackage[]{roboto} % SmallCaps, Regular is a bit bold
% \linespread{0.90} % too compact, keep font natural
\newfontfamily\fontrun[]{Roboto Condensed Light} % condensed runing heads
\ifav
  \setmainfont[
    ItalicFont={Roboto Light Italic},
  ]{Roboto}
\else\ifbooklet
  \setmainfont[
    ItalicFont={Roboto Light Italic},
  ]{Roboto}
\else
\setmainfont[
  ItalicFont={Roboto Italic},
]{Roboto Light}
\fi\fi
\renewcommand{\LettrineFontHook}{\bfseries\color{rubric}}
% \renewenvironment{labelblock}{\begin{center}\bfseries\color{rubric}}{\end{center}}

%%%%%%%%
% MISC %
%%%%%%%%

\setdefaultlanguage[frenchpart=false]{french} % bug on part


\newenvironment{quotebar}{%
    \def\FrameCommand{{\color{rubric!10!}\vrule width 0.5em} \hspace{0.9em}}%
    \def\OuterFrameSep{\itemsep} % séparateur vertical
    \MakeFramed {\advance\hsize-\width \FrameRestore}
  }%
  {%
    \endMakeFramed
  }
\renewenvironment{quoteblock}% may be used for ornaments
  {%
    \savenotes
    \setstretch{0.9}
    \normalfont
    \begin{quotebar}
  }
  {%
    \end{quotebar}
    \spewnotes
  }


\renewcommand{\headrulewidth}{\arrayrulewidth}
\renewcommand{\headrule}{{\color{rubric}\hrule}}

% delicate tuning, image has produce line-height problems in title on 2 lines
\titleformat{name=\chapter} % command
  [display] % shape
  {\vspace{1.5em}\centering} % format
  {} % label
  {0pt} % separator between n
  {}
[{\color{rubric}\huge\textbf{#1}}\bigskip] % after code
% \titlespacing{command}{left spacing}{before spacing}{after spacing}[right]
\titlespacing*{\chapter}{0pt}{-2em}{0pt}[0pt]

\titleformat{name=\section}
  [block]{}{}{}{}
  [\vbox{\color{rubric}\large\raggedleft\textbf{#1}}]
\titlespacing{\section}{0pt}{0pt plus 4pt minus 2pt}{\baselineskip}

\titleformat{name=\subsection}
  [block]
  {}
  {} % \thesection
  {} % separator \arrayrulewidth
  {}
[\vbox{\large\textbf{#1}}]
% \titlespacing{\subsection}{0pt}{0pt plus 4pt minus 2pt}{\baselineskip}

\ifaiv
  \fancypagestyle{main}{%
    \fancyhf{}
    \setlength{\headheight}{1.5em}
    \fancyhead{} % reset head
    \fancyfoot{} % reset foot
    \fancyhead[L]{\truncate{0.45\headwidth}{\fontrun\elbibl}} % book ref
    \fancyhead[R]{\truncate{0.45\headwidth}{ \fontrun\nouppercase\leftmark}} % Chapter title
    \fancyhead[C]{\thepage}
  }
  \fancypagestyle{plain}{% apply to chapter
    \fancyhf{}% clear all header and footer fields
    \setlength{\headheight}{1.5em}
    \fancyhead[L]{\truncate{0.9\headwidth}{\fontrun\elbibl}}
    \fancyhead[R]{\thepage}
  }
\else
  \fancypagestyle{main}{%
    \fancyhf{}
    \setlength{\headheight}{1.5em}
    \fancyhead{} % reset head
    \fancyfoot{} % reset foot
    \fancyhead[RE]{\truncate{0.9\headwidth}{\fontrun\elbibl}} % book ref
    \fancyhead[LO]{\truncate{0.9\headwidth}{\fontrun\nouppercase\leftmark}} % Chapter title, \nouppercase needed
    \fancyhead[RO,LE]{\thepage}
  }
  \fancypagestyle{plain}{% apply to chapter
    \fancyhf{}% clear all header and footer fields
    \setlength{\headheight}{1.5em}
    \fancyhead[L]{\truncate{0.9\headwidth}{\fontrun\elbibl}}
    \fancyhead[R]{\thepage}
  }
\fi

\ifav % a5 only
  \titleclass{\section}{top}
\fi

\newcommand\chapo{{%
  \vspace*{-3em}
  \centering % no vskip ()
  {\Large\addfontfeature{LetterSpace=25}\bfseries{\elauthor}}\par
  \smallskip
  {\large\eldate}\par
  \bigskip
  {\Large\selectfont{\eltitle}}\par
  \bigskip
  {\color{rubric}\hline\par}
  \bigskip
  {\Large TEXTE LIBRE À PARTICPATION LIBRE\par}
  \centerline{\small\color{rubric} {hurlus.fr, tiré le \today}}\par
  \bigskip
}}

\newcommand\cover{{%
  \thispagestyle{empty}
  \centering
  {\LARGE\bfseries{\elauthor}}\par
  \bigskip
  {\Large\eldate}\par
  \bigskip
  \bigskip
  {\LARGE\selectfont{\eltitle}}\par
  \vfill\null
  {\color{rubric}\setlength{\arrayrulewidth}{2pt}\hline\par}
  \vfill\null
  {\Large TEXTE LIBRE À PARTICPATION LIBRE\par}
  \centerline{{\href{https://hurlus.fr}{\dotuline{hurlus.fr}}, tiré le \today}}\par
}}

\begin{document}
\pagestyle{empty}
\ifbooklet{
  \cover\newpage
  \thispagestyle{empty}\hbox{}\newpage
  \cover\newpage\noindent Les voyages de la brochure\par
  \bigskip
  \begin{tabularx}{\textwidth}{l|X|X}
    \textbf{Date} & \textbf{Lieu}& \textbf{Nom/pseudo} \\ \hline
    \rule{0pt}{25cm} &  &   \\
  \end{tabularx}
  \newpage
  \addtocounter{page}{-4}
}\fi

\thispagestyle{empty}
\ifaiv
  \twocolumn[\chapo]
\else
  \chapo
\fi
{\it\elabstract}
\bigskip
\makeatletter\@starttoc{toc}\makeatother % toc without new page
\bigskip

\pagestyle{main} % after style

  \frontmatter  \noindent Je dédie ce livre au rocher d’hospitalité et de liberté, à ce coin de vieille terre normande où vit le noble petit peuple de la mer, à l’île de Guernesey, sévère et douce, mon asile actuel, mon tombeau probable.\par

\byline{V. H.}
 \noindent La religion, la société, la nature ; telles sont les trois luttes de l’homme. Ces trois luttes sont en même temps ses trois besoins ; il faut qu’il croie, de là le temple ; il faut qu’il crée, de là la cité ; il faut qu’il vive, de là la charrue et le navire. Mais ces trois solutions contiennent trois guerres. La mystérieuse difficulté de la vie sort de toutes les trois. L’homme a affaire à l’obstacle sous la forme superstition, sous la forme préjugé, et sous la forme élément. Un triple anankè pèse sur nous, l’anankè des dogmes,  l’anankè des lois, l’anankè des choses. Dans \emph{Notre-Dame de Paris,} l’auteur a dénoncé le premier ; dans \emph{les Misérables}, il a signalé le second ; dans ce livre, il indique le troisième.\par
A ces trois fatalités qui enveloppent l’homme se mêle la fatalité intérieure, l’anankè suprême, le cœur humain.\par

\dateline{Hauteville-House, mars 1866.}
 \mainmatter    \section[{A. Première partie. Sieur Clubin}]{A. Première partie \\
Sieur Clubin}\renewcommand{\leftmark}{A. Première partie \\
Sieur Clubin}

  \subsection[{A.I. Livre premier. De quoi se compose une mauvaise réputation}]{A.I. Livre premier \\
De quoi se compose une mauvaise réputation}
  \subsubsection[{A.I.1. Un mot écrit sur une page blanche}]{A.I.1. \\
Un mot écrit sur une page blanche}
\noindent La Christmas de 182... fut remarquable à Guernesey. Il neigea ce jour-là. Dans les îles de la Manche, un hiver où il gèle à glace est mémorable, et la neige fait événement.\par
Le matin de cette Christmas, la route qui longe la mer de Saint-Pierre-Port au Valle était toute blanche. Il avait neigé depuis minuit jusqu’à l’aube. Vers neuf heures, peu après le lever du soleil, comme ce n’était pas encore le moment pour les anglicans d’aller à l’église de Saint-Sampson et pour les wesleyens d’aller à la chapelle Eldad, le chemin était à peu près désert. Dans tout le tronçon de route qui sépare la première tour de la seconde tour, il n’y avait que trois passants, un enfant, un homme et une femme. Ces trois passants, marchant à distance les uns des autres,  n’avaient visiblement aucun lien entre eux. L’enfant, d’une huitaine d’années, s’était arrêté, et regardait la neige avec curiosité. L’homme venait derrière la femme, à une centaine de pas d’intervalle. Il allait comme elle du côté de Saint-Sampson. L’homme, jeune encore, semblait quelque chose comme un ouvrier ou un matelot. Il avait ses habits de tous les jours, une vareuse de gros drap brun, et un pantalon à jambières goudronnées, ce qui paraissait indiquer qu’en dépit de la fête il n’irait à aucune chapelle. Ses épais souliers de cuir brut, aux semelles garnies de gros clous, laissaient sur la neige une empreinte plus ressemblante à une serrure de prison qu’à un pied d’homme. La passante, elle, avait évidemment déjà sa toilette d’église ; elle portait une large mante ouatée de soie noire à faille, sous laquelle elle était fort coquettement ajustée d’une robe de popeline d’Irlande à bandes alternées blanches et roses, et, si elle n’eût eu des bas rouges, on eût pu la prendre pour une parisienne. Elle allait devant elle avec une vivacité libre et légère, et, à cette marche qui n’a encore rien porté de la vie, on devinait une jeune fille. Elle avait cette grâce fugitive de l’allure qui marque la plus délicate des transitions, l’adolescence, les deux crépuscules mêlés, le commencement d’une femme dans la fin d’un enfant. L’homme ne la remarquait pas.\par
Tout à coup, près d’un bouquet de chênes verts qui est à l’angle d’un courtil, au lieu dit les Basses-Maisons, elle se retourna, et ce mouvement fit que l’homme la regarda. Elle s’arrêta, parut le considérer un moment,  puis se baissa, et l’homme crut voir qu’elle écrivait avec son doigt quelque chose sur la neige. Elle se redressa, se remit en marche, doubla le pas, se retourna encore, cette fois en riant, et disparut à gauche du chemin, dans le sentier bordé de haies qui mène au château de Lierre. L’homme, quand elle se retourna pour la seconde fois, reconnut Déruchette, une ravissante fille du pays.\par
Il n’éprouva aucun besoin de se hâter, et, quelques instants après, il se trouva près du bouquet de chênes à l’angle du courtil. Il ne songeait déjà plus à la passante disparue, et il est probable que si, en cette minute-là, quelque marsouin eût sauté dans la mer ou quelque rouge-gorge dans les buissons, cet homme eût passé son chemin, l’œil fixé sur le rouge-gorge ou le marsouin. Le hasard fit qu’il avait les paupières baissées, son regard tomba machinalement sur l’endroit où la jeune fille s’était arrêtée. Deux petits pieds s’y étaient imprimés, et à côté il lut ce mot tracé par elle dans la neige : \emph{Gilliatt.}\par
Ce mot était son nom.\par
Il s’appelait Gilliatt.\par
Il resta longtemps immobile, regardant ce nom, ces petits pieds, cette neige, puis continua sa route, pensif.
 \subsubsection[{A.I.2. Le bu de la rue}]{A.I.2. \\
Le bu de la rue}
\noindent Gilliatt habitait la paroisse de Saint-Sampson. Il n’y était pas aimé. Il y avait des raisons pour cela.\par
D’abord il avait pour logis une maison « visionnée ». Il arrive quelquefois, à Jersey ou à Guernesey, qu’à la campagne, à la ville même, passant dans quelque coin désert ou dans une rue pleine d’habitants, vous rencontrez une maison dont l’entrée est barricadée ; le houx obstrue la porte ; on ne sait quels hideux emplâtres de planches clouées bouchent les fenêtres du rez-de-chaussée ; les fenêtres des étages supérieurs sont à la fois fermées et ouvertes, tous les châssis sont verrouillés, mais tous les carreaux sont cassés. S’il y a un beyle, une cour, l’herbe y pousse, le parapet d’enceinte s’écroule ; s’il y a un jardin, il est ortie, ronce et ciguë, et l’on peut y épier les insectes rares. Les cheminées se crevassent, le toit s’effondre ; ce qu’on voit du dedans des chambres est démantelé ; le bois est pourri, la pierre est moisie. Il y a aux murs du papier qui se  décolle. Vous pouvez y étudier les vieilles modes du papier, les griffons de l’empire, les draperies en croissant du directoire, les balustres et les cippes de Louis XVI. L’épaississement des toiles pleines de mouches indique la paix profonde des araignées. Quelquefois on aperçoit un pot cassé sur une planche. C’est là une maison « visionnée ». Le diable y vient la nuit.\par
La maison comme l’homme peut devenir cadavre. Il suffit qu’une superstition la tue. Alors elle est terrible. Ces maisons mortes ne sont point rares dans les îles de la Manche.\par
Les populations campagnardes et maritimes ne sont pas tranquilles à l’endroit du diable. Celles de la Manche, archipel anglais et littoral français, ont sur lui des notions très précises. Le diable a des envoyés par toute la terre. Il est certain que Belphégor est ambassadeur de l’enfer en France, Hutgin en Italie, Bélial en Turquie, Thamuz en Espagne, Martinet en Suisse, et Mammon en Angleterre. Satan est un empereur comme un autre. Satan César. Sa maison est très bien montée ; Dagon est grand panetier ; Succor Bénoth est chef des eunuques ; Asmodée, banquier des jeux ; Kobal, directeur du théâtre, et Verdelet, grand maître des cérémonies ; Nybbas est bouffon. Wiérus, homme savant, bon strygologue et démonographe bien renseigné, appelle Nybbas « le grand parodiste ».\par
Les pêcheurs normands de la Manche ont bien des précautions à prendre quand ils sont en mer, à cause  des illusions que le diable fait. On a longtemps cru que saint Maclou habitait le gros rocher carré Ortach, qui est au large entre Aurigny et les Casquets, et beaucoup de vieux matelots d’autrefois affirmaient l’y avoir très souvent vu de loin, assis et lisant dans un livre. Aussi les marins de passage faisaient-ils force génuflexions devant le rocher Ortach jusqu’au jour où la fable s’est dissipée et a fait place à la vérité. On a découvert et l’on sait aujourd’hui que ce qui habite le rocher Ortach, ce n’est pas un saint, mais un diable. Ce diable, un nommé Jochmus, avait eu la malice de se faire passer pendant plusieurs siècles pour saint Maclou. Au reste l’église elle-même tombe dans ces méprises. Les diables Raguhel, Oribel et Tobiel ont été saints jusqu’en 745 où le pape Zacharie, les ayant flairés, les mit dehors. Pour faire de ces expulsions, qui sont certes très utiles, il faut beaucoup se connaître en diables.\par
Les anciens du pays racontent, mais ces faits-là appartiennent au passé, que la population catholique de l’archipel normand a été autrefois, bien malgré elle, plus en communication encore avec le démon que la population huguenote. Pourquoi ? nous l’ignorons. Ce qui est certain, c’est que cette minorité fut jadis fort ennuyée par le diable. Il avait pris les catholiques en affection, et cherchait à les fréquenter, ce qui donnerait à croire que le diable est plutôt catholique que protestant. Une de ses plus insupportables familiarités, c’était de faire des visites nocturnes aux lits conjugaux catholiques, au moment où le mari était endormi tout  à fait, et la femme à moitié. De là des méprises. Patouillet pensait que Voltaire était né de cette façon. Cela n’a rien d’invraisemblable. Ce cas du reste est parfaitement connu et décrit dans les formulaires d’exorcismes, sous la rubrique : \emph{De erroribus nocturnis et de semine diabolorum}. Il a particulièrement sévi à Saint-Hélier vers la fin du siècle dernier, probablement en punition des crimes de la révolution. Les conséquences des excès révolutionnaires sont incalculables. Quoi qu’il en soit, cette survenue possible du démon, la nuit, quand on n’y voit pas clair, quand on dort, embarrassait beaucoup de femmes orthodoxes. Donner naissance à un Voltaire n’a rien d’agréable. Une d’elles, inquiète, consulta son confesseur sur le moyen d’éclaircir à temps ce quiproquo. Le confesseur répondit : — Pour vous assurer si vous avez affaire au diable ou à votre mari, tâtez le front ; si vous trouvez des cornes, vous serez sûre... — De quoi ? demanda la femme.\par
La maison qu’habitait Gilliatt avait été visionnée et ne l’était plus. Elle n’en était que plus suspecte. Personne n’ignore que lorsqu’un sorcier s’installe dans un logis hanté, le diable juge le logis suffisamment tenu, et fait au sorcier la politesse de n’y plus venir, à moins d’être appelé, comme le médecin.\par
Cette maison se nommait le Bû de la Rue. Elle était située à la pointe d’une langue de terre ou plutôt de rocher qui faisait un petit mouillage à part dans la crique de Houmet-Paradis. Il y a là une eau profonde. Cette maison était toute seule sur cette pointe presque  hors de l’île, avec juste assez de terre pour un petit jardin. Les hautes marées noyaient quelquefois le jardin. Entre le port de Saint-Sampson et la crique de Houmet-Paradis, il y a la grosse colline que surmonte le bloc de tours et de lierre appelé le château du Valle ou de l’Archange, en sorte que de Saint-Sampson on ne voyait pas le Bû de la Rue.\par
Rien n’est moins rare qu’un sorcier à Guernesey. Ils exercent leur profession dans certaines paroisses, et le dix-neuvième siècle n’y fait rien. Ils ont des pratiques véritablement criminelles. Ils font bouillir de l’or. Ils cueillent des herbes à minuit. Ils regardent de travers les bestiaux des gens. On les consulte ; ils se font apporter dans des bouteilles de « l’eau des malades », et on les entend dire à demi-voix : \emph{L’eau parait bien triste.} L’un d’eux un jour, en mars 1856, a constaté dans « l’eau » d’un malade sept diables. Ils sont redoutés et redoutables. Un d’eux a récemment ensorcelé un boulanger « ainsi que son four ». Un autre a la scélératesse de cacheter et sceller avec le plus grand soin des enveloppes « où il n’y a rien dedans ». Un autre va jusqu’à avoir dans sa maison sur une planche trois bouteilles étiquetées B. Ces faits monstrueux sont constatés. Quelques sorciers sont complaisants, et, pour deux ou trois guinées, prennent vos maladies. Alors ils se roulent sur leur lit en poussant des cris. Pendant qu’ils se tordent, vous dites : Tiens, je n’ai plus rien. D’autres vous guérissent de tous les maux en vous nouant un mouchoir autour du corps. Moyen si simple qu’on s’étonne que personne ne s’en soit  encore avisé. Au siècle dernier la cour royale de Guernesey les mettait sur un tas de fagots, et les brûlait vifs. De nos jours elle les condamne à huit semaines de prison, quatre semaines au pain et à l’eau, et quatre semaines au secret, alternant. \emph{Amant alterna catenæ.}\par
Le dernier brûlement de sorciers à Guernesey a eu lieu en 1747. La ville avait utilisé pour cela une de ses places, le carrefour du Bordage. Le carrefour du Bordage a vu brûler onze sorciers, de 1565 à 1700. En général ces coupables avouaient. On les aidait à l’aveu au moyen de la torture. Le carrefour du Bordage a rendu d’autres services à la société et à la religion. On y a brûlé les hérétiques. Sous Marie Tudor, on y brûla, entre autres huguenots, une mère et ses deux filles ; cette mère s’appelait Perrotine Massy. Une des filles était grosse. Elle accoucha dans la braise du bûcher. La chronique dit : « Son ventre éclata. » Il sortit de ce ventre un enfant vivant ; le nouveau-né roula hors de la fournaise ; un nommé House le ramassa. Le bailli Hélier Gosselin, bon catholique, fit rejeter l’enfant dans le feu.
 \subsubsection[{A.I.3. « Pour ta femme, quand tu te marieras »}]{A.I.3. \\
« Pour ta femme, quand tu te marieras »}
\noindent Revenons à Gilliatt.\par
On contait dans le pays qu’une femme, qui avait avec elle un petit enfant, était venue vers la fin de la révolution habiter Guernesey. Elle était anglaise, à moins qu’elle ne fût française. Elle avait un nom quelconque dont la prononciation guernesiaise et l’orthographe paysanne avaient fait Gilliatt. Elle vivait seule avec cet enfant qui était pour elle, selon les uns un neveu, selon les autres un fils, selon les autres un petit-fils, selon les autres rien du tout. Elle avait un peu d’argent, de quoi vivre pauvrement. Elle avait acheté une pièce de pré à la Sergentée, et une jaonnière à la Roque-Crespel, près de Rocquaine. La maison du Bû de la Rue était, à cette époque, visionnée. Depuis plus de trente ans, on ne l’habitait plus. Elle tombait en ruine. Le jardin, trop visité par la mer, ne pouvait rien produire. Outre les bruits nocturnes et les lueurs, cette maison avait cela de particulièrement effrayant  que, si on y laissait le soir sur la cheminée une pelote de laine, des aiguilles et une pleine assiettée de soupe, on trouvait le lendemain matin la soupe mangée, l’assiette vide et une paire de mitaines tricotée. On offrait cette masure à vendre avec le démon qui était dedans pour quelques livres sterling. Cette femme l’acheta, évidemment tentée par le diable. Ou par le bon marché.\par
Elle fit plus que l’acheter, elle s’y logea, elle et son enfant ; et, à partir de ce moment, la maison s’apaisa. \emph{Cette maison a ce qu’elle veut}, dirent les gens du pays. Le visionnement cessa. On n’y entendit plus de cris au point du jour. Il n’y eut plus d’autre lumière que le suif allumé le soir par la bonne femme. Chandelle de sorcière vaut torche du diable. Cette explication satisfit le public.\par
Cette femme tirait parti des quelques vergées de terre qu’elle avait. Elle avait une bonne vache à beurre jaune. Elle récoltait des mouzettes blanches, des caboches et des pommes de terre Golden Drops. Elle vendait, tout comme une autre, « des panais par le tonneau, des oignons par le cent, et des fèves par le dénerel ». Elle n’allait pas au marché, mais faisait vendre sa récolte par Guilbert Falliot, aux Abreuveurs Saint-Sampson. Le registre de Falliot constate qu’il vendit pour elle une fois jusqu’à douze boisseaux de \emph{patates dites trois mois, des plus temprunes.}\par
La maison avait été chétivement réparée, assez pour y vivre. Il ne pleuvait dans les chambres que par les très gros temps. Elle se composait d’un rez-de- chaussée et d’un grenier. Le rez-de-chaussée était partagé en trois salles, deux où l’on couchait, une où l’on mangeait. On montait au grenier par une échelle. La femme faisait la cuisine et montrait à lire à l’enfant. Elle n’allait point aux églises ; ce qui fit que, tout bien considéré, on la déclara française. N’aller « à aucune place », c’est grave.\par
En somme, c’étaient des gens que rien ne prouvait.\par
Française, il est probable qu’elle l’était. Les volcans lancent des pierres et les révolutions des hommes. Des familles sont ainsi envoyées à de grandes distances, des destinées sont dépaysées, des groupes sont dispersés et s’émiettent ; des gens tombent des nues, ceux-ci en Allemagne, ceux-là en Angleterre, ceux-là en Amérique. Ils étonnent les naturels du pays. D’où viennent ces inconnus ? C’est ce vésuve qui fume là-bas qui les a expectorés. On donne des noms à ces aérolithes, à ces individus expulsés et perdus, à ces éliminés du sort ; on les appelle émigrés, réfugiés, aventuriers. S’ils restent, on les tolère ; s’ils s’en vont, on est content. Quelquefois ce sont des êtres absolument inoffensifs, étrangers, les femmes du moins, aux événements qui les ont chassés, n’ayant ni haine ni colère, projectiles sans le vouloir, très étonnés. Ils reprennent racine comme ils peuvent. Ils ne faisaient rien à personne et ne comprennent pas ce qui leur est arrivé. J’ai vu une pauvre touffe d’herbe lancée éperdument en l’air par une explosion de mine. La révolution française, plus que toute autre explosion, a eu de ces jets lointains.\par
 La femme qu’à Guernesey on appelait \emph{la Gilliatt }était peut-être cette touffe d’herbe-là.\par
La femme vieillit, l’enfant grandit. Ils vivaient seuls, et évités. Ils se suffisaient. Louve et louveteau se pourlèchent. Ceci est encore une des formules que leur appliqua la bienveillance environnante. L’enfant devint un adolescent, l’adolescent devint un homme, et alors, les vieilles écorces de la vie devant toujours tomber, la mère mourut. Elle lui laissa le pré de la Sergentée, la joannière de la Roque-Crespel, la maison du Bû de la Rue, plus, dit l’inventaire officiel, « cent guinées d’or dans le pied d’une cauche », c’est-à-dire dans le pied d’un bas. La maison était suffisamment meublée de deux coffres de chêne, de deux lits, de six chaises et d’une table, avec ce qu’il faut d’ustensiles. Sur une planche il y avait quelques livres, et, dans un coin, une malle pas du tout mystérieuse qui dut être ouverte pour l’inventaire. Cette malle était en cuir fauve à arabesques de clous de cuivre et d’étoiles d’étain, et contenait un trousseau de femme neuf et complet en belle toile de fil de Dunkerque, chemises et jupes, plus des robes de soie en pièce, avec un papier où on lisait ceci écrit de la main de la morte : \emph{Pour ta femme, quand tu te marieras.}\par
Cette mort fut pour le survivant un accablement. Il était sauvage, il devint farouche. Le désert s’acheva autour de lui. Ce n’était que l’isolement, ce fut le vide. Tant qu’on est deux, la vie est possible. Seul, il semble qu’on ne pourra plus la traîner. On renonce à tirer. C’est la première forme du désespoir. Plus tard on  comprend que le devoir est une série d’acceptations. On regarde la mort, on regarde la vie, et l’on consent. Mais c’est un consentement qui saigne.\par
Gilliatt étant jeune, sa plaie se cicatrisa. A cet âge, les chairs du cœur reprennent. Sa tristesse, effacée peu à peu, se mêla autour de lui à la nature, y devint une sorte de charme, l’attira vers les choses et loin des hommes, et amalgama de plus en plus cette âme à la solitude.
 \subsubsection[{A.I.4. Impopularité}]{A.I.4. \\
Impopularité}
\noindent Gilliatt, nous l’avons dit, n’était pas aimé dans la paroisse. Rien de plus naturel que cette antipathie. Les motifs abondaient. D’abord, on vient de l’expliquer, la maison qu’il habitait. Ensuite, son origine. Qu’est-ce que c’était que cette femme ? et pourquoi cet enfant ? Les gens du pays n’aiment pas qu’il y ait des énigmes sur les étrangers. Ensuite, son vêtement qui était d’un ouvrier, tandis qu’il avait, quoique pas riche, de quoi vivre sans rien faire. Ensuite, son jardin, qu’il réussissait à cultiver et d’où il tirait des pommes de terre malgré les coups d’équinoxe. Ensuite, de gros livres qu’il avait sur une planche, et où il lisait.\par
D’autres raisons encore.\par
D’où vient qu’il vivait solitaire ? Le Bû de la Rue était une sorte de lazaret, on tenait Gilliatt en quarantaine ; c’est pourquoi il était tout simple qu’on  s’étonnât de son isolement, et qu’on le rendît responsable de la solitude qu’on faisait autour de lui.\par
Il n’allait jamais à la chapelle. Il sortait souvent la nuit. Il parlait aux sorciers. Une fois on l’avait vu assis dans l’herbe d’un air étonné. Il hantait le dolmen de l’Ancresse et les pierres fées qui sont dans la campagne çà et là. On croyait être sûr de l’avoir vu saluer poliment la Roque qui Chante. Il achetait tous les oiseaux qu’on lui apportait et les mettait en liberté. Il était honnête aux personnes bourgeoises dans les rues de Saint-Sampson, mais faisait volontiers un détour pour n’y point passer. Il pêchait souvent, et revenait toujours avec du poisson. Il travaillait à son jardin le dimanche. Il avait un bug-pipe, acheté par lui à des soldats écossais de passage à Guernesey, et dont il jouait dans les rochers au bord de la mer, à la nuit tombante. Il faisait des gestes comme un semeur. Que voulez-vous qu’un pays devienne avec un homme comme cela ?\par
Quant aux livres, qui venaient de la femme morte, et où il lisait, ils étaient inquiétants. Le révérend Jaquemin Hérode, recteur de Saint-Sampson, quand il était entré dans la maison pour l’enterrement de la femme, avait lu au dos de ces livres les titres que voici : \emph{Dictionnaire de Rosier, Candide,} par Voltaire, \emph{Avis au peuple sur sa santé,} par Tissot. Un gentilhomme français, émigré, retiré à Saint-Sampson, avait dit : \emph{Ce doit être le Tissot qui a porté la tête de la princesse de Lamballe.}\par
Le révérend avait remarqué sur un de ces livres  ce titre véritablement bourru et menaçant : \emph{De Rhubarbaro.}\par
Disons-le pourtant, l’ouvrage étant, comme le titre l’indique, écrit en latin, il était douteux que Gilliatt, qui ne savait pas le latin, lût ce livre.\par
Mais ce sont précisément les livres qu’un homme ne lit pas qui l’accusent le plus. L’inquisition d’Espagne a jugé ce point et l’a mis hors de doute.\par
Du reste ce n’était autre chose que le traité du docteur Tilingius \emph{sur la Rhubarbe}, publié en Allemagne en 1679.\par
On n’était pas sûr que Gilliatt ne fît pas des charmes, des philtres et des « bouilleries ». Il avait des fioles.\par
Pourquoi allait-il se promener le soir, et quelquefois jusqu’à minuit, dans les falaises ? évidemment pour causer avec les mauvaises gens qui sont la nuit au bord de la mer dans de la fumée.\par
Une fois il avait aidé la sorcière de Torteval à désembourber son chariot. Une vieille, nommée Moutonne Gahy.\par
A un recensement qui s’était fait dans l’île, interrogé sur sa profession, il avait répondu : — \emph{Pêcheur, quand il y a du poisson à prendre.} — Mettez-vous à la place des gens, on n’aime pas ces réponses-là.\par
La pauvreté et la richesse sont de comparaison. Gilliatt avait des champs et une maison, et, comparé à ceux qui n’ont rien du tout, il n’était pas pauvre. Un jour, pour l’éprouver, et peut-être aussi pour lui faire une avance, car il y a des femmes qui épouseraient le  diable riche, une fille dit à Gilliatt : Quand donc prendrez-vous femme ? Il répondit : \emph{Je prendrai femme quand la Roque qui Chante prendra homme.}\par
Cette Roque qui Chante est une grande pierre plantée droite dans un courtil proche monsieur Lemézurier de Fry. Cette pierre est fort à surveiller. On ne sait ce qu’elle fait là. On y entend chanter un coq qu’on ne voit pas, chose extrêmement désagréable. Ensuite il est avéré qu’elle a été mise dans ce courtil par les sarregousets, qui sont la même chose que les sins.\par
La nuit, quand il tonne, si l’on voit des hommes voler dans le rouge des nuées et dans le tremblement de l’air, ce sont les sarregousets. Une femme, qui demeure au Grand-Mielles, les connaît. Un soir qu’il y avait des sarregousets dans un carrefour, cette femme cria à un charretier qui ne savait quelle route prendre : \emph{Demandez-leur votre chemin ; c’est des gens bien faisants, c’est des gens bien civils à deviser au monde.} Il y a gros à parier que cette femme est une sorcière.\par
Le judicieux et savant roi Jacques I\textsuperscript{er} faisait bouillir toutes vives les femmes de cette espèce, goûtait le bouillon, et, au goût du bouillon, disait : \emph{C’était une sorcière,} ou \emph{Ce n’en était pas une.}\par
Il est à regretter que les rois d’aujourd’hui n’aient plus de ces talents-là, qui faisaient comprendre l’utilité de l’institution.\par
Gilliatt, non sans de sérieux motifs, vivait en odeur de sorcellerie. Dans un orage, à minuit, Gilliatt étant en mer seul dans une barque du côté de la Sommeilleuse, on l’entendit demander :\par
 — Y a-t-il du rang pour passer ?\par
Une voix cria du haut des roches :\par
— Voire ! hardi !\par
A qui parlait-il, si ce n’est à quelqu’un qui lui répondait ? Ceci nous semble une preuve.\par
Dans une autre soirée d’orage, si noire qu’on ne voyait rien, tout près de la Catiau-Roque, qui est une double rangée de roches où les sorciers, les chèvres et les faces vont danser le vendredi, on crut être certain de reconnaître la voix de Gilliatt mêlée à l’épouvantable conversation que voici :\par
— Comment se porte Vésin Brovard ? (C’était un maçon qui était tombé d’un toit.)\par
— Il guarit.\par
— Ver dia ! il a chu de plus haut que ce grand pau\footnote{ \noindent \emph{Pau}, poteau.
 }. C’est ravissant qu’il ne se soit rien rompu.\par
— Les gens eurent beau temps au varech la semaine passée.\par
— Plus qu’ogny\footnote{ \noindent \emph{Ogny}, aujourd’hui.
 }.\par
— Voire ! il n’y aura pas hardi de poisson au marché.\par
— Il vente trop dur.\par
— Ils ne sauraient mettre leurs rets bas.\par
— Comment va la Catherine ?\par
— Elle est de charme.\par
« La Catherine » était évidemment une sarregousette.\par
Gilliatt, selon toute apparence, faisait œuvre de nuit. Du moins, personne n’en doutait.\par
 On le voyait quelquefois, avec une cruche qu’il avait, verser de l’eau à terre. Or l’eau qu’on jette à terre trace la forme des diables.\par
Il existe sur la route de Saint-Sampson, vis-à-vis le martello numéro I, trois pierres arrangées en escalier. Elles ont porté sur leur plate-forme, vide aujourd’hui, une croix, à moins qu’elles n’aient porté un gibet. Ces pierres sont très malignes.\par
Des gens fort prud’hommes et des personnes absolument croyables affirmaient avoir vu, près de ces pierres, Gilliatt causer avec un crapaud. Or il n’y a pas de crapauds à Guernesey ; Guernesey a toutes les couleuvres, et Jersey a tous les crapauds. Ce crapaud avait dû venir de Jersey à la nage pour parler à Gilliatt. La conversation était amicale.\par
Ces faits demeurèrent constatés ; et la preuve, c’est que les trois pierres sont encore là. Les gens qui douteraient peuvent les aller voir ; et même, à peu de distance, il y a une maison au coin de laquelle on lit cette enseigne : \emph{Marchand en bétail mort et vivant, vieux cordage, fer, os et chiques ; est prompt dans son paiement et dans son attention}.\par
Il faudrait être de mauvaise foi pour contester la présence de ces pierres et l’existence de cette maison. Tout cela nuisait à Gilliatt.\par
Les ignorants seuls ignorent que le plus grand danger des mers de la Manche, c’est le Roi des Auxcriniers. Pas de personnage marin plus redoutable. Qui l’a vu fait naufrage entre une Saint-Michel et l’autre. Il est petit, étant nain, et il est sourd, étant roi. Il  sait les noms de tous ceux qui sont morts dans la mer et l’endroit où ils sont. Il connaît à fond le cimetière océan. Une tête massive en bas et étroite en haut, un corps trapu, un ventre visqueux et difforme, des nodosités sur le crâne, de courtes jambes, de longs bras, pour pieds des nageoires, pour mains des griffes, un large visage vert, tel est ce roi. Ses griffes sont palmées et ses nageoires sont onglées. Qu’on imagine un poisson qui est un spectre, et qui a une figure d’homme. Pour en finir avec lui, il faudrait l’exorciser, ou le pêcher. En attendant, il est sinistre. Rien n’est moins rassurant que de l’apercevoir. On entrevoit, au-dessus des lames et des houles, derrière les épaisseurs de la brume, un linéament qui est un être ; un front bas, un nez camard, des oreilles plates, une bouche démesurée où il manque des dents, un rictus glauque, des sourcils en chevrons, et de gros yeux gais. Il est rouge quand l’éclair est livide, et blafard quand l’éclair est pourpre. Il a une barbe ruisselante et rigide qui s’étale, coupée carrément, sur une membrane en forme de pèlerine, laquelle est ornée de quatorze coquilles, sept par devant et sept par derrière. Ces coquilles sont extraordinaires pour ceux qui se connaissent en coquilles. Le Roi des Auxcriniers n’est visible que dans la mer violente. Il est le baladin lugubre de la tempête. On voit sa forme s’ébaucher dans le brouillard, dans la rafale, dans la pluie. Son nombril est hideux. Une carapace de squames lui cache les côtes, comme ferait un gilet. Il se dresse debout au haut de ces vagues roulées qui jaillissent sous la pression des souffles et  se tordent comme les copeaux sortant du rabot du menuisier. Il se tient tout entier hors de l’écume, et, s’il y a à l’horizon des navires en détresse, blême dans l’ombre, la face éclairée de la lueur d’un vague sourire, l’air fou et terrible, il danse. C’est là une vilaine rencontre. A l’époque où Gilliatt était une des préoccupations de Saint-Sampson, les dernières personnes qui avaient vu le Roi des Auxcriniers déclaraient qu’il n’avait plus à sa pèlerine que treize coquilles. Treize ; il n’en était que plus dangereux. Mais qu’était devenue la quatorzième ? L’avait-il donnée à quelqu’un ? Et à qui l’avait-il donnée ? Nul ne pouvait le dire, et l’on se bornait à conjecturer. Ce qui est certain, c’est que M. Lubin-Mabier, du lieu les Godaines, homme ayant de la surface, propriétaire taxé à quatrevingts quartiers, était prêt à jurer sous serment qu’il avait vu une fois dans les mains de Gilliatt une coquille très singulière.\par
Il n’était point rare d’entendre de ces dialogues entre deux paysans :\par
— N’est-ce pas, mon voisin, que j’ai là un beau bœuf ?\par
— Bouffi, mon voisin.\par
— Tiens, c’est vrai tout de même.\par
— Il est meilleur en suif qu’il n’est en viande.\par
— Ver dia !\par
— Êtes-vous certain que Gilliatt ne l’a point regardé ?\par
Gilliatt s’arrêtait au bord des champs près des laboureurs et au bord des jardins près des jardiniers, et  il lui arrivait de leur dire des paroles mystérieuses :\par
— Quand le mors du diable fleurit, moissonnez le seigle d’hiver.\par
(Parenthèse : le mors du diable, c’est la scabieuse.)\par
— Le frêne se feuille, il ne gèlera plus.\par
— Solstice d’été, chardon en fleur.\par
— S’il ne pleut pas en juin, les blés prendront le blanc. Craignez la nielle.\par
— Le merisier fait ses grappes, méfiez-vous de la pleine lune.\par
— Si le temps, le sixième jour de la lune, se comporte comme le quatrième ou comme le cinquième jour, il se comportera de même, neuf fois sur douze dans le premier cas, et onze fois sur douze dans le second, pendant toute la lune.\par
— Ayez l’œil sur les voisins en procès avec vous. Prenez garde aux malices. Un cochon à qui on fait boire du lait chaud, crève. Une vache à qui on frotte les dents avec du poireau, ne mange plus.\par
— L’éperlan fraye, gare les fièvres.\par
— La grenouille se montre, semez les melons.\par
— L’hépathique fleurit, semez l’orge.\par
— Le tilleul fleurit, fauchez les prés.\par
— L’ypréau fleurit, ouvrez les bâches.\par
— Le tabac fleurit, fermez les serres.\par
Et, chose terrible, si l’on suivait ses conseils, on s’en trouvait bien.\par
Une nuit de juin qu’il joua du bug-pipe dans la dune, du côté de la Demie de Fontenelle, la pêche aux maquereaux manqua.\par
 Un soir, à la marée basse, sur la grève en face de sa maison du Bû de la Rue, une charrette chargée de varech versa. Il eut probablement peur d’être traduit en justice, car il se donna beaucoup de peine pour aider à relever la charrette, et il la rechargea lui-même.\par
Une petite fille du voisinage ayant des poux, il était allé à Saint-Pierre-Port, était revenu avec un onguent, et en avait frotté l’enfant ; et Gilliatt lui avait ôté ses poux, ce qui prouve que Gilliatt les lui avait donnés.\par
Tout le monde sait qu’il y a un charme pour donner des poux aux personnes.\par
Gilliatt passait pour regarder les puits, ce qui est dangereux quand le regard est mauvais ; et le fait est qu’un jour aux Arculons, près Saint-Pierre-Port, l’eau d’un puits devint malsaine. La bonne femme à qui était le puits dit à Gilliatt : Voyez donc cette eau. Et elle lui en montra un plein verre. Gilliatt avoua. — L’eau est épaisse, dit-il ; c’est vrai. La bonne femme, qui se méfiait, lui dit : Guérissez-moi-la donc. Gilliatt lui fit des questions : — si elle avait une étable ? — si l’étable avait un égout ? — si le ruisseau de l’égout ne passait pas tout près du puits ? — La bonne femme répondit oui. Gilliatt entra dans l’étable, travailla à l’égout, détourna le ruisseau, et l’eau du puits redevint bonne. On pensa dans le pays ce qu’on voulut. Un puits n’est pas mauvais, et ensuite bon, sans motif ; on ne trouva point la maladie de ce puits naturelle, et il est difficile de ne pas croire en effet que Gilliatt avait jeté un sort à cette eau.\par
 Une fois qu’il était allé à Jersey, on remarqua qu’il s’était logé à Saint-Clément, rue des Alleurs. Les alleurs, ce sont les revenants.\par
Dans les villages, on recueille des indices sur un homme ; on rapproche ces indices ; le total fait une réputation.\par
Il arriva que Gilliatt fut surpris saignant du nez. Ceci parut grave. Un patron de barque, fort voyageur, qui avait presque fait le tour du monde, affirma que chez les tungouses tous les sorciers saignent du nez. Quand on voit un homme saigner du nez, on sait à quoi s’en tenir. Toutefois les gens raisonnables firent remarquer que ce qui caractérise les sorciers en Tungousie peut ne point les caractériser au même degré à Guernesey.\par
Aux environs d’une Saint-Michel, on le vit s’arrêter dans un pré des courtils des Huriaux, bordant la grande route des Videclins. Il siffla dans le pré, et un moment après il y vint un corbeau, et un moment après il y vint une pie. Le fait fut attesté par un homme notable, qui depuis a été douzenier dans la Douzaine autorisée à faire un nouveau livre de Perchage du fief le Roi.\par
Au Hamel, dans la vingtaine de l’Épine, il y avait des vieilles femmes qui disaient être sûres d’avoir entendu un matin, à la piperette du jour, des hirondelles appeler Gilliatt.\par
Ajoutez qu’il n’était pas bon.\par
Un jour, un pauvre homme battait un âne. L’âne n’avançait pas. Le pauvre homme lui donna quelques  coups de sabot dans le ventre, et l’âne tomba. Gilliatt accourut pour relever l’âne, l’âne était mort. Gilliatt souffleta le pauvre homme.\par
Un autre jour, voyant un garçon descendre d’un arbre avec une couvée de petits épluque-pommiers nouveau-nés, presque sans plumes et tout nus, Gilliatt prit cette couvée à ce garçon, et poussa la méchanceté jusqu’à la reporter dans l’arbre.\par
Des passants lui en firent des reproches, il se borna à montrer le père et la mère épluque-pommiers qui criaient au-dessus de l’arbre et qui revenaient à leur couvée. Il avait un faible pour les oiseaux. C’est un signe auquel on reconnaît généralement les magiciens.\par
Les enfants ont pour joie de dénicher les nids de goélands et de mauves dans les falaises. Ils en rapportent des quantités d’œufs bleus, jaunes et verts avec lesquels on fait des rosaces sur les devantures des cheminées. Comme les falaises sont à pic, quelquefois le pied leur glisse, ils tombent, et se tuent. Rien n’est joli comme les paravents décorés d’œufs d’oiseaux de mer. Gilliatt ne savait qu’inventer pour faire le mal. Il grimpait, au péril de sa propre vie, dans les escarpements des roches marines, et y accrochait des bottes de foin avec de vieux chapeaux et toutes sortes d’épouvantails, afin d’empêcher les oiseaux d’y nicher, et, par conséquent, les enfants d’y aller.\par
C’est pourquoi Gilliatt était à peu près haï dans le pays. On le serait à moins.
 \subsubsection[{A.I.5. Autres cotés louches de gilliatt}]{A.I.5. \\
Autres cotés louches de gilliatt}
\noindent L’opinion n’était pas bien fixée sur le compte de Gilliatt.\par
Généralement on le croyait marcou, quelques-uns allaient jusqu’à le croire cambion. Le cambion est le fils qu’une femme a du diable.\par
Quand une femme a d’un homme sept enfants mâles consécutifs, le septième est marcou. Mais il ne faut pas qu’une fille gâte la série des garçons.\par
Le marcou a une fleur de lys naturelle empreinte sur une partie quelconque du corps, ce qui fait qu’il guérit les écrouelles aussi bien que les rois de France. Il y a des marcous en France un peu partout, particulièrement dans l’Orléanais. Chaque village du Gâtinais a son marcou. Il suffit, pour guérir les malades, que le marcou souffle sur leurs plaies ou leur fasse toucher sa fleur de lys. La chose réussit surtout dans la nuit du vendredi saint. Il y a une dizaine d’années, le marcou d’Ormes en Gâtinais, surnommé le Beau Marcou et consulté de toute la Beauce, était un tonnelier appelé  Foulon, qui avait cheval et voiture. On dut, pour empêcher ses miracles, faire jouer la gendarmerie. Il avait la fleur de lys sous le sein gauche. D’autres marcous l’ont ailleurs.\par
Il y a des marcous à Jersey, à Aurigny et à Guernesey. Cela tient sans doute aux droits que la France a sur le duché de Normandie. Autrement, à quoi bon la fleur de lys ?\par
Il y a aussi dans les îles de la Manche des scrofuleux ; ce qui rend les marcous nécessaires.\par
Quelques personnes s’étant trouvées présentes un jour que Gilliatt se baignait dans la mer avaient cru lui voir la fleur de lys. Questionné là-dessus, il s’était, pour toute réponse, mis à rire. Car il riait comme les autres hommes, quelquefois. Depuis ce temps-là, on ne le voyait plus se baigner ; il ne se baignait que dans des lieux périlleux et solitaires. Probablement la nuit, au clair de lune ; chose, on en conviendra, suspecte.\par
Ceux qui s’obstinaient à le croire cambion, c’est-à-dire fils du diable, se trompaient évidemment. Ils auraient dû savoir qu’il n’y a guère de cambions qu’en Allemagne. Mais le Valle et Saint-Sampson étaient, il y a cinquante ans, des pays d’ignorance.\par
Croire, à Guernesey, quelqu’un fils du diable, il y a visiblement là de l’exagération.\par
Gilliatt, par cela même qu’il inquiétait, était consulté. Les paysans venaient, avec peur, lui parler de leurs maladies. Cette peur-là contient de la confiance ; et, dans la campagne, plus le médecin est suspect, plus le remède est sûr. Gilliatt avait des médicaments  à lui, qu’il tenait de la vieille femme morte ; il en faisait part à qui les lui demandait, et ne voulait pas recevoir d’argent. Il guérissait les panaris avec des applications d’herbes ; la liqueur d’une de ses fioles coupait la fièvre ; le chimiste de Saint-Sampson, que nous appellerions pharmacien en France, pensait que c’était probablement une décoction de quinquina. Les moins bienveillants convenaient que Gilliatt était assez bon diable pour les malades quand il s’agissait de ses remèdes ordinaires ; mais, comme marcou, il ne voulait rien entendre ; si un scrofuleux lui demandait à toucher sa fleur de lys, pour toute réponse il lui fermait sa porte au nez ; faire des miracles était une chose à laquelle il se refusait obstinément, ce qui est ridicule à un sorcier. Ne soyez pas sorcier ; mais, si vous l’êtes, faites votre métier.\par
Il y avait une ou deux exceptions à l’antipathie universelle. Sieur Landoys, du Clos-Landès, était clerc greffier de la paroisse de Saint-Pierre-Port, chargé des écritures et gardien du registre des naissances, mariages et décès. Ce greffier Landoys tirait vanité de descendre du trésorier de Bretagne Pierre Landais, pendu en 1485. Un jour sieur Landoys poussa son bain trop avant dans la mer, et faillit se noyer. Gilliatt se jeta à l’eau, faillit se noyer lui aussi, et sauva Landoys. A partir de ce jour, Landoys ne dit plus de mal de Gilliatt. A ceux qui s’en étonnaient, il répondait : \emph{Pourquoi voulez-vous que je déteste un homme qui ne m’a rien fait, et qui m’a rendu service ?} Le clerc greffier en vint même à prendre Gilliatt en une certaine amitié. Ce  clerc greffier était un homme sans préjugés. Il ne croyait pas aux sorciers. Il riait de ceux qui ont peur des revenants. Quant à lui, il avait un bateau, il pêchait dans ses heures de loisir pour s’amuser, et il n’avait jamais rien vu d’extraordinaire, si ce n’est une fois au clair de lune une femme blanche qui sautait sur l’eau, et encore il n’en était pas bien sûr. Moutonne Gahy, la sorcière de Torteval, lui avait donné un petit sac qu’on s’attache sous la cravate et qui protège contre les esprits ; il se moquait de ce sac, et ne savait ce qu’il contenait ; pourtant il le portait, se sentant plus en sûreté quand il avait cette chose au cou.\par
Quelques personnes hardies se risquaient, à la suite du sieur Landoys, à constater en Gilliatt certaines circonstances atténuantes, quelques apparences de qualités, sa sobriété, son abstinence de gin et de tabac, et l’on en venait parfois jusqu’à faire de lui ce bel éloge : \emph{Il ne boit, ne fume, ne chique, ni ne snuffe.}\par
Mais être sobre, ce n’est une qualité que lorsqu’on en a d’autres.\par
L’aversion publique était sur Gilliatt.\par
Quoi qu’il en fût, comme marcou, Gilliatt pouvait rendre des services. Un certain vendredi saint, à minuit, jour et heure usités pour ces sortes de cures, tous les scrofuleux de l’île, d’inspiration ou par rendez-vous pris entre eux, vinrent en foule au Bû de la Rue, à mains jointes, et avec des plaies pitoyables, demander à Gilliatt de les guérir. Il refusa. On reconnut là sa méchanceté.
 \subsubsection[{A.I.6. La panse}]{A.I.6. \\
La panse}
\noindent Tel était Gilliatt.\par
Les filles le trouvaient laid.\par
Il n’était pas laid. Il était beau peut-être. Il avait dans le profil quelque chose d’un barbare antique. Au repos, il ressemblait à un dace de la colonne trajane. Son oreille était petite, délicate, sans lambeau, et d’une admirable forme acoustique. Il avait entre les deux yeux cette fière ride verticale de l’homme hardi et persévérant. Les deux coins de sa bouche tombaient, ce qui est amer ; son front était d’une courbe noble et sereine ; sa prunelle franche regardait bien, quoique troublée par ce clignement que donne aux pêcheurs la réverbération des vagues. Son rire était puéril et charmant. Pas de plus pur ivoire que ses dents. Mais le hâle l’avait fait presque nègre. On ne se mêle pas impunément à l’océan, à la tempête et à la nuit ; à trente ans, il en paraissait quarante-cinq. Il avait le sombre masque du vent et de la mer.\par
 On l’avait surnommé Gilliatt le Malin.\par
Une fable de l’Inde dit : Un jour Brahmâ demanda à la Force : Qui est plus fort que toi ? Elle répondit : L’Adresse. Un proverbe chinois dit : Que ne pourrait le lion, s’il était singe ! Gilliatt n’était ni lion, ni singe ; mais les choses qu’il faisait venaient à l’appui du proverbe chinois et de la fable indoue. De taille ordinaire et de force ordinaire, il trouvait moyen, tant sa dextérité était inventive et puissante, de soulever des fardeaux de géant et d’accomplir des prodiges d’athlète.\par
Il y avait en lui du gymnaste ; il se servait indifféremment de sa main droite et de sa main gauche.\par
Il ne chassait pas, mais il pêchait. Il épargnait les oiseaux, non les poissons. Malheur aux muets ! Il était nageur excellent.\par
La solitude fait des gens à talents ou des idiots. Gilliatt s’offrait sous ces deux aspects. Par moments on lui voyait « l’air étonné » dont nous avons parlé, et on l’eût pris pour une brute. Dans d’autres instants, il avait on ne sait quel regard profond. L’antique Chaldée a eu de ces hommes-là ; à de certaines heures, l’opacité du pâtre devenait transparente et laissait voir le mage.\par
En somme, ce n’était qu’un pauvre homme sachant lire et écrire. Il est probable qu’il était sur la limite qui sépare le songeur du penseur. Le penseur veut, le songeur subit. La solitude s’ajoute aux simples, et les complique d’une certaine façon. Ils se pénètrent à leur insu d’horreur sacrée. L’ombre où était l’esprit de Gilliatt se composait, en quantité presque égale, de  deux éléments obscurs tous deux, mais bien différents : en lui, l’ignorance, infirmité ; hors de lui, le mystère, immensité.\par
A force de grimper dans les rochers, d’escalader les escarpements, d’aller et de venir dans l’archipel par tous les temps, de manœuvrer la première embarcation venue, de se risquer jour et nuit dans les passes les plus difficiles, il était devenu, sans en tirer parti du reste, et pour sa fantaisie et son plaisir, un homme de mer surprenant.\par
Il était pilote né. Le vrai pilote est le marin qui navigue sur le fond plus encore que sur la surface. La vague est un problème extérieur, continuellement compliqué par la configuration sous-marine des lieux où le navire fait route. Il semblait, à voir Gilliatt voguer sur les bas-fonds et à travers les récifs de l’archipel normand, qu’il eût sous la voûte du crâne une carte du fond de la mer. Il savait tout et bravait tout.\par
Il connaissait les balises mieux que les cormorans qui s’y perchent. Les différences imperceptibles qui distinguent l’une de l’autre les quatre balises poteaux du Creux, d’Alligande, des Trémies et de la Sardrette étaient parfaitement nettes et claires pour lui, même dans le brouillard. Il n’hésitait ni sur le pieu à pomme ovale d’Anfré, ni sur le triple fer de lance de la Rousse, ni sur la boule blanche de la Corbette, ni sur la boule noire de Longue-Pierre, et il n’était pas à craindre qu’il confondît la croix de Goubeau avec l’épée plantée en terre de la Platte, ni la balise marteau des Barbées avec la balise queue d’aronde du Moulinet.\par
 Sa rare science de marin éclata singulièrement un jour qu’il y eut à Guernesey une de ces sortes de joutes marines qu’on nomme régates. La question était celle-ci : être seul dans une embarcation à quatre voiles, la conduire de Saint-Sampson à l’île de Herm qui est à une lieue, et la ramener de Herm à Saint-Sampson. Manœuvrer seul un bateau à quatre voiles, il n’est pas de pêcheur qui ne fasse cela, et la difficulté ne semble pas grande, mais voici ce qui l’aggravait. Premièrement, l’embarcation elle-même, laquelle était une de ces larges et fortes chaloupes ventrues d’autrefois, à la mode de Rotterdam, que les marins du siècle dernier appelaient des \emph{panses hollandaises.} On rencontre encore quelquefois en mer cet ancien gabarit de Hollande, joufflu et plat, et ayant à bâbord et à tribord deux ailes qui s’abattent, tantôt l’une, tantôt l’autre, selon le vent, et remplacent la quille. Deuxièmement, le retour de Herm ; retour qui se compliquait d’un lourd lest de pierres. On allait vide, mais on revenait chargé. Le prix de la joute était la chaloupe ! Elle était d’avance donnée au vainqueur. Cette panse avait servi de bateau-pilote ; le pilote qui l’avait montée et conduite pendant vingt ans était le plus robuste des marins de la Manche ; à sa mort on n’avait trouvé personne pour gouverner la panse, et l’on s’était décidé à en faire le prix d’une régate. La panse, quoique non pontée, avait des qualités, et pouvait tenter un manœuvrier. Elle était mâtée en avant, ce qui augmentait la puissance de traction de la voilure. Autre avantage, le mât ne gênait point le chargement. C’était une coque  solide ; pesante, mais vaste, et tenant bien le large ; une vraie barque commère. Il y eut empressement à se la disputer ; la joute était rude, mais le prix était beau. Sept ou huit pêcheurs, les plus vigoureux de l’île, se présentèrent. Ils essayèrent tour à tour ; pas un ne put aller jusqu’à Herm. Le dernier qui lutta était connu pour avoir franchi à la rame par un gros temps le redoutable étranglement de mer qui est entre Serk et Brecq-Hou. Ruisselant de sueur, il ramena la panse et dit : C’est impossible. Alors Gilliatt entra dans la barque, empoigna d’abord l’aviron, ensuite la grande écoute, et poussa au large. Puis, sans bitter l’écoute, ce qui eût été une imprudence, et sans la lâcher, ce qui le maintenait maître de la grande voile, laissant l’écoute rouler sur l’estrop au gré du vent sans dériver, il saisit de la main gauche la barre. En trois quarts d’heure, il fut à Herm. Trois heures après, quoiqu’un fort vent du sud se fût élevé et eût pris la rade en travers, la panse, montée par Gilliatt, rentrait à Saint-Sampson avec le chargement de pierres. Il avait, par luxe et bravade, ajouté au chargement le petit canon de bronze de Herm, que les gens de l’île tiraient tous les ans le 5 novembre en réjouissance de la mort de Guy Fawkes.\par
Guy Fawkes, disons-le en passant, est mort il y a deux cent soixante ans ; c’est là une longue joie.\par
Gilliatt, ainsi surchargé et surmené, quoiqu’il eût de trop le canon de Guy Fawkes dans sa barque et le vent du sud dans sa voile, ramena, on pourrait dire rapporta, la panse à Saint-Sampson.\par
 Ce que voyant, mess Lethierry s’écria : Voilà un matelot hardi !\par
Et il tendit la main à Gilliatt.\par
Nous reparlerons de mess Lethierry.\par
La panse fut adjugée à Gilliatt.\par
Cette aventure ne nuisit pas à son surnom de Malin.\par
Quelques personnes déclarèrent que la chose n’avait rien d’étonnant, attendu que Gilliatt avait caché dans le bateau une branche de mélier sauvage. Mais cela ne put être prouvé.\par
A partir de ce jour, Gilliatt n’eut plus d’autre embarcation que la panse. C’est dans cette lourde barque qu’il allait à la pêche. Il l’amarrait dans le très bon petit mouillage qu’il avait pour lui tout seul sous le mur même de sa maison du Bû de la Rue. A la tombée de la nuit, il jetait ses filets sur son dos, traversait son jardin, enjambait le parapet de pierres sèches, dégringolait d’un rocher à l’autre, et sautait dans la panse. De là au large.\par
Il pêchait beaucoup de poisson, mais on affirmait que la branche de mélier était toujours attachée à son bateau. Le mélier, c’est le néflier. Personne n’avait vu cette branche, mais tout le monde y croyait.\par
Le poisson qu’il avait de trop, il ne le vendait pas, il le donnait.\par
Les pauvres recevaient son poisson, mais lui en voulaient pourtant, à cause de cette branche de mélier. Cela ne se fait pas. On ne doit point tricher la mer.\par
Il était pêcheur, mais il n’était pas que cela. Il avait, d’instinct et pour se distraire, appris trois ou quatre  métiers. Il était menuisier, ferron, charron, calfat, et même un peu mécanicien. Personne ne raccommodait une roue comme lui. Il fabriquait dans un genre à lui tous ses engins de pêche. Il avait dans un coin du Bû de la Rue une petite forge et une enclume, et, la panse n’ayant qu’une ancre, il lui en avait fait, lui-même et lui seul, une seconde. Cette ancre était excellente ; l’organeau avait la force voulue, et Gilliatt, sans que personne le lui eût enseigné, avait trouvé la dimension exacte que doit avoir le jouail pour empêcher l’ancre de cabaner.\par
Il avait patiemment remplacé tous les clous du bordage de la panse par des gournables, ce qui rendait les trous de rouille impossibles.\par
De cette manière il avait beaucoup augmenté les bonnes qualités de mer de la panse. Il en profitait pour s’en aller de temps en temps passer un mois ou deux dans quelque îlot solitaire comme Chousey ou les Casquets. On disait : Tiens, Gilliatt n’est plus là. Cela ne faisait de peine à personne.
 \subsubsection[{A.I.7. À maison visionnée habitant visionnaire}]{A.I.7. \\
À maison visionnée habitant visionnaire}
\noindent Gilliatt était l’homme du songe. De là ses audaces, de là aussi ses timidités. Il avait ses idées à lui.\par
Peut-être y avait-il en Gilliatt de l’halluciné et de l’illuminé. L’hallucination hante tout aussi bien un paysan comme Martin qu’un roi comme Henri IV. L’Inconnu fait parfois à l’esprit de l’homme des surprises. Une brusque déchirure de l’ombre laisse tout à coup voir l’invisible, puis se referme. Ces visions sont quelquefois transfiguratrices ; elles font d’un chamelier Mahomet et d’une chevrière Jeanne d’Arc. La solitude dégage une certaine quantité d’égarement sublime. C’est la fumée du buisson ardent. Il en résulte un mystérieux tremblement d’idées qui dilate le docteur en voyant et le poète en prophète ; il en résulte Horeb, le Cédron, Ombos, les ivresses du laurier de Castalie mâché, les révélations du mois Busion ; il en résulte Péleïa à Dodone, Phémonoë à Delphes, Trophonius à  Lébadée, Ezéchiel sur le Kébar, Jérôme dans la Thébaïde. Le plus souvent l’état visionnaire accable l’homme, et le stupéfie. L’abrutissement sacré existe. Le fakir a pour fardeau sa vision comme le crétin son goître. Luther parlant aux diables dans le grenier de Wittemberg, Pascal masquant l’enfer avec le paravent de son cabinet, l’obi nègre dialoguant avec le dieu Bossum à face blanche, c’est le même phénomène, diversement porté par les cerveaux qu’il traverse, selon leur force et leur dimension. Luther et Pascal sont et restent grands ; l’obi est imbécile.\par
Gilliatt n’était ni si haut, ni si bas. C’était un pensif. Rien de plus.\par
Il voyait la nature un peu étrangement.\par
De ce qu’il lui était arrivé plusieurs fois de trouver dans de l’eau de mer parfaitement limpide d’assez gros animaux inattendus, de formes diverses, de l’espèce méduse, qui, hors de l’eau, ressemblaient à du cristal mou, et qui, rejetés dans l’eau, s’y confondaient avec leur milieu, par l’identité de diaphanéité et de couleur, au point d’y disparaître, il concluait que, puisque des transparences vivantes habitaient l’eau, d’autres transparences, également vivantes, pouvaient bien habiter l’air. Les oiseaux ne sont pas les habitants de l’air ; ils en sont les amphibies. Gilliatt ne croyait pas à l’air désert. Il disait : Puisque la mer est remplie, pourquoi l’atmosphère serait-elle vide ? Des créatures couleur d’air s’effaceraient dans la lumière et échapperaient à notre regard ; qui nous prouve qu’il n’y en a pas ? L’analogie indique que l’air doit avoir ses poissons  comme la mer a les siens ; ces poissons de l’air seraient diaphanes, bienfait de la prévoyance créatrice pour nous comme pour eux ; laissant passer le jour à travers leur forme et ne faisant point d’ombre et n’ayant pas de silhouette, ils resteraient ignorés de nous, et nous n’en pourrions rien saisir. Gilliatt imaginait que si l’on pouvait mettre la terre à sec d’atmosphère, et que si l’on pêchait l’air comme on pêche un étang, on y trouverait une foule d’êtres surprenants. Et, ajoutait-il dans sa rêverie, bien des choses s’expliqueraient.\par
La rêverie, qui est la pensée à l’état de nébuleuse, confine au sommeil, et s’en préoccupe comme de sa frontière. L’air habité par des transparences vivantes, ce serait le commencement de l’inconnu ; mais au delà s’offre la vaste ouverture du possible. Là d’autres êtres, là d’autres faits. Aucun surnaturalisme ; mais la continuation occulte de la nature infinie. Gilliatt, dans ce désœuvrement laborieux qui était son existence, était un bizarre observateur. Il allait jusqu’à observer le sommeil. Le sommeil est en contact avec le possible, que nous nommons aussi l’invraisemblable. Le monde nocturne est un monde. La nuit, en tant que nuit, est un univers. L’organisme matériel humain, sur lequel pèse une colonne atmosphérique de quinze lieues de haut, est fatigué le soir, il tombe de lassitude, il se couche, il se repose ; les yeux de chair se ferment ; alors dans cette tête assoupie, moins inerte qu’on ne croit, d’autres yeux s’ouvrent ; l’inconnu apparaît. Les choses sombres du monde ignoré deviennent voisines de l’homme, soit qu’il y ait communication véritable,  soit que les lointains de l’abîme aient un grossissement visionnaire ; il semble que les vivants indistincts de l’espace viennent nous regarder et qu’ils aient une curiosité de nous, les vivants terrestres ; une création fantôme monte ou descend vers nous et nous côtoie dans un crépuscule ; devant notre contemplation spectrale, une vie autre que la nôtre s’agrège et se désagrége, composée de nous-mêmes et d’autre chose ; et le dormeur, pas tout à fait voyant, pas tout à fait inconscient, entrevoit ces animalités étranges, ces végétations extraordinaires, ces lividités terribles ou souriantes, ces larves, ces masques, ces figures, ces hydres, ces confusions, ce clair de lune sans lune, ces obscures décompositions du prodige, ces croissances et ces décroissances dans une épaisseur trouble, ces flottaisons de formes dans les ténèbres, tout ce [{\corr mystère}] que nous appelons le songe et qui n’est autre chose que l’approche d’une réalité invisible. Le rêve est l’aquarium de la nuit.\par
Ainsi songeait Gilliatt.
 \subsubsection[{A.I.8. La chaise gild-holm-’ur}]{A.I.8. \\
La chaise gild-holm-’ur}
\noindent Ce serait vainement qu’on chercherait aujourd’hui, dans l’anse du Houmet, la maison de Gilliatt, son jardin, et la crique où il abritait la panse. Le Bû de la Rue n’existe plus. La petite presqu’île qui portait cette maison est tombée sous le pic des démolisseurs de falaises et a été chargée, charretée à charretée, sur les navires des brocanteurs de rochers et des marchands de granit. Elle est devenue quai, église et palais, dans la capitale. Toute cette crête d’écueils est depuis longtemps partie pour Londres.\par
Ces allongements de rochers dans la mer, avec leurs crevasses et leurs dentelures, sont de vraies petites chaînes de montagnes ; on a, en les voyant, l’impression qu’aurait un géant regardant les Cordillères. L’idiome local les appelle Banques. Ces banques ont des figures diverses. Les unes ressemblent à une épine dorsale, chaque rocher est une vertèbre ; les  autres à une arête de poisson ; les autres à un crocodile qui boit.\par
A l’extrémité de la banque du Bû de la Rue, il y avait une grande roche que les pêcheurs du Houmet appelaient la Corne de la Bête. Cette roche, sorte de pyramide, ressemblait, quoique moins élevée, au pinacle de Jersey. A marée haute, le flot la séparait de la banque, et la Corne était isolée. A marée basse, on y arrivait par un isthme de roches praticables. La curiosité de ce rocher, c’était, du côté de la mer, une sorte de chaise naturelle creusée par la vague et polie par la pluie. Cette chaise était traître. On y était insensiblement amené par la beauté de la vue ; on s’y arrêtait « pour l’amour du prospect », comme on dit à Guernesey ; quelque chose vous retenait ; il y a un charme dans les grands horizons. Cette chaise s’offrait ; elle faisait une sorte de niche dans la façade à pic du rocher ; grimper à cette niche était facile, la mer qui l’avait taillée dans le roc avait étagé au-dessous et commodément disposé une sorte d’escalier de pierres plates ; l’abîme a de ces prévenances, défiez-vous de ses politesses ; la chaise tentait, on y montait, on s’y asseyait ; là on était à l’aise ; pour siège le granit usé et arrondi par l’écume, pour accoudoirs deux anfractuosités qui semblaient faites exprès, pour dossier toute la haute muraille verticale du rocher qu’on admirait au-dessus de sa tête sans penser à se dire qu’il serait impossible de l’escalader ; rien de plus simple que de s’oublier dans ce fauteuil ; on découvrait toute la mer, on voyait au loin les navires arriver ou s’en  aller, on pouvait suivre des yeux une voile jusqu’à ce qu’elle s’enfonçât au delà des Casquets sous la rondeur de l’océan, on s’émerveillait, on regardait, on jouissait, on sentait la caresse de la brise et du flot ; il existe à Cayenne un vespertilio, sachant ce qu’il fait, qui vous endort dans l’ombre avec un doux et ténébreux battement d’ailes ; le vent est cette chauve-souris invisible ; quand il n’est pas ravageur, il est endormeur. On contemplait la mer, on écoutait le vent, on se sentait gagner par l’assoupissement de l’extase. Quand les yeux sont remplis d’un excès de beauté et de lumière, c’est une volupté de les fermer. Tout à coup on se réveillait. Il était trop tard. La marée avait grossi peu à peu. L’eau enveloppait le rocher.\par
On était perdu.\par
Redoutable blocus que celui-ci : la mer montante.\par
La marée croît insensiblement d’abord, puis violemment. Arrivée aux rochers, la colère la prend, elle écume. Nager ne réussit pas toujours dans les brisants. D’excellents nageurs s’étaient noyés à la Corne du Bû de la Rue.\par
En de certains lieux, à de certaines heures, regarder la mer est un poison. C’est comme, quelquefois, regarder une femme.\par
Les très anciens habitants de Guernesey appelaient jadis cette niche façonnée dans le roc par le flot la Chaise Gild-Holm-’Ur, ou \emph{Kidormur}. Mot celte, dit-on, que ceux qui savent le celte ne comprennent pas et que ceux qui savent le français comprennent. \emph{Qui-dort-meurt}. Telle est la traduction paysanne.\par
 On est libre de choisir entre cette traduction, \emph{Qui-dort-meurt}, et la traduction donnée en 1819, je crois, dans l’\emph{Armoricain,} par M. Athénas. Selon cet honorable celtisant, Gild-Holm-’Ur signifierait \emph{Halte-de-troupes-d’oiseaux.}\par
Il existe à Aurigny une autre chaise de ce genre, qu’on nomme la Chaise-au-Moine, si bien confectionnée par le flot, et avec une saillie de roche ajustée si à propos, qu’on pourrait dire que la mer a la complaisance de vous mettre un tabouret sous les pieds.\par
Au plein de la mer, à la marée haute, on n’apercevait plus la chaise Gild-Holm-’Ur. L’eau la couvrait entièrement.\par
La chaise Gild-Holm-’Ur était la voisine du Bû de la Rue. Gilliatt la connaissait et s’y asseyait. Il venait souvent là. Méditait-il ? Non. Nous venons de le dire, il songeait. Il ne se laissait pas surprendre par la marée.\par
  \subsection[{A.II. Livre deuxième. Mess Lethierry}]{A.II. Livre deuxième \\
Mess Lethierry}
  \subsubsection[{A.II.1. Vie agité et conscience tranquille}]{A.II.1. \\
Vie agité et conscience tranquille}
\noindent Mess Lethierry, l’homme notable de Saint-Sampson, était un matelot terrible. Il avait beaucoup navigué. Il avait été mousse, voilier, gabier, timonier, contre-maître, maître d’équipage, pilote, patron. Il était maintenant armateur. Il n’y avait pas un autre homme comme lui pour savoir la mer. Il était intrépide aux sauvetages. Dans les gros temps il s’en allait le long de la grève, regardant à l’horizon. Qu’est-ce que c’est que ça là-bas ? il y a quelqu’un en peine. C’est un chasse-marée de Weymouth, c’est un coutre d’Aurigny, c’est une bisquine de Courseulle, c’est le yacht d’un lord, c’est un anglais, c’est un français, c’est un pauvre, c’est un riche, c’est le diable, n’importe, il sautait dans une barque, appelait deux ou trois vaillants hommes, s’en passait au besoin, faisait l’équipe à lui tout seul, détachait l’amarre, prenait la rame, poussait en haute mer, montait et descendait et remontait dans  les creux du flot, plongeait dans l’ouragan, allait au danger. On le voyait de loin dans la rafale, debout sur l’embarcation, ruisselant de pluie, mêlé aux éclairs, avec la face d’un lion qui aurait une crinière d’écume. Il passait quelquefois ainsi toute sa journée dans le risque, dans la vague, dans la grêle, dans le vent, accostant les navires en perdition, sauvant les hommes, sauvant les chargements, cherchant dispute à la tempête. Le soir il rentrait chez lui, et tricotait une paire de bas.\par
Il mena cette vie cinquante ans, de dix ans à soixante, tant qu’il fut jeune. A soixante ans, il s’aperçut qu’il ne levait plus d’un seul bras l’enclume de la forge du Varclin ; cette enclume pesait trois cents livres ; et tout à coup il fut fait prisonnier par les rhumatismes. Il lui fallut renoncer à la mer. Alors il passa de l’âge héroïque à l’âge patriarcal. Ce ne fut plus qu’un bonhomme.\par
Il était arrivé en même temps aux rhumatismes et à l’aisance. Ces deux produits du travail se tiennent volontiers compagnie. Au moment où l’on devient riche, on est paralysé. Cela couronne la vie.\par
On se dit : jouissons maintenant.\par
Dans les îles comme Guernesey, la population est composée d’hommes qui ont passé leur vie à faire le tour de leur champ et d’hommes qui ont passé leur vie à faire le tour du monde. Ce sont les deux sortes de laboureurs, ceux-ci de la terre, ceux-là de la mer. Mess Lethierry était des derniers. Pourtant il connaissait la terre. Il avait eu une forte vie de travailleur. Il  avait voyagé sur le continent. Il avait été quelque temps charpentier de navire à Rochefort, puis à Cette. Nous venons de parler du tour du monde ; il avait accompli son tour de France comme compagnon dans la charpenterie. Il avait travaillé aux appareils d’épuisement des salines de Franche-Comté. Cet honnête homme avait eu une vie d’aventurier. En France il avait appris à lire, à penser, à vouloir. Il avait fait de tout, et, de tout ce qu’il avait fait, il avait extrait la probité. Le fond de sa nature, c’était le matelot. L’eau lui appartenait. Il disait : Les poissons sont chez moi. En somme toute son existence, à deux ou trois années près, avait été donnée à l’océan ; \emph{jetée à l’eau}, disait-il. Il avait navigué dans les grandes mers, dans l’Atlantique et dans le Pacifique, mais il préférait la Manche. Il s’écriait avec amour : \emph{C’est celle-là qui est rude !} Il y était né et voulait y mourir. Après avoir fait un ou deux tours du monde, sachant à quoi s’en tenir, il était revenu à Guernesey, et n’en avait plus bougé. Ses voyages désormais étaient Granville et Saint-Malo.\par
Mess Lethierry était guernesiais, c’est-à-dire normand, c’est-à-dire anglais, c’est-à-dire français. Il avait en lui cette patrie quadruple, immergée et comme noyée dans sa grande patrie l’océan. Toute sa vie et partout, il avait gardé ses mœurs de pêcheur normand.\par
Cela ne l’empêchait point d’ouvrir un bouquin dans l’occasion, de se plaire à un livre, de savoir des noms de philosophes et de poètes, et de baragouiner un peu toutes les langues.
 \subsubsection[{A.II.2. Un gout qu’il avait}]{A.II.2. \\
Un gout qu’il avait}
\noindent Gilliatt était un sauvage. Mess Lethierry en était un autre.\par
Ce sauvage avait ses élégances.\par
Il était difficile pour les mains des femmes. Dans sa jeunesse, presque enfant encore, étant entre matelot et mousse, il avait entendu le bailli de Suffren s’écrier : \emph{Voilà une jolie fille, mais quelles grandes diables de mains rouges !} Un mot d’amiral, en toute matière, commande. Au-dessus d’un oracle, il y a une consigne. L’exclamation du bailli de Suffren avait rendu Lethierry délicat et exigeant en fait de petites mains blanches. Sa main à lui, large spatule couleur acajou, était massue pour la légèreté et tenaille pour la caresse, et cassait un pavé en tombant dessus, fermée.\par
Il ne s’était jamais marié. Il n’avait pas voulu ou pas trouvé. Cela tenait peut-être à ce que ce matelot prétendait à des mains de duchesse. On ne rencontre  guère de ces mains-là dans les pêcheuses de Portbail.\par
On racontait pourtant qu’à Rochefort en Charente, il avait jadis fait la trouvaille d’une grisette réalisant son idéal. C’était une jolie fille ayant de jolies mains. Elle médisait et égratignait. Il ne fallait point s’attaquer à elle. Griffes au besoin, et d’une propreté exquise, ses ongles étaient sans reproche et sans peur. Ces charmants ongles avaient enchanté Lethierry, puis l’avaient inquiété ; et, craignant de ne pas être un jour le maître de sa maîtresse, il s’était décidé à ne point mener par-devant monsieur le maire cette amourette.\par
Une autre fois, à Aurigny, une fille lui avait plu. Il songeait aux épousailles, quand un habitant lui dit : \emph{Je vous fais mon compliment. Vous aurez là une bonne bouselière.} Il se fit expliquer l’éloge. A Aurigny, on a une mode. On prend de la bouse de vache et on la jette contre les murs. Il y a une manière de la jeter. Quand elle est sèche, elle tombe, et l’on se chauffe avec cela. On appelle ces bouses sèches des \emph{coipiaux}. On n’épouse une fille que si elle est bonne bouselière. Ce talent mit Lethierry en fuite.\par
Du reste il avait, en matière d’amour, ou d’amourette, une bonne grosse philosophie paysanne, une sagesse de matelot toujours pris, jamais enchaîné, et il se vantait de s’être, dans sa jeunesse, aisément laissé vaincre par le « cotillon ». Ce qu’on nomme aujourd’hui une crinoline, on l’appelait alors un cotillon. Cela signifie plus et moins qu’une femme.\par
Ces rudes marins de l’archipel normand ont de  l’esprit. Presque tous savent lire et lisent. On voit le dimanche de petits mousses de huit ans assis sur un rouleau de cordages un livre à la main. De tout temps ces marins normands ont été sardoniques, et ont, comme on dit aujourd’hui, fait des mots. Ce fut l’un d’eux, le hardi pilote Quéripel, qui jeta à Montgomery, réfugié à Jersey après son malencontreux coup de lance à Henri II, cette apostrophe : \emph{Tête folle a cassé tête vide}. C’est un autre, Touzeau, patron à Saint-Brelade, qui a fait ce calembour philosophique, attribué à tort à l’évêque Camus : \emph{Après la mort les papes deviennent papillons et les sires deviennent cirons.}
 \subsubsection[{A.II.3. La vieille langue de mer}]{A.II.3. \\
La vieille langue de mer}
\noindent Ces marins des Channel Islands sont de vrais vieux gaulois. Ces îles, qui aujourd’hui s’anglaisent rapidement, sont restées longtemps autochthones. Le paysan de Serk parle la langue de Louis XIV.\par
Il y a quarante ans, on retrouvait dans la bouche des matelots de Jersey et d’Aurigny l’idiome marin classique. On se fût cru en pleine marine du dix-septième siècle. Un archéologue spécialiste eût pu venir étudier là l’antique patois de manœuvre et de bataille rugi par Jean Bart dans ce porte-voix qui terrifiait l’amiral Hidde. Le [{\corr vocabulaire}] maritime de nos pères, presque entièrement renouvelé aujourd’hui, était encore usité à Guernesey vers 1820. Un navire qui tient bien le vent était « bon boulinier » ; un navire qui se range au vent presque de lui-même, malgré ses voiles d’avant et son gouvernail, était un « vaisseau ardent ». Entrer en mouvement, c’était « prendre aire » ; mettre à la cape, c’était « capeyer » ; amarrer  le bout d’une manœuvre courante, c’était « faire dormant » ; prendre le vent dessus, c’était « faire chapelle » ; tenir bon sur le câble, c’était « faire teste » ; être en désordre à bord, c’était « être en pantenne » ; avoir le vent dans les voiles, c’était « porter-plain ». Rien de tout cela ne se dit plus. Aujourd’hui on dit : \emph{louvoyer,} alors on disait : \emph{leauvoyer ;} on dit : \emph{naviguer, }on disait : \emph{naviger ;} on dit : \emph{virer vent devant}, on disait : \emph{donner vent devant ;} on dit : \emph{aller de l’avant}, on disait : \emph{tailler de l’avant ;} on dit : \emph{tirez d’accord, }on disait : \emph{halez d’accord ;} on dit : \emph{dérapez,} on disait : \emph{déplantez ;} on dit : \emph{embraquez}, on disait : \emph{abraquez ; }on dit : \emph{taquets}, on disait : \emph{binons ;} on dit : \emph{burins,} on disait : \emph{tappes ;} on dit : \emph{balancines,} on disait : \emph{valancines ;} on dit : \emph{tribord}, on disait : \emph{stribord ;} on dit : \emph{les hommes de quart à bâbord}, on disait : \emph{les basbourdis. }Tourville écrivait à Hocquincourt : \emph{nous avons singlé}. Au lieu de « la rafale », \emph{le raffal ;} au lieu de « bossoir », \emph{boussoir ;} au lieu de « drosse », \emph{drousse ;} au lieu de « loffer », \emph{faire une olofée} ; au lieu de « élonger », \emph{alonger ;} au lieu de « forte brise », \emph{survent ;} au lieu de « jouail », \emph{jas ;} au lieu de « soute », \emph{fosse ;} telle était, au commencement de ce siècle, la langue de bord des îles de la Manche. En entendant parler un pilote jersiais, Ango eût été ému. Tandis que partout les voiles \emph{faseyaient,} aux îles de la Manche elles \emph{barbeyaient.} Une saute-de-vent était une « folle-vente ». On n’employait plus que là les deux modes gothiques d’amarrage, la valture et la portugaise. On n’entendait plus que là les vieux commandements : \emph{Tour-et-choque !}   — \emph{Bosse et bitte !} — Un matelot de Granville disait déjà \emph{le clan,} qu’un matelot de Saint-Aubin ou de Saint-Sampson disait encore \emph{le canal de pouliot}. Ce qui était \emph{bout d’alonge} à Saint-Malo était à Saint-Hélier \emph{oreille d’âne.} Mess Lethierry, absolument comme le duc de Vivonne, appelait la courbure concave des ponts \emph{la tonture} et le ciseau du calfat \emph{la patarasse.} C’est avec ce bizarre idiome entre les dents que Duquesne battit Ruyter, que Duguay-Trouin battit Wasnaer, et que Tourville en 1681 embossa en plein jour la première galère qui bombarda Alger. Aujourd’hui, c’est une langue morte ! L’argot de la mer est actuellement tout autre. Duperré ne comprendrait pas Suffren.\par
La langue des signaux ne s’est pas moins transformée ; et il y a loin des quatre flammes rouge, blanche, bleue et jaune de La Bourdonnais aux dix-huit pavillons d’aujourd’hui qui, arborés deux par deux, trois par trois, et quatre par quatre, offrent aux besoins de la communication lointaine soixante-dix mille combinaisons, ne restent jamais court, et, pour ainsi dire, prévoient l’imprévu.
 \subsubsection[{A.II.4. On est vulnérable dans ce qu’on aime}]{A.II.4. \\
On est vulnérable dans ce qu’on aime}
\noindent Mess Lethierry avait le cœur sur la main ; une large main et un grand cœur. Son défaut, c’était cette admirable qualité, la confiance. Il avait une façon à lui de prendre un engagement ; c’était solennel ; il disait : \emph{J’en donne ma parole d’honneur au bon Dieu}. Cela dit, il allait jusqu’au bout. Il croyait au bon Dieu, pas au reste. Le peu qu’il allait aux églises était politesse. En mer, il était superstitieux.\par
Pourtant jamais un gros temps ne l’avait fait reculer ; cela tenait à ce qu’il était peu accessible à la contradiction. Il ne la tolérait pas plus de l’océan que d’un autre. Il entendait être obéi ; tant pis pour la mer si elle résistait ; il fallait qu’elle en prît son parti. Mess Lethierry ne cédait point. Une vague qui se cabre, pas plus qu’un voisin qui dispute, ne réussissait à l’arrêter. Ce qu’il disait était dit, ce qu’il projetait était fait. Il ne se courbait ni devant une objection, ni devant une tempête. \emph{Non,} pour lui, n’existait pas ; ni dans la  bouche d’un homme, ni dans le grondement d’un nuage. Il passait outre. Il ne permettait point qu’on le refusât. De là son entêtement dans la vie et son intrépidité sur l’océan.\par
Il assaisonnait volontiers lui-même sa soupe au poisson, sachant la dose de poivre et de sel et les herbes qu’il fallait, et se régalait autant de la faire que de la manger. Un être qu’un suroît transfigure et qu’une redingote abrutit, qui ressemble, les cheveux au vent, à Jean Bart, et, en chapeau rond, à Jocrisse, gauche à la ville, étrange et redoutable à la mer, un dos de portefaix, point de jurons, très rarement de la colère, un petit accent très doux qui devient tonnerre dans un porte-voix, un paysan qui a lu l’Encyclopédie, un guernesiais qui a vu la révolution, un ignorant très savant, aucune bigoterie, mais toutes sortes de visions, plus de foi à la Dame blanche qu’à la sainte Vierge, la forme de Polyphème, la logique de la girouette, la volonté de Christophe Colomb, quelque chose d’un taureau et quelque chose d’un enfant, un nez presque camard, des joues puissantes, une bouche qui a toutes ses dents, un froncement partout sur la figure, une face qui semble avoir été tripotée par la vague et sur laquelle la rose des vents a tourné pendant quarante ans, un air d’orage sur le front, une carnation de roche en pleine mer ; maintenant mettez dans ce visage dur un regard bon, vous aurez mess Lethierry.\par
Mess Lethierry avait deux amours : Durande et Déruchette.\par
  \subsection[{A.III. Livre troisième. Durande et Déruchette}]{A.III. Livre troisième \\
Durande et Déruchette}
  \subsubsection[{A.III.1. Babil et fumée}]{A.III.1. \\
Babil et fumée}
\noindent Le corps humain pourrait bien n’être qu’une apparence. Il cache notre réalité. Il s’épaissit sur notre lumière ou sur notre ombre. La réalité, c’est l’âme. A parler absolument, notre visage est un masque. Le vrai homme, c’est ce qui est sous l’homme. Si l’on apercevait cet homme-là, tapi et abrité derrière cette illusion qu’on nomme la chair, on aurait plus d’une surprise. L’erreur commune, c’est de prendre l’être extérieur pour l’être réel. Telle fille, par exemple, si on la voyait ce qu’elle est, apparaîtrait oiseau.\par
Un oiseau qui a la forme d’une fille, quoi de plus exquis ! Figurez-vous que vous l’avez chez vous. Ce sera Déruchette. Le délicieux être ! On serait tenté de lui dire : Bonjour, mademoiselle la bergeronnette. On  ne voit pas les ailes, mais on entend le gazouillement. Par instants, elle chante. Par le babil, c’est au-dessous de l’homme ; par le chant, c’est au-dessus. Il y a le mystère dans ce chant ; une vierge est une enveloppe d’ange. Quand la femme se fait, l’ange s’en va ; mais plus tard, il revient, apportant une petite âme à la mère. En attendant la vie, celle qui sera mère un jour est très longtemps un enfant, la petite fille persiste dans la jeune fille, et c’est une fauvette. On pense en la voyant : qu’elle est aimable de ne pas s’envoler ! Le doux être familier prend ses aises dans la maison, de branche en branche, c’est-à-dire de chambre en chambre, entre, sort, s’approche, s’éloigne, lisse ses plumes ou peigne ses cheveux, fait toutes sortes de petits bruits délicats, murmure on ne sait quoi d’ineffable à vos oreilles. Il questionne, on lui répond ; on l’interroge, il gazouille. On jase avec lui. Jaser, cela délasse de parler. Cet être a du ciel en lui. C’est une pensée bleue mêlée à votre pensée noire. Vous lui savez gré d’être si léger, si fuyant, si échappant, si peu saisissable, et d’avoir la bonté de ne pas être invisible, lui qui pourrait, ce semble, être impalpable. Ici-bas, le joli, c’est le nécessaire. Il y a sur la terre peu de fonctions plus importantes que celle-ci ; être charmant. La forêt serait au désespoir sans le colibri. Dégager de la joie, rayonner du bonheur, avoir parmi les choses sombres une exsudation de lumière, être la dorure du destin, être l’harmonie, être la grâce, être la gentillesse, c’est vous rendre service. La beauté me fait du bien en étant belle. Telle créature a cette  féerie d’être pour tout ce qui l’entoure un enchantement ; quelquefois elle n’en sait rien elle-même, ce n’en est que plus souverain ; sa présence éclaire, son approche réchauffe ; elle passe, on est content ; elle s’arrête, on est heureux ; la regarder, c’est vivre ; elle est de l’aurore ayant la figure humaine ; elle ne fait pas autre chose que d’être là, cela suffit, elle édénise la maison, il lui sort par tous les pores un paradis ; cette extase, elle la distribue à tous sans se donner d’autre peine que de respirer à côté d’eux. Avoir un sourire qui, on ne sait comment, diminue le poids de la chaîne énorme traînée en commun par tous les vivants, que voulez-vous que je vous dise, c’est divin. Ce sourire, Déruchette l’avait. Disons plus, Déruchette était ce sourire. Il y a quelque chose qui nous ressemble plus que notre visage, c’est notre physionomie ; et il y a quelque chose qui nous ressemble plus que notre physionomie, c’est notre sourire. Déruchette souriant, c’était Déruchette.\par
C’est un sang particulièrement attrayant que celui de Jersey et de Guernesey. Les femmes, les filles surtout, sont d’une beauté fleurie et candide. C’est la blancheur saxonne et la fraîcheur normande combinées. Des joues roses et des regards bleus. Il manque à ces regards l’étoile. L’éducation anglaise les amortit. Ces yeux limpides seront irrésistibles le jour où la profondeur parisienne y apparaîtra. Paris, heureusement, n’a pas encore fait son entrée dans les anglaises. Déruchette n’était pas une parisienne, mais n’était pas non plus une guernesiaise. Elle était née à Saint- Pierre-Port, mais mess Lethierry l’avait élevée. Il l’avait élevée pour être mignonne ; elle l’était.\par
Déruchette avait le regard indolent, et agressif sans le savoir. Elle ne connaissait peut-être pas le sens du mot amour, et elle rendait volontiers les gens amoureux d’elle. Mais sans mauvaise intention. Elle ne songeait à aucun mariage. Le vieux gentilhomme émigré qui avait pris racine à Saint-Sampson, disait : \emph{Cette petite fait de la flirtation à poudre}.\par
Déruchette avait les plus jolies petites mains du monde et des pieds assortis aux mains, \emph{quatre pattes de mouche}, disait mess Lethierry. Elle avait dans toute sa personne la bonté et la douceur, pour famille et pour richesse mess Lethierry, son oncle, pour travail de se laisser vivre, pour talent quelques chansons, pour science la beauté, pour esprit l’innocence, pour cœur l’ignorance ; elle avait la gracieuse paresse créole, mêlée d’étourderie et de vivacité, la gaîté taquine d\emph{e }l’enfance avec une pente à la mélancolie, des toilettes un peu insulaires, élégantes mais incorrectes, des chapeaux de fleurs toute l’année, le front naïf, le cou souple et tentant, les cheveux châtains, la peau blanche avec quelques taches de rousseur l’été, la bouche grande et saine, et sur cette bouche l’adorable et dangereuse clarté du sourire. C’était là Déruchette.\par
Quelquefois, le soir, après le soleil couché, au moment où la nuit se mêle à la mer, à l’heure où le crépuscule donne une sorte d’épouvante aux vagues, on voyait entrer dans le goulet de Saint-Sampson, sur  le soulèvement sinistre des flots, on ne sait quelle masse informe, une silhouette monstrueuse qui sifflait et crachait, une chose horrible qui râlait comme une bête et qui fumait comme un volcan, une espèce d’hydre bavant dans l’écume et traînant un brouillard, et se ruant vers la ville avec un effrayant battement de nageoires et une gueule d’où sortait de la flamme. C’était Durande.
 \subsubsection[{A.III.2. Histoire éternelle de l’utopie}]{A.III.2. \\
Histoire éternelle de l’utopie}
\noindent C’était une prodigieuse nouveauté qu’un bateau à vapeur dans les eaux de la Manche en 182.. Toute la côte normande en fut longtemps effarée. Aujourd’hui dix ou douze steamers se croisant en sens inverse sur un horizon de mer ne font lever les yeux à personne ; tout au plus occupent-ils un moment le connaisseur spécial qui distingue à la couleur de leur fumée que celui-ci brûle du charbon de Wales et celui-là du charbon de Newcastle. Ils passent, c’est bien. Wellcome, s’ils arrivent. Bon voyage, s’ils partent.\par
On était moins calme à l’endroit de ces inventions-là dans le premier quart de ce siècle, et ces mécaniques et leur fumée étaient particulièrement mal vues chez les insulaires de la Manche. Dans cet archipel puritain, où la reine d’Angleterre a été blâmée de violer la bible\footnote{ \noindent \emph{Genèse}, chap. {\scshape iii}, vers 16 : Tu enfanteras avec douleur.
 } en accouchant par le chloroforme, le bateau à  vapeur eut pour premier succès d’être baptisé \emph{le Bateau-Diable} (Devil-Boat). A ces bons pêcheurs d’alors, jadis catholiques, désormais calvinistes, toujours bigots, cela sembla être de l’enfer qui flottait. Un prédicateur local traita cette question : \emph{A-t-on le droit de faire travailler ensemble l’eau et le feu que Dieu a séparés\footnote{ \noindent \emph{Genèse}, chap. {\scshape i}, vers 4.
 } ?} Cette bête de feu et de fer ne ressemblait-elle pas à Léviathan ? N’était-ce pas refaire dans la mesure humaine le chaos ? Ce n’est pas la première fois que l’ascension du progrès est qualifiée retour au chaos.\par
\emph{Idée folle, erreur grossière, absurdité ;} tel avait été le verdict de l’académie des sciences consultée, au commencement de ce siècle, sur le bateau à vapeur par Napoléon ; les pêcheurs de Saint-Sampson sont excusables de n’être, en matière scientifique, qu’au niveau des géomètres de Paris, et, en matière religieuse, une petite île comme Guernesey n’est pas forcée d’avoir plus de lumières qu’un grand continent comme l’Amérique. En 1807, quand le premier bateau de Fulton, patronné par Livingstone, pourvu de la machine de Watt envoyée d’Angleterre, et monté, outre l’équipage, par deux français seulement, André Michaux et un autre, quand ce premier bateau à vapeur fit son premier voyage de New-York à Albany, le hasard fit que ce fut le 17 août. Sur ce, le méthodisme prit la parole, et dans toutes les chapelles les prédicateurs maudirent cette machine, déclarant que ce nombre \emph{dix-sept} était le total des dix antennes et des sept têtes  de la bête de l’Apocalypse. En Amérique on invoquait contre le navire à vapeur la bête de l’Apocalypse et en Europe la bête de la Genèse. Là était toute la différence.\par
Les savants avaient rejeté le bateau à vapeur comme impossible ; les prêtres à leur tour le rejetaient comme impie. La science avait condamné, la religion damnait. Fulton était une variété de Lucifer. Les gens simples des côtes et des campagnes adhéraient à la réprobation par le malaise que leur donnait cette nouveauté. En présence du bateau à vapeur, le point de vue religieux était ceci : — L’eau et le feu sont un divorce. Ce divorce est ordonné de Dieu. On ne doit pas désunir ce que Dieu a uni ; on ne doit pas unir ce qu’il a désuni. — Le point de vue paysan était ceci : ça me fait peur.\par
Pour oser à cette époque lointaine une telle entreprise, un bateau à vapeur de Guernesey à Saint-Malo, il ne fallait rien moins que mess Lethierry. Lui seul pouvait la concevoir comme libre penseur, et la réaliser comme hardi marin. Son côté français eut l’idée, son côté anglais l’exécuta.\par
A quelle occasion ? disons-le.
 \subsubsection[{A.III.3. Rantaine}]{A.III.3. \\
Rantaine}
\noindent Quarante ans environ avant l’époque où se passent les faits que nous racontons ici, il y avait dans la banlieue de Paris, près du mur de ronde, entre la Fosse-aux-Lions et la Tombe-Issoire, un logis suspect. C’était une masure isolée, coupe-gorge au besoin. Là demeurait avec sa femme et son enfant une espèce de bourgeois bandit, ancien clerc de procureur au Châtelet, devenu voleur tout net. Il figura plus tard en cour d’assises. Cette famille s’appelait les Rantaine. On voyait dans la masure sur une commode d’acajou deux tasses en porcelaine fleurie ; on lisait en lettres dorées sur l’une : \emph{Souvenir d’amitié,} et sur l’autre : \emph{Don d’estime.} L’enfant était dans le bouge pêle-mêle avec le crime. Le père et la mère ayant été de la demi-bourgeoisie, l’enfant apprenait à lire ; on l’élevait. La mère, pâle, presque en guenilles, donnait machinalement « de l’éducation » à son petit, le faisait épeler, et s’interrompait pour aider son mari à quelque guet-apens, ou pour se prostituer à un passant. Pendant  ce temps-là, la Croix de Jésus, ouverte à l’endroit où on l’avait quittée, restait sur la table, et l’enfant auprès, rêveur.\par
Le père et la mère, saisis dans quelque flagrant délit, disparurent dans la nuit pénale. L’enfant disparut aussi.\par
Lethierry dans ses courses rencontra un aventurier comme lui, le tira d’on ne sait quel mauvais pas, lui rendit service, lui en fut reconnaissant, le prit en gré, le ramassa, l’amena à Guernesey, le trouva intelligent au cabotage, en fit son associé. C’était le petit Rantaine devenu grand.\par
Rantaine, comme Lethierry, avait une nuque robuste, une large et puissante marge à porter des fardeaux entre les deux épaules, et des reins d’Hercule Farnèse. Lethierry et lui, c’était la même allure et la même encolure ; Rantaine était de plus haute taille. Qui les voyait de dos se promener côte à côte sur le port, disait : Voilà les deux frères. De face, c’était autre chose. Tout ce qui était ouvert chez Lethierry était fermé chez Rantaine. Rantaine était circonspect. Rantaine était maître d’armes, jouait de l’harmonica, mouchait une chandelle d’une balle à vingt pas, avait un coup de poing magnifique, récitait des vers de la Henriade et devinait les songes. Il savait par cœur \emph{les Tombeaux de Saint-Denis,} par Treneuil. Il disait avoir été lié avec le sultan de Calicut \emph{que les portugais appellent le zamorin.} Si l’on eût pu feuilleter le petit agenda qu’il avait sur lui, on y eût trouvé, entre autres notes, des mentions du genre de celle-ci : « A Lyon,  dans une des fissures du mur d’un des cachots de Saint-Joseph, il y a une lime cachée. » Il parlait avec une sage lenteur. Il se disait fils d’un chevalier de Saint-Louis. Son linge était dépareillé et marqué à des lettres différentes. Personne n’était plus chatouilleux que lui sur le point d’honneur ; il se battait et tuait. Il avait dans le regard quelque chose d’une mère d’actrice.\par
La force servant d’enveloppe à la ruse, c’était là Rantaine.\par
La beauté de son coup de poing, appliquée dans une foire sur une \emph{cabeza de moro,} avait gagné jadis le cœur de Lethierry.\par
On ignorait pleinement à Guernesey ses aventures. Elles étaient bigarrées. Si les destinées ont un vestiaire, la destinée de Rantaine devait être vêtue en arlequin. Il avait vu le monde et fait la vie. C’était un circumnavigateur. Ses métiers étaient une gamme. Il avait été cuisinier à Madagascar, éleveur d’oiseaux à Sumatra, général à Honolulu, journaliste religieux aux îles Gallapagos, poète à Oomrawuttee, franc-maçon à Haïti. Il avait prononcé en cette dernière qualité au Grand-Goâve une oraison funèbre dont les journaux locaux ont conservé ce fragment : « ... Adieu donc, belle âme ! Dans la voûte azurée des cieux où tu prends maintenant ton vol, tu rencontreras sans doute le bon abbé Léandre Crameau du Petit-Goâve. Dis-lui que, grâce à dix années d’efforts glorieux, tu as terminé l’église de l’Anse-à-Veau ! Adieu, génie transcendant, maçon modèle ! » Son masque de franc-maçon ne  l’empêchait pas, comme on voit, de porter le faux nez catholique. Le premier lui conciliait les hommes de progrès et le second les hommes d’ordre. Il se déclarait blanc pur sang, il haïssait les noirs ; pourtant il eût certainement admiré Soulouque. A Bordeaux, en 1815, il avait été verdet. A cette époque, la fumée de son royalisme lui sortait du front sous la forme d’un immense plumet blanc. Il avait passé sa vie à faire des éclipses, paraissant, disparaissant, reparaissant. C’était un coquin à feu tournant. Il savait du turc ; au lieu de \emph{guillotiné} il disait \emph{néboissé.} Il avait été esclave en Tripoli chez un thaleb, et il y avait appris le turc à coups de bâton ; sa fonction avait été d’aller le soir aux portes des mosquées et d’y lire à haute voix devant les fidèles le koran écrit sur des planchettes de bois ou sur des omoplates de chameau. Il était probablement renégat.\par
Il était capable de tout, et de pire.\par
Il éclatait de rire et fronçait le sourcil en même temps. Il disait : \emph{En politique, je n’estime que les gens inaccessibles aux influences}. Il disait : \emph{Je suis pour les mœurs.} Il était plutôt gai et cordial qu’autre chose. La forme de sa bouche démentait le sens de ses paroles. Ses narines eussent pu passer pour des naseaux. Il avait au coin de l’œil un carrefour de rides où toutes sortes de pensées obscures se donnaient rendez-vous. Le secret de sa physionomie ne pouvait être déchiffré que là. Sa patte d’oie était une serre de vautour. Son crâne était bas au sommet et large aux tempes. Son oreille difforme et encombrée de broussailles,  semblait dire : ne parlez pas à la bête qui est dans cet antre.\par
Un beau jour, à Guernesey, on ne sut plus où était Rantaine.\par
L’associé de Lethierry avait « filé », laissant vide la caisse de l’association.\par
Dans cette caisse il y avait de l’argent à Rantaine sans doute, mais il y avait aussi cinquante mille francs à Lethierry.\par
Lethierry, dans son métier de caboteur et de charpentier de navires, avait, en quarante ans d’industrie et de probité, gagné cent mille francs. Rantaine lui en emporta la moitié.\par
Lethierry, à moitié ruiné, ne fléchit pas et songea immédiatement à se relever. On ruine la fortune des gens de cœur, non leur courage. On commençait alors à parler du bateau à vapeur. L’idée vint à Lethierry d’essayer la machine Fulton, si contestée, et de relier par un bateau à feu l’archipel normand à la France. Il joua son vatout sur cette idée. Il y consacra son reste. Six mois après la fuite de Rantaine, on vit sortir du port stupéfait de Saint-Sampson un navire à fumée, faisant l’effet d’un incendie en mer, le premier steamer qui ait navigué dans la Manche.\par
Ce bateau, que la haine et le dédain de tous gratifièrent immédiatement du sobriquet « la Galiote à Lethierry », s’annonça comme devant faire le service régulier de Guernesey à Saint-Malo.
 \subsubsection[{A.III.4. Suite de l’histoire de l’utopie}]{A.III.4. \\
Suite de l’histoire de l’utopie}
\noindent La chose, on le comprend de reste, prit d’abord fort mal. Tous les propriétaires de coutres faisant le voyage de l’île guernesiaise à la côte française jetèrent les hauts cris. Ils dénoncèrent cet attentat à l’écriture sainte et à leur monopole. Quelques chapelles fulminèrent. Un révérend, nommé Elihu, qualifia le bateau à vapeur « un libertinage ». Le navire à voiles fut déclaré orthodoxe. On vit distinctement les cornes du diable sur la tête des bœufs que le bateau à vapeur apportait et débarquait. Cette protestation dura un temps raisonnable. Cependant peu à peu on finit par s’apercevoir que ces bœufs arrivaient moins fatigués, et se vendaient mieux, la viande étant meilleure ; que les risques de mer étaient moindres pour les hommes aussi ; que ce passage, moins coûteux, était plus sûr et plus court ; qu’on partait à heure fixe et qu’on arrivait à heure fixe ; que le poisson, voyageant plus vite, était plus frais, et qu’on pouvait désormais déverser sur les marchés français l’excédant des grandes pêches, si fréquentes à Guernesey ; que le beurre des admirables  vaches de Guernesey faisait plus rapidement le trajet dans le Devil-Boat que dans les sloops à voile, et ne perdait plus rien de sa qualité, de sorte que Dinan en demandait, et que Saint-Brieuc en demandait, et que Rennes en demandait ; qu’enfin il y avait, grâce à ce qu’on appelait \emph{la Galiote à Lethierry}, sécurité de voyage, régularité de communication, va-et-vient facile et prompt, agrandissement de circulation, multiplication de débouchés, extension de commerce, et qu’en somme il fallait prendre son parti de ce Devil-Boat qui violait la bible et enrichissait l’île. Quelques esprits forts se hasardèrent à approuver dans une certaine mesure. Sieur Landoys, le greffier, accorda son estime à ce bateau. Du reste, ce fut impartialité de sa part, car il n’aimait pas Lethierry. D’abord Lethierry était mess et Landoys n’était que sieur. Ensuite, quoique greffier à Saint-Pierre-Port, Landoys était paroissien de Saint-Sampson ; or ils n’étaient dans la paroisse que deux hommes, Lethierry et lui, n’ayant point de préjugés ; c’était bien le moins que l’un détestât l’autre. Être du même bord, cela éloigne.\par
Sieur Landoys néanmoins eut l’honnêteté d’approuver le bateau à vapeur. D’autres se joignirent à sieur Landoys. Insensiblement, le fait monta ; les faits sont une marée ; et, avec le temps, avec le succès continu et croissant, avec l’évidence du service rendu, l’augmentation du bien-être de tous étant constatée, il vint un jour où, quelques sages exceptés, tout le monde admira « la Galiote à Lethierry ».\par
On l’admirerait moins aujourd’hui. Ce steamer d’il  y a quarante ans ferait sourire nos constructeurs actuels. Cette merveille était difforme ; ce prodige était infirme.\par
De nos grands steamers transatlantiques d’à présent au bateau à roues et à feu que Denis Papin fit manœuvrer sur la Fulde en 1707, il n’y a pas moins de distance que du vaisseau à trois ponts le \emph{Montebello}, long de deux cents pieds, large de cinquante, ayant une grande vergue de cent quinze pieds, déplaçant un poids de trois mille tonneaux, portant onze cents hommes, cent vingt canons, dix mille boulets et cent soixante paquets de mitraille, vomissant à chaque bordée, quand il combat, trois mille trois cents livres de fer, et déployant au vent, quand il marche, cinq mille six cents mètres carrés de toile, au dromon danois du deuxième siècle, trouvé plein de haches de pierre, d’arcs et de massues, dans les boues marines de Wester-Satrup, et déposé à l’hôtel de ville de Flensbourg.\par
Cent ans juste d’intervalle, 1707-1807, séparent le premier bateau de Papin du premier bateau de Fulton. « La Galiote à Lethierry » était, à coup sûr, un progrès sur ces deux ébauches, mais était une ébauche elle-même. Cela ne l’empêchait pas d’être un chef-d’œuvre. Tout embryon de la science offre ce double aspect : monstre comme fœtus ; merveille comme germe.
 \subsubsection[{A.III.5. Le bateau-diable}]{A.III.5. \\
Le bateau-diable}
\noindent « La Galiote à Lethierry » n’était pas mâtée selon le point vélique, et ce n’était pas là son défaut, car c’est une des lois de la construction navale ; d’ailleurs, le navire ayant pour propulseur le feu, la voilure était l’accessoire. Ajoutons qu’un navire à roues est presque insensible à la voilure qu’on lui met. La Galiote était trop courte, trop ronde, trop ramassée ; elle avait trop de joue et trop de hanche ; la hardiesse n’avait pas été jusqu’à la faire légère ; la Galiote avait quelques-uns des inconvénients et quelques-unes des qualités de la Panse. Elle tanguait peu, mais roulait beaucoup. Les tambours étaient trop hauts. Elle avait trop de bau pour sa longueur. La machine, massive, l’encombrait, et, pour rendre le navire capable d’une forte cargaison, on avait dû hausser démesurément la muraille, ce qui donnait à la Galiote à peu près le défaut des vaisseaux de soixante-quatorze, qui sont un gabarit bâtard, et qu’il faut raser pour les rendre battants et  marins. Étant courte, elle eût dû virer vite, les temps employés à une évolution étant comme les longueurs des navires ; mais sa pesanteur lui ôtait l’avantage que lui donnait sa brièveté. Son maître-couple était trop large, ce qui la ralentissait, la résistance de l’eau étant proportionnelle à la plus grande section immergée et au carré de la vitesse du navire. L’avant était vertical, ce qui ne serait pas une faute aujourd’hui, mais en ce temps-là l’usage invariable était de l’incliner de quarante-cinq degrés. Toutes les courbes de la coque étaient bien raccordées, mais pas assez longues pour l’obliquité et surtout pour le parallélisme avec le prisme d’eau déplacé, lequel ne doit jamais être refoulé que latéralement. Dans les gros temps, elle tirait trop d’eau, tantôt par l’avant, tantôt par l’arrière, ce qui indiquait un vice dans le centre de gravité. La charge n’étant pas où elle devait être, à cause du poids de la machine, le centre de gravité passait souvent à l’arrière du grand mât, et alors il fallait s’en tenir à la vapeur, et se défier de la grande voile, car l’effet de la grande voile dans ce cas-là faisait arriver le vaisseau au lieu de le soutenir au vent. La ressource était, quand on était au plus près du vent, de larguer en bande la grande écoute ; le vent, de la sorte, était fixé sur l’avant par l’amure, et la grande voile ne faisait plus l’effet d’une voile de poupe. Cette manœuvre était difficile. Le gouvernail était l’antique gouvernail, non à roue comme aujourd’hui, mais à barre, tournant sur ses gonds scellés dans l’étambot et mû par une solive horizontale passant par-dessus la barre d’arcasse. Deux  canots, espèces de youyous, étaient suspendus aux pistolets. Le navire avait quatre ancres, la grosse ancre, la seconde ancre qui est l’ancre travailleuse, \emph{working-anchor}, et deux ancres d’affourche. Ces quatre ancres, mouillées avec des chaînes, étaient manœuvrées, selon les occasions, par le grand cabestan de poupe et le petit cabestan de proue. A cette époque, le guindoir à pompe n’avait pas encore remplacé l’effort intermittent de la barre d’anspec. N’ayant que deux ancres d’affourche, l’une à tribord, l’autre à bâbord, le navire ne pouvait affourcher en patte d’oie, ce qui le désarmait un peu devant certains vents. Pourtant il pouvait en ce cas s’aider de la seconde ancre. Les bouées étaient normales, et construites de manière à porter le poids de l’orin des ancres, tout en restant à flot. La chaloupe avait la dimension utile. C’était le véritable en-cas du bâtiment ; elle était assez forte pour lever la maîtresse ancre. Une nouveauté de ce navire, c’est qu’il était en partie gréé avec des chaînes, ce qui du reste n’ôtait rien de leur mobilité aux manœuvres courantes et de leur tension aux manœuvres dormantes. La mâture, quoique secondaire, n’avait aucune incorrection ; le capelage, bien serré, bien dégagé, paraissait peu. Les membrures étaient solides, mais grossières, la vapeur n’exigeant point la même délicatesse de bois que la voile. Ce navire marchait avec une vitesse de deux lieues à l’heure. En panne il faisait bien son abattée. Telle qu’elle était, « la Galiote à Lethierry » tenait bien la mer, mais elle manquait de pointe pour diviser le liquide, et l’on ne pouvait dire qu’elle eût de belles  façons. On sentait que dans un danger, écueil ou trombe, elle serait peu maniable. Elle avait le craquement d’une chose informe. Elle faisait, en roulant sur la vague, un bruit de semelle neuve.\par
Ce navire était surtout un récipient, et, comme tout bâtiment plutôt armé en marchandise qu’en guerre, il était exclusivement disposé pour l’arrimage. Il admettait peu de passagers. Le transport du bétail rendait l’arrimage difficile et très particulier. On arrimait alors les bœufs dans la cale, ce qui était une complication. Aujourd’hui on les arrime sur l’avant-pont. Les tambours du Devil-Boat Lethierry étaient peints en blanc, la coque, jusqu’à la ligne de flottaison, en couleur de feu, et tout le reste du navire, selon la mode assez laide de ce siècle, en noir.\par
Vide, il calait sept pieds, et, chargé, quatorze.\par
Quant à la machine, elle était puissante. La force était d’un cheval pour trois tonneaux, ce qui est presque une force de remorqueur. Les roues étaient bien placées, un peu en avant du centre de gravité du navire. La machine avait une pression maximum de deux atmosphères. Elle usait beaucoup de charbon, quoiqu’elle fût à condensation et à détente. Elle n’avait pas de volant à cause de l’instabilité du point d’appui, et elle y remédiait, comme on le fait encore aujourd’hui, par un double appareil faisant alterner deux manivelles fixées aux extrémités de l’arbre de rotation et disposées de manière que l’une fût toujours à son point fort quand l’autre était à son point mort. Toute la machine reposait sur une seule plaque de fonte ; de  sorte que, même dans un cas de grave avarie, aucun coup de mer ne lui ôtait l’équilibre et que la coque déformée ne pouvait déformer la machine. Pour rendre la machine plus solide encore, on avait placé la bielle principale près du cylindre, ce qui transportait du milieu à l’extrémité le centre d’oscillation du balancier. Depuis on a inventé les cylindres oscillants qui permettent de supprimer les bielles ; mais, à cette époque, la bielle près du cylindre semblait le dernier mot de la machinerie. La chaudière était coupée de cloisons et pourvue de sa pompe de saumure. Les roues étaient très grandes, ce qui diminuait la perte de force, et la cheminée était très haute, ce qui augmentait le tirage du foyer ; mais la grandeur des roues donnait prise au flot et la hauteur de la cheminée donnait prise au vent. Aubes de bois, crochets de fer, moyeux de fonte, telles étaient les roues, bien construites et, chose qui étonne, pouvant se démonter. Il y avait toujours trois aubes immergées. La vitesse du centre des aubes ne surpassait que d’un sixième la vitesse du navire ; c’était là le défaut de ces roues. En outre, le manneton des manivelles était trop long, et le tiroir distribuait la vapeur dans le cylindre avec trop de frottement. Dans ce temps-là, cette machine semblait et était admirable.\par
Cette machine avait été forgée en France à l’usine de fer de Bercy. Mess Lethierry l’avait un peu imaginée ; le mécanicien qui l’avait construite sur son épure était mort ; de sorte que cette machine était unique, et impossible à remplacer. Le dessinateur restait, mais le constructeur manquait.\par
 La machine avait coûté quarante mille francs.\par
Lethierry avait construit lui-même la Galiote sous la grande cale couverte qui est à côté de la première tour entre Saint-Pierre-Port et Saint-Sampson. Il avait été à Brême acheter le bois. Il avait épuisé dans cette construction tout son savoir-faire de charpentier de marine, et l’on reconnaissait son talent au bordage dont les coutures étaient étroites et égales, et recouvertes de sarangousti, mastic de l’Inde meilleur que le brai. Le doublage était bien mailleté. Lethierry avait enduit la carène de gallegalle. Il avait, pour remédier à la rondeur de la coque, ajusté un boute-hors au beaupré, ce qui lui permettait d’ajouter à la civadière une fausse civadière. Le jour du lancement, il avait dit : me voilà à flot ! La Galiote réussit en effet, on l’a vu.\par
Par hasard ou exprès, elle avait été lancée un 14 juillet. Ce jour-là, Lethierry, debout sur le pont entre les deux tambours, regarda fixement la mer et lui cria : — C’est ton tour ! les parisiens ont pris la Bastille ; maintenant nous te prenons, toi !\par
La Galiote à Lethierry faisait une fois par semaine le voyage de Guernesey à Saint-Malo. Elle partait le mardi matin et revenait le vendredi soir, veille du marché qui est le samedi. Elle était d’un plus fort échantillon de bois que les plus grands sloops caboteurs de tout l’archipel, et, sa capacité étant en raison de sa dimension, un seul de ses voyages valait, pour l’apport et pour le rendement, quatre voyages d’un coutre ordinaire. De là de forts bénéfices. La réputation d’un navire dépend de son arrimage, et Lethierry était un  arrimeur. Quand il ne put plus travailler en mer lui-même, il dressa un matelot pour le remplacer comme arrimeur. Au bout de deux années, le bateau à vapeur rapportait net sept cent cinquante livres sterling par an, c’est-à-dire dix-huit mille francs. La livre sterling de Guernesey vaut vingt-quatre francs, celle d’Angleterre vingt-cinq et celle de Jersey vingt-six. Ces chinoiseries sont moins chinoises qu’elles n’en ont l’air ; les banques y trouvent leur compte.
 \subsubsection[{A.III.6. Entrée de lethierry dans la gloire}]{A.III.6. \\
Entrée de lethierry dans la gloire}
\noindent « La Galiote » prospérait. Mess Lethierry voyait s’approcher le moment où il deviendrait monsieur. A Guernesey on n’est pas de plain-pied monsieur. Entre l’homme et le monsieur il y a toute une échelle à gravir ; d’abord, premier échelon, le nom tout sec, Pierre, je suppose ; puis, deuxième échelon, vésin (voisin) Pierre ; puis, troisième échelon, père Pierre ; puis, quatrième échelon, sieur Pierre ; puis, cinquième échelon, mess Pierre ; puis, sommet, monsieur Pierre.\par
Cette échelle, qui sort de terre, se continue dans le bleu. Toute la hiérarchique Angleterre y entre et s’y étage. En voici les échelons, de plus en plus lumineux : au-dessus du monsieur (\emph{gentleman}), il y a l’esq. (écuyer), au-dessus de l’esq., le chevalier (\emph{sir} viager), puis, en s’élevant toujours, le baronet (\emph{sir} héréditaire), puis le lord, \emph{laird} en Écosse, puis le baron, puis le vicomte, puis le comte (\emph{earl} en Angleterre, \emph{jarl} en Norvège), puis le marquis, puis le duc, puis le pair  d’Angleterre, puis le prince du sang royal, puis le roi. Cette échelle monte du peuple à la bourgeoisie, de la bourgeoisie au baronetage, du baronetage à la pairie, de la pairie à la royauté.\par
Grâce à son coup de tête réussi, grâce à la vapeur, grâce à sa machine, grâce au Bateau-Diable, mess Lethierry était devenu quelqu’un. Pour construire « la Galiote », il avait dû emprunter ; il s’était endetté à Brême, il s’était endetté à Saint-Malo ; mais chaque année il amortissait son passif.\par
Il avait de plus acheté à crédit, à l’entrée même du port de Saint-Sampson, une jolie maison de pierre, toute neuve, entre mer et jardin, sur l’encoignure de laquelle on lisait ce nom : \emph{les Bravées.} Le logis les Bravées, dont la devanture faisait partie de la muraille même du port, était remarquable par une double rangée de fenêtres, au nord du côté d’un enclos plein de fleurs, au sud du côté de l’océan ; de sorte que cette maison avait deux façades, l’une sur les tempêtes, l’autre sur les roses.\par
Ces façades semblaient faites pour les deux habitants, mess Lethierry et miss Déruchette.\par
La maison des Bravées était populaire à Saint-Sampson. Car mess Lethierry avait fini par être populaire. Cette popularité lui venait un peu de sa bonté, de son dévouement et de son courage, un peu de la quantité d’hommes qu’il avait sauvés, beaucoup de son succès, et aussi de ce qu’il avait donné au port de Saint-Sampson le privilège des départs et des arrivées du bateau à vapeur. Voyant que décidément le Devil- Boat était une bonne affaire, Saint-Pierre, la capitale, l’avait réclamé pour son port, mais Lethierry avait tenu bon pour Saint-Sampson. C’était sa ville natale. — C’est là que j’ai été lancé à la mer, disait-il. — De là une vive popularité locale. Sa qualité de propriétaire payant taxe faisait de lui ce qu’on appelle à Guernesey un \emph{habitant.} On l’avait nommé douzenier. Ce pauvre matelot avait franchi cinq échelons sur six de l’ordre social guernesiais ; il était mess ; il touchait au monsieur, et qui sait s’il n’arriverait pas même à franchir le monsieur ? Qui sait si un jour on ne lirait pas dans l’almanach de Guernesey au chapitre \emph{Gentry and Nobility} cette inscription inouïe et superbe : \emph{Lethierry esq. ?}\par
Mais mess Lethierry dédaignait ou plutôt ignorait le côté par lequel les choses sont vanité. Il se sentait utile, c’était là sa joie. Être populaire le touchait moins qu’être nécessaire. Il n’avait, nous l’avons dit, que deux amours, et par conséquent, que deux ambitions, Durande et Déruchette.\par
Quoi qu’il en fût, il avait mis à la loterie de la mer, et il y avait gagné le quine.\par
Le quine, c’était la Durande naviguant.
 \subsubsection[{A.III.7. Le même parrain et la même patronne}]{A.III.7. \\
Le même parrain et la même patronne}
\noindent Après avoir créé ce bateau à vapeur, Lethierry l’avait baptisé. Il l’avait nommé \emph{Durande.} La Durande, — nous ne l’appellerons plus autrement. On nous permettra également, quel que soit l’usage typographique, de ne point souligner ce nom \emph{Durande}, nous conformant en cela à la pensée de mess Lethierry pour qui la Durande était presque une personne.\par
Durande et Déruchette, c’est le même nom. Déruchette est le diminutif. Ce diminutif est fort usité dans l’ouest de la France.\par
Les saints dans les campagnes portent souvent leur nom avec tous ses diminutifs et tous ses augmentatifs. On croirait à plusieurs personnes là où il n’y en a qu’une. Ces identités de patrons et de patronnes sous des noms différents ne sont point chose rare. Lise, Lisette, Lisa, Élisa, Isabelle, Lisbeth, Betsy, cette multitude est Élisabeth. Il est probable que Mahout, Maclou, Malo et Magloire sont le même saint. Du reste, nous n’y tenons pas.\par
 Sainte Durande est une sainte de l’Angoumois et de la Charente. Est-elle correcte ? Ceci regarde les bollandistes. Correcte ou non, elle a des chapelles.\par
Lethierry, étant à Rochefort, jeune matelot, avait fait connaissance avec cette sainte, probablement dans la personne de quelque jolie charentaise, peut-être de la grisette aux beaux ongles. Il lui en était resté assez de souvenir pour qu’il donnât ce nom aux deux choses qu’il aimait ; Durande à la galiote, Déruchette à la fille.\par
Il était le père de l’une et l’oncle de l’autre.\par
Déruchette était la fille d’un frère qu’il avait eu. Elle n’avait plus ni père ni mère. Il l’avait adoptée. Il remplaçait le père et la mère.\par
Déruchette n’était pas seulement sa nièce. Elle était sa filleule. C’était lui qui l’avait tenue sur les fonts de baptême. C’était lui qui lui avait trouvé cette patronne, sainte Durande, et ce prénom, Déruchette.\par
Déruchette, nous l’avons dit, était née à Saint-Pierre-Port. Elle était inscrite à sa date sur le registre de paroisse.\par
Tant que la nièce fut enfant et tant que l’oncle fut pauvre, personne ne prit garde à cette appellation, \emph{Déruchette ;} mais quand la petite fille devint une miss et quand le matelot devint un gentleman, \emph{Déruchette }choqua. On s’en étonnait. On demandait à mess Lethierry : Pourquoi Déruchette ? Il répondait : C’est un nom qui est comme ça. On essaya plusieurs fois de la débaptiser. Il ne s’y prêta point. Un jour une belle dame de la high life de Saint-Sampson, femme d’un  forgeron riche ne travaillant plus, dit à mess Lethierry : Désormais j’appellerai votre fille \emph{Nancy.} — Pourquoi pas Lons-le-Saulnier ? dit-il. La belle dame ne lâcha point prise, et lui dit le lendemain : Nous ne voulons décidément pas de Déruchette. J’ai trouvé pour votre fille un joli nom, \emph{Marianne.} — Joli nom en effet, repartit mess Lethierry, mais composé de deux vilaines bêtes, un mari et un âne. Il maintint Déruchette.\par
On se tromperait si l’on concluait du mot ci-dessus qu’il ne voulait point marier sa nièce. Il voulait la marier, certes, mais à sa façon. Il entendait qu’elle eût un mari dans son genre à lui, travaillant beaucoup, et qu’elle ne fît pas grand’chose. Il aimait les mains noires de l’homme et les mains blanches de la femme. Pour que Déruchette ne gâtât point ses jolies mains, il l’avait tournée vers la demoiselle. Il lui avait donné un maître de musique, un piano, une petite bibliothèque, et aussi un peu de fil et d’aiguilles dans une corbeille de travail. Elle était plutôt liseuse que couseuse, et plutôt musicienne que liseuse. Mess Lethierry la voulait ainsi. Le charme, c’était tout ce qu’il lui demandait. Il l’avait élevée plutôt à être fleur qu’à être femme. Quiconque a étudié les marins comprendra ceci. Les rudesses aiment les délicatesses. Pour que la nièce réalisât l’idéal de l’oncle, il fallait qu’elle fût riche. C’est bien ce qu’entendait mess Lethierry. Sa grosse machine de mer travaillait dans ce but. Il avait chargé Durande de doter Déruchette.
 \subsubsection[{A.III.8. L’air Bonny Dundee}]{A.III.8. \\
L’air \emph{Bonny Dundee}}
\noindent Déruchette habitait la plus jolie chambre des Bravées, à deux fenêtres, meublée en acajou ronceux, ornée d’un lit à rideaux quadrillés vert et blanc, et ayant vue sur le jardin et sur la haute colline où est le château du Valle. C’est de l’autre côté de cette colline qu’était le Bû de la Rue.\par
Déruchette avait dans cette chambre sa musique et son piano. Elle s’accompagnait de ce piano en chantant l’air qu’elle préférait, la mélancolique mélodie écossaise \emph{Bonny Dundee ;} tout le soir est dans cet air, toute l’aurore était dans sa voix ; cela faisait un contraste doucement surprenant ; on disait : miss Déruchette est à son piano ; et les passants du bas de la colline s’arrêtaient quelquefois devant le mur du jardin des Bravées pour écouter ce chant si frais et cette chanson si triste.\par
Déruchette était de l’allégresse allant et venant dans la maison. Elle y faisait un printemps perpétuel.  Elle était belle, mais plus jolie que belle, et plus gentille que jolie. Elle rappelait aux bons vieux pilotes amis de mess Lethierry cette princesse d’une chanson de soldats et de matelots qui était si belle « qu’elle passait pour telle dans le régiment ». Mess Lethierry disait : Elle a un câble de cheveux.\par
Dès l’enfance, elle avait été ravissante. On avait craint longtemps son nez ; mais la petite, probablement déterminée à être jolie, avait tenu bon ; la croissance ne lui avait fait aucun mauvais tour ; son nez ne s’était ni trop allongé, ni trop raccourci ; et, en devenant grande, elle était restée charmante.\par
Elle n’appelait jamais son oncle autrement que « mon père ».\par
Il lui tolérait quelques talents de jardinière, et même de ménagère. Elle arrosait elle-même ses plates-bandes de roses trémières, de molènes pourpres, de phlox vivaces et de benoîtes écarlates ; elle cultivait le crépis rose et l’oxalide rose ; elle tirait parti du climat de cette île de Guernesey, si hospitalière aux fleurs. Elle avait, comme tout le monde, des aloès en pleine terre, et, ce qui est plus difficile, elle faisait réussir la potentille du Népaul. Son petit potager était savamment ordonné ; elle y faisait succéder les épinards aux radis et les pois aux épinards ; elle savait semer des choux-fleurs de Hollande et des choux de Bruxelles qu’elle repiquait en juillet, des navets pour août, de la chicorée frisée pour septembre, des panais ronds pour l’automne, et de la raiponce pour l’hiver. Mess Lethierry la laissait faire, pourvu qu’elle ne maniât  pas trop la bêche et le râteau et surtout qu’elle ne mît pas l’engrais elle-même. Il lui avait donné deux servantes, nommées l’une Grâce et l’autre Douce, qui sont deux noms de Guernesey. Grâce et Douce faisaient le service de la maison et du jardin, et elles avaient le droit d’avoir les mains rouges.\par
Quant à mess Lethierry, il avait pour chambre un petit réduit donnant sur le port, et attenant à la grande salle basse du rez-de-chaussée où était la porte d’entrée et où venaient aboutir les divers escaliers de la maison. Sa chambre était meublée de son branle, de son chronomètre et de sa pipe. Il y avait aussi une table et une chaise. Le plafond, à poutres, avait été blanchi au lait de chaux, ainsi que les quatre murs ; à droite de la porte était cloué l’archipel de la Manche, belle carte marine portant cette mention : \emph{W. Faden, 5, Charing Cross. Geographer to His Majesty ;} et à gauche d’autres clous étalaient sur la muraille un de ces gros mouchoirs de coton où sont figurés en couleur les signaux de toute la marine du globe, ayant aux quatre coins les étendards de France, de Russie, d’Espagne et des États-Unis d’Amérique, et au centre l’Union-Jack d’Angleterre.\par
Douce et Grâce étaient deux créatures quelconques, du bon côté du mot. Douce n’était pas méchante et Grâce n’était pas laide. Ces noms dangereux n’avaient pas mal tourné. Douce, non mariée, avait un « galant ». Dans les îles de la Manche le mot est usité ; la chose aussi. Ces deux filles avaient ce qu’on pourrait appeler le service créole, une sorte de lenteur propre à la  domesticité normande dans l’archipel. Grâce, coquette et jolie, considérait sans cesse l’horizon avec une inquiétude de chat. Cela tenait à ce qu’ayant, comme Douce, un galant, elle avait, de plus, disait-on, un mari matelot, dont elle craignait le retour. Mais cela ne nous regarde pas. La nuance entre Grâce et Douce, c’est que, dans une maison moins austère et moins innocente, Douce fût restée la servante et Grâce fût devenue la soubrette. Les talents possibles de Grâce se perdaient avec une fille candide comme Déruchette. Du reste, les amours de Douce et de Grâce étaient latents. Rien n’en revenait à mess Lethierry, et rien n’en rejaillissait sur Déruchette.\par
La salle basse du rez-de-chaussée, halle à cheminée entourée de bancs et de tables, avait, au siècle dernier, servi de lieu d’assemblée à un conventicule de réfugiés français protestants. Le mur de pierre nue avait pour tout luxe un cadre de bois noir où s’étalait une pancarte de parchemin ornée des prouesses de Bénigne Bossuet, évêque de Meaux. Quelques pauvres diocésains de cet aigle, persécutés par lui lors de la révocation de l’édit de Nantes, et abrités à Guernesey, avaient accroché ce cadre à ce mur pour porter témoignage. On y lisait, si l’on parvenait à déchiffrer une écriture lourde et une encre jaunie, les faits peu connus que voici : — « Le 29 octobre 1685, démolition des temples de Morcef et de Nanteuil, demandée au Roy par M. l’évêque de Meaux. » — « Le 2 avril 1686, arrestation de Cochard père et fils pour religion, à la prière de M. l’évêque de Meaux. Relâchés ; les Cochard  ayant abjuré. » — « Le 28 octobre 1699, M. l’évêque de Meaux envoie à M. de Pontchartrain un mémoire remontrant qu’il serait nécessaire de mettre les demoiselles de Chalandes et de Neuville, qui sont de la religion réformée, dans la maison des Nouvelles-Catholiques de Paris. » — « Le 7 juillet 1703, est exécuté l’ordre demandé au Roy par M. l’évêque de Meaux de faire enfermer à l’hôpital le nommé Baudoin et sa femme, \emph{mauvais catholiques} de Fublaines. »\par
Au fond de la salle, près de la porte de la chambre de mess Lethierry, un petit retranchement en planches qui avait été la chaire huguenote était devenu, grâce à un grillage avec chatière, « l’office » du bateau à vapeur, c’est-à-dire le bureau de la Durande, tenu par mess Lethierry en personne. Sur le vieux pupitre de chêne, un registre aux pages cotées Doit et Avoir remplaçait la bible.
 \subsubsection[{A.III.9. L’homme qui avait deviné rantaine}]{A.III.9. \\
L’homme qui avait deviné rantaine}
\noindent Tant que mess Lethierry avait pu naviguer, il avait conduit la Durande, et il n’avait pas eu d’autre pilote et d’autre capitaine que lui-même ; mais il était venu une heure, nous l’avons dit, où mess Lethierry avait dû se faire remplacer. Il avait choisi pour cela sieur Clubin, de Torteval, homme silencieux. Sieur Clubin avait sur toute la côte un renom de probité sévère. C’était l’alter ego et le vicaire de mess Lethierry.\par
Sieur Clubin, quoiqu’il eût plutôt l’air d’un notaire que d’un matelot, était un marin capable et rare. Il avait tous les talents que veut le risque perpétuellement transformé. Il était arrimeur habile, gabier méticuleux, bosseman soigneux et connaisseur, timonier robuste, pilote savant, et hardi capitaine. Il était prudent, et il poussait quelquefois la prudence jusqu’à oser, ce qui est une grande qualité à la mer. Il avait la crainte du probable tempérée par l’instinct du possible. C’était un de ces marins qui affrontent le danger  dans une proportion à eux connue et qui de toute aventure savent dégager le succès. Toute la certitude que la mer peut laisser à un homme, il l’avait. Sieur Clubin, en outre, était un nageur renommé ; il était de cette race d’hommes rompus à la gymnastique de la vague, qui restent tant qu’on veut dans l’eau, qui, à Jersey, partent du Havre-des-Pas, doublent la Colette, font le tour de l’ermitage et du château Élisabeth, et reviennent au bout de deux heures à leur point de départ. Il était de Torteval, et il passait pour avoir souvent fait à la nage le trajet redouté des Hanois à la pointe de Plainmont.\par
Une des choses qui avaient le plus recommandé sieur Clubin à mess Lethierry, c’est que, connaissant ou pénétrant Rantaine, il avait signalé à mess Lethierry l’improbité de cet homme, et lui avait dit : — \emph{Rantaine vous volera}. Ce qui s’était vérifié. Plus d’une fois, pour des objets, il est vrai, peu importants, mess Lethierry avait mis à l’épreuve l’honnêteté, poussée jusqu’au scrupule, de sieur Clubin, et il se reposait de ses affaires sur lui. Mess Lethierry disait : Toute conscience veut toute confiance.
 \subsubsection[{A.III.10. Les récits de long cours}]{A.III.10. \\
Les récits de long cours}
\noindent Mess Lethierry, mal à l’aise autrement, portait toujours ses habits de bord, et plutôt sa vareuse de matelot que sa vareuse de pilote. Cela faisait plisser le petit nez de Déruchette. Rien n’est joli comme les grimaces de la grâce en colère. Elle grondait et riait. — \emph{Bon père}, s’écriait-elle, \emph{pouah ! vous sentez le goudron.} Et elle lui donnait une petite tape sur sa grosse épaule.\par
Ce bon vieux héros de la mer avait rapporté de ses voyages des récits surprenants. Il avait vu à Madagascar des plumes d’oiseau dont trois suffisaient à faire le toit d’une maison. Il avait vu dans l’Inde des tiges d’oseille hautes de neuf pieds. Il avait vu dans la Nouvelle-Hollande des troupeaux de dindons et d’oies menés et gardés par un chien de berger qui est un oiseau, et qu’on appelle l’agami. Il avait vu des cimetières d’éléphants. Il avait vu en Afrique des gorilles, espèces d’hommes-tigres, de sept pieds de haut.  Il connaissait les mœurs de tous les singes, depuis le macaque sauvage qu’il appelait \emph{macaco bravo} jusqu’au macaque hurleur qu’il appelait \emph{macaco barbado}. Au Chili, il avait vu une guenon attendrir les chasseurs en leur montrant son petit. Il avait vu en Californie un tronc d’arbre creux tombé à terre dans l’intérieur duquel un homme à cheval pouvait faire cent cinquante pas. Il avait vu au Maroc les mozabites et les biskris se battre à coups de matraks et de barres de fer, les biskris pour avoir été traités de \emph{kelb,} qui veut dire chiens, et les mozabites pour avoir été traités de \emph{khamsi}, qui veut dire gens de la cinquième secte. Il avait vu en Chine couper en petits morceaux le pirate Chanh-thong-quan-larh-Quoi, pour avoir assassiné le âp d’un village. A Thu-dan-mot, il avait vu un lion enlever une vieille femme en plein marché de la ville. Il avait assisté à l’arrivée du grand serpent venant de Canton à Saïgon pour célébrer dans la pagode de Cho-len la fête de Quan-nam, déesse des navigateurs. Il avait contemplé chez les Moï le grand Quan-Sû. A Rio-Janeiro, il avait vu les dames brésiliennes se mettre le soir dans les cheveux de petites bulles de gaze contenant chacune une vagalumes, belle mouche à phosphore, ce qui les coiffe d’étoiles. Il avait combattu dans l’Uruguay les fourmilières et dans le Paraguay les araignées d’oiseaux, velues, grosses comme une tête d’enfant, couvrant de leurs pattes un diamètre d’un tiers d’aune, et attaquant l’homme, auquel elles lancent leurs poils qui s’enfoncent comme des flèches dans la chair et y soulèvent des pustules. Sur le fleuve Arinos,  affluent du Tocantins, dans les forêts vierges au nord de Diamantina, il avait constaté l’effrayant peuple chauve-souris, les murcilagos, hommes qui naissent avec les cheveux blancs et les yeux rouges, habitent le sombre des bois, dorment le jour, s’éveillent la nuit, et pêchent et chassent dans les ténèbres, voyant mieux quand il n’y a pas de lune. Près de Beyrouth, dans un campement d’une expédition dont il faisait partie, un pluviomètre ayant été volé dans une tente, un sorcier, habillé de deux ou trois bandelettes de cuir et ressemblant à un homme qui serait vêtu de ses bretelles, avait si furieusement agité une sonnette au bout d’une corne qu’une hyène était venue rapporter le pluviomètre. Cette hyène était la voleuse. Ces histoires vraies ressemblaient tant à des contes qu’elles amusaient Déruchette.\par
La poupée de la Durande était le lien entre le bateau et la fille. On nomme \emph{poupée} dans les îles normandes la figure taillée dans la proue, statue de bois sculptée à peu près. De là, pour dire \emph{naviguer,} cette locution locale, \emph{être entre poupe et poupée.}\par
La poupée de la Durande était particulièrement chère à mess Lethierry. Il l’avait commandée au charpentier ressemblante à Déruchette. Elle ressemblait à coups de hache. C’était une bûche faisant effort pour être une jolie fille.\par
Ce bloc légèrement difforme faisait illusion à mess Lethierry. Il le considérait avec une contemplation de croyant. Il était de bonne foi devant cette figure. Il y reconnaissait parfaitement Déruchette. C’est un peu  comme cela que le dogme ressemble à la vérité, et l’idole à Dieu.\par
Mess Lethierry avait deux grandes joies par semaine ; une joie le mardi et une joie le vendredi. Première joie, voir partir la Durande ; deuxième joie, la voir revenir. Il s’accoudait à sa fenêtre, regardait son œuvre, et était heureux. Il y a quelque chose de cela dans la Genèse. \emph{Et vidit quod esset bonum.}\par
Le vendredi, la présence de mess Lethierry à sa fenêtre valait un signal. Quand on voyait, à la croisée des Bravées, s’allumer sa pipe, on disait : Ah ! le bateau à vapeur est à l’horizon. Une fumée annonçait l’autre.\par
La Durande en rentrant au port nouait son câble sous les fenêtres de mess Lethierry à un gros anneau de fer scellé dans le soubassement des Bravées. Ces nuits-là, Lethierry faisait un admirable somme dans son branle, sentant d’un côté Déruchette endormie et de l’autre Durande amarrée.\par
Le lieu d’amarrage de la Durande était voisin de la cloche du port. Il y avait là, devant la porte des Bravées, un petit bout de quai.\par
Ce quai, les Bravées, la maison, le jardin, les ruettes bordées de haies, la plupart même des habitations environnantes, n’existent plus aujourd’hui. L’exploitation du granit de Guernesey a fait vendre ces terrains. Tout cet emplacement est occupé, à l’heure où nous sommes, par des chantiers de casseurs de pierres.
 \subsubsection[{A.III.11. Coup d’œil sur les maris éventuels}]{A.III.11. \\
Coup d’œil sur les maris éventuels}
\noindent Déruchette grandissait, et ne se mariait pas.\par
Mess Lethierry, en en faisant une fille aux mains blanches, l’avait rendue difficile. Ces éducations-là se retournent plus tard contre vous.\par
Du reste, il était, quant à lui, plus difficile encore. Le mari qu’il imaginait pour Déruchette était aussi un peu un mari pour Durande. Il eût voulu pourvoir d’un coup ses deux filles. Il eût voulu que le conducteur de l’une pût être aussi le pilote de l’autre. Qu’est-ce qu’un mari ? C’est le capitaine d’une traversée. Pourquoi pas le même patron à la fille et au bateau ? Un ménage obéit aux marées. Qui sait mener une barque sait mener une femme. Ce sont les deux sujettes de la lune et du vent. Sieur Clubin, n’ayant guère que quinze ans de moins que mess Lethierry, ne pouvait être pour Durande qu’un patron provisoire ; il fallait un pilote jeune, un patron définitif, un vrai successeur du fondateur, de l’inventeur, du créateur. Le pilote définitif  de Durande serait un peu le gendre de mess Lethierry. Pourquoi ne pas fondre les deux gendres dans un ? Il caressait cette idée. Il voyait, lui aussi, apparaître dans ses songes un fiancé. Un puissant gabier basané et fauve, athlète de la mer, voilà son idéal. Ce n’était pas tout à fait celui de Déruchette. Elle faisait un rêve plus rose.\par
Quoi qu’il en fût, l’oncle et la nièce semblaient être d’accord pour ne point se hâter. Quand on avait vu Déruchette devenir une héritière probable, les partis s’étaient présentés en foule. Ces empressements-là ne sont pas toujours de bonne qualité. Mess Lethierry le sentait. Il grommelait : fille d’or, épouseur de cuivre. Et il éconduisait les prétendants. Il attendait. Elle de même.\par
Chose singulière, il tenait peu à l’aristocratie. De ce côté-là, mess Lethierry était un anglais invraisemblable. On croira difficilement qu’il avait été jusqu’à refuser pour Déruchette un Ganduel, de Jersey, et un Bugnet-Nicolin, de Serk. On n’a pas même craint d’affirmer, mais nous doutons que cela soit possible, qu’il n’avait point accepté une ouverture venant de l’aristocratie d’Aurigny, et qu’il avait décliné les propositions d’un membre de la famille Édou, laquelle évidemment descend d’Édouard le Confesseur.
 \subsubsection[{A.III.12. Exception dans le caractère de lethierry}]{A.III.12. \\
Exception dans le caractère de lethierry}
\noindent Mess Lethierry avait un défaut ; un gros. Il haïssait, non quelqu’un, mais quelque chose, le prêtre. Un jour, lisant, — car il lisait, — dans Voltaire, — car il lisait Voltaire, — ces mots : « les prêtres sont des chats », il posa le livre, et on l’entendit grommeler à demi-voix : je me sens chien.\par
Il faut se souvenir que les prêtres, les luthériens et les calvinistes comme les catholiques, l’avaient, dans sa création du Devil-Boat local, vivement combattu et doucement persécuté. Être révolutionnaire en navigation, essayer d’ajuster un progrès à l’archipel normand, faire essuyer à la pauvre petite île de Guernesey les plâtres d’une invention nouvelle, c’était là, nous ne l’avons point dissimulé, une témérité damnable. Aussi l’avait-on un peu damné. Nous parlons ici, qu’on ne l’oublie pas, du clergé ancien, bien différent du clergé actuel, qui, dans presque toutes les églises locales, a une tendance libérale vers le  progrès. On avait entravé Lethierry de cent manières ; toute la quantité d’obstacle qu’il peut y avoir dans les prêches et dans les sermons lui avait été opposée. Détesté des hommes d’église, il les détestait. Leur haine était la circonstance atténuante de la sienne.\par
Mais, disons-le, son aversion des prêtres était idiosyncrasique. Il n’avait pas besoin pour les haïr, d’en être haï. Comme il le disait, il était le chien de ces chats. Il était contre eux par l’idée, et, ce qui est le plus irréductible, par l’instinct. Il sentait leurs griffes latentes, et il montrait les dents. Un peu à tort et à travers, convenons-en, et pas toujours à propos. Ne point distinguer est un tort. Il n’y a pas de bonne haine en bloc. Le vicaire savoyard n’eût point trouvé grâce devant lui. Il n’est pas sûr que, pour mess Lethierry, il y eût un bon prêtre. A force d’être philosophe, il perdait un peu de sagesse. L’intolérance des tolérants existe, de même que la rage des modérés. Mais Lethierry était si débonnaire qu’il ne pouvait être vraiment haineux. Il repoussait plutôt qu’il n’attaquait. Il tenait les gens d’église à distance. Ils lui avaient fait du mal, il se bornait à ne pas leur vouloir de bien. La nuance entre leur haine et la sienne, c’est que la leur était animosité, et que la sienne était antipathie.\par
Guernesey, toute petite île qu’elle est, a de la place pour deux religions. Elle contient de la religion catholique et de la religion protestante. Ajoutons qu’elle ne met point les deux religions dans la même église. Chaque culte a son temple ou sa chapelle. En Allemagne, à Heidelberg, par exemple, on n’y fait pas tant  de façons ; on coupe l’église en deux ; une moitié à saint Pierre, une moitié à Calvin ; entre deux, une cloison pour prévenir les gourmades ; parts égales ; les catholiques ont trois autels, les huguenots ont trois autels ; comme ce sont les mêmes heures d’offices, la cloche unique sonne à la fois pour les deux services. Elle appelle en même temps à Dieu et au diable. Simplification.\par
Le flegme allemand s’accommode de ces voisinages. Mais à Guernesey chaque religion est chez elle. Il y a la paroisse orthodoxe et il y a la paroisse hérétique. On peut choisir. Ni l’une, ni l’autre ; tel avait été le choix de mess Lethierry.\par
Ce matelot, cet ouvrier, ce philosophe, ce parvenu du travail, très simple en apparence, n’était pas du tout simple au fond. Il avait ses contradictions et ses opiniâtretés. Sur le prêtre, il était inébranlable. Il eût rendu des points à Montlosier.\par
Il se permettait des railleries très déplacées. Il avait des mots à lui, bizarres, mais ayant un sens. Aller à confesse, il appelait cela « peigner sa conscience ». Le peu de lettres qu’il avait, bien peu, une certaine lecture glanée çà et là, entre deux bourrasques, se compliquait de fautes d’orthographe. Il avait aussi des fautes de prononciation, pas toujours naïves. Quand la paix fut faite par Waterloo entre la France de Louis XVIII et l’Angleterre de Wellington, mess Lethierry dit : \emph{Bourmont a été le traître d’union entre les deux camps}. Une fois il écrivit papauté, \emph{pape ôté}. Nous ne pensons pas que ce fût exprès.\par
 Cet antipapisme ne lui conciliait point les anglicans. Il n’était pas plus aimé des recteurs protestants que des curés catholiques. En présence des dogmes les plus graves, son irréligion éclatait presque sans retenue. Un hasard l’ayant conduit à un sermon sur l’enfer du révérend Jaquemin Hérode, sermon magnifique rempli d’un bout à l’autre de textes sacrés prouvant les peines éternelles, les supplices, les tourments, les damnations, les châtiments inexorables, les brûlements sans fin, les malédictions inextinguibles, les colères de la toute-puissance, les fureurs célestes, les vengeances divines, choses incontestables, on l’entendit, en sortant avec un des fidèles, dire doucement : — Voyez-vous, moi, j’ai une drôle d’idée. Je m’imagine que Dieu est bon.\par
Ce levain d’athéisme lui venait de son séjour en France.\par
Quoique guernesiais, et assez pur sang, on l’appelait dans l’île « le français », à cause de son esprit \emph{improper}. Lui-même ne s’en cachait point, il était imprégné d’idées subversives. Son acharnement de faire ce bateau à vapeur, ce Devil-Boat, l’avait bien prouvé. Il disait : \emph{J’ai tété 89}. Ce n’est point là un bon lait.\par
Du reste, des contre-sens, il en faisait. Il est très difficile de rester entier dans les petits pays. En France, \emph{garder les apparences,} en Angleterre, \emph{être respectable,} la vie tranquille est à ce prix. Être respectable, cela implique une foule d’observances, depuis le dimanche bien sanctifié jusqu’à la cravate bien mise.  « Ne pas se faire montrer au doigt », voilà encore une loi terrible. Être montré au doigt c’est le diminutif de l’anathème. Les petites villes, marais de commères, excellent dans cette malignité isolante, qui est la malédiction vue par le petit bout de la lorgnette. Les plus vaillants redoutent ce raca. On affronte la mitraille, on affronte l’ouragan, on recule devant M\textsuperscript{me} Pimbêche. Mess Lethierry était plutôt tenace que logique. Mais, sous cette pression, sa ténacité même fléchissait. Il mettait, autre locution pleine de concessions latentes, et parfois inavouables, « de l’eau dans son vin ». Il se tenait à l’écart des hommes du clergé, mais il ne leur fermait point résolûment sa porte. Aux occasions officielles et aux époques voulues des visites pastorales, il recevait d’une façon suffisante, soit le recteur luthérien, soit le chapelain papiste. Il lui arrivait, de loin en loin, d’accompagner à la paroisse anglicane Déruchette, laquelle elle-même, nous l’avons dit, n’y allait qu’aux quatre grandes fêtes de l’année.\par
Somme toute, ces compromis, qui lui coûtaient, l’irritaient, et, loin de l’incliner vers les gens d’église, augmentaient son escarpement intérieur. Il s’en dédommageait par plus de moquerie. Cet être sans amertume n’avait d’âcreté que de ce côté-là. Aucun moyen de l’amender là-dessus.\par
De fait et absolument, c’était là son tempérament, et il fallait en prendre son parti.\par
Tout clergé lui déplaisait. Il avait l’irrévérence révolutionnaire. D’une forme à l’autre du culte il distinguait peu. Il ne rendait même pas justice à ce grand  progrès : ne point croire à la présence réelle. Sa myopie en ces choses allait jusqu’à ne point voir la nuance entre un ministre et un abbé. Il confondait un révérend docteur avec un révérend père. Il disait : \emph{Wesley ne vaut pas mieux que Loyola.}. Quand il voyait passer un pasteur avec sa femme, il se détournait. \emph{Prêtre marié !} disait-il, avec l’accent absurde que ces deux mots avaient en France à cette époque. Il contait qu’à son dernier voyage en Angleterre, il avait vu « l’évêchesse de Londres ». Ses révoltes sur ce genre d’unions allaient jusqu’à la colère. — Une robe n’épouse pas une robe ! s’écriait-il. — Le sacerdoce lui faisait l’effet d’un sexe. Il eût volontiers dit : « ni homme, ni femme ; prêtre ». Il appliquait avec mauvais goût, au clergé anglican et au clergé papiste, les mêmes épithètes dédaigneuses ; il enveloppait les deux « soutanes » dans la même phraséologie ; et il ne se donnait pas la peine de varier, à propos des prêtres, quels qu’ils fussent, catholiques ou luthériens, les métonymies soldatesques usitées dans ce temps-là. Il disait à Déruchette : \emph{Marie-toi avec qui tu voudras, pourvu que ce ne soit pas avec un calotin}.
 \subsubsection[{A.III.13. L’insouciance fait partie de la grace}]{A.III.13. \\
L’insouciance fait partie de la grace}
\noindent Une fois une parole dite, mess Lethierry s’en souvenait ; une fois une parole dite, Déruchette l’oubliait. Là était la nuance entre l’oncle et la nièce.\par
Déruchette, élevée comme on l’a vu, s’était accoutumée à peu de responsabilité. Il y a, insistons-y, plus d’un péril latent dans une éducation pas assez prise au sérieux. Vouloir faire son enfant heureux trop tôt, c’est peut-être une imprudence.\par
Déruchette croyait que, pourvu qu’elle fût contente, tout était bien. Elle sentait d’ailleurs son oncle joyeux de la voir joyeuse. Elle avait à peu près les idées de mess Lethierry. Sa religion se satisfaisait d’aller à la paroisse quatre fois par an. On l’a vue en toilette pour Noël. De la vie, elle ignorait tout. Elle avait tout ce qu’il faut pour être un jour folle d’amour. En attendant, elle était gaie.\par
Elle chantait au hasard, jasait au hasard, vivait devant elle, jetait un mot et passait, faisait une chose  et fuyait, était charmante. Joignez à cela la liberté anglaise. En Angleterre les enfants vont seuls, les filles sont leurs maîtresses, l’adolescence a la bride sur le cou. Telles sont les mœurs. Plus tard ces filles libres font des femmes esclaves. Nous prenons ici ces deux mots en bonne part ; libres dans la croissance, esclaves dans le devoir.\par
Déruchette s’éveillait chaque matin avec l’inconscience de ses actions de la veille. Vous l’eussiez bien embarrassée en lui demandant ce qu’elle avait fait la semaine passée. Ce qui ne l’empêchait pas d’avoir, à de certaines heures troubles, un malaise mystérieux, et de sentir on ne sait quel passage du sombre de la vie sur son épanouissement et sur sa joie. Ces azurs-là ont ces nuages-là. Mais ces nuages s’en allaient vite. Elle en sortait par un éclat de rire, ne sachant pourquoi elle avait été triste ni pourquoi elle était sereine. Elle jouait avec tout. Son espièglerie becquetait les passants. Elle faisait des malices aux garçons. Si elle eût rencontré le diable, elle n’en eût pas eu pitié, elle lui eût fait une niche. Elle était jolie, et en même temps si innocente, qu’elle en abusait. Elle donnait un sourire comme un jeune chat donne un coup de griffe. Tant pis pour l’égratigné. Elle n’y songeait plus. Hier n’existait pas pour elle ; elle vivait dans la plénitude d’aujourd’hui. Voilà ce que c’est que trop de bonheur. Chez Déruchette le souvenir s’évanouissait comme la neige fond.
 \subsection[{A.IV. Livre quatrième. Le bug-pipe}]{A.IV. Livre quatrième \\
Le bug-pipe}
  \subsubsection[{A.IV.1. Premières rougeurs d’une aurore, ou d’un incendie}]{A.IV.1. \\
Premières rougeurs d’une aurore, ou d’un incendie}
\noindent Gilliatt n’avait jamais parlé à Déruchette. Il la connaissait pour l’avoir vue de loin, comme on connaît l’étoile du matin.\par
A l’époque où Déruchette avait rencontré Gilliatt dans le chemin de Saint-Pierre-Port au Valle et lui avait fait la surprise d’écrire son nom sur la neige, elle avait seize ans. La veille précisément, mess Lethierry lui avait dit : Ne fais plus d’enfantillages. Te voilà grande fille.\par
Ce nom, \emph{Gilliatt,} écrit par cette enfant, était tombé dans une profondeur inconnue.\par
Qu’était-ce que les femmes pour Gilliatt ? lui-même n’aurait pu le dire. Quand il en rencontrait une, il lui faisait peur, et il en avait peur. Il ne parlait à une femme qu’à la dernière extrémité. Il n’avait jamais été « le galant » d’aucune campagnarde. Quand il était seul  dans un chemin et qu’il voyait une femme venir vers lui, il enjambait une clôture de courtil ou se fourrait dans une broussaille et s’en allait. Il évitait même les vieilles. Il avait vu dans sa vie une parisienne. Une parisienne de passage, étrange événement pour Guernesey à cette époque lointaine. Et Gilliatt avait entendu cette parisienne raconter en ces termes ses malheurs : « Je suis très ennuyée, je viens de recevoir des gouttes de pluie sur mon chapeau, il est abricot, et c’est une couleur qui ne pardonne pas. » Ayant trouvé plus tard, entre les feuillets d’un livre, une ancienne gravure de modes représentant « une dame de la chaussée d’Antin » en grande toilette, il l’avait collée à son mur, en souvenir de cette apparition. Les soirs d’été, il se cachait derrière les rochers de la crique Houmet-Paradis pour voir les paysannes se baigner en chemise dans la mer. Un jour, à travers une haie, il avait regardé la sorcière de Torteval remettre sa jarretière. Il était probablement vierge.\par
Ce matin de Noël où il rencontra Déruchette et où elle écrivit son nom en riant, il rentra chez lui ne sachant plus pourquoi il était sorti. La nuit venue, il ne dormit pas. Il songea à mille choses ; — qu’il ferait bien de cultiver des radis noirs dans son jardin ; que l’exposition était bonne ; — qu’il n’avait pas vu passer le bateau de Serk ; était-il arrivé quelque chose à ce bateau ? — qu’il avait vu des trique-madame en fleur, chose rare pour la saison. Il n’avait jamais su au juste ce que lui était la vieille femme qui était morte, il se dit que décidément elle devait être sa mère, et il  pensa à elle avec un redoublement de tendresse. Il pensa au trousseau de femme qui était dans la malle de cuir. Il pensa que le révérend Jaquemin Hérode serait probablement un jour ou l’autre nommé doyen de Saint-Pierre-Port subrogé de l’évêque, et que le rectorat de Saint-Sampson deviendrait vacant. Il pensa que le lendemain de Noël on serait au vingt-septième jour de la lune, et que par conséquent la haute mer serait à trois heures vingt et une minutes, la demi-retirée à sept heures quinze, la basse mer à neuf heures trente-trois, et la demi-montée à douze heures trente-neuf. Il se rappela dans les moindres détails le costume du highlander qui lui avait vendu le bug-pipe, son bonnet orné d’un chardon, sa claymore, son habit serré aux pans courts et carrés, son jupon, le scilt or philaberg, orné de la bourse sporran et du smushing-mull, tabatière de corne, son épingle faite d’une pierre écossaise, ses deux ceintures, la sashwise et le belts, son épée, le swond, son coutelas, le dirck, et le skene dhu, couteau noir à poignée noire ornée de deux cairgorums, et les genoux nus de ce soldat, ses bas, ses guêtres quadrillées et ses souliers à boucles. Cet équipement devint un spectre, le poursuivit, lui donna la fièvre, et l’assoupit. Il se réveilla au grand jour, et sa première pensée fut Déruchette.\par
Le lendemain il dormit, mais il revit toute la nuit le soldat écossais. Il se dit à travers son sommeil que les Chefs-Plaids d’après Noël seraient tenus le 21 janvier. Il rêva aussi du vieux recteur Jaquemin Hérode. En se réveillant il songea à Déruchette, et il eut contre  elle une violente colère ; il regretta de ne plus être petit, parce qu’il irait jeter des pierres dans ses carreaux.\par
Puis il pensa que, s’il était petit, il aurait sa mère, et il se mit à pleurer.\par
Il forma le projet d’aller passer trois mois à Chousey ou aux Minquiers. Pourtant il ne partit pas.\par
Il ne remit plus les pieds dans la route de Saint-Pierre-Port au Valle.\par
Il se figurait que son nom, \emph{Gilliatt}, était resté là gravé sur la terre et que tous les passants devaient le regarder.
 \subsubsection[{A.IV.2. Entrée, pas a pas, dans l’inconnu}]{A.IV.2. \\
Entrée, pas a pas, dans l’inconnu}
\noindent En revanche, il voyait tous les jours les Bravées. Il ne le faisait pas exprès, mais il allait de ce côté-là. Il se trouvait que son chemin était toujours de passer par le sentier qui longeait le mur du jardin de Déruchette.\par
Un matin, comme il était dans ce sentier, une femme du marché qui revenait des Bravées dit à une autre : \emph{Miss Lethierry aime les seakales.}\par
Il fit dans son jardin du Bû de la Rue une fosse à seakales. Le seakale est un chou qui a le goût de l’asperge.\par
Le mur du jardin des Bravées était très bas ; on pouvait l’enjamber. L’idée de l’enjamber lui eût paru épouvantable. Mais il n’était pas défendu d’entendre en passant, comme tout le monde, les voix des personnes qui parlaient dans les chambres ou dans le jardin. Il n’écoutait pas, mais il entendait. Une fois, il entendit les deux servantes, Douce et Grâce, se  quereller. C’était un bruit dans la maison. Cette querelle lui resta dans l’oreille comme une musique.\par
Une autre fois, il distingua une voix qui n’était pas comme celle des autres et qui lui sembla devoir être la voix de Déruchette. Il prit la fuite.\par
Les paroles que cette voix avait prononcées demeurèrent à jamais gravées dans sa pensée. Il se les redisait à chaque instant. Ces paroles étaient : \emph{Vous plairait-il me bailler le genêt\footnote{ \noindent Me donner le balai.
 } ?}\par
Par degrés il s’enhardit. Il osa s’arrêter. Il arriva une fois que Déruchette, impossible à apercevoir du dehors, quoique sa fenêtre fût ouverte, était à son piano, et chantait. Elle chantait son air \emph{Bonny Dundee}. Il devint très pâle, mais il poussa la fermeté jusqu’à écouter.\par
Le printemps arriva. Un jour, Gilliatt eut une vision ; le ciel s’ouvrit. Gilliatt vit Déruchette arroser des laitues.\par
Bientôt, il fit plus que s’arrêter. Il observa ses habitudes, il remarqua ses heures, et il l’attendit.\par
Il avait bien soin de ne pas se montrer.\par
Peu à peu, en même temps que les massifs se remplissaient de papillons et de roses, immobile et muet des heures entières, caché derrière ce mur, vu de personne, retenant son haleine, il s’habitua à voir Déruchette aller et venir dans le jardin. On s’accoutume au poison.\par
De la cachette où il était, il entendait souvent  Déruchette causer avec mess Lethierry sous une épaisse tonnelle de charmille où il y avait un banc. Les paroles venaient distinctement jusqu’à lui.\par
Que de chemin il avait fait ! Maintenant il en était venu à guetter et à prêter l’oreille. Hélas ! le cœur humain est un vieil espion.\par
Il y avait un autre banc, visible et tout proche, au bord d’une allée. Déruchette s’y asseyait quelquefois.\par
D’après les fleurs qu’il voyait Déruchette cueillir et respirer, il avait deviné ses goûts en fait de parfums. Le liseron était l’odeur qu’elle préférait, puis l’œillet, puis le chèvrefeuille, puis le jasmin. La rose n’était que la cinquième. Elle regardait le lys, mais elle ne le respirait pas.\par
D’après ce choix de parfums, Gilliatt la composait dans sa pensée. A chaque odeur il rattachait une perfection.\par
La seule idée d’adresser la parole à Déruchette lui faisait dresser les cheveux.\par
Une bonne vieille chineuse, que son industrie ambulante ramenait de temps en temps dans la ruette longeant l’enclos des Bravées, en vint à remarquer confusément les assiduités de Gilliatt pour cette muraille et sa dévotion à ce lieu désert. Rattacha-t-elle la présence de cet homme devant ce mur à la possibilité d’une femme derrière ce mur ? Aperçut-elle ce vague fil invisible ? Était-elle, en sa décrépitude mendiante, restée assez jeune pour se rappeler quelque chose des belles années, et savait-elle encore, dans  son hiver et dans sa nuit, ce que c’est que l’aube ? Nous l’ignorons, mais il paraît qu’une fois, en passant près de Gilliatt « faisant sa faction », elle dirigea de son côté toute la quantité de sourire dont elle était encore capable, et grommela entre ses gencives : \emph{ça chauffe.}\par
Gilliatt entendit ces mots, il en fut frappé, il murmura avec un point d’interrogation intérieur : — Ça chauffe ? Que veut dire cette vieille ? — Il répéta machinalement le mot toute la journée, mais il ne le comprit pas.\par
Un soir qu’il était à sa fenêtre du Bû de la Rue, cinq ou six jeunes filles de l’Ancresse vinrent par partie de plaisir se baigner dans la crique de Houmet. Elles jouaient dans l’eau, très naïvement, à cent pas de lui. Il ferma sa fenêtre violemment. Il s’aperçut qu’une femme nue lui faisait horreur.
 \subsubsection[{A.IV.3. L’air Bonny Dundee trouve un écho dans la colline}]{A.IV.3. \\
L’air \emph{Bonny Dundee} trouve un écho dans la colline}
\noindent Derrière l’enclos du jardin des Bravées, un angle de mur couvert de houx et de lierre, encombré d’orties, avec une mauve sauvage arborescente et un grand bouillon-blanc poussant dans les granits, ce fut dans ce recoin qu’il passa à peu près tout son été. Il était là, inexprimablement pensif. Les lézards, accoutumés à lui, se chauffaient dans les mêmes pierres au soleil. L’été fut lumineux et caressant. Gilliatt avait au-dessus de sa tête le va-et-vient des nuages. Il était assis dans l’herbe. Tout était plein de bruits d’oiseaux. Il se prenait le front à deux mains et se demandait : Mais enfin pourquoi a-t-elle écrit mon nom sur la neige ? Le vent de mer jetait au loin de grands souffles. Par intervalles, dans la carrière lointaine de la Vaudue, la trompe des mineurs grondait brusquement, avertissant les passants de s’écarter et qu’une mine allait faire explosion. On ne voyait pas le port de Saint-Sampson, mais on  voyait les pointes des mâts par-dessus les arbres. Les mouettes volaient éparses. Gilliatt avait entendu sa mère dire que les femmes pouvaient être amoureuses des hommes, que cela arrivait quelquefois. Il se répondait : Voilà. Je comprends, Déruchette est amoureuse de moi. Il se sentait profondément triste. Il se disait : Mais elle aussi, elle pense à moi de son côté ; c’est bien fait. Il songeait que Déruchette était riche, et que, lui, il était pauvre. Il pensait que le bateau à vapeur était une exécrable invention. Il ne pouvait jamais se rappeler quel quantième du mois on était. Il regardait vaguement les gros bourdons noirs à croupes jaunes et à ailes courtes qui s’enfoncent avec bruit dans les trous des murailles.\par
Un soir, Déruchette rentrait se coucher. Elle s’approcha de la fenêtre pour la fermer. La nuit était obscure. Tout à coup Déruchette prêta l’oreille. Dans cette profondeur d’ombre il y avait une musique. Quelqu’un qui était probablement sur le versant de la colline, ou au pied des tours du château du Valle, ou peut-être plus loin encore, exécutait un air sur un instrument. Déruchette reconnut sa mélodie favorite \emph{Bonny Dundee} jouée sur le bug-pipe. Elle n’y comprit rien.\par
Depuis ce moment, cette musique se renouvela de temps en temps à la même heure, particulièrement dans les nuits très noires.\par
Déruchette n’aimait pas beaucoup cela.
 \subsubsection[{A.IV.4.}]{A.IV.4.}

\begin{verse}
Pour l’oncle et le tuteur, bonshommes taciturnes,\\
Les sérénades sont des tapages nocturnes.\\
\end{verse}

\bibl{(\emph{Vers d’une comédie inédite.})}
\noindent Quatre années passèrent.\par
Déruchette approchait de ses vingt et un ans et n’était toujours pas mariée.\par
Quelqu’un a écrit quelque part : — Une idée fixe, c’est une vrille. Chaque année elle s’enfonce d’un tour. Si on veut nous l’extirper la première année, on nous tirera les cheveux ; la deuxième année, on nous déchirera la peau ; la troisième année, on nous brisera l’os ; la quatrième année, on nous arrachera la cervelle.\par
Gilliatt en était à cette quatrième année-là.\par
Il n’avait pas encore dit une parole à Déruchette. Il songeait du côté de cette charmante fille. C’était tout.\par
Il était arrivé qu’une fois, se trouvant par hasard à Saint-Sampson, il avait vu Déruchette causant avec mess Lethierry devant la porte des Bravées qui s’ouvrait sur la chaussée du port. Gilliatt s’était risqué à  approcher très près. Il croyait être sûr qu’au moment où il avait passé elle avait souri. Il n’y avait à cela rien d’impossible.\par
Déruchette entendait toujours de temps en temps le bug-pipe.\par
Ce bug-pipe, mess Lethierry aussi l’entendait. Il avait fini par remarquer cet acharnement de musique sous les fenêtres de Déruchette. Musique tendre, circonstance aggravante. Un galant nocturne n’était pas de son goût. Il voulait marier Déruchette le jour venu, quand elle voudrait et quand il voudrait, purement et simplement, sans roman et sans musique. Impatienté, il avait guetté, et il croyait bien avoir entrevu Gilliatt. Il s’était passé les ongles dans les favoris, signe de colère, et il avait grommelé : \emph{Qu’a-t-il à piper, cet animal-là ? Il aime Déruchette, c’est clair. Tu perds ton temps. Qui veut Déruchette doit s’adresser à moi, et pas en jouant de la flûte.}\par
Un événement considérable, prévu depuis longtemps, s’accomplit. On annonça que le révérend Jaquemin Hérode était nommé subrogé de l’évêque de Winchester, doyen de l’île et receveur de Saint-Pierre-Port, et qu’il quitterait Saint-Sampson pour Saint-Pierre immédiatement après avoir installé son successeur.\par
Le nouveau recteur ne pouvait tarder à arriver. Ce prêtre était un gentleman d’origine normande, monsieur Joë Ebenezer Caudray, anglaisé Cawdry.\par
On avait sur le futur recteur des détails que la bienveillance et la malveillance commentaient en sens  inverse. On le disait jeune et pauvre, mais sa jeunesse était tempérée par beaucoup de doctrine et sa pauvreté par beaucoup d’espérance. Dans la langue spéciale créée pour l’héritage et la richesse, la mort s’appelle espérance. Il était le neveu et l’héritier du vieux et opulent doyen de Saint-Asaph. Ce doyen mort, il serait riche. M. Ebenezer Caudray avait des parentés distinguées ; il avait presque droit à la qualité d’honorable. Quant à sa doctrine, on la jugeait diversement. Il était anglican, mais, selon l’expression de l’évêque Tillotson, très « libertin », c’est-à-dire très sévère. Il répudiait le pharisaïsme ; il se ralliait plutôt au presbytère qu’à l’épiscopat. Il faisait le rêve de la primitive église, où Adam avait le droit de choisir Ève, et où Frumentanus, évêque d’Hiérapolis, enlevait une fille pour en faire sa femme en disant aux parents : \emph{Elle le veut et je le veux, vous n’êtes plus son père et vous n’êtes plus sa mère, je suis l’ange d’Hiérapolis, et celle-ci est mon épouse. Le père, c’est Dieu.} S’il fallait en croire ce que l’on disait, M. Ebenezer Caudray subordonnait le texte \emph{Tes père et mère honoreras}, au texte, selon lui supérieur : \emph{La femme est la chair de l’homme. La femme quittera son père et sa mère pour suivre son mari}. Du reste, cette tendance à circonscrire l’autorité paternelle, et à favoriser religieusement tous les modes de formation du lien conjugal, est propre à tout le protestantisme, particulièrement en Angleterre et singulièrement en Amérique.
 \subsubsection[{A.IV.5. Le succès juste est toujours haï}]{A.IV.5. \\
Le succès juste est toujours haï}
\noindent Voici quel était à ce moment-là le bilan de mess Lethierry. La Durande avait tenu tout ce qu’elle avait promis. Mess Lethierry avait payé ses dettes, réparé ses brèches, acquitté les créances de Brême, fait face aux échéances de Saint-Malo. Il avait exonéré sa maison des Bravées des hypothèques qui la grevaient ; il avait racheté toutes les petites rentes locales inscrites sur cette maison. Il était possesseur d’un grand capital productif, la Durande. Le revenu net du navire était maintenant de mille livres sterling et allait croissant. A proprement parler, la Durande était toute sa fortune. Elle était aussi la fortune du pays. Le transport des bœufs étant un des plus gros bénéfices du navire, on avait dû, pour améliorer l’arrimage et faciliter l’entrée et la sortie des bestiaux, supprimer les portemanteaux et les deux canots. C’était peut-être une imprudence. La Durande n’avait plus qu’une embarcation, la chaloupe. La chaloupe, il est vrai, était excellente.\par
 Il s’était écoulé dix ans depuis le vol Rantaine.\par
Cette prospérité de la Durande avait un côté faible, c’est qu’elle n’inspirait point confiance ; on la croyait un hasard. La situation de mess Lethierry n’était acceptée que comme exception. Il passait pour avoir fait une folie heureuse. Quelqu’un qui l’avait imité à Cowes, dans l’île de Wight, n’avait pas réussi. L’essai avait ruiné ses actionnaires. Lethierry disait : C’est que la machine était mal construite. Mais on hochait la tête. Les nouveautés ont cela contre elles que tout le monde leur en veut ; le moindre faux pas les compromet. Un des oracles commerciaux de l’archipel normand, le banquier Jauge, de Paris, consulté sur une spéculation de bateaux à vapeur, avait, dit-on, répondu en tournant le dos : \emph{C’est une conversion que vous me proposez là. Conversion de l’argent en fumée}. En revanche, les bateaux à voiles trouvaient des commandites tant qu’ils en voulaient. Les capitaux s’obstinaient pour la toile contre la chaudière. A Guernesey, la Durande était un fait, mais la vapeur n’était pas un principe. Tel est l’acharnement de la négation en présence du progrès. On disait de Lethierry : \emph{C’est bon, mais il ne recommencerait pas}. Loin d’encourager, son exemple faisait peur. Personne n’eût osé risquer une deuxième Durande.
 \subsubsection[{A.IV.6. Chance qu’ont eue ces naufragés de rencontrer ce sloop}]{A.IV.6. \\
Chance qu’ont eue ces naufragés de rencontrer ce sloop}
\noindent L’équinoxe s’annonce de bonne heure dans la Manche. C’est une mer étroite qui gêne le vent et l’irrite. Dès le mois de février, il y a commencement de vents d’ouest, et toute la vague est secouée en tous sens. La navigation devient inquiète ; les gens de la côte regardent le mât de signal ; on se préoccupe des navires qui peuvent être en détresse. La mer apparaît comme un guet-apens ; un clairon invisible sonne on ne sait quelle guerre ; de grands coups d’haleine furieuse bouleversent l’horizon ; il fait un vent terrible. L’ombre siffle et souffle. Dans la profondeur des nuées la face noire de la tempête enfle ses joues.\par
Le vent est un danger ; le brouillard en est un autre.\par
Les brouillards ont été de tout temps craints des navigateurs. Dans certains brouillards sont en suspension des prismes microscopiques de glace auxquels  Mariotte attribue les halos, les parhélies et les parasélènes. Les brouillards orageux sont composites ; des vapeurs diverses, de pesanteur spécifique inégale, s’y combinent avec la vapeur d’eau, et se superposent dans un ordre qui divise la brume, en zones et fait du brouillard une véritable formation ; l’iode est en bas, le soufre au-dessus de l’iode, le brome au-dessus du soufre, le phosphore au-dessus du brome. Ceci, dans une certaine mesure, en faisant la part de la tension électrique et magnétique, explique plusieurs phénomènes, le feu Saint-Elme de Colomb et de Magellan, les étoiles volantes mêlées aux navires dont parle Sénèque, les deux flammes Castor et Pollux dont parle Plutarque, la légion romaine dont César crut voir les javelots prendre feu, la pique du château de Duino dans le Frioul que le soldat de garde faisait étinceler en la touchant du fer de sa lance, et peut-être même ces fulgurations d’en bas que les anciens appelaient « les éclairs terrestres de Saturne ». A l’équateur, une immense brume permanente semble nouée autour du globe, c’est le \emph{Cloud-ring,} l’anneau des nuages. Le Cloud-ring a pour fonction de refroidir le tropique, de même que le Gulf-stream a pour fonction de réchauffer le pôle. Sous le Cloud-ring, le brouillard est fatal. Ce sont les latitudes des chevaux, \emph{horse latitude ;} les navigateurs des derniers siècles jetaient là les chevaux à la mer, en temps d’orage pour s’alléger, en temps de calme pour économiser la provision d’eau. Colomb disait : \emph{Nube abaxo es muerte.} « Le nuage bas est la mort. » Les étrusques, qui sont pour la météorologie  ce que les chaldéens sont pour l’astronomie, avaient deux pontificats, le pontificat du tonnerre et le pontificat de la nuée ; les fulgurateurs observaient les éclairs et les aquiléges observaient le brouillard. Le collège des prêtres-augures de Tarquinies était consulté par les tyriens, les phéniciens, les pélasges, et tous les navigateurs primitifs de l’antique Marinterne. Le mode de génération des tempêtes était dès lors entrevu ; il est intimement lié au mode de génération des brouillards, et c’est, à proprement parler, le même phénomène. Il existe sur l’océan trois régions des brumes, une équatoriale, deux polaires ; les marins leur donnent un seul nom : \emph{le Pot au noir.}\par
Dans tous les parages et surtout dans la Manche, les brouillards d’équinoxe sont dangereux. Ils font brusquement la nuit sur la mer. Un des périls du brouillard, même quand il n’est pas très épais, c’est d’empêcher de reconnaître le changement de fond par le changement de couleur de l’eau ; il en résulte une dissimulation redoutable de l’approche des brisants et des bas-fonds. On est près d’un écueil sans que rien vous en avertisse. Souvent les brouillards ne laissent au navire en marche d’autre ressource que de mettre en panne ou de jeter l’ancre. Il y a autant de naufrages de brouillard que de vent.\par
Pourtant, après une bourrasque fort violente qui succéda à une de ces journées de brouillard, le sloop de poste \emph{Cashmere} arriva parfaitement d’Angleterre. Il entra à Saint-Pierre-Port au premier rayon du jour sortant de la mer, au moment même où le château Cornet  tirait son coup de canon au soleil. Le ciel s’était éclairci. Le sloop \emph{Cashmere} était attendu comme devant amener le nouveau recteur de Saint-Sampson. Peu après l’arrivée du sloop, le bruit se répandit dans la ville qu’il avait été accosté la nuit en mer par une chaloupe contenant un équipage naufragé.
 \subsubsection[{A.IV.7. Chance qu’a eue ce flaneur d’être aperçu par ce pêcheur}]{A.IV.7. \\
Chance qu’a eue ce flaneur d’être aperçu par ce pêcheur}
\noindent Cette nuit-là, Gilliatt, au moment où le vent avait molli, était allé pêcher, sans toutefois pousser la panse trop loin de la côte.\par
Comme il rentrait, à la marée montante, vers deux heures de l’après-midi, par un très beau soleil, en passant devant la Corne de la Bête pour gagner l’anse du Bû de la Rue, il lui sembla voir dans la projection de la chaise Gild-Holm-’Ur une ombre portée qui n’était pas celle du rocher. Il laissa arriver la panse de ce côté, et il reconnut qu’un homme était assis dans la chaise Gild-Holm-’Ur. La mer était déjà très haute, la roche était cernée par le flot, le retour n’était plus possible. Gilliatt fit à l’homme de grands gestes. L’homme resta immobile. Gilliatt approcha. L’homme était endormi.\par
Cet homme était vêtu de noir. — Cela a l’air d’un prêtre, pensa Gilliatt. Il approcha plus près encore, et vit un visage d’adolescent.\par
 Ce visage lui était inconnu.\par
La roche heureusement était à pic, il y avait beaucoup de fond, Gilliatt effaça et parvint à élonger la muraille. La marée soulevait assez la barque pour que Gilliatt en se haussant debout sur le bord de la panse pût atteindre aux pieds de l’homme. Il se dressa sur le bordage et éleva les mains. S’il fût tombé en ce moment-là, il est douteux qu’il eût reparu sur l’eau. La lame battait. Entre la panse et le rocher l’écrasement était inévitable.\par
Il tira le pied de l’homme endormi.\par
— Hé, que faites-vous là ?\par
L’homme se réveilla.\par
— Je regarde, dit-il.\par
Il se réveilla tout à fait et reprit :\par
— J’arrive dans le pays, je suis venu par ici en me promenant, j’ai passé la nuit en mer, j’ai trouvé la vue belle, j’étais fatigué, je me suis endormi.\par
— Dix minutes plus tard, vous étiez noyé, dit Gilliatt.\par
— Bah !\par
— Sautez dans ma barque.\par
Gilliatt maintint la barque du pied, se cramponna d’une main au rocher et tendit l’autre main à l’homme vêtu de noir, qui sauta lestement dans le bateau. C’était un très beau jeune homme.\par
Gilliatt prit l’aviron, et en deux minutes la panse arriva dans l’anse du Bû de la Rue.\par
Le jeune homme avait un chapeau rond et une cravate blanche. Sa longue redingote noire était boutonnée jusqu’à la cravate. Il avait des cheveux blonds  en couronne, le visage féminin, l’œil pur, l’air grave.\par
Cependant la panse avait touché terre. Gilliatt passa le câble dans l’anneau d’amarre, puis se tourna, et vit la main très blanche du jeune homme qui lui présentait un souverain d’or.\par
Gilliatt écarta doucement cette main.\par
Il y eut un silence. Le jeune homme le rompit.\par
— Vous m’avez sauvé la vie.\par
— Peut-être, répondit Gilliatt.\par
L’amarre était nouée. Ils sortirent de la barque.\par
Le jeune homme reprit :\par
— Je vous dois la vie, monsieur.\par
— Qu’est-ce que ça fait ?\par
Cette réponse de Gilliatt fut encore suivie d’un silence.\par
— Êtes-vous de cette paroisse ? demanda le jeune homme.\par
— Non, répondit Gilliatt.\par
— De quelle paroisse êtes-vous ?\par
Gilliatt leva la main droite, montra le ciel, et dit :\par
— De celle-ci.\par
Le jeune homme le salua et le quitta.\par
Au bout de quelques pas, le jeune homme s’arrêta, fouilla dans sa poche, en tira un livre, et revint vers Gilliatt en lui tendant le livre.\par
— Permettez-moi de vous offrir ceci.\par
Gilliatt prit le livre.\par
C’était une bible.\par
Un instant après, Gilliatt, accoudé sur son parapet, regardait le jeune homme tourner l’angle du sentier qui va à Saint-Sampson.\par
 Peu à peu il baissa la tête, oublia ce nouveau venu, ne sut plus si la chaise Gild-Holm-’Ur existait, et tout disparut pour lui dans l’immersion sans fond de la rêverie. Gilliatt avait un abîme, Déruchette.\par
Une voix qui l’appelait le tira de cette ombre.\par
— Hé, Gilliatt !\par
Il reconnut la voix et leva les yeux.\par
— Qu’y a-t-il, sieur Landoys ?\par
C’était en effet sieur Landoys qui passait sur la route à cent pas du Bû de la Rue dans son phiaton (phaéton) attelé de son petit cheval. Il s’était arrêté pour hêler Gilliatt, mais il semblait affairé et pressé.\par
— Il y a du nouveau, Gilliatt.\par
— Où ça ?\par
— Aux Bravées.\par
— Quoi donc ?\par
— Je suis trop loin pour vous conter cela.\par
Gilliatt frissonna.\par
— Est-ce que miss Déruchette se marie ?\par
— Non. Il s’en faut.\par
— Que voulez-vous dire ?\par
— Allez aux Bravées. Vous le saurez.\par
Et sieur Landoys fouetta son cheval.\par
  \subsection[{A.V. Livre cinquième. Le revolver}]{A.V. Livre cinquième \\
Le revolver}
  \subsubsection[{A.V.1. Les conversations de l’auberge jean}]{A.V.1. \\
Les conversations de l’auberge jean}
\noindent Sieur Clubin était l’homme qui attend une occasion.\par
Il était petit et jaune avec la force d’un taureau. La mer n’avait pu réussir à le hâler. Sa chair semblait de cire. Il était de la couleur d’un cierge et il en avait la clarté discrète dans les yeux. Sa mémoire était quelque chose d’imperturbable et de particulier. Pour lui, voir un homme une fois, c’était l’avoir, comme on a une note dans un registre. Ce regard laconique empoignait. Sa prunelle prenait une épreuve d’un visage et la gardait ; le visage avait beau vieillir, sieur Clubin le retrouvait. Impossible de dépister ce souvenir tenace. Sieur Clubin était bref, sobre, froid ; jamais un geste. Son air de candeur gagnait tout d’abord. Beaucoup de gens le croyaient naïf ; il avait au coin de l’œil un pli d’une bêtise étonnante. Pas de meilleur marin que lui, nous l’avons dit ; personne comme lui  pour amurer une voile, pour baisser le point du vent et pour maintenir avec l’écoute la voile orientée. Aucune réputation de religion et d’intégrité ne dépassait la sienne. Qui l’eût soupçonné eût été suspect. Il était lié d’amitié avec M. Rébuchet, changeur à Saint-Malo, rue Saint-Vincent, à côté de l’armurier, et M. Rébuchet disait : \emph{Je donnerais ma boutique à garder à Clubin.} Sieur Clubin était veuvier. Sa femme avait été l’honnête femme comme il était l’honnête homme. Elle était morte avec la renommée d’une vertu à tout rompre. Si le bailli lui eût conté fleurette, elle l’eût été dire au roi ; et si le bon Dieu eût été amoureux d’elle, elle l’eût été dire au curé. Ce couple, sieur et dame Clubin, avait réalisé dans Torteval l’idéal de l’épithète anglaise, \emph{respectâble.} Dame Clubin était le cygne ; sieur Clubin était l’hermine. Il fût mort d’une tache. Il n’eût pu trouver une épingle sans en chercher le propriétaire. Il eût tambouriné un paquet d’allumettes. Il était entré un jour dans un cabaret à Saint-Servan, et avait dit au cabaretier : J’ai déjeuné ici il y a trois ans, vous vous êtes trompé dans l’addition ; et il avait remboursé au cabaretier soixante-cinq centimes. C’était une grande probité, avec un pincement de lèvres attentif.\par
Il semblait en arrêt. Sur qui ? Sur les coquins probablement.\par
Tous les mardis il menait la Durande de Guernesey à Saint-Malo. Il arrivait à Saint-Malo le mardi soir, séjournait deux jours pour faire son chargement, et repartait pour Guernesey le vendredi matin.\par
 Il y avait alors à Saint-Malo une petite hôtellerie sur le port qu’on appelait l’Auberge Jean.\par
La construction des quais actuels a démoli cette auberge. A cette époque la mer venait mouiller la porte Saint-Vincent et la porte Dinan ; Saint-Malo et Saint-Servan communiquaient à marée basse par des carrioles et des maringottes roulant et circulant entre les navires à sec, évitant les bouées, les ancres et les cordages, et risquant parfois de crever leur capote de cuir à une basse vergue ou à une barre de clin-foc. Entre deux marées, les cochers houspillaient leurs chevaux sur ce sable où, six heures après, le vent fouettait le flot. Sur cette même grève rôdaient jadis les vingt-quatre dogues portiers de Saint-Malo, qui mangèrent un officier de marine en 1770. Cet excès de zèle les a fait supprimer. Aujourd’hui on n’entend plus d’aboiements nocturnes entre le petit Talard et le grand Talard.\par
Sieur Clubin descendait à l’Auberge Jean. C’est là qu’était le bureau français de la Durande.\par
Les douaniers et les gardes-côtes venaient prendre leurs repas et boire à l’Auberge Jean. Ils avaient leur table à part. Les douaniers de Binic se rencontraient là, utilement pour le service, avec les douaniers de Saint-Malo.\par
Des patrons de navires y venaient aussi, mais mangeaient à une autre table.\par
Sieur Clubin s’asseyait tantôt à l’une, tantôt à l’autre, plus volontiers pourtant à la table des douaniers qu’à celle des patrons. Il était bienvenu aux deux.\par
 Ces tables étaient bien servies. Il y avait des raffinements de boissons locales étrangères pour les marins dépaysés. Un matelot petit-maître de Bilbao y eût trouvé une helada. On y buvait du stout comme à Greenwich et de la gueuse brune comme à Anvers.\par
Des capitaines au long cours et des armateurs faisaient quelquefois figure à la mense des patrons. On y échangeait les nouvelles : — Où en sont les sucres ? — Cette douceur ne figure que pour de petits lots. Pourtant les bruts vont ; trois mille sacs de Bombay et cinq cents boucauts de Sagua. — Vous verrez que la droite finira par renverser Villèle. — Et l’indigo ? — On n’a traité que sept surons Guatemala. — La \emph{Nanine-Julie} est montée en rade. Joli trois-mâts de Bretagne. — Voilà encore les deux villes de la Plata en bisbille. — Quand Montevideo engraisse, Buenos-Ayres maigrit. — Il a fallu transborder le chargement du \emph{Regina-Cœli,} condamné au Callao. — Les cacaos marchent ; les sacs Caraques sont cotés deux cent trente-quatre et les sacs Trinidad soixante-treize. — Il paraît qu’à la revue du Champ de Mars on a crié : A bas les ministres ! — Les cuirs salés verts saladeros se vendent, les bœufs soixante francs et les vaches quarante-huit. — A-t-on passé le Balkan ? Que fait Diebitsch ? — A San Francisco l’anisette en pomponelles manque. L’huile d’olive Plagniol est calme. Le fromage de Gruyère en tins, trente-deux francs le quintal. — Eh bien, Léon XII est-il mort ? — Etc., etc.\par
Ces choses-là se criaient et se commentaient  bruyamment. A la table des douaniers et des gardes-côtes on parlait moins haut.\par
Les faits de police des côtes et des ports veulent moins de sonorité et moins de clarté dans le dialogue.\par
La table des patrons était présidée par un vieux capitaine au long cours, M. Gertrais-Gaboureau. M. Gertrais-Gaboureau n’était pas un homme, c’était un baromètre. Son habitude de la mer lui avait donné une surprenante infaillibilité de pronostic. Il décrétait le temps qu’il fera demain. Il auscultait le vent ; il tâtait le pouls à la marée. Il disait au nuage : montre-moi ta langue. C’est-à-dire l’éclair. Il était le docteur de la vague, de la brise, de la rafale. L’océan était son malade ; il avait fait le tour du monde comme on fait une clinique, examinant chaque climat dans sa bonne et mauvaise santé ; il savait à fond la pathologie des saisons. On l’entendait énoncer des faits comme ceci : — Le baromètre a descendu une fois, en 1796, à trois lignes au-dessous de tempête. — Il était marin par amour. Il haïssait l’Angleterre de toute l’amitié qu’il avait pour la mer. Il avait étudié soigneusement la marine anglaise pour en connaître le côté faible. Il expliquait en quoi le \emph{Sovereign} de 1637 différait du \emph{Royal William} de 1670 et de la \emph{Victory} de 1755. Il comparait les accastillages. Il regrettait les tours sur le pont et les hunes en entonnoir du \emph{Great Harry} de 1514, probablement au point de vue du boulet français, qui se logeait si bien dans ces surfaces. Les nations pour lui n’existaient que par leurs institutions  maritimes ; des synonymies bizarres lui étaient propres. Il désignait volontiers l’Angleterre par \emph{Trinity House}, l’Écosse par \emph{Northern commissionners}, et l’Irlande par \emph{Ballast Board.} Il abondait en renseignements ; il était alphabet et almanach ; il était étiage et tarif. Il savait par cœur le péage des phares, surtout des anglais ; un penny par tonne en passant devant celui-ci, un farthing devant celui-là. Il vous disait : \emph{Le phare de Small’s Rock, qui ne consommait que deux cents gallons d’huile, en brûle maintenant quinze cents gallons}. Un jour, à bord, dans une maladie grave qu’il fit, on le croyait mort, l’équipage entourait son branle, il interrompit les hoquets de l’agonie pour dire au maître charpentier : — Il serait avantageux d’adopter dans l’épaisseur des chouquets une mortaise de chaque côté pour y recevoir un réa en fonte ayant son essieu en fer et pour servir à passer les guinderesses. — De tout cela résultait une figure magistrale.\par
Il était rare que le sujet de conversation fût le même à la table des patrons et à la table des douaniers. Ce cas pourtant se présenta précisément dans les premiers jours de ce mois de février où nous ont amené les faits que nous racontons. Le trois-mâts \emph{Tamaulipas,} capitaine Zuela, venant du Chili et y retournant, appela l’attention des deux menses : A la mense des patrons on parla de son chargement, et à la mense des douaniers de ses allures.\par
Le capitaine Zuela, de Copiapo, était un chilien un peu colombien, qui avait fait avec indépendance les guerres de l’indépendance, tenant tantôt pour  Bolivar, tantôt pour Morillo, selon qu’il y trouvait son profit. Il s’était enrichi à rendre service à tout le monde. Pas d’homme plus bourbonien, plus bonapartiste, plus absolutiste, plus libéral, plus athée, et plus catholique. Il était de ce grand parti qu’on pourrait nommer le parti Lucratif. Il faisait de temps en temps en France des apparitions commerciales ; et, à en croire les ouï-dire, il donnait volontiers passage à son bord à des gens en fuite, banqueroutiers ou proscrits politiques, peu lui importait, payants. Son procédé d’embarquement était simple. Le fugitif attendait sur un point désert de la côte, et, au moment d’appareiller, Zuela détachait un canot, qui l’allait prendre. Il avait ainsi, à son précédent voyage, fait évader un contumace du procès Berton, et cette fois il comptait, disait-on, emmener des hommes compromis dans l’affaire de la Bidassoa. La police, avertie, avait l’œil sur lui.\par
Ces temps étaient une époque de fuites. La restauration était une réaction ; or les révolutions amènent des émigrations, et les restaurations entraînent des proscriptions. Pendant les sept ou huit premières années après la rentrée des Bourbons, la panique fut partout, dans la finance, dans l’industrie, dans le commerce, qui sentaient la terre trembler et où abondaient les faillites. Il y avait un sauve-qui-peut dans la politique. Lavalette avait pris la fuite, Lefebvre-Desnouettes avait pris la fuite, Delon avait pris la fuite. Les tribunaux d’exception sévissaient, plus Trestaillon. On fuyait le pont de Saumur, l’esplanade de la Réole, le mur de l’Observatoire de Paris, la tour de Taurias d’Avignon,  silhouettes lugubrement debout dans l’histoire, qu’a marquées la réaction, et où l’on distingue encore aujourd’hui cette main sanglante. A Londres le procès Thistlewood ramifié en France, à Paris le procès Trogoff, ramifié en Belgique, en Suisse et en Italie, avaient multiplié les motifs d’inquiétude et de disparition, et augmenté cette profonde déroute souterraine qui faisait le vide jusque dans les plus hauts rangs de l’ordre social d’alors. Se mettre en sûreté, tel était le souci. Être compromis, c’était être perdu. L’esprit des cours prévôtales avait survécu à l’institution. Les condamnations étaient de complaisance. On se sauvait au Texas, aux montagnes Rocheuses, au Pérou, au Mexique. Les hommes de la Loire, brigands alors, paladins aujourd’hui, avaient fondé le champ d’Asile. Une chanson de Béranger disait : \emph{Sauvages, nous sommes français ; prenez pitié de notre gloire}. S’expatrier était la ressource. Mais rien n’est moins simple que de fuir ; ce monosyllabe contient des abîmes. Tout fait obstacle à qui s’esquive. Se dérober implique se déguiser. Des personnes considérables, et même illustres, étaient réduites à des expédients de malfaiteurs. Et encore elles y réussissaient mal. Elles y étaient invraisemblables. Leur habitude de coudées franches rendait difficile leur glissement à travers les mailles de l’évasion. Un filou en rupture de ban était devant l’œil de la police plus correct qu’un général. S’imagine-t-on l’innocence contrainte à se grimer, la vertu contrefaisant sa voix, la gloire mettant un masque ? Tel passant à l’air suspect était une renommée en quête d’un faux passe-port.  Les allures louches de l’homme qui s’échappe ne prouvaient pas qu’on n’eût point devant les yeux un héros. Traits fugitifs et caractéristiques des temps, que l’histoire dite régulière néglige, et que le vrai peintre d’un siècle doit souligner. Derrière ces fuites d’honnêtes gens se faufilaient, moins surveillées et moins suspectes, les fuites des fripons. Un chenapan forcé de s’éclipser profitait du pêle-mêle, faisait nombre parmi les proscrits, et souvent, nous venons de le dire, grâce à plus d’art, semblait dans ce crépuscule plus honnête homme que l’honnête homme. Rien n’est gauche comme la probité reprise de justice. Elle n’y comprend rien et fait des maladresses. Un faussaire s’échappait plus aisément qu’un conventionnel.\par
Chose bizarre à constater, on pourrait presque dire, particulièrement pour les malhonnêtes gens, que l’évasion menait à tout. La quantité de civilisation qu’un coquin apportait de Paris ou de Londres lui tenait lieu de dot dans les pays primitifs ou barbares, le recommandait, et en faisait un initiateur. Cette aventure n’avait rien d’impossible d’échapper ici au code pour arriver là-bas au sacerdoce. Il y avait de la fantasmagorie dans la disparition, et plus d’une évasion a eu des résultats de rêve. Une fugue de ce genre conduisait à l’inconnu et au chimérique. Tel banqueroutier sorti d’Europe par ce trou à la lune a reparu vingt ans après grand vizir au Mogol ou roi en Tasmanie.\par
Aider aux évasions, c’était une industrie et, vu la fréquence du fait, une industrie à gros profits. Cette  spéculation complétait de certains commerces. Qui voulait se sauver en Angleterre s’adressait aux contrebandiers ; qui voulait se sauver en Amérique s’adressait à des fraudeurs de long cours tels que Zuela.
 \subsubsection[{A.V.2. Clubin aperçoit quelqu’un}]{A.V.2. \\
Clubin aperçoit quelqu’un}
\noindent Zuela venait quelquefois manger à l’auberge Jean. Sieur Clubin le connaissait de vue.\par
Du reste, sieur Clubin n’était pas fier ; il ne dédaignait pas de connaître de vue les chenapans. Il allait même quelquefois jusqu’à les connaître de fait, leur donnant la main en pleine rue et leur disant bonjour. Il parlait anglais au smogler et baragouinait l’espagnol avec le contrabandista. Il avait là-dessus des sentences : — On peut tirer du bien de la connaissance du mal. — Le garde-chasse cause utilement avec le braconnier. — Le pilote doit sonder le pirate ; le pirate étant un écueil. — Je goûte à un coquin comme un médecin goûte à un poison. — C’était sans réplique. Tout le monde donnait raison au capitaine Clubin. On l’approuvait de ne point être un délicat ridicule. Qui donc eût osé en médire ? Tout ce qu’il faisait était évidemment « pour le bien du service ». De lui tout était simple. Rien ne pouvait le compromettre. Le cristal  voudrait se tacher qu’il ne pourrait. Cette confiance était la juste récompense d’une longue honnêteté, et c’est là l’excellence des réputations bien assises. Quoi que fît ou quoi que semblât faire Clubin, on y entendait malice dans le sens de la vertu ; l’impeccabilité lui était acquise ; — par-dessus le marché, il était très avisé, disait-on ; — et de telle ou telle accointance qui dans un autre eût été suspecte, sa probité sortait avec un relief d’habileté. Ce renom d’habileté se combinait harmonieusement avec son renom de naïveté, sans contradiction ni trouble. Un naïf habile, cela existe. C’est une des variétés de l’honnête homme, et une des plus appréciées. Sieur Clubin était de ces hommes qui, rencontrés en conversation intime avec un escroc ou un bandit, sont acceptés ainsi, pénétrés, compris, respectés d’autant plus, et ont pour eux le clignement d’yeux satisfait de l’estime publique.\par
Le \emph{Tamaulipas} avait complété son chargement. Il était en partance et allait prochainement appareiller.\par
Un mardi soir la Durande arriva à Saint-Malo comme il faisait grand jour. Sieur Clubin, debout sur la passerelle et surveillant la manœuvre de l’approche du port, aperçut près du Petit-Bey, sur la plage de sable, entre deux rochers, dans un lieu très solitaire, deux hommes qui causaient. Il les visa de sa lunette marine, et reconnut l’un des deux hommes. C’était le capitaine Zuela. Il paraît qu’il reconnut aussi l’autre.\par
Cet autre était un personnage de haute taille, un peu grisonnant. Il portait le haut chapeau et le grave  vêtement des Amis. C’était probablement un quaker. Il baissait les yeux avec modestie.\par
En arrivant à l’Auberge Jean, sieur Clubin apprit que le \emph{Tamaulipas} comptait appareiller dans une dizaine de jours.\par
On a su depuis qu’il avait pris encore quelques autres informations.\par
A la nuit il entra chez l’armurier de la rue Saint-Vincent et lui dit :\par
— Savez-vous ce que c’est qu’un revolver ?\par
— Oui, répondit l’armurier, c’est américain.\par
— C’est un pistolet qui recommence la conversation.\par
— En effet, ça a la demande et la réponse.\par
— Et la réplique.\par
— C’est juste, monsieur Clubin. Un canon tournant.\par
— Et cinq ou six balles.\par
L’armurier entr’ouvrit le coin de sa lèvre et fit entendre ce bruit de langue qui, accompagné d’un hochement de tête, exprime l’admiration.\par
— L’arme est bonne, monsieur Clubin. Je crois qu’elle fera son chemin.\par
— Je voudrais un revolver à six canons.\par
— Je n’en ai pas.\par
— Comment ça, vous armurier ?\par
— Je ne tiens pas encore l’article. Voyez-vous, c’est nouveau. Ça débute. On ne fait encore en France que du pistolet.\par
— Diable !\par
— Ça n’est pas encore dans le commerce.\par
 — Diable !\par
— J’ai d’excellents pistolets.\par
— Je veux un revolver.\par
— Je conviens que c’est plus avantageux. Mais attendez donc, monsieur Clubin...\par
— Où ça ?\par
— Je crois savoir où. Je m’informerai.\par
— Quand pourrez-vous me rendre réponse ?\par
— D’occasion. Mais bon.\par
— Quand faut-il que je revienne ?\par
— Si je vous procure un revolver, c’est qu’il sera bon.\par
— Quand me rendrez-vous réponse ?\par
— A votre prochain voyage.\par
— Ne dites pas que c’est pour moi, dit Clubin.
 \subsubsection[{A.V.3. Clubin emporte et ne rapporte point}]{A.V.3. \\
Clubin emporte et ne rapporte point}
\noindent Sieur Clubin fit le chargement de la Durande, embarqua nombre de bœufs et quelques passagers, et quitta, comme à l’ordinaire, Saint-Malo pour Guernesey le vendredi matin.\par
Ce même jour vendredi, quand le navire fut au large, ce qui permet au capitaine de s’absenter quelques instants du pont de commandement, Clubin entra dans sa cabine, s’y enferma, prit un sac-valise qu’il avait, mit des vêtements dans le compartiment élastique, du biscuit, quelques boîtes de conserves, quelques livres de cacao en bâton, un chronomètre et une lunette marine dans le compartiment solide, cadenassa le sac, et passa dans les oreillons une aussière toute préparée pour le hisser au besoin. Puis il descendit dans la cale, entra dans la fosse aux câbles, et on le vit remonter avec une de ces cordes à nœuds armées d’un crampon qui servent aux calfats sur mer et aux voleurs sur terre. Ces cordes facilitent les escalades.\par
 Arrivé à Guernesey, Clubin alla à Torteval. Il y passa trente-six heures. Il y emporta le sac-valise et la corde à nœuds, et ne les rapporta pas.\par
Disons-le une fois pour toutes, le Guernesey dont il est question dans ce livre, c’est l’ancien Guernesey, qui n’existe, plus et qu’il serait impossible de retrouver aujourd’hui, ailleurs que dans les campagnes. Là il est encore vivant, mais il est mort dans les villes. La remarque que nous faisons pour Guernesey doit être aussi faite pour Jersey. Saint-Hélier vaut Dieppe ; Saint-Pierre-Port vaut Lorient. Grâce au progrès, grâce à l’admirable esprit d’initiative de ce vaillant petit peuple insulaire, tout s’est transformé depuis quarante ans dans l’archipel de la Manche. Où il y avait de l’ombre, il y a de la lumière. Cela dit, passons.\par
En ces temps qui sont déjà, par l’éloignement, des temps historiques, la contrebande était très active dans la Manche. Les navires fraudeurs abondaient particulièrement sur la côte ouest de Guernesey. Les personnes renseignées à outrance, et qui savent dans les moindres détails ce qui se passait il y a tout à l’heure un demi-siècle, vont jusqu’à citer les noms de plusieurs de ces navires, presque tous asturiens et guiposcoans. Ce qui est hors de doute, c’est qu’il ne s’écoulait guère de semaine sans qu’il en vînt un ou deux, soit dans la baie des Saints, soit à Plainmont. Cela avait presque les allures d’un service régulier. Une cave de la mer à Serk s’appelait et s’appelle encore les Boutiques, parce que c’était dans cette grotte qu’on  venait acheter aux fraudeurs leurs marchandises. Pour les besoins de ces commerces, il se parlait dans la Manche une espèce de langue contrebandière, oubliée aujourd’hui, et qui était à l’espagnol ce que le levantin est à l’italien.\par
Sur beaucoup de points du littoral anglais et français, la contrebande était en cordiale entente secrète avec le négoce patent et patenté. Elle avait ses entrées chez plus d’un haut financier, par la porte dérobée, il est vrai ; et elle fusait souterrainement dans la circulation commerciale et dans tout le système veineux de l’industrie. Négociant par devant, contrebandier par derrière ; c’était l’histoire de beaucoup de fortunes. Séguin le disait de Bourgain ; Bourgain le disait de Séguin. Nous ne nous faisons point garant de leurs paroles ; peut-être se calomniaient-ils l’un et l’autre. Quoi qu’il en fût, la contrebande, traquée par la loi, était incontestablement fort bien apparentée dans la finance. Elle était en rapport avec « le meilleur monde ». Cette caverne, où Mandrin coudoyait jadis le comte de Charolais, était honnête au dehors, et avait une façade irréprochable sur la société ; pignon sur rue.\par
De là beaucoup de connivences, nécessairement masquées. Ces mystères voulaient une ombre impénétrable. Un contrebandier savait beaucoup de choses et devait les taire ; une foi inviolable et rigide était sa loi. La première qualité d’un fraudeur était la loyauté. Sans discrétion pas de contrebande. Il y avait le secret de la fraude comme il y a le secret de la confession.\par
 Ce secret était imperturbablement gardé. Le contrebandier jurait de tout taire, et tenait parole. On ne pouvait se fier à personne mieux qu’à un fraudeur. Le juge-alcade d’Oyarzun prit un jour un contrebandier des Ports secs, et le fit mettre à la question pour le forcer à nommer son bailleur de fonds secret. Le contrebandier ne nomma point le bailleur de fonds. Ce bailleur de fonds était le juge-alcade. De ces deux complices, le juge et le contrebandier, l’un avait dû, pour obéir aux yeux de tous à la loi, ordonner la torture, à laquelle l’autre avait résisté, pour obéir à son serment.\par
Les deux plus fameux contrebandiers hantant Plainmont à cette époque étaient Blasco et Blasquito. Ils étaient tocayos. C’est une parenté espagnole et catholique qui consiste à avoir le même patron dans le paradis, chose, on en conviendra, non moins digne de considération que d’avoir le même père sur la terre.\par
Quand on était à peu près au fait du furtif itinéraire de la contrebande, parler à ces hommes, rien n’était plus facile et plus difficile. Il suffisait de n’avoir aucun préjugé nocturne, d’aller à Plainmont et d’affronter le mystérieux point d’interrogation qui se dresse là.
 \subsubsection[{A.V.4. Plainmont}]{A.V.4. \\
Plainmont}
\noindent Plainmont, près Torteval, est un des trois angles de Guernesey. Il y a là, à l’extrémité du cap, une haute croupe de gazon qui domine la mer.\par
Ce sommet est désert.\par
Il est d’autant plus désert qu’on y voit une maison.\par
Cette maison ajoute l’effroi à la solitude.\par
Elle est, dit-on, visionnée.\par
Hantée ou non, l’aspect en est étrange.\par
Cette maison, bâtie en granit et élevée d’un étage, est au milieu de l’herbe. Elle n’a rien d’une ruine. Elle est parfaitement habitable. Les murs sont épais et le toit est solide. Pas une pierre ne manque aux murailles, pas une tuile, au toit. Une cheminée de brique contrebute l’angle du toit. Cette maison tourne le dos à la mer. Sa façade du côté de l’océan n’est qu’une muraille. En examinant attentivement cette façade, on y distingue une fenêtre, murée. Les deux pignons offrent trois lucarnes, une à l’est, deux à l’ouest, murées  toutes trois. La devanture qui fait face à la terre a seule une porte et des fenêtres. La porte est murée. Les deux fenêtres du rez-de-chaussée sont murées. Au premier étage, et c’est là ce qui frappe tout d’abord quand on approche, il y a deux fenêtres ouvertes ; mais les fenêtres murées sont moins farouches que ces fenêtres ouvertes. Leur ouverture les fait noires en plein jour. Elles n’ont pas de vitres, pas même de châssis. Elles s’ouvrent sur l’ombre du dedans. On dirait les trous vides de deux yeux arrachés. Rien dans cette maison. On aperçoit par les croisées béantes le délabrement intérieur. Pas de lambris, nulle boiserie, la pierre nue. On croit voir un sépulcre à fenêtre permettant aux spectres de regarder dehors. Les pluies affouillent les fondations du côté de la mer. Quelques orties agitées par le vent caressent le bas des murs. A l’horizon, aucune habitation humaine. Cette maison est une chose vide où il y a le silence. Si l’on s’arrête pourtant et [{\corr si}] l’on colle son oreille à la muraille, on y entend confusément par instants des battements d’ailes effarouchées. Au-dessus de la porte murée, sur la pierre qui fait l’architrave, sont gravées ces lettres : ELM-PBILG, et cette date : 1780.\par
La nuit, la lune lugubre entre là.\par
Toute la mer est autour de cette maison. Sa situation est magnifique, et par conséquent sinistre. La beauté du lieu devient une énigme. Pourquoi aucune famille humaine n’habite-t-elle ce logis ? La place est belle, la maison est bonne. D’où vient cet abandon ? Aux questions de la raison s’ajoutent les questions de  la rêverie. Ce champ est cultivable, d’où vient qu’il est inculte ? Pas de maître. La porte murée. Qu’a donc ce lieu ? pourquoi l’homme en fuite ? que se passe-t-il ici ? S’il ne s’y passe rien, pourquoi n’y a-t-il ici personne ? Quand tout est endormi, y a-t-il ici quelqu’un d’éveillé ? La rafale ténébreuse, le vent, les oiseaux de proie, les bêtes cachées, les êtres ignorés, apparaissent à la pensée et se mêlent à cette maison. De quels passants est-elle l’hôtellerie ? On se figure des ténèbres de grêle et de pluie s’engouffrant dans les fenêtres. De vagues ruissellements de tempêtes ont laissé leurs traces sur la muraille intérieure. Ces chambres murées et ouvertes sont visitées par l’ouragan. S’est-il commis un crime là ? Il semble que, la nuit, cette maison livrée à l’ombre doit appeler au secours. Reste-t-elle muette ? en sort-il des voix ? A qui a-t-elle affaire dans cette solitude ? Le mystère des heures noires est à l’aise ici. Cette maison est inquiétante à midi ; qu’est-elle à minuit ? En la regardant, on regarde un secret. On se demande, la rêverie ayant sa logique et le possible ayant sa pente, ce que devient cette maison entre le crépuscule du soir et le crépuscule du matin. L’immense dispersion de la vie extra-humaine a-t-elle sur ce sommet désert un nœud où elle s’arrête et qui la force à devenir visible et à descendre ? l’épars vient-il y tourbillonner ? l’impalpable s’y condense-t-il jusqu’à prendre forme ? Énigmes. L’horreur sacrée est dans ces pierres. Cette ombre qui est dans ces chambres défendues est plus que de l’ombre ; c’est de l’inconnu. Après le soleil couché, les bateaux pêcheurs  rentreront, les oiseaux se tairont, le chevrier qui est derrière le rocher s’en ira avec ses chèvres, les entre-deux des pierres livreront passage aux premiers glissements des reptiles rassurés, les étoiles commenceront à regarder, la bise soufflera, le plein de l’obscurité se fera, ces deux fenêtres seront là, béantes. Cela s’ouvre aux songes ; et c’est par des apparitions, par des larves, par des faces de fantômes vaguement distinctes, par des masques dans les lueurs, par de mystérieux tumultes d’âmes et d’ombres, que la croyance populaire, à la fois stupide et profonde, traduit les sombres intimités de cette demeure avec la nuit.\par
La maison est « visionnée » ; ce mot répond à tout.\par
Les esprits crédules ont leur explication ; mais les esprits positifs ont aussi la leur. Rien de plus simple, disent-ils, que cette maison. C’est un ancien poste d’observation, du temps des guerres de la révolution et de l’empire, et des contrebandes. Elle a été bâtie là pour cela. La guerre finie, le poste a été abandonné. On n’a pas démoli la maison parce qu’elle peut redevenir utile. On a muré la porte et les fenêtres du rez-de-chaussée contre les stercoraires humains, et pour que personne n’y pût entrer ; on a muré les fenêtres des trois côtés sur la mer, à cause du vent du sud et du vent d’ouest. Voilà tout.\par
Les ignorants et les crédules insistent. D’abord, la maison n’a pas été bâtie à l’époque des guerres de la révolution. Elle porte la date — 1780 — antérieure à la révolution. Ensuite, elle n’a pas été bâtie pour être un poste ; elle porte les lettres ELM-PBILG, qui  sont le double monogramme de deux familles, et qui indiquent, suivant l’usage, que la maison a été construite pour l’établissement d’un jeune ménage. Donc, elle a été habitée. Pourquoi ne l’est-elle plus ? Si l’on a muré la porte et les croisées pour que personne ne pût pénétrer dans la maison, pourquoi a-t-on laissé deux fenêtres ouvertes ? Il fallait tout murer, ou rien. Pourquoi pas de volets ? pourquoi pas de châssis ? pourquoi pas de vitres ? pourquoi murer les fenêtres d’un côté si on ne les mure pas de l’autre ? On empêche la pluie d’entrer par le sud, mais on la laisse entrer par le nord.\par
Les crédules ont tort, sans doute, mais à coup sûr les positifs n’ont pas raison. Le problème persiste.\par
Ce qui est sûr, c’est que la maison passe pour avoir été plutôt utile que nuisible aux contrebandiers.\par
Le grossissement de l’effroi ôte aux faits leur vraie proportion. Sans nul doute, bien des phénomènes nocturnes, parmi ceux dont s’est peu à peu composé le « visionnement » de la masure, pourraient s’expliquer par des présences obscures et furtives, par de courtes stations d’hommes tout de suite rembarqués, tantôt par les précautions, tantôt par les hardiesses de certains industriels suspects se cachant pour mal faire et se laissant entrevoir pour faire peur.\par
A cette époque déjà lointaine beaucoup d’audaces étaient possibles. La police, surtout dans les petits pays, n’était pas ce qu’elle est aujourd’hui.\par
Ajoutons que, si cette masure était, comme on le dit, commode aux fraudeurs, leurs rendez-vous devaient avoir là jusqu’à un certain point leurs coudées  franches, précisément parce que la maison était mal vue. Être mal vue l’empêchait d’être dénoncée. Ce n’est guère aux douaniers et aux sergents qu’on s’adresse contre les spectres. Les superstitieux font des signes de croix et non des procès-verbaux. Ils voient ou croient voir, s’enfuient et se taisent. Il existe une connivence tacite, non voulue, mais réelle, entre ceux qui font peur et ceux qui ont peur. Les effrayés se sentent dans leur tort d’avoir été effrayés, ils s’imaginent avoir surpris un secret, ils craignent d’aggraver leur position, mystérieuse pour eux-mêmes, et de fâcher les apparitions. Ceci les rend discrets. Et, même en dehors de ce calcul, l’instinct des gens crédules est le silence ; il y a du mutisme dans l’épouvante ; les terrifiés parlent peu ; il semble que l’horreur dise : chut !\par
Il faut se souvenir que ceci remonte à l’époque où les paysans guernesiais croyaient que le mystère de la Crèche était, tous les ans, à jour fixe, répété par les bœufs et les ânes ; époque où personne, dans la nuit de Noël, n’eût osé pénétrer dans une étable, de peur d’y trouver les bêtes à genoux.\par
S’il faut ajouter foi aux légendes locales et aux récits des gens qu’on rencontre, la superstition autrefois a été quelquefois jusqu’à suspendre aux murs de cette maison de Plainmont, à des clous dont on voit encore çà et là la trace, des rats sans pattes, des chauves-souris sans ailes, des carcasses de bêtes mortes, des crapauds écrasés entre les pages d’une bible, des brins de lupin jaune, étranges ex-voto, accrochés là par d’imprudents passants nocturnes qui avaient cru voir  quelque chose, et qui, par ces cadeaux, espéraient obtenir leur pardon, et conjurer la mauvaise humeur des stryges, des larves et des brucolaques. Il y a eu de tout temps des crédules aux abacas et aux sabbats, et même d’assez haut placés. César consultait Sagane, et Napoléon mademoiselle Lenormand. Il est des consciences inquiètes jusqu’à tâcher d’obtenir des indulgences du diable. \emph{Que Dieu fasse et que Satan ne défasse pas !} c’était là une des prières de Charles-Quint. D’autres esprits sont plus timorés encore. Ils vont jusqu’à se persuader qu’on peut avoir des torts envers le mal. Être irréprochable vis-à-vis du démon, c’est une de leurs préoccupations. De là des pratiques religieuses tournées vers l’immense malice obscure. C’est un bigotisme comme un autre. Les crimes contre le démon existent dans certaines imaginations malades ; avoir violé la loi d’en bas tourmente de bizarres casuistes de l’ignorance ; on a des scrupules du côté des ténèbres. Croire à l’efficacité de la dévotion aux mystères du Brocken et d’Armuyr, se figurer qu’on a péché contre l’enfer, avoir recours pour des infractions chimériques à des pénitences chimériques, avouer la vérité à l’esprit de mensonge, faire son meâ culpâ devant le père de la Faute, se confesser en sens inverse, tout cela existe ou a existé ; les procès de magie le prouvent à chaque page de leurs dossiers. Le songe humain va jusque-là. Quand l’homme se met à s’effarer, il ne s’arrête point. On rêve des fautes imaginaires, on rêve des purifications imaginaires, et l’on fait faire le nettoyage de sa conscience par l’ombre du balai des sorcières.\par
 Quoi qu’il en soit, si cette maison a des aventures, c’est son affaire ; à part quelques hasards et quelques exceptions, nul n’y va voir, elle est laissée seule ; il n’est du goût de personne de se risquer aux rencontres infernales.\par
Grâce à la terreur qui la garde, et qui en éloigne quiconque pourrait observer et témoigner, il a été de tout temps facile de s’introduire la nuit dans cette maison, au moyen d’une échelle de corde, ou même tout simplement du premier escalier venu pris aux courtils voisins. Un en-cas de hardes et de vivres apporté là permettrait d’y attendre en toute sécurité l’éventualité et l’à-propos d’un embarquement furtif. La tradition raconte qu’il y a une quarantaine d’années un fugitif, de la politique selon les uns, du commerce selon les autres, a séjourné quelque temps caché dans la maison visionnée de Plainmont, d’où il a réussi à s’embarquer sur un bateau pêcheur pour l’Angleterre. D’Angleterre on gagne aisément l’Amérique.\par
Cette même tradition affirme que des provisions déposées dans cette masure y demeurent sans qu’on y touche ; Lucifer, comme les contrebandiers, ayant intérêt à ce que celui qui les a mises là revienne.\par
Du sommet où est cette maison, on aperçoit au sud-ouest, à un mille de la côte, l’écueil des Hanois.\par
Cet écueil est célèbre. Il a fait toutes les mauvaises actions que peut faire un rocher. C’était un des plus redoutables assassins de la mer. Il attendait en traître les navires dans la nuit. Il a élargi les cimetières de Torteval et de la Rocquaine.\par
 En 1862 on a placé sur cet écueil un phare.\par
Aujourd’hui l’écueil des Hanois éclaire la navigation qu’il fourvoyait ; le guet-apens a un flambeau à la main. On cherche à l’horizon comme un protecteur et un guide ce rocher qu’on fuyait comme un malfaiteur. Les Hanois rassurent ces vastes espaces nocturnes qu’ils effrayaient. C’est quelque chose comme le brigand devenu gendarme.\par
Il y a trois Hanois, le grand Hanois, le petit Hanois, et la Mauve. C’est sur le petit Hanois qu’est aujourd’hui le « Light Red ».\par
Cet écueil fait partie d’un groupe de pointes, quelques-unes sous-marines, quelques-unes sortant de la mer. Il les domine. Il a, comme une forteresse, ses ouvrages avancés ; du côté de la haute mer, un cordon de treize rochers ; au nord, deux brisants, les Hautes-Fourquies, les Aiguillons, et un banc de sable, l’Hérouée ; au sud, trois rochers, le Cat-Rock, la Percée et la Roque Herpin ; plus deux boues, la South Boue et la Boue le Mouet, et en outre, devant Plainmont, à fleur d’eau, le Tas de Pois d’Aval.\par
Qu’un nageur franchisse le détroit des Hanois à Plainmont, cela est malaisé, non impossible. On se souvient que c’était une des prouesses de sieur Clubin. Le nageur qui connaît ces bas-fonds a deux stations où il peut se reposer, la Roque ronde, et plus loin, en obliquant un peu à gauche, la Roque rouge.
 \subsubsection[{A.V.5. Les déniquoiseaux}]{A.V.5. \\
Les déniquoiseaux}
\noindent C’est à peu près vers cette journée de samedi, passée par sieur Clubin à Torteval, qu’il faut rapporter un fait singulier, peu ébruité d’abord dans le pays, et qui ne transpira que longtemps après. Car beaucoup de choses, nous venons de le remarquer, restent inconnues à cause même de l’effroi qu’elles ont fait à ceux qui en ont été témoins.\par
Dans la nuit du samedi au dimanche, nous précisons la date et nous la croyons exacte, trois enfants escaladèrent l’escarpement de Plainmont. Ces enfants s’en retournaient au village. Ils venaient de la mer. C’était ce qu’on appelle dans la langue locale des \emph{déniquoiseaux.} Lisez déniche-oiseaux. Partout où il y a des falaises et des trous de rochers au-dessus de la mer, les enfants dénicheurs d’oiseaux abondent. Nous en avons dit un mot déjà. On se souvient que Gilliatt s’en préoccupait, à cause des oiseaux et à cause des enfants.\par
 Les deniquoiseaux sont des espèces de gamins de l’océan, peu timides.\par
La nuit était très obscure. D’épaisses superpositions de nuées cachaient le zénith. Trois heures du matin venaient de sonner au clocher de Torteval, qui est rond et pointu et qui ressemble à un bonnet de magicien.\par
Pourquoi ces enfants revenaient-ils si tard ? Rien de plus simple. Ils étaient allés à la chasse aux nids de mauves, dans le Tas de Pois d’Aval. La saison ayant été très douce, les amours des oiseaux commençaient de très bonne heure. Ces enfants, guettant les allures des mâles et des femelles autour des gîtes, et distraits par l’acharnement de cette poursuite, avaient oublié l’heure. Le flux les avait cernés ; ils n’avaient pu regagner à temps la petite anse où ils avaient amarré leur canot, et ils avaient dû attendre sur une des pointes du Tas de Pois que la mer se retirât. De là leur rentrée nocturne. Ces rentrées-là sont attendues par la fiévreuse inquiétude des mères, laquelle, rassurée, dépense sa joie en colère, et, grossie dans les larmes, se dissipe en taloches. Aussi se hâtaient-ils, assez inquiets. Ils avaient cette manière de se hâter qui s’attarderait volontiers, et qui contient un certain désir de ne pas arriver. Ils avaient en perspective un embrassement compliqué de [{\corr giffles}].\par
Un seul de ces enfants n’avait rien à craindre, c’était un orphelin. Ce garçon était français, sans père ni mère, et content en cette minute-là de n’avoir pas de mère. Personne ne s’intéressant à lui, il ne serait  pas battu. Les deux autres étaient guernesiais, et de la paroisse même de Torteval.\par
La haute croupe de roches escaladée, les trois déniquoiseaux parvinrent sur le plateau où est la maison visionnée.\par
Ils commencèrent par avoir peur, ce qui est le devoir de tout passant, et surtout de tout enfant, à cette heure et dans ce lieu.\par
Ils eurent bien envie de se sauver à toutes jambes, et bien envie de s’arrêter pour regarder.\par
Ils s’arrêtèrent.\par
Ils regardèrent la maison.\par
Elle était toute noire et formidable.\par
C’était, au milieu du plateau désert, un bloc obscur, une [{\corr excroissance}] symétrique et hideuse, une haute masse carrée à angles rectilignes, quelque chose de semblable à un énorme autel de ténèbres.\par
La première pensée des enfants avait été de s’enfuir ; la seconde fut de s’approcher. Ils n’avaient jamais vu cette maison-là à cette heure-là. La curiosité d’avoir peur existe. Ils avaient un petit français avec eux, ce qui fit qu’ils approchèrent.\par
On sait que les français ne croient à rien.\par
D’ailleurs, être plusieurs dans un danger, rassure ; avoir peur à trois, encourage.\par
Et puis, on est chasseur, on est enfant ; à trois qu’on est, on n’a pas trente ans ; on est en quête, on fouille, on épie les choses cachées ; est-ce pour s’arrêter en chemin ? on avance la tête dans ce trou-ci, comment ne point l’avancer dans ce trou-là ? Qui est  en chasse subit un entraînement ; qui va à la découverte est dans un engrenage. Avoir tant regardé dans le nid des oiseaux, cela donne la démangeaison de regarder un peu dans le nid des spectres. Fureter dans l’enfer ; pourquoi pas ?\par
De gibier en gibier, on arrive au démon. Après les moineaux, les farfadets. On va savoir à quoi s’en tenir sur toutes ces peurs que vos parents vous ont faites. Être sur la piste des contes bleus, rien n’est plus glissant. En savoir aussi long que les bonnes femmes, cela tente.\par
Tout ce pêle-mêle d’idées, à l’état de confusion et d’instinct dans la cervelle des déniquoiseaux guernesiais, eut pour résultante leur témérité. Ils marchèrent vers la maisen.\par
Du reste, le petit qui leur servait de point d’appui dans cette bravoure en était digne. C’était un garçon résolu, apprenti calfat, de ces enfants déjà hommes, couchant au chantier sur de la paille dans un hangar, gagnant sa vie, ayant une grosse voix, grimpant volontiers aux murs et aux arbres, sans préjugés vis-à-vis des pommes près desquelles il passait, ayant travaillé à des radoubs de vaisseaux de guerre, fils du hasard, enfant de raccroc, orphelin gai, né en France, et on ne savait où, deux raisons pour être hardi, ne regardant pas à donner un double à un pauvre, très méchant, très bon, blond jusqu’au roux, ayant parlé à des parisiens. Pour le moment, il gagnait un chelin par jour à calfater des barques de poissonniers, en réparation aux Pêqueries. Quand l’envie lui en prenait,  il se donnait des vacances, et allait dénicher des oiseaux. Tel était le petit français.\par
La solitude du lieu avait on ne sait quoi de funèbre. On sentait là l’inviolabilité menaçante. C’était farouche. Ce plateau, silencieux et nu, dérobait à très courte distance dans le précipice sa courbe déclive et fuyante. La mer en bas se taisait. Il n’y avait point de vent. Les brins d’herbe ne bougeaient pas.\par
Les petits déniquoiseaux avançaient à pas lents, l’enfant français en tête, en regardant la maison.\par
L’un d’eux, plus tard, en racontant le fait, ou l’à peu près qui lui en était resté, ajoutait : « Elle ne disait rien. »\par
Ils s’approchaient en retenant leur haleine, comme on approcherait d’une bête.\par
Ils avaient gravi le roidillon qui est derrière la maison et qui aboutit du côté de la mer à un petit isthme de rochers peu praticable ; ils étaient parvenus assez près de la masure ; mais ils ne voyaient que la façade sud, qui est toute murée ; ils n’avaient pas osé tourner à gauche, ce qui les eût exposés à voir l’autre façade où il y a deux fenêtres, ce qui est terrible.\par
Cependant ils s’enhardirent, l’apprenti calfat leur ayant dit tout bas : — Virons à bâbord. C’est ce côté-là qui est le beau. Il faut voir les deux fenêtres noires.\par
Ils « virèrent à bâbord » et arrivèrent de l’autre côté de la maison.\par
Les deux fenêtres étaient éclairées.\par
Les enfants s’enfuirent.\par
Quand ils furent loin, le petit français se retourna.\par
 — Tiens, dit-il, il n’y a plus de lumière.\par
En effet, il n’y avait plus de clarté aux fenêtres. La silhouette de la masure se dessinait, découpée comme à l’emporte-pièce, sur la lividité diffuse du ciel.\par
La peur ne s’en alla point, mais la curiosité revint. Les déniquoiseaux se rapprochèrent.\par
Brusquement, aux deux fenêtres à la fois, la lumière se refit.\par
Les deux gars de Torteval reprirent leurs jambes à leur cou, et se sauvèrent. Le petit satan de français n’avança pas, mais ne recula pas.\par
Il demeura immobile, faisant face à la maison, et la regardant.\par
La clarté s’éteignit, puis brilla de nouveau. Rien de plus horrible. Le reflet faisait une vague traînée de feu sur l’herbe mouillée par la buée de la nuit. A un certain moment, la lueur dessina sur le mur intérieur de la masure de grands profils noirs qui remuaient et des ombres de têtes énormes.\par
Du reste, la masure étant sans plafonds ni cloisons et n’ayant plus que les quatre murs et le toit, une fenêtre ne peut pas être éclairée sans que l’autre le soit.\par
Voyant que l’apprenti calfat restait, les deux autres déniquoiseaux revinrent, pas à pas, l’un après l’autre, tremblants, curieux. L’apprenti calfat leur dit tout bas : — Il y a des revenants dans la maison. J’ai vu le nez d’un. — Les deux petits de Torteval se blottirent derrière le français, et haussés sur la pointe du pied, par-dessus son épaule, abrités par lui, le prenant pour  bouclier, l’opposant à la chose, rassurés de le sentir entre eux et la vision, ils regardèrent, eux aussi.\par
La masure, de son côté, semblait les regarder. Elle avait, dans cette vaste obscurité muette, deux prunelles rouges. C’étaient les fenêtres. La lumière s’éclipsait, reparaissait, s’éclipsait encore, comme font ces lumières-là. Ces intermittences sinistres tiennent probablement au va-et-vient de l’enfer. Cela s’entr’ouvre, puis se referme. Le soupirail du sépulcre a des effets de lanterne sourde.\par
Tout à coup une noirceur très opaque ayant la forme humaine se dressa sur l’une des fenêtres comme si elle venait du dehors, puis s’enfonça dans l’intérieur de la maison. Il sembla que quelqu’un venait d’entrer.\par
Entrer par la croisée, c’est l’habitude des voleurs.\par
La clarté fut un moment plus vive, puis s’éteignit et ne reparut plus. La maison redevint noire. Alors il en sortit des bruits. Ces bruits ressemblaient à des voix. C’est toujours comme cela. Quand on voit, on n’entend pas ; quand on ne voit pas, on entend.\par
La nuit sur la mer a une taciturnité particulière. Le silence de l’ombre est là plus profond qu’ailleurs. Lorsqu’il n’y a ni vent ni flot, dans cette remuante étendue où d’ordinaire on n’entend pas voler les aigles, on entendrait une mouche voler. Cette paix sépulcrale donnait un relief lugubre aux bruits qui sortaient de la masure.\par
— Voyons voir, dit le petit français.\par
Et il fit un pas vers la maison.\par
 Les deux autres avaient une telle peur qu’ils se décidèrent à le suivre. Ils n’osaient plus s’enfuir tout seuls.\par
Comme ils venaient de dépasser un assez gros tas de fagots qui, on ne sait pas pourquoi, les rassurait dans cette solitude, une chevêche s’envola d’un buisson. Cela fit un froissement de branches. Les chevêches ont une espèce de vol louche, d’une obliquité inquiétante. L’oiseau passa de travers près des enfants, en fixant sur eux la rondeur de ses yeux, clairs dans la nuit.\par
Il y eut un certain tremblement dans le groupe derrière le petit français.\par
Il apostropha la chevêche.\par
— Moineau, tu viens trop tard. Il n’est plus temps. Je veux voir.\par
Et il avança.\par
Le craquement de ses gros souliers cloutés sur les ajoncs n’empêchait pas d’entendre les bruits de la masure, qui s’élevaient et s’abaissaient avec l’accentuation calme et la continuité d’un dialogue.\par
Un moment après, il ajouta :\par
— D’ailleurs il n’y a que les bêtes qui croient aux revenants.\par
L’insolence dans le danger rallie les traînards et les pousse en avant.\par
Les deux gars de Torteval se remirent en marche, emboîtant le pas à la suite de l’apprenti calfat.\par
La maison visionnée leur faisait l’effet de grandir démesurément. Dans cette illusion d’optique de la peur  il y avait de la réalité. La maison grandissait en effet, parce qu’ils en approchaient.\par
Cependant les voix qui étaient dans la maison prenaient une saillie de plus en plus nette. Les enfants écoutaient. L’oreille aussi a ses grossissements. C’était autre chose qu’un murmure, plus qu’un chuchotement, moins qu’un brouhaha. Par instants une ou deux paroles clairement articulées se détachaient. Ces paroles, impossibles à comprendre, sonnaient bizarrement. Les enfants s’arrêtaient, écoutaient, puis recommençaient à avancer.\par
— C’est la conversation des revenants, murmura l’apprenti calfat, mais je ne crois pas aux revenants.\par
Les petits de Torteval étaient bien tentés de se replier derrière le tas de fagots ; mais ils en étaient déjà loin, et leur ami le calfat continuait de marcher vers la masure. Ils tremblaient de rester avec lui, et ils n’osaient pas le quitter.\par
Pas à pas, et perplexes, ils le suivaient.\par
L’apprenti calfat se tourna vers eux et leur dit :\par
— Vous savez que ce n’est pas vrai. Il n’y en a pas.\par
La maison devenait de plus en plus haute. Les voix devenaient de plus en plus distinctes.\par
Ils approchaient.\par
En approchant, on reconnaissait qu’il y avait dans la maison quelque chose comme de la lumière étouffée. C’était une lueur très vague, un de ces effets de lanterne sourde indiqués tout à l’heure, et qui abondent dans l’éclairage des sabbats.\par
Quand ils furent tout près, ils firent halte.\par
 Un des deux de Torteval hasarda cette observation :\par
— Ce n’est pas des revenants ; c’est des dames blanches.\par
— Qu’est-ce que c’est que ça qui pend à une fenêtre ? demanda l’autre.\par
— Ça a l’air d’une corde.\par
— C’est un serpent.\par
— C’est de la corde de pendu, dit le français avec autorité. Ça leur sert. Mais je n’y crois pas.\par
Et, en trois bonds plutôt qu’en trois pas, il fut au pied du mur de la masure. Il y avait de la fièvre dans cette hardiesse.\par
Les deux autres, frissonnants, l’imitèrent, et vinrent se coller près de lui, se serrant l’un contre son côté droit, l’autre contre son côté gauche. Les enfants appliquèrent leur oreille contre la muraille. On continuait de parler dans la maison.\par
Voici ce que disaient les fantômes :\par
— Asi, entendido està ?\footnote{ \noindent — Ainsi, c’est entendu ?\par
 — Entendu.\par
 — C’est dit ?\par
 — Dit.\par
 — Un homme attendra ici, et pourra s’en aller en Angleterre avec Blasquito ?\par
 — En payant.\par
 — En payant.\par
 — Blasquito prendra l’homme dans sa barque.\par
 — Sans chercher à savoir de quel pays il est ?\par
 — Cela ne nous regarde pas.\par
 — Sans lui demander son nom ?\par
 — On ne demande pas le nom ; on pèse la bourse.\par
 — Bien. L’homme attendra dans cette maison.\par
 — Il faudra qu’il ait de quoi manger.\par
 — Il en aura.\par
 — Où ?\par
 — Dans ce sac que j’apporte.\par
 — Très bien.\par
 — Puis-je laisser ce sac ici ?\par
 — Les contrebandiers ne sont pas des voleurs.\par
 — Et vous autres, quand partez-vous ?\par
 — Demain matin. Si votre homme était prêt, il pourrait venir avec nous.\par
 — Il n’est pas prêt.\par
 — C’est son affaire.\par
 — Combien de jours aura-t-il à attendre dans cette maison ?\par
 — Deux, trois, quatre jours. Moins ou plus.\par
 — Est-il certain que Blasquito viendra ?\par
 — Certain.\par
 — Ici ? A Plainmont ?\par
 — A Plainmont.\par
 — Quelle semaine ?\par
 — La semaine prochaine.\par
 — Quel jour ?\par
 — Vendredi, samedi, ou dimanche.\par
 — Il ne peut manquer ?\par
 — Il est mon tocayo.\par
 — Il vient par tous les temps ?\par
 — Par tous. Il n’a pas peur. Je suis Blasco, il est Blasquito.\par
 — Ainsi, il ne peut manquer de venir à Guernesey ?\par
 — Je viens un mois ; il vient l’autre mois.\par
 — Je comprends.\par
 — A compter de samedi prochain, d’aujourd’hui en huit, il ne se passera pas cinq jours sans que Blasquito arrive.\par
 — Mais si la mer était très dure ?\par
 — Egurraldia gaïztoa\footnote{ \noindent Basque. Mauvais temps.
 } ?\par
 — Oui.\par
 — Blasquito ne viendrait pas si vite, mais il viendrait.\par
 — D’où viendra-t-il ?\par
 — De Bilbao.\par
 — Où ira-t-il ?\par
 — A Portland.\par
 — C’est bien.\par
 — Ou à Torbay.\par
 — C’est mieux.\par
 — Votre homme peut être tranquille.\par
 — Blasquito ne trahira pas ?\par
 — Les lâches sont les traîtres. Nous sommes des vaillants. La mer est l’église de l’hiver. La trahison est l’église de l’enfer.\par
 — Personne n’entend ce que nous disons ?\par
 — Nous écouter et nous regarder est impossible. L’épouvante fait ici le désert.\par
 — Je le sais.\par
 — Qui oserait se hasarder à nous écouter ?\par
 — C’est vrai.\par
 — D’ailleurs on écouterait qu’on ne comprendrait pas. Nous parlons une farouche langue à nous que personne ne connaît. Puisque vous la savez, c’est que vous êtes des nôtres.\par
 — Je suis venu pour prendre des arrangements avec vous.\par
 — C’est bon.\par
 — Maintenant je m’en vais.\par
 — Soit.\par
 — Dites-moi, si le passager veut que Blasquito le conduise ailleurs qu’à Portland ou à Torbay ?\par
 — Qu’il ait des onces\footnote{ \noindent Des quadruples.
 }.\par
 — Blasquito fera-t-il ce que l’homme voudra ?\par
 — Blasquito fera ce que les onces voudront.\par
 — Faut-il beaucoup de temps pour aller à Torbay ?\par
 — Comme il plaît au vent.\par
 — Huit heures ?\par
 — Moins ou plus.\par
 — Blasquito obéira-t-il à son passager ?\par
 — Si la mer obéit à Blasquito.\par
 — Il sera bien payé.\par
 — L’or est l’or. Le vent est le vent.\par
 — C’est juste.\par
 — L’homme avec l’or fait ce qu’il peut. Dieu avec le vent fait ce qu’il veut.\par
 — L’homme qui compte partir avec Blasquito sera ici vendredi.\par
 — Bien.\par
 — A quel moment arrive Blasquito ?\par
 — A la nuit. On arrive la nuit. On part la nuit. Nous avons une femme qui s’appelle la mer, et une sœur qui s’appelle la nuit. La femme trompe quelquefois ; la sœur jamais.\par
 — Tout est convenu. Adieu, hommes.\par
 — Bonsoir. Un coup d’eau-de-vie ?\par
 — Merci.\par
 — C’est meilleur que du sirop.\par
 — J’ai votre parole.\par
 — Mon nom est Point-d’honneur.\par
 — Adieu.\par
 — Vous êtes gentilhomme et je suis chevalier.
 }\par
— Entendido.\par
— Dicho ?\par
— Dicho.\par
— Aqui esperarà un hombre, y podrà marcharse à Inglaterra con Blasquito ?\par
 — Pagando.\par
— Pagando.\par
— Blasquito tomarà al hombre en su barca.\par
— Sin tratar de conocer su pais ?\par
— No nos toca.\par
— Ni de saber su nombre ?\par
— No se pregunta el nombre, pero se pesa la boisa.\par
— Bien. Esperarà el hombre en esa casa.\par
— Tenga que corner.\par
— Tendrà.\par
— En donde ?\par
— En este saco que he traido.\par
— Muy bien.\par
— Puedo dexar el saco aqui ?\par
— Los contrabandistas no son ladrones.\par
— Y vosotros, cuando os marchais ?\par
 — Manana por la maûana. Si su hombre de usted esta para do, podria venir con nosotros.\par
— Parado no està.\par
— Hacienda suya.\par
— Cuantos dias esperara alli ?\par
— Dos, très, quatro dias. Menos o mas.\par
— Es cierto que el Blasquito vendra ?\par
— Cierto.\par
— A este Plainmont ?\par
— A este Plainmont.\par
— En qué semana ?\par
— La que viene.\par
— A quai dia ?\par
— En viernes, o sabado, o domingo.\par
— No puede faltar ?\par
— Es mi tocayo.\par
 — Por qualquiera tiempo viene ?\par
— Qualqueria. No terne. Soy el Blasco, es el Blas-quito.\par
— Asi, no pue de faltar en venir à Guernesey ?\par
— Vengo yo un mes, y viene al otro mes.\par
— Entendio.\par
— A contar del otro sabado, desde hoy en ochi-dias, ne pasaran cinco dias sin que venga el Blasquito.\par
— Pero un mar muy malo ?\par
— Egurraldia gaïztoa ?\par
— Si.\par
— No vendria el Blasquito tan pronto, pero ven-dria.\par
— De donde vendrà ?\par
— De Vilvao.\par
— A donde ira ?\par
 — A Portland.\par
— Bien.\par
— O à Torbay.\par
— Mejor.\par
— Su humbre de usted puede estar quieto.\par
— No serà traidor el Blasquito ?\par
— Los cobardes son traidores. Somos valientes. El mar es la iglesia del invierno. La traicion es la iglesia del infîerno.\par
— No seentiende lo que decimos ?\par
— Escucharnos y mirarnos es imposible. El espanto hace alli el desierto.\par
— Lo sé.\par
— Quien se atreveria à escuchar ?\par
— Es verdad.\par
— Àun quando escucharian no entienderian. Hablamos  mos una lengua fiera y nuestra que no se conoce. Despues que la sabeis, sois de nosotros.\par
— Soy venido para componer las haciendas con ustedes.\par
— Bueno.\par
— Y ahora me voy.\par
— Corriente.\par
— Digame usted, hombre. Si el pasagero quiere que el Blasquito le lleve à alguna otra parle que Portlaud o Torbay ?\par
— Tenga onzas.\par
— El Blasquito haro lo que quiera el hombre ?\par
— El Blasquito hace lo que quieren las onzas.\par
— Es menester mucho tiempo para ir à Torbay ?\par
— Como quiere el viento.\par
— Ocho horas ?\par
 — Menos, o mas.\par
— El Blasquito obedecera al pasagero ?\par
— Si le obedece el mar à el Blasquito.\par
— Bien pagado sera.\par
— El oro es el oro. El viento es el viento.\par
— Mucho.\par
— El hombre hace lo que puede con el oro. Dios con el viento hace lo que quiere.\par
— Aqui esta à viernes el que desea marcharse con Blasquito.\par
— Pues.\par
— A qué momento llega Blasquito ?\par
— A la noche. A la noche se llega, a la noche se marcha. Tenemos una muger que se llama el mar, y una hermana que se llama la noche. La muger puede faltar, la hermana no.\par
— Todo està dicho. Abur, hombres.\par
 — Buenas tardes. Un trago de aguardiente ?\par
— Gracias.\par
— Es mejor que xarope.\par
— Tengo vuestra palabra.\par
— Mi nombre es Pundonor.\par
— Vaya usted con Dios.\par
— Sois gentleman y soy caballero.\par
Il était clair que des diables seuls pouvaient parler ainsi. Les enfants n’en écoutèrent pas davantage, et cette fois prirent la fuite pour de bon, le petit français, enfin convaincu, courant plus vite que les autres.\par
Le mardi qui suivit ce samedi, sieur Clubin était de retour à Saint-Malo, ramenant la Durande.\par
Le \emph{Tamaulipas} était toujours en rade.\par
Sieur Clubin, entre deux bouffées de pipe, demanda à l’aubergiste de l’Auberge Jean :\par
— Eh bien, quand donc part-il, ce \emph{Tamaulipas ?}\par
— Après-demain jeudi, répondit l’aubergiste.\par
Ce soir-là, Clubin soupa à la table des gardes-côtes, et, contre son habitude, sortit après son souper. Il résulta de cette sortie qu’il ne put tenir le bureau de  la Durande, et qu’il manqua à peu près son chargement. Cela fut remarqué d’un homme si exact.\par
Il paraît qu’il causa quelques instants avec son ami le changeur.\par
Il rentra deux heures après que Noguette eut sonné le couvre-feu. La cloche brésilienne sonne à dix heures. Il était donc minuit.
 \subsubsection[{A.V.6. La jacressarde}]{A.V.6. \\
La jacressarde}
\noindent Il y a quarante ans, Saint-Malo possédait une ruelle dite la ruelle Coutanchez. Cette ruelle n’existe plus, ayant été comprise dans les embellissements.\par
C’était une double rangée de maisons de bois penchées les unes vers les autres, et laissant entre elles assez de place pour un ruisseau qu’on appelait la rue. On marchait les jambes écartées des deux côtés de l’eau, en heurtant de la tête ou du coude les maisons de droite ou de gauche. Ces vieilles baraques du moyen âge normand ont des profils presque humains. De masure à sorcière il n’y a pas loin. Leurs étages rentrants, leurs surplombs, leurs auvents circonflexes et leurs broussailles de ferrailles simulent des lèvres, des mentons, des nez et des sourcils. La lucarne est l’œil, borgne. La joue, c’est la muraille, ridée et dartreuse. Elles se touchent du front comme si elles complotaient un mauvais coup. Tous ces mots de l’ancienne civilisation, coupe-gorge, coupe-trogne, coupe-gueule, se rattachent à cette architecture.\par
 Une des maisons de la ruelle Coutanchez, la plus grande, la plus fameuse ou la plus famée, se nommait la Jacressarde.\par
La Jacressarde était le logis de ceux qui ne logent pas. Il y a dans toutes les villes, et particulièrement dans les ports de mer, au-dessous de la population, un résidu. Des gens sans aveu, à ce point que souvent la justice elle-même ne parvient pas à leur en arracher un, des écumeurs d’aventures, des chasseurs d’expédients, des chimistes de l’espèce escroc, remettant toujours la vie au creuset, toutes les formes du haillon et toutes les manières de le porter, les fruits secs de l’improbité, les existences en banqueroute, les consciences qui ont déposé leur bilan, ceux qui ont avorté dans l’escalade et le bris de clôture (car les grands faiseurs d’effractions planent et restent en haut), les ouvriers et les ouvrières du mal, les drôles et les drôlesses, les scrupules déchirés et les coudes percés, les coquins aboutis à l’indigence, les méchants mal récompensés, les vaincus du duel social, les affamés qui ont été les dévorants, les gagne-petit du crime, les gueux, dans la double et lamentable acception du mot ; tel est le personnel. L’intelligence humaine est là, bestiale. C’est le tas d’ordure des âmes. Cela s’amasse dans un coin, où passe de temps en temps ce coup de balai qu’on nomme une descente de police. A Saint-Malo la Jacressarde était ce coin.\par
Ce qu’on trouve dans ces repaires, ce ne sont pas les forts criminels, les bandits, les escarpes, les grands produits de l’ignorance et de l’indigence. Si le meurtre  y est représenté, c’est par quelque ivrogne brutal ; le vol n’y dépasse point le filou. C’est plutôt le crachat de la société que son vomissement. Le truand, oui ; le brigand, non. Pourtant il ne faudrait pas s’y fier. Ce dernier étage des bohèmes peut avoir des extrémités scélérates. Une fois, en jetant le filet sur l’Épi-scié qui était pour Paris ce que la Jacressarde était pour Saint-Malo, la police prit Lacenaire.\par
Ces gîtes admettent tout. La chute est un nivellement. Quelquefois l’honnêteté qui se déguenille tombe là. La vertu et la probité ont, cela s’est vu, des aventures. Il ne faut, d’emblée, ni estimer les Louvres ni mépriser les bagnes. Le respect public, de même que la réprobation universelle, veulent être épluchés. On y a des surprises. Un ange dans le lupanar, une perle dans le fumier, cette sombre et éblouissante trouvaille est possible.\par
La Jacressarde était plutôt une cour qu’une maison, et plutôt un puits qu’une cour. Elle n’avait point d’étage sur la rue. Un haut mur percé d’une porte basse était sa façade. On levait le loquet, on poussait la porte, on était dans une cour.\par
Au milieu de cette cour, on apercevait un trou rond, entouré d’une marge de pierre au niveau du sol. C’était un puits. La cour était petite, le puits était grand. Un pavage défoncé encadrait la margelle.\par
La cour, carrée, était bâtie de trois côtés. Du côté de la rue, rien ; mais en face de la porte, et à droite et à gauche, il y avait du logis.\par
Si, après la nuit tombée, on entrait là, un peu à ses  risques et périls, on entendait comme un bruit d’haleines mêlées, et, s’il y avait assez de lune ou d’étoiles pour donner forme aux linéaments obscurs qu’on avait sous les yeux, voici ce qu’on voyait.\par
La cour. Le puits. Autour de la cour, vis-à-vis la porte, un hangar figurant une sorte de fer à cheval qui serait carré, galerie vermoulue, tout ouverte, à plafond de solives, soutenue par des piliers de pierre inégalement espacés ; au centre, le puits ; autour du puits, sur une litière de paille, et faisant comme un chapelet circulaire, des semelles droites, des dessous de bottes éculées, des orteils passant par des trous de souliers, et force talons nus, des pieds d’homme, des pieds de femme, des pieds d’enfant. Tous ces pieds dormaient.\par
Au delà de ces pieds, l’œil, en s’enfonçant dans la pénombre du hangar, distinguait des corps, des formes, des têtes assoupies, des allongements inertes, des guenilles des deux sexes, une promiscuité dans du fumier, on ne sait quel sinistre gisement humain. Cette chambre à coucher était à tout le monde. On y payait deux sous par semaine. Les pieds touchaient le puits. Dans les nuits d’orage, il pleuvait sur ces pieds ; dans les nuits d’hiver, il neigeait sur ces corps.\par
Qu’était-ce que ces êtres ? Les inconnus. Ils venaient là le soir et s’en allaient le matin. L’ordre social se complique de ces larves. Quelques-uns se glissaient pour une nuit et ne payaient pas. La plupart n’avaient point mangé de la journée. Tous les vices, toutes les abjections, toutes les infections, toutes les détresses ;  le même sommeil d’accablement sur le même lit de boue. Les rêves de toutes ces âmes faisaient bon voisinage. Rendez-vous funèbre où remuaient et s’amalgamaient dans le même miasme les lassitudes, les défaillances, les ivresses cuvées, les marches et contre-marches d’une journée sans un morceau de pain et sans une bonne pensée, les lividités à paupières closes, des remords, des convoitises, des chevelures mêlées de balayures, des visages qui ont le regard de la mort, peut-être des baisers de bouches de ténèbres. Cette putridité humaine fermentait dans cette cuve. Ils étaient jetés dans ce gîte par la fatalité, par le voyage, par le navire arrivé la veille, par une sortie de prison, par la chance, par la nuit. Chaque jour la destinée vidait là sa hotte. Entrait qui voulait, dormait qui pouvait, parlait qui osait. Car c’était un lieu de chuchotement. On se hâtait de se mêler. On tâchait de s’oublier dans le sommeil, puisqu’on ne peut se perdre dans l’ombre. On prenait de la mort ce qu’on pouvait. Ils fermaient les yeux dans cette agonie pêle-mêle recommençant tous les soirs. D’où sortaient-ils ? de la société, étant la misère ; de la vague, étant l’écume.\par
N’avait point de la paille qui voulait. Plus d’une nudité traînait sur le pavé ; ils se couchaient éreintés ; ils se levaient ankylosés. Le puits, sans parapet et sans couvercle, toujours béant, avait trente pieds de profondeur. La pluie y tombait, les immondices y suintaient, tous les ruissellements de la cour y filtraient. Le seau pour tirer de l’eau était à côté. Qui avait soif, y buvait. Qui avait ennui, s’y noyait. Du sommeil dans  le fumier on glissait à ce sommeil-là. En 1819, on en retira un enfant de quatorze ans.\par
Pour ne point courir de danger dans cette maison, il fallait être « de la chose ». Les laïques étaient mal vus.\par
Ces êtres se connaissaient-ils entre eux ? Non. Ils se flairaient.\par
Une femme était la maîtresse du logis, jeune, assez jolie, coiffée d’un bonnet à rubans, débarbouillée quelquefois avec l’eau du puits, ayant une jambe de bois.\par
Dès l’aube, la cour se vidait ; les habitués s’envolaient.\par
Il y avait dans la cour un coq et des poules, grattant le fumier tout le jour. La cour était traversée par une poutre horizontale sur poteaux, figure d’un gibet pas trop dépaysée là. Souvent, le lendemain des soirées pluvieuses, on voyait sécher sur cette poutre une robe de soie mouillée et crottée, qui était à la femme jambe de bois.\par
Au-dessus du hangar, et, comme lui, encadrant la cour, il y avait un étage, et au-dessus de l’étage un grenier. Un escalier de bois pourri trouant le plafond du hangar menait en haut ; échelle branlante gravie bruyamment par la femme chancelante.\par
Les locataires de passage, à la semaine ou à la nuit, habitaient la cour ; les locataires à demeure habitaient la maison.\par
Des fenêtres, pas un carreau ; des chambranles, pas une porte ; des cheminées, pas un foyer ; c’était la  maison. On passait d’une chambre dans l’autre indifféremment par un trou carré long qui avait été la porte ou par une baie triangulaire qui était l’entre-deux des solives de la cloison. Les plâtrages tombés couvraient le plancher. On ne savait comment tenait la maison. Le vent la remuait. On montait comme on pouvait sur le glissement des marches usées de l’escalier. Tout était à claire-voie. L’hiver entrait dans la masure comme l’eau dans une éponge. L’abondance des araignées rassurait contre l’écroulement immédiat. Aucun meuble. Deux ou trois paillasses dans des coins, ventre ouvert, montrant plus de cendre que de paille. Çà et là une cruche et une terrine, servant à divers usages. Une odeur douce et hideuse.\par
Des fenêtres on avait vue sur la cour. Cette vue ressemblait à un dessus de tombereau de boueux. Les choses, sans compter les hommes, qui pourrissaient là, qui s’y rouillaient, qui y moisissaient, étaient indescriptibles. Les débris fraternisaient ; il en tombait des murailles, il en tombait des créatures. Les loques ensemençaient les décombres.\par
Outre sa population flottante, cantonnée dans la cour, la Jacressarde avait trois locataires, un charbonnier, un chiffonnier et un faiseur d’or. Le charbonnier et le chiffonnier occupaient deux des paillasses du premier ; le faiseur d’or, chimiste, logeait au grenier, qu’on appelait, on ne sait pourquoi, le galetas. On ignorait dans quel coin couchait la femme. Le faiseur d’or était un peu poëte. Il habitait, dans le toit, sous les tuiles, une chambre où il y avait une lucarne  étroite et une grande cheminée de pierre, gouffre à faire mugir le vent. La lucarne n’ayant pas de châssis, il avait cloué dessus un morceau de feuillard provenant d’une déchirure de navire. Cette tôle laissait passer peu de jour et beaucoup de froid. Le charbonnier payait d’un sac de charbon de temps en temps, le chiffonnier payait d’un setier de grain aux poules par semaine, le faiseur d’or ne payait pas. En attendant il brûlait la maison. Il avait arraché le peu qu’il y avait de boiserie, et à chaque instant il tirait du mur ou du toit une latte pour faire chauffer sa marmite à or. Sur la cloison, au-dessus du grabat du chiffonnier, on voyait deux colonnes de chiffres à la craie, tracées par le chiffonnier semaine à semaine, une colonne de 3 et une colonne de 5, selon que le setier de grain coûtait trois liards ou cinq centimes. La marmite à or du « chimiste » était une vieille bombe cassée, promue par lui chaudière, où il combinait des ingrédients. La transmutation l’absorbait. Quelquefois il parlait dans la cour aux va-nu-pieds, qui en riaient. Il disait : \emph{Ces gens-là sont pleins de préjugés.} Il était résolu à ne pas mourir sans jeter la pierre philosophale dans les vitres de la science. Son fourneau mangeait beaucoup de bois. La rampe de l’escalier y avait disparu. Toute la maison y passait, à petit feu. L’hôtesse lui disait : Vous ne me laisserez que la coque. Il la désarmait en lui faisant des vers. Telle était la Jacressarde.\par
Un enfant, qui était peut-être un nain, âgé de douze ans ou de soixante ans, goîtreux, ayant un balai à la main, était le domestique.\par
 Les habitués entraient par la porte de la cour ; le public entrait par la boutique.\par
Qu’était-ce que la boutique ?\par
Le haut mur faisant façade sur la rue était percé, à droite de l’entrée de la cour, d’une baie en équerre à la fois porte et fenêtre, avec volet et châssis, le seul volet dans toute la maison qui eût des gonds et des verrous, le seul châssis qui eût des vitres. Derrière cette devanture, ouverte sur la rue, il y avait une petite chambre, compartiment pris sur le hangar-dortoir. On lisait sur la porte de la rue cette inscription charbonnée : \emph{Ici on tient la curiosité.} Le mot était dès lors usité. Sur trois planches s’appuyant en étagère au vitrage, on apercevait quelques pots de faïence sans anse, un parasol chinois en baudruche à figures, crevé çà et là, impossible à ouvrir et à fermer, des tessons de fer ou de grès informes, des chapeaux d’homme et de femme effondrés, trois ou quatre coquilles d’ormers, quelques paquets de vieux boutons d’os et de cuivre, une tabatière avec portrait de Marie-Antoinette, et un volume dépareillé de l’Algèbre de Boisbertrand. C’était la boutique. Cet assortiment était « la curiosité ». La boutique communiquait, par une arrière-porte, avec la cour où était le puits. Il y avait une table et un escabeau. La femme à la jambe de bois était la dame de comptoir.
 \subsubsection[{A.V.7. Acheteurs nocturnes et vendeur ténébreux}]{A.V.7. \\
Acheteurs nocturnes et vendeur ténébreux}
\noindent Clubin avait été absent de l’auberge Jean le mardi toute la soirée ; il le fut encore le mercredi soir.\par
Ce soir-là, à la brune, deux hommes s’engagèrent dans la ruelle Coutanchez ; ils s’arrêtèrent devant la Jacressarde. L’un d’eux cogna à la vitre. La porte de la boutique s’ouvrit. Ils entrèrent. La femme à la jambe de bois leur fit le sourire réservé aux bourgeois. Il y avait une chandelle sur la table.\par
Ces hommes étaient deux bourgeois en effet.\par
Celui des deux qui avait cogné dit : — Bonjour, la femme. Je viens pour la chose.\par
La femme jambe de bois fit un deuxième sourire et sortit par l’arrière-porte, qui donnait sur la cour au puits. Un moment après, l’arrière-porte se rouvrit, et un homme se présenta dans l’entre-bâillement. Cet homme avait une casquette et une blouse, et la saillie d’un objet sous sa blouse. Il avait des brins de paille dans les plis de sa blouse et le regard de quelqu’un qu’on vient de réveiller.\par
 Il avança. On se regarda. L’homme en blouse avait l’air ahuri et fin. Il dit :\par
— C’est vous l’armurier ?\par
Celui qui avait cogné répondit :\par
— Oui. C’est vous le Parisien ?\par
— Dit Peaurouge. Oui.\par
— Montrez.\par
— Voici.\par
L’homme tira de dessous sa blouse un engin fort rare en Europe à cette époque, un revolver.\par
Ce revolver était neuf et brillant. Les deux bourgeois l’examinèrent. Celui qui semblait connaître la maison et que l’homme en blouse avait qualifié « l’armurier » fit jouer le mécanisme. Il passa l’objet à l’autre, qui paraissait être moins de la ville et qui se tenait le dos tourné à la lumière.\par
L’armurier reprit :\par
— Combien ?\par
L’homme en blouse répondit :\par
— J’arrive d’Amérique avec. Il y a des gens qui apportent des singes, des perroquets, des bêtes, comme si les français étaient des sauvages. Moi j’apporte ça. C’est une invention utile.\par
— Combien ? repartit l’armurier.\par
— C’est un pistolet qui fait le moulinet.\par
— Combien ?\par
— Paf. Un premier coup. Paf. Un deuxième coup. Paf... une grêle, quoi ! Ça fait de la besogne.\par
— Combien ?\par
— Il y a six canons.\par
 — Eh bien, combien ?\par
— Six canons, c’est six louis.\par
— Voulez-vous cinq louis ?\par
— Impossible. Un louis par balle. C’est le prix.\par
— Voulons-nous faire affaire ? soyons raisonnables.\par
— J’ai dit le prix juste. Examinez-moi ça, monsieur l’arquebusier.\par
— J’ai examiné.\par
— Le moulinet tourne comme monsieur Talleyrand. On pourrait mettre ce moulinet-là dans le Dictionnaire des girouettes. C’est un bijou.\par
— Je l’ai vu.\par
— Quant aux canons, c’est de forge espagnole.\par
— Je l’ai remarqué.\par
— C’est rubané. Voici comment ça se confectionne, ces rubans-là. On vide dans la forge la hotte d’un chiffonnier en vieux fer. On prend tout plein de vieille ferraille, des vieux clous de maréchal, des fers à cheval cassés...\par
— Et de vieilles lames de faulx.\par
— J’allais le dire, monsieur l’armurier. On vous fiche à tout ce bric-à-brac une bonne chaude suante, et ça vous fait une magnifique étoffe de fer...\par
— Oui, mais qui peut avoir des crevasses, des éventures, des travers.\par
— Pardine. Mais on remédie aux travers par des petites queues d’aronde, de même qu’on évite le risque des doublures en battant ferme. On corroie son étoffe de fer au gros marteau, on lui flanque deux autres chaudes suantes ; si le fer a été surchauffé, on le  rétablit par des chaudes grasses, et à petits coups. Et puis on étire l’étoffe, et puis on la roule bien sur la chemise, et avec ce fer-là, fichtre ! on vous fait ces canons-là.\par
— Vous êtes donc du métier ?\par
— Je suis de tous les métiers.\par
— Les canons sont couleur d’eau.\par
— C’est une beauté, monsieur l’armurier. Ça s’obtient avec du beurre d’antimoine.\par
— Nous disons donc que nous allons vous payer cela cinq louis ?\par
— Je me permets de faire observer à monsieur que j’ai eu l’honneur de dire six louis.\par
L’armurier baissa la voix.\par
— Écoutez, Parisien. Profitez de l’occasion. Défaites-vous de ça. Ça ne vaut rien pour vous autres, une arme comme ça. Ça fait remarquer un homme.\par
— En effet, dit Parisien, c’est un peu voyant. C’est meilleur pour un bourgeois.\par
— Voulez-vous cinq louis ?\par
— Non, six. Un par trou.\par
— Eh bien, six napoléons.\par
— Je veux six louis.\par
— Vous n’êtes donc pas bonapartiste ? vous préférez un louis à un napoléon !\par
Parisien, dit Peaurouge, sourit.\par
— Napoléon vaut mieux, dit-il, mais Louis vaut plus.\par
— Six napoléons.\par
— Six louis. C’est pour moi une différence de vingt-quatre francs.\par
 — Pas d’affaire en ce cas.\par
— Soit. Je garde le bibelot.\par
— Gardez-le.\par
— Du rabais ! par exemple ! il ne sera pas dit que je me serai défait comme ça d’une chose qui est une invention nouvelle.\par
— Bonsoir, alors.\par
— C’est un progrès sur le pistolet, que les indiens chesapeakes appellent Nortay-u-Hah.\par
— Cinq louis payés comptant, c’est de l’or.\par
— Nortay-u-Hah, cela veut dire \emph{Fusil-Court.} Beaucoup de personnes ignorent cela.\par
— Voulez-vous cinq louis, et un petit écu de rabiot ?\par
— Bourgeois, j’ai dit six.\par
L’homme qui tournait le dos à la chandelle et qui n’avait pas encore parlé faisait pendant ce dialogue pivoter le mécanisme. Il s’approcha de l’oreille de l’armurier et lui chuchota :\par
— L’objet est-il bon ?\par
— Excellent.\par
— Je donne les six louis.\par
Cinq minutes après, pendant que Parisien dit Peau-rouge serrait dans une fente secrète sous l’aisselle de sa blouse les six louis d’or qu’il venait de recevoir, l’armurier, et l’acheteur emportant dans la poche de son pantalon le revolver, sortaient de la ruelle Coutanchez.
 \subsubsection[{A.V.8. Carambolage de la bille rouge et de la bille noire}]{A.V.8. \\
Carambolage de la bille rouge et de la bille noire}
\noindent Le lendemain, qui était le jeudi, à peu de distance de Saint-Malo, près de la pointe du Décollé, à un endroit où la falaise est haute et où la mer est profonde, il se passa une chose tragique.\par
Une langue de rochers en forme de fer de lance, qui se relie à la terre par un isthme étroit, se prolonge dans l’eau et s’y achève brusquement par un grand brisant à pic ; rien n’est plus fréquent dans l’architecture de la mer. Pour arriver, en venant du rivage, au plateau de la roche à pic, on suit un plan incliné dont la montée est quelquefois assez âpre.\par
C’est sur un plateau de ce genre qu’était debout vers quatre heures du soir un homme enveloppé dans une large cape d’ordonnance et probablement armé dessous, chose facile à reconnaître à de certains plis droits et anguleux de son manteau. Le sommet où se tenait cet homme était une plate-forme assez vaste,  semée de gros cubes de roche pareils à des pavés démesurés, et laissant entre eux des passages étroits. Cette plate-forme, où croissait une petite herbe épaisse et courte, se terminait du côté de la mer par un espace libre, aboutissant à un escarpement vertical. L’escarpement, élevé d’une soixantaine de pieds au-dessus de la haute mer, semblait taillé au fil à plomb. Son angle de gauche pourtant se ruinait et offrait un de ces escaliers naturels propres aux falaises de granit, dont les marches peu commodes exigent quelquefois des enjambées de géants ou des sauts de clowns. Cette dégringolade de rochers descendait perpendiculairement jusqu’à la mer et s’y enfonçait. C’était à peu près un casse-cou. Cependant, à la rigueur, on pouvait par là s’aller embarquer sous la muraille même de la falaise.\par
La brise soufflait, l’homme serré sous sa cape, ferme sur ses jarrets, la main gauche empoignant son coude droit, clignait un œil et appuyait l’autre sur une longue-vue. Il semblait absorbé dans une attention sérieuse. Il s’était approché du bord de l’escarpement, et il se tenait là immobile, le regard imperturbablement attaché sur l’horizon. La marée était pleine. Le flot battait au-dessous de lui le bas de la falaise.\par
Ce que cet homme observait, c’était un navire au large qui faisait en effet un jeu singulier.\par
Ce navire, qui venait de quitter depuis une heure à peine le port de Saint-Malo, s’était arrêté derrière les Banquetiers. C’était un trois-mâts. Il n’avait pas jeté l’ancre, peut-être parce que le fond ne lui eût permis  d’abattre que sur le bord du câble, et parce que le navire eût serré son ancre sous le taille-mer ; il s’était borné à mettre en panne.\par
L’homme, qui était un garde-côte, comme le faisait voir sa cape d’uniforme, épiait toutes les manœuvres du trois-mâts et semblait en prendre note mentalement. Le navire avait mis en panne vent dessus vent dedans ; ce qu’indiquaient le petit hunier coiffé et le vent laissé dans le grand hunier ; il avait bordé l’artimon et orienté le perroquet de fougue au plus près, de façon à contrarier les voiles les unes par les autres, et à avoir peu d’arrivée et moins de dérive. Il ne se souciait point de se présenter beaucoup au vent, car il n’avait brassé le petit hunier que perpendiculairement à la quille. De cette façon, tombant en travers, il ne dérivait au plus que d’une demi-lieue à l’heure.\par
Il faisait encore grand jour, surtout en pleine mer et sur le haut de la falaise. Le bas des côtes devenait obscur.\par
Le garde-côte, tout à sa besogne et espionnant consciencieusement le large, n’avait pas songé à scruter le rocher à côté et au-dessous de lui. Il tournait le dos à l’espèce d’escalier peu praticable qui mettait en communication le plateau de la falaise et la mer. Il ne remarquait pas que quelque chose y remuait. Il y avait dans cet escalier, derrière une anfractuosité, quelqu’un, un homme, caché là, selon toute apparence, avant l’arrivée du garde-côte. De temps en temps, dans l’ombre, une tête sortait de dessous la roche, regardait en haut, et guettait le guetteur. Cette tête,  coiffée d’un large chapeau américain, était celle de l’homme, du quaker, qui, une dizaine de jours auparavant, parlait, dans les pierres du Petit-Bey, au capitaine Zuela.\par
Tout à coup l’attention du garde-côte parut redoubler. Il essuya rapidement du drap de sa manche le verre de sa longue-vue et la braqua avec énergie sur le trois-mâts.\par
Un point noir venait de s’en détacher.\par
Ce point noir, semblable à une fourmi sur la mer, était une embarcation.\par
L’embarcation semblait vouloir gagner la terre. Quelques marins la montaient, ramant vigoureusement.\par
Elle obliquait peu à peu et se dirigeait vers la pointe du Décollé.\par
Le guet du garde-côte était arrivé à son plus haut degré de fixité. Il ne perdait pas un mouvement de l’embarcation. Il s’était rapproché plus près encore de l’extrême bord de la falaise.\par
En ce moment un homme de haute stature, le quaker, surgit derrière le garde-côte au haut de l’escalier. Le guetteur ne le voyait pas.\par
Cet homme s’arrêta un instant, les bras pendants et les poings crispés, et, avec l’œil d’un chasseur qui vise, il regarda le dos du garde-côte.\par
Quatre pas seulement le séparaient du garde-côte ; il mit un pied en avant, puis s’arrêta ; il fit un second pas, et s’arrêta encore ; il ne faisait point d’autre mouvement que de marcher, tout le reste de son corps  était statue ; son pied s’appuyait sur l’herbe sans bruit ; il fit le troisième pas, et s’arrêta ; il touchait presque le garde-côte, toujours immobile avec sa longue-vue. L’homme ramena lentement ses deux mains fermées à la hauteur de ses clavicules, puis, brusquement, ses avant-bras s’abattirent, et ses deux poings, comme lâchés par une détente, frappèrent les deux épaules du garde-côte. Le choc fut sinistre. Le garde-côte n’eut pas le temps de jeter un cri. Il tomba la tête la première du haut de la falaise dans la mer. On vit ses deux semelles le temps d’un éclair. Ce fut une pierre dans l’eau. Tout se referma.\par
Deux ou trois grands cercles se firent dans l’eau sombre.\par
Il ne resta que la longue-vue échappée des mains du garde-côte et tombée à terre sur l’herbe.\par
Le quaker se pencha sur le bord de l’escarpement, regarda les cercles s’effacer dans le flot, attendit quelques minutes, puis se redressa en chantant entre ses dents :\par


\begin{verse}
Monsieur d’la Police est mort\\
En perdant la vie.\\
\end{verse}

\noindent Il se pencha une seconde fois. Rien ne reparut. Seulement, à l’endroit où le garde-côte s’était englouti, il s’était formé à la surface de l’eau une sorte d’épaisseur brune qui s’élargissait sur le balancement de la lame. Il était probable que le garde-côte s’était brisé le crâne sur quelque roche sous-marine. Son  sang remontait et faisait cette tache dans l’écume. Le quaker, tout en considérant cette flaque rougeâtre, reprit :\par


\begin{verse}
Un quart d’heure avant sa mort,\\
Il était encore....\\
\end{verse}

\noindent Il n’acheva pas.\par
Il entendit derrière lui une voix très douce qui disait :\par
— Vous voilà, Rantaine. Bonjour. Vous venez de tuer un homme.\par
Il se retourna, et vit à une quinzaine de pas en arrière de lui, à l’issue d’un des entre-deux des rochers, un petit homme qui avait un revolver à la main.\par
Il répondit :\par
— Comme vous voyez. Bonjour, sieur Clubin.\par
Le petit homme eut un tressaillement.\par
— Vous me reconnaissez ?\par
— Vous m’avez bien reconnu, repartit Rantaine.\par
Cependant on entendait un bruit de rames sur la mer. C’était l’embarcation observée par le garde-côte, qui approchait.\par
Sieur Clubin dit à demi-voix, comme se parlant à lui-même :\par
— Cela a été vite fait.\par
— Qu’y a-t-il pour votre service ? demanda Rantaine.\par
— Pas grand’chose. Voilà tout à l’heure dix ans que je ne vous ai vu. Vous avez dû faire de bonnes affaires. Comment vous portez-vous ?\par
 — Bien, dit Rantaine. Et vous ?\par
Rantaine fit un pas vers sieur Clubin.\par
Un petit coup sec arriva à son oreille. C’était sieur Clubin qui armait le revolver.\par
— Rantaine, nous sommes à quinze pas. C’est une bonne distance. Restez où vous êtes.\par
— Ah çà, fit Rantaine, qu’est-ce que vous me voulez ?\par
— Moi, je viens causer avec vous.\par
Rantaine ne bougea plus. Sieur Clubin reprit :\par
— Vous venez d’assassiner un garde-côte.\par
Rantaine souleva le bord de son chapeau et répondit :\par
— Vous m’avez déjà fait l’honneur de me le dire.\par
— En termes moins précis. [{\corr J’avais}] dit : un homme ; je dis maintenant : un garde-côte. Ce garde-côte portait le numéro six cent dix-neuf. Il était père de famille. Il laisse une femme et cinq enfants.\par
— Ça doit être, dit Rantaine.\par
Il y eut un imperceptible temps d’arrêt.\par
— Ce sont des hommes de choix, ces gardes-côtes, fit Clubin, presque tous d’anciens marins.\par
— J’ai remarqué, dit Rantaine, qu’en général on laisse une femme et cinq enfants.\par
Sieur Clubin continua :\par
— Devinez combien m’a coûté ce revolver.\par
— C’est une jolie pièce, répondit Rantaine.\par
— Combien l’estimez-vous ?\par
— Je l’estime beaucoup.\par
— Il m’a coûté cent quarante-quatre francs.\par
 — Vous avez dû acheter ça, dit Rantaine, à la boutique d’armes de la rue Coutanchez.\par
Clubin reprit :\par
— Il n’a pas crié. La chute coupe la voix.\par
— Sieur Clubin, il y aura de la brise cette nuit.\par
— Je suis seul dans le secret.\par
— Logez-vous toujours à l’Auberge Jean ? demanda Rantaine.\par
— Oui, on n’y est pas mal.\par
— Je me rappelle y avoir mangé de bonne choucroute.\par
— Vous devez être excessivement fort, Rantaine. Vous avez des épaules ! Je ne voudrais pas recevoir une chiquenaude de vous. Moi, quand je suis venu au monde, j’avais l’air si chétif qu’on ne savait pas si on réussirait à m’élever.\par
— On y a réussi, c’est heureux.\par
— Oui, je loge toujours à cette vieille Auberge Jean.\par
— Savez-vous, sieur Clubin, pourquoi je vous ai reconnu ? C’est parce que vous m’avez reconnu. J’ai dit : Il n’y a pour cela que Clubin.\par
Et il avança d’un pas.\par
— Replacez-vous où vous étiez, Rantaine.\par
Rantaine recula et fit cet aparté :\par
— On devient un enfant devant ces machins-là.\par
Sieur Clubin poursuivit :\par
— Situation. Nous avons à droite, du côté de Saint-Ênogat, à trois cents pas d’ici, un autre garde-côte, le numéro six cent dix-huit, qui est vivant, et à gauche, du côté de Saint-Lunaire, un poste de douane. Cela fait sept hommes armés qui peuvent être ici dans cinq  minutes. Le rocher sera cerné. Le col sera gardé. Impossible de s’évader. Il y a un cadavre au pied de la falaise.\par
Rantaine jeta un œil oblique sur le revolver.\par
— Comme vous dites, Rantaine. C’est une jolie pièce. Peut-être n’est-il chargé qu’à poudre. Mais qu’est-ce que cela fait ? Il suffit d’un coup de feu pour faire accourir la force armée. J’en ai six à tirer.\par
Le choc alternatif des rames devenait très distinct. Le canot n’était pas loin.\par
Le grand homme regardait le petit homme, étrangement. Sieur Clubin parlait d’une voix de plus en plus tranquille et douce.\par
— Rantaine, les hommes du canot qui va arriver, sachant ce que vous venez de faire ici tout à l’heure, prêteraient main-forte et aideraient à vous arrêter. Vous payez votre passage dix mille francs au capitaine Zuela. Par parenthèse, vous auriez eu meilleur marché avec les contrebandiers de Plainmont ; mais ils ne vous auraient mené qu’en Angleterre, et d’ailleurs vous ne pouvez risquer d’aller à Guernesey où l’on a l’honneur de vous connaître. Je reviens à la situation. Si je fais feu, on vous arrête. Vous payez à Zuela votre fugue dix mille francs. Vous lui avez donné cinq mille francs d’avance. Zuela garderait les cinq mille francs, et s’en irait. Voilà. Rantaine, vous êtes bien affublé. Ce chapeau, ce drôle d’habit et ces guêtres vous changent. Vous avez oublié les lunettes. Vous avez bien fait de laisser pousser vos favoris.\par
Rantaine fit un sourire assez semblable à un grincement. Clubin continua :\par
 — Rantaine, vous avez une culotte américaine à gousset double. Dans l’un il y a votre montre. Gardez-la.\par
— Merci, sieur Clubin.\par
— Dans l’autre il y a une petite boîte de fer battu qui ouvre et ferme à ressort. C’est une ancienne tabatière à matelot. Tirez-la de votre gousset et jetez-la-moi.\par
— Mais c’est un vol !\par
— Vous êtes libre de crier à la garde.\par
Et Clubin regarda fixement Rantaine.\par
— Tenez, mess Clubin..., dit Rantaine faisant un pas, et tendant sa main ouverte.\par
\emph{Mess} était une flatterie.\par
— Restez où vous êtes, Rantaine.\par
— Mess Clubin, arrangeons-nous. Je vous offre moitié.\par
Clubin exécuta un croisement de bras d’où sortait le bout de son revolver.\par
— Rantaine, pour qui me prenez-vous ? Je suis un honnête homme.\par
Et il ajouta après un silence :\par
— Il me faut tout.\par
Rantaine grommela entre ses dents : — Celui-ci est d’un fort gabarit.\par
Cependant l’œil de Clubin venait de s’allumer. Sa voix devint nette et coupante comme l’acier. Il s’écria :\par
— Je vois que vous vous méprenez. C’est vous qui vous appelez Vol, moi je m’appelle Restitution. Rantaine, écoutez. Il y a dix ans, vous avez quitté de nuit  Guernesey en prenant dans la caisse d’une association cinquante mille francs qui étaient à vous, et en oubliant d’y laisser cinquante mille francs qui étaient à un autre. Ces cinquante mille francs volés par vous à votre associé, l’excellent et digne mess Lethierry, font aujourd’hui avec les intérêts composés pendant dix ans quatrevingt mille six cent soixante-six francs soixante-six centimes. Hier vous êtes entré chez un changeur. Je vais vous le nommer. Rébuchet, rue Saint-Vincent. Vous lui avez compté soixante-seize mille francs en billets de banque français, contre lesquels il vous a donné trois bank-notes d’Angleterre de mille livres sterling chaque, plus l’appoint. Vous avez mis ces bank-notes dans la tabatière de fer, et la tabatière de fer dans votre gousset de droite. Ces trois mille livres sterling font soixante-quinze mille francs. Au nom de mess Lethierry, je m’en contenterai. Je pars demain pour Guernesey, et j’entends les lui porter. Rantaine, le trois-mâts qui est là en panne est le \emph{Tamaulipas}. Vous y avez fait embarquer cette nuit vos malles mêlées aux sacs et aux valises de l’équipage. Vous voulez quitter la France. Vous avez vos raisons. Vous allez à Arequipa. L’embarcation vient vous chercher. Vous l’attendez ici. Elle arrive. On l’entend qui nage. Il dépend de moi de vous laisser partir ou de vous faire rester. Assez de paroles. Jetez-moi la tabatière de fer.\par
Rantaine ouvrit son gousset, en tira une petite boîte et la jeta à Clubin. C’était la tabatière de fer. Elle alla rouler aux pieds de Clubin.\par
Clubin se pencha sans baisser la tête et ramassa la  tabatière de la main gauche, tenant dirigés sur Rantaine ses deux yeux et les six canons du revolver.\par
Puis il cria :\par
— Mon ami, tournez le dos.\par
Rantaine tourna le dos.\par
Sieur Clubin mit le revolver sous son aisselle, et fit jouer le ressort de la tabatière. La boîte s’ouvrit.\par
Elle contenait quatre bank-notes, trois de mille livres et une de dix livres.\par
Il replia les trois bank-notes de mille livres, les replaça dans la tabatière de fer, referma la boîte et la mit dans sa poche.\par
Puis il prit à terre un caillou. Il enveloppa ce caillou du billet de dix livres, et dit :\par
— Retournez-vous.\par
Rantaine se retourna.\par
Sieur Clubin reprit :\par
— Je vous ai dit que je me contenterais des trois mille livres. Voilà dix livres que je vous rends.\par
Et il jeta à Rantaine le billet lesté du caillou.\par
Rantaine, d’un coup de pied, lança la bank-note et le caillou dans la mer.\par
— Comme il vous plaira, fit Clubin. Allons, vous devez être riche. Je suis tranquille.\par
Le bruit de rames, qui s’était continuellement rapproché pendant ce dialogue, cessa. Cela indiquait que l’embarcation était au pied de la falaise.\par
— Votre fiacre est en bas. Vous pouvez partir, Rantaine.\par
Rantaine se dirigea vers l’escalier et s’y enfonça.\par
 Clubin vint avec précaution au bord de l’escarpement, et, avançant la tête, le regarda descendre.\par
Le canot s’était arrêté près de la dernière marche de rochers, à l’endroit même où était tombé le garde-côte.\par
Tout en regardant dégringoler Rantaine, Clubin grommela :\par
— Bon numéro six cent dix-neuf ! Il se croyait seul. Rantaine croyait n’être que deux. Moi seul savais que nous étions trois.\par
Il aperçut à ses pieds sur l’herbe la longue-vue qu’avait laissé tomber le garde-côte. Il la ramassa.\par
Le bruit des rames recommença. Rantaine venait de sauter dans l’embarcation, et le canot prenait le large.\par
Quand Rantaine fut dans le canot, après les premiers coups d’aviron, la falaise commençant à s’éloigner derrière lui, il se dressa brusquement debout, sa face devint monstrueuse, il montra le poing en bas, et cria : — Ha ! le diable lui-même est une canaille !\par
Quelques secondes après, Clubin, au haut de la falaise et braquant la longue-vue sur l’embarcation, entendait distinctement ces paroles articulées par une voix haute dans le bruit de la mer :\par
— Sieur Clubin, vous êtes un honnête homme ; mais vous trouverez bon que j’écrive à Lethierry pour lui faire part de la chose, et voici dans le canot un matelot de Guernesey qui est de l’équipage du \emph{Tamaulipas}, qui s’appelle Ahier Tostevin, qui reviendra à Saint-Malo au prochain voyage de Zuela et qui témoignera  que je vous ai remis pour mess Lethierry la somme de trois mille livres sterling.\par
C’était la voix de Rantaine.\par
Clubin était l’homme des choses bien faites. Immobile comme l’avait été le garde-côte, et à cette même place, l’œil dans la longue-vue, il ne quitta pas un instant le canot du regard. Il le vit décroître dans les lames, disparaître et reparaître, approcher le navire en panne, et l’accoster, et il put reconnaître la haute taille de Rantaine sur le pont du \emph{Tamaulipas.}\par
Quand le canot fut remonté à bord et replacé dans les pistolets, le \emph{Tamaulipas} fit servir. La brise montait de terre, il éventa toutes ses voiles, la lunette de Clubin demeura braquée sur cette silhouette de plus en plus simplifiée, et, une demi-heure après, le \emph{Tamaulipas} n’était plus qu’une corne noire s’amoindrissant à l’horizon sur le ciel blême du crépuscule.
 \subsubsection[{A.V.9. Renseignement utile aux personnes qui attendent, ou craignent, des lettres d’outre-mer}]{A.V.9. \\
Renseignement utile aux personnes qui attendent, ou craignent, des lettres d’outre-mer}
\noindent Ce soir-là encore, sieur Clubin rentra tard.\par
Une des causes de son retard, c’est qu’avant de rentrer, il était allé jusqu’à la porte Dinan où il y avait des cabarets. Il avait acheté, dans un de ces cabarets où il n’était pas connu, une bouteille d’eau-de-vie qu’il avait mise dans la large poche de sa vareuse comme s’il voulait l’y cacher ; puis, la Durande devant partir le lendemain matin, il avait fait un tour à bord pour s’assurer que tout était en ordre.\par
Quand sieur Clubin rentra à l’Auberge Jean, il n’y avait plus dans la salle basse que le vieux capitaine au long cours, M. Gertrais-Gaboureau, qui buvait sa chope et fumait sa pipe.\par
M. Gertrais-Gaboureau salua sieur Clubin entre une bouffée et une gorgée.\par
— Good bye, capitaine Clubin.\par
— Bonsoir, capitaine Gertrais.\par
 — Eh bien, voilà le \emph{Tamaulipas} parti.\par
— Ah ! dit Clubin, je n’y ai pas fait attention.\par
Le capitaine Gertrais-Gaboureau cracha et dit :\par
— Filé, Zuela.\par
— Quand ça donc ?\par
— Ce soir.\par
— Où va-t-il ?\par
— Au diable.\par
— Sans doute ; mais où ?\par
— A Arequipa.\par
— Je n’en savais rien, dit Clubin.\par
Il ajouta :\par
— Je vais me coucher.\par
Il alluma sa chandelle, marcha vers la porte, et revint.\par
— Êtes-vous allé à Arequipa, capitaine Gertrais ?\par
— Oui. Il y a des ans.\par
— Où relâche-t-on ?\par
— Un peu partout. Mais ce \emph{Tamaulipas} ne relâchera point.\par
M. Gertrais-Gaboureau vida sur le bord d’une assiette la cendre de sa pipe, et continua :\par
— Vous savez, le chasse-marée \emph{Cheval-de-Troie} et ce beau trois-mâts, le \emph{Trentemouzin}, qui sont allés à Cardiff. Je n’étais pas d’avis du départ à cause du temps. Ils sont revenus dans un bel état. Le chasse-marée était chargé de térébenthine, il a fait eau, et en faisant jouer les pompes il a pompé avec l’eau tout son chargement. Quant au trois-mâts, il a surtout souffert dans les hauts ; la guibre, la poulaine, les minots, le  jas de l’ancre à bâbord, tout ça cassé. Le bout-dehors du grand foc cassé au ras du chouque. Les haubans de focs et les sous-barbes, va-t’en voir s’ils viennent. Le mât de misaine n’a rien ; il a eu cependant une secousse sévère. Tout le fer du beaupré a manqué, et, chose incroyable, le beaupré n’est que mâché, mais il est complètement dépouillé. Le masque du navire à bâbord est à jour trois bons pieds carrés. Voilà ce que c’est que de ne pas écouter le monde.\par
Clubin avait posé sa chandelle sur la table et s’était mis à repiquer un rang d’épingles qu’il avait dans le collet de sa vareuse. Il reprit :\par
— Ne disiez-vous pas, capitaine Gertrais, que le \emph{Tamaulipas} ne relâchera point ?\par
— Non. Il va droit au Chili.\par
— En ce cas il ne pourra pas donner de ses nouvelles en route.\par
— Pardon, capitaine Clubin. D’abord il peut remettre des dépêches à tous les bâtiments qu’il rencontre faisant voile pour l’Europe.\par
— C’est juste.\par
— Ensuite il a la boîte aux lettres de la mer.\par
— Qu’appelez-vous la boîte aux lettres de la mer ?\par
— Vous ne connaissez pas ça, capitaine Clubin ?\par
— Non.\par
— Quand on passe le détroit de Magellan.\par
— Eh bien ?\par
— Partout de la neige, toujours gros temps, de vilains mauvais vents, une mer de quatre sous.\par
— Après ?\par
 — Quand vous avez doublé le cap Monmouth.\par
— Bien. Ensuite ?\par
— Ensuite vous doublez le cap Valentin.\par
— Et ensuite ?\par
— Ensuite vous doublez le cap Isidore.\par
— Et puis ?\par
— Vous doublez la pointe Anna.\par
— Bon. Mais qu’est-ce que vous appelez la boîte aux lettres de la mer ?\par
— Nous y sommes. Montagnes à droite, montagnes à gauche. Des pingouins partout, des pétrels-tempêtes. Un endroit terrible. Ah ! mille saints, mille singes ! quel bataclan, et comme ça tape ! La bourrasque n’a pas besoin qu’on aille à son secours. C’est là qu’on surveille la lisse de hourdi ! C’est là qu’on diminue la toile ! C’est là qu’on te vous remplace la grande voile par le foc, et le foc par le tourmentin ! Coups de vent sur coups de vent. Et puis quelquefois quatre, cinq, six jours de cape sèche. Souvent d’un jeu de voiles tout neuf il vous reste de la charpie. Quelle danse ! des rafales à vous faire sauter un trois-mâts comme une puce. J’ai vu sur un brick anglais, le \emph{True blue,} un petit mousse occupé à la gibboom emporté à tous les cinq cent mille millions de tonnerres de Dieu, et la gibboom avec. On va en l’air comme des papillons, quoi. J’ai vu le contre-maître de la \emph{Revenue,} une jolie goélette, arraché de dessus le fore-crostree, et tué roide. J’ai eu ma lisse cassée, et mon serre-gouttière en capilotade. On sort de là avec toutes ses voiles mangées. Des frégates de cinquante font eau comme des paniers. Et la  mauvaise diablesse de côte ! Rien de plus bourru. Des rochers déchiquetés comme par enfantillage. On approche du Port-Famine. Là, c’est pire que pire. Les plus rudes lames que j’aie vues de ma vie. Des parages d’enfer. Tout à coup on aperçoit ces deux mots écrits en rouge : \emph{Post-Office.}\par
— Que voulez-vous dire, capitaine Gertrais ?\par
— Je veux dire, capitaine Clubin, que tout de suite après qu’on a doublé la pointe Anna on voit sur un caillou de cent pieds de haut un grand bâton. C’est un poteau qui a une barrique au cou. Cette barrique, c’est la boîte aux lettres. Il a fallu que les anglais écrivent dessus : \emph{Post-Office.} De quoi se mêlent-ils ? C’est la poste de l’océan ; elle n’appartient pas à cet honorable gentleman, le roi d’Angleterre. Cette boîte aux lettres est commune. Elle appartient à tous les pavillons. \emph{Post-Office !} est-ce assez chinois ! ça vous fait l’effet d’une tasse de thé que le diable vous offrirait tout à coup. Voici maintenant comment se fait le service. Tout bâtiment qui passe expédie au poteau un canot avec ses dépêches. Le navire qui vient de l’Atlantique envoie ses lettres pour l’Europe, et le navire qui vient du Pacifique envoie ses lettres pour l’Amérique. L’officier commandant votre canot met dans le baril votre paquet et y prend le paquet qu’il y trouve. Vous vous chargez de ces lettres-là ; le navire qui viendra après vous se chargera des vôtres. Comme on navigue en sens contraire, le continent d’où vous venez, c’est celui où je vais. Je porte vos lettres, vous portez les miennes. Le baril est bitté au poteau avec une chaîne. Et il pleut ! Et il neige !  Et il grêle ! Une fichue mer ! Les satanicles volent de tous côtés. Le \emph{Tamaulipas} ira par là. Le baril a un bon couvercle à charnière, mais pas de serrure ni de cadenas. Vous voyez qu’on peut écrire à ses amis. Les lettres parviennent.\par
— C’est très drôle, murmura Clubin rêveur.\par
Le capitaine Gertrais-Gaboureau se retourna vers sa chope.\par
— Une supposition que ce garnement de Zuela m’écrit, ce gueux flanque son barbouillage dans la barrique à Magellan et dans quatre mois j’ai le griffonnage de ce gredin. — Ah çà ! capitaine Clubin, est-ce que vous partez demain ?\par
Clubin, absorbé dans une sorte de somnambulisme, n’entendit pas. Le capitaine Gertrais répéta sa question.\par
Clubin se réveilla.\par
— Sans doute, capitaine Gertrais. C’est mon jour. Il faut que je parte demain matin.\par
— Si c’était moi, je ne partirais pas. Capitaine Clubin, la peau des chiens sent le poil mouillé. Les oiseaux de mer viennent depuis deux nuits tourner autour de la lanterne du phare. Mauvais signe. J’ai un storm-glass qui fait des siennes. Nous sommes au deuxième octant de la lune ; c’est le maximum d’humidité. J’ai vu tantôt des pimprenelles qui fermaient leurs feuilles et un champ de trèfles dont les tiges étaient toutes droites. Les vers de terre sortent, les mouches piquent, les abeilles ne s’éloignent pas de leur ruche, les moineaux se consultent. On entend le son des cloches de loin.  J’ai entendu ce soir l’angélus de Saint-Lunaire. Et puis le soleil s’est couché sale. Il y aura demain un fort brouillard. Je ne vous conseille pas de partir. Je crains plus le brouillard que l’ouragan. C’est un sournois, le brouillard.
 \subsection[{A.VI. Livre sixième. Le timonier ivre et le capitaine Sobre}]{A.VI. Livre sixième \\
Le timonier ivre et le capitaine Sobre}
  \subsubsection[{A.VI.1. Les Rochers-Douvres}]{A.VI.1. \\
Les Rochers-Douvres}
\noindent A cinq lieues environ en pleine mer, au sud de Guernesey, vis-à-vis la pointe de Plainmont, entre les îles de la Manche et Saint-Malo, il y a un groupe d’écueils appelé les Rochers-Douvres. Ce lieu est funeste.\par
Cette dénomination, Douvre, \emph{Dover,} appartient à beaucoup d’écueils et de falaises. Il y a notamment près des Côtes du Nord une Roche-Douvre sur laquelle on construit un phare en ce moment, écueil dangereux, mais qu’il ne faut point confondre avec celui-ci.\par
Le point de France le plus proche du rocher Douvres est le cap Bréhant. Le rocher Douvres est un peu plus loin de la côte de France que de la première île de l’archipel normand. La distance de cet écueil à Jersey se mesure à peu près par la grande diagonale de Jersey. Si l’île de Jersey tournait sur la Corbière  comme sur un gond, la pointe Sainte-Catherine irait presque frapper les Douvres. C’est encore là un éloignement de plus de quatre lieues.\par
Dans ces mers de la civilisation les roches les plus sauvages sont rarement désertes. On rencontre des contrebandiers à Hagot, des douaniers à Binic, des celtes à Bréhat, des cultivateurs d’huîtres à Cancale, des chasseurs de lapins à Césambre, l’île de César, des ramasseurs de crabes à Brecqhou, des pêcheurs au chalut aux Minquiers, des pêcheurs à la trouble à Ecréhou. Aux rochers Douvres, personne.\par
Les oiseaux de mer sont là chez eux.\par
Pas de rencontre plus redoutée. Les Casquets où s’est, dit-on, perdue la \emph{Blanche Nef,} le banc du Calvados, les aiguilles de l’île de Wight, la Ronesse qui fait la côte de Beaulieu si dangereuse, le bas-fond de Préel qui étrangle l’entrée de Merquel et qui force de ranger à vingt brasses la balise peinte en rouge, les approches traîtres d’Étables et de Plouha, les deux druides de granit du sud de Guernesey, le vieux Anderlo et le petit Anderlo, la Corbière, les Hanois, l’île de Ras, recommandée à la frayeur par ce proverbe : — \emph{Si jamais tu passes le Ras, si tu ne meurs, tu trembleras ;} — les Mortes-Femmes, le passage de la Boue et de la Frouquie, la Déroute entre Guernesey et Jersey, la Hardent entre les Minquiers et Chausey, le Mauvais Cheval entre Boulay-Bay et Barneville, sont moins mal famés. Il vaudrait mieux affronter tous ces écueils l’un après l’autre que le rocher Douvres une seule fois.\par
Sur toute cette périlleuse mer de la Manche, qui  est la mer Égée de l’occident, le rocher Douvres n’a d’égal en terreur que l’écueil Pater-Noster entre Guernesey et Serk.\par
Et encore, de Pater-Noster on peut faire un signal ; une détresse là peut être secourue. On voit au nord la pointe Dicard, ou d’Icare, et au sud Gros-Nez. Du rocher Douvres, on ne voit rien.\par
La rafale, l’eau, la nuée, l’illimité, l’inhabité. Nul ne passe aux rochers Douvres qu’égaré. Les granits sont d’une stature brutale et hideuse. Partout l’escarpement. L’inhospitalité sévère de l’abîme.\par
C’est la haute mer. L’eau y est très profonde. Un écueil absolument isolé comme le rocher Douvres attire et abrite les bêtes qui ont besoin de l’éloignement des hommes. C’est une sorte de vaste madrépore sous-marin. C’est un labyrinthe noyé. Il y a là, à une profondeur où les plongeurs atteignent difficilement, des antres, des caves, des repaires, des entre-croisements de rues ténébreuses. Les espèces monstrueuses y pullulent. On s’entre-dévore. Les crabes mangent les poissons, et sont eux-mêmes mangés. Des formes épouvantables, faites pour n’être pas vues par l’œil humain, errent dans cette obscurité, vivantes. De vagues linéaments de gueules, d’antennes, de tentacules, de nageoires, d’ailerons, de mâchoires ouvertes, d’écailles, de griffes, de pinces, y flottent, y tremblent, y grossissent, s’y décomposent et s’y effacent dans la transparence sinistre. D’effroyables essaims nageants rôdent, faisant ce qu’ils ont à faire. C’est une ruche d’hydres.\par
 L’horrible est là, idéal.\par
Figurez-vous, si vous pouvez, un fourmillement d’holothuries.\par
Voir le dedans de la mer, c’est voir l’imagination de l’Inconnu. C’est la voir du côté terrible. Le gouffre est analogue à la nuit. Là aussi il y a sommeil, apparent du moins, de la conscience de la création. Là s’accomplissent en pleine sécurité les crimes de l’irresponsable. Là, dans une paix affreuse, les ébauches de la vie, presque fantômes, tout à fait démons, vaquent aux farouches occupations de l’ombre.\par
Il y a quarante ans, deux roches d’une forme extraordinaire signalaient de loin l’écueil Douvres aux passants de l’océan. C’étaient deux pointes verticales, aiguës et recourbées, se touchant presque par le sommet. On croyait voir sortir de la mer les deux défenses d’un éléphant englouti. Seulement c’étaient les défenses, hautes comme des tours, d’un éléphant grand comme une montagne. Ces deux tours naturelles de l’obscure ville des monstres ne laissaient entre elles qu’un étroit passage où se ruait la lame. Ce passage, tortueux et ayant dans sa longueur plusieurs coudes, ressemblait à un tronçon de rue entre deux murs. On nommait ces roches jumelles les deux Douvres : Il y avait la grande Douvre et la petite Douvre ; l’une avait soixante pieds de haut, l’autre quarante. Le va-et-vient de la vague a fini par donner un trait de scie dans la base de ces tours, et le violent coup d’équinoxe du 26 octobre 1859 en a renversé une. Celle qui reste, la petite, est tronquée et fruste.\par
 Un des plus étranges rochers du groupe Douvres s’appelle l’Homme. Celui-là subsiste encore aujourd’hui. Au siècle dernier, des pêcheurs, fourvoyés sur ces brisants, trouvèrent au haut de ce rocher un cadavre. A côté de ce cadavre, il y avait quantité de coquillages vidés. Un homme avait naufragé à ce roc, s’y était réfugié, y avait vécu quelque temps de coquillages et y était mort. De là ce nom, l’Homme.\par
Les solitudes d’eau sont lugubres. C’est le tumulte et le silence. Ce qui se fait là ne regarde plus le genre humain. C’est de l’utilité inconnue. Tel est l’isolement du rocher Douvres. Tout autour, à perte de vue, l’immense tourment des flots.
 \subsubsection[{A.VI.2. Du cognac inespéré}]{A.VI.2. \\
Du cognac inespéré}
\noindent Le vendredi matin, lendemain du départ du \emph{Tamaulipas}, la Durande partit pour Guernesey.\par
Elle quitta Saint-Malo à neuf heures.\par
Le temps était clair, pas de brume ; le vieux capitaine Gertrais-Gaboureau parut avoir radoté.\par
Les préoccupations de sieur Clubin lui avaient décidément fait à peu près manquer son chargement. Il n’avait embarqué que quelques colis d’articles de Paris pour les boutiques de \emph{fancy} de Saint-Pierre-Port, trois caisses pour l’hôpital de Guernesey, l’une de savon jaune, l’autre de chandelle à la baguette, et la troisième de cuir de semelle français et de cordouan choisi. Il rapportait de son précédent chargement une caisse de sucre crushed et trois caisses de thé conjou que la douane française n’avait pas voulu admettre. Sieur Clubin avait embarqué peu de bétail ; quelques bœufs seulement. Ces bœufs étaient dans la cale assez négligemment arrimés.\par
 Il y avait à bord six passagers, un guernesiais, deux malouins marchands de bestiaux, un « touriste », comme on disait déjà à cette époque, un parisien demi-bourgeois, probablement touriste du commerce, et un américain voyageant pour distribuer des bibles.\par
La Durande, sans compter Clubin, le capitaine, portait sept hommes d’équipage, un timonier, un matelot charbonnier, un matelot charpentier, un cuisinier, manœuvrier au besoin, deux chauffeurs et un mousse. L’un des deux chauffeurs était en même temps mécanicien. Ce chauffeur-mécanicien, très brave et très intelligent nègre hollandais, évadé des sucreries de Surinam, s’appelait Imbrancam. Le nègre Imbrancam comprenait et servait admirablement la machine. Dans les premiers temps, il n’avait pas peu contribué, apparaissant tout noir dans sa fournaise, à donner un air diabolique à la Durande.\par
Le timonier, jersiais de naissance et cotentin d’origine, se nommait Tangrouille. Tangrouille était d’une haute noblesse.\par
Ceci était vrai à la lettre. Les îles de la Manche sont, comme l’Angleterre, un pays hiérarchique. Il y existe encore des castes. Les castes ont leurs idées, qui sont leurs défenses. Ces idées des castes sont partout les mêmes, dans l’Inde comme en Allemagne. La noblesse se conquiert par l’épée et se perd par le travail. Elle se conserve par l’oisiveté. Ne rien faire, c’est vivre noblement ; quiconque ne travaille pas est honoré. Un métier fait déchoir. En France autrefois, il n’y avait d’exception que pour les verriers. Vider les bouteilles  étant un peu la gloire des gentilshommes, faire des bouteilles ne leur était point déshonneur. Dans l’archipel de la Manche, ainsi que dans la Grande-Bretagne, qui veut rester noble doit rester riche. Un workman ne peut être un gentleman. L’eût-il été, il ne l’est plus. Tel matelot descend des chevaliers bannerets et n’est qu’un matelot. Il y a trente ans, à Aurigny, un Gorges authentique, qui aurait eu des droits à la seigneurie de Gorges confisquée par Philippe-Auguste, ramassait du varech pieds nus dans la mer. Un Carteret est charretier à Serk. Une mademoiselle de Veulle, arrière-petite-nièce du bailli de Veulle, de son vivant premier magistrat de Jersey, a été domestique chez celui qui écrit ces lignes. Il existe à Jersey un drapier et à Guernesey un cordonnier nommés Gruchy qui se déclarent Grouchy et cousins du maréchal de Waterloo. Les anciens pouillés de l’évêché de Coutances font mention d’une seigneurie de Tangroville, parente évidente de Tancarville sur la basse Seine, qui est Montmorency. Au quinzième siècle Johan de Héroudeville, archer et étoffe du sire de Tangroville, portait derrière lui « son corset et ses autres harnois ». En mai 1371, à Pontorson, à la montre de Bertrand du Guesclin, « monsieur de Tangroville a fait son devoir comme chevalier bachelor ». Dans les îles normandes, si la misère survient, on est vite éliminé de la noblesse. Un changement de prononciation suffit. \emph{Tangroville} devient \emph{Tangrouillle}, et tout est dit.\par
C’est ce qui était arrivé au timonier de la Durande.\par
Il y a à Saint-Pierre-Port, au Bordage, un marchand  de ferraille appelé Ingrouille qui est probablement un Ingroville. Sous Louis le Gros, les Ingroville possédaient trois paroisses dans l’élection de Valognes. Un abbé Trigan a fait l’\emph{Histoire ecclésiastique de Normandie ;} ce chroniqueur Trigan était curé de la seigneurie de Digoville. Le sire de Digoville, s’il était tombé en roture, se nommerait \emph{Digouille.}\par
Tangrouille, ce Tancarville probable et ce Montmorency possible, avait cette antique qualité de gentilhomme, défaut grave pour un timonier, il s’enivrait.\par
Sieur Clubin s’était obstiné à le garder. Il en avait répondu à mess Lethierry.\par
Le timonier Tangrouille ne quittait jamais le navire et couchait à bord.\par
La veille du départ, quand sieur Clubin était venu, à une heure assez avancée de la soirée, faire la visite du bâtiment, Tangrouille était dans son branle et dormait.\par
Dans la nuit Tangrouille s’était réveillé. C’était son habitude nocturne. Tout ivrogne qui n’est pas son maître, a sa cachette. Tangrouille avait la sienne, qu’il nommait sa cambuse. La cambuse secrète de Tangrouille était dans la cale-à-l’eau. Il l’avait mise là pour la rendre invraisemblable. Il croyait être sûr que cette cachette n’était connue que de lui seul. Le capitaine Clubin, étant sobre, était sévère. Le peu de rhum et de gin que le timonier pouvait dérober au guet vigilant du capitaine, il le tenait en réserve dans ce coin mystérieux de la cale-à-l’eau, au fond d’une baille de sonde, et presque toutes les nuits il avait un rendez- vous amoureux avec cette cambuse. La surveillance était rigoureuse, l’orgie était pauvre, et d’ordinaire les excès nocturnes de Tangrouille se bornaient à deux ou trois gorgées, avalées furtivement. Parfois même la cambuse était vide. Cette nuit-là Tangrouille y avait trouvé une bouteille d’eau-de-vie inattendue. Sa joie avait été grande, et sa stupeur plus grande encore. De quel ciel lui tombait cette bouteille ? Il n’avait pu se rappeler quand ni comment il l’avait apportée dans le navire. Il l’avait bue immédiatement. Un peu par prudence ; de peur que cette eau-de-vie ne fût découverte et saisie. Il avait jeté la bouteille à la mer. Le lendemain, quand il prit la barre, Tangrouille avait une certaine oscillation.\par
Il gouverna pourtant à peu près comme d’ordinaire.\par
Quant à Clubin, il était, on le sait, revenu coucher à l’auberge Jean.\par
Clubin portait toujours sous sa chemise une ceinture de voyage en cuir où il gardait un en-cas d’une vingtaine de guinées et qu’il ne quittait que la nuit. Dans l’intérieur de cette ceinture, il y avait son nom, \emph{sieur Clubin}, écrit par lui-même sur le cuir brut à l’encre grasse lithographique, qui est indélébile.\par
En se levant, avant de partir, il avait mis dans cette ceinture la boîte de fer contenant les soixante-quinze mille francs en bank-notes, puis il s’était comme d’habitude bouclé la ceinture autour du corps.
 \subsubsection[{A.VI.3. Propos interrompus}]{A.VI.3. \\
Propos interrompus}
\noindent Le départ se fit allégrement. Les voyageurs, sitôt leurs valises et leurs portemanteaux installés sur et sous les bancs, passèrent cette revue du bateau à laquelle on ne manque jamais, et qui semble obligatoire tant elle est habituelle. Deux des passagers, le touriste et le parisien, n’avaient jamais vu de bateau à vapeur, et, dès les premiers tours de roue, ils admirèrent l’écume. Puis ils admirèrent la fumée. Ils examinèrent pièce à pièce, et presque brin à brin, sur le pont et dans l’entrepont, tous ces appareils maritimes d’anneaux, de crampons, de crochets, de boulons, qui à force de précision et d’ajustement sont une sorte de colossale bijouterie ; bijouterie de fer, dorée avec de la rouille par la tempête. Ils firent le tour du petit canon d’alarme amarré sur le pont, « à la chaîne comme un chien de garde », observa le touriste, et « couvert d’une blouse de toile goudronnée pour l’empêcher de s’enrhumer », ajouta le parisien. En s’éloignant de  terre, on échangea les observations d’usage sur la perspective de Saint-Malo ; un passager émit l’axiome que les approches de mer trompent, et qu’à une lieue de la côte, rien ne ressemble à Ostende comme Dunkerque. On compléta ce qu’il y avait à dire sur Dunkerque par cette observation que ses deux navires-vigies peints en rouge s’appellent l’un \emph{Ruytingen} et l’autre \emph{Mardyck.}\par
Saint-Malo s’amincit au loin, puis s’effaça.\par
L’aspect de la mer était le vaste calme. Le sillage faisait dans l’océan derrière le navire une longue rue frangée d’écume qui se prolongeait presque sans torsion à perte de vue.\par
Guernesey est au milieu d’une ligne droite qu’on tirerait de Saint-Malo en France à Exeter en Angleterre. La ligne droite en mer n’est pas toujours la ligne logique. Pourtant les bateaux à vapeur ont, jusqu’à un certain point, le pouvoir de suivre la ligne droite, refusée aux bateaux à voiles.\par
La mer, compliquée du vent, est un composé de forces. Un navire est un composé de machines. Les forces sont des machines infinies, les machines sont des forces limitées. C’est entre ces deux organismes, l’un inépuisable, l’autre intelligent, que s’engage ce combat qu’on appelle la navigation.\par
Une volonté dans un mécanisme fait contre-poids à l’infini. L’infini, lui aussi, contient un mécanisme. Les éléments savent ce qu’ils font et où ils vont. Aucune force n’est aveugle. L’homme doit épier les forces, et tâcher de découvrir leur itinéraire.\par
 En attendant que la loi soit trouvée, la lutte continue, et dans cette lutte la navigation à la vapeur est une sorte de victoire perpétuelle que le génie humain remporte à toute heure du jour sur tous les points de la mer. La navigation à la vapeur a cela d’admirable qu’elle discipline le navire. Elle diminue l’obéissance au vent et augmente l’obéissance à l’homme.\par
Jamais la Durande n’avait mieux travaillé en mer que ce jour-là. Elle se comportait merveilleusement.\par
Vers onze heures, par une fraîche brise de nord-nord-ouest, la Durande se trouvait au large des Minquiers, donnant peu de vapeur, naviguant à l’ouest, tribord amures et au plus près du vent. Le temps était toujours clair et beau. Cependant les chalutiers rentraient.\par
Peu à peu, comme si chacun songeait à regagner le port, la mer se nettoyait de navires.\par
On ne pouvait dire que la Durande tînt tout à fait sa route accoutumée. L’équipage n’avait aucune préoccupation, la confiance dans le capitaine était absolue ; toutefois, peut-être par la faute du timonier, il y avait quelque déviation. La Durande paraissait plutôt aller vers Jersey que vers Guernesey. Un peu après onze heures, le capitaine rectifia la direction et l’on mit franchement le cap sur Guernesey. Ce ne fut qu’un peu de temps perdu. Dans les jours courts le temps perdu a ses inconvénients. Il faisait un beau soleil de février.\par
Tangrouille, dans l’état où il était, n’avait plus le pied très sûr ni le bras très ferme. Il en résultait que  le brave timonier embardait souvent, ce qui ralentissait la marche.\par
Le vent était à peu près tombé.\par
Le passager guernesiais, qui tenait à la main une longue-vue, la braquait de temps en temps sur un petit flocon de brume grisâtre lentement charrié par le vent à l’extrême horizon à l’ouest, et qui ressemblait à une ouate où il y aurait de la poussière.\par
Le capitaine Clubin avait son austère mine puritaine ordinaire. Il paraissait redoubler d’attention.\par
Tout était paisible et presque riant à bord de la Durande, les passagers causaient. En fermant les yeux dans une traversée, on peut juger de l’état de la mer par le trémolo des conversations. La pleine liberté d’esprit des passagers répond à la parfaite tranquillité de l’eau.\par
Il est impossible, par exemple, qu’une conversation telle que celle-ci ait lieu autrement que par une mer très calme :\par
— Monsieur, voyez donc cette jolie mouche verte et rouge.\par
— Elle s’est égarée en mer et se repose sur le navire.\par
— Une mouche se fatigue peu.\par
— Au fait, c’est si léger. Le vent la porte.\par
— Monsieur, on a pesé une once de mouches, puis on les a comptées, et l’on en a trouvé six mille deux cent soixante-huit.\par
Le guernesiais à la longue-vue avait abordé les malouins marchands de bœufs, et leur parlage était quelque chose en ce genre :\par
 — Le bœuf d’Aubrac a le torse rond et trapu, les jambes courtes, le pelage fauve. Il est lent au travail, à cause de la brièveté des jambes.\par
— Sous ce rapport, le Salers vaut mieux que l’Aubrac.\par
— Monsieur, j’ai vu deux beaux bœufs dans ma vie. Le premier avait les jambes basses, l’avant épais, la culotte pleine, les hanches larges, une bonne longueur de la nuque à la croupe, une bonne hauteur au garrot, les maniements riches, la peau facile à détacher. Le second offrait tous les signes d’un engraissement judicieux. Torse ramassé, encolure forte, jambes légères, robe blanche et rouge, culotte retombante.\par
— Ça, c’est la race cotentine.\par
— Oui, mais ayant eu quelque rapport avec le taureau angus ou le taureau suffolk.\par
— Monsieur, vous me croirez si vous voulez, dans le midi il y a des concours d’ânes.\par
— D’ânes ?\par
— D’ânes. Comme j’ai l’honneur. Et ce sont les laids qui sont les beaux.\par
— Alors c’est comme les mulassières. Ce sont les laides qui sont les bonnes.\par
— Justement. La jument poitevine. Gros ventre, grosses jambes.\par
— La meilleure mulassière connue, c’est une barrique sur quatre poteaux.\par
— La beauté des bêtes n’est pas la même que la beauté des hommes.\par
 — Et surtout des femmes.\par
— C’est juste.\par
— Moi, je tiens à ce qu’une femme soit jolie.\par
— Moi, je tiens à ce qu’elle soit bien mise.\par
— Oui, nette, propre, tirée à quatre épingles, astiquée.\par
— L’air tout neuf. Une jeune fille, ça doit toujours sortir de chez le bijoutier.\par
— Je reviens à mes bœufs. J’ai vu vendre ces deux bœufs au marché de Thouars.\par
— Le marché de Thouars, je le connais. Les Bonneau de la Rochelle et les Bahu, les marchands de blé de Marans, je ne sais pas si vous en avez entendu parler, devaient venir à ce marché-là.\par
Le touriste et le parisien causaient avec l’américain des bibles. La conversation, là aussi, était au beau fixe.\par
— Monsieur, disait le touriste, voici quel est le tonnage flottant du monde civilisé : France, sept cent seize mille tonneaux ; Allemagne, un million ; États-Unis, cinq millions ; Angleterre, cinq millions cinq cent mille. Ajoutez le contingent des petits pavillons. Total : douze millions neuf cent quatre mille tonneaux distribués dans cent quarante-cinq mille navires épars sur l’eau du globe.\par
L’américain interrompit :\par
— Monsieur, ce sont les États-Unis qui ont cinq millions cinq cent mille.\par
— J’y consens, dit le touriste. Vous êtes américain ?\par
 — Oui, monsieur.\par
— J’y consens encore.\par
Il y eut un silence, l’américain missionnaire se demanda si c’était le cas d’offrir une bible.\par
— Monsieur, repartit le touriste, est-il vrai que vous ayez le goût des sobriquets en Amérique au point d’en affubler tous vos gens célèbres, et que vous appeliez votre fameux banquier missourien Thomas Benton, \emph{le vieux lingot ?}\par
— De même que nous nommons Zacharie Taylor \emph{le vieux Zach.}\par
— Et le général Harrison \emph{le vieux Tip,} n’est-ce pas ? et le général Jackson \emph{le vieil Hickory ?}\par
— Parce que Jackson est dur comme le bois hickory, et parce que Harrison a battu les peaux-rouges à Tippecanoe.\par
— C’est une mode byzantine que vous avez là.\par
— C’est notre mode. Nous appelons Van Buren \emph{le petit sorcier}, Seward, qui a fait faire les petites coupures des billets de banque, \emph{Billy-le-Petit,} et Douglas, le sénateur démocrate de l’Illinois, qui a quatre pieds de haut et une grande éloquence, \emph{le Petit Géant}. Vous pouvez aller du Texas au Maine, vous ne rencontrerez personne qui dise ce nom : Cass, on dit : \emph{le grand Michigantier ;} ni ce nom : Clay, on dit : \emph{le garçon de moulin à la balafre.} Clay est fils d’un meunier.\par
— J’aimerais mieux dire Clay ou Class, observa le parisien, c’est plus court.\par
— Vous manqueriez d’usage du monde. Nous nommons Corwin, qui est secrétaire de la trésorerie, \emph{le  garçon de charrette}. Daniel Webster est \emph{Dan-le-noir. }Quant à Ninfield Scott, comme sa première pensée, après avoir battu les anglais à Chippeway, a été de se mettre à table, nous l’appelons \emph{Vite-une-assiette-de-soupe.}\par
Le flocon de brume aperçu dans le lointain avait grandi. Il occupait maintenant sur l’horizon un segment d’environ quinze degrés. On eût dit un nuage se traînant sur l’eau faute de vent. Il n’y avait presque plus de brise. La mer était plate. Quoiqu’il ne fût pas midi, le soleil pâlissait. Il éclairait, mais ne chauffait plus.\par
— Je crois, dit le touriste, que le temps va changer.\par
— Nous aurons peut-être de la pluie, dit le parisien.\par
— Ou du brouillard, reprit l’américain.\par
— Monsieur, repartit le touriste, en Italie, c’est à Molfetta qu’il tombe le moins de pluie, et à Tolmezzo qu’il en tombe le plus.\par
A midi, selon l’usage de l’archipel, on sonna la cloche pour dîner. Dîna qui voulut. Quelques passagers portaient avec eux leur en-cas, et mangèrent gaîment sur le pont. Clubin ne dîna point.\par
Tout en mangeant, les conversations allaient leur train.\par
Le guernesiais, ayant le flair des bibles, s’était rapproché de l’américain. L’américain lui dit :\par
— Vous connaissez cette mer-ci ?\par
— Sans doute, j’en suis.\par
— Et moi aussi, dit l’un des malouins.\par
Le guernesiais adhéra d’un salut, et reprit :\par
— A présent, nous sommes au large, mais je  n’aurais pas aimé avoir du brouillard quand nous étions devers les Minquiers.\par
L’américain dit au malouin :\par
— Les insulaires sont plus de la mer que les côtiers.\par
— C’est vrai, nous autres gens de la côte, nous n’avons que le demi-bain.\par
— Qu’est-ce que c’est que ça, les Minquiers ? continua l’américain.\par
Le malouin répondit :\par
— C’est des cailloux très mauvais.\par
— Il y a aussi les Grelets, fit le guernesiais.\par
— Parbleu, répliqua le malouin.\par
— Et les Chouas, ajouta le guernesiais.\par
Le malouin éclata de rire.\par
— A ce compte-là, dit-il, il y a aussi les Sauvages.\par
— Et les Moines, observa le guernesiais.\par
— Et le Canard, s’écria le malouin.\par
— Monsieur, repartit le guernesiais poliment, vous avez réponse à tout.\par
— Malouin, malin.\par
Cette réponse faite, le malouin cligna de l’œil.\par
Le touriste interposa une question.\par
— Est-ce que nous avons à traverser toute cette rocaille ?\par
— Point. Nous l’avons laissée au sud-sud-est. Elle est derrière nous.\par
Et le guernesiais poursuivit :\par
— Tant gros rochers que menus, les Grelets ont cinquante-sept pointes.\par
— Et les Minquiers quarante-huit, dit le malouin.\par
 Ici le dialogue se concentra entre le malouin et le guernesiais.\par
— Il me semble, monsieur de Saint-Malo, qu’il y a trois rochers que vous ne comptez pas.\par
— Je compte tout.\par
— De la Dérée au Maître-Ile ?\par
— Oui.\par
— Et les Maisons ?\par
— Qui sont sept rochers au milieu des Minquiers. Oui.\par
— Je vois que vous connaissez les pierres.\par
— Si on ne connaissait pas les pierres, on ne serait pas de Saint-Malo.\par
— Ça fait plaisir d’entendre les raisonnements des français.\par
Le malouin salua à son tour, et dit :\par
— Les Sauvages sont trois rochers.\par
— Et les Moines deux.\par
— Et le Canard un.\par
— Le Canard, ça dit un seul.\par
— Non, car la Suarde, c’est quatre rochers.\par
— Qu’appelez-vous la Suarde ? demanda le guernesiais.\par
— Nous appelons la Suarde ce que vous appelez les Chouas.\par
— Il ne fait pas bon passer entre les Chouas et le Canard.\par
— Ça n’est possible qu’aux oiseaux.\par
— Et aux poissons.\par
— Pas trop. Dans les gros temps, ils se cognent aux murs.\par
 — Il y a du sable dans les Minquiers.\par
— Autour des Maisons.\par
— C’est huit rochers qu’on voit de Jersey.\par
— De la grève d’Azette, c’est juste. Pas huit, sept.\par
— A mer retirée, on peut se promener dans les Minquiers.\par
— Sans doute, il y a de la découverte.\par
— Et les Dirouilles ?\par
— Les Dirouilles n’ont rien de commun avec les Minquiers.\par
— Je veux dire que c’est dangereux.\par
— C’est du côté de Granville.\par
— On voit que, comme nous, vous gens de Saint-Malo, vous avez amour de naviguer dans ces mers.\par
— Oui, répondit le malouin, avec cette différence que nous disons : nous avons habitude, et que vous dites : nous avons amour.\par
— Vous êtes de bons marins.\par
— Je suis marchand de bœufs.\par
— Qui donc était de Saint-Malo, déjà ?\par
— Surcouf.\par
— Un autre ?\par
— Duguay-Trouin.\par
Ici le voyageur de commerce parisien intervint.\par
— Duguay-Trouin ? il fut pris par les anglais. Il était aussi aimable que brave. Il sut plaire à une jeune anglaise. Ce fut elle qui brisa ses fers.\par
En ce moment une voix tonnante cria :\par
— Tu es ivre !
 \subsubsection[{A.VI.4. Ou se déroulent toutes les qualités du capitaine clubin}]{A.VI.4. \\
Ou se déroulent toutes les qualités du capitaine clubin}
\noindent Tous se retournèrent.\par
C’était le capitaine qui interpellait le timonier.\par
Sieur Clubin ne tutoyait personne. Pour qu’il jetât au timonier Tangrouille une telle apostrophe, il fallait que Clubin fût fort en colère, ou voulût fort le paraître.\par
Un éclat de colère à propos dégage la responsabilité, et quelquefois la transpose.\par
Le capitaine, debout sur le pont de commandement entre les deux tambours, regardait fixement le timonier. Il répéta entre ses dents : Ivrogne ! L’honnête Tangrouille baissa la tête.\par
Le brouillard s’était développé. Il occupait maintenant près de la moitié de l’horizon. Il avançait dans tous les sens à la fois ; il y a dans le brouillard quelque chose de la goutte d’huile. Cette brume s’élargissait insensiblement. Le vent la poussait sans hâte et sans bruit. Elle prenait peu à peu possession de l’océan. Elle  venait du nord-ouest et le navire l’avait devant sa proue. C’était comme une vaste falaise mouvante et vague. Elle se coupait sur la mer comme une muraille. Il y avait un point précis où l’eau immense entrait sous le brouillard et disparaissait.\par
Ce point d’entrée dans le brouillard était encore à une demi-lieue environ. Si le vent changeait, on pouvait éviter l’immersion dans la brume ; mais il fallait qu’il changeât tout de suite. La demi-lieue d’intervalle se comblait et décroissait à vue d’œil ; la Durande marchait, le brouillard marchait aussi. Il venait au navire et le navire allait à lui.\par
Clubin commanda d’augmenter la vapeur et d’obliquer à l’est.\par
On côtoya ainsi quelque temps le brouillard, mais il avançait toujours. Le navire pourtant était encore en plein soleil.\par
Le temps se perdait dans ces manœuvres qui pouvaient difficilement réussir. La nuit vient vite en février.\par
Le guernesiais considérait cette brume. Il dit aux malouins :\par
— Que c’est un hardi brouillard.\par
— Une vraie malpropreté sur la mer, observa l’un des malouins.\par
L’autre malouin ajouta :\par
— Voilà qui gâte une traversée.\par
Le guernesiais s’approcha de Clubin.\par
— Capitaine Clubin, j’ai peur que nous ne soyons gagnés par le brouillard.\par
 Clubin répondit :\par
— Je voulais rester à Saint-Malo, mais on m’a conseillé de partir.\par
— Qui ça ?\par
— Des anciens.\par
— Au fait, reprit le guernesiais, vous avez eu raison de partir. Qui sait s’il n’y aura pas tempête demain ? Dans cette saison on peut attendre pour du pire.\par
Quelques minutes après, la Durande entrait dans le banc de brume.\par
Ce fut un instant singulier. Tout à coup ceux qui étaient à l’arrière ne virent plus ceux qui étaient à l’avant. Une molle cloison grise coupa en deux le bateau.\par
Puis le navire entier plongea sous la brume. Le soleil ne fut plus qu’une espèce de grosse lune. Brusquement, tout le monde grelotta. Les passagers endossèrent leur pardessus et les matelots leur suroît. La mer, presque sans un pli, avait la froide menace de la tranquillité. Il semble qu’il y ait un sous-entendu dans cet excès de calme. Tout était blafard et blême. La cheminée noire et la fumée noire luttaient contre cette lividité qui enveloppait le navire.\par
La dérivation à l’est était sans but désormais. Le capitaine remit le cap sur Guernesey et augmenta la vapeur.\par
Le passager guernesiais, rôdant autour de la chambre à feu, entendit le nègre Imbrancam qui parlait au chauffeur son camarade. Le passager prêta l’oreille. Le nègre disait :\par
 — Ce matin dans le soleil nous allions lentement ; à présent dans le brouillard nous allons vite.\par
Le guernesiais revint vers sieur Clubin.\par
— Capitaine Clubin, il n’y a pas de soin, pourtant ne donnons-nous pas trop de vapeur ?\par
— Que voulez-vous, monsieur ? il faut bien regagner le temps perdu par la faute de cet ivrogne de timonier.\par
— C’est vrai, capitaine Clubin.\par
Et Clubin ajouta :\par
— Je me dépêche d’arriver. C’est assez du brouillard, ce serait trop de la nuit.\par
Le guernesiais rejoignit les malouins, et leur dit :\par
— Nous avons un excellent capitaine.\par
Par intervalles de grandes lames de brume, qu’on eût dit cardées, survenaient pesamment et cachaient le soleil. Ensuite, il reparaissait plus pâle et comme malade. Le peu qu’on entrevoyait du ciel ressemblait aux bandes d’air sales et tachées d’huile d’un vieux décor de théâtre.\par
La Durande passa à proximité d’un coutre qui avait jeté l’ancre par prudence. C’était le \emph{Shealtiel} de Guernesey. Le patron du coutre remarqua la vitesse de la Durande. Il lui sembla aussi qu’elle n’était pas dans la route exacte. Elle lui parut trop appuyer à l’ouest. Ce navire à toute vapeur dans le brouillard l’étonna.\par
Vers deux heures, la brume était si épaisse que le capitaine dut quitter la passerelle et se rapprocher du timonier. Le soleil s’était évanoui, tout était brouillard. Il y avait sur la Durande une sorte d’obscurité blanche.  On naviguait dans de la pâleur diffuse. On ne voyait plus le ciel et on ne voyait plus la mer.\par
Il n’y avait plus de vent.\par
Le bidon à térébenthine suspendu à un anneau sous la passerelle des tambours n’avait pas même une oscillation.\par
Les passagers étaient devenus silencieux.\par
Toutefois le parisien, entre ses dents, fredonnait la chanson de Béranger \emph{Un jour le bon Dieu s’éveillant.}\par
Un des malouins lui adressa la parole.\par
— Monsieur vient de Paris ?\par
— Oui, monsieur. \emph{Il mit la tête à la fenêtre.}\par
— Qu’est-ce qu’on fait à Paris ?\par
— \emph{Leur planète a péri peut-être.} — Monsieur, à Paris tout marche de travers.\par
— Alors c’est sur terre comme sur mer.\par
— C’est vrai que nous avons là un fichu brouillard.\par
— Et qui peut faire des malheurs.\par
Le parisien s’écria :\par
— Mais pourquoi ça, des malheurs ? à propos de quoi, des malheurs ? à quoi ça sert-il, des malheurs ? C’est comme l’incendie de l’Odéon. Voilà des familles sur la paille. Est-ce que c’est juste ? Tenez, monsieur, je ne connais pas votre religion, mais moi je ne suis pas content.\par
— Ni moi, fit le malouin.\par
— Tout ce qui se passe ici-bas, reprit le parisien, fait l’effet d’une chose qui se détraque. J’ai dans l’idée que le bon Dieu n’y est pas.\par
Le malouin se gratta le haut de la tête comme  quelqu’un qui cherche à comprendre. Le parisien continua :\par
— Le bon Dieu est absent. On devrait rendre un décret pour forcer Dieu à résidence. Il est à sa maison de campagne et ne s’occupe pas de nous. Aussi tout va de guingois. Il est évident, mon cher monsieur, que le bon Dieu n’est plus dans le gouvernement, qu’il est en vacances, et que c’est le vicaire, quelque ange séminariste, quelque crétin avec des ailes de moineau, qui mène les affaires.\par
Moineau fut articulé \emph{moigneau}, prononciation de gamin faubourien.\par
Le capitaine Clubin, qui s’était approché des deux causeurs, posa sa main sur l’épaule du parisien.\par
— Chut ! dit-il. Monsieur, prenez garde à vos paroles. Nous sommes en mer.\par
Personne ne dit plus mot.\par
Au bout de cinq minutes, le guernesiais, qui avait tout entendu, murmura à l’oreille du malouin :\par
— Et un capitaine religieux !\par
Il ne pleuvait pas, et l’on se sentait mouillé. On ne se rendait compte du chemin qu’on faisait que par une augmentation de malaise. Il semblait qu’on entrât dans de la tristesse. Le brouillard fait le silence sur l’océan ; il assoupit la vague et étouffe le vent. Dans ce silence, le râle de la Durande avait on ne sait quoi d’inquiet et de plaintif.\par
On ne rencontrait plus de navires. Si, au loin, soit du côté de Guernesey, soit du côté de Saint-Malo, quelques bâtiments étaient en mer hors du  brouillard, pour eux la Durande, submergée dans la brume, n’était pas visible, et sa longue fumée, rattachée à rien, leur faisait l’effet d’une comète noire dans un ciel blanc.\par
Tout à coup Clubin cria :\par
— Faichien ! tu viens de donner un faux coup. Tu vas nous faire des avaries. Tu mériterais d’être mis aux fers. Va-t’en, ivrogne !\par
Et il prit la barre.\par
Le timonier humilié se réfugia dans les manœuvres de l’avant.\par
Le guernesiais dit :\par
— Nous voilà sauvés.\par
La marche continua, rapide.\par
Vers trois heures le dessous de la brume commença à se soulever, et l’on revit de la mer.\par
— Je n’aime pas ça, dit le guernesiais.\par
La brume en effet ne peut être soulevée que par le soleil ou par le vent. Par le soleil c’est bon ; par le vent c’est moins bon. Or il était trop tard pour le soleil. A trois heures, en février, le soleil faiblit. Une reprise de vent, à ce point critique de la journée, est peu désirable. C’est souvent une annonce d’ouragan.\par
Du reste, s’il y avait de la brise, on la sentait à peine.\par
Clubin, l’œil sur l’habitacle, tenant la barre et gouvernant, mâchait entre ses dents des paroles comme celles-ci qui arrivaient jusqu’aux passagers :\par
— Pas de temps à perdre. Cet ivrogne nous a retardés.\par
 Son visage était d’ailleurs absolument sans expression.\par
La mer était moins dormante sous la brume. On y entrevoyait quelques lames. Des lumières glacées flottaient à plat sur l’eau. Ces plaques de lueur sur la vague préoccupent les marins. Elles indiquent des trouées faites par le vent supérieur dans le plafond de brume. La brume se soulevait, et retombait plus dense. Parfois l’opacité était complète. Le navire était pris dans une vraie banquise de brouillard. Par intervalles ce cercle redoutable s’entr’ouvrait comme une tenaille, laissait voir un peu d’horizon, puis se refermait.\par
Le guernesiais, armé de sa longue-vue, se tenait comme une vedette à l’avant du bâtiment.\par
Une éclaircie se fit, puis s’effaça.\par
Le guernesiais se retourna effaré.\par
— Capitaine Clubin !\par
— Qu’y a-t-il ?\par
— Nous gouvernons droit sur les Hanois.\par
— Vous vous trompez, dit Clubin froidement.\par
Le guernesiais insista :\par
— J’en suis sûr.\par
— Impossible.\par
— Je viens d’apercevoir du caillou à l’horizon.\par
— Où ?\par
— Là.\par
— C’est le large. Impossible.\par
Et Clubin maintint le cap sur le point indiqué par le passager.\par
 Le guernesiais ressaisit sa longue-vue.\par
Un moment après il accourut à l’arrière.\par
— Capitaine !\par
— Eh bien ?\par
— Virez de bord.\par
— Pourquoi ?\par
— Je suis sûr d’avoir vu de la roche très haute et tout près. C’est le grand Hanois.\par
— Vous aurez vu du brouillard plus épais.\par
— C’est le grand Hanois. Virez de bord, au nom du ciel !\par
Clubin donna un coup de barre.
 \subsubsection[{A.VI.5. Clubin met le comble a l’admiration}]{A.VI.5. \\
Clubin met le comble a l’admiration}
\noindent On entendit un craquement. Le déchirement d’un flanc de navire sur un bas-fond en pleine mer est un des bruits les plus lugubres qu’on puisse rêver. La Durande s’arrêta court.\par
Du choc plusieurs passagers tombèrent et roulèrent sur le pont.\par
Le guernesiais leva les mains au ciel.\par
— Sur les Hanois ! quand je le disais !\par
Un long cri éclata sur le navire.\par
— Nous sommes perdus.\par
La voix de Clubin, sèche et brève, domina le cri.\par
— Personne n’est perdu ! Et silence !\par
Le torse noir d’Imbrancam nu jusqu’à la ceinture sortit du carré de la chambre à feu.\par
Le nègre dit avec calme :\par
— Capitaine, l’eau entre. La machine va s’éteindre.\par
Le moment fut épouvantable.\par
Le choc avait ressemblé à un suicide. On l’eût fait  exprès qu’il n’eût pas été plus terrible. La Durande s’était ruée comme si elle attaquait le rocher. Une pointe de roche était entrée dans le navire comme un clou. Plus d’une toise carrée de vaigres avait éclaté, l’étrave était rompue, l’élancement fracassé, l’avant effondré, la coque, ouverte, buvait la mer avec un bouillonnement horrible. C’était une plaie par où entrait le naufrage. Le contre-coup avait été si violent qu’il avait brisé à l’arrière les sauvegardes du gouvernail, descellé et battant. On était défoncé par l’écueil, et, autour du navire, on ne voyait rien, que le brouillard épais et compacte, et maintenant presque noir. La nuit arrivait.\par
La Durande plongeait de l’avant. C’était le cheval qui a dans les entrailles le coup de corne du taureau.\par
Elle était morte.\par
L’heure de la demi-remontée se faisait sentir sur la mer.\par
Tangrouille était dégrisé ; personne n’est ivre dans un naufrage ; il descendit dans l’entrepont, remonta et dit :\par
— Capitaine, l’eau barrotte la cale. Dans dix minutes, l’eau sera au ras des dalots.\par
Les passagers couraient sur le pont éperdus, se tordant les bras, se penchant par-dessus le bord, regardant la machine, faisant tous les mouvements inutiles de la terreur. Le touriste s’était évanoui.\par
Clubin fit signe de la main, on se tut. Il interrogea Imbrancam :\par
 — Combien de temps la machine peut-elle travailler encore ?\par
— Cinq ou six minutes.\par
Puis il interrogea le passager guernesiais :\par
— J’étais à la barre. Vous avez observé le rocher. Sur quel banc des Hanois sommes-nous ?\par
— Sur la Mauve. Tout à l’heure, dans l’éclaircie, j’ai très bien reconnu la Mauve.\par
— Étant sur la Mauve, reprit Clubin, nous avons le grand Hanois à bâbord et le petit Hanois à tribord. Nous sommes à un mille de terre.\par
L’équipage et les passagers écoutaient, frémissants d’anxiété et d’attention, l’œil fixé sur le capitaine.\par
Alléger le navire était sans but, et d’ailleurs impossible. Pour vider la cargaison à la mer, il eût fallu ouvrir les sabords et augmenter les chances d’entrée de l’eau. Jeter l’ancre était inutile ; on était cloué. D’ailleurs, sur ce fond à faire basculer l’ancre, la chaîne eût probablement surjouaillé. La machine n’étant pas endommagée et restant à la disposition du navire tant que le feu ne serait pas éteint, c’est-à-dire pour quelques minutes encore, on pouvait faire force de roues et de vapeur, reculer et s’arracher de l’écueil. En ce cas, on sombrait immédiatement. Le rocher, jusqu’à un certain point, bouchait l’avarie et gênait le passage de l’eau. Il faisait obstacle. L’ouverture désobstruée, il serait impossible d’aveugler la voie d’eau et de franchir les pompes. Qui retire le poignard d’une plaie au cœur, tue sur-le-champ le blessé. Se dégager du rocher, c’était couler à fond.\par
 Les bœufs, atteints par l’eau dans la cale, commençaient à mugir.\par
Clubin commanda :\par
— La chaloupe à la mer.\par
Imbrancam et Tangrouille se précipitèrent et défirent les amarres. Le reste de l’équipage regardait, pétrifié.\par
— Tous à la manœuvre, cria Clubin.\par
Cette fois, tous obéirent.\par
Clubin, impassible, continua, dans cette vieille langue de commandement que ne comprendraient pas les marins d’à présent :\par
— Abraquez. — Faites une marguerite si le cabestan est entravé. — Assez de virage. — Amenez. — Ne laissez pas se joindre les poulies des francs-funains. — Affalez. — Amenez vivement des deux bouts. — Ensemble. — Garez qu’elle ne pique. — Il y a trop de frottement. — Touchez les garants de la caliorne. — Attention.\par
La chaloupe était en mer.\par
Au même instant, les roues de la Durande s’arrêtèrent, la fumée cessa, le fourneau était noyé.\par
Les passagers, glissant le long de l’échelle ou s’accrochant aux manœuvres courantes, se laissèrent tomber dans la chaloupe plus qu’ils n’y descendirent. Imbrancam enleva le touriste évanoui, le porta dans la chaloupe, puis remonta sur le navire.\par
Les matelots se ruaient à la suite des passagers. Le mousse avait roulé sous les pieds. On marchait sur l’enfant.\par
 Imbrancam barra le passage.\par
— Personne avant le moço, dit-il.\par
Il écarta de ses deux bras noirs les matelots, saisit le mousse, et le tendit au passager guernesiais qui, debout dans la chaloupe, reçut l’enfant.\par
Le mousse sauvé, Imbrancam se rangea et dit aux autres :\par
— Passez.\par
Cependant Clubin était allé à sa cabine et avait fait un paquet des papiers du bord et des instruments. Il ôta la boussole de l’habitacle. Il remit les papiers et les instruments à Imbrancam et la boussole à Tangrouille, et leur dit : Descendez dans la chaloupe.\par
Ils descendirent. L’équipage les avait précédés. La chaloupe était pleine. Le flot rasait le bord.\par
— Maintenant, cria Clubin, partez.\par
Un cri s’éleva de la chaloupe.\par
— Et vous, capitaine ?\par
— Je reste.\par
Des gens qui naufragent ont peu le temps de délibérer et encore moins le temps de s’attendrir. Cependant ceux qui étaient dans la chaloupe et relativement en sûreté eurent une émotion qui n’était pas pour eux-mêmes. Toutes les voix insistèrent en même temps.\par
— Venez avec nous, capitaine.\par
— Je reste.\par
Le guernesiais, qui était au fait de la mer, répliqua :\par
— Capitaine, écoutez. Vous êtes échoué sur les Hanois. A la nage on n’a qu’un mille à faire pour  gagner Plainmont. Mais en barque on ne peut aborder qu’à la Rocquaine, et c’est deux milles. Il y a des brisants et du brouillard. Cette chaloupe n’arrivera pas à la Rocquaine avant deux heures d’ici. Il fera nuit noire. La marée monte, le vent fraîchit. Une bourrasque est proche. Nous ne demandons pas mieux que de revenir vous chercher ; mais, si le gros temps éclate, nous ne pourrons pas. Vous êtes perdu si vous demeurez. Venez avec nous.\par
Le parisien intervint :\par
— La chaloupe est pleine et trop pleine, c’est vrai, et un homme de plus ce sera un homme de trop. Mais nous sommes treize, c’est mauvais pour la barque, et il vaut encore mieux la surcharger d’un homme que d’un chiffre. Venez, capitaine.\par
Tangrouille ajouta :\par
— Tout est de ma faute, et pas de la vôtre. Ce n’est pas juste que vous demeuriez.\par
— Je reste, dit Clubin. Le navire sera dépecé par la tempête cette nuit. Je ne le quitterai pas. Quand le navire est perdu, le capitaine est mort. On dira de moi : Il a fait son devoir jusqu’au bout. Tangrouille, je vous pardonne.\par
Et croisant les bras, il cria :\par
— Attention au commandement. Largue en bande l’amarre. Partez.\par
La chaloupe s’ébranla. Imbrancam avait saisi le gouvernail. Toutes les mains qui ne ramaient pas s’élevèrent vers le capitaine. Toutes les bouches crièrent : Hurrah pour le capitaine Clubin !\par
 — Voilà un admirable homme, dit l’américain.\par
— Monsieur, répondit le guernesiais, c’est le plus honnête homme de toute la mer.\par
Tangrouille pleurait.\par
— Si j’avais eu du cœur, murmura-t-il à demi-voix, je serais demeuré avec lui.\par
La chaloupe s’enfonça dans le brouillard et s’effaça.\par
On ne vit plus rien.\par
Le frappement des rames décrut et s’évanouit.\par
Clubin resta seul.
 \subsubsection[{A.VI.6. Un intérieur d’abime éclairé}]{A.VI.6. \\
Un intérieur d’abime éclairé}
\noindent Quand cet homme se vit sur ce rocher, sous ce nuage, au milieu de cette eau, loin de tout contact vivant, loin de tout bruit humain, laissé pour mort, seul entre la mer qui montait et la nuit qui venait, il eut une joie profonde.\par
Il avait réussi.\par
Il tenait son rêve. La lettre de change à longue échéance qu’il avait tirée sur la destinée lui était payée.\par
Pour lui, être abandonné, c’était être délivré. Il était sur les Hanois, à un mille de la terre ; il avait soixante-quinze mille francs. Jamais plus savant naufrage n’avait été accompli. Rien n’avait manqué ; il est vrai que tout était prévu. Clubin, dès sa jeunesse, avait eu une idée : mettre l’honnêteté comme enjeu dans la roulette de la vie, passer pour homme probe et partir de là, attendre sa belle, laisser la martingale s’enfler, trouver le joint, deviner le moment ; ne pas  tâtonner, saisir ; faire un coup et n’en faire qu’un, finir par une rafle, laisser derrière lui les imbéciles. Il entendait réussir en une fois ce que les escrocs bêtes manquent vingt fois de suite, et, tandis qu’ils aboutissent à la potence, aboutir, lui, à la fortune. Rantaine rencontré avait été son trait de lumière. Il avait immédiatement construit son plan. Faire rendre gorge à Rantaine ; quant à ses révélations possibles, les frapper de nullité en disparaissant ; passer pour mort, la meilleure des disparitions ; pour cela perdre la Durande. Ce naufrage était nécessaire. Par-dessus le marché, s’en aller en laissant une bonne renommée, ce qui faisait de toute son existence un chef-d’œuvre. Qui eût vu Clubin dans ce naufrage eût cru voir un démon, heureux.\par
Il avait vécu toute sa vie pour cette minute-là.\par
Toute sa personne exprima ce mot : Enfin ! Une sérénité épouvantable blêmit sur ce front obscur. Son œil terne et au fond duquel on croyait voir une cloison devint profond et terrible. L’embrasement intérieur de cette âme s’y réverbéra.\par
Le for intérieur a, comme la nature externe, sa tension électrique. Une idée est un météore ; à l’instant du succès, les méditations amoncelées qui l’ont préparé s’entr’ouvrent, et il en jaillit une étincelle ; avoir en soi la serre du mal et sentir une proie dedans, c’est un bonheur qui a son rayonnement ; une mauvaise pensée qui triomphe illumine un visage ; de certaines combinaisons réussies, de certains buts atteints, de certaines félicités féroces, font apparaître et disparaître  dans les yeux des hommes de lugubres épanouissements lumineux. C’est de l’orage joyeux, c’est de l’aurore menaçante. Cela sort de la conscience, devenue ombre et nuée.\par
Il éclaira dans cette prunelle.\par
Cet éclair ne ressemblait à rien de ce qu’on peut voir luire là-haut ni ici-bas.\par
Le coquin comprimé qui était en Clubin fit explosion.\par
Clubin regarda l’obscurité immense, et ne put retenir un éclat de rire bas et sinistre.\par
Il était donc libre ! il était donc riche !\par
Son inconnue se dégageait enfin. Il résolvait son problème.\par
Clubin avait du temps devant lui. La marée montait, et par conséquent soutenait la Durande, qu’elle finirait même par soulever. Le navire adhérait solidement à l’écueil ; nul danger de sombrer. En outre, il fallait laisser à la chaloupe le temps de s’éloigner, de se perdre peut-être ; Clubin l’espérait.\par
Debout sur la Durande naufragée, il croisa les bras, savourant cet abandon dans les ténèbres.\par
L’hypocrisie avait pesé trente ans sur cet homme. Il était le mal et s’était accouplé à la probité. Il haïssait la vertu d’une haine de mal marié. Il avait toujours eu une préméditation scélérate ; depuis qu’il avait l’âge d’homme, il portait cette armature rigide, l’apparence. Il était monstre en dessous ; il vivait dans une peau d’homme de bien avec un cœur de bandit. Il était le pirate doucereux. Il était le prisonnier de l’honnêteté ; il était enfermé dans cette boîte de momie, l’innocence ;  il avait sur le dos des ailes d’ange, écrasantes pour un gredin. Il était surchargé d’estime publique. Passer pour honnête homme, c’est dur. Maintenir toujours cela en équilibre, penser mal et parler bien, quel labeur ! Il avait été le fantôme de la droiture, étant le spectre du crime. Ce contre-sens avait été sa destinée. Il lui avait fallu faire bonne contenance, rester présentable, écumer au-dessous du niveau, sourire ses grincements de dents. La vertu pour lui, c’était la chose qui étouffe. Il avait passé sa vie à avoir envie de mordre cette main sur sa bouche.\par
Et voulant la mordre, il avait dû la baiser.\par
Avoir menti, c’est avoir souffert. Un hypocrite est un patient dans la double acception du mot ; il calcule un triomphe et endure un supplice. La préméditation indéfinie d’un mauvais coup accompagnée et dosée d’austérité, l’infamie intérieure assaisonnée d’excellente renommée, donner continuellement le change, n’être jamais soi, faire illusion, c’est une fatigue. Avec tout ce noir qu’on broie en son cerveau composer de la candeur, vouloir dévorer ceux qui vous vénèrent, être caressant, se retenir, se réprimer, toujours être sur le qui-vive, se guetter sans cesse, donner bonne mine à son crime latent, faire sortir sa difformité en beauté, se fabriquer une perfection avec sa méchanceté, chatouiller du poignard, sucrer le poison, veiller sur la rondeur de son geste et sur la musique de sa voix, ne pas avoir son regard, rien n’est plus difficile, rien n’est plus douloureux. L’odieux de l’hypocrisie commence obscurément dans l’hypocrite. Boire perpétuellement  son imposture est une nausée. La douceur que la ruse donne à la scélératesse répugne au scélérat, continuellement forcé d’avoir ce mélange dans la bouche, et il y a des instants de haut-le-cœur où l’hypocrite est sur le point de vomir sa pensée. Ravaler cette salive est horrible. Ajoutez à cela le profond orgueil. Il existe des minutes bizarres où l’hypocrite s’estime. Il y a un moi démesuré dans le fourbe. Le ver a le même glissement que le dragon, et le même redressement. Le traître n’est autre chose qu’un despote gêné qui ne peut faire sa volonté qu’en se résignant au deuxième rôle. C’est de la petitesse capable d’énormité. L’hypocrite est un titan, nain.\par
Clubin se figurait de bonne foi qu’il avait été opprimé. De quel droit n’était-il pas né riche ? Il n’aurait pas demandé mieux que d’avoir de ses père et mère cent mille livres de rente. Pourquoi ne les avait-il pas ? Ce n’était pas sa faute, à lui. Pourquoi, en ne lui donnant pas toutes les jouissances de la vie, le forçait-on à travailler, c’est-à-dire à tromper, à trahir, à détruire ? Pourquoi, de cette façon, l’avait-on condamné à cette torture de flatter, de ramper, de complaire, de se faire aimer et respecter, et d’avoir jour et nuit sur la face un autre visage que le sien ? Dissimuler est une violence subie. On hait devant qui l’on ment. Enfin l’heure avait sonné. Clubin se vengeait.\par
De qui ? De tous, et de tout.\par
Lethierry ne lui avait fait que du bien ; grief de plus ; il se vengeait de Lethierry.\par
Il se vengeait de tous ceux devant lesquels il s’était  contraint. Il prenait sa revanche. Quiconque avait pensé du bien de lui était son ennemi. Il avait été le captif de cet homme-là.\par
Clubin était en liberté. Sa sortie était faite. Il était hors des hommes. Ce qu’on prendrait pour sa mort était sa vie ; il allait commencer. Le vrai Clubin dépouillait le faux. D’un coup il avait tout dissous. Il avait poussé du pied Rantaine dans l’espace, Lethierry dans la ruine, la justice humaine dans la nuit, l’opinion dans l’erreur, l’humanité entière hors de lui, Clubin. Il venait d’éliminer le monde.\par
Quant à Dieu, ce mot de quatre lettres l’occupait peu.\par
Il avait passé pour religieux. Eh bien, après ?\par
Il y a des cavernes dans l’hypocrite, ou, pour mieux dire, l’hypocrite entier est une caverne.\par
Quand Clubin se trouva seul, son antre s’ouvrit. Il eut un instant de délices ; il aéra son âme.\par
Il respira son crime à pleine poitrine.\par
Le fond du mal devint visible sur ce visage. Clubin s’épanouit. En ce moment, le regard de Rantaine à côté du sien eût semblé le regard d’un enfant nouveau-né.\par
L’arrachement du masque, quelle délivrance ! Sa conscience jouit de se voir hideusement nue et de prendre librement un bain ignoble dans le mal. La contrainte d’un long respect humain finit par inspirer un goût forcené pour l’impudeur. On en arrive à une certaine lasciveté dans la scélératesse. Il existe, dans ces effrayantes profondeurs morales si peu sondées,  on ne sait quel étalage atroce et agréable qui est l’obscénité du crime. La fadeur de la fausse bonne renommée met en appétit de honte. On dédaigne tant les hommes qu’on voudrait en être méprisé. Il y a de l’ennui à être estimé. On admire les coudées franches de la dégradation. On regarde avec convoitise la turpitude, si à l’aise dans l’ignominie. Les yeux baissés de force ont souvent de ces échappées obliques. Rien n’est plus près de Messaline que Marie Alacoque. Voyez la Cadière et la religieuse de Louviers. Clubin, lui aussi, avait vécu sous le voile. L’effronterie avait toujours été son ambition. Il enviait la fille publique et le front de bronze de l’opprobre accepté ; il se sentait plus fille publique qu’elle, et avait le dégoût de passer pour vierge. Il avait été le Tantale du cynisme. Enfin, sur ce rocher, dans cette solitude, il pouvait être franc ; il l’était. Se sentir sincèrement abominable, quelle volupté ! Toutes les extases possibles à l’enfer, Clubin les eut dans cette minute ; les arrérages de la dissimulation lui furent soldés ; l’hypocrisie est une avance ; Satan le remboursa. Clubin se donna l’ivresse d’être effronté, les hommes ayant disparu, et n’ayant plus là que le ciel. Il se dit : Je suis un gueux ! et fut content.\par
Jamais rien de pareil ne s’était passé dans une conscience humaine.\par
L’éruption d’un hypocrite, nulle ouverture de cratère n’est comparable à cela.\par
Il était charmé qu’il n’y eût là personne, et il n’eût pas été fâché qu’il y eût quelqu’un. Il eût joui d’être effroyable devant témoin.\par
 Il eût été heureux de dire en face au genre humain : Tu es idiot !\par
L’absence des hommes assurait son triomphe, mais le diminuait.\par
Il n’avait que lui pour spectateur de sa gloire.\par
Être au carcan a son charme. Tout le monde voit que vous êtes infâme.\par
Forcer la foule à vous examiner, c’est faire acte de puissance. Un galérien debout sur un tréteau dans le carrefour avec le collier de fer au cou est le despote de tous les regards qu’il contraint de se tourner vers lui. Dans cet échafaud il y a du piédestal. Être un centre de convergence pour l’attention universelle, quel plus beau triomphe ? Obliger au regard la prunelle publique, c’est une des formes de la suprématie. Pour ceux dont le mal est l’idéal, l’opprobre est une auréole. On domine de là. On est en haut de quelque chose. On s’y étale souverainement. Un poteau que l’univers voit n’est pas sans quelque analogie avec un trône.\par
Être exposé, c’est être contemplé.\par
Un mauvais règne a évidemment des joies de pilori. Néron incendiant Rome, Louis XIV prenant en traître le Palatinat, le régent George tuant lentement Napoléon, Nicolas assassinant la Pologne à la face de la civilisation, devaient éprouver quelque chose de la volupté que rêvait Clubin. L’immensité du mépris fait au méprisé l’effet d’une grandeur.\par
Être démasqué est un échec, mais se démasquer est une victoire. C’est de l’ivresse, c’est de l’imprudence insolente et satisfaite, c’est une nudité  éperdue qui insulte tout devant elle. Suprême bonheur.\par
Ces idées dans un hypocrite semblent une contradiction, et n’en sont pas une. Toute l’infamie est conséquente. Le miel est fiel. Escobar confine au marquis de Sade. Preuve : Léotade. L’hypocrite, étant le méchant complet, a en lui les deux pôles de la perversité. Il est d’un côté prêtre, et de l’autre courtisane. Son sexe de démon est double. L’hypocrite est l’épouvantable hermaphrodite du mal. Il se féconde seul. Il s’engendre et se transforme lui-même. Le voulez-vous charmant, regardez-le ; le voulez-vous horrible, retournez-le.\par
Clubin avait en lui toute cette ombre d’idées confuses. Il les percevait peu, mais il en jouissait beaucoup.\par
Un passage de flammèches de l’enfer qu’on verrait dans la nuit, c’était la succession des pensées de cette âme.\par
Clubin resta ainsi quelque temps rêveur ; il regardait son honnêteté de l’air dont le serpent regarde sa vieille peau.\par
Tout le monde avait cru à cette honnêteté, même un peu lui.\par
Il eut un second éclat de rire.\par
On l’allait croire mort, et il était riche. On l’allait croire perdu, et il était sauvé. Quel bon tour joué à la bêtise universelle !\par
Et dans cette bêtise universelle il y avait Rantaine. Clubin songeait à Rantaine avec un dédain sans bornes. Dédain de la fouine pour le tigre. Cette fugue, manquée par Rantaine, il la réussissait, lui Clubin. Rantaine  s’en allait penaud, et Clubin disparaissait triomphant. Il s’était substitué à Rantaine dans le lit de sa mauvaise action, et c’était lui Clubin qui avait la bonne fortune.\par
Quant à l’avenir, il n’avait pas de plan bien arrêté. Il avait dans la boîte de fer enfarmée dans sa ceinture ses trois bank-notes ; cette certitude lui suffisait. Il changerait de nom. Il y a des pays où soixante mille francs en valent six cent mille. Ce ne serait pas une mauvaise solution que d’aller dans un de ces coins-là vivre honnêtement avec l’argent repris à ce voleur de Rantaine. Spéculer, entrer dans le grand négoce, grossir son capital, devenir sérieusement millionnaire, cela non plus ne serait point mal.\par
Par exemple, à Costa-Rica, comme c’était le commencement du grand commerce du café, il y avait des tonnes d’or à gagner. On verrait.\par
Peu importait d’ailleurs. Il avait le temps d’y songer. Pour le moment, le difficile était fait. Dépouiller Rantaine, disparaître avec la Durande, c’était la grosse affaire. Elle était accomplie. Le reste était simple. Nul obstacle possible désormais. Rien à craindre. Rien ne pouvait survenir. Il allait atteindre la côte à la nage, à la nuit il aborderait à Plainmont, il escaladerait la falaise, il irait droit à la maison visionnée, il y entrerait sans peine au moyen de sa corde à nœuds cachée d’avance dans un trou de rocher, il trouverait dans la maison visionnée son sac-valise contenant des vêtements secs et des vivres, là il pourrait attendre, il était renseigné, huit jours ne se passeraient pas sans  que des contrebandiers d’Espagne, Blasquito probablement, touchassent à Plainmont, pour quelques guinées il se ferait transporter, non à Torbay, comme il l’avait dit à Blasco pour dérouter les conjectures et donner le change, mais à Pasages ou à Bilbao. De là il gagnerait la Vera-Cruz ou la Nouvelle-Orléans. Du reste le moment était venu de se jeter à la mer, la chaloupe était loin, une heure de nage n’était rien pour Clubin, un mille seulement le séparait de la terre, puisqu’il était sur les Hanois.\par
A ce point de la rêverie de Clubin, une déchirure se fit dans le brouillard. Le formidable rocher Douvres apparut.
 \subsubsection[{A.VI.7. L’inattendu intervient}]{A.VI.7. \\
L’inattendu intervient}
\noindent Clubin, hagard, regarda.\par
C’était bien l’épouvantable écueil isolé.\par
Impossible de se méprendre sur cette silhouette difforme. Les deux Douvres jumelles se dressaient, hideusement, laissant voir entre elles, comme un piège, leur défilé. On eût dit le coupe-gorge de l’océan.\par
Elles étaient tout près. Le brouillard les avait cachées, comme un complice.\par
Clubin, dans le brouillard, avait fait fausse route. Malgré toute son attention, il lui était arrivé ce qui arriva à deux grands navigateurs, à Gonzalez qui découvrit le cap Blanc, et à Fernandez qui découvrit le cap Vert. La brume l’avait égaré. Elle lui avait paru excellente pour l’exécution de son projet, mais elle avait ses périls. Clubin avait dévié à l’ouest et s’était trompé. Le passager guernesiais, en croyant reconnaître les Hanois, avait déterminé le coup de barre final. Clubin avait cru se jeter sur les Hanois.\par
La Durande, crevée par un des bas-fonds de l’écueil, n’était séparée des deux Douvres que de quelques encâblures.\par
 A deux cents brasses plus loin, on apercevait un massif cube de granit. On distinguait sur les pans escarpés de cette roche quelques stries et quelques reliefs pour l’escalade. Les coins rectilignes de ces rudes murailles à angle droit faisaient pressentir au sommet un plateau.\par
C’était l’Homme.\par
La roche l’Homme s’élevait plus haut encore que les roches Douvres. Sa plate-forme dominait leur double pointe inaccessible. Cette plate-forme, croulant par les bords, avait un entablement, et on ne sait quelle régularité sculpturale. On ne pouvait rien rêver de plus désolé et de plus funeste. Les lames du large venaient plisser leurs nappes tranquilles aux faces carrées de cet énorme tronçon noir, sorte de piédestal pour les spectres immenses de la mer et de la nuit.\par
Tout cet ensemble était stagnant. A peine un souffle dans l’air, à peine une ride sur la vague. On devinait sous cette surface muette de l’eau la vaste vie noyée des profondeurs.\par
Clubin avait souvent vu l’écueil Douvres de loin.\par
Il se convainquit que c’était bien là qu’il était.\par
Il ne pouvait douter.\par
Changement brusque et hideux. Les Douvres au lieu des Hanois. Au lieu d’un mille, cinq lieues de mer. Cinq lieues de mer ! l’impossible. La roche Douvres, pour le naufragé solitaire, c’est la présence, visible et palpable, du dernier moment. Défense d’atteindre la terre.\par
 Clubin frissonna. Il s’était mis lui-même dans la gueule de l’ombre. Pas d’autre refuge que le rocher l’Homme. Il était probable que la tempête surviendrait dans la nuit, et que la chaloupe de la Durande, surchargée, chavirerait. Aucun avis du naufrage n’arriverait à terre. On ne saurait même pas que Clubin avait été laissé sur l’écueil Douvres. Pas d’autre perspective que la mort de froid et de faim. Ses soixante-quinze mille francs ne lui donneraient pas une bouchée de pain. Tout ce qu’il avait échafaudé aboutissait à cette embûche. Il était l’architecte laborieux de sa catastrophe. Nulle ressource. Nul salut possible. Le triomphe se faisait précipice. Au lieu de la délivrance, la capture. Au lieu du long avenir prospère, l’agonie. En un clin d’œil, le temps qu’un éclair passe, toute sa construction avait croulé. Le paradis rêvé par ce démon avait repris sa vraie figure, le sépulcre.\par
Cependant le vent s’était élevé. Le brouillard, secoué, troué, arraché, s’en allait pêle-mêle sur l’horizon en grands morceaux informes. Toute la mer reparut.\par
Les bœufs, de plus en plus envahis par l’eau, continuaient de beugler dans la cale.\par
La nuit approchait ; probablement la tempête.\par
La Durande, peu à peu renflouée par la mer montante, oscillait de droite à gauche, puis de gauche à droite, et commençait à tourner sur l’écueil comme sur un pivot.\par
On pouvait pressentir le moment où une lame l’arracherait et la roulerait à vau-l’eau.\par
Il y avait moins d’obscurité qu’au moment du  naufrage. Quoique l’heure fût plus avancée, on voyait plus clair. Le brouillard, en s’en allant, avait emporté une partie de l’ombre. L’ouest était dégagé de toute nuée. Le crépuscule a un grand ciel blanc. Cette vaste lueur éclairait la mer.\par
La Durande était échouée en plan incliné de la poupe à la proue. Clubin monta sur l’arrière du navire qui était presque hors de l’eau. Il attacha sur l’horizon son œil fixe.\par
Le propre de l’hypocrisie c’est d’être âpre à l’espérance. L’hypocrite est celui qui attend. L’hypocrisie n’est autre chose qu’une espérance horrible ; et le fond de ce mensonge-là est fait avec cette vertu, devenue vice.\par
Chose étrange à dire, il y a de la confiance dans l’hypocrisie. L’hypocrisie se confie à on ne sait quoi d’indifférent dans l’inconnu, qui permet le mal.\par
Clubin regardait l’étendue.\par
La situation était désespérée, cette âme sinistre ne l’était point.\par
Il se disait qu’après ce long brouillard les navires restés sous la brume en panne ou à l’ancre allaient reprendre leur course, et que peut-être il en passerait quelqu’un à l’horizon.\par
Et, en effet, une voile surgit.\par
Elle venait de l’est et allait à l’ouest.\par
En approchant, la complication du navire se dessina. Il n’avait qu’un mât, et il était gréé en goélette. Le beaupré était presque horizontal. C’était un coutre.\par
 Avant une demi-heure, il côtoierait d’assez près l’écueil Douvres.\par
Clubin se dit : Je suis sauvé.\par
Dans une minute comme celle où il était, on ne pense d’abord qu’à la vie.\par
Ce coutre était peut-être étranger. Qui sait si ce n’était pas un des navires contrebandiers allant à Plainmont ? Qui sait si ce n’était pas Blasquito lui-même ? En ce cas, non-seulement la vie serait sauve, mais la fortune ; et la rencontre de l’écueil Douvres, en hâtant la conclusion, en supprimant l’attente dans la maison visionnée, en dénouant en pleine mer l’aventure, aurait été un incident heureux.\par
Toute la certitude de la réussite rentra frénétiquement dans ce sombre esprit.\par
C’est une chose étrange que la facilité avec laquelle les coquins croient que le succès leur est dû.\par
Il n’y avait qu’une chose à faire.\par
La Durande, engagée dans les rochers, mêlait sa silhouette à la leur, se confondait avec leur dentelure où elle n’était qu’un linéament de plus, y était indistincte et perdue, et ne suffirait pas, dans le peu de jour qui restait, pour attirer l’attention du navire qui allait passer.\par
Mais une figure humaine se dessinant en noir sur la blancheur crépusculaire, debout sur le plateau du rocher l’Homme et faisant des signaux de détresse, serait sans nul doute aperçue. On enverrait une embarcation pour recueillir le naufragé.\par
Le rocher l’Homme n’était qu’à deux cents brasses.  L’atteindre à la nage était simple, l’escalader était facile.\par
Il n’y avait pas une minute à perdre.\par
L’avant de la Durande était dans la roche, c’était du haut de l’arrière, et du point même où était Clubin, qu’il fallait se jeter à la nage.\par
Il commença par mouiller une sonde et reconnut qu’il y avait sous l’arrière beaucoup de fond. Les coquillages microscopiques de foraminifères et de polycystinées que le suif rapporta étaient intacts, ce qui indiquait qu’il y avait là de très creuses caves de roche, où l’eau, quelle que fût l’agitation de la surface, était toujours tranquille.\par
Il se déshabilla, laissant ses vêtements sur le pont. Des vêtements, il en trouverait sur le coutre.\par
Il ne garda que la ceinture de cuir.\par
Quand il fut nu, il porta la main à cette ceinture, la reboucla, y palpa la boîte de fer, étudia rapidement du regard la direction qu’il aurait à suivre à travers les brisants et les vagues pour gagner le rocher l’Homme, puis, se précipitant la tête la première, il plongea.\par
Comme il tombait de haut, il plongea profondément.\par
Il entra très avant sous l’eau, atteignit le fond, le toucha, côtoya un moment les roches sous-marines, puis donna une secousse pour remonter à la surface.\par
En ce moment, il se sentit saisir par le pied.
 \subsection[{A.VII. Livre septième. Imprudence de faire des questions à un livre}]{A.VII. Livre septième \\
Imprudence de faire des questions à un livre}
  \subsubsection[{A.VII.1. La perle au fond du précipice}]{A.VII.1. \\
La perle au fond du précipice}
\noindent Quelques minutes après son court colloque avec sieur Landoys, Gilliatt était à Saint-Sampson.\par
Gilliatt était inquiet jusqu’à l’anxiété. Qu’était-il donc arrivé ?\par
Saint-Sampson avait une rumeur de ruche effarouchée. Tout le monde était sur les portes. Les femmes s’exclamaient. Il y avait des gens qui semblaient raconter quelque chose et qui gesticulaient ; on faisait groupe autour d’eux. On entendait ce mot : quel malheur ! Plusieurs visages souriaient.\par
Gilliatt n’interrogea personne. Il n’était pas dans sa nature de faire des questions. D’ailleurs il était trop ému pour parler à des indifférents. Il se défiait des récits, il aimait mieux tout savoir d’un coup ; il alla droit aux Bravées.\par
Son inquiétude était telle qu’il n’eut même pas peur d’entrer dans cette maison.\par
 D’ailleurs la porte de la salle basse sur le quai était toute grande ouverte. Il y avait sur le seuil un fourmillement d’hommes et de femmes. Tout le monde entrait, il entra.\par
En entrant, il trouva contre le chambranle de la porte sieur Landoys qui lui dit à demi-voix :\par
— Vous savez sans doute à présent l’événement ?\par
— Non.\par
— Je n’ai pas voulu vous crier ça dans la route. On a l’air d’un oiseau de malheur.\par
— Qu’est-ce ?\par
— La Durande est perdue.\par
Il y avait foule dans la salle.\par
Les groupes parlaient bas, comme dans la chambre d’un malade.\par
Les assistants, qui étaient les voisins, les passants, les curieux, les premiers venus, se tenaient entassés près de la porte avec une sorte de crainte, et laissaient vide le fond de la salle où l’on voyait, à côté de Déruchette en larmes, assise, mess Lethierry, debout.\par
Il était adossé à la cloison du fond. Son bonnet de matelot tombait sur ses sourcils. Une mèche de cheveux gris pendait sur sa joue. Il ne disait rien. Ses bras n’avaient pas de mouvement, sa bouche semblait n’avoir plus de souffle. Il avait l’air d’une chose posée contre le mur.\par
On sentait, en le voyant, l’homme au dedans duquel la vie vient de s’écrouler. Durande n’étant plus, Lethierry n’avait plus de raison d’être. Il avait une âme en mer, cette âme venait de sombrer. Que devenir  maintenant ? Se lever tous les matins, se coucher tous les soirs. Ne plus attendre Durande, ne plus la voir partir, ne plus la voir revenir. Qu’est-ce qu’un reste d’existence sans but ? Boire, manger, et puis ? Cet homme avait couronné tous ses travaux par un chef-d’œuvre, et tous ses dévouements par un progrès. Le progrès était aboli, le chef-d’œuvre était mort. Vivre encore quelques années vides, à quoi bon ? Rien à faire désormais. A cet âge on ne recommence pas ; en outre, il était ruiné. Pauvre vieux bonhomme !\par
Déruchette, pleurante près de lui sur une chaise, tenait dans ses deux mains un des poings de mess Lethierry. Les mains étaient jointes, le poing était crispé. La nuance des deux accablements était là. Dans les mains jointes quelque chose espère encore ; dans le poing crispé, rien.\par
Mess Lethierry lui abandonnait son bras et la laissait faire. Il était passif. Il n’avait plus que la quantité de vie qu’on peut avoir après le coup de foudre.\par
Il y a de certaines arrivées au fond de l’abîme qui vous retirent du milieu des vivants. Les gens qui vont et viennent dans votre chambre sont confus et indistincts ; ils vous coudoient sans parvenir jusqu’à vous. Vous leur êtes inabordable et ils vous sont inaccessibles. Le bonheur et le désespoir ne sont pas les mêmes milieux respirables ; désespéré, on assiste à la vie des autres de très loin, on ignore presque leur présence ; on perd le sentiment de sa propre existence ; on a beau être en chair et en os, on ne se sent plus réel ; on n’est plus pour soi-même qu’un songe.\par
 Mess Lethierry avait le regard de cette situation-là.\par
Les groupes chuchotaient. On échangeait ce qu’on savait. Voici quelles étaient les nouvelles :\par
La Durande s’était perdue la veille sur le rocher Douvres, par le brouillard, une heure environ avant le coucher du soleil. A l’exception du capitaine qui n’avait pas voulu quitter son navire, les gens s’étaient sauvés dans la chaloupe. Une bourrasque de sud-ouest, survenue après le brouillard, avait failli les faire naufrager une seconde fois, et les avait poussés au large au delà de Guernesey. Dans la nuit ils avaient eu ce bon hasard de rencontrer le \emph{Cashmere}, qui les avait recueillis et amenés à Saint-Pierre-Port. Tout était de la faute du timonier Tangrouille, qui était en prison. Clubin avait été magnanime.\par
Les pilotes, qui abondaient dans les groupes, prononçaient ce mot, \emph{l’écueil Douvres}, d’une façon particulière. — Mauvaise auberge, disait l’un d’eux.\par
On remarquait sur la table une boussole et une liasse de registres et de carnets ; c’étaient sans doute la boussole de la Durande et les papiers de bord remis par Clubin à Imbrancam et à Tangrouille au moment du départ de la chaloupe ; magnifique abnégation de cet homme sauvant jusqu’à des paperasses à l’instant où il se laisse mourir ; petit détail plein de grandeur ; oubli sublime de soi-même.\par
On était unanime pour admirer Clubin, et, du reste, unanime aussi pour le croire, après tout, sauvé. Le coutre \emph{Shealtiel} était arrivé quelques heures après le \emph{Cashmere ;} c’était ce coutre qui apportait les derniers  renseignements. Il venait de passer vingt-quatre heures dans les mêmes eaux que la Durande. Il y avait patienté pendant le brouillard et louvoyé pendant la tempête. Le patron du \emph{Shealtiel} était présent parmi les assistants.\par
A l’instant où Gilliatt était entré, ce patron venait de faire son récit à mess Lethierry. Ce récit était un vrai rapport. Vers le matin, la bourrasque étant finie et le vent devenant maniable, le patron du \emph{Shealtiel} avait entendu des beuglements en pleine mer. Ce bruit de prairies au milieu des vagues l’avait surpris ; il s’était dirigé de ce côté. Il avait aperçu la Durande dans les rochers Douvres. L’accalmie était suffisante pour qu’il pût approcher. Il avait hélé l’épave. Le mugissement des bœufs qui se noyaient dans la cale avait seul répondu. Le patron du \emph{Shealtiel} était certain qu’il n’y avait personne à bord de la Durande. L’épave était parfaitement tenable ; et, si violente qu’eût été la bourrasque, Clubin eût pu y passer la nuit. Il n’était pas homme à lâcher prise aisément. Il n’y était point, donc il était sauvé. Plusieurs sloops et plusieurs lougres, de Granville et de Saint-Malo, se dégageant du brouillard, avaient dû, la veille au soir, ceci était hors de doute, côtoyer d’assez près l’écueil Douvres. Un d’eux avait évidemment recueilli le capitaine Clubin. Il faut se souvenir que la chaloupe de la Durande était pleine en quittant le navire échoué, qu’elle allait courir beaucoup de risques, qu’un homme de plus était une surcharge et pouvait la faire sombrer, et que c’était là surtout ce qui avait dû déterminer Clubin à rester sur  l’épave ; mais qu’une fois son devoir rempli, un navire sauveteur se présentant, Clubin n’avait, à coup sûr, fait nulle difficulté d’en profiter. On est un héros, mais on n’est pas un niais. Un suicide eût été d’autant plus absurde, que Clubin était irréprochable. Le coupable c’était Tangrouille, et non Clubin. Tout ceci était concluant ; le patron du \emph{Shealtiel} avait visiblement raison, et tout le monde s’attendait à voir Clubin reparaître d’un moment à l’autre. On préméditait de le porter en triomphe.\par
Deux certitudes ressortaient du récit du patron, Clubin sauvé et la Durande perdue.\par
Quant à la Durande, il fallait en prendre son parti, la catastrophe était irrémédiable. Le patron du \emph{Shealtiel} avait assisté à la dernière phase du naufrage. Le rocher fort aigu où la Durande était en quelque sorte clouée avait tenu bon toute la nuit, et avait résisté au choc de la tempête comme s’il voulait garder l’épave pour lui ; mais au matin, à l’instant où le \emph{Shealtiel, }constatant qu’il n’y avait personne à sauver, allait s’éloigner de la Durande, il était survenu un de ces paquets de mer qui sont comme les derniers coups de colère des tempêtes. Ce flot avait furieusement soulevé la Durande, l’avait arrachée du brisant, et, avec la vitesse et la rectitude d’une flèche lancée, l’avait jetée entre les deux roches Douvres. On avait entendu un craquement « diabolique », disait le patron. La Durande, portée par la lame à une certaine hauteur, s’était engagée dans l’entre-deux des roches jusqu’au maître-couple. Elle était de nouveau clouée, mais plus  solidement que sur le brisant sous-marin. Elle allait rester là déplorablement suspendue, livrée à tout le vent et à toute la mer.\par
La Durande, au dire de l’équipage du \emph{Shealtiel,} était déjà aux trois quarts fracassée. Elle eût évidemment coulé dans la nuit si l’écueil ne l’eût retenue et soutenue. Le patron du \emph{Shealtiel} avec sa lunette avait étudié l’épave. Il donnait avec la précision marine le détail du désastre ; la hanche de tribord était défoncée, les mâts tronqués, la voilure déralinguée, les chaînes des haubans presque toutes coupées, la claire-voie du capot-de-chambre écrasée par la chute d’une vergue, les jambettes brisées au ras du plat bord depuis le travers du grand mât jusqu’au couronnement, le dôme de la cambuse effondré, les chantiers de la chaloupe culbutés, le rouffle démonté, l’arbre du gouvernail rompu, les drosses déclouées, les pavois rasés, les bittes emportées, le traversin détruit, la lisse enlevée, l’étambot cassé. C’était toute la dévastation frénétique de la tempête. Quant à la grue de chargement scellée au mât sur l’avant, plus rien, aucune nouvelle, nettoyage complet, partie à tous les diables, avec sa guinderesse, ses moufles, sa poulie de fer et ses chaînes. La Durande était disloquée ; l’eau allait maintenant se mettre à la déchiqueter. Dans quelques jours il n’en resterait plus rien.\par
Pourtant la machine, chose remarquable et qui prouvait son excellence, était à peine atteinte dans ce ravage. Le patron du \emph{Shealtiel} croyait pouvoir affirmer que « la manivelle » n’avait point d’avarie grave. Les mâts du  navire avaient cédé, mais la cheminée de la machine avait résisté. Les gardes de fer de la passerelle de commandement étaient seulement tordues ; les tambours avaient souffert, les cages étaient froissées, mais les roues ne paraissaient pas avoir une palette de moins. La machine était intacte. C’était la conviction du patron du \emph{Shealtiel.} Le chauffeur Imbrancam, qui était mêlé aux groupes, partageait cette conviction. Ce nègre, plus intelligent que beaucoup de blancs, était l’admirateur de la machine. Il levait les bras en ouvrant les dix doigts de ses mains noires et disait à Lethierry muet : Mon maître, la mécanique est en vie.\par
Le salut de Clubin semblant assuré, et la coque de la Durande étant sacrifiée, la machine, dans les conversations des groupes, était la question. On s’y intéressait comme à une personne. On s’émerveillait de sa bonne conduite. — Voilà une solide commère, disait un matelot français. — C’est de quoi bon ! s’écriait un pêcheur guernesiais. — Il faut qu’elle ait eu de la malice, reprenait le patron du \emph{Shealtiel}, pour se tirer de là avec deux ou trois écorchures.\par
Peu à peu cette machine fut la préoccupation unique. Elle échauffa les opinions pour et contre. Elle avait des amis et des ennemis. Plus d’un, qui avait un bon vieux coutre à voiles, et qui espérait ressaisir la clientèle de la Durande, n’était pas fâché de voir l’écueil Douvres faire justice de la nouvelle invention. Le chuchotement devint brouhaha. On discuta presque avec bruit. Pourtant c’était une rumeur toujours un peu discrète, et il se faisait par intervalles de subits  abaissements de voix, sous la pression du silence sépulcral de Lethierry.\par
Du colloque engagé sur tous les points, il résultait ceci :\par
La machine était l’essentiel. Refaire le navire était possible, refaire la machine non. Cette machine était unique. Pour en fabriquer une pareille, l’argent manquerait, l’ouvrier manquerait encore plus. On rappelait que le constructeur de la machine était mort. Elle avait coûté quarante mille francs. Personne ne risquerait désormais sur une telle éventualité un tel capital ; d’autant plus que voilà qui était jugé, les navires à vapeur se perdaient comme les autres ; l’accident actuel de la Durande coulait à fond tout son succès passé. Pourtant il était déplorable de penser qu’à l’heure qu’il était, cette mécanique était encore entière et en bon état, et qu’avant cinq ou six jours elle serait probablement mise en pièces comme le navire. Tant qu’elle existait, il n’y avait, pour ainsi dire, pas de naufrage. La perte seule de la machine serait irrémédiable. Sauver la machine, ce serait réparer le désastre.\par
Sauver la machine, c’était facile à dire. Mais qui s’en chargerait ? est-ce que c’était possible ? Faire et exécuter c’est deux, et la preuve, c’est qu’il est aisé de faire un rêve et difficile de l’exécuter. Or si jamais un rêve avait été impraticable et insensé, c’était celui-ci : sauver la machine échouée sur les Douvres. Envoyer travailler sur ces roches un navire et un équipage serait absurde ; il n’y fallait pas songer. C’était la saison des coups de mer ; à la première bourrasque les  chaînes des ancres seraient sciées par les crêtes sous-marines des brisants, et le navire se fracasserait à l’écueil. Ce serait envoyer un deuxième naufrage au secours du premier. Dans l’espèce de trou du plateau supérieur où s’était abrité le naufragé légendaire mort de faim, il y avait à peine place pour un homme. Il faudrait donc que, pour sauver cette machine, un homme allât aux rochers Douvres, et qu’il y allât seul, seul dans cette mer, seul dans ce désert, seul à cinq lieues de la côte, seul dans cette épouvante, seul des semaines entières, seul devant le prévu et l’imprévu, sans ravitaillement dans les angoisses du dénûment, sans secours dans les incidents de la détresse, sans autre trace humaine que celle de l’ancien naufragé expiré de misère là, sans autre compagnon que ce mort. Et comment s’y prendrait-il d’ailleurs pour sauver cette machine ? Il faudrait qu’il fût non-seulement matelot, mais forgeron. Et à travers quelles épreuves ! L’homme qui tenterait cela serait plus qu’un héros. Ce serait un fou. Car dans de certaines entreprises disproportionnées où le surhumain semble nécessaire, la bravoure a au-dessus d’elle la démence. Et en effet, après tout, se dévouer pour de la ferraille, ne serait-ce pas extravagant ? Non, personne n’irait aux rochers Douvres. On devait renoncer à la machine comme au reste. Le sauveteur qu’il fallait ne se présenterait point. Où trouver un tel homme ?\par
Ceci, dit un peu autrement, était le fond de tous les propos murmurés dans cette foule.\par
Le patron du \emph{Shealtiel,} qui était un ancien pilote,  résuma la pensée de tous par cette exclamation à voix haute :\par
— Non ! c’est fini. L’homme qui ira là et qui rapportera la machine n’existe pas.\par
— Puisque je n’y vais pas, ajouta Imbrancam, c’est qu’on ne peut pas y aller.\par
Le patron du \emph{Shealtiel} secoua sa main gauche avec cette brusquerie qui exprime la conviction de l’impossible, et reprit :\par
— S’il existait...\par
Déruchette tourna la tête.\par
— Je l’épouserais, dit-elle.\par
Il y eut un silence.\par
Un homme très pâle sortit du milieu des groupes et dit :\par
— Vous l’épouseriez, miss Déruchette ?\par
C’était Gilliatt.\par
Cependant tous les yeux s’étaient levés. Mess Lethierry venait de se dresser tout droit. Il avait sous le sourcil une lumière étrange.\par
Il prit du poing son bonnet de matelot et le jeta à terre, puis il regarda solennellement devant lui sans voir aucune des personnes présentes, et dit :\par
— Déruchette l’épouserait. J’en donne ma parole d’honneur au bon Dieu.
 \subsubsection[{A.VII.2. Beaucoup d’étonnement sur la cote ouest}]{A.VII.2. \\
Beaucoup d’étonnement sur la cote ouest}
\noindent La nuit qui suivit ce jour-là devait être, à partir de dix heures du soir, une nuit de lune. Cependant, quelle que fût la bonne apparence de la nuit, du vent et de la mer, aucun pêcheur ne comptait sortir ni de la Hougue la Perre, ni du Bourdeaux, ni de Houmet Benet, ni du Platon, ni de Port Grat, ni de la baie Vason, ni de Perelle bay, ni de Pezeris, ni du Tielles, ni de la baie des Saints, ni de Petit Bô, ni d’aucun port ou portelet de Guernesey. Et cela était tout simple, le coq avait chanté à midi.\par
Quand le coq chante à une heure extraordinaire, la pêche manque.\par
Ce soir-là, pourtant, à la tombée de la nuit, un pêcheur qui rentrait à Omptolle eut une surprise. A la hauteur du Houmet Paradis, au delà des deux Brayes et des deux Grunes, ayant à gauche la balise des Plattes Fougères qui représente un entonnoir renversé, et à droite la balise de Saint-Sampson qui représente  une figure d’homme, il crut apercevoir une troisième balise. Qu’était-ce que cette balise ? quand l’avait-on plantée sur ce point ? quel bas-fond indiquait-elle ? La balise répondit tout de suite à ces interrogations ; elle remuait ; c’était un mât. L’étonnement du pêcheur ne décrut point. Une balise faisait question ; un mât bien plus encore. Il n’y avait point de pêche possible. Quand tout le monde rentrait, quelqu’un sortait. Qui ? pourquoi ?\par
Dix minutes après, le mât, cheminant lentement, arriva à quelque distance du pêcheur d’Omptolle. Il ne put reconnaître la barque. Il entendit ramer. Il n’y avait que le bruit de deux avirons. C’était donc vraisemblablement un homme seul. Le vent était nord ; cet homme évidemment nageait pour aller prendre le vent au delà de la pointe Fontenelle. Là, probablement, il mettrait à la voile. Il comptait donc doubler l’Ancresse et le mont Crevel. Qu’est-ce que cela voulait dire ?\par
Le mât passa, le pêcheur rentra.\par
Cette même nuit, sur la côte ouest de Guernesey, des observateurs d’occasion, disséminés et isolés, firent, à des heures diverses et sur divers points, des remarques.\par
Comme le pêcheur d’Omptolle venait d’amarrer sa barque, un charretier de varech, à un demi-mille plus loin, fouettant ses chevaux dans la route déserte des Clôtures, près du cromlech, aux environs des martellos 6 et 7, vit en mer, assez loin à l’horizon, dans un endroit peu fréquenté, parce qu’il faut le bien  connaître, devers la Roque-Nord et la Sablonneuse, une voile qu’on hissait. Il y fit d’ailleurs peu d’attention, étant pour chariot et non pour bateau.\par
Une demi-heure s’était peut-être écoulée depuis que le charretier avait aperçu cette voile, quand un plâtreur revenant de son ouvrage de la ville et contournant la mare Pelée, se trouva tout à coup presque en face d’une barque très hardiment engagée parmi les roches du Quenon, de la Rousse de Mer et de la Gripe de Rousse. La nuit était noire, mais la mer était claire, effet qui se produit souvent, et l’on pouvait distinguer au large les allées et venues. Il n’y avait en mer que cette barque.\par
Un peu plus bas, et un peu plus tard, un ramasseur de langoustes, disposant ses boutiques sur l’ensablement qui sépare le Port Soif du Port Enfer, ne comprit pas ce que faisait une barque glissant entre la Boue Corneille et la Moulrette. Il fallait être bon pilote et bien pressé d’arriver quelque part pour se risquer là.\par
Comme huit heures sonnaient au Catel, le tavernier de Cobo bay observa, avec quelque ébahissement, une voile au delà de la Boue du Jardin et des Grunettes, très près de la Suzanne et des Grunes de l’ouest.\par
Non loin de Cobo bay, sur la pointe solitaire du Coumet de la baie Vason, deux amoureux étaient en train de se séparer et de se retenir ; au moment où la fille disait au garçon : — Si je m’en vas, ce n’est pas pour l’amour de ne pas être avec toi, c’est que j’ai  mon fait à choser. Ils furent distraits de leur baiser d’adieu par une assez grosse barque qui passa très près d’eux et qui se dirigeait vers les Messellettes.\par
Monsieur Le Peyre des Norgiots, habitant le cotillon Pipet, était occupé vers neuf heures du soir à examiner un trou fait par des maraudeurs dans la haie de son courtil, la Jennerotte, et de son « friquet planté à arbres » ; tout en constatant le dommage, il ne put s’empêcher de remarquer une barque doublant témérairement le Crocq-Point à cette heure de nuit.\par
Un lendemain de tempête, avec ce qui reste d’agitation à la mer, cet itinéraire était peu sûr. On était imprudent de le choisir, à moins de savoir par cœur les passes.\par
A neuf heures et demie, à l’Équerrier, un chalutier remportant son filet s’arrêta quelque temps pour considérer entre Colombelle et la Souffleresse quelque chose qui devait être un bateau. Ce bateau s’exposait beaucoup. Il y a là des coups de vent subits très dangereux. La roche Souffleresse est ainsi nommée parce qu’elle souffle brusquement sur les barques.\par
A l’instant où la lune se levait, la marée étant pleine et la mer étant étale dans le petit détroit de Li-Hou, le gardien solitaire de l’île de Li-Hou fut très effrayé ; il vit passer entre la lune et lui une longue forme noire. Cette forme noire, haute et étroite, ressemblait à un linceul debout qui marcherait. Elle glissait lentement au-dessus des espèces de murs que font les bancs de rochers. Le gardien de Li-Hou crut reconnaître la Dame Noire.\par
 La Dame Blanche habite le Tau de Pez d’Amont, la Dame Grise habite le Tau de Pez d’Aval, la Dame Rouge habite la Silleuse au nord du Banc-Marquis, et la Dame Noire habite le Grand-Étacré, à l’ouest de Li-Houmet. La nuit, au clair de lune, ces dames sortent, et quelquefois se rencontrent.\par
A la rigueur cette forme noire pouvait être une voile. Les longs barrages de roches sur lesquels elle semblait marcher pouvaient en effet cacher la coque d’une barque voguant derrière eux, et laisser voir la voile seulement. Mais le gardien se demanda quelle barque oserait à cette heure se hasarder entre Li-Hou et la Pécheresse et les Anguillières et Lérée-Point. Et dans quel but ? Il lui parut plus probable que c’était la Dame Noire.\par
Comme la lune venait de dépasser le clocher de Saint-Pierre du Bois, le sergent du Château Rocquaine, en relevant la moitié de l’échelle pont-levis, distingua, à l’embouchure de la baie, plus loin que la Haute Canée, plus près que la Sambule, une barque à la voile qui semblait descendre du nord au sud.\par
Il existe sur la côte sud de Guernesey, en arrière de Plainmont, au fond d’une baie toute de précipices et de murailles, coupée à pic dans le flot, un port singulier qu’un français, séjournant dans l’île depuis 1855, le même peut-être qui écrit ces lignes, a baptisé \emph{le Port au quatrième étage,} nom généralement adopté aujourd’hui. Ce port, qui s’appelait alors la Moie, est un plateau de roche, à demi naturel, à demi taillé, élevé d’une quarantaine de pieds au-dessus du niveau  de l’eau, et communiquant avec les vagues par deux gros madriers parallèles en plan incliné. Les barques, hissées à force de bras par des chaînes et des poulies, montent de la mer et y redescendent le long de ces madriers qui sont comme deux rails. Pour les hommes il y a un escalier. Ce port était alors très fréquenté par les contrebandiers. Étant peu praticable, il leur était commode.\par
Vers onze heures, des fraudeurs, peut-être ceux-là mêmes sur lesquels avait compté Clubin, étaient avec leurs ballots au sommet de cette plate-forme de la Moie. Qui fraude guette ; ils épiaient. Ils furent étonnés d’une voile qui déboucha brusquement au delà de la silhouette noire du cap Plainmont. Il faisait clair de lune. Ces contrebandiers surveillèrent cette voile, craignant que ce ne fût quelque garde-côte allant s’embusquer en observation derrière le grand Hanois. Mais la voile dépassa les Hanois, laissa derrière elle au nord-ouest la Boue Blondel, et s’enfonça au large dans l’estompe livide des brumes de l’horizon.\par
— Où diable peut aller cette barque ? se dirent les contrebandiers.\par
Le même soir, un peu après le coucher du soleil, on avait entendu quelqu’un frapper à la porte de la masure du Bû de la Rue. C’était un jeune garçon vêtu de brun avec des bas jaunes, ce qui indiquait un petit clerc de la paroisse. Le Bû de la Rue était fermé, porte et volets. Une vieille pêcheuse de fruits de mer, rôdant dans la banque une lanterne à la main, avait hélé le garçon, et ces paroles s’étaient échangées  devant le Bû de la Rue entre la pêcheuse et le petit clerc.\par
— Qu’est-ce que vous voulez, gas ?\par
— L’homme d’ici.\par
— Il n’y est point.\par
— Où est-il ?\par
— Je ne sais point.\par
— Y sera-t-il demain ?\par
— Je ne sais point.\par
— Est-ce qu’il est parti ?\par
— Je ne sais point.\par
— C’est que, voyez-vous, la femme, le nouveau recteur de la paroisse, le révérend Ebenezer Caudray, voudrait lui faire une visite.\par
— Je ne sais point.\par
— Le révérend m’envoie demander si l’homme du Bû de la Rue sera chez lui demain matin.\par
— Je ne sais point.
 \subsubsection[{A.VII.3. Ne tentez pas la bible}]{A.VII.3. \\
Ne tentez pas la bible}
\noindent Dans les vingt-quatre heures qui suivirent, mess Lethierry ne dormit pas, ne mangea pas, ne but pas, baisa au front Déruchette, s’informa de Clubin dont on n’avait pas encore de nouvelles, signa une déclaration comme quoi il n’entendait former aucune plainte, et fit mettre Tangrouille en liberté.\par
Il resta toute la journée du lendemain à demi appuyé à la table de l’office de la Durande, ni debout, ni assis, répondant avec douceur quand on lui parlait. Du reste, la curiosité étant satisfaite, la solitude s’était faite aux Bravées. Il y a beaucoup de désir d’observer dans l’empressement à s’apitoyer. La porte s’était refermée ; on laissait Lethierry avec Déruchette. L’éclair qui avait passé dans les yeux de Lethierry s’était éteint ; le regard lugubre du commencement de la catastrophe lui était revenu.\par
Déruchette, inquiète, avait, sur le conseil de Grâce et de Douce, mis, sans rien dire, à côté de lui sur la  table une paire de bas qu’il était en train de tricoter quand la mauvaise nouvelle était arrivée.\par
Il sourit amèrement et dit :\par
— On me croit donc bête.\par
Après un quart d’heure de silence, il ajouta :\par
— C’est bon quand on est heureux ces manies-là.\par
Déruchette avait fait disparaître la paire de bas, et avait profité de l’occasion pour faire disparaître aussi la boussole et les papiers de bord, que mess Lethierry regardait trop.\par
Dans l’après-midi, un peu avant l’heure du thé, la porte s’ouvrit, et deux hommes entrèrent, vêtus de noir, l’un vieux, l’autre jeune.\par
Le jeune, on l’a peut-être aperçu déjà dans le cours de ce récit.\par
Ces hommes avaient tous deux l’air grave, mais d’une gravité différente ; le vieillard avait ce qu’on pourrait nommer la gravité d’état ; le jeune homme avait la gravité de nature. L’habit donne l’une, la pensée donne l’autre.\par
C’étaient, le vêtement l’indiquait, deux hommes d’église, appartenant tous deux à la religion établie.\par
Ce qui, dans le jeune homme, eût, au premier abord, frappé l’observateur, c’est que cette gravité, qui était profonde dans son regard, et qui résultait évidemment de son esprit, ne résultait pas de sa personne. La gravité admet la passion, et l’exalte en l’épurant, mais ce jeune homme était, avant tout, joli. Étant prêtre, il avait au moins vingt-cinq ans ; il en paraissait dix-huit. Il offrait cette harmonie, et aussi  ce contraste, qu’en lui l’âme semblait faite pour la passion et le corps pour l’amour. Il était blond, rose, frais, très fin et très souple dans son costume sévère, avec des joues de jeune fille et des mains délicates ; il avait l’allure vive et naturelle, quoique réprimée. Tout en lui était charme, élégance, et presque volupté. La beauté de son regard corrigeait cet excès de grâce. Son sourire sincère, qui montrait des dents d’enfant, était pensif et religieux. C’était la gentillesse d’un page et la dignité d’un évêque.\par
Sous ses épais cheveux blonds, si dorés qu’ils paraissaient coquets, son crâne était élevé, candide et bien fait. Une ride légère à double inflexion entre les deux sourcils éveillait confusément l’idée de l’oiseau de la pensée planant, ailes déployées, au milieu de ce front.\par
On sentait, en le voyant, un de ces êtres bienveillants, innocents et purs, qui progressent en sens inverse de l’humanité vulgaire, que l’illusion fait sages et que l’expérience fait enthousiastes.\par
Sa jeunesse transparente laissait voir sa maturité intérieure. Comparé à l’ecclésiastique en cheveux gris qui l’accompagnait, au premier coup d’œil il semblait le fils, au second coup d’œil il semblait le père.\par
Celui-ci n’était autre que le docteur Jaquemin Hérode. Le docteur Jaquemin Hérode appartenait à la haute église, laquelle est à peu près un papisme sans pape. L’anglicanisme était travaillé dès cette époque par les tendances qui se sont depuis affirmées et condensées dans le puséysme. Le docteur Jaquemin  Hérode était de cette nuance anglicane qui est presque une variété romaine. Il était haut, correct, étroit et supérieur. Son rayon visuel intérieur sortait à peine au dehors. Il avait pour esprit la lettre. Du reste altier. Son personnage tenait de la place. Il avait moins l’air d’un révérend que d’un monsignor. Sa redingote était un peu coupée en soutane. Son vrai milieu eût été Rome. Il était prélat de chambre, né. Il semblait avoir été créé exprès pour orner un pape, et pour marcher derrière la chaise gestatoire, avec toute la cour pontificale, \emph{in abitto paonazzo}. L’accident d’être né anglais, et une éducation théologique plus tournée vers l’ancien testament que vers le nouveau, lui avaient fait manquer cette grande destinée. Toutes ses splendeurs se résumaient en ceci, être recteur de Saint-Pierre-Port, doyen de l’île de Guernesey et subrogé de l’évêque de Winchester. C’était, sans nul doute, de la gloire.\par
Cette gloire n’empêchait pas M. Jaquemin Hérode d’être, à tout prendre, un assez bon homme.\par
Comme théologien, il était bien situé dans l’estime des connaisseurs, et il faisait presque autorité à la cour des Arches, cette Sorbonne de l’Angleterre.\par
Il avait la mine docte, un clignement d’yeux capable et exagéré, les narines velues, les dents visibles, la lèvre supérieure mince et la lèvre inférieure épaisse, plusieurs diplômes, une grosse prébende, des amis baronets, la confiance de l’évêque, et toujours une bible dans sa poche.\par
Mess Lethierry était si complètement absorbé que  tout ce que put produire l’entrée des deux prêtres fut un imperceptible froncement de sourcil.\par
M. Jaquemin Hérode s’avança, salua, rappela, en quelques mots sobrement hautains, sa promotion récente, et dit qu’il venait, selon l’usage, « introduire » près des notables, — et près de mess Lethierry en particulier, — son successeur dans la paroisse, le nouveau recteur de Saint-Sampson, le révérend Joë Ebenezer Caudray, désormais pasteur de mess Lethierry.\par
Déruchette se leva.\par
Le jeune prêtre, qui était le révérend Ebenezer, s’inclina.\par
Mess Lethierry regarda M. Ebenezer Caudray et grommela entre ses dents : mauvais matelot.\par
Grâce avança des chaises. Les deux révérends s’assirent près de la table.\par
Le docteur Hérode entama un speech. Il lui était revenu qu’il était arrivé un événement. La Durande avait fait naufrage. Il venait, comme pasteur, apporter des consolations et des conseils. Ce naufrage était malheureux, mais heureux aussi. Sondons-nous ; n’étions-nous pas enflés par la prospérité ? Les eaux de la félicité sont dangereuses. Il ne faut pas prendre en mauvaise part les malheurs. Les voies du Seigneur sont inconnues. Mess Lethierry était ruiné. Eh bien ? être riche, c’est être en danger. On a de faux amis. La pauvreté les éloigne. On reste seul. \emph{Solus eris.} La Durande rapportait, disait-on, mille livres sterling par an. C’est trop pour le sage. Fuyons les tentations,  dédaignons l’or. Acceptons avec reconnaissance la ruine et l’abandon. L’isolement est plein de fruits. On y obtient les grâces du Seigneur. C’est dans la solitude qu’Aia trouva les eaux chaudes, en conduisant les ânes de Sébéon son père. Ne nous révoltons pas contre les impénétrables décrets de la providence. Le saint homme Job, après sa misère, avait crû en richesse. Qui sait si la perte de la Durande n’aurait pas des compensations, même temporelles ? Ainsi, lui docteur Jaquemin Hérode, il avait engagé des capitaux dans une très belle opération en cours d’exécution à Sheffield ; si mess Lethierry, avec les fonds qui pouvaient lui rester, voulait entrer dans cette affaire, il y referait sa fortune ; c’était une grosse fourniture d’armes au czar en train de réprimer la Pologne. On y gagnerait trois cents pour cent.\par
Le mot czar parut réveiller Lethierry. Il interrompit le docteur Hérode.\par
— Je ne veux pas du czar.\par
Le révérend Hérode répondit :\par
— Mess Lethierry, les princes sont voulus de Dieu. Il est écrit : « Rendez à César ce qui est à César. » Le czar, c’est César.\par
Lethierry, à demi retombé dans son rêve, murmura :\par
— Qui ça, César ? Je ne connais pas.\par
Le révérend Jaquemin Hérode reprit son exhortation. Il n’insistait pas sur Sheffield. Ne pas vouloir de César, c’est être républicain. Le révérend comprenait qu’on fût républicain. En ce cas, que mess Lethierry se  tournât vers une république. Mess Lethierry pouvait rétablir sa fortune aux États-Unis mieux encore qu’en Angleterre. S’il voulait décupler ce qui lui restait, il n’avait qu’à prendre des actions dans la grande compagnie d’exploitation des plantations du Texas, laquelle employait plus de vingt mille nègres.\par
— Je ne veux pas de l’esclavage, dit Lethierry.\par
— L’esclavage, répliqua le révérend Hérode, est d’institution sacrée. Il est écrit : « Si le maître a frappé son esclave, il ne lui sera rien fait, car c’est son argent. »\par
Grâce et Douce, sur le seuil de la porte, recueillaient avec une sorte d’extase les paroles du révérend recteur.\par
Le révérend continua. C’était, somme toute, nous venons de le dire, un bon homme ; et, quels que pussent être ses dissentiments de caste ou de personne avec mess Lethierry, il venait très sincèrement lui apporter toute l’aide spirituelle, et même temporelle, dont lui, docteur Jaquemin Hérode, disposait.\par
Si mess Lethierry était ruiné à ce point de ne pouvoir coopérer fructueusement à une spéculation quelconque, russe ou américaine, que n’entrait-il dans le gouvernement et dans les fonctions salariées ? Ce sont de nobles places, et le révérend était prêt à y introduire mess Lethierry. L’office de député-vicomte était précisément vacant à Jersey. Mess Lethierry était aimé et estimé, et le révérend Hérode, doyen de Guernesey et subrogé de l’évêque, se faisait fort d’obtenir pour mess Lethierry l’emploi de député-vicomte de  Jersey. Le député-vicomte est un officier considérable ; il assiste, comme représentant de sa majesté, à la tenue des chefs-plaids, aux débats de la cohue, et aux exécutions des arrêts de justice.\par
Lethierry fixa sa prunelle sur le docteur Hérode.\par
— Je n’aime pas la pendaison, dit-il.\par
Le docteur Hérode, qui jusqu’alors avait prononcé tous les mots avec la même intonation, eut un accès de sévérité et une inflexion nouvelle.\par
— Mess Lethierry, la peine de mort est ordonnée divinement. Dieu a remis le glaive à l’homme. Il est écrit : « Œil pour œil, dent pour dent. »\par
Le révérend Ebenezer rapprocha imperceptiblement sa chaise de la chaise du révérend Jaquemin, et lui dit, de façon à n’être entendu que de lui :\par
— Ce que dit cet homme lui est dicté.\par
— Par qui ? par quoi ? demanda du même ton le révérend Jaquemin Hérode.\par
Ebenezer répondit très bas :\par
— Par sa conscience.\par
Le révérend Hérode fouilla dans sa poche, en tira un gros in-dix-huit relié avec fermoirs, le posa sur la table, et dit à voix haute :\par
— La conscience, la voici.\par
Le livre était une bible.\par
Puis le docteur Hérode s’adoucit. Son désir était d’être utile à mess Lethierry, qu’il considérait fort. Il avait, lui pasteur, droit et devoir de conseil ; pourtant mess Lethierry était libre.\par
Mess Lethierry, ressaisi par son absorption et par  son accablement, n’écoutait plus. Déruchette, assise près de lui, et pensive de son côté, ne levait pas les yeux et mêlait à cette conversation peu animée la quantité de gêne qu’apporte une présence silencieuse. Un témoin qui ne dit rien est une espèce de poids indéfinissable. Au surplus, le docteur Hérode ne semblait pas le sentir.\par
Lethierry ne répondant plus, le docteur Hérode se donna carrière. Le conseil vient de l’homme et l’inspiration vient de Dieu. Dans le conseil du prêtre il y a de l’inspiration. Il est bon d’accepter les conseils et dangereux de les rejeter. Sochoth fut saisi par onze diables pour avoir dédaigné les exhortations de Nathanaël. Tiburien fut frappé de la lèpre pour avoir mis hors de chez lui l’apôtre André. Barjésus, tout magicien qu’il était, devint aveugle pour avoir ri des paroles de saint Paul. Elxaï, et ses sœurs Marthe et Marthène, sont en enfer à l’heure qu’il est pour avoir méprisé les avertissements de Valencianus qui leur prouvait clair comme le jour que leur Jésus-Christ de trente-huit lieues de haut était un démon. Oolibama, qui s’appelle aussi Judith, obéissait aux conseils. Ruben et Pheniel écoutaient les avis d’en haut ; leurs noms seuls suffisent pour l’indiquer ; Ruben signifie \emph{fils de la vision}, et Pheniel signifie \emph{la face de Dieu}.\par
Mess Lethierry frappa du poing sur la table.\par
— Parbleu ! s’écria-t-il, c’est ma faute.\par
— Que voulez-vous dire ? demanda M. Jaquemin Hérode.\par
— Je dis que c’est ma faute.\par
 — Votre faute, quoi ?\par
— Puisque je faisais revenir Durande le vendredi.\par
M. Jaquemin Hérode murmura à l’oreille de M. Ebenezer Caudray : — Cet homme est superstitieux.\par
Il reprit en élevant la voix, et du ton de l’enseignement :\par
— Mess Lethierry, il est puéril de croire au vendredi. Il ne faut pas ajouter foi aux fables. Le vendredi est un jour comme un autre. C’est très souvent une date heureuse. Melendez a fondé la ville de Saint-Augustin un vendredi ; c’est un vendredi que Henri VII a donné sa commission à John Cabot ; les pèlerins du \emph{Mayflower} sont arrivés à Providence-Town un vendredi. Washington est né le vendredi 22 février 1732 ; Christoph Colomb a découvert l’Amérique le vendredi 12 octobre 1492.\par
Cela dit, il se leva.\par
Ebenezer, qu’il avait amené, se leva également.\par
Grâce et Douce, devinant que les révérends allaient prendre congé, ouvrirent la porte à deux battants.\par
Mess Lethierry ne voyait rien et n’entendait rien.\par
M. Jaquemin Hérode dit en aparté à M. Ebenezer Caudray : — Il ne nous salue même pas. Ce n’est pas du chagrin, c’est de l’abrutissement. Il faut croire qu’il est fou.\par
Cependant il prit sa petite bible sur la table et la tint entre ses deux mains allongées comme on tiendrait un oiseau qu’on craint de voir envoler. Cette attitude créa parmi les personnes présentes une certaine attente. Grâce et Douce avancèrent la tête.\par
 Sa voix fit tout ce qu’elle put pour être majestueuse.\par
— Mess Lethierry, ne nous séparons pas sans lire une page du saint livre. Les situations de la vie sont éclairées par les livres ; les profanes ont les sorts virgiliens, les croyants ont les avertissements bibliques. Le premier livre venu, ouvert au hasard, donne un conseil ; la bible, ouverte au hasard, fait une révélation. Elle est surtout bonne pour les affligés. Ce qui se dégage immanquablement de la sainte écriture, c’est l’adoucissement à leur peine. En présence des affligés, il faut consulter le saint livre sans choisir l’endroit, et lire avec candeur le passage sur lequel on tombe. Ce que l’homme ne choisit pas, Dieu le choisit. Dieu sait ce qu’il nous faut. Son doigt invisible est sur le passage inattendu que nous lisons. Quelle que soit cette page, il en sort infailliblement de la lumière. N’en cherchons pas d’autre, et tenons-nous-en là. C’est la parole d’en haut. Notre destinée nous est dite mystérieusement dans le texte évoqué avec confiance et respect. Écoutons, et obéissons. Mess Lethierry, vous êtes dans la douleur, ceci est le livre de consolation.\par
Le révérend Jaquemin Hérode fit jouer le ressort du fermoir, glissa son ongle à l’aventure entre deux pages, posa sa main un instant sur le livre ouvert, et se recueillit, puis, abaissant les yeux avec autorité, il se mit à lire à haute voix.\par
Ce qu’il lut, le voici :\par
« Isaac se promenait dans le chemin qui mène au puits appelé le Puits de celui qui vit et qui voit.\par
 « Rébecca, ayant aperçu Isaac, dit : Qui est cet homme qui vient au-devant de moi ?\par
« Alors Isaac la fit entrer dans sa tente, et la prit pour femme, et l’amour qu’il eut pour elle fut grand. »\par
Ebenezer et Déruchette se regardèrent.
 \section[{B. Deuxième partie. Gilliatt le Malin}]{B. Deuxième partie \\
Gilliatt le Malin}\renewcommand{\leftmark}{B. Deuxième partie \\
Gilliatt le Malin}

  \subsection[{B.I. Livre premier. L’écueil}]{B.I. Livre premier \\
L’écueil}
  \subsubsection[{B.I.1. L’endroit ou il est malaisé d’arriver et difficile de repartir}]{B.I.1. \\
L’endroit ou il est malaisé d’arriver et difficile de repartir}
\noindent La barque, aperçue sur plusieurs points de la côte de Guernesey dans la soirée précédente à des heures diverses, était, on l’a deviné, la panse. Gilliatt avait choisi le long de la côte le chenal à travers les rochers ; c’était la route périlleuse, mais c’était le chemin direct. Prendre le plus court avait été son seul souci. Les naufrages n’attendent pas, la mer est une chose pressante, une heure de retard pouvait être irréparable. Il voulait arriver vite au secours de la machine en danger.\par
Une des préoccupations de Gilliatt en quittant Guernesey parut être de ne point éveiller l’attention. Il partit de la façon dont on s’évade. Il eut un peu l’allure de se cacher. Il évita la côte est comme quelqu’un qui trouverait inutile de passer en vue de Saint-Sampson et de Saint-Pierre-Port ; il glissa, on pourrait  presque dire il se glissa, silencieusement le long de la côte opposée qui est relativement inhabitée. Dans les brisants, il dut ramer ; mais Gilliatt maniait l’aviron selon la loi hydraulique, prendre l’eau sans choc et la rendre sans vitesse, et de cette manière il put nager dans l’obscurité avec le plus de force et le moins de bruit possible. On eût pu croire qu’il allait faire une mauvaise action.\par
La vérité est que, se jetant tête baissée dans une entreprise fort ressemblante à l’impossible, et risquant sa vie avec toutes les chances à peu près contre lui, il craignait la concurrence.\par
Comme le jour commençait à poindre, les yeux inconnus qui sont peut-être ouverts dans les espaces purent voir au milieu de la mer, sur un des points où il y a le plus de solitude et de menace, deux choses entre lesquelles l’intervalle décroissait, l’une se rapprochant de l’autre. L’une, presque imperceptible dans le large mouvement des lames, était une barque à la voile ; dans cette barque il y avait un homme ; c’était la panse portant Gilliatt. L’autre, immobile, colossale, noire, avait au-dessus des vagues une surprenante figure. Deux hauts piliers soutenaient hors des flots dans le vide une sorte de traverse horizontale qui était comme un pont entre leurs sommets. La traverse, si informe de loin qu’il était impossible de deviner ce que c’était, faisait corps avec les deux jambages. Cela ressemblait à une porte. A quoi bon une porte dans cette ouverture de toutes parts qui est la mer ? On eût dit un dolmen titanique planté là, en plein océan, par  une fantaisie magistrale, et bâti par des mains qui ont l’habitude de proportionner leurs constructions à l’abîme. Cette silhouette farouche se dressait sur le clair du ciel.\par
La lueur du matin grandissait à l’est ; la blancheur de l’horizon augmentait la noirceur de la mer. En face, de l’autre côté, la lune se couchait.\par
Ces deux piliers, c’étaient les Douvres. L’espèce de masse emboîtée entre eux comme une architrave entre deux chambranles, c’était la Durande.\par
Cet écueil, tenant ainsi sa proie et la faisant voir, était terrible ; les choses ont parfois vis-à-vis de l’homme une ostentation sombre et hostile. Il y avait du défi dans l’attitude de ces rochers. Cela semblait attendre.\par
Rien d’altier et d’arrogant comme cet ensemble : le vaisseau vaincu, l’abîme maître. Les deux rochers, tout ruisselants encore de la tempête de la veille, semblaient des combattants en sueur. Le vent avait molli, la mer se plissait paisiblement, on devinait à fleur d’eau quelques brisants où les panaches d’écume retombaient avec grâce, il venait du large un murmure semblable à un bruit d’abeilles. Tout était de niveau, hors les deux Douvres, debout et droites comme deux colonnes noires. Elles étaient jusqu’à une certaine hauteur toutes velues de varech. Leurs hanches escarpées avaient des reflets d’armures. Elles semblaient prêtes à recommencer. On comprenait qu’elles étaient enracinées sous l’eau à des montagnes. Une sorte de toute-puissance tragique s’en dégageait.\par
 D’ordinaire la mer cache ses coups. Elle reste volontiers obscure. Cette ombre incommensurable garde tout pour elle. Il est très rare que le mystère renonce au secret. Certes, il y a du monstre dans la catastrophe, mais en quantité inconnue. La mer est patente et secrète ; elle se dérobe, elle ne tient pas à divulguer ses actions. Elle fait un naufrage, et le recouvre ; l’engloutissement est sa pudeur. La vague est hypocrite ; elle tue, recèle, ignore et sourit. Elle rugit, puis moutonne.\par
Ici rien de pareil. Les Douvres, élevant au-dessus des flots la Durande morte, avaient un air de triomphe. On eût dit deux bras monstrueux sortant du gouffre et montrant aux tempêtes ce cadavre de navire. C’était quelque chose comme l’assassin qui se vante.\par
L’horreur sacrée de l’heure s’y ajoutait. Le point du jour a une grandeur mystérieuse qui se compose d’un reste de rêve et d’un commencement de pensée. A ce moment trouble, un peu de spectre flotte encore. L’espèce d’immense H majuscule formée par les deux Douvres ayant la Durande pour trait d’union apparaissait à l’horizon dans on ne sait quelle majesté crépusculaire.\par
Gilliatt était vêtu de ses habits de mer, chemise de laine, bas de laine, souliers cloutés, vareuse de tricot, pantalon à poches de grosse étoffe bourrue, et sur la tête une de ces coiffes de laine rouge usitées alors dans la marine, qu’on appelait au siècle dernier \emph{galériennes}.\par
Il reconnut l’écueil et avança.\par
 La Durande était tout le contraire d’un navire coulé à fond ; c’était un navire accroché en l’air.\par
Pas de sauvetage plus étrange à entreprendre.\par
Il faisait plein jour quand Gilliatt arriva dans les eaux de l’écueil.\par
Il y avait, nous venons de le dire, peu de mer. L’eau avait seulement la quantité d’agitation que lui donne le resserrement entre les rochers. Toute manche, petite ou grande, clapote. L’intérieur d’un détroit écume toujours.\par
Gilliatt n’aborda point les Douvres sans précaution.\par
Il jeta la sonde plusieurs fois.\par
Gilliatt avait un petit débarquement à faire.\par
Habitué aux absences, il avait chez lui son en-cas de départ toujours prêt. C’était un sac de biscuit, un sac de farine de seigle, un panier de stock-fish et de bœuf fumé, un grand bidon d’eau douce, une caisse norvégienne à fleurs peintes contenant quelques grosses chemises de laine, son suroit et ses jambières goudronnées, et une peau de mouton qu’il jetait la nuit par-dessus sa vareuse. Il avait, en quittant le Bû de la Rue, mis tout cela en hâte dans la panse, plus un pain frais. Pressé de partir, il n’avait emporté d’autre engin de travail que son marteau de forgeron, sa hache et son hacherot, une scie, et une corde à nœuds armée de son grappin. Avec une échelle de cette sorte et la manière de s’en servir, les pentes revêches deviennent maniables, et un bon marin trouve des praticables dans les plus rudes escarpements. On  peut voir, dans l’île de Serk, le parti que tirent d’une corde à nœuds les pêcheurs du havre Gosselin.\par
Ses filets et ses lignes et tout son attirail de pêche étaient dans la barque. Il les y avait mis par habitude, et machinalement, car il allait, s’il donnait suite à son entreprise, séjourner quelque temps dans un archipel de brisants, et les engins de pêche n’y ont que faire.\par
Au moment où Gilliatt accosta l’écueil, la mer baissait, circonstance favorable. Les lames décroissantes laissaient à découvert au pied de la petite Douvre quelques assises plates ou peu inclinées, figurant assez bien des corbeaux à porter un plancher. Ces surfaces, tantôt étroites, tantôt larges, échelonnées avec des espacements inégaux le long du monolithe vertical, se prolongeaient en corniche mince jusque sous la Durande, laquelle faisait ventre entre les deux rochers. Elle était serrée là comme dans un étau.\par
Ces plates-formes étaient commodes pour débarquer et aviser. On pouvait décharger là, provisoirement, l’en-cas apporté dans la panse. Mais il fallait se hâter, elles n’étaient hors de l’eau que pour peu d’heures. A la mer montante, elles rentreraient sous l’écume.\par
Ce fut devant ces roches, les unes planes, les autres déclives, que Gilliatt poussa et arrêta la panse.\par
Une épaisseur mouillée et glissante de goémon les couvrait, l’obliquité augmentait çà et là le glissement.\par
Gilliatt se déchaussa, sauta pieds nus sur le goémon, et amarra la panse à une pointe de rocher.\par
Puis il s’avança le plus loin qu’il put sur l’étroite  corniche de granit, parvint sous la Durande, leva les yeux et la considéra.\par
La Durande était saisie, suspendue et comme ajustée dans les deux roches à vingt pieds environ au-dessus du flot. Il avait fallu pour la jeter là une furieuse violence de la mer.\par
Ces coups forcenés n’ont rien qui étonne les gens de mer. Pour ne citer qu’un exemple, le 25 janvier 1840, dans le golfe de Stora, une tempête finissante fit, du choc de sa dernière lame, sauter un brick, tout d’une pièce, par-dessus la carcasse échouée de la corvette \emph{la Marne,} et l’incrusta, beaupré en avant, entre deux falaises.\par
Du reste il n’y avait dans les Douvres qu’une moitié de la Durande.\par
Le navire, arraché aux vagues, avait été en quelque sorte déraciné de l’eau par l’ouragan. Le tourbillon de vent l’avait tordu, le tourbillon de mer l’avait retenu, et le bâtiment, ainsi pris en sens inverse par les deux mains de la tempête, s’était cassé comme une latte. L’arrière, avec la machine et les roues, enlevé hors de l’écume et chassé par toute la furie du cyclone dans le défilé des Douvres, y était entré jusqu’au maître-bau, et était demeuré là. Le coup de vent avait été bien assené ; pour enfoncer ce coin entre ces deux rochers, l’ouragan s’était fait massue. L’avant, emporté et roulé par la rafale, s’était disloqué sur les brisants.\par
La cale défoncée avait vidé dans la mer les bœufs noyés.\par
Un large morceau de la muraille de l’avant tenait  encore à l’arrière et pendait aux porques du tambour de gauche par quelques attaches délabrées, faciles à briser d’un coup de hache.\par
On voyait çà et là dans les anfractuosités lointaines de l’écueil des poutres, des planches, des haillons de voiles, des tronçons de chaînes, toutes sortes de débris, tranquilles sur les rochers.\par
Gilliatt regardait avec attention la Durande. La quille faisait plafond au-dessus de sa tête.\par
L’horizon, où l’eau illimitée remuait à peine, était serein. Le soleil sortait superbement de cette vaste rondeur bleue.\par
De temps en temps une goutte d’eau se détachait de l’épave et tombait dans la mer.
 \subsubsection[{B.I.2. Les perfections du désastre}]{B.I.2. \\
Les perfections du désastre}
\noindent Les Douvres étaient différentes de forme comme de hauteur.\par
Sur la petite Douvre, recourbée et aiguë, on voyait se ramifier, de la base à la cime, de longues veines d’une roche couleur brique, relativement tendre, qui cloisonnait de ses lames l’intérieur du granit. Aux affleurements de ces lames rougeâtres il y avait des cassures utiles à l’escalade. Une de ces cassures, située un peu au-dessus de l’épave, avait été si bien élargie et travaillée par les éclaboussures de la vague qu’elle était devenue une espèce de niche où l’on eût pu loger une statue. Le granit de la petite Douvre était arrondi à la surface et mousse comme de la pierre de touche, douceur qui ne lui ôtait rien de sa dureté. La petite Douvre se terminait en pointe comme une  corne. La grande Douvre, polie, unie, lisse, perpendiculaire, et comme taillée sur épure, était d’un seul morceau et semblait faite d’ivoire noir. Pas un trou, pas un relief. L’escarpement était inhospitalier ; un forçat n’eût pu s’en servir pour sa fuite ni un oiseau pour son nid. Au sommet il y avait, comme sur le rocher l’Homme, une plate-forme ; seulement cette plate-forme était inaccessible.\par
On pouvait monter sur la petite Douvre, mais non s’y maintenir ; on pouvait séjourner sur la grande, mais non y monter.\par
Gilliatt, le premier coup d’œil jeté, revint à la panse, la déchargea sur la plus large des corniches à fleur d’eau, fit de tout ce chargement, fort succinct, une sorte de ballot qu’il noua dans un prélart, y ajusta une élingue avec sa boucle de hissement, poussa ce ballot dans un recoin de roche où le flot ne pouvait l’atteindre, puis des pieds et des mains, de saillie en saillie, étreignant la petite Douvre, se cramponnant aux moindres stries, il monta jusqu’à la Durande échouée en l’air.\par
Parvenu à la hauteur des tambours, il sauta sur le pont.\par
Le dedans de l’épave était lugubre.\par
La Durande offrait toutes les traces d’une voie de fait épouvantable. C’était le viol effrayant de l’orage. La tempête se comporte comme une bande de pirates. Rien ne ressemble à un attentat comme un naufrage. La nuée, le tonnerre, la pluie, les souffles, les flots, les rochers, ce tas de complices est horrible.\par
 On rêvait sur le pont désemparé quelque chose comme le trépignement furieux des esprits de la mer. Il y avait partout des marques de rage. Les torsions étranges de certaines ferrures indiquaient les saisissements forcenés du vent. L’entre-pont était comme le cabanon d’un fou où tout était cassé.\par
Pas de bête comme la mer pour dépecer une proie. L’eau est pleine de griffes. Le vent mord, le flot dévore ; la vague est une mâchoire. C’est à la fois de l’arrachement et de l’écrasement. L’océan a le même coup de patte que le lion.\par
Le délabrement de la Durande offrait ceci de particulier qu’il était détaillé et minutieux. C’était une sorte d’épluchement terrible. Beaucoup de choses semblaient faites exprès. On pouvait dire : quelle méchanceté ! Les fractures des bordages étaient feuilletées avec art. Ce genre de ravage est propre au cyclone. Déchiqueter et amenuiser, tel est le caprice de ce dévastateur énorme. Le cyclone a des recherches de bourreau. Les désastres qu’il fait ont un air de supplices. On dirait qu’il a de la rancune ; il raffine comme un sauvage. Il dissèque en exterminant. Il torture le naufrage, il se venge, il s’amuse ; il y met de la petitesse.\par
Les cyclones sont rares dans nos climats, et d’autant plus redoutables qu’ils sont inattendus. Un rocher rencontré peut faire pivoter un orage. Il est probable que la bourrasque avait fait spirale sur les Douvres, et s’était brusquement tournée en trombe au choc de l’écueil, ce qui expliquait le jet du navire à une telle  hauteur dans ces roches. Quand le cyclone souffle, un vaisseau ne pèse pas plus au vent qu’une pierre à une fronde.\par
La Durande avait la plaie qu’aurait un homme coupé en deux ; c’était un tronc ouvert laissant échapper un fouillis de débris semblable à des entrailles. Des cordages flottaient et frissonnaient ; des chaînes se balançaient en grelottant ; les fibres et les nerfs du navire étaient à nu et pendaient. Ce qui n’était pas fracassé était désarticulé ; des fragments du mailletage du doublage étaient pareils à des étrilles hérissées de clous ; tout avait la forme de la ruine ; une barre d’anspec n’était plus qu’un morceau de fer, une sonde n’était plus qu’un morceau de plomb, un cap-de-mouton n’était plus qu’un morceau de bois ; une drisse n’était plus qu’un bout de chanvre, un touron n’était plus qu’un écheveau brouillé, une ralingue n’était plus qu’un fil dans un ourlet ; partout l’inutilité lamentable de la démolition ; rien qui ne fût décroché, décloué, lézardé, rongé, déjeté, sabordé, anéanti ; aucune adhésion dans ce monceau hideux, partout la déchirure, la dislocation, et la rupture, et ce je ne sais quoi d’inconsistant et de liquide qui caractérise tous les pêle-mêle, depuis les mêlées d’hommes qu’on nomme batailles jusqu’aux mêlées d’éléments qu’on nomme chaos. Tout croulait, tout coulait, et un ruissellement de planches, de panneaux, de ferrailles, de câbles et de poutres s’était arrêté au bord de la grande fracture de la quille, d’où le moindre choc pouvait tout précipiter dans la mer. Ce qui restait de cette puissante carène si triomphante  autrefois, tout cet arrière suspendu entre les deux Douvres et peut-être prêt à tomber, était crevassé çà et là et laissait voir par de larges trous l’intérieur sombre du navire.\par
L’écume crachait d’en bas sur cette chose misérable.
 \subsubsection[{B.I.3. Saine, mais non sauve}]{B.I.3. \\
Saine, mais non sauve}
\noindent Gilliatt ne s’attendait pas à ne trouver qu’une moitié du bâtiment. Rien dans les indications, pourtant si précises, du patron du \emph{Shealtiel}, ne faisait pressentir cette coupure du navire par le milieu. C’était probablement à l’instant où s’était faite cette coupure sous les épaisseurs aveuglantes de l’écume qu’avait eu lieu ce « craquement diabolique » entendu par le patron du \emph{Shealtiel.} Ce patron s’était sans doute éloigné au moment du dernier coup de vent, et ce qu’il avait pris pour un paquet de mer était une trombe. Plus tard, en se rapprochant pour observer l’échouement, il n’avait pu voir que la partie antérieure de l’épave, le reste, c’est-à-dire la large cassure qui avait séparé l’avant de l’arrière, lui étant caché par l’étranglement de l’écueil.\par
A cela près, le patron du \emph{Shealtiel} n’avait rien dit que d’exact. La coque était perdue, la machine était intacte.\par
Ces hasards sont fréquents dans les naufrages  comme dans les incendies. La logique du désastre nous échappe.\par
Les mâts cassés étaient tombés, la cheminée n’était pas même ployée ; la grande plaque de fer qui supportait la mécanique l’avait maintenue ensemble et tout d’une pièce. Les revêtements en planches des tambours étaient disjoints à peu près comme les lames d’une persienne ; mais à travers leurs claires-voies on distinguait les deux roues en bon état. Quelques pales manquaient.\par
Outre la machine, le grand cabestan de l’arrière avait résisté. Il avait sa chaîne, et, grâce à son robuste emboîtement dans un cadre de madriers, il pouvait rendre encore des services, pourvu toutefois que l’effort du tournevire ne fît pas fendre le plancher. Le tablier du pont fléchissait presque sur tous les points. Tout ce diaphragme était branlant.\par
En revanche le tronçon de la coque engagé entre les Douvres tenait ferme, nous l’avons dit, et semblait solide.\par
Cette conservation de la machine avait on ne sait quoi de dérisoire et ajoutait l’ironie à la catastrophe. La sombre malice de l’inconnu éclate quelquefois dans ces espèces de moqueries amères. La machine était sauvée, ce qui ne l’empêchait point d’être perdue. L’océan la gardait pour la démolir à loisir. Jeu de chat.\par
Elle allait agoniser là et se défaire pièce à pièce. Elle allait servir de jouet aux sauvageries de l’écume. Elle allait décroître jour par jour, et fondre pour ainsi dire. Qu’y faire ? Que ce lourd bloc de mécanismes et  d’engrenages, à la fois massif et délicat, condamné à l’immobilité par sa pesanteur, livré dans cette solitude aux forces démolissantes, mis par l’écueil à la discrétion du vent et du flot, pût, sous la pression de ce milieu implacable, échapper à la destruction lente, il semblait qu’il y eût folie rien qu’à l’imaginer.\par
La Durande était prisonnière des Douvres.\par
Comment la délivrer ?\par
Comment la tirer de là ?\par
L’évasion d’un homme est difficile ; mais quel problème que celui-ci : l’évasion d’une machine !
 \subsubsection[{B.I.4. Examen local préalable}]{B.I.4. \\
Examen local préalable}
\noindent Gilliatt n’était entouré que d’urgences. Le plus pressé pourtant était de trouver d’abord un mouillage pour la panse, puis un gîte pour lui-même.\par
La Durande s’étant plus tassée à bâbord qu’à tribord, le tambour de droite était plus élevé que le tambour de gauche.\par
Gilliatt monta sur le tambour de droite. De là il dominait la partie basse des brisants et, quoique le boyau des rochers, aligné à angles brisés derrière les Douvres, fit plusieurs coudes, Gilliatt put étudier le plan géométral de l’écueil.\par
Ce fut par cette reconnaissance qu’il commença.\par
Les Douvres, ainsi que nous l’avons indiqué déjà, étaient comme deux hauts pignons marquant l’entrée étroite d’une ruelle de petites falaises granitiques à devantures perpendiculaires. Il n’est point rare de trouver, dans les formations sous-marines primitives,  de ces corridors singuliers qui semblent coupés à la hache.\par
Ce défilé, fort tortueux, n’était jamais à sec, même dans les basses mers. Un courant très secoué le traversait toujours de part en part. La brusquerie des tournants était, selon le rumb de vent régnant, bonne ou mauvaise ; tantôt elle déconcertait la houle et la faisait tomber ; tantôt elle l’exaspérait. Ce dernier cas était le plus fréquent ; l’obstacle met le flot en colère et le pousse aux excès ; l’écume est l’exagération de la vague.\par
Le vent d’orage, dans ces étranglements entre deux roches, subit la même compression et acquiert la même malignité. C’est la tempête à l’état de strangurie. L’immense souffle reste immense et se fait aigu. Il est massue et dard. Il perce en même temps qu’il écrase. Qu’on se figure l’ouragan devenu vent coulis.\par
Les deux chaînes de rochers, laissant entre elles cette espèce de rue de la mer, s’étageaient plus bas que les Douvres en hauteurs graduellement décroissantes et s’enfonçaient ensemble dans le flot à une certaine distance. Il y avait là un autre goulet, moins élevé que le goulet des Douvres, mais plus étroit encore, et qui était l’entrée est du défilé. On devinait que le double prolongement des deux arêtes de roches continuait la rue sous l’eau jusqu’au rocher l’Homme placé comme une citadelle carrée à l’autre extrémité de l’écueil.\par
Du reste, à mer basse, et c’était l’instant où  Gilliatt observait, ces deux rangées de bas-fonds montraient leurs affleurements, quelques-uns à sec, tous visibles, et se coordonnant sans interruption.\par
L’Homme bordait et arc-boutait au levant la masse entière de l’écueil contrebutée au couchant par les deux Douvres.\par
Tout l’écueil, vu à vol d’oiseau, offrait un chapelet serpentant de brisants ayant à un bout les Douvres et à l’autre bout l’Homme.\par
L’écueil Douvres, pris dans son ensemble, n’était autre chose que l’émergement de deux gigantesques lames de granit se touchant presque et sortant verticalement, comme une crête, des cimes qui sont au fond de l’océan. Il y a hors de l’abîme de ces exfoliations immenses. La rafale et la houle avaient déchiqueté cette crête en scie. On n’en voyait que le haut ; c’était l’écueil. Ce que le flot cachait devait être énorme. La ruelle, où l’orage avait jeté la Durande, était l’entre-deux de ces lames colossales.\par
Cette ruelle, en zigzag comme l’éclair, avait à peu près sur tous les points la même largeur. L’océan l’avait ainsi faite. L’éternel tumulte dégage de ces régularités bizarres. Une géométrie sort de la vague.\par
D’un bout à l’autre du défilé, les deux murailles de roche se faisaient face parallèlement à une distance que le maître-couple de la Durande mesurait presque exactement. Entre les deux Douvres, l’évasement de la petite Douvre, recourbée et renversée, avait donné place aux tambours. Partout ailleurs les tambours eussent été broyés.\par
 La double façade intérieure de l’écueil était hideuse. Quand dans l’exploration du désert d’eau nommé océan on arrive aux choses inconnues de la mer, tout devient surprenant et difforme. Ce que Gilliatt, du haut de l’épave, pouvait apercevoir du défilé, faisait horreur. Il y a souvent dans les gorges granitiques de l’océan une étrange figuration permanente du naufrage. Le défilé des Douvres avait la sienne, effroyable. Les oxydes de la roche mettaient sur l’escarpement, çà et là, des rougeurs imitant des plaques de sang caillé. C’était quelque chose comme l’exsudation saignante d’un caveau de boucherie. Il y avait du charnier dans cet écueil. La rude pierre marine, diversement colorée, ici par la décomposition des amalgames métalliques mêlés à la roche, là par la moisissure, étalait par places des pourpres affreuses, des verdissements suspects, des éclaboussures vermeilles, éveillant une idée de meurtre et d’extermination. On croyait voir le mur pas essuyé d’une chambre d’assassinat. On eût dit que des écrasements d’hommes avaient laissé là leur trace ; la roche à pic avait on ne sait quelle empreinte d’agonies accumulées. En de certains endroits ce carnage paraissait ruisseler encore, la muraille était mouillée, et il semblait impossible d’y appuyer le doigt sans le retirer sanglant. Une rouille de massacre apparaissait partout. Au pied du double escarpement parallèle, épars à fleur d’eau, ou sous la lame, ou à sec dans les affouillements, de monstrueux galets ronds, les uns écarlates, les autres noirs ou violets, avaient des ressemblances de viscères ; on croyait voir des poumons  frais, ou des foies pourrissant. On eût dit que des ventres de géants avaient été vidés là. De longs fils rouges, qu’on eût pu prendre pour des suintements funèbres, rayaient du haut en bas le granit.\par
Ces aspects sont fréquents dans les cavernes de la mer.
 \subsubsection[{B.I.5. Un mot sur la collaboration secrète des éléments}]{B.I.5. \\
Un mot sur la collaboration secrète des éléments}
\noindent Pour ceux qui, par les hasards des voyages, peuvent être condamnés à l’habitation temporaire d’un écueil dans l’océan, la forme de l’écueil n’est point chose indifférente. Il y a l’écueil pyramide, une cime unique hors de l’eau ; il y a l’écueil cercle, quelque chose comme un rond de grosses pierres ; il y a l’écueil corridor. L’écueil corridor est le plus inquiétant. Ce n’est pas seulement à cause de l’angoisse du flot entre ses parois et des tumultes de la vague resserrée, c’est aussi à cause des obscures propriétés météorologiques qui semblent se dégager du parallélisme de deux roches en pleine mer. Ces deux lames droites sont un véritable appareil voltaïque.\par
Un écueil corridor est orienté. Cette orientation importe. Il en résulte une première action sur l’air et sur l’eau. L’écueil corridor agit sur le flot et sur le vent, mécaniquement, par sa forme, galvaniquement, par l’aimantation différente possible de ses plans  verticaux, masses juxtaposées et contrariées l’une par l’autre.\par
Cette nature d’écueils tire à elle toutes les forces furieuses éparses dans l’ouragan, et a sur la tourmente une singulière puissance de concentration.\par
De là, dans les parages de ces brisants, une certaine accentuation de la tempête.\par
Il faut savoir que le vent est composite. On croit le vent simple ; il ne l’est point. Cette force n’est pas seulement chimique, elle est magnétique. Il y a en elle de l’inexplicable. Le vent est électrique autant qu’aérien. De certains vents coïncident avec les aurores boréales. Le vent du banc des Aiguilles roule des vagues de cent pieds de haut, stupeur de Dumont d’Urville. — \emph{La corvette}, dit-il, \emph{ne savait à qui entendre.} — Sous les rafales australes, de vraies tumeurs maladives boursouflent l’océan, et la mer devient si horrible que les sauvages s’enfuient pour ne point la voir. Les rafales boréales sont autres ; elles sont toutes mêlées d’épingles de glace, et ces bises irrespirables refoulent en arrière sur la neige les traîneaux des esquimaux. D’autres vents brûlent. C’est le simoun d’Afrique, qui est le typhon de Chine et le samiel de l’Inde. Simoun, Typhon, Samiel ; on croit nommer des démons. Ils fondent le haut des montagnes ; un orage a vitrifié le volcan de Tolucca. Ce vent chaud, tourbillon couleur d’encre se ruant sur les nuées écarlates, a fait dire aux Védas : \emph{Voici le dieu noir qui vient voler les vaches rouges.} On sent dans tous ces faits la pression du mystère électrique.\par
 Le vent est plein de ce mystère. De même la mer. Elle aussi est compliquée ; sous ses vagues d’eau, qu’on voit, elle a ses vagues de forces, qu’on ne voit pas. Elle se compose de tout. De tous les pêle-mêle, l’océan est le plus indivisible et le plus profond.\par
Essayez de vous rendre compte de ce chaos, si énorme qu’il aboutit au niveau. Il est le récipient universel, réservoir pour les fécondations, creuset pour les transformations. Il amasse, puis disperse ; il accumule, puis ensemence ; il dévore, puis crée. Il reçoit tous les égouts de la terre, et il les thésaurise. Il est solide dans la banquise, liquide dans le flot, fluide dans l’effluve. Comme matière il est masse, et comme force il est abstraction. Il égalise et marie les phénomènes. Il se simplifie par l’infini dans la combinaison. C’est à force de mélange et de trouble qu’il arrive à la transparence. La diversité soluble se fond dans son unité. Il a tant d’éléments qu’il est l’identité. Une de ses gouttes, c’est tout lui. Parce qu’il est plein de tempêtes, il devient l’équilibre. Platon voyait danser les sphères ; chose étrange à dire, mais réelle, dans la colossale évolution terrestre autour du soleil, l’océan, avec son flux et reflux, est le balancier du globe.\par
Dans un phénomène de la mer, tous les phénomènes sont présents. La mer est aspirée par le tourbillon comme par un siphon ; un orage est un corps de pompe ; la foudre vient de l’eau comme de l’air ; dans les navires on sent de sourdes secousses, puis une odeur de soufre sort du puits des chaînes. L’océan bout. \emph{Le diable a mis la mer dans sa chaudière}, disait  Ruyter. En des certaines tempêtes qui caractérisent le remous des saisons et les entrées en équilibre des forces génésiaques, les navires battus de l’écume semblent exsuder une lueur, et des flammèches de phosphore courent sur les cordages, si mêlées à la manœuvre que les matelots tendent la main et tâchent de prendre au vol ces oiseaux de feu. Après le tremblement de terre de Lisbonne, une haleine de fournaise poussa vers la ville une lame de soixante pieds de hauteur. L’oscillation océanique se lie à la trépidation terrestre.\par
Ces énergies incommensurables rendent possibles tous les cataclysmes. A la fin de 1864, à cent lieues des côtes de Malabar, une des îles Maldives a sombré. Elle a coulé à fond comme un navire. Les pêcheurs partis le matin n’ont rien retrouvé le soir ; à peine ont-ils pu distinguer vaguement leurs villages sous la mer, et cette fois ce sont les barques qui ont assisté au naufrage des maisons.\par
En Europe où il semble que la nature se sente contrainte au respect de la civilisation, de tels événements sont rares jusqu’à l’impossibilité présumable. Pourtant Jersey et Guernesey ont fait partie de la Gaule ; et, au moment où nous écrivons ces lignes, un coup d’équinoxe vient de démolir sur la frontière d’Angleterre et d’Écosse la falaise Première des Quatre, \emph{First of the Fourth}.\par
Nulle part ces forces paniques n’apparaissent plus formidablement amalgamées que dans le surprenant détroit boréal nommé Lyse-Fiord. Le Lyse-Fiord est  le plus redoutable des écueils-boyaux de l’océan. La démonstration est là complète. C’est la mer de Norvège, le voisinage du rude golfe Stavanger, le cinquante-neuvième degré de latitude. L’eau est lourde et noire, avec une fièvre d’orages intermittents. Dans cette eau, au milieu de cette solitude, il y a une grande rue sombre. Rue pour personne. Nul n’y passe ; aucun navire ne s’y hasarde. Un corridor de dix lieues de long entre deux murailles de trois mille pieds de haut ; voilà l’entrée qui s’offre. Ce détroit a des coudes et des angles comme toutes les rues de la mer, jamais droites, étant faites par la torsion du flot. Dans le Lyse-Fiord, presque toujours la lame est tranquille, le ciel est serein ; lieu terrible. Où est le vent ? pas en haut. Où est le tonnerre ? pas dans le ciel. Le vent est sous la mer, la foudre est dans la roche. De temps en temps il y a un tremblement d’eau. A de certains moments, sans qu’il y ait un nuage en l’air, vers le milieu de la hauteur de la falaise verticale, à mille ou quinze cents pieds au-dessus des vagues, plutôt du côté sud que du côté nord, brusquement le rocher tonne, un éclair en sort, cet éclair s’élance, puis se retire, comme ces jouets qui s’allongent et se replient dans la main des enfants ; il a des contractions et des élargissements ; il se darde à la falaise opposée, rentre dans le rocher, puis en ressort, recommence, multiplie ses têtes et ses langues, se hérisse de pointes, frappe où il peut, recommence encore, puis s’éteint sinistre. Les volées d’oiseaux s’enfuient. Rien de mystérieux comme cette artillerie sortant de l’invisible. Un rocher attaque l’autre.  Les écueils s’entre-foudroient. Cette guerre ne regarde pas les hommes. Haine de deux murailles dans le gouffre.\par
Dans le Lyse-Fiord le vent tourne en effluve, la roche fait fonction de nuage et le tonnerre a des sorties de volcan. Ce détroit étrange est une pile ; il a pour éléments ses deux falaises.
 \subsubsection[{B.I.6. Une écurie pour le cheval}]{B.I.6. \\
Une écurie pour le cheval}
\noindent Gilliatt se connaissait assez en écueils pour prendre les Douvres fort au sérieux. Avant tout, nous venons de le dire, il s’agissait de mettre en sûreté la panse.\par
La double arête de récifs qui se prolongeait en tranchée sinueuse derrière les Douvres faisait elle-même groupe çà et là avec d’autres roches, et l’on y devinait des culs-de-sac et des caves se dégorgeant dans la ruelle et se rattachant au défilé principal comme des branches à un tronc.\par
La partie inférieure des brisants était tapissée de varech et la partie supérieure de lichen. Le niveau uniforme du varech sur toutes les roches marquait la ligne de flottaison de la marée pleine et de la mer étale. Les pointes que l’eau n’atteignait pas avaient cette argenture et cette dorure que donne aux granits marins le bariolage du lichen blanc et du lichen jaune.\par
Une lèpre de coquillages conoïdes couvrait la roche à de certains endroits. Carie sèche du granit.\par
 Sur d’autres points, dans les angles rentrants où s’était accumulé un sable fin ondé à la surface plutôt par le vent que par le flot, il y avait des touffes de chardon bleu.\par
Dans les redans peu battus de l’écume, on reconnaissait les petites tanières forées par l’oursin. Ce hérisson coquillage, qui marche, boule vivante, en roulant sur ses pointes, et dont la cuirasse se compose de plus de dix mille pièces artistement ajustées et soudées, l’oursin, dont la bouche s’appelle, on ne sait pourquoi, \emph{lanterne d’Aristote}, creuse le granit avec ses cinq dents qui mordent la pierre, et se loge dans le trou. C’est en ces alvéoles que les chercheurs de fruits de mer le trouvent. Ils le coupent en quatre et le mangent cru, comme l’huître. Quelques-uns trempent leur pain dans cette chair molle. De là son nom, \emph{œuf de mer.}\par
Les sommets lointains des bas-fonds, mis hors de l’eau par la marée descendante, aboutissaient sous l’escarpement même de l’Homme à une sorte de crique, murée presque de tous côtés par l’écueil. Il y avait là évidemment un mouillage possible. Gilliatt observa cette crique. Elle avait la forme d’un fer à cheval, et s’ouvrait d’un seul côté, au vent d’est, qui est le moins mauvais vent de ces parages. Le flot y était enfermé et presque dormant. Cette baie était tenable. Gilliatt d’ailleurs n’avait pas beaucoup de choix.\par
Si Gilliatt voulait profiter de la marée basse, il importait qu’il se hâtât.\par
Le temps, du reste, continuait d’être beau et doux.  L’insolente mer était maintenant de bonne humeur.\par
Gilliatt redescendit, se rechaussa, dénoua l’amarre, rentra dans sa barque et poussa en mer. Il côtoya à la rame le dehors de l’écueil.\par
Arrivé près de l’Homme, il examina l’entrée de la crique.\par
Une moire fixe dans la mobilité du flot, ride imperceptible à tout autre qu’un marin, dessinait la passe.\par
Gilliatt étudia un instant cette courbe, linéament presque indistinct dans la lame, puis il prit un peu de large afin de virer à l’aise et de faire bon chenal, et vivement, d’un seul coup d’aviron, il entra dans la petite anse.\par
Il sonda.\par
Le mouillage était excellent en effet.\par
La panse serait protégée là contre à peu près toutes les éventualités de la saison.\par
Les plus redoutables récifs ont de ces recoins paisibles. Les ports qu’on trouve dans l’écueil ressemblent à l’hospitalité du bédouin ; ils sont honnêtes et sûrs.\par
Gilliatt rangea la panse le plus près qu’il put de l’Homme, toutefois hors de la distance de talonnement, et mouilla ses deux ancres.\par
Cela fait, il croisa les bras et tint conseil avec lui-même.\par
La panse était abritée ; c’était un problème résolu ; mais le deuxième se présentait. Où s’abriter lui-même maintenant ?\par
Deux gîtes s’offraient ; la panse elle-même, avec  son coin de cabine à peu près habitable, et le plateau de l’Homme, aisé à escalader.\par
De l’un ou de l’autre de ces gîtes, on pourrait, à eau basse, et en sautant de roche en roche, gagner presque à pied sec l’entre-deux des Douvres où était la Durande.\par
Mais la marée basse ne dure qu’un moment, et tout le reste du temps on serait séparé, soit du gîte, soit de l’épave, par plus de deux cents brasses. Nager dans le flot d’un écueil est difficile ; pour peu qu’il y ait de la mer, c’est impossible.\par
Il fallait renoncer à la panse et à l’Homme.\par
Aucune station possible dans les rochers voisins.\par
Les sommets inférieurs s’effaçaient deux fois par jour sous la marée haute.\par
Les sommets supérieurs étaient sans cesse atteints par des bonds d’écume. Lavage inhospitalier.\par
Restait l’épave elle-même.\par
Pouvait-on s’y loger ?\par
Gilliatt l’espéra.
 \subsubsection[{B.I.7. Une chambre pour le voyageur}]{B.I.7. \\
Une chambre pour le voyageur}
\noindent Une demi-heure après, Gilliatt, de retour sur l’épave, montait et descendait du pont à l’entrepont et de l’entrepont à la cale, approfondissant l’examen sommaire de sa première visite.\par
Il avait, à l’aide du cabestan, hissé sur le pont de la Durande le ballot qu’il avait fait du chargement de la panse. Le cabestan s’était bien comporté. Les barres ne manquaient pas pour le virer. Gilliatt, dans ce tas de décombres, n’avait qu’à choisir.\par
Il trouva dans les débris un ciseau à froid, tombé sans doute de la baille du charpentier, et dont il augmenta sa petite caisse d’outils.\par
En outre, car dans le dénûment tout compte, il avait son couteau dans sa poche.\par
Gilliatt travailla toute la journée à l’épave, déblayant, consolidant, simplifiant.\par
Le soir venu, il reconnut ceci :\par
Toute l’épave était frissonnante au vent. Cette  carcasse tremblait à chaque pas que Gilliatt faisait. Il n’y avait de stable et de ferme que la partie de la coque, emboîtée entre les rochers, qui contenait la machine. Là, les baux s’arc-boutaient puissamment au granit.\par
S’installer dans la Durande était imprudent. C’était une surcharge ; et, loin de peser sur le navire, il importait de l’alléger.\par
Appuyer sur l’épave était le contraire de ce qu’il fallait faire.\par
Cette ruine voulait les plus grands ménagements. C’était comme un malade, qui expire. Il y aurait bien assez du vent pour la brutaliser.\par
C’était déjà fâcheux d’être contraint d’y travailler. La quantité de travail que l’épave aurait nécessairement à porter la fatiguerait certainement, peut-être au delà de ses forces.\par
En outre, si quelque accident de nuit survenait pendant le sommeil de Gilliatt, être dans l’épave, c’était sombrer avec elle. Nulle aide possible ; tout était perdu. Pour secourir l’épave, il fallait être dehors.\par
Être hors d’elle et près d’elle ; tel était le problème.\par
La difficulté se compliquait.\par
Où trouver un abri dans de telles conditions ?\par
Gilliatt songea.\par
Il ne restait plus que les deux Douvres. Elles semblaient peu logeables.\par
On distinguait d’en bas sur le plateau supérieur de la grande Douvre une espèce d’excroissance.\par
Les roches debout à cime plate, comme la grande  Douvre et l’Homme, sont des pics décapités. Ils abondent dans les montagnes et dans l’océan. Certains rochers, surtout parmi ceux qu’on rencontre au large, ont des entailles comme des arbres attaqués. Ils ont l’air d’avoir reçu un coup de cognée. Ils sont soumis en effet au vaste va-et-vient de l’ouragan, ce bûcheron de la mer.\par
Il existe d’autres causes de cataclysme, plus profondes encore. De là sur tous ces vieux granits tant de blessures. Quelques-uns de ces colosses ont la tête coupée.\par
Quelquefois, cette tête, sans qu’on puisse s’expliquer comment, ne tombe pas, et demeure, mutilée, sur le sommet tronqué. Cette singularité n’est point très rare. La Roque-au-Diable, à Guernesey, et la Table, dans la vallée d’Anweiler, offrent, dans les plus surprenantes conditions, cette bizarre énigme géologique.\par
Il était probablement arrivé à la grande Douvre quelque chose de pareil.\par
Si le renflement qu’on apercevait sur le plateau n’était pas une gibbosité naturelle de la pierre, c’était nécessairement quelque fragment restant du faîte ruiné.\par
Peut-être y avait-il dans ce morceau de roche une excavation.\par
Un trou où se fourrer ; Gilliatt n’en demandait pas davantage.\par
Mais comment atteindre au plateau ? comment gravir cette paroi verticale, dense et polie comme un caillou, à demi couverte d’une nappe de conferves visqueuses, et ayant l’aspect glissant d’une surface savonnée ?\par
 Il y avait trente pieds au moins du pont de la Durande à l’arête du plateau.\par
Gilliatt tira de sa caisse d’outils la corde à nœuds, se l’agrafa à la ceinture par le grappin, et se mit à escalader la petite Douvre. A mesure qu’il montait, l’ascension était plus rude. Il avait négligé d’ôter ses souliers, ce qui augmentait le malaise de la montée. Il ne parvint pas sans peine à la pointe. Arrivé à cette pointe, il se dressa debout. Il n’y avait guère de place que pour ses deux pieds. En faire son logis était difficile. Un stylite se fût contenté de cela ; Gilliatt, plus exigeant, voulait mieux.\par
La petite Douvre se recourbait vers la grande, ce qui faisait que de loin elle semblait la saluer ; et l’intervalle des deux Douvres, qui était d’une vingtaine de pieds en bas, n’était plus que de huit ou dix pieds en haut.\par
De la pointe où il avait gravi, Gilliatt vit plus distinctement l’ampoule rocheuse qui couvrait en partie la plate-forme de la grande Douvre.\par
Cette plate-forme s’élevait à trois toises au moins au-dessus de sa tête.\par
Un précipice l’en séparait.\par
L’escarpement de la petite Douvre en surplomb se dérobait sous lui.\par
Gilliatt détacha de sa ceinture la corde à nœuds, prit rapidement du regard les dimensions, et lança le grappin sur la plate-forme.\par
Le grappin égratigna la roche, puis dérapa. La corde à nœuds, ayant le grappin à son extrémité, retomba sous les pieds de Gilliatt le long de la petite Douvre.\par
 Gilliatt recommença, lançant la corde plus avant, et visant la protubérance granitique où il apercevait des crevasses et des stries.\par
Le jet fut si adroit et si net que le crampon se fixa.\par
Gilliatt tira dessus.\par
La roche cassa, et la corde à nœuds revint battre l’escarpement au-dessous de Gilliatt.\par
Gilliatt lança le grappin une troisième fois.\par
Le grappin ne retomba point.\par
Gilliatt fit effort sur la corde. Elle résista. Le grappin était ancré.\par
Il était arrêté dans quelque anfractuosité du plateau que Gilliatt ne pouvait voir.\par
Il s’agissait de confier sa vie à ce support inconnu.\par
Gilliatt n’hésita point.\par
Tout pressait. Il fallait aller au plus court.\par
D’ailleurs redescendre sur le pont de la Durande pour aviser à quelque autre mesure était presque impossible. Le glissement était probable, et la chute à peu près certaine. On monte, on ne redescend pas.\par
Gilliatt avait, comme tous les bons matelots, des mouvements de précision. Il ne perdait jamais de force. Il ne faisait que des efforts proportionnés. De là les prodiges de vigueur qu’il exécutait avec des muscles ordinaires ; il avait les biceps du premier venu, mais un autre cœur. Il ajoutait à la force, qui est physique, l’énergie, qui est morale.\par
La chose à faire était redoutable.\par
Franchir, pendu à ce fil, l’intervalle des deux Douvres ; telle était la question.\par
 On rencontre souvent, dans les actes de dévouement ou de devoir, de ces points d’interrogation qui semblent posés par la mort.\par
Feras-tu cela ? dit l’ombre.\par
Gilliatt exécuta une seconde traction d’essai sur le crampon ; le crampon tint bon.\par
Gilliat enveloppa sa main gauche de son mouchoir, étreignit la corde à nœuds du poing droit qu’il recouvrit de son poing gauche, puis tendant un pied en avant, et repoussant vivement de l’autre pied la roche afin que la vigueur de l’impulsion empêchât la rotation de la corde, il se précipita du haut de la petite Douvre sur l’escarpement de la grande.\par
Le choc fut dur.\par
Malgré la précaution prise par Gilliatt, la corde tourna, et ce fut son épaule qui frappa le rocher.\par
Il y eut rebondissement.\par
A leur tour ses poings heurtèrent la roche. Le mouchoir s’était dérangé. Ils furent écorchés ; c’était beaucoup qu’ils ne fussent pas brisés.\par
Gilliatt demeura un moment étourdi et suspendu.\par
Il fut assez maître de son étourdissement pour ne point lâcher la corde.\par
Un certain temps s’écoula en oscillations et en soubresauts avant qu’il pût saisir la corde avec ses pieds ; il y parvint pourtant.\par
Revenu à lui, et tenant la corde entre ses pieds comme dans ses mains, il regarda en bas.\par
Il n’était pas inquiet sur la longueur de sa corde, qui lui avait plus d’une fois servi pour de plus grandes  hauteurs. La corde, en effet, traînait sur le pont de la Durande.\par
Gilliatt, sûr de pouvoir redescendre, se mit à grimper.\par
En quelques instants il atteignit le plateau.\par
Jamais rien que d’ailé n’avait posé le pied là. Ce plateau était couvert de fientes d’oiseaux. C’était un trapèze irrégulier, cassure de ce colossal prisme granitique nommé la grande Douvre. Ce trapèze était creusé au centre comme une cuvette. Travail des pluies.\par
Gilliatt, du reste, avait conjecturé juste. On voyait à l’angle méridional du trapèze une superposition de rochers, décombres probables de l’écroulement du sommet. Ces rochers, espèce de tas de pavés démesurés, laissaient à une bête fauve qui eût été fourvoyée sur cette cime de quoi se glisser entre eux. Ils s’équilibraient pêle-mêle ; ils avaient les interstices d’un monceau de gravats. Il n’y avait là ni grotte, ni antre, mais des trous comme dans une éponge. Une de ces tanières pouvait admettre Gilliatt.\par
Cette tanière avait un fond d’herbe et de mousse.\par
Gilliatt serait là comme dans une gaine.\par
L’alcôve, à l’entrée, avait deux pieds de haut. Elle allait se rétrécissant vers le fond. Il y a des cercueils de pierre qui ont cette forme. L’amas de rochers étant adossé au sud-ouest, la tanière était garantie des ondées, mais ouverte au vent du nord.\par
Gilliatt trouva que c’était bon.\par
Les deux problèmes étaient résolus ; la panse avait un port et il y avait un logis.\par
 L’excellence de ce logis était d’être à portée de l’épave.\par
Le grappin de la corde à nœuds, tombé entre deux quartiers de roche, s’y était solidement accroché. Gilliatt l’immobilisa en mettant dessus une grosse pierre.\par
Puis il entra immédiatement en libre pratique avec la Durande.\par
Il était chez lui désormais.\par
La grande Douvre était sa maison ; la Durande était son chantier.\par
Aller et venir, monter et descendre, rien de plus simple.\par
Il dégringola vivement de la corde à nœuds sur le pont.\par
La journée était bonne, cela commençait bien, il était content, il s’aperçut qu’il avait faim.\par
Il déficela son panier de provisions, ouvrit son couteau, coupa une tranche de bœuf fumé, mordit sa miche de pain bis, but un coup au bidon d’eau douce, et soupa admirablement.\par
Bien faire et bien manger, ce sont là deux joies. L’estomac plein ressemble à une conscience satisfaite.\par
Son souper fini, il y avait encore un peu de jour. Il en profita pour commencer l’allégement, très urgent, de l’épave.\par
Il avait passé une partie de la journée à trier les décombres. Il mit de côté, dans le compartiment solide où était la machine, tout ce qui pouvait servir, bois, fer, cordage, toile. Il jeta à la mer l’inutile.\par
Le chargement de la panse, hissé par le cabestan  sur le pont, était, quelque sommaire qu’il fût, un encombrement. Gilliatt avisa l’espèce de niche creusée, à une hauteur que sa main pouvait atteindre, dans la muraille de la petite Douvre. On voit souvent dans les rochers de ces armoires naturelles, point fermées, il est vrai. Il pensa qu’il était possible de confier à cette niche un dépôt. Il mit au fond ses deux caisses, celle des outils et celle des vêtements, ses deux sacs, le seigle et le biscuit, et sur le devant, un peu trop près du bord peut-être, mais il n’avait pas d’autre place, le panier de provisions.\par
Il avait eu le soin de retirer de la caisse aux vêtements sa peau de mouton, son suroit à capuchon et ses jambières goudronnées.\par
Pour ôter prise au vent sur la corde à nœuds, il en attacha l’extrémité inférieure à une porque de la Durande.\par
La Durande ayant beaucoup de rentrée, cette porque avait beaucoup de courbure, et tenait le bout de la corde aussi bien que l’eût fait une main fermée.\par
Restait le haut de la corde. Assujettir le bas était bien, mais au sommet de l’escarpement, à l’endroit où la corde à nœuds rencontrait l’arête de la plate-forme, il était à craindre qu’elle ne fût peu à peu sciée par l’angle vif du rocher.\par
Gilliatt fouilla le monceau de décombres en réserve, et y prit quelques loques de toile à voile et, dans un tronçon de vieux câbles, quelques longs brins de fil de caret, dont il bourra ses poches.\par
Un marin eût deviné qu’il allait capitonner avec ces  morceaux de toile et ces bouts de fil le pli de la corde à nœuds sur le coupant du rocher, de façon à le préserver de toute avarie ; opération qui s’appelle fourrure.\par
Sa provision de chiffons faite, il se passa les jambières aux jambes, endossa le suroit par-dessus sa vareuse, rabattit le capuchon sur sa galérienne, se noua au cou par les deux pattes la peau de mouton, et ainsi vêtu de cette panoplie complète, il empoigna la corde, robustement fixée désormais au flanc de la grande Douvre, et il monta à l’assaut de cette sombre tour de la mer.\par
Gilliatt, en dépit de ses mains écorchées, arriva lestement au plateau.\par
Les dernières pâleurs du couchant s’éteignaient. Il faisait nuit sur la mer. Le haut de la Douvre gardait un peu de lueur.\par
Gilliatt profita de ce reste de clarté pour fourrer la corde à nœuds. Il lui appliqua, au coude qu’elle faisait sur le rocher, un bandage de plusieurs épaisseurs de toile, fortement ficelé à chaque épaisseur. C’était quelque chose comme la garniture que se mettent aux genoux les actrices pour les agonies et les supplications du cinquième acte.\par
La fourrure terminée, Gilliatt accroupi se redressa.\par
Depuis quelques instants, pendant qu’il ajustait ces loques sur la corde à nœuds, il percevait confusément en l’air un frémissement singulier.\par
Cela ressemblait, dans le silence du soir, au bruit que ferait le battement d’ailes d’une immense chauve-souris.\par
Gilliatt leva les yeux.\par
 Un grand cercle noir tournait au-dessus de sa tête dans le ciel profond et blanc du crépuscule.\par
On voit, dans les vieux tableaux, de ces cercles sur la tête des saints. Seulement ils sont d’or sur un fond sombre ; celui-ci était ténébreux sur un fond clair. Rien de plus étrange. On eût dit l’auréole de nuit de la grande Douvre.\par
Ce cercle s’approchait de Gilliatt et ensuite s’éloignait ; se rétrécissant, puis s’élargissant.\par
C’étaient des mouettes, des goëlands, des frégates, des cormorans, des mauves, une nuée d’oiseaux de mer, étonnés.\par
Il est probable que la grande Douvre était leur auberge et qu’ils venaient se coucher. Gilliatt y avait pris une chambre. Ce locataire inattendu les inquiétait.\par
Un homme là, c’est ce qu’ils n’avaient jamais vu.\par
Ce vol effaré dura quelque temps.\par
Ils paraissaient attendre que Gilliatt s’en allât.\par
Gilliatt, vaguement pensif, les suivait du regard.\par
Ce tourbillon volant finit par prendre son parti, le cercle tout à coup se défit en spirale, et ce nuage de cormorans alla s’abattre, à l’autre bout de l’écueil, sur l’Homme.\par
Là, ils parurent se consulter et délibérer. Gilliatt, tout en s’allongeant dans son fourreau de granit, et tout en se mettant sous la joue une pierre pour oreiller, entendit longtemps les oiseaux parler l’un après l’autre, chacun à son tour de croassement.\par
Puis ils se turent, et tout s’endormit, les oiseaux sur leur rocher, Gilliatt sur le sien.
 \subsubsection[{B.I.8. Impotunæque volucres}]{B.I.8. \\
Impotunæque volucres}
\noindent Gilliatt dormit bien. Pourtant il eut froid, ce qui le réveilla de temps en temps. Il avait naturellement placé ses pieds au fond et sa tête au seuil. Il n’avait pas pris le soin d’ôter de son lit une multitude de cailloux assez tranchants qui n’amélioraient pas son sommeil.\par
Par moments, il entr’ouvrait les yeux.\par
Il entendait à de certains instants des détonations profondes. C’était la mer montante qui entrait dans les caves de l’écueil avec un bruit de coup de canon.\par
Tout ce milieu où il était offrait l’extraordinaire de la vision ; Gilliatt avait de la chimère autour de lui. Le demi-étonnement de la nuit s’y ajoutant, il se voyait plongé dans l’impossible. Il se disait : Je rêve.\par
Puis il se rendormait, et, en rêve alors, il se retrouvait au Bû de la Rue, aux Bravées, à Saint-Sampson ; il entendait chanter Déruchette ; il était dans le réel. Tant qu’il dormait, il croyait veiller et vivre ; quand il se réveillait, il croyait dormir.\par
 En effet, il était désormais dans un songe.\par
Vers le milieu de la nuit, une vaste rumeur s’était faite dans le ciel. Gilliatt en avait confusément conscience à travers son sommeil. Il est probable que la brise s’élevait.\par
Une fois, qu’un frisson de froid le réveilla, il écarta les paupières un peu plus qu’il n’avait fait encore. Il y avait de larges nuées au zénith ; la lune s’enfuyait et une grosse étoile courait après elle.\par
Gilliatt avait l’esprit plein de la diffusion des songes, et ce grossissement du rêve compliquait les farouches paysages de la nuit.\par
Au point du jour il était glacé, et dormait profondément.\par
La brusquerie de l’aurore le tira de ce sommeil, dangereux peut-être. Son alcôve faisait face au soleil levant.\par
Gilliatt bâilla, s’étira, et se jeta hors de son trou.\par
Il dormait si bien qu’il ne comprit pas d’abord.\par
Peu à peu le sentiment de la réalité lui revint, et à tel point qu’il s’écria : Déjeunons !\par
Le temps était calme, le ciel était froid et serein, il n’y avait plus de nuages, le balayage de la nuit avait nettoyé l’horizon, le soleil se levait bien. C’était une seconde belle journée qui commençait. Gilliatt se sentit joyeux.\par
Il quitta son suroit et ses jambières, les roula dans la peau de mouton, la laine en dedans, noua le rouleau d’un bout de funin, et le poussa au fond de la tanière, à l’abri d’une pluie éventuelle.\par
 Puis il fit son lit, c’est-à-dire retira les cailloux.\par
Son lit fait, il se laissa glisser le long de la corde sur le pont de la Durande, et courut à la niche où il avait déposé le panier de provisions.\par
Le panier n’y était plus. Comme il était fort près du bord, le vent de la nuit l’avait emporté et jeté dans la mer.\par
Ceci annonçait l’intention de se défendre.\par
Il avait fallu au vent une certaine volonté et une certaine malice pour aller chercher là ce panier.\par
C’était un commencement d’hostilités. Gilliatt le comprit.\par
Il est très difficile, quand on vit dans la familiarité bourrue de la mer, de ne point regarder le vent comme quelqu’un et les rochers comme des personnages.\par
Il ne restait plus à Gilliatt, avec le biscuit et la farine de seigle, que la ressource des coquillages dont s’était nourri le naufragé mort de faim sur le rocher l’Homme.\par
Quant à la pêche, il n’y fallait point songer. Le poisson, ennemi des chocs, évite les brisants ; les nasses et les chaluts perdent leur peine dans les récifs, et ces pointes ne sont bonnes qu’à déchirer les filets.\par
Gilliatt déjeuna de quelques poux de roque, qu’il détacha fort malaisément du rocher. Il faillit y casser son couteau.\par
Tandis qu’il faisait ce luncheon maigre, il entendit un bizarre tumulte sur la mer. Il regarda.\par
 C’était l’essaim de goélands et de mouettes qui venait de se ruer sur une des roches basses, battant de l’aile, s’entre-culbutant, criant, appelant. Tous fourmillaient bruyamment sur le même point. Cette horde à bec et ongles pillait quelque chose.\par
Ce quelque chose était le panier de Gilliatt.\par
Le panier, lancé sur une pointe par le vent, s’y était crevé. Les oiseaux étaient accourus. Ils emportaient dans leurs becs toutes sortes de lambeaux déchiquetés. Gilliatt reconnut de loin son bœuf fumé et son stockfish.\par
Les oiseaux entraient en lutte à leur tour. Ils avaient, eux aussi, leurs représailles. Gilliatt leur avait pris leur logis ; ils lui prenaient son souper.
 \subsubsection[{B.I.9. L’écueil, et la manière de s’en servir}]{B.I.9. \\
L’écueil, et la manière de s’en servir}
\noindent Une semaine se passa.\par
Quoiqu’on fût dans une saison de pluie, il ne pleuvait pas, ce qui réjouissait fort Gilliatt.\par
Du reste, ce qu’il entreprenait dépassait, en apparence du moins, la force humaine. Le succès était tellement invraisemblable que la tentative paraissait folle.\par
Les opérations serrées de près manifestent leurs empêchements et leurs périls. Rien n’est tel que de commencer pour voir combien il sera malaisé de finir. Tout début résiste. Le premier pas qu’on fait est un révélateur inexorable. La difficulté qu’on touche pique comme une épine.\par
Gilliatt eut tout de suite à compter avec l’obstacle.\par
Pour tirer du naufrage, où elle était aux trois quarts enfoncée, la machine de la Durande, pour tenter, avec quelque chance de réussite, un tel sauvetage en un tel lieu dans une telle saison, il semblait qu’il fallût être une troupe d’hommes, Gilliatt était seul ; il fallait  tout un outillage de charpenterie et de machinerie, Gilliatt avait une scie, une hache, un ciseau et un marteau ; il fallait un bon atelier et un bon baraquement, Gilliatt n’avait pas de toit ; il fallait des provisions et des vivres, Gilliatt n’avait pas de pain.\par
Quelqu’un qui, pendant toute cette première semaine, eût vu Gilliatt travailler dans l’écueil, ne se fût pas rendu compte de ce qu’il voulait faire. Il semblait ne plus songer à la Durande ni aux deux Douvres. Il n’était occupé que de ce qu’il y avait dans les brisants ; il paraissait absorbé dans le sauvetage des petites épaves. Il profitait des marées basses pour dépouiller les récifs de tout ce que le naufrage leur avait partagé. Il allait de roche en roche ramasser ce que la mer y avait jeté, les haillons de voilure, les bouts de corde, les morceaux de fer, les éclats de panneaux, les bordages défoncés, les vergues cassées, là une poutre, là une chaîne, là une poulie.\par
En même temps il étudiait toutes les anfractuosités de l’écueil. Aucune n’était habitable, au grand désappointement de Gilliatt qui avait froid la nuit dans l’entre-deux de pavés où il logeait sur le comble de la grande Douvre, et qui eût souhaité trouver une meilleure mansarde.\par
Deux de ces anfractuosités étaient assez spacieuses ; quoique le dallage de roche naturel en fût presque partout oblique et inégal, on pouvait s’y tenir debout et y marcher. La pluie et le vent y avaient leurs aises, mais les plus hautes marées ne les atteignaient point. Elles étaient voisines de la petite Douvre, et  d’un abord possible à toute heure. Gilliatt décida que l’une serait un magasin, et l’autre une forge.\par
Avec tous les rabans de têtière et tous les rabans de pointure qu’il put recueillir, il fit des ballots des menues épaves, liant les débris en faisceaux et les toiles en paquets. Il aiguilleta soigneusement le tout. A mesure que la marée en montant venait renflouer ces ballots, il les traînait à travers les récifs jusqu’à son magasin. Il avait trouvé dans un creux de roche une guinderesse au moyen de laquelle il pouvait haler même les grosses pièces de charpente. Il tira de la mer de la même façon les nombreux tronçons de chaînes, épars dans les brisants.\par
Gilliatt était tenace et étonnant dans ce labeur. Il faisait tout ce qu’il voulait. Rien ne résiste à un acharnement de fourmi.\par
A la fin de la semaine, Gilliatt avait dans ce hangar de granit tout l’informe bric-à-brac de la tempête mis en ordre. Il y avait le coin des écouets et le coin des écoutes ; les boulines n’étaient point mêlées avec les drisses ; les bigots étaient rangés selon la quantité de trous qu’ils avaient ; les emboudinures, soigneusement détachées des organeaux des ancres brisées, étaient roulées en écheveaux ; les moques, qui n’ont point de rouet, étaient séparées des moufles ; les cabillots, les margouillets, les pataras, les gabarons, les joutereaux, les calebas, les galoches, les pantoires, les oreilles d’âne, les racages, les bosses, les boute-hors, occupaient, pourvu qu’ils ne fussent pas complètement défigurés par l’avarie, des compartiments différents ;  toute la charpente, traversins, piliers, épontilles, chouquets, mantelets, jumelles, hiloires, était entassée à part ; chaque fois que cela avait été possible, les planches des fragments de franc-bord embouffeté avaient été rentrées les unes dans les autres ; il n’y avait nulle confusion des garcettes de ris avec les garcettes de tournevire, ni des araignées avec les touées, ni des poulies de galauban avec les poulies de franc-funin, ni des morceaux de virure avec les morceaux de vibord ; un recoin avait été réservé à une partie du trelingage de la Durande, qui appuyait les haubans de hune et les gambes de hune. Chaque débris avait sa place. Tout le naufrage était là, classé et étiqueté. C’était quelque chose comme le chaos en magasin.\par
Une voile d’étai, fixée par de grosses pierres, recouvrait, fort trouée il est vrai, ce que la pluie pouvait endommager.\par
Si fracassé qu’eût été l’avant de la Durande, Gilliatt était parvenu à sauver les deux bossoirs avec leurs trois roues de poulies.\par
Il retrouva le beaupré, et il eut beaucoup de peine à en dérouler les liures ; elles étaient fort adhérentes, ayant été, comme toujours, faites au cabestan, et par un temps sec. Gilliatt pourtant les détacha, ce gros funin pouvant lui être fort utile.\par
Il avait également recueilli la petite ancre qui était demeurée accrochée dans un creux de bas-fond où la mer descendante la découvrait.\par
Il trouva dans ce qui avait été la cabine de  Tangrouille un morceau de craie, et le serra soigneusement. On peut avoir des marques à faire.\par
Un seau de cuir à incendie et plusieurs bailles en assez bon état complétaient cet en-cas de travail.\par
Tout ce qui restait du chargement de charbon de terre de la Durande fut porté dans le magasin.\par
En huit jours ce sauvetage des débris fut achevé ; l’écueil fut nettoyé, et la Durande fut allégée. Il ne resta plus sur l’épave que la machine.\par
Le morceau de la muraille de l’avant qui adhérait à l’arrière ne fatiguait point la carcasse. Il y pendait sans tiraillement, étant soutenu par une saillie de roche. Il était d’ailleurs large et vaste, et lourd à traîner, et il eût encombré le magasin. Ce panneau de muraille avait l’aspect d’un radeau. Gilliatt le laissa où il était.\par
Gilliatt, profondément pensif dans ce labeur, chercha en vain la « poupée » de la Durande. C’était une des choses que le flot avait à jamais emportées. Gilliatt, pour la retrouver, eût donné ses deux bras, s’il n’en eût pas eu tant besoin.\par
A l’entrée du magasin et en dehors, on voyait deux tas de rebut, le tas de fer, bon à reforger, et le tas de bois, bon à brûler.\par
Gilliatt était à la besogne au point du jour. Hors des heures de sommeil, il ne prenait pas un moment de repos.\par
Les cormorans, volant çà et là, le regardaient travailler.
 \subsubsection[{B.I.10. La forge}]{B.I.10. \\
La forge}
\noindent Le magasin fait, Gilliatt fit la forge.\par
La deuxième anfractuosité choisie par Gilliatt offrait un réduit, espèce de boyau, assez profond. Il avait eu d’abord l’idée de s’y installer ; mais la bise, se renouvelant sans cesse, était si continue et si opiniâtre dans ce couloir qu’il avait dû renoncer à habiter là. Ce soufflet lui donna l’idée d’une forge. Puisque cette caverne ne pouvait être sa chambre, elle serait son atelier. Se faire servir par l’obstacle est un grand pas vers le triomphe. Le vent était l’ennemi de Gilliatt, Gilliatt entreprit d’en faire son valet.\par
Ce qu’on dit de certains hommes : — propres à tout, bons à rien, — on peut le dire des creux de rocher. Ce qu’ils offrent, ils ne le donnent point. Tel creux de rocher est une baignoire, mais qui laisse fuir l’eau par une fissure ; tel autre est une chambre, mais sans plafond ; tel autre est un lit de mousse, mais mouillée ; tel autre est un fauteuil, mais de pierre.\par
 La forge que Gilliatt voulait établir était ébauchée par la nature ; mais dompter cette ébauche jusqu’à la rendre maniable, et transformer cette caverne en laboratoire, rien n’était plus âpre et plus malaisé. Avec trois ou quatre larges roches évidées en entonnoir et aboutissant à une fêlure étroite, le hasard avait fait là une espèce de vaste soufflante informe, bien autrement puissante que ces anciens grands soufflets de forge de quatorze pieds de long, lesquels donnaient en bas, par chaque coup d’haleine, quatrevingt-dix-huit mille pouces d’air. C’était ici tout autre chose. Les proportions de l’ouragan ne se calculent pas.\par
Cet excès de force était une gêne ; il était difficile de régler ce souffle.\par
La caverne avait deux inconvénients ; l’air la traversait de part en part, l’eau aussi.\par
Ce n’était point la lame marine, c’était un petit ruissellement perpétuel, plus semblable à un suintement qu’à un torrent.\par
L’écume, sans cesse lancée par le ressac sur l’écueil, quelquefois à plus de cent pieds en l’air, avait fini par emplir d’eau de mer une cuve naturelle située dans les hautes roches qui dominaient l’excavation. Le trop-plein de ce réservoir faisait, un peu en arrière, dans l’escarpement, une mince chute d’eau, d’un pouce environ, tombant de quatre ou cinq toises. Un contingent de pluie s’y ajoutait. De temps en temps un nuage versait en passant une ondée dans ce réservoir inépuisable et toujours débordant. L’eau en était saumâtre, non potable, mais limpide, quoique salée. Cette  chute s’égouttait gracieusement aux extrémités des conferves comme aux pointes d’une chevelure.\par
Gilliatt songea à se servir de cette eau pour discipliner ce vent. Au moyen d’un entonnoir, de deux ou trois tuyaux en planches menuisés et ajustés à la hâte, dont un à robinet, et d’une baille très large disposée en réservoir inférieur, sans flasque et sans contre-poids, en complétant seulement l’engin par un étranguillon en haut et des trous aspirateurs en bas, Gilliatt, qui était, nous l’avons dit, un peu forgeron et un peu mécanicien, parvint à composer, pour remplacer le soufflet de forge qu’il n’avait pas, un appareil moins parfait que ce qu’on nomme aujourd’hui une cagniardelle, mais moins rudimentaire que ce qu’on appelait jadis dans les Pyrénées une trompe.\par
Il avait de la farine de seigle, il en fit de la colle ; il avait du funin blanc, il en fit de l’étoupe. Avec cette étoupe et cette colle et quelques coins de bois, il boucha toutes les fissures du rocher, ne laissant qu’un bec d’air, fait d’un petit tronçon d’espoulette qu’il trouva dans la Durande et qui avait servi de boute-feu au pierrier de signal. Ce bec d’air était horizontalement dirigé sur une large dalle où Gilliatt mit le foyer de la forge. Un bouchon, fait d’un bout de touron, le fermait au besoin.\par
Après quoi, Gilliatt empila du charbon et du bois dans ce foyer, battit le briquet sur le rocher même, fit tomber l’étincelle sur une poignée d’étoupe, et avec l’étoupe allumée, alluma le bois et le charbon.\par
Il essaya la soufflante. Elle fit admirablement.\par
 Gilliatt sentit une fierté de cyclope, maître de l’air, de l’eau et du feu.\par
Maître de l’air ; il avait donné au vent une espèce de poumon, créé dans le granit un appareil respiratoire, et changé la soufflante en soufflet. Maître de l’eau ; de la petite cascade, il avait fait une trompe. Maître du feu ; de ce rocher inondé, il avait fait jaillir la flamme.\par
L’excavation étant presque partout à ciel ouvert, la fumée s’en allait librement, noircissant l’escarpement en surplomb. Ces rochers, qui semblaient à jamais faits pour l’écume, connurent la suie.\par
Gilliatt prit pour enclume un gros galet roulé d’un grain très dense, offrant à peu près la forme et la dimension voulues. C’était là une base de frappement fort dangereuse, et pouvant éclater. Une des extrémités de ce bloc, arrondie et finissant en pointe, pouvait à la rigueur tenir lieu de bicorne conoïde, mais l’autre bicorne, la bicorne pyramidale, manquait. C’était l’antique enclume de pierre des troglodytes. La surface, polie par le flot, avait presque la fermeté de l’acier.\par
Gilliatt regretta de ne point avoir apporté son enclume. Comme il ignorait que la Durande avait été coupée en deux par la tempête, il avait espéré trouver la baille du charpentier et tout son outillage ordinairement logé dans la cale à l’avant. Or, c’était précisément l’avant qui avait été emporté.\par
Les deux excavations, conquises sur l’écueil par Gilliatt, étaient voisines. Le magasin et la forge communiquaient.\par
 Tous les soirs, sa journée finie, Gilliatt soupait d’un morceau de biscuit amolli dans l’eau, d’un oursin ou d’un poingclos, ou de quelques châtaignes de mer, la seule chasse possible dans ces rochers, et, grelottant comme la corde à nœuds, remontait se coucher dans son trou sur la grande Douvre.\par
L’espèce d’abstraction où vivait Gilliatt s’augmentait de la matérialité même de ses occupations. La réalité à haute dose effare. Le labeur corporel avec ses détails sans nombre n’ôtait rien à la stupeur de se trouver là et de faire ce qu’il faisait. Ordinairement la lassitude matérielle est un fil qui tire à terre ; mais la singularité même de la besogne entreprise par Gilliatt le maintenait dans une sorte de région idéale et crépusculaire. Il lui semblait par moments donner des coups de marteau dans les nuages. Dans d’autres instants, il lui semblait que ses outils étaient des armes. Il avait le sentiment singulier d’une attaque latente qu’il réprimait ou qu’il prévenait. Tresser du funin, tirer d’une voile un fil de caret, arc-bouter deux madriers, c’était façonner des machines de guerre. Les mille soins minutieux de ce sauvetage finissaient par ressembler à des précautions contre des agressions intelligentes, fort peu dissimulées et très transparentes. Gilliatt ne savait pas les mots qui rendent les idées, mais il percevait les idées. Il se sentait de moins en moins ouvrier et de plus en plus belluaire.\par
Il était là comme dompteur. Il le comprenait presque. Élargissement étrange pour son esprit.\par
En outre, il avait autour de lui, à perte de vue,  l’immense songe du travail perdu. Voir manœuvrer dans l’insondable et dans l’illimité la diffusion des forces, rien n’est plus troublant. On cherche des buts. L’espace toujours en mouvement, l’eau infatigable, les nuages qu’on dirait affairés, le vaste effort obscur, toute cette convulsion est un problème. Qu’est-ce que ce tremblement perpétuel fait ? que construisent ces rafales ? que bâtissent ces secousses ? Ces chocs, ces sanglots, ces hurlements, qu’est-ce qu’ils créent ? à quoi est occupé ce tumulte ? Le flux et le reflux de ces questions est éternel comme la marée. Gilliatt, lui, savait ce qu’il faisait ; mais l’agitation de l’étendue l’obsédait confusément de son énigme. A son insu, mécaniquement, impérieusement, par pression et pénétration, sans autre résultat qu’un éblouissement inconscient et presque farouche, Gilliatt rêveur amalgamait à son propre travail le prodigieux travail inutile de la mer. Comment, en effet, ne pas subir et sonder, quand on est là, le mystère de l’effrayante onde laborieuse ? Comment ne pas méditer, dans la mesure de ce qu’on a de méditation possible, la vacillation du flot, l’acharnement de l’écume, l’usure imperceptible du rocher, l’époumonnement insensé des quatre vents ? Quelle terreur pour la pensée, le recommencement perpétuel, l’océan puits, les nuées Danaïdes, toute cette peine pour rien !\par
Pour rien, non. Mais, ô Inconnu, toi seul sais pourquoi.
 \subsubsection[{B.I.11. Découverte}]{B.I.11. \\
Découverte}
\noindent Un écueil voisin de la côte est quelquefois visité par les hommes ; un écueil en pleine mer, jamais. Qu’irait-on y chercher ? Ce n’est pas une île. Point de ravitaillement à espérer, ni arbres à fruits, ni pâturages, ni bestiaux, ni sources d’eau potable. C’est une nudité dans une solitude. C’est une roche, avec des escarpements hors de l’eau et des pointes sous l’eau. Rien à trouver là que le naufrage.\par
Ces espèces d’écueils, que la vieille langue marine appelle les Isolés, sont, nous l’avons dit, des lieux étranges. La mer y est seule. Elle fait ce qu’elle veut. Nulle apparition terrestre ne l’inquiète. L’homme épouvante la mer ; elle se défie de lui ; elle lui cache ce qu’elle est et ce qu’elle fait. Dans l’écueil, elle est rassurée ; l’homme n’y viendra pas. Le monologue des flots ne sera point troublé. Elle travaille à l’écueil, répare ses avaries, aiguise ses pointes, le hérisse, le remet à neuf, le maintient en état. Elle entreprend le  percement du rocher, désagrége la pierre tendre, dénude la pierre dure, ôte la chair, laisse l’ossement, fouille, dissèque, fore, troue, canalise, met les cœcums en communication, emplit l’écueil de cellules, imite l’éponge en grand, creuse le dedans, sculpte le dehors. Elle se fait, dans cette montagne secrète qui est à elle, des antres, des sanctuaires, des palais ; elle a on ne sait quelle végétation hideuse et splendide composée d’herbes flottantes qui mordent et de monstres qui prennent racine ; elle enfouit sous l’ombre de l’eau cette magnificence affreuse. Dans l’écueil isolé, rien ne la surveille, ne l’espionne et ne la dérange ; elle y développe à l’aise son côté mystérieux, inaccessible à l’homme. Elle y dépose ses sécrétions vivantes et horribles. Tout l’ignoré de la mer est là.\par
Les promontoires, les caps, les finistères, les nases, les brisants, les récifs, sont, insistons-y, de vraies constructions. La formation géologique est peu de chose, comparée à la formation océanique. Les écueils, ces maisons de la vague, ces pyramides et ces syringes de l’écume, appartiennent à un art mystérieux que l’auteur de ce livre a nommé quelque part l’Art de la Nature, et ont une sorte de style énorme. Le fortuit y semble voulu. Ces constructions sont multiformes. Elles ont l’enchevêtrement du polypier, la sublimité de la cathédrale, l’extravagance de la pagode, l’amplitude du mont, la délicatesse du bijou, l’horreur du sépulcre. Elles ont des alvéoles comme un guêpier, des tanières comme une ménagerie, des tunnels comme une taupinière, des cachots comme une bastille,  des embuscades comme un camp. Elles ont des portes, mais barricadées, des colonnes, mais tronquées, des tours, mais penchées, des ponts, mais rompus. Leurs compartiments sont inexorables ; ceci n’est que pour les oiseaux ; ceci n’est que pour les poissons. On ne passe pas. Leur figure architecturale se transforme, se déconcerte, affirme la statique, la nie, se brise, s’arrête court, commence en archivolte, finit en architrave ; bloc sur bloc ; Encelade est le maçon. Une dynamique extraordinaire étale là ses problèmes, résolus. D’effrayants pendentifs menacent, mais ne tombent pas. On ne sait comment tiennent ces bâtisses vertigineuses. Partout des surplombs, des porte-à-faux, des lacunes, des suspensions insensées ; la loi de ce babélisme échappe ; l’Inconnu, immense architecte, ne calcule rien, et réussit tout ; les rochers, bâtis pêle-mêle, composent un monument monstre ; nulle logique, un vaste équilibre. C’est plus que de la solidité, c’est de l’éternité. En même temps, c’est le désordre. Le tumulte de la vague semble avoir passé dans le granit. Un écueil, c’est de la tempête pétrifiée. Rien de plus émouvant pour l’esprit que cette farouche architecture, toujours croulante, toujours debout. Tout s’y entr’aide et s’y contrarie. C’est un combat de lignes d’où résulte un édifice. On y reconnaît la collaboration de ces deux querelles, l’océan et l’ouragan.\par
Cette architecture a ses chefs-d’œuvre, terribles. L’écueil Douvres en était un.\par
Celui-là, la mer l’avait construit et perfectionné avec un amour formidable. L’eau hargneuse le léchait.  Il était hideux, traître, obscur ; plein de caves.\par
Il avait tout un système veineux de trous sous-marins se ramifiant dans des profondeurs insondables. Plusieurs des orifices de ce percement inextricable étaient à sec aux marées basses. On y pouvait entrer. A ses risques et périls.\par
Gilliatt, pour les besoins de son sauvetage, dut explorer toutes ces grottes. Pas une qui ne fût effroyable. Partout, dans ces caves, se reproduisait, avec les dimensions exagérées de l’océan, cet aspect d’abattoir et de boucherie étrangement empreint dans l’entre-deux des Douvres. Qui n’a point vu, dans des excavations de ce genre, sur la muraille du granit éternel, ces affreuses fresques de la nature, ne peut s’en faire l’idée.\par
Ces grottes féroces étaient sournoises ; il ne fallait point s’y attarder. La marée haute les emplissait jusqu’au plafond.\par
Les poux de roque et les fruits de mer y abondaient.\par
Elles étaient encombrées de galets roulés, amoncelés en tas au fond des voûtes. Beaucoup de ces galets pesaient plus d’une tonne. Ils étaient de toutes proportions et de toutes couleurs ; la plupart paraissaient sanglants ; quelques-uns, couverts de conferves poilues et gluantes, semblaient de grosses taupes vertes fouillant le rocher.\par
Plusieurs de ces caves se terminaient brusquement en cul-de-four. D’autres, artères d’une circulation mystérieuse, se prolongeaient dans le rocher en fissures  tortueuses et noires. C’étaient les rues du gouffre. Ces fissures se rétrécissant sans cesse, un homme n’y pouvait passer. Un brandon allumé y laissait voir des obscurités suintantes.\par
Une fois, Gilliatt, furetant, s’aventura dans une de ces fissures. L’heure de la marée s’y prêtait. C’était une belle journée de calme et de soleil. Aucun incident de mer, pouvant compliquer le risque, n’était à redouter.\par
Deux nécessités, nous venons de l’indiquer, poussaient Gilliatt à ces explorations ; chercher, pour le sauvetage, des débris utiles, et trouver des crabes et des langoustes pour sa nourriture. Les coquillages commençaient à lui manquer dans les Douvres.\par
La fissure était resserrée et le passage presque impossible. Gilliatt voyait de la clarté au delà. Il fit effort, s’effaça, se tordit de son mieux, et s’engagea le plus avant qu’il put.\par
Il se trouvait, sans s’en douter, précisément dans l’intérieur du rocher sur la pointe duquel Clubin avait lancé la Durande. Gilliatt était sous cette pointe. Le rocher, abrupt extérieurement, et inabordable, était évidé en dedans. Il avait des galeries, des puits et des chambres comme le tombeau d’un roi d’Égypte. Cet affouillement était un des plus compliqués parmi ces dédales, travail de l’eau, sape de la mer infatigable. Les embranchements de ce souterrain sous mer communiquaient probablement avec l’eau immense du dehors par plus d’une issue, les unes béantes au niveau du flot, les autres, profonds entonnoirs invisibles.  C’était tout près de là, mais Gilliatt l’ignorait, que Clubin s’était jeté à la mer.\par
Gilliatt, dans cette lézarde à crocodiles, où les crocodiles, il est vrai, n’étaient pas à craindre, serpentait, rampait, se heurtait le front, se courbait, se redressait, perdait pied, retrouvait le sol, avançait péniblement. Peu à peu le boyau s’élargit, un demi-jour parut, et tout à coup Gilliatt fit son entrée dans une caverne extraordinaire.
 \subsubsection[{B.I.12. Le dedans d’un édifice sous mer}]{B.I.12. \\
Le dedans d’un édifice sous mer}
\noindent Ce demi-jour vint à propos.\par
Un pas de plus, Gilliatt tombait dans une eau peut-être sans fond. Ces eaux de caves ont un tel refroidissement et une paralysie si subite, que souvent les plus forts nageurs y restent.\par
Nul moyen d’ailleurs de remonter et de s’accrocher aux escarpements entre lesquels on est muré.\par
Gilliatt s’arrêta court. La crevasse d’où il sortait aboutissait à une saillie étroite et visqueuse, espèce d’encorbellement dans la muraille à pic. Gilliatt s’adossa à la muraille et regarda.\par
Il était dans une grande cave. Il avait au-dessus de lui quelque chose comme le dessous d’un crâne démesuré. Ce crâne avait l’air fraîchement disséqué. Les nervures ruisselantes des stries du rocher imitaient sur la voûte les embranchements des fibres et les sutures dentelées d’une boîte osseuse. Pour plafond, la  pierre ; pour plancher, l’eau ; les lames de la marée, resserrées entre les quatre parois de la grotte, semblaient de larges dalles tremblantes. La grotte était fermée de toutes parts. Pas une lucarne, pas un soupirail ; aucune brèche à la muraille, aucune fêlure à la voûte. Tout cela était éclairé d’en bas à travers l’eau. C’était on ne sait quel resplendissement ténébreux.\par
Gilliatt, dont les pupilles s’étaient dilatées pendant le trajet obscur du corridor, distinguait tout dans ce crépuscule.\par
Il connaissait, pour y être allé plus d’une fois, les caves de Plémont à Jersey, le Creux-Maillé à Guernesey, les Boutiques à Serk, ainsi nommées à cause des contrebandiers qui y déposaient leurs marchandises ; aucun de ces merveilleux antres n’était comparable à la chambre souterraine et sous-marine où il venait de pénétrer.\par
Gilliatt voyait en face de lui sous la vague une sorte d’arche noyée. Cette arche, ogive naturelle façonnée par le flot, était éclatante entre ses deux jambages profonds et noirs. C’est par ce porche submergé qu’entrait dans la caverne la clarté de la haute mer. Jour étrange donné par un engloutissement.\par
Cette clarté s’évasait sous la lame comme un large éventail et se répercutait sur le rocher. Ses rayonnements rectilignes, découpés en longues bandes droites sur l’opacité du fond, s’éclaircissant ou s’assombrissant d’une anfractuosité à l’autre, imitaient des interpositions de lames de verre. Il y avait du jour dans cette cave, mais du jour inconnu. Il n’y avait plus  dans cette clarté rien de notre lumière. On pouvait croire qu’on venait d’enjamber dans une autre planète. La lumière était une énigme ; on eût dit la lueur glauque de la prunelle d’un sphinx. Cette cave figurait le dedans d’une tête de mort énorme et splendide ; la voûte était le crâne, et l’arche était la bouche ; les trous des yeux manquaient. Cette bouche, avalant et rendant le flux et le reflux, béante au plein midi extérieur, buvait de la lumière et vomissait de l’amertume. De certains êtres, intelligents et mauvais, ressemblent à cela. Le rayon du soleil, en traversant ce porche obstrué d’une épaisseur vitreuse d’eau de mer, devenait vert comme un rayon d’Aldébaran. L’eau, toute pleine de cette lumière mouillée, paraissait de l’émeraude en fusion. Une nuance d’aigue-marine d’une délicatesse inouïe teignait mollement toute la caverne. La voûte, avec ses lobes presque cérébraux et ses ramifications rampantes pareilles à des épanouissements de nerfs, avait un tendre reflet de chrysoprase. Les moires du flot, réverbérées au plafond, s’y décomposaient et s’y recomposaient sans fin, élargissant et rétrécissant leurs mailles d’or avec un mouvement de danse mystérieuse. Une impression spectrale s’en dégageait ; l’esprit pouvait se demander quelle proie ou quelle attente faisait si joyeux ce magnifique filet de feu vivant. Aux reliefs de la voûte et aux aspérités du roc pendaient de longues et fines végétations baignant probablement leurs racines à travers le granit dans quelque nappe d’eau supérieure, et égrenant, l’une après l’autre, à leur extrémité, une goutte d’eau, une perle. Ces perles  tombaient dans le gouffre avec un petit bruit doux. Le saisissement de cet ensemble était indicible. On ne pouvait rien imaginer de plus charmant ni rien rencontrer de plus lugubre.\par
C’était on ne sait quel palais de la Mort, contente.
 \subsubsection[{B.I.13. Ce qu’on y voit et ce qu’on y entrevoit}]{B.I.13. \\
Ce qu’on y voit et ce qu’on y entrevoit}
\noindent De l’ombre qui éblouit, tel était ce lieu surprenant.\par
La palpitation de la mer se faisait sentir dans cette cave. L’oscillation extérieure gonflait, puis déprimait la nappe d’eau intérieure avec la régularité d’une respiration. On croyait deviner une âme mystérieuse dans ce grand diaphragme vert s’élevant et s’abaissant en silence.\par
L’eau était magiquement limpide, et Gilliatt y distinguait, à des profondeurs diverses, des stations immergées, surfaces de roches en saillie d’un vert de plus en plus foncé. Certains creux obscurs étaient probablement insondables.\par
Des deux côtés du porche sous-marin, des ébauches de cintres surbaissés, pleins de ténèbres, indiquaient de petites caves latérales, bas côtés de la caverne centrale, accessibles peut-être à l’époque des très basses marées.\par
Ces anfractuosités avaient des plafonds en plan  incliné, à angles plus ou moins ouverts. De petites plages, larges de quelques pieds, mises à nu par les fouilles de la mer, s’enfonçaient et se perdaient sous ces obliquités.\par
Çà et là des herbes longues de plus d’une toise ondulaient sous l’eau avec un balancement de cheveux au vent. On entrevoyait des forêts de goémons.\par
Hors du flot et dans le flot, toute la muraille de la cave, du haut en bas, depuis la voûte jusqu’à son effacement dans l’invisible, était tapissée de ces prodigieuses floraisons de l’océan, si rarement aperçues par l’œil humain, que les vieux navigateurs espagnols nommaient \emph{praderias del mar.} Une mousse robuste, qui avait toutes les nuances de l’olive, cachait et amplifiait les exostoses du granit. De tous les surplombs jaillissaient les minces lanières gaufrées du varech dont les pêcheurs se font des baromètres. Le souffle obscur de la caverne agitait ces courroies luisantes.\par
Sous ces végétations se dérobaient et se montraient en même temps les plus rares bijoux de l’écrin de l’océan, des éburnes, des strombes, des mitres, des casques, des pourpres, des buccins, des struthiolaires, des cérites turriculées. Les cloches des patelles, pareilles à des huttes microscopiques, adhéraient partout au rocher et se groupaient en villages, dans les rues desquels rôdaient les oscabrions, ces scarabées de la vague. Les galets ne pouvant que difficilement entrer dans cette grotte, les coquillages s’y  réfugiaient. Les coquillages sont des grands seigneurs, qui, tout brodés et tout passementés, évitent le rude et incivil contact de la populace des cailloux. L’amoncellement étincelant des coquillages faisait sous la lame, à de certains endroits, d’ineffables irradiations à travers lesquelles on entrevoyait un fouillis d’azurs et de nacres, et des ors de toutes les nuances de l’eau.\par
Sur la paroi de la cave, un peu au-dessus de la ligne de flottaison de la marée, une plante magnifique et singulière se rattachait comme une bordure à la tenture de varech, la continuait et l’achevait. Cette plante, fibreuse, touffue, inextricablement coudée et presque noire, offrait au regard de larges nappes brouillées et obscures, partout piquées d’innombrables petites fleurs couleur lapis-lazuli. Dans l’eau ces fleurs semblaient s’allumer, et l’on croyait voir des braises bleues. Hors de l’eau c’étaient des fleurs, sous l’eau c’étaient des saphirs ; de sorte que la lame, en montant et en inondant le soubassement de la grotte revêtu de ces plantes, couvrait le rocher d’escarboucles.\par
A chaque gonflement de la vague enflée comme un poumon, ces fleurs, baignées, resplendissaient ; à chaque abaissement elles s’éteignaient ; mélancolique ressemblance avec la destinée. C’était l’aspiration, qui est la vie ; puis l’expiration, qui est la mort.\par
Une des merveilles de cette caverne, c’était le roc. Ce roc, tantôt muraille, tantôt cintre, tantôt étrave ou pilastre, était par places brut et nu, puis, tout à côté, travaillé des plus délicates ciselures naturelles. On ne sait quoi, qui avait beaucoup d’esprit, se mêlait à la  stupidité massive du granit. Quel artiste que l’abîme ! Tel pan de mur, coupé carrément et couvert de rondes bosses ayant des attitudes, figurait un vague bas-relief ; on pouvait, devant cette sculpture où il y avait du nuage, rêver de Prométhée ébauchant pour Michel-Ange. Il semblait qu’avec quelques coups de marteau le génie eût pu achever ce qu’avait commencé le géant. En d’autres endroits la roche était damasquinée comme un bouclier sarrasin ou niellée comme une vasque florentine. Elle avait des panneaux qui paraissaient de bronze de Corinthe, puis des arabesques comme une porte de mosquée, puis, comme une pierre runique, des empreintes d’ongle obscures et improbables. Des plantes à ramuscules torses et à vrilles, s’entre-croisant sur les dorures du lichen, la couvraient de filigranes. Cet antre se compliquait d’un alhambra. C’était la rencontre de la sauvagerie et de l’orfévrerie dans l’auguste et difforme architecture du hasard.\par
Les magnifiques moisissures de la mer mettaient du velours sur les angles du granit. Les escarpements étaient festonnés de lianes grandiflores, adroites à ne point tomber, et qui semblaient intelligentes, tant elles ornaient bien. Des pariétaires à bouquets bizarres montraient leurs touffes à propos et avec goût. Toute la coquetterie possible à une caverne était là. La surprenante lumière édénique qui venait de dessous l’eau, à la fois pénombre marine et rayonnement paradisiaque, estompait tous les linéaments dans une sorte de diffusion visionnaire. Chaque vague était un prisme. Les contours des choses, sous ces ondoiements irisés,  avaient le chromatisme des lentilles d’optique trop convexes ; des spectres solaires flottaient sous l’eau. On croyait voir se tordre dans cette diaphanéité aurorale des tronçons d’arcs-en-ciel noyés. Ailleurs, en d’autres coins, il y avait dans l’eau un certain clair de lune. Toutes les splendeurs semblaient amalgamées là pour faire on ne sait quoi d’aveugle et de nocturne. Rien de plus troublant et de plus énigmatique que ce faste dans cette cave. Ce qui dominait, c’était l’enchantement. La végétation fantasque et la stratification informe s’accordaient et dégageaient une harmonie. Ce mariage de choses farouches était heureux. Les ramifications se cramponnaient en ayant l’air d’effleurer. La caresse du roc sauvage et de la fleur fauve était profonde. Des piliers massifs avaient pour chapiteaux et pour ligatures de frêles guirlandes toutes pénétrées de frémissement, on songeait à des doigts de fées chatouillant des pieds de béhémoths, et le rocher soutenait la plante et la plante étreignait le rocher avec une grâce monstrueuse.\par
La résultante de ces difformités mystérieusement ajustées était on ne sait quelle beauté souveraine. Les œuvres de la nature, non moins suprêmes que les œuvres du génie, contiennent de l’absolu, et s’imposent. Leur inattendu se fait obéir impérieusement par l’esprit ; on y sent une préméditation qui est en dehors de l’homme, et elles ne sont jamais plus saisissantes que lorsqu’elles font subitement sortir l’exquis du terrible.\par
Cette grotte inconnue était, pour ainsi dire, et si  une telle expression est admissible, sidéralisée. On y subissait ce que la stupeur a de plus imprévu. Ce qui emplissait cette crypte, c’était de la lumière d’apocalypse. On n’était pas bien sûr que cette chose fût. On avait devant les yeux une réalité empreinte d’impossible. On regardait cela, on y touchait, on y était ; seulement il était difficile d’y croire.\par
Était-ce du jour qui venait par cette fenêtre sous la mer ? Était-ce de l’eau qui tremblait dans cette cuve obscure ? Ces cintres et ces porches n’étaient-ils point de la nuée céleste imitant une caverne ? Quelle pierre avait-on sous les pieds ? Ce support n’allait-il point se désagréger et devenir fumée ? Qu’était-ce que cette joaillerie de coquillages qu’on entrevoyait ? A quelle distance était-on de la vie, de la terre, des hommes ? Qu’était-ce que ce ravissement mêlé à ces ténèbres ? Émotion inouïe, presque sacrée, à laquelle s’ajoutait la douce inquiétude des herbes au fond de l’eau.\par
A l’extrémité de la cave, qui était oblongue, sous une archivolte cyclopéenne d’une coupe singulièrement correcte, dans un creux presque indistinct, espèce d’antre dans l’antre et de tabernacle dans le sanctuaire, derrière une nappe de clarté verte interposée comme un voile de temple, on apercevait hors du flot une pierre à pans carrés ayant une ressemblance d’autel. L’eau entourait cette pierre de toutes parts. Il semblait qu’une déesse vînt d’en descendre. On ne pouvait s’empêcher de rêver sous cette crypte, sur cet autel, quelque nudité céleste éternellement pensive, et que l’entrée d’un homme faisait éclipser. Il était difficile de  concevoir cette cellule auguste sans une vision dedans ; l’apparition, évoquée par la rêverie, se recomposait d’elle-même ; un ruissellement de lumière chaste sur des épaules à peine entrevues, un front baigné d’aube, un ovale de visage olympien, des rondeurs de seins mystérieux, des bras pudiques, une chevelure dénouée dans de l’aurore, des hanches ineffables modelées en pâleur dans une brume sacrée, des formes de nymphe, un regard de vierge, une Vénus sortant de la mer, une Ève sortant du chaos ; tel était le songe qu’il était impossible de ne pas faire. Il était invraisemblable qu’il n’y eût point là un fantôme. Une femme toute nue, ayant en elle un astre, était probablement sur cet autel tout à l’heure. Sur ce piédestal d’où émanait une indicible extase, on imaginait une blancheur, vivante et debout. L’esprit se représentait, au milieu de l’adoration muette de cette caverne, une Amphitrite, une Téthys, quelque Diane pouvant aimer, statue de l’idéal formée d’un rayonnement et regardant l’ombre avec douceur. C’était elle qui, en s’en allant, avait laissé dans la caverne cette clarté, espèce de parfum lumière sorti de ce corps étoile. L’éblouissement de ce fantôme n’était plus là ; on n’apercevait pas cette figure, faite pour être vue seulement par l’invisible, mais on la sentait ; on avait ce tremblement, qui est une volupté. La déesse était absente, mais la divinité était présente.\par
La beauté de l’antre semblait faite pour cette présence. C’était à cause de cette déité, de cette fée des nacres, de cette reine des souffles, de cette grâce née  des flots, c’était à cause d’elle, on se le figurait du moins, que le souterrain était religieusement muré, afin que rien ne pût jamais troubler, autour de ce divin fantôme, l’obscurité qui est un respect, et le silence qui est une majesté.\par
Gilliatt, qui était une espèce de voyant de la nature, songeait, confusément ému.\par
Tout à coup, à quelques pieds au-dessous de lui, dans la transparence charmante de cette eau, qui était comme de la pierrerie dissoute, il aperçut quelque chose d’inexprimable. Une espèce de long haillon se mouvait dans l’oscillation des lames. Ce haillon ne flottait pas, il voguait ; il avait un but, il allait quelque part, il était rapide. Cette guenille avait la forme d’une marotte de bouffon avec des pointes ; ces pointes, flasques, ondoyaient ; elle semblait couverte d’une poussière impossible à mouiller. C’était plus qu’horrible, c’était sale. Il y avait de la chimère dans cette chose ; c’était un être, à moins que ce ne fût une apparence. Elle semblait se diriger vers le côté obscur de la cave et s’y enfonçait. Les épaisseurs d’eau devinrent sombres sur elle. Cette silhouette glissa et disparut, sinistre.\par
  \subsection[{B.II. Livre deuxième. Le labeur}]{B.II. Livre deuxième \\
Le labeur}
  \subsubsection[{B.II.1. Les ressources de celui a qui tout manque}]{B.II.1. \\
Les ressources de celui a qui tout manque}
\noindent Cette cave ne lâchait pas aisément les gens. L’entrée avait été peu commode, la sortie fut plus obstruée encore. Gilliatt néanmoins s’en tira, mais il n’y retourna plus. Il n’y avait rien trouvé de ce qu’il cherchait, et il n’avait pas le temps d’être curieux.\par
Il mit immédiatement la forge en activité. Il manquait d’outils, il s’en fabriqua.\par
Il avait pour combustible l’épave, l’eau pour moteur, le vent pour souffleur, une pierre pour enclume, pour art son instinct, pour puissance sa volonté.\par
Gilliatt entra ardemment dans ce sombre travail.\par
Le temps paraissait y mettre de la complaisance. Il continuait d’être sec et aussi peu équinoxial que possible. Le mois de mars était venu, mais tranquillement. Les jours s’allongeaient. Le bleu du ciel, la vaste douceur des mouvements de l’étendue, la sérénité du plein  midi, semblaient exclure toute mauvaise intention. La mer était gaie au soleil. Une caresse préalable assaisonne les trahisons. De ces caresses-là, la mer n’en est point avare. Quand on a affaire à cette femme, il faut se défier du sourire.\par
Il y avait peu de vent ; la soufflante hydraulique n’en travaillait que mieux. L’excès de vent eût plutôt gêné qu’aidé.\par
Gilliatt avait une scie ; il se fabriqua une lime ; avec la scie il attaqua le bois, avec la lime il attaqua le métal ; puis il s’ajouta les deux mains de fer du forgeron, une tenaille et une pince ; la tenaille étreint, la pince manie ; l’une agit comme le poignet, l’autre comme le doigt. L’outillage est un organisme. Peu à peu Gilliatt se donnait des auxiliaires, et construisait son armure. D’un morceau de feuillard il fit un auvent au foyer de sa forge.\par
Un de ses principaux soins fut le triage et la réparation des poulies. Il remit en état les caisses et les rouets des moufles. Il coupa l’exfoliation de toutes les solives brisées, et en refaçonna les extrémités ; il avait, nous l’avons dit, pour les nécessités de sa charpenterie, quantité de membrures emmagasinées et appareillées selon les formes, les dimensions et les essences, le chêne d’un côté, le sapin de l’autre, les pièces courbes, comme les porques, séparées des pièces droites, comme les hiloires. C’était sa réserve de points d’appui et de leviers, dont il pouvait avoir grand besoin à un moment donné.\par
Quiconque médite un palan doit se pourvoir de  poutres et de moufles ; mais cela ne suffit pas, il faut de la corde. Gilliatt restaura les câbles et les grelins. Il étira les voiles déchirées, et réussit à en extraire d’excellent fil de caret dont il composa du filin ; avec ce filin, il rabouta les cordages. Seulement ces sutures étaient sujettes à pourrir, il fallait se hâter d’employer ces cordes et ces câbles, Gilliatt n’avait pu faire que du funin blanc, il manquait de goudron.\par
Les cordages raccommodés, il raccommoda les chaînes.\par
Il put, grâce à la pointe latérale du galet enclume, laquelle tenait lieu de bicorne conique, forger des anneaux grossiers, mais solides. Avec ces anneaux il rattacha les uns aux autres les bouts de chaîne cassés, et fit des longueurs.\par
Forger seul et sans aide est plus que malaisé. Il en vint à bout pourtant. Il est vrai qu’il n’eut à façonner sur la forge que des pièces de peu de masse ; il pouvait les manier d’une main avec la pince pendant qu’il les martelait de l’autre main.\par
Il coupa en tronçons les barres de fer rondes de la passerelle de commandement, forgea aux deux extrémités de chaque tronçon, d’un côté une pointe, de l’autre une large tête plate, et cela fit de grands clous d’environ un pied de long. Ces clous, très usités en pontonnerie, sont utiles aux fixations dans les rochers.\par
Pourquoi Gilliatt se donnait-il toute cette peine ? On verra.\par
Il dut refaire plusieurs fois le tranchant de sa hache  et les dents de sa scie. Il s’était, pour la scie, fabriqué un tiers-point.\par
Il se servait dans l’occasion du cabestan de la Durande. Le crochet de la chaîne cassa. Gilliatt en reforgea un autre.\par
A l’aide de sa pince et de sa tenaille, et en se servant de son ciseau comme d’un tournevis, il entreprit de démonter les deux roues du navire ; il y parvint. On n’a pas oublié que ce démontage était exécutable ; c’était une particularité de la construction de ces roues. Les tambours qui les avaient couvertes, les emballèrent ; avec les planches des tambours, Gilliatt charpenta et menuisa deux caisses, où il déposa, pièce à pièce, les deux roues soigneusement numérotées. Son morceau de craie lui fut précieux pour ce numérotage.\par
Il rangea ces deux caisses sur la partie la plus solide du pont de la Durande.\par
Ces préliminaires terminés, Gilliatt se trouva face à face avec la difficulté suprême. La question de la machine se posa.\par
Démonter les roues avait été possible ; démonter la machine, non.\par
D’abord Gilliatt connaissait mal ce mécanisme. Il pouvait, en allant au hasard, lui faire quelque blessure irréparable. Ensuite, même pour essayer de le défaire morceau à morceau, s’il eût eu cette imprudence, il fallait d’autres outils que ceux qu’on peut fabriquer avec une caverne pour forge, un vent coulis pour soufflet, et un caillou pour enclume. En tentant de démonter la machine, on risquait de la dépecer.\par
 Ici on pouvait se croire tout à fait en présence de l’impraticable.\par
Il semblait que Gilliatt fût au pied de ce mur, l’impossible.\par
Que faire ?
 \subsubsection[{B.II.2. Comme quoi shakespeare peut se rencontrer avec eschyle}]{B.II.2. \\
Comme quoi shakespeare peut se rencontrer avec eschyle}
\noindent Gilliatt avait son idée.\par
Depuis ce maçon charpentier de Salbris qui, au seizième siècle, dans le bas âge de la science, bien avant qu’Amontons eût trouvé la première loi du frottement, Lahire la seconde et Coulomb la troisième, sans conseil, sans guide, sans autre aide qu’un enfant, son fils, avec un outillage informe, résolut en bloc, dans la descente du « gros horloge » de l’église de la Charité-sur-Loire, cinq ou six problèmes de statique et de dynamique mêlés ensemble ainsi que des roues dans un embarras de charrettes et faisant obstacle à la fois, depuis ce manœuvre extravagant et superbe qui trouva moyen, sans casser un fil de laiton et sans déchiqueter un engrenage, de faire glisser tout d’une pièce, par une simplification prodigieuse, du second étage du clocher au premier étage, cette massive cage des heures, toute en fer et en cuivre, « grande comme  la chambre du guetteur de nuit », avec son mouvement, ses cylindres, ses barillets, ses tambours, ses crochets et ses pesons, son orbe de canon et son orbe de chaussée, son balancier horizontal, ses ancres d’échappement, ses écheveaux de chaînes et de chaînettes, ses poids de pierre dont un pesait cinq cents livres, ses sonneries, ses carillons, ses jacquemarts ; depuis cet homme qui fit ce miracle, et dont on ne sait plus le nom, jamais rien de pareil à ce que méditait Gilliatt n’avait été entrepris.\par
L’opération que rêvait Gilliatt était pire peut-être, c’est-à-dire plus belle encore.\par
Le poids, la délicatesse, l’enchevêtrement des difficultés, n’étaient pas moindres de la machine de la Durande que de l’horloge de la Charité-sur-Loire.\par
Le charpentier gothique avait un aide, son fils ; Gilliatt était seul.\par
Une population était là, venue de Meung-sur-Loire, de Nevers, et même d’Orléans, pouvant, au besoin, assister le maçon de Salbris, et l’encourageant de son brouhaha bienveillant ; Gilliatt n’avait autour de lui d’autre rumeur que le vent et d’autre foule que les flots.\par
Rien n’égale la timidité de l’ignorance, si ce n’est sa témérité. Quand l’ignorance se met à oser, c’est qu’elle a en elle une boussole. Cette boussole, c’est l’intuition du vrai, plus claire parfois dans un esprit simple que dans un esprit compliqué.\par
Ignorer invite à essayer. L’ignorance est une rêverie, et la rêverie curieuse est une force. Savoir,  déconcerte parfois et déconseille souvent. Gama, savant, eût reculé devant le cap des Tempêtes. Si Christophe Colomb eût été bon cosmographe, il n’eût point découvert l’Amérique.\par
Le second qui monta sur le mont Blanc fut un savant, Saussure ; le premier fut un pâtre, Balmat.\par
Ces cas, disons-le en passant, sont l’exception, et tout ceci n’ôte rien à la science, qui reste la règle. L’ignorant peut trouver, le savant seul invente.\par
La panse était toujours à l’ancre dans la crique de l’Homme, où la mer la laissait tranquille. Gilliatt, on s’en souvient, avait tout arrangé de façon à se maintenir en libre pratique avec sa barque. Il y alla, et en mesura soigneusement le bau à plusieurs endroits, particulièrement le maître-couple. Puis il revint à la Durande, et mesura le grand diamètre du parquet de la machine. Ce grand diamètre, sans les roues, bien entendu, était de deux pieds moindre que le maîtrebau de la panse. Donc la machine pouvait entrer dans la barque.\par
Mais comment l’y faire entrer ?
 \subsubsection[{B.II.3. Le chef-d’œuvre de gilliatt vient au secours du chef-d’œuvre de lethierry}]{B.II.3. \\
Le chef-d’œuvre de gilliatt vient au secours du chef-d’œuvre de lethierry}
\noindent A quelque temps de là, un pêcheur qui eût été assez fou pour flâner en cette saison dans ces parages eût été payé de sa hardiesse par la vision entre les Douvres de quelque chose de singulier.\par
Voici ce qu’il eût aperçu : quatre madriers robustes, espacés également, allant d’une Douvre à l’autre, et comme forcés entre les rochers, ce qui est la meilleure des solidités. Du côté de la petite Douvre leurs extrémités posaient et se contrebutaient sur les reliefs du roc ; du côté de la grande Douvre, ces extrémités avaient dû être violemment enfoncées dans l’escarpement à coups de marteau par quelque puissant ouvrier debout sur la poutre même qu’il enfonçait. Ces madriers étaient un peu plus longs que l’entre-deux n’était large ; de là la ténacité de leur emboîtement ; de là aussi leur ajustement en plan incliné. Ils touchaient la grande Douvre  à angle aigu et la petite Douvre à angle obtus. Ils étaient faiblement déclives, mais inégalement, ce qui était un défaut. A ce défaut près, on les eût dit disposés pour recevoir le tablier d’un pont. A ces quatre madriers étaient attachés quatre palans garnis chacun de leur itague et de leur garant, et ayant cela de hardi et d’étrange que le moufle à deux rouets était à une extrémité du madrier et la poulie simple à l’extrémité opposée. Cet écart, trop grand pour n’être pas périlleux, était probablement exigé par les nécessités de l’opération à accomplir. Les moufles étaient forts et les poulies solides. A ces palans se rattachaient des câbles qui de loin paraissaient des fils, et, au-dessous de cet appareil aérien de moufles et de charpentes, la massive épave, la Durande, semblait suspendue à ces fils.\par
Suspendue, elle ne l’était pas encore. Perpendiculairement sous les madriers, huit ouvertures étaient pratiquées dans le pont, quatre à bâbord et quatre à tribord de la machine, et huit autres sous celles-là, dans la carène. Les câbles descendant verticalement des quatre moufles, entraient dans le pont, puis sortaient de la carène par les ouvertures de tribord, passaient sous la quille et sous la machine, rentraient dans le navire par les ouvertures de bâbord, et, remontant, traversant de nouveau le pont, revenaient s’enrouler aux quatre poulies des madriers, où une sorte de palanguin les saisissait et en faisait un trousseau relié à un câble unique et pouvant être dirigé par un seul bras. Un crochet et une moque, par le trou de laquelle passait et se dévidait le câble  unique, complétaient l’appareil et, au besoin, l’enrayaient. Cette combinaison contraignait les quatre palans à travailler ensemble, et, véritable frein des forces pendantes, gouvernail de dynamique sous la main du pilote de l’opération, maintenait la manœuvre en équilibre. L’ajustement très ingénieux de ce palanguin avait quelques-unes des qualités simplifiantes de la poulie Weston d’aujourd’hui, et de l’antique polyspaston de Vitruve. Gilliatt avait trouvé cela, bien qu’il ne connût ni Vitruve, qui n’existait plus, ni Weston qui n’existait pas encore. La longueur des câbles variait selon l’inégale déclivité des madriers, et corrigeait un peu cette inégalité. Les cordes étaient dangereuses, le funin blanc pouvait casser, il eût mieux valu des chaînes, mais des chaînes eussent mal roulé sur les palans.\par
Tout cela, plein de fautes, mais fait par un seul homme, était surprenant.\par
Du reste, nous abrégeons l’explication. On comprendra que nous omettions beaucoup de détails qui rendraient la chose claire aux gens du métier et obscure aux autres.\par
Le haut de la cheminée de la machine passait entre les deux madriers du milieu.\par
Gilliatt, sans s’en douter, plagiaire inconscient de l’inconnu, avait refait, à trois siècles de distance, le mécanisme du charpentier de Salbris, mécanisme rudimentaire et incorrect, redoutable à qui oserait le manœuvrer.\par
Disons ici que les fautes même les plus grossières  n’empêchent point un mécanisme de fonctionner tant bien que mal. Cela boite, mais cela marche. L’obélisque de la place de Saint-Pierre de Rome a été dressé contre toutes les règles de la statique. Le carrosse du czar Pierre était construit de telle sorte qu’il semblait devoir verser à chaque pas ; il roulait pourtant. Que de difformités dans la machine de Marly ! Tout y était en porte-à-faux. Elle n’en donnait pas moins à boire à Louis XIV.\par
Quoi qu’il en fût, Gilliatt avait confiance. Il avait même empiété sur le succès au point de fixer dans le bord de la panse, le jour où il y était allé, deux paires d’anneaux de fer en regard, des deux côtés de la barque, aux mêmes espacements que les quatre anneaux de la Durande auxquels se rattachaient les quatre chaînes de la cheminée.\par
Gilliatt avait évidemment un plan très complet et très arrêté. Ayant contre lui toutes les chances, il voulait mettre toutes les précautions de son côté.\par
Il faisait des choses qui semblaient inutiles, signe d’une préméditation attentive.\par
Sa manière de procéder eût dérouté, nous avons déjà fait cette remarque, un observateur, même connaisseur.\par
Un témoin de ses travaux qui l’eût vu, par exemple, avec des efforts inouïs et au péril de se rompre le cou, enfoncer à coups de marteau huit ou dix des grands clous qu’il avait forgés, dans le soubassement des deux Douvres à l’entrée du défilé de l’écueil, eût compris difficilement le pourquoi de ces clous, et se  fût probablement demandé à quoi bon toute cette peine.\par
S’il eût vu ensuite Gilliatt mesurer le morceau de la muraille de l’avant qui était, on s’en souvient, resté adhérent à l’épave, puis attacher un fort grelin au rebord supérieur de cette pièce, couper à coups de hache les charpentes disloquées qui la retenaient, la traîner hors du défilé, à l’aide de la marée descendante poussant le bas pendant que Gilliatt tirait le haut, enfin rattacher à grand’peine avec le grelin cette pesante plaque de planches et de poutres, plus large que l’entrée même du défilé, aux clous enfoncés dans la base de la petite Douvre, l’observateur eût peut-être moins compris encore, et se fût dit que si Gilliatt voulait, pour l’aisance de ses manœuvres, dégager la ruelle des Douvres de cet encombrement, il n’avait qu’à le laisser tomber dans la marée qui l’eût emporté à vau-l’eau.\par
Gilliatt probablement avait ses raisons.\par
Gilliatt, pour fixer les clous dans le soubassement des Douvres, tirait parti de toutes les fentes du granit, les élargissait au besoin, et y enfonçait d’abord des coins de bois dans lesquels il enracinait ensuite les clous de fer. Il ébaucha la même préparation dans les deux roches qui se dressaient à l’autre extrémité du détroit de l’écueil, du côté de l’est ; il en garnit de chevilles de bois toutes les lézardes, comme s’il voulait tenir ces lézardes prêtes à recevoir, elles aussi, des crampons ; mais cela parut être un simple en-cas, car il n’y enfonça point de clous. On comprend que, par  prudence dans sa pénurie, il ne pouvait dépenser ses matériaux qu’au fur et à mesure des besoins, et au moment où la nécessité se déclarait. C’était une complication ajoutée à tant d’autres difficultés.\par
Un premier travail achevé, un deuxième surgissait. Gilliatt passait sans hésiter de l’un à l’autre et faisait résolûment cette enjambée de géant.
 \subsubsection[{B.II.4. Sub re}]{B.II.4. \\
Sub re}
\noindent L’homme qui faisait ces choses était devenu effrayant.\par
Gilliatt, dans ce labeur multiple, dépensait toutes ses forces à la fois ; il les renouvelait difficilement.\par
Privations d’un côté, lassitude de l’autre, il avait maigri. Ses cheveux et sa barbe avaient poussé. Il n’avait plus qu’une chemise qui ne fût pas en loques. Il était pieds nus, le vent ayant emporté un de ses souliers, et la mer l’autre. Les éclats de l’enclume rudimentaire, et fort dangereuse, dont il se servait, lui avaient fait aux mains et aux bras de petites plaies, éclaboussures du travail. Ces plaies, écorchures plutôt que blessures, étaient superficielles, mais irritées par l’air vif et par l’eau salée.\par
Il avait faim, il avait soif, il avait froid.\par
Son bidon d’eau douce était vide. Sa farine de seigle était employée ou mangée. Il n’avait plus qu’un peu de biscuit.\par
 Il le cassait avec les dents, manquant d’eau pour le détremper.\par
Peu à peu et jour à jour ses forces décroissaient.\par
Ce rocher redoutable lui soutirait la vie.\par
Boire était une question ; manger était une question ; dormir était une question.\par
Il mangeait quand il parvenait à prendre un cloporte de mer ou un crabe ; il buvait quand il voyait un oiseau de mer s’abattre sur une pointe de rocher. Il y grimpait et y trouvait un creux avec un peu d’eau douce. Il buvait après l’oiseau, quelquefois avec l’oiseau ; car les mauves et les mouettes s’étaient accoutumées à lui, et ne s’envolaient pas à son approche. Gilliatt, même dans ses plus grandes faims, ne leur faisait point de mal. Il avait, on s’en souvient, la superstition des oiseaux. Les oiseaux, de leur côté, ses cheveux étant hérissés et horribles et sa barbe longue, n’en avaient plus peur ; ce changement de figure les rassurait ; ils ne le trouvaient plus un homme et le croyaient une bête.\par
Les oiseaux et Gilliatt étaient maintenant bons amis. Ces pauvres s’entr’aidaient. Tant que Gilliatt avait eu du seigle, il leur avait émietté de petits morceaux des galettes qu’il faisait ; à cette heure, à leur tour, ils lui indiquaient les endroits où il y avait de l’eau.\par
Il mangeait les coquillages crus ; les coquillages sont, dans une certaine mesure, désaltérants. Quant aux crabes, il les faisait cuire ; n’ayant pas de marmite, il les rôtissait entre deux pierres rougies au feu, à la manière des gens sauvages des îles Féroë.\par
 Cependant un peu d’équinoxe s’était déclaré ; la pluie était venue ; mais une pluie hostile. Point d’ondées, point d’averses, mais de longues aiguilles, fines, glacées, pénétrantes, aiguës, qui perçaient les vêtements de Gilliatt jusqu’à la peau et la peau jusqu’aux os. Cette pluie donnait peu à boire et mouillait beaucoup.\par
Avare d’assistance, prodigue de misère, telle était cette pluie, indigne du ciel. Gilliatt l’eut sur lui pendant plus d’une semaine tout le jour et toute la nuit. Cette pluie était une mauvaise action d’en haut.\par
La nuit, dans son trou de rocher, il ne dormait que par l’accablement du travail. Les grands cousins de mer venaient le piquer. Il se réveillait couvert de pustules.\par
Il avait la fièvre, ce qui le soutenait ; la fièvre est un secours, qui tue. D’instinct, il mâchait du lichen ou suçait des feuilles de cochléaria sauvage, maigres pousses des fentes sèches de l’écueil. Du reste, il s’occupait peu de sa souffrance. Il n’avait pas le temps de se distraire de sa besogne à cause de lui, Gilliatt. La machine de la Durande se portait bien. Cela lui suffisait.\par
A chaque instant, pour les nécessités de son travail, il se jetait à la nage, puis reprenait pied. Il entrait dans l’eau et en sortait, comme on passe d’une chambre de son appartement dans l’autre.\par
Ses vêtements ne séchaient plus. Ils étaient pénétrés d’eau de pluie qui ne tarissait pas et d’eau de mer qui ne sèche jamais. Gilliatt vivait mouillé.\par
 Vivre mouillé est une habitude qu’on prend. Les pauvres groupes irlandais, vieillards, mères, jeunes filles presque nues, enfants, qui passent l’hiver en plein air sous l’averse et la neige blottis les uns contre les autres aux angles des maisons dans les rues de Londres, vivent et meurent mouillés.\par
Être mouillé et avoir soif ; Gilliatt endurait cette torture bizarre. Il mordait par moments la manche de sa vareuse.\par
Le feu qu’il faisait ne le réchauffait guère ; le feu en plein air n’est qu’un demi-secours ; on brûle d’un côté et l’on gèle de l’autre.\par
Gilliatt, en sueur, grelottait.\par
Tout résistait autour de Gilliatt dans une sorte de silence terrible. Il se sentait l’ennemi.\par
Les choses ont un sombre \emph{Non possumus.}\par
Leur inertie est un avertissement lugubre.\par
Une immense mauvaise volonté entourait Gilliatt. Il avait des brûlures et des frissons. Le feu le mordait, l’eau le glaçait, la soif l’enfiévrait, le vent lui déchirait ses habits, la faim lui minait l’estomac. Il subissait l’oppression d’un ensemble épuisant. L’obstacle, tranquille, vaste, ayant l’irresponsabilité apparente du fait fatal, mais plein d’on ne sait quelle unanimité farouche, convergeait de toutes parts sur Gilliatt. Gilliatt le sentait appuyé inexorablement sur lui. Nul moyen de s’y soustraire. C’était presque quelqu’un. Gilliatt avait conscience d’un rejet sombre et d’une haine faisant effort pour le diminuer. Il ne tenait qu’à lui de fuir ; mais, puisqu’il restait, il avait affaire à l’hostilité  impénétrable. Ne pouvant le mettre dehors, on le mettait dessous. On ? l’Inconnu. Cela l’étreignait, le comprimait, lui ôtait la place, lui ôtait l’haleine. Il était meurtri par l’invisible. Chaque jour la vis mystérieuse se serrait d’un cran.\par
La situation de Gilliatt en ce milieu inquiétant ressemblait à un duel louche dans lequel il y a un traître.\par
La coalition des forces obscures l’environnait. Il sentait une résolution de se débarrasser de lui. C’est ainsi que le glacier chasse le bloc erratique.\par
Presque sans avoir l’air d’y toucher, cette coalition latente le mettait en haillons, en sang, aux abois, et, pour ainsi dire, hors de combat avant le combat. Il n’en travaillait pas moins, et sans relâche ; mais, à mesure que l’ouvrage se faisait, l’ouvrier se défaisait. On eût dit que cette fauve nature, redoutant l’âme, prenait le parti d’exténuer l’homme. Gilliatt tenait tête, et attendait. L’abîme commençait par l’user. Que ferait l’abîme ensuite ?\par
La double Douvre, ce dragon fait de granit et embusqué en pleine mer, avait admis Gilliatt. Elle l’avait laissé entrer et laissé faire. Cette acceptation ressemblait à l’hospitalité d’une gueule ouverte.\par
Le désert, l’étendue, l’espace où il y a pour l’homme tant de refus, l’inclémence muette des phénomènes suivant leurs cours, la grande loi générale implacable et passive, les flux et reflux, l’écueil, pléiade noire dont chaque pointe est une étoile à tourbillons, centre d’une irradiation de courants, on ne sait quel complot  de l’indifférence des choses contre la témérité d’un être, l’hiver, les nuées, la mer assiégeante, enveloppaient Gilliatt, le cernaient lentement, se fermaient en quelque sorte sur lui, et le séparaient des vivants comme un cachot qui monterait autour d’un homme. Tout contre lui, rien pour lui ; il était isolé, abandonné, affaibli, miné, oublié. Gilliatt avait sa cambuse vide, son outillage ébréché ou défaillant, la soif et la faim le jour, le froid la nuit, des plaies et des loques, des guenilles sur des suppurations, des trous aux habits et à la chair, les mains déchirées, les pieds saignants, les membres maigres, le visage livide, une flamme dans les yeux.\par
Flamme superbe, la volonté visible. L’œil de l’homme est ainsi fait qu’on y aperçoit sa vertu. Notre prunelle dit quelle quantité d’homme il y a en nous. Nous nous affirmons par la lumière qui est sous notre sourcil. Les petites consciences clignent de l’œil, les grandes jettent des éclairs. Si rien ne brille sous la paupière, c’est que rien ne pense dans le cerveau, c’est que rien n’aime dans le cœur. Celui qui aime veut, et celui qui veut éclaire et éclate. La résolution met le feu au regard ; feu admirable qui se compose de la combustion des pensées timides.\par
Les opiniâtres sont les sublimes. Qui n’est que brave n’a qu’un accès, qui n’est que vaillant n’a qu’un tempérament, qui n’est que courageux n’a qu’une vertu ; l’obstiné dans le vrai a la grandeur. Presque tout le secret des grands cœurs est dans ce mot : \emph{Perseve-rondo}. La persévérance est au courage ce que la roue  est au levier ; c’est le renouvellement perpétuel du point d’appui. Que le but soit sur la terre ou au ciel, aller au but, tout est là ; dans le premier cas, on est Colomb, dans le second cas, on est Jésus. La croix est folle ; de là sa gloire. Ne pas laisser discuter sa conscience ni désarmer sa volonté, c’est ainsi qu’on obtient la souffrance, et le triomphe. Dans l’ordre des faits moraux tomber n’exclut point planer. De la chute sort l’ascension. Les médiocres se laissent déconseiller par l’obstacle spécieux ; les forts, non. Périr est leur peut-être, conquérir est leur certitude. Vous pouvez donner à Étienne toutes sortes de bonnes raisons pour qu’il ne se fasse pas lapider. Le dédain des objections raisonnables enfante cette sublime victoire vaincue qu’on nomme le martyre.\par
Tous les efforts de Gilliatt semblaient cramponnés à l’impossible, la réussite était chétive ou lente, et il fallait dépenser beaucoup pour obtenir peu ; c’est là ce qui le faisait magnanime, c’est là ce qui le faisait pathétique.\par
Que, pour échafauder quatre poutres au-dessus d’un navire échoué, pour découper et isoler dans ce navire la partie sauvetable, pour ajuster à cette épave dans l’épave quatre palans avec leurs câbles, il eût fallu tant de préparatifs, tant de travaux, tant de tâtonnements, tant de nuits sur la dure, tant de jours dans la peine, c’était là la misère du travail solitaire. Fatalité dans la cause, nécessité dans l’effet. Cette misère, Gilliatt l’avait plus qu’acceptée ; il l’avait voulue. Redoutant un concurrent, parce qu’un concurrent eût  pu être un rival, il n’avait point cherché d’auxiliaire. L’écrasante entreprise, le risque, le danger, la besogne multipliée par elle-même, l’engloutissement possible du sauveteur par le sauvetage, la famine, la fièvre, le dénûment, la détresse, il avait tout pris pour lui seul. Il avait eu cet égoïsme.\par
Il était sous une sorte d’effrayante cloche pneumatique. La vitalité se retirait peu à peu de lui. Il s’en apercevait à peine.\par
L’épuisement des forces n’épuise pas la volonté. Croire n’est que la deuxième puissance ; vouloir est la première. Les montagnes proverbiales que la foi transporte ne sont rien à côté de ce que fait la volonté. Tout le terrain que Gilliatt perdait en vigueur, il le regagnait en ténacité. L’amoindrissement de l’homme physique sous l’action refoulante de cette sauvage nature aboutissait au grandissement de l’homme moral.\par
Gilliatt ne sentait point la fatigue, ou, pour mieux dire, n’y consentait pas. Le consentement de l’âme refusé aux défaillances du corps est une force immense.\par
Gilliatt voyait les pas que faisait son travail, et ne voyait que cela. C’était le misérable sans le savoir. Son but, auquel il touchait presque, l’hallucinait. Il souffrait toutes ces souffrances sans qu’il lui vînt une autre pensée que celle-ci : En avant ! Son œuvre lui montait à la tête. La volonté grise. On peut s’enivrer de son âme.\par
Cette ivrognerie-là s’appelle l’héroïsme.\par
Gilliatt était une espèce de Job de l’océan.\par
 Mais un Job luttant, un Job combattant et faisant front aux fléaux, un Job conquérant, et, si de tels mots n’étaient pas trop grands pour un pauvre matelot pêcheur de crabes et de langoustes, un Job Prométhée.
 \subsubsection[{B.II.5. Sub umbra}]{B.II.5. \\
Sub umbra}
\noindent Parfois, la nuit, Gilliatt ouvrait les yeux et regardait l’ombre.\par
Il se sentait étrangement ému.\par
L’œil ouvert sur le noir. Situation lugubre ; anxiété.\par
La pression de l’ombre existe.\par
Un indicible plafond de ténèbres ; une haute obscurité sans plongeur possible ; de la lumière mêlée à cette obscurité, on ne sait quelle lumière vaincue et sombre ; de la clarté mise en poudre ; est-ce une semence ? est-ce une cendre ? des millions de flambeaux, nul éclairage ; une vaste ignition qui ne dit pas son secret, une diffusion de feu en poussière qui semble une volée d’étincelles arrêtée, le désordre du tourbillon et l’immobilité du sépulcre, le problème offrant une ouverture de précipice, l’énigme montrant et cachant sa face, l’infini masqué de noirceur, voilà la nuit. Cette superposition pèse à l’homme.\par
Cet amalgame de tous les mystères à la fois, du mystère cosmique comme du mystère fatal, accable la tête humaine.\par
 La pression de l’ombre agit en sens inverse sur les différentes espèces d’âmes. L’homme devant la nuit se reconnaît incomplet. Il voit l’obscurité et sent l’infirmité. Le ciel noir, c’est l’homme aveugle. L’homme, face à face avec la nuit, s’abat, s’agenouille, se prosterne, se couche à plat ventre, rampe vers un trou, ou se cherche des ailes. Presque toujours il veut fuir cette présence informe de l’Inconnu. Il se demande ce que c’est ; il tremble, il se courbe, il ignore ; parfois aussi il veut y aller.\par
Aller où ?\par
Là.\par
Là ? Qu’est-ce ? et qu’y a-t-il ?\par
Cette curiosité est évidemment celle des choses défendues, car de ce côté tous les ponts autour de l’homme sont rompus. L’arche de l’infini manque. Mais le défendu attire, étant gouffre. Où le pied ne va pas, le regard peut atteindre ; où le regard s’arrête, l’esprit peut continuer. Pas d’homme qui n’essaie, si faible et si insuffisant qu’il soit. L’homme, selon sa nature, est en quête ou en arrêt devant la nuit. Pour les uns c’est un refoulement ; pour les autres c’est une dilatation. Le spectacle est sombre. L’indéfinissable y est mêlé.\par
La nuit est-elle sereine ? C’est un fond d’ombre. Est-elle orageuse ? C’est un fond de fumée. L’illimité se refuse et s’offre à la fois, fermé à l’expérimentation, ouvert à la conjecture. D’innombrables piqûres de lumière rendent plus noire l’obscurité sans fond. Escarboucles, scintillations, astres. Présences constatées dans l’Ignoré ; défis effrayants d’aller toucher à ces  clartés. Ce sont des jalons de création dans l’absolu ; ce sont des marques de distance, là où il n’y a plus de distance ; c’est on ne sait quel numérotage impossible, et réel pourtant, de l’étiage des profondeurs. Un point microscopique qui brille, puis un autre, puis un autre, puis un autre ; c’est l’imperceptible, c’est l’énorme. Cette lumière est un foyer, ce foyer est une étoile, cette étoile est un soleil, ce soleil est un univers, cet univers n’est rien. Tout nombre est zéro devant l’infini.\par
Ces univers, qui ne sont rien, existent. En les constatant, on sent la différence qui sépare n’être rien de n’être pas.\par
L’inaccessible ajouté à l’inexplicable, tel est le ciel.\par
De cette contemplation se dégage un phénomène sublime, le grandissement de l’âme par la stupeur.\par
L’effroi sacré est propre à l’homme ; la bête ignore cette crainte. L’intelligence trouve dans cette terreur auguste son éclipse, et sa preuve.\par
L’ombre est une ; de là l’horreur. En même temps elle est complexe ; de là l’épouvante. Son unité fait masse sur notre esprit, et ôte l’envie de résister. Sa complexité fait qu’on regarde de tous côtés autour de soi ; il semble qu’on ait à craindre de brusques arrivées. On se rend, et on se garde. On est en présence de Tout, d’où la soumission, et de Plusieurs, d’où la défiance. L’unité de l’ombre contient un multiple. Multiple mystérieux, visible dans la matière, sensible dans la pensée. Cela fait silence, raison de plus d’être au guet.\par
 La nuit, — celui qui écrit ceci l’a dit ailleurs, — c’est l’état propre et normal de la création spéciale dont nous faisons partie. Le jour, bref dans la durée comme dans l’espace, n’est qu’une proximité d’étoile.\par
Le prodige nocturne universel ne s’accomplit pas sans frottements, et tous les frottements d’une telle machine sont des contusions à la vie. Les frottements de la machine, c’est là ce que nous nommons le Mal.\par
Nous sentons dans cette obscurité le mal, démenti latent à l’ordre divin, blasphème implicite du fait rebelle à l’idéal. Le mal complique d’on ne sait quelle tératologie à mille têtes le vaste ensemble cosmique. Le mal est présent à tout pour protester. Il est ouragan, et il tourmente la marche d’un navire ; il est chaos, et il entrave l’éclosion d’un monde. Le bien a l’unité, le mal a l’ubiquité. Le mal déconcerte la vie, qui est une logique. Il fait dévorer la mouche par l’oiseau et la planète par la comète. Le mal est une rature à la création.\par
L’obscurité nocturne est pleine d’un vertige. Qui l’approfondit s’y submerge et s’y débat. Pas de fatigue comparable à cet examen des ténèbres. C’est l’étude d’un effacement.\par
Aucun lieu définitif où poser l’esprit. Des points de départ sans point d’arrivée. L’entre-croisement des solutions contradictoires, tous les embranchements du doute s’offrant en même temps, la ramification des phénomènes s’exfoliant sans limite sous une poussée indéfinie, toutes les lois se versant l’une dans l’autre,  une promiscuité insondable qui fait que la minéralisation végète, que la végétation vit, que la pensée pèse, que l’amour rayonne et que la gravitation aime ; l’immense front d’attaque de toutes les questions se développant dans l’obscurité sans bornes ; l’entrevu ébauchant l’ignoré ; la simultanéité cosmique en pleine apparition, non pour le regard mais pour l’intelligence, dans le grand espace indistinct ; l’invisible devenu vision. C’est l’Ombre. L’homme est là-dessous.\par
Il ne connaît pas le détail, mais il porte, en quantité proportionnée à son esprit, le poids monstrueux de l’ensemble. Cette obsession poussait les pâtres chaldéens à l’astronomie. Des révélations involontaires sortent des pores de la création ; une exsudation de science se fait en quelque sorte d’elle-même, et gagne l’ignorant. Tout solitaire, sous cette imprégnation mystérieuse, devient, souvent sans en avoir conscience, un philosophe naturel.\par
L’obscurité est indivisible. Elle est habitée. Habitée sans déplacement par l’absolu, habitée aussi avec déplacement. On s’y meut, chose inquiétante. Une formation sacrée y accomplit ses phases. Des préméditations, des puissances, des destinations voulues, y élaborent en commun une œuvre démesurée. Une vie terrible et horrible est là dedans. Il y a de vastes évolutions d’astres, la famille stellaire, la famille planétaire, le pollen zodiacal, le \emph{quid divinum} des courants, des effluves, des polarisations et des attractions ; il y a l’embrassement et l’antagonisme, un magnifique flux et reflux d’antithèse universelle, l’impondérable en  liberté au milieu des centres ; il y a la séve dans les globes, la lumière hors des globes, l’atome errant, le germe épars, des courbes de fécondation, des rencontres d’accouplement et de combat, des profusions inouïes, des distances qui ressemblent à des rêves, des circulations vertigineuses, des enfoncements de mondes dans l’incalculable, des prodiges s’entre-poursuivant dans les ténèbres, un mécanisme une fois pour toutes, des souffles de sphères en fuite, des roues qu’on sent tourner ; le savant conjecture, l’ignorant consent et tremble ; cela est et se dérobe ; c’est inexpugnable, c’est hors de portée, c’est hors d’approche. On est convaincu jusqu’à l’oppression. On a sur soi on ne sait quelle évidence noire. On ne peut rien saisir. On est écrasé par l’impalpable.\par
Partout l’incompréhensible ; nulle part l’inintelligible.\par
Et à tout cela ajoutez la question redoutable ; cette Immanence est-elle un Être ?\par
On est sous l’ombre. On regarde. On écoute.\par
Cependant la sombre terre marche et roule ; les fleurs ont conscience de ce mouvement énorme, la silène s’ouvre à onze heures du soir et l’hémérocalle à cinq heures du matin. Régularités saisissantes.\par
Dans d’autres profondeurs la goutte d’eau se fait monde, l’infusoire pullule, la fécondité géante sort de l’animalcule, l’imperceptible étale sa grandeur, le sens inverse de l’immensité se manifeste ; une diatomée en une heure produit treize cents millions de diatomées.\par
Quelle proposition de toutes les énigmes à la fois !\par
 L’irréductible est là.\par
On est contraint à la foi. Croire de force, tel est le résultat. Mais avoir foi ne suffit pas pour être tranquille. La foi a on ne sait quel bizarre besoin de forme. De là les religions. Rien n’est accablant comme une croyance sans contour.\par
Quoi qu’on pense et quoi qu’on veuille, quelque résistance qu’on ait en soi, regarder l’ombre, ce n’est pas regarder, c’est contempler.\par
Que faire de ces phénomènes ? Comment se mouvoir sous leur convergence ? Décomposer cette pression est impossible. Quelle rêverie ajuster à tous ces aboutissants mystérieux ? Que de révélations abstruses, simultanées, balbutiantes, s’obscurcissant par leur foule même, sortes de bégaiements du verbe ! L’ombre est un silence ; mais ce silence dit tout. Une résultante s’en dégage majestueusement : Dieu. Dieu, c’est la notion incompressible. Elle est dans l’homme. Les syllogismes, les querelles, les négations, les systèmes, les religions, passent dessus sans la diminuer. Cette notion, l’ombre tout entière l’affirme. Mais le trouble est sur tout le reste. Immanence formidable. L’inexprimable entente des forces se manifeste par le maintien de toute cette obscurité en équilibre. L’univers pend ; rien ne tombe. Le déplacement incessant et démesuré s’opère sans accident et sans fracture. L’homme participe à ce mouvement de translation, et la quantité d’oscillation qu’il subit, il l’appelle la destinée. Où commence la destinée ? Où finit la nature ? Quelle différence y a-t-il entre un événement et une saison, entre  un chagrin et une pluie, entre une vertu et une étoile ? Une heure, n’est-ce pas une onde ? Les engrenages en mouvement continuent, sans répondre à l’homme, leur révolution impassible. Le ciel étoilé est une vision de roues, de balanciers et de contre-poids. C’est la contemplation suprême, doublée de la suprême méditation. C’est toute la réalité, plus toute l’abstraction. Rien au delà. On se sent pris. On est à la discrétion de cette ombre. Pas d’évasion possible. On se voit dans l’engrenage, on est partie intégrante d’un Tout ignoré, on sent l’inconnu qu’on a en soi fraterniser mystérieusement avec un inconnu qu’on a hors de soi. Ceci est l’annonce sublime de la mort. Quelle angoisse, et en même temps quel ravissement ! Adhérer à l’infini, être amené par cette adhérence à s’attribuer à soi-même une immortalité nécessaire, qui sait ? une éternité possible, sentir dans le prodigieux flot de ce déluge de vie universelle l’opiniâtreté insubmersible du moi ! regarder les astres et dire : je suis une âme comme vous ; regarder l’obscurité et dire : je suis un abîme comme toi !\par
Ces énormités, c’est la Nuit.\par
Tout cela, accru par la solitude, pesait sur Gilliatt.\par
Le comprenait-il ? Non.\par
Le sentait-il ? Oui.\par
Gilliatt était un grand esprit trouble et un grand cœur sauvage.
 \subsubsection[{B.II.6. Gilliatt fait prendre position a la panse}]{B.II.6. \\
Gilliatt fait prendre position a la panse}
\noindent Ce sauvetage de la machine, médité par Gilliatt, était, nous l’avons dit déjà, une véritable évasion, et l’on connaît les patiences de l’évasion. On en connaît aussi les industries. L’industrie va jusqu’au miracle ; la patience va jusqu’à l’agonie. Tel prisonnier, Thomas, par exemple, au Mont-Saint-Michel, trouve moyen de mettre la moitié d’une muraille dans sa paillasse. Tel autre, à Tulle, en 1820, coupe du plomb sur la plate-forme promenoir de la prison, avec quel couteau ? on ne peut le deviner, fait fondre ce plomb, avec quel feu ? on l’ignore, coule ce plomb fondu, dans quel moule ? on le sait, dans un moule de mie de pain ; avec ce plomb et ce moule, fait une clef, et avec cette clef ouvre une serrure dont il n’avait jamais vu que le trou. Ces habiletés inouïes, Gilliatt les avait. Il eût monté et descendu la falaise de Boisrosé. Il était le Trenck d’une épave et le Latude d’une machine.\par
La mer, geôlière, le surveillait.\par
 Du reste, disons-le, si ingrate et si mauvaise que fût la pluie, il en avait tiré parti. Il avait un peu refait sa provision d’eau douce ; mais sa soif était inextinguible, et il vidait son bidon presque aussi rapidement qu’il l’emplissait.\par
Un jour, le dernier jour d’avril, je crois, ou le premier de mai, tout se trouva prêt.\par
La parquet de la machine était comme encadré entre les huit câbles des palans, quatre d’un côté, quatre de l’autre. Les seize ouvertures par où passaient ces câbles étaient reliées sur le pont et sous la carène par des traits de scie. Le vaigrage avait été coupé avec la scie, la charpente avec la hache, la ferrure avec la lime, le doublage avec le ciseau. La partie de la quille à laquelle se superposait la machine, était coupée carrément et prête à glisser avec la machine en la soutenant. Tout ce branle effrayant ne tenait plus qu’à une chaîne qui, elle-même, ne tenait plus qu’à un coup de lime. A ce point d’achèvement et si près de la fin, la hâte est prudence.\par
La marée était basse, c’était le bon moment.\par
Gilliatt était parvenu à démonter l’arbre des roues dont les extrémités pouvaient faire obstacle et arrêter le dérapement. Il avait réussi à amarrer verticalement cette lourde pièce dans la cage même de la machine.\par
Il était temps de finir. Gilliatt, nous venons de le dire, n’était point fatigué, ne voulant pas l’être, mais ses outils l’étaient. La forge devenait peu à peu impossible. La pierre enclume s’était fendue. La soufflante commençait à mal travailler. La petite chute  hydraulique étant d’eau marine, des dépôts salins s’étaient formés dans les jointures de l’appareil, et en gênaient le jeu.\par
Gilliatt alla à la crique de l’Homme, passa la panse en revue, s’assura que tout y était en état, particulièrement les quatre anneaux plantés à bâbord et à tribord, puis leva l’ancre, et, ramant, revint avec la panse aux deux Douvres.\par
L’entre-deux des Douvres pouvait admettre la panse. Il y avait assez de fond et assez d’ouverture. Gilliatt avait reconnu dès le premier jour qu’on pouvait pousser la panse jusque sous la Durande.\par
La manœuvre pourtant était excessive, elle exigeait une précision de bijoutier, et cette insertion de la barque dans l’écueil était d’autant plus délicate que, pour ce que Gilliatt voulait faire, il était nécessaire d’entrer par la poupe, le gouvernail en avant. Il importait que le mât et le gréement de la panse restassent en deçà de l’épave, du côté du goulet.\par
Ces aggravations dans la manœuvre rendaient l’opération malaisée pour Gilliatt lui-même. Ce n’était plus, comme pour la crique de l’Homme, l’affaire d’un coup de barre, il fallait tout ensemble pousser, tirer, ramer et sonder. Gilliatt n’y employa pas moins d’un quart d’heure. Il y parvint pourtant.\par
En quinze ou vingt minutes, la panse fut ajustée sous la Durande. Elle y fut presque embossée. Gilliatt, au moyen de ses deux ancres, affourcha la panse. La plus grosse des deux se trouva placée de façon à travailler du plus fort vent à craindre, qui était le vent  d’ouest. Puis, à l’aide d’un levier et du cabestan, Gilliatt descendit dans la panse les deux caisses contenant les roues démontées, dont les élingues étaient toutes prêtes. Ces deux caisses firent lest.\par
Débarrassé des deux caisses, Gilliatt rattacha au crochet de la chaîne du cabestan l’élingue du palanguin régulateur, destiné à enrayer les palans.\par
Pour ce que méditait Gilliatt, les défauts de la panse devenaient des qualités ; elle n’était pas pontée, le chargement aurait plus de profondeur, et pourrait poser sur la cale ; elle était mâtée à l’avant, trop à l’avant peut-être, le chargement aurait plus d’aisance, et, le mât se trouvant ainsi en dehors de l’épave, rien ne gênerait la sortie ; elle n’était qu’un sabot, rien n’est stable et solide en mer comme un sabot.\par
Tout à coup Gilliatt s’aperçut que la mer montait. Il regarda d’où venait le vent.
 \subsubsection[{B.II.7. Tout de suite un danger}]{B.II.7. \\
Tout de suite un danger}
\noindent Il y avait peu de brise, mais ce qui soufflait, soufflait de l’ouest. C’est une mauvaise habitude que le vent a volontiers dans l’équinoxe.\par
La marée montante, selon le vent qui souffle, se comporte diversement dans l’écueil Douvres. Suivant la rafale qui le pousse, le flot entre dans ce corridor soit par l’est, soit par l’ouest. Si la mer entre par l’est, elle est bonne et molle ; si elle entre par l’ouest, elle est furieuse. Cela tient à ce que le vent d’est, venant de terre, a peu d’haleine, tandis que le vent d’ouest, qui traverse l’Atlantique, apporte tout le souffle de l’immensité. Même très peu de brise apparente, si elle vient de l’ouest, est inquiétante. Elle roule les larges lames de l’étendue illimitée, et pousse trop de vague à la fois dans l’étranglement.\par
Une eau qui s’engouffre est toujours affreuse. Il en est d’une eau comme d’une foule ; une multitude est un liquide ; quand la quantité pouvant entrer est  moindre que la quantité voulant entrer, il y a écrasement pour la foule et convulsion pour l’eau. Tant que le vent du couchant règne, fût-ce la plus faible brise, les Douvres ont deux fois par jour cet assaut. La marée s’élève, le flux presse, la roche résiste, le goulet ne s’ouvre qu’avarement, le flot enfoncé de force bondit et rugit, et une houle forcenée bat les deux façades intérieures de la ruelle. De sorte que les Douvres, par le moindre vent d’ouest, offrent ce spectacle singulier : dehors, sur la mer, le calme ; dans l’écueil, un orage. Ce tumulte local et circonscrit n’a rien d’une tempête ; ce n’est qu’une émeute de vagues, mais terrible. Quant aux vents de nord et de sud, ils prennent l’écueil en travers et ne font que peu de ressac dans le boyau. L’entrée par l’est, détail qu’il faut rappeler, confine au rocher l’Homme ; l’ouverture redoutable de l’ouest est à l’extrémité opposée, précisément entre les deux Douvres.\par
C’est à cette ouverture de l’ouest que se trouvait Gilliatt avec la Durande échouée et la panse embossée.\par
Une catastrophe semblait inévitable. Cette catastrophe imminente avait, en quantité faible, mais suffisante, le vent qu’il lui fallait.\par
Avant peu d’heures, le gonflement de la marée ascendante allait se ruer de haute lutte dans le détroit des Douvres. Les premières lames bruissaient déjà. Ce gonflement, mascaret de toute l’Atlantique, aurait derrière lui la totalité de la mer. Aucune bourrasque, aucune colère ; mais une simple onde souveraine contenant en elle une force d’impulsion qui, partie de  l’Amérique pour aboutir à l’Europe, a deux mille lieues de jet. Cette onde, barre gigantesque de l’océan, rencontrerait l’hiatus de l’écueil et, froncée aux deux Douvres, tours de l’entrée, piliers du détroit, enflée par le flux, enflée par l’empêchement, repoussée par le rocher, surmenée par la brise, ferait violence à l’écueil, pénétrerait, avec toutes les torsions de l’obstacle subi et toutes les frénésies de la vague entravée, entre les deux murailles, y trouverait la panse et la Durande, et les briserait.\par
Contre cette éventualité, il fallait un bouclier. Gilliatt l’avait.\par
Il fallait empêcher la marée de pénétrer d’emblée, lui interdire de heurter tout en la laissant monter, lui barrer le passage sans lui refuser l’entrée, lui résister et lui céder, prévenir la compression du flot dans le goulet, qui était tout le danger, remplacer l’irruption par l’introduction, soutirer à la vague son emportement et sa brutalité, contraindre cette furie à la douceur. Il fallait substituer à l’obstacle qui irrite l’obstacle qui apaise.\par
Gilliatt, avec cette adresse qu’il avait, plus forte que la force, exécutant une manœuvre de chamois dans la montagne ou de sapajou dans la forêt, utilisant pour des enjambées oscillantes et vertigineuses la moindre pierre en saillie, sautant à l’eau, sortant de l’eau, nageant dans le remous, grimpant au rocher, une corde entre les dents, un marteau à la main, détacha le grelin qui maintenait suspendu et collé au soubassement de la petite Douvre le pan de muraille  de l’avant de la Durande, façonna avec des bouts de haussière des espèces de gonds rattachant ce panneau aux gros clous plantés dans le granit, fit tourner sur ces gonds cette armature de planches pareille à une trappe d’écluse, l’offrit en flanc, comme on fait d’une joue de gouvernail, au flot qui en poussa et en appliqua une extrémité sur la grande Douvre pendant que les gonds de corde retenaient sur la petite Douvre l’autre extrémité, opéra sur la grande Douvre, au moyen des clous d’attente plantés d’avance, la même fixation que sur la petite, amarra solidement cette vaste plaque de bois au double pilier du goulet, croisa sur ce barrage une chaîne comme un baudrier sur une cuirasse, et en moins d’une heure cette clôture se dressa contre la marée, et la ruelle de l’écueil fut fermée comme par une porte.\par
Cette puissante applique, lourde masse de poutres et de planches, qui, à plat, eût été un radeau, et, debout, était un mur, avait, le flot aidant, été maniée par Gilliatt avec une dextérité de saltimbanque. On pourrait presque dire que le tour était fait avant que la mer montante eût eu le temps de s’en apercevoir.\par
C’était un de ces cas où Jean Bart eût dit le fameux mot qu’il adressait au flot de la mer chaque fois qu’il esquivait un naufrage : \emph{attrapé, l’anglais !} On sait que quand Jean Bart voulait insulter l’océan, il l’appelait l’\emph{anglais.}\par
Le détroit barré, Gilliatt songea à la panse. Il dévida assez de câble sur les deux ancres pour qu’elle pût monter avec la marée. Opération analogue à ce que  les anciens marins appelaient « mouiller avec des embossures ». Dans tout ceci, Gilliatt n’était pas pris au dépourvu, le cas était prévu ; un homme du métier l’eût reconnu à deux poulies de guinderesse frappées en galoche à l’arrière de la panse, dans lesquelles passaient deux grelins dont les bouts étaient en ralingue aux organeaux des deux ancres.\par
Cependant le flux avait grossi ; la demi-montée s’était faite ; c’est à ce moment que les chocs des lames de la marée, même paisible, peuvent être rudes. Ce que Gilliatt avait combiné, se réalisa. Le flot roulait violemment vers le barrage, le rencontrait, s’y enflait, et passait dessous. Au dehors, c’était la houle, au dedans l’infiltration. Gilliatt avait imaginé quelque chose comme les fourches caudines de la mer. La marée était vaincue.
 \subsubsection[{B.II.8. Péripétie plutot que dénoument}]{B.II.8. \\
Péripétie plutot que dénoument}
\noindent Le moment redoutable était venu.\par
Il s’agissait maintenant de mettre la machine dans la barque.\par
Gilliatt fut pensif quelques instants, tenant le coude de son bras gauche dans sa main droite et son front dans sa main gauche.\par
Puis il monta sur l’épave dont une partie, la machine, devait se détacher, et dont l’autre partie, la carcasse, devait demeurer.\par
Il coupa les quatre élingues qui fixaient à tribord et à bâbord à la muraille de la Durande les quatre chaînes de la cheminée. Les élingues n’étant que de la corde, son couteau en vint à bout.\par
Les quatre chaînes, libres et sans attache, vinrent pendre le long de la cheminée.\par
De l’épave il monta dans l’appareil construit par lui, frappa du pied sur les poutres, inspecta les moufles, regarda les poulies, toucha les câbles, examina les  rallonges, s’assura que le funin blanc n’était pas mouillé profondément, constata que rien ne manquait et que rien ne fléchissait, puis, sautant du haut des hiloires sur le pont, il prit position, près du cabestan, dans la partie de la Durande qui devait rester accrochée aux Douvres. C’était là son poste de travail.\par
Grave, ému seulement de l’émotion utile, il jeta un dernier coup d’œil sur les palans, puis saisit une lime et se mit à scier la chaîne qui tenait tout en suspens.\par
On entendait le grincement de la lime dans le grondement de la mer.\par
La chaîne du cabestan, rattachée au palanguin régulateur, était à la portée de Gilliatt, tout près de sa main.\par
Tout à coup il y eut un craquement. Le chaînon que mordait la lime, plus qu’à moitié entamé, venait de se rompre ; tout l’appareil entrait en branle. Gilliatt n’eut que le temps de se jeter sur le palanguin.\par
La chaîne cassée fouetta le rocher, les huit câbles se tendirent, tout le bloc scié et coupé s’arracha de l’épave, le ventre de la Durande s’ouvrit, le plancher de fer de la machine pesant sur les câbles apparut sous la quille.\par
Si Gilliatt n’eût pas empoigné à temps le palanguin, c’était une chute. Mais sa main terrible était là ; ce fut une descente.\par
Quand le frère de Jean Bart, Pieter Bart, ce puissant et sagace ivrogne, ce pauvre pêcheur de Dunkerque qui tutoyait le grand amiral de France, sauva la galère  Langeron en perdition dans la baie d’Ambleteuse, quand, pour tirer cette lourde masse flottante du milieu des brisants de la baie furieuse, il lia la grande voile en rouleau avec des joncs marins, quand il voulut que ce fût ces roseaux qui, en se cassant d’eux-mêmes, donnassent au vent la voile à enfler, il se fia à la rupture des roseaux comme Gilliatt à la fracture de la chaîne, et ce fut la même hardiesse bizarre couronnée du même succès surprenant.\par
Le palanguin, saisi par Gilliatt, tint bon et opéra admirablement. Sa fonction, on s’en souvient, était l’amortissement des forces, ramenées de plusieurs à une seule, et réduites à un mouvement d’ensemble. Ce palanguin avait quelque rapport avec une patte de bouline ; seulement, au lieu d’orienter une voile, il équilibrait un mécanisme.\par
Gilliatt, debout et le poing au cabestan, avait, pour ainsi dire, la main sur le pouls de l’appareil.\par
Ici l’invention de Gilliatt éclata.\par
Une remarquable coïncidence de forces se produisit.\par
Pendant que la machine de la Durande, détachée en bloc, descendait vers la panse, la panse montait vers la machine. L’épave et le bateau sauveteur, s’entr’aidant en sens inverse, allaient au-devant l’un de l’autre. Ils venaient se chercher et s’épargnaient la moitié du travail.\par
Le flux, se gonflant sans bruit entre les deux Douvres, soulevait l’embarcation et l’approchait de la Durande. La marée était plus que vaincue, elle était  domestiquée. L’océan faisait partie du mécanisme.\par
Le flot montant haussait la panse sans choc, mollement, presque avec précaution et comme si elle eût été de porcelaine.\par
Gilliatt combinait et proportionnait les deux travaux, celui de l’eau et celui de l’appareil, et, immobile au cabestan, espèce de statue redoutable obéie par tous les mouvements à la fois, réglait la lenteur de la descente sur la lenteur de la montée.\par
Pas de secousse dans le flot, pas de saccade dans les palans. C’était une étrange collaboration de toutes les forces naturelles, soumises. D’un côté, la gravitation, apportant la machine ; de l’autre, la marée, apportant la barque. L’attraction des astres, qui est le flux, et l’attraction du globe, qui est la pesanteur, semblaient s’entendre pour servir Gilliatt. Leur subordination n’avait pas d’hésitation ni de temps d’arrêt, et, sous la pression d’une âme, ces puissances passives devenaient des auxiliaires actifs. De minute en minute l’œuvre avançait ; l’intervalle entre la panse et l’épave diminuait insensiblement. L’approche se faisait en silence et avec une sorte de terreur de l’homme qui était là. L’élément recevait un ordre et l’exécutait.\par
Presque au moment précis où le flux cessa de s’élever, les câbles cessèrent de se dévider. Subitement, mais sans commotion, les moufles s’arrêtèrent. La machine, comme posée par une main, avait pris assiette dans la panse. Elle y était droite, debout, immobile, solide. La plaque de soutènement s’appuyait de ses quatre angles et d’aplomb sur la cale.\par
 C’était fait.\par
Gilliatt regarda, éperdu.\par
Le pauvre être n’était point gâté par la joie. Il eut le fléchissement d’un immense bonheur. Il sentit tous ses membres plier ; et, devant son triomphe, lui qui n’avait pas eu un trouble jusqu’alors, il se mit à trembler.\par
Il considéra la panse sous l’épave, et la machine dans la panse. Il semblait n’y pas croire. On eût dit qu’il ne s’attendait pas à ce qu’il avait fait. Un prodige lui était sorti des mains, et il le regardait avec stupeur.\par
Cet effarement dura peu.\par
Gilliatt eut le mouvement d’un homme qui se réveille, se jeta sur la scie, coupa les huit câbles, puis, séparé maintenant de la panse, grâce au soulèvement du flux, d’une dizaine de pieds seulement, il y sauta, prit un rouleau de filin, fabriqua quatre élingues, les passa dans les anneaux préparés d’avance, et fixa, des deux côtés, au bord de la panse, les quatre chaînes de la cheminée encore attachées une heure auparavant au bord de la Durande.\par
La cheminée amarrée, Gilliatt dégagea le haut de la machine. Un morceau carré du tablier du pont de la Durande y adhérait. Gilliatt le décloua, et débarrassa la panse de cet encombrement de planches et de solives qu’il jeta sur le rocher. Allégement utile.\par
Du reste, la panse, comme on devait le prévoir, s’était maintenue fermement sous la surcharge de la machine. La panse ne s’était enfoncée que jusqu’à un  bon étiage de flottaison. La machine de la Durande, quoique pesante, était moins lourde que le monceau de pierres et le canon rapportés jadis de Herm par la panse.\par
Tout était donc fini. Il n’y avait plus qu’à s’en aller.
 \subsubsection[{B.II.9. Le succès repris aussitot que donné}]{B.II.9. \\
Le succès repris aussitot que donné}
\noindent Tout n’était pas fini.\par
Rouvrir le goulet fermé par le morceau de muraille de la Durande, et pousser tout de suite la panse hors de l’écueil, rien n’était plus clairement indiqué. En mer, toutes les minutes sont urgentes. Peu de vent, à peine une ride au large ; la soirée, très belle, promettait une belle nuit. La mer était étale, mais le reflux commençait à se faire sentir ; le moment était excellent pour partir. On aurait la marée descendante pour sortir des Douvres et la marée remontante pour rentrer à Guernesey. On pourrait être à Saint-Sampson au point du jour.\par
Mais un obstacle inattendu se présenta. Il y avait eu une lacune dans la prévoyance de Gilliatt.\par
La machine était libre ; la cheminée ne l’était pas.\par
La marée, en approchant la panse de l’épave suspendue en l’air, avait amoindri les périls de la descente et abrégé le sauvetage ; mais cette diminution  d’intervalle avait laissé le haut de la cheminée engagé dans l’espèce de cadre béant qu’offrait la coque ouverte de la Durande. La cheminée était prise là comme entre quatre murs.\par
Le service rendu par le flot se compliquait de cette sournoiserie. Il semblait que la mer, contrainte d’obéir, eût eu une arrière-pensée.\par
Il est vrai que ce que le flux avait fait, le reflux allait le défaire.\par
La cheminée, haute d’un peu plus de trois toises, s’enfonçait de huit pieds dans la Durande ; le niveau de l’eau allait baisser de douze pieds ; la cheminée, descendant avec la panse sur le flot décroissant, aurait quatre pieds d’aisance et pourrait se dégager.\par
Mais combien de temps fallait-il pour cette mise en liberté ? Six heures.\par
Dans six heures il serait près de minuit. Quel moyen d’essayer la sortie à pareille heure, quel chenal suivre à travers tous ces brisants déjà si inextricables le jour, et comment se risquer en pleine nuit noire dans cette embuscade de bas-fonds ?\par
Force était d’attendre au lendemain. Ces six heures perdues en faisaient perdre au moins douze.\par
Il ne fallait pas même songer à avancer le travail en rouvrant le goulet de l’écueil. Le barrage serait nécessaire à la prochaine marée.\par
Gilliatt dut se reposer.\par
Se croiser les bras, c’était la seule chose qu’il n’eût pas encore faite depuis qu’il était dans l’écueil des Douvres.\par
 Ce repos forcé l’irrita et l’indigna presque, comme s’il était de sa faute. Il se dit : Qu’est-ce que Déruchette penserait de moi, si elle me voyait là à rien faire ?\par
Pourtant cette reprise de forces n’était peut-être pas inutile.\par
La panse étant maintenant à sa disposition, il arrêta qu’il y passerait la nuit.\par
Il alla chercher sa peau de mouton sur la grande Douvre, redescendit, soupa de quelques patelles et de deux ou trois châtaignes de mer, but, ayant grand’soif, les dernières gorgées d’eau douce de son bidon presque vide, s’enveloppa de la peau dont la laine lui fit plaisir, se coucha comme un chien de garde près de la machine, rabattit sa galérienne sur ses yeux, et s’endormit.\par
Il dormit profondément. On a de ces sommeils après les choses faites.
 \subsubsection[{B.II.10. Les avertissements de la mer}]{B.II.10. \\
Les avertissements de la mer}
\noindent Au milieu de la nuit, brusquement, et comme par la détente d’un ressort, il se réveilla.\par
Il ouvrit les yeux.\par
Les Douvres au-dessus de sa tête étaient éclairées ainsi que par la réverbération d’une grande braise blanche. Il y avait sur toute la façade noire de l’écueil comme le reflet d’un feu.\par
D’où venait ce feu ?\par
De l’eau.\par
La mer était extraordinaire.\par
Il semblait que l’eau fût incendiée. Aussi loin que le regard pouvait s’étendre, dans l’écueil et hors de l’écueil, toute la mer flamboyait. Ce flamboiement n’était pas rouge ; il n’avait rien de la grande flamme vivante des cratères et des fournaises. Aucun pétillement, aucune ardeur, aucune pourpre, aucun bruit. Des traînées bleuâtres imitaient sur la vague des plis de suaire. Une large lueur blême frissonnait sur l’eau. Ce n’était pas l’incendie ; c’en était le spectre.\par
 C’était quelque chose comme l’embrasement livide d’un dedans de sépulcre par une flamme de rêve.\par
Qu’on se figure des ténèbres allumées.\par
La nuit, la vaste nuit trouble et diffuse, semblait être le combustible de ce feu glacé. C’était on ne sait quelle clarté faite d’aveuglement. L’ombre entrait comme élément dans cette lumière fantôme.\par
Les marins de la Manche connaissent tous ces indescriptibles phosphorescences, pleines d’avertissements pour le navigateur. Elles ne sont nulle part plus surprenantes que dans le Grand V, près d’Isigny.\par
A cette lumière, les choses perdent leur réalité. Une pénétration spectrale les fait comme transparentes. Les roches ne sont plus que des linéaments. Les câbles des ancres paraissent des barres de fer chauffées à blanc. Les filets des pêcheurs semblent sous l’eau du feu tricoté. Une moitié de l’aviron est d’ébène, l’autre moitié, sous la lame, est d’argent. En retombant de la rame dans le flot, les gouttes d’eau étoilent la mer. Toute barque traîne derrière elle une comète. Les matelots mouillés et lumineux semblent des hommes qui brûlent. On plonge sa main dans le flot, on la retire gantée de flamme ; cette flamme est morte, on ne la sent point. Votre bras est un tison allumé. Vous voyez les formes qui sont dans la mer rouler sous les vagues à vau-le-feu. L’écume étincelle. Les poissons sont des langues de feu et des tronçons d’éclair serpentant dans une profondeur pâle.\par
Cette clarté avait passé à travers les paupières fermées de Gilliatt. C’est grâce à elle qu’il s’était réveillé.\par
 Ce réveil vint à point.\par
Le reflux avait descendu ; un nouveau flux revenait. La cheminée de la machine, dégagée pendant le sommeil de Gilliatt, allait être ressaisie par l’épave, béante au-dessus d’elle.\par
Elle y retournait lentement.\par
Il ne s’en fallait que d’un pied pour que la cheminée rentrât dans la Durande.\par
La remontée d’un pied, c’est pour le flux environ une demi-heure. Gilliatt, s’il voulait profiter de cette délivrance déjà remise en question, avait une demi-heure devant lui.\par
Il se dressa en sursaut.\par
Si urgente que fût la situation, il ne put faire autrement que de rester quelques minutes debout, considérant la phosphorescence, méditant.\par
Gilliatt savait à fond la mer. Malgré qu’elle en eût, et quoique souvent maltraité par elle, il était depuis longtemps son compagnon. Cet être mystérieux qu’on nomme l’océan ne pouvait rien avoir dans l’idée que Gilliatt ne le devinât. Gilliatt, à force d’observation, de rêverie et de solitude, était devenu un voyant du temps, ce qu’on appelle en anglais un \emph{wheater wise.}\par
Gilliatt courut aux guinderesses et fila du câble ; puis, n’étant plus retenu par l’affourche, il saisit le croc de la panse, et, s’appuyant aux roches, la poussa vers le goulet à quelques brasses au delà de la Durande, tout près du barrage. Il y avait \emph{du rang,} comme disent les matelots de Guernesey. En moins de dix minutes, la panse fut retirée de dessous la carcasse échouée.\par
 Plus de crainte que la cheminée fût désormais reprise au piège. Le flux pouvait monter.\par
Pourtant Gilliatt n’avait point l’air d’un homme qui va partir.\par
Il considéra encore la phosphorescence, et leva les ancres ; mais ce ne fut point pour déplanter, ce fut pour affourcher de nouveau la panse, et très solidement ; près de la sortie, il est vrai.\par
Il n’avait employé jusque-là que les deux ancres de la panse, et il ne s’était pas encore servi de la petite ancre de la Durande, retrouvée, on s’en souvient, dans les brisants. Cette ancre avait été déposée par lui, toute prête aux urgences, dans un coin de la panse, avec un en-cas de haussières et de poulies de guinderesses, et son câble tout garni d’avance de bosses très cassantes, ce qui empêche la chasse. Gilliatt mouilla cette troisième ancre, en ayant soin de rattacher le câble à un grelin dont un bout était en ralingue à l’organeau de l’ancre, et dont l’autre bout se garnissait au guindoir de la panse. Il pratiqua de cette façon une sorte d’affourche en patte d’oie, bien plus forte que l’affourche à deux ancres. Ceci indiquait une vive préoccupation, et un redoublement de précautions. Un marin eût reconnu dans cette opération quelque chose de pareil au mouillage d’un temps forcé, quand on peut craindre un courant qui prendrait le navire par sous le vent.\par
La phosphorescence, que Gilliatt surveillait et sur laquelle il avait l’œil fixé, le menaçait peut-être, mais en même temps le servait. Sans elle il eût été  prisonnier du sommeil et dupe de la nuit. Elle l’avait réveillé, et elle l’éclairait.\par
Elle faisait dans l’écueil un jour louche. Mais cette clarté, si inquiétante qu’elle parût à Gilliatt, avait eu cela d’utile qu’elle lui avait rendu le danger visible et la manœuvre possible. Désormais, quand Gilliatt voudrait mettre à la voile, la panse, emportant la machine, était libre.\par
Seulement, Gilliatt semblait de moins en moins songer au départ. La panse embossée, il alla chercher la plus forte chaîne qu’il eût dans son magasin, et, la rattachant aux clous plantés dans les deux Douvres, il fortifia en dedans avec cette chaîne le rempart de vaigres et de solives déjà protégé au dehors par l’autre chaîne croisée. Loin d’ouvrir l’issue, il achevait de la barrer.\par
La phosphorescence l’éclairait encore, mais décroissait. Il est vrai que le jour commençait à poindre.\par
Tout à coup Gilliatt prêta l’oreille.
 \subsubsection[{B.II.11. À bon entendeur, salut}]{B.II.11. \\
À bon entendeur, salut}
\noindent Il lui sembla entendre, dans un lointain immense, quelque chose de faible et d’indistinct.\par
Les profondeurs ont, à de certaines heures, un grondement.\par
Il écouta une seconde fois. Le bruit lointain recommença. Gilliatt secoua la tête comme quelqu’un qui sait ce que c’est.\par
Quelques minutes après, il était à l’autre extrémité de la ruelle de l’écueil, à l’entrée vers l’est, libre, jusque-là, et, à grands coups de marteau, il enfonçait de gros clous dans le granit des deux musoirs de ce goulet voisin du rocher l’Homme, comme il avait fait pour le goulet des Douvres.\par
Les crevasses de ces rochers étaient toutes préparées et bien garnies de bois, presque tout cœur de chêne. L’écueil de ce côté étant très délabré, il y avait beaucoup de lézardes, et Gilliatt put y fixer plus de clous encore qu’au soubassement des deux Douvres.\par
 A un moment donné, et comme si l’on eût soufflé dessus, la phosphorescence s’était éteinte ; le crépuscule, d’instant en instant plus lumineux, la remplaçait.\par
Les clous plantés, Gilliatt traîna des poutres, puis des cordes, puis des chaînes, et, sans détourner les yeux de son travail, sans se distraire un instant, il se mit à construire en travers du goulet de l’Homme, avec des madriers fixés horizontalement et rattachés par des câbles, un de ces barrages à claire-voie que la science aujourd’hui a adoptés et qu’elle qualifie brise-lames.\par
Ceux qui ont vu, par exemple, à la Rocquaine à Guernesey, ou au Bourgdault en France, l’effet que font quelques pieux plantés dans le rocher, comprennent la puissance de ces ajustages si simples. Le brise-lames est la combinaison de ce qu’on nomme en France épi avec ce qu’on nomme en Angleterre dick. Les brise-lames sont les chevaux de frise des fortifications contre les tempêtes. On ne peut lutter contre la mer qu’en tirant parti de la divisibilité de cette force.\par
Cependant le soleil s’était levé, parfaitement pur. Le ciel était clair, la mer était calme.\par
Gilliatt pressait son travail. Il était calme lui aussi, mais dans sa hâte il y avait de l’anxiété.\par
Il allait, à grandes enjambées de roche en roche, du barrage au magasin et du magasin au barrage. Il revenait tirant éperdument, tantôt une porque, tantôt une hiloire. L’utilité de cet en-cas de charpentes se manifesta. Il était évident que Gilliatt était en face d’une éventualité prévue.\par
 Une forte barre de fer lui servait de levier pour remuer les poutres.\par
Le travail s’exécutait si vite que c’était plutôt une croissance qu’une construction. Qui n’a pas vu à l’œuvre un pontonnier militaire ne peut se faire une idée de cette rapidité.\par
Le goulet de l’est était plus étroit encore que le goulet de l’ouest. Il n’avait que cinq ou six pieds d’entre-bâillement. Ce peu d’ouverture aidait Gilliatt. L’espace à fortifier et à fermer étant très restreint, l’armature serait plus solide et pourrait être plus simple. Ainsi des solives horizontales suffisaient ; les pièces debout étaient inutiles.\par
Les premières traverses du brise-lames posées, Gilliatt monta dessus et écouta.\par
Le grondement devenait expressif.\par
Gilliatt continua sa construction. Il la contrebuta avec les deux bossoirs de la Durande reliés à l’enchevêtrement des solives par des drisses passées dans leurs trois roues de poulies. Il noua le tout avec des chaînes.\par
Cette construction n’était autre chose qu’une sorte de claie colossale, ayant des madriers pour baguettes et des chaînes pour osiers.\par
Cela semblait tressé autant que bâti.\par
Gilliatt multiplia les attaches, et ajouta des clous où il le fallait.\par
Ayant eu beaucoup de fer rond dans l’épave, il avait pu faire de ces clous une grosse provision.\par
Tout en travaillant, il broyait du biscuit entre ses  dents. Il avait soif, mais ne pouvait boire, n’ayant plus d’eau douce. Il avait vidé le bidon la veille à son souper.\par
Il échafauda encore quatre ou cinq charpentes, puis monta de nouveau sur le barrage. Il écouta.\par
Le bruit à l’horizon avait cessé. Tout se taisait.\par
La mer était douce et superbe ; elle méritait tous les madrigaux que lui adressent les bourgeois quand ils sont contents d’elle, — « un miroir », — « un lac », — « de l’huile », — « une plaisanterie », — « un mouton ». — Le bleu profond du ciel répondait au vert profond de l’océan. Ce saphir et cette émeraude pouvaient s’admirer l’un l’autre. Ils n’avaient aucun reproche à se faire. Pas un nuage en haut, pas une écume en bas. Dans toute cette splendeur montait magnifiquement le soleil d’avril. Il était impossible de voir un plus beau temps.\par
A l’extrême horizon une longue file noire d’oiseaux de passage rayait le ciel. Ils allaient vite. Ils se dirigeaient vers la terre. Il semblait qu’il y eût de la fuite dans leur vol.\par
Gilliatt se remit à exhausser le brise-lames.\par
Il l’éleva le plus haut qu’il put, aussi haut que le lui permit la courbure des rochers.\par
Vers midi, le soleil lui sembla plus chaud qu’il ne devait l’être. Midi est l’heure critique du jour ; Gilliatt, debout sur la robuste claire-voie qu’il achevait de bâtir, se remit à considérer l’étendue.\par
La mer était plus que tranquille, elle était stagnante. On n’y voyait pas une voile. Le ciel était  partout limpide ; seulement de bleu il était devenu blanc. Ce blanc était singulier. Il y avait à l’ouest sur l’horizon une petite tache d’apparence malsaine. Cette tache restait immobile à la même place, mais grandissait. Près des brisants, le flot frissonnait très doucement.\par
Gilliatt avait bien fait de bâtir son brise-lames.\par
Une tempête approchait.\par
L’abîme se décidait à livrer bataille.\par
  \subsection[{B.III. Livre troisième. La lutte}]{B.III. Livre troisième \\
La lutte}
  \subsubsection[{B.III.1. L’extrême touche l’extrême, et le contraire annonce le contraire}]{B.III.1. \\
L’extrême touche l’extrême, et le contraire annonce le contraire}
\noindent Rien n’est menaçant comme l’équinoxe en retard.\par
Il y a sur la mer un phénomène farouche qu’on pourrait appeler l’arrivée des vents du large.\par
En toute saison, particulièrement à l’époque des syzygies, à l’instant où l’on doit le moins s’y attendre, la mer est prise soudain d’une tranquillité étrange. Ce prodigieux mouvement perpétuel s’apaise ; il a de l’assoupissement ; il entre en langueur ; il semble qu’il va se donner relâche ; on pourrait le croire fatigué. Tous les chiffons marins, depuis le guidon de pêche jusqu’aux enseignes de guerre, pendent le long des mâts. Les pavillons amiraux, royaux, impériaux, dorment.\par
Tout à coup ces loques se mettent à remuer discrètement.\par
C’est le moment, s’il y a des nuages, d’épier la formation des cirrus ; si le soleil se couche, d’examiner  la rougeur du soir ; s’il fait nuit et s’il y a lune, d’étudier les halos.\par
Dans cette minute-là, le capitaine ou le chef d’escadre qui a la chance de posséder un de ces verres-de-tempête dont l’inventeur est inconnu, observe ce verre au microscope et prend ses précautions contre le vent du sud si la mixture a un aspect de sucre fondu, et contre le vent du nord si la mixture s’exfolie en cristallisations pareilles à des fourrés de fougères ou à des bois de sapins. Dans cette minute-là, après avoir consulté quelque gnomon mystérieux gravé par les romains, ou par les démons, sur une de ces énigmatiques pierres droites qu’on appelle en Bretagne menhir et en Irlande cruach, le pauvre pêcheur irlandais ou breton retire sa barque de la mer.\par
Cependant la sérénité du ciel et de l’océan persiste. Le matin se lève radieux et l’aurore sourit ; ce qui remplissait d’horreur religieuse les vieux devins, épouvantés qu’on pût croire à la fausseté du soleil. \emph{Solem quis dicere falsum audeat ?}\par
La sombre vision du possible latent est interceptée à l’homme par l’opacité fatale des choses. Le plus redoutable et le plus perfide des aspects, c’est le masque de l’abîme.\par
On dit : anguille sous roche ; on devrait dire : tempête sous calme.\par
Quelques heures, quelques jours parfois, se passent ainsi. Les pilotes braquent leurs longues-vues çà et là. Le visage des vieux marins a un air de sévérité qui tient à la colère secrète de l’attente.\par
 Subitement on entend un grand murmure confus. Il y a une sorte de dialogue mystérieux dans l’air.\par
On ne voit rien.\par
L’étendue demeure impassible.\par
Cependant le bruit s’accroît, grossit, s’élève. Le dialogue s’accentue.\par
Il y a quelqu’un derrière l’horizon.\par
Quelqu’un de terrible, le vent.\par
Le vent, c’est-à-dire cette populace de titans que nous appelons les Souffles.\par
L’immense canaille de l’ombre.\par
L’Inde les nommait les Marouts, la Judée les Kéroubims, la Grèce les Aquilons. Ce sont les invisibles oiseaux fauves de l’infini. Ces borées accourent.
 \subsubsection[{B.III.2. Les vents du large}]{B.III.2. \\
Les vents du large}
\noindent D’où viennent-ils ? De l’incommensurable. Il faut à leurs envergures le diamètre du gouffre. Leurs ailes démesurées ont besoin du recul indéfini des solitudes. L’Atlantique, le Pacifique, ces vastes ouvertures bleues, voilà ce qui leur convient. Ils les font sombres. Ils y volent en troupes. Le commandant Page a vu une fois sur la haute mer sept trombes à la fois. Ils sont là, farouches. Ils préméditent les désastres. Ils ont pour labeur l’enflure éphémère et éternelle du flot. Ce qu’ils peuvent est ignoré, ce qu’ils veulent est inconnu. Ils sont les sphinx de l’abîme, et Gama est leur Œdipe. Dans cette obscurité de l’étendue qui remue toujours, ils apparaissent, faces de nuées. Qui aperçoit leurs linéaments livides dans cette dispersion qui est l’horizon de la mer, se sent en présence de la force irréductible. On dirait que l’intelligence humaine les inquiète, et ils se hérissent contre elle. L’intelligence est invincible, mais l’élément est imprenable. Que faire contre  l’ubiquité insaisissable ? Le souffle se fait massue, puis redevient souffle. Les vents combattent par l’écrasement et se défendent par l’évanouissement. Qui les rencontre est aux expédients. Leur assaut, divers et plein de répercussions, déconcerte. Ils ont autant de fuite que d’attaque. Ils sont les impalpables tenaces. Comment en venir à bout ? La proue du navire Argo, sculptée dans un chêne de Dodone, à la fois proue et pilote, leur parlait. Ils brutalisaient cette proue déesse. Christophe Colomb, les voyant venir vers \emph{la Pinta}, montait sur le pont et leur adressait les premiers versets de l’évangile selon saint Jean. Surcouf les insultait. \emph{Voici la clique}, disait-il. Napier leur tirait des coups de canon. Ils ont la dictature du chaos.\par
Ils ont le chaos. Qu’en font-ils ? On ne sait quoi d’implacable. La fosse aux vents est plus monstrueuse que la fosse aux lions. Que de cadavres sous ces plis sans fond ! Les vents poussent sans pitié la grande masse obscure et amère. On les entend toujours, eux ils n’écoutent rien. Ils commettent des choses qui ressemblent à des crimes. On ne sait sur qui ils jettent les arrachements blancs de l’écume. Que de férocité impie dans le naufrage ! quel affront à la providence ! Ils ont l’air par moment de cracher sur Dieu. Ils sont les tyrans des lieux inconnus. \emph{Luoghi spaventosi}, murmuraient les marins de Venise.\par
Les espaces frémissants subissent leurs voies de fait. Ce qui se passe dans ces grands abandons est inexprimable. Quelqu’un d’équestre est mêlé à l’ombre. L’air fait un bruit de forêt. On n’aperçoit rien, et l’on  entend des cavaleries. Il est midi, tout à coup il fait nuit, un tornado passe ; il est minuit, tout à coup il fait jour, l’effluve polaire s’allume. Des tourbillons alternent en sens inverse, sorte de danse hideuse, trépignement des fléaux sur l’élément. Un nuage trop lourd se casse par le milieu, et tombe en morceaux dans la mer. D’autres nuages, pleins de pourpre, éclairent et grondent, puis s’obscurcissent lugubrement ; le nuage vidé de foudre noircit, c’est un charbon éteint. Des sacs de pluie se crèvent en brume. Là une fournaise où il pleut ; là une onde d’où se dégage un flamboiement. Les blancheurs de la mer sous l’averse éclairent des lointains surprenants ; on voit se déformer des épaisseurs où errent des ressemblances. Des nombrils monstrueux creusent les nuées. Les vapeurs tournoient, les vagues pirouettent ; les naïades ivres roulent ; à perte de vue, la mer massive et molle se meut sans se déplacer ; tout est livide ; des cris désespérés sortent de cette pâleur.\par
Au fond de l’obscurité inaccessible, de grandes gerbes d’ombre frissonnent. Par moments, il y a paroxysme. La rumeur devient tumulte, de même que la vague devient houle. L’horizon, superposition confuse de lames, oscillation sans fin, murmure en basse continue ; des jets de fracas y éclatent bizarrement ; on croit entendre éternuer des hydres. Des souffles froids surviennent, puis des souffles chauds. La trépidation de la mer annonce une épouvante qui s’attend à tout. Inquiétude. Angoisse. Terreur profonde des eaux. Subitement, l’ouragan, comme une bête, vient boire à  l’océan ; succion inouïe ; l’eau monte vers la bouche invisible, une ventouse se forme, la tumeur enfle ; c’est la trombe, le Prester des anciens, stalactite en haut, stalagmite en bas, double cône inverse tournant, une pointe en équilibre sur l’autre, baiser de deux montagnes, une montagne d’écume qui s’élève, une montagne de nuée qui descend ; effrayant coït de l’onde et de l’ombre. La trombe, comme la colonne de la bible, est ténébreuse le jour et lumineuse la nuit. Devant la trombe le tonnerre se tait. Il semble qu’il ait peur.\par
Le vaste trouble des solitudes a une gamme ; crescendo redoutable : le grain, la rafale, la bourrasque, l’orage, la tourmente, la tempête, la trombe ; les sept cordes de la lyre des vents, les sept notes de l’abîme. Le ciel est une largeur, la mer est une rondeur ; une haleine passe, il n’y a plus rien de tout cela, tout est furie et pêle-mêle.\par
Tels sont ces lieux sévères.\par
Les vents courent, volent, s’abattent, finissent, recommencent, planent, sifflent, mugissent, rient ; frénétiques, lascifs, effrénés, prenant leurs aises sur la vague irascible. Ces hurleurs ont une harmonie. Ils font tout le ciel sonore. Ils soufflent dans la nuée comme dans un cuivre, ils embouchent l’espace ; et ils chantent dans l’infini, avec toutes les voix amalgamées des clairons, des buccins, des olifants, des bugles et des trompettes, une sorte de fanfare prométhéenne. Qui les entend écoute Pan. Ce qu’il y a d’effroyable, c’est qu’ils jouent. Ils ont une colossale joie composée  d’ombre. Ils font dans les solitudes la battue des navires. Sans trêve, jour et nuit, en toute saison, au tropique comme au pôle, en sonnant dans leur trompe éperdue, ils mènent, à travers les enchevêtrements de la nuée et de la vague, la grande chasse noire des naufrages. Ils sont des maîtres de meutes. Ils s’amusent. Ils font aboyer après les roches les flots, ces chiens. Ils combinent les nuages, et les désagrègent. Ils pétrissent, comme avec des millions de mains, la souplesse de l’eau immense.\par
L’eau est souple parce qu’elle est incompressible. Elle glisse sous l’effort. Chargée d’un côté, elle échappe de l’autre. C’est ainsi que l’eau se fait l’onde. La vague est sa liberté.
 \subsubsection[{B.III.3. Explication du bruit écouté par gilliatt}]{B.III.3. \\
Explication du bruit écouté par gilliatt}
\noindent La grande venue des vents vers la terre se fait aux équinoxes. A ces époques la balance du tropique et du pôle bascule, et la colossale marée atmosphérique verse son flux sur un hémisphère et son reflux sur l’autre. Il y a des constellations qui signifient ces phénomènes, la Balance, le Verseau.\par
C’est l’heure des tempêtes.\par
La mer attend, et garde le silence.\par
Quelquefois le ciel a mauvaise mine. Il est blafard, une grande panne obscure l’obstrue. Les marins regardent avec anxiété l’air fâché de l’ombre.\par
Mais c’est son air satisfait qu’ils redoutent le plus. Un ciel riant d’équinoxe, c’est l’orage faisant patte de velours. Par ces ciels-là, la Tour des Pleureuses d’Amsterdam s’emplissait de femmes examinant l’horizon.\par
Quand la tempête vernale ou automnale tarde, c’est qu’elle fait un plus gros amas. Elle thésaurise pour le  ravage. Méfiez-vous des arrérages. Ango disait : \emph{La mer est bonne payeuse.}\par
Quand l’attente est trop longue, la mer ne trahit son impatience que par plus de calme. Seulement la tension magnétique se manifeste par ce qu’on pourrait nommer l’inflammation de l’eau. Des lueurs sortent de la vague. Air électrique, eau phosphorique. Les matelots se sentent harassés. Cette minute est particulièrement périlleuse pour les iron-clads ; leur coque de fer peut produire de fausses indications du compas, et les perdre. Le steamer transatlantique l’\emph{Yowa} a péri ainsi.\par
Pour ceux qui sont en familiarité avec la mer, son aspect, dans ces instants-là, est étrange ; on dirait qu’elle désire et craint le cyclone. De certains hyménées, d’ailleurs fort voulus par la nature, sont accueillis de cette façon. La lionne en rut fuit devant le lion. La mer, elle aussi, est en chaleur. De là son tremblement.\par
L’immense mariage va se faire.\par
Ce mariage, comme les noces des anciens empereurs, se célèbre par des exterminations. C’est une fête avec assaisonnement de désastres.\par
Cependant, de là-bas, du large, des latitudes inexpugnables, du livide horizon des solitudes, du fond de la liberté sans bornes, les vents arrivent.\par
Faites attention, voilà le fait équinoxial.\par
Une tempête, cela se complote. La vieille mythologie entrevoyait ces personnalités indistinctes mêlées à la grande nature diffuse. Éole se concerte avec Borée.  L’entente de l’élément avec l’élément est nécessaire. Ils se distribuent la tâche. On a des impulsions à donner à la vague, au nuage, à l’effluve ; la nuit est un auxiliaire, il importe de l’employer. On a des boussoles à dérouter, des fanaux à éteindre, des phares à masquer, des étoiles à cacher. Il faut que la mer coopère. Tout orage est précédé d’un murmure. Il y a derrière l’horizon chuchotement préalable des ouragans.\par
C’est là ce que, dans l’obscurité, au loin, par-dessus le silence effrayé de la mer, on entend.\par
Ce chuchotement redoutable, Gilliatt l’avait entendu. La phosphorescence avait été le premier avertissement ; ce murmure, le second.\par
Si le démon Légion existe, c’est lui, à coup sûr, qui est le Vent.\par
Le vent est multiple, mais l’air est un.\par
De là cette conséquence : tout orage est mixte. L’unité de l’air l’exige.\par
Tout l’abîme est impliqué dans une tempête. L’océan entier est dans une bourrasque. La totalité de ses forces y entre en ligne et y prend part. Une vague, c’est le gouffre d’en haut. Avoir affaire à une tourmente, c’est avoir affaire à toute la mer et à tout le ciel.\par
Messier, l’homme de la marine, l’astronome pensif de la logette de Cluny, disait : \emph{Le vent de partout est partout.} Il ne croyait point aux vents emprisonnés, même dans les mers closes. Il n’y avait point pour lui de vents méditerranéens. Il disait les reconnaître au passage. Il affirmait que tel jour, à telle heure, le Fohn du  lac de Constance, l’antique Favonius de Lucrèce, avait traversé l’horizon de Paris ; tel autre jour le Bora de l’Adriatique ; tel autre jour le Notus giratoire qu’on prétend enfermé dans le rond des Cyclades. Il en spécifiait les effluves. Il ne pensait pas que l’autan qui tourne entre Malte et Tunis et que l’autan qui tourne entre la Corse et les Baléares, fussent dans l’impossibilité de s’échapper. Il n’admettait point qu’il y eût des vents ours dans des cages. Il disait : « Toute pluie vient du tropique et tout éclair vient du pôle. » Le vent en effet se sature d’électricité à l’intersection des colures, qui marque les extrémités de l’axe, et d’eau à l’équateur ; et il nous apporte de la Ligne le liquide et des pôles le fluide.\par
Ubiquité, c’est le vent.\par
Ceci ne veut pas dire, certes, que les zones venteuses n’existent pas. Rien n’est plus démontré que ces afflations à courants continus, et un jour la navigation aérienne, servie par les air-navires que nous nommons, par manie du grec, aéroscaphes, en utilisera les lignes principales. La canalisation de l’air par le vent est incontestable ; il y a des fleuves de vent, des rivières de vent et des ruisseaux de vent ; seulement les embranchements de l’air se font à l’inverse des embranchements de l’eau ; ce sont les ruisseaux qui sortent des rivières et les rivières qui sortent des fleuves, au lieu d’y tomber ; de là, au lieu de la concentration, la dispersion.\par
C’est cette dispersion qui fait la solidarité des vents et l’unité de l’atmosphère. Une molécule déplacée  déplace l’autre. Tout le vent remue ensemble. A ces profondes causes d’amalgame, ajoutez le relief du globe, trouant l’atmosphère par toutes ses montagnes, faisant des nœuds et des torsions dans les courses du vent, et déterminant dans tous les sens des contre-courants. Irradiation illimitée.\par
Le phénomène du vent, c’est l’oscillation de deux océans l’un sur l’autre ; l’océan d’air, superposé à l’océan d’eau, s’appuie sur cette fuite et chancelle sur ce tremblement.\par
L’indivisible ne se met pas dans des compartiments. Il n’y a pas de cloison entre un flot et l’autre. Les îles de la Manche sentent la poussée du cap de Bonne-Espérance. La navigation universelle tient tête à un monstre unique. Toute la mer est la même hydre. Les vagues couvrent la mer d’une sorte de peau de poisson. Océan, c’est Ceto.\par
Sur cette unité s’abat l’innombrable.
 \subsubsection[{B.III.4. Turba, turma}]{B.III.4. \\
Turba, turma}
\noindent Pour le compas, il y a trente-deux vents, c’est-à-dire trente-deux directions ; mais ces directions peuvent se subdiviser indéfiniment. Le vent, classé par directions, c’est l’incalculable ; classé par espèces, c’est l’infini.\par
Homère reculerait devant ce dénombrement.\par
Le courant polaire heurte le courant tropical. Voilà le froid et le chaud combinés, l’équilibre commence par le choc, l’onde des vents en sort, enflée, éparse, et déchiquetée dans tous les sens en ruissellements farouches. La dispersion des souffles secoue aux quatre coins de l’horizon le prodigieux échevèlement de l’air.\par
Tous les rumbs sont là ; le vent du Gulf-Stream qui dégorge tant de brume sur Terre-Neuve, le vent du Pérou, région à ciel muet où jamais l’homme n’a entendu tonner, le vent de la Nouvelle-Écosse où vole le Grand Auk, \emph{Alca impennis,} au bec rayé, les tourbillons de Fer des mers de Chine, le vent de Mozambique qui  malmène les pangaies et les jonques, le vent électrique du Japon dénoncé par le gong, le vent d’Afrique qui habite entre la montagne de la Table et la montagne du Diable et qui se déchaîne de là, le vent de l’équateur qui passe par-dessus les vents alizés et qui trace une parabole dont le sommet est toujours à l’ouest, le vent plutonien qui sort des cratères et qui est le redoutable souffle de la flamme, l’étrange vent propre au volcan Awu qui fait toujours surgir un nuage olivâtre du nord, la mousson de Java, contre laquelle sont construites ces casemates qu’on nomme \emph{maisons d’ouragan,} la bise à embranchements que les anglais appellent \emph{bush,} buisson, les grains arqués du détroit de Malacca observés par Horsburgh, le puissant vent du sud-ouest, nommé Pampero au Chili et Rebojo à Buenos-Ayres, qui emporte le condor en pleine mer et le sauve de la fosse où l’attend, sous une peau de bœuf fraîchement écorché, le sauvage couché sur le dos et bandant son grand arc avec ses pieds, le vent chimique qui, selon Lemery, fait dans la nuée des pierres de tonnerre, l’harmattan des cafres, le chasse-neige polaire, qui s’attelle aux banquises et qui traîne les glaces éternelles, le vent du golfe de Bengale qui va jusqu’à Nijni-Novogorod saccager le triangle de baraques de bois où se tient la foire d’Asie, le vent des Cordillères, agitateur des grandes vagues et des grandes forêts, le vent des archipels d’Australie où les chasseurs de miel dénichent les ruches sauvages cachées sous les aisselles des branches de l’eucalyptus géant, le siroco, le mistral, le hurricane, les vents de sécheresse,  les vents d’inondation, les diluviens, les torrides, ceux qui jettent dans les rues de Gênes la poussière des plaines du Brésil, ceux qui obéissent à la rotation diurne, ceux qui la contrarient et qui font dire à Herrera : \emph{Malo viento torna contra el sol}, ceux qui vont par couples, d’accord pour bouleverser, l’un défaisant ce que fait l’autre, et les vieux vents qui ont assailli Christophe Colomb sur la côte de Veraguas, et ceux qui pendant quarante jours, du 21 octobre au 28 novembre 1520, ont mis en question Magellan abordant le Pacifique, et ceux qui ont démâté l’Armada et soufflé sur Philippe II. D’autres encore, et comment trouver la fin ? Les vents porteurs de crapauds et de sauterelles qui poussent des nuées de bêtes par-dessus l’océan ; ceux qui opèrent ce qu’on appelle « la saute de vent » et qui ont pour fonction d’achever les naufragés, ceux qui, d’un seul coup d’haleine, déplacent la cargaison dans le navire, et le contraignent à continuer sa route penché ; les vents qui construisent les circumcumuli, les vents qui construisent les circumstrati ; les lourds vents aveugles tuméfiés de pluie, les vents de la grêle, les vents de la fièvre, ceux dont l’approche met en ébullition les salses et les solfatares de Calabre, ceux qui font étinceler le poil des panthères d’Afrique rôdant dans les broussailles du cap de Fer, ceux qui viennent secouant hors de leur nuage, comme une langue de trigonocéphale, l’épouvantable éclair à fourche ; ceux qui apportent des neiges noires. Telle est l’armée.\par
L’écueil Douvres, au moment où Gilliatt construisait son brise-lames, en entendait le galop lointain.\par
 Nous venons de le dire, le Vent, c’est tous les vents.\par
Toute cette horde arrivait.\par
D’un côté, cette légion.\par
De l’autre, Gilliatt.
 \subsubsection[{B.III.5. Gilliatt a l’option}]{B.III.5. \\
Gilliatt a l’option}
\noindent Les mystérieuses forces avaient bien choisi le moment.\par
Le hasard, s’il existe, est habile.\par
Tant que la panse avait été remisée dans la crique de l’Homme, tant que la machine avait été emboîtée dans l’épave, Gilliatt était inexpugnable. La panse était en sûreté, la machine était à l’abri ; les Douvres, qui tenaient la machine, la condamnaient à une destruction lente, mais la protégeaient contre une surprise. Dans tous les cas, il restait à Gilliatt une ressource. La machine détruite ne détruisait pas Gilliatt. Il avait la panse pour se sauver.\par
Mais attendre que la panse fût retirée du mouillage où elle était inaccessible, la laisser s’engager dans le défilé des Douvres, patienter jusqu’à ce qu’elle fût prise, elle aussi, par l’écueil, permettre à Gilliatt d’opérer le sauvetage, le glissement et le transbordement de la machine, ne point entraver ce merveilleux  travail qui mettait tout dans la panse, consentir à cette réussite, là était le piège. Là se laissait entrevoir, sorte de linéament sinistre, la sombre ruse de l’abîme.\par
A cette heure, la machine, la panse, Gilliatt, étaient réunis dans la ruelle de rochers. Ils ne faisaient qu’un. La panse broyée à l’écueil, la machine coulée à fond, Gilliatt noyé, c’était l’affaire d’un effort unique sur un seul point. Tout pouvait être fini à la fois, en même temps, et sans dispersion ; tout pouvait être écrasé d’un coup.\par
Pas de situation plus critique que celle de Gilliatt.\par
Le sphinx possible, soupçonné par les rêveurs au fond de l’ombre, semblait lui poser un dilemme.\par
Reste, ou pars.\par
Partir était insensé, rester était effrayant.
 \subsubsection[{B.III.6. Le combat}]{B.III.6. \\
Le combat}
\noindent Gilliatt monta sur la grande Douvre.\par
De là il voyait toute la mer.\par
L’ouest était surprenant. Il en sortait une muraille. Une grande muraille de nuée, barrant de part en part l’étendue, montait lentement de l’horizon vers le zénith. Cette muraille, rectiligne, verticale, sans une crevasse dans sa hauteur, sans une déchirure à son arête, paraissait bâtie à l’équerre et tirée au cordeau. C’était du nuage ressemblant à du granit. L’escarpement de ce nuage, tout à fait perpendiculaire à l’extrémité sud, fléchissait un peu vers le nord comme une tôle ployée, et offrait le vague glissement d’un plan incliné. Ce mur de brume s’élargissait, et croissait sans que son entablement cessât un instant d’être parallèle à la ligne d’horizon, presque indistincte dans l’obscurité tombante. Cette muraille de l’air montait tout d’une pièce en silence. Pas une ondulation, pas un plissement, pas une saillie qui se déformât ou se déplaçât. Cette  immobilité en mouvement était lugubre. Le soleil, blême derrière on ne sait quelle transparence malsaine, éclairait ce linéament d’apocalypse. La nuée envahissait déjà près de la moitié de l’espace. On eût dit l’effrayant talus de l’abîme. C’était quelque chose comme le lever d’une montagne d’ombre entre la terre et le ciel.\par
C’était en plein jour l’ascension de la nuit.\par
Il y avait dans l’air une chaleur de poêle. Une buée d’étuve se dégageait de cet amoncellement mystérieux. Le ciel, qui de bleu était devenu blanc, était de blanc devenu gris. On eût dit une grande ardoise. La mer, dessous, terne et plombée, était une autre ardoise énorme. Pas un souffle, pas un flot, pas un bruit. A perte de vue, la mer déserte. Aucune voile d’aucun côté. Les oiseaux s’étaient cachés. On sentait de la trahison dans l’infini.\par
Le grossissement de toute cette ombre s’amplifiait insensiblement.\par
La montagne mouvante de vapeurs qui se dirigeait vers les Douvres était un de ces nuages qu’on pourrait appeler les nuages de combat. Nuages louches. A travers ces entassements obscurs, on ne sait quel strabisme vous regarde.\par
Cette approche était terrible.\par
Gilliatt examina fixement la nuée et grommela entre ses dents : J’ai soif, tu vas me donner à boire.\par
Il demeura quelques moments immobile, l’œil attaché sur le nuage. On eût dit qu’il toisait la tempête.\par
Sa galérienne était dans la poche de sa vareuse, il  l’en tira et s’en coiffa. Il prit, dans le trou où il avait si longtemps couché, sa réserve de hardes ; il chaussa les jambières et endossa le suroit, comme un chevalier qui revêt son armure au moment de l’action. On sait qu’il n’avait plus de souliers, mais ses pieds nus étaient endurcis aux rochers.\par
Cette toilette de guerre faite, il considéra son brise-lames, empoigna vivement la corde à nœuds, descendit du plateau de la Douvre, prit pied sur les roches d’en bas, et courut à son magasin. Quelques instants après, il était au travail. Le vaste nuage muet put entendre ses coups de marteau. Que faisait Gilliatt ? Avec ce qui lui restait de clous, de cordes et de poutres il construisait au goulet de l’est une seconde claire-voie à dix ou douze pieds en arrière de la première.\par
Le silence était toujours profond. Les brins d’herbe dans les fentes de l’écueil ne bougeaient pas.\par
Brusquement le soleil disparut. Gilliatt leva la tête.\par
La nuée montante venait d’atteindre le soleil. Ce fut comme une extinction du jour, remplacé par une réverbération mêlée et pâle.\par
La muraille de nuée avait changé d’aspect. Elle n’avait plus son unité. Elle s’était froncée horizontalement en touchant au zénith d’où elle surplombait sur le reste du ciel. Elle avait maintenant des étages. La formation de la tempête s’y dessinait comme dans une section de tranchée. On distinguait les couches de la pluie et les gisements de la grêle. Il n’y avait point d’éclair, mais une horrible lueur éparse ; car l’idée  d’horreur peut s’attacher à l’idée de lumière. On entendait la vague respiration de l’orage. Ce silence palpitait obscurément. Gilliatt, silencieux lui aussi, regardait se grouper au-dessus de sa tête tous ces blocs de brume et se composer la difformité des nuages. Sur l’horizon pesait et s’étendait une bande de brouillard couleur cendre, et au zénith une bande couleur plomb ; des guenilles livides pendaient des nuages d’en haut sur les brouillards d’en bas. Tout le fond, qui était le mur de nuages, était blafard, laiteux, terreux, morne, indescriptible. Une mince nuée blanchâtre transversale, arrivée on ne sait d’où, coupait obliquement, du nord au sud, la haute muraille sombre. Une des extrémités de cette nuée traînait dans la mer. Au point où elle touchait la confusion des vagues, on apercevait dans l’obscurité un étouffement de vapeur rouge. Au-dessous de la longue nuée pâle, de petits nuages, très bas, tout noirs, volaient en sens inverse les uns des autres comme s’ils ne savaient que devenir. Le puissant nuage du fond croissait de toutes parts à la fois, augmentait l’éclipse, et continuait son interposition lugubre. Il n’y avait plus, à l’est, derrière Gilliatt, qu’un porche de ciel clair qui allait se fermer. Sans qu’on eût l’impression d’aucun vent, une étrange diffusion de duvet grisâtre passa, éparpillée et émiettée, comme si quelque gigantesque oiseau venait d’être plumé derrière ce mur de ténèbres. Il s’était formé un plafond de noirceur compacte qui, à l’extrême horizon, touchait la mer et s’y mêlait dans de la nuit. On sentait quelque chose qui avance. C’était vaste et  lourd, et farouche. L’obscurité s’épanouissait. Tout à coup un immense tonnerre éclata.\par
Gilliatt lui-même ressentit la secousse. Il y a du songe dans le tonnerre. Cette réalité brutale dans la région visionnaire a quelque chose de terrifiant. On croit entendre la chute d’un meuble dans la chambre des géants.\par
Aucun flamboiement électrique n’accompagna le coup. Ce fut comme un tonnerre noir. Le silence se refit. Il y eut une sorte d’intervalle comme lorsqu’on prend position. Puis apparurent, l’un après l’autre et lentement, de grands éclairs informes. Ces éclairs étaient muets. Pas de grondement. A chaque éclair tout s’illuminait. Le mur de nuages était maintenant un antre. Il y avait des voûtes et des arches. On y distinguait des silhouettes. Des têtes monstrueuses s’ébauchaient ; des cous semblaient se tendre ; des éléphants portant leurs tours, entrevus, s’évanouissaient.\par
Une colonne de brume, droite, ronde et noire, surmontée d’une vapeur blanche, simulait la cheminée d’un steamer colossal englouti, chauffant sous la vague et fumant. Des nappes de nuée ondulaient. On croyait voir des plis de drapeaux. Au centre, sous des épaisseurs vermeilles, s’enfonçait, immobile, un noyau de brouillard dense, inerte, impénétrable aux étincelles électriques, sorte de fœtus hideux dans le ventre de la tempête.\par
Gilliatt subitement sentit qu’un souffle l’échevelait. Trois ou quatre larges araignées de pluie s’écrasèrent  autour de lui sur la roche. Puis il y eut un second coup de foudre. Le vent se leva.\par
L’attente de l’ombre était au comble ; le premier coup de tonnerre avait remué la mer, le deuxième fêla la muraille de nuée du haut en bas, un trou se fit, toute l’ondée en suspens versa de ce côté, la crevasse devint comme une bouche ouverte pleine de pluie, et le vomissement de la tempête commença.\par
L’instant fut effroyable.\par
Averse, ouragan, fulgurations, fulminations, vagues jusqu’aux nuages, écume, détonations, torsions frénétiques, cris, rauquements, sifflements, tout à la fois. Déchaînement de monstres.\par
Le vent soufflait en foudre. La pluie ne tombait pas, elle croulait.\par
Pour un pauvre homme, engagé, comme Gilliatt, avec une barque chargée, dans un entre-deux de rochers en pleine mer, pas de crise plus menaçante. Le danger de la marée, dont Gilliatt avait triomphé, n’était rien près du danger de la tempête. Voici quelle était la situation :\par
Gilliatt, autour de qui tout était précipice, démasquait, à la dernière minute et devant le péril suprême, une stratégie savante. Il avait pris son point d’appui chez l’ennemi même ; il s’était associé l’écueil ; le rocher Douvres, autrefois son adversaire, était maintenant son second dans cet immense duel. Gilliatt l’avait mis sous lui. De ce sépulcre, Gilliatt avait fait sa forteresse. Il s’était crénelé dans cette masure formidable de la mer. Il y était bloqué, mais muré. Il  était, pour ainsi dire, adossé à l’écueil, face à face avec l’ouragan. Il avait barricadé le détroit, cette rue des vagues. C’était du reste la seule chose à faire. Il semble que l’océan, qui est un despote, puisse être, lui aussi, mis à la raison par des barricades. La panse pouvait être considérée comme en sûreté de trois côtés. Étroitement resserrée entre les deux façades intérieures de l’écueil, affourchée en patte d’oie, elle était abritée au nord par la petite Douvre, au sud par la grande, escarpements sauvages, plus habitués à faire des naufrages qu’à en empêcher. A l’ouest elle était protégée par le tablier de poutres amarré et cloué aux rochers, barrage éprouvé qui avait vaincu le rude flux de la haute mer, véritable porte de citadelle ayant pour chambranles les colonnes mêmes de l’écueil, les deux Douvres. Rien à craindre de ce côté-là. C’est à l’est qu’était le danger.\par
A l’est il n’y avait que le brise-lames. Un brise-lames est un appareil de pulvérisation. Il lui faut au moins deux claires-voies. Gilliatt n’avait eu le temps que d’en construire une. Il bâtissait la seconde sous la tempête même.\par
Heureusement le vent arrivait du nord-ouest. La mer fait des maladresses. Ce vent, qui est l’ancien vent de galerne, avait peu d’effet sur les roches Douvres. Il assaillait l’écueil en travers, et ne poussait le flot ni sur l’un, ni sur l’autre des deux goulets du défilé, de sorte qu’au lieu d’entrer dans une rue, il se heurtait à une muraille. L’orage avait mal attaqué.\par
Mais les attaques du vent sont courbes, et il fallait  s’attendre à quelque virement subit. Si ce virement se faisait à l’est avant que la deuxième claire-voie du brise-lames fût construite, le péril serait grand. L’envahissement de la ruelle de rochers par la tempête s’accomplirait, et tout était perdu.\par
L’étourdissement de l’orage allait croissant. Toute la tempête est coup sur coup. C’est là sa force ; c’est aussi là son défaut. A force d’être une rage, elle donne prise à l’intelligence, et l’homme se défend ; mais sous quel écrasement ! Rien n’est plus monstrueux. Nul répit, pas d’interruption, pas de trêve, pas de reprise d’haleine. Il y a on ne sait quelle lâcheté dans cette prodigalité de l’inépuisable. On sent que c’est le poumon de l’infini qui souffle.\par
Toute l’immensité en tumulte se ruait sur l’écueil Douvres. On entendait des voix sans nombre. Qui donc crie ainsi ? L’antique épouvante panique était là. Par moments, cela avait l’air de parler, comme si quelqu’un faisait un commandement. Puis des clameurs, des clairons, des trépidations étranges, et ce grand hurlement majestueux que les marins nomment \emph{appel de l’océan.} Les spirales indéfinies et fuyantes du vent sifflaient en tordant le flot ; les vagues, devenues disques sous ces tournoiements, étaient lancées contre les brisants comme des palets gigantesques par des athlètes invisibles. L’énorme écume échevelait toutes les roches. Torrents en haut, baves en bas. Puis les mugissements redoublaient. Aucune rumeur humaine ou bestiale ne saurait donner l’idée des fracas mêlés à ces dislocations de la mer. La nuée canonnait, les grêlons  mitraillaient, la houle escaladait. De certains points semblaient immobiles ; sur d’autres le vent faisait vingt toises par seconde. La mer à perte de vue était blanche ; dix lieues d’eau de savon emplissaient l’horizon. Des portes de feu s’ouvraient. Quelques nuages paraissaient brûlés par les autres, et, sur des tas de nuées rouges qui ressemblaient à des braises, ils ressemblaient à des fumées. Des configurations flottantes se heurtaient et s’amalgamaient, se déformant les unes par les autres. Une eau incommensurable ruisselait. On entendait des feux de peloton dans le firmament. Il y avait au milieu du plafond d’ombre une espèce de vaste hotte renversée d’où tombaient pêle-mêle la trombe, la grêle, les nuées, les pourpres, les phosphores, la nuit, la lumière, les foudres, tant ces penchements du gouffre sont formidables !\par
Gilliatt semblait n’y pas faire attention. Il avait la tête baissée sur son travail. La deuxième claire-voie commençait à s’exhausser. A chaque coup de tonnerre il répondait par un coup de marteau. On entendait cette cadence dans ce chaos. Il était nu-tête. Une rafale lui avait emporté sa galérienne.\par
Sa soif était ardente. Il avait probablement la fièvre. Des flaques de pluie s’étaient formées autour de lui dans des trous de rochers. De temps en temps il prenait de l’eau dans le creux de sa main et buvait. Puis, sans même examiner où en était l’orage, il se remettait à la besogne.\par
Tout pouvait dépendre d’un instant. Il savait ce qui l’attendait s’il ne terminait pas à temps son brise- lames. A quoi bon perdre une minute à regarder s’approcher la face de la mort ?\par
Le bouleversement autour de lui était comme une chaudière qui bout. Il y avait du fracas et du tapage. Par instants la foudre semblait descendre un escalier. Les percussions électriques revenaient sans cesse aux mêmes pointes de rocher, probablement veinées de diorite. Il y avait des grêlons gros comme le poing. Gilliatt était forcé de secouer les plis de sa vareuse. Jusqu’à ses poches étaient pleines de grêle.\par
La tourmente était maintenant ouest, et battait le barrage des deux Douvres ; mais Gilliatt avait confiance en ce barrage, et avec raison. Ce barrage, fait du grand morceau de l’avant de la Durande, recevait sans dureté le choc du flot ; l’élasticité est une résistance ; les calculs de Stevenson établissent que, contre la vague, élastique elle-même, un assemblage de bois, d’une dimension voulue, rejointoyé et enchaîné d’une certaine façon, fait meilleur obstacle qu’un breack-water de maçonnerie. Le barrage des Douvres remplissait ces conditions ; il était d’ailleurs si ingénieusement amarré que la lame, en frappant dessus, était comme le marteau qui enfonce le clou, et l’appuyait au rocher et le consolidait ; pour le démolir, il eût fallu renverser les Douvres. La rafale, en effet, ne réussissait qu’à envoyer à la panse, par-dessus l’obstacle, quelques jets de bave. De ce côté, grâce au barrage, la tempête avortait en crachement. Gilliatt tournait le dos à cet effort-là. Il sentait tranquillement derrière lui cette rage inutile.\par
 Les flocons d’écume, volant de toutes parts, ressemblaient à de la laine. L’eau vaste et irritée noyait les rochers, montait dessus, entrait dedans, pénétrait dans le réseau des fissures intérieures, et ressortait des masses granitiques par des fentes étroites, espèces de bouches intarissables qui faisaient dans ce déluge de petites fontaines paisibles. Çà et là des filets d’argent tombaient gracieusement de ces trous dans la mer.\par
La claire-voie de renfort du barrage de l’est s’achevait. Encore quelques nœuds de corde et de chaîne, et le moment approchait où cette clôture pourrait à son tour lutter.\par
Subitement, une grande clarté se fit, la pluie discontinua, les nuées se désagrégèrent, le vent venait de sauter, une sorte de haute fenêtre crépusculaire s’ouvrit au zénith, et les éclairs s’éteignirent ; on put croire à la fin. C’était le commencement.\par
La saute de vent était du sud-ouest au nord-est.\par
La tempête allait reprendre, avec une nouvelle troupe d’ouragans. Le nord allait donner, assaut violent. Les marins nomment cette reprise redoutée \emph{la rafale de la renverse}. Le vent du sud a plus d’eau, le vent du nord a plus de foudre.\par
L’agression maintenant, venant de l’est, allait s’adresser au point faible.\par
Cette fois Gilliatt se dérangea de son travail, il regarda.\par
Il se plaça debout sur une saillie de rocher en surplomb derrière la deuxième claire-voie presque  terminée. Si la première claie du brise-lames était emportée, elle défoncerait la seconde, pas consolidée encore, et, sous cette démolition, elle écraserait Gilliatt. Gilliatt, à la place qu’il venait de choisir, serait broyé avant de voir la panse et la machine et toute son œuvre s’abîmer dans cet engouffrement. Telle était l’éventualité. Gilliatt l’acceptait, et, terrible, la voulait.\par
Dans ce naufrage de toutes ses espérances, mourir d’abord, c’est ce qu’il lui fallait ; mourir le premier ; car la machine lui faisait l’effet d’une personne. Il releva de sa main gauche ses cheveux collés sur ses yeux par la pluie, étreignit à pleine poignée son bon marteau, se pencha en arrière, menaçant lui-même, et attendit.\par
Il n’attendit pas longtemps.\par
Un éclat de foudre donna le signal, l’ouverture pâle du zénith se ferma, une bouffée d’averse se précipita, tout redevint obscur, et il n’y eut plus de flambeau que l’éclair. La sombre attaque arrivait.\par
Une puissante houle, visible dans les coups sur coups de l’éclair, se leva à l’est au delà du rocher l’Homme. Elle ressemblait à un gros rouleau de verre. Elle était glauque et sans écume et barrait toute la mer. Elle avançait vers le brise-lames. En approchant, elle s’enflait ; c’était on ne sait quel large cylindre de ténèbres roulant sur l’océan. Le tonnerre grondait sourdement.\par
Cette houle atteignit le rocher l’Homme, s’y cassa en deux, et passa outre. Les deux tronçons rejoints  ne firent plus qu’une montagne d’eau, et, de parallèle qu’elle était au brise-lames, elle y devint perpendiculaire. C’était une vague qui avait la forme d’une poutre.\par
Ce bélier se jeta sur le brise-lames. Le choc fut rugissant. Tout s’effaça dans de l’écume.\par
On ne peut se figurer, si on ne les a vues, ces avalanches de neige que la mer s’ajoute, et sous lesquelles elle engloutit des rochers de plus de cent pieds de haut, tels, par exemple, que le Grand Anderlo à Guernesey et le Pinacle à Jersey. A Sainte-Marie de Madagascar, elle saute par-dessus la pointe de Tintingue.\par
Pendant quelques instants, le paquet de mer aveugla tout. Il n’y eut plus rien de visible qu’un entassement furieux, une bave démesurée, la blancheur du linceul tournoyant au vent du sépulcre, un amas de bruit et d’orage sous lequel l’extermination travaillait.\par
L’écume se dissipa. Gilliatt était debout.\par
Le barrage avait tenu bon. Pas une chaîne rompue, pas un clou déplanté. Le barrage avait montré sous l’épreuve les deux qualités du brise-lames ; il avait été souple comme une claie et solide comme un mur. La houle s’y était dissoute en pluie.\par
Un ruissellement d’écume, glissant le long des zigzags du détroit, alla mourir sous la panse.\par
L’homme qui avait fait cette muselière à l’océan ne se reposa pas.\par
L’orage heureusement divagua pendant quelque temps. L’acharnement des vagues revint aux parties murées de l’écueil. Ce fut un répit. Gilliatt en profita pour compléter la claire-voie d’arrière.\par
 La journée s’acheva dans ce labeur. La tourmente continuait ses violences sur le flanc de l’écueil avec une solennité lugubre. L’urne d’eau et l’urne de feu qui sont dans les nuées se versaient sans se vider. Les ondulations hautes et basses du vent ressemblaient aux mouvements d’un dragon.\par
Quand la nuit vint, elle y était déjà ; on ne s’en aperçut pas.\par
Du reste, ce n’était point l’obscurité complète. Les tempêtes, illuminées et aveuglées par l’éclair, ont des intermittences de visible et d’invisible. Tout est blanc, puis tout est noir. On assiste à la sortie des visions et à la rentrée des ténèbres.\par
Une zone de phosphore, rouge de la rougeur boréale, flottait comme un haillon de flamme spectrale derrière les épaisseurs de nuages. Il en résultait un vaste blêmissement. Les largeurs de la pluie étaient lumineuses.\par
Ces clartés aidaient Gilliatt et le dirigeaient. Une fois il se tourna et dit à l’éclair : Tiens-moi la chandelle.\par
Il put, à cette lueur, exhausser la claire-voie d’arrière plus haut encore que la claire-voie d’avant. Le brise-lames se trouva presque complet. Comme Gilliatt amarrait à l’étrave culminante un câble de renfort, la bise lui souffla en plein dans le visage. Ceci lui fit dresser la tête. Le vent s’était brusquement replacé au nord-est. L’assaut du goulet de l’est recommençait. Gilliatt jeta les yeux au large. Le brise-lames allait être encore assailli. Un nouveau coup de mer venait.\par
Cette lame fut rudement assenée ; une deuxième  la suivit, puis une autre et une autre encore, cinq ou six en tumulte, presque ensemble ; enfin une dernière, épouvantable.\par
Celle-ci, qui était comme un total de forces, avait on ne sait quelle figure d’une chose vivante. Il n’aurait pas été malaisé d’imaginer dans cette intumescence et dans cette transparence des aspects d’ouïes et de nageoires. Elle s’aplatit et se broya sur le brise-lames. Sa forme presque animale s’y déchira dans un rejaillissement. Ce fut, sur ce bloc de rochers et de charpentes, quelque chose comme le vaste écrasement d’une hydre. La houle en mourant dévastait. Le flot paraissait se cramponner et mordre. Un profond tremblement remua l’écueil. Des grognements de bête s’y mêlaient. L’écume ressemblait à la salive d’un léviathan.\par
L’écume retombée laissa voir un ravage. Cette dernière escalade avait fait de la besogne. Cette fois le brise lames avait souffert. Une longue et lourde poutre, arrachée de la claire-voie d’avant, avait été lancée par-dessus le barrage d’arrière, sur la roche en surplomb choisie un moment par Gilliatt pour poste de combat. Par bonheur, il n’y était point remonté. Il eût été tué roide.\par
Il y eut dans la chute de ce poteau une singularité, qui, en empêchant le madrier de rebondir, sauva Gilliatt des ricochets et des contre-coups. Elle lui fut même utile encore, comme on va le voir, d’une autre façon.\par
Entre la roche en saillie et l’escarpement intérieur  du défilé, il y avait un intervalle, un grand hiatus assez semblable à l’entaille d’une hache ou à l’alvéole d’un coin. Une des extrémités du madrier jeté en l’air par le flot s’était en tombant engagée dans cet hiatus. L’hiatus s’en était élargi.\par
Une idée vint à Gilliatt.\par
Peser sur l’autre extrémité.\par
Le madrier, pris par un bout dans la fente du rocher qu’il avait agrandie, en sortait droit comme un bras tendu. Cette espèce de bras s’allongeait parallèlement à la façade intérieure du défilé, et l’extrémité libre du madrier s’éloignait de ce point d’appui d’environ dix-huit ou vingt pouces. Bonne distance pour l’effort à faire.\par
Gilliatt s’arc-bouta des pieds, des genoux et des poings à l’escarpement et s’adossa des deux épaules au levier énorme. La poutre était longue ; ce qui augmentait la puissance de la pesée. La roche était déjà ébranlée. Pourtant Gilliatt dut s’y reprendre à quatre fois. Il lui ruisselait des cheveux autant de sueur que de pluie. Le quatrième effort fut frénétique. Il y eut un rauquement dans le rocher, l’hiatus prolongé en fissure s’ouvrit comme une mâchoire, et la lourde masse tomba dans l’étroit entre-deux du défilé avec un bruit terrible, réplique aux coups de foudre.\par
Elle tomba droite, si cette expression est possible, c’est-à-dire sans se casser.\par
Qu’on se figure un menhir précipité tout d’une pièce.\par
La poutre-levier suivit le rocher, et Gilliatt, tout  cédant à la fois sous lui, faillit lui-même tomber.\par
Le fond était très comblé de galets en cet endroit et il y avait peu d’eau. Le monolithe, dans un clapotement d’écume, qui éclaboussa Gilliatt, se coucha entre les deux grandes roches parallèles du défilé et fit une muraille transversale, sorte de trait d’union des deux escarpements. Ses deux bouts touchaient ; il était un peu trop long, et son sommet qui était de roche mousse s’écrasa en s’emboîtant. Il résulta de cette chute un cul-de-sac singulier, qu’on peut voir encore aujourd’hui. L’eau, derrière cette barre de pierre, est presque toujours tranquille.\par
C’était là un rempart plus invincible encore que le panneau de l’avant de la Durande ajusté entre les deux Douvres.\par
Ce barrage intervint à propos.\par
Les coups de mer avaient continué. La vague s’opiniâtre toujours sur l’obstacle. La première claire-voie entamée commençait à se désarticuler. Une maille défaite à un brise-lames est une grave avarie. L’élargissement du trou est inévitable, et nul moyen d’y remédier sur place. La houle emporterait le travailleur.\par
Une décharge électrique, qui illumina l’écueil, dévoila à Gilliatt le dégât qui se faisait dans le brise-lames, les poutres déjetées, les bouts de corde et les bouts de chaîne commençant à jouer dans le vent, une déchirure au centre de l’appareil. La deuxième claire-voie était intacte.\par
Le bloc de pierre, si puissamment jeté par Gilliatt  dans l’entre-deux derrière le brise-lames, était la plus solide des barrières, mais avait un défaut ; il était trop bas. Les coups de mer ne pouvaient le rompre, mais pouvaient le franchir.\par
Il ne fallait point songer à l’exhausser. Des masses rocheuses seules pouvaient être utilement superposées à ce barrage de pierre ; mais comment les détacher, comment les traîner, comment les soulever, comment les étager, comment les fixer ? On ajoute des charpentes, on n’ajoute pas des rochers.\par
Gilliatt n’était pas Encelade.\par
Le peu d’élévation de ce petit isthme de granit préoccupait Gilliatt.\par
Ce défaut ne tarda point à se faire sentir. Les rafales ne quittaient plus le brise-lames ; elles faisaient plus que s’acharner, on eût dit qu’elles s’appliquaient. On entendait sur cette charpente cahotée une sorte de piétinement.\par
Tout à coup un tronçon d’hiloire, détaché de cette dislocation, sauta au delà de la deuxième claire-voie, vola par-dessus la roche transversale, et alla s’abattre dans le défilé, où l’eau le saisit et l’emporta dans les sinuosités de la ruelle. Gilliatt l’y perdit de vue. Il est probable que le morceau de poutre alla heurter la panse. Heureusement, dans l’intérieur de l’écueil, l’eau, enfermée de toutes parts, se ressentait à peine du bouleversement extérieur. Il y avait peu de flot, et le choc ne put être très rude. Gilliatt du reste n’avait pas le temps de s’occuper de cette avarie, s’il y avait avarie ; tous les dangers se levaient à la fois, la  tempête se concentrait sur le point vulnérable, l’imminence était devant lui.\par
L’obscurité fut un moment profonde, l’éclair s’interrompit, connivence sinistre ; la nuée et la vague ne firent qu’un ; il y eut un coup sourd.\par
Ce coup fut suivi d’un fracas.\par
Gilliatt avança la tête. La claire-voie, qui était le front du barrage, était défoncée. On voyait les pointes de poutres bondir dans la vague. La mer se servait du premier brise-lames pour battre en brèche le second.\par
Gilliatt éprouva ce qu’éprouverait un général qui verrait son avant-garde ramenée.\par
Le deuxième rang de poutres résista au choc. L’armature d’arrière était fortement liée et contre-butée. Mais la claire-voie rompue était pesante, elle était à la discrétion des flots qui la lançaient, puis la reprenaient, les ligatures qui lui restaient l’empêchaient de s’émietter et lui maintenaient tout son volume, et les qualités que Gilliatt lui avait données comme appareil de défense aboutissaient à en faire un excellent engin de destruction. De bouclier elle était devenue massue. En outre les cassures la hérissaient, des bouts de solives lui sortaient de partout et elle était comme couverte de dents et d’éperons. Pas d’arme contondante plus redoutable et plus propre à être maniée par la tempête.\par
Elle était le projectile et la mer était la catapulte.\par
Les coups se succédaient avec une sorte de régularité tragique. Gilliatt, pensif derrière cette porte  barricadée par lui, écoutait ces frappements de la mort voulant entrer.\par
Il réfléchissait amèrement que, sans cette cheminée de la Durande si fatalement retenue par l’épave, il serait en cet instant-là même, et depuis le matin, rentré à Guernesey, et au port, avec la panse en sûreté et la machine sauvée.\par
La chose redoutée se réalisa. L’effraction eut lieu. Ce fut comme un râle. Toute la charpente du brise-lames à la fois, les deux armatures confondues et broyées ensemble, vint, dans une trombe de houle, se ruer sur le barrage de pierre comme un chaos sur une montagne, et s’y arrêta. Cela ne fut plus qu’un enchevêtrement, informe broussaille de poutres, pénétrable aux flots, mais les pulvérisant encore. Ce rempart vaincu agonisait héroïquement. La mer l’avait fracassé, il brisait la mer. Renversé, il demeurait, dans une certaine mesure, efficace. La roche formant barrage, obstacle sans recul possible, le retenait par le pied. Le défilé était, nous l’avons dit, très étroit sur ce point ; la rafale victorieuse avait refoulé, mêlé et pilé tout le brise-lames en bloc dans cet étranglement ; la violence même de la poussée, en tassant la masse et en enfonçant les fractures les unes dans les autres, avait fait de cette démolition un écrasement solide. C’était détruit et inébranlable. Quelques pièces de bois seulement s’arrachèrent. Le flot les dispersa. Une passa en l’air très près de Gilliatt. Il en sentit le vent sur son front.\par
Mais quelques lames, ces grosses lames qui dans  les tourmentes reviennent avec une périodicité imperturbable, sautaient par-dessus la ruine du brise-lames. Elles retombaient dans le défilé, et, en dépit des coudes que faisait la ruelle, elles y soulevaient l’eau. Le flot du détroit commençait à remuer fâcheusement. Le baiser obscur des vagues aux rochers s’accentuait.\par
Comment empêcher à présent cette agitation de se propager jusqu’à la panse ?\par
Il ne faudrait pas beaucoup de temps à ces rafales pour mettre toute l’eau intérieure en tempête, et, en quelques coups de mer, la panse serait éventrée, et la machine coulée.\par
Gilliatt songeait, frémissant.\par
Mais il ne se déconcertait point. Pas de déroute possible pour cette âme.\par
L’ouragan maintenant avait trouvé le joint et s’engouffrait frénétiquement entre les deux murailles du détroit.\par
Tout à coup retentit et se prolongea dans le défilé, à quelque distance en arrière de Gilliatt, un craquement, plus effrayant que tout ce que Gilliatt avait encore entendu.\par
C’était du côté de la panse.\par
Quelque chose de funeste se passait là.\par
Gilliatt y courut.\par
Du goulet de l’est, où il était, il ne pouvait voir la panse à cause des zigzags de la ruelle. Au dernier tournant, il s’arrêta, et attendit un éclair.\par
L’éclair arriva et lui montra la situation.\par
Au coup de mer sur le goulet de l’est avait répondu  un coup de vent sur le goulet de l’ouest. Un désastre s’y ébauchait.\par
La panse n’avait point d’avarie visible ; affourchée comme elle était, elle donnait peu de prise ; mais la carcasse de la Durande était en détresse.\par
Cette ruine, dans une telle tempête, présentait de la surface. Elle était toute hors de l’eau, en l’air, offerte. Le trou que lui avait pratiqué Gilliatt pour en extraire la machine, achevait d’affaiblir la coque. La poutre de quille était coupée. Ce squelette avait la colonne vertébrale rompue.\par
L’ouragan avait soufflé dessus.\par
Il n’en avait point fallu davantage. Le tablier du pont s’était plié comme un livre qui s’ouvre. Le démembrement s’était fait. C’était ce craquement qui, à travers la tourmente, était parvenu aux oreilles de Gilliatt.\par
Ce qu’il vit en approchant paraissait presque irrémédiable.\par
L’incision carrée opérée par lui était devenue une plaie. De cette coupure le vent avait fait une fracture. Cette brisure transversale séparait l’épave en deux. La partie postérieure, voisine de la panse, était demeurée solide dans son étau de rochers. La partie antérieure, celle qui faisait face à Gilliatt, pendait. Une fracture, tant qu’elle tient, est un gond. Cette masse oscillait sur ses cassures, comme sur des charnières, et le vent la balançait avec un bruit redoutable.\par
Heureusement la panse n’était plus dessous.\par
Mais ce balancement ébranlait l’autre moitié de la  coque encore incrustée et immobile entre les deux Douvres. De l’ébranlement à l’arrachement il n’y a pas loin. Sous l’opiniâtreté du vent, la partie disloquée pouvait subitement entraîner l’autre, qui touchait presque à la panse, et tout, la panse avec la machine, s’engloutirait sous cet effondrement.\par
Gilliatt avait cela devant les yeux.\par
C’était la catastrophe.\par
Comment la détourner ?\par
Gilliatt était de ceux qui du danger même font jaillir le secours. Il se recueillit un moment.\par
Gilliatt alla à son arsenal et prit sa hache.\par
Le marteau avait bien travaillé, c’était le tour de la cognée.\par
Puis Gilliatt monta sur l’épave. Il prit pied sur la partie du tablier qui n’avait pas fléchi, et, penché au-dessus du précipice de l’entre-deux des Douvres, il se mit à achever les poutres brisées et à couper ce qui restait d’attaches à la coque pendante.\par
Consommer la séparation des deux tronçons de l’épave, délivrer la moitié restée solide, jeter au flot ce que le vent avait saisi, faire la part à la tempête, telle était l’opération. Elle était plus périlleuse que malaisée. La moitié pendante de la coque, tirée par le vent et par son poids, n’adhérait que par quelques points. L’ensemble de l’épave ressemblait à un diptyque dont un volet à demi décloué battrait l’autre. Cinq ou six pièces de la membrure seulement, pliées et éclatées, mais non rompues, tenaient encore. Leurs fractures criaient et s’élargissaient à chaque va-et-vient  de la bise, et la hache n’avait pour ainsi dire qu’à aider le vent. Ce peu d’attaches, qui faisait la facilité de ce travail, en faisait aussi le danger. Tout pouvait crouler à la fois sous Gilliatt.\par
L’orage atteignait son paroxysme. La tempête n’avait été que terrible, elle devint horrible. La convulsion de la mer gagna le ciel. La nuée jusque-là avait été souveraine, elle semblait exécuter ce qu’elle voulait, elle donnait l’impulsion, elle versait la folie aux vagues, tout en gardant on ne sait quelle lucidité sinistre. En bas c’était de la démence, en haut c’était de la colère. Le ciel est le souffle, l’océan n’est que l’écume. De là l’autorité du vent. L’ouragan est génie. Cependant l’ivresse de sa propre horreur l’avait troublé. Il n’était plus que tourbillon. C’était l’aveuglement enfantant la nuit.\par
Il y a dans les tourmentes un moment insensé ; c’est pour le ciel une espèce de montée au cerveau. L’abîme ne sait plus ce qu’il fait. Il foudroie à tâtons. Rien de plus affreux. C’est l’heure hideuse. La trépidation de l’écueil était à son comble. Tout orage a une mystérieuse orientation ; à cet instant-là, il la perd. C’est le mauvais endroit de la tempête. A cet instant-là, \emph{le vent}, disait Thomas Fuller, \emph{est un fou furieux}. C’est à cet instant-là que se fait dans les tempêtes cette dépense continue d’électricité que Piddington appelle \emph{la cascade d’éclairs.} C’est à cet instant-là qu’au plus noir de la nuée apparaît, on ne sait pourquoi, pour espionner l’effarement universel, ce cercle de lueur bleue que les vieux marins espagnols nommaient l’Œil  de Tempête, \emph{el ojo de tempestad}. Cet œil lugubre était sur Gilliatt.\par
Gilliatt de son côté regardait la nuée. Maintenant il levait la tête. Après chaque coup de cognée, il se dressait, hautain. Il était, ou il semblait être, trop perdu pour que l’orgueil ne lui vînt pas. Désespérait-il ? Non. Devant le suprême accès de rage de l’océan, il était aussi prudent que hardi. Il ne mettait les pieds dans l’épave que sur les points solides. Il se risquait et se préservait. Lui aussi était à son paroxysme. Sa vigueur avait décuplé. Il était éperdu d’intrépidité. Ses coups de cognée sonnaient comme des défis. Il paraissait avoir gagné en lucidité ce que la tempête avait perdu. Conflit pathétique. D’un côté l’intarissable, de l’autre l’infatigable. C’était à qui ferait lâcher prise à l’autre. Les nuées terribles modelaient dans l’immensité des masques de gorgones, tout le dégagement d’intimidation possible se produisait, la pluie venait des vagues, l’écume venait des nuages, les fantômes du vent se courbaient, des faces de météores s’empourpraient et s’éclipsaient, et l’obscurité était monstrueuse après ces évanouissements ; il n’y avait plus qu’un versement, arrivant de tous les côtés à la fois ; tout était ébullition ; l’ombre en masse débordait ; les cumulus chargés de grêle, déchiquetés, couleur cendre, paraissaient pris d’une espèce de frénésie giratoire, il y avait en l’air un bruit de pois secs secoués dans un crible, les électricités inverses observées par Volta faisaient de nuage à nuage leur jeu fulminant, les prolongements de la foudre étaient épouvantables, les  éclairs s’approchaient tout près de Gilliatt. Il semblait étonner l’abîme. Il allait et venait sur la Durande branlante, faisant trembler le pont sous son pas, frappant, taillant, coupant, tranchant, la hache au poing, blême aux éclairs, échevelé, pieds nus, en haillons, la face couverte des crachats de la mer, grand dans ce cloaque de tonnerres.\par
Contre le délire des forces, l’adresse seule peut lutter. L’adresse était le triomphe de Gilliatt. Il voulait une chute ensemble de tout le débris disloqué. Pour cela, il affaiblissait les fractures charnières sans les rompre tout à fait, laissant quelques fibres qui soutenaient le reste. Subitement il s’arrêta, tenant la cognée haute. L’opération était à point. Le morceau entier se détacha.\par
Cette moitié de la carcasse de l’épave coula entre les deux Douvres, au-dessous de Gilliatt debout sur l’autre moitié, penché et regardant. Elle plongea perpendiculairement dans l’eau, éclaboussa les rochers, et s’arrêta dans l’étranglement avant de toucher le fond. Il en resta assez hors de l’eau pour dominer le flot de plus de douze pieds ; le tablier vertical faisait muraille entre les deux Douvres ; comme la roche jetée en travers un peu plus haut dans le détroit, il laissait à peine filtrer un glissement d’écume à ses deux extrémités ; et ce fut la cinquième barricade improvisée par Gilliatt contre la tempête dans cette rue de la mer.\par
L’ouragan, aveugle, avait travaillé à cette barricade dernière.\par
Il était heureux que le resserrement des parois eût  empêché ce barrage d’aller jusqu’au fond. Cela lui laissait plus de hauteur ; en outre l’eau pouvait passer sous l’obstacle, ce qui soutirait de la force aux lames. Ce qui passe par-dessous ne saute point par-dessus. C’est là, en partie, le secret du brise-lames flottant.\par
Désormais, quoi que fit la nuée, rien n’était à craindre pour la panse et la machine. L’eau ne pouvait plus bouger autour d’elles. Entre la clôture des Douvres qui les couvrait à l’ouest et le nouveau barrage qui les protégeait à l’est, aucun coup de mer ni de vent ne pouvait les atteindre.\par
Gilliatt de la catastrophe avait tiré le salut. La nuée, en somme, l’avait aidé.\par
Cette chose faite, il prit d’une flaque de pluie un peu d’eau dans le creux de sa main, but, et dit à la nuée : Cruche !\par
C’est une joie ironique pour l’intelligence combattante de constater la vaste stupidité des forces furieuses aboutissant à des services rendus, et Gilliatt sentait cet immémorial besoin d’insulter son ennemi, qui remonte aux héros d’Homère.\par
Gilliatt descendit dans la panse et profita des éclairs pour l’examiner. Il était temps que le secours arrivât à la pauvre barque, elle avait été fort secouée dans l’heure précédente et elle commençait à s’arquer. Gilliatt, dans ce coup d’œil sommaire, ne constata aucune avarie. Pourtant il était certain qu’elle avait enduré des chocs violents. Une fois l’eau calmée, la coque s’était redressée d’elle-même ; les ancres s’étaient bien  comportées ; quant à la machine, ses quatre chaînes l’avaient admirablement maintenue.\par
Comme Gilliatt achevait cette revue, une blancheur passa près de lui et s’enfonça dans l’ombre. C’était une mouette.\par
Pas d’apparition meilleure dans les tourmentes. Quand les oiseaux arrivent, c’est que l’orage se retire.\par
Autre signe excellent, le tonnerre redoublait.\par
Les suprêmes violences de la tempête la désorganisent. Tous les marins le savent, la dernière épreuve est rude, mais courte. L’excès de foudre annonce la fin.\par
La pluie s’arrêta brusquement. Puis il n’y eut plus qu’un roulement bourru dans la nuée. L’orage cessa comme une planche qui tombe à terre. Il se cassa, pour ainsi dire. L’immense machine des nuages se défit. Une lézarde de ciel clair disjoignit les ténèbres. Gilliatt fut stupéfait, il était grand jour.\par
La tempête avait duré près de vingt heures.\par
Le vent qui avait apporté, remporta. Un écroulement d’obscurité diffuse encombra l’horizon. Les brumes rompues et fuyantes se massèrent pêle-mêle en tumulte, il y eut d’un bout à l’autre de la ligne des nuages un mouvement de retraite, on entendit une longue rumeur décroissante, quelques dernières gouttes de pluie tombèrent, et toute cette ombre pleine de tonnerres s’en alla comme une cohue de chars terribles.\par
Brusquement le ciel fut bleu.\par
Gilliatt s’aperçut qu’il était las. Le sommeil s’abat  sur la fatigue comme un oiseau de proie. Gilliatt se laissa fléchir et tomber dans la barque sans choisir la place, et s’endormit. Il resta ainsi quelques heures inerte et allongé, peu distinct des poutres et des solives parmi lesquelles il gisait.
 \subsection[{B.IV. Livre quatrième. Les doubles fonds de l’obstacle}]{B.IV. Livre quatrième \\
Les doubles fonds de l’obstacle}
  \subsubsection[{B.IV.1. Qui a faim n’est pas le seul}]{B.IV.1. \\
Qui a faim n’est pas le seul}
\noindent Quand il s’éveilla, il eut faim.\par
La mer s’apaisait. Mais il restait assez d’agitation au large pour que le départ immédiat fût impossible. La journée d’ailleurs était trop avancée. Avec le chargement que portait la panse, pour arriver à Guernesey avant minuit, il fallait partir le matin.\par
Quoique la faim le pressât, Gilliatt commença par se mettre nu, seul moyen de se réchauffer.\par
Ses vêtements étaient trempés par l’orage, mais l’eau de pluie avait lavé l’eau de mer, ce qui fait que maintenant ils pouvaient sécher.\par
Gilliatt ne garda que son pantalon, qu’il releva jusqu’aux jarrets.\par
Il étendit çà et là et fixa avec des galets sur les saillies de rocher autour de lui sa chemise, sa vareuse, son suroit, ses jambières et sa peau de mouton.\par
 Puis il pensa à manger.\par
Gilliatt eut recours à son couteau qu’il avait grand soin d’aiguiser et de tenir toujours en état, et il détacha du granit quelques poux de roque, de la même espèce à peu près que les clovisses de la Méditerranée. On sait que cela se mange cru. Mais, après tant de labeurs si divers et si rudes, la pitance était maigre. Il n’avait plus de biscuit. Quant à l’eau, elle ne lui manquait plus. Il était mieux que désaltéré, il était inondé.\par
Il profita de ce que la mer baissait pour rôder dans les rochers à la recherche des langoustes. Il y avait assez de découverte pour espérer une bonne chasse.\par
Seulement il ne réfléchissait pas qu’il ne pouvait plus rien faire cuire. S’il eût pris le temps d’aller jusqu’à son magasin, il l’eût trouvé effondré sous la pluie. Son bois et son charbon étaient noyés, et de sa provision d’étoupe, qui lui tenait lieu d’amadou, il n’y avait pas un brin qui ne fût mouillé. Nul moyen d’allumer du feu.\par
Du reste la soufflante était désorganisée ; l’auvent du foyer de la forge était descellé ; l’orage avait fait le sac du laboratoire. Avec ce qui restait d’outils échappés à l’avarie, Gilliatt, à la rigueur, pouvait encore travailler comme charpentier, non comme forgeron. Mais Gilliatt, pour l’instant, ne songeait pas à son atelier.\par
Tiré d’un autre côté par l’estomac, il s’était mis, sans plus de réflexion, à la poursuite de son repas. Il errait, non dans la gorge de l’écueil, mais en dehors, sur le revers des brisants. C’était de ce côté-là que la  Durande, dix semaines auparavant, était venue se heurter aux récifs.\par
Pour la chasse que faisait Gilliatt, l’extérieur du défilé valait mieux que l’intérieur. Les crabes, à mer basse, ont l’habitude de prendre l’air. Ils se chauffent volontiers au soleil. Ces êtres difformes aiment midi. C’est une chose bizarre que leur sortie de l’eau en pleine lumière. Leur fourmillement indigne presque. Quand on les voit, avec leur gauche allure oblique, monter lourdement, de pli en pli, les étages inférieurs des rochers comme les marches d’un escalier, on est forcé de s’avouer que l’océan a de la vermine.\par
Depuis deux mois Gilliatt vivait de cette vermine.\par
Ce jour-là pourtant les poing-clos et les langoustes se dérobaient. La tempête avait refoulé ces solitaires dans leurs cachettes et ils n’étaient pas encore rassurés. Gilliatt tenait à la main son couteau ouvert, et arrachait de temps en temps un coquillage sous le varech. Il mangeait, tout en marchant.\par
Il ne devait pas être loin de l’endroit où sieur Clubin s’était perdu.\par
Comme Gilliatt prenait le parti de se résigner aux oursins et aux châtaignes de mer, un clapotement se fit à ses pieds. Un gros crabe, effrayé de son approche, venait de sauter à l’eau. Le crabe ne s’enfonça point assez pour que Gilliatt le perdît de vue.\par
Gilliatt se mit à courir après le crabe sur le soubassement de l’écueil. Le crabe fuyait.\par
Subitement il n’y eut plus rien.\par
 Le crabe venait de se fourrer dans quelque crevasse sous le rocher.\par
Gilliatt se cramponna du poing à des reliefs de roche et avança la tête pour voir sous les surplombs.\par
Il y avait là, en effet, une anfractuosité. Le crabe avait dû s’y réfugier.\par
C’était mieux qu’une crevasse. C’était une espèce de porche.\par
La mer entrait sous ce porche, mais n’y était pas profonde. On voyait le fond couvert de galets. Ces galets étaient glauques et revêtus de conferves, ce qui indiquait qu’ils n’étaient jamais à sec. Ils ressemblaient à des dessus de têtes d’enfants avec des cheveux verts.\par
Gilliatt prit son couteau dans ses dents, descendit des pieds et des mains du haut de l’escarpement et sauta dans cette eau. Il en eut presque jusqu’aux épaules.\par
Il s’engagea sous ce porche. Il se trouvait dans un couloir fruste avec une ébauche de voûte ogive sur sa tête. Les parois étaient polies et lisses. Il ne voyait plus le crabe. Il avait pied. Il avançait dans une décroissance de jour. Il commençait à ne plus rien distinguer.\par
Après une quinzaine de pas, la voûte cessa au-dessus de lui. Il était hors du couloir. Il y avait plus d’espace, et par conséquent plus de jour ; ses pupilles d’ailleurs s’étaient dilatées ; il voyait assez clair. Il eut une surprise.\par
Il venait de rentrer dans cette cave étrange visitée par lui le mois d’auparavant.\par
 Seulement il y était rentré par la mer.\par
Cette arche qu’il avait vue noyée, c’est par là qu’il venait de passer. A de certaines marées basses, elle était praticable.\par
Ses yeux s’accoutumaient. Il voyait de mieux en mieux. Il était stupéfait. Il retrouvait cet extraordinaire palais de l’ombre, cette voûte, ces piliers, ces sangs ou ces pourpres, cette végétation à pierreries, et, au fond, cette crypte, presque sanctuaire, et cette pierre, presque autel.\par
Il se rendait peu compte de ces détails, mais il avait dans l’esprit l’ensemble, et il le revoyait.\par
Il revoyait en face de lui, à une certaine hauteur dans l’escarpement, la crevasse par laquelle il avait pénétré la première fois, et qui, du point où il était maintenant, semblait inaccessible.\par
Il revoyait près de l’arche ogive ces grottes basses et obscures, sortes de caveaux dans la cave, qu’il avait déjà observées de loin. A présent, il en était près. La plus voisine de lui était à sec et aisément abordable.\par
Plus près encore que cet enfoncement, il remarqua au-dessus du niveau de l’eau, à portée de sa main, une fissure horizontale dans le granit. Le crabe était probablement là. Il y plongea le poing le plus avant qu’il put, et se mit à tâtonner dans ce trou de ténèbres.\par
Tout à coup il se sentit saisir le bras.\par
Ce qu’il éprouva en ce moment, c’est l’horreur indescriptible.\par
Quelque chose qui était mince, âpre, plat, glacé,  gluant et vivant venait de se tordre dans l’ombre autour de son bras nu. Cela lui montait vers la poitrine. C’était la pression d’une courroie et la poussée d’une vrille. En moins d’une seconde, on ne sait quelle spirale lui avait envahi le poignet et le coude et touchait l’épaule. La pointe fouillait sous son aisselle.\par
Gilliatt se rejeta en arrière, mais put à peine remuer. Il était comme cloué. De sa main gauche restée libre il prit son couteau qu’il avait entre ses dents, et de cette main, tenant le couteau, s’arc-bouta au rocher, avec un effort désespéré pour retirer son bras. Il ne réussit qu’à inquiéter un peu la ligature, qui se resserra. Elle était souple comme le cuir, solide comme l’acier, froide comme la nuit.\par
Une deuxième lanière, étroite et aiguë, sortit de la crevasse du roc. C’était comme une langue hors d’une gueule. Elle lécha épouvantablement le torse nu de Gilliatt, et tout à coup s’allongeant, démesurée et fine, elle s’appliqua sur sa peau et lui entoura tout le corps.\par
En même temps, une souffrance inouïe, comparable à rien, soulevait les muscles crispés de Gilliatt. Il sentait dans sa peau des enfoncements ronds, horribles. Il lui semblait que d’innombrables lèvres, collées à sa chair, cherchaient à lui boire le sang.\par
Une troisième lanière ondoya hors du rocher, tâta Gilliatt, et lui fouetta les côtes comme une corde. Elle s’y fixa.\par
L’angoisse, à son paroxysme, est muette. Gilliatt ne jetait pas un cri. Il y avait assez de jour pour qu’il pût voir les repoussantes formes appliquées sur lui.  Une quatrième ligature, celle-ci rapide comme une flèche, lui sauta autour du ventre et s’y enroula.\par
Impossible de couper ni d’arracher ces courroies visqueuses qui adhéraient étroitement au corps de Gilliatt et par quantité de points. Chacun de ces points était un foyer d’affreuse et bizarre douleur. C’était ce qu’on éprouverait si l’on se sentait avalé à la fois par une foule de bouches trop petites.\par
Un cinquième allongement jaillit du trou. Il se superposa aux autres et vint se replier sur le diaphragme de Gilliatt. La compression s’ajoutait à l’anxiété ; [{\corr Gilliatt}] pouvait à peine respirer.\par
Ces lanières, pointues à leur extrémité, allaient s’élargissant comme des lames d’épée vers la poignée. Toutes les cinq appartenaient évidemment au même centre. Elles marchaient et rampaient sur Gilliatt. Il sentait se déplacer ces pressions obscures qui lui semblaient être des bouches.\par
Brusquement une large viscosité ronde et plate sortit de dessous la crevasse. C’était le centre ; les cinq lanières s’y rattachaient comme des rayons à un moyeu ; on distinguait au côté opposé de ce disque immonde le commencement de trois autres tentacules, restés sous l’enfoncement du rocher. Au milieu de cette viscosité il y avait deux yeux qui regardaient.\par
Ces yeux voyaient Gilliatt.\par
Gilliatt reconnut la pieuvre.
 \subsubsection[{B.IV.2. Le monstre}]{B.IV.2. \\
Le monstre}
\noindent Pour croire à la pieuvre, il faut l’avoir vue.\par
Comparées à la pieuvre, les vieilles hydres font sourire.\par
A de certains moments, on serait tenté de le penser, l’insaisissable qui flotte en nos songes rencontre dans le possible des aimants auxquels ses linéaments se prennent, et de ces obscures fixations du rêve il sort des êtres. L’Inconnu dispose du prodige, et il s’en sert pour composer le monstre. Orphée, Homère et Hésiode n’ont pu faire que la Chimère ; Dieu a fait la Pieuvre.\par
Quand Dieu veut, il excelle dans l’exécrable.\par
Le pourquoi de cette volonté est l’effroi du penseur religieux.\par
Tous les idéals étant admis, si l’épouvante est un but, la pieuvre est un chef-d’œuvre.\par
La baleine a l’énormité, la pieuvre est petite ; [{\corr l’hippopotame}] a une cuirasse, la pieuvre est nue ; la jararaca  a un sifflement, la pieuvre est muette ; le rhinocéros a une corne, la pieuvre n’a pas de corne ; le scorpion a un dard, la pieuvre n’a pas de dard ; le buthus a des pinces, la pieuvre n’a pas de pinces ; l’alouate a une queue prenante, la pieuvre n’a pas de queue ; le requin a des nageoires tranchantes, la pieuvre n’a pas de nageoires ; le vespertilio-vampire a des ailes onglées, la pieuvre n’a pas d’ailes ; le hérisson a des épines, la pieuvre n’a pas d’épines ; l’espadon a un glaive, la pieuvre n’a pas de glaive ; la torpille a une foudre, la pieuvre n’a pas d’effluve ; le crapaud a un virus, la pieuvre n’a pas de virus ; la vipère a un venin, la pieuvre n’a pas de venin ; le lion a des griffes, la pieuvre n’a pas de griffes ; le gypaète a un bec, la pieuvre n’a pas de bec ; le crocodile a une gueule, la pieuvre n’a pas de dents.\par
La pieuvre n’a pas de masse musculaire, pas de cri menaçant, pas de cuirasse, pas de corne, pas de dard, pas de pince, pas de queue prenante ou contondante, pas d’ailerons tranchants, pas d’ailerons onglés, pas d’épines, pas d’épée, pas de décharge électrique, pas de virus, pas de venin, pas de griffes, pas de bec, pas de dents. La pieuvre est de toutes les bêtes la plus formidablement armée.\par
Qu’est-ce donc que la pieuvre ? C’est la ventouse.\par
Dans les écueils de pleine mer, là où l’eau étale et cache toutes ses splendeurs, dans les creux de rochers non visités, dans les caves inconnues où abondent les végétations, les crustacés et les coquillages, sous les profonds portails de l’océan, le nageur qui s’y hasarde,  entraîné par la beauté du lieu, court le risque d’une rencontre. Si vous faites cette rencontre, ne soyez pas curieux, évadez-vous. On entre ébloui, on sort terrifié.\par
Voici ce que c’est que cette rencontre, toujours possible dans les roches du large.\par
Une forme grisâtre oscille dans l’eau, c’est gros comme le bras, et long d’une demi-aune environ ; c’est un chiffon ; cette forme ressemble à un parapluie fermé qui n’aurait pas de manche. Cette loque avance vers vous peu à peu. Soudain, elle s’ouvre, huit rayons s’écartent brusquement autour d’une face qui a deux yeux ; ces rayons vivent ; il y a du flamboiement dans leur ondoiement ; c’est une sorte de roue ; déployée, elle a quatre ou cinq pieds de diamètre. Épanouissement effroyable. Cela se jette sur vous.\par
L’hydre harponne l’homme.\par
Cette bête s’applique sur sa proie, la recouvre, et la noue de ses longues bandes. En dessous elle est jaunâtre, en dessus elle est terreuse ; rien ne saurait rendre cette inexplicable nuance poussière ; on dirait une bête faite de cendre qui habite l’eau. Elle est arachnide par la forme et caméléon par la coloration. Irritée, elle devient violette. Chose épouvantable, c’est mou.\par
Ses nœuds garrottent ; son contact paralyse.\par
Elle a un aspect de scorbut et de gangrène. C’est de la maladie arrangée en monstruosité.\par
Elle est inarrachable. Elle adhère étroitement à sa proie. Comment ? Par le vide. Les huit antennes, larges à l’origine, vont s’effilant et s’achèvent en aiguilles.  Sous chacune d’elles s’allongent parallèlement deux rangée de pustules décroissantes, les grosses près de la tête, les petites à la pointe. Chaque rangée est de vingt-cinq ; il y a cinquante pustules par antenne, et toute la bête en a quatre cents. Ces pustules sont des ventouses.\par
Ces ventouses sont des cartilages cylindriques, cornés, livides. Sur la grande espèce, elles vont diminuant du diamètre d’une pièce de cinq francs à la grosseur d’une lentille. Ces tronçons de tubes sortent de l’animal et y rentrent. Ils peuvent s’enfoncer dans la proie de plus d’un pouce.\par
Cet appareil de succion a toute la délicatesse d’un clavier. Il se dresse, puis se dérobe. Il obéit à la moindre intention de l’animal. Les sensibilités les plus exquises n’égalent pas la contractilité de ces ventouses, toujours proportionnée aux mouvements intérieurs de la bête et aux incidents extérieurs. Ce dragon est une sensitive.\par
Ce monstre est celui que les marins appellent poulpe, que la science appelle céphalopode, et que la légende appelle kraken. Les matelots anglais l’appellent Devil-fish, le Poisson-Diable. Ils l’appellent aussi \emph{Blood-sucker,} Suceur de sang. Dans les îles de la Manche on le nomme la pieuvre.\par
Il est très rare à Guernesey, très petit à Jersey, très gros et assez fréquent à Serk.\par
Une estampe de l’édition de Buffon par Sonnini représente un céphalopode étreignant une frégate. Denis Montfort pense qu’en effet le poulpe des hautes  latitudes est de force à couler un navire. Bory Saint-Vincent le nie, mais constate que dans nos régions il attaque l’homme. Allez à Serk, on vous montrera près de Brecq-Hou le creux de rocher où une pieuvre, il y a quelques années, a saisi, retenu et noyé un pêcheur de homards. Péron et Lamarck se trompent quand ils doutent que le poulpe, n’ayant pas de nageoires, puisse nager.\par
Celui qui écrit ces lignes a vu de ses yeux à Serk, dans la cave dite les Boutiques, une pieuvre poursuivre à la nage un baigneur. Tuée, on la mesura, elle avait quatre pieds anglais d’envergure et l’on put compter les quatre cents suçoirs. La bête agonisante les poussait hors d’elle convulsivement.\par
Selon Denis Montfort, un de ces observateurs que l’intuition à haute dose fait descendre ou monter jusqu’au magisme, le poulpe a presque des passions d’homme ; le poulpe hait. En effet, dans l’absolu, être hideux, c’est haïr.\par
Le difforme se débat sous une nécessité d’élimination qui le rend hostile.\par
La pieuvre nageant reste, pour ainsi dire, dans le fourreau. Elle nage, tous ses plis serrés. Qu’on se représente une manche cousue avec un poing dedans. Ce poing, qui est la tête, pousse le liquide et avance d’un vague mouvement ondulatoire. Ses deux yeux, quoique gros, sont peu distincts, étant de la couleur de l’eau.\par
La pieuvre en chasse ou au guet se dérobe ; elle se rapetisse, elle se condense ; elle se réduit à sa plus  simple expression. Elle se confond avec la pénombre. Elle a l’air d’un pli de la vague. Elle ressemble à tout, excepté à quelque chose de vivant.\par
La pieuvre, c’est l’hypocrite. On n’y fait pas attention ; brusquement, elle s’ouvre.\par
Une viscosité qui a une volonté, quoi de plus effroyable ! De la glu pétrie de haine.\par
C’est dans le plus bel azur de l’eau limpide que surgit cette hideuse étoile vorace de la mer. Elle n’a pas d’approche, ce qui est terrible. Presque toujours, quand on la voit, on est pris.\par
La nuit, pourtant, et particulièrement dans la saison du rut, elle est phosphorescente. Cette épouvante a ses amours. Elle attend l’hymen. Elle se fait belle, elle s’allume, elle s’illumine, et, du haut de quelque rocher, on peut l’apercevoir au-dessous de soi dans les profondes ténèbres épanouie en une irradiation blême, soleil spectre.\par
La pieuvre nage ; elle marche aussi. Elle est un peu poisson, ce qui ne l’empêche pas d’être un peu reptile. Elle rampe sur le fond de la mer. En marche elle utilise ses huit pattes. Elle se traîne à la façon de la chenille arpenteuse.\par
Elle n’a pas d’os, elle n’a pas de sang, elle n’a pas de chair. Elle est flasque. Il n’y a rien dedans. C’est une peau. On peut retourner ses huit tentacules du dedans au dehors comme des doigts de gants.\par
Elle a un seul orifice, au centre de son rayonnement. Cet hiatus unique, est-ce l’anus ? est-ce la bouche ? C’est les deux.\par
 La même ouverture fait les deux fonctions. L’entrée est l’issue. Toute la bête est froide.\par
Le carnasse de la Méditerranée est repoussant. C’est un contact odieux que cette gélatine animée qui enveloppe le nageur, où les mains s’enfoncent, où les ongles labourent, qu’on déchire sans la tuer, et qu’on arrache sans l’ôter, espèce d’être coulant et tenace qui vous passe entre les doigts ; mais aucune stupeur n’égale la subite apparition de la pieuvre, Méduse servie par huit serpents.\par
Pas de saisissement pareil à l’étreinte du céphalopode.\par
C’est la machine pneumatique qui vous attaque. Vous avez affaire au vide ayant des pattes. Ni coups d’ongle, ni coups de dents ; une scarification indicible. Une morsure est redoutable ; moins qu’une succion. La griffe n’est rien près de la ventouse. La griffe, c’est la bête qui entre dans votre chair ; la ventouse, c’est vous-même qui entrez dans la bête. Vos muscles s’enflent, vos fibres se tordent, votre peau éclate sous une pesée immonde, votre sang jaillit et se mêle affreusement à la lymphe du mollusque. La bête se superpose à vous par mille bouches infâmes ; l’hydre s’incorpore à l’homme ; l’homme s’amalgame à l’hydre. Vous ne faites qu’un. Ce rêve est sur vous. Le tigre ne peut que vous dévorer ; le poulpe, horreur ! vous aspire. Il vous tire à lui et en lui, et, lié, englué, impuissant, vous vous sentez lentement vidé dans cet épouvantable sac, qui est un monstre.\par
Au delà du terrible, être mangé vivant, il y a l’inexprimable, être bu vivant.\par
 Ces étranges animaux, la science les rejette d’abord, selon son habitude d’excessive prudence, même vis-à-vis des faits, puis elle se décide à les étudier ; elle les dissèque, elle les classe, elle les catalogue, elle leur met une étiquette ; elle s’en procure des exemplaires ; elle les expose sous verre dans les musées ; ils entrent dans la nomenclature ; elle les qualifie mollusques, invertébrés, rayonnés ; elle constate leurs voisinages : un peu au delà les calmars, un peu en deçà les sépiaires ; elle trouve à ces hydres de l’eau salée un analogue dans l’eau douce, l’argyronecte ; elle les divise en grande, moyenne et petite espèce ; elle admet plus aisément la petite espèce que la grande, ce qui est d’ailleurs, dans toutes les régions, la tendance de la science, laquelle est plus volontiers microscopique que télescopique ; elle regarde leur construction et les appelle céphalopodes, elle compte leurs antennes et les appelle octopèdes. Cela fait, elle les laisse là. Où la science les lâche, la philosophie les reprend.\par
La philosophie étudie à son tour ces êtres. Elle va moins loin et plus loin que la science. Elle ne les dissèque pas, elle les médite. Où le scalpel a travaillé, elle plonge l’hypothèse. Elle cherche la cause finale. Profond tourment du penseur. Ces créatures l’inquiètent presque sur le créateur. Elles sont les surprises hideuses. Elles sont les trouble-fête du contemplateur. Il les constate éperdu. Elles sont les formes voulues du mal. Que devenir devant ces blasphèmes de la création contre elle-même ? A qui s’en prendre ?\par
Le Possible est une matrice formidable. Le mystère  se concrète en monstres. Des morceaux d’ombre sortent de ce bloc, l’immanence, se déchirent, se détachent, roulent, flottent, se condensent, font des emprunts à la noirceur ambiante, subissent des polarisations inconnues, prennent vie, se composent on ne sait quelle forme avec l’obscurité et on ne sait quelle âme avec le miasme, et s’en vont, larves, à travers la vitalité. C’est quelque chose comme les ténèbres faites bêtes. A quoi bon ? à quoi cela sert-il ? Rechute de la question éternelle.\par
Ces animaux sont fantômes autant que monstres. Ils sont prouvés et improbables. Être est leur fait, ne pas être serait leur droit. Ils sont les amphibies de la mort. Leur invraisemblance complique leur existence. Ils touchent la frontière humaine et peuplent la limite chimérique. Vous niez le vampire, la pieuvre apparaît. Leur fourmillement est une certitude qui déconcerte notre assurance. L’optimisme, qui est le vrai pourtant, perd presque contenance devant eux. Ils sont l’extrémité visible des cercles noirs. Ils marquent la transition de notre réalité à une autre. Ils semblent appartenir à ce commencement d’êtres terribles que le songeur entrevoit confusément par le soupirail de la nuit.\par
Ces prolongements de monstres, dans l’invisible d’abord, dans le possible ensuite, ont été soupçonnés, aperçus peut-être, par l’extase sévère et par l’œil fixe des mages et des philosophes. De là la conjecture d’un enfer. Le démon est le tigre de l’invisible. La bête fauve des âmes a été dénoncée au genre humain par  deux visionnaires, l’un qui s’appelle Jean, l’autre qui s’appelle Dante.\par
Si en effet les cercles de l’ombre continuent indéfiniment, si après un anneau il y en a un autre, si cette aggravation persiste en progression illimitée, si cette chaîne, dont pour notre part nous sommes résolu à douter, existe, il est certain que la pieuvre à une extrémité prouve Satan à l’autre.\par
Il est certain que le méchant à un bout prouve à l’autre bout la méchanceté.\par
Toute bête mauvaise, comme toute intelligence perverse, est sphinx.\par
Sphinx terrible proposant l’énigme terrible. L’énigme du mal.\par
C’est cette perfection du mal qui a fait pencher parfois de grands esprits vers la croyance au dieu double, vers le redoutable bi-frons des manichéens.\par
Une soie chinoise, volée dans la dernière guerre au palais de l’empereur de la Chine, représente le requin qui mange le crocodile qui mange le serpent qui mange l’aigle qui mange l’hirondelle qui mange la chenille.\par
Toute la nature que nous avons sous les yeux est mangeante et mangée. Les proies s’entre-mordent.\par
Cependant des savants qui sont aussi des philosophes, et par conséquent bienveillants pour la création, trouvent ou croient trouver l’explication. Le but final frappe, entre autres, Bonnet de Genève, ce mystérieux esprit exact, qui fut opposé à Buffon, comme plus tard Geoffroy Saint-Hilaire l’a été à Cuvier.  L’explication serait ceci : la mort partout exige l’ensevelissement partout. Les voraces sont des ensevelisseurs.\par
Tous les êtres rentrent les uns dans les autres. Pourriture, c’est nourriture. Nettoyage effrayant du globe. L’homme, carnassier, est, lui aussi, un enterreur. Notre vie est faite de mort. Telle est la loi terrifiante. Nous sommes sépulcres.\par
Dans notre monde crépusculaire, cette fatalité de l’ordre produit des monstres. Vous dites : à quoi bon ? Le voilà.\par
Est-ce l’explication ? Est-ce la réponse à la question ? Mais alors pourquoi pas un autre ordre ? La question renaît.\par
Vivons, soit.\par
Mais tâchons que la mort nous soit progrès. Aspirons aux mondes moins ténébreux.\par
Suivons la conscience qui nous y mène.\par
Car, ne l’oublions jamais, le mieux n’est trouvé que par le meilleur.
 \subsubsection[{B.IV.3. Autre forme du combat dans le gouffre}]{B.IV.3. \\
Autre forme du combat dans le gouffre}
\noindent Tel était l’être auquel, depuis quelques instants, Gilliatt appartenait.\par
Ce monstre était l’habitant de cette grotte. Il était l’effrayant génie du lieu. Sorte de sombre démon de l’eau.\par
Toutes ces magnificences avaient pour centre l’horreur.\par
Le mois d’auparavant, le jour où pour la première fois Gilliatt avait pénétré dans la grotte, la noirceur ayant un contour entrevue par lui dans les plissements de l’eau secrète, c’était cette pieuvre.\par
Elle était là chez elle.\par
Quand Gilliatt, entrant pour la seconde fois dans cette cave à la poursuite du crabe, avait aperçu la crevasse où il avait pensé que le crabe se réfugiait, la pieuvre était dans ce trou, au guet.\par
Se figure-t-on cette attente ?\par
 Pas un oiseau n’oserait couver, pas un œuf n’oserait éclore, pas une fleur n’oserait s’ouvrir, pas un sein n’oserait allaiter, pas un cœur n’oserait aimer, pas un esprit n’oserait s’envoler, si l’on songeait aux sinistres patiences embusquées dans l’abîme.\par
Gilliatt avait enfoncé son bras dans le trou ; la pieuvre l’avait happé.\par
Elle le tenait.\par
Il était la mouche de cette araignée.\par
Gilliatt était dans l’eau jusqu’à la ceinture, les pieds crispés sur la rondeur des galets glissants, le bras droit étreint et assujetti par les enroulements plats des courroies de la pieuvre, et le torse disparaissant presque sous les replis et les croisements de ce bandage horrible.\par
Des huit bras de la pieuvre, trois adhéraient à la roche, cinq adhéraient à Gilliatt. De cette façon, cramponnée d’un côté au granit, de l’autre à l’homme, elle enchaînait Gilliatt au rocher. Gilliatt avait sur lui deux cent cinquante suçoirs. Complication d’angoisse et de dégoût. Être serré dans un poing démesuré dont les doigts élastiques, longs de près d’un mètre, sont intérieurement pleins de pustules vivantes qui vous fouillent la chair.\par
Nous l’avons dit, on ne s’arrache pas à la pieuvre. Si on l’essaie, on est plus sûrement lié. Elle ne fait que se resserrer davantage. Son effort croît en raison du vôtre. Plus de secousse produit plus de constriction.\par
Gilliatt n’avait qu’une ressource, son couteau.\par
Il n’avait de libre que la main gauche ; mais on sait  qu’il en usait puissamment. On aurait pu dire de lui qu’il avait deux mains droites.\par
Son couteau, ouvert, était dans cette main.\par
On ne coupe pas les antennes de la pieuvre ; c’est un cuir impossible à trancher, il glisse sous la lame ; d’ailleurs la superposition est telle qu’une entaille à ces lanières entamerait votre chair.\par
Le poulpe est formidable ; pourtant il y a une manière de s’en servir. Les pêcheurs de Serk la connaissent ; qui les a vus exécuter en mer de certains mouvements brusques, le sait. Les marsouins la connaissent aussi ; ils ont une façon de mordre la sèche qui lui coupe la tête. De là tous ces calmars, toutes ces sèches et tous ces poulpes sans tête qu’on rencontre au large.\par
Le poulpe, en effet, n’est vulnérable qu’à la tête.\par
Gilliatt ne l’ignorait point.\par
Il n’avait jamais vu de pieuvre de cette dimension. Du premier coup, il se trouvait pris par la grande espèce. Un autre se fût troublé.\par
Pour la pieuvre comme pour le taureau il y a un moment qu’il faut saisir ; c’est l’instant où le taureau baisse le cou, c’est l’instant où la pieuvre avance la tête ; instant rapide. Qui manque ce joint est perdu.\par
Tout ce que nous venons de dire n’avait duré que quelques minutes. Gilliatt pourtant sentait croître la succion des deux cent cinquante ventouses.\par
La pieuvre est traître. Elle tâche de stupéfier d’abord sa proie. Elle saisit, puis attend le plus qu’elle peut.\par
 Gilliatt tenait son couteau. Les succions augmentaient.\par
Il regardait la pieuvre, qui le regardait.\par
Tout à coup la bête détacha du rocher sa sixième antenne, et, la lançant sur Gilliatt, tâcha de lui saisir le bras gauche.\par
En même temps elle avança vivement la tête. Une seconde de plus, sa bouche anus s’appliquait sur la poitrine de Gilliatt. Gilliatt, saigné au flanc, et les deux bras garrottés, était mort.\par
Mais Gilliatt veillait. Guetté, il guettait.\par
Il évita l’antenne, et, au moment où la bête allait mordre sa poitrine, son poing armé s’abattit sur la bête.\par
Il y eut deux convulsions en sens inverse, celle de la pieuvre et celle de Gilliatt.\par
Ce fut comme la lutte de deux éclairs.\par
Gilliatt plongea la pointe de son couteau dans la viscosité plate, et, d’un mouvement giratoire pareil à la torsion d’un coup de fouet, faisant un cercle autour des deux yeux, il arracha la tête comme on arrache une dent.\par
Ce fut fini.\par
Toute la bête tomba.\par
Cela ressembla à un linge qui se détache. La pompe aspirante détruite, le vide se défit. Les quatre cents ventouses lâchèrent à la fois le rocher et l’homme. Ce haillon coula au fond de l’eau.\par
Gilliatt, haletant du combat, put apercevoir à ses pieds sur les galets deux tas gélatineux informes, la  tête d’un côté, le reste de l’autre. Nous disons le reste, car on ne pourrait dire le corps.\par
Gilliatt toutefois, craignant quelque reprise convulsive de l’agonie, recula hors de la portée des tentacules.\par
Mais la bête était bien morte.\par
Gilliatt referma son couteau.
 \subsubsection[{B.IV.4. Rien ne se cache et rien ne se perd}]{B.IV.4. \\
Rien ne se cache et rien ne se perd}
\noindent Il était temps qu’il tuât la pieuvre. Il était presque étouffé ; son bras droit et son torse étaient violets ; plus de deux cents tumeurs s’y ébauchaient ; le sang jaillissait de quelques-unes çà et là. Le remède à ces lésions, c’est l’eau salée. Gilliatt s’y plongea. En même temps il se frottait avec la paume de la main. Les gonflements s’effaçaient sous ces frictions.\par
En reculant et en s’enfonçant plus avant dans l’eau, il s’était, sans s’en apercevoir, rapproché de l’espèce de caveau, déjà remarqué par lui, près de la crevasse, où il avait été harponné par la pieuvre.\par
Ce caveau se prolongeait obliquement, et à sec, sous les grandes parois de la cave. Les galets qui s’y étaient amassés en avaient exhaussé le fond au-dessus du niveau des marées ordinaires. Cette anfractuosité était un assez large cintre surbaissé ; un homme y pouvait entrer en se courbant. La clarté verte de la grotte sous-marine y pénétrait, et l’éclairait faiblement.\par
 Il arriva que, tout en frictionnant en hâte sa peau tuméfiée, Gilliatt leva machinalement les yeux.\par
Son regard s’enfonça dans ce caveau.\par
Il eut un tressaillement.\par
Il lui sembla voir au fond de ce trou dans l’ombre une sorte de face qui riait.\par
Gilliatt ignorait le mot hallucination, mais connaissait la chose. Les mystérieuses rencontres avec l’invraisemblable que, pour nous tirer d’affaire, nous appelons hallucinations, sont dans la nature. Illusions ou réalités, des visions passent. Qui se trouve là les voit. Gilliatt, nous l’avons dit, était un pensif. Il avait cette grandeur d’être parfois halluciné comme un prophète. On n’est pas impunément le songeur des lieux solitaires.\par
Il crut à un de ces mirages dont, homme nocturne qu’il était, il avait eu plus d’une fois la stupeur.\par
L’anfractuosité figurait assez exactement un four à chaux. C’était une niche basse en anse de panier, dont les voussures abruptes allaient se rétrécissant jusqu’à l’extrémité de la crypte où le cailloutis de galets et la voûte de roche se rejoignaient, et où finissait le cul-de-sac.\par
Il y entra, et, penchant le front, se dirigea vers ce qu’il y avait au fond.\par
Quelque chose riait en effet.\par
C’était une tête de mort.\par
Il n’y avait pas que la tête, il y avait le squelette.\par
Un squelette humain était couché dans ce caveau.\par
 Le regard d’un homme hardi, en de pareilles rencontres, veut savoir à quoi s’en tenir.\par
Gilliatt jeta les yeux autour de lui.\par
Il était entouré d’une multitude de crabes.\par
Cette multitude ne remuait pas. C’était l’aspect que présenterait une fourmilière morte. Tous ces crabes étaient inertes. Ils étaient vides.\par
Leurs groupes, semés çà et là, faisaient sur le pavé de galets qui encombrait le caveau des constellations difformes.\par
Gilliatt, l’œil fixé ailleurs, avait marché dessus sans s’en apercevoir.\par
A l’extrémité de la crypte où Gilliatt était parvenu, il y en avait une plus grande épaisseur. C’était un hérissement immobile d’antennes, de pattes et de mandibules. Des pinces ouvertes se tenaient toutes droites et ne se fermaient plus. Les boîtes osseuses ne bougeaient pas sous leur croûte d’épines ; quelques-unes retournées montraient leur creux livide. Cet entassement ressemblait à une mêlée d’assiégeants et avait l’enchevêtrement d’une broussaille.\par
C’est sous ce monceau qu’était le squelette.\par
On apercevait sous ce pêle-mêle de tentacules et d’écailles le crâne avec ses stries, les vertèbres, les fémurs, les tibias, les longs doigts noueux avec les ongles. La cage des côtes était pleine de crabes. Un cœur quelconque avait battu là. Des moisissures marines tapissaient les trous des yeux. Des patelles avaient laissé leur bave dans les fosses nasales. Du reste il n’y avait dans ce recoin de rocher ni goémons, ni  herbes, ni un souffle d’air. Aucun mouvement. Les dents ricanaient.\par
Le côté inquiétant du rire, c’est l’imitation qu’en fait la tête de mort.\par
Ce merveilleux palais de l’abîme, brodé et incrusté de toutes les pierreries de la mer, finissait par se révéler et par dire son secret. C’était un repaire, la pieuvre y habitait ; et c’était une tombe, un homme y gisait.\par
L’immobilité spectrale du squelette et des bêtes oscillait vaguement, à cause de la réverbération des eaux souterraines qui tremblait sur cette pétrification. Les crabes, fouillis effroyable, avaient l’air d’achever leur repas. Ces carapaces semblaient manger cette carcasse. Rien de plus étrange que cette vermine morte sur cette proie morte. Sombres continuations de la mort.\par
Gilliatt avait sous les yeux le garde-manger de la pieuvre.\par
Vision lugubre, et où se laissait prendre sur le fait l’horreur profonde des choses. Les crabes avaient mangé l’homme, la pieuvre avait mangé les crabes.\par
Il n’y avait près du cadavre aucun reste de vêtement. Il avait dû être saisi nu.\par
Gilliatt, attentif et examinant, se mit à ôter les crabes de dessus l’homme. Qu’était-ce que cet homme ? Le cadavre était admirablement disséqué. On eût dit une préparation d’anatomie ; toute la chair était éliminée ; pas un muscle ne restait, pas un os ne manquait. Si Gilliatt eût été du métier, il eût pu le constater.  Les périostes dénudés étaient blancs, polis, et comme fourbis. Sans quelques verdissements de conferves çà et là, c’eût été de l’ivoire. Les cloisons cartilagineuses étaient délicatement amenuisées et ménagées. La tombe fait de ces bijouteries sinistres.\par
Le cadavre était comme enterré sous les crabes morts ; Gilliatt le déterrait.\par
Tout à coup il se pencha vivement.\par
Il venait d’apercevoir autour de la colonne vertébrale une espèce de lien.\par
C’était une ceinture de cuir qui avait évidemment été bouclée sur le ventre de l’homme de son vivant.\par
Le cuir était moisi. La boucle était rouillée.\par
Gilliatt tira à lui cette ceinture. Les vertèbres résistèrent, et il dut les rompre pour la prendre. La ceinture était intacte. Une croûte de coquillages commençait à s’y former.\par
Il la palpa et sentit un objet dur et de forme carrée dans l’intérieur. Il ne fallait pas songer à défaire la boucle. Il fendit le cuir avec son couteau.\par
La ceinture contenait une petite boîte de fer et quelques pièces d’or. Gilliatt compta vingt guinées.\par
La boîte de fer était une vieille tabatière de matelot, s’ouvrant à ressort. Elle était très rouillée et très fermée. Le ressort, complètement oxydé, n’avait plus de jeu.\par
Le couteau tira encore d’embarras Gilliatt. Une pesée de la pointe de la lame fit sauter le couvercle de la boîte.\par
La boîte s’ouvrit.\par
 Il n’y avait dedans que du papier.\par
Une petite liasse de feuilles très minces, pliées en quatre, tapissait le fond de la boîte. Elles étaient humides, mais point altérées. La boîte, hermétiquement fermée, les avait préservées. Gilliatt les déplia.\par
C’étaient trois bank-notes de mille livres sterling chacune, faisant ensemble soixante-quinze mille francs.\par
Gilliatt les replia, les remit dans la boîte, profita d’un peu de place qui y restait pour y ajouter les vingt guinées, et referma la boîte le mieux qu’il put.\par
Il se mit à examiner la ceinture.\par
Le cuir, autrefois verni à l’extérieur, était brut à l’intérieur. Sur ce fond fauve quelques lettres étaient tracées en noir à l’encre grasse. Gilliatt déchiffra ces lettres et lut : \emph{Sieur Clubin.}
 \subsubsection[{B.IV.5. Dans l’intervalle qui sépare six pouces de deux pieds il y a de quoi loger la mort}]{B.IV.5. \\
Dans l’intervalle qui sépare six pouces de deux pieds il y a de quoi loger la mort}
\noindent Gilliatt remit la boîte dans la ceinture, et mit la ceinture dans la poche de son pantalon.\par
Il laissa le squelette aux crabes, avec la pieuvre morte à côté.\par
Pendant que Gilliatt était avec la pieuvre et avec le squelette, le flux remontant avait noyé le boyau d’entrée. Gilliatt ne put sortir qu’en plongeant sous l’arche. Il s’en tira sans peine ; il connaissait l’issue, et il était maître dans ces gymnastiques de la mer.\par
On entrevoit le drame qui s’était passé là dix semaines auparavant. Un monstre avait saisi l’autre. La pieuvre avait pris Clubin.\par
Cela avait été, dans l’ombre inexorable, presque ce qu’on pourrait nommer la rencontre des hypocrisies. Il y avait eu, au fond de l’abîme, abordage entre ces deux existences faites d’attente et de ténèbres, et l’une, qui était la bête, avait exécuté l’autre, qui était l’âme. Sinistres justices.\par
 Le crabe se nourrit de charogne, la pieuvre se nourrit de crabes. La pieuvre arrête au passage un animal nageant, une loutre, un chien, un homme si elle peut, boit le sang, et laisse au fond de l’eau le corps mort. Les crabes sont les scarabées nécrophores de la mer. La chair pourrissante les attire ; ils viennent ; ils mangent le cadavre, la pieuvre les mange. Les choses mortes disparaissent dans le crabe, le crabe disparaît dans la pieuvre. Nous avons déjà indiqué cette loi.\par
Clubin avait été l’appât de la pieuvre.\par
La pieuvre l’avait retenu et noyé ; les crabes l’avaient dévoré. Un flot quelconque l’avait poussé dans la cave, au fond de l’anfractuosité où Gilliatt l’avait trouvé.\par
Gilliatt s’en revint, furetant dans les rochers, cherchant des oursins et des patelles, ne voulant plus de crabes. Il lui eût semblé manger de la chair humaine.\par
Du reste, il ne songeait plus qu’à souper le mieux possible avant de partir. Rien désormais ne l’arrêtait. Les grandes tempêtes sont toujours suivies d’un calme qui dure plusieurs jours quelquefois. Nul danger maintenant du côté de la mer. Gilliatt était résolu à partir le lendemain. Il importait de garder pendant la nuit, à cause de la marée, le barrage ajusté entre les Douvres ; mais Gilliatt comptait défaire au point du jour ce barrage, pousser la panse hors des Douvres, et mettre à la voile pour Saint-Sampson. La brise de calme qui soufflait, et qui était sud-est, était précisément le vent qu’il lui fallait.\par
 On entrait dans le premier quartier de la lune de mai ; les jours étaient longs.\par
Quand Gilliatt, sa tournée de rôdeur de rochers terminée et son estomac à peu près satisfait, revint à l’entre-deux des Douvres où était la panse, le soleil était couché, le crépuscule se doublait de ce demi-clair de lune qu’on pourrait appeler clair de croissant ; le flux avait atteint son plein, et commençait à redescendre. La cheminée de la machine debout au-dessus de la panse avait été couverte par les écumes de la tempête d’une couche de sel que la lune blanchissait.\par
Ceci rappela à Gilliatt que la tourmente avait jeté dans la panse beaucoup d’eau de pluie et d’eau de mer, et que, s’il voulait partir le lendemain, il fallait vider la barque.\par
Il avait constaté, en quittant la panse pour aller à la chasse aux crabes, qu’il y avait environ six pouces d’eau dans la cale. Sa pelle d’épuisement suffirait pour jeter cette eau dehors.\par
Arrivé à la barque, Gilliatt eut un mouvement de terreur. Il y avait dans la panse près de deux pieds d’eau.\par
Incident redoutable, la panse faisait eau.\par
Elle s’était peu à peu emplie pendant l’absence de Gilliatt. Chargée comme elle l’était, vingt pouces d’eau étaient un surcroît périlleux. Un peu plus, elle coulait. Si Gilliatt fût revenu une heure plus tard, il n’eût probablement trouvé hors de l’eau que la cheminée et le mât.\par
 Il n’y avait pas même à prendre une minute pour délibérer. Il fallait chercher la voie d’eau, la boucher, puis vider la barque, ou du moins l’alléger. Les pompes de la Durande s’étaient perdues dans le naufrage ; Gilliatt était réduit à la pelle d’épuisement de la panse.\par
Chercher la voie d’eau, avant tout. C’était le plus pressé.\par
Gilliatt se mit à l’œuvre tout de suite, sans même se donner le temps de se rhabiller, frémissant. Il ne sentait plus ni la faim, ni le froid.\par
La panse continuait de s’emplir. Heureusement il n’y avait point de vent. Le moindre clapotement l’eût coulée.\par
La lune se coucha.\par
Gilliatt, à tâtons, courbé, plus qu’à demi plongé dans l’eau, chercha longtemps. Il découvrit enfin l’avarie.\par
Pendant la bourrasque, au moment critique où la panse s’était arquée, la robuste barque avait talonné et heurté assez violemment le rocher. Un des reliefs de la petite Douvre avait fait, dans la coque, à tribord, une fracture.\par
Cette voie d’eau était fâcheusement, on pourrait presque dire perfidement, située près du point de rencontre de deux porques, ce qui, joint à l’ahurissement de la tourmente, avait empêché Gilliatt, dans sa revue obscure et rapide au plus fort de l’orage, d’apercevoir le dégât.\par
La fracture avait cela d’alarmant qu’elle était large,  et cela de rassurant que, bien qu’immergée en ce moment par la crue intérieure de l’eau, elle était au-dessus de la flottaison.\par
A l’instant où la crevasse s’était faite, le flot était rudement secoué dans le détroit, et il n’y avait plus de niveau de flottaison, la lame avait pénétré par l’effraction dans la panse, la panse sous cette surcharge s’était enfoncée de quelques pouces, et, même après l’apaisement des vagues, le poids du liquide infiltré, faisant hausser la ligne de flottaison, avait maintenu la crevasse sous l’eau. De là, l’imminence du danger. La crue avait augmenté de six pouces à vingt. Mais si l’on parvenait à boucher la voie d’eau, on pourrait vider la panse ; une fois la barque étanchée, elle remonterait à sa flottaison normale, la fracture sortirait de l’eau, et, à sec, la réparation serait aisée, ou du moins possible. Gilliatt, nous l’avons dit, avait encore son outillage de charpentier en assez bon état.\par
Mais que d’incertitudes avant d’en venir là ! que de périls ! que de chances mauvaises ! Gilliatt entendait l’eau sourdre inexorablement. Une secousse, et tout sombrait. Quelle misère ! Peut-être n’était-il plus temps.\par
Gilliatt s’accusa amèrement. Il aurait dû voir tout de suite l’avarie. Les six pouces d’eau dans la cale auraient dû l’avertir. Il avait été stupide d’attribuer ces six pouces d’eau à la pluie et à l’écume. Il se reprocha d’avoir dormi, d’avoir mangé ; il se reprocha la fatigue, il se reprocha presque la tempête et la nuit. Tout était de sa faute.\par
 Ces duretés qu’il se disait à lui-même se mêlaient au va-et-vient de son travail et ne l’empêchaient pas d’aviser.\par
La voie d’eau était trouvée, c’était le premier pas ; l’étouper était le second. On ne pouvait davantage pour l’instant. On ne fait point de menuiserie sous l’eau.\par
Une circonstance favorable, c’est que l’effraction de la coque avait eu lieu dans l’espace compris entre les deux chaînes qui assujettissaient à tribord la cheminée de la machine. L’étoupage pouvait se rattacher à ces chaînes.\par
L’eau cependant gagnait. La crue maintenant dépassait deux pieds.\par
Gilliatt avait de l’eau plus haut que les genoux.
 \subsubsection[{B.IV.6. De profundis ad altum}]{B.IV.6. \\
De profundis ad altum}
\noindent Gilliatt avait à sa disposition, dans la réserve du gréement de la panse, un assez grand prélart goudronné pourvu de longues aiguillettes à ses quatre coins.\par
Il prit ce prélart, en amarra deux coins par les aiguillettes aux deux anneaux des chaînes de la cheminée du côté de la voie d’eau, et jeta le prélart par-dessus le bord. Le prélart tomba comme une nappe entre la petite Douvre et la barque, et s’immergea dans le flot. La poussée de l’eau voulant entrer dans la cale l’appliqua contre la coque sur le trou. Plus l’eau pressait, plus le prélart adhérait. Il était collé par le flot lui-même sur la fracture. La plaie de la barque était pansée.\par
Cette toile goudronnée s’interposait entre l’intérieur de la cale et les lames du dehors. Il n’entrait plus une goutte d’eau.\par
La voie d’eau était masquée, mais n’était pas étoupée.\par
 C’était un répit.\par
Gilliatt prit la pelle d’épuisement et se mit à vider la panse. Il était grand temps de l’alléger. Ce travail le réchauffa un peu, mais sa fatigue était extrême. Il était forcé de s’avouer qu’il n’irait pas jusqu’au bout et qu’il ne parviendrait point à étancher la cale. Gilliatt avait à peine mangé, et il avait l’humiliation de se sentir exténué.\par
Il mesurait les progrès de son travail à la baisse du niveau de l’eau à ses genoux. Cette baisse était lente.\par
En outre la voie d’eau n’était qu’interrompue. Le mal était pallié, non réparé. Le prélart, poussé dans la fracture par le flot, commençait à faire tumeur dans la cale. Cela ressemblait à un poing sous cette toile, s’efforçant de la crever. La toile solide, et goudronnée, résistait ; mais le gonflement et la tension augmentaient, il n’était pas certain que la toile ne céderait pas, et d’un moment à l’autre la tumeur pouvait se fendre. L’irruption de l’eau recommencerait.\par
En pareil cas, les équipages en détresse le savent, il n’y a pas d’autre ressource qu’un tampon. On prend les chiffons de toute espèce qu’on trouve sous sa main, tout ce que dans la langue spéciale on appelle \emph{fourrures,} et l’on refoule le plus qu’on peut dans la crevasse la tumeur du prélart.\par
De ces « fourrures », Gilliatt n’en avait point. Tout ce qu’il avait emmagasiné de lambeaux et d’étoupes, avait été ou employé dans ses travaux, ou dispersé par la rafale.\par
A la rigueur, il eût pu en retrouver quelques restes  en furetant dans les rochers. La panse était assez allégée pour qu’il pût s’absenter un quart d’heure ; mais comment faire cette perquisition sans lumière ? L’obscurité était complète. Il n’y avait plus de lune ; rien que le sombre ciel étoilé. Gilliatt n’avait pas de filin sec pour faire une mèche, pas de suif pour faire une chandelle, pas de feu pour l’allumer, pas de lanterne pour l’abriter. Tout était confus et indistinct dans la barque et dans l’écueil. On entendait l’eau bruire autour de la coque blessée, on ne voyait même pas la crevasse ; c’est avec les mains que Gilliatt constatait la tension croissante du prélart. Impossible de faire en cette obscurité une recherche utile des haillons de toile et de funin épars dans les brisants. Comment glaner ces loques sans y voir clair ? Gilliatt considérait tristement la nuit. Toutes les étoiles, et pas une chandelle.\par
La masse liquide ayant diminué dans la barque, la pression extérieure augmentait. Le gonflement du prélart grossissait. Il ballonnait de plus en plus. C’était comme un abcès prêt à s’ouvrir. La situation, un moment améliorée, redevenait menaçante.\par
Un tampon était impérieusement nécessaire.\par
Gilliatt n’avait plus que ses vêtements.\par
Il les avait, on s’en souvient, mis à sécher sur les rochers saillants de la petite Douvre.\par
Il les alla ramasser et les déposa sur le rebord de la panse.\par
Il prit son suroit goudronné, et, s’agenouillant dans l’eau, il l’enfonça dans la crevasse, repoussant la  tumeur du prélart au dehors, et par conséquent la vidant. Au suroit il ajouta la peau de mouton, à la peau de mouton la chemise de laine, à la chemise la vareuse. Tout y passa.\par
Il n’avait plus sur lui qu’un vêtement, il l’ôta, et avec son pantalon il grossit et affermit l’étoupage. Le tampon était fait, et ne semblait pas insuffisant.\par
Ce tampon débordait au dehors la crevasse, avec le prélart pour enveloppe. Le flot, voulant entrer, pressait l’obstacle, l’élargissait utilement sur la fracture, et le consolidait. C’était une sorte de compresse extérieure.\par
A l’intérieur, le centre seul du gonflement ayant été refoulé, il restait tout autour de la crevasse et du tampon un bourrelet circulaire du prélart d’autant plus adhérent que les inégalités mêmes de la fracture le retenaient. La voie d’eau était aveuglée.\par
Mais rien n’était plus précaire. Ces reliefs aigus de la fracture qui fixaient le prélart, pouvaient le percer, et par ces trous l’eau rentrerait. Gilliatt, dans l’obscurité, ne s’en apercevrait même pas. Il était peu probable que ce tampon durât jusqu’au jour. L’anxiété de Gilliatt changeait de forme, mais il la sentait croître en même temps qu’il sentait ses forces s’éteindre.\par
Il s’était remis à vider la cale, mais ses bras, à bout d’efforts, pouvaient à peine soulever la pelle pleine d’eau. Il était nu, et frissonnait.\par
Gilliatt sentait l’approche sinistre de l’extrémité.\par
Une chance possible lui traversa l’esprit. Peut-être  y avait-il une voile au large. Un pêcheur qui serait par aventure de passage dans les eaux des Douvres pourrait lui venir en aide. Le moment était arrivé où un collaborateur était absolument nécessaire. Un homme et une lanterne, et tout pouvait être sauvé. A deux, on viderait aisément la cale ; dès que la barque serait étanche, n’ayant plus cette surcharge de liquide, elle remonterait, elle reprendrait son niveau de flottaison, la crevasse sortirait de l’eau, le radoub serait exécutable, on pourrait immédiatement remplacer le tampon par une pièce de bordage, et l’appareil provisoire posé sur la fracture par une réparation définitive. Sinon, il fallait attendre jusqu’au jour, attendre toute la nuit ! Retard funeste qui pouvait être la perdition. Gilliatt avait la fièvre de l’urgence. Si par hasard quelque fanal de navire était en vue, Gilliatt pourrait, du haut de la grande Douvre, faire des signaux. Le temps était calme, il n’y avait pas de vent, il n’y avait pas de mer, un homme s’agitant sur le fond étoilé du ciel avait possibilité d’être remarqué. Un capitaine de navire, et même un patron de barque, n’est pas la nuit dans les eaux des Douvres sans braquer la longue-vue sur l’écueil ; c’est de précaution.\par
Gilliatt espéra qu’on l’apercevrait.\par
Il escalada l’épave, empoigna la corde à nœuds, et monta sur la grande Douvre.\par
Pas une voile à l’horizon. Pas un fanal. L’eau à perte de vue était déserte.\par
Nulle assistance possible et nulle résistance possible.\par
 Gilliatt, chose qu’il n’avait point éprouvée jusqu’à ce moment, se sentit désarmé.\par
La fatalité obscure était maintenant sa maîtresse. Lui avec sa barque, avec la machine de la Durande, avec toute sa peine, avec toute sa réussite, avec tout son courage, il appartenait au gouffre. Il n’avait plus de ressource de lutte ; il devenait passif. Comment empêcher le flux de venir, l’eau de monter, la nuit de continuer ? Ce tampon était son unique point d’appui. Gilliatt s’était épuisé et dépouillé à le composer et à le compléter ; il ne pouvait plus ni le fortifier, ni l’affermir ; le tampon était tel quel, il devait rester ainsi, et fatalement tout effort était fini. La mer avait à sa discrétion cet appareil hâtif appliqué sur la voie d’eau. Comment se comporterait cet obstacle inerte ? C’était lui maintenant qui combattait, ce n’était plus Gilliatt. C’était ce chiffon, ce n’était plus cet esprit. Le gonflement d’un flot suffisait pour déboucher la fracture. Plus ou moins de pression ; toute la question était là.\par
Tout allait se dénouer par une lutte machinale entre deux quantités mécaniques. Gilliatt ne pouvait désormais ni aider l’auxiliaire, ni arrêter l’ennemi. Il n’était plus que le spectateur de sa vie ou de sa mort. Ce Gilliatt, qui avait été une providence, était, à la minute suprême, remplacé par une résistance inconsciente.\par
Aucune des épreuves et des épouvantes que Gilliatt avait traversées n’approchait de celle-ci.\par
En arrivant dans l’écueil Douvres, il s’était vu entouré et comme saisi par la solitude. Cette solitude  faisait plus que l’environner, elle l’enveloppait. Mille menaces à la fois lui avaient montré le poing. Le vent était là, prêt à souffler ; la mer était là, prête à rugir. Impossible de bâillonner cette bouche, le vent ; impossible d’édenter cette gueule, la mer. Et pourtant il avait lutté ; homme, il avait combattu corps à corps l’océan ; il s’était colleté avec la tempête.\par
Il avait tenu tête à d’autres anxiétés et à d’autres nécessités encore. Il avait eu affaire à toutes les détresses. Il lui avait fallu sans outils faire des travaux, sans aide remuer des fardeaux, sans science résoudre des problèmes, sans provisions boire et manger, sans lit et sans toit dormir.\par
Sur cet écueil, chevalet tragique, il avait été tour à tour mis à la question par les diverses fatalités tortionnaires de la nature, mère quand bon lui semble, bourreau quand il lui plaît.\par
Il avait vaincu l’isolement, vaincu la faim, vaincu la soif, vaincu le froid, vaincu la fièvre, vaincu le travail, vaincu le sommeil. Il avait rencontré pour lui barrer le passage les obstacles coalisés. Après le dénûment, l’élément ; après la marée, la tourmente ; après la tempête, la pieuvre ; après le monstre, le spectre.\par
Lugubre ironie finale. Dans cet écueil d’où Gilliatt avait compté sortir triomphant, Clubin mort venait de le regarder en riant.\par
Le ricanement du spectre avait raison. Gilliatt se voyait perdu. Gilliatt se voyait aussi mort que Clubin.\par
L’hiver, la famine, la fatigue, l’épave à dépecer, la  machine à transborder, les coups d’équinoxe, le vent, le tonnerre, la pieuvre, tout cela n’était rien près de la voie d’eau. On pouvait avoir, et Gilliatt avait eu, contre le froid le feu, contre la faim les coquillages du rocher, contre la soif la pluie, contre les difficultés du sauvetage l’industrie et l’énergie, contre la marée et l’orage le brise-lames, contre la pieuvre le couteau. Contre la voie d’eau, rien.\par
L’ouragan lui laissait cet adieu sinistre. Dernière reprise, estocade traître, attaque sournoise du vaincu au vainqueur. La tempête en fuite lançait cette flèche derrière elle. La déroute se retournait et frappait. C’était le coup de jarnac de l’abîme.\par
On combat la tempête ; mais comment combattre un suintement ?\par
Si le tampon cédait, si la voie d’eau se rouvrait, rien ne pouvait faire que la panse ne sombrât point. C’était la ligature de l’artère qui se dénoue. Et une fois la panse au fond de l’eau, avec cette surcharge, la machine, nul moyen de l’en tirer. Ce magnanime effort de deux mois titaniques aboutissait à un anéantissement. Recommencer était impossible. Gilliatt n’avait plus ni forge, ni matériaux. Peut-être, au point du jour, allait-il voir toute son œuvre s’enfoncer lentement et irrémédiablement dans le gouffre.\par
Chose effrayante, sentir sous soi la force sombre.\par
Le gouffre le tirait à lui.\par
Sa barque engloutie, il n’aurait plus qu’à mourir de faim et de froid, comme l’autre, le naufragé du rocher l’Homme.\par
 Pendant deux longs mois, les consciences et les providences qui sont dans l’invisible avaient assisté à ceci : d’un côté les étendues, les vagues, les vents, les éclairs, les météores, de l’autre un homme ; d’un côté la mer, de l’autre une âme ; d’un côté l’infini, de l’autre un atome.\par
Et il y avait eu bataille.\par
Et voilà que peut-être ce prodige avortait.\par
Ainsi aboutissait à l’impuissance cet héroïsme inouï, ainsi s’achevait par le désespoir ce formidable combat accepté, cette lutte de Rien contre Tout, cette Iliade à un.\par
Gilliatt éperdu regardait l’espace.\par
Il n’avait même plus un vêtement, il était nu devant l’immensité.\par
Alors, dans l’accablement de toute cette énormité inconnue, ne sachant plus ce qu’on lui voulait, se confrontant avec l’ombre, en présence de cette obscurité irréductible, dans la rumeur des eaux, des lames, des flots, des houles, des écumes, des rafales, sous les nuées, sous les souffles, sous la vaste force éparse, sous ce mystérieux firmament des ailes, des astres et des tombes, sous l’intention possible mêlée à ces choses démesurées, ayant autour de lui et au-dessous de lui l’océan, et au-dessus de lui les constellations, sous l’insondable, il s’affaissa, il renonça, il se coucha tout de son long le dos sur la roche, la face aux étoiles, vaincu, et, joignant les mains devant la profondeur terrible, il cria dans l’infini : Grâce !\par
Terrassé par l’immensité, il la pria.\par
 Il était là, seul dans cette nuit sur ce rocher au milieu de cette mer, tombé d’épuisement, ressemblant à un foudroyé, nu comme le gladiateur dans le cirque, seulement au lieu de cirque ayant l’abîme, au lieu de bêtes féroces les ténèbres, au lieu des yeux du peuple le regard de l’inconnu, au lieu des vestales les étoiles, au lieu de César, Dieu.\par
Il lui sembla qu’il se sentait se dissoudre dans le froid, dans la fatigue, dans l’impuissance, dans la prière, dans l’ombre, et ses yeux se fermèrent.
 \subsubsection[{B.IV.7. Il y a une oreille dans l’inconnu}]{B.IV.7. \\
Il y a une oreille dans l’inconnu}
\noindent Quelques heures s’écoulèrent.\par
Le soleil se leva, éblouissant.\par
Son premier rayon éclaira sur le plateau de la grande Douvre une forme immobile. C’était Gilliatt.\par
Il était toujours étendu sur le rocher.\par
Cette nudité glacée et roidie n’avait plus un frisson. Les paupières closes étaient blêmes. Il eût été difficile de dire si ce n’était pas un cadavre.\par
Le soleil paraissait le regarder.\par
Si cet homme nu n’était pas mort, il en était si près qu’il suffisait du moindre vent froid pour l’achever.\par
Le vent se mit à souffler, tiède et vivifiant ; la printanière haleine de mai.\par
Cependant le soleil montait dans le profond ciel bleu ; son rayon moins horizontal s’empourpra. Sa lumière devint chaleur. Elle enveloppa Gilliatt.\par
Gilliatt ne bougeait pas. S’il respirait, c’était de cette respiration prête à s’éteindre qui ternirait à peine un miroir.\par
 Le soleil continua son ascension, de moins en moins oblique sur Gilliatt. Le vent, qui n’avait été d’abord que tiède, était maintenant chaud.\par
Ce corps rigide et nu demeurait toujours sans mouvement ; pourtant la peau semblait moins livide.\par
Le soleil, approchant du zénith, tomba à plomb sur le plateau de la Douvre. Une prodigalité de lumière se versa du haut du ciel ; la vaste réverbération de la mer sereine s’y joignit ; le rocher commença à tiédir, et réchauffa l’homme.\par
Un soupir souleva la poitrine de Gilliatt.\par
Il vivait.\par
Le soleil continua ses caresses, presque ardentes. Le vent, qui était déjà le vent de midi et le vent d’été, s’approcha de Gilliatt comme une bouche, soufflant mollement.\par
Gilliatt remua.\par
L’apaisement de la mer était inexprimable. Elle avait un murmure de nourrice près de son enfant. Les vagues paraissaient bercer l’écueil.\par
Les oiseaux de mer, qui connaissaient Gilliatt, volaient au-dessus de lui, inquiets. Ce n’était plus leur ancienne inquiétude sauvage. C’était on ne sait quoi de tendre et de fraternel. Ils poussaient de petits cris. Ils avaient l’air de l’appeler. Une mouette, qui l’aimait sans doute, eut la familiarité de venir tout près de lui. Elle se mit à lui parler. Il ne semblait pas entendre. Elle sauta sur son épaule et lui becqueta les lèvres doucement.\par
Gilliatt ouvrit les yeux.\par
 Les oiseaux, contents et farouches, s’envolèrent.\par
Gilliatt se dressa debout, s’étira comme le lion réveillé, courut au bord de la plate-forme, et regarda sous lui dans l’entre-deux des Douvres.\par
La panse était là, intacte. Le tampon s’était maintenu ; la mer probablement l’avait peu rudoyé.\par
Tout était sauvé.\par
Gilliatt n’était plus las. Ses forces étaient réparées. Cet évanouissement avait été un sommeil.\par
Il vida la panse, mit la cale à sec et l’avarie hors de la flottaison, se rhabilla, but, mangea, fut joyeux.\par
La voie d’eau, examinée au jour, demandait plus de travail que Gilliatt n’aurait cru. C’était une assez grave avarie. Gilliatt n’eut pas trop de toute la journée pour la réparer.\par
Le lendemain, à l’aube, après avoir défait le barrage et rouvert l’issue du défilé, vêtu de ces haillons qui avaient eu raison de la voie d’eau, ayant sur lui la ceinture de Clubin et les soixante-quinze mille francs, debout dans la panse radoubée à côté de la machine sauvée, par un bon vent, par une mer admirable, Gilliatt sortit de l’écueil Douvres.\par
Il mit le cap sur Guernesey.\par
Au moment où il s’éloigna de l’écueil, quelqu’un qui eût été là l’eût entendu chanter à demi-voix l’air Bonny Dundee.
 \section[{C. Troisième partie. Déruchette}]{C. Troisième partie \\
Déruchette}\renewcommand{\leftmark}{C. Troisième partie \\
Déruchette}

  \subsection[{C.I. Livre premier. Nuit et lune}]{C.I. Livre premier \\
Nuit et lune}
  \subsubsection[{C.I.1. La cloche du port}]{C.I.1. \\
La cloche du port}
\noindent Le Saint-Sampson d’aujourd’hui est presque une ville ; le Saint-Sampson d’il y a quarante ans était presque un village.\par
Le printemps venu et les veillées d’hiver finies, on y faisait les soirées courtes, on se mettait au lit dès la nuit tombée. Saint-Sampson était une ancienne paroisse de couvre-feu ayant conservé l’habitude de souffler de bonne heure sa chandelle. On s’y couchait et on s’y levait avec le jour. Ces vieux villages normands sont volontiers poulaillers.\par
Disons en outre que Saint-Sampson, à part quelques riches familles bourgeoises, est une population de carriers et de charpentiers. Le port est un port de radoub. Tout le jour on extrait des pierres ou l’on façonne des madriers ; ici le pic, là le marteau. Maniement perpétuel de bois de chêne et du granit. Le soir  on tombe de fatigue et l’on dort comme des plombs. Les rudes travaux font les durs sommeils.\par
Un soir du commencement de mai, après avoir, pendant quelques instants, regardé le croissant de la lune dans les arbres et écouté le pas de Déruchette se promenant seule, au frais de la nuit, dans le jardin des Bravées, mess Lethierry était rentré dans sa chambre située sur le port et s’était couché. Douce et Grâce étaient au lit. Excepté Déruchette, tout dormait dans la maison. Tout dormait aussi dans Saint-Sampson. Portes et volets étaient partout fermés. Aucune allée et venue dans les rues. Quelques rares lumières, pareilles à des clignements d’yeux qui vont s’éteindre, rougissaient çà et là des lucarnes sur les toits, annonce du coucher des domestiques. Il y avait un certain temps déjà que neuf heures avaient sonné au vieux clocher roman couvert de lierre qui partage avec l’église de Saint-Brelade de Jersey la bizarrerie d’avoir pour date quatre uns ; 1111 ; ce qui signifie \emph{onze cent onze}.\par
La popularité de mess Lethierry à Saint-Sampson tenait à son succès. Le succès ôté, le vide s’était fait. Il faut croire que le guignon se gagne et que les gens point heureux ont la peste, tant est rapide leur mise en quarantaine. Les jolis fils de famille évitaient Déruchette. L’isolement autour des Bravées était maintenant tel qu’on n’y avait pas même su le petit grand événement local qui avait ce jour-là mis tout Saint-Sampson en rumeur. Le recteur de la paroisse, le révérend Joë Ebenezer Caudray, était riche. Son oncle, le magnifique doyen de Saint-Asaph, venait de mourir  à Londres. La nouvelle en avait été apportée par le sloop de poste \emph{Cashmere} arrivé d’Angleterre le matin même, et dont on apercevait le mât dans la rade de Saint-Pierre-Port. Le \emph{Cashmere} devait repartir pour Southampton le lendemain à midi, et, disait-on, emmener le révérend recteur, rappelé en Angleterre à bref délai pour l’ouverture officielle du testament, sans compter les autres urgences d’une grande succession à recueillir. Toute la journée, Saint-Sampson avait confusément dialogué. Le \emph{Cashmere,} le révérend Ebenezer, son oncle mort, sa richesse, son départ, ses promotions possibles dans l’avenir, avaient fait le fond du bourdonnement. Une seule maison, point informée, était restée silencieuse, les Bravées.\par
Mess Lethierry s’était jeté sur son branle, tout habillé. Depuis la catastrophe de la Durande, se jeter sur son branle, c’était sa ressource. S’étendre sur son grabat, c’est à quoi tout prisonnier a recours, et mess Lethierry était le prisonnier du chagrin. Il se couchait ; c’était une trêve, une reprise d’haleine, une suspension d’idées. Dormait-il ? non. Veillait-il ? non. A proprement parler, depuis deux mois et demi, — il y avait deux mois et demi de cela, — mess Lethierry était comme en somnambulisme. Il ne s’était pas encore ressaisi lui-même. Il était dans cet état mixte et diffus que connaissent ceux qui ont subi les grands accablements. Ses réflexions n’étaient pas de la pensée, son sommeil n’était pas du repos. Le jour il n’était pas un homme éveillé, la nuit il n’était pas un homme endormi. Il était debout, puis il était couché, voilà  tout. Quand il était dans son branle, l’oubli lui venait un peu, il appelait cela dormir, les chimères flottaient sur lui et en lui, le nuage nocturne, plein de faces confuses, traversait son cerveau ; l’empereur Napoléon lui dictait ses mémoires, il y avait plusieurs Déruchettes, des oiseaux bizarres étaient dans des arbres, les rues de Lons-le-Saulnier devenaient des serpents. Le cauchemar était le répit du désespoir. Il passait ses nuits à rêver, et ses jours à songer.\par
Il restait quelquefois toute une après-midi, immobile à la fenêtre de sa chambre qui donnait, on s’en souvient, sur le port, la tête basse, les coudes sur la pierre, les oreilles dans ses poings, le dos tourné au monde entier, l’œil fixé sur le vieil anneau de fer scellé dans le mur de sa maison à quelques pieds de sa fenêtre, où jadis on amarrait la Durande. Il regardait la rouille qui venait à cet anneau.\par
Mess Lethierry était réduit à la fonction machinale de vivre.\par
Les plus vaillants hommes, privés de leur idée réalisable, en arrivent là. C’est l’effet des existences vidées. La vie est le voyage, l’idée est l’itinéraire. Plus d’itinéraire, on s’arrête. Le but est perdu, la force est morte. Le sort a un obscur pouvoir discrétionnaire. Il peut toucher de sa verge même notre être moral. Le désespoir, c’est presque la destitution de l’âme. Les très grands esprits seuls résistent. Et encore.\par
Mess Lethierry méditait continuellement, si l’absorption peut s’appeler méditation, au fond d’une sorte de précipice trouble. Il lui échappait des paroles navrées  comme celle-ci : — Il ne me reste plus qu’à demander là-haut mon billet de sortie.\par
Notons une contradiction dans cette nature, complexe comme la mer dont Lethierry était, pour ainsi dire, le produit ; mess Lethierry ne priait point.\par
Être impuissant, c’est une force. En présence de nos deux grandes cécités, la destinée et la nature, c’est dans son impuissance que l’homme a trouvé le point d’appui, la prière.\par
L’homme se fait secourir par l’effroi ; il demande aide à sa crainte ; l’anxiété, c’est un conseil d’agenouillement.\par
La prière, énorme force propre à l’âme et de même espèce que le mystère. La prière s’adresse à la magnanimité des ténèbres ; la prière regarde le mystère avec les yeux mêmes de l’ombre, et, devant la fixité puissante de ce regard suppliant, on sent un désarmement possible de l’Inconnu.\par
Cette possibilité entrevue est déjà une consolation.\par
Mais Lethierry ne priait pas.\par
Du temps qu’il était heureux, Dieu existait pour lui, on pourrait dire en chair et en os ; Lethierry lui parlait, lui engageait sa parole, lui donnait presque de temps en temps une poignée de main. Mais dans le malheur de Lethierry, phénomène du reste assez fréquent, Dieu s’était éclipsé. Cela arrive quand on s’est fait un bon Dieu qui est un bonhomme.\par
Il n’y avait pour Lethierry, dans l’état d’âme où il était, qu’une vision nette, le sourire de Déruchette. Hors de ce sourire, tout était noir.\par
 Depuis quelque temps, sans doute à cause de la perte de la Durande, dont elle ressentait le contre-coup, ce charmant sourire de Déruchette était plus rare. Elle paraissait préoccupée. Ses gentillesses d’oiseau et d’enfant s’étaient éteintes. On ne la voyait plus, le matin, au coup de canon du point du jour, faire une révérence et dire au soleil levant : « \emph{Bum !... jour}. Donnez-vous la peine d’entrer. » Elle avait par moments l’air très sérieux, chose triste dans ce doux être. Elle faisait effort cependant pour rire à mess Lethierry, et pour le distraire, mais sa gaîté se ternissait de jour en jour et se couvrait de poussière, comme l’aile d’un papillon qui a une épingle à travers le corps. Ajoutons que, soit par chagrin du chagrin de son oncle, car il y a des douleurs de reflet, soit pour d’autres raisons, elle semblait maintenant incliner beaucoup vers la religion. Du temps de l’ancien recteur M. Jaquemin Hérode, elle n’allait guère, on le sait, que quatre fois l’an à l’église. Elle y était à présent fort assidue. Elle ne manquait aucun office, ni du dimanche, ni du jeudi. Les âmes pieuses de la paroisse voyaient avec satisfaction cet amendement. Car c’est un grand bonheur qu’une jeune fille, qui court tant de dangers du côté des hommes, se tourne vers Dieu.\par
Cela fait du moins que les pauvres parents ont l’esprit en repos du côté des amourettes.\par
Le soir, toutes les fois que le temps le permettait, elle se promenait une heure ou deux dans le jardin des Bravées. Elle était là, presque aussi pensive que mess Lethierry, et toujours seule. Déruchette se  couchait la dernière. Ce qui n’empêchait point Douce et Grâce d’avoir toujours un peu l’œil sur elle, par cet instinct de guet qui se mêle à la domesticité ; espionner désennuie de servir.\par
Quant à mess Lethierry, dans l’état voilé où était son esprit, ces petites altérations dans les habitudes de Déruchette lui échappaient. D’ailleurs, il n’était pas né duègne. Il ne remarquait même pas l’exactitude de Déruchette aux offices de la paroisse. Tenace dans son préjugé contre les choses et les gens du clergé, il eût vu sans plaisir ces fréquentations d’église.\par
Ce n’est pas que sa situation morale à lui-même ne fût en train de se modifier. Le chagrin est nuage et change de forme.\par
Les âmes robustes, nous venons de le dire, sont parfois, par de certains coups de malheur, destituées presque, non tout à fait. Les caractères virils, tels que Lethierry, réagissent dans un temps donné. Le désespoir a des degrés remontants. De l’accablement on monte à l’abattement, de l’abattement à l’affliction, de l’affliction à la mélancolie. La mélancolie est un crépuscule. La souffrance s’y fond dans une sombre joie.\par
La mélancolie, c’est le bonheur d’être triste.\par
Ces atténuations élégiaques n’étaient point faites pour Lethierry ; ni la nature de son tempérament, ni le genre de son malheur, ne comportaient ces nuances. Seulement, au moment où nous venons de le retrouver, la rêverie de son premier désespoir tendait, depuis une semaine environ, à se dissiper ; sans être moins triste, Lethierry était moins inerte ; il était toujours  sombre, mais il n’était plus morne ; il lui revenait une certaine perception des faits et des événements ; et il commençait à éprouver quelque chose de ce phénomène qu’on pourrait appeler la rentrée dans la réalité.\par
Ainsi, le jour, dans la salle basse, il n’écoutait pas les paroles des gens, mais il les entendait. Grâce vint un matin toute triomphante dire à Déruchette que mess Lethierry avait défait la bande d’un journal.\par
Cette demi-acceptation de la réalité est, en soi, un bon symptôme. C’est la convalescence. Les grands malheurs sont un étourdissement. On en sort par là. Mais cette amélioration fait d’abord l’effet d’une aggravation. L’état de rêve antérieur émoussait la douleur ; on voyait trouble, on sentait peu ; à présent la vue est nette, on n’échappe à rien, on saigne de tout. La plaie s’avive. La douleur s’accentue de tous les détails qu’on aperçoit. On revoit tout dans le souvenir. Tout retrouver, c’est tout regretter. Il y a dans ce retour au réel toutes sortes d’arrière-goûts amers. On est mieux, et pire. C’est ce qu’éprouvait Lethierry. Il souffrait plus distinctement.\par
Ce qui avait ramené mess Lethierry au sentiment de la réalité, c’était une secousse.\par
Disons cette secousse.\par
Une après-midi, vers le 15 ou le 20 avril, on avait entendu à la porte de la salle basse des Bravées les deux coups qui annoncent le facteur. Douce avait ouvert. C’était une lettre en effet.\par
Cette lettre venait de la mer. Elle était adressée à mess Lethierry. Elle était timbrée \emph{Lisboa}.\par
 Douce avait porté la lettre à mess Lethierry qui était enfermé dans sa chambre. Il avait pris cette lettre, l’avait machinalement posée sur sa table, et ne l’avait pas regardée.\par
Cette lettre resta une bonne semaine sur la table sans être décachetée.\par
Il arriva pourtant qu’un matin Douce dit à mess Lethierry :\par
— Monsieur, faut-il ôter la poussière qu’il y a sur votre lettre ?\par
Lethierry parut se réveiller.\par
— C’est juste, dit-il.\par
Et il ouvrit la lettre.\par
Il lut ceci :\par
\bigbreak

\dateline{« En mer, ce 10 mars.}
\noindent « Mess Lethierry, de Saint-Sampson,\par
« Vous recevrez de mes nouvelles avec plaisir.\par
« Je suis sur le \emph{Tamaulipas,} en route pour Pasrevenir. Il y a dans l’équipage un matelot Ahier-Tostevin, de Guernesey, qui reviendra, lui, et qui aura des choses à raconter. Je profite de la rencontre du navire \emph{Hernan Cortez} allant à Lisbonne pour vous faire passer cette lettre.\par
« Soyez étonné. Je suis honnête homme.\par
« Aussi honnête que sieur Clubin.\par
« Je dois croire que vous savez la chose qui est arrivée ; pourtant il n’est peut-être pas de trop que je vous l’apprenne.\par
« La voici :\par
 « Je vous ai rendu vos capitaux.\par
« Je vous avais emprunté, un peu incorrectement, cinquante mille francs. Avant de quitter Saint-Malo, j’ai remis pour vous à votre homme de confiance, sieur Clubin, trois bank-notes de mille livres chaque, ce qui fait soixante-quinze mille francs. Vous trouverez sans doute ce remboursement suffisant.\par
« Sieur Clubin a pris vos intérêts et reçu votre argent avec énergie. Il m’a paru très zélé ; c’est pourquoi je vous avertis.\par
\bigbreak
\noindent « Votre autre homme de confiance,\par
\bigbreak

\byline{« R{\scshape antaine}.}
\bigbreak
\noindent « \emph{Post-scriptum.} Sieur Clubin avait un revolver, ce qui fait que je n’ai pas de reçu. »\par
\bigbreak
\noindent Touchez une torpille, touchez une bouteille de Leyde chargée, vous ressentirez ce qu’éprouva mess Lethierry en lisant cette lettre.\par
Sous cette enveloppe, dans cette feuille de papier pliée en quatre à laquelle il avait au premier moment fait si peu attention, il y avait une commotion.\par
Il reconnut cette écriture, il reconnut cette signature. Quant au fait, tout d’abord il n’y comprit rien.\par
Commotion telle qu’elle lui remit, pour ainsi dire, l’esprit sur pied.\par
Le phénomène des soixante-quinze mille francs confiés par Rantaine à Clubin, étant une énigme, était le côté utile de la secousse, en ce qu’il forçait le cerveau de Lethierry à travailler. Faire une conjecture,  c’est pour la pensée une occupation saine. Le raisonnement est éveillé, la logique est appelée.\par
Depuis quelque temps l’opinion publique de Guernesey était occupée à rejuger Clubin, cet honnête homme pendant tant d’années si unanimement admis dans la circulation de l’estime. On s’interrogeait, on se prenait à douter, il y avait des paris pour et contre. Des lumières singulières s’étaient produites. Clubin commençait à s’éclairer, c’est-à-dire qu’il devenait noir.\par
Une information judiciaire avait eu lieu à Saint-Malo pour savoir ce qu’était devenu le garde-côte 619. La perspicacité légale avait fait fausse route, ce qui lui arrive souvent. Elle était partie de cette supposition que le garde-côte avait dû être embauché par Zuela et embarqué sur le \emph{Tamaulipas} pour le Chili. Cette hypothèse ingénieuse avait entraîné force aberrations. La myopie de la justice n’avait pas même aperçu Rantaine. Mais, chemin faisant, les magistrats instructeurs avaient levé d’autres pistes. L’obscure affaire s’était compliquée. Clubin avait fait son entrée dans l’énigme. Il s’était établi une coïncidence, un rapport peut-être, entre le départ du \emph{Tamaulipas} et la perte de la Durande. Au cabaret de la porte Dinan où Clubin croyait n’être pas connu, on l’avait reconnu ; le cabaretier avait parlé ; Clubin avait acheté une bouteille d’eau-de-vie. Pour qui ? L’armurier de la rue Saint-Vincent avait parlé ; Clubin avait acheté un revolver. Contre qui ? L’aubergiste de l’Auberge Jean avait parlé ; Clubin avait eu des absences inexplicables. Le capitaine Gertrais-Gaboureau avait parlé ; Clubin avait  voulu partir, quoique averti, et sachant qu’il allait chercher le brouillard. L’équipage de la Durande avait parlé. Au fait, le chargement était manqué et l’arrimage était mal fait, négligence aisée à comprendre, si le capitaine veut perdre le navire. Le passager guernesiais avait parlé ; Clubin avait cru naufrager sur les Hanois. Les gens de Torteval avaient parlé ; Clubin y était venu quelques jours avant la perte de la Durande, et avait dirigé sa promenade vers Plainmont voisin des Hanois. Il portait un sac-valise. « Il était parti avec, et revenu sans. » Les déniquoiseaux avaient parlé ; leur histoire avait paru pouvoir se rattacher à la disparition de Clubin, à la seule condition d’y remplacer les revenants par des contrebandiers. Enfin la maison visionnée de Plainmont elle-même avait parlé ; des gens décidés à se renseigner l’avaient escaladée, et avaient trouvé dedans, quoi ? précisément le sac-valise de Clubin. La Douzaine de Torteval avait saisi le sac, et l’avait fait ouvrir. Il contenait des provisions de bouche, une longue-vue, un chronomètre, des vêtements d’homme et du linge marqué aux initiales de Clubin. Tout cela, dans les propos de Saint-Malo et de Guernesey, se construisait, et finissait par faire un à peu près de baraterie. On rapprochait des linéaments confus ; on constatait un dédain singulier des avis, une acceptation des chances de brouillard, une négligence suspecte dans l’arrimage, une bouteille d’eau-de-vie, un timonier ivre, une substitution du capitaine au timonier, un coup de barre au moins bien maladroit. L’héroïsme à demeurer sur l’épave devenait coquinerie. Clubin du  reste s’était trompé d’écueil. L’intention de baraterie admise, on comprenait le choix des Hanois, la côte aisément gagnée à la nage, un séjour dans la maison visionnée en attendant l’occasion de fuir. Le sac-valise, cet en-cas, achevait la démonstration. Par quel lien cette aventure se rattachait-elle à l’autre aventure, celle du garde-côte, on ne le saisissait point. On devinait une corrélation ; rien de plus. On entrevoyait, du côté de cet homme, le garde-marine numéro 619, tout un drame tragique. Clubin peut-être n’y jouait pas, mais on l’apercevait dans la coulisse.\par
Tout ne s’expliquait point par la baraterie. Il y avait un revolver sans emploi. Ce revolver était probablement de l’autre affaire.\par
Le flair du peuple est fin et juste. L’instinct public excelle dans ces restaurations de la vérité faites de pièces et de morceaux. Seulement, dans ces faits d’où se dégageait une baraterie vraisemblable, il y avait de sérieuses incertitudes.\par
Tout se tenait, tout concordait ; mais la base manquait.\par
On ne perd pas un navire pour le plaisir de le perdre. On ne court point tous ces risques de brouillard, d’écueil, de nage, de refuge et de fuite, sans un intérêt. Quel avait pu être l’intérêt de Clubin ?\par
On voyait son acte, on ne voyait pas son motif.\par
De là un doute dans beaucoup d’esprits. Où il n’y a point de motif, il semble qu’il n’y ait plus d’acte.\par
La lacune était grave.\par
 Cette lacune, la lettre de Rantaine venait la combler.\par
Cette lettre donnait le motif de Clubin. Soixante-quinze mille francs à voler.\par
Rantaine était le dieu dans la machine. Il descendait du nuage une chandelle à la main.\par
Sa lettre était le coup de clarté final.\par
Elle expliquait tout, et surabondamment elle annonçait un témoignage, Ahier-Tostevin.\par
Chose décisive, elle donnait l’emploi du revolver.\par
Rantaine était incontestablement tout à fait informé. Sa lettre faisait toucher tout du doigt.\par
Aucune atténuation possible à la scélératesse de Clubin. Il avait prémédité le naufrage ; et la preuve, c’était l’en-cas apporté dans la maison visionnée. Et en le supposant innocent, en admettant le naufrage fortuit, n’eût-il pas dû, au dernier moment, décidé à son sacrifice sur l’épave, remettre les soixante-quinze mille francs pour mess Lethierry aux hommes qui se sauvaient dans la chaloupe ? L’évidence éclatait. Maintenant qu’était devenu Clubin ? Il avait probablement été victime de sa méprise. Il avait sans doute péri dans l’écueil Douvres.\par
Cet échafaudage de conjectures, très conformes, on le voit, à la réalité, occupa pendant plusieurs jours l’esprit de mess Lethierry. La lettre de Rantaine lui rendit ce service de le forcer à penser. Il eut un premier ébranlement de surprise, puis il fit cet effort de se mettre à réfléchir. Il fit l’autre effort plus difficile encore de s’informer. Il dut accepter et même chercher  des conversations. Au bout de huit jours, il était redevenu, jusqu’à un certain point, pratique ; son esprit avait repris de l’adhérence, et était presque guéri. Il était sorti de l’état trouble.\par
La lettre de Rantaine, en admettant que mess Lethierry eût pu jamais entretenir quelque espoir de remboursement de ce côté-là, fit évanouir sa dernière chance.\par
Elle ajouta à la catastrophe de la Durande ce nouveau naufrage de soixante-quinze mille francs. Elle le remit en possession de cet argent juste assez pour lui en faire sentir la perte. Cette lettre lui montra le fond de sa ruine.\par
De là une souffrance nouvelle, et très aiguë, que nous avons indiquée tout à l’heure. Il commença, chose qu’il n’avait point faite depuis deux mois, à se préoccuper de sa maison, de ce qu’elle allait devenir, de ce qu’il faudrait réformer. Petit ennui à mille pointes, presque pire que le désespoir. Subir son malheur par le menu, disputer pied à pied au fait accompli le terrain qu’il vient vous prendre, c’est odieux. Le bloc du malheur s’accepte, non sa poussière. L’ensemble accablait, le détail torture. Tout à l’heure la catastrophe vous foudroyait, maintenant elle vous chicane.\par
C’est l’humiliation aggravant l’écrasement. C’est une deuxième annulation s’ajoutant à la première, et laide. On descend d’un degré dans le néant. Après le linceul, c’est le haillon.\par
Songer à décroître. Il n’est pas de pensée plus triste.\par
 Être ruiné, cela semble simple. Coup violent ; brutalité du sort ; c’est la catastrophe une fois pour toutes. Soit. On l’accepte. Tout est fini. On est ruiné. C’est bon, on est mort. Point. On est vivant. Dès le lendemain, on s’en aperçoit. A quoi ? A des piqûres d’épingle. Tel passant ne vous salue plus, les factures des marchands pleuvent, voilà un de vos ennemis qui rit. Peut-être rit-il du dernier calembour d’Arnal, mais c’est égal, ce calembour ne lui semble si charmant que parce que vous êtes ruiné. Vous lisez votre amoindrissement même dans les regards indifférents ; les gens qui dînaient chez vous trouvent que c’était trop de trois plats à votre table ; vos défauts sautent aux yeux de tout le monde ; les ingratitudes, n’attendant plus rien, s’affichent ; tous les imbéciles ont prévu ce qui vous arrive ; les méchants vous déchirent, les pires vous plaignent. Et puis cent détails mesquins. La nausée succède aux larmes. Vous buviez du vin, vous boirez du cidre. Deux servantes ! c’est déjà trop d’une. Il faudra congédier celle-ci et surcharger celle-là. Il y a trop de fleurs dans le jardin ; on plantera des pommes de terre. On donnait ses fruits à ses amis, on les fera vendre au marché. Quant aux pauvres, il n’y faut plus songer ; n’est-on pas un pauvre soi-même ? Les toilettes, question poignante. Retrancher un ruban à une femme, quel supplice ! A qui vous donne la beauté, refuser la parure ! Avoir l’air d’un avare ! Elle va peut-être vous dire : — Quoi, vous avez ôté les fleurs de mon jardin, et voilà que vous les ôtez de mon chapeau ! — Hélas ! la condamner aux robes fanées ! La table  de famille est silencieuse. Vous vous figurez qu’autour de vous on vous en veut. Les visages aimés sont soucieux. Voilà ce que c’est que décroître. Il faut remourir tous les jours. Tomber, ce n’est rien, c’est la fournaise. Décroître, c’est le petit feu.\par
L’écroulement, c’est Waterloo ; la diminution, c’est Sainte-Hélène. Le sort, incarné en Wellington, a encore quelque dignité ; mais quand il se fait Hudson Lowe, quelle vilenie ! Le destin devient un pleutre. On voit l’homme de Campo-Formio querellant pour une paire de bas de soie. Rapetissement de Napoléon qui rapetisse l’Angleterre.\par
Ces deux phases, Waterloo et Sainte-Hélène, réduites aux proportions bourgeoises, tout homme ruiné les traverse.\par
Le soir que nous avons dit, et qui était un des premiers soirs de mai, Lethierry, laissant Déruchette errer au clair de lune dans le jardin, s’était couché plus triste que jamais.\par
Tous ces détails chétifs et déplaisants, complications des fortunes perdues, toutes ces préoccupations du troisième ordre, qui commencent par être insipides et qui finissent par être lugubres, roulaient dans son esprit. Maussade encombrement de misères. Mess Lethierry sentait sa chute irrémédiable. Qu’allait-on faire ? Qu’allait-on devenir ? Quels sacrifices faudrait-il imposer à Déruchette ? Qui renvoyer, de Douce ou de Grâce ? Vendrait-on les Bravées ? N’en serait-on pas réduit à quitter l’île ? N’être rien là où l’on a été tout, déchéance insupportable en effet.\par
 Et dire que c’était fini ! Se rappeler ces traversées liant la France à l’Archipel, ces mardis du départ, ces vendredis du retour, la foule sur le quai, ces grands chargements, cette industrie, cette prospérité, cette navigation directe et fière, cette machine où l’homme met sa volonté, cette chaudière toute-puissante, cette fumée, cette réalité ! Le navire à vapeur, c’est la boussole complétée ; la boussole indique le droit chemin, la vapeur le suit. L’une propose, l’autre exécute. Où était-elle, sa Durande ; cette magnifique et souveraine Durande, cette maîtresse de la mer, cette reine qui le faisait roi ? Avoir été dans son pays l’homme idée, l’homme succès, l’homme révolution ! y renoncer ! abdiquer ! N’être plus ! faire rire ! Être un sac où il y a eu quelque chose ! Être le passé quand on a été l’avenir ! aboutir à la pitié hautaine des idiots ! voir triompher la routine, l’entêtement, l’ornière, l’égoïsme, l’ignorance ! voir recommencer bêtement les va-et-vient des coutres gothiques cahotés sur le flot ! voir la vieillerie rajeunir ! Avoir perdu toute sa vie ! avoir été la lumière et subir l’éclipse ! Ah ! comme c’était beau sur les vagues cette cheminée altière, ce prodigieux cylindre, ce pilier au chapiteau de fumée, cette colonne plus grande que la colonne Vendôme, car sur l’une il n’y a qu’un homme et sur l’autre il y a le progrès ! L’océan était dessous. C’était la certitude en pleine mer. On avait vu cela dans cette petite île, dans ce petit port, dans ce petit Saint-Sampson ? Oui, on l’avait vu ! Quoi ! on l’a vu, et on ne le reverra plus !\par
Toute cette obsession du regret torturait Lethierry.  Il y a des sanglots de la pensée. Jamais peut-être il n’avait plus amèrement senti sa perte. Un certain engourdissement suit ces accès aigus. Sous cet appesantissement de tristesse, il s’assoupit.\par
Il resta environ deux heures les paupières fermées, dormant un peu, songeant beaucoup, fiévreux. Ces torpeurs-là couvrent un obscur travail du cerveau, très fatigant. Vers le milieu de la nuit, vers minuit, un peu avant, ou un peu après, il secoua cet assoupissement. Il se réveilla, il ouvrit les yeux, sa fenêtre faisait face à son hamac, il vit une chose extraordinaire.\par
Une forme était devant sa fenêtre. Une forme inouïe. La cheminée d’un bateau à vapeur.\par
Mess Lethierry se dressa tout d’une pièce sur son séant. Le hamac oscilla comme au branle d’une tempête. Lethierry regarda. Il y avait dans la fenêtre une vision. Le port plein de clair de lune s’encadrait dans les vitres, et sur cette clarté, tout près de la maison, se découpait, droite, ronde et noire, une silhouette superbe.\par
Un tuyau de machine était là.\par
Lethierry se précipita à bas du hamac, courut à la fenêtre, leva le châssis, se pencha dehors, et la reconnut.\par
La cheminée de la Durande était devant lui.\par
Elle était à l’ancienne place.\par
Ses quatre chaînes la maintenaient amarrée au bordage d’un bateau dans lequel, au-dessous d’elle, on distinguait une masse qui avait un contour compliqué.\par
 Lethierry recula, tourna le dos à la fenêtre, et retomba assis sur le hamac.\par
Il se retourna, et revit la vision.\par
Un moment après, le temps d’un éclair, il était sur le quai, une lanterne à la main.\par
Au vieil anneau d’amarrage, de la Durande était attachée une barque portant un peu à l’arrière un bloc massif d’où sortait la cheminée droite devant la fenêtre des Bravées. L’avant de la barque se prolongeait, en dehors du coin du mur de la maison, à fleur de quai.\par
Il n’y avait personne dans la barque.\par
Cette barque avait une forme à elle et dont tout Guernesey eût donné le signalement. C’était la panse.\par
Lethierry sauta dedans. Il courut à la masse qu’il voyait au delà du mât. C’était la machine.\par
Elle était là, entière, complète, intacte, carrément assise sur son plancher de fonte ; la chaudière avait toutes ses cloisons ; l’arbre des roues était dressé et amarré près de la chaudière ; la pompe de saumure était à sa place ; rien ne manquait.\par
Lethierry examina la machine.\par
La lanterne et la lune s’entr’aidaient pour l’éclairer.\par
Il passa tout le mécanisme en revue.\par
Il vit les deux caisses qui étaient à côté. Il regarda l’arbre des roues.\par
Il alla à la cabine. Elle était vide.\par
Il revint à la machine et la toucha. Il avança sa tête dans la chaudière. Il se mit à genoux pour voir dedans.\par
 Il posa dans le fourneau sa lanterne dont la lueur illumina toute la mécanique et produisit presque le trompe-l’œil d’une machine allumée.\par
Puis il éclata de rire, et, se redressant, l’œil fixé sur la machine, les bras tendus vers la cheminée, il cria : Au secours !\par
La cloche du port était sur le quai à quelques pas, il y courut, empoigna la chaîne et se mit à secouer la cloche impétueusement.
 \subsubsection[{C.I.2. Encore la cloche du port}]{C.I.2. \\
Encore la cloche du port}
\noindent Gilliatt en effet, après une traversée sans incident, mais un peu lente à cause de la pesanteur du chargement de la panse, était arrivé à Saint-Sampson à la nuit close, plus près de dix heures que de neuf.\par
Gilliatt avait calculé l’heure. La demi-remontée s’était faite. Il y avait de la lune et de l’eau ; on pouvait entrer dans le port.\par
Le petit havre était endormi. Quelques navires y étaient mouillés, cargues sur vergues, hunes capelées, et sans fanaux. On apercevait au fond quelques barques au radoub, à sec dans le carénage. Grosses coques démâtées et sabordées, dressant au-dessus de leur bordage troué de claires-voies les pointes courbes de leur membrure dénudée, assez semblables à des scarabées morts couchés sur le dos, pattes en l’air.\par
Gilliatt, sitôt le goulet franchi, avait examiné le port et le quai. Il n’y avait de lumière nulle part, pas plus aux Bravées qu’ailleurs. Il n’y avait point de  passants, excepté peut-être quelqu’un, un homme, qui venait d’entrer au presbytère ou d’en sortir. Et encore n’était-on pas sûr que ce fût une personne, la nuit estompant tout ce qu’elle dessine et le clair de lune ne faisant jamais rien que d’indécis. La distance s’ajoutait à l’obscurité. Le presbytère d’alors était situé de l’autre côté du port, sur un emplacement où est construite aujourd’hui une cale couverte.\par
Gilliatt avait silencieusement accosté les Bravées, et avait amarré la panse à l’anneau de la Durande sous la fenêtre de mess Lethierry.\par
Puis il avait sauté par-dessus le bordage et pris terre.\par
Gilliatt, laissant derrière lui la panse à quai, tourna la maison, longea une ruette, puis une autre, ne regarda même pas l’embranchement de sentier qui menait au Bû de la Rue, et, au bout de quelques minutes, s’arrêta dans ce recoin de muraille où il y avait une mauve sauvage à fleurs roses en juin, du houx, du lierre et des orties. C’est de là que, caché sous les ronces, assis sur une pierre, bien des fois, dans les jours d’été, et pendant de longues heures et pendant des mois entiers, il avait contemplé, par-dessus le mur, bas au point de tenter l’enjambée, le jardin des Bravées, et, à travers les branches d’arbres, deux fenêtres d’une chambre de la maison. Il retrouva sa pierre, sa ronce, toujours le mur aussi bas, toujours l’angle aussi obscur, et, comme une bête rentrée au trou, glissant plutôt que marchant, il se blottit. Une fois assis, il ne fit plus un mouvement. Il regarda. Il revoyait le jardin,  les allées, les massifs, les carrés de fleurs, la maison, les deux fenêtres de la chambre. La lune lui montrait ce rêve. Il est affreux qu’on soit forcé de respirer. Il faisait ce qu’il pouvait pour s’en empêcher.\par
Il lui semblait voir un paradis fantôme. Il avait peur que tout cela ne s’envolât. Il était presque impossible que ces choses fussent réellement sous ses yeux ; et si elles y étaient, ce ne pouvait être qu’avec l’imminence d’évanouissement qu’ont toujours les choses divines. Un souffle, et tout se dissiperait. Gilliatt avait ce tremblement.\par
Tout près, en face de lui, dans le jardin, au bord d’une allée, il y avait un banc de bois peint en vert. On se souvient de ce banc.\par
Gilliatt regardait les deux fenêtres. Il pensait à un sommeil possible de quelqu’un dans cette chambre. Derrière ce mur, on dormait. Il eût voulu ne pas être où il était. Il eût mieux aimé mourir que de s’en aller. Il pensait à une haleine soulevant une poitrine. Elle, ce mirage, cette blancheur dans une nuée, cette obsession flottante de son esprit, elle était là ! il pensait à l’inaccessible qui était endormi, et si près, et comme à la portée de son extase ; il pensait à la femme impossible assoupie, et visitée, elle aussi, par les chimères ; à la créature souhaitée, lointaine, insaisissable, fermant les yeux, le front dans la main ; au mystère du sommeil de l’être idéal ; aux songes que peut faire un songe. Il n’osait penser au delà et il pensait pourtant ; il se risquait dans les manques de respect de la rêverie, la quantité de forme féminine  que peut avoir un ange le troublait, l’heure nocturne enhardit aux regards furtifs les yeux timides, il s’en voulait d’aller si avant, il craignait de profaner en réfléchissant ; malgré lui, forcé, contraint, frémissant, il regardait dans l’invisible. Il subissait le frisson, et presque la souffrance, de se figurer un jupon sur une chaise, une mante jetée sur le tapis, une ceinture débouclée, un fichu. Il imaginait un corset, un lacet traînant à terre, des bas, des jarretières. Il avait l’âme dans les étoiles.\par
Les étoiles sont faites aussi bien pour le cœur humain d’un pauvre comme Gilliatt que pour le cœur humain d’un millionnaire. A un certain degré de passion, tout homme est sujet aux profonds éblouissements. Si c’est une nature âpre et primitive, raison de plus.\par
Être sauvage, cela s’ajoute au rêve.\par
Le ravissement est une plénitude qui déborde comme une autre. Voir ces fenêtres, c’était presque trop pour Gilliatt.\par
Tout à coup, il la vit elle-même.\par
Des branchages d’un fourré déjà épaissi par le printemps, sortit, avec une ineffable lenteur spectrale et céleste, une figure, une robe, un visage divin, presque une clarté sous la lune.\par
Gilliatt se sentit défaillir, c’était Déruchette.\par
Déruchette approcha. Elle s’arrêta. Elle fit quelques pas pour s’éloigner, s’arrêta encore, puis revint s’asseoir sur le banc de bois. La lune était dans les arbres, quelques nuées erraient parmi les étoiles pâles, la mer parlait aux choses de l’ombre à demi-voix, la ville  dormait, une brume montait de l’horizon, cette mélancolie était profonde. Déruchette inclinait le front, avec cet œil pensif qui regarde attentivement rien ; elle était assise de profil, elle était presque nu-tête, ayant un bonnet dénoué qui laissait voir sur sa nuque délicate la naissance des cheveux, elle roulait machinalement un ruban de ce bonnet autour d’un de ses doigts, la pénombre modelait ses mains de statue, sa robe était d’une de ces nuances que la nuit fait blanches, les arbres remuaient comme s’ils étaient pénétrables à l’enchantement qui se dégageait d’elle, on voyait le bout d’un de ses pieds, il y avait dans ses cils baissés cette vague contraction qui annonce une larme rentrée ou une pensée refoulée, ses bras avaient l’indécision ravissante de ne point trouver où s’accouder, quelque chose qui flotte un peu se mêlait à toute sa posture, c’était plutôt une lueur qu’une lumière et une grâce qu’une déesse, les plis du bas de sa jupe étaient exquis, son adorable visage méditait virginalement. Elle était si près que c’était terrible. Gilliatt l’entendait respirer.\par
Il y avait dans des profondeurs un rossignol qui chantait. Les passages du vent dans les branches mettaient en mouvement l’ineffable silence nocturne. Déruchette, jolie et sacrée, apparaissait dans ce crépuscule comme la résultante de ces rayons et de ces parfums ; ce charme immense et épars aboutissait mystérieusement à elle, et s’y condensait, et elle en était l’épanouissement. Elle semblait l’âme fleur de toute cette ombre.\par
Toute cette ombre flottante en Déruchette pesait  sur Gilliatt. Il était éperdu. Ce qu’il éprouvait échappe aux paroles ; l’émotion est toujours neuve et le mot a toujours servi, de là l’impossibilité d’exprimer l’émotion. L’accablement du ravissement existe. Voir Déruchette, la voir elle-même, voir sa robe, voir son bonnet, voir son ruban qu’elle tourne autour de son doigt, est-ce qu’on peut se figurer une telle chose ? Être près d’elle, est-ce que c’est possible ? L’entendre respirer, elle respire donc ! alors les astres respirent. Gilliatt frissonnait. Il était le plus misérable et le plus enivré des hommes. Il ne savait que faire. Ce délire de la voir l’anéantissait. Quoi ! c’était elle qui était là, et c’était lui qui était ici ! Ses idées, éblouies et fixes, s’arrêtaient sur cette créature comme sur une escarboucle. Il regardait cette nuque et ces cheveux. Il ne disait même pas que tout cela maintenant était à lui, qu’avant peu, demain peut-être, ce bonnet il aurait le droit de le défaire, ce ruban il aurait le droit de le dénouer. Songer jusque-là, il n’eût pas même conçu un moment cet excès d’audace. Toucher avec la pensée, c’est presque toucher avec la main. L’amour était pour Gilliatt comme le miel pour l’ours, le rêve exquis et délicat. Il pensait confusément. Il ne savait ce qu’il avait. Le rossignol chantait. Il se sentait expirer.\par
Se lever, franchir le mur, s’approcher, dire : c’est moi, parler à Déruchette, cette idée ne lui venait pas. Si elle lui fût venue, il se fût enfui. Si quelque chose de semblable à une pensée parvenait à poindre dans son esprit, c’était ceci, que Déruchette était là, qu’il n’y avait besoin de rien de plus, et que l’éternité commençait.\par
 Un bruit les tira tous les deux, elle de sa rêverie, lui de son extase.\par
Quelqu’un marchait dans le jardin. On ne voyait pas qui, à cause des arbres. C’était un pas d’homme.\par
Déruchette leva les yeux.\par
Les pas s’approchèrent, puis cessèrent. La personne qui marchait venait de s’arrêter. Elle devait être tout près. Le sentier où était le banc se perdait entre deux massifs. C’est là qu’était cette personne, dans cet entre-deux, à quelques pas du banc.\par
Le hasard avait disposé les épaisseurs des branches de telle sorte que Déruchette la voyait, mais que Gilliatt ne la voyait pas.\par
La lune projetait sur la terre, hors du massif jusqu’au banc, une ombre.\par
Gilliatt voyait cette ombre. Il regarda Déruchette.\par
Elle était toute pâle. Sa bouche entr’ouverte ébauchait un cri de surprise. Elle s’était soulevée à demi sur le banc et elle y était retombée ; il y avait dans son attitude un mélange de fuite et de fascination. Son étonnement était un enchantement plein de crainte. Elle avait sur les lèvres presque le rayonnement du sourire et une lueur de larmes dans les yeux. Elle était comme transfigurée par une présence. Il ne semblait pas que l’être qu’elle voyait fût de la terre. La réverbération d’un ange était dans son regard.\par
L’être qui n’était pour Gilliatt qu’une ombre parla. Une voix sortit du massif, plus douce qu’une voix de femme, une voix d’homme pourtant. Gilliatt entendit ces paroles :\par
 — Mademoiselle, je vous vois tous les dimanches et tous les jeudis ; on m’a dit qu’autrefois vous ne veniez pas si souvent. C’est une remarque qu’on a faite, je vous demande pardon. Je ne vous ai jamais parlé, c’était mon devoir ; aujourd’hui je vous parle, c’est mon devoir. Je dois d’abord m’adresser à vous. Le \emph{Cashmere} part demain. C’est ce qui fait que je suis venu. Vous vous promenez tous les soirs dans votre jardin. Ce serait mal à moi de connaître vos habitudes si je n’avais pas la pensée que j’ai. Mademoiselle, vous êtes pauvre ; depuis ce matin je suis riche. Voulez-vous de moi pour votre mari ?\par
Déruchette joignit ses deux mains comme une suppliante, et regarda celui qui lui parlait, muette, l’œil fixe, tremblante de la tête aux pieds.\par
La voix reprit :\par
— Je vous aime. Dieu n’a pas fait le cœur de l’homme pour qu’il se taise. Puisque Dieu promet l’éternité, c’est qu’il veut qu’on soit deux. Il y a pour moi sur la terre une femme, c’est vous. Je pense à vous comme à une prière. Ma foi est en Dieu et mon espérance est en vous. Les ailes que j’ai, c’est vous qui les portez. Vous êtes ma vie, et déjà mon ciel.\par
— Monsieur, dit Déruchette, il n’y a personne pour répondre dans la maison.\par
La voix s’éleva de nouveau :\par
— J’ai fait ce doux songe. Dieu ne défend pas les songes. Vous me faites l’effet d’une gloire. Je vous aime passionnément, mademoiselle. La sainte innocence, c’est vous. Je sais que c’est l’heure où l’on est  couché, mais je n’avais pas le choix d’un autre moment. Vous rappelez-vous ce passage de la bible qu’on nous a lu ? Genèse, chapitre vingt-cinq. J’y ai toujours songé depuis. Je l’ai relu souvent. Le révérend Hérode me disait : Il vous faut une femme riche. Je lui ai répondu : Non, il me faut une femme pauvre. Mademoiselle, je vous parle sans approcher, je me reculerai même si vous ne voulez pas que mon ombre touche vos pieds. C’est vous qui êtes la souveraine ; vous viendrez à moi si vous voulez. J’aime et j’attends. Vous êtes la forme vivante de la bénédiction.\par
— Monsieur, balbutia Déruchette, je ne savais pas qu’on me remarquait le dimanche et le jeudi.\par
La voix continua :\par
— On ne peut rien contre les choses angéliques. Toute la loi est amour. Le mariage, c’est Chanaan. Vous êtes la beauté promise. O pleine de grâce, je vous salue.\par
Déruchette répondit :\par
— Je ne croyais pas faire plus de mal que les autres personnes qui étaient exactes.\par
La voix poursuivit :\par
— Dieu a mis ses intentions dans les fleurs, dans l’aurore, dans le printemps, et il veut qu’on aime. Vous êtes belle dans cette obscurité sacrée de la nuit. Ce jardin a été cultivé par vous et dans ses parfums il y a quelque chose de votre haleine. Mademoiselle, les rencontres des âmes ne dépendent pas d’elles. Ce n’est pas de notre faute. Vous assistiez, rien de plus ; j’étais là, rien de plus. Je n’ai rien fait que de sentir que je  vous aimais. Quelquefois mes yeux se sont levés sur vous. J’ai eu tort, mais comment faire ? c’est en vous regardant que tout est venu. On ne peut s’empêcher. Il y a des volontés mystérieuses qui sont au-dessus de nous. Le premier des temples, c’est le cœur. Avoir votre âme dans ma maison, c’est à ce paradis terrestre que j’aspire, y consentez-vous ? Tant que j’ai été pauvre, je n’ai rien dit. Je sais votre âge. Vous avez vingt et un ans, j’en ai vingt-six. Je pars demain ; si vous me refusez, je ne reviendrai pas. Soyez mon engagée, voulez-vous ? Mes yeux ont déjà, plus d’une fois, malgré moi, fait aux vôtres cette question. Je vous aime, répondez-moi. Je parlerai à votre oncle dès qu’il pourra me recevoir, mais je me tourne d’abord vers vous. C’est à Rebecca qu’on demande Rebecca. A moins que vous ne m’aimiez pas.\par
Déruchette pencha le front, et murmura :\par
— Oh ! je l’adore !\par
Cela fut dit si bas que Gilliatt seul entendit.\par
Elle resta le front baissé comme si le visage dans l’ombre mettait dans l’ombre la pensée.\par
Il y eut une pause. Les feuilles d’arbres ne remuaient pas. C’était ce moment sévère et paisible où le sommeil des choses s’ajoute au sommeil des êtres, et où la nuit semble écouter le battement de cœur de la nature. Dans ce recueillement s’élevait, comme une harmonie qui complète un silence, le bruit immense de la mer.\par
La voix reprit :\par
— Mademoiselle.\par
 Déruchette tressaillit.\par
La voix continua :\par
— Hélas ! j’attends.\par
— Qu’attendez-vous ?\par
— Votre réponse.\par
— Dieu l’a entendue, dit Déruchette.\par
Alors la voix devint presque sonore, et en même temps plus douce que jamais. Ces paroles sortirent du massif, comme d’un buisson ardent :\par
— Tu es ma fiancée. Lève-toi et viens. Que le bleu profond où sont les astres assiste à cette acceptation de mon âme par ton âme, et que notre premier baiser se mêle au firmament !\par
Déruchette se leva, et demeura un instant immobile, le regard fixé devant elle, sans doute sur un autre regard. Puis, à pas lents, la tête droite, les bras pendants et les doigts des mains écartés comme lorsqu’on marche sur un support inconnu, elle se dirigea vers le massif et y disparut.\par
Un moment après, au lieu d’une ombre sur le sable il y en avait deux, elles se confondaient, et Gilliatt voyait à ses pieds l’embrassement de ces deux ombres.\par
Le temps coule de nous comme d’un sablier, et nous n’avons pas le sentiment de cette fuite, surtout dans de certains instants suprêmes. Ce couple d’un côté, qui ignorait ce témoin et ne le voyait pas, de l’autre ce témoin qui ne voyait pas ce couple, mais qui le savait là, combien de minutes demeurèrent-ils ainsi, dans cette suspension mystérieuse ? Il serait  impossible de le dire. Tout à coup, un bruit lointain éclata, une voix cria : Au secours ! et la cloche du port sonna. Ce tumulte, il est probable que le bonheur, ivre et céleste, ne l’entendit pas.\par
La cloche continua de sonner. Quelqu’un qui eût cherché Gilliatt dans l’angle du mur, ne l’y eût plus trouvé.\par
  \subsection[{C.II. Livre deuxième. La reconnaissance en plein despotisme}]{C.II. Livre deuxième \\
La reconnaissance en plein despotisme}
  \subsubsection[{C.II.1. Joie entremêlée d’angoisse}]{C.II.1. \\
Joie entremêlée d’angoisse}
\noindent Mess Lethierry agitait la cloche avec emportement. Brusquement il s’arrêta. Un homme venait de tourner l’angle du quai. C’était Gilliatt.\par
Mess Lethierry courut à lui, ou pour mieux dire se jeta sur lui, lui prit la main dans ses poings, et le regarda un moment dans les deux yeux en silence ; un de ces silences qui sont de l’explosion ne sachant par où sortir.\par
Puis avec violence, le secouant et le tirant, et le serrant dans ses bras, il fit entrer Gilliatt dans la salle basse des Bravées, en repoussa du talon la porte qui demeura entr’ouverte, s’assit, ou tomba, sur une chaise à côté d’une grande table éclairée par la lune dont le reflet blanchissait vaguement le visage de Gilliatt, et, d’une voix où il y avait des éclats de rire et des sanglots mêlés, il cria :\par
 — Ah ! mon fils ! l’homme au bug pipe ! Gilliatt ! je savais bien que c’était toi ! La panse, parbleu ! Conte-moi ça. Tu y es donc allé ! On t’aurait brûlé il y a cent ans. C’est de la magie. Il ne manque pas une vis. J’ai déjà tout regardé, tout reconnu, tout manié. Je devine que les roues sont dans les deux caisses. Te voilà donc enfin ! Je viens de te chercher dans ta cabine. J’ai sonné la cloche. Je te cherchais. Je me disais : Où est-il que je le mange ! Il faut convenir qu’il se passe des choses extraordinaires. Cet animal-là revient de l’écueil Douvres. Il me rapporte ma vie. Tonnerre ! tu es un ange. Oui, oui, oui, c’est ma machine. Personne n’y croira. On le verra, on dira : Ce n’est pas vrai. Tout y est, quoi ! Tout y est ! Il ne manque pas un serpentin. Il ne manque pas un apitage. Le tube de prise d’eau n’a pas bougé. C’est incroyable qu’il n’y ait pas eu d’avarie. Il n’y a qu’un peu d’huile à mettre. Mais comment as-tu fait ? Et dire que Durande va remarcher ! L’arbre des roues est démonté comme par un bijoutier. Donne-moi ta parole d’honneur que je ne suis pas fou.\par
Il se dressa debout, respira, et poursuivit :\par
— Jure-moi ça. Quelle révolution ! Je me pince, je sens bien que je ne rêve pas. Tu es mon enfant, tu es mon garçon, tu es le bon Dieu. Ah ! mon fils. Avoir été me chercher ma gueuse de machine ! En pleine mer ! dans ce guet-apens d’écueil ! J’ai vu des choses très farces dans ma vie. Je n’ai rien vu de tel. J’ai vu les parisiens qui sont des satans. Je t’en fiche qu’ils feraient ça. C’est pis que la Bastille. J’ai vu les gauchos  labourer dans les pampas, ils ont pour charrue une branche d’arbre qui a un coude et pour herse un fagot d’épines tiré avec une corde de cuir, ils récoltent avec ça des grains de blé gros comme des noisettes. C’est de la gnognotte à côté de toi. Tu as fait là un miracle, un pour de vrai. Ah ! le gredin ! Saute-moi donc au cou. Et on te devra tout le bonheur du pays. Vont-ils bougonner dans Saint-Sampson ! Je vais m’occuper tout de suite de refaire le bachot. C’est étonnant, la bielle n’a rien de cassé. Messieurs, il est allé aux Douvres. Je dis les Douvres. Il est allé tout seul. Les Douvres ! un caillou qu’il n’y a rien de pire. Tu sais, t’a-t-on dit ? c’est prouvé, ça a été fait exprès, Clubin a coulé Durande pour me filouter de l’argent qu’il avait à m’apporter. Il a soûlé Tangrouille. C’est long, je te raconterai un autre jour la piraterie. Moi, affreuse brute, j’avais confiance dans Clubin. Il s’y est pincé, le scélérat, car il n’a pas dû en sortir. Il y a un Dieu, canaille ! Vois-tu, Gilliatt, tout de suite, dare, dare, les fers au feu, nous allons rebâtir Durande. Nous lui donnerons vingt pieds de plus. On fait maintenant les bateaux plus longs. J’achèterai du bois à Dantzick et à Brême. A présent que j’ai la machine, on me fera crédit. La confiance reviendra.\par
Mess Lethierry s’arrêta, leva les yeux avec ce regard qui voit le ciel à travers le plafond, et dit entre ses dents : Il y en a un.\par
Puis il posa le médium de sa main droite entre ses deux sourcils, l’ongle appuyé sur la naissance du nez,  ce qui indique le passage d’un projet dans le cerveau, et il reprit :\par
— C’est égal, pour tout recommencer sur une grande échelle, un peu d’argent comptant eût bien fait mon affaire. Ah ! si j’avais mes trois bank-notes, les soixante-quinze mille francs que ce brigand de Rantaine m’a rendus et que ce brigand de Clubin m’a volés !\par
Gilliatt, en silence, chercha dans sa poche quelque chose qu’il posa devant lui. C’était la ceinture de cuir qu’il avait rapportée. Il ouvrit et étala sur la table cette ceinture dans l’intérieur de laquelle la lune laissait déchiffrer le mot \emph{Clubin ;} il tira du gousset de la ceinture une boîte, et de la boîte trois morceaux de papier pliés qu’il déplia et qu’il tendit à mess Lethierry.\par
Mess Lethierry examina les trois morceaux de papier. Il faisait assez clair pour que le chiffre 1000 et le mot \emph{thousand} y fussent parfaitement visibles. Mess Lethierry prit les trois billets, les posa sur la table l’un à côté de l’autre, les regarda, regarda Gilliatt, resta un moment interdit, puis ce fut comme une éruption après une explosion.\par
— Ça aussi ! Tu es prodigieux. Mes bank-notes ! tous les trois ! mille chaque ! mes soixante-quinze mille francs ! Tu es donc allé jusqu’en enfer. C’est la ceinture à Clubin. Pardieu ! je lis dedans son ordure de nom. Gilliatt rapporte la machine, plus l’argent ! Voilà de quoi mettre dans les journaux. J’achèterai du bois première qualité. Je devine, tu auras retrouvé la carcasse. Clubin pourri dans quelque coin. Nous prendrons le sapin à Dantzick et le chêne à Brême, nous ferons un  bon bordé, nous mettrons le chêne en dedans et le sapin en dehors. Autrefois on fabriquait les navires moins bien et ils duraient davantage ; c’est que le bois était plus assaisonné, parce qu’on ne construisait pas tant. Nous ferons peut-être la coque en orme. L’orme est bon pour les parties noyées ; être tantôt sec, tantôt trempé, ça le pourrit ; l’orme veut être toujours mouillé, il se nourrit d’eau. Quelle belle Durande nous allons conditionner ! On ne me fera pas la loi. Je n’aurai plus besoin de crédit. J’ai les sous. A-t-on jamais vu ce Gilliatt ! J’étais par terre, aplati, mort. Il me remet debout sur mes quatre fers ! Et moi qui ne pensais pas du tout à lui ! Ça m’était sorti de l’esprit. Tout me revient, à présent. Pauvre garçon ! Ah ! par exemple, tu sais, tu épouses Déruchette.\par
Gilliatt s’adossa au mur, comme quelqu’un qui chancelle, et très bas, mais très distinctement, il dit :\par
— Non.\par
Mess Lethierry eut un soubresaut.\par
— Comment, non !\par
Gilliatt répondit :\par
— Je ne l’aime pas.\par
Mess Lethierry alla à la fenêtre, l’ouvrit et la referma, revint à la table, prit les trois bank-notes, les plia, posa la boîte de fer dessus, se gratta les cheveux, saisit la ceinture de Clubin, la jeta violemment contre la muraille, et dit :\par
— Il y a quelque chose.\par
Il enfonça ses deux poings dans ses deux poches, et reprit :\par
 — Tu n’aimes pas Déruchette ! C’est donc pour moi que tu jouais du bug pipe ?\par
Gilliatt, toujours adossé au mur, pâlissait comme un homme qui tout à l’heure ne respirera plus. A mesure qu’il devenait pâle, mess Lethierry devenait rouge.\par
— En voilà un imbécile ! Il n’aime pas Déruchette ! Eh bien, arrange-toi pour l’aimer, car elle n’épousera que toi. Quelle diable d’anecdote viens-tu me conter là ! Si tu crois que je te crois ! Est-ce que tu es malade ? c’est bon, envoie chercher le médecin, mais ne dis pas d’extravagances. Pas possible que tu aies déjà eu le temps de vous quereller et de te fâcher avec elle. Il est vrai que les amoureux, c’est si bête ! Voyons, as-tu des raisons ? Si tu as des raisons, dis-les. On n’est pas une oie sans avoir des raisons. Après ça, j’ai du coton dans les oreilles, j’ai peut-être mal entendu, répète ce que tu as dit.\par
Gilliatt répliqua :\par
— J’ai dit non.\par
— Tu as dit non. Il y tient, la brute ! Tu as quelque chose, c’est sûr ! Tu as dit non ! Voilà une stupidité qui dépasse les limites du monde connu. On flanque des douches aux personnes pour bien moins que ça. Ah ! tu n’aimes pas Déruchette ! Alors c’est pour l’amour du bonhomme que tu as fait tout ce que tu as fait ! C’est pour les beaux yeux du papa que tu es allé aux Douvres, que tu as eu froid, que tu as eu chaud, que tu as crevé de faim et de soif, que tu as mangé de la vermine de rocher, que tu as eu le brouillard, la pluie et le vent pour chambre à coucher, et que tu as exécuté la chose  de me rapporter ma machine, comme on rapporte à une jolie femme son serin qui s’est échappé ! Et la tempête d’il y a trois jours ! Si tu t’imagines que je ne me rends pas compte. Tu en as eu du tirage ! C’est en faisant la bouche en cœur du côté de ma vieille caboche que tu as taillé, coupé, tourné, viré, traîné, limé, scié, charpenté, inventé, écrabouillé, et fait plus de miracles à toi tout seul que tous les saints du paradis. Ah ! idiot ! tu m’as pourtant assez ennuyé avec ton bug pipe. On appelle ça biniou en Bretagne. Toujours le même air, l’animal ! Ah ! tu n’aimes pas Déruchette ! Je ne sais pas ce que tu as. Je me rappelle bien tout à présent, j’étais là dans le coin, Déruchette a dit : Je l’épouserai. Et elle t’épousera ! Ah ! tu ne l’aimes pas ! Réflexions faites, je ne comprends rien. Ou tu es fou, ou je le suis. Et le voilà qui ne dit plus un mot. Ça n’est pas permis de faire tout ce que tu as fait, et de dire à la fin : Je n’aime pas Déruchette. On ne rend pas service aux gens pour les mettre en colère. Eh bien, si tu ne l’épouses pas, elle coiffera sainte Catherine. D’abord, j’ai besoin de toi, moi. Tu seras le pilote de Durande. Si tu t’imagines que je vais te laisser aller comme ça ! Ta, ta, ta, nenni mon cœur, je ne te lâche point. Je te tiens. Je ne t’écoute seulement pas. Où y a-t-il un matelot comme toi ! Tu es mon homme. Mais parle donc !\par
Cependant la cloche avait réveillé la maison et les environs. Douce et Grâce s’étaient levées et venaient d’entrer dans la salle basse, l’air stupéfait, sans dire mot. Grâce avait à la main une chandelle. Un groupe  de voisins, bourgeois, marins et paysans, sortis en hâte, était dehors sur le quai, considérant avec pétrification et stupeur la cheminée de la Durande dans la panse. Quelques-uns, entendant la voix de mess Lethierry dans la salle basse, commençaient à s’y glisser silencieusement par la porte entre-bâillée. Entre deux faces de commères, passait la tête de sieur Landoys qui avait ce hasard d’être toujours là où il aurait regretté de ne pas être.\par
Les grandes joies ne demandent pas mieux que d’avoir un public. Le point d’appui un peu épars qu’offre toujours une foule, leur plaît ; elles repartent de là. Mess Lethierry s’aperçut tout à coup qu’il y avait des gens autour de lui. Il accepta d’emblée l’auditoire.\par
— Ah ! vous voilà, vous autres. C’est bien heureux. Vous savez la nouvelle. Cet homme a été là et il a rapporté ça. Bonjour, sieur Landoys. Tout à l’heure quand je me suis réveillé, j’ai vu le tuyau. C’était sous ma fenêtre. Il ne manque pas un clou à la chose. On fait des gravures de Napoléon ; moi, j’aime mieux ça que la bataille d’Austerlitz. Vous sortez de votre lit, bonnes gens. La Durande vous vient en dormant. Pendant que vous mettez vos bonnets de coton et que vous soufflez vos chandelles, il y a des gens qui sont des héros. On est un tas de lâches et de fainéants, on chauffe ses rhumatismes, heureusement cela n’empêche pas qu’il y ait des enragés. Ces enragés vont où il faut aller et font ce qu’il faut faire. L’homme du Bû de la Rue arrive du rocher Douvres. Il a repêché la Durande au fond de  la mer, il a repêché l’argent dans la poche de Clubin, un trou encore plus profond. Mais comment as-tu fait ? Tout le diantre était contre toi, le vent et la marée, la marée et le vent. C’est vrai que tu es sorcier. Ceux qui disent ça ne sont déjà pas si bêtes. La Durande est revenue ! Les tempêtes ont beau avoir de la méchanceté, ça la leur coupe rasibus. Mes amis, je vous annonce qu’il n’y a plus de naufrages. J’ai visité la mécanique. Elle est comme neuve, entière, quoi ! Les tiroirs à vapeur jouent comme sur des roulettes. On dirait un objet d’hier matin. Vous savez que l’eau qui sort est conduite hors du bateau par un tube placé dans un autre tube par où passe l’eau qui entre, pour utiliser la chaleur ; eh bien, les deux tubes, ça y est. Toute la machine ! les roues aussi ! Ah ! tu l’épouseras !\par
— Qui ? la machine ? demanda sieur Landoys.\par
— Non, la fille. Oui, la machine. Les deux. Il sera deux fois mon gendre. Il sera le capitaine. Good bye, capitaine Gilliatt. Il va y en avoir une, de Durande ! On va en faire des affaires, et de la circulation, et du commerce, et des chargements de bœufs et de moutons ! Je ne donnerais pas Saint-Sampson pour Londres. Et voici l’auteur. Je vous dis que c’est une aventure. On lira ça samedi dans la gazette au père Mauger. Gilliatt le Malin est un malin. Qu’est-ce que c’est que ces louis d’or là ?\par
Mess Lethierry venait de remarquer, par l’hiatus du couvercle, qu’il y avait de l’or dans la boîte posée sur les bank-notes. Il la prit, l’ouvrit, la vida dans la paume de sa main, et mit la poignée de guinées sur la table.\par
 — Pour les pauvres. Sieur Landoys, donnez ces pounds de ma part au connétable de Saint-Sampson. Vous savez, la lettre de Rantaine ? Je vous l’ai montrée ; eh bien, j’ai les bank-notes. Voilà de quoi acheter du chêne et du sapin et faire de la menuiserie. Regardez plutôt. Vous rappelez-vous le temps d’il y a trois jours ? Quel massacre de vent et de pluie ! Le ciel tirait le canon. Gilliatt a reçu ça dans les Douvres. Ça ne l’a pas empêché de décrocher l’épave comme je décroche ma montre. Grâce à lui, je redeviens quelqu’un. La galiote au père Lethierry va reprendre son service, messieurs, mesdames. Une coquille de noix avec deux roues et un tuyau de pipe, j’ai toujours été toqué de cette invention-là. Je me suis toujours dit : j’en ferai une ! Ça date de loin ; c’est une idée qui m’est venue à Paris dans le café qui fait le coin de la rue Christine et de la rue Dauphine en lisant un journal qui en parlait. Savez-vous bien que Gilliatt ne serait pas gêné pour mettre la machine de Marly dans son gousset et pour se promener avec ? C’est du fer battu, cet homme-là, de l’acier trempé, du diamant, un marin bon jeu bon argent, un forgeron, un gaillard extraordinaire, plus étonnant que le prince de Hohenlohe. J’appelle ça un homme qui a de l’esprit. Nous sommes tous des pas grand’chose. Les loups de mer, c’est vous, c’est moi, c’est nous ; mais le lion de mer, le voici. Hurrah, Gilliatt ! Je ne sais pas ce qu’il a fait, mais certainement il a été un diable, et comment veut-on que je ne lui donne pas Déruchette !\par
Depuis quelques instants Déruchette était dans la  salle. Elle n’avait pas dit un mot, elle n’avait pas fait de bruit. Elle avait eu une entrée d’ombre. Elle s’était assise, presque inaperçue, sur une chaise en arrière de mess Lethierry debout, loquace, orageux, joyeux, abondant en gestes et parlant haut. Un peu après elle, une autre apparition muette s’était faite. Un homme vêtu de noir, en cravate blanche, ayant son chapeau à la main, s’était arrêté dans l’entre-bâillement de la porte. Il y avait maintenant plusieurs chandelles dans le groupe lentement grossi. Ces lumières éclairaient de côté l’homme vêtu de noir ; son profil d’une blancheur jeune et charmante se dessinait sur le fond obscur avec une pureté de médaille ; il appuyait son coude à l’angle d’un panneau de la porte, et il tenait son front dans sa main gauche, attitude, à son insu, gracieuse, qui faisait valoir la grandeur du front par la petitesse de la main. Il y avait un pli d’angoisse au coin de ses lèvres contractées. Il examinait et écoutait avec une attention profonde. Les assistants, ayant reconnu le révérend Ebenezer Caudray, recteur de la paroisse, s’étaient écartés pour le laisser passer, mais il était resté sur le seuil. Il y avait de l’hésitation dans sa posture et de la décision dans son regard. Ce regard par moments se rencontrait avec celui de Déruchette. Quant à Gilliatt, soit par hasard, soit exprès, il était dans l’ombre, et on ne le voyait que très confusément.\par
Mess Lethierry d’abord n’aperçut pas M. Ebenezer, mais il aperçut Déruchette. Il alla à elle, et l’embrassa avec tout l’emportement que peut avoir un baiser au  front. En même temps il étendait le bras vers le coin sombre où était Gilliatt.\par
— Déruchette, dit-il, te revoilà riche, et voilà ton mari.\par
Déruchette leva la tête avec égarement et regarda dans cette obscurité.\par
Mess Lethierry reprit :\par
— On fera la noce tout de suite, demain si ça se peut, on aura les dispenses, d’ailleurs ici les formalités ne sont pas lourdes, le doyen fait ce qu’il veut, on est marié avant qu’on ait eu le temps de crier gare, ce n’est pas comme en France, où il faut des bans, des publications, des délais, tout le bataclan, et tu pourras te vanter d’être la femme d’un brave homme, et il n’y a pas à dire, c’est que c’est un marin, je l’ai pensé dès le premier jour quand je l’ai vu revenir de Herm avec le petit canon. A présent il revient des Douvres, avec sa fortune, et la mienne, et la fortune du pays ; c’est un homme dont on parlera un jour comme il n’est pas possible ; tu as dit : je l’épouserai, tu l’épouseras ; et vous aurez des enfants, et je serai grand-père, et tu auras cette chance d’être la lady d’un gaillard sérieux, qui travaille, qui est utile, qui est surprenant, qui en vaut cent, qui sauve les inventions des autres, qui est une providence, et au moins, toi, tu n’auras pas, comme presque toutes les chipies riches de ce pays-ci, épousé un soldat ou un prêtre, c’est-à-dire l’homme qui tue ou l’homme qui ment. Mais qu’est-ce que tu fais dans ton coin, Gilliatt ? On ne te voit pas. Douce ! Grâce ! tout le monde, de la lumière. Illuminez-moi  mon gendre à giorno. Je vous fiance, mes enfants, et voilà ton mari, et voilà mon gendre, c’est Gilliatt du Bû de la Rue, le bon garçon, le grand matelot, et je n’aurai pas d’autre gendre, et tu n’auras pas d’autre mari, j’en redonne ma parole d’honneur au bon Dieu. Ah ! c’est vous, monsieur le curé, vous me marierez ces jeunes gens-là.\par
L’œil de mess Lethierry venait de tomber sur le révérend Ebenezer.\par
Douce et Grâce avaient obéi. Deux chandelles posées sur la table éclairaient Gilliatt de la tête aux pieds.\par
— Qu’il est beau ! cria Lethierry.\par
Gilliatt était hideux.\par
Il était tel qu’il était sorti, le matin même, de l’écueil Douvres, en haillons, les coudes percés, la barbe longue, les cheveux hérissés, les yeux brûlés et rouges, la face écorchée, les poings saignants ; il avait les pieds nus. Quelques-unes des pustules de la pieuvre étaient encore visibles sur ses bras velus.\par
Lethierry le contemplait.\par
— C’est mon vrai gendre. Comme il s’est battu avec la mer ! il est tout en loques ! Quelles épaules ! quelles pattes ! Que tu es beau !\par
Grâce courut à Déruchette et lui soutint la tête. Déruchette venait de s’évanouir.
 \subsubsection[{C.II.2. La malle de cuir}]{C.II.2. \\
La malle de cuir}
\noindent Dès l’aube Saint-Sampson était sur pied et Saint-Pierre-Port commençait à arriver. La résurrection de la Durande faisait dans l’île un bruit comparable à celui qu’a fait dans le midi de la France la Salette. Il y avait foule sur le quai pour regarder la cheminée sortant de la panse. On eût bien voulu voir et toucher un peu la machine ; mais Lethierry, après avoir fait de nouveau, et au jour, l’inspection triomphante de la mécanique, avait posté dans la panse deux matelots chargés d’empêcher l’approche. La cheminée, au surplus, suffisait à la contemplation. La foule s’émerveillait. On ne parlait que de Gilliatt. On commentait et on acceptait son surnom de Malin ; l’admiration s’achevait volontiers par cette phrase : « Ce n’est toujours pas agréable d’avoir dans l’île des gens capables de faire des choses comme ça. »\par
On voyait du dehors mess Lethierry assis à sa table devant sa fenêtre et écrivant, un œil sur son  papier, l’autre sur la machine. Il était tellement absorbé qu’il ne s’était interrompu qu’une fois pour « crier\footnote{ \noindent Appeler.
 } » Douce et pour lui demander des nouvelles de Déruchette. Douce avait répondu : — Mademoiselle s’est levée, et est sortie. — Mess Lethierry avait dit :\par
— Elle fait bien de prendre l’air. Elle s’est trouvée un peu mal cette nuit à cause de la chaleur. Il y avait beaucoup de monde dans la salle. Et puis la surprise, la joie, avec cela que les fenêtres étaient fermées. Elle va avoir un fier mari ! — Et il avait recommencé à écrire. Il avait déjà paraphé et scellé deux lettres adressées aux plus notables maîtres de chantiers de Brême. Il achevait de cacheter la troisième.\par
Le bruit d’une roue sur le quai lui fit dresser le cou. Il se pencha à sa fenêtre et vit déboucher du sentier par où l’on allait au Bû de la Rue un boy poussant une brouette. Ce boy se dirigeait du côté de Saint-Pierre-Port. Il y avait dans la brouette une malle de cuir jaune damasquinée de clous de cuivre et d’étain.\par
Mess Lethierry apostropha le boy.\par
— Où vas-tu, garçon ?\par
Le boy s’arrêta, et répondit :\par
— Au \emph{Cashmere}.\par
— Quoi faire ?\par
— Porter cette malle.\par
— Eh bien, tu porteras aussi ces trois lettres.\par
Mess Lethierry ouvrit le tiroir de sa table, y prit  un bout de ficelle, noua ensemble sous un nœud en croix les trois lettres qu’il venait d’écrire, et jeta le paquet au boy qui le reçut au vol dans ses deux mains.\par
— Tu diras au capitaine du \emph{Cashmere} que c’est moi qui écris, et qu’il ait soin. C’est pour l’Allemagne. Brême via London.\par
— Je ne parlerai pas au capitaine, mess Lethierry.\par
— Pourquoi ?\par
— Le \emph{Cashmere} n’est pas à quai.\par
— Ah !\par
— Il est en rade.\par
— C’est juste. A cause de la mer.\par
— Je ne pourrai parler qu’au patron de l’embarcation.\par
— Tu lui recommanderas mes lettres.\par
— Oui, mess Lethierry.\par
— A quelle heure part le \emph{Cashmere ?}\par
— A douze heures.\par
— A midi, aujourd’hui, la marée monte. Il a la marée contre.\par
— Mais il a le vent pour.\par
— Boy, dit mess Lethierry, braquant son index sur la cheminée de la machine, vois-tu ça ? ça se moque du vent et de la marée.\par
Le boy mit les lettres dans sa poche, ressaisit le brancard de la brouette, et reprit sa course vers la ville. Mess Lethierry appela : Douce ! Grâce !\par
Grâce entre-bâilla la porte.\par
— Mess, qu’y a-t-il ?\par
 — Entre, et attends.\par
Mess Lethierry prit une feuille de papier et se mit à écrire. Si Grâce, debout derrière lui, eût été curieuse et eût avancé la tête pendant qu’il écrivait, elle aurait pu lire, par-dessus son épaule, ceci :\par
\bigbreak
\noindent « J’écris à Brême pour du bois. J’ai rendez-vous toute la journée avec des charpentiers pour l’estimat. La reconstruction marchera vite. Toi, de ton côté, va chez le doyen pour avoir les dispenses. Je désire que le mariage se fasse le plus tôt possible, tout de suite serait le mieux. Je m’occupe de Durande, occupe-toi de Déruchette. »\par
\bigbreak
\noindent Il data, et signa : L{\scshape ethierry}.\par
Il ne prit point la peine de cacheter le billet, le plia simplement en quatre et le tendit à Grâce.\par
— Porte cela à Gilliatt.\par
— Au Bû de la Rue ?\par
— Au Bû de la Rue.\par
  \subsection[{C.III. Livre troisième. Départ du Cashmere}]{C.III. Livre troisième \\
Départ du \emph{Cashmere}}
  \subsubsection[{C.III.1. Le havelet tout proche de l’église}]{C.III.1. \\
Le havelet tout proche de l’église}
\noindent Saint-Sampson ne peut avoir foule sans que Saint-Pierre-Port soit désert. Une chose curieuse sur un point donné est une pompe aspirante. Les nouvelles courent vite dans les petits pays ; aller voir la cheminée de la Durande sous les fenêtres de mess Lethierry était depuis le lever du soleil la grande affaire de Guernesey. Tout autre événement s’était effacé devant celui-là. Éclipse de la mort du doyen de Saint-Asaph ; il n’était plus question du révérend Ebenezer Caudray, ni de sa soudaine richesse, ni de son départ par le \emph{Cashmere.} La machine de la Durande rapportée des Douvres, tel était l’ordre du jour. On n’y croyait pas. Le naufrage avait paru extraordinaire, mais le sauvetage semblait impossible. C’était à qui s’en assurerait  par ses yeux. Toute autre préoccupation était suspendue. De longues files de bourgeois en famille, depuis le vésin jusqu’au mess, des hommes, des femmes, des gentlemen, des mères avec enfants et des enfants avec poupées, se dirigeaient par toutes les routes vers « la chose à voir » aux Bravées et tournaient le dos à Saint-Pierre-Port. Beaucoup de boutiques dans Saint-Pierre-Port étaient fermées ; dans Commercial-Arcade, stagnation absolue de vente et de négoce ; toute l’attention était à la Durande ; pas un marchand n’avait « étrenné » ; excepté un bijoutier, lequel s’émerveillait d’avoir vendu un anneau d’or pour mariage « à une espèce d’homme paraissant fort pressé qui lui avait demandé la demeure de monsieur le doyen ». Les boutiques restées ouvertes étaient des lieux de causerie où l’on commentait bruyamment le miraculeux sauvetage. Pas un promeneur à l’Hyvreuse, qu’on nomme aujourd’hui, sans savoir pourquoi, Cambridge-Park ; personne dans High-street, qui s’appelait alors la Grand’Rue, ni dans Smith-street, qui s’appelait alors la rue des Forges ; personne dans Hauteville ; l’Esplanade elle-même était dépeuplée. On eût dit un dimanche. Une altesse royale en visite passant la revue de la milice à l’Ancresse n’eût pas mieux vidé la ville. Tout ce dérangement à propos d’un rien du tout comme ce Gilliatt faisait hausser les épaules aux hommes graves et aux personnes correctes.\par
L’église de Saint-Pierre-Port, triple pignon juxtaposé avec transept et flèche, est au bord de l’eau au fond du port presque sur le débarcadère même. Elle  donne la bienvenue à ceux qui arrivent et l’adieu à ceux qui s’en vont. Cette église est la majuscule de la longue ligne que fait la façade de la ville sur l’océan.\par
Elle est en même temps paroisse de Saint-Pierre-Port et doyenné de toute l’île. Elle a pour desservant le subrogé de l’évêque, clergyman à pleins pouvoirs.\par
Le havre de Saint-Pierre-Port, très beau et très large port aujourd’hui, était à cette époque, et il y a dix ans encore, moins considérable que le havre de Saint-Sampson. C’étaient deux grosses murailles cyclopéennes courbes partant du rivage à tribord et à bâbord, et se rejoignant presque à leur extrémité, où il y avait un petit phare blanc. Sous ce phare un étroit goulet, ayant encore le double anneau de la chaîne qui le fermait au moyen âge, donnait passage aux navires. Qu’on se figure une pince de homard entr’ouverte, c’était le havre de Saint-Pierre-Port. Cette tenaille prenait sur l’abîme un peu de mer qu’elle forçait à se tenir tranquille. Mais par le vent d’est, il y avait du flot à l’entrebâillement, le port clapotait, et il était plus sage de ne point entrer. C’est ce qu’avait fait ce jour-là le \emph{Cashmere.} Il avait mouillé en rade.\par
Les navires, quand il y avait du vent d’est, prenaient volontiers ce parti qui, en outre, leur économisait les frais de port. Dans ce cas, les bateliers commissionnés de la ville, brave tribu de marins que le nouveau port a destituée, venaient prendre dans leurs barques, soit à l’embarcadère, soit aux stations de la plage, les voyageurs, et les transportaient, eux et leurs  bagages, souvent par de très grosses mers et toujours sans accident, aux navires en partance. Le vent d’est est un vent de côté, très bon pour la traversée d’Angleterre ; on roule, mais on ne tangue pas.\par
Quand le bâtiment en partance était dans le port, tout le monde s’embarquait dans le port ; quand il était en rade, on avait le choix de s’embarquer sur un des points de la côte voisine du mouillage. On trouvait dans toutes les criques des bateliers « à volonté ».\par
Le Havelet était une de ces criques. Ce petit havre, havelet, était tout près de la ville, mais si solitaire qu’il en semblait très loin. Il devait cette solitude à l’encaissement des hautes falaises du fort George qui dominent cette anse discrète. On arrivait au Havelet par plusieurs sentiers. Le plus direct longeait le bord de l’eau ; il avait l’avantage de mener à la ville et à l’église en cinq minutes, et l’inconvénient d’être couvert par la lame deux fois par jour. Les autres sentiers, plus ou moins abrupts, s’enfonçaient dans les anfractuosités des escarpements. Le Havelet, même en plein jour, était dans une pénombre. Des blocs en porte-à-faux pendaient de toutes parts. Un hérissement de ronces et de broussailles s’épaississait et faisait une sorte de douce nuit sur ce désordre de roches et de vagues ; rien de plus paisible que cette crique en temps calme, rien de plus tumultueux dans les grosses eaux. Il y avait là des pointes de branches perpétuellement mouillées par l’écume. Au printemps c’était plein de fleurs, de nids, de parfums, d’oiseaux, de papillons et d’abeilles. Grâce aux travaux récents, ces sauvageries  n’existent plus aujourd’hui ; de belles lignes droites les ont remplacées ; il y a des maçonneries, des quais et des jardinets, le terrassement a sévi ; le goût a fait justice des bizarreries de la montagne et de l’incorrection des rochers.
 \subsubsection[{C.III.2. Les desespoirs en présence}]{C.III.2. \\
Les desespoirs en présence}
\noindent Il était un peu moins de dix heures du matin ; \emph{le quart avant}, comme on dit à Guernesey.\par
L’affluence, selon toute apparence, grossissait à Saint-Sampson. La population, enfiévrée de curiosité, versant toute au nord de l’île, le Havelet, qui est au sud, était plus désert que jamais.\par
Pourtant on y voyait un bateau, et un batelier. Dans le bateau il y avait un sac de nuit. Le batelier semblait attendre.\par
On apercevait en rade le \emph{Cashmere} à l’ancre, qui, ne devant partir qu’à midi, ne faisait encore aucune manœuvre d’appareillage.\par
Un passant qui, de quelqu’un des sentiers-escaliers de la falaise, eût prêté l’oreille, eût entendu un murmure de paroles dans le Havelet, et, s’il se fût penché par-dessus les surplombs, il eût vu, à quelque distance du bateau, dans un recoin de roches et de branches où ne pouvait pénétrer le regard du batelier, deux  personnes, un homme et une femme, Ebenezer et Déruchette.\par
Ces réduits obscurs du bord de la mer, qui tentent les baigneuses, ne sont pas toujours aussi solitaires qu’on le croit. On y est quelquefois observé et écouté. Ceux qui s’y réfugient et qui s’y abritent peuvent être aisément suivis à travers les épaisseurs des végétations et grâce à la multiplicité et à l’enchevêtrement des sentiers. Les granits et les arbres, qui cachent l’aparté, peuvent cacher aussi un témoin.\par
Déruchette et Ebenezer étaient debout en face l’un de l’autre, le regard dans le regard ; ils se tenaient les mains. Déruchette parlait. Ebenezer se taisait. Une larme amassée et arrêtée entre ses cils hésitait, et ne tombait pas.\par
La désolation et la passion étaient empreintes sur le front religieux d’Ebenezer. Une résignation poignante s’y ajoutait, résignation hostile à la foi, quoique venant d’elle. Sur ce visage, simplement angélique jusqu’alors, il y avait un commencement d’expression fatale. Celui qui n’avait encore médité que le dogme, se mettait à méditer le sort, méditation malsaine au prêtre. La foi s’y décompose. Plier sous de l’inconnu, rien n’est plus troublant. L’homme est le patient des événements. La vie est une perpétuelle arrivée ; nous la subissons. Nous ne savons jamais de quel côté viendra la brusque descente du hasard. Les catastrophes et les félicités entrent, puis sortent, comme des personnages inattendus. Elles ont leur loi, leur orbite, leur gravitation, en dehors de l’homme. La vertu  n’amène pas le bonheur, le crime n’amène pas le malheur ; la conscience a une logique, le sort en a une autre ; nulle coïncidence. Rien ne peut être prévu. Nous vivons pêle-mêle et coup sur coup. La conscience est la ligne droite, la vie est le tourbillon. Ce tourbillon jette inopinément sur la tête de l’homme des chaos noirs et des ciels bleus. Le sort n’a point l’art des transitions. Quelquefois la roue tourne si vite que l’homme distingue à peine l’intervalle d’une péripétie à l’autre et le lien d’hier à aujourd’hui. Ebenezer était un croyant mélangé de raisonnement et un prêtre compliqué de passion. Les religions célibataires savent ce qu’elles font. Rien ne défait le prêtre comme d’aimer une femme. Toutes sortes de nuages assombrissaient Ebenezer.\par
Il contemplait Déruchette, trop.\par
Ces deux êtres s’idolâtraient.\par
Il y avait dans la prunelle d’Ebenezer la muette adoration du désespoir.\par
Déruchette disait :\par
— Vous ne partirez pas. Je n’en ai pas la force. Voyez-vous, j’ai cru que je pourrais vous dire adieu, je ne peux pas. On n’est pas forcé de pouvoir. Pourquoi êtes-vous venu hier ? Il ne fallait pas venir si vous vouliez vous en aller. Je ne vous ai jamais parlé. Je vous aimais, mais je ne le savais pas. Seulement, le premier jour, quand monsieur Hérode a lu l’histoire de Rebecca et que vos yeux ont rencontré les miens, je me suis senti les joues en feu, et j’ai pensé : Oh ! comme Rebecca a dû devenir rouge ! C’est égal, avant-hier, on  m’aurait dit : Vous aimez le recteur, j’aurais ri. C’est ce qu’il y a eu de terrible dans cet amour-là. Ç’a été comme une trahison. Je n’y ai pas pris garde. J’allais à l’église, je vous voyais, je croyais que tout le monde était comme moi. Je ne vous fais pas de reproche, vous n’avez rien fait pour que je vous aime, vous ne vous êtes pas donné de peine, vous me regardiez, ce n’est pas de votre faute si vous regardez les personnes, et cela a fait que je vous ai adoré. Je ne m’en doutais pas. Quand vous preniez le livre, c’était de la lumière ; quand les autres le prenaient, ce n’était qu’un livre. Vous leviez quelquefois les yeux sur moi. Vous parliez des archanges, c’était vous l’archange. Ce que vous disiez, je le pensais tout de suite. Avant vous, je ne sais pas si je croyais en Dieu. Depuis vous, j’étais devenue une femme qui fait sa prière. Je disais à Douce : Habille-moi bien vite que je ne manque pas l’office. Et je courais à l’église. Ainsi, être amoureuse d’un homme, c’est cela. Je ne le savais pas. Je me disais : Comme je deviens dévote ! C’est vous qui m’avez appris que je n’allais pas à l’église pour le bon Dieu. J’y allais pour vous, c’est vrai. Vous êtes beau, vous parlez bien, quand vous leviez les bras au ciel, il me semblait que vous teniez mon cœur dans vos deux mains blanches. J’étais folle, je l’ignorais. Voulez-vous que je vous dise votre faute, c’est d’être entré hier dans le jardin, c’est de m’avoir parlé. Si vous ne m’aviez rien dit, je n’aurais rien su. Vous seriez parti, j’aurais peut-être été triste, mais à présent je mourrai. A présent que je sais que je vous aime, il n’est plus possible que vous vous  en alliez. A quoi pensez-vous ? Vous n’avez pas l’air de m’écouter.\par
Ebenezer répondit :\par
— Vous avez entendu ce qui s’est dit hier.\par
— Hélas !\par
— Que puis-je à cela ?\par
Ils se turent un moment. Ebenezer reprit :\par
— Il n’y a plus pour moi qu’une chose à faire. Partir.\par
— Et moi, mourir. Oh ! je voudrais qu’il n’y eût pas de mer et qu’il n’y eût que le ciel. Il me semble que cela arrangerait tout, notre départ serait le même. Il ne fallait pas me parler, vous. Pourquoi m’avez-vous parlé ? Alors ne vous en allez pas. Qu’est-ce que je vais devenir ? Je vous dis que je mourrai. Vous serez bien avancé quand je serai dans le cimetière. Oh ! j’ai le cœur brisé. Je suis bien malheureuse. Mon oncle n’est pas méchant pourtant.\par
C’était la première fois de sa vie que Déruchette disait, en parlant de mess Lethierry, \emph{mon oncle.} Jusque-là elle avait toujours dit \emph{mon père.}\par
Ebenezer recula d’un pas et fit un signe au batelier. On entendit le bruit du croc dans les galets et le pas de l’homme sur le bord de sa barque.\par
— Non, non ! cria Déruchette.\par
Ebenezer se rapprocha d’elle.\par
— Il le faut, Déruchette.\par
— Non, jamais ! Pour une machine ! Est-ce que c’est possible ? Avez-vous vu cet homme horrible hier ? Vous ne pouvez pas m’abandonner. Vous avez de  l’esprit, vous trouverez un moyen. Il ne se peut pas que vous m’ayez dit de venir vous trouver ici ce matin, avec l’idée que vous partiriez. Je ne vous ai rien fait. Vous n’avez pas à vous plaindre de moi. C’est par ce vaisseau-là que vous voulez vous en aller ? Je ne veux pas. Vous ne me quitterez pas. On n’ouvre pas le ciel pour le refermer. Je vous dis que vous resterez. D’ailleurs il n’est pas encore l’heure. Oh ! je t’aime.\par
Et, se pressant contre lui, elle lui croisa ses dix doigts derrière le cou, comme pour faire de ses bras enlacés un lien à Ebenezer et de ses mains jointes une prière à Dieu.\par
Il dénoua cette étreinte délicate qui résista tant qu’elle put.\par
Déruchette tomba assise sur une saillie de roche couverte de lierre, relevant d’un geste machinal la manche de sa robe jusqu’au coude, montrant son charmant bras nu, avec une clarté noyée et blême dans ses yeux fixes. La barque approchait.\par
Ebenezer lui prit la tête dans ses deux mains ; cette vierge avait l’air d’une veuve et ce jeune homme avait l’air d’un aïeul. Il lui touchait les cheveux avec une sorte de précaution religieuse ; il attacha son regard sur elle pendant quelques instants, puis il déposa sur son front un de ces baisers sous lesquels il semble que devrait éclore une étoile, et, d’un accent où tremblait la suprême angoisse et où l’on sentait l’arrachement de l’âme, il lui dit ce mot, le mot des profondeurs : Adieu !\par
Déruchette éclata en sanglots.\par
 En ce moment ils entendirent une voix lente et grave qui disait :\par
— Pourquoi ne vous mariez-vous pas ?\par
Ebenezer tourna la tête. Déruchette leva les yeux.\par
Gilliatt était devant eux.\par
Il venait d’entrer par un sentier latéral.\par
Gilliatt n’était plus le même homme que la veille. Il avait peigné ses cheveux, il avait fait sa barbe, il avait mis des souliers, il avait une chemise blanche de marin à grand col rabattu, il était vêtu de ses habits de matelot les plus neufs. On voyait une bague d’or à son petit doigt. Il semblait profondément calme. Son hâle était livide.\par
Du bronze qui souffre, tel était ce visage.\par
Ils le regardèrent, stupéfaits. Quoique méconnaissable, Déruchette le reconnut. Quant aux paroles qu’il venait de dire, elles étaient si loin de ce qu’ils pensaient en ce moment-là, qu’elles avaient glissé sur leur esprit.\par
Gilliatt reprit :\par
— Quel besoin avez-vous de vous dire adieu ? Mariez-vous. Vous partirez ensemble.\par
Déruchette tressaillit. Elle eut un tremblement de la tête aux pieds.\par
Gilliatt continua :\par
— Miss Déruchette a ses vingt et un ans. Elle ne dépend que d’elle. Son oncle n’est que son oncle. Vous vous aimez...\par
Déruchette interrompit doucement :\par
— Comment se fait-il que vous soyez ici ?\par
 — Mariez-vous, poursuivit Gilliatt.\par
Déruchette commençait à percevoir ce que cet homme lui disait. Elle bégaya :\par
— Mon pauvre oncle...\par
— Il refuserait si le mariage était à faire, dit Gilliatt, il consentira quand le mariage sera fait. D’ailleurs vous allez partir. Quand vous reviendrez, il pardonnera.\par
Gilliatt ajouta avec une nuance amère : — Et puis, il ne pense déjà plus qu’à rebâtir son bateau. Cela l’occupera pendant votre absence. Il a la Durande pour le consoler.\par
— Je ne voudrais pas, balbutia Déruchette, dans une stupeur où l’on sentait de la joie, laisser derrière moi des chagrins.\par
— Ils ne dureront pas longtemps, dit Gilliatt.\par
Ebenezer et Déruchette avaient eu comme un éblouissement. Ils se remettaient maintenant. Dans leur trouble décroissant, le sens des paroles de Gilliatt leur apparaissait. Un nuage y restait mêlé, mais leur affaire à eux n’était pas de résister. On se laisse faire à qui sauve. Les objections à la rentrée dans l’éden sont molles. Il y avait dans l’attitude de Déruchette, imperceptiblement appuyée sur Ebenezer, quelque chose qui faisait cause commune avec ce que disait Gilliatt. Quant à l’énigme de la présence de cet homme et de ses paroles qui, dans l’esprit de Déruchette en particulier, produisaient plusieurs sortes d’étonnements, c’étaient des questions à côté. Cet homme leur disait : Mariez-vous. Ceci était clair. S’il y avait une  responsabilité, il la prenait. Déruchette sentait confusément que, pour des raisons diverses, il en avait le droit. Ce qu’il disait de mess Lethierry était vrai. Ebenezer pensif murmura : Un oncle n’est pas un père.\par
Il subissait la corruption d’une péripétie heureuse et soudaine. Les scrupules probables du prêtre fondaient et se dissolvaient dans ce pauvre cœur amoureux.\par
La voix de Gilliatt devint brève et dure et l’on y sentait comme des pulsations de fièvre :\par
— Tout de suite. Le \emph{Cashmere} part dans deux heures. Vous avez le temps, mais vous n’avez que le temps. Venez.\par
Ebenezer le considérait attentivement.\par
Tout à coup il s’écria :\par
— Je vous reconnais. C’est vous qui m’avez sauvé la vie.\par
Gilliatt répondit :\par
— Je ne crois pas.\par
— Là-bas, à la pointe des Banques.\par
— Je ne connais pas cet endroit-là.\par
— C’est le jour même que j’arrivais.\par
— Ne perdons pas de temps, dit Gilliatt.\par
— Et, je ne me trompe pas, vous êtes l’homme d’hier soir.\par
— Peut-être.\par
— Comment vous appelez-vous ?\par
Gilliatt haussa la voix :\par
— Batelier, attendez-nous. Nous allons revenir. Miss, vous m’avez demandé comment il se faisait que  j’étais ici, c’est bien simple, je marchais derrière vous. Vous avez vingt et un ans. Dans ce pays-ci, quand les personnes sont majeures et dépendent d’elles-mêmes, on se marie en un quart d’heure. Prenons le sentier du bord de l’eau. Il est praticable, la mer ne montera qu’à midi. Mais tout de suite. Venez avec moi.\par
Déruchette et Ebenezer semblaient se consulter du regard. Ils étaient debout l’un près de l’autre, sans bouger ; ils étaient comme ivres. Il y a de ces hésitations étranges au bord de cet abîme, le bonheur. Ils comprenaient sans comprendre.\par
— Il s’appelle Gilliatt, dit Déruchette bas à Ebenezer.\par
Gilliatt reprit avec une sorte d’autorité :\par
— Qu’attendez-vous ? je vous dis de me suivre.\par
— Où ? demanda Ebenezer.\par
— Là.\par
Et Gilliatt montra du doigt le clocher de l’église.\par
Ils le suivirent.\par
Gilliatt allait devant. Son pas était ferme. Eux ils chancelaient.\par
A mesure qu’ils avançaient vers le clocher, on voyait poindre sur ces purs et beaux visages d’Ebenezer et de Déruchette quelque chose qui serait bientôt le sourire. L’approche de l’église les éclairait. Dans l’œil creux de Gilliatt, il y avait de la nuit.\par
On eût dit un spectre menant deux âmes au paradis.\par
Ebenezer et Déruchette ne se rendaient pas bien compte de ce qui allait arriver. L’intervention de cet homme était la branche où se raccroche le noyé. Ils  suivaient Gilliatt avec la docilité du désespoir pour le premier venu. Qui se sent mourir n’est pas difficile sur l’acceptation des incidents. Déruchette, plus ignorante, était plus confiante. Ebenezer songeait. Déruchette était majeure. Les formalités du mariage anglais sont très simples, surtout dans les pays autochthones où les recteurs de paroisse ont un pouvoir presque discrétionnaire ; mais le doyen néanmoins consentirait-il à célébrer le mariage sans même s’informer si l’oncle consentait ? Il y avait là une question. Pourtant, on pouvait essayer. Dans tous les cas, c’était un sursis.\par
Mais qu’était-ce que cet homme ? et si c’était lui en effet que la veille mess Lethierry avait déclaré son gendre, comment s’expliquer ce qu’il faisait là ? Lui, l’obstacle, il se changeait en providence. Ebenezer s’y prêtait, mais il donnait à ce qui se passait le consentement tacite et rapide de l’homme qui se sent sauvé.\par
Le sentier était inégal, parfois mouillé et difficile. Ebenezer, absorbé, ne faisait pas attention aux flaques d’eau et aux blocs de galets. De temps en temps, Gilliatt se retournait et disait à Ebenezer : — Prenez garde à ces pierres, donnez-lui la main.
 \subsubsection[{C.III.3. La prévoyance de l’abnégation}]{C.III.3. \\
La prévoyance de l’abnégation}
\noindent Dix heures et demie sonnaient comme ils entraient dans l’église.\par
A cause de l’heure, et aussi à cause de la solitude de la ville ce jour-là, l’église était vide.\par
Au fond pourtant, près de la table qui, dans les églises réformées, remplace l’autel, il y avait trois personnes ; c’étaient le doyen et son évangéliste ; plus le registraire. Le doyen, qui était le révérend Jaquemin Hérode, était assis ; l’évangéliste et le registraire étaient debout.\par
Le Livre, ouvert, était sur la table.\par
A côté, sur une crédence, s’étalait un autre livre, le registre de paroisse, ouvert également, et sur lequel un œil attentif eût pu remarquer une page fraîchement écrite et dont l’encre n’était pas encore séchée. Une plume et une écritoire étaient à côté du registre.\par
 En voyant entrer le révérend Ebenezer Caudray, le révérend Jaquemin Hérode se leva.\par
— Je vous attends, dit-il. Tout est prêt.\par
Le doyen, en effet, était en robe d’officiant.\par
Ebenezer regarda Gilliatt.\par
Le révérend doyen ajouta :\par
— Je suis à vos ordres, mon collègue.\par
Et il salua.\par
Ce salut ne s’égara ni à droite ni à gauche. Il était évident, à la direction du rayon visuel du doyen, que pour lui Ebenezer seul existait. Ebenezer était clergyman et gentleman. Le doyen ne comprenait dans sa salutation ni Déruchette qui était à côté, ni Gilliatt qui était en arrière. Il y avait dans son regard une parenthèse où le seul Ebenezer était admis. Le maintien de ces nuances fait partie du bon ordre et consolide les sociétés.\par
Le doyen reprit avec une aménité gracieusement altière :\par
— Mon collègue, je vous fais mon double compliment. Votre oncle est mort et vous prenez femme ; vous voilà riche par l’un et heureux par l’autre. Du reste, maintenant, grâce à ce bateau à vapeur qu’on va rétablir, miss Lethierry aussi est riche, ce que j’approuve. Miss Lethierry est née sur cette paroisse, j’ai vérifié la date de sa naissance sur le registre. Miss Lethierry est majeure, et s’appartient. D’ailleurs son oncle, qui est toute sa famille, consent. Vous voulez vous marier tout de suite à cause de votre départ, je le comprends ; mais, ce mariage étant d’un recteur de  paroisse, j’aurais souhaité un peu de solennité. J’abrégerai pour vous être agréable. L’essentiel peut tenir dans le sommaire. L’acte est déjà tout dressé sur le registre que voici, et il n’y a que les noms à remplir. Aux termes de la loi et coutume, le mariage peut être célébré immédiatement après l’inscription. La déclaration voulue pour la licence a été dûment faite. Je prends sur moi une petite irrégularité, car la demande de licence eût dû être préalablement enregistrée sept jours d’avance ; mais je me rends à la nécessité et à l’urgence de votre départ. Soit. Je vais vous marier. Mon évangéliste sera le témoin de l’époux ; quant au témoin de l’épouse...\par
Le doyen se tourna vers Gilliatt.\par
Gilliatt fit un signe de tête.\par
— Cela suffit, dit le doyen.\par
Ebenezer restait immobile. Déruchette était l’extase, pétrifiée.\par
Le doyen continua :\par
— Maintenant, toutefois, il y a un obstacle.\par
Déruchette fit un mouvement.\par
Le doyen poursuivit :\par
— L’envoyé, ici présent, de mess Lethierry, lequel envoyé a demandé pour vous la licence et a signé la déclaration sur le registre, — et du pouce de sa main gauche le doyen désigna Gilliatt, ce qui l’exemptait d’articuler ce nom quelconque, — l’envoyé de mess Lethierry m’a dit ce matin que mess Lethierry, trop occupé pour venir en personne, désirait que le mariage se fît incontinent. Ce désir, exprimé verbalement,  n’est point assez. Je ne saurais, à cause des dispenses à accorder et de l’irrégularité que je prends sur moi, passer outre si vite sans m’informer près de mess Lethierry, à moins qu’on ne me montre sa signature. Quelle que soit ma bonne volonté, je ne puis me contenter d’une parole qu’on vient me redire. Il me faudrait quelque chose d’écrit.\par
— Qu’à cela ne tienne, dit Gilliatt.\par
Et il présenta au révérend doyen un papier.\par
Le doyen se saisit du papier, le parcourut d’un coup d’œil, sembla passer quelques lignes, sans doute inutiles, et lut tout haut :\par
— « ... Va chez le doyen pour avoir les dispenses. Je désire que le mariage se fasse le plus tôt possible. Tout de suite serait le mieux. »\par
Il posa le papier sur la table, et poursuivit :\par
— Signé Lethierry. La chose serait plus respectueusement adressée à moi. Mais puisqu’il s’agit d’un collègue, je n’en demande pas davantage.\par
Ebenezer regarda de nouveau Gilliatt. Il y a des ententes d’âmes. Ebenezer sentait là une fraude ; et il n’eut pas la force, il n’eut peut-être pas même l’idée, de la dénoncer. Soit obéissance à un héroïsme latent qu’il entrevoyait, soit étourdissement de la conscience par le coup de foudre du bonheur, il demeura sans paroles.\par
Le doyen prit la plume et remplit, aidé du registraire, les blancs de la page écrite sur le registre, puis il se redressa, et, du geste, invita Ebenezer et Déruchette à s’approcher de la table.\par
 La cérémonie commença.\par
Ce fut un moment étrange.\par
Ebenezer et Déruchette étaient l’un près de l’autre devant le ministre. Quiconque a fait un songe où il s’est marié a éprouvé ce qu’ils éprouvaient.\par
Gilliatt était à quelque distance dans l’obscurité des piliers.\par
Déruchette le matin en se levant, désespérée, pensant au cercueil et au suaire, s’était vêtue de blanc. Cette idée de deuil fut à propos pour la noce. La robe blanche fait tout de suite une fiancée. La tombe aussi est une fiançaille.\par
Un rayonnement se dégageait de Déruchette. Jamais elle n’avait été ce qu’elle était en cet instant-là. Déruchette avait ce défaut d’être peut-être trop jolie et pas assez belle. Sa beauté péchait, si c’est là pécher, par excès de grâce. Déruchette au repos, c’est-à-dire en dehors de la passion et de la douleur, était, nous avons indiqué ce détail, surtout gentille. La transfiguration de la fille charmante, c’est la vierge idéale. Déruchette, grandie par l’amour et par la souffrance, avait eu, qu’on nous passe le mot, cet avancement. Elle avait la même candeur avec plus de dignité, la même fraîcheur avec plus de parfum. C’était quelque chose comme une pâquerette qui deviendrait un lys.\par
La moiteur des pleurs taries était sur ses joues. Il y avait peut-être encore une larme dans le coin du sourire. Les larmes séchées, vaguement visibles, sont une sombre et douce parure au bonheur.\par
 Le doyen, debout près de la table, posa un doigt sur la bible ouverte et demanda à haute voix :\par
— Y a-t-il opposition ?\par
Personne ne répondit.\par
— Amen, dit le doyen.\par
Ebenezer et Déruchette avancèrent d’un pas vers le révérend Jaquemin Hérode.\par
Le doyen dit :\par
— Joë Ebenezer Caudray, veux-tu avoir cette femme pour ton épouse ?\par
Ebenezer répondit :\par
— Je le veux.\par
Le doyen reprit :\par
— Durande Déruchette Lethierry, veux-tu avoir cet homme pour ton mari ?\par
Déruchette, dans l’agonie de l’âme sous trop de joie comme de la lampe sous trop d’huile, murmurai plutôt qu’elle ne prononça : — Je le veux.\par
Alors, suivant le beau rite du mariage anglican, le doyen regarda autour de lui et fit dans l’ombre de l’église cette demande solennelle :\par
— Qui est-ce qui donne cette femme à cet homme ?\par
— Moi, dit Gilliatt.\par
Il y eut un silence. Ebenezer et Déruchette sentirent on ne sait quelle vague oppression à travers leur ravissement.\par
Le doyen mit la main droite de Déruchette dans la main droite d’Ebenezer, et Ebenezer dit à Déruchette :\par
— Déruchette, je te prends pour ma femme, soit  que tu sois meilleure ou pire, plus riche ou plus pauvre, en maladie ou en santé, pour t’aimer jusqu’à la mort, et je te donne ma foi.\par
Le doyen mit la main droite d’Ebenezer dans la main droite de Déruchette, et Déruchette dit à Ebenezer :\par
— Ebenezer, je te prends pour mon mari, soit que tu sois meilleur ou pire, plus riche ou plus pauvre, en maladie ou en santé, pour t’aider et t’obéir jusqu’à la mort, et je te donne ma foi.\par
Le doyen reprit :\par
— Où est l’anneau ?\par
Ceci était l’imprévu. Ebenezer, pris au dépourvu, n’avait pas d’anneau.\par
Gilliatt ôta la bague d’or qu’il avait au petit doigt, et la présenta au doyen. C’était probablement l’anneau « de mariage » acheté le matin au bijoutier de Commercial-Arcade.\par
Le doyen posa l’anneau sur le livre, puis le remit à Ebenezer.\par
Ebenezer prit la petite main gauche, toute tremblante, de Déruchette, passa l’anneau au quatrième doigt, et dit :\par
— Je t’épouse avec cet anneau.\par
— Au nom du Père, du Fils et du Saint-Esprit, dit le doyen.\par
— Que cela soit ainsi, dit l’évangéliste.\par
Le doyen éleva la voix :\par
— Vous êtes époux.\par
— Que cela soit, dit l’évangéliste.\par
 Le doyen reprit :\par
— Prions.\par
Ebenezer et Déruchette se retournèrent vers la table et se mirent à genoux.\par
Gilliatt, resté debout, baissa la tête.\par
Eux s’agenouillaient devant Dieu, lui se courbait sous la destinée.
 \subsubsection[{C.III.4. « Pour ta femme, quand tu te marieras »}]{C.III.4. \\
« Pour ta femme, quand tu te marieras »}
\noindent A leur sortie de l’église, ils virent le \emph{Cashmere} qui commençait à appareiller.\par
— Vous êtes à temps, dit Gilliatt.\par
Ils reprirent le sentier du Havelet.\par
Ils allaient devant, Gilliatt maintenant marchait derrière.\par
C’étaient deux somnambules. Ils n’avaient pour ainsi dire que changé d’égarement. Ils ne savaient ni où ils étaient, ni ce qu’ils faisaient ; ils se hâtaient machinalement, ils ne se souvenaient plus de l’existence de rien, ils se sentaient l’un à l’autre, ils ne pouvaient lier deux idées. On ne pense pas plus dans l’extase qu’on ne nage dans le torrent. Du milieu des ténèbres, ils étaient tombés brusquement dans un Niagara de joie. On pourrait dire qu’ils subissaient l’emparadisement. Ils ne se parlaient point, se disant trop de choses avec l’âme. Déruchette serrait contre elle le bras d’Ebenezer.\par
 Le pas de Gilliatt derrière eux leur faisait par moments songer qu’il était là. Ils étaient profondément émus, mais sans dire mot ; l’excès d’émotion se résout en stupeur. La leur était délicieuse, mais accablante. Ils étaient mariés. Ils ajournaient, on se reverrait, ce que Gilliatt faisait était bien, voilà tout. Le fond de ces deux cœurs le remerciait ardemment et vaguement. Déruchette se disait qu’elle avait là quelque chose à débrouiller, plus tard. En attendant, ils acceptaient. Ils se sentaient à la discrétion de cet homme décisif et subit, qui, d’autorité, faisait leur bonheur. Lui adresser des questions, causer avec lui, était impossible. Trop d’impressions se précipitaient sur eux à la fois. Leur engloutissement est pardonnable.\par
Les faits sont parfois une grêle. Ils vous criblent. Cela assourdit. La brusquerie des incidents tombant dans des existences habituellement calmes rend très vite les événements inintelligibles à ceux qui en souffrent ou qui en profitent. On n’est pas au fait de sa propre aventure. On est écrasé sans deviner ; on est couronné sans comprendre. Déruchette, en particulier, depuis quelques heures, avait reçu toutes les commotions ; d’abord l’éblouissement, Ebenezer dans le jardin ; puis le cauchemar, ce monstre déclaré son mari ; puis la désolation, l’ange ouvrant ses ailes et prêt à partir ; maintenant c’était la joie, une joie inouïe, avec un fond indéchiffrable ; le monstre lui donnant l’ange, à elle Déruchette ; le mariage sortant de l’agonie ; ce Gilliatt, la catastrophe d’hier, le salut d’aujourd’hui. Elle ne se rendait compte de rien. Il était évident que  depuis le matin Gilliatt n’avait eu d’autre occupation que de les marier ; il avait tout fait ; il avait répondu pour mess Lethierry, vu le doyen, demandé la licence, signé la déclaration voulue ; voilà comment le mariage avait pu s’accomplir. Mais Déruchette ne le comprenait pas ; d’ailleurs lors même qu’elle eût compris comment, elle n’eût pas compris pourquoi.\par
Fermer les yeux, rendre grâces mentalement, oublier la terre et la vie, se laisser emporter au ciel par ce bon démon, il n’y avait que cela à faire. Un éclaircissement était trop long, un remercîment était trop peu. Elle se taisait dans ce doux abrutissement du bonheur.\par
Un peu de pensée leur restait, assez pour se conduire. Sous l’eau il y a des parties de l’éponge qui demeurent blanches. Ils avaient juste la quantité de lucidité qu’il fallait pour distinguer la mer de la terre et le \emph{Cashmere} de tout autre navire.\par
En quelques minutes, ils furent au Havelet.\par
Ebenezer entra le premier dans le bateau. Au moment où Déruchette allait le suivre, elle eut la sensation de sa manche doucement retenue. C’était Gilliatt qui avait posé un doigt sur un pli de sa robe.\par
— Madame, dit-il, vous ne vous attendiez pas à partir. J’ai pensé que vous auriez peut-être besoin de robes et de linge. Vous trouverez à bord du \emph{Cashmere }un coffre qui contient des objets pour femme. Ce coffre me vient de ma mère. Il était destiné à la femme que j’épouserais. Permettez-moi de vous l’offrir.\par
Déruchette se réveilla à demi de son rêve. Elle se  tourna vers Gilliatt. Gilliatt, d’une voix basse et qu’on entendait à peine, continua :\par
— Maintenant, ce n’est pas pour vous retarder, mais voyez-vous, madame, je crois qu’il faut que je vous explique. Le jour qu’il y a eu ce malheur, vous étiez assise dans la salle basse, vous avez dit une parole. Vous ne vous souvenez pas, c’est tout simple. On n’est pas forcé de se souvenir de tous les mots qu’on dit. Mess Lethierry avait beaucoup de chagrin. Il est certain que c’était un bon bateau, et qui rendait des services. Le malheur de la mer était arrivé ; il y avait de l’émotion dans le pays. Ce sont là des choses, naturellement, qu’on a oubliées. Il n’y a pas eu que ce navire-là perdu dans les rochers. On ne peut pas penser toujours à un accident. Seulement ce que je voulais vous dire, c’est que, comme on disait personne n’ira, j’y suis allé. Ils disaient c’est impossible ; ce n’était pas cela qui était impossible. Je vous remercie de m’écouter un petit instant. Vous comprenez, madame, si je suis allé là, ce n’était pas pour vous offenser. D’ailleurs la chose date de très loin. Je sais que vous êtes pressée. Si on avait le temps, si on parlait, on se souviendrait, mais cela ne sert à rien. La chose remonte à un jour où il y avait de la neige. Et puis une fois que je passais, j’ai cru que vous aviez souri. C’est comme ça que ça s’explique. Quant à hier, je n’avais pas eu le temps de rentrer chez moi, je sortais du travail, j’étais tout déchiré, je vous ai fait peur, vous vous êtes trouvée mal, j’ai eu tort, on n’arrive pas ainsi chez les personnes, je vous prie de ne pas m’en vouloir. C’est à peu près  tout ce que je voulais dire. Vous allez partir. Vous aurez beau temps. Le vent est à l’est. Adieu, madame. Vous trouvez juste que je vous parle un peu, n’est-ce pas ? ceci est une dernière minute.\par
— Je pense à ce coffre, répondit Déruchette. Mais pourquoi ne pas le garder pour votre femme, quand vous vous marierez ?\par
— Madame, dit Gilliatt, je ne me marierai probablement pas.\par
— Ce sera dommage, car vous êtes bon. Merci.\par
Et Déruchette sourit. Gilliatt lui rendit ce sourire.\par
Puis il aida Déruchette à entrer dans le canot.\par
Moins d’un quart d’heure après, le bateau où étaient Ebenezer et Déruchette abordait en rade le \emph{Cashmere.}
 \subsubsection[{C.III.5. La grande tombe}]{C.III.5. \\
La grande tombe}
\noindent Gilliatt suivit le bord de l’eau, passa rapidement dans Saint-Pierre-Port, puis se remit à marcher vers Saint-Sampson le long de la mer, se dérobant aux rencontres, évitant les routes, pleines de passants par sa faute.\par
Dès longtemps, on le sait, il avait une manière à lui de traverser dans tous les sens le pays sans être vu de personne. Il connaissait des sentiers, il s’était fait des itinéraires isolés et serpentants ; il avait l’habitude farouche de l’être qui ne se sent pas aimé ; il restait lointain. Tout enfant, voyant peu d’accueil dans les visages des hommes, il avait pris ce pli, qui depuis était devenu son instinct, de se tenir à l’écart.\par
Il dépassa l’Esplanade, puis la Salerie. De temps en temps, il se retournait et regardait, en arrière de lui, dans la rade, le \emph{Cashmere,} qui venait de mettre à la voile. Il y avait peu de vent, Gilliatt allait plus vite que le \emph{Cashmere}. Gilliatt marchait dans les roches extrêmes  du bord de l’eau, la tête baissée. Le flux commençait à monter.\par
A un certain moment il s’arrêta et, tournant le dos à la mer, il considéra pendant quelques minutes, au delà des rochers cachant la route du Valle, un bouquet de chênes. C’étaient les chênes du lieu dit les Basses-Maisons. Là, autrefois, sous ces arbres, le doigt de Déruchette avait écrit son nom, \emph{Gilliatt}, sur la neige. Il y avait longtemps que cette neige était fondue.\par
Il poursuivit son chemin.\par
La journée était charmante plus qu’aucune qu’il y eût encore eu cette année-là. Cette matinée avait on ne sait quoi de nuptial. C’était un de ces jours printaniers où mai se dépense tout entier ; la création semble n’avoir d’autre but que de se donner une fête et de faire son bonheur. Sous toutes les rumeurs, de la forêt comme du village, de la vague comme de l’atmosphère, il y avait un roucoulement. Les premiers papillons se posaient sur les premières roses. Tout était neuf dans la nature, les herbes, les mousses, les feuilles, les parfums, les rayons. Il semblait que le soleil n’eût jamais servi. Les cailloux étaient lavés de frais. La profonde chanson des arbres était chantée par des oiseaux nés d’hier. Il est probable que leur coquille d’œuf cassée par leur petit bec était encore dans le nid. Des essais d’ailes bruissaient dans le tremblement des branches. Ils chantaient leur premier chant, ils volaient leur premier vol. C’était un doux partage de tous à la fois, huppes, mésanges, piquebois, chardonnerets, bouvreuils, moines et misses. Les lilas, les muguets, les  daphnés, les glycines, faisaient dans les fourrés un bariolage exquis. Une très jolie lentille d’eau qu’il y a à Guernesey, couvrait les mares d’une nappe d’émeraude. Les bergeronnettes et les épluque-pommiers, qui font de si gracieux petits nids, s’y baignaient. Par toutes les claires-voies de la végétation on apercevait le bleu du ciel. Quelques nuées lascives s’entre-poursuivaient dans l’azur avec des ondoiements de nymphes. On croyait sentir passer des baisers que s’envoyaient des bouches invisibles. Pas un vieux mur qui n’eût, comme un marié, son bouquet de giroflées. Les prunelliers étaient en fleur, les cytises étaient en fleur ; on voyait ces monceaux blancs qui luisaient et ces monceaux jaunes qui étincelaient à travers les entrecroisements des rameaux. Le printemps jetait tout son argent et tout son or dans l’immense panier percé des bois. Les pousses nouvelles étaient toutes fraîches vertes. On entendait en l’air des cris de bienvenue. L’été hospitalier ouvrait sa porte aux oiseaux lointains. C’était l’instant de l’arrivée des hirondelles. Les thyrses des ajoncs bordaient les talus des chemins creux, en attendant les thyrses des aubépines. Le beau et le joli faisaient bon voisinage ; le superbe se complétait par le gracieux ; le grand ne gênait pas le petit ; aucune note du concert ne se perdait ; les magnificences microscopiques étaient à leur plan dans la vaste beauté universelle ; on distinguait tout comme dans une eau limpide. Partout une divine plénitude et un gonflement mystérieux faisaient deviner l’effort panique et sacré de la séve en travail. Qui brillait, brillait plus ; qui  aimait, aimait mieux. Il y avait de l’hymne dans la fleur et du rayonnement dans le bruit. La grande harmonie diffuse s’épanouissait. Ce qui commence à poindre provoquait ce qui commence à sourdre. Un trouble, qui venait d’en bas, et qui venait aussi d’en haut, remuait vaguement les cœurs, corruptibles à l’influence éparse et souterraine des germes. La fleur promettait obscurément le fruit, toute vierge songeait, la reproduction des êtres, préméditée par l’immense âme de l’ombre, s’ébauchait dans l’irradiation des choses. On se fiançait partout. On s’épousait sans fin. La vie, qui est la femelle, s’accouplait avec l’infini, qui est le mâle. Il faisait beau, il faisait clair, il faisait chaud ; à travers les haies, dans les enclos, on voyait rire les enfants. Quelques-uns jouaient aux mérelles. Les pommiers, les pêchers, les cerisiers, les poiriers, couvraient les vergers de leurs grosses touffes pâles ou vermeilles. Dans l’herbe, primevères, pervenches, achillées, marguerites, amaryllis, jacinthes, et les violettes, et les véroniques. Les bourraches bleues, les iris jaunes, pullulaient, avec ces belles petites étoiles roses qui fleurissent toujours en troupe et qu’on appelle pour cela « les compagnons ». Des bêtes toutes dorées couraient entre les pierres. Les joubarbes en floraison empourpraient les toits de chaume. Les travailleuses des ruches étaient dehors. L’abeille était à la besogne. L’étendue était pleine du murmure des mers et du bourdonnement des mouches. La nature, perméable au printemps, était moite de volupté.\par
Quand Gilliatt arriva à Saint-Sampson, il n’y avait  pas encore d’eau au fond du port, et il put le traverser à pied sec, inaperçu derrière les coques de navires au radoub. Un cordon de pierres plates espacées qu’il y a là, aide à ce passage.\par
Gilliatt ne fut pas remarqué. La foule était à l’autre bout du port, près du goulet, aux Bravées. Là son nom était dans toutes les bouches. On parlait tant de lui qu’on ne fit pas attention à lui. Gilliatt passa, caché en quelque sorte par le bruit qu’il faisait.\par
Il vit de loin la panse à la place où il l’avait amarrée, la cheminée de la machine entre ses quatre chaînes, un mouvement de charpentiers à l’ouvrage, des silhouettes confuses d’allants et de venants, et il entendit la voix tonnante et joyeuse de mess Lethierry donnant des ordres.\par
Il s’enfonça dans les ruettes.\par
Il n’y avait personne derrière les Bravées, toute la curiosité étant sur le devant. Gilliatt prit le sentier longeant le mur bas du jardin. Il s’arrêta dans l’angle où était la mauve sauvage ; il revit la pierre où il s’était assis ; il revit le banc de bois où s’était assise Déruchette. Il regarda la terre de l’allée où il avait vu s’embrasser deux ombres, qui avaient disparu.\par
Il se remit en marche. Il gravit la colline du château du Valle, puis la redescendit, et se dirigea vers le Bû de la Rue.\par
Le Houmet-Paradis était solitaire.\par
Sa maison était telle qu’il l’avait laissée le matin après s’être habillé pour aller à Saint-Pierre-Port.\par
Une fenêtre était ouverte. Par cette fenêtre on  voyait le bug pipe accroché à un clou de la muraille.\par
On apercevait sur la table la petite bible donnée en remercîment à Gilliatt par un inconnu qui était Ebenezer.\par
La clef était à la porte. Gilliatt approcha, posa la main sur cette clef, ferma la porte à double tour, mit la clef dans sa poche, et s’éloigna.\par
Il s’éloigna, non du côté de la terre, mais du côté de la mer.\par
Il traversa diagonalement son jardin, par le plus court, sans précaution pour les plates-bandes, en ayant soin toutefois de ne pas écraser les seakales, qu’il avait plantés parce que c’était un goût de Déruchette.\par
Il franchit le parapet et descendit dans les brisants.\par
Il se mit à suivre, allant toujours devant lui, la longue et étroite ligne de récifs qui liait le Bû de la Rue à ce gros obélisque de granit debout au milieu de la mer qu’on appelait la Corne de la Bête. C’est là qu’était la Chaise Gild-Holm-’Ur.\par
Il enjambait d’un récif à l’autre comme un géant sur des cimes. Faire ces enjambées sur une crête de brisants, cela ressemble à marcher sur l’arête d’un toit.\par
Une pêcheuse à la trouble qui rôdait pieds nus dans les flaques d’eau à quelque distance, et qui regagnait le rivage, lui cria : Prenez garde. La mer arrive.\par
Il continua d’avancer.\par
Parvenu à ce grand rocher de la pointe, la Corne,  qui faisait pinacle sur la mer, il s’arrêta. La terre finissait là. C’était l’extrémité du petit promontoire.\par
Il regarda.\par
Au large, quelques barques, à l’ancre, pêchaient. On voyait de temps en temps sur ces bateaux des ruissellements d’argent au soleil qui étaient la sortie de l’eau des filets. Le \emph{Cashmere} n’était pas encore à la hauteur de Saint-Sampson ; il avait déployé son grand hunier. Il était entre Herm et Jethou.\par
Gilliatt tourna le rocher. Il parvint sous la Chaise Gild-Holm-’Ur, au pied de cette espèce d’escalier abrupt que, moins de trois mois auparavant, il avait aidé Ebenezer à descendre. Il le monta.\par
La plupart des degrés étaient déjà sous l’eau. Deux ou trois seulement étaient encore à sec. Il les escalada.\par
Ces degrés menaient à la Chaise Gild-Holm-’Ur. Il arriva à la Chaise, la considéra un moment, appuya sa main sur ses yeux et la fit lentement glisser d’un sourcil à l’autre, geste par lequel il semble qu’on essuie le passé, puis il s’assit dans le creux de roche, avec l’escarpement derrière son dos et l’océan sous ses pieds.\par
Le \emph{Cashmere} en ce moment-là élongeait la grosse tour ronde immergée, gardée par un sergent et un canon, qui marque dans la rade le mi-chemin entre Herm et Saint-Pierre-Port.\par
Au-dessus de la tête de Gilliatt, dans les fentes, quelques fleurs de rocher frissonnaient. L’eau était bleue à perte de vue. Le vent étant d’est, il y avait,  peu de ressac autour de Sert, dont on ne voit de Guernesey que la côte occidentale. On apercevait au loin la France comme une brume et la longue bande jaune des sables de Carteret. Par instants, un papillon blanc passait. Les papillons ont le goût de se promener sur la mer.\par
La brise était très faible. Tout ce bleu, en bas comme en haut, était immobile. Aucun tremblement n’agitait ces espèces de serpents d’un azur plus clair ou plus foncé qui marquent à la surface de la mer les torsions latentes des bas-fonds.\par
Le \emph{Cashmere}, peu poussé du vent, avait, pour saisir la brise, hissé ses bonnettes de hune. Il s’était couvert de toile. Mais, le vent étant de travers, l’effet des bonnettes le forçait à serrer de très près la côte de Guernesey. Il avait franchi la balise de Saint-Sampson. Il atteignait la colline du château du Valle. Le moment arrivait où il allait doubler la pointe du Bû de la Rue.\par
Gilliatt le regardait venir.\par
L’air et la vague étaient comme assoupis. La marée se faisait, non par lame, mais par gonflement. Le niveau de l’eau se haussait sans palpitation. La rumeur du large, éteinte, ressemblait à un souffle d’enfant.\par
On entendait dans la direction du havre de Saint-Sampson de petits coups sourds, qui étaient des coups de marteau. C’étaient probablement les charpentiers dressant les palans et le fardier pour retirer de la panse la machine. Ces bruits parvenaient à peine à  Gilliatt, à cause de la masse de granit à laquelle il était adossé.\par
Le \emph{Cashmere} approchait avec une lenteur de fantôme.\par
Gilliatt attendait.\par
Tout à coup un clapotement et une sensation de froid le firent regarder en bas. Le flot lui touchait les pieds.\par
Il baissa les yeux, puis les releva.\par
Le \emph{Cashmere} était tout près.\par
L’escarpement où les pluies avaient creusé la Chaise Gild-Holm-’Ur était si vertical, et il y avait là tant d’eau, que les navires pouvaient sans danger, par les temps calmes, faire chenal à quelques encâblures du rocher.\par
Le \emph{Cashmere} arriva. Il surgit, il se dressa. Il semblait croître sur l’eau. Ce fut comme le grandissement d’une ombre. Le gréement se détacha en noir sur le ciel dans le magnifique balancement de la mer. Les longues voiles, un moment superposées au soleil, devinrent presque roses et eurent une transparence ineffable. Les flots avaient un murmure indistinct. Aucun bruit ne troublait le glissement majestueux de cette silhouette. On voyait sur le pont comme si on y eût été.\par
Le \emph{Cashmere} rasa presque la roche.\par
Le timonier était à la barre, un mousse grimpait aux haubans, quelques passagers, accoudés au bordage, considéraient la sérénité du temps, le capitaine fumait.\par
 Mais ce n’était rien de tout cela que voyait Gilliatt.\par
Il y avait sur le pont un coin plein de soleil. C’était là ce qu’il regardait. Dans ce soleil étaient Ebenezer et Déruchette. Ils étaient assis dans cette lumière, lui près d’elle. Ils se blottissaient gracieusement côte à côte, comme deux oiseaux se chauffant à un rayon de midi, sur un de ces bancs couverts d’un petit plafond goudronné que les navires bien aménagés offrent aux voyageurs et sur lesquels on lit, quand c’est un bâtiment anglais : \emph{For ladies only.} La tête de Déruchette était sur l’épaule d’Ebenezer, le bras d’Ebenezer était derrière la taille de Déruchette ; ils se tenaient les mains, les doigts entre-croisés dans les doigts. Les nuances d’un ange à l’autre étaient saisissables sur ces deux exquises figures faites d’innocence. L’une était plus virginale, l’autre plus sidérale. Leur chaste embrassement était expressif. Tout l’hyménée était là, toute la pudeur aussi. Ce banc était déjà une alcôve et presque un nid. En même temps, c’était une gloire ; la douce gloire de l’amour en fuite dans un nuage.\par
Le silence était céleste.\par
L’œil d’Ebenezer rendait grâce et contemplait ; les lèvres de Déruchette remuaient ; et dans ce charmant silence, comme le vent portait du côté de terre, à l’instant rapide où le sloop glissa à quelques toises de la Chaise Gild-Holm-’Ur, Gilliatt entendit la voix tendre et délicate de Déruchette qui disait :\par
— Vois donc. Il semblerait qu’il y a un homme dans le rocher.\par
 Cette apparition passa.\par
Le \emph{Cashmere} laissa la pointe du Bû de la Rue derrière lui et s’enfonça dans le plissement profond des vagues. En moins d’un quart d’heure, sa mâture et ses voiles ne firent plus sur la mer qu’une sorte d’obélisque blanc décroissant à l’horizon. Gilliatt avait de l’eau jusqu’aux genoux.\par
Il regardait le sloop s’éloigner.\par
La brise fraîchit au large. Il put voir le \emph{Cashmere} hisser ses bonnettes basses et ses focs pour profiter de cette augmentation de vent. Le \emph{Cashmere} était déjà hors des eaux de Guernesey. Gilliatt ne le quittait pas des yeux.\par
Le flot lui arrivait à la ceinture.\par
La marée s’élevait. Le temps passait.\par
Les mauves et les cormorans volaient autour de lui, inquiets. On eût dit qu’ils cherchaient à l’avertir. Peut-être y avait-il dans ces volées d’oiseaux quelque mouette venue des Douvres, qui le reconnaissait.\par
Une heure s’écoula.\par
Le vent du large ne se faisait pas sentir dans la rade, mais la diminution du \emph{Cashmere} était rapide. Le sloop était, selon toute apparence, en pleine vitesse. Il atteignait déjà presque la hauteur des Casquets.\par
Il n’y avait pas d’écume autour du rocher Gild-Holm-’Ur, aucune lame ne battait le granit. L’eau s’enflait paisiblement. Elle atteignait presque les épaules de Gilliatt.\par
Une autre heure s’écoula.\par
Le \emph{Cashmere} était au delà des eaux d’Aurigny. Le  rocher Ortach le cacha un moment. Il entra dans l’occultation de cette roche, puis en ressortit, comme d’une éclipse. Le sloop fuyait au nord. Il gagna la haute mer. Il n’était plus qu’un point ayant, à cause du soleil, la scintillation d’une lumière.\par
Les oiseaux jetaient de petits cris à Gilliatt.\par
On ne voyait plus que sa tête.\par
La mer montait avec une douceur sinistre.\par
Gilliatt, immobile, regardait le \emph{Cashmere} s’évanouir.\par
Le flux était presque à son plein. Le soir approchait. Derrière Gilliatt, dans la rade, quelques bateaux de pêche rentraient.\par
L’œil de Gilliatt, attaché au loin sur le sloop, restait fixe.\par
Cet œil fixe ne ressemblait à rien de ce qu’on peut voir sur la terre. Dans cette prunelle tragique et calme il y avait de l’inexprimable. Ce regard contenait toute la quantité d’apaisement que laisse le rêve non réalisé ; c’était l’acceptation lugubre d’un autre accomplissement. Une fuite d’étoile doit être suivie par des regards pareils. De moment en moment, l’obscurité céleste se faisait sous ce sourcil dont le rayon visuel demeurait fixé à un point de l’espace. En même temps que l’eau infinie autour du rocher Gild-Holm-’Ur, l’immense tranquillité de l’ombre montait dans l’œil profond de Gilliatt.\par
Le \emph{Cashmere}, devenu imperceptible, était maintenant une tache mêlée à la brume. Il fallait pour le distinguer savoir où il était.\par
 Peu à peu, cette tache, qui n’était plus une forme, pâlit.\par
Puis elle s’amoindrit.\par
Puis elle se dissipa.\par
A l’instant où le navire s’effaça à l’horizon, la tête disparut sous l’eau. Il n’y eut plus rien que la mer.
 \section[{NOTES}]{NOTES}\renewcommand{\leftmark}{NOTES}

  
\labelblock{NOTE I.}

\noindent Sur la page du titre on lit, dans le manuscrit original, les mentions suivantes :\par
\bigbreak
\noindent Commencé le 4 juin 1864.\par
Interrompu le 4 août.\par
Repris le 4 décembre.\par
Terminé le 29 avril 1865.\par
Publié le 12 mars 1866.\par
\bigbreak
\noindent A la page qui termine la première partie, \emph{Sieur Clubin}, on lit :\par
\bigbreak
\noindent 3 août, 8 heures 1/2 du matin.\par
Interrompu jusqu’à mon retour. Je vais partir pour mon voyage annuel, le 10 ou le 11.\par
 
\labelblock{NOTE II.}

\noindent Dans la première partie, livre VI, au chapitre II, \emph{Du cognac inespéré}, les lignes qui suivent sont biffées dans le manuscrit :\par
\bigbreak
\noindent Une volonté dans un mécanisme fait contre-poids à l’infini. L’infini, lui aussi, contient un mécanisme. Ses engrenages sont pour nous invisibles, tant ils sont démesurés. Le zodiaque est une de ses roues. La loi des saisons est liée à cette rotation. Il faut à l’aiguille aimantée six cent vingt ans pour qu’elle accomplisse son oscillation complète à l’ouest et à l’est du méridien. Ainsi l’oscillation actuelle, commencée en 1660, ne s’achèvera qu’en 2280. La loi des tempêtes est liée à cette oscillation. Dans cette révolution de six cent vingt ans, c’est tantôt le pôle asiatique, tantôt le pôle américain, qui est le pôle le plus froid. Une période de quarante et un ans ramène le maximum des taches solaires. Franklin a prouvé que les coups de vent du nord-est avaient leur source au sud-ouest. Au sud de l’équateur, les ouragans tournent dans le sens d’une montre, et au nord de l’équateur en sens inverse.\par

\labelblock{NOTE III.}

\noindent Le livre septième de la première partie, \emph{Imprudence de faire des questions à un livre}, a, dans le manuscrit, ces variantes du titre : J{\scshape oli métier que la bible fait la} et D{\scshape ieu parle aux jeunes plus souvent qu’aux vieux}.\par
\bigbreak
\noindent Autres variantes de titres :\par
Deuxième partie. Livre I. Chapitre VIII. \emph{Imporlunœque volucres}.\par
U{\scshape n romain rentrerait}.\par
 Chapitre XII. \emph{Le dedans d’un édifice sous mer}.\par
U{\scshape ne cachette de la mer}.\par
Troisième partie. \emph{Déruchette}.\par
C{\scshape e qui échappe a la mer n’échappe pas a la femme}.\par
Livre III. \emph{Départ du} Cashmere.\par
L{\scshape a mer n’avait pas dit son dernier mot}.
 


% at least one empty page at end (for booklet couv)
\ifbooklet
  \pagestyle{empty}
  \clearpage
  % 2 empty pages maybe needed for 4e cover
  \ifnum\modulo{\value{page}}{4}=0 \hbox{}\newpage\hbox{}\newpage\fi
  \ifnum\modulo{\value{page}}{4}=1 \hbox{}\newpage\hbox{}\newpage\fi


  \hbox{}\newpage
  \ifodd\value{page}\hbox{}\newpage\fi
  {\centering\color{rubric}\bfseries\noindent\large
    Hurlus ? Qu’est-ce.\par
    \bigskip
  }
  \noindent Des bouquinistes électroniques, pour du texte libre à participation libre,
  téléchargeable gratuitement sur \href{https://hurlus.fr}{\dotuline{hurlus.fr}}.\par
  \bigskip
  \noindent Cette brochure a été produite par des éditeurs bénévoles.
  Elle n’est pas faîte pour être possédée, mais pour être lue, et puis donnée.
  Que circule le texte !
  En page de garde, on peut ajouter une date, un lieu, un nom ; pour suivre le voyage des idées.
  \par

  Ce texte a été choisi parce qu’une personne l’a aimé,
  ou haï, elle a en tous cas pensé qu’il partipait à la formation de notre présent ;
  sans le souci de plaire, vendre, ou militer pour une cause.
  \par

  L’édition électronique est soigneuse, tant sur la technique
  que sur l’établissement du texte ; mais sans aucune prétention scolaire, au contraire.
  Le but est de s’adresser à tous, sans distinction de science ou de diplôme.
  Au plus direct ! (possible)
  \par

  Cet exemplaire en papier a été tiré sur une imprimante personnelle
   ou une photocopieuse. Tout le monde peut le faire.
  Il suffit de
  télécharger un fichier sur \href{https://hurlus.fr}{\dotuline{hurlus.fr}},
  d’imprimer, et agrafer ; puis de lire et donner.\par

  \bigskip

  \noindent PS : Les hurlus furent aussi des rebelles protestants qui cassaient les statues dans les églises catholiques. En 1566 démarra la révolte des gueux dans le pays de Lille. L’insurrection enflamma la région jusqu’à Anvers où les gueux de mer bloquèrent les bateaux espagnols.
  Ce fut une rare guerre de libération dont naquit un pays toujours libre : les Pays-Bas.
  En plat pays francophone, par contre, restèrent des bandes de huguenots, les hurlus, progressivement réprimés par la très catholique Espagne.
  Cette mémoire d’une défaite est éteinte, rallumons-la. Sortons les livres du culte universitaire, cherchons les idoles de l’époque, pour les briser.
\fi

\ifdev % autotext in dev mode
\fontname\font — \textsc{Les règles du jeu}\par
(\hyperref[utopie]{\underline{Lien}})\par
\noindent \initialiv{A}{lors là}\blindtext\par
\noindent \initialiv{À}{ la bonheur des dames}\blindtext\par
\noindent \initialiv{É}{tonnez-le}\blindtext\par
\noindent \initialiv{Q}{ualitativement}\blindtext\par
\noindent \initialiv{V}{aloriser}\blindtext\par
\Blindtext
\phantomsection
\label{utopie}
\Blinddocument
\fi
\end{document}
