%%%%%%%%%%%%%%%%%%%%%%%%%%%%%%%%%
% LaTeX model https://hurlus.fr %
%%%%%%%%%%%%%%%%%%%%%%%%%%%%%%%%%

% Needed before document class
\RequirePackage{pdftexcmds} % needed for tests expressions
\RequirePackage{fix-cm} % correct units

% Define mode
\def\mode{a4}

\newif\ifaiv % a4
\newif\ifav % a5
\newif\ifbooklet % booklet
\newif\ifcover % cover for booklet

\ifnum \strcmp{\mode}{cover}=0
  \covertrue
\else\ifnum \strcmp{\mode}{booklet}=0
  \booklettrue
\else\ifnum \strcmp{\mode}{a5}=0
  \avtrue
\else
  \aivtrue
\fi\fi\fi

\ifbooklet % do not enclose with {}
  \documentclass[french,twoside]{book} % ,notitlepage
  \usepackage[%
    papersize={105mm, 297mm},
    inner=12mm,
    outer=12mm,
    top=20mm,
    bottom=15mm,
    marginparsep=0pt,
  ]{geometry}
  \usepackage[fontsize=9.5pt]{scrextend} % for Roboto
\else\ifav
  \documentclass[french,twoside]{book} % ,notitlepage
  \usepackage[%
    a5paper,
    inner=25mm,
    outer=15mm,
    top=15mm,
    bottom=15mm,
    marginparsep=0pt,
  ]{geometry}
  \usepackage[fontsize=12pt]{scrextend}
\else% A4 2 cols
  \documentclass[twocolumn]{report}
  \usepackage[%
    a4paper,
    inner=15mm,
    outer=10mm,
    top=25mm,
    bottom=18mm,
    marginparsep=0pt,
  ]{geometry}
  \setlength{\columnsep}{20mm}
  \usepackage[fontsize=9.5pt]{scrextend}
\fi\fi

%%%%%%%%%%%%%%
% Alignments %
%%%%%%%%%%%%%%
% before teinte macros

\setlength{\arrayrulewidth}{0.2pt}
\setlength{\columnseprule}{\arrayrulewidth} % twocol
\setlength{\parskip}{0pt} % classical para with no margin
\setlength{\parindent}{1.5em}

%%%%%%%%%%
% Colors %
%%%%%%%%%%
% before Teinte macros

\usepackage[dvipsnames]{xcolor}
\definecolor{rubric}{HTML}{800000} % the tonic 0c71c3
\def\columnseprulecolor{\color{rubric}}
\colorlet{borderline}{rubric!30!} % definecolor need exact code
\definecolor{shadecolor}{gray}{0.95}
\definecolor{bghi}{gray}{0.5}

%%%%%%%%%%%%%%%%%
% Teinte macros %
%%%%%%%%%%%%%%%%%
%%%%%%%%%%%%%%%%%%%%%%%%%%%%%%%%%%%%%%%%%%%%%%%%%%%
% <TEI> generic (LaTeX names generated by Teinte) %
%%%%%%%%%%%%%%%%%%%%%%%%%%%%%%%%%%%%%%%%%%%%%%%%%%%
% This template is inserted in a specific design
% It is XeLaTeX and otf fonts

\makeatletter % <@@@


\usepackage{blindtext} % generate text for testing
\usepackage[strict]{changepage} % for modulo 4
\usepackage{contour} % rounding words
\usepackage[nodayofweek]{datetime}
% \usepackage{DejaVuSans} % seems buggy for sffont font for symbols
\usepackage{enumitem} % <list>
\usepackage{etoolbox} % patch commands
\usepackage{fancyvrb}
\usepackage{fancyhdr}
\usepackage{float}
\usepackage{fontspec} % XeLaTeX mandatory for fonts
\usepackage{footnote} % used to capture notes in minipage (ex: quote)
\usepackage{framed} % bordering correct with footnote hack
\usepackage{graphicx}
\usepackage{lettrine} % drop caps
\usepackage{lipsum} % generate text for testing
\usepackage[framemethod=tikz,]{mdframed} % maybe used for frame with footnotes inside
\usepackage{pdftexcmds} % needed for tests expressions
\usepackage{polyglossia} % non-break space french punct, bug Warning: "Failed to patch part"
\usepackage[%
  indentfirst=false,
  vskip=1em,
  noorphanfirst=true,
  noorphanafter=true,
  leftmargin=\parindent,
  rightmargin=0pt,
]{quoting}
\usepackage{ragged2e}
\usepackage{setspace} % \setstretch for <quote>
\usepackage{tabularx} % <table>
\usepackage[explicit]{titlesec} % wear titles, !NO implicit
\usepackage{tikz} % ornaments
\usepackage{tocloft} % styling tocs
\usepackage[fit]{truncate} % used im runing titles
\usepackage{unicode-math}
\usepackage[normalem]{ulem} % breakable \uline, normalem is absolutely necessary to keep \emph
\usepackage{verse} % <l>
\usepackage{xcolor} % named colors
\usepackage{xparse} % @ifundefined
\XeTeXdefaultencoding "iso-8859-1" % bad encoding of xstring
\usepackage{xstring} % string tests
\XeTeXdefaultencoding "utf-8"
\PassOptionsToPackage{hyphens}{url} % before hyperref, which load url package

% TOTEST
% \usepackage{hypcap} % links in caption ?
% \usepackage{marginnote}
% TESTED
% \usepackage{background} % doesn’t work with xetek
% \usepackage{bookmark} % prefers the hyperref hack \phantomsection
% \usepackage[color, leftbars]{changebar} % 2 cols doc, impossible to keep bar left
% \usepackage[utf8x]{inputenc} % inputenc package ignored with utf8 based engines
% \usepackage[sfdefault,medium]{inter} % no small caps
% \usepackage{firamath} % choose firasans instead, firamath unavailable in Ubuntu 21-04
% \usepackage{flushend} % bad for last notes, supposed flush end of columns
% \usepackage[stable]{footmisc} % BAD for complex notes https://texfaq.org/FAQ-ftnsect
% \usepackage{helvet} % not for XeLaTeX
% \usepackage{multicol} % not compatible with too much packages (longtable, framed, memoir…)
% \usepackage[default,oldstyle,scale=0.95]{opensans} % no small caps
% \usepackage{sectsty} % \chapterfont OBSOLETE
% \usepackage{soul} % \ul for underline, OBSOLETE with XeTeX
% \usepackage[breakable]{tcolorbox} % text styling gone, footnote hack not kept with breakable


% Metadata inserted by a program, from the TEI source, for title page and runing heads
\title{\textbf{ Confessions (traduction 1864, Moreau, 1864) }}
\date{401}
\author{Augustin (354, 430)}
\def\elbibl{Augustin (354, 430). 401. \emph{Confessions (traduction 1864, Moreau, 1864)}}
\def\elsource{ \href{https://www.bibliotheque-monastique.ch/bibliotheque/bibliotheque/saints/augustin/confessions/confessions.htm}{\dotuline{Bibliothèque monastique Saint Benoît}}\footnote{\href{https://www.bibliotheque-monastique.ch/bibliotheque/bibliotheque/saints/augustin/confessions/confessions.htm}{\url{https://www.bibliotheque-monastique.ch/bibliotheque/bibliotheque/saints/augustin/confessions/confessions.htm}}} }

% Default metas
\newcommand{\colorprovide}[2]{\@ifundefinedcolor{#1}{\colorlet{#1}{#2}}{}}
\colorprovide{rubric}{red}
\colorprovide{silver}{lightgray}
\@ifundefined{syms}{\newfontfamily\syms{DejaVu Sans}}{}
\newif\ifdev
\@ifundefined{elbibl}{% No meta defined, maybe dev mode
  \newcommand{\elbibl}{Titre court ?}
  \newcommand{\elbook}{Titre du livre source ?}
  \newcommand{\elabstract}{Résumé\par}
  \newcommand{\elurl}{http://oeuvres.github.io/elbook/2}
  \author{Éric Lœchien}
  \title{Un titre de test assez long pour vérifier le comportement d’une maquette}
  \date{1566}
  \devtrue
}{}
\let\eltitle\@title
\let\elauthor\@author
\let\eldate\@date


\defaultfontfeatures{
  % Mapping=tex-text, % no effect seen
  Scale=MatchLowercase,
  Ligatures={TeX,Common},
}


% generic typo commands
\newcommand{\astermono}{\medskip\centerline{\color{rubric}\large\selectfont{\syms ✻}}\medskip\par}%
\newcommand{\astertri}{\medskip\par\centerline{\color{rubric}\large\selectfont{\syms ✻\,✻\,✻}}\medskip\par}%
\newcommand{\asterism}{\bigskip\par\noindent\parbox{\linewidth}{\centering\color{rubric}\large{\syms ✻}\\{\syms ✻}\hskip 0.75em{\syms ✻}}\bigskip\par}%

% lists
\newlength{\listmod}
\setlength{\listmod}{\parindent}
\setlist{
  itemindent=!,
  listparindent=\listmod,
  labelsep=0.2\listmod,
  parsep=0pt,
  % topsep=0.2em, % default topsep is best
}
\setlist[itemize]{
  label=—,
  leftmargin=0pt,
  labelindent=1.2em,
  labelwidth=0pt,
}
\setlist[enumerate]{
  label={\bf\color{rubric}\arabic*.},
  labelindent=0.8\listmod,
  leftmargin=\listmod,
  labelwidth=0pt,
}
\newlist{listalpha}{enumerate}{1}
\setlist[listalpha]{
  label={\bf\color{rubric}\alph*.},
  leftmargin=0pt,
  labelindent=0.8\listmod,
  labelwidth=0pt,
}
\newcommand{\listhead}[1]{\hspace{-1\listmod}\emph{#1}}

\renewcommand{\hrulefill}{%
  \leavevmode\leaders\hrule height 0.2pt\hfill\kern\z@}

% General typo
\DeclareTextFontCommand{\textlarge}{\large}
\DeclareTextFontCommand{\textsmall}{\small}

% commands, inlines
\newcommand{\anchor}[1]{\Hy@raisedlink{\hypertarget{#1}{}}} % link to top of an anchor (not baseline)
\newcommand\abbr[1]{#1}
\newcommand{\autour}[1]{\tikz[baseline=(X.base)]\node [draw=rubric,thin,rectangle,inner sep=1.5pt, rounded corners=3pt] (X) {\color{rubric}#1};}
\newcommand\corr[1]{#1}
\newcommand{\ed}[1]{ {\color{silver}\sffamily\footnotesize (#1)} } % <milestone ed="1688"/>
\newcommand\expan[1]{#1}
\newcommand\foreign[1]{\emph{#1}}
\newcommand\gap[1]{#1}
\renewcommand{\LettrineFontHook}{\color{rubric}}
\newcommand{\initial}[2]{\lettrine[lines=2, loversize=0.3, lhang=0.3]{#1}{#2}}
\newcommand{\initialiv}[2]{%
  \let\oldLFH\LettrineFontHook
  % \renewcommand{\LettrineFontHook}{\color{rubric}\ttfamily}
  \IfSubStr{QJ’}{#1}{
    \lettrine[lines=4, lhang=0.2, loversize=-0.1, lraise=0.2]{\smash{#1}}{#2}
  }{\IfSubStr{É}{#1}{
    \lettrine[lines=4, lhang=0.2, loversize=-0, lraise=0]{\smash{#1}}{#2}
  }{\IfSubStr{ÀÂ}{#1}{
    \lettrine[lines=4, lhang=0.2, loversize=-0, lraise=0, slope=0.6em]{\smash{#1}}{#2}
  }{\IfSubStr{A}{#1}{
    \lettrine[lines=4, lhang=0.2, loversize=0.2, slope=0.6em]{\smash{#1}}{#2}
  }{\IfSubStr{V}{#1}{
    \lettrine[lines=4, lhang=0.2, loversize=0.2, slope=-0.5em]{\smash{#1}}{#2}
  }{
    \lettrine[lines=4, lhang=0.2, loversize=0.2]{\smash{#1}}{#2}
  }}}}}
  \let\LettrineFontHook\oldLFH
}
\newcommand{\labelchar}[1]{\textbf{\color{rubric} #1}}
\newcommand{\milestone}[1]{\autour{\footnotesize\color{rubric} #1}} % <milestone n="4"/>
\newcommand\name[1]{#1}
\newcommand\orig[1]{#1}
\newcommand\orgName[1]{#1}
\newcommand\persName[1]{#1}
\newcommand\placeName[1]{#1}
\newcommand{\pn}[1]{\IfSubStr{-—–¶}{#1}% <p n="3"/>
  {\noindent{\bfseries\color{rubric}   ¶  }}
  {{\footnotesize\autour{ #1}  }}}
\newcommand\reg{}
% \newcommand\ref{} % already defined
\newcommand\sic[1]{#1}
\newcommand\surname[1]{\textsc{#1}}
\newcommand\term[1]{\textbf{#1}}

\def\mednobreak{\ifdim\lastskip<\medskipamount
  \removelastskip\nopagebreak\medskip\fi}
\def\bignobreak{\ifdim\lastskip<\bigskipamount
  \removelastskip\nopagebreak\bigskip\fi}

% commands, blocks
\newcommand{\byline}[1]{\bigskip{\RaggedLeft{#1}\par}\bigskip}
\newcommand{\bibl}[1]{{\RaggedLeft{#1}\par\bigskip}}
\newcommand{\biblitem}[1]{{\noindent\hangindent=\parindent   #1\par}}
\newcommand{\dateline}[1]{\medskip{\RaggedLeft{#1}\par}\bigskip}
\newcommand{\labelblock}[1]{\medbreak{\noindent\color{rubric}\bfseries #1}\par\mednobreak}
\newcommand{\salute}[1]{\bigbreak{#1}\par\medbreak}
\newcommand{\signed}[1]{\bigbreak\filbreak{\raggedleft #1\par}\medskip}

% environments for blocks (some may become commands)
\newenvironment{borderbox}{}{} % framing content
\newenvironment{citbibl}{\ifvmode\hfill\fi}{\ifvmode\par\fi }
\newenvironment{docAuthor}{\ifvmode\vskip4pt\fontsize{16pt}{18pt}\selectfont\fi\itshape}{\ifvmode\par\fi }
\newenvironment{docDate}{}{\ifvmode\par\fi }
\newenvironment{docImprint}{\vskip6pt}{\ifvmode\par\fi }
\newenvironment{docTitle}{\vskip6pt\bfseries\fontsize{18pt}{22pt}\selectfont}{\par }
\newenvironment{msHead}{\vskip6pt}{\par}
\newenvironment{msItem}{\vskip6pt}{\par}
\newenvironment{titlePart}{}{\par }


% environments for block containers
\newenvironment{argument}{\itshape\parindent0pt}{\vskip1.5em}
\newenvironment{biblfree}{}{\ifvmode\par\fi }
\newenvironment{bibitemlist}[1]{%
  \list{\@biblabel{\@arabic\c@enumiv}}%
  {%
    \settowidth\labelwidth{\@biblabel{#1}}%
    \leftmargin\labelwidth
    \advance\leftmargin\labelsep
    \@openbib@code
    \usecounter{enumiv}%
    \let\p@enumiv\@empty
    \renewcommand\theenumiv{\@arabic\c@enumiv}%
  }
  \sloppy
  \clubpenalty4000
  \@clubpenalty \clubpenalty
  \widowpenalty4000%
  \sfcode`\.\@m
}%
{\def\@noitemerr
  {\@latex@warning{Empty `bibitemlist' environment}}%
\endlist}
\newenvironment{quoteblock}% may be used for ornaments
  {\begin{quoting}}
  {\end{quoting}}

% table () is preceded and finished by custom command
\newcommand{\tableopen}[1]{%
  \ifnum\strcmp{#1}{wide}=0{%
    \begin{center}
  }
  \else\ifnum\strcmp{#1}{long}=0{%
    \begin{center}
  }
  \else{%
    \begin{center}
  }
  \fi\fi
}
\newcommand{\tableclose}[1]{%
  \ifnum\strcmp{#1}{wide}=0{%
    \end{center}
  }
  \else\ifnum\strcmp{#1}{long}=0{%
    \end{center}
  }
  \else{%
    \end{center}
  }
  \fi\fi
}


% text structure
\newcommand\chapteropen{} % before chapter title
\newcommand\chaptercont{} % after title, argument, epigraph…
\newcommand\chapterclose{} % maybe useful for multicol settings
\setcounter{secnumdepth}{-2} % no counters for hierarchy titles
\setcounter{tocdepth}{5} % deep toc
\markright{\@title} % ???
\markboth{\@title}{\@author} % ???
\renewcommand\tableofcontents{\@starttoc{toc}}
% toclof format
% \renewcommand{\@tocrmarg}{0.1em} % Useless command?
% \renewcommand{\@pnumwidth}{0.5em} % {1.75em}
\renewcommand{\@cftmaketoctitle}{}
\setlength{\cftbeforesecskip}{\z@ \@plus.2\p@}
\renewcommand{\cftchapfont}{}
\renewcommand{\cftchapdotsep}{\cftdotsep}
\renewcommand{\cftchapleader}{\normalfont\cftdotfill{\cftchapdotsep}}
\renewcommand{\cftchappagefont}{\bfseries}
\setlength{\cftbeforechapskip}{0em \@plus\p@}
% \renewcommand{\cftsecfont}{\small\relax}
\renewcommand{\cftsecpagefont}{\normalfont}
% \renewcommand{\cftsubsecfont}{\small\relax}
\renewcommand{\cftsecdotsep}{\cftdotsep}
\renewcommand{\cftsecpagefont}{\normalfont}
\renewcommand{\cftsecleader}{\normalfont\cftdotfill{\cftsecdotsep}}
\setlength{\cftsecindent}{1em}
\setlength{\cftsubsecindent}{2em}
\setlength{\cftsubsubsecindent}{3em}
\setlength{\cftchapnumwidth}{1em}
\setlength{\cftsecnumwidth}{1em}
\setlength{\cftsubsecnumwidth}{1em}
\setlength{\cftsubsubsecnumwidth}{1em}

% footnotes
\newif\ifheading
\newcommand*{\fnmarkscale}{\ifheading 0.70 \else 1 \fi}
\renewcommand\footnoterule{\vspace*{0.3cm}\hrule height \arrayrulewidth width 3cm \vspace*{0.3cm}}
\setlength\footnotesep{1.5\footnotesep} % footnote separator
\renewcommand\@makefntext[1]{\parindent 1.5em \noindent \hb@xt@1.8em{\hss{\normalfont\@thefnmark . }}#1} % no superscipt in foot
\patchcmd{\@footnotetext}{\footnotesize}{\footnotesize\sffamily}{}{} % before scrextend, hyperref


%   see https://tex.stackexchange.com/a/34449/5049
\def\truncdiv#1#2{((#1-(#2-1)/2)/#2)}
\def\moduloop#1#2{(#1-\truncdiv{#1}{#2}*#2)}
\def\modulo#1#2{\number\numexpr\moduloop{#1}{#2}\relax}

% orphans and widows
\clubpenalty=9996
\widowpenalty=9999
\brokenpenalty=4991
\predisplaypenalty=10000
\postdisplaypenalty=1549
\displaywidowpenalty=1602
\hyphenpenalty=400
% Copied from Rahtz but not understood
\def\@pnumwidth{1.55em}
\def\@tocrmarg {2.55em}
\def\@dotsep{4.5}
\emergencystretch 3em
\hbadness=4000
\pretolerance=750
\tolerance=2000
\vbadness=4000
\def\Gin@extensions{.pdf,.png,.jpg,.mps,.tif}
% \renewcommand{\@cite}[1]{#1} % biblio

\usepackage{hyperref} % supposed to be the last one, :o) except for the ones to follow
\urlstyle{same} % after hyperref
\hypersetup{
  % pdftex, % no effect
  pdftitle={\elbibl},
  % pdfauthor={Your name here},
  % pdfsubject={Your subject here},
  % pdfkeywords={keyword1, keyword2},
  bookmarksnumbered=true,
  bookmarksopen=true,
  bookmarksopenlevel=1,
  pdfstartview=Fit,
  breaklinks=true, % avoid long links
  pdfpagemode=UseOutlines,    % pdf toc
  hyperfootnotes=true,
  colorlinks=false,
  pdfborder=0 0 0,
  % pdfpagelayout=TwoPageRight,
  % linktocpage=true, % NO, toc, link only on page no
}

\makeatother % /@@@>
%%%%%%%%%%%%%%
% </TEI> end %
%%%%%%%%%%%%%%


%%%%%%%%%%%%%
% footnotes %
%%%%%%%%%%%%%
\renewcommand{\thefootnote}{\bfseries\textcolor{rubric}{\arabic{footnote}}} % color for footnote marks

%%%%%%%%%
% Fonts %
%%%%%%%%%
\usepackage[]{roboto} % SmallCaps, Regular is a bit bold
% \linespread{0.90} % too compact, keep font natural
\newfontfamily\fontrun[]{Roboto Condensed Light} % condensed runing heads
\ifav
  \setmainfont[
    ItalicFont={Roboto Light Italic},
  ]{Roboto}
\else\ifbooklet
  \setmainfont[
    ItalicFont={Roboto Light Italic},
  ]{Roboto}
\else
\setmainfont[
  ItalicFont={Roboto Italic},
]{Roboto Light}
\fi\fi
\renewcommand{\LettrineFontHook}{\bfseries\color{rubric}}
% \renewenvironment{labelblock}{\begin{center}\bfseries\color{rubric}}{\end{center}}

%%%%%%%%
% MISC %
%%%%%%%%

\setdefaultlanguage[frenchpart=false]{french} % bug on part


\newenvironment{quotebar}{%
    \def\FrameCommand{{\color{rubric!10!}\vrule width 0.5em} \hspace{0.9em}}%
    \def\OuterFrameSep{\itemsep} % séparateur vertical
    \MakeFramed {\advance\hsize-\width \FrameRestore}
  }%
  {%
    \endMakeFramed
  }
\renewenvironment{quoteblock}% may be used for ornaments
  {%
    \savenotes
    \setstretch{0.9}
    \normalfont
    \begin{quotebar}
  }
  {%
    \end{quotebar}
    \spewnotes
  }


\renewcommand{\headrulewidth}{\arrayrulewidth}
\renewcommand{\headrule}{{\color{rubric}\hrule}}

% delicate tuning, image has produce line-height problems in title on 2 lines
\titleformat{name=\chapter} % command
  [display] % shape
  {\vspace{1.5em}\centering} % format
  {} % label
  {0pt} % separator between n
  {}
[{\color{rubric}\huge\textbf{#1}}\bigskip] % after code
% \titlespacing{command}{left spacing}{before spacing}{after spacing}[right]
\titlespacing*{\chapter}{0pt}{-2em}{0pt}[0pt]

\titleformat{name=\section}
  [block]{}{}{}{}
  [\vbox{\color{rubric}\large\raggedleft\textbf{#1}}]
\titlespacing{\section}{0pt}{0pt plus 4pt minus 2pt}{\baselineskip}

\titleformat{name=\subsection}
  [block]
  {}
  {} % \thesection
  {} % separator \arrayrulewidth
  {}
[\vbox{\large\textbf{#1}}]
% \titlespacing{\subsection}{0pt}{0pt plus 4pt minus 2pt}{\baselineskip}

\ifaiv
  \fancypagestyle{main}{%
    \fancyhf{}
    \setlength{\headheight}{1.5em}
    \fancyhead{} % reset head
    \fancyfoot{} % reset foot
    \fancyhead[L]{\truncate{0.45\headwidth}{\fontrun\elbibl}} % book ref
    \fancyhead[R]{\truncate{0.45\headwidth}{ \fontrun\nouppercase\leftmark}} % Chapter title
    \fancyhead[C]{\thepage}
  }
  \fancypagestyle{plain}{% apply to chapter
    \fancyhf{}% clear all header and footer fields
    \setlength{\headheight}{1.5em}
    \fancyhead[L]{\truncate{0.9\headwidth}{\fontrun\elbibl}}
    \fancyhead[R]{\thepage}
  }
\else
  \fancypagestyle{main}{%
    \fancyhf{}
    \setlength{\headheight}{1.5em}
    \fancyhead{} % reset head
    \fancyfoot{} % reset foot
    \fancyhead[RE]{\truncate{0.9\headwidth}{\fontrun\elbibl}} % book ref
    \fancyhead[LO]{\truncate{0.9\headwidth}{\fontrun\nouppercase\leftmark}} % Chapter title, \nouppercase needed
    \fancyhead[RO,LE]{\thepage}
  }
  \fancypagestyle{plain}{% apply to chapter
    \fancyhf{}% clear all header and footer fields
    \setlength{\headheight}{1.5em}
    \fancyhead[L]{\truncate{0.9\headwidth}{\fontrun\elbibl}}
    \fancyhead[R]{\thepage}
  }
\fi

\ifav % a5 only
  \titleclass{\section}{top}
\fi

\newcommand\chapo{{%
  \vspace*{-3em}
  \centering % no vskip ()
  {\Large\addfontfeature{LetterSpace=25}\bfseries{\elauthor}}\par
  \smallskip
  {\large\eldate}\par
  \bigskip
  {\Large\selectfont{\eltitle}}\par
  \bigskip
  {\color{rubric}\hline\par}
  \bigskip
  {\Large TEXTE LIBRE À PARTICPATION LIBRE\par}
  \centerline{\small\color{rubric} {hurlus.fr, tiré le \today}}\par
  \bigskip
}}

\newcommand\cover{{%
  \thispagestyle{empty}
  \centering
  {\LARGE\bfseries{\elauthor}}\par
  \bigskip
  {\Large\eldate}\par
  \bigskip
  \bigskip
  {\LARGE\selectfont{\eltitle}}\par
  \vfill\null
  {\color{rubric}\setlength{\arrayrulewidth}{2pt}\hline\par}
  \vfill\null
  {\Large TEXTE LIBRE À PARTICPATION LIBRE\par}
  \centerline{{\href{https://hurlus.fr}{\dotuline{hurlus.fr}}, tiré le \today}}\par
}}

\begin{document}
\pagestyle{empty}
\ifbooklet{
  \cover\newpage
  \thispagestyle{empty}\hbox{}\newpage
  \cover\newpage\noindent Les voyages de la brochure\par
  \bigskip
  \begin{tabularx}{\textwidth}{l|X|X}
    \textbf{Date} & \textbf{Lieu}& \textbf{Nom/pseudo} \\ \hline
    \rule{0pt}{25cm} &  &   \\
  \end{tabularx}
  \newpage
  \addtocounter{page}{-4}
}\fi

\thispagestyle{empty}
\ifaiv
  \twocolumn[\chapo]
\else
  \chapo
\fi
{\it\elabstract}
\bigskip
\makeatletter\@starttoc{toc}\makeatother % toc without new page
\bigskip

\pagestyle{main} % after style

  
\chapteropen
 \chapter[{I. L'enfance}]{I. L'enfance}\phantomsection
\label{I}\renewcommand{\leftmark}{I. L'enfance}


\begin{argument}\noindent Invocation. — Ses premières années. — Péchés de son enfance . — Haine de l’étude . — Amour du jeu.
\end{argument}


\chaptercont
\section[{Chapitre premier, grandeur de Dieu.}]{Chapitre premier, grandeur de Dieu.}

\begin{quoteblock}
\noindent « Vous êtes grand, Seigneur, et infiniment louable (Ps, CXLIV, 3) ; grande est votre puissance, et il n’est point de mesure à votre sagesse (Ps. CXLVI, 5). »\end{quoteblock}

\noindent \pn{1}Et c’est vous que l’homme veut louer, chétive partie de votre création, être de boue, promenant sa mortalité, et par elle le témoignage de son péché, et la preuve éloquente que vous résistez, Dieu que vous êtes, aux superbes (I Petr. V, 5 ) ! Et pourtant il veut vous louer, cet homme, chétive partie de votre création ! Vous l’excitez à se complaire dans vos louanges ; car vous nous avez faits pour vous, et notre cœur est inquiet jusqu’à ce qu’il repose en vous. Donnez-moi, Seigneur, de savoir et de comprendre si notre premier acte est de vous invoquer ou de vous louer, et s’il faut, d’abord, vous connaître ou vous invoquer. Mais qui vous invoque en vous ignorant ? On peut invoquer autre que vous dans cette ignorance. Ou plutôt ne vous invoque-t-on pas pour vous connaître ?\par

\begin{quoteblock}
\noindent « Mais est-ce possible, sans croire ? Et comment croire, sans apôtre (Rom. X, 14) ? »\end{quoteblock}


\begin{quoteblock}
\noindent « Ceux. là loueront le Seigneur, qui le recherchent (Ps. XXI, 27). »\end{quoteblock}

\noindent  Car le cherchant, ils le trouveront, et le trouvant, ils le loueront. Que je vous cherche Seigneur, en vous invoquant, et que je vous invoque en croyant en vous ; car vous nous avez été annoncé. Ma foi vous invoque, Seigneur, cette foi que vous m’avez donnée, que vous m’avez inspirée par l’humanité de votre Fils, par le ministère de votre apôtre.
\section[{Chapitre II, Dieu est en l’Homme ; l’Homme est en Dieu.}]{Chapitre II, Dieu est en l’Homme ; l’Homme est en Dieu.}
\noindent \pn{2}Et comment invoquerai-je mon Dieu, mon Dieu et Seigneur ? car l’invoquer, c’est l’appeler en moi. Et quelle place est en moi, pour qu’en moi vienne mon Dieu ? pour que Dieu vienne en moi, Dieu qui a fait le ciel et la terre ? Quoi ! Seigneur mon Dieu, est-il en moi de quoi vous contenir ? Mais le ciel et la terre que vous avez faits, et dans qui vous m’avez fait, vous contiennent-ils ? Or, de ce que sans vous rien ne serait, suit-il que tout ce qui est, vous contienne ? Donc, puisque je suis, comment vous demandé-je de venir en moi, qui ne puis être sans que vous soyez en moi ? et pourtant je ne suis point aux lieux profonds, et vous y êtes ;\par

\begin{quoteblock}
\noindent « car si je descends en enfer je vous y trouve (Ps CXXXVIII,8). »\end{quoteblock}

\noindent  Je ne serais donc point, mon Dieu, je ne serais point du tout si vous n’étiez en moi. Que dis-je ? je ne serais point si je n’étais en vous,\par

\begin{quoteblock}
\noindent « de qui, par qui et en qui toutes choses sont (Rom. XI, 36.»\end{quoteblock}

\noindent    Il est ainsi, Seigneur, il est ainsi. Où donc vous appelé-je, puisque je suis en vous ? D’où viendrez-vous en moi ? car où me retirer hors du ciel et de la terre, pour que de là vienne en moi mon Dieu qui a dit :\par

\begin{quoteblock}
\noindent « C’est moi qui remplis le ciel et la terre (Jérém. XXIII, 24) ? »\end{quoteblock}

\section[{Chapitre III, Dieu est tout entier partout.}]{Chapitre III, Dieu est tout entier partout.}
\noindent \pn{3}Etes-vous donc contenu par le ciel et la terre, parce que vous les remplissez ? ou les remplissez-vous, et reste-t-il encore de vous, puisque vous n’en êtes pas contenu ? Et où répandez-vous, hors du ciel et de la terre, le trop plein de votre être ? Mais avez-vous besoin d’être contenu, vous qui contenez tout, puisque vous n’emplissez qu’en contenant ? Les vases qui sont pleins de vous ne vous font pas votre équilibre ; car s’ils se brisent, vous ne vous répandez pas ; et lorsque vous vous répandez sur nous, vous ne tombez pas, mais vous nous élevez ; et vous ne vous écoulez pas, mais vous recueillez. Remplissant tout, est-ce de vous tout entier que vous remplissez toutes choses ? Ou bien, tout ne pouvant vous contenir, contient-il partie de vous, et toute chose en même temps cette même partie ? ou bien chaque être, chacune ; les plus grands, davantage ; les moindres, moins ? Y a-t-il donc en vous, plus et moins ? Ou plutôt n’êtes-vous pas tout entier partout, et, nulle part, contenu tout entier ?
\section[{Chapitre IV, Grandeurs ineffables de Dieu.}]{Chapitre IV, Grandeurs ineffables de Dieu.}
\noindent \pn{4}Qu’êtes-vous donc, mon Dieu ? qu’êtes-vous, sinon le Seigneur Dieu ?\par

\begin{quoteblock}
\noindent « Car quel autre Seigneur que le Seigneur, quel autre Dieu que notre Dieu (Ps XVII, 32) ? »\end{quoteblock}

\noindent  O très-haut, très-bon, très-puissant, tout-puissant, très-miséricordieux et très-juste, très-caché et très-présent, très-beau et très-fort, stable et incompréhensible, immuable et remuant tout, jamais nouveau, jamais ancien, renouvelant tout et conduisant à leur insu les superbes au dépérissement, toujours en action, toujours en repos, amassant sans besoin, vous portez, remplissez et protégez ; vous créez, nourrissez et perfectionnez, cherchant lorsque rien ne vous manque !\par
Votre amour est sans passion ; votre jalousie sans inquiétude ; votre repentance, sans douleur ; votre colère, sans trouble ; vos œuvres changent, vos conseils ne changent pas. Vous recouvrez ce que vous trouvez et n’avez jamais perdu. Jamais pauvre, vous aimez le gain ; jamais avare, et vous exigez des usures. On vous donne de surérogation pour vous rendre débiteur ; et qu’avons-nous qui ne soit vôtre ? Vous rendez sans devoir ; en payant, vous donnez et ne perdez rien. Et qu’ai-je dit, mon Dieu, ma vie, mes délices saintes ? Et que dit-on de vous en parlant de vous ? Mais malheur à qui se tait de vous ! car sa parole est muette.
\section[{Chapitre V, Dites à mon âme : je suis ton salut.}]{Chapitre V, Dites à mon âme : je suis ton salut.}
\noindent \pn{5}Qui me donnera de me reposer en vous ? Qui vous fera descendre en mon cœur ? Quand trouverai-je l’oubli de mes maux dans l’ivresse de votre présence, dans le charme de vos embrassements, ô mon seul bien ? Que m’êtes. vous ? Par pitié, déliez ma langue ! Que vous suis-je moi-même, pour que vous m’ordonniez de vous aimer, et, si je désobéis, que votre’ colère s’allume contre moi et me menace de grandes misères ? En est-ce donc une petite que de ne vous aimer pas ? Ah ! dites-moi, au non de vos miséricordes, Seigneur mon Dieu, dites-moi ce que vous m’êtes.\par

\begin{quoteblock}
\noindent « Dites à mon âme : Je suis ton salut (Ps XXXIV, 3). »\end{quoteblock}

\noindent  Parlez haut, que j’entende. L’oreille de mon cœur est devant vous, Seigneur ; ouvrez-la, et\par

\begin{quoteblock}
\noindent « dites à mon âme : Je suis ton salut. »\end{quoteblock}

\noindent  Que je coure après cette voix, et que je m’attache à vous ! Ne me voilez pas votre face. Que je meure pour la voir ! Que je meure pour vivre de sa vue !\par
\pn{6}La maison de mon âme est étroite pour vous recevoir, élargissez-la. Elle tombe en ruines, réparez-la. Çà et là elle blesse vos yeux, je l’avoue et le sais ; mais qui la balayera ? À quel autre que vous crierai-je :\par

\begin{quoteblock}
\noindent « Purifiez-moi de mes secrètes souillures, Seigneur, et n’imputez pas celles d’autrui à votre serviteur (Ps XVIII, 13-14) ?»\end{quoteblock}


\begin{quoteblock}
\noindent « Je crois, c’est pourquoi je parle ; Seigneur, vous le savez (Ps CXV, 10). »\end{quoteblock}


\begin{quoteblock}
\noindent « Ne vous ai-je pas, contre moi-même, accusé mes crimes, ô mon Dieu, et ne m’avez-vous pas remis la malice de mon cœur (Ps XXXI, 5) ? »\end{quoteblock}


\begin{quoteblock}
\noindent « Je n’entre point en jugement   avec vous qui êtes la vérité (Job IX 2,3).»\end{quoteblock}


\begin{quoteblock}
\noindent « Et je ne veux pas me tromper moi-même, de peur que mon iniquité ne mente à elle-même (Ps XXVI, 12).»\end{quoteblock}


\begin{quoteblock}
\noindent « Non, je ne conteste pas avec vous ; car si vous pesez les iniquités, Seigneur, Seigneur, qui pourra tenir Ps CXXIX,3) ? »\end{quoteblock}

\section[{Chapitre VI, Enfance de l’Homme ; éternité de Dieu.}]{Chapitre VI, Enfance de l’Homme ; éternité de Dieu.}
\noindent \pn{7}Mais pourtant laissez-moi parler à votre miséricorde, moi, terre et cendre. Laissez-moi pourtant parler, puisque c’est à votre miséricorde et non à l’homme moqueur que je parle. Et vous aussi, peut-être, vous riez-vous de moi ? mais vous aurez bientôt pitié. Qu’est-ce donc que je veux dire, Seigneur mon Dieu, sinon que j’ignore d’où je suis venu ici, en cette mourante vie, ou peut-être cette mort vivante ? Et j’ai été reçu dans les bras de votre miséricorde, comme je l’ai appris des père et mère de ma chair, de qui et en qui vous m’avez formé dans le temps ; car moi je ne m’en souviens pas.\par
J’ai donc reçu les consolations du lait humain. Ni ma mère, ni mes nourrices ne s’emplissaient les mamelles : mais vous, Seigneur, vous me donniez par elles l’aliment de l’enfance, selon votre institution et l’ordre profond de vos richesses. Vous me donniez aussi de ne pas vouloir plus que vous ne me donniez, et à mes nourrices de vouloir me donner ce qu’elles avaient reçu de vous ; car c’était par une affection prédisposée qu’elles me voulaient donner ce que votre opulence leur prodiguait. Ce leur était un bien que le bien qui me venait d’elles, dont elles étaient la source, sans en être le principe. De vous, ô Dieu, tout bien, de vous, mon Dieu, tout mon salut. C’est ce que depuis m’a dit votre voix criant en moi par tous vos dons intérieurs et extérieurs. Car alors que savais-je ? Sucer, savourer avec délices, pleurer aux offenses de ma chair, rien de plus.\par
\pn{8}Et puis je commençai à rire, en dormant d’abord, ensuite éveillé. Tout cela m’a été dit de moi, et je l’ai cru, car il en est ainsi des autres enfants ; autrement je n’ai nul souvenir d’alors. Et peu à peu je remarquais où j’étais, et je voulais montrer mes volontés à qui pouvait les accomplir ; mais en vain : elles étaient au dedans, on était au dehors ; et nul sens né donnait à autrui entrée dans mon âme. Aussi je me démenais de tous mes membres, de toute ma voix, de ce peu de signes, semblables à mes volontés, que je pouvais, tels que je les pouvais, et toutefois en désaccord avec elles. Et quand on ne m’obéissait point, faute de me comprendre ou pour ne pas me nuire, je m’emportais contre ces grandes personnes insoumises et libres, refusant d’être mes esclaves, et je me vengeais d’elles en pleurant. Tels j’ai observé les enfants que j’ai pu voir, et ils m’ont mieux révélé à moi-même, sans me connaître, que ceux qui m’avaient connu en m’élevant.\par
\pn{9}Et voici que dès longtemps mon enfance est morte, et je suis vivant. Mais vous, Seigneur, vous vivez toujours, sans que rien meure en vous, parce qu’avant la naissance des siècles et avant tout ce qui peut être nommé au delà, vous êtes, vous êtes Dieu et Seigneur de tout ce que vous avez créé ; en vous demeurent les causes de tout ce qui passe, et les immuables origines de toutes choses muables, et les raisons éternelles et vivantes de toutes choses irrationnelles et temporelles.\par
Dites-moi, dites à votre suppliant ; dans votre miséricorde, dites à votre misérable serviteur ; dites-moi, mon Dieu, si mon enfance a succédé à quelque âge expiré déjà, et si cet âge est celui que j’ai passé dans le sein de ma mère ? J’en ai quelques indications, j’ai vu moi-même des femmes enceintes. Mais avant ce temps, mon Dieu, mes délices, ai-je été quelque part et quelque chose ? Qui pourrait me répondre ? Personne, ni père, ni mère, ni l’expérience des autres, ni ma mémoire. Ne vous moquez-vous pas de moi à de telles questions, vous qui m’ordonnez de vous louer et de vous glorifier de ce que je connais ?\par
\pn{10}Je vous glorifie, Seigneur du ciel et de la terre, et vous rends hommage des prémices de ma vie et de mon enfance dont je n’ai point souvenir. Mais vous avez permis à l’homme de conjecturer ce qu’il fut par ce qu’il voit en autrui, et de croire beaucoup de lui sur la foi de simples femmes. Déjà j’étais alors, et je vivais ; et déjà, sur le seuil de l’enfance, je cherchais des signes pour manifester mes sentiments. Et de qui un tel animal peut-il être, sinon de vous, Seigneur ? et qui serait donc l’artisan de lui-même ? Est-il autre source d’où être et vivre découle en nous, sinon votre toute-puissance,   ô Seigneur, pour qui être et vivre est tout un, parce que l’Être par excellence et la souveraine vie, c’est vous-même ; car vous êtes le Très-Haut, et vous ne changez pas ; et le jour d’aujourd’hui ne passe point pour vous, et pourtant il passe en vous, parce qu’en vous toutes choses sont, et rien ne trouverait passage si votre main ne contenait tout. Et comme vos années ne manquent point, vos années, c’est aujourd’hui. Et combien de nos jours, et des jours de nos pères ont passé par votre aujourd’hui et en ont reçu leur être et leur durée ; et d’autres passeront encore, qui recevront de lui leur mesure d’existence. Mais vous, vous êtes le même ; ce n’est pas demain, ce n’est pas hier, c’est aujourd’hui que vous ferez, c’est aujourd’hui que vous avez fait.\par
Que m’importe si tel ne comprend pas ? Qu’il se réjouisse, celui-là même, en disant J’ignore. Oui, qu’il se réjouisse ; qu’il préfère vous trouver en ne trouvant pas, à ne vous trouver pas en trouvant.
\section[{Chapitre VII, L’enfant est pécheur.}]{Chapitre VII, L’enfant est pécheur.}
\noindent \pn{11}Ayez pitié, mon Dieu ! Malheur aux péchés des hommes ! Et c’est l’homme qui parle ainsi, et vous avez pitié de lui, parce que vous l’avez fait, et non le péché qui est en lui. Qui va me rappeler les péchés de mon enfance ?\par

\begin{quoteblock}
\noindent « Car personne n’est pur de péchés devant vous, pas même l’enfant dont la vie sur la terre est d’un jour (Job XXV, 4). »\end{quoteblock}

\noindent  Qui va me les rappeler, si petit enfant que ce soit, en qui je vois de moi ce dont je n’ai pas souvenance ?\par
Quel était donc mon péché d’alors ? Etait-ce de pleurer avidement après la mamelle ? Or, si je convoitais aujourd’hui avec cette même avidité la nourriture de mon âge, ne serais-je pas ridicule et répréhensible ? Je l’étais donc alors. Mais comme je ne pouvais comprendre la réprimande, ni l’usage, ni la raison ne permettaient de me reprendre. Vice réel toutefois que ces premières inclinations, car en croissant nous les déracinons, et rejetons loin de nous, et je n’ai jamais vu homme de sens, pour retrancher le mauvais, jeter le bon. Etait-il donc bien, vu l’âge si tendre, de demander en pleurant ce qui ne se pouvait impunément donner ; de s’emporter avec violence contre ceux sur qui l’on n’a aucun droit, personnes libres, âgées, père, mère, gens sages, ne se prêtant pas au premier désir ; de les frapper, en tâchant de leur faire tout le mal possible, pour avoir refusé une pernicieuse obéissance ? Ainsi, la faiblesse du corps au premier âge est innocente, l’âme ne l’est pas. Un enfant que j’ai vu et observé était jaloux. Il ne parlait pas encore, et regardait, pâle et farouche, son frère de lait. Chose connue ; les mères et nourrices prétendent conjurer ce mal par je ne sais quels enchantements. Mais est-ce innocence dans ce petit être, abreuvé à cette source de lait abondamment épanché de n’y pas souffrir près de lui un frère indigent dont ce seul aliment soutient la vie ? Et l’on endure ces défauts avec caresse, non pour être indifférents ou légers, mais comme devant passer au cours de l’âge. Vous les tolérez alors, plus tard ils vous révoltent.\par
\pn{12}Seigneur mon Dieu, vous avez donné à l’enfant et la vie, et ce corps muni de ses sens, formé de ses membres, orné de sa figure ; vous avez intéressé tous les ressorts vitaux à sa conservation harmonieuse : et vous m’ordonnez de vous louer dans votre ouvrage, de vous confesser, de glorifier votre nom, ô Très-Haut (Ps XCI, 2), parce que vous êtes le Dieu tout puissant et bon, n’eussiez-vous rien fait que ce que nul ne peut faire que vous seul, principe de toute mesure, forme parfaite qui formez tout, ordre suprême qui ordonnez tout.\par
Or, cet âge, Seigneur, que je ne me souviens pas d’avoir vécu, que je ne connais que sur la foi d’autrui, le témoignage de mes conjectures, l’exemple des autres enfants, témoignage fidèle néanmoins, cet âge, j’ai honte de le rattacher à cette vie à moi, que je vis dans le siècle. Pour moi il est égal enténèbres d’oubli à celui que j’ai passé au sein de ma mère. Que si même e j’ai été conçu en iniquité, si le sein\par

\begin{quoteblock}
\noindent « de ma mère m’a nourri dans le péché (Ps L, 7) »\end{quoteblock}

\noindent  où donc, je vous prie, mon Dieu, où votre esclave, Seigneur, où donc et quand fut-il innocent ? Mais je laisse ce temps : quel rapport de lui à moi, puisque je n’en retrouve aucun vestige ?
 \section[{Chapitre VIII, Comment il apprend a parler.}]{Chapitre VIII, Comment il apprend a parler.}
\noindent \pn{13}Dans la traversée de ma vie jusqu’à ce jour, ne suis-je pas venu de la première enfance à la seconde, ou plutôt celle-ci n’est-elle pas survenue en moi, succédant à la première ? Et l’enfance ne s’est pas retirée ; où serait-elle allée ? Et pourtant elle n’était plus ; car déjà, l’enfant à la mamelle était devenu l’enfant qui essaye la parole. Et je me souviens de cet âge ; et j’ai remarqué depuis comment alors j’appris à parler, non par le secours d’un maître qui m’ait présenté les mots dans certain ordre méthodique comme les lettres bientôt après me furent montrées, mais de moi-même et par la seule force de l’intelligence que vous m’avez donnée, mon Dieu. Car ces cris, ces accents variés, cette agitation de tous les membres, n’étant que des interprètes infidèles ou inintelligibles, qui trompaient mon cœur impatient de faire obéir à ses volontés, j’eus recours à ma mémoire pour m’emparer des mots qui frappaient mon oreille, et quand une parole décidait un geste, un mouvement vers un objet, rien ne m’échappait, et je connaissais que le son précurseur était le nom de la chose qu’on voulait désigner, Ce vouloir m’était révélé par le mouvement du corps, langage naturel et universel que parlent la face, le regard, le geste, le ton de. la voix où se produit le mouvement de l’âme qui veut, possède, rejette ou fuit.\par
Attentif au fréquent retour de ces paroles exprimant des pensées différentes dans une syntaxe invariable, je notais peu à peu leur signification, et dressant ma langue à les articuler, je m’en servis enfin pour énoncer mes volontés. Et je parvins ainsi à pratiquer l’échange des signes expressifs de nos sentiments, et j’entrai plus avant dans l’orageuse société de la vie humaine, sous l’autorité de mes parents et la conduite des hommes plus âgés.
\section[{Chapitre IX, Aversion pour l’étude ; horreur des châtiments.}]{Chapitre IX, Aversion pour l’étude ; horreur des châtiments.}
\noindent \pn{14}O Dieu, mon Dieu, quelles misères, quelles déceptions n’ai-je pas subies, à cet âge, où l’on ne me proposait d’autre règle de bien vivre qu’une docile attention aux conseils de faire fortune dans le siècle, et d’exceller dans cette science verbeuse, servile instrument de l’ambition et de la cupidité des hommes. Puis je fus livré à l’école pour apprendre les lettres ; malheureux, je n’en voyais pas l’utilité, et pourtant ma paresse était châtiée. On le trouvait bon ; nos devanciers dans la vie nous avaient préparé ces sentiers d’angoisses qu’il fallait traverser ; surcroît de labeur et de souffrance pour les enfants d’Adam.\par
Nous trouvâmes alors, Seigneur, des hommes qui vous priaient, et d’eux nous apprîmes à sentir, autant qu’il nous était possible, que vous étiez Quelqu’un de grand, qui pouviez, sans apparaître à nos sens, nous exaucer et nous secourir. Tout enfant, je vous priais, comme mon refuge et mon asile, et, à vous invoquer, je rompais les liens de ma langue, et je vous priais, tout petit, avec grande ferveur, afin de n’être point battu à l’école. Et quand, pour mon bien, vous ne m’écoutiez pas (Ps XXI, 3), tous, jusqu’à mes parents si éloignés de me vouloir la moindre peine, se riaient de mes férules, ma grande et griève peine d’alors.\par
\pn{15}Seigneur, où est le cœur magnanime, s’il en est un seul ? car je ne parle pas de l’insensibilité stupide ; où est le cœur dont l’amour vous enlace d’une assez forte étreinte pour ne plus jeter qu’un oeil indifférent sur ces appareils sinistres, chevalets, ongles de fer, cruels instruments de mort, dont l’effroi élève vers vous des supplications universelles qui les conjurent ? Où est ce cœur ? Et pourrait-il pousser l’héroïsme du dédain, jusqu’à rire de l’épouvante d’autrui, comme mes parents riaient des châtiments que m’infligeait un maître ? Car je ne les redoutais. pas moins, et je ne vous priais pas moins de me les éviter ; et je péchais toutefois, faute d’écrire, de lire, d’apprendre autant qu’on l’exigeait de moi.\par
Je ne manquais pas, Seigneur, de mémoire ou de vivacité d’esprit ; votre bonté m’en avait assez libéralement doté pour cet âge. Seulement j’aimais à jouer, et j’étais puni par qui faisait de même ; mais les jeux des hommes s’appellent affaires, et ils punissent ceux des enfants, et personne n’a pitié ni des enfants, ni des hommes. Un juge équitable pourrait-il cependant approuver qu’un enfant fût châtié pour se laisser détourner, par le jeu de paume, d’une étude qui sera plus tard entre ses mains   un jeu moins innocent ? Et que faisait donc celui qui me battait ? Une misérable dispute, où il était vaincu par un collègue, le pénétrait de plus amers dépits que je n’en éprouvais à perdre une partie de paume contre un camarade.
\section[{Chapitre X, Amour du jeu.}]{Chapitre X, Amour du jeu.}
\noindent \pn{16}Et néanmoins je péchais, Seigneur mon Dieu, ordonnateur et créateur de toutes choses naturelles, sauf les péchés dont vous n’êtes que régulateur ; Seigneur mon Dieu, je péchais en désobéissant à des parents, à des maîtres ; car je pouvais bien user dans la suite de ces connaissances qu’on m’imposait n’importe à quelle intention. Ce n’était pas meilleur choix qui me rendait désobéissant, c’était l’amour du jeu ; j’aimais toutes les vanités du combat et de la victoire ; et les récits fabuleux qui, chatouillant mon oreille, y provoquaient de plus vives démangeaisons ; et ma curiosité soulevée chaque jour, et débordant de mes yeux, m’entraînait aux spectacles et aux jeux qui divertissent les hommes. Que désirent donc toutefois ces magistrats pour leurs enfants, sinon la survivance des dignités qui les appellent à présider les jeux ? Et ils veulent qu’on les châtie, si ce plaisir les détourne d’études, qui, de leur aveu, doivent conduire leurs fils à ce frivole honneur. Regardez tout cela, Seigneur, avec miséricorde ; délivrez-nous, nous qui vous invoquons ; délivrez aussi ceux qui ne vous invoquent pas encore, pour qu’ils vous invoquent et soient délivrés.
\section[{Chapitre XI, Malade, il demande le baptême.}]{Chapitre XI, Malade, il demande le baptême.}
\noindent \pn{17}J’avais ouï parler, dès le berceau, de la vie éternelle qui nous est promise par l’humilité du Seigneur notre Dieu, abaissé jusqu’à notre orgueil ; et j’étais marqué du signe de sa croix, assaisonné du sel divin, dès ma sortie du sein de ma mère, qui a beaucoup espéré en vous. Vous savez, Seigneur, qu’étant encore enfant, surpris un jour d’une violente oppression d’estomac, j’allais mourir ; vous savez, mon Dieu, vous qui étiez déjà mon gardien, de quel élan de cœur, de quelle foi je demandai le baptême de votre Christ, mon Dieu et Seigneur, à la piété de ma mère et de notre mère commune, votre Église. Et déjà, dans son trouble, celle dont le chaste cœur concevait avec plus d’amour encore l’enfantement de mon salut éternel en votre foi, la mère de ma chair, appelait à la hâte mon initiation aux sacrements salutaires, où j’allais être lavé, en vous confessant, Seigneur Jésus, pour la rémission des péchés, quand soudain je me sentis soulagé. Ainsi fut différée ma purification, comme si je dusse nécessairement me souiller de nouveau en recouvrant la vie ; on craignait de moi une rechute dans la fange de mes péchés, plus grave et plus dangereuse au sortir du bain céleste.\par
Ainsi, déjà, je croyais, et ma mère croyait, et toute la maison, mon père excepté, qui pourtant ne put jamais abolir en moi les droits de la piété maternelle, ni me détourner de croire en Jésus-Christ, lui qui n’y croyait pas encore. Elle n’oubliait rien pour que vous me fussiez un père, mon Dieu, plutôt que lui, et ici vous l’aidiez à l’emporter sur son mari, à qui, toute supérieure qu’elle fût, elle obéissait, parce qu’en cela elle obéissait à vos ordres.\par
\pn{18}Pardon, mon Dieu, je voudrais savoir, si vous le voulez, par quel conseil mon baptême a été différé. Est-ce pour mon bien que les rênes furent ainsi lâchées à mes instincts pervers ? Ou me trompé-je ? Mais d’où vient que sans cesse ce mot nous frappe l’oreille : Laissez-le, laissez-le faire ; il n’est pas encore baptisé ? Et pourtant, s’agit-il de la santé du corps, on ne dit pas : Laissez-le se blesser davantage, car il n’est pas encore guéri.\par
Oh ! que n’ai-je obtenu cette guérison prompte ! Que n’ai-je, avec le concours des miens, placé la santé de mon âme sous la tutelle de votre grâce qui me l’eût rendue ! Mieux eût valu. Mais quels flots, quels orages de tentations se levaient sur ma jeunesse ! Ma mère les voyait ; et elle aimait mieux livrer le limon informe à leurs épreuves que l’image divine à leurs profanations.
\section[{Chapitre XII, Dieu tournait à son profit l’imprévoyance même qui dirigeait ses études.}]{Chapitre XII, Dieu tournait à son profit l’imprévoyance même qui dirigeait ses études.}
\noindent \pn{19}Ainsi, à cet âge même, que l’on redoutait moins pour moi que l’adolescence, je n’aimais point l’étude ; je haïssais d’y être contraint, et   l’on m’y contraignait, et il m’en advenait bien : – je n’eusse rien appris sans contrainte – mais moi je faisais mal ; car faire à contrecœur quelque chose de bon n’est pas bien faire. Et ceux même qui me forçaient à l’étude ne faisaient pas bien ; mais bien m’en advenait par vous, mon Dieu. Eux ne voyaient pour moi, dans ce qu’ils me pressaient d’apprendre, qu’un moyen d’assouvir l’insatiable convoitise de cette opulence qui n’est que misère, de cette gloire qui n’est qu’infamie.\par

\begin{quoteblock}
\noindent \emph{Mais} « Vous qui savez le compte des cheveux de notre tête (Matth. X, 30) » ;\end{quoteblock}

\noindent vous tourniez leur erreur à mon profit, et ma paresse, au châtiment que je méritais, si petit enfant, si grand pécheur. Ainsi, du mal qu’ils faisaient, vous tiriez mon bien, et de mes péchés, ma juste rétribution. Car vous avez ordonné, et il est ainsi, que tout esprit qui n’est pas dans l’ordre soit sa peine à lui-même.
\section[{Chapitre XIII, Vanité des fictions poétiques qu’il aimait.}]{Chapitre XIII, Vanité des fictions poétiques qu’il aimait.}
\noindent \pn{20}Mais d’où venait mon aversion pour la langue grecque, exercice de mes premières années ? C’est ce que je ne puis encore pénétrer. J’étais passionné pour la latine, telle que l’enseignent, non les premiers maîtres, mais ceux que l’on appelle grammairiens ; car ces éléments, où l’on apprend à lire, écrire, compter, ne me donnaient pas moins d’ennuis et de tourments que toutes mes études grecques. Et d’où venait ce dégoût, sinon du péché et de la vanité de la vie ? J’étais chair, esprit absent de lui-même et ne sachant plus y rentrer (Ps. LXXVII, 39). Plus certaines et meilleures étaient ces premières leçons qui m’ont donné la faculté de lire ce qui me tombe sous les yeux, d’écrire ce qu’il me plaît, que celles où j’apprenais de force les courses errantes de je ne sais quel Enée, oublieux de mes propres erreurs, et gémissant sur la mort de Didon, qui se tue par amour, quand je n’avais pas une larme pour déplorer, ô mon Dieu, ô ma vie, cette mort de mon âme que ces jeux j emportaient loin de vous.\par
\pn{21}Eh ! quoi de plus misérable qu’un malheureux sans miséricorde pour lui-même, pleurant Didon, morte pour aimer Enée, et ne se pleurant pas, lui qui meurt faute de vous aimer ! O Dieu, lumière de mon cœur, pain de la bouche intérieure de mon âme, vertu fécondante de mon intelligence, époux de ma pensée, je ne vous aimais pas ; je vous étais infidèle, et mon infidélité entendait de toutes parts cette voix : « Courage ! courage ! » car l’amour de ce monde est un divorce adultère d’avec vous. Courage ! courage ! dit cette voix, pour faire rougir, si l’on n’est pas homme comme un autre. Et ce n’est pas ma misère que je pleurais ; je pleurais Didon\par

\begin{quoteblock}
\noindent « expirée, livrant au fil du glaive sa destinée dernière Enéide (VI, 456) »\end{quoteblock}

\noindent , quand je me livrais moi-même à vos dernières créatures au lieu de vous, terre retournant à la terre. Cette lecture m’était-elle interdite, je souffrais de ne pas lire ce qui me faisait souffrir. Telles folies passent pour études plus nobles et plus fécondes que celle qui m’apprit à lire et à écrire.\par
\pn{22}Mais qu’aujourd’hui, mon Dieu, votre vérité me dise et crie dans mon âme : Il n’en est pas ainsi ! il n’en est pas ainsi ! Ces premiers enseignements sont bien les meilleurs. Car me voici tout prêt à oublier les aventures d’Enée et fables pareilles, plutôt que l’art d’écrire et de lire. Des voiles, sans doute, pendent au seuil des écoles de grammaire ; mais ils couvrent moins la profondeur d’un mystère que la vanité d’une erreur.\par
Qu’ils se récrient donc contre moi, ces maîtres insensés ! je ne les crains plus, à cette heure où je vous confesse, ô mon Dieu, tous les pensers de mon âme et me plais à marquer l’égarement de mes voies, afin d’aimer la rectitude des vôtres. Qu’ils se récrient contre moi, vendeurs ou acheteurs de grammaire ! Je leur demande s’il est vrai qu’Enée soit autrefois venu à Carthage, comme le poète l’atteste ; et les moins instruits l’ignorent, les plus savants le nient. Mais si je demande par quelles lettres s’écrit le nom d’Enée, tous ceux qui savent lire me répondront vrai, selon la convention et l’usage qui ont, parmi les hommes, déterminé ces signes. Et si je demande encore quel oubli serait le plus funeste à la vie humaine, l’oubli de l’art de lire et d’écrire, ou celui de ces fictions poétiques, qui ne prévoit la réponse de quiconque ne s’est pas oublié lui-même ?\par
Je péchais donc enfant, en préférant ainsi la vanité à l’utile ; ou plutôt je haïssais l’utile et j’aimais la vanité. « Un et un sont deux, deux et deux quatre, » était pour moi une odieuse chanson ; et je ne savais pas de plus   beau spectacle qu’un fantôme de cheval de bois rempli d’hommes armés, que l’incendie de Troie et l’ombre de Créuse (Enéide, II).
\section[{Chapitre XIV, Son aversion pour la langue grecque.}]{Chapitre XIV, Son aversion pour la langue grecque.}
\noindent \pn{23}Pourquoi donc haïssais-je ainsi la langue grecque, pleine de ces fables ? Car Homère excelle à ourdir telles fictions. Doux menteur, il était toutefois amer à mon enfance. Je crois bien qu’il en est ainsi de Virgile pour les jeunes Grecs, contraints de l’apprendre avec autant de difficulté que j ‘apprenais leur poète.\par
La difficulté d’apprendre cette langue étrangère assaisonnait de fiel la douce saveur des fables grecques. Pas un mot qui me fût connu ; et puis, des menaces terribles de châtiments pour me forcer d’apprendre. J’ignorais de même le latin au berceau ; et cependant, par simple attention, sans crainte, ni tourment, je l’avais appris, dans les embrassements de mes nourrices, les joyeuses agaceries, les riantes caresses. Ainsi je l’appris sans être pressé du poids menaçant de la peine, sollicité seulement par mon âme en travail de ses conceptions, et qui ne pouvait rien enfanter qu’à l’aide des paroles retenues, sans leçons, à les entendre de la bouche des autres, dont l’oreille recevait les premières confidences de mes impressions. Preuve qu’en cette étude une nécessité craintive est un précepteur moins puissant qu’une libre curiosité. Mais l’une contient les flottants caprices de l’autre,, grâce à vos lois, mon Dieu, vos lois qui depuis la férule de l’école jusqu’à l’épreuve du martyre, nous abreuvant d’amertumes salutaires, savent nous rappeler à vous, loin du charme empoisonneur qui nous avait retirés de vous.
\section[{Chapitre XV, Prière.}]{Chapitre XV, Prière.}
\noindent \pn{24}Exaucez, Seigneur, ma prière ; que mon âme ne défaille pas sous votre discipline ; et que je ne défaille pas à vous confesser vos miséricordes qui m’ont retiré de toutes mes déplorables voies ! Soyez-moi plus doux que les séductions qui m’égaraient ! Que je vous aime fortement, et que j’embrasse votre main de toute mon âme, pour que vous me sauviez de toute tentation jusqu’à la fin.\par
Et n’êtes-vous pas, Seigneur, mon roi et mon Dieu ? Que tout ce que mon enfance apprit d’utile, vous serve ; si je parle, si j’écris, si je lis, si je compte, que tout en moi vous serve ; car, au temps où j’apprenais des choses vaines, vous me donniez la discipline, et vous m’avez enfin remis les péchés de ma complaisance dans les vanités. Ce n’est point que ces folies ne m’aient laissé le souvenir de plusieurs mots utiles ; souvenir que l’on pourrait devoir à des lectures moins frivoles, et qui ne sèmeraient aucun piège sous les pas des enfants.
\section[{Chapitre XVI, Contre les fables impudiques.}]{Chapitre XVI, Contre les fables impudiques.}
\noindent \pn{25}Mais, malheur à toi, torrent de la coutume ! Qui te résistera ? Ne seras-tu jamais à sec ? Jusques à quand rouleras-tu les fils d’Eve dans cette profonde et terrible mer, que traversent à grand’peine les passagers de la croix ? Ne m’as-tu pas montré Jupiter tout à la fois tonnant et adultère ? Il ne pouvait être l’un et l’autre ; mais on voulait autoriser l’imitation d’un véritable adultère par la fiction d’un ton. nerre menteur. Est-il un seul de ces maîtres fièrement drapés dont l’oreille soit assez à jeun pour entendre ce cri de vérité qui part d’un homme sorti de la poussière de leurs écoles :\par

\begin{quoteblock}
\noindent « Inventions d’Homère ! Il humanise « les dieux ! Il eût mieux fait de diviniser les « hommes (Cicér. Tuscul. 1) ! »\end{quoteblock}

\noindent  Mais la vérité, c’est que le poète, dans ses fictions, assimilait aux dieux les hommes criminels, afin que le crime cessât de passer pour crime, et qu’en le commettant, on parût imiter non plus les hommes de perdition, mais les dieux du ciel.\par
\pn{26}Et néanmoins, ô torrent d’enfer ! en toi se plongent les enfants des hommes ; ils rétribuent de telles leçons ; ils les honorent de la publicité du forum ; elles sont professées à la face des lois qui, aux récompenses privées, ajoutent le salaire public ; et tu roules tes cailloux avec fracas, en criant : Ici l’on apprend la langue ; ici l’on acquiert l’éloquence nécessaire à développer et à persuader sa pensée. N’aurions-nous donc jamais su « pluie d’or, « sein de femme, déception, voûtes célestes » et semblables mots du même passage, si Térence n’eût amené sur la scène un jeune débauché se proposant Jupiter pour modèle d’impudicité,   charmé de voir en peinture, sur une muraille, « comment le Dieu verse une pluie d’or dans le sein de Danaé et trompe cette femme.» Voyez donc comme il s’anime à la débauche sur ce divin exemple. « Eh ! quel Dieu encore ! s’écrie-t-il ; Celui qui fait trembler de son tonnerre la voûte profonde des cieux. Pygmée que je suis, j’aurais honte de l’imiter ! Non, non ! je l’ai imité et de grand cœur (Térenc. Eunuc. Act. 3, scèn.5). »\par
Ces impuretés ne nous aident en rien à retenir telles paroles, mais ces paroles enhardissent l’impureté. Je n’accuse pas les paroles, vases précieux et choisis, mais le vin de l’erreur que nous y versaient des maîtres ivres. Si nous ne buvions, on nous frappait, et il ne nous était pas permis d’en appeler à un juge sobre. Et cependant, mon Dieu, devant qui mon âme évoque désormais ces souvenirs sans alarme, j’apprenais cela volontiers, je m’y plaisais, malheureux ! aussi étais-je appelé un enfant de grande espérance !
\section[{Chapitre XVII, Vanité de ses études.}]{Chapitre XVII, Vanité de ses études.}
\noindent \pn{27}Permettez-moi, mon Dieu, de parler encore de mon intelligence, votre don ; en quels délires elle s’abrutissait ! Grande affaire, et qui me troublait l’âme par l’appât de la louange, par la crainte de la honte et des châtiments, quand il s’agissait d’exprimer les plaintes amères de Junon, « impuissante à détourner de «l’Italie le chef des Troyens ! (Enéide, I, 36-75) » plaintes que je savais imaginaires ; mais on nous forçait de nous égarer sur les traces de ces mensonges poétiques, et de dire en libre langage ce que le poète dit en vers. Et celui-là méritait le plus d’éloges qui, fidèle à la dignité du personnage mis en scène, produisait un sentiment plus naïf de colère et de douleur, ajustant à ses pensées un vêtement convenable d’expression.\par
Eh ! à quoi bon, ô ma vraie vie, ô mon Dieu ! à quoi bon cet avantage sur la plupart de mes condisciples et rivaux, de voir mes compositions plus applaudies ? Vent et fumée que tout cela ! N’était-il pas d’autre sujet pour exercer mon intelligence et ma langue ? Vos louanges, Seigneur, vos louanges dictées par vos Écritures mêmes, eussent soutenu le pampre pliant de mon cœur. Il n’eût pas été emporté dans le vague des bagatelles, triste proie des oiseaux sinistres ; car il est plus d’une manière de sacrifier aux anges prévaricateurs.
\section[{Chapitre XVIII, Hommes plus fidèles aux lois de la grammaire qu’aux commandements de Dieu.}]{Chapitre XVIII, Hommes plus fidèles aux lois de la grammaire qu’aux commandements de Dieu.}
\noindent \pn{28}Eh ! quelle merveille que je me dissipasse ainsi dans les vanités, et que, loin de vous, mon Dieu, je me répandisse au dehors, quand on me proposait pour modèles des hommes qui rappelant d’eux-mêmes quelque bonne action, rougissaient d’être repris d’un barbarisme ou d’un solécisme échappé ; et qui, déployant, au récit de leurs débauches, toutes les richesses d’une élocution nombreuse, exacte et choisie, se glorifiaient des applaudissements ?\par
Vous voyez cela, Seigneur, et vous vous taisez,\par

\begin{quoteblock}
\noindent « patient, miséricordieux et vrai (Ps. LXXXV, 15). »\end{quoteblock}

\noindent  Vous tairez-vous donc toujours ? Mais à cette heure même vous retirez de ce dévorant abîme l’âme qui vous cherche, altérée de vos délices ; celui dont le cœur vous dit :\par

\begin{quoteblock}
\noindent « J’ai cherché votre visage ; votre visage, Seigneur, je le chercherai toujours (Ps XXVI, 8). »\end{quoteblock}

\noindent  On en est loin dans les ténèbres des passions. Ce n’est point le pied, ce n’est point l’espace qui nous éloigne de vous, qui nous ramène à vous. Et le plus jeune de vos fils a-t-il donc pris un cheval, un char, un vaisseau, s’est-il envolé sur des ailes visibles, s’est-il dérobé d’un pas agile, pour livrer en pays lointain aux prodigalités de sa vie ce qu’il avait reçu de vous au départ ? Père tendre, qui lui aviez tout donné alors, plus tendre encore à la détresse de son retour (Luc XV, 12-32). Mais non, c’est l’entraînement de la passion qui nous jette dans les ténèbres, et loin de votre face.\par
\pn{29}Voyez, Seigneur mon Dieu, dans votre inaltérable patience, voyez avec quelle fidélité les enfants des hommes observent le pacte grammatical qu’ils ont reçu de leurs devanciers dans le langage, avec quelle négligence ils se dérobent au pacte éternel de leur salut qu’ils ont reçu de vous. Et si un homme qui possède ou enseigne cette antique législation des sons, oublie, contrairement aux règles, l’aspiration de la première syllabe, en disant « omme, » il blesse plus les autres que si, au mépris de vos commandements, il haïssait l’homme, son frère ; comme si l’ennemi le plus funeste était plus funeste à l’homme que la haine même qui le soulève ; comme si le persécuteur ravageait autrui plus qu’il ne ravage son propre cœur ouvert à la haine.\par
Et certes, cette science des lettres n’est pas   plus intérieure que la conscience écrite de ne pas faire au prochain ce qu’on n’en voudrait pas souffrir. Oh ! que vous êtes secret, habitant des hauteurs dans le silence ! ô Dieu, seul grand, dont l’infatigable loi sème les cécités vengeresses sur les passions illégitimes ! Cet homme aspire à la renommée de l’éloquence ; il est debout devant un homme qui juge, en présence d’une foule d’hommes ; il s’acharne sur son ennemi avec la plus cruelle animosité, merveilleusement attentif à éviter toute erreur de langage, à ne pas dire : « Entre aux hommes » ; et il ne se tient pas en garde contre la fureur de son âme qui l’entraîne à supprimer un homme « d’entre les hommes. »
\section[{Chapitre XIX, Fautes des enfants, vices des hommes.}]{Chapitre XIX, Fautes des enfants, vices des hommes.}
\noindent \pn{30}J’étais exposé, malheureux enfant, sur le seuil de cette morale ; c’était l’apprentissage des tristes combats que je devais combattre ; jaloux, déjà, d’éviter un barbarisme, et non l’envie qu’une telle faute m’inspirait contre qui n’en faisait pas. Je reconnais et confesse devant vous, mon Dieu, ces faiblesses qui me faisaient louer de ces hommes. Leur plaire était alors pour moi le bien-vivre ; car je ne voyais pas ce gouffre de honte où je plongeais loin de votre regard. Etait-il donc rien de plus impur que moi ? Jusque-là, qu’abusant par mille mensonges, un précepteur, des maîtres, des parents, épris eux-mêmes de ces vanités, je les offensais par mon amour du jeu, ma passion des spectacles frivoles, mon ardeur inquiète à imiter ces bagatelles.\par
Je dérobais aussi au cellier, à la table de mes parents, soit pour obéir à l’impérieuse gourmandise, soit pour avoir à donner aux enfants qui me vendaient le plaisir que nous trouvions à jouer ensemble. Et au jeu même, vaincu par le désir d’une vaine supériorité, j’usurpais souvent de déloyales victoires. Mais quelle était mon impatience et la violence de mes reproches, si je découvrais qu’on me trompât, comme je trompais les autres ! Pris sur le fait à mon tour, et accusé, loin de céder, j ‘entrais en fureur.\par
Est-ce donc là l’innocence du premier âge ? Il n’en est pas, Seigneur, il n’en est pas ; pardonnez-moi, mon Dieu. Aujourd’hui précepteur, maître, noix, balle, oiseau ; demain magistrats, rois, trésors, domaines, esclaves ; c’est tout un, grossissant au flot successif des années, comme aux férules succèdent les supplices. C’est donc l’image de l’humilité, que vous avez aimée dans la faiblesse corporelle de l’enfance, ô notre roi, lorsque vous avez dit :\par

\begin{quoteblock}
\noindent « Le royaume des cieux est à ceux qui leur ressemblent (Matth. XIX, 14), »\end{quoteblock}

\section[{Chapitre XX, Il rend grâce à Dieu des dons qu’il a reçus de lui dans son enfance.}]{Chapitre XX, Il rend grâce à Dieu des dons qu’il a reçus de lui dans son enfance.}
\noindent \pn{31}Et cependant, Seigneur, à vous créateur et conservateur de l’univers, tout-puissant et tout bon, à vous notre Dieu, grâces soient rendues, ne m’eussiez-vous donné que d’être enfant ! Car dès lors même, j’avais l’être, et havie, et le sentiment ; et je veillais à préserver cet ensemble de tout moi-même, ce dessin de l’unité si cachée par qui j’étais ; je gardais par le sens intérieur l’intégrité de tous mes sens, et dans cette petitesse d’existence, dans cette petitesse de pensées, j’aimais la vérité. Je ne voulais pas être trompé ; ma mémoire était forte ; mon élocution polie ; l’amitié me charmait ; je fuyais la douleur, la honte, l’ignorance. Quelle admirable merveille qu’un tel animal !\par
Tout cela, don de mon Dieu ! je ne me suis moi-même rien donné. Tout cela est bon et moi-même, qui suis tout cela. Donc celui qui m’a fait est bon, et lui-même est mon bien ; et l’élan de mon cœur lui rend hommage de tous ces biens répandus sur mes premières années. Or je péchais ; car ce n’était point en lui, mais dans ses créatures, les autres et moi, que je cherchais plaisirs, grandeurs et vérités, me précipitant ainsi dans la douleur, la confusion, l’erreur. Grâces à vous, mes délices, ma gloire, ma confiance, mon Dieu ! Grâces à vous de tous vos dons ! Mais conservez-les-moi ; car ainsi vous me conserverez moi-même ; et tout ce que vous m’avez donné aura croissance et perfection ; et je serai avec vous, puisque c’est vous qui m’avez donné d’être.
\chapterclose


\chapteropen
 \chapter[{II. La seizième année}]{II. La seizième année}\phantomsection
\label{II}\renewcommand{\leftmark}{II. La seizième année}


\begin{argument}\noindent Désordres de sa première jeunesse. — Ses débauches à l’âge de seize ans. — Larcin dont il s’accuse sévèrement.
\end{argument}


\chaptercont
\section[{Chapitre premier, Désordres de sa jeunesse.}]{Chapitre premier, Désordres de sa jeunesse.}
\noindent \pn{1}Je veux rappeler mes impuretés passées, et les charnelles corruptions de mon âme, non que je les aime, mais afin de vous aimer, mon Dieu. C’est par amour de votre amour que je reviens sur mes voies infâmes dans l’amertume de mon souvenir, pour savourer votre douceur, ô Délices véritables, Béatitude et Sécurité de délices, qui recueillez en vous toutes les puissances de mon être dispersées en mille vanités loin de vous, mon centre unique\par
Car je brûlais, dès mon adolescence, de me rassasier de basses voluptés ; et je n’eus pas honte de prodiguer la sève de ma vie à d’innombrables et ténébreuses amours, et ma beauté s’est flétrie, et je n’étais plus que pourriture à vos yeux, alors que je me plaisais à moi-même et désirais plaire aux yeux des hommes.
\section[{Chapitre II, Ses débauches à seize ans.}]{Chapitre II, Ses débauches à seize ans.}
\noindent \pn{2}Ma plus vive jouissance n’était-elle pas d’aimer et d’être aimé ? Mais je ne m’en tenais pas à ces liens d’âme à âme, sur la chaste lisière de l’amitié spirituelle. D’impures vapeurs s’exhalaient des fangeuses convoitises de ma chair, de l’effervescence de la puberté ; elles couvraient et offusquaient mon cœur : la sérénité de l’amour était confondue avec les nuages de la débauche. L’une et l’autre fermentaient ensemble, et mon imbécile jeunesse était entraînée dans les précipices des passions et plongeait dans le gouffre du libertinage.\par
Votre colère s’était amassée contre moi, et je l’ignorais. Au bruit des chaînes de ma mortalité, j’étais devenu sourd, j’expiais la superbe de mon âme. Et je m’éloignais de vous, et vous me laissiez ; et je m’élançais, et je débordais, et je me répandais, et je me fondais en adultères, et vous vous taisiez ! O ma tardive joie, vous vous taisiez alors, et, toujours plus loin de vous, je m’avançais dans les aridités fécondes en douleurs, avili dans l’orgueil, agité dans la fatigue !\par
\pn{3}Qui eût alors modéré ma peine ? Qui m’eût borné à l’usage légitime de la fugitive beauté des créatures éphémères et de leurs délices, pour que les flots de ma jeunesse ne débordassent pas du moins la plage conjugale, s’ils ne pouvaient s’apaiser dans le but de la procréation des enfants, selon la prescription de votre loi, Seigneur, qui réglez la génération de notre mortalité, et pouvez étendre une main adoucie pour émousser des épines inconnues au paradis ? car votre toute-puissance est tout près de nous, lors même que nous sommes loin de vous. Que n’ai-je du moins écouté plus attentivement la voix de vos nuées :\par

\begin{quoteblock}
\noindent « Ils souffriront des tribulations dans leur chair. Et moi je vous les épargne. Il est bon à l’homme de ne point toucher de femme. Celui qui est sans femme pense aux choses de Dieu, à plaire à Dieu. Celui qui est lié par le mariage pense aux choses du monde, à plaire à sa femme (I Cor. VII, 28, I, 32,33,34). »\end{quoteblock}

\noindent  Que n’ai-je ouvert l’oreille à cette voix ! eunuque de volonté en vue du royaume des cieux (Matth. XIX, 12), dans l’attente plus heureuse de vos embrassements ?\par
\pn{4}Mais je brûlais, malheureux, et livré au torrent qui m’entraînait loin de vous, je m’affranchis de tous vos commandements, sans échapper à votre verge. Qui le pourrait ? Vous   étiez toujours présent dans la miséricorde de vos rigueurs, abreuvant des plus amers dégoûts toutes mes joies illégitimes, pour m’entraîner à chercher les joies exemptes de dégoûts. Et où les eussé-je trouvées hors de vous,\par

\begin{quoteblock}
\noindent « qui faites entrer la douleur dans le précepte (Ps. XCIII, 20) ; qui frappez pour guérir ; qui tuez pour nous empêcher de mourir à vous (Deut. XXXII, 39) ? »\end{quoteblock}

\noindent Où étais-je, et dans quel lointain exil des délices de votre maison, à cette seizième année de l’âge de ma chair, qui prit alors le sceptre sur moi ; esclave volontaire, livré sans réserve à la frénésie de cette passion, que notre dégradation affranchit de tout frein, mais que votre loi condamne ? On ne se mit point en peine d’offrir le mariage au-devant de ma chute ; on n’avait à cœur que de me faire apprendre à bien dire, à persuader par ma parole.
\section[{Chapitre III, Vices de son éducation.}]{Chapitre III, Vices de son éducation.}
\noindent \pn{5}Et, cette même année, ramené de Madaure, ville voisine de notre séjour et mon premier pèlerinage littéraire et oratoire, j’avais interrompu mes études. On préparait la dépense d’un plus lointain exil à Carthage, mon père, humble citoyen du municipe de Thagaste, consultant moins sa fortune que son ambition. Eh ! pour qui ce récit ? Pas pour vous, mon Dieu ; mais en m’adressant à vous, je parle à tous les hommes mes frères, si peu qu’ils soient ceux à qui ces pages tomberont entre les mains. Et pourquoi ? Pour que tout lecteur considère avec moi de quel profond abîme il nous faut crier vers vous. Et néanmoins se confesser de cœur, vivre de foi, quoi de plus près de votre oreille ? Quelles louanges alors ne prodiguait-on pas à mon père pour fournir, au delà de ses ressources, au studieux et lointain voyage de son fils ? Combien de citoyens beaucoup plus opulents que lui étaient loin d’avoir tel souci de leurs enfants ? Et ce même père ne s’inquiétait pas si je croissais pour vous, si j’étais chaste, pourvu que je fusse disert, ou plutôt désert sans votre culture, ô Dieu, bon, vrai, seul maître du champ de mon cœur ?\par
\pn{6}Or, à cet âge de seize ans, des affaires domestiques ayant mis entre mes études un intervalle de vacances oisives, je vécus chez mes pare et mère, et c’est alors que les ronces des désirs impurs s’élevèrent au-dessus de ma tête, et nulle main n’était là pour les arracher. Loin de là ; mon père s’aperçoit un jour, au bain, de ma pubescence qui, déjà, me couvrait d’un manteau de frémissantes inquiétudes, et, tressaillant comme à l’aspect de ses petits-fils, dans sa- joie, il en fait part à ma mère. Joie de l’ivresse où ce monde vous oublie, vous, son Créateur, pour aimer vos créatures au lieu de vous, enivré qu’il est du vin invisible d’une volonté pervertie et livrée aux vils penchants. Mais déjà dans le cœur de ma mère vous aviez commencé votre temple et jeté les assises de votre sainte habitation. Mon père n’était encore, lui, que simple catéchumène, et tout récemment. Elle frémit donc de pieuse épouvante, et trembla ; quoique je ne fusse pas encore fidèle, elle craignit pour moi ces voies tortueuses où s’engagent ceux qui vous présentent le dos et non la face.\par
\pn{7}Hélas ! osé-je encore dire que vous gardiez le silence, ô mon Dieu, quand je m’éloignais de vous ? Etait-ce ainsi que vous vous taisiez pour moi ? Et de qui étaient donc ces suaves paroles, que, par la bouche de ma mère, votre servante fidèle, vous me disiez à l’oreille ? Et rien n’en descendait dans mon cœur pour l’incliner à l’obéissance. Elle me recommandait instamment, et m’avertit un jour en secret, avec quelle sollicitude ! je m’en souviens, de me dérober à tout amour impudique et surtout adultère. Je prenais cela pour des avis de femme, que j’eusse rougi d’écouter. Et c’étaient les vôtres, et je l’ignorais ; et je pensais que vous vous taisiez, et que seule elle parlait, elle par qui vous me parliez ; et c’est vous que je méprisais en elle, moi son fils, fils de votre servante, et votre serviteur. Mais je ne savais pas, et je me précipitais avec tant d’aveuglement, qu’entre ceux de mon âge j’étais honteux de mon infériorité de honte ; car je les entendais se vanter de leurs excès, et se glorifier d’autant plus qu’ils étaient plus infâmes ; et j’avais à cœur de pécher ; soif de plaisir et soif de gloire. Qu’y a-t-il de blâmable que le vice ? Moi, crainte du blâme, je devenais plus vicieux. Et à défaut de crime réel pour m’égaler aux plus corrompus, je feignais ce que je n’avais point fait ; j’avais peur de paraître d’autant plus méprisable que j’étais plus innocent, d’autant plus vil que j’étais plus chaste.\par
\pn{8}Voilà avec quels compagnons je courais les places de Babylone, et me roulais dans sa fange   comme dans des eaux de senteur et de parfums de cinnamome. Et pour m’attacher plus victorieusement au principe du péché, l’ennemi invisible me foulait aux pieds, et me séduisait, si facile que j’étais à séduire ! Sortie du cœur de la cité abominable, mais culminant, lente encore, dans les voies du retour, la mère de ma chair m’avertit bien de garder la pudeur, et pourtant cette confidence de son mari n’éveilla pas en elle la pensée de resserrer dans les limites de l’amour conjugal, sinon de couper au vif ces instincts passionnés dont les germes, déjà si funestes, offraient à ses alarmes le présage des plus grands dangers. Elle négligea le remède, dans la crainte que toute mon espérance ne fût entravée par la chaîne du mariage ; non pas cette espérance de la vie future qu’elle plaçait en vous, ma pieuse mère, mais l’espérance d’un avenir littéraire dont ils étaient l’un et l’autre trop jaloux pour moi ; lui, parce qu’il ne songeait guère à vous, et rêvait des vanités pour moi ; elle, parce que loin de croire que ces études me fussent nuisibles, elle les regardait comme des échelons qui devaient m’élever jusqu’à votre possession.\par
Telles sont les conjectures que hasardent mes souvenirs sur les dispositions de mes parents. Et puis au lieu d’user d’une sage sévérité, on lâchait la bride en mes divertissements à la multitude de mes passions déréglées, et un épais brouillard interceptait sans cesse à ma vue, ô mon Dieu, la lumière de votre vérité !\par

\begin{quoteblock}
\noindent « Et mon iniquité naissait comme de mon embonpoint (Ps. LXXII, 7). »\end{quoteblock}

\section[{Chapitre IV, Larcin}]{Chapitre IV, Larcin}
\noindent \pn{9}Le larcin est condamné par votre loi divine, Seigneur, et par cette loi écrite au cœur des hommes, que leur iniquité même n’efface pas. Quel voleur souffre volontiers d’être volé ? Quel riche pardonne à l’indigent poussé par la détresse ? Eh bien ! moi, j’ai voulu voler, et j’ai volé sans nécessité, sans besoin, par dégoût de la justice, par plénitude d’iniquité ; car j’ai dérobé ce que j’avais meilleur, et en abondance. Et ce n’est pas de l’objet convoité par mon larcin, mais du larcin même et du péché que je voulais jouir. Dans le voisinage de nos vignes était un poirier chargé de fruits qui n’avaient aucun attrait de saveur ou de beauté. Nous allâmes, une troupe de jeunes vauriens, secouer et dépouiller cet arbre, vers le milieu de la nuit, ayant prolongé nos jeux jusqu’à cette heure, selon notre détestable habitude, et nous en rapportâmes de grandes charges, non pour en faire régal, si toutefois nous y goûtâmes, mais ne fût-ce que pour les jeter aux pourceaux : simple plaisir de faire ce qui était défendu. Voici ce cœur, ô Dieu ! ce cœur que vous avez vu en pitié au fond de l’abîme. Le voici, ce cœur ; qu’il vous dise ce qu’il allait chercher là, pour être gratuitement mauvais, sans autre sujet de malice que la malice même. Hideuse qu’elle était, je l’ai aimée ; j’ai aimé à périr ; j’ai aimé ma difformité ; non l’objet qui me rendait difforme , mais ma difformité même, je l’ai aimée ! Âme souillée, détachée de votre appui pour sa ruine, n’ayant dans la honte d’autre appétit que la honte ! Mais quelle honte !
\section[{Chapitre V, On ne fait point le mal sans intérêt.}]{Chapitre V, On ne fait point le mal sans intérêt.}
\noindent \pn{10}La beauté des corps, tels que l’or, l’argent… , a son attrait. L’attouchement est flatté par une convenance de rapport, et à chaque sens correspond une certaine modification des objets. L’honneur temporel, la puissance de commander et de vaincre ont leur beauté, d’où naît aussi la soif de la vengeance. Et, pour atteindre à ces jouissances, nous ne devons pas sortir de vous, Seigneur, ni dévier de votre loi. Cette vie même que nous vivons ici-bas a pour nous charmer sa mesure de beauté et sa juste proportion avec toutes les beautés inférieures. Le nœud si cher de l’amitié humaine trouve sa douceur dans l’unité de plusieurs âmes.\par
Cause de péché que tout cela, quand le déréglement de nos affections abandonne, pour ces biens infimes, les plus excellents, les plus sublimes, vous, Seigneur notre Dieu, et votre vérité et votre loi. Ces biens d’ici-bas ont leur charme, mais qu’est-il auprès de mon Dieu, créateur de l’univers, unique joie du juste, délices des cœurs droits ?\par
\pn{11}Recherche-t-on la cause d’un crime, on n’y croit d’ordinaire, que s’il apparaît un désir d’obtenir, une crainte de perdre quelqu’un de ces biens infimes dont nous parlons, car ils ont leur grâce et leur beauté ; niais qu’ils sont bas et rampants, si l’on songe aux trésors de la gloire et de la béatitude ! Il a été   homicide. Pourquoi ? Il convoitait la femme ou l’héritage de son frère, il a voulu le voler pour vivre, ou se mettre en garde contre ses larcins ; il brûlait de venger une offense. Aurait-il tué pour le plaisir même du meurtre ?\par
Est-ce croyable ? Car s’il est dit de cet homme, monstre de démence et de cruauté, qu’il était gratuitement méchant et cruel, nous savons néanmoins pourquoi.\par

\begin{quoteblock}
\noindent « Il craignait, dit l’historien, que le repos n’énervât sa main ou son cœur (Sallust. Guerr. De Cat., C. IX). »\end{quoteblock}

\noindent  Mais ici encore, pourquoi ? Il voulait que cette pratique du crime le rendît maître de Rome, fît tomber dans ses mains honneurs, richesses, autorité ; l’affranchît de la crainte des lois, et de cette détresse où le réduisaient la perte de sa fortune et la conscience de ses crimes. Ce Catilina n’aimait donc pas ses forfaits mêmes, mais la fin qui le portait à les commettre.
\section[{Chapitre VI, Il se trouve dans les péchés une imitation fausse des perfections divines.}]{Chapitre VI, Il se trouve dans les péchés une imitation fausse des perfections divines.}
\noindent \pn{12}Qu’ai-je donc aimé en toi, malheureux larcin, crime nocturne de mes seize ans ? Tu n’étais pas beau, étant un larcin ; es-tu même quelque chose, pour que je parle à toi ? Ces fruits volés par nous étaient beaux, parce qu’ils étaient votre œuvre, beauté infinie, créateur de toutes choses, Dieu bon, Dieu souverain bien et mon bien véritable. Ces fruits étaient beaux ; mais ce n’était pas eux que convoitait mon âme misérable ; j’en avais de meilleurs en abondance ; je ne les ai donc cueillis que pour voler. Car aussitôt je les jetai, ne savourant que l’iniquité, ma seule jouissance, ma seule joie. Si j’en approchai quelqu’un de ma bouche, je n’y goûtai que la saveur de mon crime.\par
Et maintenant, Seigneur mon Dieu, je cherche ce qui m’a plu dans ce larcin, et je n’y vois aucune ombre de beauté. Je ne parle point de cette beauté qui réside dans l’équité, dans la prudence ; ou bien, dans l’esprit de l’homme, sa mémoire, ses sens, sa vie végétative ; ni de la splendide harmonie des corps célestes, et de la terre et de la mer se peuplant de créatures par une continuelle succession de naissances et de morts ; ni même de cette beauté menteuse, voile des vices décevants.\par
\pn{13}Car l’orgueil contrefait l’élévation ; et vous seul, ô mon Dieu, êtes élevé au-dessus de tous les êtres. L’ambition, que cherche-t-elle, sinon les honneurs et la gloire ? Et vous seul devez être honoré, seul glorifié dans tous les. siècles. La tyrannie veut se faire craindre ; et qui est à craindre que vous seul, ô Dieu ? Votre pouvoir se laisse-t-il jamais rien ravir, rien soustraire ? Quand, où, par qui se pourrait-il ? Et les profanes caresses veulent surprendre l’amour ; mais quoi de plus caressant que votre amour ? Quoi de plus heureusement aimable que la beauté resplendissante et souveraine de votre vérité ? La curiosité se donne pour la passion de la science ; et vous seul possédez la science universelle et suprême. L’ignorance même et la stupidité ne se couvrent-elles pas du nom de simplicité et d’innocence, parce que rien ne saurait être plus simple que vous ? Rien de plus innocent que vous, car c’est dans leurs œuvres que les méchants trouvent leur ennemi. La paresse prétend n’être que l’appétence du repos ; et quel repos assuré que dans le Seigneur ? Le luxe se dit magnificence ; mais vous êtes la source vive et inépuisable des incorruptibles délices. La profusion se farde des traits de la libéralité ; - mais vous êtes l’opulent dispensateur de toutes largesses. L’avarice veut beaucoup posséder, et vous possédez tout. L’envie dispute la prééminence ; quoi de plus éminent que vous ? La colère cherche la vengeance ; qui se venge plus justement que vous ? La crainte frémit des soudaines rencontres, menaçantes pour ce qu’elle aime ; elle veille à sa sécurité : mais pour vous est-il rien d’étrange, rien de soudain ? Qui vous sépare de ce que vous aimez ? Hors de vous, où est la constante sécurité ? La tristesse se consume dans la perte des jouissances passionnées, parce qu’elle voudrait qu’il lui fût aussi impossible qu’à vous de rien perdre.\par
\pn{14}Ainsi l’âme devient adultère, lorsque, détournée de vous, elle cherche hors de vous ce qu’elle ne trouve, pur et sans mélange, qu’en revenant à vous. Ceux-là vous imitent avec perversité, qui s’éloignent de vous, qui s’élèvent contre vous. Et toutefois, en vous imitant ainsi, ils montrent que vous êtes le créateur de l’univers, et que vous ne laissez aucune place où l’on puisse se retirer entièrement de vous. Et moi, qu’ai-je donc aimé dans ce larcin ? En quoi ai-je imité mon Dieu ? faux et criminel imitateur ! Ai-je pris plaisir à   enfreindre la loi par la ruse, au défaut de la puissance ; et, sous les liens de la servitude, affectant une liberté boiteuse, ai-je trouvé dans la faculté de violer impunément la justice une ténébreuse image de la Toute-Puissance ? C’est l’esclave qui fuit son maître et n’atteint qu’une ombre ! O corruption ! ô monstre de vie ! ô abîme de mort ! Ce qui était illicite a-t-il pu me plaire, et par cela seul qu’il était illicite ?
\section[{Chapitre VII, Actions de grâces.}]{Chapitre VII, Actions de grâces.}
\noindent \pn{15}Que rendrai-je au Seigneur qui délivre mon âme du trouble de ces souvenirs ? Que je vous aime, Seigneur, que je vous rende grâces et confesse votre nom, ô vous qui m’avez remis tant de criminelles et abominables œuvres ! À votre grâce, à votre miséricorde je rapporte d’avoir fondu la glace de mes péchés. À votre grâce je rapporte tout ce que je n’ai pas fait de mal. Eh ! de quoi n’étais-je point capable ayant aimé le crime sans intérêt ? Et je confesse que tout m’est pardonné, et le mal que j’ai fait de gré, et celui que m’a épargné votre miséricorde.\par
Quel mortel, méditant sur son infirmité, oserait attribuer à ses propres forces sa chasteté et son innocence, et se croirait en droit de vous moins aimer, comme s’il eût eu moins besoin de ce miséricordieux pardon que vous accordez au repentir des pécheurs ? Que l’homme qui, docile à l’appel de votre voix, a évité tous ces désordres dont je publie le souvenir et l’aveu, se garde de rire s’il me voit guéri par le même médecin à qui il doit de n’avoir pas été, ou plutôt d’avoir été moins malade ; qu’il vous en aime autant, qu’il vous en aime davantage, reconnaissant que celui qui me délivre est le même qui l’a préservé des mortelles défaillances du péché.
\section[{Chapitre VIII, Ce qu’il avait aimé dans ce larcin.}]{Chapitre VIII, Ce qu’il avait aimé dans ce larcin.}
\noindent \pn{16}Malheureux ! quel avantage trouvais-je donc alors dans ces actions, dont aujourd’hui la pensée me fait rougir (Rom. VI, 21), et surtout dans ce vol où je n’aimai que lui ; rien que lui, rien sans doute, car lui-même n’était rien… pour moi cependant un surcroît de misère ! et pourtant seul je ne l’eusse pas fait. Ma mémoire me représente bien mon âme alors ; non, seul, je ne l’eusse pas fait. C’est donc, en outre, la société de mes complices que j’ai aimée. J’ai donc aimé autre chose que le vol ? Mais quoi ? rien ; car cela même encore n’est rien.\par
Qu’y a-t-il donc là en réalité ? Qui me l’enseignera, que Celui qui éclaire mon cœur et en dissipe les ténèbres ? Quelle est enfin la cause de cet acte coupable ? Mon esprit la recherche ; il la poursuit ; il veut la pénétrer. Si j’aimai ces fruits, si je les désirai, que ne les volai-je seul ? Ne suffisait-il pas à ma convoitise de commettre l’iniquité sans envenimer par le frottement de la complicité les démangeaisons de mon désir ? Mais ce plaisir que ces fruits ne me donnaient pas, je ne le trouvais dans le péché que par cette association de pécheurs.
\section[{Chapitre IX, Liaisons funestes.}]{Chapitre IX, Liaisons funestes.}
\noindent \pn{17}Quel était donc cet instinct de mon âme ? Vil et honteux instinct ! Ame misérable, tu t’es livrée à lui ! Quel était enfin cet instinct maudit ?\par

\begin{quoteblock}
\noindent « Oh ! qui peut sonder l’abîme des péchés (Ps. XVIII, 13) ? »\end{quoteblock}

\noindent  C’était un rire malin qui nous chatouillait le cœur à l’idée de tromper un homme et de l’irriter. Pourquoi donc avais-je du plaisir à n’être pas seul ? Seul, est-il plus difficile de rire ? Il est vrai ; et cependant un homme est seul, et le rire s’empare de lui, si un objet trop ridicule frappe ses sens ou son esprit. Mais moi, je n’eusse rien fait seul ; non, seul, je n’eusse rien fait.\par
Oui, mon Dieu, voici devant vous la vivante souvenance de mon âme ! Seul, je n’eusse pas commis ce larcin, n’en aimant pas l’objet, n’aimant que lui-même. Seul, je n’eusse trouvé aucun plaisir à le faire , je ne l’eusse point fait. O amitié ennemie, subtile séduction de l’esprit, ardeur de nuire et de dérober, inspirée par l’entrain et le jeu, sans cupidité, sans passion vindicative, sur un seul mot Allons, dérobons ! et l’on rougit de rougir encore !  
\section[{Chapitre X, Élan vers Dieu.}]{Chapitre X, Élan vers Dieu.}
\noindent \pn{18}Qui démêlera ces tortueux replis, ce nœud inextricable ? Il recèle la honte ; je n’y veux plus penser ; je ne le veux plus voir. C’est vous que je veux, ô justice, ô innocence, si belle aux chastes regards, dont la jouissance nous laisse insatiables ! En vous est la paix profonde et la vie inaltérable. Celui qui entre en vous,\par

\begin{quoteblock}
\noindent « entre dans la joie de son Seigneur (Matth. XXV, 21). »\end{quoteblock}

\noindent  Libre de toute crainte, il demeure souverainement bien dans le Bien souverain. J’ai dérivé loin de vous, et je me suis égaré, mon Dieu ; mon adolescence s’est écoulée hors de votre stabilité, et je suis devenu à moi-même une contrée d’indigence.  
\chapterclose


\chapteropen
\chapter[{III. Les études à Carthage}]{III. Les études à Carthage}\phantomsection
\label{III}\renewcommand{\leftmark}{III. Les études à Carthage}


\begin{argument}\noindent Amours impurs. — Il tombe à dix-neuf ans dans l’hérésie des Manichéens. — Prières et larmes de sa mère. Paroles prophétiques d’un évêque.
\end{argument}


\chaptercont
\section[{Chapitre I, Amours impures.}]{Chapitre I, Amours impures.}
\noindent \pn{1}Je vins à Carthage, où bientôt j’entendis bouillir autour de moi la chaudière des sales amours. Je n’aimais pas encore, et j’aimais à aimer ; et par une indigence secrète, je m’en voulais de n’être pas encore assez indigent. Je cherchais un objet à mon amour, aimant à aimer ; et je haïssais ma sécurité, ma voie exempte de piéges. Mon cœur défaillait, vide de la nourriture intérieure, de vous-même, mon Dieu ; et ce n’était pas de cette faim-là que je me sentais affamé ; je n’avais pas l’appétit des aliments incorruptibles : non que j’en fusse rassasié ; je n’étais dégoûté que par inanition. Et mon âme était mal portante et couverte de plaies, et se jetant misérablement hors d’elle-même, elle mendiait ces vifs attouchements qui devaient envenimer son ulcère. C’est la vie que l’on aime dans les créatures aimées, être aimé m’était encore plus doux, quand la personne aimante se donnait toute à moi.\par
Je souillais donc la source de l’amitié des ordures de la concupiscence ; je couvrais sa sérénité du nuage infernal de la débauche. Hideux et infâme, dans la plénitude de ma vanité, je prétendais encore à l’urbanité élégante. Et je tombai dans l’amour où je désirais être pris, ô mon Dieu, ô ma miséricorde, de quelle amertume votre bonté a assaisonné ce miel ! Je fus aimé, j’en vins aux liens secrets de la jouissance, et, joyeux, je m’enlaçais dans un réseau d’angoisses, pour être bientôt livré aux verges de fer brûlantes de la jalousie, des soupçons, des craintes, des colères et des querelles.
\section[{Chapitre II, Théâtres.}]{Chapitre II, Théâtres.}
\noindent \pn{2}Je me laissais ravir au théâtre, plein d’images de mes misères, et d’aliments à ma flamme. Mais qu’est-ce donc ? et comment l’homme veut-il s’apitoyer au spectacle des aventures lamentables et tragiques qu’il ne voudrait pas lui-même souffrir ? Et cependant, spectateur, il veut en souffrir de la douleur, et cette douleur même est son plaisir. Qu’est-ce donc, sinon une pitoyable maladie d’esprit ? Car notre émotion est d’autant plus vive, que nous sommes moins guéris de ces passions quoique pâtir s’appelle misère, et compatir, miséricorde. Mais quelle est cette compatissance pour des fictions scéniques ? Appelle-t-on l’auditeur au secours ? Non, il est convié seulement à se douloir ; et il applaudit l’acteur, en raison de la douleur qu’il reçoit. Et si la représentation de ces infortunes, antiques ou imaginaires, le laisse sans impressions douloureuses, il se retire le dédain et la critique à la bouche. Est-il douloureusement ému, il demeure attentif, et pleure avec joie.\par
\pn{3}Mais tout homme veut se réjouir ; d’où vient donc cet amour des larmes et de la douleur ? Le plaisir, que la misère exclut, se trouve-t-il dans la commisération ? Et ce sentiment fait-il aimer la douleur dont il ne saurait se passer ? L’amour est la source de ces sympathies. Où va cependant, où s’écoule ce flot ? Au torrent de poix bouillante, au gouffre ardent des noires voluptés, où il change et se confond lui-même, égaré si loin et déchu de la limpidité céleste. Faut-il donc répudier la compassion ? Nullement. La douleur est donc parfois aimable ; mais garde-toi de l’impureté, ô mon   âme, sous la tutelle de mon Dieu, Dieu de nos pères, qui doit être loué et exalté dans tous les siècles (Dan. III, 32) ; garde-toi de l’impureté, car je ne suis pas aujourd’hui fermé à la commisération. Mais alors, au théâtre, j’entrais dans la joie de ces amants qui se possédaient dans le crime, et pourtant ce n’était que feinte et jeux imaginaires. Alors qu’ils étaient perdus l’un pour l’autre, je me sentais comme une compatissante tristesse ; et pourtant je jouissais de ce double sentiment.\par
Aujourd’hui, j’ai plus en pitié la joie dans le vice, que les prétendues souffrances nées de la ruine d’une pernicieuse volupté, et de la perte d’une félicité malheureuse. Assurément, c’est là une compassion vraie ; mais la douleur n’y est plus un plaisir. Car si la charité approuve celui qui plaint douloureusement un affligé, néanmoins, une pitié vraiment fraternelle préférerait qu’il n’y eût point une douleur à plaindre. Et, en effet, la bonne volonté ne saurait pas plus vouloir le mal, que le vrai miséricordieux désirer qu’il y ait des misérables pour exercer sa miséricorde.\par
Il est donc certaine douleur permise, il n’en est point que l’on doive aimer. Ainsi, Seigneur, mon Dieu, vous qui aimez les âmes d’un amour infiniment plus pur que nous, votre compassion pour elles est d’autant plus incorruptible, que vous ne sentez l’atteinte d’aucune douleur. Mais l’homme en est-il capable ?\par
\pn{4}Malheureux que j’étais, j’aimais à me douloir, et je cherchais des sujets de douleurs. Dans ces infortunes étrangères et fausses, ces infortunes rie saltimbanques, jamais le jeu d’un histrion ne me plaisait, ne m’attachait par un charme plus fort que celui des larmes qui jaillissaient de mes yeux. Faut-il s’en étonner ? Pauvre brebis égarée de votre troupeau, et impatiente de votre houlette, j’étais couvert d’une lèpre honteuse.\par
Et voilà d’où venait mon amour pour ces douleurs, non toutefois jusqu’au désir d’en être pénétré plus avant. Car je n’eusse pas aimé souffrir ce qui me plaisait à voir ; mais ces récits, ces fictions m’effleuraient vivement la chair, et, comme l’ongle envenimé, elles soulevaient bientôt une brûlante tumeur, distillant le pus et la sanie. Telle était ma vie ; était-ce une vie ? ô mon Dieu !
\section[{Chapitre III, Insolence de la jeunesse de Carthage.}]{Chapitre III, Insolence de la jeunesse de Carthage.}
\noindent \pn{5}Et votre miséricorde fidèle planait de loin, les ailes étendues sur moi. En quelles dissolutions ne me suis-je pas consumé ? Loin de vous, j’ai suivi une curiosité sacrilège, qui m’amena au plus profond de l’infidélité, au culte trompeur des démons, à qui j’offrais comme un sacrifice de mes actes criminels, et dans tous je sentais votre fouet. N’ai-je pas osé, même pendant la célébration d’une solennité sainte, dans votre sanctuaire, convoiter l’impudicité et marchander des fruits de mort ? Votre main alors s’est appesantie davantage sur moi, mais non en raison de ma faute, ô mon Dieu, mon immense miséricorde, mon refuge contre ces épouvantables pécheurs, avec qui je m’égarais présomptueux, la tête haute, toujours plus loin de vous, aimant mes voies et non les vôtres, aimant ma liberté d’esclave fugitif.\par
\pn{6}Ces études, prétendues honnêtes, avaient leur aboutissant au forum de la chicane ; et j’aspirais à me distinguer là où les succès se mesurent aux mensonges. Tel est l’aveuglement des hommes, et, cet aveuglement même, ils s’en glorifient ! Et déjà je l’emportais à l’école du rhéteur ; et ma joie était superbe, et j’étais gonflé de vent. Mais pourtant, plus retenu que les autres, Seigneur, vous le savez, j’étais bien éloigné de « démolir » avec les « démolisseurs. » (Ce nom de furies et de démons reçoit une acception d’urbanité.) Et je vivais avec eux, impudent dans ma pudeur, puisque je n’étais pas comme eux ; et je trouvais parfois du plaisir dans leur familiarité, malgré l’horreur que m’inspiraient leurs actes, ces « démolitions » effrontées dont ils assaillaient la modestie de l’étranger, faisant de son trouble l’objet de leurs jeux iniques et la pâture de leurs malignes joies. Quoi de plus semblable aux actes des démons ? Et pouvaient-ils s’appeler mieux que démolisseurs ? Mais, démolisseurs démolis, livrés aux secrètes risées et aux séductions des esprits de mensonge, au moment même où ils se plaisaient à railler et à tromper autrui. 
 \section[{Chapitre IV, Il se passionne pour la sagesse à la lecture de l’Hortensius de Cicéron.}]{Chapitre IV, Il se passionne pour la sagesse à la lecture de l’{\itshape Hortensius} de Cicéron.}
\noindent \pn{7}C’est en telle compagnie que, dans un âge encore tendre, j’étudiais l’éloquence où je désirais exceller, à malheureuses et damnables fins, les joies de la vanité humaine. Et l’ordre suivi dans cette étude m’avait mis sous les yeux un certain livre de Cicéron, dont on admire plus généralement la langue que le cœur. Ce livre contient une exhortation à la philosophie, c’est l’{\itshape Hortensius}. Sa lecture changea mes sentiments ; elle changea les prières que je vous adressais à vous-même, Seigneur ; elle rendit tout autres mes vœux et mes désirs. Je ne vis soudain que bassesse dans l’espérance du siècle, et je convoitai l’immortelle sagesse avec un incroyable élan de cœur, et déjà je commençais à sue lever pour revenir à vous. Car je ne songeais plus à raffiner mon langage, unique fruit que payaient pour un fils de dix-neuf ans les épargnes de ma mère, veuve depuis plus de deux années ; non, je ne rapportais plus à la vanité du langage la lecture de ce livre ; il m’avait persuadé ce qu’il disait et non pas son bien dire.\par
\pn{8}Oh ! comme je brûlais, mon Dieu I comme je brûlais de revoler de la terre à vous ! et je ne savais pas ce que vous faisiez en moi. Car la sagesse est en vous, et ce n’est que l’amour de la sagesse, nommé par les Grecs philosophie, que cette lecture allumait en moi. Il est des hommes qui se servent de la philosophie pour tromper, et, de ce nom si grand, si séduisant, si vénérable, ils colorent et fardent leurs erreurs. Et tous les prétendus sages de son temps ou des siècles antérieurs, l’auteur de l’{\itshape Hortensius} les note et les montre du doigt, rendant sans le vouloir témoignage à l’avertissement salutaire que votre Esprit a publié par votre saint et fidèle serviteur :\par

\begin{quoteblock}
\noindent « Prenez garde que personne ne vous surprenne par la philosophie, par de vaines subtilités, selon les traditions des hommes, selon les principes d’une fausse science naturelle, et non selon le Christ ; car en lui habite corporellement toute la plénitude de la divinité (Coloss. II, 8,9). »\end{quoteblock}

\noindent Et en ce temps, vous le savez, lumière de mon cœur, j’ignorais encore ces paroles de l’Apôtre, et ce qui me plaisait uniquement en cette exhortation, c’est que ne proposant à mon choix aucune secte, mais la sagesse elle-même quelle qu’elle fût, elle m’excitait à l’aimer, à la rechercher, à la poursuivre, à l’atteindre et à l’embrasser fortement ; et je brûlais, et je débordais d’enthousiasme. Une chose seule ralentissait un peu mes transports ; le nom du Christ n’était pas là. Ce nom, suivant le dessein de votre miséricorde, Seigneur, ce nom de mon Sauveur votre Fils, avait été amoureusement bu par mon tendre cœur avec le lait même de ma mère, et il était demeuré au fond ; et, sans ce nom, nul livre, si rempli qu’il fût de beautés, d’élégance et de vérité, ne pouvait me ravir tout entier.
\section[{Chapitre V, Son mépris pour l’écriture.}]{Chapitre V, Son mépris pour l’écriture.}
\noindent \pn{9}Je pris donc la résolution d’appliquer mon esprit à la sainte Écriture, et de connaître ce qu’elle était. Je le sais aujourd’hui : une chose qui ne se dévoile ni à la pénétration des superbes, ni à la simplicité des enfants ; entrée basse, voûtes immenses, partout un voile de mystères ! Et je n’étais pas capable d’y entrer, ni de plier ma tête à son allure. Car alors je n’en pensais pas comme j’en parle aujourd’hui : elle me semblait indigne d’être mise en parallèle avec la majesté cicéronienne. Mon orgueil répudiait sa simplicité, et mon regard ne pénétrait pas ses profondeurs. Et c’était pourtant cette Écriture qui veut croître avec les petits : mais je dédaignais d’être petit ; et enflé de vaine gloire, je me croyais grand.
\section[{Chapitre VI, Il tombe dans l’erreur des manichéens.}]{Chapitre VI, Il tombe dans l’erreur des manichéens.}
\noindent \pn{10}Aussi, je rencontrai des hommes, au superbe délire, charnels et parleurs ; leur bouche recélait un piége diabolique, une glu composée du mélange des syllabes de votre nom, et des noms de Notre-Seigneur Jésus-Christ et du Paraclet notre consolateur, l’Esprit-Saint. Ces noms résidaient toujours sur leurs lèvres, mais ce n’était qu’un son vainement articulé ; leur cœur était vide du vrai. Et ils disaient : Vérité, vérité ; ils me la nommaient sans cesse, et jamais elle n’était en eux. Ils débitaient l’erreur, non-seulement sur vous, qui êtes vraiment la vérité, mais sur ce monde élémentaire, votre ouvrage, où, par delà les vérités mêmes connues des philosophes j’ai dû m’élancer, grâce   à votre amour, ô mon Père, ô bonté souveraine, beauté de toutes les beautés !\par
Vérité, vérité, combien alors même, et du plus profond de mon âme, je soupirais pour vous, quand, si souvent, et de mille manières, et de vive voix, ces hommes faisaient autour de moi bruire votre nom dans leurs nombreux et longs ouvrages ! Et les mets qu’ils servaient à mon appétit de vérité, c’étaient, au lieu de vous, « la lune, le soleil,» chefs-d’œuvre de vos mains, mais votre œuvre, et non pas vous, ni même votre œuvre suprême ; car vos créatures spirituelles sont encore plus excellentes que ces corps éclatants de lumière et roulant dans les cieux.\par
Et ce n’était pas de ces créatures excellentes, c’était de vous seule, ô vérité sans changement et sans ombre (Jacq. I,17), que j’avais faim et soif ; et l’on ne présentait à ma table que de splendides fantômes. Et mieux eût valu attacher mon amour à ce soleil, vrai du moins pour les yeux, qu’à ces mensonges, qui, par les yeux, trompent l’esprit. Et toutefois je les prenais pour vous, et je m’en nourrissais, mais sans avidité, car mon palais ne me rendait pas la saveur de votre réalité ; et vous n’étiez rien de toutes ces vaines fictions, où je trouvais moins aliment qu’épuisement. La nourriture imaginaire de nos songes est semblable à la nourriture de nos veilles ; et elle laisse notre sommeil à jeun. Mais ces vanités ne vous ressemblaient en rien, comme depuis votre parole me l’a fait connaître ; ce n’étaient que rêves insensés, corps fantastiques, bien éloignés de la certitude de ces corps réels, soit célestes, soit terrestres, que nous voyons de l’oeil charnel, de l’oeil des brutes et des oiseaux ; corps plus vrais néanmoins dans leur réalité que dans notre imagination ; mais combien notre imagination est plus vraie que cette induction chimérique qui se plaît à en soupçonner d’immenses, d’infinis, pur néant, dont alors je me repaissais à vide !\par
Mais vous, mon amour, en qui je me meurs pour être fort, vous n’êtes ni ces corps que nous voyons dans les cieux, ni ceux que nous ne pouvons voir de si bas ; car ils ne sont que vos créatures, et même ne résident pas au faîte de votre création. Combien donc êtes-vous loin de ces folles conceptions, de ces chimères de corps qui n’ont aucun être, qui ont moins de certitude que les images mêmes des corps réels, entités plus certaines que ces images, et qui ne sont pas vous : vous n’êtes pas même l’âme qui est leur vie, cette vie des corps meilleure et plus certaine que les corps ; mais vous êtes la vie des âmes, la vie des vies, indépendante et immuable vie, ô vie de mon âme !\par
\pn{11}Où étiez-vous alors, à quelle distance de moi ? Et je voyageais loin de vous, sevré même du gland dont je paissais les pourceaux (Luc, XV, 16) . Combien les fables des grammairiens et des poètes sont préférables à ces mensonges ! Ces vers, cette poésie, cette Médée qui s’envole, sont encore plus utiles que les cinq éléments, bizarrement travestis pour correspondre aux cinq cavernes de ténèbres, néant qui tue l’âme crédule. La poésie, l’art des vers sont encore des aliments de vérité. Et je déclamais le vol de Médée, sans l’affirmer ; je l’entendais déclamer, sans y croire ; mais ces autres folies, je les ai crues. Malheur ! malheur ! Par quels degrés ai-je roulé au fond de l’abîme ? O mon Dieu, je vous confesse mon erreur, à vous qui avez eu pitié de moi, quand je ne vous la confessais pas encore ; je vous cherchais, dans une laborieuse et haletante pénurie de vérité ; je vous cherchais non par l’intelligence raisonnable qui m’élève au-dessus des animaux, mais par le sens charnel ; et vous étiez intérieur à l’intimité, supérieur aux sommités de mon âme. Je rencontrai l’énigme de Salomon, cette femme hardie, pauvre en sagesse, assise devant sa porte, où elle crie :\par

\begin{quoteblock}
\noindent « Mangez avec plaisir le pain caché ; buvez avec délices les eaux dérobées (prov. IX, 17)»\end{quoteblock}

\noindent  Cette femme me séduisit, parce qu’elle me trouva tout au dehors habitant l’oeil de ma chair, et ruminant en moi tout ce qu’il m’avait donné à dévorer.
\section[{Chapitre VII, Folies des manichéens.}]{Chapitre VII, Folies des manichéens.}
\noindent \pn{12}Car je ne soupçonnais pas cette autre nature qui seule est en vérité, et je me démenais en subtilités pour complaire à ces ridicules imposteurs, quand ils me demandaient d’où vient le mal ; si Dieu est borné aux limites d’une forme corporelle ; s’il a des cheveux et des ongles ; et s’il faut tenir pour justes ceux qui avaient plusieurs femmes, tuaient des hommes et sacrifiaient des animaux ? Ces questions   troublaient mon ignorance ; je me retirais de la vérité, et me figurais aller vers elle, parce que je ne savais pas que le mal n’est que la privation du bien, privation dont le dernier terme est le néant. Et pouvais-je le voir, moi dont la vue s’arrêtait au corps, et l’esprit au fantôme ?\par
Et je ne savais pas que « Dieu est un esprit » qui n’a point de membres mesurables en longueur et largeur, dont l’être n’est point masse, car la masse est moindre en sa partie, qu’en son tout. Et fût-elle infinie, elle est moindre dans un espace défini, que dans son étendue infinie ; et elle n’est pas toute en tous lieux, comme l’esprit, comme Dieu, et j’ignorais entièrement ce qui est en nous, par quoi nous sommes semblables à Dieu, et en quel sens 1’Écriture a raison de dire que\par

\begin{quoteblock}
\noindent « nous sommes faits à son image (Genèse I, 27). »\end{quoteblock}

\noindent \pn{13}Et je ne connaissais pas cette vraie justice intérieure, qui ne juge pas sur la coutume, mais sur la loi de rectitude du Dieu tout-puissant qui ordonne les mœurs des pays et des jours, selon les pays et les jours, toujours et partout la même, pas autre en d’autres lieux, pas autre en d’autres temps devant qui sont justes Abraham, Isaac, Jacob, Moïse, David et tous ces hommes loués de la bouche de Dieu, jugés injustes par les ignorants qui jugent au jour de l’homme, et soumettent la conduite universelle du genre humain au point de vue de leur siècle et de leur foyer. Novice aux armes, tu ignores à quel membre s’ajuste ce casque, ce cuissard ; tu prends le casque pour chaussure, le cuissard pour te couvrir la tête ; et tu prétends en murmurant que l’armure n’est pas à ta taille ! Un jour, après l’heure du midi, toute vente est prohibée : ce marchand va-t-il se révolter contre cette défense, parce qu’elle n’existait pas ce matin ? Trouveras-tu étrange, dans une maison, que tel serviteur touche des objets interdits à celui qui verse à boire, que l’on fasse à l’écurie ce qui n’est pas permis à table ? Et faut-il s’étonner que sous le même toit, dans la même troupe d’esclaves, même permission ne soit donnée ni partout, ni à tous ? Telle est l’erreur de ceux qui ne peuvent souffrir qu’il ait été permis aux justes des anciens jours ce qui n’est pas permis aux justes d’aujourd’hui ; et que Dieu ait fait tel commandement à ceux-ci, tel à ceux-là, pour des raisons temporelles, tous néanmoins demeurant esclaves de l’éternelle justice ; et cependant, dans un même homme, dans un même jour, sous un même toit, ce qui sied à un membre répugne à l’autre, ce qui est loisible maintenant cessera de l’être dans une heure ; ce qui est permis ou ordonné là, est ici justement défendu, et puni. Est-ce à dire que la justice est différente et muable ? Non ; mais les temps qu’elle gouverne changent dans leur fuite, car ils sont temps. Et les hommes trop courts de jours et de vue pour embrasser dans leur ensemble les principes régulateurs des siècles passés et des différentes sociétés humaines en les rattachant aux éléments contemporains, mais apercevant sans peine ce qui, dans un seul corps, un seul jour, une seule maison convient à tel membre, à tel moment, à tel lieu, à telle personne, se soumettent à l’ordre particulier, et se révoltent contre l’ordre général.\par
\pn{14}J’ignorais alors ces vérités, et je n’y songeais pas ; elles frappaient mes yeux de toutes parts, et je ne voyais pas. Et quand je chantais des vers, je savais bien qu’il ne m’était pas permis de jeter au hasard un pied quelconque, qu’il fallait le placer différemment suivant la variété des mesures, et que, dans un même vers, le même pied ne pouvait se répéter partout ; quoique l’art lui-même, qui présidait à mes chants, soit invariable dans sa législation, constant et universel. Et je ne considérais pas que la justice souveraine des bonnes et saintes âmes, contient, d’une manière infiniment plus excellente et plus sublime, toutes les règles qu’elle a données, partout invariable et appropriant néanmoins à la variété des temps, non pas l’universalité, mais la convenance particulière de ses préceptes. Aveugle que j’étais, je blâmais ces saints patriarches qui ont usé du présent suivant l’inspiration et le commandement de Dieu, et annoncé l’avenir qu’il dévoilait à leurs yeux !
\section[{Chapitre VIII, Ce que Dieu commande devient permis.}]{Chapitre VIII, Ce que Dieu commande devient permis.}
\noindent \pn{15}Où, quand, est-il injuste d’aimer Dieu de tout son cœur, de toute son âme, de tout son esprit, et son prochain comme soi-même ? Au rebours, les crimes contre nature, tels que ceux de Sodome, appellent partout et toujours l’horreur et le châtiment. Que si tous les   peuples imitaient Sodome, ils seraient tenus de la même culpabilité devant la loi divine, qui n’a pas fait les hommes pour user ainsi d’eux-mêmes. Car c’est violer l’alliance qui doit être entre nous et Dieu, que de profaner par de vils appétits de débauche la nature dont il est l’auteur.\par
Pour les délits contraires aux coutumes locales, ils se doivent éviter selon la diversité des mœurs : le pacte social établi dans une ville, chez un peuple, par l’usage ou la loi, ne saurait être enfreint suivant le caprice d’un citoyen ou d’un étranger. Il y a difformité dans toute partie en désaccord avec son tout,\par
Mais quand Dieu ordonne contre la coutume, contre la loi, où que ce soit, c’est chose à faire, n’eût-elle jamais été faite ; à renouveler, si elle est oubliée ; n’est-elle pas établie ? il faut l’établir. S’il est permis à un roi, dans la ville où il règne, d’ordonner ce que nul avant lui et ce que lui-même n’avait point encore voulu ; lui obéir, ce n’est pas violer l’ordre de la ville, c’est le violer plutôt, que de ne pas lui obéir ; car le pacte fondamental de la société humaine est l’obéissance aux rois. Combien donc est-il plus raisonnable de voler à l’exécution des volontés du grand Roi de l’univers ? Dans la hiérarchie des pouvoirs humains, la préséance de l’autorité supérieure sur la moindre est reconnue par le sujet ; à Dieu la préséance absolue.\par
\pn{16}Même réprobation de tout crime où se trouve le désir de nuire par propos outrageants, par acte de violence, soit inimitié vindicative, soit convoitise d’un bien étranger qui précipite le brigand sur le voyageur, soit précautions de la peur fatales à qui l’inspire, soit envie du misérable qui jalouse un heureux, de l’heureux qui craint ou souffre de trouver un égal ; soit simple goût du mal d’autrui, qui séduit les spectateurs des combats de l’arène, et les rieurs et les railleurs. Voilà les grands chefs d’iniquité qui ont leurs racines dans la triple concupiscence de dominer, de voir, de sentir, tantôt séparées, tantôt réunies. Et la vie est mauvaise, qui s’élève contre les nombres trois et sept, contre l’harmonieuse harpe à dix cordes, votre décalogue, ô Dieu, toute puissance et toute suavité ! Mais quels crimes peuvent vous atteindre, vous que rien ne corrompt ? Quels forfaits vous intéressent, vous à qui rien ne peut nuire ? Et néanmoins vous vous portez vengeur de tout ce que les hommes attentent contre eux-mêmes, parce qu’en vous offensant ils traitent leurs âmes avec impiété, car l’iniquité est infidèle contre elle-même ; parce qu’ils dépravent ou ruinent leur nature que vous avez faite et ordonnée, soit par l’abus des choses permises, soit par l’impur désir et l’usage contre nature des choses défendues ; parce qu’ils entreprennent contre vous dans les révoltes de leur cœur et les blasphèmes de leur parole, et regimbent contre l’aiguillon ; parce que, brisant toutes les barrières de la société humaine, ils s’applaudissent avec audace des factions et des cabales qu’élève leur intérêt ou leur ressentiment. Et ces désordres arrivent, lorsqu’on vous abandonne, source de la vie, seul et véritable créateur et modérateur du monde ; lorsqu’un orgueil privé poursuit d’un amour étroit un objet d’erreur. Aussi n’est-ce que par l’humble piété qu’on a retour vers vous ; vous nous délivrez alors de l’habitude du mal. Propice à l’aveu du pécheur, vous exaucez les gémissements de l’esclavage ; vous brisez les fers que nous nous sommes forgés à nous-mêmes, pourvu que nous ne dressions plus contre vous cette corne infernale d’une fausse liberté, jaloux d’avoir davantage, au risque de tout perdre, préférant notre bien particulier à vous, seul Bien de tous les êtres.
\section[{Chapitre IX, Dieu juge autrement que les Hommes.}]{Chapitre IX, Dieu juge autrement que les Hommes.}
\noindent \pn{17}Mais en outre de cette multitude de souillures et d’iniquités, il est des péchés commis dans les voies de retour, qui, justement blâmés suivant la lettre de la loi de perfection, trouvent faveur comme espérance du fruit à venir, comme l’herbe présage de la moisson. Et il est des actes qui, coupables en apparence, sont néanmoins innocents parce qu’ils ne portent atteinte ni à vous, Seigneur mon Dieu, ni à la société civile ; ainsi certaines satisfactions dominées à l’entretien de la vie, selon les habitudes d’une époque, sans qu’on ait sujet d’accuser une convoitise déréglée ; ainsi l’exercice rigoureux d’une autorité légitime, imputable au désir de réprimer plutôt qu’au besoin de nuire. Combien d’actions répréhensibles aux yeux des hommes, autorisées par votre témoignage ; combien louées par eux, que votre justice condamne ? si différentes sont souvent   l’apparence de l’action, l’intention du cœur, et la donnée secrète des circonstances !\par
Mais quand soudain vous commandez um chose extraordinaire, jusqu’alors défendue pas vous, tinssiez-vous cachées pour un temps les raisons de votre commandement, fût-il contraire aux conventions sociales de quelques hommes ; qui doute qu’il ne faille obéir, puisqu’il n’est de société légitime que celle qui vous obéit ? Mais heureux ceux qui savent que c’est de vous que le commandement est venu. Toutes les actions de vos serviteurs sont l’expression des nécessités du présent ou la figure de l’avenir.
\section[{Chapitre X, Extravagance des manichéens.}]{Chapitre X, Extravagance des manichéens.}
\noindent \pn{18}Dans mon ignorance, je me raillais de ces hommes divins, vos serviteurs et vos prophètes. Et que faisais-je en riant des saints que vous apprêter à rire de moi ? J’en étais venu peu à peu à la niaiserie de croire que la figue que l’on cueille et l’arbre maternel pleurent des larmes de lait ; et que si un saint selon Manès eût mangé cette figue, innocent toutefois du crime de l’avoir cueillie, c’étaient des anges mêlés à son haleine, c’étaient même des parcelles de Dieu, que, dans les soupirs de l’oraison, la digestion de ce fruit rapportait à ses lèvres ; parcelles du Dieu souverain et véritable à jamais comprimées dans cette substance végétale, si elles n’eussent été dégagées par la dent et l’estomac de l’élu. Malheureux ! je croyais qu’il valait mieux avoir pitié des productions de la terre que des hommes pour qui elle produit. Car si tout autre qu’un Manichéen m’eût demandé quelque chose pour apaiser sa faim, le fruit donné à cet homme m’eût paru comme dévoué au dernier supplice.
\section[{Chapitre XI, Prières et larmes de sa mère.}]{Chapitre XI, Prières et larmes de sa mère.}
\noindent \pn{19}Et vous avez étendu votre main d’en-haut, et de ces profondes ténèbres vous avez retiré mon âme (Ps. CXLIII, 7). Car, devant vous, votre fidèle servante, ma mère, me pleurait avec plus de larmes que d’autres mères n’en répandent sur un cercueil. Elle voyait ma mort à cette foi, à cet esprit qu’elle tenait de vous, et vous l’avez exaucée, Seigneur. Vous l’avez exaucée, et n’avez pas dédaigné ces larmes dont le torrent arrosait la terre sous ses yeux partout où elle versait sa prière, et vous l’avez exaucée. Car d’où pouvait venir ce songe, qui lui donna tant de consolation qu’elle m’accorda de partager sa demeure et sa table, dont naguère elle m’avait éloigné, dans l’aversion et l’horreur que lui inspiraient mes hérétiques blasphèmes ?\par
Elle se voyait debout sur une règle de bois, quand vient à elle un jeune homme rayonnant de lumière, serein, et qui souriait à sa douleur morne et profonde. Il lui demande la cause de sa tristesse et de ses larmes journalières, de ce ton qui ne s’informe pas, mais qui veut instruire ; et sur sa réponse qu’elle pleurait ma perte, il lui commande de ne se plus mettre en peine, et de faire attention qu’où elle était, là j’étais aussi, moi. Elle regarda, et me vit à côté d’elle, sur la même règle, debout. Oh ! assurément vous aviez l’oreille à son cœur, Bonté toute-puissante, qui prenez soin de chacun de nous comme s’il était seul, de tous comme de chacun.\par
\pn{20}Et, nouveau témoignage de votre grâce, lorsqu’au récit de sa vision, je cherchais à l’entraîner vers l’espérance d’être un jour elle-même ce que j’étais, elle me répondit sur l’heure sans hésiter : — Non, il ne m’a pas été dit, où il est, tu seras, mais, il sera où tu es. — Je vous confesse, Seigneur, mon souvenir, autant que ma mémoire me le représente, souvenir plus d’une fois rappelé ; je fus frappé de cette parole lancée par ma mère, qui, vigilante à la garde de votre oracle, sans se laisser troubler par le mensonge d’une spécieuse interprétation, vit aussitôt ce qu’il fallait voir, ce que certainement je n’avais pas vu avant sa réponse, Oui, je fus plus frappé de cette parole que de la vision même, présage de ses joies futures, si tardives, et consolation de sa tristesse présente.\par
Car neuf années s’écoulèrent encore, où, me débattant dans les fanges de l’abîme et les ténèbres du mensonge, après de fréquents efforts pour me relever, et de cruelles rechutes, je gravitais toujours plus au fond. Et cependant cette veuve, chaste, pieuse et sobre, telle que vous les aimez, plus vive à l’espérance, mais non moins assidue à pleurer et gémir, ne cessait aux heures de ses prières d’élever pour moi en votre présence la voix de ses soupirs.   Et ses prières pénétraient jusques à vous, et vous me laissiez toujours rouler et plonger dans la nuit !
\section[{Chapitre XII, Parole prophétique d’un évêque.}]{Chapitre XII, Parole prophétique d’un évêque.}
\noindent \pn{21}Mais vous avez rendu un autre oracle, dont je me souviens. Il est beaucoup de choses que je passe sous silence, pour courir à celles qui me pressent de vous rendre témoignage ; il en est beaucoup que j’ai oubliées. Cet oracle, vous l’avez rendu par la bouche d’un évêque, votre serviteur, nourri dans votre Église, exercé au maniement de vos Écritures. Elle le priait un jour de vouloir bien entrer en conférence avec moi, pour réfuter mes erreurs, me faire désapprendre le mal et m’enseigner le bien (elle sollicitait ainsi toute personne qu’elle trouvait capable) ; mais il s’en excusa avec une prudence que j’ai reconnue depuis, et lui répondit : que j’étais encore indocile, étant tout plein des nouveautés de cette hérésie, et des succès de disputes où j’avais, lui disait-elle, embarrassé quelques ignorants. — Laissez-le, ajouta-t-il. Seulement, priez le Seigneur pour lui. Lui-même reconnaîtra par ses lectures toute l’erreur et toute l’impiété de sa créance.\par
Ensuite il raconta que lui aussi, tout enfant, avait été livré aux Manichéens par sa mère qu’ils avaient séduite ; qu’il avait non-seulement lu, mais transcrit de sa main presque tous leurs ouvrages, et que sans dispute, sans lutte d’arguments, il avait vu tout à coup combien cette secte était à fuir ; il l’avait fuie. Comme ma mère, loin de se rendre à ses paroles, le pressait d’instances et de larmes nouvelles, pour qu’il me vît et discutât contre moi : — « Allez, lui dit-il avec une certaine impatience, laissez-moi, et vivez toujours ainsi. Il est impossible que l’enfant de telles larmes périsse.» — Ma mère, dans nos entretiens, rappelait souvent qu’elle avait reçu cette réponse comme une voix sortie du ciel.  
\chapterclose


\chapteropen
\chapter[{IV. Les neuf années manichéennes}]{IV. Les neuf années manichéennes}\phantomsection
\label{IV}\renewcommand{\leftmark}{IV. Les neuf années manichéennes}


\begin{argument}\noindent Neuf années d’erreur. — Sa passion pour l’astrologie. — Mort d’un ami ; violence de sa douleur. — Ses livres de la Beauté et de la Convenance. — Force et vivacité de son intelligence.
\end{argument}


\chaptercont
\section[{Chapitre premier, Neuf années d’erreur.}]{Chapitre premier, Neuf années d’erreur.}
\noindent \pn{1}Pendant ces neuf années de mon âge, de dix-neuf à vingt-huit, je demeurai dans cet esclavage, séduit et séducteur, au gré de mes instincts déréglés ; je trompais en public par les sciences dites libérales ; en secret, par le mensonge d’une fausse religion : ici, jouet de l’orgueil, là, de la superstition, partout de la vanité. Épris du vide de la gloire populaire, j’en étais venu à jalouser les applaudissements du théâtre, les luttes de poésie, la poursuite des couronnes de foin, les bagatelles des spectacles, toutes les intempérances du libertinage. Et demandant d’autre part d’être purifié de ces souillures, j’apportais des aliments à ces saints, à ces élus de Manès, pour que l’alambic de leur estomac en exprimât à mon intention des anges et des dieux libérateurs. Telle était l’extravagance des opinions et des pratiques que je professais avec mes amis, par moi et comme moi séduits.\par
Qu’ils me raillent, ces superbes, qui n’ont pas encore le bonheur d’être humiliés et écrasés par vous, mon Dieu : moi je confesse mes ignominies pour votre gloire ; permettez-moi, je vous en conjure, donnez-moi de promener aujourd’hui mes souvenirs par tous les détours de mes erreurs passées, et\par

\begin{quoteblock}
\noindent « de vous immoler « une victime de joie (Ps XVI, 6).»\end{quoteblock}

\noindent  Car, sans vous, que suis-je à moi-même, qu’un guide malheureux penché sur les précipices ? Et que suis-je, dans la santé de l’âme, qu’un nourrisson allaité de votre lait, et qui se repaît de vous, incorruptible nourriture ? Et qu’est-ce que l’homme, quelque homme que ce soit, puisqu’il est homme ? Qu’ils nous raillent donc, les forts et les puissants ; mais confessons toujours à vous nos infirmités et notre indigence.
\section[{Chapitre II, Il enseigne la rhétorique. — son commerce illégitime avec une femme. — il rejette les offres d’un devin.}]{Chapitre II, Il enseigne la rhétorique. — son commerce illégitime avec une femme. — il rejette les offres d’un devin.}
\noindent \pn{2}J’enseignais alors la rhétorique, l’escrime de la faconde, maître vénal blessé par l’intérêt ; je préférais pourtant, vous le savez, Seigneur, avoir ce qu’on appelle de bons disciples, et en toute simplicité, je leur apprenais l’artifice, non pour s’élever jamais contre la vie de l’innocent, mais pour sauver parfois une tête coupable. Et vous, mon Dieu, vous m’avez vu de loin chanceler sur la voie glissante, vous avez distingué, dans une épaisse fumée, les étincelles de cette probité qui me dévouait à l’instruction de ces amateurs de vanité, de ces chercheurs de mensonge dont j’étais le compagnon. En ces mêmes années, j’avais une femme qui ne m’était pas unie par la sainteté du mariage, mais que l’imprudence d’un vague désir m’avait fait trouver. Seule femme toutefois que je connusse ; je lui gardais la foi ; mais je ne laissais pas de mesurer par ma propre expérience tout l’intervalle qui sépare les convenances d’une légitime union, dont la fin est de transmettre la vie, et cette liaison de voluptueuses amours, dont les fruits naissent contre nos vœux, quoique leur naissance force notre tendresse.\par
\pn{3}Je me souviens encore qu’ayant voulu disputer au concours le prix d’un chant scénique, un devin me fit demander ce que je lui donnerais pour remporter la victoire ; mais, plein d’horreur de ces abominables sacriléges, je   répondis que, s’agît-il d’une couronne d’or impérissable, je ne souffrirais pas que ma victoire coûtât la vie à une mouche. Je savais qu’il immolerait un odieux sacrifice d’animaux, pour me gagner par cette offrande les suffrages des démons. Mais ce ne fut pas au regard de votre chaste amour que je répudiai ce crime, ô Dieu de mon cœur ! je ne savais pas vous aimer, ne pouvant concevoir que des splendeurs corporelles. Et l’âme qui soupire après de telles chimères ne vous est-elle pas infidèle, courtisane du mensonge, pâture des vents ? Et je ne voulais pas que pour moi l’on sacrifiât aux démons, à qui ma superstitieuse créance me sacrifiait chaque jour. Mais n’est-ce pas repaître les vents (Osée, XII, 1) que d’alimenter ces esprits qui font de nos erreurs leurs malignes délices ?
\section[{Chapitre III, Sa passion pour l’astrologie.}]{Chapitre III, Sa passion pour l’astrologie.}
\noindent \pn{4}Je ne cessais donc de consulter ces imposteurs, que l’on nomine astrologues, parce qu’ils semblaient n’offrir aucun sacrifice, ni adresser aucune prière aux esprits, pour la divination de l’avenir. Mais la véritable piété chrétienne repousse et condamne aussi leur science. C’est à vous, Seigneur, qu’il faut confesser et dire :\par

\begin{quoteblock}
\noindent « Ayez pitié de moi, guérissez mon âme, parce que j’ai péché contre vous (Ps. XI, 5). »\end{quoteblock}

\noindent  Et loin d’abuser de votre indulgence jusques au libertinage du péché, il faut avoir souvenir de cette parole du Seigneur\par

\begin{quoteblock}
\noindent « Voilà que tu es guéri, garde-toi de pécher désormais, de peur qu’il ne t’arrive pis (Jean, V, 14). C’est cette ordonnance salutaire qu’ils s’efforcent d’effacer, ceux qui disent : Le ciel vous forme une fatale nécessité de pécher. C’est à Vénus, c’est à Mars, c’est à Saturne qu’il faut s’en prendre. On veut ainsi que l’homme soit pur ; l’homme ! chair et sang, orgueilleuse pourriture ! on veut accuser Celui qui a créé les cieux et ordonne leurs mouvements. Et quel est-il, sinon vous-même, ô Dieu de douceur, source de justice, « qui rendez à chacun selon ses œuvres (Matth. XVI, 27) et ne méprisez pas un cœur contrit et humilié(PS. L, 19) ? »\end{quoteblock}

\noindent \pn{5}Je connaissais alors un homme d’un grand esprit, très-habile et très-célèbre dans la médecine ; j’avais reçu de sa main la couronne poétique ; mais c’était le proconsul, et non le médecin, qui avait couronné ma tête malade. Vous vous réservez la cure de ces maladies, ô vous,\par

\begin{quoteblock}
\noindent « qui résistez aux superbes et faites grâce aux humbles (I Pier. V,5) ! »\end{quoteblock}

\noindent  Et cependant, n’est-ce pas vous qui n’avez cessé de m’assister par ce vieillard, qui n’avez cessé par sa main de soigner mon âme ? J’étais entré dans son intimité, et ses entretiens, sans fard d’expression, mais sérieux et agréables par la vivacité des pensées, trouvaient en moi un auditeur attentif et assidu, Aussitôt qu’il apprit, dans nos entretiens, ma passion pour les livres d’astrologie, il me conseilla avec une bienveillance paternelle de les jeter là, pour ne pas accorder à ces futilités le soin que réclament les choses nécessaires. Il ajouta qu’il s’était livré sérieusement à cette étude dans ses premières années, et avait pensé d’en faire profession pour vivre ; que s’étant élevé à l’intelligence d’Hippocrate, il ne serait pas demeuré au-dessous de cette nouvelle étude, et ne l’avait finalement abandonnée pour la médecine, que parce qu’en reconnaissant toutes les erreurs, sa probité lui avait défendu de tromper les hommes pour gagner sa vie. — Mais vous, me dit-il, qui pour vivre honorablement avez la rhétorique, vous qu’une libre curiosité, et non le besoin de l’existence, attache à ces mensonges, vous pouvez m’en croire, puisque je n’ai approfondi ces malheureuses connaissances que pour en faire mon gagne-pain.\par
Je lui demandai d’où venait que plusieurs prédictions se trouvassent véritables, et il me répondit, comme il put, qu’il fallait l’attribuer à la puissance du sort, universellement répandue dans la nature. Vous consultez un poète au hasard, disait-il, vous feuilletez ses chants, dans une intention bien éloignée de celle qui les inspire, et vous trouvez souvent une conformité merveilleuse à votre pensée ; il ne faut donc pas s’étonner qu’une âme humaine, émue d’un instinct supérieur, sans savoir ce qui se passe en elle, par hasard et non par science, rende parfois un son qui s’accorde à l’état et à la conduite d’une autre âme.\par
\pn{6}Voilà ce que j ‘appris de lui, ou de vous par lui ; et ce que plus tard je devais rechercher par moi-même, vous l’avez esquissé d’un premier trait dans ma mémoire. Car alors, ni lui, ni mon cher Nébridius, sage et excellent jeune homme, plein de mépris railleurs pour cet art   divinatoire, ne purent me persuader de le rejeter ; je cédais à l’autorité de ceux qui en ont écrit, et je n’avais point encore trouvé de raison certaine., telle que j’en cherchais, qui me prouvât à l’évidence que le hasard, et non le calcul des mouvements célestes, décidait de la vérité de ces prédictions.
\section[{Chapitre IV, Mort d’un ami.}]{Chapitre IV, Mort d’un ami.}
\noindent \pn{7}En ces premières années de mon enseignement dans ma ville natale, je m’étais fait un ami, que la parité d’études et d’âge m’avait rendu bien cher ; il fleurissait comme moi sa fleur d’adolescence. Enfants, nous avions grandi ensemble ; nous avions été à l’école, nous avions joué ensemble. Mais il ne m’était pas alors aussi cher que depuis, quoique notre amitié n’ait jamais été vraie ; car l’amitié n’est pas vraie si vous ne la liez vous-même entre ceux qui s’attachent à vous\par

\begin{quoteblock}
\noindent « par la charité, que répand dans nos cœurs l’Esprit-Saint qui nous est donné (Rom. V,5) »\end{quoteblock}

\noindent  Et pourtant, elle m’était bien douce cette liaison entretenue au foyer des mêmes sentiments. Je l’avais détourné de la vraie foi, dont son enfance n’avait pas été profondément imbue, pour l’amener à ces fables de superstition et de mort qui coûtaient tant de larmes à ma mère. Il s’égarait d’esprit avec moi, cet homme dont mon âme ne pouvait plus se passer. Mais vous voilà ! … toujours penché sur la trace de vos fugitifs, Dieu des vengeances et source des miséricordes, qui nous ramenez à vous par des voies admirables… vous voilà ! et vous retirez cet homme de la vie ; à peine avions-nous fourni une année d’amitié, amitié qui m’était douce au delà de tout ce que mes jours d’alors ont connu de douceur !\par
\pn{8}Quel homme pourrait énumérer, seul, les trésors de clémence dont, à lui seul, il a fait l’épreuve ? Que fites-vous alors, ô Dieu, et combien impénétrable est l’abîme de vos jugements ? Dévoré de fièvre, il gisait sans connaissance dans une sueur mortelle. On désespéra de lui, et il fut baptisé à son insu, sans que je m’en misse en peine, persuadé qu’un peu d’eau répandue sur son corps insensible ne saurait effacer de son âme les sentiments que je lui avais inspirés. Il en fut autrement ; il se trouva mieux, et en voie de salut. Et aussitôt que je pus lui parler (ce qui me fut possible aussitôt qu’il put parler lui-même, car je ne le quittais pas, tant nos deux existences étaient confondues), je voulus rire, pensant qu’il rirait avec moi de ce baptême qu’il avait reçu en absence d’esprit et de sentiment : il savait alors l’avoir reçu. Et il eut horreur de moi, comme d’un ennemi, et soudain, avec une admirable liberté, il me commanda, si je voulais demeurer son ami, de cesser ce langage. Surpris et troublé, je contins tous les mouvements de mon âme, attendant que sa convalescence me permît de l’entreprendre à mon gré. Mais il fut soustrait à ma folie, pour être réservé dans votre sein à ma consolation. Peu de jours après, en mon absence, la fièvre le reprend et il meurt.\par
\pn{9}La douleur de sa perte voila mon cœur de ténèbres. Tout ce que je voyais n’était plus que mort. Et la patrie m’était un supplice, et la maison paternelle une désolation singulière. Tous les témoignages de mon commerce avec lui, sans lui, étaient pour moi un cruel martyre. Mes yeux le demandaient partout, et il m’était refusé. Et tout m’était odieux, parce que tout était vide de lui, et que rien ne pouvait plus me dire : Il vient, le voici ! comme pendant sa vie, quand il était absent. J’étais devenu un problème à moi-même, et j’interrogeais mon âme, « pourquoi elle était triste et me troublait ainsi, » et elle n’imaginait rien à me répondre. Et si je lui disais :\par

\begin{quoteblock}
\noindent « Espère en Dieu (Ps. XLI, 6), »\end{quoteblock}

\noindent  elle me désobéissait avec justice, parce qu’il était meilleur et plus vrai, cet homme, deuil de mon cœur, que ce fantôme en qui je voulais espérer. Le seul pleurer m’était doux, seul charme à qui mon âme avait donné la survivance de mon ami.
\section[{Chapitre V, Pourquoi les larmes sont-elles douces aux affligés ?}]{Chapitre V, Pourquoi les larmes sont-elles douces aux affligés ?}
\noindent \pn{10}Et maintenant, Seigneur, tout cela est passé ; et le temps a soulagé ma blessure. Puis-je approcher de votre bouche l’oreille de mon cœur ? . O vous, qui êtes la vérité, me direz-vous : Pourquoi les larmes sont douces aux malheureux ? — Mais peut-être, quoique présent partout, avez-vous rejeté loin de vous notre misère ? Et vous demeurez en vous-même, tandis que nous roulons dans l’instabilité. Et pourtant, si votre oreille ne s’inclinait   à nos pleurs, que resterait-il de notre espérance ? D’où vient donc que l’on cueille à l’arbre amer de la vie ces fruits si doux de gémissements, de pleurs, de soupirs et de plaintes ? Qui leur donne cette saveur ? Est-ce l’espérance que vous nous entendez ? Cela est vrai de la prière, mue du désir d’arriver jusqu’à vous. Mais quoi de semblable dans une telle affliction, dans cette funèbre douleur où j’étais enseveli ? Je n’espérais pas le voir revivre, mes pleurs ne demandaient pas ce retour ; je gémissais pour gémir, je pleurais pour pleurer. Car j’étais malheureux, j’avais perdu la joie de mon âme. Serait-ce donc qu’affadi de regrets, dans l’horreur où le plonge une perte chère, le cœur se réveille au goût amer des larmes ?
\section[{Chapitre VI, Violence de sa douleur.}]{Chapitre VI, Violence de sa douleur.}
\noindent \pn{11}Eh ! pourquoi toutes ces paroles ? Ce n’est pas le temps de vous interroger, mais de se confesser à vous. J’étais malheureux, et malheureux le cœur enchaîné de l’amour des choses mortelles ! Leur perte le déchire, et il sent alors cette réalité de misère qui l’opprimait avant même qu’il les eût perdues. Voilà comme j’étais alors, et je pleurais amèrement, et je me reposais dans l’amertume. Ainsi j’étais malheureux, et cette malheureuse vie m’était encore plus chère que mon ami. Je l’eusse voulu changer, mais non la perdre plutôt que de l’avoir perdu, lui. Et je ne sais si j’eusse voulu me donner pour lui, comme on le dit, pure fiction peut-être, d’Oreste et de Pylade, jaloux de mourir l’un pour l’autre ou ensemble, parce que survivre était pour eux pire que la mort. Mais je ne sais quel sentiment bien différent s’élevait en moi ; profond dégoût de vivre et crainte de mourir. Je crois que, plus je l’aimais, plus la mort qui me l’avait enlevé, m’apparaissait comme une ennemie cruelle, odieuse, terrible ; prête à dévorer tous les hommes, puisqu’elle venait de l’engloutir. Ainsi j’étais alors ; oui, je m’en souviens.\par
O mon Dieu ! voici mon cœur ; le voici ! voyez dedans tous mes souvenirs ; ô vous ! mon espérance, qui me purifiez des souillures de telles affections, élevant mes yeux jusqu’à vous, et débarrassant mes pieds de ces entraves (Ps. XXIV, 15). Je m’étonnais de voir vivre les autres mortels, parce qu’il était mort, celui que j’avais aimé comme s’il n’eût jamais dû mourir ; et je m’étonnais encore davantage, lui mort, de vivre, moi, qui étais un autre lui-même. II parle bien de son ami le poète qui i’appelle : Moitié de mon âme (Horac. Od. liv. II, ch. VI). Oui, j’ai senti que son âme et la mienne n’avaient été qu’une âme en deux corps ; c’est pourquoi la vie m’était en horreur, je ne voulais plus vivre, réduit à la moitié de moi-même. Et peut-être ne craignais-je ainsi de mourir, que de peur d’ensevelir tout entier celui que j’avais tant aimé (Rétr. Liv. II, ch. VI).
\section[{Chapitre VII, Il quitte Thagaste.}]{Chapitre VII, Il quitte Thagaste.}
\noindent \pn{12}O démence ! qui ne sait pas aimer les hommes selon l’homme. Homme insensé que j’étais alors, si impatient des afflictions humaines ! Oppressé, troublé, je soupirais, je pleurais, incapable de repos et de conseil ; je portais mon âme déchirée et sanglante, et qui ne voulait plus se laisser porter par moi, et je ne savais où la poser. Le charme des bois, les jeux et les chants , l’air embaumé , les banquets splendides, les voluptés du lit et de la table, la lecture, la poésie, rien ne pouvait la distraire. Tout m’était en horreur ; la lumière elle-même ; et tout ce qui n’était pas lui m’était odieux et nuisible, hormis les gémissements et les larmes, qui seuls donnaient quelque repos à ma douleur.\par
Et dès qu’une distraction en éloignait mon âme, je pliais sous le fardeau de ma misère, que vous seul, Seigneur, pouviez soulever et guérir. Je le savais, mais je manquais de volonté et de force, d’autant plus que vous n’étiez à ma pensée rien de solide ni de certain. Ce n’était pas vous, mais un vain fantôme, mais mon erreur, qui était mon Dieu. Vainement je voulais y appuyer mon âme ; elle manquait dans ce vide et retombait sur moi, Et je me restais à moi-même mon unique lieu, lieu de malheur, où je ne pouvais rester, et dont je ne pouvais sortir. Où mon cœur se fût-il enfui de mon cœur ? où me serais-je précipité hors de moi-même ? où me serais-je dérobé à ma poursuite ? Et cependant j ‘abandonnai ma patrie ; car mes yeux le cherchaient moins où ils n’étaient pas accoutumés à le voir, et de Thagaste je vins à Carthage.
 \section[{Chapitre VIII, Sa douleur diminue avec le temps.}]{Chapitre VIII, Sa douleur diminue avec le temps.}
\noindent \pn{13}Le temps n’est pas oisif ; et nos sentiments portent la trace de son cours ; il fait dans notre âme de merveilleuses œuvres. Et il venait, il passait jour à jour, et son flot m’apportait d’autres images, d’autres souvenirs, et me rendait peu à peu le goût de mes premières joies ; ma douleur se repliait devant elles : et c’étaient, sinon de nouvelles douleurs, du moins des germes d’afflictions futures que je semais en moi. Car la douleur eût-elle si facilement pénétré dans l’intimité de mon être, si je n’avais répandu mon âme sur le sable, en aimant un mortel comme s’il ne devait pas mourir ? Or, je trouvais distraction et soulagement dans les consolations de mes amis qui aimaient avec moi ce que j’aimais au lieu de vous. Longue fiction, long mensonge, voluptés adultères de l’esprit, stimulées par le commerce de la parole. Mais si l’un de mes amis venait à mourir, ce mensonge ne laissait pas de vivre.\par
Ces liaisons s’emparaient de mon âme par des charmes encore plus puissants ; échanges de doux propos, d’enjouement, de bienveillants témoignages ; agréables lectures, badinages honnêtes, affectueuses civilités ; rares dissentiments, sans aigreur, comme on en a avec soi-même ; léger assaisonnement de contradiction, sel qui relève l’unanimité trop constante ; instruction réciproque ; impatients regrets des amis absents, joyeux accueil à leur bienvenue.\par
Tous ces doux témoignages que les cœurs amis expriment de l’air, de la langue, des yeux, par mille mouvements pleins de caresses, sont comme autant de foyers où les esprits se fondent et se réduisent à l’unité.
\section[{Chapitre IX, L’amitié n’est vraie qu’en Dieu.}]{Chapitre IX, L’amitié n’est vraie qu’en Dieu.}
\noindent \pn{14}Voilà ce que l’on aime dans les amis, ce qu’on aime de tel amour, que la conscience humaine se trouve coupable de ne pas rendre affection pour affection ; elle ne veut de la personne aimée que le témoignage d’une affection partagée. De là le deuil des morts chéris, les ténèbres de la douleur, les douces jouissances changées en amertume dans le cœur plein de larmes, et la perte de la vie en ceux qui meurent devenant la mort des vivants. Heureux qui vous aime, et son ami en vous, et son ennemi pour vous ! Celui-là seul ne perd aucun être cher, à qui tous sont chers en celui qui ne se perd jamais. Et quel est-il, sinon notre Dieu, Dieu qui a fait le ciel et la terre, qui les remplit, et en les remplissant les a faits ? Et personne ne vous perd que celui qui vous quitte. Et celui qui vous quitte, où va-t-il, où se réfugie-t-il, sinon de vous en vous, de votre amour dans votre colère ? Où pourra-t-il ne pas trouver votre loi dans sa peine ? car votre loi est la vérité, et la vérité, c’est vous.
\section[{Chapitre X, L’âme ne peut trouver son repos dans les créatures.}]{Chapitre X, L’âme ne peut trouver son repos dans les créatures.}
\noindent \pn{15}\par

\begin{quoteblock}
\noindent « Dieu des vertus, convertissez-nous, montrez-nous votre face, et nous serons sauvés (Ps. LXXIX, 4).»\end{quoteblock}

\noindent Hors de vous, où peut se tourner l’âme de l’homme, sans poser sur une douleur, quelle que soit la beauté des créatures, où, loin d’elle et de vous, elle cherche son repos ? Mais elles ne seraient rien, si elles n’étaient par vous, ces beautés qui se lèvent et se couchent. En se levant, elles commencent d’être, elles croissent pour atteindre leur perfection ; arrivées là, elles vieillissent et meurent ; car tout vieillit et tout meurt. Ainsi, aussitôt nées, elles tendent à être, et plus elles s’empressent de croître afin d’être, plus elles se hâtent de n’être plus. Telle est la condition de leur existence. Voilà la part que vous leur avez faite ; elles sont d’un ensemble de choses qui ne coexistent jamais toutes à la fois, mais qui par leur fuite et leur succession produisent ce tout dont elles sont partie. Et n’est-ce pas ainsi que notre discours s’accomplit par les signes et les sons ? Jamais il n’existera en totalité, si chaque parole ne passe, après avoir prononcé son rôle, pour qu’une autre lui succède.\par
Que mon âme vous loue de telles œuvres, Dieu leur créateur, mais qu’elle n’y demeure point attachée par l’appât de cet amour qui captive les sens ; car elles vont toujours où elles allaient, pour ne plus être, et déchirent de désirs pernicieux l’âme avide d’être et de se reposer dans ce qu’elle aime. Mais l’âme peut-elle trouver son repos dans leur instabilité ?   Elles fuient, et l’instant même de leur présence se dérobe au sens charnel. Lent est le sens de la chair, parce qu’il est le sens de la chair. et la manière d’être de la chair. Il suffit à sa fin, mais il est impuissant pour saisir ce qui court d’un point désigné à un autre. Car votre Verbe créateur dit à l’être créé : Tu iras d’ici là.
\section[{Chapitre XI, Les créatures changent ; Dieu seul est immuable.}]{Chapitre XI, Les créatures changent ; Dieu seul est immuable.}
\noindent \pn{16}Ne sois pas vaine, ô mon âme ! prends garde de perdre l’ouïe du cœur dans le tumulte de tes vanités. Ecoute donc aussi : Le Verbe lui-même te crie de revenir ; là est le lieu du repos inaltérable, où l’amour n’est pas renoncé s’il ne renonce lui-même. Vois ; ces objets passent, d’autres leur succèdent, et de ces éléments particuliers se forme l’universalité de l’ordre inférieur. Et moi, est-ce que je passe ? dit le Verbe de Dieu. Fixe ici ta demeure place ici tout ce que tu as reçu d’ici, ô mon âme !, car tu dois être lasse de mensonges. Remets à la vérité tout ce que tu tiens de la vérité, et tu ne perdras rien ; tes plaies seront fermées, tes langueurs guéries, tout ton être éphémère rétabli, renouvelé, lié à toi-même ; il ne te portera plus au lieu où il descend ; mais il subsistera avec toi, appuyé à la stabilité permanente de Dieu.\par
\pn{17}Pourquoi t’égarer à suivre ta chair ? Elle-même, que ne revient-elle à te suivre ? Que connais - tu par elle ? Quelques parties d’un tout que tu ignores, et tu te complais en si peu ! Mais si le sens charnel était capable de comprendre ce tout, et s’il n’eût reçu pour ton châtiment de justes bornes, tes désirs hâteraient le passage de tout ce qui existe dans le présent, afin de jouir de l’ensemble. C’est par ce sens charnel que tu entends la parole, et tu ne demandes pas l’immobilité des syllabes, mais leur rapide écoulement, et l’arrivée des dernières pour entendre le tout. Et toutes choses forment un certain ensemble, non par coexistence, mais par Succession, et le tout a plus de charmes que la partie, quand il se laisse voir aux sens. Mais combien est plus excellent Celui qui a fait cet ensemble de toutes choses ? Et celui-là, c’est notre Dieu. Et il ne passe pas, parce que rien ne lui succède.
\section[{Chapitre XII, Les âmes trouvent en Dieu le repos et l’immutabilité.}]{Chapitre XII, Les âmes trouvent en Dieu le repos et l’immutabilité.}
\noindent \pn{18}Si les corps te plaisent, prends-en sujet de louer Dieu ; réfléchis ton amour vers leur Auteur, de peur qu’en t’arrêtant à ce qui te plaît, tu ne lui déplaises.\par
Si les âmes te plaisent, aime-les en Dieu. Muables en elles-mêmes, elles sont fixes et immuables en lui ; sans lui elles s’évanouiraient dans le néant. Qu’elles soient donc aimées en lui. Entraîne avec toi vers lui toutes celles que tu peux, et dis-leur : Aimons-le, aimons-le. Il a tout fait, et il n’est pas loin de ses créatures. Il ne s’est pas retiré après les avoir faites, mais c’est en lui comme de lui qu’elles ont leur être. Voici où il est ; où réside le goût de la vérité, dans l’intimité du cœur ; mais le cœur s’est détourné de lui,\par

\begin{quoteblock}
\noindent « Revenez à votre cœur, hommes de péchés (Isaïe, XLVI, 8) »\end{quoteblock}

\noindent  et rattachez-vous à Celui qui vous a faits. Demeurez avec lui, et vous serez debout. Reposez-vous en lui, et vous serez tranquilles.\par
Où allez-vous ? au milieu des précipices ? où allez-vous ? Le bien que vous aimez vient de lui. Bien véritable et doux tant que vous l’aimerez pour Dieu, il deviendra justement amer, si vous avez l’injustice de l’aimer sans son Auteur. Pourquoi marcher, marcher encore dans ces sentiers rudes et laborieux ? Le repos n’est pas où vous le cherchez. Cherchez votre recherche ; mais il n’est pas où vous cherchez. Vous cherchez la vie bienheureuse dans la région de la mort ; elle n’est pas là. Comment la vie bienheureuse serait-elle où la vie même n’est pas ?\par
\pn{19}Et notre véritable Vie est descendue ici-bas, et elle s’est chargée de notre mort, et elle a tué notre mort par l’abondance de sa vie. Et sa voix a retenti comme un tonnerre, afin que nous revinssions â lui dans le secret d’où il s’est élancé vers nous, quand, descendu dans le sein virginal, où il a épousé la créature humaine, la chair mortelle pour la soustraire à la mort,\par

\begin{quoteblock}
\noindent « il est sorti comme l’époux de sa couche, et comme un géant qui dévore sa carrière (Ps. XVIII, 6). »\end{quoteblock}

\noindent  Il ne s’est point arrêté, mais il a couru, criant par ses paroles, ses actions, sa mort, sa vie, sa descente souterraine et son ascension, que nous retournions à lui. Et il a   disparu de nos yeux, afin que, rentrant dans notre cœur, nous l’y trouvions. Il s’est retiré, et le voilà, il est ici. Il n’a pas voulu être longtemps avec nous, et il ne nous a pas quittés. il est retourné d’où il n’était jamais sorti ; car\par

\begin{quoteblock}
\noindent « le monde a été fait par lui ; et il était dans ce monde (Jean, I, 10), et dans ce monde il est venu sauver les pécheurs (I Tim. ,15) »\end{quoteblock}

\noindent  C’est de lui que mon âme implore sa guérison,\par

\begin{quoteblock}
\noindent « parce qu’elle a péché contre lui (Ps XL, 5). Fils des hommes, jusques à quand porterez-vous un cœur appesanti (Ps. IV, 3) ? »\end{quoteblock}

\noindent  La vie est descendue vers vous, et vous ne voulez pas monter vers elle et vivre ? Mais où monterez-vous, puisque vous êtes en haut, le front dans les cieux (Ps LXXII, 9) ? Descendez pour monter, pour monter jusqu’à Dieu : car vous êtes tombés en montant contre lui. Dis-leur cela, ô mon âme ! afin qu’ils pleurent dans cette vallée de larmes, dis, et emporte-les avec toi vers Dieu ; car tu parles par son Esprit, si ta parole est brûlante de charité.
\section[{Chapitre XIII, D’où procède l’amour, livres qu’il avait écrits sur la beauté et la convenance.}]{Chapitre XIII, D’où procède l’amour, livres qu’il avait écrits sur la beauté et la convenance.}
\noindent \pn{20}C’est ce que j’ignorais alors ; j’aimais les beautés inférieures ; et je descendais à l’abîme, et je disais à mes amis : Qu’aimons-nous qui ne soit beau ? Qu’est-ce donc que le beau ? et qu’est-ce que la beauté ? Quel est cet attrait qui nous attache aux objets de notre affection ? S’ils étaient sans charme et sans beauté, ils ne feraient aucune impression sur nous. Et je considérais que, dans les corps eux-mêmes, il faut distinguer ce qui en est comme le tout, et partant la beauté ; et ce qui plaît par un simple rapport de convenance, comme la proportion d’un membre au corps, d’une chaussure au pied, etc. Cette source de pensées jaillit dans mon esprit du plus profond de mon cœur, et j’écrivis sur le beau et le convenable deux ou trois livres, je crois ; vous le savez, mon Dieu, car cela m’est échappé. Je n’ai plus ces livres, ils se sont égarés, je ne sais comment.
\section[{Chapitre XIV, Il avait dédié ces livres à l’orateur Hiérius. — estime pour les absents : d’où vient-elle ?}]{Chapitre XIV, Il avait dédié ces livres à l’orateur Hiérius. — estime pour les absents : d’où vient-elle ?}
\noindent \pn{21}Eh ! qui put me porter alors, Seigneur mon Dieu, à les dédier à Hiérius, orateur de Rome ? je ne le connaissais pas même de vue ; je l’aimais sur sa brillante réputation de savoir, et l’on m’avait rapporté de lui certaines paroles qui m’avaient plu. Mais en réalité, l’estime des autres et l’enthousiasme que leur inspirait un Syrien, initié d’abord aux lettres grecques, pour devenir plus tard un modèle d’éloquence latine et d’érudition philosophique, voilà ce qui décidait mon admiration. Eh quoi ! on entend louer un homme, et on l’aime aussitôt, quoique absent ? Est-ce que l’amour passe de la bouche du panégyriste dans le cœur de l’auditeur ? non ; mais l’amour de l’un allume l’amour de l’autre. On aime l’objet de la louange lorsqu’on est assuré qu’elle part du cœur, et que l’affection la donne.\par
\pn{22}C’est ainsi que j’aimais alors les hommes, d’après le jugement des hommes, et non d’après le vôtre qui ne trompe jamais, ô mon Dieu ! Et toutefois mes éloges n’avaient rien de commun avec ceux que l’on accorde à un habile conducteur, à un chasseur de l’amphithéâtre honoré des suffrages populaires ; mon estime était d’un autre ordre, elle était grave, elle louait comme j’eusse désiré d’être loué moi-même. Or, je n’étais nullement jaloux d’être aimé et loué comme les histrions, quoique je fusse le premier à les louer et à les aimer ; je préférais l’obscurité à telle renommée, la haine même à telles faveurs. Mais comment peut se maintenir dans une même âme l’équilibre de ces affections différentes et contraires ? Comment puis-je aimer en cet homme ce que je hais en moi, ce que je repousse si loin de moi, homme comme lui ? Tu ne voudrais pas être, cela te fût-il possible, ce bon cheval que tu aimes ; mais en-peux-tu dire autant de l’histrion, ton semblable ? J’aime donc dans un homme ce que je haïrais d’être moi-même, tout homme que je suis ? Immense abîme que l’homme, dont les cheveux mêmes vous sont comptés, Seigneur, sans qu’un seul s’égare ; et il est encore plus aisé pourtant de les nombrer que les affections et les mouvements de son cœur !\par
\pn{23}Quant à ce rhéteur, le sentiment que j’avais pour lui était de nature à me faire envier   d’être ce qu’il était ; et mes vaniteuses présomptions m’égaraient ; et je flottais à tout vent, et je ne laissais pas d’être secrètement gouverné par vous. Et d’où ai-je appris, et comment puis-je vous confesser avec certitude que j’empruntais plutôt mon amour pour cet homme à l’amour de ses partisans qu’aux raisons mêmes de leurs éloges ? Si, en effet, au lieu de le louer on l’eût blâmé, et que ces sujets de louanges eussent été des sujets de censure et de mépris, j’eusse été loin de m’enflammer à son égard. Et cependant l’homme et les choses restaient les mêmes ; l’opinion seule était différente. Voilà où tombe l’âme infirme, qui ne se tient pas encore à la base solide de la vérité. Au souffle capricieux de l’opinion, elle va, elle plie, elle tourne et revient ; et la lumière se voile pour elle ; elle ne distingue plus la vérité, la vérité qui est devant elle !\par
Et c’était un triomphe pour moi, que mon discours et mes études vinssent à la connaissance de cet homme. S’il m’approuvait, je redoublais d’ardeur ; sinon, j’étais blessé dans mon cœur plein de vanité et vide de cette constance qui n’est qu’en vous. Et cependant je me plaisais toujours à méditer sur le beau et le convenable, sujet du livre que je lui avais adressé, et mon admiration louait, sans écho, ce monument de ma pensée.
\section[{Chapitre XV, Son esprit obscurci par les images sensibles ne pouvait concevoir les substances spirituelles.}]{Chapitre XV, Son esprit obscurci par les images sensibles ne pouvait concevoir les substances spirituelles.}
\noindent \pn{24}Mais je ne saisissais pas, dans les merveilles de votre art, le pivot de cette grande vérité, ô Tout-Puissant,\par

\begin{quoteblock}
\noindent « seul auteur de tant de merveilles (Ps LXXI, 18) »\end{quoteblock}

\noindent  et mon esprit se promenait parmi les formes corporelles, distinguait le beau et le convenable, définissait l’un, ce qui est par soi-même ; l’autre, ce qui a un rapport de proportion avec un objet ; principes que j’établissais sur des exemples sensibles. Et je portais mes pensées sur la nature de l’esprit, et la fausse idée que j’avais des êtres spirituels ne me permettait pas de voir la vérité ; et son éclat même pénétrait mes yeux, et je détournais mon âme éblouie de la réalité incorporelle pour l’attacher aux linéaments, aux couleurs, aux grandeurs palpables.\par
Et comme je ne pouvais rien voir de tel dans mon esprit, je croyais impossible de le saisir lui-même. Mais apercevant dans la vertu une paix aimable, dans le vice une discorde odieuse ; là, je remarquais l’unité ; ici, la division. Et dans cette unité, je plaçais l’âme raisonnable, l’essence de la vérité et du souverain bien ; dans cette division, je ne sais quelle substance de vie irraisonnable, je ne sais quelle essence de souverain mal, dont je faisais non-seulement une réalité, mais une véritable vie, un être indépendant de vous, mon Dieu, de vous, de qui toutes choses procèdent. Misérable rêveur, j’appelais l’une Monas, spiritualité sans sexe ; l’autre Dyas, principe des colères homicides, des emportements, de la débauche ; et je ne savais ce que je disais. J’ignorais et n’avais pas encore appris que nulle substance n’est le mal, et que notre principe intérieur n’est pas le bien souverain et immuable.\par
\pn{25}Il y a violence criminelle, quand l’esprit livre son activité à un mouvement pervers, quand il soulève les flots turbulents de sa fureur ; libertinage, quand l’âme ne gouverne plus l’inclination qui l’entraîne aux voluptés charnelles. Et de même cette rouille du préjugé et de l’erreur qui flétrit la vie, vient d’un dérèglement de la raison. Tel était alors l’état de la mienne. Car j’ignorais qu’elle dût être éclairée d’une autre lumière pour participer de la vérité, n’étant pas elle-même l’essence de la vérité.\par

\begin{quoteblock}
\noindent « C’est vous qui allumerez ma lampe, Seigneur mon Dieu ; c’est vous qui éclairerez mes ténèbres (Ps. XVII, 29) et tous, nous avons reçu de votre plénitude, parce que vous êtes la vraie lumière qui éclaire tout homme venant en ce monde (Jean I, 16,9), lumière sans vicissitudes et sans ombre (Jacq. I, 17). »\end{quoteblock}

\noindent \pn{26}Mais je faisais effort vers vous, et vous me repoussiez loin de vous, afin que je goûtasse la mort ; car vous résistez aux superbes. Et quoi de plus superbe que cette démence inouïe qui prétend être naturellement ce que vous êtes ? Sujet au changement, et le sentant bien à mon désir d’être sage pour devenir meilleur, j’aimais mieux vous supposer muable que de n’être pas moi-même ce que vous êtes. Vous me repoussiez donc, et vous résistiez à l’extravagance de mes pensées, et j’imaginais à loisir des formes corporelles ; chair, j’accusais la chair ; esprit égaré et ne revenant pas encore à vous (Ps. LXXVII, 39), j’allais, je me promenais dans un monde   imaginaire, d’êtres qui ne sont ni en vous, ni en moi, ni dans les corps ; et ce n’étaient point les créations de votre Vérité, mais les fictions de ma vanité que je formais sur les corps. Et je disais à vos simples enfants, aux fidèles, mes concitoyens, dont alors j’étais séparé par un exil que j’ignorais, je leur disais avec ma sotte loquacité : Comment mon âme, créature de Dieu, est-elle dans l’erreur ? Et je ne pouvais souffrir que l’on me répondît : Comment Dieu est-il dans l’erreur ? Et je soutenais que votre immuable nature était entraînée dans l’erreur plutôt que de reconnaître que la mienne, muable, et volontairement égarée, subissait l’erreur comme la peine de son crime.\par
\pn{27}J’avais vingt-six à vingt-sept ans, lorsque j’écrivis ces livres ; et je roulais dans ma fantaisie ces inanités d’images, bourdonnantes à l’oreille de mon cœur. Et je voulais pourtant, ô douce vérité, la rendre attentive à l’ouïe intérieure de vos mélodies, quand je méditais sur la beauté et la convenance, jaloux de me tenir devant vous, de vous entendre pour frémir d’allégresse comme à la voix de l’époux(Jean, III, 29) et je ne le pouvais, car la voix de l’erreur m’entraînait hors de moi, et le poids de mon orgueil me précipitait dans l’abîme. Vous ne donniez pas alors la joie et l’allégresse à mon entendement, et mes os ne tressaillaient pas, n’étant point encore humiliés (Ps. L, 10).
\section[{Chapitre XVI, Génie de Saint Augustin.}]{Chapitre XVI, Génie de Saint Augustin.}
\noindent \pn{28}Et de quoi me servait alors qu’à l’âgé de vingt ans environ, ayant eu entre les mains ce livre d’Aristote, qu’on appelle les dix catégories, je le compris seul à la simple lecture ? Et cependant à ce nom de catégories, les joues du rhéteur de Carthage, mon maître, se gonflaient d’emphase, et plusieurs autres réputés habiles avaient également éveillé en moi comme une attente inquiète de quelque chose d’extraordinaire et de divin. J’en conférai depuis avec d’autres qui disaient n’avoir compris cet ouvrage qu’à grand’peine, à l’aide d’excellents maîtres, non-seulement par enseignement de vive voix, mais par des figures tracées sur le sable, et ils ne m’en purent rien apprendre que ma lecture solitaire ne m’eût fait connaître.\par
Et ces catégories me semblaient parler assez clairement des substances, l’homme par exemple ; et de ce qui est en elles, comme la figure de l’homme ; quel il est, quelle est sa taille, sa hauteur ; de qui il est frère ou parent ; où il est établi ; quand il est né ; s’il est debout, assis ; chaussé ou armé ; actif ou passif ; tout ce qui est enfin compris, soit dans ces neuf genres, dont j’ai touché quelques exemples, soit dans le genre lui-même de la substance, où les exemples sont innombrables.\par
\pn{29}Quel bien me faisait ou plutôt quel mal ne me faisait pas cette connaissance ? Je voulais que tout ce qui est fût compris dans ces dix prédicaments ; et vous-même, comment vous concevais-je, ô mon Dieu, simplicité, immutabilité parfaite ? Ma pensée matérielle se figurait votre grandeur et votre beauté réunies en vous comme l’accident dans le sujet ; comme si vous n’étiez pas vous-même votre grandeur et votre beauté, tandis que le corps ne tient pas de son essence corporelle sa grandeur et sa beauté ; car, fût-il moins grand et moins beau, en serait-il moins corps ? Chimère que tout ce que je pensais de vous, et non vérité ; inventions de ma misère, et non réalités de votre béatitude ! Et votre ordre s’accomplissait en moi : la terre me produisait des chardons et des ronces ; je ne pouvais arriver qu’au prix de mes sueurs à gagner mon pain (Gen. III, 18, 19).\par
\pn{30}Et que me servait encore d’avoir lu et compris seul tout ce que j’avais pu lire de livres sur les arts qu’on appelle libéraux, infâme esclave de mes passions ! Je me complaisais dans ces lectures, sans reconnaître d’où venait tout ce qu’il y avait de vrai et de certain. Je tournais le dos à la lumière, la face aux objets éclairés, et mes yeux qui les voyaient lumineux, ne recevaient pas eux-mêmes le rayon. Tout ce que j’ai compris, sans peine et sans maître, de l’art de parler et de raisonner, de la géométrie, de la musique et des nombres, vous le savez, Seigneur mon Dieu ; la promptitude de l’intelligence et la vivacité du raisonnement sont des dons de votre libéralité ; mais au lieu de vous en faire un sacrifice, je ne m’en suis servi que pour ma perte. J’ai revendiqué la meilleure part de mon héritage, je n’ai pas conservé ma force pour vous (Ps. LVIII, 10) et « loin de vous dans une terre étrangère » je l’ai prodiguée aux caprices des passions, ces folles courtisanes (Luc, XV, 12, 13, 30). Pour si mauvais usage  , que me servait un tel bien ? Car je ne m’apercevais des difficultés que ces sciences offraient aux esprits les plus vifs et les plus studieux, qu’en cherchant à leur en donner les solutions ; et le plus intelligent, c’était le moins lent à me suivre dans mes explications.\par
Et que m’en revenait-il encore, puisque je vous considérais, Seigneur mon Dieu, vérité suprême, comme un corps lumineux et immense, et moi comme un fragment de ce corps ? O excès de perversité ! voilà donc où j’en étais ! Et je ne rougis pas, mon Dieu, de confesser vos miséricordes sur moi, et de vous invoquer, moi qui ne rougissais pas alors de professer publiquement mes blasphèmes et d’aboyer contre vous. Et que me servait ce génie qui dévorait la science ? que me servait d’avoir, sans nulle assistance de maîtres, dénoué les plus inextricables ouvrages, quand une honteuse et sacrilége ignorance m’entraînait si loin des doctrines de la piété ? Et quel obstacle était-ce pour vos petits que la lenteur de leur esprit, si, demeurant toujours près de vous, ils attendaient en sûreté au nid de votre Église la venue de leurs plumes, ces ailes de la charité que fait croître l’aliment d’une foi sainte ?\par
O Seigneur, ô mon Dieu !\par

\begin{quoteblock}
\noindent « espérons en l’abri de vos ailes (Ps. LXII, 8) »\end{quoteblock}

\noindent  protégez-nous, portez-nous. Vous nous porterez tout petits,\par

\begin{quoteblock}
\noindent « et vous nous porterez jusqu’aux cheveux blancs (Is. XLVI, 4) »\end{quoteblock}

\noindent  car notre force n’est force qu’avec vous ; elle n’est que faiblesse quand nous ne sommes qu’avec nous-mêmes. Tout notre bien vit en vous, et-notre rupture avec vous a fait notre corruption. Retournons à vous, Seigneur, pour n’être plus mortellement détournés. C’est en vous que vit notre bien, bien parfait, qui est vous-même. Craindrons-nous de ne plus retrouver au retour la demeure dont nous nous sommes précipités ? S’est-elle écroulée en notre absence cette demeure, qui est votre éternité ?
\chapterclose


\chapteropen
 \chapter[{V. La vingt-neuvième année}]{V. La vingt-neuvième année}\phantomsection
\label{V}\renewcommand{\leftmark}{V. La vingt-neuvième année}


\begin{argument}\noindent Il se dégoûte des doctrines manichéennes à l’âge de vingt-neuf ans. — Il va à Rome, puis à Milan pour enseigner la rhétorique. — Ayant entendu saint Ambroise, il rompt avec les manichéens et demeure catéchumène dans l’Église.
\end{argument}


\chaptercont
\section[{Chapitre premier, Que mon âme vous loue, Seigneur, pour vous aimer.}]{Chapitre premier, Que mon âme vous loue, Seigneur, pour vous aimer.}
\noindent \pn{1}Recevez le sacrifice de mes confessions, cette offrande de ma langue, formée, excitée par vous à confesser votre nom. Guérissez toutes les puissances de mon âme ; qu’elles s’écrient :\par

\begin{quoteblock}
\noindent « Seigneur, qui est semblable à vous ? (Ps XXXIV, 10) »\end{quoteblock}

\noindent  Celui qui se confesse à vous, ne vous apprend rien de ce qui se passe en lui ; car votre regard ne reste pas à la porte d’un cœur fermé, et votre main n’est pas repoussée par la dureté des hommes ; votre miséricorde ou votre justice la rompt, quand il vous plaît ;\par

\begin{quoteblock}
\noindent « et personne ne se peut dérober à votre chaleur (Ps XVIII, 7).»\end{quoteblock}

\noindent Que mon âme vous loue pour vous aimer ; qu’elle confesse vos miséricordes pour vous louer ! Votre création est un hymne permanent en votre honneur ; les esprits, par leur propre bouche ; les êtres animés et les êtres corporels, par la bouche de ceux qui les contemplent, publient vos louanges ; et notre âme se réveille de ses langueurs, elle se soulève vers vous en s’appuyant sur vos œuvres, pour arriver jusqu’à vous, Artisan de tant de merveilles ; là, est sa vraie nourriture ; là, sa véritable force.
\section[{Chapitre II, Où fuit l’impie, en fuyant Dieu ?}]{Chapitre II, Où fuit l’impie, en fuyant Dieu ?}
\noindent \pn{2}Où vont, où fuient loin de vous ces hommes sans repos et sans équité ? Vous les voyez ; votre regard perce leurs ténèbres ; laideur obscure qui fait ressortir la beauté de l’ensemble. Quel mal ont-ils pu vous faire ? Quelle atteinte porter à votre empire qui demeure dans sa justice et son inviolabilité du plus haut des cieux au plus profond des abîmes ? Où ont-ils fui, en fuyant votre face ? Où pouvaient-ils vous échapper ? Ils ont fui, pour ne pas voir Celui qui les voit ; pour ne vous rencontrer qu’étant aveugles.\par

\begin{quoteblock}
\noindent « Vous n’abandonnez rien de ce que vous avez fait (Sag. XI, 25) »\end{quoteblock}

\noindent Les injustes vous ont rencontré, pour leur juste supplice ; ils se sont dérobés à votre douceur, pour trouver votre rectitude et tomber dans votre âpreté. Ils ignorent que vous êtes partout, vous, que le lieu ne comprend pas, et que seul vous êtes présent même à ceux qui vous fuient.\par
Qu’ils se retournent donc et qu’ils vous cherchent ; car pour être abandonné de ses créatures, le Créateur ne les abandonne pas. Qu’ils se retournent, et qu’ils vous cherchent ! Mais vous êtes dans leur cœur ; dans le cœur de ceux qui vous confessent, qui ‘se jettent dans vos bras, qui pleurent dans votre sein au retour de leurs pénibles voies. Père tendre, vous essuyez leurs larmes, et ils pleurent encore, et ils trouvent leur joie dans ces pleurs ; car, ce n’est pas un homme de chair et de sang, mais vous-même, Seigneur, qui les consolez, vous, leur Créateur, qui les créez une seconde fois ! Et où étais-je, quand je vous cherchais ? Et vous étiez devant moi ; mais absent de moi-même, et ne me trouvant pas, que j’étais loin de vous trouver !
\section[{Chapitre III, Faustus. — aveuglement des philosophes.}]{Chapitre III, Faustus. — aveuglement des philosophes.}
\noindent \pn{3}Je vais parler, en présence de mon Dieu, de la vingt-neuvième année de mon âge. Il y avait alors à Carthage un évêque manichéen,   nommé Faustus, grand lacet du diable, qui avait fait tomber plusieurs à l’appât de son éloquence. Tout en l’admirant, je savais néanmoins la distinguer des vérités que j’étais avide d’apprendre : et je regardais moins au vase du discours, qu’au mets de science que ce célèbre Faustus servait à mon esprit. Car sa réputation me l’avait annoncé comme riche en savoir et profond dans les sciences libérales.\par
Et comme j’avais lu un grand nombre de philosophes, et retenu leurs doctrines, j’en comparais quelques-unes avec ces longues rêveries des Manichéens, et je trouvais plus de probabilité aux sentiments, de ceux qui\par

\begin{quoteblock}
\noindent « ont pu pénétrer dans l’économie du monde, quoiqu’ils n’en aient jamais trouvé le Maître (Sag. XIII, 1). Car vous êtes grand, Seigneur, vous approchez votre regard des abaissements et vous l’éloignez des hauteurs (Ps CXXXVII, 6) ; »\end{quoteblock}

\noindent  vous ne vous découvrez qu’aux cœurs contrits, et vous êtes impénétrable aux superbes ; leur curieuse industrie sût-elle d’ailleurs le compte des étoiles et des grains de sable, la mesure de l’étendue céleste, eût-elle exploré la route des astres !\par
\pn{4}C’est par leur esprit, c’est par le génie que vous leur avez donné, qu’ils cherchent ces secrets ; ils en découvrent beaucoup ; ils annoncent plusieurs années d’avance les éclipses de soleil et de lune, et le jour, et l’heure, et le degré ; et leur calcul ne les trompe pas, et il arrive selon leurs prédictions, et ils ont écrit les lois de leurs découvertes qu’on lit encore aujourd’hui, et qui servent a prédire quelle année, quel mois de l’année, quel jour du mois, quelle heure du jour, en quel point de son disque la lune ou le soleil doit subir une éclipse, et il arrivera comme il est prédit.\par
Et les hommes admirent, les ignorants sont dans la stupeur, et les savants se glorifient et s’élèvent. Et, dans leur superbe impie, ils se retirent de votre lumière ; infaillibles prophètes des éclipses du soleil, ils ne se doutent pas. de celle qu’ils souffrent eux-mêmes à cette heure. Ils ne recherchent pas avec une pieuse reconnaissance de qui ils tiennent ce génie de recherche. Et. s’ils vous découvrent comme leur auteur, ils ne se donnent pas à vous, pour que vous conserviez votre ouvrage ; et ils ne vous immolent pas l’homme qu’ils ont fait en eux, ils, ils ne vous offrent en sacrifice ni ces oiseaux de leurs téméraires pensées, ni ces monstres de leur curiosité qui leur font une voie secrète aux profondeurs de l’abîme, ni ces boucs de leurs impudicités, afin que votre feu, Seigneur, dévore toute cette mort palpitante, et les engendre à l’immortalité.\par
\pn{5}Mais ils ne savent pas la voie, votre Verbe, par qui vous avez fait tous les objets qu’ils nombrent, et eux-mêmes qui les nombrent, et le sens qui leur découvre ce qu’ils nombrent, et l’esprit qui leur donne la capacité de nombrer ;\par

\begin{quoteblock}
\noindent « votre sagesse seule exclut le nombre (Ps. CXLVI, 5).»\end{quoteblock}

\noindent  Et votre Fils unique s’est fait notre sagesse, notre justice et notre sanctification (I Cor. I, 30) : il a été nombré parmi nous, il a payé le tribut à César (Matth. XXII, 21). Oh ! ils ne savent pas cette voie qui fait descendre de soi-même vers lui, pour monter par lui jusqu’à lui ! Ils ne savent pas cette voie, et ils se croient élevés et rayonnants comme les astres, et les voilà froissés contre terre ;\par

\begin{quoteblock}
\noindent « et les ténèbres ont envahi la folie de leur cœur (Rom. I, 21) ! »\end{quoteblock}

\noindent  Ils disent sur la créature beaucoup de vérités, et ils necherchent pas avec piété la Vérité créatrice ; c’est pourquoi ils ne la trouvent pas ; ou s’ils la trouvent,\par

\begin{quoteblock}
\noindent « ils la reconnaissent pour Dieu, sans l’honorer comme Dieu, sans lui rendre grâces ; mais ils se dissipent dans la vanité de leurs pensées, et ils se disent sages en s’appropriant ce qui est à vous, et, en retour, leur aveugle perversité vous attribue ce qui leur appartient ; ils vous chargent de leurs mensonges, vous qui êtes la vérité ; « ils transforment la gloire du Dieu incorruptible en la ressemblance et l’image de l’homme corruptible, des oiseaux, des quadrupèdes et des serpents ; ils changent votre vérité en mensonge ; ils adorent et servent la créature de préférence au Créateur (Rom. I, 21-25).»\end{quoteblock}

\noindent \pn{6}Ces hommes, néanmoins, m’avaient révélé beaucoup de vérités naturelles, et j’en saisissais la raison par l’ordre et le calcul des temps, par les visibles témoignages des astres ; et je comparais ces observations aux discours de Manès qui a écrit sur ce sujet de longues extravagances où je ne trouvais la raison ni des solstices, ni des équinoxes, ni des éclipses, ni d’aucun phénomène dont la philosophie du siècle avait su m’informer. Et j’étais tenu de croire à des rêveries en désaccord parfait avec les règles mathématiques et l’observation de mes yeux.  
\section[{Chapitre IV, Malheur à la science qui ignore Dieu !}]{Chapitre IV, Malheur à la science qui ignore Dieu !}
\noindent \pn{7}Seigneur, Dieu de vérité, vous plaît-il celui qui sait tout cela ? Malheureux qui le sait et vous ignore ! Heureux qui l’ignore et vous connaît ! Et celui qui a cette double science n’est heureux que par vous seul, si, vous connaissant, il vous glorifie comme Dieu, s’il vous rend hommage, s’il ne se dissipe pas dans la vanité de ses pensées. Mieux vaut celui qui sait posséder un arbre et vous rendre grâces de ses fruits, sans savoir la hauteur de sa tige et l’étendue de ses branches, que celui qui sait la mesure des rameaux et le compte des feuilles, sans en jouir, sans en connaître, sans en aimer le Créateur ; ainsi, le fidèle a ce monde pour trésor ; tout ce qu’il renonce, il le retrouve en vous, ô Souverain de l’univers ! et quoiqu’il ignore la marche de l’étoile polaire, n’est-ce pas folie de mettre en doute la supériorité de cet humble croyant sur cet arpenteur du ciel, ce calculateur des étoiles, ce peseur des éléments, qui vous néglige, vous l’Ordonnateur de toutes choses\par

\begin{quoteblock}
\noindent « selon la mesure, le nombre et le poids (sag. XI, 21) ? »\end{quoteblock}

\section[{Chapitre V, Folie de Manès.}]{Chapitre V, Folie de Manès.}
\noindent \pn{8}Eh ! qui demandait à un Manès d’écrire sur des sujets entièrement étrangers à la science de la piété ? Vous avez dit à l’homme :\par

\begin{quoteblock}
\noindent « Voici la science, c’est la piété (Job, XXVIII, 28 selon les Sept.) »\end{quoteblock}

\noindent science qu’il eût pu ignorer en possédant la science humaine ; et celle-là même lui manquait, et il avait l’impudence d’enseigner ce qu’il ignorait ; pouvait-il donc être initié à la science des saints ? C’est vanité que de professer les connaissances que l’on possède dans l’ordre naturel, c’est piété que de confesser votre nom. Aussi a-t-il été permis à cet homme de multiplier ses divagations scientifiques, afin que son ignorance, évidente aux yeux des vrais savants, fit apprécier la valeur de ses opinions sur les choses cachées. Il ne voulait pas qu’on fit médiocre état de lui, cherchant même à faire croire que le Consolateur, l’Esprit-Saint, qui prodigue à vos fidèles sa céleste opulence, résidait personnellement en lui, dans toute la plénitude de son autorité. Aussi, toutes fois qu’on le surprend en flagrante erreur au sujet du ciel, des étoiles, des mouvements du soleil et de la lune, quoique la doctrine de la religion n’y soit nullement intéressée, son outrecuidance n’en paraît pas moins sacrilège ; car il ne débite pas seulement l’ignorance, mais le mensonge, avec un tel délire d’orgueil, qu’il voudrait autoriser ces discours par la prétendue divinité de sa personne.\par
\pn{9}Qu’un de mes frères en Jésus-Christ soit, à l’égard de ces connaissances, dans l’ignorance ou l’erreur, je prends ses opinions en patience. Rien n’y fait obstacle à son avancement ; son ignorance de la situation et de l’état d’une créature corporelle ne lui donne aucun sentiment indigne de vous, Seigneur, créateur de toutes choses. Mais elle lui devient funeste, s’il l’identifie avec les doctrines essentielles de la piété, et s’il s’obstine à affirmer ce qu’il ignore. Cette faible enfance au berceau de la foi, trouve dans la charité une mère qui la soutient, jusqu’à ce que le nouvel homme s’élève à cette perfection virile, qui cesse de flotter à tout vent de doctrine (Eph. IV, 13, 14). Et ce docteur, ce guide, ce maître, ce souverain, assez hardi pour persuader à ses disciples que ce n’était pas un homme, mais votre Esprit-Saint qu’ils suivaient en lui, qui ne le tiendrait pour un insensé, dont la folie, convaincue d’imposture, ne mérite que haine et mépris ? Cependant je n’étais pas encore assuré que l’on ne pût expliquer selon sa doctrine les vicissitudes de la durée des jours et des nuits, l’alternative elle-même de la nuit et du jour, les défaillances des astres, et les autres phénomènes que mes lectures m’avaient présentés, en sorte que, dans les points, douteux et de complète incertitude, ma foi en sa sainteté inclinait ma créance à son autorité.
\section[{Chapitre VI, Éloquence de Faustus et son ignorance.}]{Chapitre VI, Éloquence de Faustus et son ignorance.}
\noindent \pn{10}Et pendant ces neuf années où mon esprit s’égarait à les suivre, j’attendais avec impatience la venue de ce Faustus ; car ceux de la secte que j ‘avais rencontrés jusqu’alors, et qui tous manquaient de réponses à mes objections, me l’annonçaient comme devant, dès l’abord et au premier entretien, me donner facile solution de ces difficultés, et de plus graves encore, qui pourraient inquiéter ma pensée.  \par
Il vint, et je vis un homme doux, de parole agréable, et gazouillant les mêmes contes avec beaucoup plus de charme qu’aucun d’eux. Mais que faisait à ma soif toute la bonne grâce d’un échanson qui ne m’offrait que de précieux vases ? Mon oreille était déjà rassasiée de ces discours ; ils ne me semblaient pas plus solides pour être éloquents, ni plus vrais pour être plus polis. Et je ne jugeais pas de la sagesse de son âme à la convenance de sa physionomie et aux grâces de son élocution. Ceux qui me l’avaient vanté étaient de mauvais juges, qui ne l’estimaient docte et sage que parce qu’ils cédaient au charme de sa parole.\par
J’ai connu une autre espèce d’hommes à qui la vérité même est suspecte, et qui refusent de s’y rendre quand elle est proposée en beaux termes. Mais déjà, mon Dieu, vous m’aviez enseigné par des voies admirables et secrètes ; et je crois que je tiens de vous cet enseignement, parce qu’il est vrai, et que nul autre que vous n’enseigne la vérité, où et d’où qu’elle vienne. J’avais donc appris de vous que ce n’est point raison qu’une chose semble vraie pour être dite avec éloquence, ni fausse parce que les sons, s’élancent des lèvres sans harmonie ; ni au rebours, qu’une chose soit vraie par là même qu’elle est énoncée sans politesse, ni fausse parce qu elle est vêtue de brillantes paroles ; mais qu’il en est de la sagesse et de la folie comme d’aliments bons ou mauvais, et des expressions comme de vases d’or et d’argile ou ces aliments peuvent être indifféremment servis.\par
\pn{11}Le vif désir que j’avais eu si longtemps devoir cet homme trouvait quelque satisfaction dans le mouvement et la vivacité de ses discours, dans la propriété de son langage, qui se pliait comme un vêtement à sa pensée. J’admirais cette éloquence avec plusieurs, et je la publiais plus haut que nul autre ; mais je souffrais avec peine que son nombreux auditoire ne me permît pas de lui proposer mes doutes, de lui communiquer les perplexités de ma pensée en conférence familière, dans un libre entretien. Je pris toutefois l’occasion en temps et lieu convenables, en compagnie de mes intimes amis, et je lui dérobai une audience.\par
Je lui proposai plusieurs questions qui m’embarrassaient ; et je m’assurai bientôt qu’étranger à toutes les sciences, il n’avait même de ]a grammaire qu’une connaissance assez vulgaire. Il avait lu quelques discours de Cicéron, certains passages de Sénèque, quelques tirades de poésie, et ce qu’il avait trouvé dans les écrivains de sa secte de plus élégant et de plus pur. L’exercice journalier de la parole lui avait donné cette facilité d’élocution, qu’une certaine mesure dans l’esprit, accompagnée de, grâce naturelle, rendait plus agréable et plus propre à séduire. N’est-ce pas la vérité, Seigneur mon Dieu, arbitre de ma conscience ? Vous voyez à nu mon cœur et ma mémoire, ô vous qui déjà me conduisiez par les plus secrètes voies de votre Providence, et présentiez à ma face la laideur de mes égarements, pour que leur vue m’en donnât la haine.
\section[{Chapitre VII, Il se dégoute des doctrines manichéennes.}]{Chapitre VII, Il se dégoute des doctrines manichéennes.}
\noindent \pn{12}Aussitôt que son incapacité dans les sciences où j’avais cru qu’il excellait, me parut évidente, je désespérais de lui pour éclaircir et résoudre mes doutes sur des questions dont l’ignorance l’eût laissé dans la vérité de la piété, s’il n’eût pas été manichéen. Les livres de cette secte. sont remplis de contes interminables sur le ciel, les astres, le soleil, la lune ; et, les ayant comparés aux calculs astronomiques que j’avais lus ailleurs, pour juger si les raisons manichéennes valaient mieux ou autant que les autres, je n’attendais plus de Faustus aucune explication satisfaisante.\par
Je soumis toutefois mes difficultés à son examen ; mais il se refusa avec autant de prudence que de modestie à soulever ce fardeau. Il connaissait son insuffisance et ne rougit pas de l’avouer. Il n’était point de ces parleurs que j’avais souvent essuyés, qui, en voulant m’instruire, ne me disaient rien ; le cœur ne manquait point à cet homme, et s’il n’était dans la rectitude devant vous, il ne laissait pas d’être en garde sur lui-même. N’ignorant point entièrement son ignorance, il ne voulut pas s’engager par une discussion téméraire, dans un défilé sans issue, sans possibilité de retour. Cette franchise me le rendit encore plus aimable. La modeste confession de l’esprit est plus belle que la science même que je poursuivais ; et, en toute question difficile ou subtile, il n’en fit jamais autrement.\par
\pn{13}Ainsi, mon zèle pour les doctrines manichéennes se ralentit. Désespérant de plus en plus de leurs autres docteurs, à l’insuffisance du plus renommé d’entre eux, je bornai mes   rapports avec lui à des entretiens sur l’art oratoire dont il était épris, et que j’enseignais aux jeunes gens de Carthage ; à des lectures dont il était curieux par ouï-dire, ou que je jugeais conformes à la tournure de son esprit. Tout effort d’ailleurs pour avancer dans cette secte cessa de ma part, sitôt que je connus cet homme. Je n’en vins pas toutefois à rompre avec eux, mais je me résignai provisoirement, faute de mieux, à rester là où je m’étais jeté en aveugle, attendant qu’une lumière nouvelle déterminât un meilleur choix. Ainsi, ce Faustus, qui avait été pour plusieurs un lacet mortel, relâchait déjà, à son insu et sans le vouloir, les nœuds où j’étais pris. Vos mains, ô mon Dieu, actives dans le secret de votre Providence, n’abandonnaient pas mon âme ; et les larmes de ma mère, ce sang de son cœur qui coulait nuit et jour, montaient vers vous en sacrifice pour moi. Telle a été votre conduite à mon égard, admirable et cachée. Oui, votre conduite, ô mon Dieu ! Car\par

\begin{quoteblock}
\noindent « c’est le Seigneur, qui dirige les pas de l’homme, et l’homme désirera sa voie (Ps. XXXVI, 23).»\end{quoteblock}

\noindent Et qui peut procurer le salut, que la main toute-puissante qui refait ce qu’elle a fait ?
\section[{Chapitre VIII, Il va à Rome malgré sa mère.}]{Chapitre VIII, Il va à Rome malgré sa mère.}
\noindent \pn{14}C’est donc par un ordre inconnu de votre Providence, qu’il me fut persuadé d’aller a Rome, pour y enseigner la rhétorique plutôt, qu’à Carthage. Et d’où me vint cette persuasion, je ne manquerai pas de vous le confesser, parce qu’ici les abîmes de vos secrets, et la présence permanente de votre miséricorde sur nous, se découvrent à ma pensée et sollicitent mes louanges. Je ne me laissai pas conduire à Rome par l’espoir que m’y promettaient mes amis, de considération et d’avantages plus grands, quoique de telles raisons fussent alors toutes-puissantes sur mon esprit ; mais la plus forte, la seule même qui me décida, c’est que j’avais ouï dire que la jeunesse y était plus studieuse, plus patiente de l’ordre et de la répression ; qu’un maître n’y voyait jamais sa classe insolemment envahie par des disciples étrangers à ses leçons, et qu’on ne pouvait même y être admis que sur sa permission.\par
Or, rien n’est comparable à la honteuse et brutale licence des écoliers de Carthage. Ils forcent l’entrée des cours avec fureur et leur démence effrontée bouleverse l’ordre que chaque maître y établit dans l’intérêt de ses disciples. Ils commettent, avec une impudente stupidité, mille insolences que la loi devrait punir, si elles ne comptaient sur le patronage de la coutume. Malheureux, qui font, comme licite, ce qui sera toujours illicite devant votre loi éternelle ; qui croient à l’impunité, déjà punis par leur cécité morale, et souffrant incomparablement plus qu’ils ne font souffrir. Ces brutales habitudes dont, écolier, j’avais su me préserver, maître ; j’étais contraint de les endurer. Voilà ce qui m’attirait où un témoignage unanime m’assurait qu’il ne se passait rien de semblable.\par

\begin{quoteblock}
\noindent \emph{Mais} « Vous, mon espérance et mon héritage dans la terre des vivants (Ps CXLI, 6) »\end{quoteblock}

\noindent Vous m’inspiriez ce désir de migration pour le salut de mon âme, vous prêtiez des épines à Carthage pour m’en arracher, des charmes à Rome pour m’y attirer, et cela par l’entremise de ces hommes, amateurs de cette mort vivante ; les uns m’étalant leurs insolences, les autres leurs vaines promesses, et, afin de redresser mes pas, vous vous serviez en secret de leur malice et de la mienne. Ces perturbateurs de mon repos étaient possédés d’une aveugle frénésie ; ces tauteurs de mes espérances n’avaient de goût que pour la terre, et moi, qui détestais à Carthage une réalité de misère, je poursuivais a Rome un mensonge de félicité.\par
\pn{15}Mais pourquoi sortir d’ici et aller là ? vous le saviez, mon Dieu, sans m’en instruire, sans en instruire ma mère, a qui mon départ déchira l’âme, et qui me suivit jusqu’à la mer. Elle s’attachait à moi avec force, pour me retenir ou pour me suivre ; et je la trompai, ne témoignant d’autre dessein que celui d’accompagner un ami prêt à faire voile au premier vent favorable. Et je mentis à ma mère, et à quelle mère ! et je pris la fuite. Vous m’avez pardonné dans votre miséricorde ; vil, et souillé, vous m’avez préservé des eaux de la mer, pour m’amener à l’eau de votre grâce, qui, en me purifiant, devait sécher ces torrents de larmes dont ma mère marquait chaque jour la place des prières qu’elle versait pour moi. Et comme elle refusait de s’en retourner sans moi, je lui persuadai, non sans peine, de passer la nuit dans un monument dédié à saint Cyprien, non loin du vaisseau. Cette même nuit, je partis à   la dérobée, et elle demeura à prier et à pleurer, Et que vous demandait-elle, mon Dieu, avec tant de larmes ? de ne pas permettre mon voyage. Mais vous, dans la hauteur de vos conseils, touchant au ressort le plus vif de ses désirs, vous n’avez tenu compte de sa prière d’un jour, pour faire de moi selon sa prière de chaque jour.\par
Le vent souffla ; il emplit nos voiles, et déroba le rivage à nos regards. Elle vint le matin au bord de la mer, folle de douleur, remplissant de ses plaintes et de ses cris votre oreille inexorable à ce désespoir ; et vous m’entraîniez par la main de mes passions, où je devais en finir avec elles ; et votre justice meurtrissait du fouet de la douleur sa charnelle tendresse. Elle aimait ma présence auprès d’elle, comme une mère, et plus que beaucoup de mères ; et elle ne savait pas tout ce que vous lui apprêtiez de joies par cette absence. Elle ne le savait pas. Et de là, ces pleurs, ces sanglots, ces angoisses qui accusaient un reste de l’hérédité coupable d’Eve ; elle cherchait en pleurant ce qu’elle avait enfanté dans les pleurs. Mais après s’être répandue en plaintes sur ma fraude et ma cruauté, elle se remit à vous prier pour moi, rentra dans son intérieur, tandis que je voguais vers Rome.
\section[{Chapitre IX, Il tombe malade. — prières de sa mère.}]{Chapitre IX, Il tombe malade. — prières de sa mère.}
\noindent \pn{16}Et une maladie, terrible châtiment du corps, m’y attendait ; et déjà je m’acheminais vers l’enfer, chargé de tout ce que j’avais commis de crimes contre vous, contre moi, contre les autres, fardeau sinistre qui aggravait encore ce lien d’iniquité originelle qui nous fait tous mourir en Adam. Vous ne m’en aviez encore remis aucun en Jésus-Christ, et sa croix n’avait pas encore rompu ce contrat d’inimitié que mes péchés avaient formé entre vous et moi. Et l’eût-il rompu avec ce fantôme de croix que je rêvais ? Aussi fausse que me semblait la mort de sa chair, aussi véritable était celle de mon âme ; et aussi vraie qu’était la mort de sa chair, aussi fausse était la vie de mon âme qui se refusait à cette créance. Et la fièvre redoublait, et je m’en allais, et je périssais. Où pouvais-je aller, en m’en allant ainsi, sinon au supplice du feu, à des tourments dignes de mes œuvres, selon l’ordre de votre vérité ? Et elle ne le savait pas, et elle priait pour moi, loin de moi. Mais vous, partout présent, où elle était, vous l’écoutiez, et où j’étais, vous aviez pitié de moi, et vous me rendiez la santé du corps quand ce cœur sacrilège était encore malade. Car, dans ce péril extrême, je ne songeais pas au baptême ; enfant, j’étais bien meilleur, alors que je le demandai à la piété de ma mère, ainsi que mon souvenir vous l’a confessé. Mais j’avais grandi pour ma honte, et je riais, dans ma folie, des conseils du Médecin céleste qui ne m’a pas permis de mourir ainsi d’une double mort. Cette blessure au cœur de ma mère eût été incurable. Non, je ne puis dire tout ce qu’elle avait d’âme pour moi, et combien plus de souffrances lui coûtait le fils de son esprit que l’enfant de sa chair.\par
\pn{17}Oh ! non, je ne sais pas comment elle eût guéri, si ma mort, et une telle mort, eût traversé les entrailles de son amour. Et où pouvaient aller tant de prières, vives, fréquentes, continuelles, nulle part qu’à vous ? Et vous, Dieu des miséricordes, eussiez-vous méprisé le cœur contrit et humilié d’une veuve chaste, sobre, exacte à l’aumône, rendant tout hommage et tout devoir à vos saints, ne laissant passer aucun jour sans participer à l’offrande de votre autel ; soir et matin, assidue à votre Église, non pour engager de vaines causeries avec les vieilles, mais pour vous entendre dans vos paroles, pour être entendue de vous dans ses prières ?\par
Et ces larmes, qui ne vous demandaient ni or, ni argent, aucun bien passager ou périssable, mais le salut de l’âme de son fils, auriez-vous pu les mépriser ? Auriez - vous donc rebuté celle que votre grâce faisait votre suppliante ? Oh ! non, Seigneur ; vous lui étiez présent, vous l’entendiez, vous agissiez dans l’ordre de votre prédestination immuable. Loin, loin de moi ce doute impie que vous pussiez la tromper par ces visions, par ces réponses, dont j’ai rappelé les unes, omis les autres qu’elle gardait toutes dans la foi de son cœur, et que sa prière vous représentait sans cesse comme des billets souscrits de votre sang. Miséricorde infinie ! vous remettez leurs dettes à vos débiteurs, et vous voulez bien pourtant les reconnaître pour créanciers de vos promesses !  
\section[{Chapitre X, Il s’éloigne du manichéisme, dont il retient encore plus d’une erreur.}]{Chapitre X, Il s’éloigne du manichéisme, dont il retient encore plus d’une erreur.}
\noindent \pn{18}Vous m’avez donc rétabli de cette maladie et vous avez sauvé le fils de votre servante dans ce corps d’un jour, pour avoir à lui rendre une santé plus précieuse et plus sûre. Et je conservais, à Rome, des liaisons avec ces Saints trompés et trompeurs, et non-seulement avec les Auditeurs dont faisait partie l’hôte de ma maladie et de ma convalescence, mais aussi avec les Elus.\par
Je croyais encore que ce n’est pas nous qui péchons, mais je ne sais quelle nature étrangère qui pèche en nous ; et il plaisait à mon orgueil d’être en dehors du péché, et en faisant le mal, de ne pas m’en reconnaître coupable devant vous pour obtenir de votre miséricorde la guérison de mon âme ; et j’aimais à l’excuser en accusant je ne sais quel autre qui était en moi, sans être moi. Et pourtant le tout était moi, et mon impiété seule m’avait divisé contre moi-même, et c’était là le péché, le plus incurable, de ne me croire point pécheur ; et mon exécrable iniquité préférait, ô Dieu tout-puissant, votre défaite en moi, pour ma ruine, à votre victoire sur moi pour mon salut. Vous n’aviez donc pas encore placé la sentinelle, à l’entrée de ma bouche, et la porte de circonspection autour de mes lèvres, afin que mon cœur ne se laissât pas glisser aux paroles de malice pour excuser ses crimes, à l’exemple des artisans d’iniquité (Ps. CXL, 3,4).\par
\pn{19}C’est pourquoi je vivais encore avec leurs élus ,et toutefois sans espoir de rien acquérir désormais dans cette doctrine, et attendant mieux, je m’y tenais toujours, mais avec plus de tiédeur et d’indifférence. Il me vint même à l’esprit que les philosophes, dits Académiciens, avaient été plus sages que les autres en soutenant qu’il faut douter de tout, et que l’homme n’est capable d’aucune vérité. Je pensais, selon l’opinion commune, que telle était leur doctrine, dont alors je ne pénétrais pas le vrai sens. Je ne me fis donc pas scrupule d’ébranler la trop grande confiance de mon hôte dans les fables qui remplissent les livres manichéens. Je ne laissais pas toutefois d’entretenir avec ces hérétiques des relations plus familières qu’avec les autres hommes, et quoique moins ardente à la défense de leurs opinions, mon intimité avec eux (car Rome en recèle un grand nombre), ralentissait l’ardeur de mes recherches, alors surtout que je désespérais, ô Dieu du ciel et de la terre, créateur du visible et de l’invisible, de trouver dans votre Église la vérité dont ils m’avaient détourné. Il me semblait si honteux de vous supposer notre figure charnelle, et nos membres avec les limites de leurs contours ! Et comme, en voulant me représenter mon Dieu, ma pensée s’attachait toujours à une masse corporelle (rien à mes yeux ne pouvait être sans être ainsi), la principale, ou plutôt la seule et invincible cause de mes erreurs était là. 20. Et de là, cette croyance insensée que le Mal avait une substance corporelle, masse terreuse, difformité pesante, qu’ils appelaient terre, et une autre subtile et déliée, comme le corps de l’air, esprit de malice infiltré, suivant eux, dans ce monde élémentaire. Et un reste de piété quelconque me défendant de croire qu’un Dieu bon eût créé aucune nature mauvaise, j’établissais deux natures contraires et antagonistes, infinies toutes deux ; mais celle du bien plus infinie que celle du mal.\par
Et de ce principe de corruption découlaient tous mes blasphèmes. Mon esprit faisait-il effort pour recourir à la foi catholique, j’étais repoussé, car la foi catholique n’était pas ce que je la supposais ; et je me trouvais plus religieux, ô Dieu à qui vos miséricordes sur moi rendent témoignage, de vous croire infini de toutes parts, sauf le point où le principe mauvais en lutte contre vous me forçait à vous reconnaître une limite, que de vous tenir pour borné, aux formes du corps humain.\par
Et mieux valait, selon moi, croire que vous n’avez point créé le mal (le mal dont mon ignorance faisait non-seulement une substance, mais une substance corporelle, ne pouvant se figurer l’esprit autrement que comme un corps subtil répandu dans l’espace), que de vous prendre pour l’auteur de ce qui me paraissait la nature du mal. Notre Sauveur lui-même, votre Fils unique, je le regardais comme une extension émanée de votre étendue lumineuse pour notre salut, en sorte que je ne croyais de lui que le néant que j’imaginais. Aussi, lui attribuant cette substance, je m’assurais qu’elle ne pouvait naître de la vierge Marie qu’en se mêlant à la chair et je ne pouvais admettre ce mélange sans souillure d’un être de ma fantaisie. Je craignais donc, en le croyant né dans la chair, d’être conduit à le croire souillé par la chair.   Que vos enfants en esprit se rient de moi avec douceur et amour, s’ils viennent à lire ces confessions mais enfin, tel j’étais alors.
\section[{Chapitre XI, Ridicules réponses des manichéens.}]{Chapitre XI, Ridicules réponses des manichéens.}
\noindent \pn{21}Je ne pensais pas d’ailleurs qu’il fût possible de défendre ce qu’ils attaquaient dans vos Écritures ; mais néanmoins je désirais parfois en conférer en détail avec quelque docteur profondément versé dans l’intelligence des saints Livres, et voir ce qu’il en penserait. Déjà même, à Carthage, j’avais été touché des discours d’un certain Helpidius, qui, dans des conférences publiques contre les Manichéens, les pressait par certains passages de l’Écriture, dont ils paraissaient fort embarrassés ; car ils craignaient d’avancer en public leur réponse, qu’ils nous communiquaient en secret, à savoir, que les livres du Nouveau Testament avaient été falsifiés par je ne sais quels Juifs, qui voulaient enter la loi juive sur la foi chrétienne ; mais ils ne représentaient eux-mêmes aucun exemplaire authentique. Pour moi, envahi, étouffé par ces pensers matériels, qui affaissaient sous leur poids mon esprit haletant, je ne pouvais plus respirer l’air pur et vif de votre vérité.
\section[{Chapitre XII, Déloyauté de la jeunesse romaine.}]{Chapitre XII, Déloyauté de la jeunesse romaine.}
\noindent \pn{22}Déjà je remplissais avec zèle l’intention de mon voyage à Rome ; j’enseignais la rhétorique à quelques jeunes gens réunis chez moi, dont j’étais connu, et qui me faisaient connaître. O voici que j’apprends qu’il se pratique à Rome certaines choses, inouïes en Afrique. On n’y voit, il est vrai, aucune de ces violences ordinaires à l’impudente jeunesse de Carthage ; mais il s’y fait, me dit-on, entre jeunes gens, de soudains complots pour frauder leur maître de sa récompense, et ils passent chez un autre, transfuges avares de la bonne foi et-de l’équité ! Et je me sentais plein de haine pour ces âmes viles ; mais cette haine n’était pas légitime, car c’était peut-être le préjudice que j’en devais souffrir, plutôt que l’iniquité même de leur action, qui la soulevait.\par
Et néanmoins elles sont bien hideuses ces âmes infidèles ; prostituées à l’amour des frivoles jouets du temps, et de ce trésor de boue dont la prise souille la main, dans les embrassements de ce monde éphémère, elles méprisent votre clémence éternelle, qui nous rappelle, qui pardonne à l’épouse adultère aussitôt qu’elle revient à vous. Et je hais encore aujourd’hui ces hommes de honte et de difformité, quoique je les aime en vue de leur correction, afin qu’ils préfèrent à l’argent la science qu’on leur enseigne, et qu’ils vous préfèrent à la science, ô Dieu, vérité, félicité inaltérable, paix des âmes pures ! Mais alors mon intérêt me donnait plus de haine contre leur perversité, que le vôtre ne m’inspirait de désir pour leur amendement.
\section[{Chapitre XIII, Il se rend à Milan pour y enseigner la rhétorique. — Saint Ambroise.}]{Chapitre XIII, Il se rend à Milan pour y enseigner la rhétorique. — Saint Ambroise.}
\noindent \pn{23}On demanda de Milan au préfet de Rome un maître de rhétorique pour cette ville, qui s’engageait même à faire les frais du voyage, et je sollicitai cet emploi par des amis infatués de toutes les erreurs manichéennes, dont, à leur insu comme au mien, mon départ allait me délivrer. Un sujet proposé fit goûter mon éloquence au préfet Symmaque, qui m’envoya. À Milan, j’allai trouver l’évêque Ambroise, connu partout comme l’une des plus grandes âmes du monde, et votre pieux serviteur. Son zèle éloquent distribuait alors à votre peuple la pure substance de votre froment, la joie de vos huiles, la sobre intempérance de votre vin. Aveugle, votre main me menait à lui, pour qu’il me menât à vous, les yeux ouverts. Cet homme de Dieu m’accueillit comme un père, et se réjouit de ma venue avec la charité d’un évêque.\par
Et je me pris à l’aimer, et ce n’était pas d’abord le docteur de la vérité (j’avais perdu tout espoir de la trouver dans votre Église), mais l’homme bienveillant pour moi que j’aimais en lui. J’étais assidu à ses instructions publiques, non avec l’intention requise, mais pour m’assurer si le fleuve de son éloquence répondait à sa réputation, si la renommée en exagérait ou resserrait le cours, et je demeurais suspendu aux formes de sa parole, insouciant et dédaigneux du fond ; et j’étais flatté de la douceur de ces discours, plus savants, avec moins de charme et de séduction que ceux de Faustus ; je parle selon l’art des rhéteurs ; pour le sens, nulle comparaison. L’un s’égarait dans les mensonges de Manès, l’autre enseignait la plus saine doctrine du salut, Mais le salut est   loin des pécheurs, tel que j’étais alors, et cependant j’en approchais peu à peu, sans le savoir.
\section[{Chapitre XIV, Il rompt avec les manichéens, et demeure catéchumène dans l’Église.}]{Chapitre XIV, Il rompt avec les manichéens, et demeure catéchumène dans l’Église.}
\noindent \pn{24}Indifférent à la vérité, je n’étais attentif qu’à l’art de ses discours. Et, en moi, ce vain souci avait survécu, l’espoir que la voie qui mène à vous fût ouverte à l’homme. Toutefois, les paroles que j’aimais amenaient à mon esprit les choses elles-mêmes dont j’étais insouciant. Elles étaient inséparables, et mon cœur ne pouvait s’ouvrir à l’éloquence, sans que la vérité y entrât de compagnie, par degrés néanmoins. Je vis d’abord que tout ce qu’il avançait pouvait se défendre, et la foi catholique s’affirmer sans témérité contre les attaques des Manichéens, que j’avais crus jusqu’alors irrésistibles. Je fus surtout ébranlé, à l’entendre résoudre suivant l’esprit plusieurs passages obscurs de l’Ancien Testament, dont l’interprétation littérale me donnait la mort.\par
Eclairé par l’exposition du sens spirituel, je réprouvais déjà ce découragement qui m’avait fait croire impossible toute résistance aux ennemis, aux moqueurs de la Loi et des Prophètes. Toutefois, je ne me croyais pas tenu d’entrer dans la voie du catholicisme, parce qu’il pouvait avoir aussi de doctes et éloquents défenseurs, ni de condamner le parti que j’avais embrassé, parce que la défense lui présentait des armes égales. Ainsi la foi catholique cessant de me paraître vaincue, ne se levait pas encore victorieuse devant moi.\par
\pn{25}J’employai tous les ressorts de mon esprit à la découverte de quelque raison décisive pour convaincre de fausseté les opinions manichéennes. Si mon esprit eût pu se représenter une substance spirituelle, il eût brisé tous ces jouets d’erreur et les eût balayés de mon imagination ; mais je ne pouvais. Néanmoins, quant à ce monde extérieur, domaine de nos sens charnels, je trouvais beaucoup plus de probabilité dans les sentiments de la plupart des philosophes ; et de sérieuses réflexions, des comparaisons réitérées, appuyaient ce jugement.\par
Ainsi doutant de tout, suivant les maximes présumées de l’Académie, et flottant à toute incertitude, je résolus de quitter les Manichéens, ne croyant pas devoir, dans cette crise d’irrésolution, rester attaché à une secte qui déjà cédait dans mon estime à telle école philosophique. Mais à ces philosophes, vides du nom rédempteur de Jésus, je refusais de remettre la cure des langueurs de mon âme. Je me décidai donc à demeurer catéchumène dans l’Église catholique, l’Église de mon père et de ma mère, en attendant un phare de certitude pour diriger ma course.
\chapterclose


\chapteropen
 \chapter[{VI. La trentième année}]{VI. La trentième année}\phantomsection
\label{VI}\renewcommand{\leftmark}{VI. La trentième année}


\begin{argument}\noindent Sainte Monique retrouve son fils à Milan. — Assiduité d’Augustin aux prédications de saint Ambroise. — Son ami Alypius. — Projet de vie en commun avec ses amis. — Sa crainte de la mort et du jugement.
\end{argument}


\chaptercont
\section[{Chapitre Premier, Sainte Monique suit son fils à Milan.}]{Chapitre Premier, Sainte Monique suit son fils à Milan.}
\noindent \pn{1}O mon espérance dès ma jeunesse, où donc vous cachiez-vous à moi ? où vous étiez-vous retiré ? N’est-ce pas vous qui m’aviez fait si différent des brutes de la terre et des oiseaux du ciel ? Vous m’aviez donné la lumière qui leur manque, et je marchais dans la voie ténébreuse et glissante ; je vous cherchais hors de moi et je ne trouvais pas le Dieu de mon cœur. J’avais roulé dans la mer profonde, et j’étais dans la défiance et le désespoir de trouver jamais la vérité.\par
Et déjà j’avais auprès de moi ma mère. Elle était accourue, forte de sa piété, me suivant par mer et par terre, sûre de vous dans tous les dangers. Au milieu des hasards de la mer, elle encourageait les matelots mêmes qui encouragent d’ordinaire les novices affronteurs de l’abîme, et leur promettait l’heureux terme de la traversée, parce que, dans une vision, vous lui en aviez fait la promesse. Elle me trouva dans le plus grand des périls, le désespoir de rencontrer la vérité. Et cependant, quand je lui annonçai que je n’étais plus manichéen, sans être encore chrétien catholique, elle ne tressaillit pas de joie, comme à une nouvelle imprévue : son âme ne portait plus le deuil d’un fils perdu sans espoir ; mais ses pleurs coulaient toujours pour vous demander sa résurrection ; sa pensée était le cercueil où elle me présentait à Celui qui peut dire :\par

\begin{quoteblock}
\noindent « Jeune homme, je te l’ordonne, lève- toi ! (Luc VII, 14, 15) »\end{quoteblock}

\noindent afin que le fils de la veuve, reprenant la vie et la parole, fût rendu par vous à sa mère .\par
Son cœur ne fut donc point troublé par la joie en apprenant qu’une si grande quantité de larmes n’avait pas en vain coulé. Sans être encore acquis à la vérité, j’étais du moins soustrait à l’erreur. Mais certaine que vous n’en resteriez pas à la moitié du don que vous aviez promis tout entier, elle me dit avec un grand calme, et d’un cœur plein de confiance, qu’elle était persuadée en Jésus-Christ, qu’avant de sortir de cette vie, elle me verrait catholique fidèle. Ainsi elle me parla : mais en votre présence, ô source des miséricordes, elle redoublait de prières et de larmes afin qu’il vous plût d’accélérer votre secours et d’illuminer mes ténèbres ; plus fervente que jamais à l’église, et suspendue aux lèvres d’Ambroise, à la source\par

\begin{quoteblock}
\noindent « d’eau vive qui court jusqu’à la vie éternelle (Jean IV, 14) ; »\end{quoteblock}

\noindent elle l’aimait comme un ange de Dieu, elle savait que c’était lui qui, me réduisant aux perplexités du doute, avait décidé cette crise, dangereux, mais infaillible passage de la maladie à la santé.
\section[{Chapitre II, Elle se rend à la défense de Saint Ambroise.}]{Chapitre II, Elle se rend à la défense de Saint Ambroise.}
\noindent \pn{2}Ma mère ayant apporté aux tombeaux des martyrs, selon l’usage de l’Afrique, du pain, du vin et des gâteaux de riz, le portier de l’église lui opposa la défense de l’évêque ; elle reçut cet ordre avec une pieuse soumission, et je l’admirai si prompte à condamner sa coutume plutôt qu’à discuter la défense (Saint Augustin, devenu évêque, imita saint Ambroise et attaqua cette coutume dont abusait l’intempérance. (Voir lett. 22 à Aurélien de Carthage, et lett. 29 à Alypius.). L’intempérance ne livrait aucun assaut à son esprit, et l’amour du vin ne l’excitait pas à la haine de la vérité, comme tant de personnes, hommes   et femmes, pour qui les chansons de sobriété sont le verre d’eau qui donne des nausées à l’ivrogne. Lorsqu’elle apportait sa corbeille remplie des offrandes funèbres, elle en goûtait et distribuait le reste, ne se réservant que quelques gouttes de vin, autant que l’honneur des saintes mémoires en pouvait demander à son extrême sobriété. Si le même jour célébrait plus d’un pieux anniversaire, elle portait sur tous les monuments un seul petit flacon de vin trempé et tiède, qu’elle partageait avec les siens en petites libations ; car elle satisfaisait à sa piété et non à son plaisir.\par
Sitôt qu’elle eut appris que le saint évêque, le grand prédicateur de votre parole, avait défendu cette pratique même aux plus sobres observateurs, pour refuser aux ivrognes toute occasion de se gorger d’intempérance dans ces nouveaux banquets funèbres trop semblables à la superstition païenne, elle y renonça de grand cœur, et au lieu d’une corbeille garnie de terrestres offrandes, elle sut apporter aux tombeaux des martyrs une âme pleine des vœux les plus épurés ; se réservant de donner aux pauvres selon son pouvoir, il lui suffit de participer, dans ces saints lieux, à la communion du corps du Seigneur, dont les membres, imitateurs de sa croix, ont reçu la couronne du martyre.\par
Il me semble toutefois, Seigneur mon Dieu, et tel est le sentiment de mon cœur en votre présence, qu’il n’eût pas été facile d’obtenir de ma mère le retranchement de cette pratique, si la défense en eût été portée par un autre moins aimé d’elle qu’Ambroise, qu’elle chérissait comme l’instrument de mon salut ; et lui l’aimait pour sa vie exemplaire, son assiduité à l’église, sa ferveur spirituelle dans l’exercice des bonnes œuvres ; il ne pouvait se taire de ses louanges en me voyant, et me félicitait d’avoir une telle mère. Il ne savait pas quel fils elle avait en moi, qui doutais de toutes ces grandes vérités, et ne croyais pas qu’on pût trouver le chemin de la vie.
\section[{Chapitre III, Occupations de Saint Ambroise.}]{Chapitre III, Occupations de Saint Ambroise.}
\noindent \pn{3}Mes gémissements et mes prières ne vous appelaient pas encore à mon secours ; mon esprit inquiet cherchait et discutait sans repos. Et j’estimais Ambroise lui-même un homme heureux suivant le siècle, à le voir honoré des plus hautes puissances de la terre : son célibat seul me semblait pénible. Mais tout ce qu’il nourrissait d’espérance, tout ce qu’il avait de luttes à soutenir contre les séductions de sa propre grandeur, tout ce qu’il trouvait de consolations dans l’adversité, de charmes dans la voix secrète qui lui parlait au fond du cœur, tout ce qu’il goûtait de savoureuses joies en ruminant le pain de vie, je n’en avais nul pressentiment, nulle expérience, et lui ne se doutait pas de mes angoisses et de la fosse profonde où j’allais tomber. Il m’était impossible de l’entretenir de ce que je voulais, comme je le voulais ; une armée de gens nécessiteux me dérobait cette audience et cet entretien il était le serviteur de leurs infirmités. S’ils lui laissaient quelques instants, il réconfortait son corps par les aliments nécessaires et son esprit par la lecture.\par
Quand il lisait, ses yeux couraient les pages dont son esprit perçait le sens ; sa voix et sa langue se reposaient. Souvent en franchissant le seuil de sa porte, dont l’accès n’était jamais défendu, où l’on entrait sans être annoncé, je le trouvais lisant tout bas et jamais autrement. Je m’asseyais, et après être demeuré dans un long silence (qui eût osé troubler une attention si profonde ?) je me retirais, présumant qu’il lui serait importun d’être interrompu dans ces rapides instants, permis au délassement de son esprit fatigué du tumulte de tant d’affaires. Peut-être évitait-il une lecture à haute voix, de peur d’être surpris par un auditeur attentif en quelque passage obscur ou difficile, qui le contraignit à dépenser en éclaircissement ou en dispute, le temps destiné aux ouvrages dont il s’était proposé l’examen ; et puis, la nécessité de ménager sa voix qui se brisait aisément, pouvait être encore une juste raison de lecture muette. Enfin, quelle que fût l’intention de cette habitude, elle ne pouvait être que bonne en un tel homme.\par
\pn{4}Il m’était donc impossible d’interroger à mon désir votre saint oracle qui résidait dans son cœur, sauf quelques demandes où il ne fallait qu’un mot de réponse. Cependant mes vives sollicitudes épiaient un jour de loisir où elles pussent s’épancher en lui, elles ne le trouvaient jamais. Sans doute, je ne laissais jamais passer le jour du Seigneur sans l’entendre expliquer au peuple avec certitude la parole de vérité (II Tim. II, 15), et je m’assurais de plus en   plus que l’on pouvait démêler tous ces nœuds de subtiles calomnies que ces imposteurs ourdissaient contre les divines Écritures. Mais quand j’eus appris, qu’en croyant l’homme fait à votre image, vos fils spirituels, à qui votre grâce a donné une seconde naissance au sein de l’unité catholique, ne vous croyaient point pour cela limité aux formes du corps humain, quoique je ne pusse alors concevoir le plus léger, le plus vague soupçon d’une substance spirituelle ; néanmoins j’eus honte, dans ma joie, d’avoir, tant d’années durant, aboyé, non pas contre la foi catholique, mais contre les seules chimères de mes pensées charnelles d’autant plus téméraire et impie, que je censurais en maître ce que je devais étudier en disciple. O très-haut et très-prochain, très-caché et très-présent, Et re sans parties plus ou moins grandes, tout entier partout, et tout entier nulle part, vous n’êtes point cette forme corporelle, et pourtant vous avez fait l’homme à votre image, l’homme qui de la tête aux pieds tient dans un espace.
\section[{Chapitre IV, Assiduité d’Augustin aux sermons de Saint Ambroise.}]{Chapitre IV, Assiduité d’Augustin aux sermons de Saint Ambroise.}
\noindent \pn{5}Ne sachant donc de quelle manière votre image pouvait résider dans l’homme, ne devais-je pas frapper à la porte et demander comment il fallait croire, loin de m’écrier dans l’insolence de mon erreur : Voilà ce que vous croyez ? J’étais d’autant plus vivement rongé du désir intérieur de tenir la certitude, que, jouet et dupe de vaines promesses, j’avais plus longtemps, à ma honte, débité comme certains tant de peut-être, avec toute la puérilité de l’erreur et de la passion : j’en ai vu clairement depuis la fausseté. Certain aussi de les avoir tenus pour certains, j’étais déjà certain de leur incertitude, lors même que j’élevais contre votre Église mes aveugles accusations ; et sans être sûr qu’elle enseignât la vérité, je savais bien qu’elle n’enseignait pas ce que ma témérité lui reprochait. Ainsi je me sentais confondre et changer, et je me réjouissais, ô mon Dieu, que votre Église unique, corps de votre Fils unique, où, tout enfant, on mit sur mes lèvres le nom du Christ, ne se nourrît pas de bagatelles puériles, et que nul article de sa pure doctrine ne vous fît cette violence, ô Créateur de toutes choses, de vous resserrer, sous forme humaine, dans un espace limité, si large et si vaste qu’il pût être !\par
\pn{6}Je me réjouissais encore que l’ancienne Loi et les Prophètes ne me fussent plus proposés à lire du même oeil qui m’y faisait remarquer tant d’absurdités, quand je reprochais à vos saints les sentiments que je leur prêtais. Et j’aimais à entendre Ambroise recommander souvent, au peuple, dans ses sermons, cette règle suprême\par

\begin{quoteblock}
\noindent « La lettre tue et l’esprit vivifie (II Cor. III, 6). »\end{quoteblock}

\noindent Et, lorsqu’en soulevant le voile mystique, il découvrait l’esprit là où la lettre semblait enseigner une erreur, il ne disait rien qui me déplût, quoique je ne susse pas encore s’il disait la vérité. Je retenais mon cœur sur le penchant de l’adhésion, de peur du précipice ; et cette suspension même m’étouffait. Je voulais être aussi sûr de ce qui échappait à ma vue que de sept et trois sont dix. Je n’étais pas, il est vrai, assez insensé pour croire que je pusse ici me tromper ; mais je voulais avoir la même compréhension de toute vérité, soit corporelle et éloignée de mes sens, soit spirituelle, quoique ma pensée ne sût rien se représenter sans corps. Or, je devais croire pour guérir, pour que les yeux de mon esprit, dégagés de leur voile, pussent s’arrêter en quelque sorte sur votre vérité éternelle, sans révolution et sans éclipse.\par
Mais trop souvent celui qui a passé par le mauvais médecin n’ose plus se fier même au bon. Ainsi mon âme souffrante, que la foi seule pouvait guérir, de peur d’être trompée par la foi, se refusait à sa guérison. Elle résistait à ce puissant remède préparé par vos mains, et que vous prodiguez à l’univers avec souveraine efficace.
\section[{Chapitre V, Nécessité de croire ce que l’on ne comprend pas encore.}]{Chapitre V, Nécessité de croire ce que l’on ne comprend pas encore.}
\noindent \pn{7}Toutefois, je préférais dès lors la doctrine catholique, jugeant qu’elle commande avec plus de modestie et entière sincérité, de croire ce qui n’est point démontré (soit qu’on ait affaire à qui ne peut porter la démonstration, soit qu’il n’y ait point de démonstration possible), tandis que leurs téméraires promesses de science, appât dérisoire à la crédulité, ne sont qu’un ramas de fables et d’absurdités   qu’ils ne peuvent soutenir, et dont ensuite ils imposent la créance. Et votre main miséricordieuse et douce, ô Seigneur ! prenant et façonnant mon cœur peu à peu, je remarquais quelle infinité de faits je croyais, dont je n’avais été ni témoin, ni contemporain ; tant d’événements dans l’histoire des nations, tant de récits de lieux, de villes, d’actions, contés par des amis, des médecins, par tous les hommes, qu’il faut admettre sous peine de rompre toutes les relations de la vie. Une foi inébranlable ne m’assurait-elle pas des auteurs de ma naissance ? et que pouvais-je en savoir, si je ne croyais au témoignage ?\par
Ainsi vous m’avez persuadé que, loin de blâmer ceux qui ajoutent foi à vos Écritures, dont vous avez si puissamment établi l’autorité chez presque tous les peuples du monde, les incrédules seuls sont répréhensibles, et ne doivent point être écoutés quand ils nous disent D’où savez-vous si ces livres ont été communiqués au genre humain par l’Esprit du vrai Dieu, qui est la vérité même ? Et c’est précisément là ce qu’il me fallait croire, puisque, dans ces luttes sophistiques de questions captieuses, dans ces conflits de philosophes dont j’avais lu les livres, rien n’avait pu déraciner en moi la croyance que vous êtes, tout en ignorant ce que vous êtes, ni me faire douter que la conduite des choses humaines appartînt à votre Providence.\par
\pn{8}Ma foi, à cet égard, était, il est vrai, tantôt plus forte, tantôt plus faible ; mais toujours ai-je cru que vous êtes, et que vous prenez souci de nous, quoique je ne susse que penser de votre substance, ou de la voie qui conduit, qui ramène à vous. Ainsi donc, impuissante à trouver la vérité par raison pure, notre faiblesse a besoin de l’appui des saints Livres, et je commençai dès lors à croire que vous n’auriez point investi cette Écriture d’une autorité si haute et si universelle, s’il ne vous avait plu d’être cru, d’être cherché par elle. Quant aux absurdités où je me choquais d’ordinaire, quelques explications plausibles données devant moi m’en faisaient déjà rapporter l’inconnu étrange à la profondeur des mystères. Et son autorité m’apparaissait d’autant plus vénérable et plus digne de foi, que, s’offrant à la main de tout lecteur, elle n’en conservait pas moins dans la profondeur du sens la majesté de ses secrets ; accessible par la nudité de l’expression, par l’abaissement du langage, et toutefois exerçant les cœurs les plus méditatifs ; recevant tous les hommes en son vaste sein, n’en faisant passer qu’un petit nombre jusqu’à vous à travers le fin tissu de son voile, mais beaucoup plus néanmoins que si, au faite d’autorité où elle est élevée, elle ne rassemblait le genre humain dans le giron de son humilité sainte. Ainsi je méditais, et vous veniez à moi. Je soupirais, et vous prêtiez l’oreille. Je flottais, et vous me gouverniez. J’allais par la voie large du siècle, et vous ne m’abandonniez pas.
\section[{Chapitre VI, Misère de l’ambition.}]{Chapitre VI, Misère de l’ambition.}
\noindent \pn{9}J’aspirais aux honneurs, aux richesses, au mariage, et j’étais votre risée. Et je trouvais dans ces désirs mille épines douloureuses ; et vous m’étiez d’autant plus propice que vous me rendiez plus amer ce qui n’était pas vous. Voyez mon cœur, ô Seigneur ! qui m’avez inspiré ces souvenirs et cette confession. Que désormais s’attache à vous mon âme que vous avez dégagée des gluants appâts de la mort ! Quelle était sa misère ! Et vous ne cessiez de piquer sa plaie vive, afin qu’au mépris de tout elle se convertît à vous, qui êtes au-dessus de tout, sans qui rien ne serait ; qu’elle se convertît et guérît.\par
Quelle était la grandeur de mon mal, et quelle fut, pour me le faire sentir, l’habileté de votre traitement, alors que je me disposais à prononcer un panégyrique de l’empereur, où je devais débiter force mensonges qui eussent été accueillis par des applaudissements complices ! et mon cœur était haletant de soucis, j’étais possédé de la fièvre des pensers dévorants, quand, passant par une rue de Milan, j’aperçus un pauvre, aviné, je crois, et en joyeuse humeur. Je soupirai, et, m’adressant à quelques amis qui se trouvaient avec moi, je déplorai nos laborieuses folies. Tous nos efforts, si pénibles, et tels que ceux dont j’étais alors consumé, traînant sous l’aiguillon des passions cette charge de misère, de plus en plus lourde à mesure qu’on la traîne, avaient-ils d’autre but que cette sécurité joyeuse, où ce mendiant nous avait précédés, où peut-être nous n’arriverions jamais ? Quelques pièces d’argent mendiées lui avaient suffi pour acquérir ce que je poursuivais dans ces âpres défilés, par mille sentiers d’angoisse, la joie d’une félicité temporelle.   Il n’avait pas, sans doute, une joie véritable ; mais l’objet de mon ambitieuse ardeur était bien plus faux encore. Il était du moins sûr de sa joie, et j’étais soucieux. Il était libre ; moi, rongé d’inquiétudes. Que si l’on m’eût demandé mon choix entre la joie ou la crainte, il n’eût pas été douteux ; et si de nouveau l’on eût offert à mon choix d’être tel que cet homme, ou tel que j’étais alors, j’eusse préféré d’être moi avec mon fardeau de sollicitudes et de craintes, mais par aveuglement, et non par rectitude. Devais-je donc me préférer à lui, pour être plus savant, si ma science ne me donnait pas plus de joie, et si je n’en usais que pour plaire aux hommes, non pas afin de les instruire, mais uniquement de leur plaire ? C’est pourquoi vous brisiez mes os avec la verge de votre discipline.\par
\pn{10}Loin donc de mon âme ceux qui lui disent : Il y a joie et joie. Ce mendiant trouvait la sienne dans l’ivresse, et tu cherchais la tienne dans la gloire. Et quelle gloire, Seigneur, celle qui n’est pas en vous ? Mensonge de joie mensonge de gloire : seulement, cette gloire était plus captieuse à mon esprit. La nuit allait cuver son ivresse, et moi j’avais dormi, je m’étais levé, j’allais dormir et me lever avec la mienne, combien de jours encore ? Oui, il y a joie et joie. Celle des saintes espérances est infiniment distante de la vaine allégresse de ce malheureux. Mais alors même, grande était la distance de lui à moi. Plus heureux que moi, il ne se sentait point d’aise, quand les soucis me déchiraient les entrailles ; et il avait acheté son vin en souhaitant mille prospérités aux cœurs charitables, tandis que c’était au prix du mensonge que je marchandais la vanité.\par
Je tins alors à mes amis plus d’un discours semblable , et mes réflexions sur mon état étaient fréquentes, et je le trouvais alarmant ; et j’en souffrais, et cette affliction redoublait le malaise. Et si quelque prospérité semblait me sourire, j’avais peine à avancer la main ; voulais-je la saisir, elle était envolée.
\section[{Chapitre VII, Son ami Alypius.}]{Chapitre VII, Son ami Alypius.}
\noindent \pn{11}Tel était le sujet ordinaire de nos plaintes entre amis, et principalement de mes entretiens intimes avec Alypius et Nebridius. Alypius, né dans la même cité, d’une des premières familles municipales, était plus jeune que moi. Il avait suivi mes leçons à mon début dans notre ville natale et puis à Carthage ; et il m’aimait beaucoup, parce que je lui paraissais savant et bon. Et moi je l’aimais à cause du grand caractère de vertu qu’il développait déjà dans un âge encore tendre. Cependant le gouffre de l’immoralité et des spectacles frivoles, béant à Carthage, l’avait englouti dans le délire des jeux du cirque. Il y était misérablement Plongé, lorsque je professais en public l’art oratoire, mais il n’assistait pas encore à mes cours, à cause de certaine mésintelligence élevée entre son père et moi. J’appris avec douleur cette pernicieuse passion ; j’allais perdre, peut-être avais-je déjà perdu ma plus haute espérance. Et je n’avais, pour l’avertir ou le réprimer , ni le droit d’une bienveillance amicale, ni l’autorité d’un maître. Je croyais qu’il partageait à mon égard les sentiments de son père ; mais il n’en était rien. Car, loin de s’en inquiéter, il me saluait et venait même à mon auditoire m’écouter quelques instants et se retirait.\par
\pn{12}Et néanmoins, il m’était sorti de l’esprit de l’entretenir, pour le conjurer de ne pas sacrifier une aussi belle intelligence à l’aveugle entraînement de ces misérables jeux. Mais vous, Seigneur, qui ne lâchez jamais les rênes dont vous gouvernez vos créatures, vous n’aviez pas oublié qu’il devait être, entre vos enfants, l’un des premiers ministres de vos mystères. Et pour que l’honneur de son redressement vous revînt tout entier, vous m’en fîtes l’instrument, mais l’instrument involontaire. Un jour que je tenais ma séance ordinaire, il vint, me salua, prit place entre mes disciples, et se mit à m’écouter avec attention. Et par hasard, la leçon que j’avais entre les mains me parut demander, pour son explication, une comparaison empruntée aux jeux du cirque, qui dût jeter sur mes paroles plus d’agrément et de lumières, avec un assaisonnement de raillerie piquante contre les esclaves d’une telle manie.\par
Vous savez, mon Dieu, que je ne songeais nullement alors à en guérir Alypius. Mais il saisit le trait pour lui, ne le croyant adressé qu’à lui seul un autre m’en eût voulu, lui s’en voulut à lui-même ; excellent jeune homme, et qui m’en aima encore de plus vive amitié ! N’aviez-vous pas déjà dit depuis longtemps, dans vos Écritures :\par

\begin{quoteblock}
\noindent « Reprends le sage et il t’aimera (Prov. IX, 8) ? »\end{quoteblock}

\noindent Et néanmoins ce ne fut   pas moi qui le, repris ; mais vous, à qui, soit de gré, soit à notre insu, nous servons tous d’instruments selon l’ordre de votre sagesse et de votre justice. Ce fut vous qui fîtes de mon cœur et de ma langue des charbons ardents pour brûler et guérir le mal dont se mourait cette âme de précieuse espérance. Que celui-là taise vos louanges qui ne considère pas vos miséricordes ; elles parlent en votre honneur du fond de mes moelles. J’avais dit, et aussitôt Alypius s’élança hors de l’abîme où un aveugle plaisir l’avait précipité ; sa magnanime résolution secoua son âme et en fit tomber toutes les ordures du cirque, où il ne revint jamais depuis. Bientôt après, triomphant de la résistance de son père, il emporta la permission de me prendre pour maître. Redevenu mon disciple, il s’engagea avec moi dans les superstitions des Manichéens, aimant en eux cet extérieur de continence qu’il croyait naturel et vrai. Mais cette continence était loin de leur cœur ; ce n’était qu’un piége tendu aux âmes généreuses (Prov. VI, 26) qui n’atteignant pas encore aux profondeurs de la vertu, se laissent prendre à la superficie où glissent son ombre et sa trompeuse image.
\section[{Chapitre VIII, Alypius entraîné aux sanglants spectacles du cirque.}]{Chapitre VIII, Alypius entraîné aux sanglants spectacles du cirque.}
\noindent \pn{13}Nourri par ses parents dans l’enchantement des voies du siècle, loin de les délaisser, il m’avait précédé à Rome pour y apprendre le droit ; et là, il fut pris d’une étrange passion pour les combats de gladiateurs, et de la façon la plus étrange. Il avait pour ces spectacles autant d’aversion que d’horreur, quand un jour, quelques condisciples de ses amis, au sortir de table, le rencontrent, et malgré l’obstination de ses refus et de sa résistance, l’entraînent à l’amphithéâtre avec une violence amicale, au moment de ces cruels et funestes jeux. En vain il s’écriait :\par

\begin{quoteblock}
\noindent « Vous pouvez entraîner mon corps et le placer près de vous, mais pourrez-vous ouvrir à ces spectacles mon âme et mes yeux ? J’y serai absent, et je triompherai et d’eux et de vous.»\end{quoteblock}

\noindent Il eut beau dire, ils l’emmenèrent avec eux, curieux peut-être d’éprouver s’il pourrait tenir sa promesse.\par
Ils arrivent, prennent place où ils peuvent ; tout respirait l’ardeur et la volupté du sang. Mais lui, fermant la porte de ses yeux, défend à son âme de descendre dans cette arène barbare ; heureux s’il eût encore condamné ses oreilles ! car, à un incident du combat, un grand cri s’étant élevé de toutes parts, il est violemment ému, cède à la curiosité, et se croyant peut-être assez en garde pour braver, et vaincre même après avoir vu, il ouvre les yeux. Alors son âme est plus grièvement blessée que le malheureux même qu’il a cherché d’un ardent regard, il tombe plus misérable que celui dont la chute a soulevé cette clameur : entré par son oreille, ce cri a ouvert ses yeux pour livrer passage au coup qui frappe et renverse un cœur plus téméraire que fort, d’autant plus faible qu’il plaçait sa confiance en lui-même au lieu de vous. À peine a-t-il vu ce sang, il y boit du regard la cruauté. Dès lors il ne détourne plus l’oeil ; il l’arrête avec complaisance ; il se désaltère à la coupe des furies, et sans le savoir, il fait ses délices de ces luttes féroces ; il s’enivre des parfums du carnage. Ce n’était plus ce même homme qui venait d’arriver, c’était l’un des habitués de cette foule barbare ; c’était le véritable compagnon de ses condisciples. Que dirai-je encore ? il devint spectateur, applaudisseur, furieux enthousiaste, il remporta de ce lieu une effrayante impatience d’y revenir. Ardent, autant et plus. que ceux qui l’avaient entraîné, il entraînait les autres. Et c’est pourtant de si bas que votre main puissante et miséricordieuse l’a retiré, et vous lui avez appris .à ne point s’assurer en lui, mais en vous, bien longtemps après néanmoins.
\section[{Chapitre IX, Alypius soupçonné d’un larcin.}]{Chapitre IX, Alypius soupçonné d’un larcin.}
\noindent \pn{14}Ce souvenir restait dans sa mémoire comme un préservatif à l’avenir. Semblable avertissement lui avait été déjà donné, lorsqu’il était mon disciple à Carthage. C’était vers le milieu du jour ; il se promenait au Forum, pensant à une déclamation qu’il devait prononcer selon la coutume dans les exercices de l’école, quand surviennent les gardes du palais qui l’arrêtent comme voleur. Vous l’aviez permis, mon Dieu, sans doute afin qu’il apprît, devant être un jour si grand, combien il importe que l’homme, juge de l’homme, ne prononce pas sur le sort de son semblable avec une crédulité téméraire.\par
 Il se promenait donc seul, devant le tribunal, avec ses tablettes et son stylet, lorsqu’un jeune écolier, franc voleur, secrètement muni d’une hache, sans être aperçu de lui, s’approche des barreaux de plomb en saillie sur les devantures de la voie des Orfèvres, et se met à les couper. Au bruit de la hache, on s’écrie à l’intérieur et on envoie des gens pour saisir le coupable. Entendant leurs voix, celui-ci prend la fuite et jette son instrument, de peur d’être surpris armé. Alypius qui ne l’avait pas vu entrer, le voit sortir et fuir rapidement. Il s’approche pour s’informer ; étonné de trouver une hache, il s’arrête à la considérer. On l’aperçoit, seul, tenant l’outil dont le bruit avait donné l’alarme. On l’arrête, on l’entraîne, on appelle tous les habitants du voisinage, on le montre en triomphe comme un voleur pris en flagrant délit qu’on va livrer au juge.\par
\pn{15}Mais la leçon devait se borner là. Vous vîntes aussitôt, Seigneur, au secours de son innocence, dont vous étiez le seul témoin. Comme on le menait à la prison ou au supplice, il se trouva à la rencontre un architecte, spécialement chargé de la conservation des bâtiments publics. Les gens qui le tiennent sont charmés qu’à leur passage vienne précisément s’offrir celui qui d’ordinaire les soupçonnait des larcins commis au Forum ; il en allait enfin connaître les auteurs. Or, cet homme avait plus d’une fois vu Alypius chez un sénateur qu’il allait souvent saluer. Il le reconnaît, lui prend la main et, le tirant à part, lui demande la cause de ce désordre, et apprend ce qui s’est passé. La foule s’émeut et murmure avec menace ; l’architecte commande qu’on le suive. On passe devant la maison du jeune homme coupable. À la porte se trouvait un enfant, trop petit pour être retenu dans sa révélation par la crainte de compromettre son maître, qu’il avait accompagné au Forum. Alypius le voit et le désigne à l’architecte, qui, montrant la hache à l’enfant, lui demande à qui elle est : à nous, répond à l’instant celui-ci. On l’interroge de nouveau ; tout se découvre. Ainsi, le crime retomba sur cette maison, à la confusion de la multitude, qui déjà triomphait d’Alypius. Dispensateur futur de votre parole, et juge de tant d’affaires en votre Église, il sortit de ce danger avec plus d’instruction et d’expérience.
\section[{Chapitre X, Intégrité d’Alypius. — ardeur de Nebridius à la recherche de la vérité.}]{Chapitre X, Intégrité d’Alypius. — ardeur de Nebridius à la recherche de la vérité.}
\noindent \pn{16}Je l’avais rencontré à Rome, où il s’unit à moi d’amitié si étroite qu’il me suivit à Milan pour ne point se séparer de moi, et aussi pour utiliser sa science du droit, suivant le désir de ses parents plutôt que le sien. Eprouvé déjà par trois emplois, où son désintéressement n’avait pas moins étonné les autres qu’il n’était surpris lui-même de la préférence qu’on pouvait accorder à l’or sur la probité, une dernière tentative contre sa fermeté avait mis en œuvre tous les ressorts de la séduction et de la terreur. Il remplissait à Rome les fonctions d’assesseur, auprès du comte des revenus d’Italie, quand un sénateur, puissant par ses bienfaits et son crédit, accoutumé à ne pas trouver d’obstacles, voulut se permettre je ne sais quoi de contraire à la loi. Alypius s’y oppose. On lui promet une récompense, qu’il dédaigne ; on essaie de menaces, qu’il foule aux pieds ; tous admirant cette constance qui ne pliait pas devant un homme, bien connu pour avoir mille moyens d’être utile ou de nuire ; cette fermeté d’âme également indifférente au désir de son amitié et à la crainte de sa haine. Le magistrat lui-même, dont Alypius était le conseiller, quoique opposé à cette injuste prétention, n’osait cependant refuser hautement ; mais s’excusant sur l’homme juste, il alléguait sa résistance ; et s’il fléchissait, Alypius était en effet décidé à résigner ses fonctions.\par
Son amour pour les lettres, seul, faillit le séduire ; il eût pu, avec le gain du prétoire, se procurer des manuscrits ; mais il consulta la justice et prit une résolution meilleure, préférant le véto de l’équité au permis de l’occasion. Cela n’est rien sans doute, mais « qui est fidèle dans les petites choses l’est dans les grandes ; » et rien ne saurait anéantir cet oracle sorti de la bouche de votre vérité :\par

\begin{quoteblock}
\noindent « Si vous n’avez pas été fidèle dispensateur d’un faux trésor, qui vous confiera le véritable ? Si vous n’avez pas été fidèle dépositaire du bien d’autrui, qui vous rendra celui qui est à vous (Luc, XVI, 10-12) ? »\end{quoteblock}

\noindent Tel était l’homme si étroitement lié avec moi, et comme moi chancelant, irrésolu sur le genre de vie à suivre.\par
\pn{17}Et Nebridius aussi qui avait abandonné   son pays, voisin de Carthage, et Carthage même, son séjour ordinaire, et le vaste domaine de son père, et sa mère qui ne songeait pas à le suivre — il avait tout quitté pour venir à Milan vivre avec moi dans la poursuite passionnée de la vérité et de la sagesse. Il soupirait comme moi, il flottait comme moi, ardent à la recherche de la vie bienheureuse, profond dans l’examen des plus difficiles problèmes. Voilà donc trois bouches affamées, exhalant entre elles leur mutuelle indigence, et attendant de vous leur nourriture au temps marqué. Et, dans l’amertume dont votre miséricorde abreuvait notre vie séculière, considérant le but de nos souffrances, nous ne trouvions plus que ténèbres. Nous nous détournions en gémissant, et nous disions : Jusques à quand ? Et tout en le répétant, nous poursuivions toujours, parce qu’il ne nous apparaissait rien de certain que nous pussions saisir en lâchant le reste.
\section[{Chapitre XI, Vives perplexités d’Augustin.}]{Chapitre XI, Vives perplexités d’Augustin.}
\noindent \pn{18}Et je ne pouvais, sans un profond étonnement, repasser dans ma mémoire tout ce long temps écoulé depuis la dix-neuvième année de mon âge, où je m’étais si vivement épris de la sagesse, résolu d’abandonner à sa rencontre les vaines espérances et les trompeuses chimères de mes passions. Et déjà j’accomplissais mes trente ans, embourbé dans la même fange, avide de jouir des objets présents, périssables, et qui divisaient mon âme. Je trouverai demain, disais-je ; demain la vérité paraîtra, et je la saisirai. Et puis, Faustus va venir, il m’expliquera tout. O grands maîtres de l’Académie ! on ne peut rien tenir de certain pour régler la vie. Mais non, cherchons mieux ; ne désespérons pas. Voici déjà que les absurdités de l’Écriture ne sont plus des absurdités ; une interprétation différente satisfait la raison. Arrêtons-nous sur les degrés où, entant, mes parents m’avaient déposé, jusqu’à ce que se présente la vérité pure.\par
Mais où, mais quand la chercher ? Ambroise n’a pas une heure à me donner, je n’en ai pas une pour lire. Et puis, où trouver des livres ? quand et comment s’en procurer ? à qui en emprunter ? Réglons le temps ; ménageons-nous des heures pour le salut de notre âme. Une grande espérance se lève. La foi catholique n’enseigne pas ce dont l’accusait la vanité de mon erreur. Ceux qui la connaissent condamnent comme un blasphème la croyance que Dieu soit borné aux limites d’un corps humain ; et j’hésite à frapper pour qu’on achève de m’ouvrir ? La matinée est donnée à mes disciples : que fais-je le reste du jour ? pourquoi cette négligence ? Mais trouverai-je un moment pour rendre visite à des amis puissants, dont le crédit m’est nécessaire ? pour préparer ces leçons que je vends ? pour donner quelque relâche à mon esprit fatigué de tant de soins ?\par
\pn{19}Périssent toutes ces vanités, périsse tout ce néant ; employons-nous à la seule recherche de la vérité. Cette vie est misérable et l’heure de la mort incertaine ; si elle nous surprend, en quel état sortirons-nous d’ici ? Où apprendrons-nous ce que nous y aurons négligé d’apprendre ? ou plutôt ne nous faudra-t-il pas expier cette négligence ? Et si la mort allait trancher tout souci avec ce nœud de chair ? Si tout finissait ainsi ? Encore s’en faut-il enquérir. Mais non ; blasphème qu’un tel doute ! Ce n’est pas un rien, ce. n’est pas un néant qui élève la foi chrétienne à cette hauteur d’autorité par tout l’univers. Le doigt de Dieu n’aurait pas opéré pour nous tant de merveilles, si la mort du corps absorbait la vie de l’âme. Que tardons-nous, que ne laissons-nous là l’espoir du siècle, pour nous appliquer tout entier à chercher Dieu et la vie bienheureuse ? .\par
Mais attends encore ; n’est-il plus de charme dans ce monde ? a-t-il perdu ses puissantes séductions ? n’en détache pas ton cœur à la légère. Il serait honteux de revenir à lui après l’avoir quitté. Vois, à quoi tient-il que tu n’arrives à une charge honorable ? Que pourrais-tu souhaiter après ? N’ai-je pas en effet des amis puissants ? Quel que soit mon empressement à limiter mes espérances, je puis toujours aspirer à une présidence de tribunal ; et je prendrai une femme dont la fortune sera suffisante à mon état, et là se borneront mes désirs. Combien d’hommes illustres et dignes de servir d’exemples, ont vécu mariés, et fidèles à la sagesse !\par
\pn{20}Ainsi disais-je ; et les vents contraires de mes perplexités jetaient mon cœur çà et là ; et le temps passait ; et je tardais à me convertir à vous, Seigneur mon Dieu ; je différais de jour en jour de vivre en vous, et je ne différais pas un seul jour de mourir en moi-même. Aimant la vie bienheureuse, je la redoutais dans son   séjour, et en la fuyant je la cherchais. Je croyais que je serais trop malheureux d’être à jamais privé des embrassements d’une femme ; et le remède de votre miséricorde, efficace contre cette infirmité, ne venait pas à ma pensée, faute d’en avoir fait l’épreuve ; car j’attribuais la continence aux propres forces de l’homme, et cependant je sentais ma faiblesse. J’ignorais, insensé, qu’il est écrit :\par

\begin{quoteblock}
\noindent « Nul n’est chaste, si vous ne lui en donnez la force (Sagesse, VIII, 21). »\end{quoteblock}

\noindent Et vous me l’eussiez donnée, si le gémissement intérieur de mon âme eût frappé à votre oreille ; si ma foi vive eût jeté dans votre sein tous mes soucis.
\section[{Chapitre XII, Ses entretiens avec Alyplus sur le mariage et le célibat.}]{Chapitre XII, Ses entretiens avec Alyplus sur le mariage et le célibat.}
\noindent \pn{21}Alypius me détournait du mariage, et me représentait sans cesse que ces liens ne nous permettraient plus de vivre assurés de nos loisirs, dans l’amour de la sagesse, comme nous le désirions depuis longtemps. Il était d’une chasteté d’autant plus admirable, qu’il avait eu commerce avec les femmes dans sa première jeunesse ; mais il s’en était détaché, avec remords et mépris, pour vivre dans une parfaite continence. Et moi je lui opposais l’exemple d’hommes mariés qui étaient demeurés dans la pratique de la sagesse, le service dé Pieu, la fidélité aux devoirs de l’amitié. Mais que j’étais loin d’une telle force d’âme ! Esclave de cette fièvre charnelle dont j’étais dévoré, je traînais ma chaîne, dans une mortelle ivresse, et je tremblais qu’on ne vînt la rompre, et ma plaie vive, frémissante sous l’anneau secoué, repoussait la parole d’un bon conseiller, la main d’un libérateur. Que dis-je ? le serpent, par ma bouche, parlait à Alypius ; ma langue formait les nœuds et semait dans sa voie les doux piéges où son pied innocent et libre allait s’embarrasser.\par
\pn{22}Ce lui était un prodige de me voir, moi qu’il estimait, pris à l’appât de la volupté, jusqu’à lui avouer même, dans nos conversations, qu’il me serait impossible de garder le célibat ; et pour me défendre contre son étonnement, je lui disais que ce plaisir qu’il avait ravi au passage, et dont un vague souvenir lui rendait le mépris si facile, n’avait rien de comparable aux délices de cette liaison dans laquelle je vivais. Que si la sanction du mariage venait à légitimer de telles jouissances, quel sujet aurait-il donc d’être surpris de mon impuissance à mépriser une telle vie ? Il finissait par la désirer lui-même, cédant moins aux sollicitations du plaisir qu’à celles de la curiosité. Il voulait savoir, disait-il, quel était enfin ce bonheur sans lequel ma vie, qui lui plaisait, ne me paraissait plus une vie, mais un supplice.\par
Libre de mes fers, son esprit s’étonnait de mon esclavage, et de l’étonnement il se laissait aller au désir d’en faire l’essai, pour tomber peut-être de cette expérience dans la servitude même qui l’étonnait, parce qu’il voulait se fiancer à la mort, et que l’homme qui aime le péril y tombe (Eccli. III, 27). Car nous n’étions, l’un et l’autre, que faiblement touchés des devoirs qui donnent seuls quelque dignité au mariage, la continence et l’éducation des enfants. Pour moi, je n’en aimais guère que l’enivrante habitude d’assouvir cette insatiable concupiscence dont j’étais la proie ; et lui allait trouver la captivité dans son étonnement de ma servitude. Voilà où nous en étions, jusqu’à ce que votre grandeur, fidèle à notre boue, prit en pitié notre misère, et vint à notre secours par de merveilleuses et secrètes voies.
\section[{Chapitre XIII, Sa mère n’obtient de Dieu aucune révélation sur le mariage de son fils.}]{Chapitre XIII, Sa mère n’obtient de Dieu aucune révélation sur le mariage de son fils.}
\noindent \pn{23}Et l’on pressait activement l’affaire de mon mariage. J’avais fait une demande ; j’étais accueilli ; ma mère s’y employait avec zèle, d’autant que le mariage devait me conduire à l’eau salutaire du baptême ; elle sentait avec joie que je m’en approchais chaque jour davantage ; et ma profession de foi allait accomplir ses vœux et vos promesses. Mais lorsque, à ma prière et selon l’instinct de son désir, elle vous suppliait, de l’accent le plus passionné du cœur, de lui révéler en songe quelque chose de cette future alliance, vous n’avez jamais voulu l’entendre. Elle voyait de vaines et fantastiques images rassemblées par la vive préoccupation de l’esprit ; elle me les racontait avec mépris ; ce n’était plus cette confiance qui lui attestait l’impression de votre doigt. Certain goût ineffable lui donnait, disait-elle, le discernement   de vos révélations et des songes de son âme. On pressait néanmoins mon mariage ; la jeune fille était demandée, mais il s’en fallait de deux années qu’elle fût nubile ; et comme elle me plaisait, on prit le parti d’attendre.
\section[{Chapitre XIV, Projet de vie en commun avec ses amis.}]{Chapitre XIV, Projet de vie en commun avec ses amis.}
\noindent \pn{24}Nous étions plusieurs amis ensemble, qui, dégoûtés des turbulentes inquiétudes de la vie humaine, objet habituel de nos réflexions et de nos entretiens, avions presque résolu de nous retirer de la foule pour vivre en paix. Notre plan était de mettre en commun ce que nous pourrions avoir, de faire une seule famille, un seul héritage, notre sincère amitié faisant disparaître le tien et le mien, le bien de chacun serait à tous, le bien de tous à chacun ; nous pouvions être dix dans cette communauté, et plusieurs. d’entre nous étaient fort riches ; Romanianus, en particulier, citoyen de notre municipe, qu’une tourmente d’affaires avait jeté à la cour de l’empereur, et mon intime ami dès l’enfance. Il était le plus ardent à presser ce dessein, et il nous le persuadait avec d’autant plus d’autorité, qu’il avait la prépondérance de la fortune. Nous avions décidé que deux d’entre nous seraient chargés, comme magistrats annuels, de l’administration des affaires, les autres vivant en repos. Mais quand on vint à demander sites femmes y consentiraient, plusieurs étant déjà mariés, et nous aspirant à l’être, l’argile si bien façonné de cette illusion nouvelle éclata entre nos mains, et nous en rejetâmes les débris.\par
Et nous voilà retombés dans nos soupirs, dans nos gémissements, dans les voies du siècle larges et battues, et notre cœur roulait le flot de ses pensées devant l’éternelle stabilité de votre conseil (Ps. XXXII, 2). Du haut de ce conseil, riant de nos résolutions, vous prépariez les vôtres, attendant le temps propre pour nous donner la nourriture, et pour ouvrir la main qu’il allait combler nos âmes de bénédiction (Ps CXLIV, 15, 16.).
\section[{Chapitre XV, La femme qu’il entretenait étant retournée en Afrique, il en prend une autre.}]{Chapitre XV, La femme qu’il entretenait étant retournée en Afrique, il en prend une autre.}
\noindent \pn{25}Cependant mes péchés se multipliaient ; et quand on vint arracher de mes côtés, comme un obstacle à mon mariage, la femme qui vivait avec moi, il fallut déchirer le cœur où elle avait racine, et la blessure saigna longtemps. Mais elle, à son retour en Afrique, vous fit vœu de renoncer au commerce de l’homme. Elle me laissait le fils naturel qu’elle m’avait donné. Et moi malheureux, incapable d’imiter une femme, impatient de cette attente de deux années pour obtenir la main qui m’était promise, n’étant point amoureux du mariage, mais esclave de la volupté, je trouvai une autre femme, comme pour soutenir et irriter la maladie de mon âme, en lui continuant cette honteuse escorte de plaisirs jusqu’à l’avènement de l’épouse. Ainsi la blessure dont la première séparation m’avait navré, ne guérissait pas : niais après de cuisantes douleurs, elle tournait en sanie : et le mal, plus languissant, n’en était que plus désespéré.
\section[{Chapitre XVI, Sa crainte de la mort et du jugement.}]{Chapitre XVI, Sa crainte de la mort et du jugement.}
\noindent \pn{26}Louange à vous ! gloire à vous ! ô source des miséricordes. Je devenais de jour en jour plus déplorable, et vous plus prochain. Vous avanciez déjà la main qui allait me retirer et me laver de cette boue, et je ne m’en doutais pas. Et rien ne me rappelait du fond de l’abîme des voluptés charnelles que la crainte de la mort et de votre jugement futur, si profonde en mon cœur que tant de doctrines contraires n’avaient jamais pu l’en bannir. Et je discutais avec Alypius et Nebridius les raisons finales des biens et des maux, leur avouant que, dans mon esprit, Epicure eût obtenu la palme, si j’avais pu cesser de croire à la survivance de l’âme après la mort, et à la rémunération des œuvres qu’Epicure n’admit jamais. Si nous étions immortels, leur disais-je, vivant dans une perpétuelle volupté des sens, sans aucune crainte de la perdre, pourquoi ne serions-nous pas heureux ? Et que nous faudrait-il encore ? Et je ne voyais pas que cette pensée même témoignait de ma misère et de la profondeur de mon naufrage ; aveugle, je n’apercevais pas la lumière de cette beauté chaste et   pure qu’il faut embrasser sans passion, invisible au regard de la chair, visible seulement à l’oeil intérieur.\par
Et, malheureux, je ne concevais pas de quelle source coulait en moi ce plaisir que la présence de mes amis me faisait trouver au récit de ces honteuses misères. Car, au sein même des joies charnelles, je n’eusse pu vivre heureux, même selon l’homme sensuel d’alors, sans ces amis que j’aimais et, dont je me sentais aimé sans intérêt. O voies tortueuses ! malheur à l’âme téméraire qui, en se retirant de vous, espère trouver mieux que vous ! Elle se tourne, elle se retourne en vain, sur le dos, sur les flancs, sur le ventre ; tout lui est dur. Et vous seul êtes son repos. Et vous voici ! et vous nous délivrez de nos lamentables erreurs ! et vous nous mettez dans votre voie, et vous nous consolez et dites : « Courez, je vous soutiendrai ; je vous conduirai au but, et là, je vous soutiendrai encore. 
\chapterclose


\chapteropen
 \chapter[{VII. La découverte du néo-platonisme}]{VII. La découverte du néo-platonisme}\phantomsection
\label{VII}\renewcommand{\leftmark}{VII. La découverte du néo-platonisme}


\begin{argument}\noindent Peines de son esprit dans la recherche du mal. — Par quels degrés il s’élève à la connaissance de Dieu. — Erreur de ses sentiments sur la personne de Notre-Seigneur Jésus-Christ.
\end{argument}


\chaptercont
\section[{Chapitre premier, Il ne pouvait concevoir Dieu que comme une substance infiniment étendue.}]{Chapitre premier, Il ne pouvait concevoir Dieu que comme une substance infiniment étendue.}
\noindent \pn{1}Et déjà était morte mon adolescence honteuse et criminelle ; et j’entrais dans la jeunesse, et plus j’avançais en âge, plus je m’égarais en de ridicules chimères, ne pouvant concevoir d’autre substance que celle qui se voit par les yeux. Je ne vous prêtais plus, il est vrai, mon Dieu, les formes humaines, depuis que j’avais commencé d’ouvrir l’esprit à la sagesse ; je m’étais toujours préservé de cette erreur ; et je la voyais, avec joie, condamnée par la foi de votre Église catholique, notre mère spirituelle. Mais de quelle autre manière vous concevoir ? je l’ignorais, et je m’évertuais à vous comprendre, homme que j’étais, et quel homme ! vous le souverain, le seul et vrai Dieu. Et je croyais de toutes les forces de mon être que vous êtes incorruptible, inviolable, immuable ; car, malgré mon ignorance du comment et du pourquoi, je voyais cependant avec certitude que ce qui est sujet à la corruption est au-dessous de l’incorruptible ; et je préférais sans hésiter l’inviolable à ce qui souffre violence, et l’immuable au muable.\par
Mon cœur protestait violemment contre ces vanités de ma fantaisie, et je cherchais à dissiper d’un seul coup l’essaim bourdonnant d’impuretés qui offusquaient le regard de ma pensée ; à peine éloigné, il revenait soudain fondre plus pressé sur mes yeux aveuglés ; et tout en renonçant à cette vaine imagination de forme humaine, je ne pouvais néanmoins me débarrasser de l’idée d’une substance corporelle pénétrant le monde dans toute son étendue, et répandue, hors du monde, dans l’infini ; et, toutefois, je lui maintenais, en tant qu’incorruptible, inviolable et immuable, la prééminence sur ce qui est sujet à corruption, déchéance et changement. Tout être, à qui je refusais l’étendue, ne me semblait plus qu’un rien ; mais rien absolu, et non ce vide que ferait dans l’étendue la disparition de tout corps. Car l’étendue serait toujours, malgré cette vacuité de tout corps élémentaire ou céleste, vide étendu, spacieux néant.\par
\pn{2}Et dans cette pléthore de cœur, m’obscurcissant moi-même à mes propres yeux, je pensais que tout ce qui ne m’apparaissait point à l’état d’extension ou de diffusion, de concentration ou de renflement, n’était que pur néant. Car les formes sur lesquelles se promènent mes yeux, étaient les seules images que parcourût ma pensée, et je ne m’apercevais pas que cette action intérieure qui me figurait ces images, ne leur était en rien semblable, et qu’elle ne pouvait les imaginer sans être elle-même quelque chose de grand.\par
Et vous, ô vie de ma vie, c’est ainsi que je vous croyais grand ; répandu, suivant moi, dans tout le corps de l’univers, et le débordant partout à l’infini, le ciel et la terre et toute créature vous possédaient, terminés en vous ; vous, nulle part. Mais comme le corps de l’air étendu sur la terre ne résiste point à la lumière du soleil qui le traverse, qui le pénètre sans le déchirer ou le diviser et le remplit tout entier, j’imaginais que vous passiez ainsi par le corps du ciel et de l’air, de la mer et même par celui de la terre, également pénétrable en ses parties les plus grandes et les moindres à l’immanation de votre présence, qui imprimait, comme une respiration subti1e, le mouvement intérieur et extérieur à toutes vos créatures.\par
Telles étaient mes conjectures ; ma pensée ne pouvait aller au delà, et c’était encore une   erreur. Car il fallait admettre qu’une plus grande partie de la terre en contenait une plus grande de vous, et une plus petite, une moindre, votre présence se distribuant de manière qu’il en tenait plus dans le corps de l’éléphant que dans celui du passereau ; beaucoup plus grand, il prenait beaucoup plus de place ; et ainsi les divisions de votre essence se proportionnaient aux inégalités des corps. Et toutes fois il n’en est pas ainsi ; mais vous n’aviez point encore éclairé mes ténèbres.
\section[{Chapitre II. Objection de Nebridius contre les manichéens.}]{Chapitre II. Objection de Nebridius contre les manichéens.}
\noindent \pn{3}Il me suffisait, Seigneur, pour confondre ces imposteurs dupes, et ces bavards muets, car leur bouche est toujours muette pour votre Verbe ; il me suffisait de cette objection que Nebridius, à Carthage même, leur présentait d’ordinaire, et qui avait fortement remué tous ceux qui, comme moi, l’avaient entendue. Qu’aurait pu faire contre vous, leur demandait-il, cette nation de ténèbres qu’ils vous opposent comme une armée ennemie, si vous n’eussiez pas voulu combattre contre elle ? Si l’on répond qu’elle pouvait nuire, vous n’êtes plus ni inviolable, ni incorruptible. Si l’on convient de son impuissance, on ne peut plus apporter aucune raison à cette lutte ; lutte si opiniâtre, qu’une partie de vous-même, un de vos membres, une production de votre propre substance engagée parmi ces puissances ennemies et les natures indépendantes de votre création, s’y trouve infectée d’une telle corruption, que, précipitée de la béatitude dans la misère, elle a besoin d’un libérateur et d’un purificateur : or, à les en croire, cette partie de vous-même est l’âme de l’homme, que votre Verbe vient, libre, délivrer de ses chaînes ; pur, de ses souillures ; intact, de sa corruption, et toutefois corruptible lui-même, puisqu’il n’est qu’une seule et même substance avec elle.\par
Donc, s’ils reconnaissent que tout ce que vous êtes, c’est-à-dire la substance dont vous êtes, est incorruptible, toutes leurs hypothèses sont fausses. et odieuses. S’ils vous tiennent pour corruptible, cela seul est un blasphème, abominable à proférer. C’était assez pour se presser la poitrine avec dégoût et vomir ces pernicieux docteurs, qui, renfermés dans un cercle dont ils ne pouvaient sortir sans un horrible sacrilège de cœur et de langue, étaient condamnés à penser et à parler ainsi de vous.
\section[{Chapitre III, Peine qu’il éprouve à concevoir l’origine du mal.}]{Chapitre III, Peine qu’il éprouve à concevoir l’origine du mal.}
\noindent \pn{4}Mais tout en vous reconnaissant incapable de souillure, d’altération et de changement, si ferme que je fusse dans la croyance que vous êtes notre Seigneur, vrai Dieu, créateur de nos âmes et de nos corps, et non-seulement des âmes et des corps, mais de tout être et de toute chose, je ne saisissais pas encore toutefois le nœud de l’origine du mal. Et néanmoins, quelle qu’elle fût, je sentais que je devais conduire mes réflexions avec assez de prudence pour ne pas être réduit à trouver le Dieu immuable sujet au changement, et à ne point me laisser surprendre par l’objet de ma poursuite. Et j’y songeais avec sécurité, certain qu’il n’y avait qu’erreur dans les discours de ces hommes que je fuyais de toute mon âme parce qu’il était évident pour moi qu’ils recherchaient la cause du mal en esprit de malice, aimant mieux croire votre substance susceptible de le souffrir, que la leur capable de le faire.\par
\pn{5}Et je m’appliquais à saisir cette vérité souvent affirmée devant moi, que le libre arbitre de la volonté est la cause du mal de nos actions, et l’équité de vos jugements, du mal de nos souffrances. Mais ici ma faible vue s’obscurcissait. En vain je travaillais à retirer les yeux de mon âme de cet abîme de ténèbres, j’y plongeais de nouveau ; et je réitérais mes efforts, et je plongeais toujours.\par
Une chose me soulevait un peu vers votre lumière, c’est que je n’étais pas plus certain de vivre que d’avoir une volonté. Ainsi, quand je voulais ou ne voulais pas, j’avais toute certitude que ce n’était pas autre que moi qui voulait ou ne voulait pas ; et je soupçonnais déjà que là résidait la cause de mon péché. Quant aux actes où je me portais malgré moi, je me sentais plutôt souffrir qu’agir, et je présumais que c’était moins une faute qu’un châtiment, dont je me reconnaissais justement frappé, en songeant à votre justice.\par
Mais je me demandais ensuite : Qui m’a fait ? n’est-ce pas mon Dieu qui est bon, qui est la bonté même ? D’où m’est venu de vouloir le mal, de ne pas vouloir le bien, mon crime,   mon supplice ? Qui a donc semé et planté en moi ce grain d’amertume, moi dont tout l’être est venu de mon Dieu, souverainement doux. Si le diable en est l’auteur, d’où lui-même est-il le diable ? Que si, par la malice de sa volonté, d’ange il est devenu démon, d’où lui est venue cette volonté mauvaise qui l’a fait diable, lui que son créateur, souverainement bon, avait fait ange de bonté ? Et ces pensées étaient un poids mortel qui me coulait à fond, mais toutefois je ne descendais pas jusqu’au gouffre d’horreur, où l’on ne vous confesse plus, où l’on vous soumet au mal pour ne pas reconnaître le crime de l’homme.
\section[{Chapitre IV, Dieu étant le souverain bien est nécessairement incorruptible.}]{Chapitre IV, Dieu étant le souverain bien est nécessairement incorruptible.}
\noindent \pn{6}Je faisais donc tous mes efforts pour découvrir le reste, comme j’avais déjà découvert que l’incorruptible est meilleur que le corruptible, vous reconnaissant ainsi, qui que vous fussiez, pour incorruptible. Car jamais esprit n’a pu et ne pourra concevoir rien de meilleur que vous, suprême et souverain Bien. Or, comme il est d’évidente certitude que l’incorruptible est préférable au corruptible, préférence qui alors même ne me semblait pas douteuse, j’aurais pu saisir par la pensée quelque chose de meilleur que mon Dieu, si- lui n’eût été l’incorruptible.\par
Ainsi persuadé de la prééminence de l’incorruptible sur le corruptible, c’est dans cette excellence que je devais vous chercher ; c’est par là que je devais concevoir d’où procède le mal, c’est-à-dire la corruption même, qui ne peut nullement atteindre votre substance, car la corruption n’a aucune prise sur notre Dieu, ni par sa volonté, ni par la nécessité, ni par survenance fortuite, parce qu’il est Dieu, qu’il ne veut que le bien, et qu’il est lui-même le bien essentiel, et que se corrompre n’est plus de l’essence du bien. Et rien ne vous contraint d’agir malgré vous, parce que votre volonté n’est pas plus grande que votre puissance ; et pour qu’elle le fût, il faudrait que vous fussiez plus grand que vous-même, car la volonté, car la puissance de Dieu, c’est Dieu même. Et qui peut vous surprendre, vous qui connaissez tout ; rien ne pouvant exister que par votre connaissance ? Et faut-il tant s’arrêter à chercher pourquoi cette substance, qui est Dieu, est incorruptible, puisque si elle ne l’était pas, elle ne serait pas Dieu ?
\section[{Chapitre V, Ses doutes sur l’origine du mal.}]{Chapitre V, Ses doutes sur l’origine du mal.}
\noindent \pn{7}Et je cherchais la source du mal, et je la cherchais mal, et je n’apercevais pas le mal de ma recherche même, et je faisais comparaître aux regards de mon esprit la création universelle, et tout ce qui est visible dans son étendue, la terre, la mer, l’air, les astres, les plantes et les animaux mortels ; et tout ce qui est invisible, comme le firmament, les anges et les substances spirituelles ; et mon imagination les distribuait en divers lieux comme des êtres corporels. Et je faisais de votre création une grande masse que je classais par espèces de corps, ou réels, ou que mon erreur substituait aux esprits. Et cette masse, je me la représentais immense, non pas selon son immensité réelle qu’il m’était impossible d’atteindre, mais selon les seules limites que lui assignait mon imagination. Et je me la représentais, Seigneur, de toutes parts environnée et pénétrée de votre essence ; et je me figurais une mer sans fond et sans rivage, solitaire dans l’infini, qui contiendrait une éponge d’une immensité finie, et toute pleine de l’immense mer.\par
Ainsi je croyais vos créatures finies, pleines de votre infini, et je me disais : Voici Dieu, voilà ses créatures, Dieu bon, infiniment meilleur qu’elles, mais dont la bonté n’a pu les faire que bonnes, et c’est ainsi qu’il les environne et les remplit. Où est donc le mal, d’où vient-il, et par où s’est-il glissé ? quelle est sa racine ? quel est son germe ? Mais peut-être, n’est-il pas. Pourquoi donc redoutons-nous, pourquoi fuyons-nous ce qui n’est pas ? Et si notre crainte est vaine, cette crainte même est un mal ; c’est un mal que ce néant qui so1licite et tourmente notre cœur, mal d’autant plus pénible, qu’avec moins de sujet de craindre il nous livre à la crainte. Ainsi donc, ou nous avons la crainte du mal, ou nous avons le mal de la crainte.\par
Et d’où vient cela ? Car Dieu tout bon n’a rien fait que de bon Bien souverain, ses créatures, il est vrai, ne sont que des participations diminuées de sa bonté ;mais, toutefois, Créateur et créatures, tout est bon. D’où procède enfin le mal ? Est-ce de la matière, qu’il a mise en œuvre ? Elle recélait peut-être   lorsqu’il lui donna la forme et l’ordre, un élément mauvais, qu’il y laissa sans le convertir en bien. Et pourquoi ? Etait-il impuissant à convertir, à changer l’essence de cette matière, pour qu’il n’y restât aucun vestige de mal, lui qui est Tout-Puissant ? Pourquoi a-t-il voulu tirer quelque chose d’une pareille matière, et pourquoi, avec cette toute-puissance, ne l’a-t-il pas plutôt réduite au néant ? Pouvait-elle donc exister contre sa volonté ? Que si elle était éternelle, pourquoi l’a-t-il laissée ainsi tout une éternité et s’est-il décidé si tard à en faire quelque chose ? Et s’il lui est venu soudaine volonté de faire, que n’a-t-il fait plutôt qu’elle cessât d’être, et que lui seul fût, comme le Bien véritable, souverain, infini ? Ou enfin, s’il n’était pas bien que la main de celui qui est tout bon demeurât stérile d’œuvre bonne, ne devait-il pas dissiper et rendre au néant cette matière mauvaise pour en instituer une bonne, dont il eût créé toutes choses ? car il ne serait pas tout-puissant s’il ne pouvait rien faire de bon qu’à l’aide de cette matière que lui-même n’aurait pas faite.\par
Et voilà tout ce que roulait de pensers mon pauvre cœur, gros de tous les mordants soucis dont le pénétraient la crainte de la mort et la tristesse de n’avoir point trouvé la vérité. Je portais néanmoins, enracinée dans mon âme, la foi de l’Église catholique en votre Christ notre Sauveur et Maître ; et bien qu’elle fût encore en moi avec des défauts et des fluctuations illégitimes, elle tenait pourtant dans mon esprit, et y prenait chaque jour davantage.
\section[{Chapitre VI, Vaines prédictions des astrologues.}]{Chapitre VI, Vaines prédictions des astrologues.}
\noindent \pn{8}J’avais déjà rejeté loin les trompeuses prédictions des astrologues et l’impiété de leurs délires. Oh ! que vos miséricordes, mon Dieu, en publient aussi vos louanges du fond des entrailles de mon âme ! C’est vous qui m’avez détrompé, et vous seul ; car qui nous ressuscite de la mort de toute erreur, que la vie qui ne saurait mourir ; que la sagesse, dont la lumière se suffisant à elle-même, éclaire les ténèbres des âmes, qui gouverne le monde et connaît jusqu’à la feuille qu’emporte le vent ? Vous avez pris en pitié mon obstination à combattre le sage vieillard Vindicianus, et Nebridius, ce jeune homme d’un esprit incomparable, lorsqu’ils soutenaient, l’un avec force, l’autre avec moins d’assurance, mais fréquemment, qu’il n’est point de science de l’avenir ; que si le sort dispose souvent selon les conjectures des hommes, ce n’est pas à la science des devins, mais à la multitude de leurs prophéties qu’il faut l’attribuer ; on peut prédire vrai à force de prédire. Vous m’avez donc amené un ami, assez peu savant en astrologie, mais zélé consulteur d’astrologues, quoiqu’il eût appris de son père un fait qui, à son insu, ruinait la vanité de cette science.\par
Cet homme, nommé Firminus, instruit dans les lettres et l’éloquence, me consultant un jour comme l’un de ses plus chers amis, sur, quelques grandes espérances qu’il bâtissait dans le siècle, pour savoir ce que j’en augurais d’après son horoscope, je ne refusai pas de lui donner mes conjectures et tout ce que ma pensée trouvait à tâtons, mais, inclinant déjà vers l’opinion de Nebridius, j’ajoutai que je commençais à tenir tout cela. pour vain et ridicule. Alors il me conta que son père, fort curieux de cette science, avait un ami voué à la même étude, et que, mettant en commun leur laborieuse passion pour ces puérilités, ils observaient chez eux le moment de la naissance des animaux domestiques, et précisaient en même temps la situation du ciel, pour fonder sur ces marques l’expérience de leur art.\par
Il disait donc avoir appris de son père, que lorsque sa mère était enceinte de lui Firminus, le sein d’une servante de cet ami grossit en même temps, ce qui ne put longtemps échapper au regard d’un maître si exact observateur de la naissance de ses chiens. Il arriva donc qu’ayant calculé les jour, heure et minute de la délivrance, l’un de sa femme, l’autre de sa servante, elles accouchèrent ensemble, en sorte qu’ils figurèrent nécessairement le même ascendant, l’un à son fils, l’autre à son esclave. Car, au moment où les deux femmes avaient ressenti les premières douleurs, ils s’informèrent mutuellement de ce qui se passait chez eux, et tinrent des serviteurs prêts à partir, au moment précis de la naissance. Maîtres absolus comme ils l’étaient, ils furent ponctuellement obéis. Et la rencontre des envoyés, disait-il, s’était opérée à une distance de l’une et de l’autre maison si précisément égale, qu’il fut de part et d’autre impossible de signaler la moindre différence dans l’aspect des astres, et dans le calcul des   moments. Et cependant Firminus, né dans un rang élevé parmi les siens, se promenait par les plus riantes voies du siècle, comblé de richesses et d’honneurs, tandis que l’esclave vivait toujours courbé sous le même fardeau de servitude, au témoignage même de celui qui le connaissait bien.\par
\pn{9}Ayant entendu ce récif, que le caractère du narrateur me rendait digne de foi, toutes les résistances de mes doutes tombèrent. Et aussitôt je cherchais à guérir Firminus de cette curiosité, lui montrant que j’aurais dû, pour lui dire vrai, remarquer, à l’aspect des astres de sa nativité, le rang que ses parents tenaient dans leur ville, son héritage considérable, sa naissance ingénue, son éducation honnête, son instruction libérale. Qui si cet esclave, né sous de communes influences, m’eût consulté, il eût fallu, pour lui annoncer aussi la vérité, que j’eusse reconnu, dans ces mêmes signes, la misère et la servilité de sa condition ; circonstances bien différentes et bien éloignées des premières. Or, comment l’observation des mêmes signes m’eût-elle fourni des réponses qui devaient être différentes pour être vraies, une réponse semblable étant une erreur ? D’où je conclus avec certitude que ce qui se dit de vrai après l’examen des constellations, se dit, non par science, mais par hasard, et que le faux doit être imputé, non à l’imperfection de l’art, mais au mensonge de tout calcul fondé sur le sort.\par
\pn{10}Ce récit ayant ouvert la voie à mes pensées, je ruminais en moi-même comment, en attaquant ceux qui trafiquent de telles rêveries, insensés que je désirais ardemment réfuter et couvrir de ridicule, je leur enlèverais jusqu’au moyen d’alléguer pour défense que Firminus m’avait abusé par un conte, ou que lui-même s’était laissé tromper par son père. Et je dirigeai mes réflexions sur ceux qui naissent jumeaux, dont souvent la naissance se suit de si près, que le moment d’intervalle, quelle que soit l’influence qu’ils lui prêtent dans l’ordre des événements, se joue des calculs de l’observation humaine et des figures que l’astrologue doit consulter pour la vérité de ses prédictions. Mais cette vérité même est un rêve. L’examen des mêmes signes lui eût fait tirer le même horoscope d’Esaü et de Jacob, dont la vie fut si différente. Sa prédiction eût donc été fausse. Car, pour dire la vérité, il aurait dû, de l’inspection des mêmes étoiles, augurer des fortunes différentes. Ce n’est donc pas la science, mais le hasard qui lui eût présenté la vérité.\par
C’est vous, Seigneur, juste modérateur de l’univers, c’est vous qui, par une action secrète, à l’insu de tous, consulteurs et consultés, faites sortir de l’abîme de vos justices une réponse conforme aux mérites cachés des âmes. Et que l’homme ne s’élève pas jusqu’à dire : Qu’est-ce donc ? pourquoi ? Qu’il se taise ! qu’il se taise ; car il est homme.
\section[{Chapitre VII, Tourments de son esprit dans la recherche de l’origine du mal.}]{Chapitre VII, Tourments de son esprit dans la recherche de l’origine du mal.}
\noindent \pn{11}Et déjà, ô mon libérateur, vous m’aviez affranchi de ces liens ; et j’étais encore engagé dans la recherche de l’origine du mal, et je ne trouvais pas d’issue. Mais vous ne permettiez pas aux tourmentes de ma pensée de m’enlever à la ferme croyance que vous êtes, et que votre substance est immuable, que vous êtes la providence et la justice des hommes, et que vous leur avez ouvert en Jésus-Christ, votre Fils, Notre-Seigneur, et dans les saintes Écritures fondées sur l’autorité de l’Église catholique, la voie de salut vers cette vie qui doit commencer à la mort.\par
Ces vérités sauves, et inébranlablement fortifiées dans mon esprit, je cherchais, avec angoisse, d’où vient le mal. Oh ! quelles étaient alors les tranchées de mon âme en travail ! quels étaient ses gémissements, mon Dieu ! Et vous-étiez là, écoutant, à mon insu. Et lorsque, dans le silence, je poursuivais ma recherche avec effort, c’étaient d’éclatants appels à votre miséricorde que ces muettes contritions de ma pensée.\par
Vous saviez ce que je souffrais, et nul ne le savait. Qu’était-ce, en effet, ce que ma parole en faisait passer dans l’oreille de mes plus chers amis ? La parole, le temps eût-il suffi pour leur faire entendre le bruit des flots de mon âme ? Mais ils entraient tous dans votre oreille, vous ne perdiez rien des rugissantes lamentations de ce cœur. Et mon désir était devant vous, et la lumière de mes yeux n’était plus avec moi (Ps. XXXVII, 9-11). Car elle était en moi, et j’étais hors de moi-même, Il n’est pas de lieu pour elle ; et je ne portais mon esprit que sur les objets qui occupent un lieu, et je n’y trouvais   pas où reposer, et je n’y pouvais demeurer, et dire : Cela suffit, je suis bien ; et il ne m’était plus permis de revenir où j’eusse été mieux. Supérieur à ces objets, inférieur à vous, je vous suis soumis, ô ma véritable joie, et vous m’avez soumis tout ce que vous avez fait au-dessous de moi.\par
Et tel est le tempérament de rectitude, la moyenne région où est le salut : demeurant l’image de mon Dieu, ma fidélité à vous servir m’eût assuré la domination sur mon corps. Mais mon orgueil s’est dressé contre vous, je me suis élancé contre mon Seigneur sous le bouclier d’un cœur endurci (Job, XV, 26), et tout ce que je foulais aux pieds s’est élevé au-dessus de ma tête, pour m’opprimer, sans trève, sans relâche. Tous ces corps, je les rencontrais en foule, en masse serrée, sur le passage de mes yeux ; je voulais rentrer dans ma pensée, et leurs images m’interceptaient le retour, et je croyais entendre : Où vas-tu, indigne et infâme ?\par
Et telles étaient les excroissances de ma plaie, parce que vous m’aviez humilié comme un blessé superbe (Ps. LXXXVIII, 11.) ; le gonflement de mon âme me séparait de vous, et l’enflure de ma face me fermait les yeux.
\section[{Chapitre VIII. Dieu entretenait son inquiétude jusqu’à ce qu’il connut la vérité.}]{Chapitre VIII. Dieu entretenait son inquiétude jusqu’à ce qu’il connut la vérité.}
\noindent \pn{12}Et vous, Seigneur, vous demeurez éternellement, mais votre colère contre nous n’est pas éternelle, puisque vous avez eu pitié de ma boue et de ma cendre, et que votre regard a daigné réformer toutes mes difformités.\par
Votre main piquait d’un secret aiguillon mon cœur agité pour entretenir son impatience, jusqu’à ce que l’évidence intérieure lui eût dévoilé votre certitude, et, mon enflure diminuait à votre contact puissant et caché, et l’oeil de mon âme, trouble et ténébreux, guérissait de jour en jour par le cuisant collyre des douleurs salutaires.
\section[{Chapitre IX, Il avait trouvé la divinité du verbe dans les livres des platoniciens, mais non pas l’humilité de son incarnation.}]{Chapitre IX, Il avait trouvé la divinité du verbe dans les livres des platoniciens, mais non pas l’humilité de son incarnation.}
\noindent \pn{13}Et voulant d’abord me faire connaître comment vous résistez aux superbes et donnez votre grâce aux humbles (I Pierre, V, 5) et quelles prodigalités de miséricorde a répandues sur la terre l’humilité de votre Verbe fait chair et habitant parmi nous, vous m’avez remis, par les mains d’un homme, monstre de vaine gloire, plusieurs livres platoniciens, traduits de grec en latin, où j’ai lu, non en propres termes, mais dans une frappante identité de sens, appuyé de nombreuses raisons,\par

\begin{quoteblock}
\noindent « qu’au commencement était le Verbe ; que le Verbe était en Dieu, et que le Verbe était Dieu ; qu’il était au commencement en Dieu, que tout a été fait par lui et rien sans lui : que ce qui a été fait a vie en lui ; que la vie est la lumière, des hommes, que cette lumière luit dans les ténèbres, et que les ténèbres ne l’ont point comprise. » \emph{Et que l’âme de l’homme,} « tout en rendant témoignage de la lumière, n’est pas elle-même la lumière, mais que le Verbe de Dieu, Dieu lui-même, est la vraie lumière qui éclaire tout homme venant en ce monde ; » \emph{et} « qu’il était dans le monde, et que le monde a été fait par lui, et que le monde ne l’a point connu. Mais qu’il soit venu chez lui, que les siens ne l’aient pas reçu, et qu’à ceux qui l’ont reçu il ait donné le pouvoir d’être faits enfants de Dieu, à ceux-là qui croient en son nom. »\end{quoteblock}

\noindent C’est ce que je n’ai pas lu dans ces livres.\par
\pn{14}J’y ai lu encore :\par

\begin{quoteblock}
\noindent « Que le Verbe-Dieu est né non de la chair, ni du sang, ni de la volonté de l’homme, ni de la volonté de la chair ; qu’il est né de Dieu. » \emph{Mais} « que le Verbe se soit fait chair, et qu’il ait habité parmi nous (Jean, I, 1-14). »\end{quoteblock}

\noindent C’est ce que je n’y ai pas lu.\par
J’ai découvert encore plus d’un passage témoignant par diverses expressions,\par

\begin{quoteblock}
\noindent « que le Fils consubstantiel au Père, n’a pas cru faire un larcin d’être égal à Dieu, »\end{quoteblock}

\noindent parce que naturellement il n’est pas autre que lui. Mais qu’il\par

\begin{quoteblock}
\noindent « se soit anéanti, abaissé à la forme d’un esclave, à la ressemblance de l’homme, qu’il ait été trouvé homme dans tout ce qui a paru de lui, qu’il se soit humilié, qu’il se soit fait obéissant jusqu’à la mort, à la mort de la   croix ! — pourquoi Dieu l’a ressuscité des « morts et lui a donné un nom au-dessus de tout autre nom, afin qu’à ce nom de Jésus tout genou fléchisse au ciel, sur la terre dans les enfers, et que toute langue confesse que Jésus Notre-Seigneur est dans la gloire de Dieu son Père (Philip. II, 6-11) ;»\end{quoteblock}

\noindent c’est ce que ces livres ne disent pas.\par
Qu’il est avant les temps , au delà des temps, dans une immuable pérennité, comme votre Fils, coéternel à vous ; que, pour être heureuses, les âmes reçoivent de sa plénitude(Jean I, 16), et que pour être sages, elles sont renouvelées par la communion de la sagesse résidant en lui ; cela est bien ici.\par

\begin{quoteblock}
\noindent « Mais qu’il soit mort dans le temps pour les impies (Rom. V, 6) ; que vous n’ayez point épargné votre Fils unique, et que pour nous tous vous l’ayez livré (Ibid. VIII, 32),»\end{quoteblock}

\noindent c’est ce qui n’est pas ici. Vous avez caché ces choses aux sages, et les avez révélées aux petits, afin de faire venir à lui les souffrants et les surchargés, pour qu’il les soulage. Car il est doux et humble de cœur (Matth. XI, 25, 28, 29), il conduit les hommes de douceur et de mansuétude dans la justice, il leur enseigne ses voies, et à la vue de notre humilité et de nos souffrances, il nous remet tous nos péchés ((Ps. XXIV, 9,18). Mais élevés sur le cothurne d’une doctrine soi-disant plus sublime, les hommes d’orgueil ne l’entendent point nous dire :\par

\begin{quoteblock}
\noindent « Apprenez de moi que je suis doux et humble de cœur, et vous trouverez le repos de vos âmes (Matth. XI, 29) »\end{quoteblock}

\noindent s’ils connaissent Dieu, ils ne l’honorent pas, ils ne le glorifient pas comme Dieu ; ils se dissipent dans la vanité de leurs pensées, et leur cœur insensé se remplit de ténèbres ; se proclamant sages, ils deviennent fous.\par
\pn{15}Ainsi cette lecture même me montrait la profanation de votre incorruptible gloire transportée à des idoles, aux statues formées à la ressemblance de l’homme corruptible, à l’image des oiseaux, des bêtes et des serpents (Rom. I, 21, 23) ; » fatal mets d’Egypte qui fait perdre à Esaü son droit d’aînesse (Genès. XXV, 33, 34.), et frappe de déchéance votre peuple premier-né, dont le cœur tourné vers ta terre de Pharaon, adorant une brute au lieu de vous, incline votre image, son âme, devant l’image d’un veau qui rumine son foin (Exod. XXXII, 1-6 ; Ps. CV, 19, 20.) !\par
Voilà ce que je trouvai dans ces écrits, mais je ne goûtai pas de cette profane nourriture ; car il vous a plu, Seigneur, de lever l’opprobre de Jacob, et de soumettre l’aîné au plus jeune (Rom. IX, 13) ; et vous avez appelé les nations à votre héritage. Et je venais à vous, sorti des rangs étrangers, et mes désirs se tournaient vers l’or que votre peuple emporta de la maison de servitude par votre commandement (Exod. III, 22 ; XI, 2), parce qu’il était à vous, où qu’il fût. N’avez-vous pas dit aux Athéniens par votre Apôtre :\par

\begin{quoteblock}
\noindent « C’est en lui que nous avons la vie, le mouvement et l’être (Act. XVII, 28), »\end{quoteblock}

\noindent comme plusieurs d’entre eux l’avaient déjà dit ? Et je ne m’arrêtai pas devant ces idoles égyptiennes servies dans l’or de vos vases par ces insensés\par

\begin{quoteblock}
\noindent « qui transforment la vérité divine en mensonge, et rendent à la créature le culte et l’hommage dus au Créateur (Rom. I, 25). »\end{quoteblock}

\section[{Chapitre X, Il découvre que Dieu est la lumière immuable.}]{Chapitre X, Il découvre que Dieu est la lumière immuable.}
\noindent \pn{16}Ainsi averti de revenir à moi, j’entrai dans le plus secret de mon âme, aidé de votre secours. J’entrai, et j’aperçus de l’oeil intérieur, si faible qu’il fût, au-dessus de cet oeil intérieur, au-dessus de mon intelligence, la lumière immuable ; non cette lumière évidente au regard charnel, non pas une autre, de même nature, dardant d’un plus vaste foyer de plus vifs rayons et remplissant l’espace de sa grandeur. Cette lumière était d’un ordre tout différent. Et elle n’était point au-dessus de mon esprit, ainsi que l’huile est au-dessus de l’eau, et le ciel au-dessus de la terre ; elle m’était supérieure, comme auteur de mon être ; je lui étais inférieur comme son ouvrage. Qui connaît la vérité voit cette lumière, et qui voit cette lumière connaît l’éternité. L’amour est l’oeil qui la voit,\par
O éternelle vérité ! ô vraie charité !ô chère éternité ! vous êtes mon Dieu ; après vous je soupire, jour et nuit ; et dès que je pus vous découvrir, vous m’avez soulevé, pour me faire voir qu’il me restait infiniment à voir, et que je n’avais pas encore les yeux pour voir. Et vous éblouissiez ma faible vue de votre vive et pénétrante clarté, et je frissonnais d’amour et d’horreur. Et je me trouvais bien loin de vous, aux régions souterraines où j’entendais à peine votre voix descendue d’en-haut :\par

\begin{quoteblock}
\noindent « Je suis la nourriture des forts ; crois, et tu me mangeras. Et je ne passerai pas dans ta substance,   comme les aliments de ta chair ; c’est toi qui passeras dans la mienne. »\end{quoteblock}

\noindent Et j’appris alors que vous éprouviez l’homme à cause de son iniquité, et qu’ainsi\par

\begin{quoteblock}
\noindent « vous aviez « fait sécher mon âme comme l’araignée (Ps. XXXVIII, 12). »\end{quoteblock}

\noindent Et je disais : N’est-ce donc rien que la vérité, parce qu’elle ne s’étend, à mes yeux, ni dans l’espace fini, ni dans l’infini ? Et vous m’avez crié de loin : Erreur, je suis celui qui est (Exod. III, 14) ! Et j’ai entendu, comme on entend dans le cœur, Et je n’avais plus aucun sujet de douter. Et j ‘eusse douté plutôt de ma vie que de l’existence de la vérité,\par

\begin{quoteblock}
\noindent « où atteint le regard de l’intelligence à travers les créatures visibles (Rom. I, 20). »\end{quoteblock}

\section[{Chapitre XI, Les créatures sont et ne sont pas.}]{Chapitre XI, Les créatures sont et ne sont pas.}
\noindent \pn{17}Et arrêtant ma vue sur tous les objets au-dessous de vous, je les reconnus, ni pour être absolument, ni pour n’être absolument pas. Ils sont, puisqu’ils sont par vous ; ils ne sont pas, puisqu’ils ne sont pas ce que vous êtes. II n’est en vérité que ce qui demeure immuablement. Donc,\par

\begin{quoteblock}
\noindent « il m’est bon de m’attacher à Dieu((Ps. LXXII, 20), »\end{quoteblock}

\noindent car, si je ne demeure en lui, je ne saurais demeurer en moi-même.\par

\begin{quoteblock}
\noindent « Et c’est lui qui, dans son immuable permanence, renouvelle toutes choses (Sag. VII, 27). Et vous êtes mon Seigneur, parce que vous n’avez pas besoin de mes biens (Ps. XV, 2). »\end{quoteblock}

\section[{Chapitre XII, Toute substance est bonne d’origine.}]{Chapitre XII, Toute substance est bonne d’origine.}
\noindent \pn{18}Et il me parut évident que ce n’est qu’en tant que bonnes, que les choses se corrompent. Que si elles étaient de souveraine ou de nulle bonté, elles ne pourraient se corrompre. Souverainement bonnes, elles seraient incorruptibles ; nullement bonnes, que laisseraient-elles à corrompre ? Car la corruption nuit, et ne saurait nuire sans diminuer le bien. Donc, ou la corruption n’est point nuisible, ce qui ne se peut, ou, ce qui est indubitable, tout ce qui se corrompt est privé d’un bien. Être privé de tout bien, c’est le néant. Être, et ne plus pouvoir se corrompre, serait un état meilleur : la permanence dans l’incorruptibilité. Or, quoi de plus extravagant que de prétendre que la perte de tout bien améliore ? Donc la privation de tout bien anéantit. Donc, ce qui est, tant qu’il est, est bon. Donc, tout ce qui est, est bon. Et ce mal, dont je cherchais partout l’origine, n’est pas une substance ; s’il était substance, il serait un bien. Car, ou il serait incorruptible, et sa bonté serait grande, ou il serait corruptible, ce qui ne se peut sans bonté.\par
Ainsi je le vis clairement : vous n’avez rien fait que de bon, et il n’est absolument aucune substance que vous n’ayez faite ; et vous n’avez pas doué toutes choses d’une égale bonté, c’est pourquoi elles sont toutes ; chacune en effet est bonne, et toutes ensemble sont très-bonnes, car notre Dieu a fait tout très-bon (Gen. I ; Eccl. XXXIX, 21).
\section[{Chapitre XIII, Toutes les créatures louent Dieu.}]{Chapitre XIII, Toutes les créatures louent Dieu.}
\noindent \pn{19}Et pour vous le mal n’est pas ; il n’est pas non plus pour l’universalité de votre œuvre ; car il n’est rien en dehors pour y pouvoir pénétrer par violence et altérer l’ordre que vous avez imposé. Mais dans le détail seulement, le mal, c’est quelque. disconvenance, convenance plus loin et devenant bien, de substances bonnes en soi. Et tous ces êtres sans convenances entre eux, conviennent à l’ordre inférieur que nous appelons la terre, qui a son atmosphère convenable de nuages et de vents.\par
Et loin de moi de désirer que ces choses ne soient pas, bien qu’à les voir séparément je les puisse désirer meilleures ! Mais fussent-elles seules, je devrais encore vous en louer, car, du fond de la terre, « les dragons et les abîmes témoignent que vous êtes digue de louanges ; et le feu, la grêlé, la neige, la glace et la trombe orageuse qui obéissent à votre parole ; les montagnes et les collines, les arbres fruitiers et les cèdres, les bêtes et les troupeaux, les oiseaux et les reptiles, les rois de la terre et les peuples, les princes et les juges de la terre, les jeunes gens et les vierges, les vieillards et les enfants, glorifient votre nom.\par
Et à la pensée que vous êtes également loué au ciel,\par

\begin{quoteblock}
\noindent « que dans les hauteurs infinies, ô mon Dieu ! vos anges et vos puissances chantent vos louanges ; que le soleil, la lune, les étoiles et la lumière, les cieux des cieux, et les eaux qui planent sur les cieux , publient votre   nom (Ps. CXLVIII, 1-12), »\end{quoteblock}

\noindent je ne souhaitais plus rien de meilleur : car embrassant l’ensemble, je trouvais bien les êtres supérieurs plus excellents que les inférieurs, mais l’ensemble, après mûr examen, plus excellent que les supérieurs isolés.
\section[{Chapitre XIV, Il s’éveille enfin a la vraie connaissance de Dieu.}]{Chapitre XIV, Il s’éveille enfin a la vraie connaissance de Dieu.}
\noindent \pn{20}Il n’est pas en santé d’esprit celui qui trouve à reprendre dans votre création ; et mon jugement n’était pas sain, quand je m’élevais contre plusieurs de vos ouvrages. Et comme mon âme n’était pas assez hardie pour trouver à reprendre mon Dieu, elle refusait de reconnaître pour votre œuvre tout ce qui lui déplaisait. Et elle était tombée dans la vaine opinion des deux substances, et elle ne pouvait s’y reposer, et elle parlait un langage d’emprunt.\par
Et, au sortir de cette erreur, elle s’était fait un Dieu répandu dans un espace infini, et ce Dieu elle le prenait pour vous, et elle l’avait placé dans son cœur, et elle s’était faite de nouveau le temple de son idole, abominable à vos yeux. Mais lorsque vous eûtes, à mon insu, attiré sur vous ma tête appesantie,\par

\begin{quoteblock}
\noindent « et clos mes yeux pour qu’ils ne vissent plus la vanité, »\end{quoteblock}

\noindent je me reposais un peu de moi-même, et ma démence s’assoupit. Et je me réveillai en vous, et je vous vis infini, mais d’un autre infini, et cette vue ne devait rien à l’œil charnel.
\section[{Chapitre XV, Vérité et fausseté dans les créatures.}]{Chapitre XV, Vérité et fausseté dans les créatures.}
\noindent \pn{21}Et je jetai les yeux sur le reste, et j e vis que tout vous est redevable d’être, et que tout est fini en vous autrement qu’en un lieu, mais parce que vous tenez tout dans votre main toute vérité ; et tout est vrai, en tant qu’être, et la fausseté n’est que la créance à l’être de ce qui n’est pas. Et je reconnus que tout a sa convenance particulière, non - seulement de lieu, mais de temps ; et que vous, seul Être éternel, ne vous êtes pas mis à l’ouvrage après des séries incalculables de temps, parce que les espaces des temps, passés ou à venir, ne sauraient ni passer, ni venir, sans l’action de votre permanence.
\section[{Chapitre XVI, Ce que c’est que le péché.}]{Chapitre XVI, Ce que c’est que le péché.}
\noindent \pn{22}Et je sentis par expérience qu’il ne faut pas s’étonner que le pain, agréable à l’organe sain, afflige le palais blessé, et qu’aux yeux malades soit odieuse la lumière si aimable à l’oeil pur. Et votre justice déplaît aux hommes d’iniquité : comment donc pourraient leur plaire et la vipère et le vermisseau, créés par vous toutefois dans une bonté convenable à l’ordre inférieur avec lequel les impies ont d’autant plus d’affinité, qu’ils vous sont moins semblables, comme les bons tendent d’autant plus à l’ordre supérieur qu’ils sont plus semblables à vous ?\par
Et je cherchai ce que c’était que l’iniquité, et je trouvai qu’il n’y avait point là substance, mais hideuse prévarication de la volonté détournée de vous, ô mon Dieu, substance souveraine ; mais prostitution de toutes les puissances intérieures (Eccli. X, 10) et enflure au dehors.
\section[{Chapitre XVII, Par quels degrés il s’élève a la connaissance de Dieu.}]{Chapitre XVII, Par quels degrés il s’élève a la connaissance de Dieu.}
\noindent \pn{23}Et je m’étonnais de vous aimer, et non plus un fantôme au lieu de vous. Et je ne m’en tenais pas à jouir de mon Dieu, mais j’étais ravi vers vous par votre beauté, et bientôt un poids malheureux me détachait de vous, et je retombais sur ce sol en gémissant ; et ce poids, c’étaient les habitudes de la chair.\par
Mais votre souvenir était toujours avec moi, et je ne doutais nullement que vous ne fussiez le seul être à qui je dusse m’attacher, quoique je fusse encore loin de pouvoir m’attacher à vous ; parce que\par

\begin{quoteblock}
\noindent « la chair corruptible appesantit l’âme, et que cette maison de boue fait retomber l’esprit et abat l’essor de ses pensées (Sag. IX, 15). »\end{quoteblock}

\noindent J’étaie encore certain\par

\begin{quoteblock}
\noindent « que depuis la création de l’univers, vos vertus invisibles, votre « puissance éternelle et votre divinité, se révèlent à l’homme par l’intelligence de vos œuvres (Rom. I, 20). »\end{quoteblock}

\noindent Je cherchai donc d’où me venait cette admiration éclairée de la beauté des corps célestes ou terrestres, et quelle règle m’offrait son appui lorsque jugeant, selon la vérité, des objets muables, je disais : Cela doit être, cela ne doit pas être ainsi ; et je découvris,   au-dessus de mon intelligence muable, l’éternité immuable de la vérité.\par
Et je montai par degrés, du corps à l’âme qui sent par le corps, et de là à cette faculté intérieure à qui le sens corporel annonce la présence des objets externes, limite où s’arrête l’instinct des animaux ; j’atteignis enfin cette puissance raisonnable, juge de tous les rapports des sens.\par
Et voilà que se reconnaissant en moi sujette au changement, cette puissance s’élève à la pure intelligence, emmène sa pensée loin de l’habitude et des troublantes distractions de la fantaisie, pour découvrir quelle est la lumière qui l’inonde quand elle déclare hautement l’immuable préférable au muable. Et cet immuable, d’où le connaît-elle ? Car si elle n’en avait quelque connaissance, elle ne le préférerait point au muable. Enfin, elle jette sur l’Être même un tremblant coup d’oeil. Alors, « vos perfections invisibles se dévoilèrent à moi par l’intelligence de vos œuvres,» mais je n’y pus fixer mon regard émoussé. Rendu à ma faiblesse ordinaire, je n’avais plus avec moi qu’un amoureux souvenir et le regret de ne pouvoir goûter au mets dont le parfum m’avait séduit.
\section[{Chapitre XVIII, Jésus-Christ seul est la voie du salut.}]{Chapitre XVIII, Jésus-Christ seul est la voie du salut.}
\noindent \pn{24}Et je cherchais la voie où l’on trouve la force pour jouir de vous, et je ne la trouvais pas que je n’eusse embrassé\par

\begin{quoteblock}
\noindent « le Médiateur de Dieu et des hommes, Jésus-Christ homme (I Tim. II, 5) ; Dieu souverain, béni dans tous les siècles (Rom. IX, 5) ; »\end{quoteblock}

\noindent qui nous appelle par ces paroles\par

\begin{quoteblock}
\noindent « Je suis la voie, la vérité, la vie (Jean, XIV, 6) ; »\end{quoteblock}

\noindent et qui unit à notre chair une nourriture dont ma faiblesse était incapable. Car le Verbe s’est fait chair (Ibid. I, 14), afin que votre sagesse, par qui vous avez tout créé, devînt le lait de notre enfance.\par
Et je n’étais pas humble, pour connaître mon humble maître Jésus-Christ, et les profonds enseignements de son infirmité. Car votre Verbe, l’éternelle vérité, planant infiniment au-dessus des dernières cimes de votre création, élève à soi les infériorités soumises. C’est dans les basses régions qu’il s’est bâti avec notre boue une humble masure, pour faire tomber du haut d’eux-mêmes ceux qu’il voulait réduire, afin de les amener à lui, guérissant l’orgueil au profit de l’amour. Il a voulu que leur foi en eux cessât de les égarer, qu’ils s’humiliassent dans leur infirmité, en voyant à leurs pieds, infirme sous les haillons de notre tunique charnelle, la Divinité même, et que las, se couchant sur elle, elle les enlevât avec elle en se relevant.
\section[{Chapitre XIX, Il prenait jésus-christ pour un homme d’éminente sagesse.}]{Chapitre XIX, Il prenait jésus-christ pour un homme d’éminente sagesse.}
\noindent \pn{25}Mais je pensais autrement, et mes sentiments sur Notre-Seigneur Jésus-Christ étaient ceux que l’on peut avoir d’un homme éminent en sagesse, d’un homme incomparable ; sa miraculeuse naissance d’une vierge, son dévouement tout divin pour nous, avaient, suivant moi, investi son enseignement de cette autorité souveraine qui inspirait, à son exemple, le mépris des biens temporels en vue du gain de l’immortalité.\par
Mais tout ce qu’il y avait de mystère saint dans le Verbe fait chair, c’est ce que je ne pouvais pas même soupçonner. Seulement, la tradition écrite, m’apprenant qu’il a mangé, bu, dormi, marché ; qu’il a connu la joie et la tristesse, qu’il a conversé avec nous, me faisait comprendre que cette chair n’avait pu s’unir à votre Verbe que par l’intermédiaire de l’âme et de l’esprit de l’homme. Qui l’ignore, entre ceux qui connaissent l’immutabilité de votre Verbe ? Et alors même, toute la connaissance qu’il m’était possible d’en avoir ne me laissait sur ce point aucun doute. Car mouvoir les membres du corps au gré de la volonté, et ne les mouvoir plus ; être affecté de quelque passion, puis devenir indifférent ; exprimer par des signes de sages pensées, puis demeurer dans le silence, sont les traits distinctifs de la mobilité d’âme et d’esprit. Que si ces témoignages étaient faussement rendus de lui, tout le reste serait suspect de mensonge, et l’Écriture ne présenterait à la foi du genre humain aucune espérance de salut.\par
Or, ce qui est écrit étant vrai, je reconnaissais tout l’homme en Jésus-Christ, et non pas le corps seul de l’homme ou le corps et l’âme sans l’esprit ; je reconnaissais l’homme même. Mais ce n’était pas la Vérité en personne, c’était, selon moi, une sublime exaltation de la   nature humaine, admise en lui à une participation privilégiée de la sagesse, qui lui assurait la prééminence sur les autres hommes.\par
Alypius pensait que, dans leur croyance d’un Dieu vêtu de chair, les catholiques ne trouvaient en Jésus-Christ que le Dieu et la chair, et il ne croyait point qu’ils affirmassent en lui l’esprit et l’âme de l’homme. Et comme il était fermement persuadé que tout ce que la tradition conserve de lui dans la mémoire humaine n’avait pu s’accomplir en l’absence du principe vital et raisonnable, il ne venait qu’à pas lents à la foi catholique. Mais bientôt découvrant dans cette erreur l’hérésie des Apollinaristes, il embrassa avec joie la foi de l’Église.\par
Pour moi, je n’appris, je l’avoue, que quelque temps après , quelle dissidence sur le mystère du Verbe incarné s’élève entre la vérité catholique et le mensonge de Photin. Les contradictions de l’hérésie mettent en saillie les sentiments de votre Église, et produisent au jour la saine doctrine. « Il fallait qu’il y eût des hérésies, pour que les cœurs à l’épreuve fussent signalés entre les faibles (I Cor. XI, 19).
\section[{Chapitre XX, Les livres des platoniciens l’avaient rendu plus savant, mais plus vain.}]{Chapitre XX, Les livres des platoniciens l’avaient rendu plus savant, mais plus vain.}
\noindent \pn{26}Les livres des Platoniciens que je lisais alors, m’ayant convié à la recherche de la vérité incorporelle, j’aperçus, par l’intelligence de vos ouvrages, vos perfections invisibles. Et là, contraint de m’arrêter, je sentis que les ténèbres de mon âme offusquaient ma contemplation ; j’étais certain que vous êtes, et que vous êtes infini, sans cependant vous répandre par les espaces finis ou infinis ; mais toujours vous-même, dans l’intégrité de votre substance, et la constance de vos mouvements ; j’étais certain que tout être procède de vous, par cette seule raison fondamentale qu’il est ; certain de tout cela, j’étais néanmoins trop faible pour jouir de vous.\par
Et je parlais comme ayant la science, et si je n’eusse cherché la voie dans le Christ Sauveur, cette science n’allait qu’à ma perte. Je voulais déjà passer pour sage, tout plein encore de mon supplice, et je ne pleurais pas, et je m’enflais de ma sagesse.\par
Car où était cette charité qui bâtit sur les fondations de l’humilité, sur Jésus-Christ lui-même ? Et ces livres pouvaient-ils me l’enseigner ? Et, sans doute, vous me les avez fait tomber entre les mains avant que j’eusse médité vos Écritures, pour qu’il me souvînt en quels sentiments ils m’avaient laissé ; et que dans la suite, pénétré de la douceur de vos saints livres, pansé de mes blessures par votre main, je susse quel discernement il faut faire de la présomption et de l’aveu ; de qui voit où il faut aller, sans voir par où, et de qui sait le chemin conduisant non-seulement à la vue, mais à la possession de la patrie bienheureuse. Peut-être, formé d’abord par vos saintes Lettres, dont l’habitude familière m’eût fait goûter votre douce saveur, pour tomber ensuite dans la lecture de ces livres, j’eusse été détaché du solide fondement de la piété, ou bien même demeurant le cœur imbibé de sentiments salutaires, j’aurais pu croire que la lecture de ces philosophies suffit pour en produire de semblables.
\section[{Chapitre XXI, Il trouve dans l’écriture l’humilité et la vraie voie du salut.}]{Chapitre XXI, Il trouve dans l’écriture l’humilité et la vraie voie du salut.}
\noindent \pn{27}Je dévorai donc avidement ces vénérables dictées de votre Esprit, et surtout l’apôtre Paul ; et, en un moment, s’évanouirent ces difficultés où il m’avait paru quelquefois en contradiction avec lui-même, et son texte en désaccord avec les témoignages de la Loi et des Prophètes. Et je saisis l’unité de physionomie de ces chastes éloquences, et je connus cette joie où l’on tremble.\par
Et j’appris aussitôt que tout ce que j’avais lu de vrai dans ces autres livres s’enseignait ici avec l’idée toujours présente de votre grâce, afin que celui qui voit ne se glorifie pas, comme s’il n’eût pas reçu, non-seulement ce qu’il voit, mais aussi de voir. (Qu’a-t-il, en effet, qu’il n’ait reçu (I Cor. IV, 7) ?) afin que votre parole lui donne non-seulement les yeux pour voir, mais aussi la force pour embrasser votre immutabilité ; afin que le voyageur encore trop éloigné pour vous découvrir, prenne la bonne route, vienne à vous, vous voie et vous embrasse.\par
Que si l’homme se plaît dans la loi de Dieu, selon l’homme intérieur, que fera-t-il de cette autre loi , incarnée dans ses membres, qui   combat contre la loi de son esprit, et le traîne captif sous cette loi de péché qui lui est incorporée (Rom. VII, 22, 23. ) ? Car\par

\begin{quoteblock}
\noindent « vous êtes juste, Seigneur ; ce sont nos péchés, nos iniquités, nos offenses, qui ont appesanti sur nous votre main (Dan. III, 27-32).»\end{quoteblock}

\noindent Et votre justice nous a livrés à l’antique pécheur, au prince de la mort, qui a persuadé à notre volonté l’imitation de sa volonté déchue de votre vérité (Jean VIII, 44).\par
Que fera cet homme de misère ?\par

\begin{quoteblock}
\noindent « Qui le délivrera du corps de cette mort, sinon votre grâce par Jésus-Christ Notre-Seigneur (Rom. VII, 25), »\end{quoteblock}

\noindent que vous avez engendré coéternel à vous-même, et créé au commencement de vos voies (Prov. VIII, 22), en qui le prince du monde n’a rien trouvé digne de mort (Jean, XIV, 30) ; Victime innocente, dont le sang a effacé l’arrêt de notre condamnation (Coloss. II, 14).\par
Voilà où ces livres sont muets. Ces pages profanes nous offrent-elles cet air de piété, ces larmes de pénitence, ce sacrifice que vous aimez des tribulations spirituelles d’un cœur contrit et humilié (Ps. L. 19) : et le salut de votre peuple, et la cité votre épouse (Apoc. XXI, 2), et ce gage de l’Esprit-Saint (II Cor. V, 5), ce calice de notre rançon ? On n’y entend point ces cantiques :\par

\begin{quoteblock}
\noindent « Mon âme ne sera-t-elle point soumise à Dieu ? à Dieu dont elle attend son salut ? Car il est mon Dieu, mon Sauveur, mon Tuteur, et je ne serai plus ébranlé (Ps. LXI, 2). »\end{quoteblock}

\noindent Personne n’y entend cet appel :\par

\begin{quoteblock}
\noindent « Venez à moi, vous tous qui êtes affligés. (Matth. XI, 28, 29, 35) »\end{quoteblock}

\noindent Ils dédaignent, ces superbes, d’apprendre de lui qu’il est doux et humble de cœur. C’est là ce que vous avez caché aux sages, aux savants, et révélé aux humbles.\par
Oui, autre chose est d’apercevoir du haut d’un roc sauvage la patrie de la paix, sans trouver le chemin qui y mène, et de s’épuiser en vains efforts, par des sentiers perdus, pour échapper aux embûches de ces fugitifs, déserteurs de Dieu , guerroyant contre l’homme sous la conduite de leur prince tout ensemble lion et dragon ; autre chose, de suivre la véritable route, protégée par l’armée du souverain empereur, où n’osent marauder les transfuges de la milice céleste : car cette voie ils l’évitent comme un supplice. Et ma substance s’assimilait merveilleusement ces vérités : à la lecture du moindre de vos apôtres (I Cor. XV, 9), je considérais vos œuvres, et j’admirais (Habac. III, 2).
\chapterclose


\chapteropen
 \chapter[{VIII. La conversion}]{VIII. La conversion}\phantomsection
\label{VIII}\renewcommand{\leftmark}{VIII. La conversion}


\begin{argument}\noindent Arrivé à la trente-deuxième année, il va trouver le vieillard Simplicianus. — Il apprend la conversion de Victorinus, rhéteur célèbre. — Potitianus lui fait le récit de la vie de saint Antoine. — Agitation de son âme pendant ce récit. — Lutte entre la chair et l’esprit. — Derniers combats. — Il se rend à cette voix du ciel : Prends, lis ! Prends, lis !
\end{argument}


\chaptercont
\section[{Chapitre premier, Augustin va trouver le vieillard Simplicianus.}]{Chapitre premier, Augustin va trouver le vieillard Simplicianus.}
\noindent \pn{1}Mon Dieu, que mes souvenirs soient des actions de grâces, et que je publie vos miséricordes sur moi ! Que toutes mes puissances intérieures se pénètrent de votre amour, qu’elles s’écrient :\par

\begin{quoteblock}
\noindent « Seigneur, qui est semblable à vous ? (Ps. XXXIV, 10)»\end{quoteblock}

\noindent Vous avez brisé mes liens ; que mon cœur vous sacrifie un sacrifice de louange (Ps. CXV, 17). Je raconterai comment vous les avez brisés, et tous ceux qui vous adorent diront à ce récit : Béni soit le Seigneur au ciel et sur la terre ! Grand et admirable est son nom.\par
Vos paroles s’étaient gravées au fond de mon âme, et votre présence l’assiégeait de toutes parts. J’étais certain de votre éternelle vie, quoiqu’elle ne m’apparût qu’en énigme et comme en un miroir (I Cor. XIII, 12). Il ne me restait plus aucun doute que votre incorruptible substance ne fût le principe de toute substance, et ce n’était pas plus de certitude de vous, mais plus de stabilité en vous que je désirais. Car dans ma vie temporelle tout chancelait, et mon cœur était à purifier du vieux levain ; et la voie, le Sauveur lui-même me plaisait, mais je redoutais les épines de son étroit sentier.\par
Et votre secrète inspiration me fit trouver bon d’aller vers Simplicianus, qui me semblait un de vos fidèles serviteurs ; en lui résidaient les lumières de votre grâce. J’avais appris que dès sa jeunesse il avait vécu dans la piété la plus fervente. Il était vieux alors, et ces long jours, passés dans l’étude de vos voies, me garantissaient sa savante expérience ; et je ne fus pas trompé. Je voulais, en le consultant sur les perplexités de mon âme, savoir de lui le traitement propre à la guérir, à la remettre dans votre chemin.\par
\pn{2}Car je voyais bien votre Église remplie, mais chacun y suivait un sentier différent. Je souffrais de vivre dans le siècle, et je m’étais à charge à moi-même ; l’ardeur de mes passions déjà ralentie ne trouvait plus dans l’espoir des honneurs et de la fortune un aliment à la patience d’un joug si lourd. Ces espérances perdaient leurs délices, au prix de votre douceur et de la beauté de votre maison que j’aimais (Ps. XXV, 8). Mais le lien le plus fort qui me retînt, c’était la femme. Et l’Apôtre ne me défendait pas le mariage, quoiqu’il nous convie à un état plus parfait, lui qui veut que tous les hommes soient comme il était lui-même (I Cor. VII, 7).\par
Trop faible encore, je me cherchais une place plus douce ; aussi je me traînais dans tout le reste, plein de langueur, rongé de soucis et pressentant certains ennuis, dont je déclinais le fardeau, dans cette vie conjugale qui enchaînait tous mes vœux. J’avais appris de la bouche de la Vérité même, qu’il est des eunuques volontaires pour le royaume des cieux : mais,\par

\begin{quoteblock}
\noindent « entende, qui peut entendre, (Matth. XIX, 12) »\end{quoteblock}

\noindent ajoute l’Homme-Dieu.\par

\begin{quoteblock}
\noindent «Vanité que l’homme qui n’a pas la science de Dieu, à qui la vue du bien n’a pas dévoilé celui qui est (Sag. XIII, 1).»\end{quoteblock}

\noindent J’étais déjà sorti de ce néant. Je m’élevais plus haut ; guidé parle témoignage universel de votre création, je vous avais trouvé, ô mon Créateur, et en vous votre Verbe, Dieu un avec vous et le Saint-Esprit, par qui vous avez tout créé. Il est encore une autre sorte d’impies qui connaissent Dieu, mais sans le glorifier comme Dieu (Rom. I, 21), sans lui rendre hommage. Voilà le   précipice où j’étais tombé, et votre droite m’en retira (Ps. XVII, 36) et me mit en voie de convalescence. Car, vous avez dit à l’homme :\par

\begin{quoteblock}
\noindent « La piété est la vraie science (Job. XXVIII, 28). Ne désire point passer pour sage (Prov. III, 7), « parce que ceux qui se proclamaient sages sont devenus fous (Ro. I, 21,22). »\end{quoteblock}

\noindent Et j’avais déjà trouvé la perle précieuse qu’il fallait acheter au prix de tous mes biens (Matth. 13, 46), et j’hésitais encore.
\section[{Chapitre II, Simplicianus lui raconte la conversion de Victorinus-le-rhéteur.}]{Chapitre II, Simplicianus lui raconte la conversion de Victorinus-le-rhéteur.}
\noindent \pn{3}J’allai donc vers Simplicianus, père selon la grâce de l’évêque Ambroise, qui l’aimait véritablement comme un père. Je le fis entrer dans le dédale de mes erreurs. Et lorsque je lui racontai que j’avais lu quelques ouvrages platoniciens, traduits en latin par Victorinus, rhéteur à Rome, qui, m’avait-on dit, était mort chrétien, il me félicita de n’être point tombé sur ces autres philosophes pleins de mensonges et de déceptions, professeurs de science charnelle (Coloss. II, 8), tandis que la doctrine platonicienne nous suggère de toutes les manières Dieu et son Verbe. Puis, pour m’exhorter à l’humilité du Christ , cachée aux sages et révélée aux petits (Matth. XI, 25), il réunit tous ses souvenirs sur ce même Victorinus, qu’il avait intimement connu pendant son séjour à Rome. Ce qu’il ma dit de lui, je ne le tairai pas. Adorable chef-d’œuvre de puissance et de grâce ! Ce vieillard, si docte en toute science libérale, qui avait lu, discuté, éclairci tant de livres écrits par les philosophes ; maître de tant de sénateurs illustres, à qui la gloire de son enseignement avait mérité l’honneur le plus rare aux yeux de la cité du monde une statue sur le Forum ; jusqu’au déclin de son âge, adorateur des idoles, initié aux mystères sacriléges, si chers alors à presque toute cette noblesse, à ce peuple de Rome, honteusement épris de tant de monstres divinisés, et d’Isis, et de l’aboyeur Anubis, qui, un jour, avaient levé les armes contre Neptune, Vénus et Minerve (Enéid. Liv. VIII, 678-700) ; vaincus à qui Rome victorieuse sacrifiait, abominables dieux que ce Victorinus avait défendus tant d’années de sa bouche prostituée à la terre ; merveille ineffable ! ce vieillard n’a point eu honte de se faire l’esclave de votre Christ, d’être lavé comme celui qui vient de naître, à la source pure ; il a plié sa tête au joug de l’humilité, et l’orgueil de son front à l’opprobre de la croix !\par
\pn{4}Seigneur, Seigneur, ô vous qui avez abaissé les cieux et en êtes descendu, qui avez touché les montagnes et les avez embrasées (Ps. CXLIII, 5), par quels charmes vous êtes-vous insinué dans cette âme ? Il lisait, me dit Simplicianus, la sainte Écriture, il faisait une étude assidue et profonde de tous les livres chrétiens, et disait à Simplicianus, loin du monde, en secret et dans l’intimité\par

\begin{quoteblock}
\noindent « Sais-tu que me voilà chrétien ? Je ne le croirai pas, répondait son ami, je ne te compterai pas au nombre des chrétiens, que je ne t’aie vu dans l’église du Christ. »\end{quoteblock}

\noindent Et lui reprenait avec ironie :\par

\begin{quoteblock}
\noindent « Sont-ce donc les murailles qui font le chrétien ? (Ps. XXVIII, 5) »\end{quoteblock}

\noindent Il répétait souvent qu’il était décidément chrétien ; même réponse de Simplicianus, même ironie des murailles. Il appréhendait de blesser ses amis, superbes démonolâtres, et il s’attendait que de ces sommets de Babylone, de ces cèdres du Liban que Dieu n’avait pas encore brisés, il roulerait sur lui d’accablantes inimitiés.\par
Mais en plongeant plus profondément dans ces lectures, il y puisa de la fermeté, il craignit\par

\begin{quoteblock}
\noindent « d’être désavoué du Christ devant ses saints anges, s’il craignait de le confesser devant les hommes (Matth. X, 33) ; »\end{quoteblock}

\noindent et reconnaissant qu’il serait coupable d’un grand crime s’il rougissait des sacrés mystères de l’humilité de votre Verbe, lui qui n’avait pas rougi des sacriléges mystères de ces démons superbes dont il s’était rendu le superbe imitateur, il dépouilla toute honte de vanité, et revêtit la pudeur de la vérité, et tout à coup, il surprit Simplicianus par ces mots : « Allons à l’église ; je veux être chrétien ! » Et lui, ne se sentant pas de joie, l’y conduisit à l’instant. Aussitôt qu’il eut reçu les premières instructions sur les mystères, il donna son nom pour être régénéré dans le baptême, à l’étonnement de Rome, à la joie de l’Église. Les superbes, à cette vue, frémissaient, ils grinçaient des dents, ils séchaient de rage (Ps. XCI, 10) mais votre serviteur, ô Dieu, avait son espérance au Seigneur, et il ne voyait plus les vanités et les folies du mensonge (Ps. XXXIX,5).\par
\pn{5}Puis, quand l’heure fut venue de faire la profession de foi, qui consiste en certaines paroles retenues de mémoire, et que récitent ordinairement d’un lieu plus élevé, en présence des   fidèles de Rome, ceux qui demandent l’accès de votre grâce ; les prêtres, ajouta Simplicianus, offrirent à Victorinus de réciter en particulier, comme c’était l’usage de le proposer aux personnes qu’une solennité publique pouvait intimider ; mais lui aima mieux professer son salut en présence de la multitude sainte. Car ce n’était pas le salut qu’il enseignait dans ses leçons d’éloquence, et pourtant il avait professé publiquement. Et combien peu devait-il craindre de prononcer votre parole devant l’humble troupeau, lui qui ne craignait pas tant d’insensés auditeurs de la sienne ?\par
Il monta ; son nom, répandu tout bas par ceux qui le connaissaient, éleva dans l’assemblée un murmure de joie. Et de qui, dans cette enceinte, n’était-il pas connu ? Et la voix contenue de l’allégresse générale frémissait : Victorinus ! Victorinus ! Un transport soudain, à sa vue, avait rompu le silence, le désir de l’entendre le rétablit aussitôt. Il prononça le symbole de vérité avec une admirable foi, et tous eussent voulu l’enlever dans leur cœur ; et tous l’y portaient dans les bras de leur joie et de leur amour.
\section[{Chapitre III, D’où vient que - l’on ressent tant de joie de la conversion des pécheurs.}]{Chapitre III, D’où vient que - l’on ressent tant de joie de la conversion des pécheurs.}
\noindent \pn{6}Dieu de bonté, que se passe-t-il dans l’homme pour qu’il ressente plus de joie du saint d’une âme désespérée et de sa délivrance d’un plus grand péril, que s’il eût toujours bien espéré d’elle, ou que le péril eût été moins grand ? Et vous aussi, Père des miséricordes, vous vous réjouissez plus d’un seul pénitent que de quatre-vingt-dix-neuf justes qui a’ont pas besoin de pénitence. Et nous, c’est avec une consolante émotion que nous apprenons que le bon pasteur rapporte sur ses épaules, à la joie des anges, la brebis égarée ; et que la drachme est rendue à votre trésor par la femme qui l’a retrouvée, et dont les voisines partagent le contentement. Et les solennelles réjouissances de votre maison font rouler des larmes dans les yeux qui ont lu que\par

\begin{quoteblock}
\noindent « votre Fils était mort, et qu’il est ressuscité, qu’il était perdu, et qu’il est retrouvé (Luc, XV.) »\end{quoteblock}

\noindent Vous vous réjouissez en nous et en vos anges, sanctifiés par votre charité sainte. Car vous, toujours le même, vous avez toujours la même connaissance de ce qui n’est, ni toujours, ni le même.\par
\pn{7}Que se passe-t-il donc dans l’âme qui lui fait trouver plus de joie à la recouvrance qu’en la possession continuelle de ce qu’elle aime ? Tout l’atteste, tout est plein de témoignages qui nous crient : Il est ainsi. Un empereur victorieux triomphe, et il n’eût vaincu s’il n’eût combattu. Et plus a-été grand le péril au combat, plus vive est l’allégresse dans le triomphe. Un vaisseau est battu de la tempête, le naufrage est imminent ; les matelots pâlissent aux portes de la mort : le ciel et la mer s’apaisent ; l’excès de la joie naît de l’excès de la crainte. Une personne aimée est malade, son pouls est de mauvais augure ; tous ceux qui désirent sa guérison sont malades de cœur : elle est sauvée, mais elle n’a pas encore recouvré ses forces pour marcher, et déjà c’est un bonheur tel qu’il n’en fut jamais lorsqu’elle jouissait de toute la vigueur de la santé.\par
Et les plaisirs mêmes de cette vie, ce n’est point seulement par les contrariétés qui surprennent notre volonté, mais encore au prix de certaines peines étudiées et volontaires, que nous les achetons. La volupté du boire et du manger n’existe qu’en tant que précédée de l’angoisse de la faim et de la soif. Et les ivrognes cherchent dans des aliments salés une irritation dont la boisson, qui l’apaise, fait un plaisir. Et la coutume veut que l’on diffère de livrer une fiancée, de peur que l’époux ne dédaigne la main que ses soupirs n’auraient pas longtemps attendue.\par
\pn{8}Ainsi, et dans l’abomination des voluptés humaines, et dans les plaisirs licites et permis, et dans la sincérité d’une amitié pure, et dans ce retour de l’enfant « qui était mort et qui est « ressuscité, qui était perdu et qui est retrouvé (Luc XV, 24, 32), toujours une grande joie est précédée d’un aiguillon douloureux. Quoi donc ! Seigneur mon Dieu, vous êtes à vous-même votre éternelle joie ; quelques êtres, autour, de vous, se réjouissent éternellement de vous, et cette partie du monde souffre une continuelle alternative de défaillance et d’accroissement, de guerre et de paix ? Est-ce la condition de son être ? est-ce ainsi que vous l’avez fait, quand, depuis les hauteurs des cieux jusqu’aux profondeurs de la terre, depuis le commencement jusqu’à la fin des siècles, depuis l’ange jusqu’au vermisseau, depuis le premier des mouvements jusqu’au dernier, vous avez placé toute sorte de biens, chacun en son lieu, et   réglé vos œuvres parfaites chacune en son temps ? Grand Dieu ! que vous êtes sublime dans les hauteurs et profond dans les abîmes ! Vous n’êtes jamais loin, et pourtant quelle peine pour retourner à vous !
\section[{Chapitre IV, Pourquoi les conversions célèbres doivent inspirer une joie plus vive.}]{Chapitre IV, Pourquoi les conversions célèbres doivent inspirer une joie plus vive.}
\noindent \pn{9}Agissez, Seigneur, faites ; réveillez-nous, rappelez-nous ; embrasez et ravissez ; soyez flamme et douceur ; aimons, courons. Combien reviennent à vous d’un enfer d’aveuglement plus profond que Victorinus, et s’approchent, et reçoivent le rayon de votre lumière ? Et ils ne le reçoivent qu’avec le pouvoir de devenir enfants de Dieu (1 Jean, I, 9,12). Mais, moins connus du monde, la joie de leur retour est moins vive, même en ceux qui les connaissent. La joie générale est individuellement plus féconde ; le feu gagne au contact, et la flamme s’élance. Et puis, les hommes connus de plusieurs autorisent et devancent de plus nombreuses conversions. C’est pourquoi leurs prédécesseurs mêmes se livrent à cette joie de prosélytisme qui en prévoit de nouvelles.\par
Car, loin de ma pensée que, sous votre tente, le riche ait la préséance sur le pauvre, et le puissant sur le faible, puisque vous avez fait choix des plus faibles pour confondre les forts ; et des objets du monde les plus vils et les plus méprisables, et de ce qui est comme n’étant pas, pour anéantir ce qui est (I Cor. I, 27, 28). Et cependant, le moindre de vos apôtres (Ibid. XV, 9), dont la voix a fait entendre cet oracle de votre sagesse, vainqueur de l’orgueil du proconsul Paul, qu’il fit passer sous le joug de douceur de votre Christ et enrôla sous les drapeaux du plus grand des rois, cet apôtre de Saul voulut s’appeler Paul (Act. XIII, 7, 12), en souvenir, de cet éclatant triomphe. Car l’ennemi est plus glorieusement vaincu dans celui qu’il possède avec plus d’empire, et par qui il en possède plusieurs. Il tient les grands par l’orgueil de leur renommée, et le vulgaire par l’autorité de leurs exemples.\par
Or, plus on aimait à se figurer le cœur de Victorinus comme une citadelle inexpugnable où Satan s’était renfermé, et sa langue comme un dard fort et acéré, dont il avait tué tant d’âmes, plus l’enthousiasme de vos enfants dut éclater, en voyant le fort enchaîné par notre Roi (Matth. XII, 29) ; ses vases conquis purifiés, consacrés à votre culte, et devenus les instruments du Seigneur pour toute bonne œuvre (II Tim. II, 21).
\section[{Chapitre V, Tyrannie de l’habitude.}]{Chapitre V, Tyrannie de l’habitude.}
\noindent \pn{10}L’homme de Dieu m’avait fait ce récit de Victorinus, et je brûlais déjà de l’imiter. Telle avait été l’intention de Simplicianus. Et quand il ajouta qu’au temps de l’empereur Julien où un édit défendit aux chrétiens d’enseigner les lettres et l’art oratoire, Victorinus s’était empressé d’obéir à cette loi, désertant l’école de faconde plutôt que votre Verbe, qui donne l’éloquence à la langue de l’enfant (Sag. X, 21), il ne me parut pas moins heureux que fort d’avoir trouvé tant de loisir pour vous.\par
C’est après un tel loisir que je soupirais, non plus dans les liens étrangers, mais dans les fers de ma volonté. Le démon tenait dans sa main mon vouloir, et il m’en avait fait une chaîne, et il m’en avait lié. Car la volonté pervertie fait la passion ; l’asservissement à la passion fait la coutume ; le défaut de résistance à la coutume fait la nécessité. Et ces nœuds d’iniquité étaient comme les anneaux de cette chaîne dont m’enlaçait le plus dur esclavage. Cette volonté nouvelle qui se levait en moi de vous servir sans intérêt, de jouir de vous, mon Dieu, seule joie véritable, cette volonté était trop faible pour vaincre la force invétérée de l’autre. Ainsi deux volontés en moi, une vieille, une nouvelle, l’une charnelle, l’autre spirituelle, étaient aux prises, et cette lutte brisait mon âme.\par
\pn{11}Ainsi ma propre expérience me donnait l’intelligence de ces paroles :\par

\begin{quoteblock}
\noindent « La chair convoite contre l’esprit et l’esprit contre la chair. (Galat. V, 17) »\end{quoteblock}

\noindent De part et d’autre, c’était toujours moi ; mais il avait plus de moi dans ce que j’aimais que dans ce que je haïssais en moi. Là, en effet, il n’y avait déjà presque plus de moi, car je le souffrais plutôt contre mon gré que je ne le faisais volontairement. Et cependant la coutume s’était par moi aguerrie contre moi, puisque ma volonté m’avait amené où je ne voulais pas Et de quel droit eussé-je protesté contre le juste châtiment inséparable de mon péché ? Et je n’avais plus alors l’excuse qui me faisait attribuer mon impuissance à mépriser le   siècle pour vous servir, aux indécisions de me doutes. Car j’étais certain de la vérité ; mais engagé à la terre, je refusais d’entrer à votre solde, et je craignais autant la délivrance des obstacles qu’il en faut craindre l’esclavage.\par
\pn{12}Ainsi, le fardeau du siècle pesait sur moi comme le doux accablement du sommeil ; et les méditations que j’élevais vers vous ressemblaient aux efforts d’un homme qui veut s’éveiller, et vaincu par la profondeur de sou assoupissement, y replonge. Et il n’est personne qui veuille dormir toujours, et la raison, d’un commun accord, préfère la veille ; mais souvent on hésite à secouer le joug qui engourdit les membres, et l’ennui du sommeil cède au charme plus doux que l’on y trouve, quoique l’heure du lever soit venue ; ainsi je ne doutais pas qu’il ne voulût mieux me livrer à votre amour que de m’abandonner à ma passion. Le premier parti- me plaisait, il était vainqueur ; je goûtais l’autre, et j’étais vaincu. Et je ne savais que répondre à votre parole :\par

\begin{quoteblock}
\noindent « Lève-toi, toi qui dort Lève-toi d’entre les morts, et le Christ t’illuminera (Ephés. V, 14) ! »\end{quoteblock}

\noindent Et vous m’entouriez d’évidents témoignages ; et convaincu de la vérité, je n’avais à vous opposer que ces paroles de lenteur et de somnolence. : Tout à l’heure ! encore un instant ! laissez-moi un peu ! Mais ce tout à l’heure devenait jamais ; ce laissez-moi un peu durait toujours.\par
Vainement je me plaisais en votre loi, selon l’homme intérieur, puisqu’une autre loi luttait dans ma chair contre la roi de mon esprit, et m’entraînait captif de la loi du péché, incarnée dans mes membres. Car la loi du péché, c’est la violence de la coutume qui entraîne l’esprit et le retient contre son gré, mais non contre la justice, puisqu’il s’est volontairement asservi. Malheureux homme ! qui me délivrera du corps de cette mort, sinon votre grâce par Jésus-Christ Notre Seigneur (Rom. VII, 22-25) ?
\section[{Chapitre VI, Récit de Potitianus.}]{Chapitre VI, Récit de Potitianus.}
\noindent \pn{13}Comment vous m’avez délivré de cette chaîne étroite de sensualité et de l’esclavage du siècle, je vais le raconter, à la gloire de votre nom, Seigneur, mon rédempteur et mon secours. Je vivais dans une anxiété toujours croissante, et sans cesse soupirant après vous. Je fréquentais votre Église, autant que me le permettait ce fardeau d’affaires qui me faisait gémir. Avec moi était Alypius, sorti pour la troisième fois de sa charge d’assesseur, attendant en liberté des acheteurs de conseils, comme j’avais des chalands d’éloquence, si toutefois l’éloquence est une marchandise que l’enseignement puisse livrer. Nous avions obtenu de l’amitié de Nebridius de suppléer comme grammairien notre cher Verecundus, citoyen de Milan, qui en avait témoigné le vif désir, nous demandant, au nom de l’amitié, quelqu’un de nous pour lui prêter fidèle assistance, dont il avait grand besoin.\par
Ce ne fut donc pas l’intérêt qui décida Nebridius ; les lettres, s’il eût voulu, lui offraient un plus bel avenir ; mais sa bienveillance lui fit un devoir de se rendre à notre prière ; doux et excellent ami ! Sa conduite fut un modèle de prudence ; il évita soigneusement d’être connu des personnes éminentes dans le siècle, épargnant ainsi toute inquiétude à son esprit, qu’il voulait conserver libre et assuré d’autant d’heures de loisir qu’il pourrait s’en réserver, pour rechercher la sagesse par méditation, lecture ou entretien.\par
\pn{14}Un jour qu’il était absent, je ne sais pourquoi, nous eûmes la visite, Alypius et moi, d’un de nos concitoyens d’Afrique, Potitianus, l’un des premiers officiers militaires du palais. J’ai oublié ce qu’il voulait de nous. Nous nous assîmes pour nous entretenir. II aperçut par hasard, sur une table de jeu qui était devant nous, un volume. Il le prit, l’ouvrit, c’était l’apôtre Paul. Il ne s’y attendait certainement pas, croyant trouver quelque ouvrage nécessaire à cette profession qui dévorait ma vie. Il sourit, et me félicita du regard, étonné d’avoir surpris auprès de moi ce livre, et ce livre seul. Car il était chrétien zélé, souvent prosterné, dans votre église, en de fréquentes et longues oraisons. Je lui avouai que cette lecture était ma principale étude. Alors, il fut amené par la conversation a nous parler d’Antoine, solitaire d’Egypte, dont le nom si glorieux parmi vos serviteurs nous était jusqu’alors inconnu. Il s’en aperçut et s’arrêta sur ce sujet ; il révéla ce grand homme à notre ignorance, dont il ne pouvait assez s’étonner.\par
Nous étions dans la stupeur de l’admiration au récit de ces irréfragables merveilles de si récente mémoire, presque contemporaines, opérées dans la vraie foi, dans l’Église catholique. Et nous étions tous surpris, nous d’apprendre, lui de nous apprendre ces faits extraordinaires.  \par
\pn{15}Et ses paroles roulèrent de là sur ces saints troupeaux de monastères, et les parfums de vertu divine qui s’en exhalent, sur ces fécondes aridités du désert, dont nous ne savions rien. Et à Milan même, hors des murs, était un cloître rempli de bons frères, élevé sous l’aile d’Ambroise, et nous l’ignorions. Il continuait de parler, et nous écoutions en silence ; et il en vint à nous conter, qu’un jour, à Trèves, l’empereur passant l’après-midi aux spectacles du cirque, trois de ses compagnons et lui allèrent se promener dans les jardins attenant aux murs de la ville ; et comme ils marchaient deux à deux, l’un avec lui, les deux autres ensemble, ils se séparèrent. Ceux-ci, chemin faisant, entrèrent dans une cabane où vivaient quelques-uns de ces pauvres volontaires, vos serviteurs, à qui le royaume des cieux appartient (Matth. V, III) , et ils trouvèrent un manuscrit de la vie d’Antoine.\par
L’un d’eux se met à lire ; il admire, son cœur brûle, et tout en lisant, il songe à embrasser une telle vie, à quitter la milice du siècle pour vous servir : ils étaient l’un et l’autre agents des affaires de l’empereur. Rempli soudain d’un divin amour et d’une sainte honte, il s’irrite contre lui-même, et jetant les yeux sur son ami :\par

\begin{quoteblock}
\noindent «Dis-moi, je te prie, où donc tendent tous nos travaux ? Que cherchons-nous ? pour qui portons-nous les armes ? Quel peut être notre plus grand espoir au palais que d’être amis de l’empereur ? Et dans cette fortune, quelle fragilité ! que de périls ! Et combien de périls pour arriver au plus grand péril ? Et puis, quand cela sera-t-il ? Mais, ami de Dieu, si je veux l’être, je le suis, et sur l’heure. »\end{quoteblock}

\noindent Il parlait ainsi, dans la crise de l’enfantement de sa nouvelle vie ; et puis, ses yeux reprenant leur course dans ces saintes pages, il lisait, et il changeait au dedans, là où votre oeil voyait, et son esprit se dépouillait du monde, comme on vit bientôt après. Et il lisait, et les flots de son âme roulaient frémissants ; il vit et prit le meilleur parti, et il était à vous déjà, lorsqu’il dit à son ami :\par

\begin{quoteblock}
\noindent « C’en est fait, je romps avec tout notre espoir ; je veux servir Dieu, et à cette heure, en ce lieu, je me mets à l’œuvre. Si tu n’es pas pour me suivre, ne me détourne pas. (Luc XIV, 26, 35) »\end{quoteblock}

\noindent L’autre répond qu’il veut aussi conquérir sa part de gloire et de butin. Et tous deux, déjà vos serviteurs, bâtissent la tour qui s’élève avec ce que l’on perd pour vous suivre. Potitianus et son compagnon, après s’être promenés dans une autre partie du jardin, arrivèrent, en les cherchant, à cette retraite, et les avertirent qu’il était temps de rentrer, parce que le jour baissait. Mais eux, déclarant leur dessein , comment cette volonté leur était venue et s’était affermie en eux, prièrent leurs amis de ne pas contrarier leur résolution, s’ils refusaient de la partager. Ceux-ci, ne se sentant pas changés, pleurèrent néanmoins sur eux-mêmes, disait Potitianus. Ils félicitèrent pieusement leurs camarades, se recommandant à leurs prières. Ils retournèrent au palais, le cœur traînant toujours à terre, et les autres, le cœur attaché au ciel, restèrent dans la cabane. Tous deux avaient des fiancées qui, à cette nouvelle, vous consacrèrent leur virginité.
\section[{Chapitre VII, Agitation de son âme pendant le récit de Potitianus.}]{Chapitre VII, Agitation de son âme pendant le récit de Potitianus.}
\noindent \pn{16}Tel fut le récit de Potitianus. Mais vous, Seigneur, pendant qu’il parlait vous me retourniez vers moi-même ; vous effaciez ce dos que je me présentais pour ne pas me voir, et vous me placiez devant ma face pour que je visse enfin toute ma laideur et ma difformité, et mes taches, et mes souillures, et mes ulcères. Et je voyais, et j’avais horreur, et impossible de fuir de moi ! Et si je m’efforçais de détourner mes yeux de moi, cet homme venait avec son récit ; et vous m’opposiez de nouveau à moi, et vous me creviez les yeux de moi-même, pour que mon iniquité me fût évidente et odieuse. Je la connaissais bien, mais par dissimulation, par connivence, je l’oubliais.\par
\pn{17}Alors aussi, plus je me sentais d’ardent amour pour ces confiances salutaires livrées sans réserve à votre cure, plus j’avais, au retour sur moi, de haine et d’imprécations contre moi-même. Tant d’années, tant d’existence taries ! Douze ans et plus, depuis cette dix-neuvième année de mon âge, où la lecture de l’Hortensius de Cicéron avait éveillé en moi l’amour de la sagesse ; et je différais encore de sacrifier ce vain bonheur terrestre à la poursuite de cette félicité dont la recherche seule, même sans possession, serait encore préférable à la découverte du plus riche trésor, à la royauté des nations, à l’empressement de ces nombreuses esclaves, les voluptés corporelles.\par
 Mais, malheureux que j’étais, malheureux au seuil même de l’adolescence, je vous avais demandé la chasteté, et je vous avais dit : Donnez-moi la chasteté et la continence, mais pas encore. Je craignais d’être trop tôt exaucé, trop tôt guéri de ce mal de concupiscence que j’aimais mieux assouvir qu’éteindre. Et je m’étais égaré dans les voies d’une superstition sacrilége ; et je n’y trouvais point de certitude, et je la préférais pourtant aux doctrines dont je n’étais pas le pieux disciple, mais l’ardent ennemi.\par
\pn{18}Et depuis, je n’avais remis de jour en jour, comme je croyais, à rejeter les espérances du siècle et m’attacher à vous seul, que faute d’apercevoir ce fanal directeur de ma course. Mais le jour était arrivé où je me trouvais tout nu devant moi, et ma conscience me criait : Où es-tu, langue, qui disais que l’incertitude du vrai t’empêchait seule de jeter là ton bagage de vanité ? Eh bien ! tout est certain maintenant ; la vérité te presse ; à de plus libres épaules sont venues des ailes qui emportent des âmes, à qui il n’a fallu ni le pesant labeur de tant de recherches, ni d’années de méditation.\par
Ainsi je me rongeais intérieurement, j’étais pénétré de confusion et de honte, quand Potitianus parlait. Son discours, et le motif de sa visite cessant, il se retira. Et alors, que ne me dis-je pas à moi-même ? De quels coups le fouet de mes pensées meurtrit mon âme, l’excitant à me suivre dans mes efforts pour vous joindre ? Et-elle était rétive. Elle refusait et ne s’excusait pas. Toutes les raisons étaient épuisées. Il ne lui restait qu’une peur muette : elle appréhendait comme la mort, de se sentir tirer la bride à l’abreuvoir de la coutume, où elle buvait une consomption mortelle.
\section[{Chapitre VIII, Lutte intérieure.}]{Chapitre VIII, Lutte intérieure.}
\noindent \pn{19}Alors, pendant cette violente rixe au logis intérieur, où je poursuivais mon âme dans le plus secret réduit de mon cœur, le visage troublé comme l’esprit, j’interpelle Alypius, je m’écrie : Eh quoi ! que faisons-nous là ?, N’as-tu pas entendu ? Les ignorants se lèvent ; ils forcent le ciel, et nous, avec notre science, sans cœur, nous voilà vautrés dans la chair et dans le sang ! Est-ce honte de les suivre ? N’avons-nous pas honte de ne pas même les suivre ? Telles furent mes paroles. Et mon agitation m’emporta brusquement loin de lui. Il se taisait, surpris, et me regardait. Car mon accent était étrange. Et mon front ; mes joues, mes yeux, le teint de mon visage, le ton de ma voix, racontaient bien plus mon esprit que les paroles qui m’échappaient.\par
Notre demeure avait un petit jardin dont nous avions la jouissance, comme du reste de la maison ; car le propriétaire, notre hôte n’y habitait pas. C’est là que m’avait jeté la tempête de mon cœur ; là, personne ne pouvait interrompre ce sanglant débat que j’avais engagé contre moi-même ,dont vous saviez l’issue, et moi, non. Mais cette fureur m’enfantait à la raison, cette mort à la vie ; sachant ce que j’étais de mal, j’ignorais ce qu’en un moment j’allais être de bien.\par
Je me retirai au jardin ; Alypius me suivait pas à pas. Car j’étais seul, même en sa présence. Et pouvait-il me quitter dans une telle crise ? Nous nous assîmes, le plus loin possible de la maison. Et mon esprit frémissait, et les vagues de mon indignation se soulevaient contre moi, de ce que je ne passais pas encore à votre volonté, à votre alliance, ô mon Dieu, où toutes les puissances de mon âme me poussaient en me criant : Courage I Et leurs louanges me soulevaient vers le Ciel : Et pour cela il ne fallait ni navire, ni char ; il ne fallait pas même faire ce pas qui nous séparait de la maison. Car non-seulement aller, mais arriver à vous, n’était autre chose que vouloir, mais d’une volonté forte et pleine, et non d’une volonté languissante et boiteuse, se dressant à demi et se débattant contre l’autre moitié d’elle-même qui retombe.\par
\pn{20}Et dans cette angoisse de mes indécisions, je faisais plusieurs de ces mouvements de corps que souvent des hommes veulent et ne peuvent faire, soit absence des membres, ou qu’ils soient emprisonnés dans des liens, paralysés de langueur, retenus par quelque entrave. Si je m’arrache les cheveux, si je me frappe le front, si j’embrasse mes genoux de mes doigts entrelacés, je le fais parce que je l’ai voulu. Et je pouvais le vouloir sans le faire, si la mobilité de mes membres ne m’eût obéi. Combien donc ai-je fait de choses, où vouloir et pouvoir n’était pas tout un. Et alors je ne faisais pas ce que je désirais d’un désir incomparablement plus puissant, et il ne s’agissait que de vouloir pour pouvoir, c’est- à-dire de vouloir pour vouloir. Car ici la puissance n’était autre que la volonté ; vouloir, c’était faire ; et pourtant rien   ne se faisait ; et mon corps obéissait plutôt à la volonté la plus imperceptible de l’âme qui d’un signe lui commandait un mouvement, que l’âme ne s’obéissait à elle-même pour accomplir dans la volonté seule sa plus forte volonté.
\section[{Chapitre IX, L’esprit commande au corps ; il est obéi : l’esprit se commande, et il se résiste}]{Chapitre IX, L’esprit commande au corps ; il est obéi : l’esprit se commande, et il se résiste}
\noindent \pn{21}D’où vient ce prodige ? quelle en est la cause ? Faites luire votre miséricorde ! que j’interroge ces mystères de vengeance, et qu’ils me répondent ! que je pénètre cette nuit de tribulation qui couvre les fils d’Adam ! D’où vient, pourquoi ce prodige ? L’esprit commande au corps ; il est obéi ; l’esprit se commande, et il se résiste. L’esprit commande à la main de se mouvoir, et l’agile docilité de l’organe nous laisse à peine distinguer le maître de l’esclave ; et l’esprit est esprit, la main est corps. L’esprit commande de vouloir à l’esprit, à lui-même, et il n’obéit pas. D’où vient ce prodige ? la cause ? Celui-là, dis-je, se commande de vouloir, qui ne commanderait s’il ne voulait ; et ce qu’il commande ne se fait pas !\par
Mais il ne veut qu’à demi ; donc, il ne commande qu’à demi. Car, tant il veut, tant il commande ; et tant il est désobéi, tant il ne veut pas. Si la volonté dit : Sois la volonté ! autrement : que je sois ! Elle n’est pas entière dans son commandement, et partant elle n’est pas obéie ; car si elle était entière, elle ne se commanderait pas d’être, elle serait déjà. Ce n’est donc pas un prodige que cette volonté partagée, qui est et n’est pas ; c’est la faiblesse de l’esprit malade, qui, soulevé par la main de la vérité, ne se relève qu’à demi, et retombe de tout le poids de l’habitude. Et il n’existe ainsi deux volontés que parce qu’il en est toujours une incomplète, et que ce qui manque à l’une s’ajoute à l’autre.
\section[{Chapitre X, Deux volontés ; un seul esprit.}]{Chapitre X, Deux volontés ; un seul esprit.}
\noindent \pn{22}Périssent de votre présence, mon Dieu, comme parleurs de vanités, comme séducteurs d’âmes ceux qui, apercevant deux volontés délibérantes, affirment deux esprits de deux natures, l’une bonne, l’autre mauvaise. Mauvais eux-mêmes, par ce sentiment mauvais, ils peuvent être bons, s’ils donnent un tel assentiment aux doctrines et aux hommes de vérité, que votre Apôtre puisse leur dire :\par

\begin{quoteblock}
\noindent « Vous avez été ténèbres autrefois, et vous êtes maintenant lumière dans le Seigneur (Ephés. V, 8). »\end{quoteblock}

\noindent Ceux-ci voulant être lumière en eux-mêmes, et non dans le Seigneur, par cette pensée téméraire que l’âme est une même nature que Dieu, sont devenus d’épaisses ténèbres, parce que leur sacrilége arrogance les a retirés de vous,\par

\begin{quoteblock}
\noindent « Lumière de tout homme venant au monde (Jean, I, 9.) »\end{quoteblock}

\noindent Songez donc à ce que vous dites et rougissez ;\par

\begin{quoteblock}
\noindent « approchez de lui, recevez sa lumière et votre visage ne rougira plus (Ps. XXXIII, 6). »\end{quoteblock}

\noindent Quand je délibérais pour entrer au service du Seigneur mon Dieu, ce que j’avais résolu depuis longtemps, qui voulait ? moi. Qui ne voulait pas ? moi. L’un et l’autre était moi, demi voulant, à demi ne voulant pas. Et je me querellais moi-même, et je me divisais contre moi. Et ce schisme, élevé malgré moi, n’attestait pas la présence d’un esprit étranger, mais le châtiment de mon âme. Et je n’en étais pas l’artisan, mais le péché qui habitait en moi. J’expiais la coupable liberté d’Adam, mon père (Rom. VIII, 14).\par
\pn{23}Car s’il est autant de natures contraires que de volontés ennemies, ce n’est plus deux natures, c’est plusieurs qu’il faut affirmer. Qu’un homme délibère d’aller à leur assemblée ou au théâtre, ces hérétiques s’écrient : Voilà les deux natures ; l’une bonne qui le conduit ici, l’autre mauvaise qui l’en éloigne. Autrement d’où peut venir cette contrariété de deux volontés en lutte ? Et moi je les dis mauvaises toutes deux, et celle qui conduit à eux, et celle qui attire au théâtre. Ils pensent, eux, que la première ne peut être que bonne. Mais si quelqu’un de nous, flottant à la merci de deux volontés engagées, délibère d’aller au théâtre ou à notre église, ne balanceront-ils pas à répondre ? Car ou ils avoueront, ce qu’ils refusent, que c’est la volonté bonne qui fait entrer dans notre église, comme elle y a introduit ceux que la communion des mystères y retient ; ou ils seront tenus d’admettre le conflit de deux mauvaises natures, de deux mauvais esprits en un seul homme, et ils démentiront leur assertion ordinaire d’un bon et d’un mauvais ; ou, rendus à la vérité, ils cesseront de nier que, lorsqu’on délibère, ce   soit une même âme livrée aux flux et reflux de ses volontés.\par
\pn{24}Qu’ils n’osent donc plus dire, en voyant dans un seul homme deux volontés aux prises, que ce sont deux esprits contraires, émanés de deux substances contraires, et deux principes contraires ; deux antagonistes, l’un bon, l’autre mauvais. Car vous, Dieu de vérité, vous les improuvez, vous les réfutez, vous les confondez. Et de même, dans deux volontés mauvaises, quand un homme délibère s’il ôtera la, vie à son semblable par le fer ou le poison ; s’il usurpera tel héritage ou tel autre, ne pouvant les usurper tous deux ; s’il écoutera la luxure qui achète la volupté, ou l’avarice qui sarde l’argent ; s’il ira au cirque ou au théâtre, ouverts le même jour ; ou bien, nouvelle indécision, s’il entrera dans cette maison faire un larcin auquel l’occasion le convie ; ou bien, autre incertitude, y commettre un adultère dont il trouve la facilité ; et si toutes ces circonstances concourent dans le même instant, si toutes ces volontés se pressent dans le même désir, ne pouvant s’accomplir à la fois, l’esprit n’est-il pas déchiré par cette querelle intestine de quatre volontés, plus encore, que sollicitent tant d’objets de convoitise ? Et pourtant ils ne calculent pas une telle quantité de substances différentes.\par
Et de même des volontés bonnes. Car je leur demande s’il est bon de se plaire à la lecture de l’Apôtre, au chant d’un saint cantique, s’il est bon d’expliquer l’Evangile ? À chaque demande, même réponse : oui. Mais si tous ces pieux exercices nous plaisent également, au même instant, le cœur de l’homme n’est-il pas distendu par cette diversité de volonté qui délibèrent sur l’objet à saisir de préférence ? Et ces volontés sont bonnes, et elles se combattent jusqu’à ce que soit déterminé le point où se porte une et entière cette volonté qui se divisait en plusieurs.\par
Ainsi, lorsque l’éternité nous élève à ses sublimes délices, et que le plaisir d’un bien temporel nous rattache ici-bas, c’est une même âme qui veut l’un ou l’autre, mais d’une demi-volonté ; et de là ces épines qui la déchirent quand la vérité détermine une préférence qui ne peut vaincre l’habitude.
\section[{Chapitre XI,Derniers combats.}]{Chapitre XI,Derniers combats.}
\noindent \pn{25}Ainsi je souffrais et je me torturais, m’accusant moi-même avec une amertume inconnue, me retournant et me roulant dans mes liens, jusqu’à ce j’eusse rompu tout entière cette chaîne qui ne me retenait plus que par un faible anneau, mais qui me retenait pourtant. Et vous me pressiez, Seigneur, au plus secret de mon âme, et votre sévère miséricorde me flagellait à coups redoublés et de crainte et de honte, pour prévenir une langueur nouvelle qui, retardant la rupture de ce faible et dernier chaînon, lui rendrait une nouvelle force d’étreinte.\par
Car je me disais au dedans de moi : Allons ! allons ! point de retard ! Et mon cœur suivait déjà ma parole ; et j’allais agir, et je n’agissais pas. Et je ne retombais pas dans l’abîme de ma vie passée, mais j’étais debout sur le bord, et je respirais. Et puis je faisais effort, et pour arriver, atteindre, tenir, de quoi s’en fallait-il ? Et je n’arrivais pas, et je n’atteignais pas, et je ne tenais rien ; hésitant à mourir à la mort, à vivre à la vie, je me laissais dominer plutôt par le mal, ce compagnon d’enfance, que par ce mieux étranger. Et plus l’insaisissable instant où mon être allait changer devenait proche, plus il me frappait d’épouvante ; ni ramené, ni détourné, pourtant, mon pas était suspendu.\par
\pn{26}Et ces bagatelles de bagatelles, ces vanités de vanités, mes anciennes maîtresses, me tiraient par ma robe de chair, et me disaient tout bas : Est-ce que tu nous renvoies ? Quoi ! dès ce moment, nous ne serons plus avec toi, pour jamais ? Et, dès ce moment, ceci, cela, ne te sera plus permis, et pour jamais ? Et tout ce qu’elles me suggéraient dans ce que j’appelle ceci, cela, ce qu’elles me suggéraient, ô mon Dieu ! que votre miséricorde l’efface de l’âme de votre serviteur ! Quelles souillures ! quelles infamies ! Et elles ne m’abordaient plus de front, querelleuses et hardies, mais par de timides chuchotements murmurés à mon épaule, par de furtives attaques ; elles sollicitaient un regard de mon dédain. Elles me retardaient toutefois dans mon hésitation à les repousser, à me débarrasser d’elles pour me rendre où j’étais appelé. Car la violence de l’habitude me disait : Pourras-tu vivre sans elles ?\par
\pn{27}Et déjà elle-même ne me parlait plus   que d’une voix languissante. Car, du côté où je tournais mon front, et où je redoutais de passer, se dévoilait la chaste et sereine majesté de la continence, m’invitant, non plus avec le sourire de la courtisane, mais par d’honnêtes caresses, à m’approcher d’elle sans crainte ; et elle étendait, pour me recevoir et m’embrasser, ses pieuses mains, toutes pleines de bons exemples ; enfants, jeunes filles, jeunesse nombreuse, tous les âges, veuves vénérables, femmes vieillies dans la virginité, et dans ces saintes âmes, la continence n’était pas stérile ; elle enfantait ces générations de joies célestes qu’elle doit, Seigneur, à votre conjugal amour !\par
Et elle semblait me dire, d’une douce et encourageante ironie : Quoi ! ne pourras-tu ce qui est possible à ces enfants, à ces femmes ? Est-ce donc en eux-mêmes, et non dans le Seigneur leur Dieu, que cela leur est possible ? C’est le Seigneur leur Dieu qui me donne à eux. Tu t’appuies sur toi-même, et tu chancelles ? Et cela t’étonne ? Jette-toi hardiment sur lui, n’aie pas peur ; il ne se dérobera pas pour te laisser tomber. Jette-toi hardiment, il te recevra, il te guérira ! Et je rougissais, parce que j’entendais encore le murmure des vanités : et je restais hésitant, suspendu. Et elle me parlait encore, et je croyais entendre : Sois sourd à la voix de ces membres de terre, afin de les mortifier. Les délices qu’ils te racontent sont-elles comparables aux suavités de la loi du Seigneur ton Dieu (Ps. CXVIII, 85) ? Cette lutte intestine n’était qu’un duel de moi avec moi. Et Alypius, attaché à mes côtés, attendait en silence l’issue de cette étrange révolution.
\section[{Chapitre XII, « Prends, lis ! prends, lis ! »}]{Chapitre XII, « Prends, lis ! prends, lis ! »}
\noindent \pn{28}Quand, du fond le plus intérieur, ma pensée eut retiré et amassé toute ma misère devant les yeux de mon cœur, il s’y éleva un affreux orage, chargé d’une pluie de larmes.\par
Et pour les répandre avec tous mes soupirs, je me levai, je m’éloignai d’Alypius. La solitude allait me donner la liberté de mes pleurs. Et je me retirai assez loin pour n’être pas importuné, même d’une si chère présence.\par
Tel était mon état, et il s’en aperçut, car je ne sais quelle parole m’était échappée où vibrait un son de voix gros de larmes. Et je m’étais levé. Il demeura à la place où nous nous étions assis, dans une profonde stupeur. Et moi j’allai m’étendre, je ne sais comment, sous un figuier, et je lâchai les rênes à mes larmes, et les sources de mes yeux ruisselèrent, comme le sang d’un sacrifice agréable. Et je vous parlai, non pas en ces termes, mais en ce sens :\par

\begin{quoteblock}
\noindent « Eh ! jusques à quand, Seigneur (Ps. VI, 4) ? jusques à quand, Seigneur, serez-vous irrité ? Ne gardez pas souvenir de mes iniquités passées (Ps. LXXXIII, 5, 8). »\end{quoteblock}

\noindent Car je sentais qu’elles me retenaient encore. Et je m’écriais en sanglots : Jusques à quand ? jusques à quand ? Demain ?… demain ?… Pourquoi pas à l’instant ; pourquoi pas sur l’heure en finir avec ma honte ?\par
\pn{29}Je disais et je pleurais dans toute l’amertume d’un cœur brisé. Et tout à coup j’entends sortir d’une maison voisine comme une voix d’enfant ou de jeune fille qui chantait et répétait souvent : « PRENDS, LIS ! PRENDS, LIS ! » Et aussitôt, changeant de visage, je cherchai sérieusement à me rappeler si c’était un refrain en usage dans quelque jeu d’enfant ; et rien de tel ne me revint à la mémoire. Je réprimai l’essor de mes larmes, et je me levai, et ne vis plus là qu’un ordre divin d’ouvrir le livre de l’Apôtre, et de lire le premier chapitre venu. Je savais qu’Antoine, survenant, un jour, à la lecture de l’Evangile, avait saisi, comme adressées à lui-même, ces paroles :\par

\begin{quoteblock}
\noindent « Va, vends -ce que tu as, donne-le aux pauvres, et tu auras un trésor dans le ciel ; viens, suis-moi (Matth. XIX, 21) ; »\end{quoteblock}

\noindent et qu’un tel oracle l’avait aussitôt converti à vous.\par
Je revins vite à la place où Alypius était assis ; car, en me levant, j’y avais laissé le livre de l’Apôtre. Je le pris, l’ouvris, et lus en silence le premier chapitre où se jetèrent mes yeux :\par

\begin{quoteblock}
\noindent « Ne vivez pas dans les festins, dans les débauches, ni dans les voluptés impudiques, ni en conteste, ni en jalousie ; mais revêtez-vous de Notre-Seigneur Jésus-Christ, et ne cherchez pas à flatter votre chair dans ses désirs. »\end{quoteblock}

\noindent Je ne voulus pas, je n’eus pas besoin d’en lire davantage. Ces ligues à peine achevées ; il se répandit dans mon cœur comme une lumière de sécurité qui dissipa les ténèbres de mon incertitude.\par
\pn{30}Alors, ayant laissé dans le livre la trace de mon doigt ou je ne sais quelle autre marque, je le fermai, et, d’un visage tranquille, je déclarai tout à Alypius. Et lui me révèle à son tour ce   qui à mon insu se passait en lui. Il demande à voir ce que j’avais lu ; je le lui montre, et lisant plus loin que moi, il recueille les paroles suivantes que je n’avais pas remarquées :\par

\begin{quoteblock}
\noindent « Assistez le faible dans la foi (Rom. XIV, 1). »\end{quoteblock}

\noindent Il prend cela pour lui, et me l’avoue. Fortifié par cet avertissement dans une résolution bonne et sainte, et en harmonie avec cette pureté de mœurs dont j’étais loin depuis longtemps, il se joint à moi sans hésitation et sans trouble.\par
À l’instant, nous allons trouver ma mère, nous lui contons ce qui arrive, elle se réjouit ; comment cela est arrivé, elle tressaille de joie, elle triomphe. Et elle vous bénissait,\par

\begin{quoteblock}
\noindent « ô vous qui êtes puissant à exaucer au delà de nos demandes, au delà de nos pensées Ephés. III, 20), »\end{quoteblock}

\noindent car vous lui aviez bien plus accordé en moi que ne vous avaient demandé ses plaintes et ses larmes touchantes. J’étais tellement converti à vous que je ne cherchais plus de femme, que j’abdiquais toute espérance dans le siècle, élevé désormais sur cette règle de foi, où votre révélation m’avait jadis montré debout à ma mère. Et son deuil était changé (Ps. XXIV, 12) en une joie bien plus abondante qu’elle n’avait espéré, bien plus douce et plus chaste que celle qu’elle attendait des enfants de ma chair. 
\chapterclose


\chapteropen
 \chapter[{IX. Le baptême et le deuil}]{IX. Le baptême et le deuil}\phantomsection
\label{IX}\renewcommand{\leftmark}{IX. Le baptême et le deuil}


\begin{argument}\noindent Il renonce à sa profession. — Sa retraite dans la villa de Verecundus. — Son baptême. — Mort de sa mère.
\end{argument}


\chaptercont
\section[{Chapitre premier, Actions de grâces !}]{Chapitre premier, Actions de grâces !}
\noindent \pn{1}\par

\begin{quoteblock}
\noindent « O Seigneur, je suis votre serviteur ; je suis votre serviteur, et le fils de votre servante. Vous avez brisé mes liens, je vous sacrifierai un sacrifice de louanges (Ps. CXV, 16, 17) ! »\end{quoteblock}

\noindent Que mon cœur, que ma langue vous louent, et que tous mes os s’écrient :\par

\begin{quoteblock}
\noindent « Seigneur, qui est semblable à vous ? Qu’ils parlent, et répondez-moi ; et dites à mon âme : Je suis ton salut (Ps. XXXIV, 10-3). »\end{quoteblock}

\noindent Qui étais-je ? et quel étais-je ? Combien de mal en mes actions ; et, sinon dans mes actions, dans mes paroles ; et, sinon dans mes paroles, dans ma volonté ? Mais vous, Seigneur de bonté et de miséricorde, vous avez mesuré d’un regard la profondeur de ma mort, et vous avez retiré du fond de mon cœur un abîme de corruption. Et il ne s’agissait pourtant que de ne pas vouloir ma volonté, et de vouloir la vôtre !\par
Mais où était donc, durant le cours de tant d’années, et de quels secrets et profonds replis s’est exhumé soudain mon libre arbitre, pour incliner ma tête sous votre aimable joug, et mes épaules sous votre léger fardeau (Matth. XI, 30), ô Christ, ô Jésus, mon soutien et mon rédempteur ? Quelles soudaines délices ne trouvai-je pas dans le renoncement aux délices des vanités ? En être quitté, avait été ma crainte, et les quitter, était ma joie. Car vous les chassiez de chez moi, ô véritable, ô souveraine douceur ! vous les chassiez, et, à leur place, vous entriez plus aimable que toute volupté, mais non au sang et à la chair ; plus éclatant que toute lumière, mais plus intérieur que tout secret ; plus élevé que toute grandeur, mais non pour ceux qui s’élèvent en eux-mêmes. Déjà mon esprit était libre du cuisant souci de parvenir aux honneurs, aux richesses, de rouler dans l’impureté, et d’irriter la lèpre de mes intempérances ; et je gazouillais déjà sous vos yeux, ô ma lumière, ô mon opulence, ô mon salut, Seigneur, mon Dieu !
\section[{Chapitre II, Il renonce à sa profession.}]{Chapitre II, Il renonce à sa profession.}
\noindent \pn{2}Et je résolus en votre présence de dérober doucement, et sans éclat, le ministère de ma parole au trafic du vain langage ; ne voulant plus désormais que des enfants, indifférents à votre foi, à votre paix, ne respirant que frénésie de mensonge et guerres de forum, vinssent prendre à ma bouche les armes qu’elle vendait à leur fureur.\par
Et il ne restait heureusement que fort peu de jours jusqu’aux vacances d’automne, et je résolus d’attendre en patience le moment du congé annuel pour ne plus revenir mettre en vente votre esclave racheté. Tel était mon dessein en votre présence, et en présence de mes seuls amis. Et il était convenu entre nous de n’en rien ébruiter, quoiqu’au sortir de la vallée de larmes (Ps. LXXXIII, 6-7), chantant le cantique des degrés, nous fussions par vous armés de flèches perçantes et de charbons dévorants contre la langue perfide (Ps. CXIX, 3-5) qui nous combat, à titre de conseillère, et nous aime comme l’aliment qu’elle engloutit.\par
\pn{3}Vous aviez blessé mon cœur des flèches de votre amour ; et je portais dans mes entrailles vos paroles qui les traversaient ; et les exemples de vos serviteurs, que de ténèbres vous avez laits lumière, et, de mort, vie, s’élevaient comme un ardent bûcher pour brûler et consumer en moi ce fardeau de langueur qui m’entraînait vers l’abîme ; et j’étais pénétré   d’une ardeur si vive, que tout vent de contradiction, soufflé par la langue rusée, irritait ma flamme loin de l’éteindre.\par
Mais la gloire de votre nom, que vous avez sanctifié par toute la terre, assurant des approbateurs à mon vœu et à ma résolution, c’eût été, suivant moi, vanité que de ne pas attendre la prochaine venue des vacances, et d’afficher ma retraite d’une profession exposée aux regards publics, au risque de faire dire que je n’avais devancé le retour si voisin des loisirs d’automne qu’afin de me signaler. Et à quoi bon livrer mes intentions aux téméraires conjectures, aux vains propos, et appeler le blasphème sur une inspiration sainte ?\par
\pn{4}Et, cet été même, l’extrême fatigue de l’enseignement public avait engagé ma poitrine ; je tirais péniblement ma respiration, et des douleurs internes témoignaient de la lésion du poumon ; une voix claire et soutenue m’était refusée. La crainte me troubla d’abord d’être forcé par nécessité de me dérober à ce pénible exercice, ou de l’interrompre jusqu’à guérison ou convalescence ; mais quand la pleine volonté de m’employer à vous seul, pour vous contempler, ô mon Dieu, se leva et prit racine en moi, vous le savez, Seigneur, je fus heureux même de cette sincère excuse, pour modérer le déplaisir des parents qui ne permettaient pas la liberté à l’instituteur de leur fils.\par
Plein de cette joie, j’attendais avec patience que ce reste de temps s’écoulât : une vingtaine de jours peut-être ; et il me fallait de la constance pour les attendre, parce que la passion s’était retirée, qui soulevait la moitié de ma charge ; et j’en serais demeuré accablé, si la patience n’eût pris la place de la passion. Quelqu’un de vos serviteurs, mes frères, me reprochera-t-il d’avoir pu, le cœur déjà brûlant de vous servir, m’asseoir encore une heure dans la chaire du mensonge ? Je ne veux pas me justifier. Mais vous, Seigneur, très-miséricordieux, ne m’avez-vous point pardonné ce péché, et ne me l’avez-vous point remis dans l’eau sainte, avec tant d’autres hideuses et mortelles souillures ?
\section[{Chapitre III, Sainte mort de ses amis Nebridius et Verecundus.}]{Chapitre III, Sainte mort de ses amis Nebridius et Verecundus.}
\noindent \pn{5}Notre bonheur devenait une sollicitude poignante pour Verecundus, qui, retenu dans le siècle par le lien le plus étroit, se voyait sur le point d’être sevré de notre commerce. Epoux, infidèle encore, d’une chrétienne, sa femme était la plus forte entrave qui le retardât à l’entrée des voies nouvelles ; et il ne voulait être chrétien que de la manière dont il ne pouvait l’être.\par
Mais avec quelle bienveillance il nous offrit sa campagne pour toute la durée de notre séjour ! Vous lui en rendrez la récompense, Seigneur, à la résurrection des justes ; car une partie de la dette lui est déjà payée. Ce fut en notre absence ; nous étions à Rome, quand, atteint d’une maladie grave, il se fit chrétien, et sortit de cette vie avec la foi. Et vous eûtes pitié, non de lui seul, mais de nous encore. C’eût été pour notre cœur une trop cruelle torture, de nous souvenir d’un tel ami .et de sa tendre affection pour nous, sans le compter entre les brebis de votre troupeau.\par
Grâces à vous, mon Dieu, nous sommes à vous. J’en prends à témoin et vos assistances et vos consolations ; ô fidèle prometteur, vous rendrez à Verecundus, en retour de l’hospitalité de Cassiacum, où nous nous reposâmes des tourmentes du siècle, la fraîcheur à jamais verdoyante de votre paradis, car vous lui avez remis ses péchés sur la terre, sur votre montagne, la montagne opime, la montagne féconde (Ps. LXVII, 16). Telles étaient alors ses anxiétés.\par
\pn{6}Pour Nebridius, il partageait notre joie, quoique n’étant pas encore chrétien, pris au piége d’une pernicieuse erreur qui lui faisait regarder comme un fantôme la vérité de la chair de votre Fils ; s’il s’en retirait néanmoins étranger aux sacrements de votre Église, il demeurait ardent investigateur de la vérité. Peu de temps après ma conversion et ma renaissance dans le baptême, devenu lui-même fidèle catholique, modèle de continence et de chasteté, il embrassa votre service, en Afrique, parmi les siens ; il avait rendu toute sa famille chrétienne, quand vous le délivrâtes de la prison charnelle ; et maintenant, il vit au sein d’Abraham !\par
 Quoi qu’on puisse entendre par ce sein (Voir ce que plus tard saint Augustin pensait du sein d’Abraham, dans le Traité de l’Âme et de son origine, ch. XVI, n. 24) , c’est là qu’il vit, mon Nebridius, mon doux ami ; de votre affranchi, devenu votre fils adoptif ; c’est là qu’il vit. Et quel autre lieu digne d’une telle âme ? II vit au séjour dont il me faisait tant de questions à moi, à moi homme de boue et de misère ! Il n’approche plus son oreille de ma bouche, mais sa bouche spirituelle de votre source, et il se désaltère à loisir dans votre sagesse ; éternellement heureux. Et pourtant je ne crois pas qu’il s’enivre là jusques à m’oublier, quand vous, ô Seigneur, vous qu’il boit, conservez mon souvenir.\par
Voilà où nous en étions ; consolant Verecundus attristé de notre conversion, sans nous en moins aimer, et l’exhortant au degré de perfection compatible avec son état, c’est-à-dire la vie conjugale. Nous attendions que Nebridius nous suivit, étant si près de nous, et il allait le faire, lorsqu’enfin ils s’écoulèrent, ces jours qui nous semblaient si nombreux et si longs dans notre impatience de ces libres loisirs, où nous pourrions chanter de tout notre amour :\par

\begin{quoteblock}
\noindent « Mon cœur vous appelle ; je cherche « votre visage ; Seigneur, je le chercherai toujours (Ps. XXVI, 8). »\end{quoteblock}

\section[{Chapitre IV, Son enthousiasme à la lecture. des psaumes.}]{Chapitre IV, Son enthousiasme à la lecture. des psaumes.}
\noindent \pn{7}Enfin le jour arriva où j’allais être de fait libre de ma profession, comme déjà je l’étais en esprit. Et je fus libre. Et le Seigneur affranchit ma langue comme il avait affranchi mon cœur. Et je vous bénissais avec joie en allant à cette villa avec tout ce qui m’était cher. Comment j’y employai des études déjà consacrées à votre service, mais qui, dans cette halte soudaine, soufflaient encore la superbe de l’école, c’est ce que témoignent les livres de mes conférences dans l’intimité (Voy. Rétract. Ch. I, II, III, IV), et de mes entretiens solitaires en votre présence, et les lettres que j’écrivais à Nebridius absent. Mais le temps suffirait-il à rappeler toutes les grâces dont vous nous avez alors comblés ? Et puis il me tarde de passer à des objets plus importante.\par
Ma mémoire me rappelle à vous, Seigneur, et il m’est doux de vous confesser par quels aiguillons intérieurs vous m’avez dompté, comment vous m’avez aplani en abaissant les montagnes et les collines de mes pensées, comment vous avez redressé mes voies obliques et adouci mes aspérités, et comment vous avez soumis Alypius, le frère de mon cœur, au nom de votre Fils unique, Notre-Seigneur et Sauveur Jésus-Christ, dont son dédain repoussait le nom de nos écrits. Il aimait mieux y respirer l’odeur des cèdres de la philosophie, déjà brisés en moi par le Seigneur, que l’humble végétation de l’Église, ces herbes salutaires, mortelles aux serpents.\par
\pn{8}Quels élans, mon Dieu, m’emportaient vers vous, en lisant les psaumes de David, cantiques fidèles, hymnes de piété qui bannissent l’esprit d’orgueil ; novice à l’amour pur, je partageais les loisirs de ma retraite avec Alypius, catéchumène comme moi, et avec ma mère, qui ne pouvait me quitter, femme ayant la foi d’un homme, et, avec le calme de l’âge, la charité d’une mère, la piété d’une chrétienne.\par
De quels élans m’emportaient vers vous ces psaumes, et de quelle flamme ils me consumaient pour vous ! Et je brûlais de les chanter à toute la terre, s’il était possible, pour anéantir l’orgueil du genre humain ! Et ne se chantent-ils pas par toute la terre ? et qui peut se dérober à votre chaleur (Ps. XVIII, 7) ?\par
Quelle violente et douloureuse indignation m’exaltait contre les Manichéens, et quelle commisération m’inspiraient leur ignorance de ces mystères, de ces divins remèdes, et le délire de leur fureur contre l’antidote qui leur eût rendu la raison ! J’eusse voulu qu’ils se fussent trouvés là, près de moi et m’écoutant à mon insu, observant et ma face et ma voix, quand je lisais le psaume quatrième, et ce que ce psaume faisait de moi :\par

\begin{quoteblock}
\noindent « Je vous ai invoqué, et vous m’avez entendu, Dieu de ma justice ; j’étais dans la tribulation, et vous m’avez dilaté ; ayez pitié de moi, Seigneur, exaucez ma prière.»\end{quoteblock}

\noindent Que n’étaient-ils là, m’écoutant, mais à mon insu, pour qu’ils n’eussent pas lieu de croire que ce fût à eux que s’adressaient tous les traits dont j’entrecoupais ces paroles ! Et puis j’eusse autrement parlé, me sentant écouté et vu ; et, quand j’eusse parlé de même, ils n’eussent pas accueilli ma parole comme elle partait en moi et pour moi, sous vos yeux, de la tendre familiarité du cœur.\par
\pn{9}Je frissonnais d’épouvante, et j’étais enflammé d’espérance, et je tressaillais vers votre   miséricorde, ô Père ! Et mon âme sortait par mes yeux et ma voix, quand, s’adressant à nous, votre Esprit d’amour nous dit :\par

\begin{quoteblock}
\noindent «Fils des hommes, jusques à quand ces cœurs appesantis ? Pourquoi aimez-vous la vanité, et cherchez-vous le mensonge ? »\end{quoteblock}

\noindent N’avais-je pas aimé la vanité ? n’avais-je pas cherché le mensonge ? Et cependant, Seigneur, vous aviez exalté déjà votre Saint, le ressuscitant des morts, et le plaçant à votre droite (Marc, XII, 19),d’où il devait faire descendre le Consolateur promis, l’Esprit de vérité (Jean, XIV, 16-17) ; et déjà il l’avait envoyé (Act. II, 1-4) ; mais je ne le savais pas.\par
Il l’avait envoyé, parce qu’il était déjà glorifié, ressuscité des morts et monté au ciel.\par

\begin{quoteblock}
\noindent « Car, avant la gloire de Jésus, l’Esprit n’était pas encore donné (Jean, VII, 39).»\end{quoteblock}

\noindent Et le Prophète s’écrie : Jusques à quand ces cœurs appesantis ? « Pourquoi aimez-vous la vanité, et cherchez-vous le mensonge ? Apprenez donc que le « Seigneur a exalté son Saint. » Il s’écrie : Jusques à quand ? Il s’écrie : Apprenez ! — Et moi, dans ma longue ignorance, j’ai aimé la vanité, j’ai cherché le mensonge ! C’est pourquoi j’écoutais en frémissant, je me souvenais d’avoir été un de ceux que ces paroles accusent. J’avais pris pour la vérité ces fantômes de vanité et de mensonge. Et quels accents, forts et profonds, retentissaient dans ma mémoire endolorie ! Oh ! que n’ont-ils été entendus de ceux qui aiment encore la vanité et cherchent le mensonge ! Peut-être en eussent-ils été troublés, peut-être eussent-ils vomi leur erreur ; et vous eussiez exaucé les cris de leur cœur élevés jusqu’à vous ; car c’est de la vraie mort de la chair qu’est mort Celui qui intercède pour nous.\par
\pn{10}Et puis je lisais : « Entrez en fureur, mais sans pécher. » Et combien étais-je touché de ces paroles, ô mon Dieu, moi qui avais appris à m’emporter contre mon passé pour dérober au péché mon avenir ? Et de quel juste emportement, puisque ce n’était point une autre nature, race de ténèbres, qui péchait en moi, comme le prétendent ceux qui\par

\begin{quoteblock}
\noindent « thésaurisent contre eux la colère, pour ce jour de colère où la justice sera révélée (Rom. II, 5). »\end{quoteblock}

\noindent Et mes biens n’étaient plus au dehors, et ce n’était plus dans ce soleil que je les cherchais de l’oeil charnel. Ceux qui cherchent leur joie au dehors se dissipent comme la fumée, et se répandent comme l’eau sur les objets visibles et temporels, et leur famélique pensée n’en lèche que les images.. Oh ! s’ils se fatiguaient de leur indigence, et disaient : « Qui nous « montrera le Bien ? » Oh ! s’ils entendaient notre réponse : « La lumière de votre visage, Seigneur, s’est imprimée dans nous. » Car nous ne sommes pas cette lumière qui éclaire tout homme (Jean , 1,9), mais nous sommes éclairés par vous, pour devenir, de ténèbres que nous étions, lumière en vous (Ephés. V, 8).\par
Oh ! s’ils voyaient cette lumière intérieure, éternelle, que je frémissais, moi, qui déjà la goûtais, de ne pouvoir leur montrer, s’ils m’eussent apporté leur cœur dans des yeux détournés de vous, en me disant : « Qui nous montrera le Bien ? » Car c’est là, c’est dans la chambre secrète où je m’étais emporté contre moi-même ; où, pénétré de componction, je vous avais offert l’holocauste de ma caducité, et jeté les prémices de mon renouvellement au sein de votre espérance ; c’est là que j‘avais commencé de savourer votre douceur, et que mon cœur avait reçu votre joie. Et je m’écriais à la vérité de cette lecture, sanctionnée par le sens intérieur. Et je ne voulais plus me diviser dans la multiplicité des biens terrestres, bourreau et victime du temps, lorsque la simple éternité me mettait en possession d’un autre froment, d’un autre vin, d’une autre huile.\par
\pn{11}Et le verset suivant arrachait à mon cœur un long cri : « Oh ! dans sa paix ! oh ! dans lui-même ! » ô bienheureuse parole ! « Je prendrai mon repos et mon sommeil ! » Et qui nous fera résistance quand l’autre parole s’accomplira :\par

\begin{quoteblock}
\noindent « La mort est engloutie dans la victoire (I Cor. XV, 54). »\end{quoteblock}

\noindent Et vous êtes cet Être fort qui ne change pas ; et en vous le repos oublieux de toutes les peines ; parce que nul autre n’est avec vous ; parce qu’il ne faut pas se mettre en quête de tout ce qui n’est pas vous. « Mais vous m’avez affermi, Seigneur, dans la simplicité de l’espérance. »\par
Je lisais, et brûlais, et ne savais quoi faire à ces morts sourds, parmi lesquels j’avais dardé ma langue empoisonnée, aboyeur aveugle et acharné contre ces lettres saintes, lettres distillant le miel céleste, radieuses de votre lumière ; et je me consumais d’indignation contre les ennemis de cette Écriture.\par
\pn{12}Quand épuiserai-je tous les souvenirs de ces heureuses vacances ? Mais je n’ai pas   oublié et ne tairai point l’aiguillon de votre fouet, et l’admirable célérité de votre miséricorde. Vous me torturiez alors par une cruelle souffrance de dents ; et le mal était arrivé à un tel excès, que, ne pouvant plus parler , il me vint à l’esprit d’inviter mes amis présents à vous prier pour moi, ô Dieu, maître de toute santé. J’écrivis mon désir sur des tablettes, et je les leur donnai à lire. À peine le sentiment de la prière eut-il fléchi nos genoux, que cette douleur disparut. Mais quelle douleur ! et comment s’évanouit-elle ? Je fus épouvanté, je l’avoue, Seigneur, mon Dieu ; non, de ma vie je n’avais rien éprouvé de semblable. Et l’impression de votre volonté entra au plus profond de moi-même ; et, dans ma foi exultante, je louai votre nom. Et cette foi ne me laissait pas en sécurité sur mes fautes passées, que le baptême ne m’avait pas encore remises.
\section[{Chapitre V, Il consulte Saint Ambroise.}]{Chapitre V, Il consulte Saint Ambroise.}
\noindent \pn{13}Les vacances étant écoulées, je fis savoir aux citoyens de Milan qu’ils eussent à chercher pour leurs enfants un autre vendeur de paroles, parce que j’avais résolu de me consacrer à votre service, une poitrine souffrante et une respiration gênée m’interdisant d’ailleurs l’exercice de ma profession. J’instruisis par lettres votre serviteur, le saint évêque Ambroise, de mes erreurs passées et de mon présent désir, lui demandant quel livre de vos Écritures je devais lire de préférence pour me mieux préparer à l’immense grâce que j’allais recevoir. Il m’ordonna le prophète Isaïe, sans doute comme le plus clair révélateur de l’Evangile et de la vocation des païens. Mais, dès les premières lignes, ne pouvant pénétrer le sens et pensant que le reste me serait également inintelligible, j’en remis la lecture au temps où je serais plus aguerri à la parole du Seigneur.
\section[{Chapitre VI, Il reçoit le baptême avec Alypius son ami, et Adéodatus son fils. — génie de cet enfant. — sa mort.}]{Chapitre VI, Il reçoit le baptême avec Alypius son ami, et Adéodatus son fils. — génie de cet enfant. — sa mort.}
\noindent \pn{14}Le temps étant venu de m’enrôler sous vos enseignes, nous revînmes de la campagne à Milan. Alypius voulut renaître en vous avec moi ; il avait déjà revêtu l’humilité nécessaire à la communion de vos sacrements ; intrépide dompteur de son corps, jusqu’à fouler pieds nus ce sol couvert de glaces ; prodige d’austérité. Nous nous associâmes l’enfant Adéodatus, ce fils charnel de mon péché, nature que vous aviez comblée. À peine âgé de quinze ans, il surpassait en génie des hommes avancés dans la vie et dans la science.\par
Ce sont vos dons que je publie, Seigneur mon Dieu, Créateur de toutes choses. et puissant Réformateur de nos difformités. Car il n’y avait en cet enfant de moi que le péché ; et s’il était élevé dans votre crainte, c’est vous qui me l’aviez inspiré, nul autre. Oui, ce sont vos dons que je publie. Il est un livre écrit par moi, intitulé Le Maître ; mon interlocuteur, c’est cet enfant ; et les réponses faites sous son nom sont, vous le savez, mon Dieu, ses pensées de seize ans. Il s’est révélé à moi par des signes plus admirables encore. Ce génie-là m’effrayait. Et quel autre que vous pourrait accomplir de tels chefs-d’œuvre ?\par
Vous avez bientôt, de cette terre, fait disparaître sa vie ; et je me souviens de lui avec sécurité ; son enfance, sa première jeunesse, rien de cet être ne me laissant à craindre pour lui. Nous nous l’associâmes comme un frère dans votre grâce, à élever sous vos yeux ; et nous reçûmes le baptême, et le remords inquiet de notre vie passée prit congé de nous. Et je ne me rassasiais pas en ces premiers jours de la contemplation si douce des profondeurs de votre conseil pour le salut du genre humain. À ces hymnes, à ces cantiques célestes, quel torrent de pleurs faisaient jaillir de mon âme violemment remuée les suaves accents de votre Église ! Ils coulaient dans mon oreille, et versaient votre vérité dans mon cœur ; ils soulevaient en moi les plus vifs élans d’amour ; et mes larmes roulaient, larmes délicieuses !
\section[{Chapitre VII, Découverte des corps de Saint Gervais et de Saint Protais.}]{Chapitre VII, Découverte des corps de Saint Gervais et de Saint Protais.}
\noindent \pn{15}L’Église de Milan venait d’adopter cette pratique consolante et sainte, ce concert mélodieux où les frères confondaient avec amour leurs voix et leurs cœurs. Il y avait à peu près un an ; Justine, mère du jeune empereur Valentinien, séduite par l’hérésie des Ariens, persécutait votre Ambroise. Le peuple fidèle passait les nuits dans l’église, prêt à mourir   avec son évêque, votre serviteur. Et ma mère, votre servante, voulant des premières sa part d’angoisses et de veilles, n’y vivait que d’oraisons. Nous-mêmes, encore froids à la chaleur de votre Esprit, nous étions frappés de ce trouble, de cette consternation de toute une ville. Alors, pour préserver le peuple des ennuis de sa tristesse, il fut décidé que l’on chanterait des hymnes et des psaumes, selon l’usage de l’Église d’Orient, depuis ce jour continué parmi nous, et imité dans presque toutes les parties de votre grand bercail.\par
\pn{16}C’est alors que dans une vision vous révélâtes à votre évêque le lieu qui recélait les corps des martyrs Gervais et Protais. Vous les aviez conservés tant d’années à l’abri de la corruption, dans le trésor de votre secret, sachant le moment de les produire, pour mettre un frein à la fureur d’une simple femme, mais d’une femme impératrice. Retrouvés et exhumés, on les transfère solennellement à la basilique ambroisienne, et les possédés sont délivrés des esprits immondes, de l’aveu même de ces démons, et un citoyen très-connu, aveugle depuis plusieurs années, demande et apprend la cause de l’enthousiasme du peuple il se lève, il prie son guide de le conduire à ces pieux restes. Arrivé là, il est admis à toucher avec un mouchoir le cercueil où reposaient ces morts saintes et précieuses à votre regard (Ps. CXV, 15). Il touche, porte le linge à ses yeux, ses yeux s’ouvrent. Le bruit en court sur l’heure ; tout s’anime du vif éclat de vos louanges. Et le cœur de la femme ennemie, sans être rendu à la santé de la foi, n’en fut pas moins réprimé dans ses fureurs de persécution.\par
Grâces à vous, mon Dieu ! où et d’où avez-vous rappelé mes souvenirs, pour que je révélasse, à votre gloire, ce grand événement que mon oubli avait passé sous silence. Et cependant, lorsque tout exhalait ainsi la fragrante odeur de vos parfums, nous ne courions pas après vous (Cantiq. I, 3) ! Et c’est ce qui faisait couler de mes yeux, à cette heure, une telle abondance de larmes en écoutant vos cantiques. J’avais soupiré si longtemps après vous, et enfin je respirais tout l’air qui peut entrer dans cette chaumine d’argile.
\section[{Chapitre VIII, Mort de Sainte Monique. — son éducation.}]{Chapitre VIII, Mort de Sainte Monique. — son éducation.}

\begin{quoteblock}
\noindent « Ô vous qui rassemblez sous le même toit les cœurs unanimes (Ps. LXVII, 7), »\end{quoteblock}

\noindent \pn{17}vous nous avez alors associé un homme jeune encore, de notre municipe, Evodius, officier de l’empereur, converti et baptisé avant nous, qui avait quitté la milice du siècle pour la vôtre. Réunis, décidés à vivre dans une communauté de résolutions saintes, nous cherchions le lieu propice au dessein de vous servir, et retournant ensemble en Afrique, nous étions à l’embouchure du Tibre, quand je perdis ma mère. J’abrège, j’ai hâte d’arriver. Recevez mes confessions, mon Dieu, et les actions de grâces que je vous rends, même en silence, de tant de faveurs sans nombre. Mais je ne tairai point tout ce que mon âme engendre de pensées sur votre servante, dont la chair m’a engendré au temps et le cœur à l’éternité. Ce n’est pas son opulence, mais vos libéralités répandues sur elle, que je veux publier. Car elle n’était pas elle-même l’auteur de sa vie, l’auteur de son éducation. C’est vous qui l’avez créée ; son père et sa mère ne savaient pas quelle œuvre se produisait par eux. Et qui l’éleva dans votre crainte ? La verge du Christ, la conduite de votre Fils unique dans une maison fidèle, membre sain de votre Église.\par
Et elle ne se louait pas tant du zèle de sa mère à l’instruire, que de la surveillance d’une vieille servante qui avait porté son père tout petit, ainsi que les jeunes filles ont coutume de porter à dos les petits enfants. Ce souvenir, sa vieillesse, la pureté de ses mœurs, lui assuraient, dans une maison chrétienne, la vénération de ses maîtres, qui lui avaient commis la conduite de leurs filles ; son zèle répondait à tant de confiance ; elle était, au besoin, d’une sainte rigueur pour les corriger, et toujours d’une admirable prudence pour les instruire. Hors les heures de leur modeste repas à la table de leurs parents, fussent-elles dévorées de soif, elle ne leur permettait pas même de boire de l’eau, prévenant une habitude funeste, et disant avec un grand sens :\par

\begin{quoteblock}
\noindent « Vous buvez de l’eau aujourd’hui, parce que le vin n’est pas en votre pouvoir ; mais, quand vous serez dans la maison de vos maris, maîtresses des celliers, vous dédaignerez l’eau, sans renoncer à l’habitude de boire. »\end{quoteblock}

 \noindent Par ce sage tempérament de préceptes et d’autorité, elle réprimait les avides désirs de la première jeunesse, et elle réglait la soif même de ces jeunes filles à cette mesure de bienséance qui exclut jusqu’au désir de ce qu’elle ne permet pas. 18. Et néanmoins, c’est l’aveu que votre servante faisait à son fils, le goût du vin s’était glissé chez elle. Quand ses parents l’envoyaient, suivant l’usage, comme une sobre enfant, puiser le vin à la cuve, après avoir baissé le vase pour le remplir, et avant de le verser dans un flacon, elle en goûtait un peu de l’extrémité des lèvres, tentation bientôt vaincue par la répugnance. Car cela ne venait pas d’un honteux penchant : c’était ce vif entrain du premier âge, ce bouillonnement d’espiéglerie que le poids de l’autorité apaise dans les jeunes cœurs.\par
Or, ajoutant, chaque jour, goutte à goutte,\par

\begin{quoteblock}
\noindent « parce que le mépris des petites choses amène insensiblement la chute » (Eccli. XIX, 1)\end{quoteblock}

\noindent elle était tombée dans l’habitude de boire, avec plaisir, à petite coupe presque pleine. Où était alors cette vieille gouvernante si sage ? où étaient ses austères défenses ? Eh ! quelle en eût été la force contre cette maladie cachée, si votre grâce salutaire, ô Seigneur, ne veillait sur nous ? En l’absence de son père, de sa mère, de tout ce qui prenait soin d’elle, vous, toujours présent, qui avez créé, qui appelez à vous, et, par la voie même des hommes de perversité, opérez le bien pour le salut des âmes ; que lites-vous alors, ô mon Dieu ? par quel traitement l’avez-vous guérie ? N’avez-vous pas tiré d’une autre âme un sarcasme froid et aigu, invisible acier dont votre main, céleste opérateur, trancha vif cette gangrène ? Une servante qui l’accompagnait d’ordinaire à la cuve, se disputant un jour, comme souvent il arrive, avec sa jeune maîtresse, seule à seule, lui lança ce reproche avec l’épithète effrontée et sanglante d’ivrognesse. Elle, percée de ce trait, voit sa laideur, la réprouve et s’en dépouille. Tant il est vrai que si les amis corrompent par la flatterie, les ennemis corrigent souvent par le reproche ; et votre justice ne leur rend pas, suivant leur action, mais suivant leur volonté. Car, dans sa colère, cette servante ne voulait que piquer sa maîtresse et non la guérir. Aussi le fit-elle en secret, soit que le temps et le lieu de la querelle en eût ainsi décidé, soit qu’elle craignît elle-même un châtiment pour une révélation si tardive. Mais vous, Seigneur, providence du ciel et de la terre, qui faites dériver à votre usage le lit profond chu torrent et réglez le cours turbulent des siècles, c’est par la démence d’une âme que vous avez guéri l’autre, afin que sur un tel exemple nul n’attribue à son ascendant personnel l’influence décisive d’une parole salutaire.
\section[{Chapitre IX, Vertus de Sainte Monique.}]{Chapitre IX, Vertus de Sainte Monique.}
\noindent \pn{19}Formée à la modestie et à la sagesse, plutôt soumise par vous à ses parents que par eux à vous, à peine nubile, elle fut remise à un homme qu’elle servit comme son maître ; jalouse de l’acquérir à votre épargne, elle n’employait, pour vous prouver à lui, d’autre langage que sa vertu. Et vous la rendiez belle de cette beauté qui lui gagna l’admiration et les respectueux amour de son mari. Elle souffrit ses infidélités avec tant de patience que jamais nuage ne s’éleva entre eux à ce sujet. Elle attendait que votre miséricorde lui donnât avec la foi la chasteté. Naturellement affectueux, elle le savait prompt et irascible, et n’opposait à ses emportements que calme et silence. Aussitôt qu’elle le voyait remis et apaisé, il le lui rendait à propos raison de sa conduite, s’il était arrivé qu’il eût cédé trop légèrement à sa vivacité.\par
Quand plusieurs des femmes de la ville, mariées à des hommes plus doux, portaient sur leur visage quelque trace des sévices domestiques, accusant, dans l’intimité de l’entretien, les mœurs de leurs maris, ma mère accusait leur langue, et leur donnait avec enjouement ce sérieux avis, qu’à dater de l’heure où lecture leur avait été faite de leur contrat de noces, elles avaient dû le regarder comme l’acte authentique de leur esclavage, et ce souvenir de leur condition devait comprimer en elles toute révolte contre leurs maîtres. Et comme ces femmes, connaissant l’humeur violente de Patricius, ne pouvaient témoigner assez d’étonnement qu’on n’eût jamais ouï dire qu’il eût frappé sa femme, ou que leur bonne intelligence eût souffert un seul jour d’interruption, elles lui en demandaient l’explication secrète ; et elle leur enseignait le plan de conduite dont je viens de parler. Celles qui en faisaient l’essai, avaient lieu de s’en   féliciter ; celles qui n’en tenaient compte, demeuraient dans le servage et l’oppression.\par
\pn{20}Sa belle-mère, au commencement, s’était laissé prévenir contre elle sur de perfides insinuations d’esclaves ; mais désarmée par une patience infatigable de douceur et de respects, elle dénonça d’elle-même à son fils ces langues envenimées qui troublaient la paix du foyer, et sollicita leur châtiment. Lui, se rendant à son désir et à l’intérêt de l’union et de l’ordre domestique, châtia les coupables au gré de sa mère. Et elle promit pareille récompense à qui, pour lui plaire, lui dirait du mal de sa belle-fille. Cette leçon ayant découragé la médisance, elles vécurent depuis dans le charme de la plus affectueuse bienveillance.\par
\pn{21}Votre fidèle servante, dont le sein, grâce à vous, m’a donné la vie, ô mon Dieu, ma miséricorde, avait encore reçu de vous un don bien précieux. Entre les dissentiments et les animosités, elle n’intervenait que pour pacifier. Confidente de ces propos pleins de fiel et d’aigreur, nausées d’invectives dont l’intempérance de la haine se soulage sur l’ennemie absente en présence d’une amie, elle ne rapportait de l’une à l’autre que les paroles qui pouvaient servir à les réconcilier. Cette vertu me paraîtrait bien insignifiante, si une triste expérience ne m’eût appris combien est infini le nombre de ceux qui, frappés de je ne sais quelle contagieuse épidémie de péchés, ne se contentent pas de rapporter à l’ennemi irrité les propos de l’ennemi irrité, mais en ajoutent encore qu’il n’a pas tenus ; quand, au contraire, l’esprit d’humanité ne doit compter pour rien de s’abstenir de ces malins rapports qui excitent et enveniment la haine, s’il ne se met en devoir de l’éteindre par de bonnes paroles, ainsi qu’elle en usait, docile écolière du Maître intérieur.\par
\pn{22}Enfin elle parvint à vous gagner son mari sur la fin de sa vie temporelle, et le croyant ne lui donna plus les mêmes sujets de chagrin que l’infidèle.\par
Elle était aussi la servante de vos serviteurs. Tous ceux d’entre eux de qui elle était connue, vous louaient, vous glorifiaient, vous chérissent en elle, parce qu’ils sentaient votre présence dans son cœur, attestée par les fruits de sa sainte vie. Elle n’avait eu qu’un mari ; elle avait acquitté envers ses parents sa dette de reconnaissance, et gouverné sa famille avec, piété ; ses bonnes œuvres lui, rendaient témoignage (I Tim. V, 4, 9, 10). Ses fils qu’elle avait nourris, elles les enfantait autant de fois qu’elle les voyait s’éloigner de vous. Enfin, quand nous tous, vos serviteurs, mon Dieu, puisque votre libéralité nous permet ce nom, vivions ensemble, avant son sommeil suprême, dans l’union de votre amour et la grâce de votre baptême, elle nous soignait comme si nous eussions été tous ses enfants, elle nous servait comme si chacun de nous eût été son père.
\section[{Chapitre X, Entretien de Sainte Monique avec son fils sur le bonheur de la vie éternelle.}]{Chapitre X, Entretien de Sainte Monique avec son fils sur le bonheur de la vie éternelle.}
\noindent \pn{23}À l’approche du jour où elle devait sortir de cette vie, jour que nous ignorions, et connu de vous, il arriva, je crois, par votre disposition secrète, que nous nous trouvions seuls, elle et moi, appuyés contre une fenêtre, d’où la vue s’étendait sur le jardin de la maison où nous étions descendus, au port d’Ostie. C’est là que, loin de la foule, après les fatigues d’une longue route, nous attendions le moment de la traversée.\par
Nous étions seuls, conversant avec une ineffable douceur, et dans l’oubli du passé, dévorant l’horizon de l’avenir (Philip. III, 13), nous cherchions entre nous, en présence de la Vérité que vous êtes, quelle sera pour les saints cette vie éternelle\par

\begin{quoteblock}
\noindent « que l’oeil n’a pas vue, que l’oreille n’a pas entendue, et où n’atteint pas le cœur de l’homme (I Cor. II, 9). »\end{quoteblock}

\noindent Et nous aspirions des lèvres de l’âme aux sublimes courants de votre fontaine, fontaine de vie qui réside en vous (Ps. XXXV, 10), afin que, pénétrée selon sa mesure de la rosée céleste, notre pensée pût planer dans les hauteurs.\par
\pn{24}Et nos discours arrivant à cette conclusion, que la plus vive joie des sens dans le plus vif éclat des splendeurs corporelles, loin de soutenir le parallèle avec la félicité d’une telle vie, ne méritait pas même un nom, portés par un nouvel élan d’amour vers Celui qui est, nous nous promenâmes par les échelons des corps jusqu’aux espaces célestes d’où les étoiles, la lune et le soleil nous envoient leur lumière ; et montant encore plus haut dans nos, pensées, dans nos paroles, dans l’admiration de vos œuvres, nous traversâmes nos âmes pour atteindre, bien au-delà, cette région d’inépuisable abondance, où vous rassasiez éternellement   Israël de la nourriture de vérité, et où la vie est la sagesse créatrice de ce qui est, de ce qui a été, de ce qui sera ; sagesse incréée, qui est ce qu’elle a été, ce qu’elle sera toujours ; ou plutôt en qui ne se trouvent ni avoir été, ni devoir être, mais l’être seul, parce qu’elle est éternelle ; car avoir été et devoir être exclut l’éternité.\par
Et en parlant ainsi, dans nos amoureux élans vers cette vie, nous y touchâmes un instant d’un bond de cœur, et nous soupirâmes en y laissant captives les prémices de l’esprit, et nous redescendîmes dans le bruit de la voix, dans la parole qui commence et finit. Et qu’y a-t-il là de semblable à votre Verbe, Notre-Seigneur, dont l’immuable permanence en soi renouvelle toutes choses (Sag. VII, 27) ?\par
\pn{25}Nous disions donc : qu’une âme soit ; en qui les révoltes de la chair, le spectacle de la terre, des eaux, de l’air et des cieux, fassent silence, qui se fasse silence à elle-même qu’oublieuse de soi, elle franchisse le seuil intérieur ; songes, visions fantastiques, toute langue, tout signe, tout ce qui passe, venant à se taire ; car tout cela dit à qui sait entendre :\par
Je ne suis pas mon ouvrage ; celui qui m’a fait est Celui qui demeure dans l’éternité (Ps. XCIX, 3,5) ; que cette dernière voix s’évanouisse dans le silence, après avoir élevé notre âme vers l’Auteur de toutes choses, et qu’il parle lui seul, non par ses créatures, mais par lui-même, et que son Verbe nous parle, non plus par la langue charnelle, ni par la voix de l’ange, ni par le bruit de la nuée, ni par l’énigme de la parabole ; mais qu’il nous parle lui seul que nous aimons en tout, qu’en l’absence de tout il nous parle ; que notre pensée, dont l’aile rapide atteint en ce moment même l’éternelle sagesse immuable au-dessus de tout, se soutienne dans cet essor, et que, toute vue d’un ordre inférieur cessante, elle seule ravisse, captive, absorbe le contemplateur dans ses secrètes joies ; qu’enfin la vie éternelle soit semblable à cette fugitive extase, qui nous fait soupirer encore ; n’est-ce pas la promesse de cette parole :\par

\begin{quoteblock}
\noindent « Entre dans la joie de ton Seigneur (Matth. XXV, 21) ? »\end{quoteblock}

\noindent Et quand cela ? Sera-ce alors que\par

\begin{quoteblock}
\noindent « nous ressusciterons tous, sans néanmoins être tous changés (I Cor. XV, 51) ?»\end{quoteblock}

\noindent \pn{26}Telles étaient les pensées, sinon les paroles, de notre entretien. Et vous savez, Seigneur, que ce jour même où nous parlions ainsi, où le monde avec tous ses charmes nous paraissait si bas, elle me dit :\par

\begin{quoteblock}
\noindent « Mon fils, en ce qui me regarde, rien ne m’attache plus à cette vie. Qu’y ferais-je ? pourquoi y suis-je encore ? J’ai consommé dans le siècle toute mon espérance. Il était une seule chose pour laquelle je désirais séjourner quelque peu dans cette vie, c’était « de te voir chrétien catholique avant de mourir. Mon Dieu me l’a donné avec surabondance, puisque je te vois mépriser toute félicité terrestre pour le servir. Que fais-je encore ici ? »\end{quoteblock}

\section[{Chapitre XI, Dernières paroles de Sainte Monique.}]{Chapitre XI, Dernières paroles de Sainte Monique.}
\noindent \pn{27}Ce que je répondis à ces paroles, je ne m’en souviens pas bien ; mais à cinq ou six jours de là, la fièvre la mit au lit. Un jour dans sa maladie, elle perdit connaissance et fut un moment enlevée à tout ce qui l’entourait. Nous accourûmes ; elle reprit bientôt ses sens, et nous regardant mon frère et moi, debout auprès d’elle ; elle nous dit comme nous interrogeant : « Où étais-je ? » Et à l’aspect de notre douleur muette : « Vous laisserez ici, votre mère ! » Je gardais le silence et je retenais mes pleurs. Mon frère dit quelques mots exprimant le vœu qu’elle achevât sa vie dans sa patrie plutôt que sur une terre étrangère. Elle l’entendit, et, le visage ému, le réprimant des yeux pour de telles pensées, puis me regardant : « Vois comme il parle, » me dit-elle ; et s’adressant à tous deux : « Laissez ce corps partout ; et que tel souci ne vous trouble pas. Ce que je vous demande seulement, c’est de vous souvenir de moi à l’autel du Seigneur, partout où vous serez. » Nous ayant témoigné sa censée comme elle pouvait l’exprimer, elle se tut, et le progrès de la maladie redoublait ses souffrances.\par
\pn{28}Alors, méditant sur vos dons, ô Dieu invisible, ces dons que vous semez dans le cœur de vos fidèles pour en récolter d’admirables moissons, je me réjouissais et vous rendais grâces au souvenir de cette vive préoccupation qui l’avait toujours inquiétée de sa sépulture, dont elle avait fixé et préparé la place auprès du corps de son mari ; parce qu’ayant vécu dans une étroite union, elle voulait encore, ô insuffisance de l’esprit humain pour les choses   divines ! ajouter à ce bonheur, et qu’il fût dit par les hommes qu’après un voyage d’outremer, une même terre couvrait la terre de leurs corps réunis dans la mort même.\par
Quand donc ce vide de son cœur avait-il commencé d’être comblé par la plénitude de votre grâce ? Je l’ignorais, et cette révélation qu’elle venait de faire ainsi me pénétrait d’admiration et de joie. Mais déjà, dans mon entretien à la fenêtre, ces paroles : « Que fais-je ici ? » témoignaient assez qu’elle ne tenait plus à mourir dans sa patrie. J’appris encore depuis, qu’à Ostie même, un jour, en mon absence, elle avait parlé avec une confiance toute maternelle à plusieurs de mes amis du mépris de cette vie et du bonheur de la mort. Admirant la vertu que vous aviez donnée à une femme, ils lui demandaient si elle ne redouterait pas de laisser son corps si loin de son pays : «Rien n’est loin de Dieu, répondit-elle ; et il n’est pas à craindre qu’à la fin des siècles,il ne reconnaisse pas la place où il doit me ressusciter. » Ce fut ainsi que, le neuvième jour de sa maladie, dans la cinquante-sixième année de sa vie, et la trente-troisième de mon âge, cette âme pieuse et sainte vit tomber les chaînes corporelles.
\section[{Chapitre XII, Douleur de Saint Augustin.}]{Chapitre XII, Douleur de Saint Augustin.}
\noindent \pn{29}Je lui fermais les yeux, et dans le fond de mon cœur affluait une douleur immense, prête à déborder en ruisseaux de larmes ; et mes yeux, sur l’impérieux commandement de l’âme, ravalaient leur courant jusqu’à demeurer secs, et cette lutte me déchirait. Aussitôt qu’elle eut rendu le dernier soupir, l’enfant Adéodatus jeta un grand cri ; nous le réprimâmes ; il se tut.\par
C’est ainsi que ce que j’avais en moi d’enfance, et qui voulait s’écouler en pleurs, était réprimé par la voix virile du cœur et se taisait. Car nous ne pensions pas qu’il fût juste de mener ce deuil avec les sanglots et les gémissements, qui accompagnent d’ordinaire les morts crues malheureuses ou sans réveil. Mais sa mort n’était ni malheureuse, ni entière. Nous en avions pour garants sa vertu, sa foi sincère et les raisons les plus certaines.\par
\pn{30}Qu’est-ce donc qui me faisait si cruellement souffrir au fond de moi, sinon la rupture soudaine de cette habitude, tant douce et chère, de vivre ensemble ; blessure vive à mon âme ? Je me félicitais toutefois du témoignage qu’elle m’avait rendu jusque dans sa dernière maladie, quand, souriante à mes soins, elle m’appelait bon fils, et redisait avec l’affection la plus tendre, qu’elle n’avait jamais entendu de ma bouche un trait dur ou injurieux lancé contre elle. Et pourtant, ô Dieu notre créateur, cette respectueuse déférence était-elle en rien comparable au service d’esclave qu’elle me rendait ? Aussi, c’était le délaissement de cette grande consolation qui navrait mon âme, et ma vie se déchirait qui n’était qu’une avec la sienne.\par
\pn{31}Quand on eut arrêté les pleurs de cet enfant, Evodius prit le psautier et se mit à chanter ce psaume auquel nous répondions tous :\par

\begin{quoteblock}
\noindent « Je chanterai, Seigneur, à votre gloire, vos miséricordes et vos jugements (Ps. C, 1). »\end{quoteblock}

\noindent Apprenant ce qui se passait, un grand nombre de nos frères et de femmes pieuses accoururent, et pendant que les funèbres devoirs s’accomplissaient suivant l’usage, je me retirai où la bienséance voulait, avec ceux qui ne jugeaient pas convenable de me laisser seul.\par
Je dis alors quelques paroles conformes à la circonstance ; je cherchais avec le baume de vérité à calmer mon martyre, connu de vous, et qu’ils ignoraient, attentifs à mes discours et me croyant insensible à la douleur. Mais moi, à votre oreille, où nul d’eux ne pouvait entendre, je gourmandais la mollesse de mes sentiments, et je fermais le passage au cours de mon affliction, et elle me cédait un peu, et elle revenait à l’instant avec une fureur nouvelle, sans toutefois forcer la barrière des larmes, le calme du visage ; seul, je savais tout ce que je refoulais dans mon cœur. Et comme je m’en voulais de laisser tant de prise sur moi aux accidents humains, cette fatalité de votre justice et de notre misère, ma douleur elle-même était une douleur ; j’étais livré à une double agonie.\par
\pn{32}Le corps porté à l’église, j’y vais, j’en reviens, sans une larme, pas même à ces prières que nous versâmes au moment où l’on vous offrît pour elle le sacrifice de notre rédemption, alors que le cadavre est déjà penché sur le bord de la fosse où on va le descendre : à ces prières mêmes, pas une larme ; mais, tout le jour, ma tristesse fut secrète et profonde, et l’esprit troublé, je vous demandais, comme je pouvais, de guérir ma peine, et vous ne m’écoutiez pas,   afin sans doute que cette seule épreuve achevât de graver dans ma mémoire quelle est la force des liens de la coutume sur l’âme même qui ne se nourrit plus de la parole de mensonge.\par
J’imaginai d’aller au bain, ayant appris qu’ainsi les Grecs l’avaient nommé, comme bannissant les inquiétudes de l’esprit. J’y vais, et je le confesse à votre miséricorde, ô Père des orphelins, j’en sors tel que j’y suis entré. Il n’avait point fait transpirer l’amertume de mon cœur.\par
Et puis je m’endormis, et à mon réveil, je sentis ma douleur bien diminuée ; et, seul au lit, je me rappelai ces vers de votre Ambroise, que je sentais si véritables\par

\begin{quoteblock}
\noindent « O Dieu créateur, modérateur des cieux, qui jetez sur le jour le splendide manteau de la lumière, répandez sur la nuit les grâces du sommeil ; afin que le repos rende au labeur ordinaire les membres épuisés, soulage les fatigues de l’esprit, et brise le joug inquiet de l’affliction ! »\end{quoteblock}

\noindent \pn{33}Et peu à peu je rentrais dans mes premières pensées sur votre servante, et me rappelant son pieux amour pour vous, et pour moi cette tendresse prévenante et sainte qui tout à coup me manquait, je goûtai la douceur de pleurer en votre présence sur elle et pour elle, sur moi et pour moi. Et je donnai congé à mes pleurs, jusqu’alors retenus, de couler à loisir ; et, soulevé sur ce lit de larmes, mon cœur trouva du repos, entendu de vous seul, et non pas d’un homme juge superbe de ma douleur.\par
Et maintenant, Seigneur, je vous le confesse en ces lignes. Lise et interprète à son gré qui voudra. Et celui-là, s’il m’accuse comme d’un péché, d’avoir donné à peine une heure de larmes à ma mère, morte pour un temps à mes yeux, ma mère qui m’avait pleuré tant d’années pour me faire vivre aux vôtres, qu’il se garde de rire, mais que plutôt, s’il est de grande charité, lui-même vous offre ses pleurs pour mes péchés, à vous, Père de tous les frères de votre Christ.
\section[{Chapitre XIII, Il prie pour sa mère.}]{Chapitre XIII, Il prie pour sa mère.}
\noindent \pn{34}Aujourd’hui, le cœur guéri de cette blessure que l’affection charnelle rendait peut être trop vive, je répands devant vous, mon Dieu, pour cette femme, votre servante, de bien autres pleurs ; pleurs de l’esprit frappé des périls de toute âme qui meurt en Adam. Il est vrai que, vivifié en Jésus-Christ (I Cor. XV, 22), elle a vécu dans les liens de la chair de manière à glorifier votre nom par sa foi et ses mœurs ; mais toutefois je n’oserais dire que, depuis que vous l’eûtes régénérée par le baptême, il ne soit sorti de sa bouche aucune parole contraire à vos préceptes. Et n’a-t-il pas été dit par la Vérité, votre Fils :\par

\begin{quoteblock}
\noindent « Celui, qui appelle son frère insensé est passible du feu (Matth. V, 22) ? »\end{quoteblock}

\noindent Et malheur à la vie même exemplaire, si vous la scrutez dans l’absence de la miséricorde. Mais comme vous ne recherchez pas nos fautes à la rigueur, nous avons le confiant espoir de trouver quelque place dans votre indulgence. Et d’autre part, quel homme, en comptant ses mérites véritables, fait autre choses que de compter vos dons ? Oh ! si les hommes se connaissaient, comme celui qui se glorifie se glorifierait dans le Seigneur (II Cor. X, 17) !\par
\pn{35}- Ainsi donc, ô ma gloire ! ô ma vie ! O Dieu de mon cœur ! mettant à part ses bonnes œuvres, dont je vous rends grâces avec joie, je vous prie à cette heure pour les péchés de ma mère ; exaucez-moi, au nom du Médecin suspendu au bois infâme, qui aujourd’hui, assis à votre droite, sans cesse intercède pour nous (Rom. VIII, 34). Je sais qu’elle a fait miséricorde, et de toute son âme remis la dette aux débiteurs. Remettez-lui donc la sienne (Matth. VI, 12) ; et s’il en est qu’elle ait contractée, tant d’années durant qu’elle a vécu après avoir reçu l’eau salutaire, remettez-lui, Seigneur, remettez-lui, je vous en conjure ; n’entrez pas avec elle en jugement (Ps. CXLII, 2). Que votre miséricorde s’élève au-dessus de votre justice (Jacq. II, 13) ! Vos paroles sont véritables, et vous avez promis aux miséricordieux miséricorde (Matth. 5,7) Et vous leur avez donné de l’être, vous qui avez pitié de qui il vous plaît d’avoir pitié, et faites grâce à qui il vous plaît de faire grâce (Exod. XXXIII, 19).\par
\pn{36}Et n’auriez-vous pas déjà fait ce que je vous demande ? je le crois ; mais encore, agréez, Seigneur, cette offrande de mon désir (Ps. CXVIII, 108). Car aux approches du jour de sa dissolution elle ne songea pas à faire somptueusement ensevelir, embaumer son corps ; elle ne souhaita point un monument choisi ; elle se soucia peu de reposer au pays de ses pères ; non, ce n’est pas là ce qu’elle nous recommanda ;  elle exprima ce seul vœu que l’on fit mémoire d’elle à votre autel : elle n’avait laissé passer aucun jour de sa vie sans assister à ses mystères. Elle savait bien que là se dispensait la sainte Victime par qui a été effacée la cédule qui nous était contraire j, et vaincu, l’ennemi qui, dans l’exacte vérification de nos fautes, cherche partout une erreur, et ne trouve rien à redire en l’Auteur de notre victoire. Qui lui rendra son sang innocent ? Qui lui rendra le prix dont il a payé notre délivrance ? C’est au sacrement de cette Rédemption que votre servante a attaché son âme (Coloss. II, 14) par le lien de la foi.\par
Que personne ne l’arrache à votre protection ; que, ni par force, ni par ruse, le lion-dragon ne se dresse entre elle et vous. Elle ne dira pas qu’elle ne doit rien, de peur d’être convaincue par la malice de l’accusateur, et de lui être adjugée ; mais elle répondra que sa dette lui est remise par Celui à qui personne ne peut rendre ce qu’il a acquitté pour nous sans devoir.\par
\pn{37}Qu’elle repose donc en paix avec l’homme qui fut son unique mari, qu’elle servit avec une patience dont elle vous destinait les fruits, voulant le gagner à vous. Inspirez aussi, Seigneur mon Dieu, inspirez à vos serviteurs, mes frères, à vos enfants, mes maîtres, que je veux servir de mon cœur, de ma voix et de ma plume ; tous tant qu’ils soient qui liront ces pages, inspirez-leur de se souvenir, à votre autel, de Monique, votre servante, et de Patricius, dans le temps son époux, dont la chair, grâce à vous, m’a introduit dans cette vie ; comment ? je l’ignore : qu’ils se souviennent, avec une affection pieuse, de ceux qui ont été mes parents à cette lumière défaillante ; mes frères en vous, notre Père, et en notre mère universelle ; mes futurs concitoyens dans l’éternelle Jérusalem, après laquelle le pèlerinage de votre peuple soupire depuis le départ jusqu’au retour ; et que sollicitées par ces Confessions, les prières de plusieurs lui obtiennent plus abondamment que mes seules prières, cette grâce qu’elle me demandait à son heure suprême. 
\chapterclose


\chapteropen
 \chapter[{X. Temps présent, mémoire et désir}]{X. Temps présent, mémoire et désir}\phantomsection
\label{X}\renewcommand{\leftmark}{X. Temps présent, mémoire et désir}


\begin{argument}\noindent Confession du cœur. — Ce qu’il sait avec certitude, c’est qu’il aime Dieu. — Il le cherche et le trouve dans sa mémoire. —Puissance incompréhensible dont il décrit les merveilles. — Il s’interroge sur la triple tentation de la volupté, de la curiosité et de l’orgueil. — Il remet à Notre-Seigneur Jésus-Christ, seul médiateur entre Dieu et les hommes, la guérison des maux de son âme.
\end{argument}


\chaptercont
\section[{Chapitre premier, Élévation.}]{Chapitre premier, Élévation.}
\noindent \pn{1}Que je vous connaisse, intime connaisseur de l’homme ! que je vous connaisse comme vous me connaissez (I Cor. XIII, 12) ! Force de mon âme, pénétrez-la, transformez-la, pour qu’elle soit vôtre et par vous possédée sans tache et sans ride (Ephés. V, 27) ! C’est là tout mon espoir, toute ma parole ! Ma joie est dans cet espoir lorsqu’elle n’est pas insensée. Quant au reste des choses de cette vie, moins elles valent de larmes, plus on leur en donne ; plus elles sont déplorables, moins on les pleure ! Mais, vous l’avez dit, vous aimez la vérité, Seigneur (Ps, L, 8) ; et celui qui l’accomplit vient à la lumière (Jean, III, 21) : qu’elle soit donc dans mon cœur qui se confesse à vous, qu’elle soit dans cet écrit qui me confesse à tous !
\section[{Chapitre II, Confession du cœur.}]{Chapitre II, Confession du cœur.}
\noindent \pn{2}Et quand même je vous fermerais mon cœur, que pourrais-je vous dérober ? Vos yeux, Seigneur, ne voient-ils pas à nu l’abîme de la conscience humaine ? C’est vous que je cacherais â moi-même, sans me cacher à vous. Et maintenant que mes gémissements témoignent que je me suis en dégoût, voilà qu’aimable et glorieux vous attirez mon cœur et mes désirs, afin que je rougisse de moi, que je me rejette et vous élise ; afin que je ne trouve grâce devant moi-même, comme devant vous, que grâce à vous.\par
Quel que je sois, vous me connaissez donc toujours, Seigneur ; et j’ai dit cependant quel fruit je recueillais de ma confession. Je vous la fais, non de la bouche et de la voix, mais en paroles de l’âme, en cris de la pensée qu’entend votre oreille. En effet, suis-je mauvais, c’est me confesser à vous que de me déplaire à moi-même ; suis-je pieux, c’est me confesser à vous que de ne pas m’attribuer les bons élans de mon âme. Car c’est vous, mon Dieu ! qui bénissez le juste (Ps. V, 13), mais vous l’avez d’abord justifié comme pécheur (Rom. IV, 5).\par
Ma confession en votre présence, Seigneur, est donc explicite et tacite : silence des lèvres, cris d’amour ! Que dis-je de bon aux hommes que vous n’ayez d’abord entendu au fond de moi-même, et que pouvez-vous entendre de tel en moi-même que vous ne m’ayez dit d’abord ?
\section[{Chapitre III, Pourquoi il confesse ce que la grâce à fait de lui.}]{Chapitre III, Pourquoi il confesse ce que la grâce à fait de lui.}
\noindent \pn{3}Pour entendre mes Confessions comme s’ils devaient, eux ! guérir toutes mes langueurs, qu’y a-t-il donc des hommes à moi ? Race curieuse de la vie d’autrui et paresseuse à redresser la sienne : Pourquoi s’informent-ils de ce que je suis, quand ils refusent d’apprendre de vous ce qu’ils sont ? Et d’où savent-ils, lorsque c’est moi qui leur parle de moi, que je dis vrai, puisque pas un homme ne sait ce qui se passe dans l’homme, si ce n’est l’esprit de l’homme qui est en lui (I Cor. II, 11) ? Mais qu’ils vous écoutent parler d’eux-mêmes, ils ne pourront dire : Le Seigneur a menti. Qu’est-ce en effet que vous écouter, sinon se connaître ? Et   qui nierait ce qu’il sait ainsi, ne mentirait-il pas à lui-même ?\par
Mais comme entre ceux qu’elle unit des liens de sa fraternité, la charité croit tout (I Cor. XIII, 7) ; je me confesse à vous, Seigneur, de sorte que les autres m’entendent. Je ne puis leur démontrer la vérité de ma confession, et toutefois ceux dont la charité ouvre les oreilles croient à ma parole.\par
\pn{4}Cependant, ô Médecin intérieur, montrez-moi bien l’utilité de ce que je vais faire. Car la confession de mes iniquités passées, que vous avez remises et couvertes (Ps. XXXI, 1) pour béatifier en vous cette âme transformée par la foi et par votre sacrement, peut ranimer les cœurs contre l’engourdissement et le : Je ne puis ! du désespoir ; les éveiller à l’amour de votre miséricorde, aux douceurs de votre grâce, cette force des faibles à qui elle a révélé leur faiblesse ! Et pour les justes, c’est une consolation d’entendre les péchés de ceux qui en sont affranchis, non pour ces péchés eux-mêmes, mais parce qu’ils ont été et ne sont plus.\par
Mais de quel fruit, Seigneur mon Dieu, à qui chaque jour se confesse ma conscience, plus assurée en l’espoir de votre miséricorde qu’en son innocence ; de quel fruit est-il donc, je vous le demande, que par ces lignes je confesse aux hommes devant vous, non ce que j’ai été, mais ce que je suis aujourd’hui ? Quant au passé, j’en ai reconnu et signalé l’avantage. Et maintenant beaucoup de ceux qui me connaissent ou ne me connaissent pas, qui m’ont entendu ou bien ont entendu parler de moi, désirent savoir ce qu’il en est au temps même de ces confessions ; ils n’ont pas l’oreille à mon cœur où je suis tel que je suis ; ils veulent donc m’entendre avouer ce que je puis être au fond de moi-même où l’oeil, ni l’oreille, ni l’intelligence ne peuvent pénétrer. Ils sont prêts à me croire sans plus de preuve ; la charité, qui les sanctifie, leur dit que je ne mens pas en leur parlant de moi, et c’est elle en eux qui me donne créance.
\section[{Chapitre IV, Quel fruit il espère de cette confession.}]{Chapitre IV, Quel fruit il espère de cette confession.}
\noindent \pn{5}Mais dans quel intérêt le désirent-ils ? Veulent-ils se réjouir avec moi en apprenant combien l’impulsion de votre grâce m’a rapproché de vous, et sachant combien je suis retardé par le poids de moi-même, prier pour moi ? À ceux-là je me révélerai. Car il n’est pas d’un faible intérêt, Seigneur mon Dieu, que grâces vous soient rendues par plusieurs à mon sujet, et que vous soyez par plusieurs sollicité pour moi. Que le cœur de mes frères aime en moi ce que vous leur enseignez d’aimable ; qu’il plaigne en moi ce que vous leur enseignez à plaindre. Mais ces sentiments, je ne les demande qu’au cœur de mes frères, et non pas à l’étranger,.\par

\begin{quoteblock}
\noindent « non pas au fils de l’étranger dont «la bouche parle le mensonge, dont la main « est une main d’iniquité (Ps. CXLIII, 8). »\end{quoteblock}

\noindent Je ne les demande qu’au cœur fraternel, qui, s’il m’approuve, se réjouit de moi, s’il m’improuve, s’attriste pour moi, et, dans la louange et le blâme, m’aime toujours.\par
C’est à ceux-là que je veux me dévoiler qu’ils respirent à la vue de mes biens, qu’ils soupirent à la vue de mes maux. Mes biens sont votre ouvrage et vos dons ; mes maux sont mes crimes et votre justice. Qu’ils respirent là, qu’ils soupirent ici ! Que les hymnes, que les larmes s’élèvent en votre présence de ces âmes fraternelles, vos vivants encensoirs (Apoc. VIII, 3) ! Et vous, Seigneur, touché des parfums de votre temple saint,\par

\begin{quoteblock}
\noindent «ayez pitié de moi, selon a grandeur de votre miséricorde (Ps. L. 1), »\end{quoteblock}

\noindent pour la gloire de votre nom ; poursuivez votre œuvre ; consommez mes imperfections.\par
\pn{6}Voilà le fruit de ma confession présente, c’est l’aveu même, non plus en présence de vous seul, dans le secret de la joie qui appréhende et de la tristesse qui espère (Philip. II, 12), mais publié à la face des enfants des hommes, associés à ma foi et à mon allégresse, hôtes comme moi de la mortalité, citoyens de ma cité, voyageurs comme moi, prédécesseurs, successeurs et compagnons de mon pèlerinage.\par
Ceux-là sont vos serviteurs, mes frères, que vous avez faits vos fils ; mes maîtres, que vous m’avez commandé de servir, si je veux vivre de vous avec vous. Et votre Verbe ne s’est pas contenté d’enseigner comme précepteur, il a pris les devants comme guide. Et je l’imite d’action et de parole, je l’imite sous vos ailes, à travers de grands périls. Mais sous ce voile protecteur mon âme vous est soumise, et mon infirmité vous est connue.\par
Je ne suis qu’un petit enfant, mais j’ai un Père qui vit toujours ; j’ai un tuteur puissant.   Et celui-là même m’a donné la vie, qui me prend sous sa tutelle ; et celui-là, c’est vous, ô mon tout-bien ! ô tout-puissant ! qui êtes avec moi dès avant que je sois avec vous ! Je révélerai donc à ceux que vous m’ordonnez de servir, ce que je suis aujourd’hui, ce peu que je suis encore.\par

\begin{quoteblock}
\noindent « Mais je ne me juge pas (I Cor. IV, 3). »\end{quoteblock}

\noindent Qu’on m’écoute dans l’esprit où je parle.
\section[{Chapitre V, L’Homme ne se connaît pas entièrement lui-même.}]{Chapitre V, L’Homme ne se connaît pas entièrement lui-même.}
\noindent \pn{7}C’est vous, Seigneur, qui êtes mon juge, parce que, bien que nul homme ne sache rien de l’homme que l’esprit de l’homme\par

\begin{quoteblock}
\noindent « qui est en lui (I Cor. II, 11), »\end{quoteblock}

\noindent cependant il est quelque chose de l’homme que ne sait pas même l’esprit de l’homme qui est en lui. Mais vous savez tout de lui, Seigneur, qui l’avez fait. Et moi, qui m’abaisse sous votre regard, qui ne vois en moi que terre et que cendre, je sais pourtant de vous une chose que j’ignore de moi. Et certes, ne vous voyant pas encore face à face, mais en énigme et par miroir (Ibid. XIII, 12), dans cet exil, errant loin de vous, plus présent à moi-même qu’à vous, je sais néanmoins que vous êtes inviolable, et j’ignore à quelles tentations je suis ou ne suis pas capable de résister.\par
Et j’ai l’espérance que, fidèle comme vous l’êtes, ne permettant pas que nous soyons tentés au delà de nos forces, vous nous donnez la puissance de sortir vainqueurs de la tentation, afin que vous puissiez persévérer (I Cor. X, 13). Je confesserai donc, de moi, ce que je sais, et aussi ce que j’ignore. Car ce que je connais de moi, je le connais à votre lumière, et ce que j’ignore de moi, je l’ignore jusqu’à ce que votre face change mes ténèbres en midi (Isaïe, LVIII, 10).
\section[{Chapitre VI, Ce qu’il sait avec certitude, c’est qu’il aime Dieu.}]{Chapitre VI, Ce qu’il sait avec certitude, c’est qu’il aime Dieu.}
\noindent \pn{8}Ce que je sais, de toute la certitude de la conscience, Seigneur, c’est que je vous aime. Vous avez percé mon cœur de votre parole, et à l’instant je vous aimai. Le ciel et la terre et tout ce qu’ils contiennent ne me disent-ils pas aussi de toutes parts qu’il faut que je vous aime ? Et ils ne cessent de le dire aux hommes,\par

\begin{quoteblock}
\noindent « afin qu’ils demeurent sans excuse (Rom. I, 20). »\end{quoteblock}

\noindent Mais le langage de votre miséricorde est plus intérieur en celui dont vous daignez avoir pitié, et à qui il vous plaît de faire grâce (Ibid, IX ; 15) ; autrement le ciel et la terre racontent vos louanges à des sourds.\par
Qu’aimé-je donc en vous aimant ? Ce n’est point la beauté selon l’étendue, ni la gloire selon le temps, ni l’éclat de cette lumière amie à nos yeux, ni les douces mélodies du chant, ni la suave odorance des fleurs et des parfums, ni la manne, ni le miel, ni les délices de la volupté.\par
Ce n’est pas là ce que j’aime en aimant mon Dieu, et pourtant j’aime une lumière, une mélodie, une odeur, un aliment, une volupté, en aimant mon Dieu ; cette lumière, cette mélodie, cette odeur, cet aliment, cette volupté, suivant l’homme intérieur ; lumière, harmonie, senteur, saveur, amour de l’âme, qui défient les limites de l’étendue, et les mesures du temps, et le souffle des vents, et la dent de la faim, et le dégoût de la jouissance, Voilà ce que j’aime en aimant mon Dieu.\par
\pn{9}Et qu’est-ce enfin ? J’ai interrogé la terre, et elle m’a dit : « Ce n’est pas moi. » Et tout ce qu’elle porte m’a fait même aveu. J’ai interrogé la mer et les abîmes, et les êtres animés qui glissent sous les eaux, et ils ont répondu : « Nous ne sommes pas ton Dieu ; cherche au-dessus de nous. » J’ai interrogé les vents, et l’air avec ses habitants m’a dit de toutes parts : « Anaximènes se trompe ; je ne suis pas Dieu. » J’interroge le ciel, le soleil, la lune, les étoiles, et ils me répondent : « Nous ne sommes pas non plus le Dieu que tu cherches. » Et je dis enfin à tous les objets qui se pressent aux portes de mes sens : « Parlez-moi de mon Dieu, puisque vous ne l’êtes pas ; dites-moi de lui quelque chose. » Et ils me crient d’une voix éclatante :\par

\begin{quoteblock}
\noindent « C’est lui qui nous a faits (Ps. XCIX, 3). »\end{quoteblock}

\noindent La voix seule de mon désir interrogeait les créatures, et leur seule beauté était leur réponse. Et je me retournai vers moi-même, et je me suis dit : Et toi, qu’es-tu ? Et j’ai répondu : « Homme ». Et deux êtres sont sous mon obéissance ; l’un extérieur, le corps ; l’autre en moi et caché, l’âme. Auquel devais-je plutôt demander mon Dieu, vainement cherché, à travers le voile de mon corps, depuis la terre jusqu’au ciel, aussi loin que je puisse lancer en émissaires les rayons de mes yeux ?\par
 Il valait mieux consulter l’être intérieur, car tous les envoyés des corps s’adressaient au tribunal de ce juge secret des réponses du ciel et de la terre et des créatures qui s’écriaient Nous ne sommes pas Dieu, mais son ouvrage. L’homme intérieur se sert de l’autre comme instrument de sa connaissance externe ; moi, cet homme intérieur, moi esprit, j’ai cette connaissance par le sens corporel. J’ai demandé mon Dieu à l’univers, et il m’a répondu : Je ne suis pas Dieu, je suis son œuvre.\par
\pn{10}Mais l’univers n’offre-t-il pas même apparence à quiconque jouit de l’intégrité de ses sens ? Pourquoi donc ne tient-il pas à tous même langage ? Animaux grands et petits le voient, sans pouvoir l’interroger, en l’absence d’une raison maîtresse qui préside aux rapports des sens. Les hommes ont ce pouvoir afin que les grandeurs invisibles de Dieu soient aperçues par l’intelligence de ses ouvrages (Rom. I, 20). Mais ils cèdent à l’amour des créatures ; et, devenus leurs esclaves, ils ne peuvent plus être leurs juges.\par
Et elles ne répondent qu’à ceux qui les interrogent comme juges ; et ce n’est point que leur langage, ou plutôt leur nature, varie, si l’un ne fait que voir, si l’autre, en voyant, interroge ; mais dans leur apparente constance, muettes pour celui-ci, elles parlent à celui-là, ou plutôt elles parlent à tous, mais elles ne sont entendues que des hommes qui confrontent ces dispositions sensibles avec le témoignage intérieur de la vérité. Car la Vérité me dit : Ton Dieu n’est ni le ciel, ni la terre, ni tout autre corps. Et leur nature même dit aux yeux : Toute grandeur corporelle est moindre en sa partie qu’en son tout. Et tu es supérieure à tout cela ; c’est à toi que je parle, ô mon âme, puisque tu donnes à ton corps cette vie végétative, que nul corps ne donne à un autre. Mais ton Dieu est la vie même de la vie.
\section[{Chapitre VII, Dieu ne peut être connu par les sens.}]{Chapitre VII, Dieu ne peut être connu par les sens.}
\noindent \pn{11}Qu’aimai-je donc, en aimant mon Dieu ? Quel est Celui qui domine de si haut les sommités de mon âme ? Mon âme elle-même me servira d’échelon pour monter à lui. Je franchirai cette force de vitalité qui me lie à mon corps et en remplit les organes de sa sève. Elle ne peut me faire trouver Dieu ; autrement elle le ferait trouver\par

\begin{quoteblock}
\noindent « au cheval, au mulet qui « n’ont pas la raison (Ps. XXXI, 9), »\end{quoteblock}

\noindent et dont les corps vivent du même principe.\par
Il est une autre puissance qui, non-seulement donne la vie, mais la sensibilité à cette chair que Dieu m’a faite ; défend à l’oeil d’entendre, à l’oreille de voir, ordonne à l’un de se tenir prêt pour que je voie, à l’autre pour que j’entende, et maintient tous les sens chacun à son poste et dans sa fonction, pour qu’ils prêtent la diversité de leur ministère à l’active unité du moi, de l’homme esprit. Mais je franchirai encore cette puissance qui m’est commune avec le cheval et le mulet, également doués de la sensibilité corporelle.
\section[{Chapitre VIII, De la mémoire.}]{Chapitre VIII, De la mémoire.}
\noindent \pn{12}Je franchirai donc ces puissances de mon être, pour monter par degrés jusqu’à Celui qui m’a fait. Et j’entre dans les domaines, dans les vastes palais de ma mémoire, où sont renfermés les trésors de ces innombrables images entrées par la porte des sens. Là, demeurent toutes nos pensées, qui augmentent, diminuent ou changent ces épargnes thésaurisées par nos sens ; et enfin tout dépôt, toute réserve, que le gouffre de l’oubli n’a pas encore enseveli.\par
Quand je suis là, je me fais représenter ce que je veux. Certains objets paraissent sur-le-champ, d’autres se font chercher davantage ; il faut les tirer comme d’un recoin obscur ; d’autres s’élancent en essaim, et tandis que l’on demande l’un d’eux, accourant tous à la fois, ils semblent dire : N’est-ce pas nous ? Et la main de mon esprit les éloigne de la face de mon souvenir, jusqu’à ce que l’objet désiré sorte de ses ténèbres et de sa retraite. D’autres enfin se suggérant sans peine au rang où je les appelle, les premiers cèdent la place aux suivants, pour rentrer à leur poste et reparaître à ma volonté. Ce qui arrive exactement lorsque je fais un récit de mémoire.\par
\pn{13}Là se conservent, distinctes et sans mélange, les espèces introduites chacune par une entrée particulière : la lumière, les couleurs, les figures corporelles, par les yeux ; tous les sons, par l’oreille ; toutes les odeurs, par le passage des narines ; toutes les saveurs, par la voie du palais ; et par le sens universel tout   objet dur ou mol, chaud ou froid, doux ou rude, grave ou léger, qui affecte le corps, soit au dehors, soit- au dedans. La mémoire les reçoit toutes à son vaste foyer, où, au besoin, je les compte et lès passe en revue. Ineffables replis, dédalé profond, où tout entre par le seuil qui l’attend et se range avec ordre ! Et ce n’est pas toutefois la réalité, mais l’image de la réalité sentie, qui entre pour revenir au rappel de la-pensée.\par
Qui pourrait dire comment se forment ces images ? et l’on sait toutefois par quel sens elles sont recueillies et mises en réserve. Car, alors que je demeure dans les ténèbres et le silence, ma mémoire me représente à volonté les couleurs, distingue le blanc du noir, et les sons ne font pas incursion sur les réminiscences de mes yeux, et, quoique présents, ils semblent se retirer et se tenir à part : je les demande, si je veux, et ils viennent aussitôt. Parfois encore, la langue immobile et le gosier silencieux, je chante comme il me plaît, sans que l’image des couleurs qui cohabite, me trouble ni m’interrompe quand je revois le trésor que l’oreille m’a versé. Ainsi, je visite au caprice du souvenir, ces magasins approvisionnés par les sens ; et je distingue, sans rien odorer, la senteur des lis de celle des violettes ; et je préfère le miel au vin chaud, le poli à l’aspérité, par réminiscence du palais et de la main. Et tout cela se passe en moi, dans l’immense galerie de ma mémoire.\par
\pn{14}J’y fais comparaître le ciel, la terre, la mer, avec toutes les impressions que j’en ai reçues, hors celles que j’ai oubliées. Là, je me rencontre moi-même, je me reprends au temps, au lieu, aux circonstances d’une action et au sentiment dont j’étais affecté dans cette action. Là résident les souvenirs de toutes les révélations de l’expérience personnelle ou du témoignage ; de cette trame du passé j’ourdis le tissu des expériences et les témoignages accueillis sur la foi de mon expérience, des événements et des espérances futures, et je forme de tout cela comme un présent que je médite ; et dans ces vastes plis de mon intelligence, peuplés de tant d’images, je me dis à moi-même : Je ferai ceci ou cela, et il s’ensuivra ceci ou cela. Oh ! si telle ou telle chose pouvait arriver ! Plaise à Dieu ! à Dieu ne plaise ! Et je me parle ainsi, et les images des objets qui m’intéressent sortent du pécule de ma mémoire ; car en leur absence il me serait impossible d’en parler.\par
\pn{15}Que cette puissance de la mémoire est grande ! Grande, ô mon Dieu ! sanctuaire impénétrable, infini ! Eh ! qui pourrait aller au fond ? Et c’est une puissance de mon esprit, une propriété de ma nature, et moi-même je ne comprends pas tout ce que je suis. L’esprit est donc trop étroit pour se contenir lui-même ? Et où donc déborde ce qu’il ne peut contenir de lui ? Serait-ce hors de lui ? ou plutôt, n’est-ce pas en lui ? Et d’où vient ce défaut de contenance ?\par
Ici je me sens confondu d’admiration et d’épouvante. Et les hommes vont admirer les cimes des monts, les vagues de la mer, le vaste cours des fleuves, le circuit de l’Océan, et le mouvement des astres ; et ils se laissent là, et ils n’admirent pas, chose admirable ! qu’au moment où je parle de tout cela, je n’en vois rien par les yeux ; incapable d’en parler pourtant, si tout cela, montagnes, vagues, fleuves, astres que j’ai vus, Océan, auquel je crois, n’offrait intérieurement à ma mémoire les mêmes immensités où s’élanceraient mes regards. Et toutefois lorsque ma vue s’est portée sur ces spectacles, elle ne les a pas engloutis ; et les réalités ne sont pas en moi, mais seulement les images, et je sais par quel sens chaque impression est entrée.
\section[{Chapitre IX, Mémoire des sciences.}]{Chapitre IX, Mémoire des sciences.}
\noindent \pn{16}Là, ne s’arrête pas l’immense capacité de ma mémoire. Elle porte en ses flancs tout ce que j’ai retenu de la science, et que l’oubli ne m’a pas encore dérobé. Et ces perceptions, je les garde à l’écart plus intérieurement, non pas en lieu, ni en images, mais en réalité, Car ce que je sais de la grammaire et de la dialectique, du nombre et de l’espèce des questions, n’est pas entré dans ma mémoire comme l’image, qui laisse la réalité à la porte, évanouie aussitôt qu’apparue ; comme la voix imprimant à l’ouïe une trace qui la fait vibrer encore lorsqu’elle a cessé de raisonner ; comme l’odeur qui, dans son passage, dissipée au vent, pénètre l’odorat et porte à mémoire d’une image qui se reproduit au désir de la réminiscence ; comme l’aliment qui n’a plus de saveur qu’au palais de la mémoire ; ou comme l’objet que la main a touché, dont l’éloignement n’efface pas l’empreinte : car les réalités de cet ordre ne sont pas présentées à la mémoire,   mais leurs seules images, qui, saisies avec une étonnante rapidité , sont rangées dans des cellules merveilleuses, d’où elles sont tirées merveilleusement par la main du souvenir.
\section[{Chapitre X, Les sciences n’entrent pas dans la mémoire par les sens.}]{Chapitre X, Les sciences n’entrent pas dans la mémoire par les sens.}
\noindent \pn{17}Quand j’entends dire qu’un objet comporte trois sortes de questions, savoir : s’il est, ce qu’il est, quel il est, je m’empare bien de l’image des sons dont ces paroles se forment, je sais qu’ils ont traversé l’air avec bruit, et qu’ils ne sont plus. Mais les réalités mêmes, exprimées par ces sons, je ne les ai perçues par aucun sens corporel ; je ne les ai nulle part que dans mon esprit, et c’est elles-mêmes, non leur image, qui habitent dans ma mémoire. Par où sont-elles entrées en moi ? qu’elles le déclarent, si elles peuvent. Je visite toutes les portes de ma chair, et je n’en trouve pas une qui leur ait donné passage.\par
Les yeux disent : Si elles sont colorées, nous les avons annoncées ; si elles sont sonores, disent les oreilles, nous les avons introduites ; si elles sont odorantes, disent les narines, c’est par nous qu’elles ont passé. Le goût, dit encore : S’il n’est pas question de saveur, ne me demande rien. Et le tact : S’il ne s’agit pas de corps, je n’ai point touché, et, partant, je n’ai rien dit. Par où et comment se sont-elles glissées dans ma mémoire ? je l’ignore : car, en les apercevant, ce n’est pas sur le témoignage d’une intelligence étrangère que je les ai crues, mais j’ai reconnu leur vérité dans mon esprit, je les lui ai remises comme un dépôt, pour me les rendre à mon désir. Elles étaient donc en moi avant que je ne les connusse, sans être dans ma mémoire ; mais où donc, et comment, quand on m’en a parlé, les ai-je reconnues, en disant : Il est ainsi, c’est vrai ; si elles n’étaient déjà dans ma mémoire, mais ensevelies au loin, et à de telles profondeurs, que peut-être, sans indication, ma pensée ne les eût jamais exhumées ?
\section[{Chapitre XI, Acquérir la science, c’est rassembler les notions dispersées dans l’esprit.}]{Chapitre XI, Acquérir la science, c’est rassembler les notions dispersées dans l’esprit.}
\noindent \pn{18}Ainsi, obtenir les notions qui ne se communiquent point à nos sens par image, mais dont nous percevons en nous la réalité même, par intuition directe, n’est après tout que rassembler dans l’esprit ce que la mémoire contient çà et là, en recommandant à la pensée de réunir ces fragments épars et négligés pour les placer sous la main de l’attention.\par
Et combien ma mémoire mortelle en son sein de notions de cet ordre, déjà toutes trouvées et comme rangées sous ma main ; ce qui s’appelle apprendre et connaître ? Que je cesse de les visiter de temps en temps, elles s’écoulent et gagnent le fond des plus lointains replis, où il faut que la pensée, les retrouve comme si elle les découvrait de nouveau, et les rassemble du même lieu (car elles ne changent pas de demeure), afin de les connaître, c’est-à-dire de les rallier dans leur dispersion ; d’où vient l’expression de COGITARE, fréquentatif de COGERE, rassembler, comme AGITO l’est d’AGO, et FACTITO de FACIO. Mais l’intelligence s’est approprié ce verbe, et l’emploie à la désignation exclusive de ces ralliements intérieurs dont elle forme sa pensée.
\section[{Chapitre XII, Mémoire des mathématiques.}]{Chapitre XII, Mémoire des mathématiques.}
\noindent \pn{19}La mémoire renferme aussi les propriétés et les lois innombrables du nombre et de la mesure ; et nulle d’elles ne lui a été transmise par impression sensible, car elles ne sont ni colorées, ni sonores, ni odorantes, ni savoureuses, ni tangibles. J’ai bien entendu le son des mots qui les désignent quand on en parle ; niais autre est le son, autre la réalité ; l’un est grec ou latin ; l’autre n’est ni grec ni latin ; elle ne connaît aucune langue.\par
J’ai vu tirer des lignes aussi déliées qu’un fil d’araignée ; mais il est un autre ordre de lignes, qui se présentent sans image, sans que l’oeil charnel les annonce. Elles sont évidentes à l’esprit qui les reconnaît, en l’absence de toute préoccupation corporelle. Les sons m’ont encore signalé les nombres nombrés ; mais il n’en est pas ainsi des nombres nombrants qui sont sans images, et partant d’une réalité absolue. Rie de moi qui ne me comprend pas ; rieur, tu me feras pitié. 
 \section[{Chapitre XIII, Mémoire des opérations de l’esprit.}]{Chapitre XIII, Mémoire des opérations de l’esprit.}
\noindent \pn{20}Et il me souvient de toutes ces notions ; et il nie souvient comment je les ai obtenues. Et il me souvient de tous les faux raisonnements élevés contre elles. Et le souvenir de ces erreurs est vrai ; et le discernement que j’ai fait du faux et du vrai sur ces points controversés est présent à mon souvenir.\par
Et je vois encore qu’il faut faire différence entre ce discernement actuel, et le souvenir de ce même discernement, souvent réitéré dans les opérations de ma pensée. Il me souvient donc d’avoir exercé souvent cet acte d’intelligence ; et ce discernement actuel, cette intellection d’aujourd’hui, je les serre dans ma mémoire pour me les rappeler à l’avenir tels qu’à cette heure je les conçois. J’ai donc souvenir de m’être souvenu, et c’est encore par la force de ma mémoire que je me souviendrai de mon présent ressouvenir.
\section[{Chapitre XIV, Memoire des affections de l’âme.}]{Chapitre XIV, Memoire des affections de l’âme.}
\noindent \pn{21}Et la mémoire conserve aussi les passions de mon esprit, non pas comme elles y sont lorsqu’il en est affecté ; elle les conserve dans les conditions de sa puissance. Car je me remémore mes joies, mes tristesses, mes craintes d’autrefois, mes désirs passés, libre en ce moment de tristesse et de joie, de désir et de crainte. Et parfois, au contraire, je me rappelle mes tristesses avec joie, et mes joies avec tristesse.\par
Qu’il en arrive ainsi à l’égard des affections sensibles, rien d’étonnant ; l’esprit est un être, et le corps un autre. Que je me souvienne avec joie d’une douleur que mon corps ne souffre plus, j’en suis donc peu surpris. Mais la mémoire n’est autre que l’esprit. En effet, si je recommande une chose au souvenir d’un homme, je lui dis : Mets-toi bien dans l’esprit. S’il m’arrive d’oublier, ne dirai-je pas : Je n’avais pas à l’esprit… ., il m’est passé de l’esprit… , donnant à la mémoire même le nom d’esprit ?\par
Cela étant, d’où vient donc qu’au moment où je me rappelle avec joie ma tristesse passée, la joie est dans mon esprit et la tristesse dans ma mémoire ; que l’esprit se réjouit de cette joie, sans que la mémoire s’attriste de cette tristesse ? Est-ce que la mémoire est indépendante de l’esprit ? Qui l’oserait dire ? En serait-elle comme l’estomac, et la joie et la tristesse comme des aliments doux et amers qui passent et séjournent dans ses cavités, mais dépourvus de saveur ? Il serait ridicule de presser davantage cette similitude, qui n’est pas toutefois sans vérité.\par
\pn{22}Or, quand je dis que l’âme est troublée par quatre passions, le désir, la joie, la crainte et la tristesse, c’est à la mémoire que j’emprunte tous mes raisonnements sur ce sujet, et toutes mes divisions et définitions selon le genre et la différence ; et ce souvenir des passions ne m’affecte d’aucun trouble passionné. Et il m’eût été impossible de les rappeler, si pourtant elles n’eussent été présentes au trésor où je puise.\par
Mais la mémoire ne serait-elle pas la rumination de l’esprit ? Pourquoi donc alors la réminiscence de la joie ou de la tristesse serait-elle sans amertume ou sans douceur au palais de la pensée ? Est-ce donc ce point de différence qui exclut toute similitude ? Qui se résignerait, en effet, à proférer ces mots de tristesse et de crainte, s’il fallait autant de fois qu’on en parle s’attrister ou craindre ? Et cependant il nous serait impossible d’en parler, si nous ne trouvions dans notre mémoire non seulement l’image que le son de ces mots y grave par les sens, mais encore les notions des réalités introduites sans frapper à aucune porte charnelle, et sur la foi de sentiments antérieurs confiés par l’esprit à la mémoire, qui souvent elle-même les retient sans mandat.
\section[{Chapitre XV, Comment les réalités absentes se représentent à la mémoire.}]{Chapitre XV, Comment les réalités absentes se représentent à la mémoire.}
\noindent \pn{23}Est-ce par image ou non ? qui pourrait le dire ? Je nomme une pierre, je nomme le soleil, en l’absence des objets, mais en présence de leur image. Je nomme la douleur du corps sans en éprouver aucune, et pourtant si son image ne la représente dans ma mémoire, je ne sais de quoi je parle ; je ne la distingue plus du plaisir. Je nomme la santé du corps, lorsque mon corps est sain, pénétré de la réalité même ; et toutefois, si son image n’était fixée dans ma mémoire, le son de ce mot n’éveillerait aucun sens à mon souvenir. Et ce nom de santé ne serait, pour les malades, qu’un emprunt à un vocabulaire inconnu, si   la puissance de leur mémoire ne retenait l’image de la réalité absente. Je nomme les nombres nombrants, et les voilà dans ma mémoire, eux-mêmes et non leur image. Je nomme l’image du soleil, et elle est dans ma mémoire ; et ce n’est pas l’image de l’image que je me représente, mais l’image elle-même toujours docile à mon rappel. Je nomme la mémoire et je reconnais ce que je nomme. Et où puis-je le reconnaître, sinon dans la mémoire ? Serait-ce donc par son image, et non par son essence, qu’elle serait présente à elle-même ?
\section[{Chapitre XVI, La mémoire se souvient de l’oubli.}]{Chapitre XVI, La mémoire se souvient de l’oubli.}
\noindent \pn{24}Mais quoi ! lorsque je nomme l’oubli, je reconnais ce que je nomme ; et comment le reconnaîtrais-je, si je ne m’en souvenais ? Et je ne parle pas du son de ce mot, je parle de l’objet dont il est le signe, qu’il me serait impossible de reconnaître si la signification du son m’était échappée. Ainsi, quand il me souvient de la mémoire, c’est par elle-même qu’elle se représente à elle-même ; quand il me souvient de l’oubli, oubliance et mémoire viennent aussitôt à moi ; mémoire, qui me fait souvenir ; oubliance, dont je me souviens.\par
Mais qu’est-ce que l’oubli, sinon une absence de mémoire ? Comment donc est-il présent, pour que je me souvienne de lui, lui dont la présence m’interdit le souvenir ? Or, s’il est vrai que, pour se rappeler, la mémoire doive retenir, et que faute de se rappeler l’oubli, il soit impossible de reconnaître la signification de ce mot, il suit que la mémoire retient l’oubli. La cause de l’oubli comparaît donc en nous pour le prévenir ? N’en faut-il pas inférer que ce n’est point par elle-même, mais par image, qu’elle revient à la mémoire ? Que, si elle était présente elle-même, elle ne nous ferait pas souvenir, mais oublier, Qui pourra pénétrer, qui pourra comprendre ces phénomènes ?\par
\pn{25}J’y succombe, Seigneur, et c’est sous moi que je succombe. Et me voilà pour moi-même un sol ingrat, qui rit de ma peine et boit mes sueurs. Et je ne sonde pas maintenant la profondeur des voûtes célestes, je ne mesure pas les distances des astres, je ne recherche pas la loi de l’équilibre terrestre ; non, c’est dans ma mémoire qui n’est que moi, c’est dans mon esprit qui n’est que moi, que je me perds. Que tout ce que je ne suis pas soit loin de moi, rien d’étonnant ; mais quoi de plus près de moi que moi-même ? Et voilà que je ne puis comprendre la puissance de ma mémoire, moi qui, sans elle, ne pourrais pas même me nommer !\par
Je me souviens donc de l’oubli j’en suis certain ; et comment l’expliquer ? Dirai-je que dans ma mémoire ne réside pas ce dont je me souviens ? Dirai-je que l’oubli n’y réside que pour m’empêcher d’oublier ? Egale absurdité. Dirai-je encore que ma mémoire ne conserve que l’image de l’oubli, et non l’oubli même ? Le puis-je, s’il est nécessaire que l’impression de l’image dans la mémoire soit devancée par la présence de l’objet même dont se détache l’image ? C’est ainsi que je me souviens de Carthage, et des lieux que j’ai parcourus, et des visages que j ‘ai vus, et de tous les rapports que m’ont transmis les sens : ainsi de la douleur, ainsi de la santé. Ces réalités étaient là quand ma mémoire s’empara de leur image, et me la réfléchit en leur présence, pour les reproduire, absentes, à mon souvenir.\par
Que si l’oubli demeure dans ma mémoire, non par lui-même, mais en image, il a donc fallu sa présence pour que son image lui fût dérobée ? Et s’il était présent, comment a-t-il pu graver son image, là où sa présence efface toute empreinte ? Et pourtant, si incompréhensible et inexplicable que soit ce mystère, je suis certain de me souvenir de l’oubli, ce meurtrier du souvenir.
\section[{Chapitre XVII, Dieu est au delà de la mémoire.}]{Chapitre XVII, Dieu est au delà de la mémoire.}
\noindent \pn{26}C’est quelque chose de grand que la puissance de la mémoire. Une sorte d’horreur me glace, ô mon Dieu, quand je pénètre dans cette multiplicité profonde, infinie ! Et cela, c’est mon esprit ; et cela, c’est moi-même. Que suis-je donc, ô mon Dieu ? quelle nature suis-je ? Variété vivante, puissante immensité !\par
Et voilà que je cours par les champs de ma mémoire ; et je visite ces antres, ces cavernes innombrables, peuplées à l’infini d’innombrables espèces, qui habitent par image, comme les corps ; par elles-mêmes, comme les sciences ; par je ne sais quelles notions, quels signes, comme les affections morales qui, n’opprimant plus l’esprit, restent néanmoins captives de la mémoire, quoique rien ne soit dans la mémoire qui ne soit dans l’esprit. Je vais, je cours, je vole çà et là, et pénètre partout, aussi avant   que possible, et de limites, nulle part ! Tant est vaste l’empire de ma mémoire ! tant est profonde la vie de l’homme vivant de la vie mortelle.\par
Que faire, ô ma vraie vie, ô mon Dieu ? Je franchirai aussi cette puissance de mon être, qui s’appelle mémoire, je la franchirai pour m’élancer vers vous, douce lumière. Que me répondez-vous ? Et voilà que, montant par mon esprit jusqu’à vous, qui demeurez au-dessus de moi, je laisse au-dessous cette puissance qui s’appelle mémoire, jaloux de vous atteindre où l’on peut vous atteindre ; de m’attacher à vous, où l’on peut s’attacher à vous. Car les brutes et les oiseaux ont la mémoire pour retrouver leurs tanières, leurs nids, leurs habitudes. Sans la mémoire ils n’auraient aucune faculté d’accoutumance.\par
Je passe donc par delà ma mémoire pour arriver à Celui qui m’a séparé des animaux, et m’a fait plus sage que les oiseaux du ciel. Je passe par delà ma mémoire. Mais où vous trouverai-je, bonté vraie, sécurité de délices ? où vous trouverai-je ? Si je vous trouve hors de ma mémoire , votre souvenir m’est donc échappé. Et, si je vous oublie, comment vous trouver ?
\section[{Chapitre XVIII, Il faut conserver la mémoire d’un objet perdu pour le retrouver.}]{Chapitre XVIII, Il faut conserver la mémoire d’un objet perdu pour le retrouver.}
\noindent \pn{27}La femme qui a perdu sa drachme et l’a cherchée avec sa lampe (Luc, XV, 8), s’en souvient pour la trouver ; autrement pourrait-elle, en la trouvant, la reconnaître ? Je me rappelle d’avoir cherché et retrouvé beaucoup d’objets perdus. Mais comment le sais-je ? Quand j’étais en quête de ma perte, on me disait : N’est-ce pas cela ? Et je répondais non, tant que l’objet ne m’était pas représenté ; et vainement, échappé à ma mémoire, m’eût-il été remis sous les yeux, je ne l’eusse pas retrouvé, faute de le reconnaître. Et il en est toujours ainsi toutes les fois qu’on cherche et recouvre ce qu’on avait perdu.\par
C’est que, s’il s’agit d’un objet visible, pour être soustrait au regard, il ne l’est pas à la mémoire qui le retient par son image, et, sur cette image intérieure, le reconnaît en le retrouvant ; car nous ne pouvons retrouver sans reconnaître, ni reconnaître sans nous souvenir : la mémoire garde l’objet, perdu pour les yeux.
\section[{Chapitre XIX, Comment la mémoire retrouve un objet oublié.}]{Chapitre XIX, Comment la mémoire retrouve un objet oublié.}
\noindent \pn{28}Mais quoi ! si la mémoire elle-même laisse échapper l’objet ; quand, par exemple, nous l’avons oublié et le cherchons pour nous en souvenir, où le cherchons-nous, sinon dans la mémoire ? Nous en présente-t-elle un autre, nous le repoussons, et ce n’est qu’en présence de l’objet même de notre recherche que nous disons : Le voici. Et, pour cela, il faut le reconnaître ; pour le reconnaître, il faut se souvenir, et pourtant nous l’avons oublié. Il n’est donc pas entièrement perdu ; c’est donc à l’aide de ce qui nous reste, que nous cherchons ce qui nous échappe. La mémoire se sent dépourvue de son lest ordinaire, et, comme disloquée par l’absence d’un membre, elle réclame ce qui lui manque.\par
Ainsi qu’à nos yeux ou à notre pensée s’offre un homme connu de nous, dont le nom nous fuit, tout nom qui ne se lie point à l’idée de la personne est rejeté, jusqu’à ce que se représente enfin celui qui s’adapte naturellement à cette image de connaissance. Mais d’où revient-il, sinon de la mémoire ? Car, le reconnaissons-nous sur l’avis d’un tiers, c’est encore elle qui le reproduit. Ce nom, en effet, n’est pas un étranger qui sollicite notre créance, mais un hôte de retour, dont nous constatons l’identité. Autrement, quel avis pourrait éveiller un souvenir entièrement effacé dans notre esprit ? Ce n’est donc pas tout à fait oublier une chose que de se souvenir de l’avoir oubliée ; et nous ne pourrions chercher un objet perdu, si aucun souvenir ne nous en était resté.
\section[{Chapitre XX, Chercher Dieu, c’est chercher la vie heureuse.}]{Chapitre XX, Chercher Dieu, c’est chercher la vie heureuse.}
\noindent \pn{29}Est-ce ainsi que je vous cherche, Seigneur ? Vous chercher, c’est chercher la vie bienheureuse. Oh ! que je vous cherche, pour que mon âme vive. Elle est la vie de mon corps, et vous êtes sa vie. Est-ce donc ainsi que je cherche la vie bienheureuse ? Car je ne l’ai pas trouvée, tant que je n’ai pas dit là où il faut le dire : C’est assez ! Est-ce ainsi que je la cherche ? Est-ce par souvenir, comme si je l’eusse oubliée,   avec conscience de mon oubli ? Est-ce par désir de l’inconnu ? soit que je n’en aie jamais rien su, soit que j’aie tout oublié jusqu’à la mémoire de mon oubli.\par
Mais n’est-ce pas cette vie heureuse après laquelle tous les hommes soupirent et que nul ne dédaigne ? Où l’ont-ils connue pour la désirer ainsi ? où l’ont-ils vue pour l’aimer ? Il faut donc qu’elle soit avec nous ; comment ? je l’ignore ; il faut qu’elle soit en nous ; mais à différentes mesures. L’heureux en espérance la possède, moins que l’heureux en réalité, plus que celui qui est déshérité et de la réalité et de l’espérance. Mais celui-là même la possède à certain degré, puisqu’il la désire, et d’un désir incontestable.\par
Quelle est donc cette notion dans l’homme ? je ne sais. Réside-t-elle dans sa mémoire ? c’est le problème qui m’intéresse ; car alors, il faut que nous ayons été autrefois heureux. Est-ce individuellement, est-ce dans ce premier homme, premier pécheur, en qui nous sommes tous morts, premier père de nos misères ?\par
C’est ce que je n’examine pas maintenant, je ne veux que savoir si la vie heureuse est dans la mémoire. Elle ne peut nous être entièrement inconnue, puisque nous l’aimons ; puisqu’à ce nom, il n’est personne qui ne confesse le désir de la réalité. Est-ce donc le son qui nous en plaît ? Qu’importe au Grec ce mot latin dont il ignore le sens ; mais le synonyme grec ne le laisse pas indifférent. Car elle ne connaît ni la Grèce, ni Rome, celle qu’envient et Grecs et Latins, et tout homme en toute langue ; elle est donc connue de tous les hommes. Trouvez un mot compris de tous pour leur demander s’ils veulent être heureux : oui, répondront-ils sans hésiter. Ce qui serait impossible, si ce nom n’exprimait une réalité conservée dans leur mémoire.
\section[{Chapitre XXI, Comment l’idée de la béatitude peut être dans la mémoire.}]{Chapitre XXI, Comment l’idée de la béatitude peut être dans la mémoire.}
\noindent \pn{30}Mais en est-il de ce souvenir comme de celui de Carthage que l’on a vue ? Non. La vie heureuse n’est pas un corps ; les yeux ne l’ont pas aperçue. S’en souvient-on comme des nombres ? Non : leur notion ne laisse pas d’autre désir. Mais la notion de la vie heureuse nous inspire l’amour et le désir de sa possession.\par
S’en souvient-on comme de l’éloquence ? Non. Quoique ce mot suggère à plusieurs qui ne sont pas éloquents, le souvenir et le désir de la chose même, preuve qu’elle existe dans leur esprit, c’est néanmoins par les sens qu’ils ont remarqué l’éloquence d’autrui, avec un plaisir qui leur en a donné le goût ; goût dérivé du plaisir ; plaisir, d’une notion intérieure niais nul de nos sens ne nous révèle en autrui la vie heureuse.\par
En est-il donc comme du souvenir de la joie ? Peut-être. Car si je me souviens de la joie dans la tristesse, je puis me souvenir de la vie heureuse dans ma misère. Et cette joie ne me fut jamais sensible, ni à la vue, ni à l’ouïe, ni à l’odorat, ni au goût, ni au toucher ; pur sentiment de l’esprit, dont l’impression, conservée dans ma mémoire, réveille en moi le dédain ou le désir, suivant la diversité des objets qui l’ont fait naître. Il fut un temps où je me réjouissais de la honte, et mon cœur ne se souvient de ces joies qu’avec horreur ; j’ai parfois goûté le plaisir du bien, et je m’en souviens avec un désir, qui, sevré de l’occasion, me rappelle avec tristesse ma joie passée.\par
\pn{31}Mais où, mais quand ai-je vécu ma vie heureuse pour m’en souvenir, pour l’aimer, pour la désirer ? Et il ne s’agit pas ici de mon désir ou du vœu de quelques hommes ; car en est-il un qui ne veuille être heureux ? Une notion moins sûre permettrait-elle une volonté si certaine ?\par
Demandez à deux hommes s’ils veulent porter les armes , peut-être l’un dira oui l’autre non ; demandez-leur s’ils veulent être heureux, tous deux répondront sans hésiter que tel est leur désir, et le même désir appelle l’un aux armes et en détourne l’autre. Ne serait-ce pas que, trouvant leur plaisir, l’un ici, l’autre là, tous deux s’accordent néanmoins dans leur volonté d’être heureux, comme ils s’accorderaient dans la réponse à la question s’ils veulent avoir sujet de joie ; et cette joie même, c’est ce qu’ils appellent bonheur, l’unique but qu’ils poursuivent par des voies différentes. Or, comme la joie est chose que tout homme, un jour, a ressentie, il faut que ce nom de bonheur en représente la connaissance à la mémoire.
 \section[{Chapitre XXII, Dieu, unique joie du cœur.}]{Chapitre XXII, Dieu, unique joie du cœur.}
\noindent \pn{32}Loin, mon Dieu, loin du cœur de votre serviteur humilié devant vous ; de trouver son bonheur en toutes joies ! Car il en est une refusée aux impies (Isaïe, XLVIII, 22), connue de vos serviteurs qui vous aiment ; cette joie, c’est vous. Et voilà la vie heureuse, se réjouir en vous, de vous et pour vous ; la voilà, il n’en est point d’autre. La placer ailleurs, c’est poursuivre une autre joie que la véritable. Et cependant, la volonté qui s’en éloigne s’attache encore à son image.
\section[{Chapitre XXIII, Amour naturel des hommes pour la vérité ils ne la haïssent que lorsqu’elle contrarie leurs passions.}]{Chapitre XXIII, Amour naturel des hommes pour la vérité ils ne la haïssent que lorsqu’elle contrarie leurs passions.}
\noindent \pn{33}Tous les hommes ne veulent donc pas être heureux, car il en est qui, refusant de se réjouir en vous, seule vie bienheureuse, refusent leur félicité. Serait-ce plutôt que, malgré leur désir, les révoltes de la chair contre l’esprit, et de l’esprit contre la chair, les réduisent à l’impuissance de leur vouloir (Galat. V, 17), les précipitent dans la faiblesse de leur force, dont ils se contentent, faute d’une volonté qui prête la force à leur faiblesse ?\par
Je leur demande à tous s’ils ne préfèrent pas la joie de la vérité à celle du mensonge. Et ils n’hésitent pas plus ici que pour la réponse à la question du bonheur. Car la vie heureuse c’est la joie de la vérité ; c’est la joie en vous, qui êtes la vérité (Jean, XIV, 6), ô Dieu ! ma lumière, mon salut (Ps. XXVI, 1), mon Dieu. Nous voulons tous cette vie bienheureuse, nous voulons tous cette vie, seule bienheureuse ; nous voulons tous la joie de la vérité.\par
J’en ai vu plusieurs qui voulaient tromper, nul qui voulût l’être. Où donc les hommes ont-ils pris cette connaissance du bonheur, si ce n’est où ils ont pris celle de la vérité ? car ils aiment la vérité, puisqu’ils ne veulent pas être trompés. Et ils ne peuvent aimer la vie heureuse, qui n’est que la joie de la vérité, sans aimer la vérité. Et ils ne sauraient l’aimer, si la mémoire n’en avait aucune idée.\par
Pourquoi donc n’y cherchent-ils pas leur joie, pour y trouver leur félicité ? C’est qu’ils sont fortement préoccupés de ces vanités qui leur créent plus de misères que ce faible souvenir ne leur laisse de bonheur. Il est encore une faible lumière dans l’âme de l’homme. Qu’il marche, qu’il marche, tant qu’elle luit, de peur d’être surpris par les ténèbres (Jean, XII, 31).\par
\pn{34}Mais d’où vient que la vérité engendre la haine ? D’où vient que l’on voit, un ennemi dans l’homme qui l’annonce en votre nom, si l’on aime la vie heureuse qui n’est que la joie de la vérité ? C’est qu’elle est tant aimée, que ceux même qui ont un autre amour veulent que l’objet de cet amour soit la vérité ; et refusant d’être trompés, ils ne veulent pas être convaincus d’erreur. Et de l’amour de ce qu’ils prennent pour la vérité vient leur haine de la vérité même. Ils aiment sa lumière et haïssent son regard. Voulant tromper sans l’être, ils l’aiment quand elle se manifeste, et la haïssent quand elle les découvre ; mais par une juste rémunération, les dévoilant malgré eux, elle leur reste voilée.\par
C’est ainsi, oui c’est ainsi que l’esprit, humain, dans cet état de cécité, de langueur, de honte et d’infirmité, prétend se cacher et que tout lui soit découvert ; et il arrive, au contraire, qu’il n’échappe pas à la vérité qui lui échappe. Et néanmoins dans cet état de misère, il préfère ses joies à celles du mensonge. Il sera donc heureux lorsque, sans crainte d’aucun trouble, il jouira de la seule Vérité, mère de toutes les autres.
\section[{Chapitre XXIV, Dieu se trouve dans la mémoire.}]{Chapitre XXIV, Dieu se trouve dans la mémoire.}
\noindent \pn{35}Ai-je assez dévoré les espaces de ma mémoire à vous chercher, mon Dieu ? et je ne vous ai pas trouvé hors d’elle ! Non, je n’ai rien trouvé de vous que je ne me sois rappelé, depuis le jour où vous m’avez été enseigné. Depuis ce jour, je ne vous ai pas oublié.\par
Où j’ai trouvé la vérité, là j’ai trouvé mon Dieu, la vérité même, alors connue, dès lors présente à ma mémoire. Et, depuis que je vous sais, vous n’en êtes pas sorti, et je vous y trouve toutes les fois que votre souvenir me convie à vos délices. Voilà mes voluptés saintes, don de votre miséricorde, qui a jeté un regard sur ma pauvreté.
 \section[{Chapitre XXV, Dans quelle partie de la mémoire trouvons-nous Dieu ?}]{Chapitre XXV, Dans quelle partie de la mémoire trouvons-nous Dieu ?}
\noindent \pn{36}Mais où demeurez-vous dans ma mémoire, vous, Seigneur ? où y demeurez-vous ? Quelle chambre vous y êtes-vous faite ? Quel sanctuaire vous êtes-vous bâti ? Vous lui avez fait cet honneur d’habiter en elle, je le sais ; mais c’est votre logement que j’y cherche. Lorsque mon cœur s’est rappelé mon Dieu, j’ai traversé toutes ces régions de souvenir qui me sont communes avec les bêtes ; ne vous trouvant pas entre les images des objets sensibles, je vous ai demandé à la résidence où je mets en dépôt les affections de mon esprit ; mais vainement : j’ai pénétré au siége même de l’esprit, hôte de ma mémoire, car l’esprit se souvient aussi de soi-même ; et vous n’y étiez pas, parce que vous n’êtes ni une image sensible, ni une affection du principe vivant en nous, comme la joie, la tristesse, le désir, la crainte, le souvenir, l’oubli, ni l’esprit lui-même, mais le Seigneur, Dieu de l’esprit.\par
Instabilité que tout cela, et pourtant vous, éternel et immuable, vous avez daigné demeurer dans ma mémoire depuis que je vous ai connu. Et je demande encore où vous habitez en elle, comme si elle était lieu ? Mais certes vous habitez en elle, puisque je me souviens de vous depuis l’heure où je vous ai connu, et c’est en elle que je vous retrouve, lorsque votre souvenir se représente à mon cœur.
\section[{Chapitre XXVI, Dieu est la vérité que les hommes consultent.}]{Chapitre XXVI, Dieu est la vérité que les hommes consultent.}
\noindent \pn{37}Mais où donc vous ai-je trouvé pour vous apprendre ? Vous n’étiez pas dans ma mémoire avant de m’être connu. Où donc vous ai-je trouvé, sinon en vous, au-dessus de moi ? Entre vous et nous le lieu n’existe pas, et nous nous approchons, nous nous éloignons de vous sans distance. Vérité, oracle universel, vous siégez partout pour répondre à ceux qui vous consultent ; vos réponses fournissent en tous lieux à tant de consulteurs divers ! Vous parlez clairement, mais tous n’entendent pas de même. Tous conforment leurs demandes à leurs volontés, mais vous n’y conformez pas toujours vos réponses. Celui-là seul est votre zélé serviteur, qui a moins en vue d’entendre de vous ce qu’il veut, que de vouloir ce qu’il a entendu de vous.
\section[{Chapitre XXVII, Ravissement de cœur devant Dieu.}]{Chapitre XXVII, Ravissement de cœur devant Dieu.}
\noindent \pn{38}Je vous ai aimée tard, beauté si ancienne, beauté si nouvelle, je vous ai aimée tard. Mais quoi ! vous étiez au dedans, moi au dehors de moi-même ; et c’est au dehors que je vous cherchais ; et je poursuivais de ma laideur la beauté de vos créatures. Vous étiez avec moi, et je n’étais pas avec vous ; retenu loin de vous par tout ce qui, sans vous, ne serait que néant. Vous m’appelez, et voilà que votre cri force la surdité de mon oreille ; votre splendeur rayonne, elle chasse mon aveuglement ; votre parfum, je le respire, et voilà que je soupire pour vous ; je vous ai goûté, et me voilà dévoré de faim et de soif ; vous m’avez touché, et je brûle du désir de votre paix.
\section[{Chapitre XXVIII, Misère de cette vie.}]{Chapitre XXVIII, Misère de cette vie.}
\noindent \pn{39}Quand je vous serai uni de tout moi-même, plus de douleur alors, plus de travail ; ma vie sera toute vivante, étant toute pleine de vous. L’âme que vous remplissez devient légère ; trop vide encore de vous, je pèse sur moi. Mes joies déplorables combattent mes tristesses salutaires, et de quel côté demeure la victoire ? je l’ignore. Hélas ! Seigneur, ayez pitié de moi. Mes tristesses coupables sont aux prises avec mes saintes joies ; et de quel côté demeure la victoire ? je l’ignore encore. Hélas ! Seigneur, ayez pitié de moi ! pitié, Seigneur ! vous voyez ; je ne vous dérobe point mes plaies. O médecin, je suis malade ! ô miséricorde, vous voyez- ma misère ! Ah ! n’est-ce pas une tentation continuelle que la vie de l’homme sur la terre (Job, VII, 1) ? Qui veut les afflictions et les épreuves ? Vous ordonnez de les souffrir, et non de les aimer. On n’aime point ce que l’on souffre, quoiqu’on en aime la souffrance. On se réjouit de souffrir, mais on choisirait de n’avoir pas tel sujet de joie. Dans le malheur, je désire la prospérité ; heureux, je crains le malheur. Entre ces deux   écueils, est-il pour la vie humaine un abri contre la tentation ? Malheur, oui, malheur aux prospérités du siècle livrées à la crainte de l’adversité et aux séductions de la joie ! Malheur, trois fois malheur aux adversités du siècle, livrées au désir de la prospérité ! dures à souffrir, écueil où la patience fait naufrage ! N’est-ce pas une tentation continuelle que la vie de l’homme sur la terre ?
\section[{Chapitre XXIX, La grâce de Dieu est notre seul appui.}]{Chapitre XXIX, La grâce de Dieu est notre seul appui.}
\noindent \pn{40}Et toute mon espérance n’est que dans la grandeur de votre miséricorde. Donnez-moi ce que vous m’ordonnez, et ordonnez-moi ce qu’il vous plaît. Vous me commandez, la continence.\par

\begin{quoteblock}
\noindent « Et je sais, dit votre serviteur, que « nul ne peut l’avoir, si Dieu ne la lui donne. Et savoir même d’où vient ce don en est un de la sagesse (Sag. VIII, 21).»\end{quoteblock}

\noindent La continence nous recompose et ramène à l’unité les fractions multiples de nous-mêmes. Car ce n’est pas assez vous aimer que d’aimer avec vous quelque chose que l’on n’aime pas pour vous. O amour toujours brûlant sans jamais s’éteindre ; amour, mon Dieu, embrasez-moi ! Vous m’ordonnez la continence ; donnez-moi ce que vous m’ordonnez, et ordonnez-moi ce qu’il vous plaît.
\section[{Chapitre XXX, Triple tentation de la volupté, de la curiosité et de l’orgueil.}]{Chapitre XXX, Triple tentation de la volupté, de la curiosité et de l’orgueil.}
\noindent \pn{41}Vous m’ordonnez formellement de proscrire la concupiscence de la chair, la concupiscence des yeux, et l’ambition du siècle (Jean, II, 16). Vous défendez l’amour illégitime ; et, quant au mariage, si vous l’avez permis, vous avez conseillé mieux. Et vous m’avez donné de faire selon votre désir, avant même d’être appelé au ministère de vos sacrements.\par
Mais elles vivent encore dans ma mémoire, dont j’ai tant parlé, ces images qu’une triste accoutumance y a fixées. Faibles et pâles, tant que je veille, elles attendent mon sommeil pour m’insinuer un plaisir, pour me dérober une ombre de consentement et d’action. Vaines illusions, assez puissantes toutefois sur mon âme et sur ma chair pour obtenir de moi, quand je dors, ce que les réalités demandent en vain à mon réveil. Suis-je donc alors autre que moi-même, Seigneur mon Dieu ? Et cependant quelle différence entre moi et moi, dans cet instant de passage au sommeil, de retour à la veille !\par
Où est cette raison vigilante contre de telles séductions ? Supérieure aux atteintes des réalités mêmes, se ferme-t-elle avec les yeux ? s’assoupit-elle avec les sens ? D’où vient donc que souvent nous résistons endormis, fidèles au souvenir de nos bonnes résolutions ? Nul attrait flatteur ne triomphe alors de notre chaste persévérance. Et toutefois, quand il en arrive autrement, nous sommes si absents de nous-mêmes que nous retrouvons, au réveil, le repos de notre conscience : la douleur de ce qui s’est passé en nous n’est point un remords pour la volonté qui dormait.\par
\pn{42}Mais votre main, Dieu tout-puissant, n’a-t-elle pas le pouvoir de guérir toutes les langueurs de mon âme, et de verser une grâce abondante sur les mouvements impurs de mon sommeil ? Une nouvelle effusion de miséricordes, Seigneur, pour que mon âme, dégagée des appâts de la concupiscence, me suive, et que je vous l’amène ; qu’elle ne se révolte. plus contre soi ; que, loin de se livrer, endormie, aux imaginations impures et brutales, jusqu’à séduire la chair, elle refuse la moindre adhésion ! Eloignez de moi toute surprise, la plus faible même, celle qui fuirait devant un souffle de chasteté exhalé dans mon sommeil : il vous en coûtera peu de m’accorder cette grâce en cette vie, à l’âge où je suis ; ô vous, qui êtes assez puissant pour nous, exaucer au delà de nos prières, au delà de nos pensées (Ephés. III, 20).\par
Et j’ai dit à mon bon Maître ce que je suis encore dans ces ressentiments de ma misère ; et, pénétré d’une joie craintive (Ps. II, 11), je me réjouis, Seigneur, de ce que vous m’avez donné, et je m’afflige de rester inachevé, et j’espère que vous accomplirez en moi votre œuvre de clémence, jusqu’à la paix définitive que mes puissances intérieures et extérieures feront avec vous, au jour où la mort sera engloutie dans la victoire (I Cor. XV, 54).  
\section[{Chapitre XXXI, De la volupté dans les aliments.}]{Chapitre XXXI, De la volupté dans les aliments.}
\noindent \pn{43}Le jour me suggère un autre ennemi ; et plût à Dieu qu’il pût lui suffire ! Nous réparons, par le boire et le manger, les ruines journalières du corps, jusqu’au moment où, détruisant l’aliment et l’estomac, vous éteindrez mon indigence par une admirable plénitude, et revêtirez cette chair corruptible d’une éternelle incorruptibilité (I Cor. XV, 53). Aujourd’hui toutefois, cette nécessité m’est douce, et je combats cette douceur pour ne pas m’y laisser prendre : guerre de tous les instants que je me fais par le jeûne, et les rigueurs qui réduisent le corps en servitude (Cor. IX, 27) ; et pourtant je ne puis éviter le plaisir qui chasse les douleurs du besoin : car la faim et la soif sont aussi des douleurs, brûlantes et meurtrières comme la fièvre, si les aliments ne les soulagent ; et votre bonté consolante mettant à la disposition de nôtre misère les tributs du ciel, de la terre et des eaux, nos angoisses deviennent des délices.\par
\pn{44}Vous n’avez enseigné à ne prendre les aliments que comme des remèdes. Mais quand je passe de l’inquiétude du besoin au repos qui eh suit la satisfaction, le piège de la concupiscence m’attend au, passage ; car ce passage lui-même est un plaisir ; et il n’est pas d’autre voie, et c’est la nécessité qui m’y pousse. L’entretien de la vie est la seule raison du boire et du manger, et néanmoins un dangereux plaisir marche de compagnie ; esclave qui trop souvent cherche à devancer son maître, revendiquant pour lui-même ce que je prétends n’accorder qu’à l’intérêt légitime. Et puis, les limites de l’un ne sont pas celles de l’autre ; ce qui suffit à la nécessité ne suffit pas au plaisir ; et parfois, il devient difficile de reconnaître si nous accordons un secours à la requête du besoin, ou un excès aux perfides sollicitations de la convoitise. Notre pauvre âme sourit à cette incertitude, charmée d’y trouver une excuse pour couvrir, du prétexte de la santé, une complaisance coupable. À ces tentations, je résiste chaque jour avec effort, et j’appelle à mon secours votre bras salutaire ; et je vous remets toutes mes perplexités : car je n’ai pas encore sur ce point la stabilité du conseil.\par
\pn{45}J’entends la voix de mon Dieu :\par

\begin{quoteblock}
\noindent « Ne laissez pas appesantir vos cœurs par l’intempérance et l’ivrognerie (Luc, XXI, 34). »\end{quoteblock}

\noindent Ce dernier vice est loin de moi ; votre miséricorde ne lui permettra jamais de m’approcher. Mais la sensualité s’insinue quelquefois chez votre serviteur. Que votre miséricorde la tienne éloignée de lui. Nul ne peut être continent, si vous ne lui en donnez la grâce. Vous accordez beaucoup à nos prières ; le bien même que nous avons reçu avant de vous prier, c’est vous qui nous l’avez donné, c’est de vous que nous tenons encore de nous savoir redevables. Je n’ai jamais été sujet à l’intempérance, mais j’ai connu des intempérants que vous avez rendus sobres. Vous faites les uns ce qu’ils ont toujours été, les autres ce qu’ils n’ont pas été toujours, pour qu’ils sachent, les uns et les autres, à qui ils doivent rendre grâces.\par
Vous me dites encore :\par

\begin{quoteblock}
\noindent « Ne marche pas à la suite de tes convoitises, et détourne-toi de ta volonté (Ecclési. XVIII, 30).»\end{quoteblock}

\noindent Votre grâce m’a fait entendre cette autre parole que j’aime :\par

\begin{quoteblock}
\noindent « Que nous mangions, ou ne mangions pas, rien de plus pour nous, rien de moins (I Cor. VIII, 8), »\end{quoteblock}

\noindent c’est-à-dire que je ne trouverai là ni mon opulence, ni ma détresse. Et cette parole encore :\par

\begin{quoteblock}
\noindent « J’ai appris à me contenter de l’état où je suis ; je sais vivre dans l’abondance, et je sais souffrir le besoin. Je peux tout en celui qui me fortifie (Philipp. IV, 11-13). »\end{quoteblock}

\noindent Voilà comme parle un soldat du ciel ; est-ce notre langage, poussière que nous sommes ? Mais souvenez-vous, Seigneur, que nous sommes poussière ; que c’est de poussière que vous avez fait cet homme, perdu et retrouvé (Ps. CII, 14 ; Gen. III, 19 ; Luc, XV, 24, 32.). Et ce n’est pas en lui qu’il a trouvé sa force, celui-là, poussière comme nous, qui darde au souffle de votre inspiration ces paroles brûlantes dans mon cœur : « Je peux tout en celui qui me fortifie. » Oh ! fortifiez-moi, pour que je puisse ! Donnez-moi ce que vous m’ordonnez ; et ordonnez-moi ce qu’il vous plaît. Et il confesse, lui, qu’il a tout reçu, et que toute sa gloire est dans le Seigneur (I Cor. I, 30, 31). Il veut recevoir aussi, cet autre, que j’entends vous adresser cette prière :\par

\begin{quoteblock}
\noindent « Délivrez-moi des désirs de la sensualité (Ecclés. XXIII, 6).»\end{quoteblock}

\noindent N’est-il pas évident, ô Dieu saint, que vous donnez tout, jusqu’à l’obéissance à vos commandements ?\par
\pn{46}Vous m’avez enseigné, ô bon Père,\par

\begin{quoteblock}
\noindent « que tout est pur pour les cœurs purs ; » \emph{mais que c’est un mal de se mettre à table au scandale de son frère} (Rom. XIV, 20)\end{quoteblock}

\noindent  ; que toutes vos créatures sont   bonnes ;\par

\begin{quoteblock}
\noindent « qu’il ne faut rien refuser de ce que l’on peut recevoir en action de grâces (I Tim. IV, 4) ; »\end{quoteblock}

\noindent que ce n’est point\par

\begin{quoteblock}
\noindent « notre aliment qui nous rend recommandables à Dieu (I Cor. VIII, 8), que l’on se garde de juger sur le manger et le boire (Coloss. II, 16) ; que celui qui mange ne méprise pas celui qui s’abstient ; que celui qui s’abstient ne méprise pas celui qui mange (Rom. XIV, 3). »\end{quoteblock}

\noindent Grâces à vous de tous ces enseignements que j’ai retenus ; louanges à vous, mon Dieu, qui avez frappé à mon oreille pour introduire la lumière dans mon cœur. Délivrez-moi de toute tentation.\par
Non que je craigne l’impureté de l’aliment, je crains l’impureté de la convoitise. Je sais qu’il a été permis à Noé de se nourrir de toute chair (Gen. IX, 2,3) ; qu’Hélie a demandé à la chair l’apaisement de sa faim (III Rois, XVII, 6) ; que l’abstinence admirable de Jean n’a pas été souillée de sa pâture de sauterelles (Matth. III, 4) ; je sais aussi qu’Esaü s’est laissé surprendre par un désir de lentilles (Gen. XXV, 34) ; que David s’est accusé lui-même d’avoir désiré un peu d’eau (II Rois, XIII, 15-17) ; que notre Roi a été tenté, non de chair, mais de pain (Matth. IV, 3). Aussi le peuple, dans le désert, mérita-t-il d’être réprouvé, non pour avoir eu désir de la chair, mais parce que ce désir le fit murmurer contre le Seigneur (Nomb. XI).\par
\pn{47}Entouré de ces tentations, je lutte chaque jour contre la concupiscence du boire et du manger. Car ce n’est pas chose que je puisse me retrancher pour jamais, comme le désir de la femme. Il me faut donc tenir à ma bouche un frein qui se relâche et se retire à propos. Et, Seigneur, quel est celui qui ne s’emporte quelquefois au delà des barrières de la nécessité ? S’il en est un, il est grand, qu’il vous glorifie de sa perfection ! Moi, je ne suis pas cet homme ; je suis un pécheur, et je glorifie pourtant votre nom, assuré que Celui qui a vaincu le siècle (Jean, XVI, 33) intercède auprès de vous pour mes péchés (Rom. VIII, 34), qu’il m’a compté entre les membres infirmes de son corps, dont vos yeux ne dédaignent pas les imperfections, et qui sont tous inscrits au livre de vie(Ps. CXXXVIII, 16).
\section[{Chapitre XXXII, Plaisir de l’odorat.}]{Chapitre XXXII, Plaisir de l’odorat.}
\noindent \pn{48}Les odeurs me hissent assez indifférent à leur charme. Absentes, je ne les recherche pas, je ne répudie pas leur présence ; je suis disposé à m’en passer. Du moins me semble-t-il ainsi, et je me trompe peut-être. Car ne faut-il pas gémir sur cette nuit profonde qui, nous voilant les ressorts de notre être, interdit à l’esprit, lorsqu’il se consulte lui-même sur sa puissance, toute créance facile à ses réponses, parce qu’il ignore d’ordinaire ce qu’il recèle en lui, si l’expérience ne le lui découvre ? Et nul homme ne doit être en sécurité dans cette vie qui n’est, tout entière, qu’une tentation (Job, VII, 1) ; de mauvais devenu meilleur, rien ne garantit que de meilleur il ne devienne pire. Il n’est qu’un espoir, qu’une confiance, qu’une promesse sûre, votre miséricorde.
\section[{Chapitre XXXIII, Plaisir de l’ouïe. — du chant d’Église.}]{Chapitre XXXIII, Plaisir de l’ouïe. — du chant d’Église.}
\noindent \pn{49}Les voluptés de l’oreille m’avaient captivé par des liens plus forts ; mais vous les avez brisés ; vous m’avez délivré de cet esclavage. Cependant, je l’avoue, aux accents que vivifient vos paroles chantées par une voix douce et savante, je ne puis me défendre d’une certaine complaisance, impuissante toutefois à me retenir quand il me plaît de me retirer. Suaves mélodies, n’est-ce pas justice qu’admises avec les saintes pensées qui sont leur âme, je leur fasse dans la mienne une place d’honneur ? mais j’ai peine à garder une juste mesure.\par
Car il me semble que je leur accorde parfois plus qu’il ne convient, sentant que par cette harmonie, les paroles sacrées pénètrent mon esprit d’une plus vive flamme d’amour ; et je vois que les affections de l’âme et leurs nuances variées retrouvent chacune sa note dans les modulations de la voix, et je ne sais quelle secrète sympathie qui les réveille. Mais le charme sensible, à qui il ne faut pas laisser le loisir d’énerver l’âme, me trompe souvent, quand la sensation se lasse de marcher après la raison, et prétend autoriser de là faveur d’être admise à sa suite, ses efforts pour la précéder et la conduire. C’est là que je pèche sans m’en apercevoir, mais bientôt je m’en aperçois.\par
\pn{50}D’autres fois, un excès de précautions   contre de telles surprises me jette dans un excès de rigidité, et je voudrais éloigner de mon oreille et de l’Église même ces touchantes harmonies, compagnes ordinaires des psaumes de David. Il me parait alors plus sûr de s’en tenir à ce que j’ai souvent ouï dire d’Athanase, évêque d’Alexandrie, qu’il les faisait réciter avec une légère inflexion de voix, plus semblable à une lecture qu’à un chant.\par
Et cependant quand je me rappelle ces larmes que les chants de votre Église me firent répandre aux premiers jours où je recouvrai la foi, et qu’aujourd’hui même je me sens encore ému, non de ces accents, mais des paroles modulées avec leur expression juste par une voix pure et limpide, je reconnais de nouveau la grande utilité de cette institution. Ainsi je flotte entre le danger de l’agréable et l’expérience de l’utile, et j’incline plutôt, sans porter toutefois une décision irrévocable, au maintien du chant dans l’Église, afin que le charme de l’oreille élève aux mouvements de la piété l’esprit trop faible encore. Mais pourtant, lorsqu’il m’arrive d’être moins touché du verset que du chant, c’est un péché, je l’avoue, qui mérite pénitence : je voudrais alors ne pas entendre chanter.\par
Voilà où j’en suis. Pleurez avec moi, pleurez pour moi, vous qui sondez en vous-mêmes la source vive des bonnes œuvres ; car, pour vous, qui la négligez, ces plaintes ne vous ton client guère. Mais, Seigneur mon Dieu, témoin de cette laborieuse étude de moi-même, ma langueur est sous vos yeux ; voyez, entendez-moi ; donnez-moi un regard de pitié, guérissez-moi.
\section[{Chapitre XXXIV, Volupté des yeux.}]{Chapitre XXXIV, Volupté des yeux.}
\noindent \pn{51}Reste la volupté des yeux de ma chair, dont je vais publier les confessions à l’oreille de votre temple , des âmes fraternelles et pieuses ; ainsi j’aurai parlé de toutes les tentations charnelles qui me frappent encore, tandis que je gémis,\par

\begin{quoteblock}
\noindent « et soupire après cette habitation céleste dont Je brûle d’être revêtu comme d’un second vêtement (II Cor. V, 2). »\end{quoteblock}

\noindent La beauté, la variété des formes, l’agrément et la vivacité des couleurs charment les yeux. Que mon âme ne demeure pas attachée à ces objets ; que Dieu la retienne, Dieu leur auteur,\par

\begin{quoteblock}
\noindent «dont toutes les œuvres sont bonnes (Ecclési. XXXIX,39) ; »\end{quoteblock}

\noindent mais lui seul est mon bien, et non pas elles. Et elles me sollicitent, tant que je veille pendant la durée du jour ; et il ne m’est pas donné de m’en reposer, comme je me repose des chants qui ont cessé, quelquefois de tout bruit, dans un profond silence. Car la reine des couleurs elle-même, cette lumière qui inonde tout ce que nous voyons, se glisse partout où je suis pendant le jour, me pénètre par mille insinuations charmeresses, alors même que je porte ailleurs l’activité de ma pensée. Elle s’insinue si profondément, qu’à sa disparition soudaine nous la recherchons avec inquiétude ; et son absence prolongée nous attriste l’âme.\par
\pn{52}O lumière que voyait Tobie l’aveugle, lorsqu’il enseignait à son fils le chemin de la vie, et, sans s’égarer, y marchait devant lui d’un pied sûr, du pied de la charité (Tob. IV) ! Lumière que voyait Isaac, malgré la nuit pesante dont la vieillesse avait voilé ses yeux ! lumière par laquelle il sut connaître, en les bénissant, ses fils qu’il bénissait sans les connaître (Gen. XXVII) ! Lumière que voyait Jacob, dont le grand âge, aussi, avait éteint la vue, quand son cœur, rayonnant de clartés, mesura d’un regard toutes les générations du peuple futur, désignées dans ses fils ; quand ses mains mystérieusement croisées sur les enfants de Joseph, se refusèrent à l’ordre extérieur que leur père voulait rétablir ; car elles étaient imposées selon le discernement intérieur (Gen. XLIXX).\par
Voilà la lumière même ; elle est une ; elle ne fait qu’un de tous ceux qui la voient et qui l’aiment. Mais cette lumière corporelle, dont je parlais, assaisonne la vie pour les aveugles amants du siècle, d’enivrantes et perfides douceurs. Et à ceux toutefois qui savent vous en rendre hommage, ô Dieu créateur de toutes choses, elle sert de degré pour monter à votre gloire, et non pour descendre au fond de leur sommeil. C’est ainsi que je veux être.\par
Je lutte contre les séductions des yeux, de peur que mes pieds ne s’y embarrassent à l’entrée de vos voies ; et j’élève vers vous mes yeux invisibles, afin que les nœuds qui arrêtent mes pas soient rompus (Ps. XXIV).Vous les dégagez souvent, car souvent ils s’engagent. Vous ne cessez de me délivrer, et je ne cesse de me prendre aux piéges semés partout ; vigilant défenseur d’Israël, vous ne dormez, vous ne sommeillez jamais (Ps. CXX, 4).  \par
\pn{53}Que de séductions sans nombre dans les œuvres de l’art et de l’industrie, vêtements, vases, tableaux, statues ; abus d’une nécessité, abus même d’une intention pieuse ; nouveaux enivrements que les hommes ajoutent aux convoitises des yeux ; répandus au dehors à la suite de leurs œuvres, oubliant en eux-mêmes Celui qui les a faits, ils gâtent en se défigurant le chef-d’œuvre divin.\par
Ici même, ô mon Dieu ! ô ma gloire ! ici je trouve à glorifier votre nom ; ô mon sanctificateur ! je vous offre un sacrifice de louanges ! car ces beautés que vous faites passer de l’âme à la main de l’artiste, procèdent de cette beauté, supérieure à nos âmes, et vers laquelle mon âme soupire nuit et jour. Mais ces amateurs, ces fabricants de beautés extérieures, empruntent à l’invisible la lumière qui les leur fait agréer, et non la règle qui en dirige l’usage. Elle est présente, et ils ne la voient pas. C’est en vain qu’elle leur dit de ne pas aller plus loin, et de vous conserver toute leur force (Ps. LVIII, 10), au lieu de la dissiper dans ces délices énervantes.\par
Et moi qui en parle ainsi, qui en parle avec discernement, j’engage encore mes pas aux filets de ces beautés ; mais vous me délivrez, Seigneur, vous me délivrez,\par

\begin{quoteblock}
\noindent « parce que votre miséricorde est toujours présente à mes « yeux (Ps. XXV,3).»\end{quoteblock}

\noindent Ma faiblesse se laisse prendre, votre miséricorde me délivre ; parfois sans souffrance, quand je tombe par mégarde ; parfois avec douleur, quand le lien s’est resserré.
\section[{Chapitre XXXV, Curiosité.}]{Chapitre XXXV, Curiosité.}
\noindent \pn{54}Ajoutez une autre tentation qui nous environne de périls multipliés. Outre la concupiscence de la chair, mêlée à toutes les impressions sensibles, à toutes les voluptés dont le fol amour consume ceux qui se retirent de vous, il se glisse encore dans l’âme, par les sens, un nouveau désir, ne demandant plus du plaisir à la chair, mais des expériences ; vaine curiosité qui se couvre du nom de connaissance et de savoir. Or, comme elle consiste dans l’appétit de connaître, et que la vue est le premier organe de nos connaissances, l’Esprit-Saint l’a nommée concupiscence des yeux (I Jean, II, 16).\par
Voir appartient aux yeux, mais nous attribuons cette expression aux autres sens, quand nous les appliquons à connaître. Car nous ne disons pas d’un objet : Ecoute comme il rayonne, sens comme il brille, goûte comme il resplendit, touche comme il éclate. Un seul mot pour tout cela, vois ; et non-seulement, vois quelle lumière, ce qui est exclusivement du ressort des yeux, mais encore, vois quel son, vois quelle odeur, vois quelle saveur, vois quelle dureté. Aussi l’expérience générale des sens, avons-nous dit, est-elle nommée concupiscence des yeux. Quoique, en effet, la vision soit leur fonction particulière, les autres sens l’usurpent néanmoins, quand, à l’exemple des yeux, ils explorent quelque vérité.\par
\pn{55}Or, on discerne sans peine si l’intérêt du plaisir ou celui de la curiosité fait agir les sens. Le plaisir recherche la beauté, l’harmonie, les odeurs, les saveurs, les doux attouchements, la curiosité veut essayer même de leurs contraires, non pour affronter une impression pénible, mais par fantaisie d’éprouver et de savoir. Quel plaisir, en effet, peut nous offrir l’aspect d’un cadavre déchiré, qui fait horreur ? En est-il un gisant, tous accourent pour rapporter de cette vue la consternation, la pâleur. Ils craignent maintenant de le revoir dans leur sommeil. Eh ! qui les a contraints, éveillés, de le voir ? Quel ouï-dire leur a donné l’espérance d’y trouver quelque beauté ? — Ainsi des autres sens ; mais il serait trop long de poursuivre.\par
C’est cette maladie qui invente les raffinements des spectacles ; c’est elle qui prétend pénétrer les secrets les plus cachés de la nature, inutiles à connaître, et dont les hommes ne désirent rien que la connaissance ; c’est elle qui sollicite les efforts prévaricateurs de la magie ; c’est elle enfin qui, dans la religion même, va jusqu’à tenter Dieu, et lui demande des prodiges par fantaisie, et non par charité.\par
\pn{56}Dans cette immense forêt, remplie d’embûches et de périls, combien de coupes n’ai-je pas déjà faites ? que n’ai-je pas retranché dans mon cœur ? grâce à votre assistance, ô Dieu de mon salut ! Et cependant, la vie de chaque jour étant assaillie de ces essaims d’objets qui bourdonnent autour d’elle, quand oserai-je dire que nul d’entre eux ne fixe mon regard, et que je défie tous les piéges d’une vaine curiosité ? À cette heure, il est vrai, je suis indifférent au plaisir du théâtre ; je me soucie peu de connaître le cours des astres ; jamais mon âme n’a interrogé les ombres ; et j’abhorre tout   pacte sacrilége. Mais, ô Seigneur mon Dieu, à qui je dois le service du plus humble esclave, par quelles insinuations perfides l’ennemi ne me suggère-t-il pas de vous demander quelque miracle ? Et je vous conjure, par notre Roi, par notre patrie sainte, la chaste et pure Jérusalem, qu’un coupable consentement, jusqu’à présent éloigné de mon âme, s’en éloigne de plus en plus chaque jour. Mais quand je vous sollicite pour la santé d’un frère, le but de mes instances est bien différent ; vous faites comme il vous plaît, et vous me donnez la grâce, vous ne me la refuserez jamais, d’embrasser votre volonté. 57. Et cependant combien de bagatelles et de frivolités méprisables séduisent encore chaque jour notre curiosité ? Qui pourrait compter nos tentations et nos chutes ? Combien de fois souffrons-nous, par certaine condescendance pour les faibles, de vains récits que, peu à peu, nous écoutons avec plaisir ? Je ne vais plus au cirque voir un chien courir après un lièvre ; mais que le hasard dans le champ où je passe, m’en donne le spectacle, me voilà peut-être détourné d’une méditation profonde ; cette chasse inattendue m’attire ; elle ne m’oblige pas de tourner bride, mais de laisser courre mon cœur. Et si, en me donnant la preuve de ma faiblesse, vous ne m’inspirez aussitôt de ramener mon esprit de cette vue à une pensée qui m’élève jusqu’à vous, ou bien de passer outre avec mépris, je reste amusé de cette puérile distraction.\par
Que dis-je ? sans sortir de ma maison, un lézard, qui prend des mouches, une araignée, qui les enveloppe de ses fils, n’est-ce pas assez pour captiver mes yeux ? La petitesse de ces animaux diminue-t-elle donc l’action de ma curiosité ? Je passe de là à vous louer, Créateur, ordonnateur admirable de toutes choses ; mais cette fin n’était pas le principe de mon attention : autre chose est de se relever prompte, ment ou de ne tomber jamais. Et toute ma vie est pleine de faux pas ; et la grandeur de votre clémence est mon unique espoir. Car, dès lors que notre âme, prostituée à ces vains objets, se remplit de conceptions frivoles, il arrive que nos prières sont souvent interrompues et troublées ; et en votre présence, la voix de notre cœur veut-elle monter jusqu’à vous, une irruption de pensées misérables, accourues je ne sais d’où, vient traverser un acte si important.
\section[{Chapitre XXXVI, Orgueil.}]{Chapitre XXXVI, Orgueil.}
\noindent \pn{58}Et ceci, est-ce pure bagatelle dont il faille tenir peu de compte ? Et notre espérance peut-elle être ailleurs que dans la miséricorde bien connue, qui a commencé l’œuvre de notre conversion ?\par
Et vous savez à quel point vous m’avez changé, me guérissant d’abord de la passion de la vengeance, pour devenir secourable à mes autres iniquités, dissiper toutes mes langueurs, racheter ma vie de la corruption, pour me donner la couronne de grâce et de miséricorde, et prodiguer vos biens à la merci de mes désirs (Ps. CIII, 3-5). Vous m’avez inspiré votre crainte, qui éteint l’orgueil, et apprivoisé ma tête à votre joug. Et je le porte aujourd’hui, et ce fardeau m’est doux ; vous me l’aviez promis, vous tenez votre promesse (Matth. XI, 30) et il était en effet léger, à mon insu, quand je craignais de m’y soumettre. Mais dites-moi, Seigneur, seul dominateur exempt d’orgueil, parce que vous êtes le seul Maître véritable, et qui n’en connaît point d’autre, dites-moi, suis-je délivré, ou pourrai-je l’être jamais dans cette vie, de ce troisième genre de tentation ?\par
\pn{59}Vouloir être craint et aimé des hommes, sans autre raison que le désir d’une joie qui n’est pas vraie, c’est une vie misérable, c’est une honteuse insolence. Et voilà pourquoi notre cœur est sans amour pour vous, et notre crainte sans pureté. Aussi, vous répandez sur les humbles la grâce que vous refusez aux superbes (I Pierre, V,5) ; vous tonnez sur les ambitions du siècle, et les fondements des montagnes tremblent.\par
Or, comme l’intérêt de la société humaine y fait un devoir de l’amour et de la crainte, l’ennemi de notre véritable félicité nous presse, et par tous les pièges qu’il sème sous nos pas, il nous crie : Courage, courage ! Il veut que notre avidité à recueillir nous laisse surprendre ; il veut que nos joies se déplacent et quittent votre vérité pour se fixer au mensonge des hommes ; il veut que nous prenions plaisir à nous faire aimer et craindre, non pour vous, mais au lieu de vous. Et, nous rendant semblables à lui-même, il veut nous gagner, non pas à l’union de la charité, mais au partage de son supplice, lui qui a mis son trône sur l’aquilon, afin que vos coupables et difformes imitateurs   tombent dans ses fers (Isaïe, XIV, 13-15) ténébreux et glacés. Mais nous, Seigneur, nous sommes votre petit troupeau (Luc, XII, 32) ; nous voilà ; prenez votre houlette. Etendez vos ailes sur nous ; que leur ombre soit notre asile. Soyez notre gloire ; que l’on ne nous aime que pour vous ; que votre Verbe seul se fasse craindre en nous. Celui qui veut être loué des hommes, malgré votre blâme, ne trouvera pas d’homme pour le défendre à votre tribunal, ni pour le soustraire à votre arrêt. Et il ne s’agit point d’un pécheur flatté dans les mauvais instincts de son âme, ni d’un impie dont on bénit l’iniquité (Ps. X, 13), mais d’un homme loué pour quelque grâce reçue de vous ; s’il jouit plutôt de la louange que de cette faveur divine qui, en est l’objet, votre blâme accompagne ces louanges ; et celui qui les donne vaut mieux que celui qui les reçoit ; l’un aime dans l’homme le don de Dieu, l’autre préfère au don de Dieu celui de l’homme.
\section[{Chapitre XXXVII, Disposition de son âme touchant le blâme et la louange.}]{Chapitre XXXVII, Disposition de son âme touchant le blâme et la louange.}
\noindent \pn{60}Voilà les tentations dont nous sommes assaillis, Seigneur, chaque jour, sans relâche. Chaque jour la langue humaine est la fournaise de notre épreuve. C’est, encore ici que vous nous commandez la continence. Donnez-moi ce que vous m’ordonnez ; ordonnez-moi ce qu’il vous plaît. Vous savez ici les gémissements que mon cœur exhale, et les torrents de larmes que roulent mes yeux. Inhabile à discerner jusqu’à quel point je suis allégé de ce fardeau de corruption, je tremble pour mes maux secrets (Ps. XVIII, 3), connus de votre regard, et que le mien ignore.\par
Les autres tentations me laissent toujours quelque moyen de m’examiner, celle-ci presque jamais ; car pour les voluptés charnelles, pour les convoitises de la vaine science, je vois l’empire que j ‘ai gagné sur mon esprit, par la privation volontaire ou l’absence de ces impressions. Et je m’interroge alors, en mesurant le degré de vide que j’éprouve. Quant à la richesse, que l’on ne poursuit que pour satisfaire l’une de ces trois concupiscences, ou deux ou toutes ensemble, l’esprit se trouve-t-il dans l’impossibilité de deviner s’il la méprise en la possédant, qu’il la congédie pour s’éprouver. Est-ce à dire que, pour nous assurer de notre force à supporter le jeûne de la louange, il faille vivre mal, et en venir à un tel cynisme, que personne ne puisse nous connaître sans horreur ? Qui pourrait penser ou dire pareille extravagance ? Mais si la louange est la compagne ordinaire et obligée d’une vie exemplaire et de bonnes œuvres, il ne faut pas plus renoncer à la vertu qu’à son cortège. Et cependant, sans privation et sans absence, puis-je avoir le secret de ma résignation ?\par
\pn{61}Que vais-je donc ici vous confesser, Seigneur ? Eh bien ! je vous dirai que je me plais à la louange, mais encore plus à la vérité qu’à la louange. Car s’il m’était donné de choisir la louange des hommes pour salaire d’erreur ou de démence, ou leur blâme pour prix de mon inébranlable attachement à la vérité, mon choix ne serait pas douteux.\par
Je voudrais bien, toutefois, que le suffrage des lèvres d’autrui n’ajoutât rien à la joie que je ressens de ce peu de bien qui est en moi. Mais, je l’avoue, le bon témoignage l’augmente et le blâme la diminue. Et quand cette affliction, d’esprit me trouble, il me vient une excuse ; ce qu’el1e vaut, vous le savez, mon Dieu ; pour moi, elle me laisse dans le doute. Or, vous ne nous avez pas seulement ordonne la continence qui enseigne ce dont notre amour doit s’abstenir, mais encore la justice qui lui montre où il se doit diriger ; et vous nous commandez d’unir à votre amour celui du prochain, Il me semble donc que c’est l’avancement de l’un de mes frères que j’aime ou que j’espère, quand je me plais aux louanges intelligentes qu’il donne, et que c’est encore pour lui que je m’afflige quand je l’entends prononcer un blâme ignorant ou injuste. Quelquefois ,même je m’attriste des témoignages flatteurs que l’on me rend, soit que l’on approuve en moi ce qui me déplaît de moi-même, soit que l’on estime au delà de leur valeur des avantages secondaires. Eh ! que sais-je ? Ce sentiment ne vient-il pas de ma répugnance aux éloges en désaccord avec l’opinion que j’ai de moi ? Non qu’alors je sois touché de l’intérêt du prochain ; mais c’est que le bien que j’aime en moi m’est encore plus agréable quand je ne suis pas seul à l’aimer. Et, en effet, est-ce donc me louer que de contredire mes sentiments sur moi, en louant ce qui me déplaît, en exaltant des   qualités indifférentes ? Suis-je donc ici un mystère pour moi-même ?\par
\pn{62}Mais ne vois-je pas en vous, ô Vérité, que l’intérêt seul du prochain doit me rendre sensible à la louange ? Est-ce ainsi que je suis ? je l’ignore. Et, en cela, je vous connais mieux que moi-même. Oh ! révélez-moi à moi, mon Dieu ; que je signale aux prières de mes frères les secrètes blessures de mon âme.\par
Encore un retour sur moi : je veux me sonder plus à fond. Si la seule utilité du prochain me fait agréer la louange, d’où vient que le blâme jeté à un autre m’intéresse moins que celui qui me touche ? Pourquoi suis-je plus vivement blessé du trait qui m’atteint que de celui dont une même injustice frappe un frère en ma présence ? Est-ce encore là un secret qui m’échappe ? Et que n’ai-je déjà pris mon parti de me tromper moi-même, et de trahir devant vous la vérité et de cœur et de bouche ! Eloignez de moi, Seigneur, cette folie, de peur que mes paroles ne soient pour moi l’huile qui parfume la tête du pécheur (Ps. CXL, 5) !
\section[{Chapitre XXXVIII, Vaine gloire, poison subtil.}]{Chapitre XXXVIII, Vaine gloire, poison subtil.}
\noindent \pn{63}Je suis pauvre et dénué, et tout ce que j’ai de mieux, c’est cette déplaisance de moi-même dont le gémissement intérieur me rend témoignage, et qui ne se lassera de poursuvre votre miséricorde, que vous n’ayez soulagé mes défaillances, en consommant ma régénération dans la paix ignorée de l’oeil superbe.\par
Les paroles de notre bouche, nos actions qui se produisent à la connaissance des hommes, amènent la plus dangereuse tentation, cet amour de la louange, qui recrute, au profit de certaine qualité personnelle, des suffrages mendiés, et trouve encore à me séduire par les reproches mêmes que je me fais. Souvent l’homme tire une vanité nouvelle du mépris même de la vaine gloire ; et la vaine gloire rentre en lui par ce mépris dont il se glorifie.
\section[{Chapitre XXXIX, Complaisance en soi-même.}]{Chapitre XXXIX, Complaisance en soi-même.}
\noindent \pn{64}Il est encore en nous un autre ennemi, une tentation de même nature ; cette complaisance en soi qui se repaît de son inanité, se souciant peu de plaire ou déplaire au prochain. Or, celui qui se plaît à lui-même, vous déplaît souverainement, soit qu’il prenne en lui pour bien ce qui n’est pas bien, ou qu’il revendique comme son bien propre celui qu’il tient de vous ; soit que, reconnaissant votre don, il l’attribue à ses mérites, ou qu’enfin il confesse votre grâce, mais avec cette joie de l’égoïsme qui envie aux autres les mêmes faveurs. Parmi tant de périls et d’épreuves, vous le voyez, mon cœur tremble ; et, si le mal s’est apaisé, c’est bien moins absence de blessures que célérité de la main dont j’ai senti l’action salutaire.
\section[{Chapitre XL, Coup d’oeil sur tout ce qu’il a dit.}]{Chapitre XL, Coup d’oeil sur tout ce qu’il a dit.}
\noindent \pn{65}Dans ce long pèlerinage de ma pensée, où ne m’avez-vous pas accompagné, ô Vérité ? avez-vous cessé de m’enseigner ce qu’il fallait rechercher ou fuir, quand je vous consultais, en vous communiquant selon mon pouvoir les découvertes de l’oeil intérieur ? J’ai voyagé hors de moi-même par le sens qui m’ouvre le monde ; j’ai observé la vie de mon corps et l’action de mes sens. Et je suis entré dans les profondeurs de ma mémoire, dans ces nombreuses et immenses retraites, peuplées d’une infinité d’images ; et je les ai considérées avec épouvante ; et j’ai vu que je ne pouvais rien distinguer sans vous, et j’ai reconnu que vous étiez fort différent de tout cela.\par
Fort différent aussi de moi-même, de moi, qui, dans cette exploration intérieure, cherchais à faire le discernement exact, et la juste appréciation de mes découvertes : soit que les réalités me fussent transmises par les sens, soit que, mêlées à ma nature, je les interrogeasse en moi-même ; soit que je m’attachasse au nombre et au signalement de leurs introducteurs, et que, repassant tous ces trésors enfermés dans ma mémoire, ma pensée exhumât les uns et mît les autres en réserve. Oui, vous êtes fort différent de moi, qui fais cela, et de la puissance intérieure par qui je le fais ; et vous n’êtes pas cette puissance, parce que vous êtes la lumière immuable que je consulte sur l’être, la qualité, la valeur de toutes choses. Ainsi j’écoutais, et j’écoute souvent vos leçons et vos commandements. Votre voix fait mes délices, et, dans ce peu de loisirs que me laisse la nécessité de mes travaux, cette joie sainte est mon asile.  .\par
Et, dans tous ces objets que je parcours à la clarté de votre lumière, je ne trouve de lieu sûr pour mon âme qu’en vous ; il n’est que vous, où mon être épars puisse se rassembler pour y demeurer à jamais tout entier. Et parfois vous me pénétrez d’un sentiment étrange, douceur inconnue, qui, devenant en moi parfaite et durable, serait je ne sais quoi qui ne serait plus cette vie. Mais je retombe sous le poids de ma chaîne, et le torrent m’entraîne, et je suis lié ; et je pleure, et mes larmes ne relâchent pas mes liens. Le fardeau de l’habitude m’emporte au fond. Où je puis être, je ne veux ; où je veux, je ne puis ; double misère.
\section[{Chapitre XLI, Ce qui le rejetait loin de Dieu.}]{Chapitre XLI, Ce qui le rejetait loin de Dieu.}
\noindent \pn{66}Et j’ai reconnu dans cette triple convoitise la source de mes coupables infirmités, et j’ai demandé mon salut à votre bras. Car j’ai vu votre gloire avec un cœur blessé, et, tout ébloui, j’ai dit : Qui peut voir jusque-là ? Et j’étais rejeté loin de la splendeur de vos regards (Ps. XXX, 23). Vous êtes la Vérité qui préside sur toutes choses. Et mon insatiable avarice ne voulait pas vous perdre ; elle voulait posséder le mensonge avec vous. Ainsi le menteur ne veut pas que la vérité lui soit inconnue. Je vous avais donc perdu, parce que vous ne souffrez pas qu’on vous possède sans répudier l’héritage du mensonge.
\section[{Chapitre XLII, Égarement des superbes qui ont eu recours aux anges déchus comme médiateurs entre Dieu et les Hommes.}]{Chapitre XLII, Égarement des superbes qui ont eu recours aux anges déchus comme médiateurs entre Dieu et les Hommes.}
\noindent \pn{67}Qui trouver, capable de me réconcilier avec vous ? Devais-je solliciter les anges ? et par quelles prières ? par quels sacrifices ? Plusieurs, ai-je ouï dire, travaillant pour revenir à vous, et ne le pouvant d’eux-mêmes, ont tenté cette voie, et, tombés bientôt dans un désir curieux de visions étranges, ils ont mérité d’être livrés à l’illusion. Superbes, ils vous cherchaient avec tout le faste de la science, le cœur haut et non contrit ; la conformité d’esprit a attiré sur eux les complices de leur orgueil, les puissances de l’air (Ephés. II, 2), dont les prestiges les ont égarés lorsqu’ils cherchaient un médiateur, médecin de leur âme, sans le trouver ; car ils n’avaient devant eux que le diable transfiguré en ange de lumière (II Cor. XI, 14).\par
Chair superbe, ce qui l’a séduite, c’est que le séducteur n’était pas revêtu de chair ! Hommes mortels et pécheurs ! Mais vous, Seigneur, dont ils cherchaient la paix avec orgueil, vous êtes indépendant de la mort et du péché. Or, il fallait au médiateur entre l’homme et Dieu (I Tim. II, 5) une ressemblance avec Dieu et une ressemblance avec l’homme. Entièrement semblable à l’homme, il était loin de Dieu ; entièrement semblable à Dieu, il était loin de l’homme - ; il n’était plus médiateur. Ainsi ce faux médiateur, à qui votre justice secrète permet de séduire l’orgueil, a quelque chose de commun avec l’homme : c’est le péché ; il prétend quelque chose de commun avec Dieu : libre du vêtement charnel de la mortalité, il se donne pour immortel. Mais, comme « la mort est la solde du péché (Rom. VI, 23), il entre, par la communauté du péché, dans la communauté de la mort.
\section[{Chapitre XLIII, Jésus-Christ seul médiateur.}]{Chapitre XLIII, Jésus-Christ seul médiateur.}
\noindent \pn{68}Mais le Médiateur de vérité, que le secret de votre miséricorde a fait connaître aux humbles, et que vous avez envoyé pour leur enseigner, par son exemple, l’humilité même, ce Médiateur de Dieu et des hommes, JÉSUS-CHRIST homme, est apparu entre les pécheurs mortels et le JUSTE immortel, mortel avec les hommes, Juste avec Dieu ; et comme la vie et la paix sont la solde de la justice, par la justice qui l’unit à Dieu, il est venu ruiner dans les impies justifiés la mort dont il voulut être comme eux tributaire. C’est lui qui a été montré de loin aux saints des anciens jours, pour qu’ils fussent sauvés par la foi au sang qu’il devait répandre, comme nous le sommes par la foi en son sang répandu. Car ce n’est qu’en sa qualité d’homme qu’il est médiateur ; en tant que Verbe, il n’est plus terme MOYEN, il est ÉGAL à Dieu, Dieu en Dieu, et avec le Saint-Esprit un seul Dieu.\par
\pn{69}Oh ! de quel amour nous avez-vous donc aimés, Père infiniment bon ? vous n’épargnez pas votre Fils unique, vous le livrez pour nous, pécheurs que nous sommes (Rom. VIII, 32). De quel amour   nous avez-vous donc aimés ? Pour nous,\par

\begin{quoteblock}
\noindent « Celui qui n’a point regardé comme une usurpation d’être égal à vous, s’est rendu obéissant jusqu’à la mort de la croix (Philip. II, 6), lui seul libre entre les morts (Ps. LXXXVII, 6-8), ayant la puissance de « quitter son âme et la puissance de la reprendre (Jean, X, 18) ; »\end{quoteblock}

\noindent pour nous, en votre nom, vainqueur et victime, et vainqueur parce qu’il est victime ; pour nous, en votre nom, sacrificateur et sacrifice, et sacrificateur parce qu’il est sacrifice, lui qui, d’esclaves, nous fait vos enfants, parce qu’il est votre Fils et pour nous esclave. Oh ! c’est avec justice que sur lui repose cette ferme espérance que vous guérirez toutes mes langueurs, par lui qui est assis à votre droite, et sans cesse y intercède pour nous (Rom. VIII, 34) ; autrement je tomberais dans le désespoir ; car nombreuses et grandes sont mes infirmités, nombreuses et grandes ! mais plus grande encore est la vertu de vos remèdes. Nous eussions pu croire votre Verbe trop éloigné de l’alliance de l’homme, et désespérer de nous s’il ne s’était fait chair, s’il n’eût demeuré parmi nous.\par
\pn{70}Plié sous la crainte de mes péchés et le fardeau de ma misère, j’avais délibéré dans mon cœur et presque résolu de fuir au désert ; mais vous m’en avez empêché, me rassurant par cette parole :\par

\begin{quoteblock}
\noindent « Le CHRIST est mort pour tous, afin que ceux qui vivent ne vivent plus à eux-mêmes, mais à celui qui est mort pour eux (I Cor. V, 15). »\end{quoteblock}

\noindent Eh bien ! Seigneur, je jette tous mes soucis en votre sein, pour vivre, pour goûter les merveilles de votre loi (Ps. CXVIII, 18). Vous savez mon ignorance et ma faiblesse ; enseignez-moi, guérissez-moi. Ce Fils unique\par

\begin{quoteblock}
\noindent « en qui sont cachés tous les « trésors de la sagesse et de la science m’a « racheté de son sang (Coloss. II, 3). »\end{quoteblock}

\noindent Loin de moi les calomnies des superbes. Je médite ma rançon, et je la mange, et je la bois, et je la distribue ; pauvre encore, je désire en être rassasié avec ceux qui la mangent et en sont rassasiés ; qui louent le Seigneur parce qu’ils le cherchent (Ps. XXI, 27).
\chapterclose


\chapteropen
 \chapter[{XI. La Création du monde et le temps}]{XI. La Création du monde et le temps}\phantomsection
\label{XI}\renewcommand{\leftmark}{XI. La Création du monde et le temps}


\begin{argument}\noindent Il demande à Dieu l’intelligence des Écritures. — Il cherche à expliquer les premières paroles de la Genèse : « Dans le principe Dieu fit le ciel et la terre. » — Il répond à cette question : « Que faisait Dieu avant la création du monde ? » — Point de temps avant la création. — Qu’est-ce que le temps ? — Quelle est la mesure du temps ?
\end{argument}


\chaptercont
\section[{Chapitre premier, La confession de nos misères dilate notre amour.}]{Chapitre premier, La confession de nos misères dilate notre amour.}
\noindent \pn{1}Eh quoi ! ce que je vous dis, l’ignorez-vous donc, ô Dieu, possesseur de l’éternité ? L’ignorez-vous, ou avez-vous besoin du temps, pour voir ce qui se passe dans le temps ? Pourquoi donc vous présenter le cours et la suite de tant de choses ? Non pour vous les apprendre, sans doute, mais pour susciter vers vous dans mon cœur et dans les cœurs qui me liront de nouvelles flammes, afin qu’un seul cri s’élève :\par

\begin{quoteblock}
\noindent « Le Seigneur est grand et infiniment digne de louanges (Ps. XCV, 4) »\end{quoteblock}

\noindent Je l’ai dit, et je le dis encore ; c’est l’amour de votre amour qui m’a suggéré cette pensée. Nous prions, et cependant la Vérité nous dit :\par

\begin{quoteblock}
\noindent « Votre Père sait ce qu’il vous faut, avant même que vous lui demandiez rien (Matth. VI, 8). »\end{quoteblock}

\noindent Ainsi la confession de nos misères et de vos miséricordes dilate notre amour pour vous ; elle appelle sur nous cette grâce qui doit consommer notre délivrance et nous sortir de nous-mêmes, séjour de malheur, pour nous faire entrer en vous, souveraine béatitude. Car vous nous avez appelés à la pauvreté volontaire, à la douceur, à la faim et à la soif de la justice, à l’amour des larmes, et de la compassion, et de la pureté intérieure, et de la paix (Matth. V, 3-9). Et je vous ai tout raconté, suivant mes forces et ma volonté, car vous avez voulu le premier que j’élevasse jusqu’à vous, Seigneur mon Dieu, les louanges de votre bonté et de vos miséricordes éternelles (Ps. CXVII, 1).
\section[{Chapitre II, Il demande à Dieu l’intelligence des écritures.}]{Chapitre II, Il demande à Dieu l’intelligence des écritures.}
\noindent \pn{2}Et ma plume serait-elle un organe capable de publier par quelles inspirations quelles saintes terreurs, par quelles consolations, quelles secrètes conduites vous m’avez amené au ministère de votre parole et à la dispensation de vos sacrements ? Et puis, eussé-je la force d’être un narrateur fidèle, chaque goutte de temps me coûte si cher !\par
Et depuis longtemps je brûle de méditer votre loi, et de vous confesser à cet égard mes lumières et mon ignorance ; les premiers reflets de vos rayons, et la lutte des ténèbres qui règnent encore dans mon âme, jusqu’à ce que ma faiblesse soit absorbée par votre force. Et je ne veux pas répandre sur d’autres soins les heures de loisir que me laissent les besoins de la nature, le délassement nécessaire de l’esprit, et le service que nous devons aux hommes, ou que nous leur rendons sans leur devoir.\par
\pn{3}Seigneur mon Dieu, prêtez l’oreille à ma prière ; que votre clémence exauce mon désir. Ce n’est pas pour moi seul que ce cœur palpite ; il se passionne encore pour l’intérêt de ses frères. Et vous voyez dans ce cœur qu’il est ainsi. Oh ! que je vous offre en sacrifice ce servage de pensées et de paroles dont je suis redevable ; et donnez-moi de quoi vous offrir.\par

\begin{quoteblock}
\noindent « Je suis indigent et pauvre (Ps. LXXXV, 1), et vous êtes « riche ; et vous versez vos libéralités sur tous « ceux qui vous invoquent (Rom. X, 12) »\end{quoteblock}

\noindent ô vous dont la Providence ne trouble pas la Sécurité. Retranchez en moi toute témérité, tout mensonge,   par la circoncision du cœur et des lèvres. Que vos Écritures soient mes chastes délices. Que je n’y trouve ni à m’égarer, ni à égarer les autres. Voyez, Seigneur ; ayez pitié, Seigneur mon Dieu, lumière des aveugles, vertu des faibles ; encore leur lumière et leur vertu, quand ils ont recouvré la vue et la force ; voyez mon âme, entendez ses cris du fond de l’abîme. Car, là même, si vous n’y êtes pas aux écoutes, où adresser nos pas et nos cris ?\par

\begin{quoteblock}
\noindent « À vous est le jour, à vous est la nuit (Ps. LXXIII, 16) »\end{quoteblock}

\noindent D’un coup d’oeil, vous réglez le vol des moments. Faites-moi largesse de temps pour méditer les secrets de votre loi ; ne la fermez pas à ceux qui frappent. Car ce n’est pas en vain que vous avez dicté tant de pages mystérieuses : forêts sacrées, n’ont-elles pas aussi leurs cerfs qui se retirent, s’abritent, courent, se reposent, paissent et ruminent sous leur ombre ? Seigneur, amenez-moi à votre perfection ; révélez-moi ces mystères. Oh ! votre parole est ma joie ; votre voix m’est plus douce que le charme des voluptés. Donnez-moi ce que j’aime ; votre voix est mon amour, et vous m’avez donné de l’aimer. Ne soyez pas infidèle à vos dons ; ne dédaignez pas votre pauvre plante que la soif dévore. Que je proclame à votre gloire toutes mes découvertes dans vos saints livres ! Que j’écoute la voix, de vos louanges (Ps. XXV, 7) ! Que je m’enivre de vous, en considérant les merveilles de votre loi, depuis ce jour premier-né des jours où vous avez fait le ciel et la terre, jusqu’à notre avènement au royaume de votre cité sainte.\par
\pn{4}Seigneur, ayez pitié de moi, exaucez mes vœux. Rien de la terre, je crois, n’est leur objet ; ni l’or, ni l’argent, ni les pierres précieuses, ni le luxe, ni les honneurs, ni la puissance, ni les plaisirs de la chair, ni les besoins qui nous suivent dans le trajet de la vie ; toutes choses d’ailleurs données par surcroît à qui cherche votre royaume et votre justice (Matth. VI, 33). Voyez, Seigneur mon Dieu, où s’élance mon désir.\par

\begin{quoteblock}
\noindent « Les impies m’ont raconté leur ivresse ; mais qu’est-ce auprès de votre loi , Seigneur (Ps. CXVIII) ? »\end{quoteblock}

\noindent Et voilà où mes vœux aspirent. Voyez, ô Père, regardez, voyez et agréez ; que sous l’oeil propice de votre miséricorde, je frappe à la porte de vos paroles saintes, et que la grâce m’ouvre leur sanctuaire. Je vous en conjure par Notre-Seigneur Jésus-Christ, votre Fils, l’homme de votre droite, fils de l’homme, que vous vous êtes fait (Ps. LXXIX, 18) médiateur entre vous et nous, par qui vous nous avez cherchés, quand nous n’étions plus en quête de vous, afin que cette sollicitude réveillât la nôtre. Je vous en conjure, au nom de votre Verbe, par qui vous avez fait toutes vos créatures, dont je suis ; au nom de votre Fils unique, par qui vous avez appelé à l’adoption le peuple des croyants, dont je suis encore ; au nom de Celui qui est assis à votre droite et y intercède pour nous ;\par

\begin{quoteblock}
\noindent « en qui sont cachés tous les trésors de « la sagesse et de la science (Coloss. II, 3) ; »\end{quoteblock}

\noindent c’est lui que je cherche dans vos livres saints. Moïse a écrit de lui (Jean V, 46) : C’est lui-même, c’est la Vérité, qui l’a dit.
\section[{Chapitre III, Il implore la vérité, qui a parlé par Moïse.}]{Chapitre III, Il implore la vérité, qui a parlé par Moïse.}
\noindent \pn{5}Oh ! que j’entende, que je comprenne comment, dans le PRINCIPE, vous avez créé le ciel et la terre (Gen,. I, 1) ! Moïse l’a écrit ; il l’a écrit et s’en est allé ; il a passé outre, allant de vous à vous ; et il n’est plus là devant moi. Que n’est-il encore ici-bas ! je m’attacherais à lui, et je le supplierais, et je le conjurerais en votre nom de me dévoiler ces mystères, et j’ouvrirais une oreille aride aux accents de ses lèvres. S’il me répondait dans la langue d’Héber, ce ne serait qu’un vain bruit qui frapperait mon organe, sans faire impression à mon esprit ; s’il me parlait dans la mienne, je l’entendrais ; mais d’où saurais-je qu’il me dirait la vérité ? et, quand je le saurais, le saurais-je de lui ? Non, ce serait au dedans de moi, dans la plus secrète résidence de ma pensée, que la vérité même, qui n’est ni hébraïque, ni grecque, ni latine, ni barbare, parlant sans organe, sans voix, sans murmure de syllabes, me dirait : Il dit vrai ; et aussitôt, dans une pleine certitude, je dirais à ce saint serviteur : Tu dis vrai. Mais je ne puis l’interroger ; c’est donc vous, ô Vérité ! dont il était plein ; c’est vous, mon Dieu, que j’implore ; oubliez mes offenses, et ce que vous avez donné d’écrire à votre grand Prophète, oh ! donnez-moi de l’entendre.  
\section[{Chapitre IV, Le ciel et la terre nous crient qu’ils ont été créés.}]{Chapitre IV, Le ciel et la terre nous crient qu’ils ont été créés.}
\noindent \pn{6}Et voilà donc le ciel et la terre ! Ils sont. Ils crient qu’ils ont été faits ; car ils varient et changent. Or ce qui est, sans avoir été créé, n’a rien en soi qui précédemment n’ait point été ; caractère propre du changement et de la vicissitude. Et ils ne se sont pas faits ; leur voix nous crie : C’est parce que nous avons été faits que nous sommes ; nous n’étions donc pas, avant d’être, pour nous faire nous-mêmes. L’évidence est leur voix. Vous les avez donc créés, Seigneur ; vous êtes beau, et ils sont beaux ; vous êtes bon, et ils sont bons ; vous êtes, et ils sont. Mais ils n’ont ni la beauté, ni la bonté, ni l’être de la même manière que vous, ô Créateur ; car, auprès de vous, ils n’ont ni beauté, ni bonté, ni être. Nous savons cela grâce à vous ; et notre science, comparée à la vôtre, n’est qu’ignorance.
\section[{Chapitre V, L’univers créé de rien.}]{Chapitre V, L’univers créé de rien.}
\noindent \pn{7}Comment donc avez-vous fait le ciel et la terre ? et quelle machine avez-vous appliquée à cette construction sublime ? L’artiste modèle un corps sur un autre, suivant la fantaisie de l’âme qui a la puissance de réaliser l’idéal que l’oeil intérieur lui découvre en elle. Et d’où lui viendrait ce pouvoir, si elle-même n’était votre ouvrage ?\par
L’artisan façonne une matière préexistante, ayant en soi de quoi devenir ce qu’il la fait, comme la terre, la pierre, le bois ou l’or, etc. Et d’où ces objets tiennent-ils leur être, si vous n’en êtes le créateur ? C’est vous qui avez créé le corps de l’ouvrier, et l’esprit qui commande à ses organes ; vous êtes l’auteur de cette matière qu’il travaille, de cette intelligence qui conçoit l’art, et voit en elle ce qu’elle veut produire au dehors ; de ces sens interprètes fidèles qui font passer dans l’ouvrage les conceptions de l’âme, et rapportent à l’âme ce qui s’est accompli, afin qu’elle consulte la vérité, juge intérieur, sur la valeur de l’ouvrage. Toutes ces créatures vous glorifient, et vous proclament le Créateur du monde.\par
Mais vous, comment les avez-vous faites ? comment avez-vous fait le ciel et la terre ? O Dieu ! Ce n’est ni sur la terre, ni dans le ciel, que vous avez fait le ciel et la terre ; ni dans les airs, ni dans les eaux qui en dépendent. Ce n’est pas dans l’univers que vous avez créé l’univers ; où pouvait-il être, pour être créé, avant d’être créé pour être ? Et vous n’aviez rien aux mains qui vous fût matière du ciel et de la terre. Eh ! d’où vous serait venue cette matière, que vous n’eussiez pas créée pour en former votre ouvrage ? Que dire, enfin, sinon que cela est, parce que vous êtes ? Et vous avez parlé, et cela fut, et votre seule parole a tout fait (Ps. XXXII, 9,6).
\section[{Chapitre VI, Comment Dieu a parlé.}]{Chapitre VI, Comment Dieu a parlé.}
\noindent \pn{8}Mais quelle a été cette parole ? S’est-elle formée comme cette voix descendue de la nue :\par

\begin{quoteblock}
\noindent « Celui-ci est mon Fils bien-aimé(Matth. III, 17 ; XVII, 5) ? »\end{quoteblock}

\noindent Cette voix retentit et passe ; elle commence et finit ; ses syllabes résonnent et s’évanouissent, la seconde après la première, la troisième après la seconde, ainsi de suite, jusqu’à la dernière, et le silence après elle. Il est donc évident et clair que cette voix fut l’expression d’une créature, organe temporel de votre éternelle volonté. Et l’oreille extérieure transmet ces paroles, formées dans le temps, à l’âme intelligente dont l’oreille intérieure s’approche de votre Verbe éternel. Et l’âme a comparé ces accents fugitifs à l’éternité silencieuse de votre Verbe, et elle s’est dit :\par

\begin{quoteblock}
\noindent « Quelle différence ! les uns sont infiniment au-dessous de moi ; ils ne sont même pas, car ils fuient, car ils passent ; mais au-dessus de moi, le Verbe de mon Dieu demeure éternellement (I Pierre, I, 25).»\end{quoteblock}

\noindent Que si vous avez commandé par des paroles passagères comme leur son l’existence du ciel et de la terre ; si c’est ainsi que vous les avez faits, il y avait donc déjà, avant le ciel et la terre, quelque créature corporelle, dont l’acte mesuré par le temps fit vibrer cette voix dans la mesure du temps. Or, nulle substance corporelle n’était avant le ciel et la terre ; ou, s’il en existait une, il faut reconnaître que vous aviez formé sans parole successive l’être qui devait articuler votre commandement : Que le ciel et la terre soient. Car cet organe de vos desseins, quel qu’il fût, ne pouvait être, si vous ne l’eussiez fait. Or, pour produire le corps dont ces paroles devaient sortir, de quelle parole vous êtes-vous servi ?
 \section[{Chapitre VII, Le verbe divin, fils de Dieu, coéternel au Père.}]{Chapitre VII, Le verbe divin, fils de Dieu, coéternel au Père.}
\noindent \pn{9}Vous nous appelez donc plus haut ; vous nous appelez à l’intelligence du Verbe-Dieu, Dieu en vous, Verbe qui se prononce et prononce tout de toute éternité ; parole sans fin, sans succession, sans écoulement ; qui dit éternellement, et tout à la fois, toutes choses. Autrement le temps et la vicissitude seraient en vous, et, dès lors, plus de véritable éternité, plus de véritable immortalité. C’est ainsi, je le sais, mon Dieu, et grâces à vous ! Je le sais, et vous bénis, Seigneur, et, avec moi, quiconque n’a pas un cœur ingrat au bienfait éclatant de votre lumière.\par
Nous savons, Seigneur, nous savons que, n’être plus ce qu’on était, qu’être ce qu’on n’était pas, c’est là naître et mourir. Aussi, rien en votre Verbe ne passe, rien ne succède, parce qu’il est immortel, parce qu’il est éternel en vérité. Et c’est par ce Verbe, coéternel avec vous, que vous dites, de toute éternité, et tout à la fois, toute ce que vous dites, et qu’il est ainsi que vous dites. Et votre parole est votre seule action ; et néanmoins ce n’est ni tout à la fois, ni de toute éternité, que s’est accomplie l’œuvre de votre parole.
\section[{Chapitre VIII, Le verbe éternel est notre unique maître.}]{Chapitre VIII, Le verbe éternel est notre unique maître.}
\noindent \pn{10}Eh ! comment cela, Seigneur mon Dieu ? J’entrevois bien quelque chose, mais comment l’exprimer ? je l’ignore. N’est-ce point que tout être qui commence et finit, ne commence et ne finit d’être qu’au temps où la raison, en qui rien ne finit, rien ne commence, la raison éternelle connaît qu’il doit commencer ou finir ? Et, cette raison, c’est votre Verbe, le principe de tout, la voix intérieure qui nous parle (Jean VIII, 25) ; comme lui-même l’a dit dans 1’Evangile par la voix de la chair ; comme il l’a fait entendre humainement à l’oreille des hommes, afin que l’on crût en lui, qu’on le cherchât intérieurement, et qu’on le trouvât dans l’éternelle vérité, où ce bon, cet unique maître des âmes enseigne tous ses disciples.\par
C’est là, Seigneur, que j’entends votre voix me dire : Que la vraie parole est celle qui nous enseigne ; et que la parole qui n’enseigne pas, n’est plus une parole. Or, qui nous enseigne, sinon l’immuable vérité ? car la créature changeante ne nous instruit qu’en tant qu’elle nous amène à cette vérité stable, notre lumière, notre appui, notre joie ; la voix de l’Epoux (Jean III, 29), qui nous réunit à notre principe. Et il est ce principe, et sans son immuable permanence nous ne saurions où revenir de nos égarements. Or, quand nous revenons de l’erreur, c’est la connaissance qui nous ramène ; et il nous enseigne cette connaissance, parce qu’il est le principe et la voix qui nous parle.
\section[{Chapitre IX, Le verbe parle à notre cœur.}]{Chapitre IX, Le verbe parle à notre cœur.}
\noindent \pn{11}C’est dans ce Principe, ô Dieu, que vous avez fait le ciel et la terre ; c’est dans votre Verbe, votre Fils, votre vertu, votre sagesse, votre vérité ; par une parole, par une opération admirable. Qui pourra comprendre cette merveille ? qui pourra la raconter ? Quelle est cette lumière qui par intervalle m’éclaire, et frappe mon cœur sans le blesser ; le glace d’épouvante, et l’embrase d’amour : épouvante, en tant que je suis si loin ; amour, en tant que je suis plus près d’elle ?\par
C’est la sagesse, la sagesse elle-même, dont le rayon déchire par intervalle les nuages de mon âme, qui, souvent infidèle à cette lumière, retombe dans ses ténèbres, sous le fardeau de son supplice : car ma détresse a épuisé mes forces (Ps. XXX, 2) ; je suis incapable même de porter mon bonheur, tant que votre pitié, Seigneur, secourable à mes iniquités, n’aura pas\par

\begin{quoteblock}
\noindent «guéri toutes mes langueurs. Mais vous rachèterez ma vie de la corruption ; vous me couronnerez de compassion et de miséricorde ; vous rassasierez de vos biens tout mon désir ; et ma jeunesse sera renouvelée comme celle de l’aigle (Ps. CII, 3-5) ; »\end{quoteblock}

\noindent car l’espérance est notre salut ; et nous attendons vos promesses en patience (Rom. VIII, 24, 25). Entende en soi qui pourra votre parole intérieure, moi je m’écrie, sur la foi de votre oracle :\par

\begin{quoteblock}
\noindent « Que vos œuvres sont glorieuses, Seigneur ! Vous avez tout fait dans votre Sagesse (Ps. CIII, 24). »\end{quoteblock}

\noindent Elle est le principe ; et c’est dans ce principe que vous avez créé le ciel et la terre.  
\section[{Chapitre X, La volonté de Dieu n’a pas de commencement.}]{Chapitre X, La volonté de Dieu n’a pas de commencement.}
\noindent \pn{12}Ne sont-ils pas tous remplis des ruines de leur vétusté, ceux qui nous disent : Que faisait Dieu avant de créer le ciel et la terre ? S’il demeurait dans l’inaction, pourquoi eu est-il sorti, pourquoi y est-il rentré ? S’il s’est accompli en Dieu un acte nouveau, une volonté nouvelle, pour donner l’être à une créature qui n’était pas sortie du néant, est-il une éternité vraie là où naît une volonté qui n’était pas ? car la volonté de Dieu n’est pas la créature. Elle est antérieure à la créature. Nulle création sans préexistence de la volonté créatrice. La volonté de Dieu appartient donc à sa substance. Que s’il est survenu dans la substance divine quelque chose de nouveau, on ne peut plus en vérité la dire éternelle. Et si Dieu a voulu de toute éternité l’existence de la créature, pourquoi, elle aussi, n’est-elle pas éternelle ?
\section[{Chapitre XI, Le temps ne saurait être la mesure de l’éternité.}]{Chapitre XI, Le temps ne saurait être la mesure de l’éternité.}
\noindent \pn{13}Ceux qui parlent ainsi ne vous comprennent pas encore , ô Sagesse de Dieu lumière des esprits ; ils ne comprennent pas comment vous créez, en vous, et par vous-même, et ils aspirent à la science de votre éternité ; mais leur cœur flotte sur les vagues du passé et de l’avenir, à la merci de la vanité.\par
Qui l’arrêtera, ce cœur, qui le fixera pour qu’il s’ouvre stable un instant, à l’intuition des splendeurs de l’immobile éternité, qu’il la compare à la mobilité des temps, et trouve toute comparaison impossible ; qu’il ne voie dans la durée qu’une succession de mouvements qui ne peuvent se développer à la fois ; observant, au contraire, que rien de l’éternité ne passe, et qu’elle demeure toute présente, tandis qu’il n’est point de temps qui soit tout entier présent ; car l’avenir suit le passé qu’il chasse devant lui ; et tout passé, tout avenir tient son être et son cours de l’éternité toujours présente ? Qui fixera le cœur de l’homme, afin qu’il demeure et considère comment ce qui demeure, comment l’éternité, jamais passée, jamais future, dispose et du passé et de l’avenir ? Est-ce ma main, est-ce ma parole, la main de mon esprit, qui aurait cette puissance ?
\section[{Chapitre XII, Ce que Dieu faisait avant la création du monde.}]{Chapitre XII, Ce que Dieu faisait avant la création du monde.}
\noindent \pn{14}Et je réponds à cette demande : Que faisait Dieu avant de créer le ciel et la terre ? Je réponds, non comme celui qui éluda, dit on, les assauts d’une telle question par cette plaisanterie : Dieu préparait des supplices aux sondeurs de mystères. Rire n’est pas répondre. Et je ne réponds pas ainsi. Et j’aimerais mieux confesser mon ignorance, que d’appeler la raillerie sur une demande profonde, et l’éloge sur une réponse ridicule.\par
Mais je dis, ô mon Dieu, que vous êtes le père de toute créature, et s’il faut entendre toute créature par ces noms du ciel et de la terre, je le déclare hautement : avant de créer le ciel et la terre, Dieu ne faisait rien. Car ce qu’il eût pu faire alors, ne saurait être que créature. Oh ! que n’ai-je la connaissance de tout ce qu’il m’importe de connaître, comme je sais que la créature n’était pas avant la création !
\section[{Chapitre XIII, Point de temps avant la création.}]{Chapitre XIII, Point de temps avant la création.}
\noindent \pn{15}Un esprit léger s’élance déjà peut-être dans un passé de siècles imaginaires, et s’étonne que le Tout-Puissant, créateur et conservateur du monde, l’architecte du ciel et de la terre, ait laissé couler un océan d’âges infinis sans entreprendre ce grand ouvrage. Qu’il sorte de son sommeil, et considère l’inanité de son étonnement ! Car d’où serait venu ce cours de siècles sans nombre dont vous n’eussiez pas été l’auteur, vous, l’auteur et le fondateur des siècles ? Quel temps eût pu être, sans votre institution ? Et comment se fût-il écoulé, ce temps qui n’eût pu être ?\par
Puisque vous êtes l’artisan de tous les temps, si l’on suppose quelque temps avant que vous eussiez créé le ciel et la terre, pourquoi donc prétendre que vous demeuriez dans l’inaction ? Car ce temps même était votre ouvrage, et nul temps n’a pu courir avant que vous eussiez fait le temps. Que si avant le ciel et la terre il n’était point de temps, pourquoi demander ce que vous faisiez ALORS ? Car, où le TEMPS n’était pas, ALORS ne pouvait être.\par
\pn{16}Et ce n’est point par le temps que vous précédez les temps, autrement vous ne seriez   pas avant tous les temps. Mais vous précédez les temps passés par l’éminence de votre éternité toujours présente ; vous dominez les temps à venir, parce qu’ils sont à venir, et qu’aussitôt venus, ils seront passés.\par

\begin{quoteblock}
\noindent « Et vous, vous « êtes toujours le même, et vos années ne s’évanouissent point (Ps. CI, 28). »\end{quoteblock}

\noindent Vos années ne vont ni ne viennent, et les nôtres vont et viennent afin d’arriver toutes. Vos années demeurent toutes à la fois, parce qu’elles demeurent. Elles ne se chassent pas pour se succéder, parce qu’elles ne passent pas. Et les nôtres ne seront toutes, que lorsque toutes auront cessé d’être. Vos années ne sont qu’un jour ; et ce jour est sans semaine, il est aujourd’hui ; et votre aujourd’hui ne cède pas au lendemain, il ne succède pas à la veille. Votre aujourd’hui, c’est l’éternité. Ainsi vous avez engendré coéternel à vous-même Celui à qui vous avez dit :\par

\begin{quoteblock}
\noindent « Je t’ai engendré aujourd’hui (Ps. II,7 ; Héb. V, 7). »\end{quoteblock}

\noindent Vous avez fait tous les temps, et vous êtes avant tous les temps, et il ne fut pas de temps où le temps n’était pas.
\section[{Chapitre XIV, Qu’est-ce que le temps ?}]{Chapitre XIV, Qu’est-ce que le temps ?}
\noindent \pn{17}Il n’y a donc pas eu de temps où vous n’ayez rien fait, puisque vous aviez déjà fait le temps. Et nul temps ne vous est coéternel, car vous demeurez ; et si le temps demeurait, il cesserait d’être temps. Qu’est-ce donc que le temps ? Qui pourra le dire clairement et en peu de mots ? Qui pourra le saisir même par la pensée, pour traduire cette conception en paroles ? Quoi de plus connu, quoi de plus familièrement présent à nos entretiens, que le temps ? Et quand nous en parlons, nous concevons ce que nous disons ; et nous concevons ce qu’on nous dit quand on nous en parle.\par
Qu’est-ce donc que le temps ? Si personne ne m’interroge, je le sais ; si je veux répondre à cette demande, je l’ignore. Et pourtant j’affirme hardiment, que si rien ne passait, il n’y aurait point de temps passé ; que si rien n’advenait, il n’y aurait point de temps à venir, et que si rien n’était, il n’y aurait point de temps présent. Or, ces deux temps, le passé et l’avenir, comment sont-ils, puisque le passé n’est plus, et que l’avenir n’est pas encore ? Pour le présent, s’il était toujours présent sans voler au passé, il ne serait plus temps ; il serait l’éternité. Si donc le présent, pour être temps, doit s’en aller en passé, comment pouvons-nous dire qu’une chose soit, qui ne peut être qu’à la condition de n’être plus ? Et peut-on dire, en vérité, que le temps soit, sinon parce qu’il tend à n’être pas ?
\section[{Chapitre XV, Quelle est la mesure du temps ?}]{Chapitre XV, Quelle est la mesure du temps ?}
\noindent \pn{18}Et cependant nous disons qu’un temps est long et qu’un temps est court, et nous ne le disons que du passé et de l’avenir ; ainsi, par exemple, cent ans passés, cent ans à venir, voilà ce que nous appelons longtemps ; et, peu de temps : dix jours écoulés, dix jours à attendre. Mais comment peut être long ou court ce qui n’est pas ? car le passé n’est plus, et l’avenir n’est pas encore. Cessons donc de dire : Ce temps est long ; disons du passé : il a été long ; et : il sera long, de l’avenir.\par
Seigneur mon Dieu, ma lumière, votre vérité ne se moquera-t-elle pas de l’homme qui parle ainsi ? Car ce long passé, est-ce quand il était déjà passé qu’il a été long, ou quand il était encore présent ? En effet, il n’a pu être long que tant qu’il fut quelque chose qui pût être long. Mais, passé, il n’était déjà plus ; et comment pouvait-il être long, lui qui n’avait plus d’être ? Ne disons plus donc : Le passé a été long : car nous ne retrouverons pas ce qui a été long, puisque du moment où il passe, il n’est plus. Disons : Ce temps présent a été long, car il était long en tant que présent. Il ne s’était pas encore écoulé au non-être, il était donc quelque chose qui pouvait être long. Mais aussitôt qu’il a passé, aussitôt il a cessé d’être long, en cessant d’être.\par
\pn{19}Voyons donc, ô âme de l’homme, si le temps présent peut être long ; car tu as reçu la faculté de concevoir et de mesurer ses pauses.\par
Que vas-tu me répondre ? Est-ce un long temps que cent années présentes ? Vois d’abord si cent années peuvent être présentes. Est-ce la première qui s’accomplit ? elle seule est présente ; les quatre-vingt-dix—neuf autres sont à venir ; et, partant, ne sont pas encore. Est-ce la seconde ? il en est une déjà passée ; une présente ; le reste est futur. Ainsi de toute année que nous fixerons comme présente dans la révolution d’un siècle ; tout ce qui la devance est passé ; tout ce qui la suit est futur. Cent années ne sauraient donc être présentes.   Mais vois si du moins l’année actuelle est elle-même présente. Est-ce son premier mois qui court ? les autres sont à venir. Est-ce le second ? le premier est déjà passé ; le reste n’est pas encore ; ainsi l’année actuelle n’est pas tout entière présente : et, partant, ce n’est pas une année présente ; car l’année, c’est douze mois, dont chacun à Son tour est présent ; le reste, passé ou futur. Et le mois courant, même, n’est pas présent, mais un seul de ses jours. Est-il le premier ? le reste est dans l’avenir. Est-il le dernier ? le reste est dans le passé. Est-il intermédiaire ? il est entre ce qui n’est plus et ce qui n’est pas encore.\par
\pn{20}Voilà donc ce temps présent que nous avons trouvé le seul qu’on pût appeler long ; le voilà réduit à peine à l’espace d’un jour. Et ce jour même, encore, discutons-le ; non, ce seul jour n’est pas tout entier présent : car il s’accomplit en vingt-quatre heures, douze de jour, douze de nuit, dont la première précède, et la dernière suit toutes les autres, l’intermédiaire suit et précède.\par
Et cette même heure se compose elle-même de parcelles fugitives. Tout ce qui s’en détache, s’envole dans le passé ; ce qui en reste est avenir. Que si l’on conçoit un point dans le temps sans division possible de moment, c’est ce point-là seul qu’on peut nommer présent. Et ce point vole, rapide, de l’avenir au passé, durée sans étendue ; car s’il est étendu, il se divise en passé et avenir.\par
Ainsi, le présent est sans étendue. Où donc est le temps que nous puissions appeler long ? Est-ce l’avenir ! Non : car il ne peut être long sans être. Nous disons donc : Il sera long. Mais quand le sera-t-il ? Non sans doute tant qu’il sera avenir, n’étant pas encore, pour être long. Que s’il ne doit être long qu’au moment où, de futur, il commencera d’être ce qu’il n’est pas encore, c’est-à-dire présent, ayant un être, et de quoi être long, n’oublions pas que le présent nous a crié à haute voix : Non, je ne saurais être long.
\section[{Chapitre XVI, Comment se mesure le temps ?}]{Chapitre XVI, Comment se mesure le temps ?}
\noindent \pn{21}Et pourtant, Seigneur, nous apercevons bien les intervalles des temps, nous les comparons entre eux, et nous disons les uns plus longs, les autres plus courts ; nous mesurons encore la différence ; nous constatons qu’elle est double, triple, etc., ou nous affirmons l’égalité. Mais notre aperception qui mesure les temps ne mesure que leur passage : car le passé, qui n’est plus, l’avenir, qui n’est pas encore, peuvent-ils se mesurer, à moins que l’on ne prétende que le néant soit mesurable ? Ce n’est donc que dans sa fuite que le temps s’aperçoit et se mesure. Est-il passé ? il n’est point mesurable, car il n’est plus,
\section[{Chapitre XVII,Ou est le passé, ou est l’avenir ?}]{Chapitre XVII,Ou est le passé, ou est l’avenir ?}
\noindent \pn{22}Je cherche, ô Père, je n’affirme rien ; mon Dieu, soyez l’arbitre et le guide de mes efforts. Qui oserait me dire qu’il n’existe pas trois temps, comme notre enfance l’a appris, comme nous l’enseignons à l’enfance : le passé, le présent et l’avenir, mais que le présent seul existe, les deux autres n’étant point ? Ou bien faut-il dire qu’ils sont ; et que le temps sort d’une retraite inconnue, quand, de futur, il devient présent, et qu’il rentre dans une autre, également inconnue, quand, de présent, il devient passé ? Car si l’avenir n’est pas encore, où donc l’ont vu ceux qui l’ont prédit ? Ce qui n’est pas peut-il se voir ? Et les narrateurs du passé seraient-ils vrais, si ce passé n’était -visible à leur esprit ? Et pourraient-ils se voir, l’un et l’autre, s’ils n’étaient que pur néant ? Il faut donc que le passé et l’avenir aient un être.
\section[{Chapitre XVIII, Comment le passé et l’avenir sont présents.}]{Chapitre XVIII, Comment le passé et l’avenir sont présents.}
\noindent \pn{23}Permettez-moi, Seigneur, de chercher encore. Ô mon espérance, éloignez le trouble de mes efforts. S’il est vrai que l’avenir et le passé soient, où sont-ils ? Si cette connaissance est encore au-dessus de moi, je sais pourtant que, où qu’ils soient, ils n’y sont ni passé, ni futur, mais présent : le futur, comme tel, n’y est pas encore ; le passé, comme tel, n’y est déjà plus. Où donc qu’ils soient, quels qu’ils soient, ils ne sont qu’en tant que présent. Ainsi dans un récit véritable d’événements passés, la mémoire ne reproduit pas les réalités qui ne sont plus, mais les mots nés des images qu’elles ont laissées en passant par nos sens, comme les traces de leurs pas. Mon enfance évanouie est dans le passé, évanoui comme elle. Mais quand j’y pense, quand j’en parle, je revois son   image dans le temps présent, parce qu’elle est encore dans ma mémoire.\par
Est-ce ainsi que se prédit l’avenir ? Est-ce en présence d’images, messagères de ce qui n’es pas encore ? Mon Dieu, je confesse ici mon ignorance. Mais ce dont je suis certain, c’est que d’ordinaire nous préméditons nos actes futurs ; que cette préméditation est présente, tandis que l’acte prémédité, en tant que futur, n’est pas encore. Notre préméditation commençant à se réaliser, l’acte sera, non plus à venir mais présent.\par
\pn{24}Quel que soit donc ce secret pressentiment de l’avenir, on ne saurait voir que ce qui est. Or, ce qui est déjà, n’est point à venir, mais présent. Ainsi voit l’avenir, ce n’est pas voir ces réalités futures qui ne sont pas encore, mais peut-être les causes et les symptômes qui existent déjà ; prémices de l’avenir déjà présentes aux regards de la pensée qui, le conçoit ; et cette conception est déjà dans l’esprit, et elle est présente à la vision prophétique.\par
Une preuve éloquente entre tant de témoignages. Je vois l’aurore et je prédis le lever du soleil. Ce que je vois est présent, ce que je prédis est futur ; non pas le soleil qui est déjà, mais son lever qui n’est pas encore : et si mon esprit ne se l’imaginait, comme au moment où j’en parle, cette prédiction serait impossible. Or, cette aurore, que je vois dans le ciel, n’est pas le lever du soleil, quoiqu’elle le devance, non plus que cette image que je vois dans mon esprit, mais leur présence coïncidente me fait augurer le phénomène futur. Ainsi, l’avenir n’est pas encore ; donc il n’est pas, donc il ne peut se voir ; mais il se peut prédire d’après des circonstances déjà présentes et visibles.
\section[{Chapitre XIX,De la prescience de l’avenir.}]{Chapitre XIX,De la prescience de l’avenir.}
\noindent \pn{25}Mais dites, Monarque souverain de votre création, comment enseignez-vous aux âmes les événements futurs ? Ne les avez-vous pas révélés à vos prophètes ? Dites, comment enseignez-vous l’avenir, vous pour qui rien n’est avenir ; ou plutôt comment enseignez-vous ce qui de l’avenir est déjà présent ? Car le néant pourrait-il s’enseigner ? C’est un secret, je le sens, supérieur à mon intelligence ; faible par elle-même , ma vue n’y saurait atteindre (Ps. CXXVIII, 6) ; mais vous serez sa force, si vous voulez, ô douce lumière des yeux de mon âme !
\section[{Chapitre XX, Quel nom donner aux différences du temps ?}]{Chapitre XX, Quel nom donner aux différences du temps ?}
\noindent \pn{26}Or, ce qui devient évident et clair, c’est que le futur et le passé ne sont point ; et, rigoureusement, on ne saurait admettre ces trois temps : passé, présent et futur ; mais peut-être dira-t-on avec vérité : Il y a trois temps, le présent du passé, le présent du présent et le présent de l’avenir. Car ce triple mode de présence existe dans l’esprit ; je ne le vois pas ailleurs. Le présent du passé, c’est la mémoire ; le présent du présent, c’est l’attention actuelle ; le présent de l’avenir, c’est son attente. Si l’on m’accorde de l’entendre ainsi, je vois et je confesse trois temps ; et que l’on dise encore, par un abus de l’usage : Il y a trois temps, le passé, le présent et l’avenir ; qu’on le dise, peu m’importe ; je ne m’y oppose pas : j’y consens, pourvu qu’on entende ce qu’on dit, et que l’on ne pense point que l’avenir soit déjà, que le passé soit encore. Nous avons bien peu de locutions justes, beaucoup d’inexactes ; mais on ne laisse pas d’en comprendre l’intention.
\section[{Chapitre XXI, Comment mesurer le temps ?}]{Chapitre XXI, Comment mesurer le temps ?}
\noindent \pn{27}Nous mesurons le temps à son passage, ai-je dit plus haut ; en sorte que nous pouvons affirmer qu’un temps est double d’un autre, ou égal à un autre, ou tel autre rapport que cette mesure exprime. Ainsi donc c’est à son passage que nous mesurons le temps. D’où le sais-tu ? dira-t-on peut-être. Je sais, répondrai-je, que nous le mesurons ; que nous ne saurions mesurer ce qui n’est pas, et que le passé ou l’avenir n’est qu’un néant. Or, comment mesurons-nous le temps présent, puisqu’il est sans étendue ? Il ne se mesure qu’à son passage ; passé, il ne se mesure plus ; car il n’est plus rien de mesurable.\par
Mais d’où vient, par où passe, où va le temps, quand on le mesure ? D’où, sinon de l’avenir ? Par où, sinon par le présent ? Où, sinon dans le passé ? Sorti de ce qui n’est pas encore, il passe par l’inétendue pour arriver à ce qui n’est plus. Comment donc mesurer le temps, si ce n’est pas certains espaces ? Ces distinctions des temps simples, doubles, triples ou égaux,   qu’est-ce autre chose que des espaces de temps Quel espace est donc pour nous la mesure du temps qui passe ? Est-ce l’avenir d’où il vient Mais mesure-t-on ce qui n’est pas encore ? Est-ce le présent par où il passe ? Mais l’inétendu se mesure-t-il ? Est-ce le passé où il entre ? Mais comment mesurer ce qui n’est plus ?
\section[{Chapitre XXII, Il demande à Dieu la connaissance de ce mystère.}]{Chapitre XXII, Il demande à Dieu la connaissance de ce mystère.}
\noindent \pn{28}Mon esprit brûle de connaître cette énigme profonde. Je vous en conjure, Seigneur mon Dieu, mon bon père, je vous en conjure au nom du Christ, ne fermez pas à mon désir l’accès d’une question si ordinaire et si mystérieuse. Laissez-moi pénétrer dans ses replis ; que la lumière de votre miséricorde les éclaire, Seigneur ! À qui m’adresser ? à quel autre confesser plus utilement mon ignorance qu’à vous, ô Dieu, qui ne désapprouvez pas le zèle ardent où m’emporte l’étude de vos Écritures ? Donnez- moi ce que j’aime. Car j’aime, et vous m’avez donné d’aimer. Donnez-moi mon amour, ô Père qui savez ne donner que de vrais biens à vos fils (Matth. VII, 2) Donnez-moi de connaître cette, vérité que je poursuis. C’est une porte fermée à tous mes labeurs, si vous ne l’ouvrez vous-même.\par
Par le Christ, au nom du Saint des saints, je vous en conjure, que nul ne me trouble ici. Je crois,\par

\begin{quoteblock}
\noindent « et ma foi inspire ma parole (Ps. CXV, 1). »\end{quoteblock}

\noindent J’espère et je ne vis qu’à l’espérance de contempler les délices du Seigneur. Et vous avez fait mes jours périssables, et ils passent (Ps. XXXVIII,6). Et comment ? je l’ignore. Et nous avons sans cesse à la bouche ces mots : époque et temps. Combien de temps a-t-il mis à ce discours, à cette œuvre ? Qu’il y a longtemps que je n’ai vu cela ! Et, cette syllabe longue est le double de temps de cette brève. Nous parlons et on nous parle tous les jours ainsi ; nous comprenons et sommes compris. Rien de plus clair et de plus usité ; rien en même temps de plus caché ; rien, jusqu’ici, de plus impénétrable.
\section[{Chapitre XXIII, Nature du temps.}]{Chapitre XXIII, Nature du temps.}
\noindent \pn{29}J’ai entendu dire à un savant que le temps, c’est le mouvement du soleil, de la lune et des astres ; je ne suis pas de cet avis ; car, pourquoi le mouvement de tout autre corps ne serait-il pas le temps ? Quoi ! le cours des astres demeurant suspendu, si la roue d’un potier continuait à tourner, n’y aurait-il plus de temps pour mesurer ses tours ? Ne nous serait-il plus possible d’exprimer l’égalité de leurs intervalles ou la différence de leurs mouvements, si les vitesses sont différentes ? Et en énonçant ces rapports, ne serait-ce pas dans le temps que nous parlerions ? N’y aurait-il dans nos paroles ni longues, ni brèves ? Et comment les reconnaître, sinon à l’inégale durée de leur son ? O Dieu ! accordez à l’homme de trouver en un point la lumière- qui lui découvre toute grandeur et toute petitesse ! Il est, je le sais, des astres et dès flambeaux célestes qui mesurent les saisons, les temps, les années et les jours (Gen. I, 14). C’est une vérité, et je ne prétendrais jamais que le mouvement de cette roue du potier fût notre jour, sans lui refuser toutefois d’être un temps, n’en déplaise à ce philosophe.\par
\pn{30}Ce que je veux savoir, moi, c’est la puissance et la nature du temps, qui nous sert de mesure aux mouvements des corps, et nous permet de dire, par exemple : Tel mouvement dure une fois plus que tel autre ; car enfin le jour n’est pas seulement la présence rapide du soleil sur l’horizon, mais encore le cercle qu’il décrit de l’orient à l’orient, et qui règle le nombre des jours écoulés, les nuits mêmes comprises, dont le compte n’est jamais séparé. Ainsi le jour n’étant accompli que par le mouvement du soleil et sa révolution d’orient en orient, est-ce le mouvement, est-ce la durée du mouvement, est-ce l’un et l’autre ensemble qui forment le jour ? Est-ce le mouvement ? Alors, une heure serait le jour, si cet espace de temps suffisait au soleil pour achever sa carrière :\par
Est-ce le jour entier ? Alors il n’y aurait point de jour si, d’un lever à l’autre, il ne s’écoulait pas plus d’une heure, et s’il fallait vingt-quatre révolutions solaires pour former le jour. Est-ce à la fois le mouvement et le temps ? Alors le soleil accomplirait son tour en une heure, et, supposé qu’il s’arrêtât, le même intervalle que sa course mesure d’un matin à l’autre se serait écoulé, qu’il n’y aurait pas eu de véritable jour.\par
Ainsi, je ne me demande plus, qu’est-ce qu’on nomme le jour, mais qu’est-ce que le temps ? ce temps, mesure du mouvement solaire, que nous dirions moindre de moitié, si   douze heures avaient suffi au parcours de l’espace accoutumé. En comparant cette différence de temps, ne dirions-nous pas que l’un est double de l’autre, alors même que la course du soleil d’orient en orient serait tantôt plus longue, tantôt plus courte de moitié ? Qu’on ne vienne donc plus me dire : Le temps, c’est le mouvement des corps célestes. Quand le soleil s’arrêta à la prière d’un homme (Josué, X, 13), pour lui laisser le loisir d’achever sa victoire, le temps s’arrêta-t-il avec le soleil ? Et n’est-ce point dans l’espace de temps nécessaire que le combat se continua et finit ? Je vois donc enfin que le temps est une sorte d’étendue. Mais n’est-ce pas une illusion ? suis-je bien certain de. le voir ? Ô vérité, ô lumière ! éclairez-moi.
\section[{Chapitre XXIV, Le temps est-il la mesure du mouvement ?}]{Chapitre XXIV, Le temps est-il la mesure du mouvement ?}
\noindent \pn{31}Si l’on me dit : Le temps, c’est le mouvement des corps, m’ordonnez-vous de le croire ? Non, vous ne l’ordonnez pas. Nul corps ne saurait se mouvoir que dans le temps. Vous le dites, et je l’entends ; mais que ce mouvement soit le temps, c’est ce que je n’entends pas ; ce n’est pas vous qui le dites. Lorsqu’en effet un corps se meut, c’est par le temps que je mesure la durée de ce mouvement, depuis son origine jusqu’à sa fin. Si je ne l’ai pas vu commencer, et si sa durée ne me permet pas de le voir finir, il n’est point en ma puissance de le mesurer, si ce n’est peut-être du moment où j’ai commencé à celui où j’ai cessé de le voir. Si je l’ai vu longtemps, j’affirme la longueur du temps sans la déterminer ; car cette détermination suppose un rapport de différence ou d’égalité. Si, supposé un mouvement circulaire, nous pouvions remarquer le point de l’espace où prend sa course et où la termine le corps mobile, ou l’une de ses parties, nous pourrions dire en combien de temps s’est accompli de tel point à tel autre, le mouvement de ce corps ou de l’une de ses parties.\par
Ainsi le mouvement d’un corps étant distinct de la mesure de sa durée, peut-on chercher encore à qui appartient le nom de temps ? Souvent ce corps se meut d’un mouvement inégal, souvent il demeure en repos, et le temps n’est pas moins la mesure de son repos que de son mouvement. Et nous disons : Son immobilité a duré autant, deux ou trois fois plus, deux ou trois fois moins que son mouvement ; et, nous le disons, d’après une mesure exacte ou approximative. Donc le mouvement des corps n’est pas le temps.
\section[{Chapitre XXV,Allumez ma lampe, Seigneur, éclairez mes ténèbres.}]{Chapitre XXV,Allumez ma lampe, Seigneur, éclairez mes ténèbres.}
\noindent \pn{32}Et je vous le confesse, Seigneur, j’ignore encore ce que c’est que le temps ; et pourtant, Seigneur, je vous le confesse aussi, je n’ignore point que c’est dans le temps que je parle, et qu’il y a déjà longtemps que je parle du temps, et que ce longtemps est une certaine teneur de durée. Eh ! comment donc puis-je le savoir, ignorant ce que c’est que le temps ? Ne serait-ce point que je ne sais peut-être comment exprimer ce que je sais ? Malheureux que je suis, j’ignore même ce que j’ignore ! Mais vous êtes témoin, Seigneur, que le mensonge est loin de moi. Mon cœur est comme ma parole.\par

\begin{quoteblock}
\noindent « Allumez ma lampe, Seigneur mon Dieu, éclairez mes ténèbres (Ps. XVII, 25). »\end{quoteblock}

\section[{Chapitre XXVI, Le temps n’est pas la mesure du temps.}]{Chapitre XXVI, Le temps n’est pas la mesure du temps.}
\noindent \pn{33}Mon âme ne vous fait-elle pas un aveu sincère quand elle déclare en votre présence qu’elle mesure le temps ? Est-il donc vrai, mon Dieu, que je le mesure, sans connaître ce que je mesure ? Je mesure le mouvement des corps par le temps, et le temps lui-même, ne saurais-je le mesurer ? Et me serait-il possible de mesurer la durée et l’étendue d’un mouvement, sans mesurer le temps où il s’accomplit ?\par
Mais sur quelle mesure puis-je apprécier le temps même ? Un temps plus long est-il la mesure d’un plus court, comme la coudée est la mesure d’une solive ? comme une syllabe longue nous paraît être la mesure d’une brève, quand nous disons que l’une est double de l’autre ; comme la longueur d’un poème s’évalue sur la longueur des vers, la longueur des vers sur celle des pieds, la longueur des pieds sur celle des syllabes, et les syllabes longues sur les brèves : évaluation qui ne repose pas sur l’étendue des pages, car elle serait alors mesure d’espace ; et non plus mesure de temps. Mais lorsque les paroles passent, en les prononçant, nous disons : Ce poème est long, il se   compose de tant de vers ; ces vers sont longs, ils se tiennent sur tant de pieds ; ces pieds sont longs, ils renferment tant de syllabes ; cette syllabe est longue, car elle est double d’une brève.\par
Toutefois, ce n’est pas encore là une mesure certaine du temps ; car un vers plus court prononcé lentement peut avoir plus de durée qu’un long débité plus vite ; ainsi d’un poème, d’un pied, d’une syllabe. D’où j’infère que le temps n’est qu’une étendue. Mais quelle est la substance de cette étendue ? Je l’ignore. Et ne serait-ce pas mon esprit même ? Car, ô mon Dieu ! qu’est-ce que je mesure, quand je dis indéfiniment : tel temps est plus long que tel autre ; ou définiment ce temps est double de celui-là ? C’est bien le temps que je mesure, j’en suis certain ; mais ce n’est point l’avenir, qui n’est pas encore ; ce n’est point le présent, qui est inétendu ; ce n’est point le passé, qui n’est plus. Qu’est-ce donc que je mesure ? Je l’ai dit ; ce n’est point le temps passé, c’est le passage du temps.
\section[{Chapitre XXVII, Comment nous mesurons le temps.}]{Chapitre XXVII, Comment nous mesurons le temps.}
\noindent \pn{34}Courage, mon esprit ; redouble d’attention et d’efforts ! Dieu est notre aide :\par

\begin{quoteblock}
\noindent « nous sommes son ouvrage et non pas le notre (Ps. XCIX, 3) »\end{quoteblock}

\noindent attention où l’aube de la vérité commence à poindre. Une voix corporelle se fait entendre ; le son continue ; et puis il cesse. Et voilà le silence ; et la voix est passée ; et il n’y a plus rien : avant le son elle était à venir, et ne pouvait se mesurer, n’étant pas encore ; elle ne le peut plus, n’étant plus. Elle le pouvait donc, quand elle vibrait, puisqu’elle était ; sans stabilité, toutefois ; car elle venait et passait. Et n’est-ce point cette instabilité même qui la rendait mesurable ? Son passage ne lui donnait-il pas une étendue dans certain espace de temps, qui formait sa mesure, le présent étant sans espace ?\par
S’il en est ainsi, écoute ; voici une nouvelle voix : elle commence, se soutient et continue sans interruption : mesurons-la, pendant qu’elle se fait entendre ; le son expiré, elle sera passée, elle ne sera plus. Mesurons-la donc ; évaluons son étendue. Mais elle dure encore ; et sa mesure ne peut se prendre que de son commencement à sa fin : car c’est l’intervalle même de ces deux termes, quels qu’ils soient, que nous mesurons. Ainsi, la voix qui dure encore n’est pas mesurable. Peut-on apprécier son étendue ? sa différence ou son égalité avec une autre ? Et, quand elle aura cessé de vibrer, elle aura cessé d’être. Comment donc la mesurer ? Toutefois le temps se mesure ; mais ce n’est ni celui qui doit être, ni celui qui n’est déjà plus, ni celui qui est sans étendue, ni celui qui est sans limites ; ce n’est donc ni le temps à venir, ni le passé, ni le temps présent, ni celui qui passe que nous mesurons ; et toutefois nous mesurons le temps.\par
\pn{35}Ce vers : « DEUS CREATOR OMNIUM » est de huit syllabes, alternativement brèves et longues ; quatre brèves, la première, la troisième, la cinquième et la septième, simples par rapport aux seconde, quatrième, sixième et huitième, qui durent le double de temps. Je le sens bien en les prononçant : et il en est ainsi, au rapport de l’évidence sensible. Autant que j’en puis croire ce témoignage, je mesure une longue par une brève, et je la sens double de celle-ci. Mais elles ne résonnent que l’une après l’autre, et si la brève précède la longue, comment retenir la brève pour l’appliquer comme mesure à la longue, puisque la longue ne commence que lorsque la brève a fini ? Et cette longue même, je ne la mesure pas tant qu’elle est présente ; puisque je ne saurais la mesurer avant sa fin : cette fin, c’est sa fuite. Qu’est-ce donc que je mesure ? où est la brève, qui mesure ? où est la longue, à mesurer ? Leur son rendu, envolées, passées toutes deux, et elles ne sont plus ! et pourtant je les mesure, et je réponds hardiment, sur la foi de mes sens, que l’une est simple, l’autre double en durée ; ce que je ne puis assurer, qu’elles ne soient passées et finies. Ce n’est donc pas elles que je mesure, puisqu’elles ne sont plus, mais quelque chose qui demeure dans ma mémoire, profondément imprimé.\par
\pn{36}C’est en toi, mon esprit, que je mesure le temps. Ne laisse pas bourdonner à ton oreille : Comment ? comment ? et ne laisse pas bourdonner autour de toi l’essaim de tes impressions ; oui, c’est en toi que je mesure l’impression qu’y laissent les réalités qui passent ; impression survivante à leur passage. Elle seule demeure présente ; je la mesure, et non les objets qui l’ont fait naître par leur passage. C’est elle que je mesure quand je mesure le temps : donc, le temps n’est autre chose que cette impression, ou il échappe à ma mesure.\par
 Mais quoi ! ne mesurons-nous pas le silence ? Ne disons-nous pas : Ce silence a autant de durée que cette parole ? Et notre pensée ne se représente-t-elle pas alors la durée du son, comme s’il régnait encore ; et cet espace ne lui sert-il pas de mesure pour calculer l’étendue silencieuse ? Ainsi, la voix et les lèvres muettes, nous récitons intérieurement des poèmes, des vers, des discours, quels qu’en soient le mouvement et les proportions ; et nous apprécions la durée, le rapport successif des mots, des syllabes, comme si notre bouche en articulait le son. Je veux soutenir le ton de ma voix, la durée préméditée de mes paroles est un espace, déjà franchi en silence, et confié à la garde de ma mémoire. Je commence, ma voix résonne jusqu’à ce qu’elle arrive au but déterminé. Que dis-je ? elle a résonné, et résonnera. Ce qui s’est écoulé d’elle, son évanoui ; le reste, son futur. Et la durée s’accomplit par l’action présente de l’esprit, poussant l’avenir au passé, qui grossit du déchet de l’avenir, jusqu’au moment où, l’avenir étant épuisé, tout n’est plus que passé.
\section[{Chapitre XXVIII, L’Esprit est la mesure du temps.}]{Chapitre XXVIII, L’Esprit est la mesure du temps.}
\noindent \pn{37}Mais qu’est-ce donc que la diminution ou l’épuisement de l’avenir qui n’est pas encore ? Qu’est-ce que l’accroissement du passé qui n’est plus, si ce n’est que dans l’esprit, où cet effet s’opère, il se rencontre trois termes l’attente, l’attention et le souvenir ? L’objet de l’attente passe par l’attention, pour tourner en souvenir. L’avenir n’est pas encore ; qui le nie ? et pourtant son attente est déjà dans notre esprit. Le passé n’est plus, qui en doute ? et pourtant son souvenir est encore dans notre esprit. Le présent est sans étendue, il n’est qu’un point fugitif ; qui l’ignore ? et pourtant l’attention est durable ; elle par qui doit passer ce qui court à l’absence : ainsi, ce n’est pas le temps à venir, le temps absent ; ce n’est pas le temps passé, le temps évanoui qui est long un long avenir, c’est une longue attente de l’avenir ; un long passé, c’est un long souvenir du passé.\par
\pn{38}Je veux réciter un cantique ; je l’ai retenu. Avant de commencer, c’est une attente intérieure qui s’étend à l’ensemble. Ai-je commencé ? tout ce qui accroît successivement au pécule du passé entre au domaine de ma mémoire : alors, toute la vie de ma pensée n’est que mémoire : par rapport à ce que j’ai dit ; qu’attente, par rapport à ce qui me reste à dire. Et pourtant mon attention reste présente, elle qui précipite ce qui n’est pas encore à n’être déjà plus. Et, à mesure que je continue ce récit, l’attente s’abrége, le souvenir s’étend jusqu’au moment où l’attente étant toute consommée, mon attention sera tout entière passée dans ma mémoire. Et il en est ainsi, non-seulement du cantique lui-même, mais de chacune de ses parties, de chacune de ses syllabes : ainsi d’une hymne plus longue, dont ce cantique n’est peut-être qu’un verset ; ainsi de la vie entière de l’homme, dont les actions de l’homme sont autant de parties ; ainsi de cette mer des générations humaines, dont chaque vie est un flot.
\section[{Chapitre XXIX, De l’union avec Dieu.}]{Chapitre XXIX, De l’union avec Dieu.}
\noindent \pn{39}Mais\par

\begin{quoteblock}
\noindent « votre miséricorde vaut mieux que toutes les vies (Ps. LXII, 4) ; »\end{quoteblock}

\noindent et toute ma vie à moi n’est qu’une dissipation ; et votre main m’a rassemblé en mon Seigneur, fils de l’homme, médiateur en votre unité et nous, multitude, multiplicité et division, « afin qu’en lui j’appréhende celui qui m’a appréhendé par lui ; » et que ralliant mon être dissipé au caprice de mes anciens jours, je demeure à la suite de votre unité, sans souvenance de ce qui n’est plus, sans aspiration inquiète vers ce qui doit venir et passer, mais recueilli « dans l’immutabilité toujours présente, » et ravi par un attrait sans distraction à la poursuite de cette « palme que votre voix me promet dans la « gloire (Philip. III, 12, 13, 14) » où j’entendrai l’hymne de vos louanges, où je contemplerai votre joie sans avenir et sans passé.\par

\begin{quoteblock}
\noindent \emph{Maintenant} « mes années s’écoulent dans les « gémissements (Ps. XXX, II), »\end{quoteblock}

\noindent et vous, ô ma consolation, ô Seigneur, ô mon Père ! vous êtes éternel. Et moi je suis devenu la proie des temps, dont l’ordre m’est inconnu ; et ils m’ont partagé ; et les tourmentes de la vicissitude déchirent mes pensées, ces entrailles de mon âme, tant que le jour n’est pas venu où, purifié de mes souillures et fondu au feu de votre amour, je m’écoulerai tout en vous.
 \section[{Chapitre XXX, Point de temps sans œuvre.}]{Chapitre XXX, Point de temps sans œuvre.}
\noindent \pn{40}Et alors en vous, dans votre vérité, type de mon être, je serai ferme et stable ; et je n’aurai plus à essuyer les questions des hommes, frappés, par la déchéance, de cette hydropisie de curiosité qui demande : Que faisait Dieu avant de créer le ciel et la terre ? ou Comment, lui est venu la pensée de faire quelque chose, puisqu’il n’avait jamais rien fait jusque-là ?\par
Inspirez-leur, ô mon Dieu, des pensées meilleures que leurs paroles ! Qu’ils reconnaissent que JAMAIS ne saurait être où le TEMPS n’est pas ! Ainsi dire qu’on n’a jamais rien fait, n’est-ce pas dire que rien ne se fait que dans le temps ? Hommes, concevez donc qu’il ne peut y avoir de temps sans œuvre, et voyez l’inanité de votre langage ! Qu’ils fixent leur attention, Seigneur,\par

\begin{quoteblock}
\noindent « sur ce qui demeure présent devant eux (Phlip. III, 13) ; »\end{quoteblock}

\noindent qu’ils comprennent que vous êtes avant tous les temps, Créateur éternel de tous les temps ; que vous n’admettez au partage de votre éternité aucun temps, aucune créature, en fût-il une qui eût devancé les temps !
\section[{Chapitre XXXI, Dieu connaît autrement que les Hommes.}]{Chapitre XXXI, Dieu connaît autrement que les Hommes.}
\noindent \pn{41}Ô Seigneur, ô mon. Dieu, combien est profond l’abîme de votre secret ! Combien les tristes suites de mon iniquité m’en ont jeté loin ! Guérissez mes yeux ; qu’ils s’ouvrent à la joie de votre lumière. Certes, s’il était un esprit assez grand, assez étendu en science et en prescience, pour avoir du passé et de l’avenir une connaissance aussi présente que l’est à ma pensée celle de ce cantique, notre admiration pour lui ne tiendrait-elle pas de l’épouvante ? Rien, en effet, rien qui lui fût inconnu dans la vicissitude des siècles, passés ou à venir tous seraient sous son regard, comme ce cantique, que je chante, est tout entier devant moi ; car je sais ce qu’il s’en est écoulé de versets depuis le commencement, et ce qu’il en reste à courir jusqu’à la fin. Mais loin de moi la pensée d’assimiler une telle connaissance à la vôtre, ô Créateur du monde, Créateur des âmes et des corps ! Loin de moi cette pensée ! Votre science du passé et de l’avenir est bien autrement admirable et cachée. Le cantique que je chante ou, que j’entends chanter m’affecte de sentiments divers ; ma pensée se partage en attente des paroles futures, en souvenir des paroles expirées ; mais rien de tel ne survient dans votre immuable éternité ; c’est que vous êtes vraiment éternel, ô Créateur des esprits !\par
Vous avez connu dès le principe le ciel et la terre, sans succession de connaissance, et vous avez créé dès le principe le ciel et la terre sans division d’action. Que l’esprit ouvert, que l’esprit fermé à l’intelligence de ces pensées confessent votre nom ! Oh ! que vous êtes grand ! et les humbles sont votre famille. Vous les relevez de la poussière (Ps. CXLV, 8) ; et ils n’ont plus de chute à craindre, car vous êtes leur élévation.
\chapterclose


\chapteropen
 \chapter[{XII. Le ciel et la terre}]{XII. Le ciel et la terre}\phantomsection
\label{XII}\renewcommand{\leftmark}{XII. Le ciel et la terre}


\begin{argument}\noindent Le Ciel, création des natures spirituelles. — La Terre, création de la matière primitive. — Profondeur de l’Écriture. —Des divers sens qu’elle peut réunir. — Tous les sens prévus par le Saint-Esprit.
\end{argument}


\chaptercont
\section[{Chapitre premier, La recherche de la vérité est pénible.}]{Chapitre premier, La recherche de la vérité est pénible.}
\noindent \pn{1}Sollicité, sous les haillons de cette vie, par les paroles de votre sainte Écriture, mon cœur, ô Dieu ! est en proie aux plus vives perplexités. Et de là ce luxe indigent de langage qu’étale d’ordinaire l’intelligence humaine ; car la recherche de la vérité coûte plus de paroles que sa découverte, la demande d’une grâce plus de temps que le succès ; et la porte est plus dure à frapper que l’aumône à recevoir. Mais nous avons votre promesse ; qui pourrait la détruire ?\par

\begin{quoteblock}
\noindent « Si Dieu est pour nous, qui sera contre nous ? (Rom. VIII, 31) Demandez, et vous recevrez ; cherchez, et vous trouverez ; frappez, et il vous sera ouvert : car qui demande, reçoit ; qui cherche, trouve, et on ouvre à qui frappe (Matth. VII, 7-8).»\end{quoteblock}

\noindent Telles sont vos promesses ; et qui craindra d’être trompé, quand la Vérité même s’engage ?
\section[{Chapitre II, Deux sortes de Cieux.}]{Chapitre II, Deux sortes de Cieux.}
\noindent \pn{2}L’humilité de ma langue confesse à votre majesté sublime que vous avez fait le ciel que je vois, cette terre que je fouie, et dont vous avez façonné la terre que je porte avec moi. Mais, Seigneur, où est ce ciel du ciel dont le Psalmiste parle ainsi : (Le ciel du ciel est au Seigneur, et il a donné la terre aux enfants des « hommes (Ps. CXIII, 16) ?» Où est ce ciel invisible, auprès duquel le visible n’est que terre ? Car cet ensemble matériel n’est pas revêtu dans toutes ses parties d’une égale beauté, et surtout aux régions inférieures dont ce monde est la dernière. Mais à l’égard de ce ciel des cieux, les cieux de notre terre ne sont que terre. Et l’on peut affirmer sans crainte que ces deux grands corps ne sont que terre par rapport à ce ciel inconnu qui est au Seigneur, et non aux enfants des hommes.
\section[{Chapitre III, Des ténèbres répandues sur la surface de l’abîme.}]{Chapitre III, Des ténèbres répandues sur la surface de l’abîme.}
\noindent \pn{3}« Or la terre était invisible et informe » espèce d’abîme profond, sur qui ne planait aucune lumière, chaos inapparent. C’est pourquoi vous avez dicté ces paroles :\par

\begin{quoteblock}
\noindent « Les ténèbres étaient à la surface de l’abîme (Gen. I, 2) »\end{quoteblock}

\noindent Qu’est-ce que les ténèbres, sinon l’absence de la lumière ? Et si la lumière eût été déjà, où donc eût-elle été, sinon au-dessus des choses, les dominant de ses clartés ? Et si la lumière n’étant pas encore, la présence des ténèbres c’est son absence. Les ténèbres étaient, — c’est-à-dire, la lumière n’était pas, comme il y a silence où il n’y a point de son. Qu’est-ce en effet que le règne du silence, sinon la vacuité du son ? N’est-ce pas vous, Seigneur, qui enseignez ainsi cette âme qui vous parle ? n’est-ce. pas vous qui lui enseignez qu’avant de recevoir de vous la forme et l’ordre, cette. matière n’était, qu’une confusion, sans couleur, sans figure, sans corps, sans esprit ; non pas un pur néant toutefois, mais je ne sais quelle informité dépourvue d’apparence ?  
\section[{Chapitre IV, Matière primitive.}]{Chapitre IV, Matière primitive.}
\noindent \pn{4}Et cela, comment le désigner pour être compris des intelligences plus lentes, autrement que par une dénomination vulgaire ? Où trouver, dans toutes les parties du monde, quelque chose de plus analogue à cette informité vague, que la terre et l’abîme ? car, placés l’un et l’autre au dernier échelon de l’existence, sont-ils comparables aux créatures supérieures, revêtues de gloire et de lumière ? Pourquoi donc n’admettrais-je pas que, par complaisance pour la faiblesse de l’homme, l’Écriture ait nommé « terre invisible et sans forme » cette informité matérielle, que vous aviez créée d’abord dans cette aride nudité, pour en faire un monde paré de formes et de beauté ?
\section[{Chapitre V, Sa nature.}]{Chapitre V, Sa nature.}
\noindent \pn{5}Et lorsque notre pensée y cherche ce que les sens en peuvent atteindre, en se disant Ce n’est ni une forme intelligible, comme la vie, comme la justice, puisqu’elle est matière des corps ; ni une forme sensible, puisque ni la vue, ni le sens n’ont de prise sur ce qui est invisible et sans forme ; quand l’esprit de l’homme, dis-je, se parle ainsi, il faut qu’il se condamne à l’ignorance pour la connaître, et se résigne à l’ignorer en la connaissant.
\section[{Chapitre VI, Comment il faut la concevoir.}]{Chapitre VI, Comment il faut la concevoir.}
\noindent \pn{6}S’il faut, Seigneur, que ma voix et ma plume publient à votre gloire tout ce que vous m’avez appris sur cette matière primitive j’avoue qu’autrefois entendant son nom dans la bouche de gens qui m’en parlaient, sans pouvoir m’en donner une intelligence qu’ils n’avaient pas eux-mêmes, ma pensée se la représentait sous une infinité de formes diverses ; ou plutôt ce n’était pas elle que ma pensée se représentait, c’était un pêle-mêle de formes horribles, hideuses, mais pêle-mêle de formes que je nommais informe, non pour être dépourvu de formes, mais pour en affecter d’inouïes, d’étranges, et telles qu’une réalité semblable offerte à mes yeux eût rempli ma faible nature de trouble et d’horreur. Cet être de mon imagination n’était donc pas informe par absence de formes, mais par rapport à des formes plus belles. Et cependant la raison me démontrait que, pour concevoir un être absolument informe, il fallait le dépouiller des derniers restes de forme, et je ne pouvais ; j’avais plutôt fait de tenir pour néant l’objet auquel la forme était refusée, que de concevoir un milieu entre la forme et rien, entre le néant et la réalité formée, une informité, un presque néant.\par
Et ma raison cessa de consulter mon esprit tout rempli d’images formelles, qu’il varie et combine à son gré. J’attachai sur les corps eux-mêmes un regard plus attentif, et je méditai plus profondément sur cette mutabilité qui les fait cesser d’être ce qu’ils étaient, et devenir ce qu’ils n’étaient pas ; alors je soupçonnai que ce passage d’une forme à l’autre se faisait par je ne sais quoi d’informe, qui n’était pas absolument rien. Mais le soupçon ne me suffisait pas ; je désirais une connaissance certaine.\par
Et maintenant, si ma voix et ma plume vous confessaient toutes les lumières dont vous avez éclairé pour moi ces obscurités, quel lecteur pourrait prêter une attention assez durable ? Et toutefois mon cœur ne laissera pas de vous glorifier et de vous chanter un cantique d’actions de grâces ; car les paroles me manquent pour exprimer ce que vous m’avez révélé. Il est donc vrai que la mutabilité des choses est la possibilité de toutes les formes qu’elles subissent. Elle-même, qu’est-elle donc ? Un esprit ? un corps ? esprit, corps, d’une certaine nature ? Si l’on pouvait dire un certain néant qui est et n’est pas, je la définirais ainsi. Et pourtant il fallait bien qu’elle eût une sorte d’être pour revêtir ces formes visibles et harmonieuses.
\section[{Chapitre VII, Le ciel plus excellent que la terre.}]{Chapitre VII, Le ciel plus excellent que la terre.}
\noindent \pn{7}Et cette matière, quelle qu’elle fût, d’où pouvait-elle tirer son être, sinon de vous, par qui toutes choses sont tout ce qu’elles sont ? Mais d’autant plus éloignées de vous qu’elles vous sont moins semblables ; car cet éloignement n’est point une distance. Ainsi donc, ô Seigneur, toujours stable au-dessus de la mobilité des temps et de la diversité des lieux, le même, toujours le même ; saint, saint, saint ; Seigneur, Dieu tout-puissant (Isaïe) ! c’est dans le Principe procédant   de vous, dans votre sagesse née de votre substance, que vous avez créé, créé quelque chose de rien.\par
Vous avez fait le ciel et la terre, sans les tirer de vous. Car ils seraient égaux à votre Fils unique, et par conséquent à vous ; et ce qui ne procède pas de vous ne saurait, sans déraison, être égal à vous. Existait-il donc hors de vous, ô Dieu, trinité une, unité trinitaire, existait-il rien dont vous les eussiez pu former ? C’est donc de rien que vous avez fait le ciel et la terre, tant et si peu. Artisan tout puissant et bon de toute espèce de biens, vous avez fait le ciel si grand, la terre si petite. Vous étiez ; et rien avec vous dont vous pussiez les former tous deux ; l’un si près de vous, l’autre si près du néant ; l’un qui n’a que vous au-dessus de lui, l’autre qui n’a rien au-dessous d’elle.
\section[{Chapitre VIII, Matière primitive faite de rien.}]{Chapitre VIII, Matière primitive faite de rien.}
\noindent \pn{8}Mais ce ciel du ciel est à vous, Seigneur ; et cette terre, que vous avez donnée aux enfants des hommes (Ps. CXIII, 15) pour la voir et la toucher, n’était pas alors telle que nos yeux la voient, et que notre main la touche ; elle était invisible et informe, abîme que nulle lumière ne dominait.\par

\begin{quoteblock}
\noindent « Les ténèbres étaient répandues sur l’abîme(Gen. I, 2) »\end{quoteblock}

\noindent c’est-à-dire nuit plus profonde qu’au plus profond de l’abîme aujourd’hui. Car cet abîme des eaux, visible maintenant, reçoit dans ses gouffres mêmes un certain degré de lumière sensible aux poissons et aux êtres animés qui rampent dans son sein. Mais tout cet abîme primitif était presque un néant dans cette entière absence de la forme. Toutefois, il était déjà quelque chose qui pût la recevoir. Ainsi donc vous formez le monde d’une matière informe, convertie par vous de rien en un presque rien, dont vous faites sortir ces chefs-d’œuvre qu’admirent les enfants des hommes.\par
Chose admirable, en effet, que ce ciel corporel, ce firmament étendu entre les eaux et. les eaux, œuvre du second jour qui suivit la naissance de la lumière ; création d’un mot\par

\begin{quoteblock}
\noindent « Qu’il soit ! et il fut (Gen. I, 6,7) ;»\end{quoteblock}

\noindent firmament nommé par vous ciel, mais ciel de cette terre, de cette mer que vous fîtes le troisième jour, en douant d’une forme visible cette matière informe que vous aviez créée avant tous les jours. Un ciel était déjà, qui les avait précédés, mais c’était le ciel de nos cieux : car, dans le principe, vous créâtes le ciel et la terre. Pour cette terre dès lors créée, ce n’était qu’une matière informe, puisqu’elle était invisible, sans ordre, abîme ténébreux. C’est de cette terre obscure, inordonnée, de cette informité, de ce presque rien, que vous deviez produire tous les êtres par qui subsiste ce monde instable et changeant. Et c’est en ce monde que commence à paraître la mutabilité qui nous donne le sentiment et la mesure des temps ; car ils naissent de la succession des choses, de la vicissitude et de l’altération des formes dont l’origine est cette matière primitive, cette terre invisible.
\section[{Chapitre IX, Le ciel du ciel.}]{Chapitre IX, Le ciel du ciel.}
\noindent \pn{9}Aussi le Maître de votre grand serviteur, en racontant que vous avez créé dans le principe le ciel et la terre, l’Esprit-Saint ne dit mot des temps, est muet sur les jours. Car, ce ciel du ciel, que vous avez fait dans le principe, est une créature spirituelle, qui sans vous être coéternelle, ô Trinité, participe néanmoins à votre éternité. L’ineffable bonheur de contempler votre présence arrête sa mobilité, et depuis son origine, invinciblement attachée à vous, elle s’est élevée au-dessus des vicissitudes du temps. Et cette terre invisible, informe, n’a pas été non plus comptée dans l’œuvre des jours ; car, où l’ordre, où la forme ne sont pas, rien n’arrive, rien ne passe, et dès lors point de jours, point de succession de temps.
\section[{Chapitre X, Invocation.}]{Chapitre X, Invocation.}
\noindent \pn{10}Ô vérité, lumière de mon cœur ! ne laissez pas la parole à mes ténèbres. Entraîné au courant de l’instabilité, la nuit m’a pénétré ; mais c’est du fond de ma chute que je me suis senti renaître à votre amour. Egaré, j’ai retrouvé votre souvenir ; j’ai entendu votre voix me rappeler ; et le bruit des passions rebelles, me permettait à peine de l’entendre. Et me voici, maintenant, tout en nage, hors d’haleine, revenu à votre fontaine sainte. Oh ! ne souffrez pas qu’on m’en repousse. Que je m’y désaltère, que j’y puise la vie, que je ne sois pas ma vie à moi-même. De ma propre vie j’ai mal vécu, j’ai été ma mort ; en vous je   revis. Parlez-moi, instruisez-moi ! Je crois au témoignage de vos livres saints ; mais quels profonds mystères sous leurs paroles !
\section[{Chapitre XI, Ce que Dieu lui a enseigné.}]{Chapitre XI, Ce que Dieu lui a enseigné.}
\noindent \pn{11}Seigneur, vous m’avez déjà dit à l’oreille du cœur, d’une voix forte, que vous êtes éternel,\par

\begin{quoteblock}
\noindent « seul en possession de l’immortalité (I Tim. VI, 16) ; »\end{quoteblock}

\noindent parce que rien ne change en vous, ni forme, ni mouvement ; que votre volonté n’est point sujette à l’inconstance des temps ; car une volonté variable ne saurait être une volonté immortelle. Je vois clairement cette vérité en votre présence ; qu’elle m’apparaisse chaque jour plus claire, je vous en conjure ! et qu’à l’ombre de vos ailes, je demeure humblement dans cette connaissance que vous m’avez révélée ! Seigneur, vous m’avez encore dit à l’oreille du cœur, d’une voix forte, que vous êtes l’auteur de toutes les natures, de toutes les substances qui ne sont pas ce que vous êtes, et sont néanmoins ; qu’il n’est rien qui ne soit votre ouvrage, hors le néant et ce mouvement de la volonté qui, s’éloignant de vous, abandonne l’être par excellence pour l’être inférieur : car ce mouvement est une défaillance et un péché ; qu’enfin nul péché, soit au faîte, soit au dernier degré de votre création, ne saurait vous nuire ou troubler votre ordre souverain. Je vois clairement cette vérité en votre présence ; qu’elle m’apparaisse chaque jour plus claire, je vous en conjure ! et qu’à l’ombre de vos ailes, je demeure humblement dans cette connaissance que vous m’avez révélée !\par
\pn{12}Seigneur, vous m’avez dit encore à l’oreille du cœur, d’une voix forte, que cette créature même ne vous est pas coéternelle, qui n’a d’autre volonté que la vôtre, qui, s’enivrant des intarissables délices d’une possession chaste et permanente, ne trahit nulle part et jamais sa mutabilité de nature, et, liée de tout son amour à votre présente éternité, n’a point d’avenir à attendre, point de passé dont la fuite ne lui laisse qu’un souvenir, supérieure à la vicissitude, étrangère aux atteintes du temps. Ô créature bienheureuse ! si elle existe ; heureuse de cet invincible attachement à votre béatitude ; heureuse d’être à jamais la demeure de votre éternité, et le miroir de votre lumière ! Et qui mérite mieux le nom de ciel du ciel que ce temple spirituel, plongé dans l’ivresse de votre joie sans que rien incline ailleurs sa défaillance ; pure intelligence, unie par le lien d’une paix divine aux esprits de sainteté, habitants de votre cité sainte, cité céleste, et par delà tous les cieux.\par
\pn{13}De là vienne à l’âme la grâce de comprendre jusqu’où son malheureux pèlerinage l’a éloignée de vous, et si elle a déjà soif de vous ; si ses larmes sont devenues son pain, quand chaque jour on lui demande : Où est ton Dieu (Ps. XLI, 3,4,11) ? Si elle ne vous adresse d’autre vœu, d’autre prière, qu’afin d’habiter votre maison tous les jours de sa vie (Ps. XXVI, 4). Et quelle est sa vie que vous-même, et quels sont vos jours que votre éternité ; puisque vos années ne manquent jamais, et que vous êtes le même (Ps. CI, 28) ?\par
Que l’âme qui le peut comprenne donc combien votre éternité plane au-dessus de tous les temps, puisque les intelligences, votre temple, qui n’ont pas voyagé aux régions étrangères, demeurent par leur fidélité à votre amour affranchies des caprices du temps. Je vois clairement cette vérité en votre présence ; qu’elle m’apparaisse chaque jour plus claire, je vous en conjure ! et, qu’à l’ombre de vos ailes, je demeure humblement dans cette connaissance que vous m’avez révélée ! 14. Mais je ne sais quoi d’informe se trouve dans les changements qui altèrent les choses de l’ordre inférieur. Et quel autre que l’insensé, égaré dans le vide, et flottant sur les vagues chimères de son cœur, pourrait me dire que, si toute forme était arrivée par réduction successive à l’anéantissement, la seule existence de cette informité, support réel de toute transformation, suffirait à produire les vicissitudes du temps ? Chose impossible : car, point de temps, sans variété de mouvements, et point de variété, sans formes.
\section[{Chapitre XII, Deux ordres de créatures.}]{Chapitre XII, Deux ordres de créatures.}
\noindent \pn{15}J’ai considéré ces vérités, mon Dieu, autant que vous m’en avez fait la grâce ; autant que vous m’avez excité à frapper, autant qu’il vous a plu de m’ouvrir ; et je trouve deux créatures, que vous avez faites hors du temps ; quoiqu’elles ne vous soient, ni l’une ni l’autre, coéternelles : l’une si parfaite, que, dans la joie non interrompue de votre contemplation,   inaccessible à l’impression de l’inconstance, elle demeure sans changer, malgré sa mutabilité naturelle, et jouit de votre immuable éternité ; et l’autre si informe, que, dépourvue de l’être suffisant pour accuser le mouvement ou le repos, elle n’offre aucune prise à la domination du temps. Mais vous ne l’avez pas laissée dans cette informité, puisque dans le principe, avant les jours, vous avez formé ce ciel et cette terre, dont je parle.\par

\begin{quoteblock}
\noindent « Or, la terre était invisible, informe, et les ténèbres couvraient l’abîme (Gen. I, 2).»\end{quoteblock}

\noindent Par ces paroles s’insinue peu à peu, dans les esprits qui ne peuvent concevoir la privation de la forme autrement que comme l’absence de l’être, la notion de cette informité, germe d’un autre ciel, d’une terre visible et ordonnée, source des eaux transparentes, et de toutes les merveilles que la tradition comprend dans l’œuvre des jours, parce que les évolutions de formes et de mouvements, prescrites à leur nature, la soumettent aux vicissitudes des temps.
\section[{Chapitre XIII, Créatures spirituelles ; matière informe.}]{Chapitre XIII, Créatures spirituelles ; matière informe.}
\noindent \pn{16}Lorsque la voix de votre Écriture parle ainsi :\par

\begin{quoteblock}
\noindent « Dans le principe, Dieu créa le ciel et « la terre : or , la terre était invisible, informe ; et les ténèbres couvraient la face de « l’abîme (Ibid. 2) ; »\end{quoteblock}

\noindent sans assigner aucun jour à cette création ; je pense que par ce ciel, ciel de nos cieux, on doit entendre le ciel spirituel où l’intelligence n’est qu’une intuition qui voit tout d’un coup, non pas en partie, ni en énigme, ou comme en un miroir, mais de pleine évidence, face à face (I Cor. XIII, 12), d’un regard invariable et fixe ; claire vue, sans succession, sans instabilité de temps ; et par cette terre, la terre invisible et informe que le temps ne pouvait atteindre. Ceci, puis cela, telle est la pâture de la vicissitude ; mais le changement peut-il être où la forme n’est pas ? C’est donc, suivant moi, de ces deux créatures, produites, l’une dans la perfection, l’autre dans l’indigence de la forme ; ciel d’une part, mais ciel du ciel ; terre de l’autre, mais terre invisible et informe, que l’Écriture dit sans mention de jour : « Dans le « principe, Dieu fit le ciel et la terre. » Car elle dit aussitôt quelle terre. Et comme elle rapporte au second jour la création du firmament, qui fut appelé ciel, elle insinue la distinction de cet autre ciel né avant les jours.
\section[{Chapitre XIV, Profondeur des écritures.}]{Chapitre XIV, Profondeur des écritures.}
\noindent \pn{17}Etonnante profondeur de vos Écritures ! leur surface semble nous sourire, comme à des petits enfants ; mais quelle profondeur, ô mon Dieu ! insondable profondeur ! À la considérer, je me sens un vertige d’effroi, effroi de respect, tremblement d’amour ! Oh ! de quelle haine je hais ses ennemis ! Que ne les passez-vous au fil de votre glaive doublement acéré, afin de les retrancher du nombre de vos ennemis ? Que j’aimerais les voir ainsi frappés de mort à eux-mêmes pour vivre à vous ! Il en est d’autres, non plus détracteurs, mais admirateurs respectueux de la Genèse, qui me disent : « Le Saint-Esprit, qui a dicté ces paroles à Moïse, son serviteur, n’a pas voulu qu’elles fussent prises dans le sens où tu les interprètes, mais dans celui-ci, dans le nôtre. » Seigneur, notre Dieu, je vous prends pour arbitre ! voilà ma réponse.
\section[{Chapitre XV, Vérités constantes, malgré la diversité des interprétations.}]{Chapitre XV, Vérités constantes, malgré la diversité des interprétations.}
\noindent \pn{18}Taxerez-vous de fausseté ce que la vérité m’a dit d’une voix forte à l’oreille du cœur ; tout ce qu’elle m’a révélé de l’éternité du Créateur, à savoir que sa substance ne varie point dans le temps et que sa volonté n’est point hors de sa substance ? Volonté sans succession, une, pleine et constante ; sans contradiction et sans caprice, car le caprice, c’est le changement, et ce qui change n’est pas éternel. Or, notre Dieu est l’éternité même. Démentirez-vous encore la même voix, qui m’a dit : L’attente des choses à venir devient une vision directe quand elles sont présentes. Sont-elles passées ? cette vision n’est plus que mémoire. Mais toute connaissance qui varie est muable ; et ce qui est muable n’est pas éternel. Or, notre Dieu est l’éternité même. Je rassemble, je réunis ces vérités, et vois que ce n’est point une survenance de volonté en Dieu, qui a créé le monde, et que sa science ne souffre rien d’éphémère.\par
\pn{19}Contradicteurs, qu’avez-vous à répondre ? Ai-je avancé une erreur ? — Non, — Quoi   donc ? Est-ce une erreur de prétendre que toute nature formée, que toute matière capable de forme, ne tiennent leur être que de Celui qui est la souveraine bonté, parce qu’il est le souverain être ? Non, dites-vous. Quoi donc ? Que niez-vous ? serait-ce l’existence d’une créature supérieure, dont le chaste amour embrasse si étroitement le vrai Dieu, le Dieu de l’éternité, que, sans lui être coéternelle, elle ne se détache jamais de lui pour tomber dans le torrent des jours, et se repose dans la contemplation de son unique vérité ? Aimé de cette heureuse créature, de tout l’amour que vous exigez, ô Dieu, vous vous montrez à elle, et vous lui suffisez, et elle ne se détourne jamais de vous, pas même pour se tourner vers elle. Voilà cette maison de Dieu, qui n’est faite d’aucun élément emprunté à la terre, ou aux cieux corporels ; demeure spirituelle ; admise à la jouissance de votre éternité, parce qu’elle demeure dans une pureté éternelle. Vous l’avez fondée à jamais ; tel est votre ordre, et il ne passe point (PS CXLVIII, 6). Et cependant elle ne vous est point coéternelle ; elle a commencé, car elle a été créée.\par
\pn{20}Nous ne trouvons pas, il est vrai, de temps avant elle, selon cette parole :\par

\begin{quoteblock}
\noindent « La sagesse a été créée la première (Ecclési. I, 4), »\end{quoteblock}

\noindent non pas cette Sagesse dont vous êtes le père, ô mon Dieu, égale et coéternelle à vous-même, par qui toutes choses ont été créées, principe en qui vous avez fait le ciel et la terre. Mais cette sagesse créature, substance intelligente, lumière par la contemplation de votre lumière, car, toute créature qu’elle est, elle porte aussi le nom de sagesse mais la lumière illuminante diffère de la lumière illuminée ; la sagesse créatrice, de la sagesse créée ; comme la justice justifiante, de la justice opérée par la justification. Ne sommes-nous pas appelés aussi votre justice ? L’un de vos serviteurs n’a-t-il pas dit :\par

\begin{quoteblock}
\noindent « Afin que nous soyons la justice de Dieu en lui (II Cor. V, 21) ? »\end{quoteblock}

\noindent Il est donc une sagesse créée la première ; et cette sagesse n’est autre chose que ces essences intelligentes, membres de votre Ville Sainte, notre mère, qui est en haut, libre (Galat. IV, 26), éternelle dans les cieux ; et quels cieux, sinon ces cieux sublimes, vos hymnes vivantes ; ce ciel des cieux (Ps. CXLVIII, 4) qui est à vous ? Sans doute, nous ne trouvons pas de temps qui précède cette sagesse. Créée la première, elle devance la création du temps ; mais avant elle préexiste l’éternité du Créateur dont elle tire sa naissance, non pas selon le temps, qui n’était pas encore, mais suivant sa condition d’être créée.\par
\pn{24}Elle procède donc de vous, ô mon Dieu ! toutefois bien différente de vous, loin d’être vous-même. Il est vrai que, ni avant elle, ni en elle, nous ne trouvons aucun temps ; que, demeurant toujours devant votre face, sans défaillance, sans infidélité , cette constance l’élève au-dessus du changement ; mais sa nature, qui le comporte, ne serait plus qu’une froide nuit, si son amour ne trouvait dans l’intimité de votre union un éternel midi de lumière et de chaleur. Rayonnante demeure, palais resplendissant ; oh ! que ta beauté m’est chère, résidence de la gloire de mon Dieu (Ps. XXV, 8) ! sublime ouvrier qui réside dans son ouvrage, combien je soupire vers toi du fond de ce lointain exil, et je conjure ton Créateur de me posséder aussi, de me posséder en toi ; car ce Créateur est le mien. Je me suis égaré comme une brebis perdue (Ps. CXVIII, 16), mais je compte sur les épaules du bon pasteur, ton divin architecte, pour être reporté dans ton enceinte (Luc, XV, 5).\par
\pn{22}Que répondez-vous maintenant, contradicteurs à qui je parlais, vous qui pourtant reconnaissez Moïse pour un fidèle serviteur de Dieu, et ses livres pour les oracles du Saint-Esprit ? Dites, n’est-ce pas là cette maison de Dieu qui, sans lui être coéternelle, a néanmoins son éternité propre dans les cieux ? Vainement vous cherchez en elle la vicissitude et le temps, vous ne les trouverez jamais ; n’est-elle pas exaltée au-dessus de toute étendue fugitive la créature qui puise sa félicité dans une permanente union avec Dieu (Ps. LXII, 28) ? Oui sans doute. Eh bien ! que trouvez-vous donc à reprendre dans toutes ces vérités que le cri de mon cœur a fait remonter vers mon Dieu, quand je prêtais l’oreille intérieure à la voix de ses louanges ? Dites, où est donc l’erreur ? Est-ce dans cette opinion que la matière était informe ; que, là où la forme n’est pas, l’ordre ne saurait être ; que l’absence de l’ordre faisait l’absence du temps, et qu’il n’y avait pourtant là qu’un presque néant, qui, doué toutefois d’une sorte d’être, ne le pouvait tenir que du principe de tout être, et de toute existence ? C’est ce que nous accordons encore, dites-vous.
 \section[{Chapitre XVI, Contre les contradicteurs de la vérité.}]{Chapitre XVI, Contre les contradicteurs de la vérité.}
\noindent \pn{23}Je veux m’entretenir un instant en votre présence, ô mon Dieu ! avec ceux qui reconnaissent pour véritables toutes les révélations dont la parole de votre vérité a éclairé mon âme. Pour ceux qui les nient, qu’ils s’assourdissent eux-mêmes tant qu’ils voudront de leurs aboiements ; je les inviterai de toutes mes forces à rentrer dans le calme, pour préparer en eux la voie à votre Verbe. S’ils s’y refusent, s’ils me repoussent, je vous en supplie, mon Dieu,\par

\begin{quoteblock}
\noindent « ne me laissez pas dans votre silence (Ps. XXVII, 1) ; »\end{quoteblock}

\noindent oh ! parlez à mon cœur en vérité : car il n’appartient qu’à vous de parler ainsi ; et ces insensés, qu’ils restent dehors soulevant de leur souffle la terre poudreuse qui aveugle leurs yeux ; et j’entrerai dans le plus secret de mon âme ; et mes chants vous diront mon amour ; et mes gémissements, les ineffables souffrances de mon pèlerinage, et mon cœur, toujours élevé en haut dans la chère souvenance de Jérusalem, n’aura de soupirs que pour Jérusalem, ma patrie, Jérusalem, ma mère, Jérusalem et vous, son roi, son soleil, son père, son protecteur, son époux, ses chastes et puissantes délices, son immuable joie ; joie au-dessus de toute parole ; sa félicité parfaite, son bien unique et véritable, vous, le seul bien, le bien en vérité et par excellence ; non, mes soupirs ne se tairont pas que vous ne m’ayez reçu dans la paix de cette mère chérie, dépositaire des prémices de mon esprit, foyer d’où s’élancent vers moi toutes ces lumières ; et que votre main n’ait rassemblé les dissipations, réformé les difformités de mon âme, pour la soutenir dans une impérissable beauté, ô ma miséricorde ! ô mon Dieu !\par
Quant à ceux qui ne contestent point ces vérités, dont la vénération, d’accord avec la nôtre, élève au plus haut point d’autorité les saintes Écritures tracées par Moïse, votre saint serviteur, mais qui trouvent à reprendre dans mes paroles, voici ce que je leur réponds : « Seigneur notre Dieu, soyez l’arbitre entre mes humbles révélations et leurs censures. »
\section[{Chapitre XVII, Ce que l’on doit entendre par le ciel et la terre.}]{Chapitre XVII, Ce que l’on doit entendre par le ciel et la terre.}
\noindent \pn{24}Tout cela est vrai, disent-ils ; mais ce n’est pas ces deux ordres de créatures que Moïse avait en vue lorsqu’il écrivait sous la dictée du Saint-Esprit :\par

\begin{quoteblock}
\noindent « Dans le principe, Dieu fit le ciel et la terre (Gen. I, 1). »\end{quoteblock}

\noindent Non, il n’a pas désigné par le ciel une essence spirituelle ou intelligente, ravie dans l’éternelle contemplation de Dieu, ni par la terre une matière informe. — Qu’entend-il donc ? — Ce que nous disons, répondent-ils ; il n’entend pas, il n’exprime pas autre chose que nous. — Quoi donc enfin ? — Sous les noms de ciel et de terre, il a d’abord compris sommairement et en peu de mots tout ce monde visible, pour distinguer ensuite en détail, selon le nombre des jours, ce qu’il a plu au Saint-Esprit de nommer en général le ciel et la terre. Car, s’adressant au peuple juif, à ce troupeau d’hommes grossiers et charnels, il ne voulait lui signaler que la partie visible des œuvres de Dieu. Mais par « cette terre invisible et informe, par cet abîme de ténèbres » qui servit de matière à l’œuvre successive des six jours, à la création et à l’ordonnance de ce monde visible, ils m’accordent que l’on peut entendre cette matière informe dont j’ai parlé.\par
\pn{25}Un autre dira peut-être que cette confusion de matière informe a été d’abord désignée sous le nom de ciel et terre, parce qu’elle est comme la matière de ce monde visible et de l’ensemble des natures qui s’y manifestent, souvent appelées ainsi. Ne peut-on pas dire aussi que c’est avec assez de raison que toutes les substances invisibles et visibles sont dénommées ciel et terre ; et que ces deux termes comprennent la création entière accomplie dans le Principe, c’est-à-dire dans la Sagesse divine ; mais que tous les êtres étant sortis du néant, et non de la substance de Dieu, puisqu’ils ne participent pas à sa nature et qu’ils ont en eux-mêmes le principe de la mutabilité, soit qu’ils demeurent comme l’éternelle maison du Seigneur, soit qu’ils changent comme l’âme et le corps de l’homme ; la matière de toutes choses visibles et invisibles encore dénuée de la forme, capable toutefois de la recevoir pour devenir le ciel et la terre, a été justement nommée « terre invisible et informe, abîme de ténèbres, » sauf cette distinction nécessaire entre la terre   invisible et sans ordre ou la matière corporelle avant l’investiture de la forme ; et les ténèbres répandues sur l’abîme ou la matière spirituelle avant la compression de sa fluide mobilité et le « FIAT LUX » de votre sagesse.\par
\pn{26}Un autre peut dire encore, s’il lui plaît, que ces paroles de l’Écriture : « Dans le principe Dieu fit le ciel et la terre, » ne sauraient s’entendre des créatures invisibles et visibles arrivées à la perfection de leur être ; mais qu’elles désignent une informe ébauche de forme et de création, germe obscur où s’agitaient confusément, sans distinction de formes et de qualités, les substances qui, dans l’ordre où elles sont aujourd’hui disposées, s’appellent le ciel ou le monde des esprits, la terre ou le monde des corps.
\section[{Chapitre XVIII, On peut donner plusieurs sens a l’écriture.}]{Chapitre XVIII, On peut donner plusieurs sens a l’écriture.}
\noindent \pn{27}J’écoute, je pèse ces opinions ; mais loin de moi toute dispute.\par

\begin{quoteblock}
\noindent « La dispute n’est bonne qu’à ruiner la foi des auditeurs (II Tim. II, 4), tandis que la loi édifie « ceux qui en savent le bon usage ; son but est l’amour qui naît d’un cœur pur, d’une bonne conscience et d’une foi sincère (I Tim. I, 8,5), »\end{quoteblock}

\noindent et le divin Maître n’ignore pas quels sont les deux commandements où il a réduit la loi et les prophètes (Matth. XXII, 40). Que m’importe donc, ô mon Dieu, ô lumière de mes yeux intérieurs, que m’importe, tant que mon amour confesse votre gloire, que ces paroles soient susceptibles d’interprétations différentes ? Que m’importe, dis-je, qu’un autre tienne pour le sens vrai de Moïse, un sens étranger au mien ? Nous cherchons tous dans la lecture de ces livres, à pénétrer et à comprendre la pensée de l’homme de Dieu, et le reconnaissant pour véridique, oserions-nous lui attribuer ce que nous savons ou croyons faux ? Ainsi donc, tandis que chacun s’applique à trouver l’intention de l’auteur inspiré, où est le mal, si à votre clarté, ô lumière des intelligences sincères, je découvre un sens que vous me démontrez véritable, quoique ce sens ne soit pas le sien, et, malgré cette différence, laisse le sien dans toute sa vérité ?
\section[{Chapitre XIX, Vérités incontestables.}]{Chapitre XIX, Vérités incontestables.}
\noindent \pn{28}C’est une vérité, Seigneur, que vous avez créé le ciel et la terre, c’est une vérité que votre Sagesse est le principe en qui vous avez créé toutes choses (Ps. CIII, 24) ; c’est une vérité que ce monde visible présente deux grandes divisions, le ciel et la terre, et que ces deux mots résument toutes les créatures. C’est une vérité que tout être muable nous suggère l’idée d’une certaine informité, ou susceptibilité de forme, d’altération et de changement. C’est une vérité que le temps est sans pouvoir sur l’être muable par sa nature, mais immuable par son intime union avec la forme immuable. C’est une vérité, que l’informité, ce presque néant, est également exempte des révolutions du temps. C’est une vérité que la matière d’une entité peut porter par anticipation le nom de cette entité même ; qu’ainsi on a pu nommer le ciel et la terre, ce je ne sais quoi d’informe, dont le ciel et la terre ont été formés. C’est une vérité, que de toutes les réalités formelles, rien n’est plus voisin de l’informité que la terre et l’abîme. C’est une vérité que tout être créé et formé, que toute possibilité de création et de forme, est votre ouvrage, ô Principe de toutes choses ! C’est une vérité, que tout être informe qui est formé, était d’abord dans l’informité pour passer à la forme.
\section[{Chapitre XX, Interprétations diverses des premières paroles de la genèse.}]{Chapitre XX, Interprétations diverses des premières paroles de la genèse.}
\noindent \pn{29}De toutes ces vérités, dont ne doutent point ceux à qui vous avez fait la grâce d’ouvrir les yeux de l’âme et de croire fermement que Moïse n’a parlé que suivant l’Esprit de vérité, l’un en choisit une et dit : « Dans le principe, Dieu fit le ciel et la terre, » c’est-à-dire Dieu fit dans son Verbe, coéternel à lui-même, des créatures intelligentes ou spirituelles, sensibles ou corporelles. Un autre : « Dans le principe, Dieu fit le ciel et la terre, » c’est-à-dire Dieu fit dans son Verbe, coéternel à lui-même, ce monde corporel avec cet ensemble de réalités évidentes à nos yeux et à notre esprit.\par
Cet autre : « Dans le principe, Dieu fit le ciel « et la terre, » c’est-à-dire dans son Verbe coéternel à lui-même, Dieu fit la matière informe   de toute création spirituelle et corporelle. Celui-ci : « Dans le principe, Dieu fit le ciel et la terre, » c’est-à-dire dans son Verbe coéternel à lui-même, Dieu créa le germe informe du monde corporel, la matière où étaient confondus le ciel et la terre, qui depuis ont reçu l’ordonnance et la forme dont nos yeux sont témoins. Celui-là dit enfin : « Dans le principe, Dieu fit le ciel et la terre, » c’est-à-dire aux préliminaires de sen œuvre, Dieu créa cette matière, grosse du ciel et de la terre, qui depuis sont sortis de son sein avec les formes qu’ils manifestent et les êtres qu’ils renferment.
\section[{Chapitre XXI, Explications différentes de ces mots : « la Terre était invisible. »}]{Chapitre XXI, Explications différentes de ces mots : « la Terre était invisible. »}
\noindent \pn{30}De même, quant à l’intelligence des paroles suivantes, chacun trouve une vérité dont il s’empare. L’un s’exprime ainsi : « La terre « était invisible, informe, et les ténèbres couvraient l’abîme ; » c’est-à-dire : cette création corporelle, ouvrage de Dieu, était la matière de toutes les réalités corporelles, mais sans forme, sans ordre et sans lumière. Un autre dit : « La terre était invisible, informe ; et les ténèbres couvraient l’abîme ; » c’est-à-dire : cet ensemble qu’on appelle le ciel et la terre, n’était encore qu’une matière informe et ténébreuse, d’où devaient sortir ce ciel corporel, cette terre corporelle, avec toutes les réalités corporelles connues de nos sens. Celui-ci : « La terre était invisible, informe, et les ténèbres couvraient l’abîme ; » c’est-à-dire : cet ensemble, qui a reçu le nom de ciel et de terre, n’était encore qu’une matière informe et ténébreuse, qui devait produire le ciel intelligible, autrement dit le ciel du ciel (Ps. CXIII, 16), et la terre ; c’est-à-dire toute la nature apparente, y compris les corps célestes ; en un mot, le monde invisible et le monde visible.\par
Un autre : « La terre était invisible, informe, « et les ténèbres couvraient l’abîme. » Ce n’est pas ce chaos que l’Écriture appelle le ciel et la terre ; mais, après avoir signalé la création des esprits et des corps, elle désigne sous le nom de terre invisible et sans ordre, d’abîme ténébreux, cette matière préexistante dont Dieu les avait formés. Un autre vient et dit : « La terre était «invisible, informe, et les ténèbres couvraient l’abîme ; » c’est-à-dire : il y avait déjà une matière informe, d’où l’action créatrice, préalablement attestée par l’Écriture, a tiré le ciel et la terre, en d’autres termes, cette masse de l’univers, partagée en deux grandes divisions : l’une supérieure, et l’autre inférieure, avec tous les êtres qu’elles présentent à notre connaissance.
\section[{Chapitre XXII, Plusieurs créations de Dieu passées sous silence.}]{Chapitre XXII, Plusieurs créations de Dieu passées sous silence.}
\noindent \pn{31}Vainement voudrait-on réfuter ces deux dernières opinions, en disant : Si vous ne voulez pas admettre que cette informité matérielle soit désignée par le nom de ciel et de terre, il existait donc quelque chose, indépendant de l’action créatrice, dont Dieu s’est servi pour faire le ciel et la terre ? Car l’Écriture ne dit point que Dieu ait créé cette matière, à moins qu’elle ne soit exprimée par la dénomination de ciel et de terre, ou de terre seulement, lorsqu’il dit : « Dans le principe, Dieu fit le ciel et la terre : or, la terre était invisible et informe ;» et, quand même le Saint-Esprit eût voulu désigner, par ces derniers mots, la matière informe, nous ne pourrions toujours entendre que cette création divine, attestée par ce verset : « Dieu fit le ciel et la terre. »\par
Mais, répondront les tenants de ces deux opinions, nous ne nions pas que cette matière soit l’œuvre de Dieu, principe de tout bien : car si nous disons que Ce qui a déjà reçu l’être et la forme est bien, à un plus haut degré que Ce qui n’en a que la capacité, nous n’en admettons pas moins que ce dernier état ne soit un bien. Quant au silence de l’Écriture sur la création de cette informité matérielle, on pourrait également l’objecter à l’égard des chérubins et des séraphins (Isaïe VI, 2 ; XXXVII, 16), et de tant d’autres esprits célestes, distingués par l’Apôtre en trônes, dominations, principautés, puissances (Coloss. I, 16), dont l’Écriture se tait, quoiqu’ils soient évidemment l’œuvre de Dieu.\par
Si l’on veut que tout soit compris dans ces mots : « Il fit le ciel et la terre, » que dirons- nous donc des eaux sur lesquelles l’Esprit de Dieu était porté ? Si, par le nom de terre, il faut implicitement les entendre, comment ce nom peut-il exprimer une matière informe, s’il désigne aussi ces eaux que nos yeux voient si transparentes et si belles ? Et, si on le prend   ainsi, pourquoi l’Écriture dit-elle que de cette matière informe a été formé le firmament, nommé ciel, sans faire mention des eaux ? Sont-elles donc encore invisibles et informes, ces eaux dont nous admirons le limpide cristal ? Ont-elles été revêtues de leur parure lorsque Dieu dit :\par

\begin{quoteblock}
\noindent « Que les eaux, inférieures au « firmament, se rassemblent (Gen. I, 9) ! »\end{quoteblock}

\noindent et cette réunion est-elle leur création ? Mais que dira-t-on des eaux supérieures au firmament ? Informes, eussent-elles reçu une place si honorable ? Et nulle part l’Écriture ne dit quelle parole les a formées.\par
Ainsi, la Genèse garde le silence sur la création de certains êtres ; et, ni la rectitude de la foi, ni la certitude de la raison, ne permettent de douter que Dieu les ait créés. Quel autre qu’un insensé oserait conclure qu’ils lui sont coéternels, de ce que la Genèse affirme leur existence sans parler de leur création ? Eh ! pourquoi donc refuserions-nous de concevoir, à la lumière de la vérité, que cette terre invisible et sans ordre, abîme de ténèbres, soit l’œuvre de Dieu, tirée du néant ; non coéternelle à lui, quoique le récit divin omette le moment de sa création ?
\section[{Chapitre XXIII, Deux espèces de doutes dans l’interprétation de l’écriture.}]{Chapitre XXIII, Deux espèces de doutes dans l’interprétation de l’écriture.}
\noindent \pn{32}J’écoute, je pèse ces sentiments divers, selon la portée de ma faiblesse, que je confesse à mon Dieu, dont elle est connue, et je vois qu’il peut naître deux sortes de débats sur les témoignages que nous ont laissés les plus fidèles oracles de la tradition. Ils peuvent porter, d’une part, sur la vérité des choses ; de l’autre, sur l’intention qui en dicte le récit : car il est différent de chercher la vérité en discutant le problème de la création, ou de préciser le sens que Moïse, ce grand serviteur de notre foi, attache à sa parole.\par
À l’égard de la première difficulté, loin de moi ceux qui prennent leurs mensonges pour la vérité ! À l’égard de la seconde, loin de moi ceux qui prétendent que Moïse affirme l’erreur ! Mais, ô Seigneur, paix et joie en vous, avec ceux qui se nourrissent de la vérité dans l’étendue de l’amour ! Approchons-nous ensemble de votre sainte parole, et cherchons votre pensée dans l’intention de votre serviteur, dont la plume est votre interprète.
\section[{Chapitre XXIV, Difficultés de déterminer le vrai sens de Moïse entre plusieurs également vrais.}]{Chapitre XXIV, Difficultés de déterminer le vrai sens de Moïse entre plusieurs également vrais.}
\noindent \pn{33}Mais, entre tant de solutions différentes et toutes véritables, qui de nous osera dire avec confiance : Voici la pensée de Moïse ; voici le sens où il veut que l’on prenne son récit ? Qui l’osera- dire avec cette hardiesse qui affirme la vérité d’une interprétation, qu’elle ait été ou non dans la pensée de Moïse ?\par
Et moi, mon Dieu, moi, votre serviteur, qui vous ai voué ce sacrifice de mes confessions, et demande à votre miséricorde la grâce d’accomplir ce vœu, je déclare avec assurance, que vous êtes, par votre Verbe immuable, l’auteur de toutes les créatures invisibles et visibles. Mais puis-je soutenir avec la même puissance de conviction, que Moïse n’avait pas en vue d’autres sens, lorsqu’il écrivait : « Dans le principe, Dieu fit le ciel et la terre ? » Je vois dans votre vérité la certitude de ma parole, et je ne puis lire dans l’esprit de Moïse si telle était sa pensée en s’exprimant ainsi. Car peut-être a-t-il entendu par « Principe » le Commencement de l’œuvre, et, par les mots de ciel et de terre, les créatures spirituelles et corporelles, non dans la perfection de leur être, mais à l’état d’ébauche informe. Je vois bien que, de ces deux sens, ni l’un, ni l’autre ne blesse la vérité. Mais lequel des deux énonce le prophète, c’est ce que je ne vois pas de même ; sans toutefois douter un seul instant que, quelle qu’ait été la pensée de cet homme divin, que je l’aie ou non présentée, c’est la vérité qu’il a vue, son expression propre qu’il lui a donnée.
\section[{Chapitre XXV, Contre ceux qui cherchent a faire prévaloir leur sentiment.}]{Chapitre XXV, Contre ceux qui cherchent a faire prévaloir leur sentiment.}
\noindent \pn{34}Que l’on ne vienne donc plus m’importuner, en disant : Moïse n’a pas eu ta pensée, mais la mienne. Encore, si l’on me disait : D’où sais-tu que le sens de Moïse est celui que tu tires de ses paroles ? Je n’aurais pas le droit de m’offenser, et je répondrais par les raisons précédentes, ou j’en développerais de nouvelles, si j’avais affaire à un esprit moins  accommodant. Mais que l’on me dise : tu te trompes, le vrai sens est le mien ; tout en m’accordant que la vérité est dans les deux ; alors, ô mon Dieu, ô vie des pauvres, vous, dont le sein exclut la contradiction, répandez en mon âme une rosée de douceur, afin que je supporte avec patience ceux qui me parlent ainsi, non qu’ils soient les hommes de Dieu, non qu’ils aient lu dans l’esprit de votre serviteur, mais parce qu’ils sont hommes de superbe, moins pénétrés de l’intelligence des pensées de Moïse, que de l’amour de leurs propres pensées ; et qu’en aiment-ils ? non pas la vérité, mais eux-mêmes : car autrement ils auraient, pour les pensées d’un autre, reconnues véritables, l’amour que j’ai pour leurs pensées, quand elles sont vraies, et je les aime, non pas comme leurs pensées, mais comme vraies ; et, à ce titre, n’étant plus à eux, mais à la vérité. Or, s’ils n’aiment dans leur opinion que la vérité, dès lors cette opinion est mienne aussi, car les amants de la vérité vivent d’un commun patrimoine.\par
Ainsi, quand ils soutiennent que leur sentiment, et non le mien, est celui de Moïse, c’est une prétention qui m’offense, et que je repousse. Leur sentiment fût-il vrai, la témérité de leur affirmation n’est plus de la science, mais de l’audace ; elle ne sort pas de la lumière de la vérité, mais des vapeurs de l’orgueil. Et c’est pourquoi, Seigneur, vos jugements sont redoutables ; car votre vérité n’est ni à moi, ni à lui, ni à tel autre ; elle est à nous tous, que votre voix appelle hautement à sa communion, avec la terrible menace d’en être privés à jamais, si nous voulons en faire notre bien privé. Celui qui prétend s’attribuer en propre l’héritage dont vous avez mis la jouissance en commun, et revendique comme son bien le pécule universel, celui-là est bientôt réduit de ce fonds social à son propre fonds, c’est-à-dire de la vérité au mensonge :\par

\begin{quoteblock}
\noindent « car celui qui professe le mensonge parle de son propre fonds (Jean, VIII, 44).»\end{quoteblock}

\noindent \pn{35}Ô mon Dieu ! ô le plus équitable des juges, et la vérité même, écoutez ma réponse à ce dur contradicteur. C’est en votre présence que je parle ; c’est en présence de mes frères qui font un légitime usage de la loi, en la rapportant à l’amour, sa fin véritable (I Tim. I, 8). Ecoutez, Seigneur, et jugez ma réponse. Voici donc ce que je lui demande avec une charité fraternelle, et dans un esprit de paix :\par
Quand nous voyons l’un et l’autre que ce que tu dis est vrai, l’un et l’autre que ce que je dis est vrai, de grâce, où le voyons-nous ? Assurément ce n’est pas en toi que je le vois, ce n’est pas en moi que tu le vois ; nous le voyons tous deux dans l’immuable vérité qui plane sur nos esprits. Et si nous sommes d’accord sur cette lumière du Seigneur qui nous éclaire, pourquoi disputons-nous sur la pensée d’un homme, qui ne saurait se voir comme cette vérité immuable ? Qu’en effet Moïse nous apparaisse et nous dise : Telle est ma pensée ; nous ne la verrions pas, nous croirions à sa parole.\par
Ainsi, suivant le conseil de l’Apôtre, gardons-nous de prendre orgueilleusement parti pour une opinion contre une autre (I Cor. IV, 6). Aimons le Seigneur notre Dieu de tout notre cœur, de toute notre âme, de tout notre esprit, et le prochain comme nous-mêmes (Deut. VI, 5 ; Matth. XXII, 37). C’est à ces deux commandements de l’amour que Moïse a rapporté les pensées de ses saintes Écritures. En pouvons-nous douter, et ne serait-ce pas démentir Dieu même que d’attribuer à son serviteur une intention différente de celle qu’affirme de lui le divin témoignage ? Vois donc ; entre tant de fouilles fécondes que l’on peut faire dans ce terrain de vérité, ne serait-ce pas une folie que de revendiquer la découverte du vrai sens de Moïse, au risque d’offenser par de pernicieuses disputes cette charité, unique fin des paroles dont nous poursuivons l’explication ?
\section[{Chapitre XXVI, Il est digne de l’écriture de renfermer plusieurs sens sous les mêmes paroles.}]{Chapitre XXVI, Il est digne de l’écriture de renfermer plusieurs sens sous les mêmes paroles.}
\noindent \pn{36}Eh quoi ! ô mon Dieu ! gloire de mon humilité et repos de mes labeurs, qui daignez écouter l’aveu de mes fautes et me les pardonner, quand vous m’ordonnez d’aimer mon prochain comme moi-même, puis-je penser que Moïse, votre serviteur fidèle ait reçu de moindres faveurs que je n’en eusse désiré moi-même et sollicité de votre grâce, si, me faisant naître en son temps pour m’élever à la hauteur de son ministère, et prenant à votre service mon cœur et ma langue, vous m’eussiez choisi pour dispensateur de ces saintes Écritures, qui devaient être dans la suite si profitable à tous les peuples, et du faîte de leur   autorité dominer universellement les paroles du mensonge et les doctrines de l’orgueil ?\par
Oui, si j’eusse été Moïse (pourquoi non ? ne sommes-nous pas sortis tous du même limon,\par

\begin{quoteblock}
\noindent « et qu’est-ce que l’homme ? est-il quelque chose si vous ne vous souvenez de lui (Ps. VIII, 5) ?) »\end{quoteblock}

\noindent , oui, si j’eusse été Moïse, et que vous m’eussiez enjoint d’écrire le livre de la Genèse, je vous aurais demandé un style doué de telles propriétés de puissance et de mesure, que les intelligences encore incapables de concevoir la création ne pussent récuser mes paroles comme au-dessus de leur portée, et que les intelligences plus élevées y trouvassent en peu de mots toute vérité qui s’offrît à leur pensée et qu’enfin, si votre lumière dévoilait à certains esprits quelques vérités nouvelles, aucune d’elles ne fût hors du sens de votre prophète.
\section[{Chapitre XXVII, Abondance de l’écriture.}]{Chapitre XXVII, Abondance de l’écriture.}
\noindent \pn{37}Une source est plus abondante en son humble bassin, pour fournir, au cours des ruisseaux qu’elle alimente, qu’aucun de ces ruisseaux qui en dérivent et parcourent de longues distances ; de même le récit de votre prophète, où vos serviteurs devaient tant puiser, fait jaillir en un filet de paroles des courants de vérité, que des saignées fécondes dirigent çà et là par de lointaines sinuosités de langage.\par
Quelques-uns, à la lecture des premières lignes, se représentent Dieu comme un homme, ou comme un être corporel, doué d’une puissance infinie, qui, par une étrange soudaineté de vouloir, aurait produit hors de lui, dans une étendue distante de lui-même, ces deux corps immenses et contenant toutes choses, l’un supérieur, l’autre inférieur. Et s’ils entendent ces mots : « Dieu dit :, Que cela soit, et cela fut, » ils se figurent une parole qui commence et finit, qui résonne et passe dans le temps, et dont le son expire à peine, que l’être appelé commence à surgir ; enfin, je ne sais quelles imaginations venues du commerce de la chair. Ceux-là sont de petits enfants. L’Écriture incline son langage jusqu’à leur bassesse, qu’elle recueille en son sein maternel. Et déjà l’édifice du salut s’élève en eux par la foi qui les assure que Dieu seul a créé tous les êtres dont l’admirable variété frappe leurs sens. Mais si l’un de ces nourrissons, dans l’orgueil de sa faiblesse, méprisant l’humilité des divines paroles, s’élance hors du berceau, le malheureux ! il va tomber, Seigneur, jetez un regard de compassion sur ce petit du passereau, il est encore sans plumes ; les passants vont le fouler aux pieds ; envoyez un de vos anges pour le reporter dans son nid, afin qu’il vive, en y demeurant tant qu’il ne sera pas en état de voler.
\section[{Chapitre XXVIII, Des divers sens qu’elle peut recevoir.}]{Chapitre XXVIII, Des divers sens qu’elle peut recevoir.}
\noindent \pn{38}Pour les autres, ces paroles ne sont plus un nid, mais un verger fertile où ils voltigent tout joyeux, à la vue des fruits cachés sous le feuillage ; et ils les cherchent, et ils les cueillent en gazouillant. Car ils découvrent à la lecture ou à l’audition de ces paroles, que votre éternelle permanence, ô Dieu, demeure au-dessus de tous les temps passés et futurs, et qu’il n’est pourtant aucune créature temporelle qui ne soit votre ouvrage.\par
Et ils voient que votre volonté, n’étant pas autre que vous-même, ne saurait subir aucun changement, et que ce n’est point par survenance de résolution soudaine et sans précédent, que vous avez, créé le monde. Ils savent que vous avez produit tout être, non pas en tirant de vous une ressemblance parfaite de vous-même, mais du néant la plus informe dissemblance, capable cependant de recevoir une forme par l’impression du caractère de votre substance. Ils savent que puisant en vous seul, chacune suivant la contenance et la propriété de son être, toutes les créatures sont très-bonnes, soit que, fixées auprès de vous, elles demeurent dans votre stabilité, soit que, successivement éloignées de vous par la distance des temps et des lieux, elles opèrent ou attestent cette splendide harmonie qui révèle votre gloire. Voilà ce qu’ils voient, et ils se réjouissent, autant qu’il leur est possible ici-bas, dans la lumière de votre vérité.\par
\pn{39}L’un en considérant le début de la Genèse, « dans le principe Dieu créa,» porte sa pensée sur l’éternelle Sagesse, ce principe qui nous parle. Un autre entend par ces mêmes paroles. le commencement de la création ; elles sont, pour lui, équivalentes à celles-ci : « Dieu créa « d’abord. » Et parmi ceux qui s’accordent à reconnaître, dans ce principe, la Sagesse par   laquelle vous avez fait le ciel et la terre, l’un prétend que, sous les noms de ciel et de terre, il faut entendre la matière primitive de l’un et de l’autre. Celui-ci n’accorde ces noms qu’aux natures distinctes et formées. Celui-là veut que le nom de ciel désigne la nature spirituelle, accomplie dans sa forme, et que le nom de terre désigne la matière corporelle dans son informité.\par
Même diversité d’opinions entre ceux qui, sous les noms de ciel et de terre, conçoivent la matière informe dont le ciel et la terre devaient être formés ; l’un y voit la source commune des créatures corporelles et intelligentes ; l’autre, de cette seule création matérielle, dont le vaste sein renferme toutes les natures évidentes à nos sens. Ceux enfin qui entendent par ces paroles des créatures disposées dans la perfection de l’ordre et de la forme, comprennent : l’un, les créatures invisibles et visibles ; l’autre, les seules visibles, c’est-à-dire ce ciel lumineux qui éblouit nos regards, et cette terre, région de ténèbres, avec tous les êtres qu’ils contiennent.
\section[{Chapitre XXIX, De combien de manières une chose peut être avant une autre.}]{Chapitre XXIX, De combien de manières une chose peut être avant une autre.}
\noindent \pn{40}Mais celui qui prend le principe dans le sens de commencement, n’a d’autre ressource pour ne pas sortir de la vérité, que d’entendre par le ciel et la terre, la matière du ciel et de la terre, c’est-à-dire de toutes les créatures intelligentes et corporelles. Car s’il entendait la création déjà formée, on aurait le droit de lui demander : Si Dieu a créé au commencement, qu’a-t-il fait ensuite ? Et ne pouvant rien trouver depuis la création de l’univers, il ne saurait décliner cette objection : « Comment Dieu a-t-il créé d’abord, s’il n’a plus créé depuis ? »\par
Que s’il prétend que la matière a été d’abord créée dans l’informité pour recevoir ensuite la forme, l’absurdité cesse ; pourvu qu’il sache bien distinguer la priorité de nature, comme l’éternité divine qui précède toutes choses ; la priorité de temps et de choix, comme celle de la fleur sur le fruit, et du fruit sur la fleur ; la priorité d’origine, comme celle du son sur le chant. Les deux priorités intermédiaires se conçoivent aisément ; il n’en est pas ainsi de la première et de la dernière. Car est-il une vue plus rare, une connaissance plus difficile, Seigneur, que celle de votre éternité immuable, créatrice de tout ce qui change, précédant ainsi tout ce qui est ?\par
Et puis, où est l’esprit assez pénétrant pour discerner, sans grand effort, quelle est la priorité du son sur le chant ? Priorité réelle ; car le chant est un son formé, et un objet peut être sans forme, et ce qui n’est pas ne peut en recevoir. Telle est la priorité de la matière sur l’objet qui en est tiré ; priorité, non d’action, puisqu’elle est plutôt passive ; non de temps, car nous ne commençons point par des sons dépourvus de la forme mélodieuse, pour les dégrossir ensuite et les façonner selon le rhythme et la mesure, comme on travaille le chêne ou l’argent dont on veut tirer un coffre ou un vase. Ces dernières matières précèdent, en effet, dans le temps, les formes qu’on leur donne ; mais il n’en est pas ainsi du chant. L’entendre, c’est entendre le son : il ne résonne pas d’abord sans avoir de forme, pour recevoir ensuite celle du chant. Tout ce qui résonne passe, et il n’en reste rien que l’art puisse reprendre et ordonner. Ainsi le chant roule dans le son, et le son est sa matière, car c’est le son même qui se transforme en chant ; et, comme je le disais, la matière ou le son précède la forme ou le chant ; non comme puissance productrice, car le son n’est pas le compositeur du chant, mais il dépend de l’âme harmonieuse qui le produit à l’aide de ses organes. Il n’a ni la priorité du temps, car le chant et le son marchent de compagnie ; ni la priorité de choix, car le son n’est pas préférable au chant, puisque le chant est un son revêtu de charme : il n’a que la priorité d’origine, car ce n’est pas le chant qui reçoit la forme pour devenir son, mais le son pour devenir chant.\par
Comprenne qui pourra par cet exemple, que ce n’est qu’en tant qu’origine du ciel et de la terre que la matière primitive a été créée d’abord et appelée le ciel et la terre ; et qu’il n’y a point là précession de temps, parce qu’il faut la forme pour développer le temps : or, elle était informe, mais néanmoins déjà liée au temps. Et toutefois, quoique placée au dernier degré de l’être (l’informité étant infiniment au-dessous de toute forme), il est impossible d’en parler sans lui donner une priorité de temps fictive. Enfin, elle-même est précédée par l’éternité du Créateur, qui de néant la fait être.  
\section[{Chapitre XXX, L’Écriture veut être interprétée en Esprit de charité.}]{Chapitre XXX, L’Écriture veut être interprétée en Esprit de charité.}
\noindent \pn{41}Que la vérité même établisse l’union entre tant d’opinions de vérité différente ! Que la miséricorde du Seigneur nous permette de faire un légitime usage de la loi, en la rapportant au précepte de l’amour ! Ainsi donc, si l’on me demande quel est, suivant moi, le sens de Moïse, ce n’est pas l’objet de mes confessions. Si je ne le publie pas devant vous, c’est que je l’ignore. Et je sais pourtant que toutes ces opinions sont vraies, sauf ces pensers charnels, dont j’ai parlé. Et ceux qui tombent dans ces pensers sont néanmoins du nombre de ces petits d’heureuse espérance, qui ne s’effarouchent pas des paroles sacrées ; ces paroles. si sublimes dans leur humilité, si prodigues dans leur parcimonie. Pour nous, qui, j’ose le dire, n’interprétons le texte saint que suivant la vérité, si c’est pour elle-même et non pour la vanité de nos sentiments que notre cœur soupire, aimons-nous mutuellement ; aimons-nous en vous, ô Dieu, source de vérité, et honorons votre serviteur, oracle de votre Esprit, dispensateur de vos Écritures ; et que notre vénération nous préserve de douter qu’en les écrivant sous votre dictée, il n’ait aperçu les lumières les plus vives et les fruits les meilleurs.
\section[{Chapitre XXXI, Moïse a pu entendre tous les sens véritables qui peuvent se donner à ses paroles.}]{Chapitre XXXI, Moïse a pu entendre tous les sens véritables qui peuvent se donner à ses paroles.}
\noindent \pn{42}Tu me dis : « Le sens de Moïse est le « mien ; » et il me dit : « Non, le sens de Moïse est le mien ; » et moi je dis avec plus de piété : Pourquoi l’un et l’autre ne serait-il pas le sien, si l’un et l’autre est véritable ? Et j’en dis autant d’un troisième, d’un quatrième, d’un autre sens quelconque avoué de la vérité ; pourquoi refuserais-je de croire qu’ils ont été vus par ce grand serviteur du seul Dieu, dont la parole toute divine se prête à la variété de tant d’interprétations vraies ?\par
Pour moi, je le déclare hardiment, et du fond du cœur, si j’écrivais quelque chose qui dût être investi d’une autorité suprême, j’aimerais mieux contenir tous les sens raisonnables qu’on pourrait donner à mes paroles, que de les limiter à un sens précis, exclusif de toute autre pensée, n’eût-elle même rien de faux qui pût blesser la mienne. Loin de moi, mon Dieu, cette témérité de croire qu’un si grand prophète n’eût pas mérité de votre grâce une telle faveur ! Oui, il a eu en vue et en esprit, lorsqu’il traçait ces paroles, tout ce que nous avons pu découvrir de vrai ; toute vérité qui nous a fui ou nous fuit encore, et qui toutefois s’y peut découvrir.
\section[{Chapitre XXXII, Tous les sens véritables prévus par le Saint-Esprit.}]{Chapitre XXXII, Tous les sens véritables prévus par le Saint-Esprit.}
\noindent \pn{43}Enfin, Seigneur, qui n’êtes pas chair et sang, mais Dieu, si l’homme n’a pas tout vu, votre Esprit Saint, mon guide vers la terre des vivants (Ps. CXLII, 10), pouvait-il ignorer tous les sens de ces paroles dont vous deviez briser les sceaux dans l’avenir, quand même votre interprète ne les eût entendues qu’en l’un des sens véritables qu’elles admettent ? Et, s’il est ainsi, la pensée de Moïse est sans doute la plus excellente. Mais, ô mon Dieu, ou faites-nous la connaître, ou révélez-nous cette autre qu’il vous plaira, et, soit que vous nous découvriez le même sens que vous avez dévoilé à votre serviteur, soit qu’à l’occasion de ces paroles, vous en découvriez un autre, que votre vérité soit notre aliment et nous préserve d’être le jouet de l’erreur.\par
Est-ce assez de pages, Seigneur mon Dieu, en est-ce assez sur ce peu de vos paroles ? Et quelles forces et quel temps suffiraient à un tel examen de tous vos livres ? Permettez-moi donc de resserrer les témoignages que j’en recueille à la gloire de votre nom ; que, dans cette multiplicité de sens qui se sont offerts et peuvent s’offrir encore à ma pensée, votre inspiration fixe mon choix sur un sens vrai, certain, édifiant, afin que, s’il m’arrive de rencontrer celui de votre antique ministre, but où mes efforts doivent tendre, cette fidèle confession vous en rende grâces ; sinon, permettez-moi du moins d’exprimer ce que votre vérité voudra me faire publier sur sa parole, comme elle lui a inspiré à lui-même la parole qui lui a plu. 
\chapterclose


\chapteropen
 \chapter[{XIII. Lumière de la Création}]{XIII. Lumière de la Création}\phantomsection
\label{XIII}\renewcommand{\leftmark}{XIII. Lumière de la Création}


\begin{argument}\noindent Toute créature tient l’être de la pure bonté de Dieu. — Il découvre dans les premières paroles de la Genèse et la Trinité de Dieu et la propriété de la personne du Saint-Esprit. — Image de la Trinité dans l’Homme. — Dieu procède dans l’institution de l’Église comme dans la création du monde. — Sens mystique de la création.
\end{argument}


\chaptercont
\section[{Chapitre premier, Invocation. — gratuite munificence de Dieu.}]{Chapitre premier, Invocation. — gratuite munificence de Dieu.}
\noindent \pn{1}Je vous invoque, ô mon Créateur, mon Dieu et ma miséricorde, qui avez gardé mon souvenir quand j’avais perdu le vôtre. Je vous appelle dans mon âme, et vous la préparez à vous recevoir en lui inspirant ce vif désir de votre possession. Oh ! répondez aujourd’hui à cet appel que vous avez devancé, quand vos cris réitérés, venant de si loin à mon oreille, me pressaient de me retourner et d’appeler à moi Celui qui m’appelait à lui. Seigneur, vous avez effacé tous mes péchés, afin de n’avoir point à solder les œuvres de mon infidélité, et vous avez prévenu mes œuvres méritantes, afin de me rendre selon le bien opéré en moi par vos mains, dont je suis l’ouvrage. Car vous étiez avant que je fusse, et je n’étais rien à qui vous pussiez donner d’être. Et me voilà toutefois, je suis par votre bonté qui a devancé tout ce que vous m’avez donné d’être, tout ce dont vous m’avez fait. Vous n’aviez pas besoin de moi, et je ne suis pas tel que ce peu de bien que je suis vous seconde, mon Seigneur et mon Dieu ; que mes services vous soulagent, comme si vous vous lassiez en agissant ; que votre puissance souffrit de l’absence de mon hommage ; que vous réclamiez mon culte, comme la terre réclame ma culture, sous peine de stérilité ; mais vous voulez mes soins, vous voulez mon culte, afin que je trouve en vous le bien de mon être ; car vous m’avez donné l’être qui me rend capable de ce bien.
\section[{Chapitre II, Toute créature tient l’être de la pure bonté de Dieu.}]{Chapitre II, Toute créature tient l’être de la pure bonté de Dieu.}
\noindent \pn{2}C’est de la plénitude de votre bonté que vos créatures ont reçu l’être ; vous avez voulu qu’un bien fût qui ne pût procéder que de vous, inutile, inégal à vous-même. Etiez-vous donc redevable au ciel, à la terre, que vous avez créés dans le principe ? Je le demande à ces créatures spirituelles et corporelles que vous avez formées dans votre sagesse, leur étiez-vous redevable de cet être, même imparfait, même informe, dans l’ordre spirituel ou corporel, être tendant au désordre et à l’éloignement de votre ressemblance ? L’être spirituel, fût-il informe, est supérieur au corps formé ; et cet être corporel, fût-il informe, est supérieur au néant ; et tous deux demeureraient comme une esquisse informe de votre Verbe, si ce même Verbe ne les eût rappelés à votre unité, en leur donnant la forme, et cette excellence qu’ils tiennent de votre souveraine bonté. Leur étiez-vous redevable de cette informité même, où ils ne pouvaient être que par vous ?\par
\pn{3}Etiez-vous redevable à la matière corporelle de l’être, même invisible et sans ordre ? car elle n’eût pas même été cela, si vous ne l’eussiez faite ; et n’étant pas, comment pouvait-elle mériter de vous son être ? Et cette ébauche de créature spirituelle, lui étiez-vous redevable de cet être même ténébreux et flottant, semblable à l’abîme, dissemblable à vous, où elle serait encore, si votre Verbe ne l’eût ramenée à son principe, et, en l’illuminant, ne l’eût faite lumière, non pas égale, mais conforme à votre égalité formelle ? Pour un   corps, être et être beau, n’est pas tout un ; autrement tous seraient beaux : ainsi, pour l’esprit créé, ce n’est pas tout un que de vivre, et de vivre sage ; autrement il serait immuable dans sa sagesse. Mais il lui est bon de s’attacher toujours à vous, de peur qu’abandonné de la lumière dont il se retire, il ne retombe dans cette vie de ténèbres, semblable à l’abîme. Et nous aussi, créatures spirituelles par notre âme, autrefois loin de vous, notre lumière,\par

\begin{quoteblock}
\noindent « n’avons-nous pas été ténèbres en cette « vie (Ephés. V, 8) »\end{quoteblock}

\noindent et ne luttons-nous pas encore contre les dernières obscurités de cette nuit jusqu’au jour où nous serons justice dans votre Fils, élevés à la hauteur des montagnes saintes, après avoir été une profondeur d’abîme sondée par vos jugements (Ps. XXXV, 7) ?
\section[{Chapitre III, Tout procède de la grâce de Dieu.}]{Chapitre III, Tout procède de la grâce de Dieu.}
\noindent \pn{4}Quant à ces paroles que vous dites au début de la création :\par

\begin{quoteblock}
\noindent « Que la lumière soit, et la lumière fut (Gen. I, 3), »\end{quoteblock}

\noindent je les applique sans inconvénient à la créature spirituelle, parce qu’elle était déjà vie quelconque, pour recevoir votre lumière. Mais si elle n’avait pas mérité de vous cette vie capable de votre lumière, avait-elle mérité davantage le don que vous lui en avez fait ? Car son informité n’eût pu vous plaire, si elle ne fût devenue lumière, non par nature, mais par l’intuition de votre 1umière illuminante, par son union avec elle, afin que ces préludes de vie et cette béatitude de vie, elle ne les dût qu’à votre grâce, qui la tourne, par un heureux changement, vers ce qui est également incapable de pis et de mieux, vers vous, seul être simple, pour qui vivre c’est vivre heureux, parce que vous êtes à vous-même votre béatitude.
\section[{Chapitre IV, Dieu n’avait pas besoin des créatures.}]{Chapitre IV, Dieu n’avait pas besoin des créatures.}
\noindent \pn{5}Que manquerait-il donc à votre félicité, félicité qui est vous-même, quand toutes ces créatures demeureraient encore dans le néant ou l’informité ? Aviez-vous besoin d’elles ? et n’est-ce point par la plénitude de votre bonté que vous les avez faites ? Et votre joie était-elle intéressée au complément de leur être ? Loin que vous soyez imparfait, pour attendre votre perfection de la leur , parfait comme vous l’êtes, leur imperfection vous déplaît, et vous les perfectionnez pour qu’elles vous plaisent. Car votre Esprit-Saint était porté au-dessus des eaux (Gen. 1, 2), et non par les eaux, comme s’il se fût reposé sur elles, lui qui fait reposer en soi ceux en qui l’on dit « qu’il repose (Is. XI, 2) » Mais c’est votre volonté incorruptible, immuable, se suffisant à elle-même, qui était portée au-dessus de cette vie, votre création, en qui la vie et la béatitude ne sont pas même chose, puisqu’elle ne laisse pas de vivre dans la fluctuation de ses ténèbres, et qu’il lui faut se tourner vers son auteur, puiser de plus en plus la vie à la source de la vie, voir la lumière dans sa lumière (Ps. XXXV, 10), et en recevoir perfection, gloire, béatitude.
\section[{Chapitre V, De la trinité.}]{Chapitre V, De la trinité.}
\noindent \pn{6}Et maintenant m’apparaît comme en énigme votre Trinité, mon Dieu. C’est dans le Principe de votre sagesse, qui est notre sagesse, ô Père ! née de vous, égale et coéternelle à vous, c’est dans votre Fils que vous avez fait le ciel et la terre. Et que n’ai-je pas dit sur le ciel du ciel, sur la terre invisible et sans forme, sur cet abîme de ténèbres, qui serait livré à toutes les tourmentes de l’informité spirituelle, s’il ne se fût fixé devant Celui par qui il était vie quelconque, et dont la lumière allait répandre sur cette vie la forme et la beauté, pour qu’elle devînt ce ciel du ciel, créé depuis, et résidant entre les eaux ? Et déjà, par ce nom de Dieu, j’atteignais le Père, qui a tout fait, par celui de Principe, le Fils en qui il a tout fait ; et, dans ma ferme croyance que mon Dieu est une Trinité, je consultais les paroles saintes, qui me répondent : « Et l’Esprit était porté au-dessus des eaux. » Et voilà mon Dieu-Trinité, Père, Fils et Saint-Esprit, seul Dieu, Créateur de toutes les créatures.
\section[{Chapitre VI, Comment l’esprit de Dieu était porté au-dessus des eaux.}]{Chapitre VI, Comment l’esprit de Dieu était porté au-dessus des eaux.}
\noindent \pn{7}Mais, ô lumière de vérité, je place près de vous ce cœur qui ne m’enseignerait que vanités ; dissipez ses ténèbres, et dites-moi, je vous en conjure par votre charité, notre mère,   dites-moi, je vous en supplie, pourquoi n’est-ce qu’après avoir nommé le ciel et la terre, invisible et sans forme, et les ténèbres répandues sur l’abîme, que votre Écriture nomme l’Esprit-Saint ? Etait-il donc nécessaire, pour nous en suggérer la connaissance, de le représenter comme « porté au-dessus, » en désignant d’abord au-dessus de quoi ? Ce n’était ni au-dessus du Père, ni au-dessus du Fils, ni sans doute au-dessus de rien. Il fallait donc indiquer d’abord au-dessus de quoi il était porté, lui dont il était impossible de parler, sans le dire « porté. » Mais pourquoi ?
\section[{Chapitre VII, Effets du Saint-Esprit.}]{Chapitre VII, Effets du Saint-Esprit.}
\noindent \pn{8}Et maintenant suive qui pourra de l’esprit le vol de l’Apôtre dans cette parole sublime :\par

\begin{quoteblock}
\noindent « La charité se répand dans nos cœurs « par le Saint-Esprit qui nous est donné ((Rom. V, 5) ; »\end{quoteblock}

\noindent soit qu’il nous enseigne les voies spirituelles et les voies suréminentes de l’amour, soit qu’il fléchisse le genou devant vous, pour nous obtenir la grâce d’être initiés\par

\begin{quoteblock}
\noindent « à la science suréminente de la charité du Christ (Ephés. III, 14-19). »\end{quoteblock}

\noindent Et voilà pourquoi, suréminent dès le principe, il paraissait au-dessus des eaux.\par
Mais à qui parler ? mais comment parler de ce poids de concupiscence qui gravite vers l’abîme, et de l’attraction sublime de la charité par la vertu de votre Esprit, qui « planait sur « les eaux ? » Quel sera mon auditeur ? quelle sera ma parole ? On plonge, on surnage ; et il n’y a là ni fond, ni rive. Quelle similitude plus dissemblable ? Ce sont nos affections, ce sont nos amours, c’est l’impureté de notre esprit que précipite l’amour des soins de la terre ; et c’est la sainteté de votre Esprit qui nous soulève vers le ciel, par l’amour de la paix éternelle, afin que nos cœurs s’élèvent en haut jusqu’à vous, où votre Esprit plane sur les eaux, et que notre âme, après la traversée de ces eaux mobiles de la vie (Ps. CXXIII, 5), aborde à la suréminence du repos.
\section[{Chapitre VIII, L’union avec Dieu, unique félicité des êtres intelligents.}]{Chapitre VIII, L’union avec Dieu, unique félicité des êtres intelligents.}
\noindent \pn{9}L’esprit de l’ange, l’âme de l’homme se sont dissipés dans leur chute comme l’eau qui s’écoule, et ils ont signalé l’abîme ténébreux où serait ensevelie toute créature spirituelle, si vous n’eussiez dit au comencement : « Que la lumière soit ! » ralliant à vous l’obéissance des esprits habitants de la cité céleste, pour assurer leur paix au sein de votre Esprit qui demeure immuable au-dessus de tout ce qui change. Autrement ce ciel du ciel ne serait par lui-même qu’abîme et ténèbres ;\par

\begin{quoteblock}
\noindent « et maintenant il est lumière dans le Seigneur (Ephés. V, 8). »\end{quoteblock}

\noindent Et, en vérité, cette inquiétude malheureuse des intelligences déchues de votre lumière, leur splendide vêtement, et réduites aux haillons de leurs ténèbres, parle assez haut ; témoin éloquent de l’excellence où. vous avez élevé cette créature raisonnable, qui ne saurait se suffire : car il ne lui faut rien moins que vous-même pour qu’elle ait sa béatitude et son repos.\par

\begin{quoteblock}
\noindent « Vous êtes, ô mon Dieu, la lumière de nos ténèbres (Ps. XVII, 29), »\end{quoteblock}

\noindent notre robe de gloire ;\par

\begin{quoteblock}
\noindent « et notre nuit rayonne comme le jour à son midi (Ps. CXXXVIII, 12). »\end{quoteblock}

\noindent Oh ! donnez-vous à moi, mon Dieu ! rendez-vous à moi ! Je vous aime ; et si mon amour est encore trop faible, rendez-le plus fort. Je ne saurais mesurer ce qu’il manque à mon amour ; et combien il est au-dessous du degré qu’il doit atteindre, pour que ma vie se précipite dans vos embrassements, et ne s’en détache point qu’elle n’ait disparu tout entière dans les plus secrètes clartés de votre visage (Ps. XXX, 21). Tout ce que je sais, c’est que partout ailleurs qu’en vous, hors de moi, comme en moi, je ne trouve que malaise, et toute richesse qui n’est pas mon Dieu, n’est pour moi qu’indigence.
\section[{Chapitre IX, Pourquoi il est dit, seulement du Saint-Esprit, qu’il était porté sur les eaux.}]{Chapitre IX, Pourquoi il est dit, seulement du Saint-Esprit, qu’il était porté sur les eaux.}
\noindent \pn{10}Mais le Père, mais le Fils, n’étaient-ils pas portés au-dessus des eaux ? Si l’on se fait une idée de corps et d’espace, ces paroles ne conviennent plus même au Saint-Esprit. Si l’on y voit l’immuable suréminence de la divinité qui demeure au-dessus de tout ce qui change, le   Père, et le Fils ; et le Saint-Esprit étaient ensemble portés sur les eaux. Pourquoi donc l’Écriture ne parle-t-elle que de votre Esprit ? pourquoi parle-t-elle de lui seul, comme s’il y avait lieu là où le lieu n’est pas, en celui de qui seul il a été dit qu’il est votre don ? Le don où nous jouissons du repos, où nous jouissons de vous-même ; repos des âmes, lieu des esprits !\par
C’est là où nous élève l’amour ; et votre divin Esprit retire notre humilité des portes de la mort (Ps. IX, 5) ; et\par

\begin{quoteblock}
\noindent « notre paix est dans notre bonne volonté (Luc, II, 14) »\end{quoteblock}

\noindent . Le corps tend à son lieu par son poids ; et ce poids ne tend pas seulement en bas, mais au lieu qui lui est propre. La pierre tombe ; le feu s’élance ; l’un et l’autre gravite suivant son poids et suivant son centre. L’huile versée dans l’eau monte au-dessus de l’eau ; l’eau versée dans l’huile descend au-dessous de l’huile ; l’un et l’autre suit son poids et cherche son centre. Hors de l’ordre, trouble ; dans l’ordre, repos. Mon poids, c’est mon amour ; où que je tende, c’est lui qui m’emporte. C’est votre don, c’est votre Esprit qui allume, qui volatilise notre cœur. Il nous embrase et nous enlève. Nous montons à l’échelle de l’âme (Ps. LXXXIII, 6), en chantant le cantique des degrés. C’est le feu de l’amour, c’est votre feu divin qui nous consume et nous ravit au centre de la paix, au sein de Jérusalem ; et\par

\begin{quoteblock}
\noindent « je trouve ma joie dans cette heureuse promesse : Nous irons à la maison du Seigneur (Ps. CXXXI, 1). »\end{quoteblock}

\noindent Et c’est la bonne volonté qui nous y fait une place ; et nous n’avons plus rien à vouloir, que cette demeure éternelle.
\section[{Chapitre X, Bonheur des pures intelligences.}]{Chapitre X, Bonheur des pures intelligences.}
\noindent \pn{11}Ô béatitude de la créature qui n’a jamais connu d’autre état que cette félicité, où elle ne se fût jamais élevée d’elle-même, si, à l’instant immédiat de sa création, votre Don, porté sur toutes choses muables, ne l’eût exaltée à l’appel de votre voix.\par

\begin{quoteblock}
\noindent « Que la lumière soit, et la « lumière fut (Gen. I, 3). »\end{quoteblock}

\noindent En nous, il y a distinction de temps : temps où nous sommes ténèbres ; temps où nous devenons lumière (Ephés. V, 8). Mais, en parlant de ces pures intelligences, l’Écriture ne fait qu’indiquer ce qu’elles eussent été sans l’illumination divine ; et elle les suppose à l’état de fluctuation ténébreuse, pour nous signaler la cause de leur gloire surnaturelle : c’est-à-dire leur union lumineuse avec la lumière sans ombre et sans défaillance. Entende qui peut ; qui ne peut, vous invoque ! — Car, enfin, que me veut-on ? Suis-je la lumière qui éclaire tout homme venant au monde (Jean 1,9) ?
\section[{Chapitre XI, Image de la trinité dans l’Homme.}]{Chapitre XI, Image de la trinité dans l’Homme.}
\noindent \pn{12}Où est l’homme qui comprend la toute-puissante Trinité ? où est l’homme qui n’en parle ? et peut-on dire qu’il en parle ? Bien rare est l’intelligence qui en parle avec la science de sa parole. Et l’on conteste, et l’on dispute ; et c’est un mystère qui demeure voilé aux âmes où la paix n’est pas. Je voudrais que les hommes observassent en eux-mêmes un triple phénomène ; simplitude infiniment différente de la Trinité sainte, mais que j’offre à leur méditation, pour leur faire sentir et reconnaître l’infini de la distance. Ce triple phénomène, le voici : être, connaître, vouloir : car je suis, je connais, je veux : je suis celui qui connaît et qui veut. Je connaît que je suis et que je veux, et je veux être et connaître.\par
Comprenne qui pourra combien notre âme est inséparable de ces trois phénomènes, qui tous trois ne font qu’une même vie, qu’une même raison, qu’une même essence, inséparablement distinctes. Homme, te voilà en présence de toi-même ; regarde en toi ; vois, et réponds moi !\par
Et si tu trouves quelque lueur dans ces mystères de ton être, ne crois pas en avoir pénétré plus avant dans les mystères de l’Être immuable au-dessus de tout, immuable dans son être, immuable dans sa connaissance, immuable dans sa volonté : car, est-ce à cause de cette triplicité, que Dieu est Trinité ; ou cette triplicité réside-t-elle en chaque personne divine, chacune étant unité-trinaire ; ou bien, dans le cercle incompréhensible, infini, d’une simplicité multiple, est-il unité féconde, principe, connaissance et fin de soi-même, qui se suffit immuablement ? Quel esprit aurait la force de dégager cette terrible inconnue ? Quelle parole, quel sentiment seraient exempts de témérité ?  
\section[{Chapitre XII, Dieu procède en l’institution de l’Église comme dans la création du monde.}]{Chapitre XII, Dieu procède en l’institution de l’Église comme dans la création du monde.}
\noindent \pn{13}Poursuis ta confession, ô ma foi ; dis au Seigneur, ton Dieu : Saint, saint, saint ! ô mon Seigneur ! ô mon Dieu ! C’est en votre nom que nous sommes baptisés, Père, Fils et Saint-Esprit ! c’est en votre nom que nous baptisons, Père, Fils et Saint-Esprit ! Car Dieu a fait en nous, par son Christ, un nouveau ciel, une nouvelle terre : c’est-à-dire les membres spirituels et les membres charnels de son Église ; et notre terre, avant que la doctrine sainte ne l’eût douée de sa forme, était invisible aussi ; elle était informe et couverte des ténèbres de l’ignorance,\par

\begin{quoteblock}
\noindent « parce que vous avez châtié l’iniquité de l’homme (Ps. XXXVIII, 12)\end{quoteblock}


\begin{quoteblock}
\noindent dans le profond abîme de vos jugements (Ps. XXXV, 7). »\end{quoteblock}

\noindent Mais votre Esprit-Saint est porté sur les eaux, et votre miséricorde n’abandonne pas notre misère ; et vous dites :\par

\begin{quoteblock}
\noindent « Que la lumière soit ! — Faites pénitence ; le royaume des cieux est proche (Matth. III, 12) ! — Faites pénitence ; que la lumière soit ! »\end{quoteblock}

\noindent Et, dans le trouble de notre âme, « nous nous sommes souvenus de vous, Seigneur, aux bords du Jourdain, » auprès de la montagne élevée à votre hauteur, et qui s’est abaissée pour nous (Ps. XLI, 7). Et nos ténèbres nous ont fait horreur ; et nous nous sommes tournés vers vous ; et la lumière a été faite.\par

\begin{quoteblock}
\noindent « Et nous voilà, ténèbres autrefois, maintenant lumière dans le Seigneur (Ephés. 5,8). »\end{quoteblock}

\section[{Chapitre XIII, Notre renouvellement n’est jamais parfait en cette vie.}]{Chapitre XIII, Notre renouvellement n’est jamais parfait en cette vie.}
\noindent \pn{14}Et nous ne sommes encore lumière que par la foi, et non par la claire vue (II Cor. V, 7).\par

\begin{quoteblock}
\noindent « Car notre salut est en espérance ; or, l’espérance qui se voit n’est plus espérance (Rom. VIII, 24).»\end{quoteblock}

\noindent C’est encore « un abîme qui appelle un abîme, » mais par la voix de vos cataractes (Ps. XLI, 8). Il est encore abîme, celui qui dit :\par

\begin{quoteblock}
\noindent « Je n’ai pu vous parler comme à des êtres spirituels, mais comme à des êtres charnels (I Cor. III, 1).»\end{quoteblock}

\noindent Et lui-même reconnaît qu’il n’a pas encore touché le but, et oubliant tout ce qui est derrière, il tend à ce qui est devant lui (Philip. III, 13) ; il gémit sous le fardeau de malheur, et son âme est altérée du Dieu vivant, comme le cerf soupire après l’eau des fontaines ; et il s’écrie :\par

\begin{quoteblock}
\noindent « Oh ! quand arriverai-je (Ps. XLI, 2,3) !»\end{quoteblock}

\noindent Et il aspire à être revêtu de sa céleste demeure (Ps. XXXV, 7), et il appelle les ténèbres de l’abîme inférieur et leur dit :\par

\begin{quoteblock}
\noindent « Ne vous conformez pas au siècle, mais réformez-vous dans le renouvellement de l’esprit (Rom. XII, 20). Ne soyez pas comme les enfants « sans intelligence ; mais, comme les plus petits d’entre eux, soyez sans malice, pour arriver à la perfection de l’esprit (I Cor. XIV, 20). »\end{quoteblock}


\begin{quoteblock}
\noindent « Ô Galates insensés ! s’écrie-t-il, qui vous a donc fascinés (Gal. III, 1) ? »\end{quoteblock}

\noindent Mais ce n’est plus sa voix, c’est la vôtre qui retentit ; la vôtre, ô Dieu, qui du haut des cieux avez fait descendre votre Esprit (Act. II, 2) par Celui qui monté dans les cieux a ouvert les cataractes de ses grâces, afin qu’un fleuve de joie inondât votre cité sainte (Ps XLV, 5). C’est après elle que soupire ce fidèle ami de l’époux, qui possède déjà les prémices de l’esprit ; mais il gémit encore dans l’attente de l’adoption céleste, qui doit affranchir son corps (Rom. VIII, 23) ; il soupire après la patrie. Il est membre de l’épouse du Christ, il est jaloux pour elle : il est l’ami de l’époux, et il est jaloux, non pour soi, mais pour lui (Jean III, 19) ; et ce n’est point par sa voix, mais par celle de vos torrents, qu’il appelle à lui cet autre abîme (Ps. XLI,8) objet de sa sainte jalousie. Il craint que le serpent, dont la ruse séduisit Eve, ne nous détourne de cette chasteté spirituelle que nous devons à notre époux, votre Fils unique (II Cor. II, 3). Oh ! quelle sera la splendeur de sa lumière, lorsque nous le verrons tel qu’il est (I Jean, III, 2) ; et qu’elles seront taries toutes ces larmes, qui sont le pain de mes jours et de mes nuits ; car on ne cesse de me dire : Où est ton Dieu (Ps. XLI, 4) ?
\section[{Chapitre XIV, L’âme est soutenue par la foi et l’espérance.}]{Chapitre XIV, L’âme est soutenue par la foi et l’espérance.}
\noindent \pn{15}Et moi-même je m’écrie souvent : Où êtes-vous, mon Dieu, où êtes-vous ? Et je respire quelques instants en vous, quand mon âme répand hors d’elle-même l’effusion de son allégresse et de vos louanges (Ps. XLI, 5). Mais elle demeure triste, parce qu’elle retombe et devient abîme, ou plutôt elle sent qu’elle est abîme encore. Et, ce flambeau dont vous éclairez mes pas dans la nuit, la foi me dit :\par

\begin{quoteblock}
\noindent « Pourquoi es-tu triste, ô mon âme, et pourquoi me   troubles-tu ? Espère dans le Seigneur (Ps. XLII, 5, 6).»\end{quoteblock}

\noindent Son Verbe est la lampe qui luit sur ton chemin (Ps. CXVIII, 105). Espère et persévère, jusqu’à ce que la nuit, mère des impies, soit passée, et avec elle la colère du Seigneur ; colère dont nous fûmes enfants nous-mêmes, alors que nous étions ténèbres. Et nous traînons la fin de notre nuit en ce corps que le péché a fait mourir (Rom. VIII, 10), dans l’attente de l’aube qui dissipera toutes les ombres (Cant. II, 17).\par
Espère dans le Seigneur. Au lever de ce jour, je serai debout pour le contempler, et j’en publierai à jamais la splendeur. Au matin de l’éternité je serai debout, et je verrai le Dieu de mon salut (Ps. V, 5 ; XLII, 5) ; celui qui vivifiera nos corps mortels par l’Esprit, cet hôte intérieur (Rom. VIII, 11), porté dans sa miséricorde sur le flot de nos ténèbres ; celui de qui nous avons reçu dans l’exil de cette vie le gage d’être à l’avenir lumière ; qui nous sauve dès ici-bas par l’espérance, et de ténèbres que nous étions, nous transforme en fils de jour et de lumière (II Cor. I, 22 ; Ephés. V, 8 ; Rom. VIII, 24 ; I Thess. V, 5). Seul en ce sombre crépuscule de la connaissance humaine, vous pouvez distinguer les cœurs et les éprouver, pour appeler la lumière jour, et les ténèbres nuit (Gen. I, 5). Eh ! quel autre que vous peut faire ce discernement des âmes ? Qu’avons-nous, que nous n’ayons reçu de vous (I. Cor. IV, 7) ? Ne sommes-nous pas une même argile dont vous formez ici des vases d’honneur, là des vases d’ignominie (Rom. IX, 21) ?
\section[{Chapitre XV,L’écriture Sainte comparée au firmament et les anges aux eaux supérieures (Gen. i, 6).}]{Chapitre XV,L’écriture Sainte comparée au firmament et les anges aux eaux supérieures (Gen. i, 6).}
\noindent \pn{16}Mais quel autre que vous, Seigneur, a étendu au-dessus de nous ce firmament divin de vos Écritures ?\par

\begin{quoteblock}
\noindent « Le ciel sera roulé comme un livre (Is. XXIV, 4), et il est maintenant étendu comme une peau (Ps. CIII, 2) »\end{quoteblock}

\noindent Seigneur, l’autorité de votre divine Écriture n’en est que plus sublime, quand les mortels, par qui vous l’avez publiée, ont passé par la mort. Et vous savez, Seigneur, que vous avez revêtu de peaux les premiers hommes, devenus mortels par le péché (Gen. III, 21). Et vous avez étendu comme une peau le firmament de vos saints livres, ces paroles d’une concordance admirable, que vous avez posées au-dessus de nous par le ministère d’hommes mortels. Et leur mort même a étendu avec plus de force le firmament d’autorité de vos paroles qu’ils ont annoncées : il est étendu sur ce monde inférieur, plus fort et plus haut que pendant leur vie. Car vous n’aviez pas encore étendu ce ciel comme une peau ; vous n’aviez pas encore rempli la terre du bruit de leur mort.\par
\pn{17}Oh ! faites-nous voir, Seigneur, ces cieux, ouvrage de vos mains. Dissipez ce nuage dont vous les voilez à nos yeux. Là résident ces oracles qui inspirent la sagesse aux petits enfants (Ps. XVIII, 8). Exaltez votre gloire, mon Dieu, par la bouche de ces enfants à la mamelle, qui bégaient à peine. Non, je ne sache pas d’autres livres plus puissants pour anéantir l’orgueil, pour détruire l’ennemi (Ps. VIII, 4, 3) qui se retranche contre votre miséricorde dans la justification de ses crimes. Non, Seigneur, je ne connais point de paroles plus chastes, plus persuasives d’humilité, plus capables de m’apprivoiser à votre joug, et d’engager mon cœur à un service d’amour. Père infiniment bon, initiez-moi à leur intelligence ; accordez cette grâce à ma soumission, puisque vous ne les avez si solidement affermies qu’en faveur des âmes soumises.\par
\pn{18}Il est d’autres eaux au-dessus de ce firmament ; eaux immortelles, je crois, et pures de la corruption de la terre. Que ces eaux louent votre nom ! que, par delà les cieux vos louanges s’élèvent de ces chœurs angéliques, qui n’ont pas besoin de considérer et de lire notre Firmament pour connaître votre Verbe ! Car ils voient votre face (Matth. XVIII, 10), et lisent sans succession de syllabes les décrets de votre éternelle volonté. C’est à la fois lecture, élection et dilection : ils lisent toujours, et ce qu’ils lisent ne passe point ; ils lisent par élection et par dilection l’immuable stabilité de votre conseil : livre toujours ouvert, et qui ne sera jamais roulé, parce que vous êtes vous-même ce livre, et que vous l’êtes éternellement ; parce que vous avez créé vos anges supérieurs à ce firmament, que vous avez affermi au-dessus de l’infirmité des peuples de la terre, afin que cette infirmité, levant ses regards jusqu’à lui, y lise la miséricorde, qui daigne annoncer dans le temps le Créateur des temps : car\par

\begin{quoteblock}
\noindent « votre miséricorde, Seigneur, est dans le ciel, et votre vérité s’élève jusqu’aux nues (Ps. XXXV, 6).»\end{quoteblock}

\noindent Les nues passent, mais le ciel demeure ; les   prédicateurs de votre parole passent de cette vie dans une autre, mais votre Écriture s’étend sur tous les peuples jusqu’à la fin des siècles.\par

\begin{quoteblock}
\noindent « Le ciel même et la terre passeront, mais vos paroles ne passeront point (Matth. XXIV, 35). »\end{quoteblock}

\noindent — Cette peau sera pliée, et l’herbe qu’elle couvrait se flétrira dans sa beauté, mais votre Verbe demeure éternellement (Is. XL, 6). Nous ne le voyons maintenant que dans l’énigme des nues et le miroir du ciel (I Cor. XIII, 12) ; il ne nous apparaît pas tel qu’il est ; car nous-mêmes, malgré l’amour de votre Fils pour nous, « nous ne voyons pas encore ce que nous serons après cette vie. » Il nous a regardés à travers le voile de sa chair ; il nous a comblés de ses caresses, et embrasés de son amour ; et nous courons après l’odeur de ses parfums. Mais, au jour de son apparition, nous serons semblables à lui, parce que nous le verrons tel qu’il est. (I Jean III, 2)» Tel qu’il est, Seigneur : ainsi nous le verrons, mais ainsi nous ne le voyons pas encore.
\section[{Chapitre XVI, Nul ne connaît Dieu, comme Dieu se connaît, lui-même.}]{Chapitre XVI, Nul ne connaît Dieu, comme Dieu se connaît, lui-même.}
\noindent \pn{19}Vous seul savez ce que vous êtes absolument, parce que seul vous êtes l’Être immuable, l’immuable connaissance et la volonté immuable ; votre volonté est, et connaît immuablement ; votre connaissance est, et veut immuablement. Et vous ne trouvez pas juste que la lumière immuable soit connue, comme elle se connaît elle-même, de la lumière illuminée et muable. Aussi, mon âme est-elle\par

\begin{quoteblock}
\noindent « en votre présence comme une terre sans eau (Ps. CXIII, 6). »\end{quoteblock}

\noindent car elle ne peut pas plus faire jaillir d’elle-même la source qui la désaltère que la lumière qui l’illumine. Comme nous ne verrons la lumière que dans votre lumière, c’est en vous seul que nous pouvons puiser la vie (Ps. XXXV, 10).
\section[{Chapitre XVII, Comment on peut entendre la création de la mer et de la terre (Gen. I, 9, 11).}]{Chapitre XVII, Comment on peut entendre la création de la mer et de la terre (Gen. I, 9, 11).}
\noindent \pn{20}Quelle main a rassemblé en un même corps ces eaux d’amertume ? Elles tendent toutes et toujours à une même fin : le bonheur du temps et de la terre, malgré la diversité et l’agitation de leurs courants. Quel autre que vous, Seigneur, a dit aux eaux de se réunir en un même lieu ? Quel autre que vous a fait surgir la terre aride et altérée de votre grâce ? Seigneur,\par

\begin{quoteblock}
\noindent « cette mer est à vous ; elle est votre ouvrage ; et cette terre aride a été formée par vos mains (Ps. XCIV, 5). »\end{quoteblock}

\noindent Ce n’est point l’amertume des volontés, mais la réunion des eaux, qui a reçu le nom de mer. Car vous réprimez aussi les mauvaises passions des âmes ; vous fixez les limites qu’il leur est défendu de franchir ; enceinte puissante où leurs flots se brisent sur eux-mêmes (Job XXXVIII, 10, 11) ; et vous formez ainsi la mer du monde, et vous la gouvernez selon l’ordre de votre empire absolu sur toutes choses.\par
\pn{21}Mais ces âmes altérées de vous, présentes à vos regards, et séparées, pour une autre fin, de l’orageuse société de la mer, elles sont la Terre, que vous arrosez d’une eau mystérieuse et douce, pour qu’elle porte son fruit. Et cette terre fructifie, et docile au commandement du Seigneur, son Dieu, notre âme germe des œuvres de miséricorde, « selon son espèce, » l’amour et le soulagement du prochain dans les nécessités temporelles ; et ces fruits conservent la semence qui doit reproduire leur principe : car c’est du sentiment de notre misère que procède notre compassion pour l’indigence, et nous porte à la soulager comme nous voudrions l’être nous-mêmes dans une semblable détresse. Et il ne s’agit pas seulement d’une germination légère, d’une assistance facile, mais de cette végétation forte, de ce patronage héroïque de la charité, qui étend ses rameaux fructueux pour soustraire au bras du fort la faible victime, en l’abritant sous l’ombrage vigoureux de la justice.
\section[{Chapitre XVIII, Les justes peuvent être comparés aux astres (Gen. I, 14).}]{Chapitre XVIII, Les justes peuvent être comparés aux astres (Gen. I, 14).}
\noindent \pn{22}Oui, Seigneur, oui, je vous en supplie, vous dont l’influence répand dans les âmes une sève de joie et de force, Seigneur, que la vérité sorte de la terre, que la justice abaisse ses regards du haut du ciel (Ps. LXXXIV, 12) ; et « que des astres nouveaux étincellent dans le firmament ! » Partageons notre pain avec celui qui a faim ; recevons sous notre toit le pauvre qui n’a point de gîte ; couvrons celui qui est nu ; et ne méprisons pas les concitoyens de notre boue. Dès que notre terre aura produit ces fruits,   voyez, dites : « Cela est bon ; » et que notre lumière\par

\begin{quoteblock}
\noindent « jette son éclat en son temps (Is. LVIII, 7, 8) ; »\end{quoteblock}

\noindent que cette première végétation de bonnes œuvres nous élève aux contemplations délicieuses du Verbe de vie, et que nous apparaissions alors dans le monde comme des constellations attachées au firmament de votre Écriture.\par
C’est là que, conversant avec nous, vous nous enseignez le discernement des choses de l’esprit et des choses des sens ; comme celui du jour et de la nuit, ou des âmes spirituelles et des âmes asservies aux sens, afin que vous ne soyez plus seul à faire, dans le secret de votre connaissance, comme avant la création du firmament, la division de la lumière et des ténèbres ; mais que les enfants de votre esprit, placés à leur firmament, dans un ordre qui révèle l’infusion présente de votre grâce, brillent au-dessus de la terre, signalent la division du jour et de la nuit, et annoncent la révolution des temps : car\par

\begin{quoteblock}
\noindent « l’antique institution est passée, et la nouvelle se lève (I Cor. V, 17), et notre salut est plus près de nous que lorsque nous avons commencé de croire ; la nuit a précédé, et le jour arrive (Rom. XIII, 11) ; et vous couronnerez l’année de votre bénédiction (LXIV, 2), quand vous enverrez des ouvriers dans votre moisson (Matth. IX, 38) ensemencée par d’autres mains (Jean IV, 38) ; »\end{quoteblock}

\noindent quand vous enverrez de nouveaux ouvriers à de nouvelles semailles, dont la moisson ne se fera qu’à la fin (Matth. XIII, 39). Ainsi, vous accomplissez les vœux du juste, et vous bénissez ses années ; mais vous, vous êtes toujours le même, et vous recueillez, au grenier de vos années sans fin (Ps. CI, 28), nos années passagères ; car votre conseil éternel verse sur la terre, aux saisons marquées, les biens célestes.\par
\pn{23}L’un reçoit, par l’Esprit, la parole de sagesse, astre de lumière, qui plaît aux amis de la vérité, comme l’aurore du jour ; à l’autre, vous donnez, par le même Esprit, la parole de science, astre inférieur ; à celui-ci, la foi ; à celui-là, la puissance de guérir ; à l’un, le don des miracles ; à l’autre, le discernement des esprits ; à l’autre, le don des langues. Et toutes ces grâces sont comme autant de constellations, ouvrage d’un seul et même Esprit, qui distribue ses dons à chacun comme il lui plaît, et fait répandre à ces étoiles des irradiations salutaires (I Cor. XII, 7, 11).\par
La parole de science renferme les mystères sacrés, signes célestes, qui, selon les temps, ont eu leurs phases, commue la lune ; mais cette parole, et les autres dons spirituels, que j’assimile aux étoiles, ne sont, en comparaison des splendeurs de cette sagesse, que les premières heures de la nuit. Toutefois ils sont nécessaires à ceux en qui la chair n’est pas encore absorbée par l’esprit (I Cor. III, 1), et que votre grand serviteur ne peut entretenir dans la langue de sagesse qu’il parlait avec les parfaits (Ibid. II, 6).\par
Mais que l’enfant, dans le Christ, cet enfant que nourrit la mamelle, en attendant qu’il soit capable d’un aliment plus solide, et que ses yeux puissent soutenir le rayon du soleil, que l’homme animal ne se croie pas abandonné dans une nuit ténébreuse, mais qu’il se contente de la clarté de la lune et des étoiles. C’est ainsi, ô sagesse infinie ! que vous conversez avec nous dans le firmament de vos Écritures, pour nous élever à la contemplation admirable qui sait distinguer toutes choses, quoique nous soyons encore enfermés dans le cercle des augures, des temps, des années et des jours.
\section[{Chapitre XIX, Voie de la perfection (Gen. I, 14).}]{Chapitre XIX, Voie de la perfection (Gen. I, 14).}
\noindent \pn{24}Mais d’abord « lavez-vous, purifiez-vous ; faites disparaître toute souillure et de vos âmes et de mes regards, » afin que la terre intérieure s’élève. Apprenez à faire le bien ;\par

\begin{quoteblock}
\noindent « rendez justice à l’orphelin, et maintenez le droit de la veuve (Is. I, 16, 17), »\end{quoteblock}

\noindent afin que cette terre se couvre de fertiles pâturages et d’arbres chargés de fruits. Venez, je veux vous instruire ; attachés au firmament du ciel, vous serez les flambeaux du monde.\par
Le riche demande au bon Maître ce qu’il doit faire pour obtenir la vie éternelle. Ecoute ce bon Maître que tu crois un homme et rien de plus, mais qui est bon, parce qu’il est Dieu ; il te dit : « Si tu veux arriver à la vie, observe les commandements ; » sépare du sol de ton cœur les eaux amères de la malice et de la corruption ; garde-toi du meurtre, de l’adultère, du vol ; ne porte point faux témoignage, afin que la terre paraisse et germe le respect des père et mère, et l’amour du prochain. — J’ai fait tout cela, répond le riche.\par
D’où viennent donc tant d’épines, si la terre est fertile ? Va, déracine ces sauvages buisson   de l’avarice ; vends ce que tu as, donne-le aux pauvres, et ton aumône te couvrira de fruits ; et tu auras un trésor dans le ciel ; et puis, suis le Seigneur, si tu veux être parfait et devenir le compagnon de ceux à qui il parle le langage de la sagesse, lui qui sait et te fera savoir ce que c’est que le jour, ce que c’est que la nuit, afin que les astres brillent aussi pour toi au firmament de son ciel ; chose impossible, si ton cœur n’y est déjà ; et là ne sera jamais ton cœur, si là n’est point ton trésor, comme te l’a dit le bon Maître (Matth. VI, 21). Mais la tristesse se répandit sur la terre stérile, et les épines étouffèrent la parole (Ibid. XIX, 16, 22).\par
\pn{25}Pour vous, race d’élection, faibles du monde, qui avez tout quitté pour suivre le Seigneur, allez et confondez les puissances du siècle. Que vos pieds radieux marchent sur sa trace ! Etincelez au firmament (I Pierre, II, 19), afin que les cieux racontent sa gloire, en discernant la lumière des parfaits qui sont encore loin des anges, et les ténèbres des petits déjà sauvés de vos mépris ! Brillez sur toute la terre ! Que ce jour, éblouissant des clartés de ce soleil, annonce au jour le Verbe de sagesse, et que cette nuit soit le clair de lune qui annonce à la nuit le Verbe de science (Ps. XVIII, 2). La lune et les étoiles luisent sur la nuit, sans être obscurcies par ses ténèbres ; elles lui donnent toute la lumière qu’elle peut recevoir. Et, comme si Dieu eût dit : Que les astres soient dans le firmament du ciel : voici soudain un grand bruit venu d’en-haut, comme un tourbillon violent, et des langues de feu rayonnent et se divisent en s’arrêtant sur la tête de chacun d’eux (Actes, II, 2, 3) : et il se fit comme un firmament d’astres possesseurs du Verbe de vie. Courez partout, flammes de sainteté, feux admirables ! Car vous êtes la lumière du monde, et le boisseau ne vous couvre pas. Celui à qui vous vous êtes attachés a été exalté dans la gloire, et il vous a exaltés. Courez donc, et révélez-vous à toutes les nations.
\section[{Chapitre XX, Sens mystique de ces paroles : « que les eaux produisent les reptiles et les oiseaux (Gen. I, 20). »}]{Chapitre XX, Sens mystique de ces paroles : « que les eaux produisent les reptiles et les oiseaux (Gen. I, 20). »}
\noindent \pn{26}Que la mer conçoive aussi, qu’elle enfante vos œuvres, et que les eaux produisent les reptiles des âmes vivantes ! Car en séparant le pur de l’impur, vous êtes devenus la bouche de Dieu (Jérém. XV, 19), et c’est par vous qu’il dit : « Produisent les eaux, » non pas des âmes vivantes, productions de la terre ; « mais des reptiles d’âmes vivantes, et les oiseaux qui volent au-dessus de la terre ! » Ces reptiles, mon Dieu, sont vos sacrements qui, par les œuvres des saints, se sont glissés à travers les flots des tentations du siècle, pour régénérer les peuples dans le baptême en votre nom. Et ainsi se sont produites de grandes merveilles, « semblables aux baleines monstrueuses, » et les voix de vos messagers ont plané sur la terre et sous le ciel de votre Écriture, autorité protectrice, qui s’étend partout où leur vol se dirige.\par

\begin{quoteblock}
\noindent « Et ce ne sont pas de sourds et vains accents que leurs paroles ; toute la terre en a été l’écho ; elles ont atteint les extrémités du monde (Ps. XVIII, 4, 5) »\end{quoteblock}

\noindent car votre bénédiction, Seigneur, les a multipliées.\par
\pn{27}Mais n’est-ce pas erreur ? et ne confondrais-je pas les connaissances claires qui résident au firmament, et les œuvres corporelles qui s’opèrent sous ce firmament au sein orageux de la mer ? Non ; car ces mêmes connaissances, qui demeurent dans la fixité de leur certitude, et sans s’accroître par génération, comme les lumières de la sagesse et de la science, exercent cependant dans l’ordre réel une action différente et multiple, dont votre bénédiction féconde encore et multiplie les effets. Ô Dieu, vous nous consolez de l’infirmité de nos sens mortels, en permettant qu’une vérité, notion simple dans l’esprit, emprunte aux signes corporels, plus d’une figure, plus d’une expression.\par
Voilà les productions des eaux, mais grâce à votre parole ; productions nées de la misère des peuples devenus étrangers à votre vérité éternelle ; productions que les eaux ont fait jaillir de leur sein, comme un remède dont votre Verbe adoucissait leur languissante amertume.\par
\pn{28}Et toutes vos œuvres sont belles, car elles sortent de votre main ; mais que vous êtes incomparablement plus beau, divin auteur, du monde ! Oh ! si Adam ne se fût point détaché de vous, ses flancs n’eussent pas été la source de cet océan amer, de ce genre humain, curiosité sans fond, éternel orage de superbe, flot de mobilité ! Et alors les dispensateurs de votre vérité n’auraient pas eu besoin d’employer au sein des ondes tant de signes sensibles et   corporels, tant de paroles symboliques, tant d’opérations mystérieuses.\par
Ce sont là, suivant moi, ces reptiles, ces oiseaux qui s’insinuent parmi les hommes pour les initier et les soumettre aux symboles sacramentels. Mais ils ne pourraient aller au delà, si votre Esprit n’élevait la voix de leur âme à un degré supérieur, et si leur cœur, après les paroles du premier échelon, n’aspirait au faîte de l’échelle sainte.
\section[{Chapitre XXI, Interprétation mystique des animaux terrestres (Gen. I, 24).}]{Chapitre XXI, Interprétation mystique des animaux terrestres (Gen. I, 24).}
\noindent \pn{29}Et ce n’est plus une mer profonde, c’est une terre séparée par votre Verbe des ondes d’amertume, qui produit, non pas des oiseaux et des reptiles d’âmes vivantes, mais l’âme vive ; car elle n’a plus besoin, comme au temps où elle était cachée sous les eaux, du baptême nécessaire aux païens, cette voie qui seule donne entrée au royaume des cieux, depuis que vous avez interdit tout autre en l’ouvrant. Et cette âme ne demande plus des merveilles extraordinaires pour faire naître sa foi. Elle n’a plus besoin, pour croire, de signes et de miracles visibles (Jean, IV, 48) : terre de foi, et déjà séparée des flots amers de l’infidélité, que lui importe\par

\begin{quoteblock}
\noindent « le don des langues, témoignage pour les infidèles et non pour les fidèles (I Cor. XIV, 22) ? »\end{quoteblock}

\noindent Et ces oiseaux, que votre parole a tirés des eaux, sont désormais inutiles à cette terre que vous avez affermie au-dessus des eaux. Faites descendre en elle ce Verbe par vos envoyés. Car nous ne pouvons que raconter leurs œuvres, mais c’est vous qui opérez en eux l’œuvre qu’ils produisent : l’âme vivante.\par
Et la terre produit aussi ; cette terre mystique, cause de l’opération de vos serviteurs sur elle ; comme la mer était la cause de l’opération de ces reptiles d’âmes vivantes et de ces oiseaux dont le vol rase le firmament du ciel. Oiseaux, reptiles, dont cette terre n’a plus besoin, quoiqu’au festin dressé par vous à vos fidèles Ps. XXII, 5), elle mange le poisson mystérieux (Lux, XXIV, 43), tiré des profondeurs de l’abîme pour nourrir la terre. Et les oiseaux, ces enfants de la mer, ne laissent pas de multiplier sur la terre.\par
Car, si l’infidélité des hommes a été la cause des premières prédications de la bonne nouvelle, les missionnaires de la parole n’en continuent pas moins d’exhorter les fidèles et de multiplier sur eux chaque jour leurs bénédictions. Mais c’est du fond de la terre purifiée que sort l’âme vive : car il n’est profitable qu’aux seuls fidèles de renoncer à l’amour du siècle, pour faire revivre en vous leur âme morte dans la vie de ces délices (Tim. V, 6), délices mortelles, ô Dieu, vivifiantes délices d’un cœur pur !\par
\pn{30}Que vos ministres travaillent donc sur cette terre, non plus, comme sur les eaux infidèles, par des symboles, des miracles, des paroles mystérieuses, afin d’entretenir la crainte de l’inconnu dans le sein de l’ignorance, mère de l’étonnement ; crainte salutaire, seule entrée qui conduise à la foi les enfants d’Adam, oublieux du Seigneur, et se cachant de sa face (Gen. III,8) pour devenir un abîme ! Non, plus ainsi ! Mais qu’ils travaillent comme sur une terre nouvelle, séparée des gouffres de l’abîme, qu’ils forment les fidèles sur le modèle de leur vie, qu’ils les invitent à l’imitation de leurs exemples.\par
Et les fidèles n’entendent plus seulement pour entendre, mais pour pratiquer.\par

\begin{quoteblock}
\noindent «Cherchez le Seigneur, et votre âme vivra (Ps. LXVIII, 33) ; votre terre produira une âme vivante. Ne vous conformez pas au siècle (Rom. XII, 2) »\end{quoteblock}

\noindent tenez-vous-en éloignés ; et votre âme vivra par la fuite des objets dont le désir la fait mourir. Réprimez en vous la violence sauvage de l’orgueil, les molles indolences de la volupté, et les insinuations d’une science menteuse, et voilà les animaux féroces apprivoisés, les chevaux domptés, les serpents sans venin : vivante allégorie des divers mouvements de l’âme. Le faste de la vanité, les séductions de la chair, le venin de la curiosité sont, en effet, les mouvements d’une âme morte, mais dont la mort n’est pas assez complète pour que tout mouvement en elle soit anéanti : elle meurt, il est vrai, en s’éloignant de la source de vie, mais elle a pris la forme du siècle, dont le torrent l’emporte.\par
\pn{31}Votre parole, ô Dieu, source de la vie éternelle, demeure et ne s’écoule point. Aussi, nous défend-elle, elle-même, de nous éloigner d’elle, en nous disant : « Ne vous conformez « pas au siècle, » afin que votre terre, abreuvée à la source de vie, produise une âme vivante, secondée par le Verbe que vos évangélistes ont publié, une âme pure, imitatrice des imitateurs   du Christ. Et tel est le sens de ces mots : « Selon son espèce : » car l’homme ne se plaît à imiter que ceux qu’il aime. « Soyez comme moi, dit l’Apôtre, car je suis comme vous.»\par
Ainsi, cette âme vive n’est peuplée que d’animaux apprivoisés, dont les actions témoignent la douceur. C’est le précepte que vous avez donné :\par

\begin{quoteblock}
\noindent « Agissez en vos œuvres avec douceur, et vous serez aimé de tous les hommes (Ecclési. III,19). »\end{quoteblock}

\noindent Et ces troupeaux inférieurs ne se trouveront pas mieux pour être dans l’abondance ; ni plus mal pour être dans la disette ; et ces serpents innocents seront sans venin pour nuire, mais pleins de prudence pour éviter les morsures ; et ils ne donneront à la contemplation de la nature temporelle qu’autant qu’il est nécessaire pour s’élever de la vue de l’ordre temporel à la vue intelligente de l’ordre éternel (Rom. I, 20) Ces animaux deviennent les serviteurs de la raison, quand ils ont reçu le frein qui les préserve de la mort ; et ils vivent alors, et leur être est bon.
\section[{Chapitre XXII, Vie de l’âme renouvelée (Gen. I, 26).}]{Chapitre XXII, Vie de l’âme renouvelée (Gen. I, 26).}
\noindent \pn{32}Oui, Seigneur, mon Dieu et mon Créateur, quand nos affections seront dégagées de l’amour du siècle, et de cette vie de péché, qui nous faisait mourir ; quand notre âme commencera de vivre de la vraie vie, docile à la parole que vous avez fait entendre par la bouche de l’Apôtre :\par

\begin{quoteblock}
\noindent « Ne vous conformez pas au siècle ; Mais réformez vous en renouvellement de l’esprit (Rom. XII, 2). »\end{quoteblock}

\noindent Et il ne s’agit plus de se produire « suivant son espèce, » d’imiter ses prédécesseurs, et de régler sa vie sur l’autorité d’un homme plus parfait. Non : car vous n’avez pas dit : Que l’homme soit fait selon son espèce, mais :\par

\begin{quoteblock}
\noindent « Faisons l’homme à notre image et ressemblance (Gen. I, 26), »\end{quoteblock}

\noindent afin que nous aussi nous ayons la faculté de reconnaître quelle est votre volonté. C’est pourquoi le grand dispensateur de vos mystères, père de tant de fils, selon l’Evangile (I Cor. IV, 15) ne voulant pas toujours avoir des enfants à la mamelle, nourrissons à porter dans ses bras (I Thess. II, 7), s’écrie :\par

\begin{quoteblock}
\noindent « Réformez-vous en renouvellement d’esprit, pour reconnaître la volonté de Dieu, pour savoir ce qui est bon, ce qui lui plaît, ce qui est parfait (Rom. XII, 2). »\end{quoteblock}

\noindent Aussi, ne dites-vous pas : Que l’homme soit fait, mais : « Faisons l’homme ; » et non : selon son espèce, mais : « à notre image et ressemblance.» Renouvelé spirituellement, et voyant votre vérité des yeux de l’intelligence, il n’a plus besoin d’un maître, d’un modèle de son espèce. C’est de vous, et c’est en vous qu’il connaît votre volonté ; ce qui est bon, ce qui vous plaît. Et vous lui donnez la puissance de contempler la Trinité de votre Unité, et l’Unité de votre Trinité. Aussi, vous dites d’abord au pluriel : « Faisons l’homme ; » puis vous ajoutez : « Et Dieu fit l’homme. » Vous dites : « À notre image ; » et vous ajoutez : « À l’image de Dieu. » Ainsi, l’homme est renouvelé dans la connaissance de Dieu,\par

\begin{quoteblock}
\noindent « selon l’image de Celui qui l’a créé (Gen. I, 27) »\end{quoteblock}


\begin{quoteblock}
\noindent « et transformé en esprit, il juge de tout ce qu’il doit juger, et n’est jugé de personne. (Coloss. III, 10) »\end{quoteblock}

\section[{Chapitre XXIII, De qui l’homme spirituel peut juger (Gen. I, 26)}]{Chapitre XXIII, De qui l’homme spirituel peut juger (Gen. I, 26)}
\noindent \pn{33}Or, « l’homme spirituel juge de tout, » et c’est ce que l’Écriture appelle avoir puissance sur les poissons de la mer, les oiseaux du ciel, les animaux domestiques et sauvages, sur toute la terre, sur tout ce qui rampe à sa surface. Et, cette puissance, il l’exerce par cette intelligence qui le rend capable de pénétrer\par

\begin{quoteblock}
\noindent « ce qui est de l’Esprit de Dieu (I Cor. II, 14). »\end{quoteblock}


\begin{quoteblock}
\noindent « Déchu de la gloire, par défaut d’intelligence, n’est-il pas descendu au rang des brutes, ne leur est-il pas devenu semblable (Ps. XLVIII, 13) ? »\end{quoteblock}

\noindent Et nous, mon Dieu, nous, enfants de la grâce dans votre Église,\par

\begin{quoteblock}
\noindent « nous, votre ouvrage, créés dans les bonnes œuvres (Ephés. II, 10), »\end{quoteblock}

\noindent nous sommes juges spirituels, soit que nous ayons l’autorité selon l’esprit, soit que nous obéissions spirituellement. « Vous avez fait l’homme mâle et femelle ; » et il en est ainsi dans la création de votre grâce, où cependant il n’y a plus ni mâle ni femelle, suivant le sexe corporel ; ni Juif ni Grec, ni libre ni esclave (Galat. III, 28). Et ces hommes de l’esprit, soit qu’ils commandent, soit qu’ils obéissent, sont juges spirituels (Cor. II, 15). Mais leur jugement ne s’exerce pas sur les pensées spirituelles qui brillent au firmament. Il ne leur appartient pas de prononcer sur une autorité si sublime ; de s’élever en juges de votre   livre saint, lors même que des ombres y voilent la lumière. Car nous lui devons la soumission de notre intelligence, et une ferme assurance dans la rectitude et la vérité de toute lettre close à nos yeux. L’homme,\par

\begin{quoteblock}
\noindent « même spirituel, et renouvelé dans la connaissance de Dieu, selon l’image du Créateur (Coloss. III, 10), »\end{quoteblock}

\noindent doit être l’observateur et non pas le juge de la loi (Jacq. IV, 11).\par
Son jugement ne va pas non plus à discerner les hommes de l’esprit des hommes de la chair, s’il ne les connaît par leurs œuvres, comme\par

\begin{quoteblock}
\noindent « l’arbre se connaît par son fruit (Math. VII, 20). »\end{quoteblock}

\noindent Votre regard seul les voit, mon Dieu ; vous les connaissez déjà, Seigneur, et vous les aviez déjà distingués ; vous les aviez appelés dans le secret de votre conseil, avant même de créer le firmament.\par
Quoique spirituel, il ne juge pas non plus des générations turbulentes du siècle.\par

\begin{quoteblock}
\noindent « Pourquoi jugerait-il ceux de dehors (I Cor. V, 12), »\end{quoteblock}

\noindent puisqu’il ignore quels sont dans ce nombre les élus appelés à goûter un jour la douceur de votre grâce, et les âmes qui doivent demeurer éternellement dans l’amertume de l’impiété ?\par
\pn{34}Ainsi donc, en formant l’homme à votre image, vous ne lui avez donné puissance ni sur les astres du ciel, ni sur le ciel secret, ni sur ce jour, ni sur cette nuit, que vous avez nommés avant la création, ni sur cette réunion des eaux qui s’appelle la mer ; il n’a reçu puissance que sur les poissons de la mer, sur les oiseaux du ciel, sur tous les animaux, sur toute la terre, sur tout ce qui rampe à sa surface.\par
Il juge, il approuve ou condamne ce qu’il trouve bien ou mal, et dans la solennité du sacrement initiateur qui consacre à votre service ceux que votre miséricorde a pêchés au fond des eaux ; et dans ce banquet sacré où le mystique poisson, tiré du fond de l’abîme, est servi pour nourrir la terre ; et dans les discours, dans les paroles, oiseaux fidèles, qui volent sous le firmament de l’autorité des saintes Écritures ; interprétations, expositions, discussions, controverses, bénédictions, prières, que les lèvres prononcent en formules sonores, afin que le peuple puisse répondre ainsi soit-il ! L’abîme du siècle, et la cécité de cette chair qui n’a pas d’yeux pour voir les pensées, telle est la cause de l’emploi des sons et du bruit dont on frappe les oreilles. Et voilà comment ces oiseaux qui se multiplient sur la terre sont néanmoins originaires des eaux.\par
L’homme spirituel juge encore, approuve ou condamne ce qu’il trouve bien ou mal, dans les œuvres, dans les mœurs des fidèles ; il juge des aumônes comme des fruits de la terre ; il juge de l’âme vivante qui sait, par la charité, les jeûnes, et les pieuses pensées apprivoiser ses passions ; il juge de tout ce qui se produit par des effets sensibles ; il est juge enfin, là où il a le pouvoir de corriger et de reprendre.
\section[{Chapitre XXIV, Pourquoi Dieu a béni l’Homme, les poissons et les oiseaux ?}]{Chapitre XXIV, Pourquoi Dieu a béni l’Homme, les poissons et les oiseaux ?}
\noindent \pn{35}Qu’ai-je lu ? Quel est ce mystère ? Voilà, Seigneur, que vous bénissez les hommes, afin qu’ils croissent, qu’ils multiplient, qu’ils remplissent la terre (Gen. I, 27). N’y a-t-il point là un secret dont vous voulez nous insinuer quelque connaissance ? Pourquoi n’avez-vous pas également béni la lumière que vous avez nommée jour, et le firmament du ciel, et les flambeaux célestes, et les étoiles, et la terre et la mer ? Je dirais, ô Dieu ! qui avez créé l’homme à votre image, je dirais que vous avez voulu accorder à l’homme la faveur singulière de votre bénédiction, si vous n’eussiez béni de même les poissons pour qu’ils croissent, multiplient et peuplent les eaux de la mer ; si vous n’eussiez béni les oiseaux pour qu’ils multiplient sur la terre (Gen I, 22).\par
Je dirais encore que votre bénédiction repose sur tous les êtres qui perpétuent leur espèce par la génération, si je voyais que votre divine main se fût étendue sur les plantes, les arbres et les animaux de la terre. Mais il n’a été dit ni aux végétaux, ni aux bêtes, ni aux serpents : Croissez et multipliez, quoiqu’ils s’accroissent par génération et se conservent dans leur espèce, comme les poissons, les oiseaux et les hommes.\par
\pn{36}Dirai-je donc, ô vérité ! ma lumière, qu’il n’y a là que vaines paroles tombées sans dessein ? Non, non, loin de moi, ô Père de toute piété ! loin de l’esclave de votre Verbe une semblable pensée ! Et si je ne puis pénétrer le sens de votre parole, qu’il l’entende mieux que moi, qu’il y puise selon la   contenance intellectuelle qu’il a reçue de vous, celui de mes frères qui est meilleur, qui est plus intelligent que moi. Mais agréez, Seigneur, cet humble aveu, qu’il monte en votre présence. Oui, je crois que ce n’est pas en vain que vous avez parlé, et je ne tairai pas les pensées que votre parole me suggère. Je les sens vraies, et je ne vois rien qui m’empêche d’interpréter ainsi les expressions figurées de vos livres saints ; car, multiplicité de signes, simplicité de sens : multiplicité de sens, simplicité de signes ; l’amour de Dieu et du prochain n’est-il pas une notion simple ? Quelle multiplicité de formules mystiques, de langues et de locutions sans nombre pour le traduire par une expression sensible ? Et c’est ainsi que les vivantes productions des eaux croissent et multiplient. Attention, lecteur ; qui que tu sois ! l’Écriture n’énonce qu’un mot, elle ne fait entendre qu’une parole :\par

\begin{quoteblock}
\noindent « Dans le principe, Dieu créa le ciel et la terre (Gen I, 1). »\end{quoteblock}

\noindent Eh bien ! qu’est-ce qui en multiplie l’interprétation ? Est-ce l’erreur ? non, mais la variété des espèces intellectuelles. Et c’est ainsi que la postérité humaine croît et multiplie.\par
\pn{37}Car, à considérer la nature même des choses dans le sens propre et non dans le sens allégorique, cette parole, « croissez et multipliez, » convient à tout ce qui se reproduit par semence. Si nous nous attachons au sens figuré, interprétation conforme, suivant moi, à l’esprit de l’Écriture, qui certes n’attribue pas en vain cette bénédiction aux seules générations des hommes, aux seules productions des eaux, nous voyons bien, il est vrai, multitude dans le ciel et la terre, ou le monde des esprits et le monde des corps ; dans la lumière et les ténèbres, ou les âmes des justes et des impies ; dans le firmament étendu entre les eaux, ou les saints dispensateurs de la loi divine ; dans la mer, ou l’océan d’amertume des sociétés humaines ; dans la terre séparée des ondes, ou les âmes purifiées au feu de l’amour ; dans les plantes séminales et les arbres fruitiers, ou les œuvres de miséricorde pratiquées en cette vie ; dans les flambeaux suspendus à la voûte céleste, ou les dons spirituels qui brillent pour édifier ; dans l’âme vivante, ou les affections soumises à la règle : dans cet ensemble de la création, nous découvrons multitude, fécondité, accroissement. Mais quant à ce mode de multiplication et de développement, qui fait qu’une seule vérité s’exprime par plusieurs énonciations, et qu’une seule énonciation s’entend en plusieurs sens vrais, c’est ce que nous ne trouvons que dans les signes sensibles de la pensée, et les conceptions de l’intelligence. Les signes sensibles, ce sont les générations de la mer, multipliées dans l’abîme de notre indigence ; les conceptions de l’intelligence, ce sont les générations humaines, c’est la fécondité de notre raison. Et voilà pourquoi, Seigneur, je crois que vous n’avez dit qu’aux seules générations des hommes et des eaux : « Croissez et « multipliez. » Et je crois que par cette bénédiction vous nous avez conféré la puissance de donner plusieurs expressions à une conception simple, et la faculté d’attacher plusieurs sens à une énonciation simple, mais obscure.\par
Ainsi se remplissent les eaux de la mer, dont les différents souffles de l’esprit remuent les courants ; ainsi la postérité humaine peuple la terre, séparée des eaux par l’amour de la vérité, et soumise à l’empire de la raison.
\section[{Chapitre XXV, Les fruits de la terre figurent les œuvres de piété (Gen. I, 29).}]{Chapitre XXV, Les fruits de la terre figurent les œuvres de piété (Gen. I, 29).}
\noindent \pn{38}Seigneur mon Dieu, je veux encore dire les pensées que la suite de vos paroles m’inspire, et je les dirai sans crainte. Je dirai la vérité ; c’est au souffle de votre volonté que je parle. Et je ne puis croire que jamais la vérité sorte de mes lèvres que par votre inspiration, car vous êtes la vérité même (Jean XIV, 6) ; tout homme est menteur (Ps. CXV, 11) et celui dont la parole est mensonge parle de son propre fonds (Jean VIII, 44). Moi, je veux dire la vérité, je ne parlerai donc que par vous.\par
Vous nous avez donné « pour nourriture toutes les plantes séminales répandues sur la terre, et tous les fruits qui recèlent en eux-mêmes leur semence reproductive ;» et ce n’est pas à nous seuls que vous les avez donnés, mais encore aux oiseaux du ciel, aux animaux terrestres et aux serpents. Ils n’ont point été donnés aux poissons et aux géants de l’abîme.\par
Je disais donc que ces fruits de la terre sont la figure allégorique des œuvres de miséricorde qui sortent du sol fertile de l’âme pour   soulager les misères de la vie. Le pieux Onésiphore était une de ces charitables terres, et vous fîtes miséricorde à toute sa maison, parce qu’il assista souvent votre serviteur Paul, et ne rougit jamais de ses chaînes (II Tim. I, 16). Tels étaient les frères qui se couvrirent des mêmes fruits, en lui apportant de Macédoine de quoi fournir à sa détresse (II Cor. XI, 9). Et avec quelle douleur il déplore la stérilité des arbres qui ne lui donnèrent point le fruit qu’ils lui devaient !\par

\begin{quoteblock}
\noindent « Au temps de ma première défense, personne ne me vint en aide, mais tous m’abandonnèrent. Dieu leur pardonne II (Tim. IV, 16) !»\end{quoteblock}

\noindent Des secours ne sont-ils pas bien dus aux maîtres spirituels qui initient notre raison à l’intelligence des saints mystères ? Ces secours sont les fruits que la terre doit à l’homme ; ils leur sont dus comme âme vivante qui anime la sève reproductive de leurs vertus ; ils leur sont dus comme oiseaux célestes, dont la voix s’est répandue aux extrémités de la terre (Ps. XVIII, 5) pour l’ensemencer de bénédictions.
\section[{Chapitre XXVI, Le fruit des œuvres de miséricorde est dans la bonne volonté.}]{Chapitre XXVI, Le fruit des œuvres de miséricorde est dans la bonne volonté.}
\noindent \pn{39}Or, ces fruits ne sont un aliment que pour ceux qui y trouvent une joie sainte, et cette joie n’est pas aux esclaves\par

\begin{quoteblock}
\noindent « asservis au culte de leur ventre (Philip. III, 19) »\end{quoteblock}

\noindent Et même en ceux qui donnent, ce n’est pas l’aumône qui est le fruit, c’est l’intention de l’aumône. Aussi je comprends la joie de ce grand apôtre, qui vivait pour son Dieu et non pour son ventre, je la comprends bien ; mon âme sympathise à cette joie. Il venait de recevoir par Epaphrodite les dons des Philippiens : mais est-ce de ces dons qu’il se réjouit ? Non, je vois la cause de sa joie, et cette cause est le fruit qu’il savoure. Car il dit en vérité :\par

\begin{quoteblock}
\noindent « J’ai ressent une joie ineffable dans le Seigneur, de ce que votre amour pour moi a commencé de refleurir ; non que cet amour se fût flétri en vous, mais il était voilé par la tristesse (Philip. IV, 10)»\end{quoteblock}

\noindent Une longue tristesse les avait donc desséchés ; et comme de stériles rameaux, ils ne portaient plus de fruits charitables ; et il se réjouit de les voir refleurir ; il se réjouit non pour lui-même des secours dont ils ont assisté son indigence ; car il ajoute :\par

\begin{quoteblock}
\noindent « Ce n’est pas qu’il me manque rien ; dès longtemps j’ai appris à me contenter de l’état où je me trouve ; je sais vivre pauvrement, je sais vivre dans l’abondance. Je suis fait à tout ; je suis à l’épreuve de tout : de la faim et des aliments, de l’opulence et de la disette. Je peux tout en Celui qui me fortifie (Philip. IV, 11, 13).»\end{quoteblock}

\noindent \pn{40}Quelle est donc la cause de ta joie, ô grand Paul ? Dis, quelle est cette joie ? Quel est ce fruit dont tu goûtes la saveur, « homme renouvelé par la connaissance de Dieu, à l’image de ton Créateur ? » Ame vivante, peuplée de vertus ! Langue aux ailes de feu qui proclame dans le monde les divins mystères ! C’est bien aux âmes comme la tienne que l’on doit cette nourriture d’amour. Dis, de quel fruit te nourris-tu ? de joie ? Ecoutons-le : « Oui, dit-il, oui, vous avez bien fait d’entrer en communion avec mes souffrances. » Voilà sa joie, voilà sa nourriture. Ils ont bien fait, non parce qu’il a eu quelque relâche à ses angoisses, lui qui vous disait :\par

\begin{quoteblock}
\noindent « Dans la tribulation vous avez dilaté mon cœur (Ps. IV, 2), »\end{quoteblock}

\noindent lui qui sait souffrir l’abondance et la disette, en vous son unique force.\par

\begin{quoteblock}
\noindent « Vous savez, ajoute-t-il, vous savez, Philippiens, que depuis mon départ de Macédoine pour les premières prédications de 1’Evangile, nulle autre Église n’a eu communication avec moi en ce qui est de donner et de recevoir ; je n’ai rien reçu que de vous seuls, qui m’avez envoyé par deux fois à Thessalonique de quoi subvenir à mes besoins (Philip. IV, 14-16). »\end{quoteblock}

\noindent Et maintenant il se réjouit de leur retour aux bonnes œuvres ; il se réjouit des nouveaux fruits et de la nouvelle fertilité du champ spirituel.\par
\pn{41}Serait-ce donc dans son intérêt ? car il dit : « Vous avez envoyé à ma détresse ? » La source de sa joie serait-elle là ? Non, non ! Et comment le savons-nous ? Lui-même nous l’apprend :\par

\begin{quoteblock}
\noindent « Ce n’est pas le don, c’est le fruit que je cherche (Ibid. 17). »\end{quoteblock}

\noindent J’ai appris de vous, mon Dieu, à distinguer entre le don et le fruit. Le don, c’est l’objet que donne celui qui assiste une indigence : c’est l’argent, la nourriture, le breuvage, le vêtement, l’abri, tout secours enfin : le fruit, c’est la volonté droite et sincère de celui qui donne. Car le divin Maître ne se borne pas à dire : « Celui qui reçoit un prophète ; » il ajoute : « en qualité de prophète ; » « celui qui reçoit un juste, » mais « en qualité de juste, recevront la récompense,   l’un du prophète, l’autre du juste. » Il ne dit pas seulement : « Celui qui donnera un verre d’eau au dernier des miens ; » il ajoute\par

\begin{quoteblock}
\noindent « en qualité de mon disciple ; en vérité je vous le dis, celui-là ne perdra point sa récompense (Matth. X, 41, 42).»\end{quoteblock}

\noindent Recueillir un prophète, recueillir un juste, donner au disciple un verre d’eau, voilà le don : agir ainsi en vue de leur qualité de prophète, de juste et de disciple, voilà le fruit. C’est le fruit que la veuve offrait à Elie qu’elle savait l’homme de Dieu, et qu’elle nourrissait à ce titre. Et ce n’est que le don qu’il recevait du corbeau dans le désert (III Rois, XVII, 6, 16). Ce don n’était pas la nourriture de l’homme intérieur, mais de l’homme extérieur, qui, seul en Elie, pouvait défaillir faute de cet aliment.
\section[{Chapitre XXVII, Signification des poissons et des baleines.}]{Chapitre XXVII, Signification des poissons et des baleines.}
\noindent \pn{42}Je veux dire toute la vérité en votre présence, Seigneur. Quand des hommes d’ignorance et d’infidélité, qui ne peuvent être gagnés à votre service que par l’initiation des sacrements et la grandeur des miracles , ces poissons, ces géants de l’abîme, accueillent vos serviteurs, pour nourrir leur faim, pour les soulager dans les besoins de la vie présente, sans connaître quels doivent être la raison et le but suprêmes de l’aumône et de l’hospitalité ; ces infidèles ne donnent et vos enfants ne reçoivent aucune nourriture ; car les uns n’agissent pas dans une volonté droite et sainte, et les autres ne voyant qu’un don et point de fruit, ne ressentent aucune joie. Or, l’âme ne se nourrit que des objets de sa joie. Et voilà pourquoi ces poissons et ces baleines ne sauraient vivre des productions qui ne naissent que d’une terre séparée des eaux de l’abîme et purifiée de leur amertume.
\section[{Chapitre XXVIII, Pourquoi Dieu dit que ses œuvres étaient très bonnes (Gen. I, 31).}]{Chapitre XXVIII, Pourquoi Dieu dit que ses œuvres étaient très bonnes (Gen. I, 31).}
\noindent \pn{43}Et à la vue de toutes vos œuvres, ô Dieu, vous les avez dites très-bonnes. Nous les voyons aussi et nous les trouvons très-bonnes. À chacun de vos ouvrages, en particulier, aussitôt que vous eûtes dit : Qu’il soit ! et aussitôt il fut, vous l’avez vu, et vous l’avez trouvé bon. J’ai compté sept fois écrit que vous aviez trouvé bonne l’œuvre qui sortait de vos mains ; et, la huitième fois, à l’aspect de tous vos ouvrages, vous les avez trouvés, non-seulement bons, mais très-bons dans leur ensemble. Chaque partie, prise isolément, n’est que bonne ; mais l’ensemble est très-bon. Et la beauté de tout objet sensible rend témoignage à votre parole. Un corps, dans l’harmonieuse beauté de tous ses membres, est beaucoup plus beau que chacun de ces membres, dont la beauté particulière concourt à la beauté de l’ensemble.
\section[{Chapitre XXIX, Comment Dieu à vu huit fois que ses œuvres étaient bonnes.}]{Chapitre XXIX, Comment Dieu à vu huit fois que ses œuvres étaient bonnes.}
\noindent \pn{44}Et j’ai recherché avec attention s’il est vrai que vous eussiez vu sept ou huit fois que vos œuvres étaient bonnes (car elles vous plaisaient) ; et je n’ai pu découvrir dans votre vue divine aucun temps qui me fît comprendre comment vous avez vu vos œuvres à tant de reprises. Et je me suis écrié : Seigneur, votre Écriture n’est-elle pas véritable, dictée par vous qui l’êtes, qui êtes la vérité même ? Pourquoi donc me dites-vous que le temps n’est pas dans votre vue ? Et voilà votre Écriture qui me raconte l’approbation que vous avez donnée jour par jour à l’œuvre de vos mains. Et j’ai compté combien de fois, et j’en ai trouvé le nombre.\par
Et comme vous êtes mon Dieu, vous me répondez d’une voix forte, d’une voix qui brise ma surdité intérieure, vous me criez :\par

\begin{quoteblock}
\noindent « O homme, mon Écriture est ma parole. Mais elle parle dans le temps ; et le temps n’atteint pas jusqu’à mon Verbe, qui demeure avec moi dans mon éternité. Ce que tu vois par mon Esprit, c’est moi qui le vois ; et ce que tu dis par mon Esprit, c’est moi qui le dis : mais tu vois dans le temps, et ce n’est pas dans le temps que je vois ; tu parles dans le temps, et ce n’est pas dans le temps que je parle. »\end{quoteblock}

 \section[{Chapitre XXX, Rêveries manichéennes.}]{Chapitre XXX, Rêveries manichéennes.}
\noindent \pn{45}J’entends, mon Dieu ; votre vérité a laissé tomber sur mon âme une goutte de douceur infinie ; et j’ai compris qu’il est des hommes à qui vos œuvres déplaisent. Ils disent que la nécessité en a tiré plusieurs de vos mains, comme la mécanique des cieux et la disposition des astres, dont l’être émane, non de votre puissance créatrice, mais d’une matière préexistante, procédant d’ailleurs, et que vous avez rassemblée, resserrée, reliée, pour en bâtir ces remparts du monde, trophée de votre victoire sur vos ennemis, forteresse élevée contre toute révolte à venir.\par
Ils prétendent encore qu’il en est d’autres qui ne vous doivent ni leur être, ni leur composition, comme les corps de chair, les insectes, et tout ce qui tient à la terre par racines : ils y voient l’ouvrage d’une puissance ennemie, esprit que vous n’avez point créé, nature malfaisante en lutte contre vous, qui produit et qui forme tous ces êtres dans les plus passes régions de ce monde. Insensés ! ils ne parlent ainsi que faute de voir vos œuvres par votre Esprit, et de vous reconnaître dans vos œuvres.
\section[{Chapitre XXXI, Le fidèle voit par l’esprit de dieu, et dieu voit en lui que ses œuvres sont bonnes.}]{Chapitre XXXI, Le fidèle voit par l’esprit de dieu, et dieu voit en lui que ses œuvres sont bonnes.}
\noindent \pn{46}Mais nous, qui les voyons par votre Esprit, les voyons-nous ? et n’est-ce pas plutôt vous-même qui les voyez en nous ? Si donc nous les voyons bonnes, c’est vous qui les voyez bonnes. Dans tout ce qui nous plaît à cause de vous, c’est vous qui nous plaisez ; et tout ce qui nous plaît par votre Esprit, vous plaît en nous.\par

\begin{quoteblock}
\noindent « Quel homme, en effet, connaît ce qui est de l’homme, sinon l’esprit de l’homme qui est en lui ? Et l’Esprit de Dieu connaît seul ce qui est de Dieu. Aussi, dit l’Apôtre, nous n’avons pas reçu l’esprit, du monde, mais l’Esprit qui vient de Dieu, afin de connaître les dons de Dieu (I Cor. II, 11, 12).»\end{quoteblock}

\noindent Et cette parole m’autorise, et je dis : Non, personne ne sait ce qui est de Dieu, que l’Esprit de Dieu. Comment savons-nous donc nous-mêmes ce que Dieu nous a donné ? Mais j’entends la réponse : si nous ne le savons que par son Esprit, qui le sait, sinon le seul Esprit de Dieu ? Il est dit en vérité à ceux qui parlent par l’Esprit de Dieu :\par

\begin{quoteblock}
\noindent « Ce n’est pas vous qui parlez (Matth. X, 20) ; »\end{quoteblock}

\noindent et l’on peut dire en vérité à ceux qui connaissent par l’Esprit de Dieu : Ce n’est pas vous qui connaissez ; et l’on peut encore dire en vérité à ceux qui voient par l’Esprit de Dieu : « Ce n’est pas vous qui voyez. » Ainsi, quand nous voyons par l’Esprit de Dieu qu’une chose est bonne, ce n’est pas nous, c’est Dieu qui la voit bonne.\par
Et l’un tient pour mauvais ce qui est bon, suivant la doctrine de ces insensés ; et l’autre en reconnaît la bonté, mais il est de ceux qui ne savent point vous aimer dans vos créatures, dont ils préfèrent la jouissance à la vôtre. Celui-ci juge bonne l’œuvre bonne ; et est Dieu même qui voit en lui ; et il aime Dieu dans son œuvre, amour qui ne saurait naître sans le don de l’Esprit :\par

\begin{quoteblock}
\noindent « car l’amour se répand. Dans nos cœurs par l’Esprit saint qui nous est donné (Rom. V, 5) : »\end{quoteblock}

\noindent Esprit par qui nous voyons que tout être, quel qu’il soit, est bon, parce qu’il procède de Celui qui n’est pas seulement un être, mais l’Être lui-même.
\section[{Chapitre XXXII, Vue de la création.}]{Chapitre XXXII, Vue de la création.}
\noindent \pn{47}Seigneur, grâces vous soient rendues ! nous voyons le ciel et la terre, c’est-à-dire les régions supérieures et inférieures du monde ; ou le monde des esprits et celui des corps ; et, pour l’embellissement des parties qui forment l’ensemble ou de l’univers visible, ou de l’universalité des êtres, nous voyons la lumière créée et séparée des ténèbres. Nous voyons le firmament du ciel, soit ce premier corps du monde, élevé entre la sublimité des eaux spirituelles et l’infériorité des eaux corporelles (Voy. Rétr. Liv. II, Chap. VI, n°2), soit ces espaces de l’air, ce ciel où les oiseaux volent entre les eaux que les vapeurs condensent au-dessus d’eux-mêmes et qui retombent en rosées sereines, et les eaux plus lourdes, qui coulent sur la terre.\par
Nous voyons, par les plaines de la mer, la beauté de ces masses d’eaux attroupées ; et nous voyous la terre, d’abord dans sa nudité, puis, recevant avec la forme, l’ordre, la beauté et la force végétative. Nous voyons les astres   briller sur nos têtes ; le soleil suffire seul au jour ; la lune et les étoiles consoler la nuit ; notes radieuses de l’harmonie des temps. Nous voyons ces humides immensités se peupler de poissons, de monstres énormes, d’oiseaux divers : car l’évaporation de l’eau donne au corps de l’air cette consistance qui soutient leur vol.\par
Nous voyons la face de la terre ornée de ces races variées d’animaux, et l’homme, créé à votre image, investi d’autorité sur eux par cette divine ressemblance, par le privilége de l’intelligence et de la raison. Et comme il est, dans son âme, un conseil dominant et une obéissance soumise, ainsi, dans notre nature corporelle, la femme est créée pour l’homme, quoique également admise au don de la raison, et son sexe l’assujettit à l’homme, comme la puissance active et passionnée, soumise à l’esprit, conçoit de l’esprit le règlement de ses actions : voilà ce que nous voyons ; chacune de ces œuvres est bonne ; et leur ensemble est très-bon.
\section[{Chapitre XXXIII, Dieu a créé le monde d’une matière créée par lui au même temps.}]{Chapitre XXXIII, Dieu a créé le monde d’une matière créée par lui au même temps.}
\noindent \pn{48}Que vos œuvres vous louent, afin que nous vous aimions ; et que nous vous aimions, afin que vos œuvres vous louent, ces œuvres qui ont, dans le temps, leur commencement et leur fin, leur lever et leur coucher, leur progrès et leur déclin, leur beauté et leur défaillance ! Elles ont donc leur régulière vicissitude de matin et de soir, dans une évidence plus ou moins manifeste. Car elles sont toutes votre création, tirées du néant, et non pas de Vous-même ; non pas d’une autre substance, étrangère, antérieure à vous, mais d’une manière créée par vous, dans le même temps, et que vous avez fait passer, sans succession, de l’informité à la forme.\par
Ainsi, quelle que soit la différence entre la matière du ciel et de la terre, entre la beauté du ciel et de la terre, c’est du néant que vous avez créé la matière, c’est de cette matière informe que vous avez formé la beauté du monde, et néanmoins la création de la forme a suivi celle de la matière immédiatement et sans intervalle.
\section[{Chapitre XXXIV, Sens mystique de la creation.}]{Chapitre XXXIV, Sens mystique de la creation.}
\noindent \pn{49}Et j’ai médité sur le sens que vous avez voulu figurer par l’ordre de vos œuvres, et par l’ordre du récit inspiré de leur création ; et j’ai vu qu’elles sont bonnes en particulier, très bonnes, dans leur ensemble ; et dans votre Verbe, votre Fils unique, je vois le ciel et la terre, le chef et le corps de l’Église, prédestinés avant le temps, avant la naissance du matin et du soir. Et, dès que vous avez commencé d’exécuter dans les temps les conceptions de votre éternité, afin de dévoiler vos secrets, de rendre l’ordre au chaos d’iniquités qui pesait sur nous et nous entraînait loin de vous dans l’abîme des ténèbres, où votre Esprit saint planait, pour nous secourir au temps marqué ; vous avez justifié les impies, vous les avez séparés des pécheurs ; vous avez établi l’autorité de votre Écriture, comme un firmament entre l’autorité où vous élevez les eaux supérieures, et la soumission à cette autorité que vous imposez aux inférieures ; et vous avez réuni comme un troupeau la coupable unanimité des volontés infidèles, pour faire briller les saintes affections des fidèles qui devaient produire en votre nom des fruits de miséricorde, distribuant aux pauvres les biens de la terre pour gagner le ciel. Et vous avez allumé dans ce firmament des astres intelligents, dépositaire du Verbe de la vie éternelle, vos saints serviteurs, comblés des dons spirituels, investis d’une autorité sublime ; et puis, ces sacrements, ces miracles visibles, ces paroles consacrées, signes célestes du firmament de votre Écriture, qui appellent vos bénédictions sur les fidèles eux-mêmes ; toutes ces œuvres, instruments de la conversion des races infidèles, c’est à l’aide de la matière que vous les avez opérées ; et vous avez formé l’âme vivante de vos fidèles, par la vertu de ces facultés aimantes, soumises au sévère règlement de la continence.\par
Et cette âme raisonnable, désormais soumise à vous seul, assez libre pour se passer du secours et de l’autorité de tout exemple humain, vous l’avez renouvelée à votre image et ressemblance ; et vous avez soumis la femme à l’homme, l’activité raisonnable à cette puissante raison de l’esprit, et comme vos ministres sont toujours nécessaires aux fidèles en cette vie pour les amener à la perfection, vous   avez voulu que les fidèles leur payassent, dans le temps, un tribut charitable, dont l’éternité soldera l’intérêt. Et nous voyons toutes ces œuvres, et nous les voyons très-bonnes, ou plutôt, vous les voyez en nous ; puisque votre grâce a répandu sur nous l’Esprit qui nous donne la force de les voir et de vous aimer en elles.
\section[{Chapitre XXXV, « seigneur, donnez-nous votre paix ! »}]{Chapitre XXXV, « seigneur, donnez-nous votre paix ! »}
\noindent \pn{50}Source de tous nos biens, Seigneur mon Dieu, donnez-nous votre paix ! la paix de votre repos, la paix de votre sabbat ! la paix sans déclin ! Car cet ordre admirable, et cette belle harmonie de tant de créatures excellentes, passeront, le jour où leur destination sera remplie.\par
Ils auront leur soir, comme ils ont eu leur matin.
\section[{Chapitre XXXVI, Le septième jour n’a pas eu de soir.}]{Chapitre XXXVI, Le septième jour n’a pas eu de soir.}
\noindent \pn{51}Or, le septième jour est sans soir et sans coucher, parce que vous l’avez sanctifié, pour qu’il demeure éternellement. Et le repos que vous prenez après l’œuvre admirable de votre repos, nous fait entendre, par l’oracle de votre sainte Écriture, que nous aussi, après l’accomplissement de notre œuvre, dont votre grâce fait la bonté, nous devons nous reposer dans le sabbat de la vie éternelle !
\section[{Chapitre XXXVII, Comment Dieu se repose en nous.}]{Chapitre XXXVII, Comment Dieu se repose en nous.}
\noindent \pn{52}Alors votre repos en nous sera, comme aujourd’hui votre opération en nous. Et notre repos sera le vôtre, comme aujourd’hui nos œuvres sont les vôtres ; car vous, Seigneur, vous êtes à la fois le mouvement et le repos éternel. Votre vue, votre action, votre repos ne connaissent pas le temps ; et cependant vous faites notre vue dans le temps, vous faites le temps, et le repos qui nous sort du temps.
\section[{Chapitre XXXVIII, Différence entre la connaissance de Dieu et celle des Hommes.}]{Chapitre XXXVIII, Différence entre la connaissance de Dieu et celle des Hommes.}
\noindent \pn{53}Nous voyons donc toutes vos créatures, parce qu’elles sont ; et, au rebours, elles sont, parce que vous les voyez. Et nous voyons, au dehors, qu’elles sont ; intérieurement, qu’elles sont bonnes. Mais vous, vous les voyez faites, là où vous les avez vues à faire. Aujourd’hui, nous sommes portés à faire le bien que notre cœur a conçu par votre Esprit. Hier, loin de vous, le mal nous entraînait. Mais vous, ô Dieu, l’unique et souveraine bonté, jamais vous n’avez cessé de faire le bien. Il est quelques bonnes œuvres que nous faisons, grâce à vous, mais elles ne sont pas éternelles. C’est après ces œuvres que nous espérons l’éternel repos dans la gloire de votre sanctification. Mais vous, le seul bien qui n’a besoin de nul autre, vous ne sortez jamais de votre repos ; votre repos, c’est vous-même.\par
Et l’homme peut-il donner à l’homme l’intelligence de ces mystères de gloire ? L’ange à l’ange, ou l’ange à l’homme ? Non ; c’est à vous qu’il faut demander, c’est en vous qu’il faut chercher, c’est à vous-même qu’il faut frapper ; ainsi l’on reçoit, ainsi l’on trouve, ainsi l’on entre (Matth. VII, 8).\par
Ainsi soit-il.
\chapterclose

 


% at least one empty page at end (for booklet couv)
\ifbooklet
  \pagestyle{empty}
  \clearpage
  % 2 empty pages maybe needed for 4e cover
  \ifnum\modulo{\value{page}}{4}=0 \hbox{}\newpage\hbox{}\newpage\fi
  \ifnum\modulo{\value{page}}{4}=1 \hbox{}\newpage\hbox{}\newpage\fi


  \hbox{}\newpage
  \ifodd\value{page}\hbox{}\newpage\fi
  {\centering\color{rubric}\bfseries\noindent\large
    Hurlus ? Qu’est-ce.\par
    \bigskip
  }
  \noindent Des bouquinistes électroniques, pour du texte libre à participation libre,
  téléchargeable gratuitement sur \href{https://hurlus.fr}{\dotuline{hurlus.fr}}.\par
  \bigskip
  \noindent Cette brochure a été produite par des éditeurs bénévoles.
  Elle n’est pas faîte pour être possédée, mais pour être lue, et puis donnée.
  Que circule le texte !
  En page de garde, on peut ajouter une date, un lieu, un nom ; pour suivre le voyage des idées.
  \par

  Ce texte a été choisi parce qu’une personne l’a aimé,
  ou haï, elle a en tous cas pensé qu’il partipait à la formation de notre présent ;
  sans le souci de plaire, vendre, ou militer pour une cause.
  \par

  L’édition électronique est soigneuse, tant sur la technique
  que sur l’établissement du texte ; mais sans aucune prétention scolaire, au contraire.
  Le but est de s’adresser à tous, sans distinction de science ou de diplôme.
  Au plus direct ! (possible)
  \par

  Cet exemplaire en papier a été tiré sur une imprimante personnelle
   ou une photocopieuse. Tout le monde peut le faire.
  Il suffit de
  télécharger un fichier sur \href{https://hurlus.fr}{\dotuline{hurlus.fr}},
  d’imprimer, et agrafer ; puis de lire et donner.\par

  \bigskip

  \noindent PS : Les hurlus furent aussi des rebelles protestants qui cassaient les statues dans les églises catholiques. En 1566 démarra la révolte des gueux dans le pays de Lille. L’insurrection enflamma la région jusqu’à Anvers où les gueux de mer bloquèrent les bateaux espagnols.
  Ce fut une rare guerre de libération dont naquit un pays toujours libre : les Pays-Bas.
  En plat pays francophone, par contre, restèrent des bandes de huguenots, les hurlus, progressivement réprimés par la très catholique Espagne.
  Cette mémoire d’une défaite est éteinte, rallumons-la. Sortons les livres du culte universitaire, cherchons les idoles de l’époque, pour les briser.
\fi

\ifdev % autotext in dev mode
\fontname\font — \textsc{Les règles du jeu}\par
(\hyperref[utopie]{\underline{Lien}})\par
\noindent \initialiv{A}{lors là}\blindtext\par
\noindent \initialiv{À}{ la bonheur des dames}\blindtext\par
\noindent \initialiv{É}{tonnez-le}\blindtext\par
\noindent \initialiv{Q}{ualitativement}\blindtext\par
\noindent \initialiv{V}{aloriser}\blindtext\par
\Blindtext
\phantomsection
\label{utopie}
\Blinddocument
\fi
\end{document}
