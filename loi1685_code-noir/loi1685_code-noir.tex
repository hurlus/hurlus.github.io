%%%%%%%%%%%%%%%%%%%%%%%%%%%%%%%%%
% LaTeX model https://hurlus.fr %
%%%%%%%%%%%%%%%%%%%%%%%%%%%%%%%%%

% Needed before document class
\RequirePackage{pdftexcmds} % needed for tests expressions
\RequirePackage{fix-cm} % correct units

% Define mode
\def\mode{a4}

\newif\ifaiv % a4
\newif\ifav % a5
\newif\ifbooklet % booklet
\newif\ifcover % cover for booklet

\ifnum \strcmp{\mode}{cover}=0
  \covertrue
\else\ifnum \strcmp{\mode}{booklet}=0
  \booklettrue
\else\ifnum \strcmp{\mode}{a5}=0
  \avtrue
\else
  \aivtrue
\fi\fi\fi

\ifbooklet % do not enclose with {}
  \documentclass[twoside]{book} % ,notitlepage
  \usepackage[%
    papersize={105mm, 297mm},
    inner=12mm,
    outer=12mm,
    top=20mm,
    bottom=15mm,
    marginparsep=3pt,
    marginpar=7mm,
  ]{geometry}
  \usepackage[fontsize=9.5pt]{scrextend} % for Roboto
\else\ifav % A5
  \documentclass[twoside]{book} % ,notitlepage
  \usepackage[%
    a5paper
  ]{geometry}
  \usepackage[fontsize=12pt]{scrextend}
\else% A4 2 cols
  \documentclass[twocolumn]{report}
  \usepackage[%
    a4paper,
    inner=15mm,
    outer=10mm,
    top=25mm,
    bottom=18mm,
    marginparsep=0pt,
  ]{geometry}
  \setlength{\columnsep}{20mm}
  \usepackage[fontsize=9.5pt]{scrextend}
\fi\fi

%%%%%%%%%%%%%%
% Alignments %
%%%%%%%%%%%%%%
% before teinte macros

\setlength{\arrayrulewidth}{0.2pt}
\setlength{\columnseprule}{\arrayrulewidth} % twocol

%%%%%%%%%%
% Colors %
%%%%%%%%%%
% before Teinte macros

\usepackage[dvipsnames]{xcolor}
\definecolor{rubric}{HTML}{0c71c3} % the tonic
\def\columnseprulecolor{\color{rubric}}
\colorlet{borderline}{rubric!30!} % definecolor need exact code
\definecolor{shadecolor}{gray}{0.95}
\definecolor{bghi}{gray}{0.5}

%%%%%%%%%%%%%%%%%
% Teinte macros %
%%%%%%%%%%%%%%%%%
%%%%%%%%%%%%%%%%%%%%%%%%%%%%%%%%%%%%%%%%%%%%%%%%%%%
% <TEI> generic (LaTeX names generated by Teinte) %
%%%%%%%%%%%%%%%%%%%%%%%%%%%%%%%%%%%%%%%%%%%%%%%%%%%
% This template is inserted in a specific design
% It is XeLaTeX and otf fonts

\makeatletter % <@@@

\setlength{\parskip}{0pt} % 1pt allow better vertical justification
\setlength{\parindent}{1.5em}

\usepackage{alphalph} % for alph couter z, aa, ab…
\usepackage{blindtext} % generate text for testing
\usepackage{booktabs} % for tables: \toprule, \midrule…
\usepackage[strict]{changepage} % for modulo 4
\usepackage{contour} % rounding words
\usepackage[nodayofweek]{datetime}
\usepackage{enumitem} % <list>
\usepackage{etoolbox} % patch commands
\usepackage{fancyvrb}
\usepackage{fancyhdr}
\usepackage{float}
\usepackage{fontspec} % XeLaTeX mandatory for fonts
\usepackage{footnote} % used to capture notes in minipage (ex: quote)
\usepackage{graphicx}
\usepackage{lettrine} % drop caps
\usepackage{lipsum} % generate text for testing
\usepackage{relsize} % \smaller \larger (ex: quotes in body and footnotes)
\usepackage{manyfoot} % for parallel footnote numerotation
\usepackage[framemethod=tikz,]{mdframed} % maybe used for frame with footnotes inside
\usepackage[defaultlines=2,all]{nowidow} % at least 2 lines by par (works well!)
\usepackage{pdftexcmds} % needed for tests expressions
\usepackage{poetry} % <l>, bad for theater
\usepackage{polyglossia} % bug Warning: "Failed to patch part"
\usepackage[%
  indentfirst=false,
  vskip=1em,
  noorphanfirst=true,
  noorphanafter=true,
  leftmargin=\parindent,
  rightmargin=0pt,
]{quoting}
\usepackage{ragged2e}
\usepackage{setspace} % \setstretch for <quote>
\usepackage{scrextend} % KOMA-common, used for addmargin
\usepackage{tabularx} % <table>
\usepackage[explicit]{titlesec} % wear titles, !NO implicit
\usepackage{tikz} % ornaments
\usepackage{tocloft} % styling tocs
\usepackage[fit]{truncate} % used im runing titles
\usepackage{unicode-math}
\usepackage[normalem]{ulem} % breakable \uline, normalem is absolutely necessary to keep \emph
\usepackage{xcolor} % named colors
\usepackage{xparse} % @ifundefined
\XeTeXdefaultencoding "iso-8859-1" % bad encoding of xstring
\usepackage{xstring} % string tests
\XeTeXdefaultencoding "utf-8"

\defaultfontfeatures{
  % Mapping=tex-text, % no effect seen
  Scale=MatchLowercase,
  Ligatures={TeX,Common},
}
\newfontfamily\zhfont{Noto Sans CJK SC}

% Metadata inserted by a program, from the TEI source, for title page and runing heads
\title{Code noir\par
\medskip
\emph{Édit du Roy, servant de règlement pour le Gouvernement et l’Administration de Justice \& la Police des Isles Françoises de l’Amérique, \& pour la Discipline et le Commerce des Nègres \& Esclaves dans ledit Pays}}
\date{1685}
\author{Lois}
\def\elbibl{Lois. 1685. \emph{Code noir}}
\def\elabstract{%
 
\labelblock{Préface hurlue}

 \noindent  Le \emph{code noir} (1685) est souvent invoqué comme totem, quoique pas toujours analysé. La loi s’écrivait alors dans la langue de Boileau, \emph{« ce que l'on conçoit bien s’énonce clairement »}, montrant distinctement et brutalement un ordre social catholique, oppressant avec la bonne conscience de la charité.\par
 La première question abordée par ce texte n’est pas l’esclavage, mais la persécution des juifs, dès l’article I. Il serait bon de rappeler aux antisémites qui prétendent défendre la mémoire de l’esclavage : il n’y a pas de gagnants dans la course victimaire. Toute hiérarchie, même entre les victimes, produit des inférieurs et des opprimés.\par
 \quoteskip\begin{quoteblock}
 \noindent Louis, par la grâce de Dieu, Roy de France\par
 (I) enjoignons à tous nos Officiers de chasser hors de nos Îles tous les \textbf{Juifs} qui y ont établi leur résidence auxquels comme aux ennemis déclarés du nom Chrétien
 \end{quoteblock}\quoteskip
 \noindent Dès l’article III, le code noir s’en prend ensuite aux protestants. En 1685, Louis XIV venait de révoquer l’\emph{Édit de tolérance} promulgué à Nantes par Henri IV en 1598, pour mettre fin aux guerres de religion. Les \emph{Isles Françoises de l’Amérique} sont l’occasion de commencer le nettoyage religieux, pour en faire un paradis catholique.\par
 \quoteskip\begin{quoteblock}
 \noindent (III) Interdisons tout exercice Public d’autre Religion que la Catholique, Apostolique \& Romaine ; voulons que les contrevenants soient punis comme rebelles \& désobéissants à nos Commandements\par
 (VIII) Déclarons nos Sujets qui ne sont pas de la Religion Catholique, Apostolique \& Romaine incapables de contracter à l’avenir aucuns \textbf{mariages} valables. Déclarons \textbf{bâtards} les enfants qui naîtront de telles conjonctions
 \end{quoteblock}\quoteskip
 \noindent Pour notre époque où deux tiers des enfants naissent entre concubins, on ne mesure pas que le monopole catholique de l’administration du mariage est un arrêt de mort sociale pour les autres confessions. Se marier sous la loi juive ou réformée, c’est n’avoir que des \emph{réputés} bâtards qui n’étaient pas seulement réprouvés, mais surtout, ils ne pouvaient plus défendre leur droit à héritage devant une justice, et donc risquaient d’être écartés de toute propriété. Le génie français n’est pas de terroriser les hérétiques en les brûlant mais de les araser discrètement, doucement et lentement, jusqu’à ce qu’ils ne soient plus qu’humiliation et cendres.\par
 La Réforme menace le pouvoir, non seulement parmi les maîtres, mais surtout si elle se diffuse parmi les \emph{nègres} (ou \emph{esclave}, c’est synonyme dans ce texte).\par
 \quoteskip\begin{quoteblock}
 \noindent (III) Défendons toutes assemblées pour cet effet, lesquelles nous déclarons conventicules, illicites \& séditieuses\par
 (XVI) Défendons pareillement aux esclaves appartenant à différents Maîtres, de s’attrouper, soit le jour ou la nuit, sous prétextes de \textbf{noces} ou autrement, soit chez l’un de leurs Maîtres ou ailleurs, \& encore moins dans les grands chemins ou lieux écartés, à peine de punition corporelle qui ne pourra être moindre que du fouet \& de la fleur de Lys ; \& encas de fréquentes récidives \& autres circonstances aggravantes, pourront être punis de mort : ce que nous laissons à l’arbitrage des Juges. Enjoignons à tous nos sujets de courir sus aux contrevenants, de les arrêter \& conduire en prison, bien qu’ils ne soient Officiers, \& qu’il n’y ait contre eux encore aucun décret.
 \end{quoteblock}\quoteskip
 \noindent La loi garde la mémoire de rares moments de liberté que les esclaves pouvaient prendre sous prétexte de mariage ou de culte. La fleur de Lys est la peine subie par Milady dans \emph{les trois mousquetaires} (Dumas était de père mulâtre), marquée du signe du roi au fer rouge. Le texte fait un appel explicite, à \emph{courir sus aux contrevenants}, à la délation et à la police privée, couvrant l’égoïsme de ces \emph{bons} citoyens qui en profitent pour dénoncer un concurrent commercial.\par
 Est-ce qu’il y a un progrès entre l’esclavage antique et l’esclavage moderne ? Comparer les conditions matérielles serait difficile, mais le statut juridique donne des indices. Pour Aristote comme pour Louis XIV, l’esclave est un \emph{bien meuble}, qui s’achète et se vend, comme du bétail. Mais, l’esclave chrétien a une âme, ce qui complique un peu l’aliénation, c’est un \emph{meuble instruit}. Observons.\par
 \quoteskip\begin{quoteblock}
 \noindent (II) Tous les Esclaves qui seront dans nos Îles seront baptisés \& \textbf{instruits} dans la Religion Catholique, Apostolique \& Romaine.\par
 (XLIV) Déclarons les esclaves être \textbf{meubles}\par
 (XLVI) Dans les saisies des esclaves, seront observées les formalités prescrites par nos Ordonnances \& les Coutumes pour les saisies des choses mobiliaires […] aux exceptions suivantes.\par
 (XLVII) Ne pourront être saisis \& vendus séparément, le Mari \& la Femme \& leurs enfants impubères, s’ils sont tous sous la puissance du même Maître
 \end{quoteblock}\quoteskip
 \noindent Un \emph{meuble instruit} n’a aucune propriété, même son corps, mais puisqu’il faut bien qu’il soit une âme, qui vivra après la mort, alors un roi très chrétien se doit d’administrer son droit au paradis. Le premier devoir de l’esclave ici-bas est de croître et se multiplier (pour le bien de son maître). Les îles françaises de 1685 organisent déjà ce que connaîtront les État-Unis après 1807  : une économie d’\emph{élevage}, et plus seulement de capture, car l’anglais domine déjà les mers et entrave le commerce triangulaire. Le code noir organise la \emph{reproduction} des esclaves, selon une interprétation astucieuse du dogme sacré du mariage.\par
 \quoteskip\begin{quoteblock}
 \noindent (X) Lesdites solennités […] pour les mariages, seront observées tant à l’égard des personnes libres que des esclaves, sans néanmoins que le \textbf{consentement} du père \& de la mère de l’esclave y soit nécessaire, mais celui du Maître seulement.\par
 (XI) Défendons très expressément aux Curés de procéder aux mariages des esclaves, s’ils ne font apparoir du consentement de leurs Maîtres. Défendons aussi aux Maîtres d’user d’aucunes \textbf{contraintes} sur leurs esclaves pour les marier contre leur gré.
 \end{quoteblock}\quoteskip
 \noindent Esclaves et maîtres partagent le même mariage catholique avec consentement sacré entre époux. Les esclaves sont moins libres que des animaux sauvages, puisqu’ils ne peuvent pas enfanter sans l’autorisation de leur maître ; mais un peu plus que les animaux domestiques, puisque les maîtres ne peuvent pas les forcer à s’aimer, ni leur retirer leurs enfants (avant la puberté), contrairement aux veaux ou aux agneaux.\par
 La loi introduit un dernier mécanisme sexiste d’aliénation.\par
 \quoteskip\begin{quoteblock}
 \noindent (XII) Les enfants qui naîtront des mariages entre esclaves seront esclaves \& appartiendront aux Maîtres des femmes esclaves \& non à ceux de leurs maris, si le mari \& la femme ont des Maîtres différents.\par
 (XIII) Voulons que si le mari esclave a épousé une femme libre, les enfants, tant mâles que filles suivent la condition de leur mère, \& soient libres comme elle, nonobstant la servitude de leur père ; \& que si le père est libre \& la mère esclave, les enfants soient esclaves pareillement.
 \end{quoteblock}\quoteskip
 \noindent Les héritiers de l’esclavage aux États-Unis ont noté que la mère esclave avait un statut relativement plus élevé que les hommes de sa condition, contrairement aux maîtres qui dominaient leurs épouses. Ce raffinement de la loi permet de comprendre que la domination patriarcale s’effectue par le \emph{patrimoine}, par la propriété, qui protège la mère et permet d’enfanter en sécurité.\par
 Empêché de propriété, l’esclave mâle ne peut pas protéger sa femme et ses enfants, il n’est que sa force de travail, tandis que son épouse a plus de valeur, puisqu’elle est potentiellement plusieurs forces de travail à naître. La propriété des enfants esclaves passe par la mère (jusqu’à son maître), mais pas par le père, réputé putatif et incertain, contrairement au libre, qui peut reconnaître ses enfants (et adopter).\par
 \bigbreak
 \noindent Ce voyage assez désagréable dans les idées claires et distinctes de la France d’alors devrait convaincre que l’esclavage est l’affaire de tous.\par
 Un blanc bourgeois mâle n’a pas à demander pardon de ce qu’il n’a pas fait, et demander pardon à qui d’ailleurs ? Personne n’est maudit jusqu’à la quinzième génération, pas plus les coupables que les victimes. Déjà du temps de Louis XIV, la responsabilité était personnelle.\par
 Par contre, un blanc, nous, tous, devons comprendre et digérer ce que cette culture française a fait, et peut refaire, avec cette langue assez subtile pour instituer légalement l’injustice la plus odieuse, en toute bonne conscience de grand style et bonne rationalité.\par
 \vfill\null
 \newpage

}
\def\elsource{ \href{https://www.assemblee-nationale.fr/histoire/esclavage/code-noir.pdf}{\dotuline{Assemblée Nationale}}\footnote{\href{https://www.assemblee-nationale.fr/histoire/esclavage/code-noir.pdf}{\url{https://www.assemblee-nationale.fr/histoire/esclavage/code-noir.pdf}}} }
\def\eltitlepage{%
{\centering\parindent0pt
  {\LARGE\addfontfeature{LetterSpace=25}\bfseries Lois\par}\bigskip
  {\Large 1685\par}\bigskip
  {\LARGE
\bigskip\textbf{Code noir}\par
\bigskip\emph{Édit du Roy, servant de règlement pour le Gouvernement et l’Administration de Justice \& la Police des Isles Françoises de l’Amérique, \& pour la Discipline et le Commerce des Nègres \& Esclaves dans ledit Pays}\par

  }
}

}

% Default metas
\newcommand{\colorprovide}[2]{\@ifundefinedcolor{#1}{\colorlet{#1}{#2}}{}}
\colorprovide{rubric}{red}
\colorprovide{silver}{lightgray}
\@ifundefined{syms}{\newfontfamily\syms{DejaVu Sans}}{}
\newif\ifdev
\@ifundefined{elbibl}{% No meta defined, maybe dev mode
  \newcommand{\elbibl}{Titre court ?}
  \newcommand{\elbook}{Titre du livre source ?}
  \newcommand{\elabstract}{Résumé\par}
  \newcommand{\elurl}{http://oeuvres.github.io/elbook/2}
  \author{Éric Lœchien}
  \title{Un titre de test assez long pour vérifier le comportement d’une maquette}
  \date{1566}
  \devtrue
}{}
\let\eltitle\@title
\let\elauthor\@author
\let\eldate\@date




% generic typo commands
\newcommand{\astermono}{\medskip\centerline{\color{rubric}\large\selectfont{\syms ✻}}\medskip\par}%
\newcommand{\astertri}{\medskip\par\centerline{\color{rubric}\large\selectfont{\syms ✻\,✻\,✻}}\medskip\par}%
\newcommand{\asterism}{\bigskip\par\noindent\parbox{\linewidth}{\centering\color{rubric}\large{\syms ✻}\\{\syms ✻}\hskip 0.75em{\syms ✻}}\bigskip\par}%

% lists
\newlength{\listmod}
\setlength{\listmod}{\parindent}
\setlist{
  itemindent=!,
  listparindent=\listmod,
  labelsep=0.2\listmod,
  parsep=0pt,
  % topsep=0.2em, % default topsep is best
}
\setlist[itemize]{
  label=—,
  leftmargin=0pt,
  labelindent=1.2em,
  labelwidth=0pt,
}
\setlist[enumerate]{
  label={\arabic*°},
  labelindent=0.8\listmod,
  leftmargin=\listmod,
  labelwidth=0pt,
}
% list for big items
\newlist{decbig}{enumerate}{1}
\setlist[decbig]{
  label={\bf\color{rubric}\arabic*.},
  labelindent=0.8\listmod,
  leftmargin=\listmod,
  labelwidth=0pt,
}
\newlist{listalpha}{enumerate}{1}
\setlist[listalpha]{
  label={\bf\color{rubric}\alph*.},
  leftmargin=0pt,
  labelindent=0.8\listmod,
  labelwidth=0pt,
}
\newcommand{\listhead}[1]{\hspace{-1\listmod}\emph{#1}}

\renewcommand{\hrulefill}{%
  \leavevmode\leaders\hrule height 0.2pt\hfill\kern\z@}

% General typo
\DeclareTextFontCommand{\textlarge}{\large}
\DeclareTextFontCommand{\textsmall}{\small}

% commands, inlines
\newcommand{\anchor}[1]{\Hy@raisedlink{\hypertarget{#1}{}}} % link to top of an anchor (not baseline)
\newcommand\abbr[1]{#1}
\newcommand{\autour}[1]{\tikz[baseline=(X.base)]\node [draw=rubric,thin,rectangle,inner sep=1.5pt, rounded corners=3pt] (X) {\color{rubric}#1};}
\newcommand\corr[1]{#1}
\newcommand{\ed}[1]{ {\color{silver}\sffamily\footnotesize (#1)} } % <milestone ed="1688"/>
\newcommand\expan[1]{#1}
\newcommand\foreign[1]{\emph{#1}}
\newcommand\gap[1]{#1}
\renewcommand{\LettrineFontHook}{\color{rubric}}
\newcommand{\initial}[2]{\lettrine[lines=2, loversize=0.3, lhang=0.3]{#1}{#2}}
\newcommand{\initialiv}[2]{%
  \let\oldLFH\LettrineFontHook
  % \renewcommand{\LettrineFontHook}{\color{rubric}\ttfamily}
  \IfSubStr{QJ’}{#1}{
    \lettrine[lines=4, lhang=0.2, loversize=-0.1, lraise=0.2]{\smash{#1}}{#2}
  }{\IfSubStr{É}{#1}{
    \lettrine[lines=4, lhang=0.2, loversize=-0, lraise=0]{\smash{#1}}{#2}
  }{\IfSubStr{ÀÂ}{#1}{
    \lettrine[lines=4, lhang=0.2, loversize=-0, lraise=0, slope=0.6em]{\smash{#1}}{#2}
  }{\IfSubStr{A}{#1}{
    \lettrine[lines=4, lhang=0.2, loversize=0.2, slope=0.6em]{\smash{#1}}{#2}
  }{\IfSubStr{V}{#1}{
    \lettrine[lines=4, lhang=0.2, loversize=0.2, slope=-0.5em]{\smash{#1}}{#2}
  }{
    \lettrine[lines=4, lhang=0.2, loversize=0.2]{\smash{#1}}{#2}
  }}}}}
  \let\LettrineFontHook\oldLFH
}
\newcommand{\labelchar}[1]{\textbf{\color{rubric} #1}}
\newcommand{\lnatt}[1]{\reversemarginpar\marginpar[\sffamily\scriptsize #1]{}}
\newcommand{\milestone}[1]{\autour{\footnotesize\color{rubric} #1}} % <milestone n="4"/>
\newcommand\name[1]{#1}
\newcommand\orig[1]{#1}
\newcommand\orgName[1]{#1}
\newcommand\persName[1]{#1}
\newcommand\placeName[1]{#1}
\newcommand{\pn}[1]{\IfSubStr{-—–¶}{#1}% <p n="3"/>
  {\noindent{\bfseries\color{rubric}   ¶  }}
  {{\footnotesize\autour{#1}}}}
\newcommand\reg{}
% \newcommand\ref{} % already defined
\newcommand\sic[1]{#1}
\newcommand\surname[1]{\textsc{#1}}
\newcommand\term[1]{\textbf{#1}}
\newcommand\zh[1]{{\zhfont #1}}


\def\mednobreak{\ifdim\lastskip<\medskipamount
  \removelastskip\nopagebreak\medskip\fi}
\def\bignobreak{\ifdim\lastskip<\bigskipamount
  \removelastskip\nopagebreak\bigskip\fi}

% commands, blocks

\newcommand{\byline}[1]{\bigskip{\RaggedLeft{#1}\par}\bigskip}
% \setlength{\RaggedLeftLeftskip}{2em plus \leftskip}
\newcommand{\bibl}[1]{{\RaggedLeft\normalfont #1\par}}
\newcommand{\biblitem}[1]{{\noindent\hangindent=\parindent   #1\par}}
\newcommand{\castItem}[1]{{\noindent\hangindent=\parindent #1\par}}
\newcommand{\dateline}[1]{\medskip{\RaggedLeft{#1}\par}\bigskip}
\newcommand{\docAuthor}[1]{{\large\bigskip #1 \par\bigskip}}
\newcommand{\docDate}[1]{#1 \ifvmode\par\fi }
\newcommand{\docImprint}[1]{\ifvmode\medskip\fi #1 \ifvmode\par\fi }
\newcommand{\labelblock}[1]{\medbreak{\noindent\color{rubric}\bfseries #1}\par\mednobreak}
\newcommand{\question}[1]{\bigbreak{\RaggedRight\noindent\emph{#1}\par}\mednobreak}
\newcommand{\salute}[1]{\bigbreak{#1}\par\medbreak}
\newcommand{\signed}[1]{\medskip{\RaggedLeft #1\par}\bigbreak} % supposed bottom
\newcommand{\speaker}[1]{\medskip{\Centering\sffamily #1 \par\nopagebreak}} % supposed bottom
\newcommand{\stagescene}[1]{{\Centering\sffamily\textsf{#1}\par}\bigskip}
\newcommand{\stageblock}[1]{\begingroup\leftskip\parindent\noindent\it\sffamily\footnotesize #1\par\endgroup} % left margin, better than list envs
\newcommand{\lpar}[1]{\noindent\hangindent=2\parindent  #1\par} % sp/l
\newcommand{\trailer}[1]{{\Centering\bigskip #1\par}} % sp/l

% environments for blocks (some may become commands)
\newenvironment{borderbox}{}{} % framing content
\newenvironment{citbibl}{\ifvmode\hfill\fi}{\ifvmode\par\fi }
\newenvironment{msHead}{\vskip6pt}{\par}
\newenvironment{msItem}{\vskip6pt}{\par}


% environments for block containers
\newenvironment{argument}{\itshape\parindent0pt}{\bigskip}
\newenvironment{biblfree}{}{\ifvmode\par\fi }
\newenvironment{bibitemlist}[1]{%
  \list{\@biblabel{\@arabic\c@enumiv}}%
  {%
    \settowidth\labelwidth{\@biblabel{#1}}%
    \leftmargin\labelwidth
    \advance\leftmargin\labelsep
    \@openbib@code
    \usecounter{enumiv}%
    \let\p@enumiv\@empty
    \renewcommand\theenumiv{\@arabic\c@enumiv}%
  }
  \sloppy
  \clubpenalty4000
  \@clubpenalty \clubpenalty
  \widowpenalty4000%
  \sfcode`\.\@m
}%
{\def\@noitemerr
  {\@latex@warning{Empty `bibitemlist' environment}}%
\endlist}
\newenvironment{docTitle}{\LARGE\bigskip\bfseries\onehalfspacing}{\bigskip}
% leftskip makes big bugs in Lexmark printing \sffamily
\newenvironment{epigraph}{\begin{addmargin}[2\parindent]{0em}\sffamily\large\setstretch{0.95}}{\end{addmargin}\bigskip}
\newenvironment{quoteblock}
  {\begin{quoting}\setstretch{0.9}} %
  {\end{quoting}}
\newenvironment{frametext}
  {\begin{mdframed}[default]} %
  {\end{mdframed}}

\quotingsetup{vskip=0pt}
\newcommand{\quoteskip}{\medskip}
\newenvironment{titlePage}
  {\Centering}
  {}






% table () is preceded and finished by custom command
\renewcommand\tabularxcolumn[1]{m{#1}}% for vertical centering text in X column
\newcommand{\tableopen}[1]{%
  \ifnum\strcmp{#1}{wide}=0{%
    \begin{center}
  }
  \else\ifnum\strcmp{#1}{long}=0{%
    \begin{center}
  }
  \else{%
    \begin{center}
  }
  \fi\fi
}
\newcommand{\tableclose}[1]{%
  \ifnum\strcmp{#1}{wide}=0{%
    \end{center}
  }
  \else\ifnum\strcmp{#1}{long}=0{%
    \end{center}
  }
  \else{%
    \end{center}
  }
  \fi\fi
}


% text structure
\newcommand\chapteropen{} % before chapter title
\newcommand\chaptercont{} % after title, argument, epigraph…
\newcommand\chapterclose{} % maybe useful for multicol settings
\setcounter{secnumdepth}{-2} % no counters for hierarchy titles
\setcounter{tocdepth}{5} % deep toc
\renewcommand\tableofcontents{\@starttoc{toc}}
% toclof format
% \renewcommand{\@tocrmarg}{0.1em} % Useless command?
% \renewcommand{\@pnumwidth}{0.5em} % {1.75em}
\renewcommand{\@cftmaketoctitle}{}
\setlength{\cftbeforesecskip}{\z@ \@plus.2\p@}

\@ifclassloaded{article}{%
  \typeout{class: article}%
}{%
  \renewcommand{\cftchapfont}{}
  \renewcommand{\cftchapdotsep}{\cftdotsep}
  \renewcommand{\cftchapleader}{\normalfont\cftdotfill{\cftchapdotsep}}
  \renewcommand{\cftchappagefont}{\bfseries}
  \setlength{\cftbeforechapskip}{0pt}
  \setlength{\cftchapnumwidth}{1em}
}
\renewcommand{\cftsecfont}{\normalfont}
\renewcommand{\cftsecpagefont}{\normalfont}
% \renewcommand{\cftsubsecfont}{\small\relax}
\renewcommand{\cftsecdotsep}{\cftdotsep}
\renewcommand{\cftsecpagefont}{\normalfont}
\renewcommand{\cftsecleader}{\normalfont\cftdotfill{\cftsecdotsep}}
\setlength{\cftsecindent}{1em}
\setlength{\cftsubsecindent}{2em}
\setlength{\cftsubsubsecindent}{3em}
\setlength{\cftsecnumwidth}{1em}
\setlength{\cftsubsecnumwidth}{1em}
\setlength{\cftsubsubsecnumwidth}{1em}

% footnotes
\newif\ifheading
\newcommand*{\fnmarkscale}{\ifheading 0.70 \else 1 \fi}
\renewcommand\footnoterule{\vspace*{0.3cm}\hrule height \arrayrulewidth width 3cm \vspace*{0.3cm}}
\setlength\footnotesep{1.5\footnotesep} % footnote separator
\renewcommand\@makefntext[1]{\parindent 1.5em \noindent \hb@xt@1.8em{\hss{\normalfont\@thefnmark . }}#1} % no superscipt in foot
\patchcmd{\@footnotetext}{\footnotesize}{\footnotesize\sffamily}{}{} % before scrextend, hyperref
\DeclareNewFootnote{A}[alph] % for editor notes
\renewcommand*{\thefootnoteA}{\alphalph{\value{footnoteA}}} % z, aa, ab…

% poem
\setlength{\poembotskip}{0pt}
\setlength{\poemtopskip}{0pt}
\setlength{\poemindent}{0pt}
\setlength{\poemmaxlinewd}{\linewidth}
\poemlinenumsfalse

%   see https://tex.stackexchange.com/a/34449/5049
\def\truncdiv#1#2{((#1-(#2-1)/2)/#2)}
\def\moduloop#1#2{(#1-\truncdiv{#1}{#2}*#2)}
\def\modulo#1#2{\number\numexpr\moduloop{#1}{#2}\relax}

% orphans and widows, nowidow package in test
% from memoir package
\clubpenalty=9996
\widowpenalty=9999
\brokenpenalty=4991
\predisplaypenalty=10000
\postdisplaypenalty=1549
\displaywidowpenalty=1602
\hyphenpenalty=400
% report h or v overfull ?
\hbadness=4000
\vbadness=4000
% good to avoid lines too wide
\emergencystretch 3em
\pretolerance=750
\tolerance=2000
\def\Gin@extensions{.pdf,.png,.jpg,.mps,.tif}

\PassOptionsToPackage{hyphens}{url} % before hyperref and biblatex, which load url package
\usepackage{hyperref} % supposed to be the last one, :o) except for the ones to follow
\hypersetup{
  % pdftex, % no effect
  pdftitle={\elbibl},
  % pdfauthor={Your name here},
  % pdfsubject={Your subject here},
  % pdfkeywords={keyword1, keyword2},
  bookmarksnumbered=true,
  bookmarksopen=true,
  bookmarksopenlevel=1,
  pdfstartview=Fit,
  breaklinks=true, % avoid long links, overrided by url package
  pdfpagemode=UseOutlines,    % pdf toc
  hyperfootnotes=true,
  colorlinks=false,
  pdfborder=0 0 0,
  % pdfpagelayout=TwoPageRight,
  % linktocpage=true, % NO, toc, link only on page no
}
\urlstyle{same} % after hyperref



\makeatother % /@@@>
%%%%%%%%%%%%%%
% </TEI> end %
%%%%%%%%%%%%%%

\setmainlanguage{french}
%%%%%%%%%%%%%
% footnotes %
%%%%%%%%%%%%%
\renewcommand{\thefootnote}{\bfseries\textcolor{rubric}{\arabic{footnote}}} % color for footnote marks

%%%%%%%%%
% Fonts %
%%%%%%%%%
% \linespread{0.90} % too compact, keep font natural
\ifav % A5
  \usepackage{DejaVuSans} % correct
  \setsansfont{DejaVuSans} % seen, if not set, problem with printer
\else\ifbooklet
  \usepackage[]{roboto} % SmallCaps, Regular is a bit bold
  \setmainfont[
    ItalicFont={Roboto Light Italic},
  ]{Roboto}
  \setsansfont{Roboto Light} % seen, if not set, problem with printer
  \newfontfamily\fontrun[]{Roboto Condensed Light} % condensed runing heads
\else
  \usepackage[]{roboto} % SmallCaps, Regular is a bit bold
  \setmainfont[
    ItalicFont={Roboto Italic},
  ]{Roboto Light}
  \setsansfont{Roboto Light} % seen, if not set, problem with printer
  \newfontfamily\fontrun[]{Roboto Condensed Light} % condensed runing heads
\fi\fi
\renewcommand{\LettrineFontHook}{\bfseries\color{rubric}}
% \renewenvironment{labelblock}{\begin{center}\bfseries\color{rubric}}{\end{center}}

%%%%%%%%
% MISC %
%%%%%%%%

\setdefaultlanguage[frenchpart=false]{french} % bug on part


\newenvironment{quotebar}{%
    \def\FrameCommand{{\color{rubric!10!}\vrule width 0.5em} \hspace{0.9em}}%
    \def\OuterFrameSep{0pt} % séparateur vertical
    \MakeFramed {\advance\hsize-\width \FrameRestore}
  }%
  {%
    \endMakeFramed
  }
\renewenvironment{quoteblock}% may be used for ornaments
  {%
    \savenotes
    \setstretch{0.9}
    \begin{quotebar}
    \smallskip
  }
  {%
    \smallskip
    \end{quotebar}
    \spewnotes
  }


\renewcommand{\headrulewidth}{\arrayrulewidth}
\renewcommand{\headrule}{{\color{rubric}\hrule}}
\renewcommand{\lnatt}[1]{\marginpar{\sffamily\scriptsize #1}}

\titleformat{name=\chapter} % command
  [display] % shape
  {\vspace{1.5em}\centering} % format
  {} % label
  {0pt} % separator between n
  {}
[{\color{rubric}\huge\textbf{#1}}\bigskip] % after code
% \titlespacing{command}{left spacing}{before spacing}{after spacing}[right]
\titlespacing*{\chapter}{0pt}{-2em}{0pt}[0pt]

\titleformat{name=\section}
  [display]{}{}{}{}
  [\vbox{\color{rubric}\large\centering\textbf{#1}}]
\titlespacing{\section}{0pt}{0pt plus 4pt minus 2pt}{\baselineskip}

\titleformat{name=\subsection}
  [block]
  {}
  {} % \thesection
  {} % separator \arrayrulewidth
  {}
[\vbox{\large\textbf{#1}}]
% \titlespacing{\subsection}{0pt}{0pt plus 4pt minus 2pt}{\baselineskip}

\ifaiv
  \fancypagestyle{main}{%
    \fancyhf{}
    \setlength{\headheight}{1.5em}
    \fancyhead{} % reset head
    \fancyfoot{} % reset foot
    \fancyhead[L]{\truncate{0.45\headwidth}{\fontrun\elbibl}} % book ref
    \fancyhead[R]{\truncate{0.45\headwidth}{ \fontrun\nouppercase\leftmark}} % Chapter title
    \fancyhead[C]{\thepage}
  }
  \fancypagestyle{plain}{% apply to chapter
    \fancyhf{}% clear all header and footer fields
    \setlength{\headheight}{1.5em}
    \fancyhead[L]{\truncate{0.9\headwidth}{\fontrun\elbibl}}
    \fancyhead[R]{\thepage}
  }
\else
  \fancypagestyle{main}{%
    \fancyhf{}
    \setlength{\headheight}{1.5em}
    \fancyhead{} % reset head
    \fancyfoot{} % reset foot
    \fancyhead[RE]{\truncate{0.9\headwidth}{\fontrun\elbibl}} % book ref
    \fancyhead[LO]{\truncate{0.9\headwidth}{\fontrun\nouppercase\leftmark}} % Chapter title, \nouppercase needed
    \fancyhead[RO,LE]{\thepage}
  }
  \fancypagestyle{plain}{% apply to chapter
    \fancyhf{}% clear all header and footer fields
    \setlength{\headheight}{1.5em}
    \fancyhead[L]{\truncate{0.9\headwidth}{\fontrun\elbibl}}
    \fancyhead[R]{\thepage}
  }
\fi

\ifav % a5 only
  \titleclass{\section}{top}
\fi

\newcommand\chapo{{%
  \vspace*{-3em}
  \centering\parindent0pt % no vskip ()
  \eltitlepage
  \bigskip
  {\color{rubric}\hline}
  \bigskip
  {\Large TEXTE LIBRE À PARTICIPATIONS LIBRES\par}
  \centerline{\small\color{rubric} {\href{https://hurlus.fr}{\dotuline{hurlus.fr}}}, tiré le \today}\par
  \bigskip
}}

\newcommand\cover{{%
  \thispagestyle{empty}
  \centering\parindent0pt
  \eltitlepage
  \vfill\null
  {\color{rubric}\setlength{\arrayrulewidth}{2pt}\hline}
  \vfill\null
  {\Large TEXTE LIBRE À PARTICIPATIONS LIBRES\par}
  \centerline{\href{https://hurlus.fr}{\dotuline{hurlus.fr}}, tiré le \today}\par
}}

\begin{document}
\pagestyle{empty}
\ifbooklet{
  \cover\newpage
  \thispagestyle{empty}\hbox{}\newpage
  \cover\newpage\noindent Les voyages de la brochure\par
  \bigskip
  \begin{tabularx}{\textwidth}{l|X|X}
    \textbf{Date} & \textbf{Lieu}& \textbf{Nom/pseudo} \\ \hline
    \rule{0pt}{25cm} &  &   \\
  \end{tabularx}
  \newpage
  \addtocounter{page}{-4}
}\fi

\thispagestyle{empty}
\ifaiv
  \twocolumn[\chapo]
\else
  \chapo
\fi
{\it\elabstract}
\bigskip
\makeatletter\@starttoc{toc}\makeatother % toc without new page
\bigskip

\pagestyle{main} % after style
\setcounter{footnote}{0}
\setcounter{footnoteA}{0}
  
\labelblock{Donné à Versailles au mois de Mars 1685.}

\noindent Louis, par la grâce de Dieu, Roy de France \& de Navarre : À tous présents \& à venir : Salut. Comme nous devons également nos soins à tous les Peuples que la divine Providence a mis sous notre obéissance, Nous avons bien voulu faire examiner en notre présence les mémoires qui nous ont été envoyés par nos officiers de nos Îles de l’Amérique, par lesquels ayant été informé du besoin qu’ils ont de notre autorité \& de notre justice pour y maintenir la discipline de l’Église Catholique, Apostolique \& Romaine, \& pour y régler ce qui concerne l’État \& la qualité des Esclaves dans lesdites Îles ; \& désirant y pourvoir \& leur faire connaître qu’encore qu’ils habitent des climats infiniment éloignés de notre séjour ordinaire, nous leur sommes toujours présent non seulement par l’étendue de notre puissance, mais encore par la promptitude de notre application à les soutenir dans leurs nécessités. À ces causes de l’avis de notre Conseil \& de notre Certaine science, pleine puissance \& autorité Royale, nous avons dit, statué \& ordonné, disons, statuons et ordonnons, voulons \& nous plaît ce qui ensuit.\par
\bigbreak
\labelchar{{\scshape Article} I.} Voulons \& entendons que l’Édit du feu Roi de glorieuse mémoire, notre très honoré Seigneur \& Père, du 23 avril 1615, soit exécuté dans nos Îles ; ce faisant, enjoignons à tous nos Officiers de chasser hors de nos Îles tous les Juifs qui y ont établi leur résidence auxquels comme aux ennemis déclarés du nom Chrétien. Nous commandons d’en sortir dans trois mois à compter du jour de la publication des Présentes, à peine de confiscation de corps \& de biens.\par
\labelchar{II.} Tous les Esclaves qui seront dans nos Îles seront baptisés \& instruits dans la Religion Catholique, Apostolique \& Romaine. Enjoignons aux Habitants qui achèteront des Nègres nouvellement arrivés d’en avertir les Gouverneur \& Intendant desdites Îles dans huitaine au plus tard, à peine d’amende arbitraire, lesquels donneront les ordres nécessaires pour les faire instruire \& baptiser dans le temps convenable.\par
\labelchar{III.} Interdisons tout exercice Public d’autre Religion que la Catholique, Apostolique \& Romaine ; voulons que les contrevenants soient punis comme rebelles \& désobéissants à nos Commandements, Défendons toutes assemblées pour cet effet, lesquelles nous déclarons conventicules, illicites \& séditieuses, sujets à la même peine qui aura lieu, même contre les Maîtres qui les permettront \& souffriront à l’égard de leurs Esclaves.\par
\labelchar{IV.} Ne seront préposés aucuns Commandeurs à la direction des Nègres, qui ne fassent profession de la Religion Catholique, Apostolique \& Romaine, à peine de confiscation desdits Nègres contre les Maîtres qui les auront préposés \& de punition arbitraire Contre les commandeurs qui auront accepté ladite direction.\par
\labelchar{V.} Défendons à nos sujets de la religion prétendue réformée d’apporter aucun trouble ni empêchement a nos autres Sujets, même à leurs esclaves, dans le libre exercice de la Religion Catholique, Apostolique \& Romaine, à peine de punition exemplaire.\par
\labelchar{VI.} Enjoignons à tous nos Sujets, de quelque qualité \& condition qu’ils soient d’observer les jours de Dimanches \& Fêtes, qui sont gardés par nos Sujets de la Religion Catholique, Apostolique \& Romaine. Leur défendons de travailler ni de faire travailler leurs esclaves auxdits jours depuis l’heure de minuit jusqu’à l’autre minuit, soit à la culture de la terre, à la manufacture des Sucres, \& à tous autres ouvrages, à peine d’amende \& de punition arbitraire contre les Maîtres, \& de confiscation tant des Sucres que desdits Esclaves qui seront surpris par nos Officiers dans le travail.\par
\labelchar{VII.} Leur défendons pareillement de tenir le marché des Nègres \& de tous autres marchés lesdits jours sur pareille peines, \& de confiscation des marchandises qui se trouveront alors au marché \& d’amende arbitraire contre les Marchands.\par
\labelchar{VIII.} Déclarons nos Sujets qui ne sont pas de la Religion Catholique, Apostolique \& Romaine incapables de contracter à l’avenir aucuns mariages valables. Déclarons bâtards les enfants qui naîtront de telles conjonctions, que nous voulons être tenus \& réputés, tenons \& réputons, pour vrais concubinages.\par
\labelchar{IX.} Les hommes libres qui auront eu un ou plusieurs enfants de leur concubinage avec des esclaves, ensemble les Maîtres qui les auront soufferts, seront chacun condamnés en une amende de 2 000 livres de sucre, \& s’ils sont les Maîtres de l’esclave de laquelle ils auront eu lesdits enfants, voulons, outre l’amende, qu’ils soient privés de l’esclave \& des enfants, \& qu’elle \& eux soient confisqués au profit l’Hôpital, sans jamais pouvoir être affranchis. N’entendons toutefois le présent article avoir lieu lorsque l’homme libre qui n’était point marié à une autre personne durant son concubinage avec son esclave, épousera dans les formes observées par l’Église ladite esclave, qui sera affranchie par ce moyen \& les enfants rendus libres \& légitimes.\par
\labelchar{X.} Lesdites solennités prescrites par l’Ordonnance de Blois articles 40, 41, 42, [1579] \& par la Déclaration du mois de novembre 1639, pour les mariages, seront observées tant à l’égard des personnes libres que des esclaves, sans néanmoins que le consentement du père \& de la mère de l’esclave y soit nécessaire, mais celui du Maître seulement.\par
\labelchar{XI.} Défendons très expressément aux Curés de procéder aux mariages des esclaves, s’ils ne font apparoir du consentement de leurs Maîtres. Défendons aussi aux Maîtres d’user d’aucunes contraintes sur leurs esclaves pour les marier contre leur gré.\par
\labelchar{XII.} Les enfants qui naîtront des mariages entre esclaves seront esclaves \& appartiendront aux Maîtres des femmes esclaves \& non à ceux de leurs maris, si le mari \& la femme ont des Maîtres différents.\par
\labelchar{XIII.} Voulons que si le mari esclave a épousé une femme libre, les enfants, tant mâles que filles suivent la condition de leur mère, \& soient libres comme elle, nonobstant la servitude de leur père ; \& que si le père est libre \& la mère esclave, les enfants soient esclaves pareillement.\par
\labelchar{XIV.} Les Maîtres seront tenus de faire enterrer en Terre-Sainte dans les Cimetières destinés a cet effet, leurs esclaves baptisés : \& à l’égard de ceux qui mourront sans avoir reçu le Baptême, ils seront enterrés la nuit dans quelque champ voisin du lieu où ils seront décédés.\par
\labelchar{XV.} Défendons aux esclaves de porter aucunes armes offensives, ni de gros bâtons, à peine du fouet, \& de confiscation des armes au profit de celui qui les en trouvera saisis ; à l’exception seulement de ceux qui seront envoyés à la chasse par leur Maître, \& qui seront porteurs de leurs billets ou marques connues.\par
\labelchar{XVI.} Défendons pareillement aux esclaves appartenant à différents Maîtres, de s’attrouper, soit le jour ou la nuit, sous prétextes de noces ou autrement, soit chez l’un de leurs Maîtres ou ailleurs, \& encore moins dans les grands chemins ou lieux écartés, à peine de punition corporelle qui ne pourra être moindre que du fouet \& de la fleur de Lys ; \& encas de fréquentes récidives \& autres circonstances aggravantes, pourront être punis de mort : ce que nous laissons à l’arbitrage des Juges. Enjoignons à tous nos sujets de courir sus aux contrevenants, de les arrêter \& conduire en prison, bien qu’ils ne soient Officiers, \& qu’il n’y ait contre eux encore aucun décret.\par
\labelchar{XVII.} Les Maîtres qui seront convaincus d’avoir permis ou toléré telles assemblées composées d’autres esclaves que de ceux qui leur appartiennent, seront condamnés en leurs propres \& privés noms, de réparer tout le dommage qui aura été fait à ses voisins à l’occasion desdites assemblées, \& à 10 écus d’amende pour la première fois, \& au double en cas de récidive.\par
\labelchar{XVIII.} Défendons aux esclaves de vendre des cannes de sucre, pour quelque cause ou occasion que ce soit, même avec la permission de leur Maître, à peine du fouet contre les esclaves, de 10 livres tournois contre leurs Maîtres qui l’auronr permis, \& de pareille amende contre l’acheteur.\par
\labelchar{XIX.} Leur défendons aussi d’exposer en vente au marché ni de porter dans des maisons particulières pour vendre aucune sorte de denrées, même des fruits, légumes, bois à brûler, herbes pour la nourriture des bestiaux \& leurs manufactures, sans permission expresse de leurs Maîtres par un billet, ou par des marques connues, à peine de revendication des choses ainsi vendues, sans restitution du prix par leurs Maîtres, \& de 6 livres tournois d’amende a leur profit contre les acheteurs.\par
\labelchar{XX.} Voulons à cet effet que deux personnes soient préposées par nos Officiers dans chaque marché pour examiner les denrées \& marchandises qui y seront apportées par les esclaves, ensemble les billets \& marques de leurs Maîtres.\par
\labelchar{XXI.} Permettons à tous nos sujets habitants des Îles de se saisir de toutes les choses dont ils trouveront les esclaves chargés, lorsqu’ils n’auront point de billets de leurs Maîtres, ni de marques connues pour être rendues incessamment à leurs Maîtres, si les habitations sont voisines du lieu où les esclaves auront été surpris en délit, sinon elles seront incessamment envoyées à l’Hôpital pour y être en dépôt jusqu’à ce que les Maîtres en aient été avertis.\par
\labelchar{XXII.} Seront tenus les Maîtres de fournir par chacune semaine à leurs esclaves âgés de 10 ans \& au-dessus pour leur nourriture, deux pots \& demi mesure du pays de farine de Manioc, ou trois cassaves pesant 2 livres \& demie au moins, ou choses équivalant, avec 2 livres de bœuf salé, ou 3 livres de poisson ou autres choses à proportion, \& aux enfants, depuis qu’ils sont sevrés jusqu’à l’âge de 10 ans, la moitié des vivres ci-dessus.\par
\labelchar{XXIII.} Leur défendons de donner aux esclaves de l’eau-de-vie de canne ou guildent, pour tenir lieu de la subsistance mentionnée au précédent article.\par
\labelchar{XXIV.} Leur défendons pareillement de se décharger de la nourriture \& subsistance de leurs esclaves en leur permettant de travailler certain jour de la semaine pour leur compte particulier.\par
\labelchar{XXV.} Seront tenus les Maîtres de fournir à chacun esclave, par chacun an deux habits de toile ou quatre aunes de toile, au gré desdits Maîtres.\par
\labelchar{XXVI.} Les esclaves qui ne seront point nourris, vêtus \& entretenus par leurs Maîtres selon que nous l’avons ordonné par ces Présentes, pourront en donner avis à notre Procureur \& mettre leurs mémoires entre ses mains, sur lesquels \& même d’office, si les avis viennent d’ailleurs, les Maîtres seront poursuivis à sa Requête \& sans frais, ce que nous voulons être observé pour les crieries \& traitements barbares \& inhumains des Maîtres envers leurs esclaves.\par
\labelchar{XXVII.} Les esclaves infirmes par vieillesse, maladie ou autrement, soit que la maladie soit incurable ou non, seront nourris \& entretenus par leurs Maîtres, \& en cas qu’ils les eussent abandonnés, lesdits esclaves seront adjugés à l’Hôpital, auquel les Maîtres seront condamnés de payer 6 sols par chacun jour pour la nourriture \& l’entretien de chacun esclave.\par
\labelchar{XXVIII.} Déclarons les esclaves ne pouvoir rien avoir qui ne soit à leur Maître, \& tout ce qui leur vient par industrie ou par la libéralité d’autres personnes, ou autrement à quelque titre que ce soit, être acquis en pleine propriété à leur Maître, sans que les enfants des esclaves, leurs pères \& mères, leurs parents \& tous autres libres ou esclaves puissent rien prétendre par successions, dispositions entre vifs ou à cause mort, lesquelles dispositions nous déclarons nulles, ensemble toutes les promesses \& obligations qu’ils auraient faites, comme étant faites par gens incapables de disposer \& contracter de leur chef.\par
\labelchar{XXIX.} Voulons néanmoins que les Maîtres soient tenus de ce que leurs esclaves auront fait par leur commandement, ensemble de ce qu’ils auront géré \& négocié dans la boutique, \& pour l’espèce particulière du commerce, à laquelle les Maîtres les aura préposés : ils seront tenus seulement jusqu’à concurrence de ce qui aura tourné au profit des Maîtres ; le pécule desdits esclaves que leurs Maîtres leur auront permis sera tenu, après que leurs Maîtres en auront déduit par préférence ce qui pourra leur être du, sinon que le pécule consistant en tout ou partie en marchandises, dont les esclaves auraient permission de faire trafic à part, sur lesquelles leurs Maîtres viendront seulement par contribution au sol la livre avec les autres créanciers.\par
\labelchar{XXX.} Ne pourront les esclaves être pourvus d’Office ni de Commission ayant quelque fonction publique, ni être constitués agents par autres que leurs Maîtres, pour gérer \& administrer aucun négoce ni arbitre, en perte, ou témoins, tant en Matière Civile que Criminelle \& en cas qu’ils soient ouïs en témoignage, leur déposition ne serviront que de mémoires pour aider les Juges à s’éclaircir d’ailleurs, sans qu’on en puisse tirer aucune présomption ni conjecture ni adminlcule de preuve.\par
\labelchar{XXXI.} Ne pourront aussi les esclaves être partie, ni en jugement ni en Matière Civile, tant en demandant que défendant, ni être parties Civile en Matière Criminelle, \& de poursuivre en Matière Criminelle la réparation des outrages \& excès qui auront été commis contre les esclaves.\par
\labelchar{XXXII.} Pourront les esclaves être poursuivis criminellement, sans qu’il soit besoin de rendre leur Maître partie, sinon en cas de complicité : \& seront lesdits esclaves accusés, jugés en première Instance par les Juges ordinaires \& par appel au Conseil Souverain sur la même instruction, avec les mêmes formalités que les personnes libres.\par
\labelchar{XXXIII.} L’esclave qui aura frappé son Maître, ou la femme de son Maître, sa Maîtresse, ou leurs enfants avec contusion de sang, ou au visage, sera puni de mort.\par
\labelchar{XXXIV.} Et quant aux excès \& voies de fait qui seront commis par les esclaves contre les personnes libres : Voulons qu’ils soient sévèrement punis, même de mort s’il y échet.\par
\labelchar{XXXV.} Les vols qualifiés, même ceux de chevaux, cavales, mulets, bœufs \& vaches qui auront été faits par les esclaves, ou par ceux affranchis, seront punis de peines afflictives, même de mort si le cas le requiert.\par
\labelchar{XXXVI.} Les vols de moutons, chèvres, cochons, volailles, cannes de sucre, pois, manioc, ou autres légumes faits par les esclaves, seront punis selon la qualité du vol, par les Juges qui pourront, s’il y échet, les condamner d’être battus de verges par l’Exécuteur de la Haute-Justice \& marqués à l’épaule d’une fleur de lys.\par
\labelchar{XXXVII.} Seront tenus les Maîtres en cas de vol ou autrement des dommages causés par leurs esclaves, outre la peine corporelle des esclaves, réparer les torts en leur nom, s’ils n’aiment mieux abandonner l’esclave à celui auquel le tort a été fait, ce qu’ils seront tenus d’opter dans 3 jours, à compter du jour de la condamnation, autrement ils en seront déchus.\par
\labelchar{XXXVIII.} L’esclave fugitif qui aura été en fuite pendant un mois à compter du jour que son Maître l’aura dénoncé en Justice, aura les oreilles coupées, \& sera marqué d’une fleur de lys sur une épaule : \& s’il récidive un autre mois à compter pareillement du jour de la dénonciation, aura le jarret coupé \& sera marqué d’une fleur de lys sur l’autre épaule, \& la troisième fois, il sera puni de mort.\par
\labelchar{XXXIX.} Les affranchis qui auront donné retraite dans leurs maisons aux esclaves fugitifs, seront condamnés par corps envers le Maître en l’amende de 300 livres de sucre par chacun jour de rétention.\par
\labelchar{XL.} L’esclave puni de mort sur la dénonciation de son Maître non complice du crime pour lequel il aura été condamné, sera estimé avant l’exécution par deux des principaux habitants de l’Île qui seront nommés d’office par le Juge, \& le prix de l’estimation en sera payé au Maître pour à quoi satisfaire il sera imposé par l’Intendant sur chacune tête de Nègre payant droit, la somme portée par l’estimation, laquelle sera régalée sur chacun desdits Nègres, \& levée par le Fermier du Domaine Royal d’Occident pour éviter à frais.\par
\labelchar{XLI.} Défendons aux Juges, à nos Procureurs \& aux Greffiers de prendre aucune taxe dans les Procès Criminels contre les esclaves, à peine de concussion.\par
\labelchar{XLII.} Pourront pareillement les Maîtres, lorsqu’ils croiront que leurs esclaves l’auront mérité, les faire enchaîner \& les faire battre de verges ou de cordes, leur défendant de leur donner la torture, ni de leur faire aucune mutilation de membre, à peine de confiscation des esclaves \& d’être procédé contre les Maîtres extraordinairement.\par
\labelchar{XLIII.} Enjoignons à nos Officiers de poursuivre criminellement les Maîtres ou les Commandeurs qui auront tué un esclave sous leur puissance ou sous leur direction, \& de punir le Maître selon l’atrocité des circonstances , \& en cas qu’il y ait lieu de l’absolution, permettons à nos Officiers de renvoyer tant les Maîtres que Commandeurs absous, sans qu’ils aient besoin de nos grâces.\par
\labelchar{XLIV.} Déclarons les esclaves être meubles, \& comme tels entrent en la communauté, n’avoir point de suite par hypothèque, \& partager également entre les cohéritiers sans préciput, ni droit d’aînesse, n’être sujets au douaire Coutumier, au Retrait Féodal \& Lignager, aux Droits Féodaux \& Seigneuriaux, aux formalités des Décrets, ni au retranchement des quatre Quints, en cas de disposition à cause de mort ou testamentaire.\par
\labelchar{XLV.} N’entendons toutefois priver nos sujets de la faculté de les stipuler propres à leurs personnes \& aux leurs de leur côté \& ligne, ainsi qu’il se pratique pour les sommes de deniers \& autres choses mobiliaires.\par
\labelchar{XLVI.} Dans les saisies des esclaves, seront observées les formalités prescrites par nos Ordonnances \& les Coutumes pour les saisies des choses mobiliaires. Voulons que les deniers en provenant soient distribués par ordre des saisies ; \& en cas de déconfiture au sol la livre, après que les dettes privilégiées auront été payées \& généralement que la condition des esclaves soit réglée en toutes affaires, comme celle des autres choses mobiliaires, aux exceptions suivantes.\par
\labelchar{XLVII.} Ne pourront être saisis \& vendus séparément, le Mari \& la Femme \& leurs enfants impubères, s’ils sont tous sous la puissance du même Maître, déclarons nulles les saisies \& ventes séparées qui en seront faites, ce que nous voulons avoir lieu dans les aliénations volontaires, si peine que feront les aliénateurs d’être privés de celui ou de ceux qu’ils auront gardés qui seront adjugés aux acquéreurs, sans qu’ils soient tenus de faire aucun supplément de prix.\par
\labelchar{XLVIII.} Ne pourront aussi les esclaves travaillant actuellement dans les sucreries, indigoteries \& habitations, âgés de 14 ans \& au-dessus jusqu’à 60 ans, être saisis pour dettes, sinon pour ce qui sera dû du prix de leur achat, ou que la sucrerie ou indigoterie ou habitation dans laquelle ils travaillent soit saisis réellement ; défendons, à peine de nullité de procéder par saisie réelle \& adjudication par décret sur les sucreries, indigoteries ni habitations, sans y comprendre les nègres de l’âge susdit y travaillant actuellement.\par
\labelchar{XLIX.} Les Fermiers Judiciaires des sucreries, indigoteries, ou habitations saisies réellement conjointement avec les esclaves, seront tenues de payer le prix entier de leur bail, sans qu’ils puissent compter parmi les fruits \& droits de leur bail qu’ils percevront les enfants qui seront nés des esclaves pendant le cours d’icelui qui n’y entrent point.\par
\labelchar{L.} Voulons que nonobstant toutes conventions contraires que nous déclarons nulles, que lesdits enfants appartiennent à la partie saisie si les créanciers sont satisfaits d’ailleurs ou à l’adjudicataire s’il intervient un décret, \& qu’à cet effet, mention soit faite dans la dernière affiche avant l’interposition du décret des enfants nés des esclaves depuis la saisie réelle : que dans la même affiche il sera fait mention des esclaves décédés depuis la saisie réelle dans laquelle ils étaient compris.\par
\labelchar{LI.} Voulons pour éviter aux frais \& aux longueurs des procédures, que la distribution du prix entier de l’adjudication conjointement de fonds \& des esclaves \& de ce qui proviendra du prix des Baux judiciaires, soit faite entre les Créanciers selon l’ordre de leurs privilèges \& hypothèques, sans distinguer ce qui est provenu du prix des fonds, d’avec ce qui est procédant du prix des esclaves.\par
\labelchar{LII.} Et néanmoins les droits Féodaux \& Seigneuriaux ne seront payés qu’à proportion du prix des fonds.\par
\labelchar{LIII.} Ne seront reçus les Lignagers \& seigneurs Féodaux à retirer les fonds décrétés, s’ils ne retirent les esclaves vendus conjointement avec les fonds, ni les adjudicataires à retenir les esclaves sans les fonds.\par
\labelchar{LIV.} Enjoignons aux Gardiens Nobles \& Bourgeois, Usufruitiers, Amodiateurs \& autres jouissants des fonds, auxquels sont attachés des esclaves qui travaillent, de gouverner lesdits esclaves comme bons pères de famille, sans qu’ils soient tenus, après leur administration finie, de rendre le prix de ceux qui seront décédés ou diminués par maladie, vieillesse ou autrement sans leur faute \& sans qu’ils puissent aussi retenir comme fruits de leurs profits, les enfants nés desdits esclaves durant leur administration, lesquels nous voulons être conservés \& rendus à ceux qui en sont les Maîtres \& Propriétaires.\par
\labelchar{LV.} Les Maîtres âgés de 20 ans pourront affranchir leurs Esclaves par tous actes entre-vifs ou à cause de mort, sans qu’ils soient tenus de rendre raison de leur affranchissement ni qu’ils aient besoin d’avis de parents, encore qu’ils soient mineurs de 25 ans.\par
LVI, Les esclaves qui auront été faits légataires universels par leurs Maîtres ou nommés Exécuteurs de leurs Testaments, ou Tuteurs de leurs enfants, seront tenus \& réputés, les tenons \& réputons pour affranchis.\par
\labelchar{LVII.} Déclarons leurs affranchissements faits dans nos Îles, leur tenir lieu de naissance dans nos dites Îles, \& les esclaves affranchis n’avoir besoin de nos Lettres de naturalité pour jouir des avantages de nos Sujets naturels dans notre Royaume, Terres \& Pays de notre obéissance, encore qu’ils soient nés dans les Pays Étrangers.\par
\labelchar{LVIII.} Commandons aux affranchis de porter un respect singulier à leurs anciens Maîtres, à leurs Veuves, \& à leurs Enfants, en sorte que l’injure qu’ils leur auront faite soit punie plus grièvement que si elle était faite à une autre personne : les déclarons toutefois francs \& quittes envers eux de toutes autres charges, services \& droits utiles que leurs anciens Maîtres voudraient prétendre, tant sur leurs personnes, que sur leurs biens \& successions en qualité de Patrons.\par
\labelchar{LIX.} Octroyons aux affranchis les mêmes droits, privilèges \& immunités dont jouissent les personnes nées libres, voulons qu’ils méritent une liberté acquise, \& qu’elle produise en eux, tant pour leurs personnes que pour leurs biens, les mêmes effets que le bonheur de la liberté naturelle cause à nos autres sujets.\par
\labelchar{LX.} Déclarons les confiscations \& les amendes, qui n’ont point de destination particulière par ces présentes nous appartenir, pour être payées à ceux qui sont préposés à la recette de nos revenus. Voulons néanmoins que distraction soit faite du tiers desdites confiscations \& amendes au profit de l’Hôpital établi dans l’Île où elles auront été adjugées.\par
SI DONNONS EN MANDEMENT à nos Amés \& Féaux les Gens tenant notre Conseil Souverain établi à la Martinique, Garde-Louppe, Saint-Christophe, que ces Présentes ils aient à les faire lire, publier \& enregistrer, \& le contenu en icelles, garder \& observer de point en point selon leur forme \& teneur, sans y contrevenir, ni permettre qu’il y soit contrevenu en quelque sorte \& manière que ce soit, nonobstant tous Édits, Déclarations, Arrêts \& Usages à ce contraires, auxquels nous avons dérogé \& dérogeons par ces dites Présentes. CAR tel est notre plaisir, \& afin que ce soit chose ferme et stable à toujours, nous y avons fait mettre notre Scel. {\scshape Donné} à Versailles au mois de Mars 1685, \& de notre Règne le 42\textsuperscript{e}. Signé, {\scshape Louis} ; \emph{Et plus bas.} Par le Roy, {\scshape Colbert}. \emph{Visa}, {\scshape Le Tellier} : Et scellé du Grand Sceau de Cire verte en lacs de soye verte \& rouge.\par
\emph{Lu, publié \& enregistré le présent Édit, oui \& ce requérant le Procureur Général du Roy, pour être exécuté selon sa forme \& teneur, \& sera à la diligence dudit Procureur Général, envoyé copies d’icelui aux Sièges Ressortissant du Conseil, pour y être pareillement, lu, publié \& enregistré. Fait \& donné au Conseil Souverain de la Côte Saint Domingue, tenu au petit Gouave, le 6 Mai 1687.} Signé, {\scshape Moriceau}.
 


% at least one empty page at end (for booklet couv)
\ifbooklet
  \pagestyle{empty}
  \clearpage
  % 2 empty pages maybe needed for 4e cover
  \ifnum\modulo{\value{page}}{4}=0 \hbox{}\newpage\hbox{}\newpage\fi
  \ifnum\modulo{\value{page}}{4}=1 \hbox{}\newpage\hbox{}\newpage\fi


  \hbox{}\newpage
  \ifodd\value{page}\hbox{}\newpage\fi
  {\centering\color{rubric}\bfseries\noindent\large
    Hurlus ? Qu’est-ce.\par
    \bigskip
  }
  \noindent Des bouquinistes électroniques, pour du texte libre à participations libres,
  téléchargeable gratuitement sur \href{https://hurlus.fr}{\dotuline{hurlus.fr}}.\par
  \bigskip
  \noindent Cette brochure a été produite par des éditeurs bénévoles.
  Elle n’est pas faite pour être possédée, mais pour être lue, et puis donnée, ou déposée dans une boîte à livres.
  En page de garde, on peut ajouter une date, un lieu, un nom ;
  comme une fiche de bibliothèque en papier qui enregistre \emph{les voyages de la brochure}.
  \par

  Ce texte a été choisi parce qu’une personne l’a aimé,
  ou haï, elle a pensé qu’il partipait à la formation de notre présent ;
  sans le souci de plaire, vendre, ou militer pour une cause.
  \par

  L’édition électronique est soigneuse, tant sur la technique
  que sur l’établissement du texte ; mais sans aucune prétention scolaire, au contraire.
  Le but est de s’adresser à tous, sans distinction de science ou de diplôme.
  \par

  Cet exemplaire en papier a été tiré sur une imprimante personnelle
   ou une photocopieuse. Tout le monde peut le faire.
  Il suffit de
  télécharger un fichier sur \href{https://hurlus.fr}{\dotuline{hurlus.fr}},
  d’imprimer, et agrafer (puis lire et donner).\par

  \bigskip

  \noindent PS : Les hurlus furent aussi des rebelles protestants qui cassaient les statues dans les églises catholiques. En 1566 démarra la révolte des gueux dans le pays de Lille. L’insurrection enflamma la région jusqu’à Anvers où les gueux de mer bloquèrent les bateaux espagnols.
  Ce fut une rare guerre de libération dont naquit un pays toujours libre : les Pays-Bas.
  En plat pays francophone, par contre, restèrent des bandes de huguenots, les hurlus, progressivement réprimés par la très catholique Espagne.
  Cette mémoire d’une défaite est éteinte, rallumons-la. Sortons les livres du culte universitaire, débusquons les idoles de l’époque, pour les démonter.
\fi

\end{document}
