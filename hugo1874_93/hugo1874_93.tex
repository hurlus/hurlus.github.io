%%%%%%%%%%%%%%%%%%%%%%%%%%%%%%%%%
% LaTeX model https://hurlus.fr %
%%%%%%%%%%%%%%%%%%%%%%%%%%%%%%%%%

% Needed before document class
\RequirePackage{pdftexcmds} % needed for tests expressions
\RequirePackage{fix-cm} % correct units

% Define mode
\def\mode{a4}

\newif\ifaiv % a4
\newif\ifav % a5
\newif\ifbooklet % booklet
\newif\ifcover % cover for booklet

\ifnum \strcmp{\mode}{cover}=0
  \covertrue
\else\ifnum \strcmp{\mode}{booklet}=0
  \booklettrue
\else\ifnum \strcmp{\mode}{a5}=0
  \avtrue
\else
  \aivtrue
\fi\fi\fi

\ifbooklet % do not enclose with {}
  \documentclass[french,twoside]{book} % ,notitlepage
  \usepackage[%
    papersize={105mm, 297mm},
    inner=12mm,
    outer=12mm,
    top=20mm,
    bottom=15mm,
    marginparsep=0pt,
  ]{geometry}
  \usepackage[fontsize=9.5pt]{scrextend} % for Roboto
\else\ifav
  \documentclass[french,twoside]{book} % ,notitlepage
  \usepackage[%
    a5paper,
    inner=25mm,
    outer=15mm,
    top=15mm,
    bottom=15mm,
    marginparsep=0pt,
  ]{geometry}
  \usepackage[fontsize=12pt]{scrextend}
\else% A4 2 cols
  \documentclass[twocolumn]{report}
  \usepackage[%
    a4paper,
    inner=15mm,
    outer=10mm,
    top=25mm,
    bottom=18mm,
    marginparsep=0pt,
  ]{geometry}
  \setlength{\columnsep}{20mm}
  \usepackage[fontsize=9.5pt]{scrextend}
\fi\fi

%%%%%%%%%%%%%%
% Alignments %
%%%%%%%%%%%%%%
% before teinte macros

\setlength{\arrayrulewidth}{0.2pt}
\setlength{\columnseprule}{\arrayrulewidth} % twocol
\setlength{\parskip}{0pt} % classical para with no margin
\setlength{\parindent}{1.5em}

%%%%%%%%%%
% Colors %
%%%%%%%%%%
% before Teinte macros

\usepackage[dvipsnames]{xcolor}
\definecolor{rubric}{HTML}{800000} % the tonic 0c71c3
\def\columnseprulecolor{\color{rubric}}
\colorlet{borderline}{rubric!30!} % definecolor need exact code
\definecolor{shadecolor}{gray}{0.95}
\definecolor{bghi}{gray}{0.5}

%%%%%%%%%%%%%%%%%
% Teinte macros %
%%%%%%%%%%%%%%%%%
%%%%%%%%%%%%%%%%%%%%%%%%%%%%%%%%%%%%%%%%%%%%%%%%%%%
% <TEI> generic (LaTeX names generated by Teinte) %
%%%%%%%%%%%%%%%%%%%%%%%%%%%%%%%%%%%%%%%%%%%%%%%%%%%
% This template is inserted in a specific design
% It is XeLaTeX and otf fonts

\makeatletter % <@@@


\usepackage{blindtext} % generate text for testing
\usepackage[strict]{changepage} % for modulo 4
\usepackage{contour} % rounding words
\usepackage[nodayofweek]{datetime}
% \usepackage{DejaVuSans} % seems buggy for sffont font for symbols
\usepackage{enumitem} % <list>
\usepackage{etoolbox} % patch commands
\usepackage{fancyvrb}
\usepackage{fancyhdr}
\usepackage{float}
\usepackage{fontspec} % XeLaTeX mandatory for fonts
\usepackage{footnote} % used to capture notes in minipage (ex: quote)
\usepackage{framed} % bordering correct with footnote hack
\usepackage{graphicx}
\usepackage{lettrine} % drop caps
\usepackage{lipsum} % generate text for testing
\usepackage[framemethod=tikz,]{mdframed} % maybe used for frame with footnotes inside
\usepackage{pdftexcmds} % needed for tests expressions
\usepackage{polyglossia} % non-break space french punct, bug Warning: "Failed to patch part"
\usepackage[%
  indentfirst=false,
  vskip=1em,
  noorphanfirst=true,
  noorphanafter=true,
  leftmargin=\parindent,
  rightmargin=0pt,
]{quoting}
\usepackage{ragged2e}
\usepackage{setspace} % \setstretch for <quote>
\usepackage{tabularx} % <table>
\usepackage[explicit]{titlesec} % wear titles, !NO implicit
\usepackage{tikz} % ornaments
\usepackage{tocloft} % styling tocs
\usepackage[fit]{truncate} % used im runing titles
\usepackage{unicode-math}
\usepackage[normalem]{ulem} % breakable \uline, normalem is absolutely necessary to keep \emph
\usepackage{verse} % <l>
\usepackage{xcolor} % named colors
\usepackage{xparse} % @ifundefined
\XeTeXdefaultencoding "iso-8859-1" % bad encoding of xstring
\usepackage{xstring} % string tests
\XeTeXdefaultencoding "utf-8"
\PassOptionsToPackage{hyphens}{url} % before hyperref, which load url package

% TOTEST
% \usepackage{hypcap} % links in caption ?
% \usepackage{marginnote}
% TESTED
% \usepackage{background} % doesn’t work with xetek
% \usepackage{bookmark} % prefers the hyperref hack \phantomsection
% \usepackage[color, leftbars]{changebar} % 2 cols doc, impossible to keep bar left
% \usepackage[utf8x]{inputenc} % inputenc package ignored with utf8 based engines
% \usepackage[sfdefault,medium]{inter} % no small caps
% \usepackage{firamath} % choose firasans instead, firamath unavailable in Ubuntu 21-04
% \usepackage{flushend} % bad for last notes, supposed flush end of columns
% \usepackage[stable]{footmisc} % BAD for complex notes https://texfaq.org/FAQ-ftnsect
% \usepackage{helvet} % not for XeLaTeX
% \usepackage{multicol} % not compatible with too much packages (longtable, framed, memoir…)
% \usepackage[default,oldstyle,scale=0.95]{opensans} % no small caps
% \usepackage{sectsty} % \chapterfont OBSOLETE
% \usepackage{soul} % \ul for underline, OBSOLETE with XeTeX
% \usepackage[breakable]{tcolorbox} % text styling gone, footnote hack not kept with breakable


% Metadata inserted by a program, from the TEI source, for title page and runing heads
\title{\textbf{ Quatrevingt-Treize }}
\date{1874}
\author{Victor Hugo}
\def\elbibl{Victor Hugo. 1874. \emph{Quatrevingt-Treize}}
\def\elsource{ \href{http://gallica.bnf.fr/ark:/12148/bpt6k37503p}{\dotuline{http://gallica.bnf.fr/ark:/12148/bpt6k37503p}}\footnote{\href{http://gallica.bnf.fr/ark:/12148/bpt6k37503p}{\url{http://gallica.bnf.fr/ark:/12148/bpt6k37503p}}}  \href{http://efele.net/ebooks/livres/000210}{\dotuline{http://efele.net/ebooks/livres/000210}}\footnote{\href{http://efele.net/ebooks/livres/000210}{\url{http://efele.net/ebooks/livres/000210}}} }

% Default metas
\newcommand{\colorprovide}[2]{\@ifundefinedcolor{#1}{\colorlet{#1}{#2}}{}}
\colorprovide{rubric}{red}
\colorprovide{silver}{lightgray}
\@ifundefined{syms}{\newfontfamily\syms{DejaVu Sans}}{}
\newif\ifdev
\@ifundefined{elbibl}{% No meta defined, maybe dev mode
  \newcommand{\elbibl}{Titre court ?}
  \newcommand{\elbook}{Titre du livre source ?}
  \newcommand{\elabstract}{Résumé\par}
  \newcommand{\elurl}{http://oeuvres.github.io/elbook/2}
  \author{Éric Lœchien}
  \title{Un titre de test assez long pour vérifier le comportement d’une maquette}
  \date{1566}
  \devtrue
}{}
\let\eltitle\@title
\let\elauthor\@author
\let\eldate\@date


\defaultfontfeatures{
  % Mapping=tex-text, % no effect seen
  Scale=MatchLowercase,
  Ligatures={TeX,Common},
}


% generic typo commands
\newcommand{\astermono}{\medskip\centerline{\color{rubric}\large\selectfont{\syms ✻}}\medskip\par}%
\newcommand{\astertri}{\medskip\par\centerline{\color{rubric}\large\selectfont{\syms ✻\,✻\,✻}}\medskip\par}%
\newcommand{\asterism}{\bigskip\par\noindent\parbox{\linewidth}{\centering\color{rubric}\large{\syms ✻}\\{\syms ✻}\hskip 0.75em{\syms ✻}}\bigskip\par}%

% lists
\newlength{\listmod}
\setlength{\listmod}{\parindent}
\setlist{
  itemindent=!,
  listparindent=\listmod,
  labelsep=0.2\listmod,
  parsep=0pt,
  % topsep=0.2em, % default topsep is best
}
\setlist[itemize]{
  label=—,
  leftmargin=0pt,
  labelindent=1.2em,
  labelwidth=0pt,
}
\setlist[enumerate]{
  label={\bf\color{rubric}\arabic*.},
  labelindent=0.8\listmod,
  leftmargin=\listmod,
  labelwidth=0pt,
}
\newlist{listalpha}{enumerate}{1}
\setlist[listalpha]{
  label={\bf\color{rubric}\alph*.},
  leftmargin=0pt,
  labelindent=0.8\listmod,
  labelwidth=0pt,
}
\newcommand{\listhead}[1]{\hspace{-1\listmod}\emph{#1}}

\renewcommand{\hrulefill}{%
  \leavevmode\leaders\hrule height 0.2pt\hfill\kern\z@}

% General typo
\DeclareTextFontCommand{\textlarge}{\large}
\DeclareTextFontCommand{\textsmall}{\small}

% commands, inlines
\newcommand{\anchor}[1]{\Hy@raisedlink{\hypertarget{#1}{}}} % link to top of an anchor (not baseline)
\newcommand\abbr[1]{#1}
\newcommand{\autour}[1]{\tikz[baseline=(X.base)]\node [draw=rubric,thin,rectangle,inner sep=1.5pt, rounded corners=3pt] (X) {\color{rubric}#1};}
\newcommand\corr[1]{#1}
\newcommand{\ed}[1]{ {\color{silver}\sffamily\footnotesize (#1)} } % <milestone ed="1688"/>
\newcommand\expan[1]{#1}
\newcommand\foreign[1]{\emph{#1}}
\newcommand\gap[1]{#1}
\renewcommand{\LettrineFontHook}{\color{rubric}}
\newcommand{\initial}[2]{\lettrine[lines=2, loversize=0.3, lhang=0.3]{#1}{#2}}
\newcommand{\initialiv}[2]{%
  \let\oldLFH\LettrineFontHook
  % \renewcommand{\LettrineFontHook}{\color{rubric}\ttfamily}
  \IfSubStr{QJ’}{#1}{
    \lettrine[lines=4, lhang=0.2, loversize=-0.1, lraise=0.2]{\smash{#1}}{#2}
  }{\IfSubStr{É}{#1}{
    \lettrine[lines=4, lhang=0.2, loversize=-0, lraise=0]{\smash{#1}}{#2}
  }{\IfSubStr{ÀÂ}{#1}{
    \lettrine[lines=4, lhang=0.2, loversize=-0, lraise=0, slope=0.6em]{\smash{#1}}{#2}
  }{\IfSubStr{A}{#1}{
    \lettrine[lines=4, lhang=0.2, loversize=0.2, slope=0.6em]{\smash{#1}}{#2}
  }{\IfSubStr{V}{#1}{
    \lettrine[lines=4, lhang=0.2, loversize=0.2, slope=-0.5em]{\smash{#1}}{#2}
  }{
    \lettrine[lines=4, lhang=0.2, loversize=0.2]{\smash{#1}}{#2}
  }}}}}
  \let\LettrineFontHook\oldLFH
}
\newcommand{\labelchar}[1]{\textbf{\color{rubric} #1}}
\newcommand{\milestone}[1]{\autour{\footnotesize\color{rubric} #1}} % <milestone n="4"/>
\newcommand\name[1]{#1}
\newcommand\orig[1]{#1}
\newcommand\orgName[1]{#1}
\newcommand\persName[1]{#1}
\newcommand\placeName[1]{#1}
\newcommand{\pn}[1]{\IfSubStr{-—–¶}{#1}% <p n="3"/>
  {\noindent{\bfseries\color{rubric}   ¶  }}
  {{\footnotesize\autour{ #1}  }}}
\newcommand\reg{}
% \newcommand\ref{} % already defined
\newcommand\sic[1]{#1}
\newcommand\surname[1]{\textsc{#1}}
\newcommand\term[1]{\textbf{#1}}

\def\mednobreak{\ifdim\lastskip<\medskipamount
  \removelastskip\nopagebreak\medskip\fi}
\def\bignobreak{\ifdim\lastskip<\bigskipamount
  \removelastskip\nopagebreak\bigskip\fi}

% commands, blocks
\newcommand{\byline}[1]{\bigskip{\RaggedLeft{#1}\par}\bigskip}
\newcommand{\bibl}[1]{{\RaggedLeft{#1}\par\bigskip}}
\newcommand{\biblitem}[1]{{\noindent\hangindent=\parindent   #1\par}}
\newcommand{\dateline}[1]{\medskip{\RaggedLeft{#1}\par}\bigskip}
\newcommand{\labelblock}[1]{\medbreak{\noindent\color{rubric}\bfseries #1}\par\mednobreak}
\newcommand{\salute}[1]{\bigbreak{#1}\par\medbreak}
\newcommand{\signed}[1]{\bigbreak\filbreak{\raggedleft #1\par}\medskip}

% environments for blocks (some may become commands)
\newenvironment{borderbox}{}{} % framing content
\newenvironment{citbibl}{\ifvmode\hfill\fi}{\ifvmode\par\fi }
\newenvironment{docAuthor}{\ifvmode\vskip4pt\fontsize{16pt}{18pt}\selectfont\fi\itshape}{\ifvmode\par\fi }
\newenvironment{docDate}{}{\ifvmode\par\fi }
\newenvironment{docImprint}{\vskip6pt}{\ifvmode\par\fi }
\newenvironment{docTitle}{\vskip6pt\bfseries\fontsize{18pt}{22pt}\selectfont}{\par }
\newenvironment{msHead}{\vskip6pt}{\par}
\newenvironment{msItem}{\vskip6pt}{\par}
\newenvironment{titlePart}{}{\par }


% environments for block containers
\newenvironment{argument}{\itshape\parindent0pt}{\vskip1.5em}
\newenvironment{biblfree}{}{\ifvmode\par\fi }
\newenvironment{bibitemlist}[1]{%
  \list{\@biblabel{\@arabic\c@enumiv}}%
  {%
    \settowidth\labelwidth{\@biblabel{#1}}%
    \leftmargin\labelwidth
    \advance\leftmargin\labelsep
    \@openbib@code
    \usecounter{enumiv}%
    \let\p@enumiv\@empty
    \renewcommand\theenumiv{\@arabic\c@enumiv}%
  }
  \sloppy
  \clubpenalty4000
  \@clubpenalty \clubpenalty
  \widowpenalty4000%
  \sfcode`\.\@m
}%
{\def\@noitemerr
  {\@latex@warning{Empty `bibitemlist' environment}}%
\endlist}
\newenvironment{quoteblock}% may be used for ornaments
  {\begin{quoting}}
  {\end{quoting}}

% table () is preceded and finished by custom command
\newcommand{\tableopen}[1]{%
  \ifnum\strcmp{#1}{wide}=0{%
    \begin{center}
  }
  \else\ifnum\strcmp{#1}{long}=0{%
    \begin{center}
  }
  \else{%
    \begin{center}
  }
  \fi\fi
}
\newcommand{\tableclose}[1]{%
  \ifnum\strcmp{#1}{wide}=0{%
    \end{center}
  }
  \else\ifnum\strcmp{#1}{long}=0{%
    \end{center}
  }
  \else{%
    \end{center}
  }
  \fi\fi
}


% text structure
\newcommand\chapteropen{} % before chapter title
\newcommand\chaptercont{} % after title, argument, epigraph…
\newcommand\chapterclose{} % maybe useful for multicol settings
\setcounter{secnumdepth}{-2} % no counters for hierarchy titles
\setcounter{tocdepth}{5} % deep toc
\markright{\@title} % ???
\markboth{\@title}{\@author} % ???
\renewcommand\tableofcontents{\@starttoc{toc}}
% toclof format
% \renewcommand{\@tocrmarg}{0.1em} % Useless command?
% \renewcommand{\@pnumwidth}{0.5em} % {1.75em}
\renewcommand{\@cftmaketoctitle}{}
\setlength{\cftbeforesecskip}{\z@ \@plus.2\p@}
\renewcommand{\cftchapfont}{}
\renewcommand{\cftchapdotsep}{\cftdotsep}
\renewcommand{\cftchapleader}{\normalfont\cftdotfill{\cftchapdotsep}}
\renewcommand{\cftchappagefont}{\bfseries}
\setlength{\cftbeforechapskip}{0em \@plus\p@}
% \renewcommand{\cftsecfont}{\small\relax}
\renewcommand{\cftsecpagefont}{\normalfont}
% \renewcommand{\cftsubsecfont}{\small\relax}
\renewcommand{\cftsecdotsep}{\cftdotsep}
\renewcommand{\cftsecpagefont}{\normalfont}
\renewcommand{\cftsecleader}{\normalfont\cftdotfill{\cftsecdotsep}}
\setlength{\cftsecindent}{1em}
\setlength{\cftsubsecindent}{2em}
\setlength{\cftsubsubsecindent}{3em}
\setlength{\cftchapnumwidth}{1em}
\setlength{\cftsecnumwidth}{1em}
\setlength{\cftsubsecnumwidth}{1em}
\setlength{\cftsubsubsecnumwidth}{1em}

% footnotes
\newif\ifheading
\newcommand*{\fnmarkscale}{\ifheading 0.70 \else 1 \fi}
\renewcommand\footnoterule{\vspace*{0.3cm}\hrule height \arrayrulewidth width 3cm \vspace*{0.3cm}}
\setlength\footnotesep{1.5\footnotesep} % footnote separator
\renewcommand\@makefntext[1]{\parindent 1.5em \noindent \hb@xt@1.8em{\hss{\normalfont\@thefnmark . }}#1} % no superscipt in foot
\patchcmd{\@footnotetext}{\footnotesize}{\footnotesize\sffamily}{}{} % before scrextend, hyperref


%   see https://tex.stackexchange.com/a/34449/5049
\def\truncdiv#1#2{((#1-(#2-1)/2)/#2)}
\def\moduloop#1#2{(#1-\truncdiv{#1}{#2}*#2)}
\def\modulo#1#2{\number\numexpr\moduloop{#1}{#2}\relax}

% orphans and widows
\clubpenalty=9996
\widowpenalty=9999
\brokenpenalty=4991
\predisplaypenalty=10000
\postdisplaypenalty=1549
\displaywidowpenalty=1602
\hyphenpenalty=400
% Copied from Rahtz but not understood
\def\@pnumwidth{1.55em}
\def\@tocrmarg {2.55em}
\def\@dotsep{4.5}
\emergencystretch 3em
\hbadness=4000
\pretolerance=750
\tolerance=2000
\vbadness=4000
\def\Gin@extensions{.pdf,.png,.jpg,.mps,.tif}
% \renewcommand{\@cite}[1]{#1} % biblio

\usepackage{hyperref} % supposed to be the last one, :o) except for the ones to follow
\urlstyle{same} % after hyperref
\hypersetup{
  % pdftex, % no effect
  pdftitle={\elbibl},
  % pdfauthor={Your name here},
  % pdfsubject={Your subject here},
  % pdfkeywords={keyword1, keyword2},
  bookmarksnumbered=true,
  bookmarksopen=true,
  bookmarksopenlevel=1,
  pdfstartview=Fit,
  breaklinks=true, % avoid long links
  pdfpagemode=UseOutlines,    % pdf toc
  hyperfootnotes=true,
  colorlinks=false,
  pdfborder=0 0 0,
  % pdfpagelayout=TwoPageRight,
  % linktocpage=true, % NO, toc, link only on page no
}

\makeatother % /@@@>
%%%%%%%%%%%%%%
% </TEI> end %
%%%%%%%%%%%%%%


%%%%%%%%%%%%%
% footnotes %
%%%%%%%%%%%%%
\renewcommand{\thefootnote}{\bfseries\textcolor{rubric}{\arabic{footnote}}} % color for footnote marks

%%%%%%%%%
% Fonts %
%%%%%%%%%
\usepackage[]{roboto} % SmallCaps, Regular is a bit bold
% \linespread{0.90} % too compact, keep font natural
\newfontfamily\fontrun[]{Roboto Condensed Light} % condensed runing heads
\ifav
  \setmainfont[
    ItalicFont={Roboto Light Italic},
  ]{Roboto}
\else\ifbooklet
  \setmainfont[
    ItalicFont={Roboto Light Italic},
  ]{Roboto}
\else
\setmainfont[
  ItalicFont={Roboto Italic},
]{Roboto Light}
\fi\fi
\renewcommand{\LettrineFontHook}{\bfseries\color{rubric}}
% \renewenvironment{labelblock}{\begin{center}\bfseries\color{rubric}}{\end{center}}

%%%%%%%%
% MISC %
%%%%%%%%

\setdefaultlanguage[frenchpart=false]{french} % bug on part


\newenvironment{quotebar}{%
    \def\FrameCommand{{\color{rubric!10!}\vrule width 0.5em} \hspace{0.9em}}%
    \def\OuterFrameSep{\itemsep} % séparateur vertical
    \MakeFramed {\advance\hsize-\width \FrameRestore}
  }%
  {%
    \endMakeFramed
  }
\renewenvironment{quoteblock}% may be used for ornaments
  {%
    \savenotes
    \setstretch{0.9}
    \normalfont
    \begin{quotebar}
  }
  {%
    \end{quotebar}
    \spewnotes
  }


\renewcommand{\headrulewidth}{\arrayrulewidth}
\renewcommand{\headrule}{{\color{rubric}\hrule}}

% delicate tuning, image has produce line-height problems in title on 2 lines
\titleformat{name=\chapter} % command
  [display] % shape
  {\vspace{1.5em}\centering} % format
  {} % label
  {0pt} % separator between n
  {}
[{\color{rubric}\huge\textbf{#1}}\bigskip] % after code
% \titlespacing{command}{left spacing}{before spacing}{after spacing}[right]
\titlespacing*{\chapter}{0pt}{-2em}{0pt}[0pt]

\titleformat{name=\section}
  [block]{}{}{}{}
  [\vbox{\color{rubric}\large\raggedleft\textbf{#1}}]
\titlespacing{\section}{0pt}{0pt plus 4pt minus 2pt}{\baselineskip}

\titleformat{name=\subsection}
  [block]
  {}
  {} % \thesection
  {} % separator \arrayrulewidth
  {}
[\vbox{\large\textbf{#1}}]
% \titlespacing{\subsection}{0pt}{0pt plus 4pt minus 2pt}{\baselineskip}

\ifaiv
  \fancypagestyle{main}{%
    \fancyhf{}
    \setlength{\headheight}{1.5em}
    \fancyhead{} % reset head
    \fancyfoot{} % reset foot
    \fancyhead[L]{\truncate{0.45\headwidth}{\fontrun\elbibl}} % book ref
    \fancyhead[R]{\truncate{0.45\headwidth}{ \fontrun\nouppercase\leftmark}} % Chapter title
    \fancyhead[C]{\thepage}
  }
  \fancypagestyle{plain}{% apply to chapter
    \fancyhf{}% clear all header and footer fields
    \setlength{\headheight}{1.5em}
    \fancyhead[L]{\truncate{0.9\headwidth}{\fontrun\elbibl}}
    \fancyhead[R]{\thepage}
  }
\else
  \fancypagestyle{main}{%
    \fancyhf{}
    \setlength{\headheight}{1.5em}
    \fancyhead{} % reset head
    \fancyfoot{} % reset foot
    \fancyhead[RE]{\truncate{0.9\headwidth}{\fontrun\elbibl}} % book ref
    \fancyhead[LO]{\truncate{0.9\headwidth}{\fontrun\nouppercase\leftmark}} % Chapter title, \nouppercase needed
    \fancyhead[RO,LE]{\thepage}
  }
  \fancypagestyle{plain}{% apply to chapter
    \fancyhf{}% clear all header and footer fields
    \setlength{\headheight}{1.5em}
    \fancyhead[L]{\truncate{0.9\headwidth}{\fontrun\elbibl}}
    \fancyhead[R]{\thepage}
  }
\fi

\ifav % a5 only
  \titleclass{\section}{top}
\fi

\newcommand\chapo{{%
  \vspace*{-3em}
  \centering % no vskip ()
  {\Large\addfontfeature{LetterSpace=25}\bfseries{\elauthor}}\par
  \smallskip
  {\large\eldate}\par
  \bigskip
  {\Large\selectfont{\eltitle}}\par
  \bigskip
  {\color{rubric}\hline\par}
  \bigskip
  {\Large TEXTE LIBRE À PARTICPATION LIBRE\par}
  \centerline{\small\color{rubric} {hurlus.fr, tiré le \today}}\par
  \bigskip
}}

\newcommand\cover{{%
  \thispagestyle{empty}
  \centering
  {\LARGE\bfseries{\elauthor}}\par
  \bigskip
  {\Large\eldate}\par
  \bigskip
  \bigskip
  {\LARGE\selectfont{\eltitle}}\par
  \vfill\null
  {\color{rubric}\setlength{\arrayrulewidth}{2pt}\hline\par}
  \vfill\null
  {\Large TEXTE LIBRE À PARTICPATION LIBRE\par}
  \centerline{{\href{https://hurlus.fr}{\dotuline{hurlus.fr}}, tiré le \today}}\par
}}

\begin{document}
\pagestyle{empty}
\ifbooklet{
  \cover\newpage
  \thispagestyle{empty}\hbox{}\newpage
  \cover\newpage\noindent Les voyages de la brochure\par
  \bigskip
  \begin{tabularx}{\textwidth}{l|X|X}
    \textbf{Date} & \textbf{Lieu}& \textbf{Nom/pseudo} \\ \hline
    \rule{0pt}{25cm} &  &   \\
  \end{tabularx}
  \newpage
  \addtocounter{page}{-4}
}\fi

\thispagestyle{empty}
\ifaiv
  \twocolumn[\chapo]
\else
  \chapo
\fi
{\it\elabstract}
\bigskip
\makeatletter\@starttoc{toc}\makeatother % toc without new page
\bigskip

\pagestyle{main} % after style

   \section[{Première partie. En mer}]{Première partie \\
En mer}\phantomsection
\label{p1}\renewcommand{\leftmark}{Première partie \\
En mer}

  \subsection[{Livre premier}]{Livre premier}\phantomsection
\label{p1l1}
\subsubsection[{Le bois de la Saudraie}]{Le bois de la Saudraie}\phantomsection
\label{p1l1c1}
\noindent Dans les derniers jours de mai 1793, un des bataillons parisiens amenés en Bretagne par Santerre fouillait le redoutable bois de la Saudraie en Astillé. On n’était pas plus de trois cents, car le bataillon était décimé par cette rude guerre. C’était l’époque où, après l’Argonne, Jemmapes et Valmy, du premier bataillon de Paris, qui était de six cents volontaires, il restait vingt-sept hommes, du deuxième trente-trois, et du troisième cinquante-sept. Temps des luttes épiques.\par
Les bataillons envoyés de Paris en Vendée comptaient neuf cent douze hommes. Chaque bataillon avait trois pièces de canon. Ils avaient été rapidement  mis sur pied. Le 25 avril, Gohier étant ministre de la justice et Bouchotte étant ministre de la guerre, la section du Bon-Conseil avait proposé d’envoyer des bataillons de volontaires en Vendée ; le membre de la commune Lubin avait fait le rapport ; le 1\textsuperscript{er} mai, Santerre était prêt à faire partir douze mille soldats, trente pièces de campagne et un bataillon de canonniers. Ces bataillons, faits si vite, furent si bien faits, qu’ils servent aujourd’hui de modèles ; c’est d’après leur mode de composition qu’on forme les compagnies de ligne ; ils ont changé l’ancienne proportion entre le nombre des soldats et le nombre des sous-officiers.\par
Le 28 avril, la commune de Paris avait donné aux volontaires de Santerre cette consigne : \emph{Point de grâce. Point de quartier.} A la fin de mai, sur les douze mille partis de Paris, huit mille étaient morts.\par
Le bataillon engagé dans le bois de la Saudraie se tenait sur ses gardes. On ne se hâtait point. On regardait à la fois à droite et à gauche, devant soi et derrière soi ; Kléber a dit : \emph{Le soldat a un œil dans le dos}. Il y avait longtemps qu’on marchait. Quelle heure pouvait-il être ? à quel moment du jour en était-on ? Il eût été difficile de le dire, car il y a toujours une sorte de soir dans de si sauvages halliers, et il ne fait jamais clair dans ce bois-là.\par
Le bois de la Saudraie était tragique. C’était dans ce taillis que, dès le mois de novembre 1792, la guerre civile avait commencé ses crimes ; Mousqueton, le boiteux féroce, était sorti de ces épaisseurs funestes ; la quantité de meurtres qui s’étaient commis  là faisait dresser les cheveux. Pas de lieu plus épouvantable. Les soldats s’y enfonçaient avec précaution. Tout était plein de fleurs ; on avait autour de soi une tremblante muraille de branches d’où tombait la charmante fraîcheur des feuilles ; des rayons de soleil trouaient çà et là ces ténèbres vertes ; à terre, le glaïeul, la flambe des marais, le narcisse des prés, la gênotte, cette petite fleur qui annonce le beau temps, le safran printanier, brodaient et passementaient un profond tapis de végétation où fourmillaient toutes les formes de la mousse, depuis celle qui ressemble à la chenille jusqu’à celle qui ressemble à l’étoile. Les soldats avançaient pas à pas, en silence, en écartant doucement les broussailles. Les oiseaux gazouillaient au-dessus des bayonnettes.\par
La Saudraie était un de ces halliers où jadis, dans les temps paisibles, on avait fait la Houiche-ba, qui est la chasse aux oiseaux pendant la nuit ; maintenant on y faisait la chasse aux hommes.\par
Le taillis était tout de bouleaux, de hêtres et de chênes ; le sol plat ; la mousse et l’herbe épaisse amortissaient le bruit des hommes en marche ; aucun sentier, ou des sentiers tout de suite perdus ; des houx, des prunelliers sauvages, des fougères, des haies d’arrête-bœuf, de hautes ronces ; impossibilité de voir un homme à dix pas. Par instants passait dans le branchage un héron ou une poule d’eau indiquant le voisinage des marais.\par
On marchait. On allait à l’aventure, avec inquiétude, et en craignant de trouver ce qu’on cherchait.\par
 De temps en temps on rencontrait des traces de campements, des places brûlées, des herbes foulées, des bâtons en croix, des branches sanglantes. Là on avait fait la soupe, là on avait dit la messe, là on avait pansé des blessés. Mais ceux qui avaient passé avaient disparu. Où étaient-ils ? Bien loin peut-être ? peut-être là tout près, cachés, l’espingole au poing ? Le bois semblait désert. Le bataillon redoublait de prudence. Solitude, donc défiance. On ne voyait personne ; raison de plus pour redouter quelqu’un. On avait affaire à une forêt mal famée.\par
Une embuscade était probable.\par
Trente grenadiers, détachés en éclaireurs, et commandés par un sergent, marchaient en avant à une assez grande distance du gros de la troupe. La vivandière du bataillon les accompagnait. Les vivandières se joignent volontiers aux avant-gardes. On court des dangers, mais on va voir quelque chose. La curiosité est une des formes de la bravoure féminine.\par
Tout à coup les soldats de cette petite troupe d’avant-garde eurent ce tressaillement connu des chasseurs qui indique qu’on touche au gîte. On avait entendu comme un souffle au centre d’un fourré, et il semblait qu’on venait de voir un mouvement dans les feuilles. Les soldats se firent signe.\par
Dans l’espèce de guet et de quête confiée aux éclaireurs, les officiers n’ont pas besoin de s’en mêler ; ce qui doit être fait se fait de soi-même.\par
En moins d’une minute le point où l’on avait remué fut cerné ; un cercle de fusils braqués l’entoura ;  le centre obscur du hallier fut couché en joue de tous les côtés à la fois, et les soldats, le doigt sur la détente, l’œil sur le lieu suspect, n’attendirent plus pour le mitrailler que le commandement du sergent.\par
Cependant la vivandière s’était hasardée à regarder à travers les broussailles, et, au moment où le sergent allait crier : Feu ! cette femme cria : Halte !\par
Et se tournant vers les soldats : — Ne tirez pas, camarades !\par
Et elle se précipita dans le taillis. On l’y suivit.\par
Il y avait quelqu’un là en effet.\par
Au plus épais du fourré, au bord d’une de ces petites clairières rondes que font dans les bois les fourneaux à charbon en brûlant les racines des arbres, dans une sorte de trou de branches, espèce de chambre de feuillage, entr’ouverte comme une alcôve, une femme était assise sur la mousse, ayant au sein un enfant qui tétait et sur ses genoux les deux têtes blondes de deux enfants endormis.\par
C’était là l’embuscade.\par
— Qu’est-ce que vous faites ici, vous ? cria la vivandière.\par
La femme leva la tête.\par
La vivandière ajouta, furieuse :\par
— Êtes-vous folle d’être là !\par
Et elle reprit :\par
— Un peu plus, vous étiez exterminée !\par
Et, s’adressant aux soldats, la vivandière ajouta :\par
— C’est une femme.\par
— Pardine, nous le voyons bien ! dit un grenadier.\par
 La vivandière poursuivit :\par
— Venir dans les bois se faire massacrer ! a-t-on idée de faire des bêtises comme ça !\par
La femme stupéfaite, effarée, pétrifiée, regardait autour d’elle, comme à travers un rêve, ces fusils, ces sabres, ces bayonnettes, ces faces farouches.\par
Les deux enfants se réveillèrent et crièrent.\par
— J’ai faim, dit l’un.\par
— J’ai peur, dit l’autre.\par
Le petit continuait de téter.\par
La vivandière lui adressa la parole.\par
— C’est toi qui as raison, lui dit-elle.\par
La mère était muette d’effroi.\par
Le sergent lui cria :\par
— N’ayez pas peur, nous sommes le bataillon du Bonnet-Rouge.\par
La femme trembla de la tête aux pieds. Elle regarda le sergent, rude visage dont on ne voyait que les sourcils, les moustaches, et deux braises qui étaient les deux yeux.\par
— Le bataillon de la ci-devant Croix-Rouge, ajouta la vivandière.\par
Et le sergent continua :\par
— Qui es-tu, madame ?\par
La femme le considérait, terrifiée. Elle était maigre, jeune, pâle, en haillons ; elle avait le gros capuchon des paysannes bretonnes et la couverture de laine rattachée au cou avec une ficelle. Elle laissait voir son sein nu avec une indifférence de femelle. Ses pieds, sans bas ni souliers, saignaient.\par
 — C’est une pauvre, dit le sergent.\par
Et la vivandière reprit de sa voix soldatesque et féminine, douce en dessous :\par
— Comment vous appelez-vous ?\par
La femme murmura dans un bégaiement presque indistinct :\par
— Michelle Fléchard.\par
Cependant la vivandière caressait avec sa grosse main la petite tête du nourrisson.\par
— Quel âge a ce môme ? demanda-t-elle.\par
La mère ne comprit pas. La vivandière insista.\par
— Je vous demande l’âge de ça.\par
— Ah ! dit la mère. Dix-huit mois.\par
— C’est vieux, dit la vivandière. Ça ne doit plus téter. Il faudra me sevrer ça. Nous lui donnerons de la soupe.\par
La mère commençait à se rassurer. Les deux petits qui s’étaient réveillés étaient plus curieux qu’effrayés. Ils admiraient les plumets.\par
— Ah ! dit la mère, ils ont bien faim.\par
Et elle ajouta :\par
— Je n’ai plus de lait.\par
— On leur donnera à manger, cria le sergent, et à toi aussi. Mais ce n’est pas tout ça. Quelles sont tes opinions politiques ?\par
La femme regarda le sergent, et ne répondit pas.\par
— Entends-tu ma question ?\par
Elle balbutia :\par
— J’ai été mise au couvent toute jeune, mais je  me suis mariée, je ne suis pas religieuse. Les sœurs m’ont appris à parler français. On a mis le feu au village. Nous nous sommes sauvés si vite que je n’ai pas eu le temps de mettre des souliers.\par
— Je te demande quelles sont tes opinions politiques ?\par
— Je ne sais pas ça.\par
Le sergent poursuivit :\par
— C’est qu’il y a des espionnes. Ça se fusille, les espionnes. Voyons. Parle. Tu n’es pas bohémienne ? Quelle est ta patrie ?\par
Elle continua de le regarder comme ne comprenant pas. Le sergent répéta :\par
— Quelle est ta patrie ?\par
— Je ne sais pas, dit-elle.\par
— Comment ! tu ne sais pas quel est ton pays ?\par
— Ah ! mon pays. Si fait.\par
— Eh bien, quel est ton pays ?\par
La femme répondit :\par
— C’est la métairie de Siscoignard, dans la paroisse d’Azé.\par
Ce fut le tour du sergent d’être stupéfait. Il demeura un moment pensif. Puis il reprit :\par
— Tu dis ?\par
— Siscoignard.\par
— Ce n’est pas une patrie, ça.\par
— C’est mon pays.\par
Et la femme, après un instant de réflexion, ajouta :\par
— Je comprends, monsieur. Vous êtes de France, moi je suis de Bretagne.\par
 — Eh bien ?\par
— Ce n’est pas le même pays.\par
— Mais c’est la même patrie ! cria le sergent.\par
La femme se borna à répondre :\par
— Je suis de Siscoignard.\par
— Va pour Siscoignard ! reprit le sergent. C’est de là qu’est ta famille ?\par
— Oui.\par
— Que fait-elle ?\par
— Elle est toute morte. Je n’ai plus personne.\par
Le sergent, qui était un peu beau parleur, continua l’interrogatoire.\par
— On a des parents, que diable ! ou on en a eu. Qui es-tu ? Parle.\par
La femme écouta, ahurie, cet — \emph{ou on en a eu} — qui ressemblait plus à un cri de bête fauve qu’à une parole humaine.\par
La vivandière sentit le besoin d’intervenir. Elle se remit à caresser l’enfant qui tétait, et donna une tape sur la joue aux deux autres.\par
— Comment s’appelle la téteuse ? demanda-t-elle ; car c’est une fille, ça.\par
La mère répondit : Georgette.\par
— Et l’aîné ? car c’est un homme, ce polisson-là.\par
— René-Jean.\par
— Et le cadet ? car lui aussi, il est un homme, et joufflu encore !\par
— Gros-Alain, dit la mère.\par
— Ils sont gentils, ces petits, dit la vivandière ; ça vous a déjà des airs d’être des personnes.\par
 Cependant le sergent insistait.\par
— Parle donc, madame. As-tu une maison ?\par
— J’en avais une.\par
— Où ça ?\par
— A Azé.\par
— Pourquoi n’es-tu pas dans ta maison ?\par
— Parce qu’on l’a brûlée.\par
— Qui ça ?\par
— Je ne sais pas. Une bataille.\par
— D’où viens-tu ?\par
— De là.\par
— Où vas-tu ?\par
— Je ne sais pas.\par
— Arrive au fait. Qui es-tu ?\par
— Je ne sais pas.\par
— Tu ne sais pas qui tu es ?\par
— Nous sommes des gens qui nous sauvons.\par
— De quel parti es-tu ?\par
— Je ne sais pas.\par
— Es-tu des bleus ? Es-tu des blancs ? Avec qui es-tu ?\par
— Je suis avec mes enfants.\par
Il y eut une pause. La vivandière dit :\par
— Moi, je n’ai pas eu d’enfants. Je n’ai pas eu le temps.\par
Le sergent recommença.\par
— Mais tes parents ! Voyons, madame, mets-nous au fait de tes parents. Moi, je m’appelle Radoub, je suis sergent, je suis de la rue du Cherche-Midi, mon père et ma mère en étaient, je peux parler de mes  parents. Parle-nous des tiens. Dis-nous ce que c’était que tes parents.\par
— C’étaient les Fléchard. Voilà tout.\par
— Oui, les Fléchard sont les Fléchard, comme les Radoub sont les Radoub. Mais on a un état. Quel était l’état de tes parents ? Qu’est-ce qu’ils faisaient ? Qu’est-ce qu’ils font ? Qu’est-ce qu’ils fléchardaient, tes Fléchard ?\par
— C’étaient des laboureurs. Mon père était infirme et ne pouvait travailler à cause qu’il avait reçu des coups de bâton que le seigneur, son seigneur, notre seigneur, lui avait fait donner, ce qui était une bonté, parce que mon père avait pris un lapin, pour le fait de quoi on était jugé à mort ; mais le seigneur avait fait grâce, et avait dit : Donnez-lui seulement cent coups de bâton ; et mon père était demeuré estropié.\par
— Et puis ?\par
— Mon grand-père était huguenot. Monsieur le curé l’a fait envoyer aux galères. J’étais toute petite.\par
— Et puis ?\par
— Le père de mon mari était un faux-saulnier. Le roi l’a fait pendre.\par
— Et ton mari, qu’est-ce qu’il fait ?\par
— Ces jours-ci il se battait.\par
— Pour qui ?\par
— Pour le roi.\par
— Et puis ?\par
— Dame, pour son seigneur.\par
— Et puis ?\par
 — Dame, pour monsieur le curé.\par
— Sacré mille noms de noms de brutes ! cria un grenadier.\par
La femme eut un soubresaut d’épouvante.\par
— Vous voyez, madame, nous sommes des Parisiens, dit gracieusement la vivandière.\par
La femme joignit les mains et cria :\par
— O mon Dieu seigneur Jésus !\par
— Pas de superstitions ! reprit le sergent.\par
La vivandière s’assit à côté de la femme, et attira entre ses genoux l’aîné des enfants, qui se laissa faire. Les enfants sont rassurés comme ils sont effarouchés, sans qu’on sache pourquoi. Ils ont on ne sait quels avertissements intérieurs.\par
— Ma pauvre bonne femme de ce pays-ci, vous avez de jolis mioches, c’est toujours ça. On devine leur âge. Le grand a quatre ans, son frère a trois ans. Par exemple, la momignarde qui tette est fameusement gouliafre. Ah ! la monstre ! Veux-tu bien ne pas manger ta mère comme ça ! Voyez-vous, madame, ne craignez rien. Vous devriez entrer dans le bataillon. Vous feriez comme moi. Je m’appelle Houzarde. C’est un sobriquet. Mais j’aime mieux m’appeler Houzarde que mamzelle Bicorneau, comme ma mère. Je suis la cantinière, comme qui dirait celle qui donne à boire quand on se mitraille et qu’on s’assassine. Le diable et son train. Nous avons à peu près le même pied, je vous donnerai des souliers à moi. J’étais à Paris le 10 août. J’ai donné à boire à Westermann. Ça a marché. J’ai vu guillotiner Louis XVI. Louis Capet, qu’on  appelle. Il ne voulait pas. Dame, écoutez donc. Dire que le 13 janvier il faisait cuire des marrons et qu’il riait avec sa famille ! Quand on l’a couché de force sur la bascule, qu’on appelle, il n’avait plus ni habit ni souliers ; il n’avait que sa chemise, une veste piquée, une culotte de drap gris et des bas de soie gris. J’ai vu ça, moi. Le fiacre où on l’a amené était peint en vert. Voyez-vous, venez avec nous. On est des bons garçons dans le bataillon, vous serez la cantinière numéro deux, je vous montrerai l’état. Oh ! c’est bien simple ! on a son bidon et sa clochette, on s’en va dans le vacarme, dans les feux de peloton, dans les coups de canon, dans le hourvari, en criant : Qui est-ce qui veut boire un coup, les enfants ? Ce n’est pas plus malaisé que ça. Moi, je verse à boire à tout le monde. Ma foi oui. Aux blancs comme aux bleus, quoique je sois une bleue. Et même une bonne bleue. Mais je donne à boire à tous. Les blessés, ça a soif. On meurt sans distinction d’opinion. Les gens qui meurent, ça devrait se serrer la main. Comme c’est godiche de se battre ! Venez avec nous. Si je suis tuée, vous aurez ma survivance. Voyez-vous, j’ai l’air comme ça, mais je suis une bonne femme et un brave homme. Ne craignez rien.\par
Quand la vivandière eut cessé de parler, la femme murmura :\par
— Notre voisine s’appelait Marie-Jeanne et notre servante s’appelait Marie-Claude.\par
Cependant le sergent Radoub admonestait le grenadier.\par
 — Tais-toi. Tu as fait peur à madame. On ne jure pas devant les dames.\par
— C’est que c’est tout de même un véritable massacrement pour l’entendement d’un honnête homme, répliqua le grenadier, que de voir des iroquois de la Chine qui ont eu leur beau-père estropié par le seigneur, leur grand-père galérien par le curé, et leur père pendu par le roi, et qui se battent, nom d’un petit bonhomme ! et qui se fichent en révolte, et qui se font écrabouiller pour le seigneur, le curé et le roi !\par
Le sergent cria :\par
— Silence dans les rangs !\par
— On se tait, sergent, reprit le grenadier ; mais ça n’empêche pas que c’est ennuyeux qu’une jolie femme comme ça s’expose à se faire casser la gueule pour les beaux yeux d’un calotin.\par
— Grenadier, dit le sergent, nous ne sommes pas ici au club de la section des Piques. Pas d’éloquence.\par
Et il se tourna vers la femme.\par
— Et ton mari, madame ? que fait-il ? Qu’est-ce qu’il est devenu ?\par
— Il est devenu rien, puisqu’on l’a tué.\par
— Où ça ?\par
— Dans la haie.\par
— Quand ça ?\par
— Il y a trois jours.\par
— Qui ça ?\par
— Je ne sais pas.\par
— Comment ! tu ne sais pas qui a tué ton mari ?\par
— Non.\par
 — Est-ce un bleu ? Est-ce un blanc ?\par
— C’est un coup de fusil.\par
— Et il y a trois jours ?\par
— Oui.\par
— De quel côté ?\par
— Du côté d’Ernée. Mon mari est tombé. Voilà.\par
— Et depuis que ton mari est mort, qu’est-ce que tu fais ?\par
— J’emporte mes petits.\par
— Où les emportes-tu ?\par
— Devant moi.\par
— Où couches-tu ?\par
— Par terre.\par
— Qu’est-ce que tu manges ?\par
— Rien.\par
Le sergent eut cette moue militaire qui fait toucher le nez par les moustaches.\par
— Rien ?\par
— C’est-à-dire des prunelles, des mûres dans les ronces, quand il y en a de reste de l’an passé, des graines de myrtille, des pousses de fougère.\par
— Oui. Autant dire rien.\par
L’aîné des enfants, qui semblait comprendre, dit : J’ai faim.\par
Le sergent tira de sa poche un morceau de pain de munition et le tendit à la mère. La mère rompit le pain en deux morceaux et les donna aux enfants. Les petits mordirent avidement.\par
— Elle n’en a pas gardé pour elle, grommela le sergent.\par
 — C’est qu’elle n’a pas faim, dit un soldat.\par
— C’est qu’elle est la mère, dit le sergent.\par
Les enfants s’interrompirent.\par
— A boire, dit l’un.\par
— A boire, répéta l’autre.\par
— Il n’y a pas de ruisseau dans ce bois du diable, dit le sergent.\par
La vivandière prit le gobelet de cuivre qui pendait à sa ceinture à côté de sa clochette, tourna le robinet du bidon qu’elle avait en bandoulière, versa quelques gouttes dans le gobelet et approcha le gobelet des lèvres des enfants.\par
Le premier but et fit la grimace.\par
Le second but et cracha.\par
— C’est pourtant bon, dit la vivandière.\par
— C’est du coupe-figure ? demanda le sergent.\par
— Oui, et du meilleur. Mais ce sont des paysans.\par
Et elle essuya son gobelet.\par
Le sergent reprit :\par
— Et comme ça, madame, tu te sauves ?\par
— Il faut bien.\par
— A travers champs, va comme je te pousse !\par
— Je cours de toutes mes forces, et puis je marche, et puis je tombe.\par
— Pauvre paroissienne ! dit la vivandière.\par
— Les gens se battent, balbutia la femme. Je suis tout entourée de coups de fusil. Je ne sais pas ce qu’on se veut. On m’a tué mon mari. Je n’ai compris que ça.\par
Le sergent fit sonner à terre la crosse de son fusil, et cria :\par
 — Quelle bête de guerre ! nom d’une bourrique !\par
La femme continua :\par
— La nuit passée, nous avons couché dans une émousse.\par
— Tous les quatre ?\par
— Tous les quatre.\par
— Couché ?\par
— Couché.\par
— Alors, dit le sergent, couché debout.\par
Et il se tourna vers les soldats.\par
— Camarades, un gros vieux arbre creux et mort où un homme peut se fourrer comme dans une gaîne, ces sauvages appellent ça une émousse. Qu’est-ce que vous voulez ? Ils ne sont pas forcés d’être de Paris.\par
— Coucher dans le creux d’un arbre ! dit la vivandière, et avec trois enfants !\par
— Et, reprit le sergent, quand les petits gueulaient, pour les gens qui passaient et qui ne voyaient rien du tout, ça devait être drôle d’entendre un arbre crier \emph{papa, maman !}\par
— Heureusement, c’est l’été, soupira la femme.\par
Elle regardait la terre, résignée, ayant dans les yeux l’étonnement des catastrophes.\par
Les soldats silencieux faisaient cercle autour de cette misère.\par
Une veuve, trois orphelins, la fuite, l’abandon, la solitude, la guerre grondant tout autour de l’horizon, la faim, la soif, pas d’autre nourriture que l’herbe, pas d’autre toit que le ciel.\par
Le sergent s’approcha de la femme et fixa ses  yeux sur l’enfant qui tétait. La petite quitta le sein, tourna doucement la tête, regarda avec ses belles prunelles bleues l’effrayante face velue, hérissée et fauve qui se penchait sur elle, et se mit à sourire.\par
Le sergent se redressa, et l’on vit une grosse larme rouler sur sa joue et s’arrêter au bout de sa moustache comme une perle.\par
Il éleva la voix.\par
— Camarades, de tout ça je conclus que le bataillon va devenir père. Est-ce convenu ? Nous adoptons ces trois enfants-là.\par
— Vive la République ! crièrent les grenadiers.\par
— C’est dit, fit le sergent.\par
Et il étendit les deux mains au-dessus de la mère et des enfants.\par
— Voilà, dit-il, les enfants du bataillon du Bonnet-Rouge.\par
La vivandière sauta de joie.\par
— Trois têtes dans un bonnet ! cria-t-elle.\par
Puis elle éclata en sanglots, embrassa éperdument la pauvre veuve, et lui dit :\par
— Comme la petite a déjà l’air gamine !\par
— Vive la République ! répétèrent les soldats.\par
Et le sergent dit à la mère :\par
— Venez, citoyenne.
 \subsection[{Livre deuxième. La corvette claymore}]{Livre deuxième \\
La corvette claymore}\phantomsection
\label{p1l2}
\subsubsection[{I. Angleterre et France mêlées}]{I \\
Angleterre et France mêlées}\phantomsection
\label{p1l2c1}
\noindent Au printemps de 1793, au moment où la France, attaquée à la fois à toutes ses frontières, avait la pathétique distraction de la chute des Girondins, voici ce qui se passait dans l’archipel de la Manche.\par
Un soir, le 1\textsuperscript{er} juin, à Jersey, dans la petite baie déserte de Bonnenuit, une heure environ avant le coucher du soleil, par un de ces temps brumeux qui sont commodes pour s’enfuir parce qu’ils sont dangereux pour naviguer, une corvette mettait à la voile. Ce bâtiment était monté par un équipage français, mais faisait partie de la flottille anglaise placée en station et comme en sentinelle à la pointe orientale de l’île. Le prince de La Tour-d’Auvergne, qui était de la maison de Bouillon, commandait la flottille anglaise,  et c’était par ses ordres, et pour un service urgent et spécial, que la corvette en avait été détachée.\par
Cette corvette, immatriculée à la Trinity-House sous le nom de \emph{the Claymore}, était en apparence une corvette de charge, mais en réalité une corvette de guerre. Elle avait la lourde et pacifique allure marchande ; il ne fallait pas s’y fier pourtant. Elle avait été construite à deux fins, ruse et force ; tromper, s’il est possible, combattre, s’il est nécessaire. Pour le service qu’elle avait à faire cette nuit-là, le chargement avait été remplacé dans l’entre-pont par trente caronades de fort calibre. Ces trente caronades, soit qu’on prévît une tempête, soit plutôt qu’on voulût donner une figure débonnaire au navire, étaient à la serre, c’est-à-dire fortement amarrées en dedans par de triples chaînes et la volée appuyée aux écoutilles tamponnées ; rien ne se voyait au dehors ; les sabords étaient aveuglés ; les panneaux étaient fermés ; c’était comme un masque mis à la corvette. Ces caronades étaient à roue de bronze à rayons, ancien modèle, dit « modèle radié ». Les corvettes d’ordonnance n’ont de canons que sur le pont ; celle-ci, faite pour la surprise et l’embûche, était à pont désarmé, et avait été construite de façon à pouvoir porter, comme on vient de le voir, une batterie d’entre-pont. \emph{La Claymore }était d’un gabarit massif et trapu, et pourtant bonne marcheuse ; c’était la coque la plus solide de toute la marine anglaise, et au combat elle valait presque une frégate, quoiqu’elle n’eût pour mât d’artimon qu’un mâtereau avec une simple brigantine. Son gouvernail,  de forme rare et savante, avait une membrure courbe presque unique qui avait coûté cinquante livres sterling dans les chantiers de Southampton.\par
L’équipage, tout français, était composé d’officiers émigrés et de matelots déserteurs. Ces hommes étaient triés ; pas un qui ne fût bon marin, bon soldat et bon royaliste. Ils avaient le triple fanatisme du navire, de l’épée et du roi.\par
Un demi-bataillon d’infanterie de marine, pouvant au besoin être débarqué, était amalgamé à l’équipage.\par
La corvette \emph{Claymore} avait pour capitaine un chevalier de Saint-Louis, le comte du Boisberthelot, un des meilleurs officiers de l’ancienne marine royale, pour second le chevalier de La Vieuville qui avait commandé aux gardes-françaises la compagnie où Hoche avait été sergent, et pour pilote le plus sagace patron de Jersey, Philip Gacquoil.\par
On devinait que ce navire avait à faire quelque chose d’extraordinaire. Un homme en effet venait de s’y embarquer, qui avait tout l’air d’entrer dans une aventure. C’était un haut vieillard, droit et robuste, à figure sévère, dont il eût été difficile de préciser l’âge, parce qu’il semblait à la fois vieux et jeune ; un de ces hommes qui sont pleins d’années et pleins de force, qui ont des cheveux blancs sur le front et un éclair dans le regard ; quarante ans pour la vigueur et quatrevingts ans pour l’autorité. Au moment où il était monté sur la corvette, son manteau de mer s’était entr’ouvert, et l’on avait pu le voir vêtu, sous ce manteau, de larges braies dites \emph{bragou-bras}, de  bottes-jambières, et d’une veste en peau de chèvre montrant en dessus le cuir passementé de soie, et en dessous le poil hérissé et sauvage, costume complet de paysan breton. Ces anciennes vestes bretonnes étaient à deux fins, servaient aux jours de fête comme aux jours de travail, et se retournaient, offrant à volonté le côté velu et le côté brodé ; peaux de bête toute la semaine, habits de gala le dimanche. Le vêtement de paysan que portait ce vieillard était, comme pour ajouter à une vraisemblance cherchée et voulue, usé aux genoux et aux coudes, et paraissait avoir été longtemps porté, et le manteau de mer, de grosse étoffe, ressemblait à un haillon de pêcheur. Ce vieillard avait sur la tête le chapeau rond du temps, à haute forme et à large bord, qui, rabattu, a l’aspect campagnard, et, relevé d’un côté par une ganse à cocarde, a l’aspect militaire. Il portait ce chapeau rabaissé à la paysanne, sans ganse ni cocarde.\par
Lord Balcarras, gouverneur de l’île, et le prince de La Tour-d’Auvergne, l’avaient en personne conduit et installé à bord. L’agent secret des princes, Gélambre, ancien garde du corps de M. le comte d’Artois, avait lui-même veillé à l’aménagement de sa cabine, poussant le soin et le respect, quoique fort bon gentilhomme, jusqu’à porter derrière ce vieillard sa valise. En le quittant pour retourner à terre, M. de Gélambre avait fait à ce paysan un profond salut ; lord Balcarras lui avait dit : \emph{Bonne chance, général}, et le prince de la Tour-d’Auvergne lui avait dit : \emph{Au revoir, mon cousin}.\par
« Le paysan », c’était en effet le nom sous lequel  les gens de l’équipage s’étaient mis tout de suite à désigner leur passager, dans les courts dialogues que les hommes de mer ont entre eux ; mais, sans en savoir plus long, ils comprenaient que ce paysan n’était pas plus un paysan que la corvette de guerre n’était une corvette de charge.\par
Il y avait peu de vent. \emph{La Claymore} quitta Bonnenuit, passa devant Boulay-Bay, et fut quelque temps en vue, courant des bordées ; puis elle décrut dans la nuit croissante, et s’effaça.\par
Une heure après, Gélambre, rentré chez lui à Saint-Hélier, expédia, par l’exprès de Southampton, à M. le comte d’Artois, au quartier général du duc d’York, les quatre lignes qui suivent :\par
« Monseigneur, le départ vient d’avoir lieu. Succès certain. Dans huit jours toute la côte sera en feu, de Granville à Saint-Malo. »\par
Quatre jours auparavant, par émissaire secret, le représentant Prieur de la Marne, en mission près de l’armée des côtes de Cherbourg, et momentanément en résidence à Granville, avait reçu, écrit de la même écriture que la dépêche précédente, le message qu’on va lire :\par
« Citoyen représentant, le 1\textsuperscript{er} juin, à l’heure de la marée, la corvette de guerre \emph{Claymore}, à batterie masquée, appareillera pour déposer sur la côte de France un homme dont voici le signalement : haute taille, vieux, cheveux blancs, habits de paysan, mains d’aristocrate. Je vous enverrai demain plus de détails. Il débarquera le 2 au matin. Avertissez la croisière, capturez la corvette, faites guillotiner l’homme. »
 \subsubsection[{II. Nuit sur le navire et sur le passager}]{II \\
Nuit sur le navire et sur le passager}\phantomsection
\label{p1l2c2}
\noindent La corvette, au lieu de prendre par le sud et de se diriger vers Sainte-Catherine, avait mis le cap au nord, puis avait tourné à l’ouest et s’était résolûment engagée entre Serk et Jersey dans le bras de mer qu’on appelle le Passage de la Déroute. Il n’y avait alors de phare sur aucun point de ces deux côtes.\par
Le soleil s’était bien couché ; la nuit était noire, plus que ne le sont d’ordinaire les nuits d’été ; c’était une nuit de lune, mais de vastes nuages, plutôt de l’équinoxe que du solstice, plafonnaient le ciel, et, selon toute apparence, la lune ne serait visible que lorsqu’elle toucherait l’horizon, au moment de son coucher. Quelques nuées pendaient jusque sur la mer et la couvraient de brume.\par
Toute cette obscurité était favorable.\par
L’intention du pilote Gacquoil était de laisser Jersey à gauche et Guernesey à droite, et de gagner, par une marche hardie entre les Hanois et les Douvres, une baie quelconque du littoral de Saint-Malo, route moins courte que par les Minquiers, mais plus  sûre, la croisière française ayant pour consigne habituelle de faire surtout le guet entre Saint-Hélier et Granville.\par
Si le vent s’y prêtait, si rien ne survenait, et en couvrant la corvette de toile, Gacquoil espérait toucher la côte de France au point du jour.\par
Tout allait bien, la corvette venait de dépasser Gros-Nez ; vers neuf heures, le temps fit mine de bouder, comme disent les marins, et il y eut du vent et de la mer ; mais ce vent était bon, et cette mer était forte sans être violente. Pourtant, à de certains coups de lame, l’avant de la corvette embarquait.\par
Le « paysan » que lord Balcarras avait appelé \emph{général}, et auquel le prince de La Tour-d’Auvergne avait dit : \emph{mon cousin}, avait le pied marin et se promenait avec une gravité tranquille sur le pont de la corvette. Il n’avait pas l’air de s’apercevoir qu’elle était fort secouée. De temps en temps il tirait de la poche de sa veste une tablette de chocolat dont il cassait et mâchait un morceau, ses cheveux blancs n’empêchant pas qu’il eût toutes ses dents.\par
Il ne parlait à personne, si ce n’est, par instants, bas et brièvement, au capitaine, qui l’écoutait avec déférence et semblait considérer ce passager comme plus commandant que lui-même.\par
\emph{La Claymore}, habilement pilotée, côtoya, inaperçue dans le brouillard, le long escarpement nord de Jersey, serrant de près la côte, à cause du redoutable écueil Pierres-de-Leeq qui est au milieu du bras de mer entre Jersey et Serk. Gacquoil, debout à la  barre, signalant tour à tour la Grève de Leeq, Gros-Nez, Plémont, faisait glisser la corvette parmi ces chaînes de récifs, en quelque sorte à tâtons, mais avec certitude, comme un homme qui est de la maison et qui connaît les êtres de l’océan. La corvette n’avait pas de feu à l’avant, de crainte de dénoncer son passage dans ces mers surveillées. On se félicitait du brouillard. On atteignit la Grande-Étape ; la brume était si épaisse qu’à peine distinguait-on la haute silhouette du Pinacle. On entendit dix heures sonner au clocher de Saint-Ouen, signe que le vent se maintenait vent-arrière. Tout continuait d’aller bien ; la mer devenait plus houleuse à cause du voisinage de la Corbière.\par
Un peu après dix heures, le comte du Boisberthelot et le chevalier de La Vieuville reconduisirent l’homme aux habits de paysan jusqu’à sa cabine, qui était la propre chambre du capitaine. Au moment d’y entrer, il leur dit en baissant la voix :\par
— Vous le savez, messieurs, le secret importe. Silence jusqu’au moment de l’explosion. Vous seuls connaissez ici mon nom.\par
— Nous l’emporterons au tombeau, répondit Boisberthelot.\par
— Quant à moi, repartit le vieillard, fussé-je devant la mort, je ne le dirais pas.\par
Et il entra dans sa chambre.
 \subsubsection[{III. Noblesse et roture mêlées}]{III \\
Noblesse et roture mêlées}\phantomsection
\label{p1l2c3}
\noindent Le commandant et le second remontèrent sur le pont et se mirent à marcher côte à côte en causant. Ils parlaient évidemment de leur passager, et voici à peu près le dialogue que le vent dispersait dans les ténèbres.\par
Boisberthelot grommela à demi-voix à l’oreille de La Vieuville :\par
— Nous allons voir si c’est un chef.\par
La Vieuville répondit :\par
— En attendant, c’est un prince.\par
— Presque.\par
— Gentilhomme en France, mais prince en Bretagne.\par
— Comme les La Trémoille, comme les Rohan.\par
— Dont il est l’allié.\par
Boisberthelot reprit :\par
— En France et dans les carrosses du roi, il est marquis comme je suis comte et comme vous êtes chevalier.\par
— Ils sont loin les carrosses ! s’écria La Vieuville. Nous en sommes au tombereau.\par
 Il y eut un silence.\par
Boisberthelot repartit :\par
— A défaut d’un prince français, on prend un prince breton.\par
— Faute de grives... Non, faute d’un aigle, on prend un corbeau.\par
— J’aimerais mieux un vautour, dit Boisberthelot.\par
Et La Vieuville répliqua :\par
— Certes ! un bec et des griffes.\par
— Nous allons voir.\par
— Oui, reprit La Vieuville, il est temps qu’il y ait un chef. Je suis de l’avis de Tinténiac : \emph{un chef et de la poudre !} Tenez, commandant, je connais à peu près tous les chefs possibles et impossibles ; ceux d’hier, ceux d’aujourd’hui et ceux de demain ; pas un n’est la caboche de guerre qu’il nous faut. Dans cette diable de Vendée, il faut un général qui soit en même temps un procureur ; il faut ennuyer l’ennemi, lui disputer le moulin, le buisson, le fossé, le caillou, lui faire de mauvaises querelles, tirer parti de tout, veiller à tout, massacrer beaucoup, faire des exemples, n’avoir ni sommeil ni pitié. A cette heure, dans cette armée de paysans, il y a des héros, il n’y a pas de capitaines. D’Elbée est nul, Lescure est malade, Bonchamps fait grâce ; il est bon, c’est bête. La Rochejaquelein est un magnifique sous-lieutenant ; Silz est un officier de rase campagne, impropre à la guerre d’expédients ; Cathelineau est un charretier naïf, Stofflet est un garde-chasse rusé, Bérard est inepte, Boulainvilliers est ridicule, Charette est horrible. Et je ne parle pas  du barbier Gaston. Car, mordemonbleu ! à quoi bon chamailler la révolution et quelle différence y a-t-il entre les républicains et nous si nous faisons commander les gentilshommes par les perruquiers ?\par
— C’est que cette chienne de révolution nous gagne, nous aussi.\par
— Une gale qu’a la France !\par
— Gale du tiers état, reprit Boisberthelot. L’Angleterre seule peut nous tirer de là.\par
— Elle nous en tirera, n’en doutez pas, capitaine.\par
— En attendant, c’est laid.\par
— Certes, des manants partout ; la monarchie qui a pour général en chef Stofflet, garde-chasse de M. de Maulevrier, n’a rien à envier à la république qui a pour ministre Pache, fils du portier du duc de Castries. Quel vis-à-vis que cette guerre de la Vendée : d’un côté Santerre le brasseur, de l’autre Gaston le merlan !\par
— Mon cher La Vieuville, je fais un certain cas de ce Gaston. Il n’a point mal agi dans son commandement de Guéménée. Il a gentiment arquebusé trois cents bleus après leur avoir fait creuser leur fosse par eux-mêmes.\par
— A la bonne heure, mais je l’eusse fait tout aussi bien que lui.\par
— Pardieu, sans doute. Et moi aussi.\par
— Les grands actes de guerre, reprit La Vieuville, veulent de la noblesse dans qui les accomplit. Ce sont choses de chevaliers et non de perruquiers.\par
— Il y a pourtant dans ce tiers état, répliqua  Boisberthelot, des hommes estimables. Tenez, par exemple, cet horloger Joly. Il avait été sergent au régiment de Flandre, il se fait chef vendéen, il commande une bande de la côte ; il a un fils, qui est républicain, et, pendant que le père sert dans les blancs, le fils sert dans les bleus. Rencontre. Bataille. Le père fait prisonnier son fils, et lui brûle la cervelle.\par
— Celui-là est bien, dit La Vieuville.\par
— Un Brutus royaliste, reprit Boisberthelot.\par
— Cela n’empêche pas qu’il est insupportable d’être commandé par un Coquereau, un Jean-Jean, un Moulins, un Focart, un Bouju, un Chouppes !\par
— Mon cher chevalier, la colère est la même de l’autre côté. Nous sommes pleins de bourgeois ; ils sont pleins de nobles. Croyez-vous que les sans-culottes soient contents d’être commandés par le comte de Canclaux, le vicomte de Miranda, le vicomte de Beauharnais, le comte de Valence, le marquis de Custine et le duc de Biron !\par
— Quel gâchis !\par
— Et le duc de Chartres !\par
— Fils d’Égalité. Ah çà, quand sera-t-il roi, celui-là ?\par
— Jamais.\par
— Il monte au trône. Il est servi par ses crimes.\par
— Et desservi par ses vices, dit Boisberthelot.\par
Il y eut encore un silence, et Boisberthelot poursuivit :\par
— Il avait pourtant voulu se réconcilier. Il était  venu voir le roi. J’étais là, à Versailles, quand on lui a craché dans le dos.\par
— Du haut du grand escalier ?\par
— Oui.\par
— On a bien fait.\par
— Nous l’appelions Bourbon le Bourbeux.\par
— Il est chauve, il a des pustules, il est régicide, pouah !\par
Et La Vieuville ajouta :\par
— Moi, j’étais à Ouessant avec lui.\par
— Sur le \emph{Saint-Esprit ?}\par
— Oui.\par
— S’il eût obéi au signal de tenir le vent que lui faisait l’amiral d’Orvilliers, il empêchait les Anglais de passer.\par
— Certes.\par
— Est-il vrai qu’il se soit caché à fond de cale ?\par
— Non. Mais il faut le dire tout de même.\par
Et La Vieuville éclata de rire.\par
Boisberthelot repartit :\par
— Il y a des imbéciles. Tenez, ce Boulainvilliers dont vous parliez, La Vieuville, je l’ai connu, je l’ai vu de près. Au commencement, les paysans étaient armés de piques ; ne s’était-il pas fourré dans la tête d’en faire des piquiers ? Il voulait leur apprendre l’exercice de la pique-en-biais et de la pique-traînante-le-fer-devant. Il avait rêvé de transformer ces sauvages en soldats de ligne. Il prétendait leur enseigner à émousser les angles d’un carré et à faire des bataillons à centre vide. Il leur baragouinait la vieille langue militaire ;  pour dire un chef d’escouade, il disait un \emph{cap d’escadre,} ce qui était l’appellation des caporaux sous Louis XIV. Il s’obstinait à créer un régiment avec tous ces braconniers ; il avait des compagnies régulières dont les sergents se rangeaient en rond tous les soirs, recevant le mot et le contre-mot du sergent de la colonelle qui les disait tout bas au sergent de la lieutenance, lequel les disait à son voisin qui les transmettait au plus proche, et ainsi d’oreille en oreille jusqu’au dernier. Il cassa un officier qui ne s’était pas levé tête nue pour recevoir le mot d’ordre de la bouche du sergent. Vous jugez comme cela a réussi. Ce butor ne comprenait pas que les paysans veulent être menés à la paysanne, et qu’on ne fait pas des hommes de caserne avec des hommes des bois. Oui, j’ai connu ce Boulainvilliers-là.\par
Ils firent quelques pas, chacun songeant de son côté.\par
Puis la causerie continua.\par
— A propos, se confirme-t-il que Dampierre soit tué ?\par
— Oui, commandant.\par
— Devant Condé ?\par
— Au camp de Pamars. D’un boulet de canon.\par
Boisberthelot soupira.\par
— Le comte de Dampierre. Encore un des nôtres qui était des leurs !\par
— Bon voyage ! dit La Vieuville.\par
— Et Mesdames ? où sont-elles ?\par
— A Trieste.\par
— Toujours ?\par
 — Toujours.\par
Et La Vieuville s’écria :\par
— Ah ! cette république ! que de dégâts pour peu de chose ! Quand on pense que cette révolution est venue pour un déficit de quelques millions !\par
— Se défier des petits points de départ, dit Boisberthelot.\par
— Tout va mal, reprit La Vieuville.\par
— Oui, La Rouarie est mort, Du Dresnay est idiot. Quels tristes meneurs que tous ces évêques, ce Coucy, l’évêque de la Rochelle, ce Beaupoil Saint-Aulaire, l’évêque de Poitiers, ce Mercy, l’évêque de Luçon, amant de madame de L’Eschasserie !...\par
— Laquelle s’appelle Servanteau, vous savez, commandant ; L’Eschasserie est un nom de terre.\par
— Et ce faux évêque d’Agra, qui est curé de je ne sais quoi !\par
— De Dol. Il s’appelle Guillot de Folleville. Il est brave, du reste, et se bat.\par
— Des prêtres quand il faudrait des soldats ! Des évêques qui ne sont pas des évêques ! des généraux qui ne sont pas des généraux !\par
La Vieuville interrompit Boisberthelot.\par
— Commandant, vous avez le \emph{Moniteur} dans votre cabine ?\par
— Oui.\par
— Qu’est-ce donc qu’on joue à Paris dans ce moment-ci ?\par
— \emph{Adèle et Paulin}, et \emph{la Caverne}.\par
— Je voudrais voir ça.\par
 — Vous le verrez. Nous serons à Paris dans un mois.\par
Boisberthelot réfléchit un instant et ajouta :\par
— Au plus tard. M. Windham l’a dit à milord Hood.\par
— Mais alors, commandant, tout ne va pas si mal ?\par
— Tout irait bien, parbleu, à la condition que la guerre de Bretagne fût bien conduite.\par
La Vieuville hocha la tête.\par
— Commandant, reprit-il, débarquerons-nous l’infanterie de marine ?\par
— Oui, si la côte est pour nous ; non, si elle est hostile. Quelquefois il faut que la guerre enfonce les portes, quelquefois il faut qu’elle se glisse. La guerre civile doit toujours avoir dans sa poche une fausse clef. On fera le possible. Ce qui importe, c’est le chef.\par
Et Boisberthelot, pensif, ajouta :\par
— La Vieuville, que penseriez-vous du chevalier de Dieuzie ?\par
— Du jeune ?\par
— Oui.\par
— Pour commander ?\par
— Oui.\par
— Que c’est encore un officier de plaine et de bataille rangée. La broussaille ne connaît que le paysan.\par
— Alors, résignez-vous au général Stofflet et au général Cathelineau.\par
La Vieuville rêva un moment, et dit :\par
— Il faudrait un prince, un prince de France, un prince du sang. Un vrai prince.\par
— Pourquoi ? Qui dit prince...\par
 — Dit poltron. Je le sais, commandant. Mais c’est pour l’effet sur les gros yeux bêtes des gars.\par
— Mon cher chevalier, les princes ne veulent pas venir.\par
— On s’en passera.\par
Boisberthelot fit ce mouvement machinal qui consiste à se presser le front avec la main, comme pour en faire sortir une idée.\par
Il reprit :\par
— Enfin, essayons de ce général-ci.\par
— C’est un grand gentilhomme.\par
— Croyez-vous qu’il suffira ?\par
— Pourvu qu’il soit bon, dit La Vieuville.\par
— C’est-à-dire féroce, dit Boisberthelot.\par
Le comte et le chevalier se regardèrent.\par
— Monsieur du Boisberthelot, vous avez dit le mot. Féroce. Oui, c’est là ce qu’il nous faut. Ceci est la guerre sans miséricorde. L’heure est aux sanguinaires. Les régicides ont coupé la tête à Louis XVI, nous arracherons les quatre membres aux régicides. Oui, le général nécessaire est le général Inexorable. Dans l’Anjou et dans le haut Poitou, les chefs font les magnanimes, on patauge dans la générosité, rien ne va. Dans le Marais et dans le pays de Retz, les chefs sont atroces, tout marche. C’est parce que Charette est féroce qu’il tient tête à Parrein. Hyène contre hyène.\par
Boisberthelot n’eut pas le temps de répondre à La Vieuville. La Vieuville eut la parole brusquement coupée par un cri désespéré, et en même temps on  entendit un bruit qui ne ressemblait à aucun des bruits qu’on entend. Ce cri et ces bruits venaient du dedans du navire.\par
Le capitaine et le lieutenant se précipitèrent vers l’entre-pont, mais ne purent y entrer. Tous les canonniers remontaient éperdus.\par
Une chose effrayante venait d’arriver.
 \subsubsection[{IV. Tormentum belli}]{IV \\
Tormentum belli}\phantomsection
\label{p1l2c4}
\noindent Une des caronades de la batterie, une pièce de vingt-quatre, s’était détachée.\par
Ceci est le plus redoutable peut-être des événements de mer. Rien de plus terrible ne peut arriver à un navire de guerre au large et en pleine marche.\par
Un canon qui casse son amarre devient brusquement on ne sait quelle bête surnaturelle. C’est une machine qui se transforme en un monstre. Cette masse court sur ses roues, a des mouvements de bille de billard, penche avec le roulis, plonge avec le tangage, va, vient, s’arrête, paraît méditer, reprend sa course, traverse comme une flèche le navire d’un bout à l’autre, pirouette, se dérobe, s’évade, se cabre, heurte, ébrèche, tue, extermine. C’est un bélier qui bat à sa fantaisie une muraille. Ajoutez ceci : le bélier est de fer, la muraille est de bois. C’est l’entrée en liberté de la matière ; on dirait que cet esclave éternel se venge ; il semble que la méchanceté qui est dans ce que nous appelons les objets inertes sorte et éclate tout à coup ; cela a l’air de perdre patience  et de prendre une étrange revanche obscure ; rien de plus inexorable que la colère de l’inanimé. Ce bloc forcené a les sauts de la panthère, la lourdeur de l’éléphant, l’agilité de la souris, l’opiniâtreté de la cognée, l’inattendu de la houle, les coups de coude de l’éclair, la surdité du sépulcre. Il pèse dix mille, et il ricoche comme une balle d’enfant. Ce sont des tournoiements brusquement coupés d’angles droits. Et que faire ? Comment en venir à bout ? Une tempête cesse, un cyclône passe, un vent tombe, un mât brisé se remplace, une voie d’eau se bouche, un incendie s’éteint ; mais que devenir avec cette énorme brute de bronze ? De quelle façon s’y prendre ? Vous pouvez raisonner un dogue, étonner un taureau, fasciner un boa, effrayer un tigre, attendrir un lion ; aucune ressource avec ce monstre, un canon lâché. Vous ne pouvez pas le tuer, il est mort. Et en même temps, il vit. Il vit d’une vie sinistre qui lui vient de l’infini. Il a sous lui son plancher qui le balance. Il est remué par le navire qui est remué par la mer qui est remuée par le vent. Cet exterminateur est un jouet. Le navire, les flots, les souffles, tout cela le tient ; de là sa vie affreuse. Que faire à cet engrenage ? Comment entraver ce mécanisme monstrueux du naufrage ? Comment prévoir ces allées et venues, ces retours, ces arrêts, ces chocs ? Chacun de ses coups au bordage peut défoncer le navire. Comment deviner ces affreux méandres ? On a affaire à un projectile qui se ravise, qui a l’air d’avoir des idées, et qui change à chaque instant de direction. Comment arrêter  ce qu’il faut éviter ? L’horrible canon se démène, avance, recule, frappe à droite, frappe à gauche, fuit, passe, déconcerte l’attente, broie l’obstacle, écrase les hommes comme des mouches. Toute la terreur de la situation est dans la mobilité du plancher. Comment combattre un plan incliné qui a des caprices ? Le navire a, pour ainsi dire, dans le ventre la foudre prisonnière qui cherche à s’échapper ; quelque chose comme un tonnerre roulant sur un tremblement de terre.\par
En un instant tout l’équipage fut sur pied. La faute était au chef de pièce qui avait négligé de serrer l’écrou de la chaîne d’amarrage et mal entravé les quatre roues de la caronade ; ce qui donnait du jeu à la semelle et au châssis, désaccordait les deux plateaux, et avait fini par disloquer la brague. Le combleau s’était cassé, de sorte que le canon n’était plus ferme à l’affût. La brague fixe, qui empêche le recul, n’était pas encore en usage à cette époque. Un paquet de mer étant venu frapper le sabord, la caronade mal amarrée avait reculé et brisé sa chaîne, et s’était mise à errer formidablement dans l’entre-pont.\par
Qu’on se figure, pour avoir une idée de ce glissement étrange, une goutte d’eau courant sur une vitre.\par
Au moment où l’amarre cassa, les canonniers étaient dans la batterie. Les uns groupés, les autres épars, occupés aux ouvrages de mer que font les marins en prévoyance d’un branle-bas de combat. La caronade, lancée par le tangage, fit une trouée dans  ce tas d’hommes et en écrasa quatre du premier coup, puis, reprise et décochée par le roulis, elle coupa en deux un cinquième misérable, et alla heurter à la muraille de bâbord une pièce de la batterie qu’elle démonta. De là le cri de détresse qu’on venait d’entendre. Tous les hommes se pressèrent à l’escalier-échelle. La batterie se vida en un clin d’œil.\par
L’énorme pièce avait été laissée seule. Elle était livrée à elle-même. Elle était sa maîtresse, et la maîtresse du navire. Elle pouvait en faire ce qu’elle voulait. Tout cet équipage d’hommes accoutumés à rire dans la bataille tremblait. Dire l’épouvante est impossible.\par
Le capitaine Boisberthelot et le lieutenant La Vieuville, deux intrépides pourtant, s’étaient arrêtés au haut de l’escalier, et, muets, pâles, hésitants, regardaient dans l’entre-pont. Quelqu’un les écarta du coude et descendit.\par
C’était leur passager, le paysan, l’homme dont ils venaient de parler le moment d’auparavant.\par
Arrivé au bas de l’escalier-échelle, il s’arrêta.
 \subsubsection[{V. Vis et vir}]{V \\
Vis et vir}\phantomsection
\label{p1l2c5}
\noindent Le canon allait et venait dans l’entre-pont. On eût dit le chariot vivant de l’Apocalypse. Le falot de marine, oscillant sous l’étrave de la batterie, ajoutait à cette vision un vertigineux balancement d’ombre et de lumière. La forme du canon s’effaçait dans la violence de sa course, et il apparaissait, tantôt noir dans la clarté, tantôt reflétant de vagues blancheurs dans l’obscurité.\par
Il continuait l’exécution du navire. Il avait déjà fracassé quatre autres pièces et fait dans la muraille deux crevasses, heureusement au-dessus de la flottaison, mais par où l’eau entrerait, s’il survenait une bourrasque. Il se ruait frénétiquement sur la membrure ; les porques très robustes résistaient, les bois courbes ont une solidité particulière ; mais on entendait leurs craquements sous cette massue démesurée, frappant, avec une sorte d’ubiquité inouïe, de tous les côtés à la fois. Un grain de plomb secoué dans une bouteille n’a pas des percussions plus insensées et plus rapides. Les quatre roues passaient et repassaient  sur les hommes tués, les coupaient, les dépeçaient et les déchiquetaient, et des cinq cadavres avaient fait vingt tronçons qui roulaient à travers la batterie ; les têtes mortes semblaient crier ; des ruisseaux de sang se tordaient sur le plancher selon les balancements du roulis. Le vaigrage, avarié en plusieurs endroits, commençait à s’entr’ouvrir. Tout le navire était plein d’un bruit monstrueux.\par
Le capitaine avait promptement repris son sang-froid, et sur son ordre on avait jeté par le carré, dans l’entre-pont, tout ce qui pouvait amortir et entraver la course effrénée du canon, les matelas, les hamacs, les rechanges de voiles, les rouleaux de cordages, les sacs d’équipage, et les ballots de faux assignats dont la corvette avait tout un chargement, cette infamie anglaise étant regardée comme de bonne guerre.\par
Mais que pouvaient faire ces chiffons, personne n’osant descendre pour les disposer comme il eût fallu ? En quelques minutes ce fut de la charpie.\par
Il y avait juste assez de mer pour que l’accident fût aussi complet que possible. Une tempête eût été désirable ; elle eût peut-être culbuté le canon, et, une fois les quatre roues en l’air, on eût pu s’en rendre maître.\par
Cependant le ravage s’aggravait. Il y avait des écorchures et même des fractures aux mâts, qui, emboîtés dans la charpente de la quille, traversent les étages des navires et y font comme de gros piliers ronds. Sous les frappements convulsifs du canon, le mât de misaine s’était lézardé, le grand mât lui-même était entamé. La batterie se disloquait. Dix pièces sur  trente étaient hors de combat ; les brèches au bordage se multipliaient, et la corvette commençait à faire eau.\par
Le vieux passager descendu dans l’entre-pont semblait un homme de pierre au bas de l’escalier. Il jetait sur cette dévastation un œil sévère. Il ne bougeait point. Il paraissait impossible de faire un pas dans la batterie.\par
Chaque mouvement de la caronade en liberté ébauchait l’effondrement du navire. Encore quelques instants, et le naufrage était inévitable.\par
Il fallait périr ou couper court au désastre ; prendre un parti ; mais lequel ?\par
Quelle combattante que cette caronade !\par
Il s’agissait d’arrêter cette épouvantable folle.\par
Il s’agissait de colleter cet éclair.\par
Il s’agissait de terrasser cette foudre.\par
Boisberthelot dit à La Vieuville :\par
— Croyez-vous en Dieu, chevalier ?\par
La Vieuville répondit :\par
— Oui. Non. Quelquefois.\par
— Dans la tempête ?\par
— Oui. Et dans des moments comme celui-ci.\par
— Il n’y a en effet que Dieu qui puisse nous tirer de là, dit Boisberthelot.\par
Tous se taisaient, laissant la caronade faire son fracas horrible.\par
Du dehors, le flot battant le navire répondait aux chocs du canon par des coups de mer. On eût dit deux marteaux alternant.\par
 Tout à coup, dans cette espèce de cirque inabordable où bondissait le canon échappé, on vit un homme apparaître, une barre de fer à la main. C’était l’auteur de la catastrophe, le chef de pièce coupable de négligence et cause de l’accident, le maître de la caronade. Ayant fait le mal, il voulait le réparer. Il avait empoigné une barre d’anspect d’une main, une drosse à nœud coulant de l’autre main, et il avait sauté par le carré dans l’entre-pont.\par
Alors une chose farouche commença ; spectacle titanique ; le combat du canon contre le canonnier ; la bataille de la matière et de l’intelligence, le duel de la chose contre l’homme.\par
L’homme s’était posté dans un angle, et, sa barre et sa corde dans ses deux poings, adossé à une porque, affermi sur ses jarrets qui semblaient deux piliers d’acier, livide, calme, tragique, comme enraciné dans le plancher, il attendait.\par
Il attendait que le canon passât près de lui.\par
Le canonnier connaissait sa pièce, et il lui semblait qu’elle devait le connaître. Il vivait depuis longtemps avec elle. Que de fois il lui avait fourré la main dans la gueule ! C’était son monstre familier. Il se mit à lui parler comme à son chien. — Viens, disait-il. Il l’aimait peut-être.\par
Il paraissait souhaiter qu’elle vînt à lui.\par
Mais venir à lui, c’était venir sur lui. Et alors il était perdu. Comment éviter l’écrasement ? Là était la question. Tous regardaient, terrifiés. Pas une poitrine ne respirait librement, excepté peut-être celle du  vieillard qui était seul dans l’entre-pont avec les deux combattants, témoin sinistre.\par
Il pouvait lui-même être broyé par la pièce. Il ne bougeait pas.\par
Sous eux le flot, aveugle, dirigeait le combat.\par
Au moment où, acceptant ce corps-à-corps effroyable, le canonnier vint provoquer le canon, un hasard des balancements de la mer fit que la caronade demeura un moment immobile et comme stupéfaite. — Viens donc ! lui disait l’homme. Elle semblait écouter.\par
Subitement elle sauta sur lui. L’homme esquiva le choc.\par
La lutte s’engagea. Lutte inouïe. Le fragile se colletant avec l’invulnérable. Le belluaire de chair attaquant la bête d’airain. D’un côté une force, de l’autre une âme.\par
Tout cela se passait dans une pénombre. C’était comme la vision indistincte d’un prodige.\par
Une âme, chose étrange, on eût dit que le canon en avait une, lui aussi ; mais une âme de haine et de rage. Cette cécité paraissait avoir des yeux. Le monstre avait l’air de guetter l’homme. Il y avait, on l’eût pu croire du moins, de la ruse dans cette masse. Elle aussi choisissait son moment. C’était on ne sait quel gigantesque insecte de fer ayant ou semblant avoir une volonté de démon. Par moments, cette sauterelle colossale cognait le plafond bas de la batterie, puis elle retombait sur ses quatre roues comme un tigre sur ses quatre griffes, et se remettait à courir  sur l’homme. Lui, souple, agile, adroit, se tordait comme une couleuvre sous tous ces mouvements de foudre. Il évitait les rencontres, mais les coups auxquels il se dérobait tombaient sur le navire et continuaient de le démolir.\par
Un bout de chaîne cassée était resté accroché à la caronade. Cette chaîne s’était enroulée on ne sait comment dans la vis du bouton de culasse. Une extrémité de la chaîne était fixée à l’affût. L’autre, libre, tournoyait éperdument autour du canon dont elle exagérait tous les soubresauts. La vis la tenait comme une main fermée, et cette chaîne, multipliant les coups de bélier par des coups de lanière, faisait autour du canon un tourbillon terrible, fouet de fer dans un poing d’airain. Cette chaîne compliquait le combat.\par
Pourtant l’homme luttait. Même, par instants, c’était l’homme qui attaquait le canon ; il rampait le long du bordage, sa barre et sa corde à la main ; et le canon avait l’air de comprendre, et, comme s’il devinait un piège, fuyait. L’homme, formidable, le poursuivait.\par
De telles choses ne peuvent durer longtemps. Le canon sembla se dire tout à coup : Allons ! il faut en finir ! et il s’arrêta. On sentit l’approche du dénoûment. Le canon, comme en suspens, semblait avoir ou avait, car pour tous c’était un être, une préméditation féroce. Brusquement, il se précipita sur le canonnier. Le canonnier se rangea de côté, le laissa passer, et lui cria en riant : A refaire ! Le canon,  comme furieux, brisa une caronade à bâbord ; puis, ressaisi par la fronde invisible qui le tenait, il s’élança à tribord sur l’homme, qui échappa. Trois caronades s’effondrèrent sous la poussée du canon ; alors, comme aveugle et ne sachant plus ce qu’il faisait, il tourna le dos à l’homme, roula de l’arrière à l’avant, détraqua l’étrave, et alla faire une brèche à la muraille de proue. L’homme s’était réfugié au pied de l’escalier, à quelques pas du vieillard témoin. Le canonnier tenait sa barre d’anspect en arrêt. Le canon parut l’apercevoir, et, sans prendre la peine de se retourner, recula sur l’homme avec une promptitude de coup de hache. L’homme acculé au bordage était perdu. Tout l’équipage poussa un cri.\par
Mais le vieux passager jusqu’alors immobile s’était élancé, lui-même plus rapide que toutes ces rapidités farouches. Il avait saisi un ballot de faux assignats, et, au risque d’être écrasé, il avait réussi à le jeter entre les roues de la caronade. Ce mouvement décisif et périlleux n’eût pas été exécuté avec plus de justesse et de précision par un homme rompu à tous les exercices décrits dans le livre de Durosel sur la \emph{Manœuvre du canon de mer}.\par
Le ballot fit l’effet d’un tampon. Un caillou enraye un bloc, une branche d’arbre détourne une avalanche. La caronade trébucha. Le canonnier à son tour, saisissant ce joint redoutable, plongea sa barre de fer entre les rayons d’une des roues d’arrière. Le canon s’arrêta.\par
Il penchait. L’homme, d’un mouvement de levier  imprimé à la barre, le fit basculer. La lourde masse se renversa, avec le bruit d’une cloche qui s’écroule, et l’homme se ruant à corps perdu, ruisselant de sueur, passa le nœud coulant de la drosse au cou de bronze du monstre terrassé.\par
C’était fini. L’homme avait vaincu. La fourmi avait eu raison du mastodonte ; le pygmée avait fait le tonnerre prisonnier.\par
Les soldats et les marins battirent des mains.\par
Tout l’équipage se précipita avec des câbles et des chaînes, et en un instant le canon fut amarré.\par
Le canonnier salua le passager.\par
— Monsieur, lui dit-il, vous m’avez sauvé la vie.\par
Le vieillard avait repris son attitude impassible, et ne répondit pas.
 \subsubsection[{VI. Les deux plateaux de la balance}]{VI \\
Les deux plateaux de la balance}\phantomsection
\label{p1l2c6}
\noindent L’homme avait vaincu, mais on pouvait dire que le canon avait vaincu aussi. Le naufrage immédiat était évité, mais la corvette n’était point sauvée. Le délabrement du navire paraissait irrémédiable. Le bordage avait cinq brèches, dont une fort grande à l’avant ; vingt caronades sur trente gisaient dans leur cadre. La caronade ressaisie et remise à la chaîne était elle-même hors de service ; la vis du bouton de culasse était forcée, et par conséquent le pointage impossible. La batterie était réduite à neuf pièces. La cale faisait eau. Il fallait tout de suite courir aux avaries et faire jouer les pompes.\par
L’entre-pont, maintenant qu’on le pouvait regarder, était effroyable à voir. Le dedans d’une cage d’éléphant furieux n’est pas plus démantelé.\par
Quelle que fût pour la corvette la nécessité de ne pas être aperçue, il y avait une nécessité plus impérieuse encore, le sauvetage immédiat. Il avait fallu éclairer le pont par quelques falots plantés çà et là dans le bordage.\par
 Cependant, tout le temps qu’avait duré cette diversion tragique, l’équipage étant absorbé par une question de vie ou de mort, on n’avait guère su ce qui se passait hors de la corvette. Le brouillard s’était épaissi ; le temps avait changé ; le vent avait fait du navire ce qu’il avait voulu ; on était hors de route, à découvert de Jersey et de Guernesey, plus au sud qu’on ne devait l’être ; on se trouvait en présence d’une mer démontée. De grosses vagues venaient baiser les plaies béantes de la corvette, baisers redoutables. Le bercement de la mer était menaçant. La brise devenait bise. Une bourrasque, une tempête peut-être, se dessinait. On ne voyait pas à quatre lames devant soi.\par
Pendant que les hommes d’équipage réparaient en hâte et sommairement les ravages de l’entre-pont, aveuglaient les voies d’eau et remettaient en batterie les pièces échappées au désastre, le vieux passager était remonté sur le pont.\par
Il s’était adossé au grand mât.\par
Il n’avait point pris garde à un mouvement qui avait eu lieu dans le navire. Le chevalier de La Vieuville avait fait mettre en bataille des deux côtés du grand mât les soldats d’infanterie de marine, et, sur un coup de sifflet du maître d’équipage, les matelots occupés à la manœuvre s’étaient rangés debout sur les vergues.\par
Le comte du Boisberthelot s’avança vers le passager.\par
Derrière le capitaine marchait un homme hagard,  haletant, les habits en désordre, l’air satisfait pourtant.\par
C’était le canonnier qui venait de se montrer si à propos dompteur de monstres, et qui avait eu raison du canon.\par
Le comte fit au vieillard vêtu en paysan le salut militaire, et lui dit :\par
— Mon général, voilà l’homme.\par
Le canonnier se tenait debout, les yeux baissés, dans l’attitude d’ordonnance.\par
Le comte du Boisberthelot reprit :\par
— Mon général, en présence de ce qu’a fait cet homme, ne pensez-vous pas qu’il y a pour ses chefs quelque chose à faire ?\par
— Je le pense, dit le vieillard.\par
— Veuillez donner des ordres, repartit Boisberthelot.\par
— C’est à vous de les donner. Vous êtes le capitaine.\par
— Mais vous êtes le général, reprit Boisberthelot.\par
Le vieillard regarda le canonnier.\par
— Approche, dit-il.\par
Le canonnier fit un pas.\par
Le vieillard se tourna vers le comte du Boisberthelot, détacha la croix de Saint-Louis du capitaine, et la noua à la vareuse du canonnier.\par
— Hurrah ! crièrent les matelots.\par
Les soldats de marine présentèrent les armes.\par
Et le vieux passager, montrant du doigt le canonnier ébloui, ajouta :\par
— Maintenant, qu’on fusille cet homme.\par
La stupeur succéda à l’acclamation.\par
 Alors, au milieu d’un silence de tombe, le vieillard éleva la voix. Il dit :\par
— Une négligence a compromis ce navire. A cette heure il est peut-être perdu. Être en mer, c’est être devant l’ennemi. Un navire qui fait une traversée est une armée qui livre une bataille. La tempête se cache, mais ne s’absente pas. Toute la mer est une embuscade. Peine de mort à toute faute commise en présence de l’ennemi. Il n’y a pas de faute réparable. Le courage doit être récompensé, et la négligence doit être punie.\par
Ces paroles tombaient l’une après l’autre, lentement, gravement, avec une sorte de mesure inexorable, comme des coups de cognée sur un chêne.\par
Et le vieillard, regardant les soldats, ajouta :\par
— Faites.\par
L’homme à la veste duquel brillait la croix de Saint-Louis courba la tête.\par
Sur un signe du comte du Boisberthelot, deux matelots descendirent dans l’entre-pont, puis revinrent apportant le hamac-suaire ; l’aumônier du bord, qui depuis le départ était en prière dans le carré des officiers, accompagnait les deux matelots ; un sergent détacha de la ligne de bataille douze soldats qu’il rangea sur deux rangs, six par six ; le canonnier, sans dire un mot, se plaça entre les deux files. L’aumônier, le crucifix en main, s’avança et se mit près de lui. — Marche, dit le sergent. — Le peloton se dirigea à pas lents vers l’avant ; les deux matelots, portant le suaire, suivaient.\par
 Un morne silence se fit sur la corvette. Un ouragan lointain soufflait.\par
Quelques instants après, une détonation éclata dans les ténèbres, une lueur passa, puis tout se tut, et l’on entendit le bruit que fait un corps en tombant dans la mer.\par
Le vieux passager, toujours adossé au grand mât, avait croisé les bras, et songeait.\par
Boisberthelot, dirigeant vers lui l’index de sa main gauche, dit bas à La Vieuville :\par
— La Vendée a une tête.
 \subsubsection[{VII. Qui met a la voile met a la loterie}]{VII \\
Qui met a la voile met a la loterie}\phantomsection
\label{p1l2c7}
\noindent Mais qu’allait devenir la corvette ?\par
Les nuages, qui toute la nuit s’étaient mêlés aux vagues, avaient fini par s’abaisser tellement qu’il n’y avait plus d’horizon et que toute la mer était comme sous un manteau. Rien que le brouillard. Situation toujours périlleuse, même pour un navire bien portant.\par
A la brume s’ajoutait la houle.\par
On avait mis le temps à profit ; on avait allégé la corvette en jetant à la mer tout ce qu’on avait pu déblayer du dégât fait par la caronade, les canons démontés, les affûts brisés, les membrures tordues ou déclouées, les pièces de bois ou de fer fracassées ; on avait ouvert les sabords, et l’on avait fait glisser sur des planches dans les vagues les cadavres et les débris humains enveloppés dans des prélarts.\par
La mer commençait à n’être plus tenable. Non que la tempête devînt précisément imminente ; il semblait au contraire qu’on entendît décroître l’ouragan qui  bruissait derrière l’horizon, et la rafale s’en allait au nord ; mais les lames restaient très hautes, ce qui indiquait un mauvais fond de mer, et, malade comme était la corvette, elle était peu résistante aux secousses, et les grandes vagues pouvaient lui être funestes.\par
Gacquoil était à la barre, pensif.\par
Faire bonne mine à mauvais jeu, c’est l’habitude des commandants de mer.\par
La Vieuville, qui était une nature d’homme gai dans les désastres, accosta Gacquoil.\par
— Eh bien, pilote, dit-il, l’ouragan rate. L’envie d’éternuer n’aboutit pas. Nous nous en tirerons. Nous aurons du vent. Voilà tout.\par
Gacquoil, sérieux, répondit :\par
— Qui a du vent a du flot.\par
Ni riant, ni triste, tel est le marin. La réponse avait un sens inquiétant. Pour un navire qui fait eau, avoir du flot, c’est s’emplir vite. Gacquoil avait souligné ce pronostic d’un vague froncement de sourcil. Peut-être, après la catastrophe du canon et du canonnier, La Vieuville avait-il dit, un peu trop tôt, des paroles presque joviales et légères. Il y a des choses qui portent malheur quand on est au large. La mer est secrète ; on ne sait jamais ce qu’elle a. Il faut prendre garde.\par
La Vieuville sentit le besoin de redevenir grave.\par
— Où sommes-nous, pilote ? demanda-t-il.\par
Le pilote répondit :\par
— Nous sommes dans la volonté de Dieu.\par
Un pilote est un maître ; il faut toujours le laisser  faire et il faut souvent le laisser dire. D’ailleurs cette espèce d’homme parle peu. La Vieuville s’éloigna.\par
La Vieuville avait fait une question au pilote, ce fut l’horizon qui répondit.\par
La mer se découvrit tout à coup.\par
Les brumes qui traînaient sur les vagues se déchirèrent, tout l’obscur bouleversement des flots s’étala à perte de vue dans un demi-jour crépusculaire, et voici ce qu’on vit.\par
Le ciel avait comme un couvercle de nuages ; mais les nuages ne touchaient plus la mer ; à l’est apparaissait une blancheur qui était le lever du jour, à l’ouest blêmissait une autre blancheur qui était le coucher de la lune. Ces deux blancheurs faisaient sur l’horizon, vis-à-vis l’une de l’autre, deux bandes étroites de lueur pâle entre la mer sombre et le ciel ténébreux.\par
Sur ces deux clartés se dessinaient, droites et immobiles, des silhouettes noires.\par
Au couchant, sur le ciel éclairé par la lune se découpaient trois hautes roches, debout comme des peulvens celtiques.\par
Au levant, sur l’horizon pâle du matin se dressaient huit voiles rangées en ordre et espacées d’une façon redoutable.\par
Les trois roches étaient un écueil ; les huit voiles étaient une escadre.\par
On avait derrière soi les Minquiers, un rocher qui avait mauvaise réputation, devant soi la croisière française. A l’ouest l’abîme, à l’est le carnage ; on était entre un naufrage et un combat.\par
 Pour faire face à l’écueil, la corvette avait une coque trouée, un gréement disloqué, une mâture ébranlée dans sa racine ; pour faire face à la bataille, elle avait une artillerie dont vingt et un canons sur trente étaient démontés, et dont les meilleurs canonniers étaient morts.\par
Le point du jour était très faible, et l’on avait un peu de nuit devant soi. Cette nuit pouvait même durer encore assez longtemps, étant surtout faite par les nuages, qui étaient hauts, épais et profonds, et avaient l’aspect solide d’une voûte.\par
Le vent qui avait fini par emporter les brumes d’en bas drossait la corvette sur les Minquiers.\par
Dans l’excès de fatigue et de délabrement où elle était, elle n’obéissait presque plus à la barre, elle roulait plutôt qu’elle ne voguait, et, souffletée par le flot, elle se laissait faire par lui.\par
Les Minquiers, écueil tragique, étaient plus âpres encore en ce temps-là qu’aujourd’hui. Plusieurs tours de cette citadelle de l’abîme ont été rasées par l’incessant dépècement que fait la mer ; la configuration des écueils change ; ce n’est pas en vain que les flots s’appellent les lames, chaque marée est un trait de scie. A cette époque, toucher les Minquiers, c’était périr.\par
Quant à la croisière, c’était cette escadre de Cancale, devenue depuis célèbre sous le commandement de ce capitaine Duchesne que Léquinio appelait « le Père Duchêne ».\par
La situation était critique. La corvette avait, sans  le savoir, pendant le déchaînement de la caronade, dévié et marché plutôt vers Granville que vers Saint-Malo. Quand même elle eût pu naviguer et faire voile, les Minquiers lui barraient le retour vers Jersey et la croisière lui barrait l’arrivée en France.\par
Du reste, de tempête point. Mais, comme l’avait dit le pilote, il y avait du flot. La mer, roulant sous un vent rude et sur un fond déchirant, était sauvage.\par
La mer ne dit jamais tout de suite ce qu’elle veut. Il y a de tout dans le gouffre, même de la chicane. On pourrait presque dire que la mer a une procédure ; elle avance et recule, elle propose et se dédit, elle ébauche une bourrasque et elle y renonce, elle promet l’abîme et ne le tient pas, elle menace le nord et frappe le sud. Toute la nuit la corvette \emph{la Claymore} avait eu le brouillard et craint la tourmente ; la mer venait de se démentir, mais d’une façon farouche ; elle avait esquissé la tempête et réalisé l’écueil. C’était toujours, sous une autre forme, le naufrage.\par
Et à la perte sur les brisants s’ajoutait l’extermination par le combat. Un ennemi complétait l’autre.\par
La Vieuville s’écria à travers son vaillant rire :\par
— Naufrage ici, bataille là. Des deux côtés nous avons le quine.
 \subsubsection[{VIII, 9 = 380}]{VIII \\
9 = 380}\phantomsection
\label{p1l2c8}
\noindent La corvette n’était presque plus qu’une épave.\par
Dans la blême clarté éparse, dans la noirceur des nuées, dans les mobilités confuses de l’horizon, dans les mystérieux froncements des vagues, il y avait une solennité sépulcrale. Excepté le vent soufflant d’un souffle hostile, tout se taisait. La catastrophe sortait du gouffre avec majesté. Elle ressemblait plutôt à une apparition qu’à une attaque. Rien ne bougeait dans les rochers, rien ne remuait dans les navires. C’était on ne sait quel colossal silence. Avait-on affaire à quelque chose de réel ? On eût dit un rêve passant sur la mer. Les légendes ont de ces visions ; la corvette était en quelque sorte entre l’écueil démon et la flotte fantôme.\par
Le comte du Boisberthelot donna à demi-voix des ordres à La Vieuville qui descendit dans la batterie, puis le capitaine saisit sa longue-vue et vint se placer à l’arrière à côté du pilote.\par
Tout l’effort de Gacquoil était de maintenir la corvette debout au flot ; car, prise de côté par le vent et par la mer, elle eût inévitablement chaviré.\par
— Pilote, dit le capitaine, où sommes-nous ?\par
— Sur les Minquiers.\par
 — De quel côté ?\par
— Du mauvais.\par
— Quel fond ?\par
— Roche criarde.\par
— Peut-on s’embosser ?\par
— On peut toujours mourir, dit le pilote.\par
Le capitaine dirigea sa lunette d’approche vers l’ouest et examina les Minquiers ; puis il la tourna vers l’est et considéra les voiles en vue.\par
Le pilote continua, comme se parlant à lui-même :\par
— C’est les Minquiers. Cela sert de reposoir à la mouette rieuse quand elle s’en va de Hollande et au grand goëland à manteau noir.\par
Cependant le capitaine avait compté les voiles.\par
Il y avait bien en effet huit navires correctement disposés et dressant sur l’eau leur profil de guerre. On apercevait au centre la haute stature d’un vaisseau à trois ponts.\par
Le capitaine questionna le pilote.\par
— Connaissez-vous ces voiles ?\par
— Certes ! répondit Gacquoil.\par
— Qu’est-ce ?\par
— C’est l’escadre.\par
— De France.\par
— Du diable.\par
Il y eut un silence. Le capitaine reprit :\par
— Toute la croisière est-elle là ?\par
— Pas toute.\par
En effet, le 2 avril, Valazé avait annoncé à la Convention que dix frégates et six vaisseaux de ligne  croisaient dans la Manche. Ce souvenir revint à l’esprit du capitaine.\par
— Au fait, dit-il, l’escadre est de seize bâtiments. Il n’y en a ici que huit.\par
— Le reste, dit Gacquoil, traîne par là-bas sur toute la côte, et espionne.\par
Le capitaine, tout en regardant à travers sa longue-vue, murmura :\par
— Un vaisseau à trois ponts, deux frégates de premier rang, cinq de deuxième rang.\par
— Mais moi aussi, grommela Gacquoil, je les ai espionnés.\par
— Bons bâtiments, dit le capitaine. J’ai un peu commandé tout cela.\par
— Moi, dit Gacquoil, je les ai vus de près. Je ne prends pas l’un pour l’autre. J’ai leur signalement dans la cervelle.\par
Le capitaine passa sa longue-vue au pilote.\par
— Pilote, distinguez-vous bien le bâtiment de haut bord ?\par
— Oui, mon commandant, c’est le vaisseau \emph{la Côte-d’Or}.\par
— Qu’ils ont débaptisé, dit le capitaine. C’était autrefois \emph{les États-de-Bourgogne}. Un navire neuf. Cent vingt-huit canons.\par
Il tira de sa poche un carnet et un crayon, et écrivit sur le carnet le chiffre 128.\par
Il poursuivit :\par
— Pilote, quelle est la première voile à bâbord ?\par
— C’est \emph{l’Expérimentée}.\par
 — Frégate de premier rang. Cinquante-deux canons. Elle était en armement à Brest il y a deux mois.\par
Le capitaine marqua sur son carnet le chiffre 52.\par
— Pilote, reprit-il, quelle est la deuxième voile à bâbord ?\par
— \emph{La Dryade}.\par
— Frégate de premier rang. Quarante canons de dix-huit. Elle a été dans l’Inde. Elle a une belle histoire militaire.\par
Et il écrivit au-dessous du chiffre 52 le chiffre 40 ; puis, relevant la tête :\par
— A tribord, maintenant.\par
— Mon commandant, ce sont toutes des frégates de second rang. Il y en a cinq.\par
— Quelle est la première à partir du vaisseau ?\par
— \emph{La Résolue}.\par
— Trente-deux pièces de dix-huit. Et la seconde ?\par
— \emph{Le Richemont.}\par
— Même force. Après ?\par
— \emph{L’Athée}\footnote{ \noindent \emph{Archives de la marine.} État de la flotte en mars 1793.
 }\par
— Drôle de nom pour aller en mer. Après ?\par
— \emph{La Calypso}.\par
— Après ?\par
— \emph{La Preneuse}.\par
— Cinq frégates de trente-deux chacune.\par
Le capitaine écrivit au-dessous des premiers chiffres, 160.\par
— Pilote, dit-il, vous les reconnaissez bien.\par
 — Et vous, répondit Gacquoil, vous les connaissez bien, mon commandant. Reconnaître est quelque chose, connaître est mieux.\par
Le capitaine avait l’œil fixé sur son carnet et additionnait entre ses dents.\par
— Cent vingt-huit, cinquante-deux, quarante, cent soixante.\par
En ce moment, La Vieuville remontait sur le pont.\par
— Chevalier, lui cria le capitaine, nous sommes en présence de trois cent quatrevingts pièces.\par
— Soit, dit La Vieuville.\par
— Vous revenez de l’inspection, La Vieuville ; combien décidément avons-nous de pièces en état de faire feu ?\par
— Neuf.\par
— Soit, dit à son tour Boisberthelot.\par
Il reprit la longue-vue des mains du pilote, et regarda l’horizon.\par
Les huit navires silencieux et noirs semblaient immobiles, mais ils grandissaient.\par
Ils se rapprochaient insensiblement.\par
La Vieuville fit le salut militaire.\par
— Commandant, dit La Vieuville, voici mon rapport. Je me défiais de cette corvette \emph{Claymore}. C’est toujours ennuyeux d’être embarqué brusquement sur un navire qui ne vous connaît pas ou qui ne vous aime pas. Navire anglais, traître aux Français. La chienne de caronade l’a prouvé. J’ai fait la visite. Bonnes ancres. Ce n’est pas du fer de loupe ; c’est forgé avec des barres soudées au martinet. Les cigales  des ancres sont solides. Câbles excellents, faciles à débiter, ayant la longueur d’ordonnance, cent vingt brasses. Force munitions. Six canonniers morts. Cent soixante-onze coups à tirer par pièce.\par
— Parce qu’il n’y a plus que neuf pièces, murmura le capitaine.\par
Boisberthelot braqua sa longue-vue sur l’horizon. La lente approche de l’escadre continuait.\par
Les caronades ont un avantage, trois hommes suffisent pour les manœuvrer ; mais elles ont un inconvénient, elles portent moins loin et tirent moins juste que les canons. Il fallait donc laisser arriver l’escadre à portée de caronade.\par
Le capitaine donna ses ordres à voix basse. Le silence se fit dans le navire. On ne sonna point le branle-bas, mais on l’exécuta. La corvette était aussi hors de combat contre les hommes que contre les flots. On tira tout le parti possible de ce reste d’un navire de guerre. On accumula près des drosses, sur le passavant, tout ce qu’il y avait d’aussières et de grelins de rechange pour raffermir au besoin la mâture. On mit en ordre le poste des blessés. Selon la mode navale d’alors, on bastingua le pont, ce qui est une garantie contre les balles, mais non contre les boulets. On apporta les passe-balles, bien qu’il fût un peu tard pour vérifier les calibres ; mais on n’avait pas prévu tant d’incidents. Chaque matelot reçut une giberne et mit dans sa ceinture une paire de pistolets et un poignard. On plia les branles ; on pointa l’artillerie ; on prépara la mousqueterie ; on disposa les haches et  les grappins ; on tint prêtes les soutes à gargousses et les soutes à boulets ; on ouvrit la soute aux poudres. Chaque homme prit son poste. Tout cela sans dire une parole et comme dans la chambre d’un mourant. Ce fut rapide et lugubre.\par
Puis on embossa la corvette. Elle avait six ancres comme une frégate. On les mouilla toutes les six ; l’ancre de veille à l’avant, l’ancre de toue à l’arrière, l’ancre de flot du côté du large, l’ancre de jusant du côté des brisants, l’ancre d’affourche à tribord, et la maîtresse-ancre à bâbord.\par
Les neuf caronades qui restaient vivantes furent mises en batterie toutes les neuf d’un seul côté, du côté de l’ennemi.\par
L’escadre, non moins silencieuse, avait, elle aussi, complété sa manœuvre. Les huit bâtiments formaient maintenant un demi-cercle dont les Minquiers faisaient la corde. \emph{La Claymore,} enfermée dans ce demi-cercle, et d’ailleurs garrottée par ses propres ancres, était adossée à l’écueil, c’est-à-dire au naufrage.\par
C’était comme une meute autour d’un sanglier, ne donnant pas de voix, mais montrant les dents.\par
Il semblait de part et d’autre qu’on s’attendait.\par
Les canonniers de \emph{la Claymore} étaient à leurs pièces.\par
Boisberthelot dit à La Vieuville :\par
— Je tiendrais à commencer le feu.\par
— Plaisir de coquette, dit La Vieuville.
 \subsubsection[{IX. Quelqu’un échappe}]{IX \\
Quelqu’un échappe}\phantomsection
\label{p1l2c9}
\noindent Le passager n’avait pas quitté le pont, il observait tout, impassible.\par
Boisberthelot s’approcha de lui.\par
— Monsieur, lui dit-il, les préparatifs sont faits. Nous voilà maintenant cramponnés à notre tombeau, nous ne lâcherons pas prise. Nous sommes prisonniers de l’escadre ou de l’écueil. Nous rendre à l’ennemi ou sombrer dans les brisants, nous n’avons pas d’autre choix. Il nous reste une ressource, mourir. Combattre vaut mieux que naufrager. J’aime mieux être mitraillé que noyé ; en fait de mort, je préfère le feu à l’eau. Mais mourir, c’est notre affaire à nous autres, ce n’est pas la vôtre, à vous. Vous êtes l’homme choisi par les princes, vous avez une grande mission, diriger la guerre de Vendée. Vous de moins, c’est peut-être la monarchie perdue ; vous devez donc vivre. Notre honneur à nous est de rester ici, le vôtre est d’en sortir. Vous allez, mon général, quitter le navire. Je vais vous donner un homme et un canot. Gagner la côte par un détour n’est pas impossible. Il n’est pas encore jour.  Les lames sont hautes, la mer est obscure, vous échapperez. Il y a des cas où fuir, c’est vaincre.\par
Le vieillard fit, de sa tête sévère, un grave signe d’acquiescement.\par
Le comte du Boisberthelot éleva la voix.\par
— Soldats et matelots ! cria-t-il.\par
Tous les mouvements s’arrêtèrent, et, de tous les points du navire, les visages se tournèrent vers le capitaine.\par
Il poursuivit :\par
— L’homme qui est parmi nous représente le roi. Il nous est confié, nous devons le conserver. Il est nécessaire au trône de France ; à défaut d’un prince, il sera, c’est du moins notre attente, le chef de la Vendée. C’est un grand officier de guerre. Il devait aborder en France avec nous, il faut qu’il y aborde sans nous. Sauver la tête, c’est tout sauver.\par
— Oui ! oui ! oui ! crièrent toutes les voix de l’équipage.\par
Le capitaine continua :\par
— Il va courir, lui aussi, de sérieux dangers. Atteindre la côte n’est pas aisé. Il faudrait que le canot fût grand pour affronter la haute mer, et il faut qu’il soit petit pour échapper à la croisière. Il s’agit d’aller atterrir à un point quelconque, qui soit sûr, et plutôt du côté de Fougères que du côté de Coutances. Il faut un matelot solide, bon rameur et bon nageur ; qui soit du pays et qui connaisse les passes. Il y a encore assez de nuit pour que le canot puisse s’éloigner de la corvette sans être aperçu. Et puis, il va y avoir de  la fumée qui achèvera de le cacher. Sa petitesse l’aidera à se tirer des bas-fonds. Où la panthère est prise, la belette échappe. Il n’y a pas d’issue pour nous, il y en a pour lui. Le canot s’éloignera à force de rames, les navires ennemis ne le verront pas ; et d’ailleurs, pendant ce temps-là, nous ici, nous allons les amuser. Est-ce dit ?\par
— Oui ! oui ! oui ! cria l’équipage.\par
— Il n’y a pas une minute à perdre, reprit le capitaine. Y a-t-il un homme de bonne volonté ?\par
Un matelot dans l’obscurité sortit des rangs, et dit :\par
— Moi.
 \subsubsection[{X. Échappe-t-il ?}]{X \\
Échappe-t-il ?}\phantomsection
\label{p1l2c10}
\noindent Quelques instants après, un de ces petits canots qu’on appelle you-yous et qui sont spécialement affectés au service des capitaines s’éloignait du navire. Dans ce canot il y avait deux hommes, le vieux passager qui était à l’arrière, et le matelot « de bonne volonté » qui était à l’avant. La nuit était encore très obscure. Le matelot, conformément aux indications du capitaine, ramait vigoureusement dans la direction des Minquiers. Aucune autre issue n’était d’ailleurs possible.\par
On avait jeté au fond du canot quelques provisions, un sac de biscuit, une longe de bœuf fumé et un baril d’eau.\par
Au moment où le you-you prit la mer, La Vieuville, goguenard devant le gouffre, se pencha par-dessus l’étambot du gouvernail de la corvette, et ricana cet adieu au canot :\par
— C’est bon pour s’échapper, et excellent pour se noyer.\par
— Monsieur, dit le pilote, ne rions plus.\par
L’écart se fit vite et il y eut promptement bonne distance entre la corvette et le canot. Le vent et  le flot étaient d’accord avec le rameur, et la petite barque fuyait rapidement, ondulant dans le crépuscule et cachée par les grands plis des vagues.\par
Il y avait sur la mer on ne sait quelle sombre attente.\par
Tout à coup, dans ce vaste et tumultueux silence de l’océan, il s’éleva une voix qui, grossie par le porte-voix comme par le masque d’airain de la tragédie antique, semblait presque surhumaine.\par
C’était le capitaine Boisberthelot qui prenait la parole.\par
— Marins du roi, cria-t-il, clouez le pavillon blanc au grand mât. Nous allons voir se lever notre dernier soleil.\par
Et un coup de canon partit de la corvette.\par
— Vive le roi ! cria l’équipage.\par
Alors on entendit au fond de l’horizon un autre cri, immense, lointain, confus, distinct pourtant :\par
— Vive la République !\par
Et un bruit pareil au bruit de trois cents foudres éclata dans les profondeurs de l’océan.\par
La lutte commençait.\par
La mer se couvrit de fumée et de feu.\par
Les jets d’écume que font les boulets en tombant dans l’eau piquèrent les vagues de tous les côtés.\par
\emph{La Claymore} se mit à cracher de la flamme sur les huit navires. En même temps toute l’escadre groupée en demi-lune autour de \emph{la Claymore} faisait feu de toutes ses batteries. L’horizon s’incendia. On eût dit un volcan qui sort de la mer. Le vent tordait cette  immense pourpre de la bataille où les navires apparaissaient et disparaissaient comme des spectres. Au premier plan, le squelette noir de la corvette se dessinait sur ce fond rouge.\par
On distinguait à la pointe du grand mât le pavillon fleurdelysé.\par
Les deux hommes qui étaient dans le canot se taisaient.\par
Le bas-fond triangulaire des Minquiers, sorte de trinacrie sous-marine, est plus vaste que l’île entière de Jersey ; la mer le couvre ; il a pour point culminant un plateau qui émerge des plus hautes marées et duquel se détachent au nord-est six puissants rochers rangés en droite ligne, qui font l’effet d’une grande muraille écroulée çà et là. Le détroit entre le plateau et les six écueils n’est praticable qu’aux barques d’un très faible tirant d’eau. Au delà de ce détroit on trouve le large.\par
Le matelot qui s’était chargé du sauvetage du canot engagea l’embarcation dans le détroit. De cette façon il mettait les Minquiers entre la bataille et le canot. Il nagea avec adresse dans l’étroit chenal, évitant les récifs à bâbord comme à tribord ; les rochers maintenant masquaient la bataille. La lueur de l’horizon et le fracas furieux de la canonnade commençaient à décroître, à cause de la distance qui augmentait ; mais, à la continuité des détonations, on pouvait comprendre que la corvette tenait bon et qu’elle voulait épuiser, jusqu’à la dernière, ses cent-quatrevingt-onze bordées.\par
Bientôt le canot se trouva dans une eau libre, hors  de l’écueil, hors de la bataille, hors de la portée des projectiles.\par
Peu à peu le modelé de la mer devenait moins sombre, les luisants brusquement noyés de noirceurs s’élargissaient, les écumes compliquées se brisaient en jets de lumière, des blancheurs flottaient sur les méplats des vagues. Le jour parut.\par
Le canot était hors de l’atteinte de l’ennemi ; mais le plus difficile restait à faire. Le canot était sauvé de la mitraille, mais non du naufrage. Il était en haute mer, coque imperceptible, sans pont, sans voile, sans mât, sans boussole, n’ayant de ressource que la rame, en présence de l’océan et de l’ouragan, atome à la merci des colosses.\par
Alors, dans cette immensité, dans cette solitude, levant sa face que blêmissait le matin, l’homme qui était à l’avant du canot regarda fixement l’homme qui était à l’arrière, et lui dit :\par
— Je suis le frère de celui que vous avez fait fusiller.
 \subsection[{Livre troisième. Halmalo}]{Livre troisième \\
Halmalo}\phantomsection
\label{p1l3}
\subsubsection[{I. La parole, c’est le verbe}]{I \\
La parole, c’est le verbe}\phantomsection
\label{p1l3c1}
\noindent Le vieillard redressa lentement la tête.\par
L’homme qui lui parlait avait environ trente ans. Il avait sur le front le hâle de la mer ; ses yeux étaient étranges ; c’était le regard sagace du matelot dans la prunelle candide du paysan. Il tenait puissamment les rames dans ses deux poings. Il avait l’air doux.\par
On voyait à sa ceinture un poignard, deux pistolets et un rosaire.\par
— Qui êtes-vous ? dit le vieillard.\par
— Je viens de vous le dire.\par
— Qu’est-ce que vous me voulez ?\par
L’homme quitta les avirons, croisa les bras et répondit :\par
 — Vous tuer.\par
— Comme vous voudrez, dit le vieillard.\par
L’homme haussa la voix.\par
— Préparez-vous.\par
— A quoi ?\par
— A mourir.\par
— Pourquoi ? demanda le vieillard.\par
Il y eut un silence. L’homme sembla un moment comme interdit de la question. Il reprit :\par
— Je dis que je veux vous tuer.\par
— Et je vous demande pourquoi.\par
Un éclair passa dans les yeux du matelot.\par
— Parce que vous avez tué mon frère.\par
Le vieillard repartit avec calme :\par
— J’ai commencé par lui sauver la vie.\par
— C’est vrai. Vous l’avez sauvé d’abord et tué ensuite.\par
— Ce n’est pas moi qui l’ai tué.\par
— Qui donc l’a tué ?\par
— Sa faute.\par
Le matelot, béant, regarda le vieillard ; puis ses sourcils reprirent leur froncement farouche.\par
— Comment vous appelez-vous ? dit le vieillard.\par
— Je m’appelle Halmalo, mais vous n’avez pas besoin de savoir mon nom pour être tué par moi.\par
En ce moment le soleil se leva. Un rayon frappa le matelot en plein visage et éclaira vivement cette figure sauvage. Le vieillard le considérait attentivement.\par
La canonnade, qui se prolongeait toujours, avait maintenant des interruptions et des saccades d’agonie.  Une vaste fumée s’affaissait sur l’horizon. Le canot, que ne maniait plus le rameur, allait à la dérive.\par
Le matelot saisit de sa main droite un des pistolets de sa ceinture et de sa main gauche son chapelet.\par
Le vieillard se dressa debout.\par
— Tu crois en Dieu ? dit-il.\par
— Notre Père qui est au ciel, répondit le matelot.\par
Et il fit le signe de la croix.\par
— As-tu ta mère ?\par
— Oui.\par
Il fit un deuxième signe de croix. Puis il reprit :\par
— C’est dit. Je vous donne une minute, monseigneur.\par
Et il arma le pistolet.\par
— Pourquoi m’appelles-tu monseigneur ?\par
— Parce que vous êtes un seigneur. Cela se voit.\par
— As-tu un seigneur, toi ?\par
— Oui. Et un grand. Est-ce qu’on vit sans seigneur ?\par
— Où est-il ?\par
— Je ne sais pas. Il a quitté le pays. Il s’appelle monsieur le marquis de Lantenac, vicomte de Fontenay, prince en Bretagne ; il est le seigneur des Sept-Forêts. Je ne l’ai jamais vu, ce qui ne l’empêche pas d’être mon maître.\par
— Et si tu le voyais, lui obéirais-tu ?\par
— Certes. Je serais donc un païen, si je ne lui obéissais pas ! on doit obéissance à Dieu, et puis au roi qui est comme Dieu, et puis au seigneur qui est comme le roi. Mais ce n’est pas tout ça, vous avez tué mon frère, il faut que je vous tue.\par
 Le vieillard répondit :\par
— D’abord, j’ai tué ton frère, j’ai bien fait.\par
Le matelot crispa son poing sur son pistolet.\par
— Allons, dit-il.\par
— Soit, dit le vieillard.\par
Et, tranquille, il ajouta :\par
— Où est le prêtre ?\par
Le matelot le regarda.\par
— Le prêtre ?\par
— Oui, le prêtre. J’ai donné un prêtre à ton frère. Tu me dois un prêtre.\par
— Je n’en ai pas, dit le matelot.\par
Et il continua :\par
— Est-ce qu’on a des prêtres en pleine mer ?\par
On entendait les détonations convulsives du combat de plus en plus lointain.\par
— Ceux qui meurent là-bas ont le leur, dit le vieillard.\par
— C’est vrai, murmura le matelot. Ils ont monsieur l’aumônier.\par
Le vieillard poursuivit :\par
— Tu perds mon âme, ce qui est grave.\par
Le matelot baissa la tête, pensif.\par
— Et en perdant mon âme, reprit le vieillard, tu perds la tienne. Écoute. J’ai pitié de toi. Tu feras ce que tu voudras. Moi, j’ai fait mon devoir tout à l’heure, d’abord en sauvant la vie à ton frère et ensuite en la lui ôtant, et je fais mon devoir à présent en tâchant de sauver ton âme. Réfléchis. Cela te regarde. Entends-tu les coups de canon dans ce moment-ci ? Il  y a là des hommes qui périssent, il y a là des désespérés qui agonisent, il y a là des maris qui ne reverront plus leur femme, des pères qui ne reverront plus leur enfant, des frères qui, comme toi, ne reverront plus leur frère. Et par la faute de qui ? par la faute de ton frère à toi. Tu crois en Dieu, n’est-ce pas ? Eh bien, tu sais que Dieu souffre en ce moment ; Dieu souffre dans son fils très chrétien le roi de France qui est enfant comme l’enfant Jésus et qui est en prison dans la tour du Temple ; Dieu souffre dans son église de Bretagne ; Dieu souffre dans ses cathédrales insultées, dans ses évangiles déchirés, dans ses maisons de prière violées ; Dieu souffre dans ses prêtres assassinés. Qu’est-ce que nous venions faire, nous, dans ce navire qui périt en ce moment ? Nous venions secourir Dieu. Si ton frère avait été un bon serviteur, s’il avait fidèlement fait son office d’homme sage et utile, le malheur de la caronade ne serait pas arrivé, la corvette n’eût pas été désemparée, elle n’eût pas manqué sa route, elle ne fût pas tombée dans cette flotte de perdition, et nous débarquerions à cette heure en France, tous, en vaillants hommes de guerre et de mer que nous sommes, sabre au poing, drapeau blanc déployé, nombreux, contents, joyeux, et nous viendrions aider les braves paysans de Vendée à sauver la France, à sauver le roi, à sauver Dieu. Voilà ce que nous venions faire, voilà ce que nous ferions. Voilà ce que, moi, le seul qui reste, je viens faire. Mais tu t’y opposes. Dans cette lutte des impies contre les prêtres, dans cette lutte des  régicides contre le roi, dans cette lutte de Satan contre Dieu, tu es pour Satan. Ton frère a été le premier auxiliaire du démon, tu es le second. Il a commencé, tu achèves. Tu es pour les régicides contre le trône, tu es pour les impies contre l’église. Tu ôtes à Dieu sa dernière ressource. Parce que je ne serai point là, moi qui représente le roi, les hameaux vont continuer de brûler, les familles de pleurer, les prêtres de saigner, la Bretagne de souffrir, et le roi d’être en prison, et Jésus-Christ d’être en détresse. Et qui aura fait cela ? Toi. Va, c’est ton affaire. Je comptais sur toi pour tout le contraire. Je me suis trompé. Ah oui, c’est vrai, tu as raison, j’ai tué ton frère. Ton frère avait été courageux, je l’ai récompensé ; il avait été coupable, je l’ai puni. Il avait manqué à son devoir, je n’ai pas manqué au mien. Ce que j’ai fait, je le ferais encore. Et, je le jure par la grande sainte Anne d’Auray qui nous regarde, en pareil cas, de même que j’ai fait fusiller ton frère, je ferais fusiller mon fils. Maintenant, tu es le maître. Oui, je te plains. Tu as menti à ton capitaine. Toi, chrétien, tu es sans foi ; toi, Breton, tu es sans honneur ; j’ai été confié à ta loyauté et accepté par ta trahison ; tu donnes ma mort à ceux à qui tu as promis ma vie. Sais-tu qui tu perds ici ? C’est toi. Tu prends ma vie au roi et tu donnes ton éternité au démon. Va, commets ton crime, c’est bien. Tu fais bon marché de ta part de paradis. Grâce à toi, le diable vaincra, grâce à toi, les églises tomberont, grâce à toi, les païens continueront de fondre les cloches et d’en faire des canons ; on mitraillera les  hommes avec ce qui sauvait les âmes. En ce moment où je parle, la cloche qui a sonné ton baptême tue peut-être ta mère. Va, aide le démon. Ne t’arrête pas. Oui, j’ai condamné ton frère, mais, sache cela, je suis un instrument de Dieu. Ah ! tu juges les moyens de Dieu ! tu vas donc te mettre à juger la foudre qui est dans le ciel ? Malheureux, tu seras jugé par elle. Prends garde à ce que tu vas faire. Sais-tu seulement si je suis en état de grâce ? Non. Va tout de même. Fais ce que tu voudras. Tu es libre de me jeter en enfer et de t’y jeter avec moi. Nos deux damnations sont dans ta main. Le responsable devant Dieu, ce sera toi. Nous sommes seuls et face à face dans l’abîme. Continue, termine, achève. Je suis vieux et tu es jeune, je suis sans armes et tu es armé ; tue-moi.\par
Pendant que le vieillard, debout, d’une voix plus haute que le bruit de la mer, disait ces paroles, les ondulations de la vague le faisaient apparaître tantôt dans l’ombre, tantôt dans la lumière ; le matelot était devenu livide ; de grosses gouttes de sueur lui tombaient du front ; il tremblait comme la feuille ; par moments il baisait son rosaire ; quand le vieillard eut fini, il jeta son pistolet et tomba à genoux.\par
— Grâce, monseigneur ! pardonnez-moi ! cria-t-il ; vous parlez comme le bon Dieu. J’ai tort. Mon frère a eu tort. Je ferai tout pour réparer son crime. Disposez de moi. Ordonnez. J’obéirai.\par
— Je te fais grâce, dit le vieillard.
 \subsubsection[{II. Mémoire de paysan vaut science de capitaine}]{II \\
Mémoire de paysan vaut science de capitaine}\phantomsection
\label{p1l3c2}
\noindent Les provisions qui étaient dans le canot ne furent pas inutiles.\par
Les deux fugitifs, obligés à de longs détours, mirent trente-six heures à atteindre la côte. Ils passèrent une nuit en mer ; mais la nuit fut belle, avec trop de lune cependant pour des gens qui cherchaient à se dérober.\par
Ils durent d’abord s’éloigner de France et gagner le large vers Jersey.\par
Ils entendirent la suprême canonnade de la corvette foudroyée, comme on entend le dernier rugissement du lion que les chasseurs tuent dans les bois. Puis le silence se fit sur la mer.\par
Cette corvette \emph{la Claymore} mourut de la même façon que \emph{le Vengeur ;} mais la gloire l’a ignoré. On n’est pas héros contre son pays.\par
Halmalo était un marin surprenant. Il fit des miracles de dextérité et d’intelligence ; cette improvisation d’un itinéraire à travers les écueils, les vagues et le guet de l’ennemi fut un chef-d’œuvre. Le vent avait décru et la mer était devenue maniable.\par
 Halmalo évita les Caux des Minquiers, contourna la Chaussée-aux-Bœufs, s’y abrita, afin de prendre quelques heures de repos dans la petite crique qui s’y fait au nord à mer basse, et, redescendant au sud, trouva moyen de passer entre Granville et les îles Chausey sans être aperçu ni de la vigie de Chausey ni de la vigie de Granville. Il s’engagea dans la baie de Saint-Michel, ce qui était hardi à cause du voisinage de Cancale, lieu d’ancrage de la croisière.\par
Le soir du second jour, environ une heure avant le coucher du soleil, il laissa derrière lui le mont Saint-Michel, et vint atterrir à une grève qui est toujours déserte, parce qu’elle est dangereuse ; on s’y enlise.\par
Heureusement la marée était haute.\par
Halmalo poussa l’embarcation le plus avant qu’il put, tâta le sable, le trouva solide, y échoua le canot et sauta à terre.\par
Le vieillard après lui enjamba le bord et examina l’horizon.\par
— Monseigneur, dit Halmalo, nous sommes ici à l’embouchure du Couesnon. Voilà Beauvoir à tribord et Huisnes à bâbord. Le clocher devant nous, c’est Ardevon.\par
Le vieillard se pencha dans le canot, y prit un biscuit qu’il mit dans sa poche, et dit à Halmalo :\par
— Prends le reste.\par
Halmalo mit dans le sac ce qui restait de viande avec ce qui restait de biscuit, et chargea le sac sur son épaule. Cela fait, il dit :\par
 — Monseigneur, faut-il vous conduire ou vous suivre ?\par
— Ni l’un ni l’autre.\par
Halmalo stupéfait regarda le vieillard.\par
Le vieillard continua :\par
— Halmalo, nous allons nous séparer. Être deux ne vaut rien. Il faut être mille, ou seul.\par
Il s’interrompit et tira d’une de ses poches un nœud de soie verte, assez pareil à une cocarde, au centre duquel était brodée une fleur de lys en or. Il reprit :\par
— Sais-tu lire ?\par
— Non.\par
— C’est bien. Un homme qui lit, ça gêne. As-tu bonne mémoire ?\par
— Oui.\par
— C’est bien. Écoute, Halmalo. Tu vas prendre à droite et moi à gauche. J’irai du côté de Fougères, toi du côté de Bazouges. Garde ton sac qui te donne l’air d’un paysan. Cache tes armes. Coupe-toi un bâton dans les haies. Rampe dans les seigles qui sont hauts. Glisse-toi derrière les clôtures. Enjambe les échaliers pour aller à travers champs. Laisse à distance les passants. Évite les chemins et les ponts. N’entre pas à Pontorson. Ah ! tu auras à traverser le Couesnon. Comment le passeras-tu ?\par
— A la nage.\par
— C’est bien. Et puis il y a un gué. Sais-tu où il est ?\par
— Entre Ancey et Vieux-Viel.\par
— C’est bien. Tu es vraiment du pays.\par
 — Mais la nuit vient. Où monseigneur couchera-t-il ?\par
— Je me charge de moi. Et toi, où coucheras-tu ?\par
— Il y a des émousses. Avant d’être matelot, j’ai été paysan.\par
— Jette ton chapeau de marin qui te trahirait. Tu trouveras bien quelque part une carapousse.\par
— Oh ! un tapabor, cela se trouve partout. Le premier pêcheur venu me vendra le sien.\par
— C’est bien. Maintenant, écoute. Tu connais les bois ?\par
— Tous.\par
— De tout le pays ?\par
— Depuis Noirmoutier jusqu’à Laval.\par
— Connais-tu aussi les noms ?\par
— Je connais les bois, je connais les noms, je connais tout.\par
— Tu n’oublieras rien ?\par
— Rien.\par
— C’est bien. A présent, attention. Combien peux-tu faire de lieues par jour ?\par
— Dix, quinze, dix-huit. Vingt, s’il le faut.\par
— Il le faudra. Ne perds pas un mot de ce que je vais te dire. Tu iras au bois de Saint-Aubin.\par
— Près de Lamballe ?\par
— Oui. Sur la lisière du ravin qui est entre Saint-Rieul et Plédéliac il y a un gros châtaignier. Tu t’arrêteras là. Tu ne verras personne.\par
— Ce qui n’empêche pas qu’il y aura quelqu’un. Je sais.\par
 — Tu feras l’appel. Sais-tu faire l’appel ?\par
Halmalo enfla ses joues, se tourna du côté de la mer, et l’on entendit le hou-hou de la chouette.\par
On eût dit que cela venait des profondeurs nocturnes. C’était ressemblant et sinistre.\par
— Bien, dit le vieillard. Tu en es.\par
Il tendit à Halmalo le nœud de soie verte.\par
— Voici mon nœud de commandement. Prends-le. Il importe que personne encore ne sache mon nom. Mais ce nœud suffit. La fleur de lys a été brodée par Madame Royale dans la prison du Temple.\par
Halmalo mit un genou en terre. Il reçut avec un tremblement le nœud fleurdelysé, et en approcha ses lèvres ; puis s’arrêtant comme effrayé de ce baiser :\par
— Le puis-je ? demanda-t-il.\par
— Oui, puisque tu baises le crucifix.\par
Halmalo baisa la fleur de lys.\par
— Relève-toi, dit le vieillard.\par
Halmalo se releva et mit le nœud dans sa poitrine.\par
Le vieillard poursuivit :\par
— Écoute bien ceci. Voici l’ordre : \emph{Insurgez-vous. Pas de quartier}. Donc, sur la lisière du bois de Saint-Aubin tu feras l’appel. Tu le feras trois fois. A la troisième fois tu verras un homme sortir de terre.\par
— D’un trou sous les arbres. Je sais.\par
— Cet homme, c’est Planchenault, qu’on appelle aussi Cœur-de-Roi. Tu lui montreras ce nœud. Il comprendra. Tu iras ensuite, par des chemins que tu inventeras, au bois d’Astillé ; tu y trouveras un homme cagneux qui est surnommé Mousqueton, et  qui ne fait miséricorde à personne. Tu lui diras que je l’aime, et qu’il mette en branle ses paroisses. Tu iras ensuite au bois de Couesbon qui est à une lieue de Ploërmel. Tu feras l’appel de la chouette ; un homme sortira d’un trou ; c’est M. Thuault, sénéchal de Ploërmel, qui a été de ce qu’on appelle l’assemblée constituante, mais du bon côté. Tu lui diras d’armer le château de Couesbon, qui est au marquis de Guer, émigré. Ravins, petits bois, terrain inégal, bon endroit. M. Thuault est un homme droit et d’esprit. Tu iras ensuite à Saint-Ouen-les-Toits, et tu parleras à Jean Chouan, qui est à mes yeux le vrai chef. Tu iras ensuite au bois de Ville-Anglose, tu y verras Guitter, qu’on appelle Saint-Martin, tu lui diras d’avoir l’œil sur un certain Courmesnil, qui est gendre du vieux Goupil de Préfeln et qui mène la jacobinière d’Argentan. Retiens bien tout. Je n’écris rien parce qu’il ne faut rien écrire. La Rouarie a écrit une liste ; cela a tout perdu. Tu iras ensuite au bois de Rougefeu où est Miélette qui saute par-dessus les ravins en s’arc-boutant sur une longue perche.\par
— Cela s’appelle une ferte.\par
— Sais-tu t’en servir ?\par
— Je ne serais donc pas Breton et je ne serais donc pas paysan ? La ferte, c’est notre amie. Elle agrandit nos bras et allonge nos jambes.\par
— C’est-à-dire qu’elle rapetisse l’ennemi et raccourcit le chemin. Bon engin.\par
— Une fois, avec ma ferte, j’ai tenu tête à trois gabelous qui avaient des sabres.\par
 — Quand ça ?\par
— Il y a dix ans.\par
— Sous le roi ?\par
— Mais oui.\par
— Tu t’es donc battu sous le roi ?\par
— Mais oui.\par
— Contre qui ?\par
— Ma foi, je ne sais pas. J’étais faux-saulnier.\par
— C’est bien.\par
— On appelait cela se battre contre les gabelles. Les gabelles, est-ce que c’est la même chose que le roi ?\par
— Oui. Non. Mais il n’est pas nécessaire que tu comprennes cela.\par
— Je demande pardon à monseigneur d’avoir fait une question à monseigneur.\par
— Continuons. Connais-tu la Tourgue ?\par
— Si je connais la Tourgue ? j’en suis.\par
— Comment ?\par
— Oui, puisque je suis de Parigné.\par
— En effet, la Tourgue est voisine de Parigné.\par
— Si je connais la Tourgue ! le gros château rond qui est le château de famille de mes seigneurs ! Il y a une grosse porte de fer qui sépare le bâtiment neuf du bâtiment vieux et qu’on n’enfoncerait pas avec du canon. C’est dans le bâtiment neuf qu’est le fameux livre sur saint Barthélemy qu’on venait voir par curiosité. Il y a des grenouilles dans l’herbe. J’ai joué tout petit avec ces grenouilles-là. Et la passe souterraine ! je la connais. Il n’y a peut-être plus que moi qui la connaisse.\par
 — Quelle passe souterraine ? Je ne sais pas ce que tu veux dire.\par
— C’était pour autrefois, dans les temps, quand la Tourgue était assiégée. Les gens du dedans pouvaient se sauver dehors en passant par un passage sous terre qui va aboutir à la forêt.\par
— En effet, il y a un passage souterrain de ce genre au château de la Jupellière, et au château de la Hunaudaye, et à la tour de Campéon ; mais il n’y a rien de pareil à la Tourgue.\par
— Si fait, monseigneur. Je ne connais pas ces passages-là dont monseigneur parle. Je ne connais que celui de la Tourgue, parce que je suis du pays. Et encore, il n’y a guère que moi qui sache cette passe-là. On n’en parlait pas. C’était défendu, parce que ce passage avait servi du temps des guerres de M. de Rohan. Mon père savait le secret et il me l’a montré. Je connais le secret pour entrer et le secret pour sortir. Si je suis dans la forêt, je puis aller dans la tour, et si je suis dans la tour, je puis aller dans la forêt. Sans qu’on me voie. Et quand les ennemis entrent, il n’y a plus personne. Voilà ce que c’est que la Tourgue. Ah ! je la connais.\par
Le vieillard demeura un moment silencieux.\par
— Tu te trompes évidemment ; s’il y avait un tel secret, je le saurais.\par
— Monseigneur, j’en suis sûr. Il y a une pierre qui tourne.\par
— Ah bon ! Vous autres paysans, vous croyez aux pierres qui tournent, aux pierres qui chantent, aux  pierres qui vont boire la nuit au ruisseau d’à côté. Tas de contes.\par
— Mais puisque je l’ai fait tourner, la pierre...\par
— Comme d’autres l’ont entendue chanter. Camarade, la Tourgue est une bastille sûre et forte, facile à défendre ; mais celui qui compterait sur une issue souterraine pour s’en tirer serait naïf.\par
— Mais, monseigneur...\par
Le vieillard haussa les épaules.\par
— Ne perdons pas de temps. Parlons de nos affaires.\par
Ce ton péremptoire coupa court à l’insistance de Halmalo.\par
Le vieillard reprit :\par
— Poursuivons. Écoute. De Rougefeu tu iras au bois de Montchevrier, où est Bénédicité, qui est le chef des Douze. C’est encore un bon. Il dit son \emph{Benedicite} pendant qu’il fait arquebuser les gens. En guerre, pas de sensiblerie. De Montchevrier, tu iras...\par
Il s’interrompit.\par
— J’oubliais l’argent.\par
Il prit dans sa poche et mit dans la main de Halmalo une bourse et un portefeuille.\par
— Voilà dans ce portefeuille trente mille francs en assignats, quelque chose comme trois livres dix sous ; il faut dire que les assignats sont faux, mais les vrais valent juste autant ; et voici dans cette bourse, attention, cent louis en or. Je te donne tout ce que j’ai. Je n’ai plus besoin de rien ici. D’ailleurs il vaut mieux qu’on ne puisse pas trouver d’argent sur moi. Je reprends. De Montchevrier, tu iras à Antrain,  où tu verras M. de Frotté ; d’Antrain, à la Jupellière, où tu verras M. de Rochecotte ; de la Jupellière, à Noirieux, où tu verras l’abbé Baudouin. Te rappelleras-tu tout cela ?\par
— Comme mon \emph{Pater}.\par
— Tu verras M. Dubois-Guy à Saint-Brice-en-Cogle, M. de Turpin à Morannes, qui est un bourg fortifié, et le prince de Talmont à Château-Gonthier.\par
— Est-ce qu’un prince me parlera ?\par
— Puisque je te parle.\par
Halmalo ôta son chapeau.\par
— Tout le monde te recevra bien en voyant cette fleur de lys de Madame. N’oublie pas qu’il faut que tu ailles dans des endroits où il y a des montagnards et des patauds. Tu te déguiseras. C’est facile. Ces républicains sont si bêtes, qu’avec un habit bleu, un chapeau à trois cornes et une cocarde tricolore on passe partout. Il n’y a plus de régiments, il n’y a plus d’uniformes, les corps n’ont pas de numéros ; chacun met la guenille qu’il veut. Tu iras à Saint-Mhervé. Tu y verras Gaulier, dit Grand-Pierre. Tu iras au cantonnement de Parné où sont les hommes aux visages noircis. Ils mettent du gravier dans leurs fusils et double charge de poudre pour faire plus de bruit ; ils font bien. Mais surtout dis-leur de tuer, de tuer, de tuer. Tu iras au camp de la Vache-Noire qui est sur une hauteur au milieu du bois de la Charnie, puis au camp de l’Avoine, puis au camp Vert, puis au camp des Fourmis. Tu iras au Grand-Bordage, qu’on appelle aussi le Haut-des-Prés, et qui est habité par une veuve  dont Treton, dit l’Anglais, a épousé la fille. Le Grand-Bordage est dans la paroisse de Quélaines. Tu visiteras Épineux-le-Chevreuil, Sillé-le-Guillaume, Parannes, et tous les hommes qui sont dans tous les bois. Tu auras des amis, et tu les enverras sur la lisière du Haut et du Bas Maine ; tu verras Jean Treton dans la paroisse de Vaisges, Sans-Regret au Bignon, Chambord à Bonchamps, les frères Corbin à Maisoncelles, et le Petit-Sans-Peur à Saint-Jean-sur-Erve. C’est le même qui s’appelle Bourdoiseau. Tout cela fait, et le mot d’ordre, \emph{Insurgez-vous, Pas de quartier}, donné partout, tu joindras la grande armée, l’armée catholique et royale, où elle sera. Tu verras MM. d’Elbée, de Lescure, de La Rochejaquelein, ceux des chefs qui vivront alors. Tu leur montreras mon nœud de commandement. Ils savent ce que c’est. Tu n’es qu’un matelot, mais Cathelineau n’est qu’un charretier. Tu leur diras de ma part ceci : Il est temps de faire les deux guerres ensemble ; la grande et la petite. La grande fait plus de tapage, la petite plus de besogne. La Vendée est bonne, la Chouannerie est pire ; et en guerre civile, c’est la pire qui est la meilleure. La bonté d’une guerre se juge à la quantité de mal qu’elle fait.\par
Il s’interrompit.\par
— Halmalo, je te dis tout cela. Tu ne comprends pas les mots, mais tu comprends les choses. J’ai pris confiance en toi en te voyant manœuvrer le canot ; tu ne sais pas la géométrie et tu fais des mouvements de mer surprenants ; qui sait mener une barque peut  piloter une insurrection ; à la façon dont tu as manié l’intrigue de la mer, j’affirme que tu te tireras bien de toutes mes commissions. Je reprends. Tu diras donc ceci aux chefs, à peu près, comme tu pourras, mais ce sera bien : J’aime mieux la guerre des forêts que la guerre des plaines ; je ne tiens pas à aligner cent mille paysans sous la mitraille des soldats bleus et sous l’artillerie de monsieur Carnot ; avant un mois je veux avoir cinq cent mille tueurs, embusqués dans les bois. L’armée républicaine est mon gibier. Braconner, c’est guerroyer. Je suis le stratège des broussailles. Bon, voilà encore un mot que tu ne saisiras pas ; c’est égal ; tu saisiras ceci : Pas de quartier ! et des embuscades partout ! Je veux faire plus de chouannerie que de Vendée. Tu ajouteras que les Anglais sont avec nous. Prenons la république entre deux feux. L’Europe nous aide. Finissons-en avec la révolution. Les rois lui font la guerre des royaumes, faisons-lui la guerre des paroisses. Tu diras cela. As-tu compris ?\par
— Oui. Il faut tout mettre à feu et à sang.\par
— C’est ça.\par
— Pas de quartier.\par
— A personne. C’est ça.\par
— J’irai partout.\par
— Et prends garde. Car dans ce pays-ci on est facilement un homme mort.\par
— La mort, cela ne me regarde point. Qui fait son premier pas use peut-être ses derniers souliers.\par
— Tu es un brave.\par
— Et si l’on me demande le nom de monseigneur ?\par
 — On ne doit pas le savoir encore. Tu diras que tu ne le sais pas, et ce sera la vérité.\par
— Où reverrai-je monseigneur ?\par
— Où je serai.\par
— Comment le saurai-je ?\par
— Parce que tout le monde le saura. Avant huit jours on parlera de moi, je ferai des exemples, je vengerai le roi et la religion, et tu reconnaîtras bien que c’est de moi qu’on parle.\par
— J’entends.\par
— N’oublie rien.\par
— Soyez tranquille.\par
— Pars maintenant. Que Dieu te conduise. Va.\par
— Je ferai tout ce que vous m’avez dit. J’irai. Je parlerai. J’obéirai. Je commanderai.\par
— Bien.\par
— Et si je réussis...\par
— Je te ferai chevalier de Saint-Louis.\par
— Comme mon frère. Et si je ne réussis pas, vous me ferez fusiller.\par
— Comme ton frère.\par
— C’est dit, monseigneur.\par
Le vieillard baissa la tête et sembla tomber dans une sévère rêverie. Quand il releva les yeux, il était seul. Halmalo n’était plus qu’un point noir s’enfonçant dans l’horizon.\par
Le soleil venait de se coucher.\par
Les goëlands et les mouettes à capuchon rentraient ; la mer, c’est dehors.\par
On sentait dans l’espace cette espèce d’inquiétude  qui précède la nuit ; les rainettes coassaient, les jaquets s’envolaient des flaques d’eau en sifflant, les mauves, les freux, les carabins, les grolles, faisaient leur vacarme du soir ; les oiseaux de rivage s’appelaient ; mais pas un bruit humain. La solitude était profonde. Pas une voile dans la baie, pas un paysan dans la campagne. A perte de vue l’étendue déserte. Les grands chardons des sables frissonnaient. Le ciel blanc du crépuscule jetait sur la grève une vaste clarté livide. Au loin les étangs dans la plaine sombre ressemblaient à des plaques d’étain posées à plat sur le sol. Le vent soufflait du large.\par
  \subsection[{Livre quatrième. Tellmarch}]{Livre quatrième \\
Tellmarch}\phantomsection
\label{p1l4}
\subsubsection[{I. Le haut de la dune}]{I \\
Le haut de la dune}\phantomsection
\label{p1l4c1}
\noindent Le vieillard laissa disparaître Halmalo, puis serra son manteau de mer autour de lui, et se mit en marche. Il cheminait à pas lents, pensif. Il se dirigeait vers Huisnes, pendant que Halmalo s’en allait vers Beauvoir.\par
Derrière lui se dressait, énorme triangle noir, avec sa tiare de cathédrale et sa cuirasse de forteresse, avec ses deux grosses tours du levant, l’une ronde, l’autre carrée, qui aident la montagne à porter le poids de l’église et du village, le mont Saint-Michel, qui est à l’océan ce que Chéops est au désert.\par
Les sables mouvants de la baie du mont Saint-Michel déplacent insensiblement leurs dunes. Il y avait à cette époque entre Huisnes et Ardevon une dune très haute, effacée aujourd’hui. Cette dune, qu’un coup d’équinoxe a nivelée, avait cette rareté d’être ancienne et de porter à son sommet une pierre milliaire érigée au douzième siècle en commémoration du concile  tenu à Avranches contre les assassins de saint Thomas de Cantorbéry. Du haut de cette dune on découvrait tout le pays, et l’on pouvait s’orienter.\par
Le vieillard marcha vers cette dune et y monta.\par
Quand il fut sur le sommet, il s’adossa à la pierre milliaire, s’assit sur une des quatre bornes qui en marquaient les angles, et se mit à examiner l’espèce de carte de géographie qu’il avait sous les pieds. Il semblait chercher une route dans un pays d’ailleurs connu. Dans ce vaste paysage, trouble à cause du crépuscule, il n’y avait de précis que l’horizon, noir sur le ciel blanc.\par
On y apercevait les groupes de toits de onze bourgs et villages ; on distinguait à plusieurs lieues de distance tous les clochers de la côte, qui sont très hauts, afin de servir au besoin de points de repère aux gens qui sont en mer.\par
Au bout de quelques instants, le vieillard sembla avoir trouvé dans ce clair-obscur ce qu’il cherchait ; son regard s’arrêta sur un enclos d’arbres, de murs et de toitures, à peu près visible au milieu de la plaine et des bois, et qui était une métairie ; il eut ce hochement de tête satisfait d’un homme qui se dit mentalement : C’est là ; et il se mit à tracer avec son doigt dans l’espace l’ébauche d’un itinéraire à travers les haies et les cultures. De temps en temps il examinait un objet informe et peu distinct, qui s’agitait au-dessus du toit principal de la métairie, et il semblait se demander : Qu’est-ce que c’est ? cela était incolore et confus à cause de l’heure ; ce n’était pas  une girouette puisque cela flottait, et il n’y avait aucune raison pour que ce fût un drapeau.\par
Il était las, il restait volontiers assis sur cette borne où il était, et il se laissait aller à cette sorte de vague oubli que donne aux hommes fatigués la première minute de repos.\par
Il y a une heure du jour qu’on pourrait appeler l’absence de bruit, c’est l’heure sereine, l’heure du soir. On était dans cette heure-là. Il en jouissait ; il regardait, il écoutait, quoi ? la tranquillité. Les farouches eux-mêmes ont leur instant de mélancolie. Subitement, cette tranquillité fut, non troublée, mais accentuée par des voix qui passaient. C’étaient des voix de femmes et d’enfants. Il y a parfois dans l’ombre de ces carillons de joie inattendus. On ne voyait point, à cause des broussailles, le groupe d’où sortaient les voix, mais ce groupe cheminait au pied de la dune et s’en allait vers la plaine et la forêt. Ces voix montaient claires et fraîches jusqu’au vieillard pensif ; elles étaient si près qu’il n’en perdait rien.\par
Une voix de femme disait :\par
— Dépêchons-nous, la Flécharde. Est-ce par ici ?\par
— Non, c’est par là.\par
Et le dialogue continuait entre les deux voix, l’une haute, l’autre timide.\par
— Comment appelez-vous cette métairie que nous habitons en ce moment ?\par
— L’Herbe-en-Pail.\par
— En sommes-nous encore loin ?\par
— A un bon quart d’heure.\par
 — Dépêchons-nous d’aller manger la soupe.\par
— C’est vrai que nous sommes en retard.\par
— Il faudrait courir. Mais vos mômes sont fatigués. Nous ne sommes que deux femmes, nous ne pouvons pas porter trois mioches. Et puis, vous en portez déjà un, vous, la Flécharde. Un vrai plomb. Vous l’avez sevrée, cette goinfre, mais vous la portez toujours. Mauvaise habitude. Faites-moi donc marcher ça. Ah ! tant pis, la soupe sera froide.\par
— Ah ! les bons souliers que vous m’avez donnés là ! On dirait qu’ils sont faits pour moi.\par
— Ça vaut mieux que d’aller nu-pattes.\par
— Dépêche-toi donc, René-Jean.\par
— C’est pourtant lui qui nous a retardées. Il faut qu’il parle à toutes les petites paysannes qu’on rencontre. Ça fait son homme.\par
— Dame, il va sur cinq ans.\par
— Dis donc, René-Jean, pourquoi as-tu parlé à cette petite dans le village ?\par
Une voix d’enfant, qui était une voix de garçon, répondit :\par
— Parce que c’est une que je connais.\par
La femme reprit :\par
— Comment ! tu la connais ?\par
— Oui, répondit le petit garçon, puisqu’elle m’a donné des bêtes ce matin.\par
— Voilà qui est fort ! s’écria la femme, nous ne sommes dans le pays que depuis trois jours, c’est gros comme le poing, et ça vous a déjà une amoureuse !\par
Les voix s’éloignèrent. Tout bruit cessa.
 \subsubsection[{II. Aures habet et non audiet}]{II \\
Aures habet et non audiet}\phantomsection
\label{p1l4c2}
\noindent Le vieillard restait immobile. Il ne pensait pas ; à peine songeait-il. Autour de lui tout était sérénité, assoupissement, confiance, solitude. Il faisait grand jour encore sur la dune, mais presque nuit dans la plaine et tout à fait nuit dans les bois. La lune montait à l’orient. Quelques étoiles piquaient le bleu pâle du zénith. Cet homme, bien que plein de préoccupations violentes, s’abîmait dans l’inexprimable mansuétude de l’infini. Il sentait monter en lui cette aube obscure, l’espérance, si le mot espérance peut s’appliquer aux attentes de la guerre civile. Pour l’instant, il lui semblait qu’en sortant de cette mer qui venait d’être si inexorable, et en touchant la terre, tout danger s’était évanoui. Personne ne savait son nom, il était seul, perdu pour l’ennemi, sans trace derrière lui, car la surface de la mer ne garde rien, caché, ignoré, pas même soupçonné. Il sentait on ne sait quel apaisement suprême. Un peu plus il se serait endormi.\par
Ce qui, pour cet homme en proie, au dedans comme au dehors, à tant de tumultes, donnait un charme étrange à cette heure calme qu’il traversait,  c’était, sur la terre comme au ciel, un profond silence.\par
On n’entendait que le vent qui venait de la mer ; mais le vent est une basse continue, et cesse presque d’être un bruit, tant il devient une habitude.\par
Tout à coup il se dressa debout.\par
Son attention venait d’être brusquement réveillée ; il considéra l’horizon. Quelque chose donnait à son regard une fixité particulière.\par
Ce qu’il regardait, c’était le clocher de Cormeray qu’il avait devant lui au fond de la plaine. On ne sait quoi d’extraordinaire se passait en effet dans ce clocher.\par
La silhouette de ce clocher se découpait nettement ; on voyait la tour surmontée de sa pyramide, et, entre la tour et la pyramide, la cage de la cloche, carrée, à jour, sans abat-vent, et ouverte aux regards des quatre côtés, ce qui est la mode des clochers bretons.\par
Or cette cage apparaissait alternativement ouverte et fermée ; à intervalles égaux, sa haute fenêtre se dessinait toute blanche, puis toute noire ; on voyait le ciel à travers, puis on ne le voyait plus ; il y avait clarté, puis occultation ; et l’ouverture et la fermeture se succédaient d’une seconde à l’autre avec la régularité du marteau sur l’enclume.\par
Le vieillard avait ce clocher de Cormeray devant lui, à une distance d’environ deux lieues ; il regarda à sa droite le clocher de Baguer-Pican, également droit sur l’horizon ; la cage de ce clocher s’ouvrait et se fermait comme celle de Cormeray.\par
Il regarda à sa gauche le clocher de Tanis ; la cage  du clocher de Tanis s’ouvrait et se fermait comme celle de Baguer-Pican.\par
Il regarda tous les clochers de l’horizon l’un après l’autre, à sa gauche les clochers de Courtils, de Précey, de Crollon et de la Croix-Avranchin ; à sa droite les clochers de Raz-sur-Couesnon, de Mordrey et des Pas ; en face de lui, le clocher de Pontorson. La cage de tous ces clochers était alternativement noire et blanche.\par
Qu’est-ce que cela voulait dire ?\par
Cela signifiait que toutes les cloches étaient en branle.\par
Il fallait, pour apparaître et disparaître ainsi, qu’elles fussent furieusement secouées.\par
Qu’était-ce donc ? Évidemment le tocsin.\par
On sonnait le tocsin, on le sonnait frénétiquement, on le sonnait partout, dans tous les clochers, dans toutes les paroisses, dans tous les villages. Et l’on n’entendait rien.\par
Cela tenait à la distance qui empêchait les sons d’arriver et au vent de mer qui soufflait du côté opposé et qui emportait tous les bruits de la terre hors de l’horizon.\par
Toutes ces cloches forcenées appelant de toutes parts, et en même temps ce silence, rien de plus sinistre.\par
Le vieillard regardait et écoutait.\par
Il n’entendait pas le tocsin, et il le voyait. Voir le tocsin, sensation étrange.\par
A qui en voulaient ces cloches ?\par
Contre qui ce tocsin ?
 \subsubsection[{III. Utilité des gros caractères}]{III \\
Utilité des gros caractères}\phantomsection
\label{p1l4c3}
\noindent Certainement quelqu’un était traqué.\par
Qui ?\par
Cet homme d’acier eut un frémissement.\par
Ce ne pouvait être lui. On n’avait pu deviner son arrivée. Il était impossible que les représentants en mission fussent déjà informés ; il venait à peine de débarquer. La corvette avait évidemment sombré sans qu’un homme échappât. Et dans la corvette même, excepté Boisberthelot et La Vieuville, personne ne savait son nom.\par
Les clochers continuaient leur jeu farouche. Il les examinait et les comptait machinalement, et sa rêverie, poussée d’une conjecture à l’autre, avait cette fluctuation que donne le passage d’une sécurité profonde à une incertitude terrible. Pourtant, après tout, ce tocsin pouvait s’expliquer de bien des façons, et il finissait par se rassurer en se répétant : En somme, personne ne sait mon arrivée et personne ne sait mon nom.\par
Depuis quelques instants il se faisait un léger bruit au-dessus de lui et derrière lui. Ce bruit ressemblait  au froissement d’une feuille d’arbre agitée. Il n’y prit d’abord pas garde ; puis, comme le bruit persistait, on pourrait dire insistait, il finit par se retourner. C’était une feuille en effet, mais une feuille de papier. Le vent était en train de décoller au-dessus de sa tête une large affiche appliquée sur la pierre milliaire. Cette affiche était placardée depuis peu de temps, car elle était encore humide et offrait prise au vent qui s’était mis à jouer avec elle et qui la détachait.\par
Le vieillard avait gravi la dune du côté opposé et n’avait pas vu cette affiche en arrivant.\par
Il monta sur la borne où il était assis, et posa sa main sur le coin du placard que le vent soulevait ; le ciel était serein, les crépuscules sont longs en juin ; le bas de la dune était ténébreux, mais le haut était éclairé ; une partie de l’affiche était imprimée en grosses lettres, et il faisait encore assez de jour pour qu’on pût les lire. Il lut ceci :\par

\labelblock{République française, une et indivisible.}

\noindent « Nous, Prieur de la Marne, représentant du peuple en mission près de l’armée des Côtes-de-Cherbourg, — ordonnons : — Le ci-devant marquis de Lantenac, vicomte de Fontenay, soi-disant prince breton, furtivement débarqué sur la côte de Granville, est mis hors la loi. — Sa tête est mise à prix. — Il sera payé à qui le livrera, mort ou vivant, la somme de soixante mille livres. — Cette somme ne  sera point payée en assignats, mais en or. — Un bataillon de l’armée des Côtes-de-Cherbourg sera immédiatement envoyé à la rencontre et à la recherche du ci-devant marquis de Lantenac. — Les communes sont requises de prêter main-forte. — Fait en la maison commune de Granville, le 2 juin 1793. — Signé :\par

\byline{« P{\scshape rieur de la} M{\scshape arne}. »}
\noindent Au-dessous de ce nom il y avait une autre signature, qui était en beaucoup plus petit caractère, et qu’on ne pouvait lire à cause du peu de jour qui restait.\par
Le vieillard rabaissa son chapeau sur ses yeux, croisa sa cape de mer jusque sous son menton, et descendit rapidement la dune. Il était évidemment inutile de s’attarder sur ce sommet éclairé.\par
Il y avait été peut-être trop longtemps déjà ; le haut de la dune était le seul point du paysage qui fût resté visible.\par
Quand il fut en bas et dans l’obscurité, il ralentit le pas.\par
Il se dirigeait dans le sens de l’itinéraire qu’il s’était tracé vers la métairie, ayant probablement des raisons de sécurité de ce côté-là.\par
Tout était désert. C’était l’heure où il n’y a plus de passants.\par
Derrière une broussaille, il s’arrêta, défit son manteau, retourna sa veste du côté velu, rattacha à son cou son manteau qui était une guenille nouée d’une corde, et se remit en route.\par
 Il faisait clair de lune.\par
Il arriva à un embranchement de deux chemins où se dressait une vieille croix de pierre. Sur le piédestal de la croix on distinguait un carré blanc qui était vraisemblablement une affiche pareille à celle qu’il venait de lire. Il s’en approcha.\par
— Où allez-vous ? lui dit une voix.\par
Il se retourna.\par
Un homme était là dans les haies, de haute taille comme lui, vieux comme lui, comme lui en cheveux blancs, et plus en haillons encore que lui-même. Presque son pareil.\par
Cet homme s’appuyait sur un long bâton.\par
L’homme reprit :\par
— Je vous demande où vous allez.\par
— D’abord où suis-je ? dit-il avec un calme presque hautain.\par
L’homme répondit :\par
— Vous êtes dans la seigneurie de Tanis, et j’en suis le mendiant, et vous en êtes le seigneur.\par
— Moi ?\par
— Oui, vous, monsieur le marquis de Lantenac.
 \subsubsection[{IV. Le caimand}]{IV \\
Le caimand}\phantomsection
\label{p1l4c4}
\noindent Le marquis de Lantenac, nous le nommerons par son nom désormais, répondit gravement :\par
— Soit. Livrez-moi.\par
L’homme poursuivit :\par
— Nous sommes tous deux chez nous ici, vous dans le château, moi dans le buisson.\par
— Finissons. Faites. Livrez-moi, dit le marquis.\par
L’homme continua :\par
— Vous alliez à la métairie d’Herbe-en-Pail, n’est-ce pas ?\par
— Oui.\par
— N’y allez point.\par
— Pourquoi ?\par
— Parce que les bleus y sont.\par
— Depuis quand ?\par
— Depuis trois jours.\par
— Les habitants de la ferme et du hameau ont-ils résisté ?\par
— Non. Ils ont ouvert toutes les portes.\par
— Ah ! dit le marquis.\par
L’homme montra du doigt le toit de la métairie  qu’on apercevait à quelque distance par-dessus les arbres.\par
— Voyez-vous le toit, monsieur le marquis ?\par
— Oui.\par
— Voyez-vous ce qu’il y a dessus ?\par
— Qui flotte ?\par
— Oui.\par
— C’est un drapeau.\par
— Tricolore, dit l’homme.\par
C’était l’objet qui avait déjà attiré l’attention du marquis quand il était au haut de la dune.\par
— Ne sonne-t-on pas le tocsin ? demanda le marquis.\par
— Oui.\par
— A cause de quoi ?\par
— Évidemment à cause de vous.\par
— Mais on ne l’entend pas ?\par
— C’est le vent qui empêche.\par
L’homme continua :\par
— Vous avez vu votre affiche ?\par
— Oui.\par
— On vous cherche.\par
Et, jetant un regard du côté de la métairie, il ajouta :\par
— Il y a là un demi-bataillon.\par
— De républicains ?\par
— Parisiens.\par
— Eh bien, dit le marquis, marchons.\par
Et il fit un pas vers la métairie.\par
L’homme lui saisit le bras.\par
 — N’y allez pas.\par
— Et où voulez-vous que j’aille ?\par
— Chez moi.\par
Le marquis regarda le mendiant.\par
— Écoutez, monsieur le marquis, ce n’est pas beau chez moi, mais c’est sûr. Une cabane plus basse qu’une cave. Pour plancher un lit de varech, pour plafond un toit de branches et d’herbes. Venez. A la métairie vous seriez fusillé. Chez moi vous dormirez. Vous devez être las ; et demain matin les bleus se seront remis en marche, et vous irez où vous voudrez.\par
Le marquis considérait cet homme.\par
— De quel côté êtes-vous donc ? demanda le marquis ; êtes-vous républicain ? êtes-vous royaliste ?\par
— Je suis un pauvre.\par
— Ni royaliste, ni républicain ?\par
— Je ne crois pas.\par
— Êtes-vous pour ou contre le roi ?\par
— Je n’ai pas le temps de ça.\par
— Qu’est-ce que vous pensez de ce qui se passe ?\par
— Je n’ai pas de quoi vivre.\par
— Pourtant vous venez à mon secours.\par
— J’ai vu que vous étiez hors la loi. Qu’est-ce que c’est que cela, la loi ? On peut donc être dehors. Je ne comprends pas. Quant à moi, suis-je dans la loi ? suis-je hors la loi ? Je n’en sais rien. Mourir de faim, est-ce être dans la loi ?\par
— Depuis quand mourez-vous de faim ?\par
— Depuis toute ma vie.\par
— Et vous me sauvez ?\par
 — Oui.\par
— Pourquoi ?\par
— Parce que j’ai dit : Voilà encore un plus pauvre que moi. J’ai le droit de respirer, lui il ne l’a pas.\par
— C’est vrai. Et vous me sauvez !\par
— Sans doute. Nous voilà frères, monseigneur. Je demande du pain, vous demandez la vie. Nous sommes deux mendiants.\par
— Mais savez-vous que ma tête est mise à prix ?\par
— Oui.\par
— Comment le savez-vous ?\par
— J’ai lu l’affiche.\par
— Vous savez lire ?\par
— Oui. Et écrire aussi. Pourquoi serais-je une brute ?\par
— Alors, puisque vous savez lire, et puisque vous avez lu l’affiche, vous savez qu’un homme qui me livrerait gagnerait soixante mille francs ?\par
— Je le sais.\par
— Pas en assignats.\par
— Oui, je sais, en or.\par
— Vous savez que soixante mille francs, c’est une fortune ?\par
— Oui.\par
— Et que quelqu’un qui me livrerait ferait sa fortune ?\par
— Eh bien, après ?\par
— Sa fortune !\par
— C’est justement ce que j’ai pensé. En vous voyant, je me suis dit : Quand je pense que quelqu’un  qui livrerait cet homme-ci gagnerait soixante mille francs et ferait sa fortune ! Dépêchons-nous de le cacher.\par
Le marquis suivit le pauvre.\par
Ils entrèrent dans un fourré. La tanière du mendiant était là. C’était une sorte de chambre qu’un grand vieux chêne avait laissé prendre chez lui à cet homme ; elle était creusée sous ses racines et couverte de ses branches. C’était obscur, bas, caché, invisible. Il y avait place pour deux.\par
— J’ai prévu que je pouvais avoir un hôte, dit le mendiant.\par
Cette espèce de logis sous terre, moins rare en Bretagne qu’on ne croit, s’appelle en langue paysanne \emph{carnichot}. Ce nom s’applique aussi à des cachettes pratiquées dans l’épaisseur des murs.\par
C’est meublé de quelques pots, d’un grabat de paille ou de goémon lavé et séché, d’une grosse couverture de créseau, et de quelques mèches de suif avec un briquet et des tiges creuses de brane-ursine pour allumettes.\par
Ils se courbèrent, rampèrent un peu, pénétrèrent dans la chambre où les grosses racines de l’arbre découpaient des compartiments bizarres, et s’assirent sur un tas de varech sec qui était le lit. L’intervalle de deux racines par où l’on entrait et qui servait de porte donnait quelque clarté. La nuit était venue, mais le regard se proportionne à la lumière, et l’on finit par trouver toujours un peu de jour dans l’ombre. Un reflet du clair de lune blanchissait vaguement  l’entrée. Il y avait dans un coin une cruche d’eau, une galette de sarrasin et des châtaignes.\par
— Soupons, dit le pauvre.\par
Ils se partagèrent les châtaignes, le marquis donna son morceau de biscuit, ils mordirent à la même miche de blé noir et burent à la cruche l’un après l’autre.\par
Ils causèrent.\par
Le marquis se mit à interroger cet homme.\par
— Ainsi, tout ce qui arrive ou rien, c’est pour vous la même chose ?\par
— A peu près. Vous êtes des seigneurs, vous autres. Ce sont vos affaires.\par
— Mais enfin, ce qui se passe...\par
— Ça se passe là-haut.\par
Le mendiant ajouta :\par
— Et puis il y a des choses qui se passent encore plus haut, le soleil qui se lève, la lune qui augmente ou diminue, c’est de celles-là que je m’occupe.\par
Il but une gorgée à la cruche, et dit : — La bonne eau fraîche !\par
Et il reprit :\par
— Comment trouvez-vous cette eau, monseigneur ?\par
— Comment vous appelez-vous ? dit le marquis.\par
— Je m’appelle Tellmarch, et l’on m’appelle le Caimand.\par
— Je sais. Caimand est un mot du pays.\par
— Qui veut dire mendiant. On me surnomme aussi le Vieux.\par
Il poursuivit :\par
— Voilà quarante ans qu’on m’appelle le Vieux.\par
 — Quarante ans ! mais vous étiez jeune.\par
— Je n’ai jamais été jeune. Vous l’êtes toujours, vous, monsieur le marquis. Vous avez des jambes de vingt ans, vous escaladez la grande dune ; moi, je commence à ne plus marcher, au bout d’un quart de lieue je suis las. Nous sommes pourtant du même âge ; mais les riches, ça a sur nous un avantage, c’est que ça mange tous les jours. Manger conserve.\par
Le mendiant, après un silence, continua :\par
— Les pauvres, les riches, c’est une terrible affaire. C’est ce qui produit les catastrophes. Du moins, ça me fait cet effet-là. Les pauvres veulent être riches, les riches ne veulent pas être pauvres. Je crois que c’est un peu là le fond. Je ne m’en mêle pas. Les événements sont les événements. Je ne suis ni pour le créancier, ni pour le débiteur. Je sais qu’il y a une dette et qu’on la paye. Voilà tout. J’aurais mieux aimé qu’on ne tuât pas le roi, mais il me serait difficile de dire pourquoi. Après ça, on me répond : Mais, autrefois, comme on vous accrochait les gens aux arbres pour rien du tout ! Tenez, moi, pour un méchant coup de fusil tiré à un chevreuil du roi, j’ai vu pendre un homme qui avait une femme et sept enfants. Il y a à dire des deux côtés.\par
Il se tut encore, puis ajouta :\par
— Vous comprenez, je ne sais pas au juste, on va, on vient, il se passe des choses ; moi, je suis là sous les étoiles.\par
Tellmarch eut encore une interruption de rêverie, puis continua :\par
 — Je suis un peu rebouteux, un peu médecin, je connais les herbes, je tire parti des plantes, les paysans me voient attentif devant rien, et cela me fait passer pour sorcier. Parce que je songe, on croit que je sais.\par
— Vous êtes du pays ? dit le marquis.\par
— Je n’en suis jamais sorti.\par
— Vous me connaissez ?\par
— Sans doute. La dernière fois que je vous ai vu, c’est à votre dernier passage, il y a deux ans. Vous êtes allé d’ici en Angleterre. Tout à l’heure j’ai aperçu un homme au haut de la dune. Un homme de grande taille. Les hommes grands sont rares ; c’est un pays d’hommes petits, la Bretagne. J’ai bien regardé, j’avais lu l’affiche. J’ai dit : Tiens ! Et quand vous êtes descendu, il y avait de la lune, je vous ai reconnu.\par
— Pourtant, moi, je ne vous connais pas.\par
— Vous m’avez vu, mais vous ne m’avez pas vu.\par
Et Tellmarch le Caimand ajouta :\par
— Je vous voyais, moi. De mendiant à passant, le regard n’est pas le même.\par
— Est-ce que je vous avais rencontré autrefois ?\par
— Souvent, puisque je suis votre mendiant. J’étais le pauvre du bas du chemin de votre château. Vous m’avez dans l’occasion fait l’aumône ; mais celui qui donne ne regarde pas, celui qui reçoit examine et observe. Qui dit mendiant, dit espion. Mais moi, quoique souvent triste, je tâche de ne pas être un mauvais espion. Je tendais la main, vous ne voyiez que la main, et vous y jetiez l’aumône dont j’avais  besoin le matin pour ne pas mourir de faim le soir. On est des fois des vingt-quatre heures sans manger. Quelquefois un sou c’est la vie. Je vous dois la vie, je vous la rends.\par
— C’est vrai, vous me sauvez.\par
— Oui, je vous sauve, monseigneur.\par
Et la voix de Tellmarch devint grave.\par
— A une condition.\par
— Laquelle ?\par
— C’est que vous ne venez pas ici pour faire le mal.\par
— Je viens ici pour faire le bien, dit le marquis.\par
— Dormons, dit le mendiant.\par
Ils se couchèrent côte à côte sur le lit de varech. Le mendiant fut tout de suite endormi. Le marquis, bien que très las, resta un moment rêveur, puis, dans cette ombre, il regarda le pauvre, et se coucha. Se coucher sur ce lit, c’était se coucher sur le sol ; il en profita pour coller son oreille à terre, et il écouta. Il y avait sous la terre un sombre bourdonnement ; on sait que le son se propage dans les profondeurs du sol ; on entendait le bruit des cloches.\par
Le tocsin continuait.\par
Le marquis s’endormit.
 \subsubsection[{V. Signé Gauvain}]{V \\
Signé Gauvain}\phantomsection
\label{p1l4c5}
\noindent Quand il se réveilla, il faisait jour.\par
Le mendiant était debout, non dans la tanière, car on ne pouvait s’y tenir droit, mais dehors et sur le seuil. Il était appuyé sur son bâton. Il y avait du soleil sur son visage.\par
— Monseigneur, dit Tellmarch, quatre heures du matin viennent de sonner au clocher de Tanis. J’ai entendu les quatre coups ; donc le vent a changé, c’est le vent de terre. Je n’entends aucun autre bruit ; donc le tocsin a cessé. Tout est tranquille dans la métairie et dans le hameau d’Herbe-en-Pail. Les bleus dorment ou sont partis. Le plus fort du danger est passé ; il est sage de nous séparer. C’est mon heure de m’en aller.\par
Il désigna un point de l’horizon.\par
— Je m’en vais par là.\par
Et il désigna le point opposé.\par
— Vous, allez-vous-en par ici.\par
Le mendiant fit au marquis un grave salut de la main.\par
Il ajouta en montrant ce qui restait du souper :\par
— Emportez des châtaignes, si vous avez faim.\par
 Un moment après, il avait disparu sous les arbres.\par
Le marquis se leva, et s’en alla du côté que lui avait indiqué Tellmarch.\par
C’était l’heure charmante que la vieille langue paysanne normande appelle la « piperette du jour ». On entendait jaser les cardrounettes et les moineaux de haie. Le marquis suivit le sentier par où ils étaient venus la veille. Il sortit du fourré et se retrouva à l’embranchement de routes marqué par la croix de pierre. L’affiche y était, blanche et comme gaie au soleil levant. Il se rappela qu’il y avait au bas de l’affiche quelque chose qu’il n’avait pu lire la veille à cause de la finesse des lettres et du peu de jour qu’il faisait. Il alla au piédestal de la croix. L’affiche se terminait en effet, au-dessous de la signature, P{\scshape rieur de la} M{\scshape arne}, par ces deux lignes en petits caractères :\par
« L’identité du ci-devant marquis de Lantenac constatée, il sera immédiatement passé par les armes. — Signé : \emph{Le chef de bataillon, commandant la colonne d’expédition,} G{\scshape auvain}. »\par
— Gauvain ! dit le marquis.\par
Il s’arrêta profondément pensif, l’œil fixé sur l’affiche.\par
— Gauvain ! répéta-t-il.\par
Il se remit en marche, se retourna, regarda la croix, revint sur ses pas, et lut l’affiche encore une fois.\par
Puis, il s’éloigna à pas lents. Quelqu’un qui eût été près de lui l’eût entendu murmurer à demi-voix : Gauvain !\par
Du fond des chemins creux où il se glissait, on ne  voyait pas les toits de la métairie qu’il avait laissée à sa gauche. Il côtoyait une éminence abrupte, toute couverte d’ajoncs en fleur, de l’espèce dite longue-épine. Cette éminence avait pour sommet une de ces pointes de terre qu’on appelle dans le pays une « hure ». Au pied de l’éminence, le regard se perdait tout de suite sous les arbres. Les feuillages étaient comme trempés de lumière. Toute la nature avait la joie profonde du matin.\par
Tout à coup ce paysage fut terrible. Ce fut comme une embuscade qui éclate. On ne sait quelle trombe faite de cris sauvages et de coups de fusil s’abattit sur ces champs et ces bois pleins de rayons, et l’on vit s’élever, du côté où était la métairie, une grande fumée coupée de flammes claires, comme si le hameau et la ferme n’étaient plus qu’une botte de paille qui brûlait. Ce fut subit et lugubre, le passage brusque du calme à la furie, une explosion de l’enfer en pleine aurore, l’horreur sans transition. On se battait du côté d’Herbe-en-Pail. Le marquis s’arrêta.\par
Il n’est personne qui, en pareil cas, ne l’ait éprouvé, la curiosité est plus forte que le danger ; on veut savoir, dût-on périr. Il monta sur l’éminence au bas de laquelle passait le chemin creux. De là on était vu, mais on voyait. Il fut sur la hure en quelques minutes. Il regarda.\par
En effet, il y avait une fusillade et un incendie. On entendait des clameurs, on voyait du feu. La métairie était comme le centre d’on ne sait quelle catastrophe. Qu’était-ce ? La métairie d’Herbe-en-Pail était-elle  attaquée ? Mais par qui ? Était-ce un combat ? N’était-ce pas plutôt une exécution militaire ? Les bleus, et cela leur était ordonné par un décret révolutionnaire, punissaient très souvent, en y mettant le feu, les fermes et les villages réfractaires ; on brûlait, pour l’exemple, toute métairie et tout hameau qui n’avaient point fait les abatis d’arbres prescrits par la loi et qui n’avaient pas ouvert et taillé dans les fourrés des passages pour la cavalerie républicaine. On avait notamment exécuté ainsi tout récemment la paroisse de Bourgon, près d’Ernée. Herbe-en-Pail était-il dans le même cas ? Il était visible qu’aucune des percées stratégiques commandées par le décret n’avait été faite dans les halliers et dans les enclos de Tanis et d’Herbe-en-Pail. Était-ce le châtiment ? Était-il arrivé un ordre à l’avant-garde qui occupait la métairie ? Cette avant-garde ne faisait-elle pas partie d’une de ces colonnes d’expédition surnommées \emph{colonnes infernales ?}\par
Un fourré très hérissé et très fauve entourait de toutes parts l’éminence au sommet de laquelle le marquis s’était placé en observation. Ce fourré, qu’on appelait le bocage d’Herbe-en-Pail, mais qui avait les proportions d’un bois, s’étendait jusqu’à la métairie, et cachait, comme tous les halliers bretons, un réseau de ravins, de sentiers et de chemins creux, labyrinthes où les armées républicaines se perdaient.\par
L’exécution, si c’était une exécution, avait dû être féroce, car elle fut courte. Ce fut, comme toutes les choses brutales, tout de suite fait. L’atrocité des guerres civiles comporte ces sauvageries. Pendant que  le marquis, multipliant les conjectures, hésitant à descendre, hésitant à rester, écoutait et épiait, ce fracas d’extermination cessa, ou pour mieux dire se dispersa. Le marquis constata dans le hallier comme l’éparpillement d’une troupe furieuse et joyeuse. Un effrayant fourmillement se fit sous les arbres. De la métairie on se jetait dans le bois. Il y avait des tambours qui battaient la charge. On ne tirait plus de coups de fusil ; cela ressemblait maintenant à une battue ; on semblait fouiller, poursuivre, traquer ; il était évident qu’on cherchait quelqu’un ; le bruit était diffus et profond ; c’était une confusion de paroles de colère et de triomphe, une rumeur composée de clameurs ; on n’y distinguait rien. Brusquement, comme un linéament se dessine dans une fumée, quelque chose devint articulé et précis dans ce tumulte, c’était un nom, un nom répété par mille voix, et le marquis entendit nettement ce cri : — Lantenac ! Lantenac ! le marquis de Lantenac !\par
C’était lui qu’on cherchait.
 \subsubsection[{VI. Les péripéties de la guerre civile}]{VI \\
Les péripéties de la guerre civile}\phantomsection
\label{p1l4c6}
\noindent Et subitement, autour de lui, et de tous les côtés à la fois, le fourré se remplit de fusils, de bayonnettes et de sabres, un drapeau tricolore se dressa dans la pénombre, le cri \emph{Lantenac !} éclata à son oreille, et, à ses pieds, à travers les ronces et les branches, des faces violentes apparurent.\par
Le marquis était seul, debout sur un sommet, visible de tous les points du bois. Il voyait à peine ceux qui criaient son nom, mais il était vu de tous. S’il y avait mille fusils dans le bois, il était là comme une cible. Il ne distinguait rien dans le taillis que des prunelles ardentes fixées sur lui.\par
Il ôta son chapeau, en retroussa le bord, arracha une longue épine sèche à un ajonc, tira de sa poche une cocarde blanche, fixa avec l’épine le bord retroussé et la cocarde à la forme du chapeau, et, remettant sur la tête le chapeau dont le bord relevé laissait voir son front et sa cocarde, il dit d’une voix haute, parlant à toute la forêt à la fois :\par
— Je suis l’homme que vous cherchez. Je suis le marquis de Lantenac, vicomte de Fontenay, prince  breton, lieutenant-général des armées du roi. Finissons-en. En joue ! Feu !\par
Et, écartant de ses deux mains sa veste de peau de chèvre, il montra sa poitrine nue.\par
Il baissa les yeux, cherchant du regard les fusils braqués, et se vit entouré d’hommes à genoux.\par
Un immense cri s’éleva : — Vive Lantenac ! Vive monseigneur ! Vive le général !\par
En même temps des chapeaux sautaient en l’air, des sabres tournoyaient joyeusement, et l’on voyait dans tout le taillis se dresser des bâtons au bout desquels s’agitaient des bonnets de laine brune.\par
Ce qu’il avait autour de lui, c’était une bande vendéenne.\par
Cette bande s’était agenouillée en le voyant.\par
La légende raconte qu’il y avait dans les vieilles forêts thuringiennes des êtres étranges, race des géants, plus et moins qu’hommes, qui étaient considérés par les Romains comme des animaux horribles, et par les Germains comme des incarnations divines, et qui, selon la rencontre, couraient la chance d’être exterminés ou adorés.\par
Le marquis éprouva quelque chose de pareil à ce que devait ressentir un de ces êtres quand, s’attendant à être traité comme un monstre, il était brusquement traité comme un dieu.\par
Tous ces yeux pleins d’éclairs redoutables se fixaient sur le marquis avec une sorte de sauvage amour.\par
Cette cohue était armée de fusils, de sabres, de  faulx, de pioches, de bâtons ; tous avaient de grands feutres ou des bonnets bruns, avec des cocardes blanches, une profusion de rosaires et d’amulettes, de larges culottes ouvertes au genou, des casaques de poil, des guêtres de cuir, le jarret nu, les cheveux longs, quelques-uns l’air féroce, tous l’œil naïf.\par
Un homme, jeune et de belle mine, traversa ces gens agenouillés et monta à grands pas vers le marquis. Cet homme était, comme les paysans, coiffé d’un feutre à bord relevé et à cocarde blanche, et vêtu d’une casaque de poil, mais il avait les mains blanches et une chemise fine, et il portait par-dessus sa veste une écharpe de soie blanche à laquelle pendait une épée à poignée dorée.\par
Parvenu sur la hure, il jeta son chapeau, détacha son écharpe, mit un genou en terre, présenta au marquis l’écharpe et l’épée, et dit :\par
— Nous vous cherchions en effet, nous vous avons trouvé. Voici l’épée de commandement. Ces hommes sont maintenant à vous. J’étais leur commandant, je monte en grade, je suis votre soldat. Acceptez notre hommage, monseigneur. Donnez vos ordres, mon général.\par
Puis il fit un signe, et des hommes qui portaient un drapeau tricolore sortirent du bois. Ces hommes montèrent jusqu’au marquis et déposèrent le drapeau à ses pieds. C’était le drapeau qu’il venait d’entrevoir à travers les arbres.\par
— Mon général, dit le jeune homme qui lui avait présenté l’épée et l’écharpe, ceci est le drapeau que  nous venons de prendre aux bleus qui étaient dans la ferme d’Herbe-en-Pail. Monseigneur, je m’appelle Gavard. J’ai été au marquis de La Rouarie.\par
— C’est bien, dit le marquis.\par
Et, calme et grave, il ceignit l’écharpe.\par
Puis il tira l’épée, et, l’agitant nue au-dessus de sa tête :\par
— Debout ! dit-il, et vive le roi !\par
Tous se levèrent.\par
Et l’on entendit dans les profondeurs du bois une clameur éperdue et triomphante : \emph{Vive le roi ! Vive notre marquis ! Vive Lantenac !}\par
Le marquis se tourna vers Gavard.\par
— Combien donc êtes-vous ?\par
— Sept mille.\par
Et tout en descendant de l’éminence, pendant que les paysans écartaient les ajoncs devant les pas du marquis de Lantenac, Gavard continua :\par
— Monseigneur, rien de plus simple. Tout cela s’explique d’un mot. On n’attendait qu’une étincelle. L’affiche de la république, en révélant votre présence, a insurgé le pays pour le roi. Nous avions en outre été avertis sous main par le maire de Granville qui est un homme à nous ; le même qui a sauvé l’abbé Olivier. Cette nuit on a sonné le tocsin.\par
— Pour qui ?\par
— Pour vous.\par
— Ah ! dit le marquis.\par
— Et nous voilà, reprit Gavard.\par
— Et vous êtes sept mille ?\par
 — Aujourd’hui. Nous serons quinze mille demain. C’est le rendement du pays. Quand M. Henri de La Rochejaquelein est parti pour l’armée catholique, on a sonné le tocsin, et en une nuit six paroisses, Isernay, Corqueux, les Échaubroignes, les Aubiers, Saint-Aubin et Nueil, lui ont amené dix mille hommes. On n’avait pas de munitions, on a trouvé chez un maçon soixante livres de poudre de mine, et M. de La Rochejaquelein est parti avec cela. Nous pensions bien que vous deviez être quelque part dans cette forêt, et nous vous cherchions.\par
— Et vous avez attaqué les bleus dans la ferme d’Herbe-en-Pail ?\par
— Le vent les avait empêchés d’entendre le tocsin. Ils ne se défiaient pas ; les gens du hameau, qui sont patauds, les avaient bien reçus. Ce matin, nous avons investi la ferme, les bleus dormaient, et en un tour de main la chose a été faite. J’ai un cheval. Daignez-vous l’accepter, mon général ?\par
— Oui.\par
Un paysan amena un cheval blanc militairement harnaché. Le marquis, sans user de l’aide que lui offrait Gavard, monta à cheval.\par
— Hurrah ! crièrent les paysans. Car les cris anglais sont fort usités sur la côte bretonne-normande, en commerce perpétuel avec les îles de la Manche.\par
Gavard fit le salut militaire et demanda :\par
— Quel sera votre quartier général, monseigneur ?\par
— D’abord la forêt de Fougères.\par
 — C’est une de vos sept forêts, monsieur le marquis.\par
— Il faut un prêtre.\par
— Nous en avons un.\par
— Qui ?\par
— Le vicaire de la Chapelle-Erbrée.\par
— Je le connais. Il a fait le voyage de Jersey.\par
Un prêtre sortit des rangs, et dit :\par
— Trois fois.\par
Le marquis tourna la tête.\par
— Bonjour, monsieur le vicaire. Vous allez avoir de la besogne.\par
— Tant mieux, monsieur le marquis.\par
— Vous aurez du monde à confesser. Ceux qui voudront. On ne force personne.\par
— Monsieur le marquis, dit le prêtre, Gaston, à Guéménée, force les républicains à se confesser.\par
— C’est un perruquier, dit le marquis. Mais la mort doit être libre.\par
Gavard, qui était allé donner quelques consignes, revint.\par
— Mon général, j’attends vos commandements.\par
— D’abord, le rendez-vous est à la forêt de Fougères. Qu’on se disperse et qu’on y aille.\par
— L’ordre est donné.\par
— Ne m’avez-vous pas dit que les gens d’Herbe-en-Pail avaient bien reçu les bleus ?\par
— Oui, mon général.\par
— Vous avez brûlé la ferme ?\par
— Oui.\par
 — Avez-vous brûlé le hameau ?\par
— Non.\par
— Brûlez-le.\par
— Les bleus ont essayé de se défendre ; mais ils étaient cent cinquante et nous étions sept mille.\par
— Qu’est-ce que c’est que ces bleus-là ?\par
— Des bleus de Santerre.\par
— Qui a commandé le roulement de tambours pendant qu’on coupait la tête au roi. Alors c’est un bataillon de Paris ?\par
— Un demi-bataillon.\par
— Comment s’appelle ce bataillon ?\par
— Mon général, il y a sur le drapeau : Bataillon du Bonnet-Rouge.\par
— Des bêtes féroces.\par
— Que faut-il faire des blessés ?\par
— Achevez-les.\par
— Que faut-il faire des prisonniers ?\par
— Fusillez-les.\par
— Il y en a environ quatrevingts.\par
— Fusillez tout.\par
— Il y a deux femmes.\par
— Aussi.\par
— Il y a trois enfants.\par
— Emmenez-les. On verra ce qu’on en fera.\par
Et le marquis poussa son cheval.
 \subsubsection[{VII. Pas de grâce (mot d’ordre de la commune) pas de quartier (mot d’ordre des princes)}]{VII \\
Pas de grâce (mot d’ordre de la commune) pas de quartier (mot d’ordre des princes)}\phantomsection
\label{p1l4c7}
\noindent Pendant que ceci se passait près de Tanis, le mendiant s’en était allé vers Crollon. Il s’était enfoncé dans les ravins, sous les vastes feuillées sourdes, inattentif à tout et attentif à rien, comme il l’avait dit lui-même, rêveur plutôt que pensif, car le pensif a un but et le rêveur n’en a pas, errant, rôdant, s’arrêtant, mangeant çà et là une pousse d’oseille sauvage, buvant aux sources, dressant la tête par moments à des fracas lointains, puis rentrant dans l’éblouissante fascination de la nature, offrant ses haillons au soleil, entendant peut-être le bruit des hommes, mais écoutant le chant des oiseaux.\par
Il était vieux et lent ; il ne pouvait aller loin ; comme il l’avait dit au marquis de Lantenac, un quart de lieue le fatiguait ; il fit un court circuit vers la Croix-Avranchin, et le soir était venu quand il s’en retourna.\par
Un peu au delà de Macey, le sentier qu’il suivait le conduisit sur une sorte de point culminant dégagé d’arbres, d’où l’on voit de très loin et d’où l’on découvre tout l’horizon de l’ouest jusqu’à la mer.\par
 Une fumée appela son attention.\par
Rien de plus doux qu’une fumée, rien de plus effrayant. Il y a les fumées paisibles et il y a les fumées scélérates. Une fumée, l’épaisseur et la couleur d’une fumée, c’est toute la différence entre la paix et la guerre, entre la fraternité et la haine, entre l’hospitalité et le sépulcre, entre la vie et la mort. Une fumée qui monte dans les arbres peut signifier ce qu’il y a de plus charmant au monde, le foyer, ou ce qu’il y a de plus affreux, l’incendie ; et tout le bonheur comme tout le malheur de l’homme sont parfois dans cette chose éparse au vent.\par
La fumée que regardait Tellmarch était inquiétante.\par
Elle était noire avec des rougeurs subites, comme si le brasier d’où elle sortait avait des intermittences et achevait de s’éteindre, et elle s’élevait au-dessus d’Herbe-en-Pail.\par
Tellmarch hâta le pas et se dirigea vers cette fumée. Il était bien las, mais il voulait savoir ce que c’était.\par
Il arriva au sommet d’un coteau auquel étaient adossés le hameau et la métairie.\par
Il n’y avait plus ni métairie ni hameau.\par
Un tas de masures brûlait, et c’était là Herbe-en-Pail.\par
Il y a quelque chose de plus poignant à voir brûler qu’un palais, c’est une chaumière. Une chaumière en feu est lamentable. La dévastation s’abattant sur la misère, le vautour s’acharnant sur le ver de terre, il y a là on ne sait quel contre-sens qui serre le cœur.\par
 A en croire la légende biblique, un incendie regardé change une créature humaine en statue ; Tellmarch fut un moment cette statue. Le spectacle qu’il avait sous les yeux le fit immobile. Cette destruction s’accomplissait en silence. Pas un cri ne s’élevait ; pas un soupir humain ne se mêlait à cette fumée ; cette fournaise travaillait et achevait de dévorer ce village sans qu’on entendît d’autre bruit que le craquement des charpentes et le pétillement des chaumes. Par moments la fumée se déchirait, les toits effondrés laissaient voir les chambres béantes, le brasier montrait tous ses rubis, des guenilles écarlates et de pauvres vieux meubles couleur de pourpre se dressaient dans des intérieurs vermeils, et Tellmarch avait le sinistre éblouissement du désastre.\par
Quelques arbres d’une châtaigneraie contiguë aux maisons avaient pris feu et flambaient.\par
Il écoutait, tâchant d’entendre une voix, un appel, une clameur ; rien ne remuait, excepté les flammes ; tout se taisait, excepté l’incendie. Est-ce donc que tous avaient fui ?\par
Où était ce groupe vivant et travaillant d’Herbe-en-Pail ? Qu’était devenu tout ce petit peuple ?\par
Tellmarch descendit du coteau.\par
Une énigme funèbre était devant lui. Il s’en approchait sans hâte et l’œil fixe. Il avançait vers cette ruine avec une lenteur d’ombre ; il se sentait fantôme dans cette tombe.\par
Il arriva à ce qui avait été la porte de la métairie, et il regarda dans la cour qui, maintenant, n’avait plus  de murailles et se confondait avec le hameau groupé autour d’elle.\par
Ce qu’il avait vu n’était rien. Il n’avait encore aperçu que le terrible, l’horrible lui apparut.\par
Au milieu de la cour il y avait un monceau noir, vaguement modelé d’un côté par la flamme, de l’autre par la lune ; ce monceau était un tas d’hommes, ces hommes étaient morts.\par
Il y avait autour de ce tas une grande mare qui fumait un peu ; l’incendie se reflétait dans cette mare, mais elle n’avait pas besoin du feu pour être rouge ; c’était du sang.\par
Tellmarch s’approcha. Il se mit à examiner, l’un après l’autre, ces corps gisants ; tous étaient des cadavres.\par
La lune éclairait, l’incendie aussi.\par
Ces cadavres étaient des soldats. Tous étaient pieds nus ; on leur avait pris leurs souliers ; on leur avait aussi pris leurs armes ; ils avaient encore leurs uniformes qui étaient bleus ; çà et là on distinguait, dans l’amoncellement des membres et des têtes, des chapeaux troués avec des cocardes tricolores. C’étaient des républicains. C’étaient ces Parisiens qui, la veille encore, étaient là tous vivants, et tenaient garnison dans la ferme d’Herbe-en-Pail. Ces hommes avaient été suppliciés, ce qu’indiquait la chute symétrique des corps ; ils avaient été foudroyés sur place, et avec soin. Ils étaient tous morts. Pas un râle ne sortait du tas.\par
Tellmarch passa cette revue des cadavres, sans en omettre un seul ; tous étaient criblés de balles.\par
 Ceux qui les avaient mitraillés, pressés probablement d’aller ailleurs, n’avaient pas pris le temps de les enterrer.\par
Comme il allait se retirer, ses yeux tombèrent sur un mur bas qui était dans la cour, et il vit quatre pieds qui passaient de derrière l’angle de ce mur.\par
Ces pieds avaient des souliers ; ils étaient plus petits que les autres ; Tellmarch approcha. C’étaient des pieds de femmes.\par
Deux femmes étaient gisantes côte à côte derrière le mur, fusillées aussi.\par
Tellmarch se pencha sur elles. L’une de ces femmes avait une sorte d’uniforme ; à côté d’elle était un bidon brisé et vidé ; c’était une vivandière. Elle avait quatre balles dans la tête. Elle était morte.\par
Tellmarch examina l’autre. C’était une paysanne. Elle était blême et béante. Ses yeux étaient fermés. Elle n’avait aucune plaie à la tête. Ses vêtements, dont les fatigues sans doute avaient fait des haillons, s’étaient ouverts dans sa chute, et laissaient voir son torse à demi nu. Tellmarch acheva de les écarter, et vit à une épaule la plaie ronde que fait une balle ; la clavicule était cassée. Il regarda ce sein livide.\par
— Mère et nourrice, murmura-t-il.\par
Il la toucha. Elle n’était pas froide.\par
Elle n’avait pas d’autre blessure que la clavicule cassée et la plaie à l’épaule.\par
Il posa la main sur le cœur et sentit un faible battement. Elle n’était pas morte.\par
 Tellmarch se redressa debout et cria d’une voix terrible :\par
— Il n’y a donc personne ici ?\par
— C’est toi, le caimand ! répondit une voix, si basse qu’on l’entendait à peine.\par
Et en même temps une tête sortit d’un trou de ruine.\par
Puis une autre face apparut dans une autre masure.\par
C’étaient deux paysans qui s’étaient cachés ; les seuls qui survécussent.\par
La voix connue du caimand les avait rassurés et les avait fait sortir des recoins où ils se blottissaient.\par
Ils avancèrent vers Tellmarch, fort tremblants encore.\par
Tellmarch avait pu crier, mais ne pouvait parler ; les émotions profondes sont ainsi.\par
Il leur montra du doigt la femme étendue à ses pieds.\par
— Est-ce qu’elle est encore en vie ? dit l’un des paysans.\par
Tellmarch fit de la tête signe que oui.\par
— L’autre femme est-elle vivante ? demanda l’autre paysan.\par
Tellmarch fit signe que non.\par
Le paysan qui s’était montré le premier reprit :\par
— Tous les autres sont morts, n’est-ce pas ? J’ai vu cela. J’étais dans ma cave. Comme on remercie Dieu dans ces moments-là de n’avoir pas de famille ! Ma maison brûlait. Seigneur Jésus ! on a tout tué. Cette femme-ci avait des enfants. Trois enfants. Tout petits !  Les enfants criaient : Mère ! La mère criait : Mes enfants ! On a tué la mère et on a emmené les enfants. J’ai vu cela, mon Dieu ! mon Dieu ! mon Dieu ! Ceux qui ont tout massacré sont partis. Ils étaient contents. Ils ont emmené les petits et tué la mère. Mais elle n’est pas morte, n’est-ce pas, elle n’est pas morte ? Dis donc, le caimand, est-ce que tu crois que tu pourrais la sauver ? Veux-tu que nous t’aidions à la porter dans ton carnichot ?\par
Tellmarch fit signe que oui.\par
Le bois touchait à la ferme. Ils eurent vite fait un brancard avec des feuillages et des fougères. Ils placèrent sur le brancard la femme toujours immobile, et se mirent en marche dans le hallier, les deux paysans portant le brancard, l’un à la tête, l’autre aux pieds, Tellmarch soutenant le bras de la femme, et lui tâtant le pouls.\par
Tout en cheminant, les deux paysans causaient, et, par-dessus la femme sanglante dont la lune éclairait la face pâle, ils échangeaient des exclamations effarées.\par
— Tout tuer !\par
— Tout brûler !\par
— Ah ! monseigneur Dieu ! est-ce qu’on va être comme ça à présent ?\par
— C’est ce grand homme vieux qui l’a voulu.\par
— Oui, c’est lui qui commandait.\par
— Je ne l’ai pas vu quand on a fusillé. Est-ce qu’il était là ?\par
— Non. Il était parti. Mais c’est égal, tout s’est fait par son commandement.\par
 — Alors, c’est lui qui a tout fait.\par
— Il avait dit : Tuez ! brûlez ! pas de quartier !\par
— C’est un marquis.\par
— Oui, puisque c’est notre marquis.\par
— Comment s’appelle-t-il donc déjà ?\par
— C’est monsieur de Lantenac.\par
Tellmarch leva les yeux au ciel et murmura entre ses dents :\par
— Si j’avais su !
 
\hline
\section[{Deuxième partie. À Paris}]{Deuxième partie \\
À Paris}\phantomsection
\label{p2}\renewcommand{\leftmark}{Deuxième partie \\
À Paris}

  \subsection[{Livre premier. Cimourdain}]{Livre premier \\
Cimourdain}\phantomsection
\label{p2l1}
\subsubsection[{I. Les rues de paris dans ce temps-la}]{I \\
Les rues de paris dans ce temps-la}\phantomsection
\label{p2l1c1}
\noindent On vivait en public ; on mangeait sur des tables dressées devant les portes ; les femmes assises sur les perrons des églises faisaient de la charpie en chantant \emph{la Marseillaise ;} le parc Monceaux et le Luxembourg étaient des champs de manœuvre ; il y avait dans tous les carrefours des armureries en plein travail, on fabriquait des fusils sous les yeux des passants qui battaient des mains ; on n’entendait que ce mot dans toutes les bouches : \emph{Patience. Nous sommes en révolution.} On souriait héroïquement. On allait au spectacle comme à Athènes pendant la guerre du Péloponèse ; on voyait affichés au coin des rues : \emph{Le Siège de Thionville. — La Mère de famille sauvée des flammes. — Le Club des Sans-Soucis. — L’Aînée des papesses Jeanne. — Les Philosophes soldats. — L’Art  d’aimer au village}. — Les Allemands étaient aux portes ; le bruit courait que le roi de Prusse avait fait retenir des loges à l’Opéra. Tout était effrayant et personne n’était effrayé. La ténébreuse loi des suspects, qui est le crime de Merlin de Douai, faisait la guillotine visible au-dessus de toutes les têtes. Un procureur, nommé Séran, dénoncé, attendait qu’on vînt l’arrêter, en robe de chambre et en pantoufles, et en jouant de la flûte à sa fenêtre. Personne ne semblait avoir le temps. Tout le monde se hâtait. Pas un chapeau qui n’eût une cocarde. Les femmes disaient : \emph{Nous sommes jolies sous le bonnet rouge}. Paris semblait plein d’un déménagement. Les marchands de bric-à-brac étaient encombrés de couronnes, de mitres, de sceptres en bois doré et de fleurs de lys, défroques des maisons royales. C’était la démolition de la monarchie qui passait. On voyait chez les fripiers des chapes et des rochets à vendre au \emph{décroche-moi-ça}. Aux Porcherons et chez Ramponneau, des hommes affublés de surplis et d’étoles, montés sur des ânes caparaçonnés de chasubles, se faisaient verser le vin du cabaret dans les ciboires des cathédrales. Rue Saint-Jacques, des paveurs, pieds nus, arrêtaient la brouette d’un colporteur qui offrait des chaussures à vendre, se cotisaient, et achetaient quinze paires de souliers qu’ils envoyaient à la Convention pour nos soldats. Les bustes de Franklin, de Rousseau, de Brutus, et il faut ajouter de Marat, abondaient ; au-dessous d’un de ces bustes de Marat, rue Cloche-Perce, était accroché sous verre, dans un cadre de bois noir, un réquisitoire  contre Malouet, avec faits à l’appui, et ces deux lignes en marge : « Ces détails m’ont été donnés par la maîtresse de Sylvain Bailly, bonne patriote qui a des bontés pour moi. — Signé : M{\scshape arat}. » Sur la place du Palais-Royal, l’inscription de la fontaine : \emph{Quantos effundit in usus !} était cachée par deux grandes toiles peintes à la détrempe, représentant l’une, Cahier de Gerville dénonçant à l’Assemblée nationale le signe de ralliement des « chiffonnistes » d’Arles ; l’autre, Louis XVI ramené de Varennes dans son carrosse royal, et sous ce carrosse une planche liée par des cordes portant à ses deux bouts deux grenadiers, la bayonnette au fusil. Peu de grandes boutiques étaient ouvertes ; des merceries et des bimbeloteries roulantes circulaient traînées par des femmes, éclairées par des chandelles, les suifs fondant sur les marchandises ; des boutiques en plein vent étaient tenues par des ex-religieuses en perruque blonde ; telle ravaudeuse, raccommodant des bas dans une échoppe, était une comtesse ; telle couturière était une marquise ; madame de Boufflers habitait un grenier d’où elle voyait son hôtel. Des crieurs couraient, offrant les « papiers-nouvelles ». On appelait \emph{écrouelleux} ceux qui cachaient leur menton dans leur cravate. Les chanteurs ambulants pullulaient. La foule huait Pitou, le chansonnier royaliste, vaillant d’ailleurs, car il fut emprisonné vingt-deux fois, et fut traduit devant le tribunal révolutionnaire pour s’être frappé le bas des reins en prononçant le mot \emph{civisme ;} voyant sa tête en danger, il s’écria : \emph{Mais c’est le contraire de ma tête qui  est coupable !} ce qui fit rire les juges et le sauva. Ce Pitou raillait la mode des noms grecs et latins ; sa chanson favorite était sur un savetier qu’il appelait \emph{Cujus}, et dont il appelait la femme \emph{Cujusdam}. On faisait des rondes de carmagnole ; on ne disait pas \emph{le cavalier et la dame}, on disait « le citoyen et la citoyenne ». On dansait dans les cloîtres en ruine, avec des lampions sur l’autel, à la voûte deux bâtons en croix portant quatre chandelles, et des tombes sous la danse. On portait des vestes bleu de tyran. On avait des épingles de chemise « au bonnet de la Liberté » faites de pierres blanches, bleues et rouges. La rue de Richelieu se nommait rue de la Loi ; le faubourg Saint-Antoine se nommait le faubourg de Gloire ; il y avait sur la place de la Bastille une statue de la Nature. On se montrait certains passants connus, Chatelet, Didier, Nicolas et Garnier-Delaunay, qui veillaient à la porte du menuisier Duplay ; Voullant, qui ne manquait pas un jour de guillotine et suivait les charretées de condamnés, et qui appelait cela « aller à la messe rouge » ; Montflabert, juré révolutionnaire et marquis, lequel se faisait appeler \emph{Dix-Août}. On regardait défiler les élèves de l’École militaire, qualifiés par les décrets de la Convention « aspirants à l’école de Mars », et par le peuple « pages de Robespierre ». On lisait les proclamations de Fréron, dénonçant les suspects du crime de « négociantisme ». Les « muscadins », ameutés aux portes des mairies, raillaient les mariages civils, s’attroupaient au passage de l’épousée et de l’époux, et  disaient : « mariés \emph{municipaliter} ». Aux Invalides, les statues des saints et des rois étaient coiffées du bonnet phrygien. On jouait aux cartes sur la borne des carrefours ; les jeux de cartes étaient, eux aussi, en pleine révolution, les rois étaient remplacés par les génies, les dames par les libertés, les valets par les égalités, et les as par les lois. On labourait les jardins publics ; la charrue travaillait aux Tuileries. A tout cela était mêlée, surtout dans les partis vaincus, on ne sait quelle hautaine lassitude de vivre ; un homme écrivait à Fouquier-Tinville : « Ayez la bonté de me délivrer de la vie. Voici mon adresse. » Champcenetz était arrêté pour s’être écrié en plein Palais-Royal : « A quand la révolution de Turquie ? Je voudrais voir la république à la Porte. Partout des journaux. Des garçons perruquiers crêpaient en public des perruques de femmes, pendant que le patron lisait à haute voix le \emph{Moniteur ;} d’autres commentaient au milieu des groupes, avec force gestes, le journal \emph{Entendons-nous}, de Dubois-Crancé, ou la \emph{Trompette du Père Bellerose}. Quelquefois les barbiers étaient en même temps charcutiers, et l’on voyait des jambons et des andouilles pendre à côté d’une poupée coiffée de cheveux d’or. Des marchands vendaient sur la voie publique « des vins d’émigrés » ; un marchand affichait des vins de \emph{cinquante-deux espèces ;} d’autres brocantaient des pendules en lyre et des sophas à la duchesse ; un perruquier avait pour enseigne ceci : « Je rase le clergé, je peigne la noblesse, j’accommode le tiers-état. » On allait se faire tirer les cartes  par Martin, au n\textsuperscript{o} 173 de la rue d’Anjou, ci-devant Dauphine. Le pain manquait, le charbon manquait, le savon manquait ; on voyait passer des bandes de vaches laitières arrivant des provinces. A la Vallée, l’agneau se vendait quinze francs la livre. Une affiche de la Commune assignait à chaque bouche une livre de viande par décade. On faisait queue aux portes des marchands ; une de ces queues est restée légendaire, elle allait de la porte d’un épicier de la rue du Petit-Carreau jusqu’au milieu de la rue Montorgueil. Faire queue, cela s’appelait « tenir la ficelle », à cause d’une longue corde que prenaient dans leur main, l’un derrière l’autre, ceux qui étaient à la file. Les femmes dans cette misère étaient vaillantes et douces. Elles passaient les nuits à attendre leur tour d’entrer chez le boulanger. Les expédients réussissaient à la révolution ; elle soulevait cette vaste détresse avec deux moyens périlleux, l’assignat et le maximum ; l’assignat était le levier, le maximum était le point d’appui. Cet empirisme sauva la France. L’ennemi, aussi bien l’ennemi de Coblentz que l’ennemi de Londres, agiotait sur l’assignat. Des filles allaient et venaient, offrant de l’eau de lavande, des jarretières et des cadenettes, et faisant l’agio ; il y avait les agioteurs du Perron de la rue Vivienne, en souliers crottés, en cheveux gras, en bonnet à poil à queue de renard, et les mayolets de la rue de Valois, en bottes cirées, le cure-dents à la bouche, le chapeau velu sur la tête, tutoyés par les filles. Le peuple leur faisait la chasse, ainsi qu’aux voleurs, que les royalistes appelaient  « citoyens actifs ». Du reste, très peu de vols. Un dénûment farouche, une probité stoïque. Les va-nu-pieds et les meurt-de-faim passaient, les yeux gravement baissés, devant les devantures des bijoutiers du Palais-Égalité. Dans une visite domiciliaire que fit la section Antoine chez Beaumarchais, une femme cueillit dans le jardin une fleur ; le peuple la souffleta. Le bois coûtait quatre cents francs, argent, la corde ; on voyait dans les rues des gens scier leur bois de lit ; l’hiver, les fontaines étaient gelées ; l’eau coûtait vingt sous la voie ; tout le monde se faisait porteur d’eau. Le louis d’or valait trois mille neuf cent cinquante francs. Une course en fiacre coûtait six cents francs. Après une journée de fiacre, on entendait ce dialogue : — Cocher, combien vous dois-je ? — Six mille livres. Une marchande d’herbe vendait pour vingt mille francs par jour. Un mendiant disait : \emph{Par charité, secourez-moi ! il me manque deux cent trente livres pour payer mes souliers.} A l’entrée des ponts, on voyait des colosses sculptés et peints par David que Mercier insultait : \emph{Énormes polichinelles de bois}, disait-il. Ces colosses figuraient le Fédéralisme et la Coalition terrassés. Aucune défaillance dans ce peuple. La sombre joie d’en avoir fini avec les trônes. Les volontaires affluaient, offrant leurs poitrines. Chaque rue donnait un bataillon. Les drapeaux des districts allaient et venaient, chacun avec sa devise. Sur le drapeau du district des Capucins on lisait : \emph{Nul ne nous fera la barbe}. Sur un autre : \emph{Plus de noblesse, que dans le cœur}. Sur tous les murs, des affiches, grandes,  petites, blanches, jaunes, vertes, rouges, imprimées, manuscrites, où on lisait ce cri : \emph{Vive la République !} Les petits enfants bégayaient \emph{Ça ira !}\par
Ces petits enfants, c’était l’immense avenir.\par
Plus tard, à la ville tragique succéda la ville cynique ; les rues de Paris ont eu deux aspects révolutionnaires très distincts, avant et après le 9 thermidor ; le Paris de Saint-Just fit place au Paris de Tallien ; et, ce sont là les continuelles antithèses de Dieu, immédiatement après le Sinaï, la Courtille apparut.\par
Un accès de folie publique, cela se voit. Cela s’était déjà vu quatrevingts ans auparavant. On sort de Louis XIV comme on sort de Robespierre, avec un grand besoin de respirer ; de là la Régence qui ouvre le siècle et le Directoire qui le termine. Deux saturnales après deux terrorismes. La France prend la clef des champs, hors du cloître puritain comme hors du cloître monarchique, avec une joie de nation échappée.\par
Après le 9 thermidor, Paris fut gai, d’une gaîté égarée. Une joie malsaine déborda. A la frénésie de mourir succéda la frénésie de vivre, et la grandeur s’éclipsa. On eut un Trimalcion qui s’appela Grimod de La Reynière ; on eut l’\emph{Almanach des Gourmands}. On dîna au bruit des fanfares dans les entre-sols du Palais-Royal, avec des orchestres de femmes battant du tambour et sonnant de la trompette ; « le rigaudinier », l’archet au poing, régna ; on soupa « à l’orientale » chez Méot, au milieu des cassolettes pleines de parfums. Le peintre Boze peignait ses filles, innocentes et charmantes têtes de seize ans, « en guillotinées »,  c’est-à-dire décolletées avec des chemises rouges. Aux danses violentes dans les églises en ruine succédèrent les bals de Ruggieri, de Luquet, de Wenzel, de Mauduit, de la Montansier ; aux graves citoyennes qui faisaient de la charpie succédèrent les sultanes, les sauvages, les nymphes ; aux pieds nus des soldats couverts de sang, de boue et de poussière succédèrent les pieds nus des femmes ornés de diamants ; en même temps que l’impudeur, l’improbité reparut ; il y eut en haut les fournisseurs et en bas « la petite pègre » ; un fourmillement de filous emplit Paris, et chacun dut veiller sur son « luc », c’est-à-dire sur son portefeuille ; un des passe-temps était d’aller voir, place du Palais-de-Justice, les voleuses au tabouret, on était obligé de leur lier les jupes ; à la sortie des théâtres, des gamins offraient des cabriolets en disant : \emph{Citoyen et citoyenne, il y a place pour deux ;} on ne criait plus \emph{le Vieux Cordelier} et \emph{l’Ami du peuple}, on criait \emph{la Lettre de Polichinelle} et \emph{la Pétition des Galopins ;} le marquis de Sade présidait la section des Piques, place Vendôme. La réaction était joviale et féroce ; les \emph{Dragons de la Liberté} de 92 renaissaient sous le nom de \emph{Chevaliers du Poignard}. En même temps surgit sur les tréteaux ce type, Jocrisse. On eut les « merveilleuses », et au delà des merveilleuses les « inconcevables » ; on jura par sa \emph{paole victimée} et par sa \emph{paole verte ;} on recula de Mirabeau jusqu’à Bobèche. C’est ainsi que Paris va et vient ; il est l’énorme pendule de la civilisation ; il touche tour à tour un pôle et l’autre, les Thermopyles et Gomorrhe. Après 93, la révolution traversa  une occultation singulière, le siècle sembla oublier de finir ce qu’il avait commencé, on ne sait quelle orgie s’interposa, prit le premier plan, fit reculer au second l’effrayante apocalypse, voila la vision démesurée, et éclata de rire après l’épouvante ; la tragédie disparut dans la parodie, et au fond de l’horizon une fumée de carnaval effaça vaguement Méduse.\par
Mais en 93, où nous sommes, les rues de Paris avaient encore tout l’aspect grandiose et farouche des commencements. Elles avaient leurs orateurs, Varlet qui promenait une baraque roulante du haut de laquelle il haranguait les passants ; leurs héros, dont un s’appelait « le capitaine des bâtons ferrés » ; leurs favoris, Guffroy, l’auteur du pamphlet \emph{Rougiff}. Quelques-unes de ces popularités étaient malfaisantes ; d’autres étaient saines. Une entre toutes était honnête et fatale ; c’était celle de Cimourdain.
 \subsubsection[{II. Cimourdain}]{II \\
Cimourdain}\phantomsection
\label{p2l1c2}
\noindent Cimourdain était une conscience pure, mais sombre. Il avait en lui l’absolu. Il avait été prêtre, ce qui est grave. L’homme peut, comme le ciel, avoir une sérénité noire ; il suffit que quelque chose fasse en lui la nuit. La prêtrise avait fait la nuit dans Cimourdain. Qui a été prêtre l’est.\par
Ce qui fait la nuit en nous peut laisser en nous les étoiles. Cimourdain était plein de vertus et de vérités, mais qui brillaient dans des ténèbres.\par
Son histoire était courte à faire. Il avait été curé de village et précepteur dans une grande maison ; puis un petit héritage lui était venu, et il s’était fait libre.\par
C’était par-dessus tout un opiniâtre. Il se servait de la méditation comme on se sert d’une tenaille ; il ne se croyait le droit de quitter une idée que lorsqu’il était arrivé au bout ; il pensait avec acharnement. Il savait toutes les langues de l’Europe et un peu les autres ; cet homme étudiait sans cesse, ce qui l’aidait à porter sa chasteté ; mais rien de plus dangereux qu’un tel refoulement.\par
Prêtre, il avait, par orgueil, hasard ou hauteur  d’âme, observé ses vœux ; mais il n’avait pu garder sa croyance. La science avait démoli sa foi ; le dogme s’était évanoui en lui. Alors, s’examinant, il s’était senti comme mutilé, et, ne pouvant se défaire prêtre, il avait travaillé à se refaire homme ; mais d’une façon austère ; on lui avait ôté la famille, il avait adopté la patrie ; on lui avait refusé une femme, il avait épousé l’humanité. Cette plénitude énorme, au fond, c’est le vide.\par
Ses parents, paysans, en le faisant prêtre, avaient voulu le faire sortir du peuple ; il était rentré dans le peuple.\par
Et il y était rentré passionnément. Il regardait les souffrants avec une tendresse redoutable. De prêtre il était devenu philosophe, et de philosophe athlète. Louis XV vivait encore que déjà Cimourdain se sentait vaguement républicain. De quelle république ? De la république de Platon peut-être, et peut-être aussi de la république de Dracon.\par
Défense lui était faite d’aimer, il s’était mis à haïr. Il haïssait les mensonges, la monarchie, la théocratie, son habit de prêtre ; il haïssait le présent ; et il appelait à grands cris l’avenir ; il le pressentait, il l’entrevoyait d’avance, il le devinait effrayant et magnifique ; il comprenait, pour le dénoûment de la lamentable misère humaine, quelque chose comme un vengeur qui serait un libérateur. Il adorait de loin la catastrophe.\par
En 1789, cette catastrophe était arrivée, et l’avait trouvé prêt. Cimourdain s’était jeté dans ce vaste renouvellement humain avec logique, c’est-à-dire, pour un esprit de sa trempe, inexorablement. La logique ne  s’attendrit pas. Il avait vécu les grandes années révolutionnaires, et avait eu le tressaillement de tous ces souffles, 89, la chute de la Bastille, la fin du supplice des peuples ; 90, le 19 juin, la fin de la féodalité ; 91, Varennes, la fin de la royauté ; 92, l’avènement de la république. Il avait vu se lever la révolution ; il n’était pas homme à avoir peur de cette géante ; loin de là, cette croissance de tout l’avait vivifié ; et, quoique déjà presque vieux, — il avait cinquante ans et un prêtre est plus vite vieux qu’un autre homme, — il s’était mis à croître, lui aussi. D’année en année, il avait regardé les événements grandir, et il avait grandi comme eux. Il avait craint d’abord que la révolution n’avortât, il l’observait, elle avait la raison et le droit, il exigeait qu’elle eût le succès ; et, à mesure qu’elle effrayait, il se sentait rassuré. Il voulait que cette Minerve, couronnée des étoiles de l’avenir, fût aussi Pallas, et eût pour bouclier le masque aux serpents. Il voulait que son œil divin pût au besoin jeter aux démons la lueur infernale, et leur rendre terreur pour terreur.\par
Il était arrivé ainsi à 93.\par
93 est la guerre de l’Europe contre la France et de la France contre Paris. Et qu’est-ce que la révolution ? C’est la victoire de la France sur l’Europe et de Paris sur la France. De là l’immensité de cette minute épouvantable, 93, plus grande que tout le reste du siècle.\par
Rien de plus tragique, l’Europe attaquant la France, et la France attaquant Paris. Drame qui a la stature de l’épopée.\par
93 est une année intense. L’orage est là dans toute  sa colère et dans toute sa grandeur. Cimourdain s’y sentait à l’aise. Ce milieu éperdu, sauvage et splendide convenait à son envergure. Cet homme avait, comme l’aigle de mer, un profond calme intérieur, avec le goût du risque au dehors. Certaines natures ailées, farouches et tranquilles sont faites pour les grands vents. Les âmes de tempête, cela existe.\par
Il avait une pitié à part, réservée seulement aux misérables. Devant l’espèce de souffrance qui fait horreur, il se dévouait. Rien ne lui répugnait. C’était là son genre de bonté. Il était hideusement secourable, et divinement. Il cherchait les ulcères pour les baiser. Les belles actions laides à voir sont les plus difficiles à faire ; il préférait celles-là. Un jour à l’Hôtel-Dieu, un homme allait mourir, étouffé par une tumeur à la gorge, abcès fétide, affreux, contagieux peut-être, et qu’il fallait vider sur-le-champ. Cimourdain était là ; il appliqua sa bouche à la tumeur, la pompa, recrachant à mesure que sa bouche était pleine, vida l’abcès, et sauva l’homme. Comme il portait encore à cette époque son habit de prêtre, quelqu’un lui dit : — Si vous faisiez cela au roi, vous seriez évêque. — Je ne le ferais pas au roi, répondit Cimourdain. L’acte et la réponse le firent populaire dans les quartiers sombres de Paris.\par
Si bien qu’il faisait de ceux qui souffrent, qui pleurent et qui menacent ce qu’il voulait. A l’époque des colères contre les accapareurs, colères si fécondes en méprises, ce fut Cimourdain qui, d’un mot, empêcha le pillage d’un bateau chargé de savon sur  le port Saint-Nicolas, et qui dissipa les attroupements furieux arrêtant les voitures à la barrière Saint-Lazare.\par
Ce fut lui qui, dix jours après le 10 août, mena le peuple jeter bas les statues des rois. En tombant elles tuèrent. Place Vendôme, une femme, Reine Violet, fut écrasée par Louis XIV au cou duquel elle avait mis une corde qu’elle tirait. Cette statue de Louis XIV avait été cent ans debout ; elle avait été érigée le 12 août 1692 ; elle fut renversée le 12 août 1792. Place de la Concorde, un nommé Guinguerlot ayant appelé les démolisseurs : canailles ! fut assommé sur le piédestal de Louis XV. La statue fut mise en pièces. Plus tard on en fit des sous. Le bras seul échappa ; c’était le bras droit que Louis XV étendait avec un geste d’empereur romain. Ce fut sur la demande de Cimourdain que le peuple donna et qu’une députation porta ce bras à Latude, l’homme enterré trente-sept ans à la Bastille. Quand Latude, le carcan au cou, la chaîne au ventre, pourrissait vivant au fond de cette prison par ordre de ce roi dont la statue dominait Paris, qui lui eût dit que cette prison tomberait, que cette statue tomberait, qu’il sortirait du sépulcre et que la monarchie y entrerait, que lui, le prisonnier, il serait le maître de cette main de bronze qui avait signé son écrou, et que de ce roi de boue il ne resterait que ce bras d’airain ?\par
Cimourdain était de ces hommes qui ont en eux une voix, et qui l’écoutent. Ces hommes-là semblent distraits ; point ; ils sont attentifs.\par
 Cimourdain savait tout et ignorait tout. Il savait tout de la science et ignorait tout de la vie. De là sa rigidité. Il avait les yeux bandés comme la Thémis d’Homère. Il avait la certitude aveugle de la flèche qui ne voit que le but et qui y va. En révolution rien de redoutable comme la ligne droite. Cimourdain allait devant lui, fatal.\par
Cimourdain croyait que, dans les genèses sociales, le point extrême est le terrain solide ; erreur propre aux esprits qui remplacent la raison par la logique. Il dépassait la Convention ; il dépassait la Commune ; il était de l’Évêché.\par
La réunion, dite l’Évêché, parce qu’elle tenait ses séances dans une salle du vieux palais épiscopal, était plutôt une complication d’hommes qu’une réunion. Là assistaient, comme à la Commune, ces spectateurs silencieux et significatifs qui avaient sur eux, comme dit Garat, « autant de pistolets que de poches ». L’Évêché était un pêle-mêle étrange ; pêle-mêle cosmopolite et parisien, ce qui ne s’exclut point, Paris étant le lieu où bat le cœur des peuples. Là était la grande incandescence plébéienne. Près de l’Évêché la Convention était froide et la Commune était tiède. L’Évêché était une de ces formations révolutionnaires, pareilles aux formations volcaniques ; l’Évêché contenait de tout, de l’ignorance, de la bêtise, de la probité, de l’héroïsme, de la colère, et de la police. Brunswick y avait des agents. Il y avait là des hommes dignes de Sparte et des hommes dignes du bagne. La plupart étaient forcenés et honnêtes. La  Gironde, par la bouche d’Isnard, président momentané de la Convention, avait dit un mot monstrueux : — \emph{Prenez garde, Parisiens. Il ne restera pas pierre sur pierre de votre ville, et l’on cherchera un jour la place où fut Paris}. — Ce mot avait créé l’Évêché. Des hommes, et, nous venons de le dire, des hommes de toutes nations, avaient senti la nécessité de se serrer autour de Paris. Cimourdain s’était rallié à ce groupe.\par
Ce groupe réagissait contre les réacteurs. Il était né de ce besoin public de violence qui est le côté redoutable et mystérieux des révolutions. Fort de cette force, l’Évêché s’était tout de suite fait sa part. Dans les commotions de Paris, c’était la Commune qui tirait le canon, c’était l’Évêché qui sonnait le tocsin.\par
Cimourdain croyait, dans son ingénuité implacable, que tout est équité au service du vrai ; ce qui le rendait propre à dominer les partis extrêmes. Les coquins le sentaient honnête, et étaient contents. Des crimes sont flattés d’être présidés par une vertu. Cela les gêne, et leur plaît. Palloy, l’architecte qui avait exploité la démolition de la Bastille, vendant ces pierres à son profit, et qui, chargé de badigeonner le cachot de Louis XVI, avait, par zèle, couvert le mur de barreaux, de chaînes et de carcans ; Gonchon, l’orateur suspect du faubourg Saint-Antoine, dont on a retrouvé plus tard les quittances ; Fournier, l’Américain qui, le 17 juillet, avait tiré sur Lafayette un coup de pistolet payé, disait-on, par Lafayette ; Henriot, qui sortait de Bicêtre, et qui avait été valet,  saltimbanque, voleur et espion avant d’être général et de pointer des canons sur la Convention ; La Reynie, l’ancien grand vicaire de Chartres, qui avait remplacé son bréviaire par le \emph{Père Duchêne ;} tous ces hommes étaient tenus en respect par Cimourdain, et, à de certains moments, pour empêcher les pires de broncher, il suffisait qu’ils sentissent en arrêt devant eux cette redoutable candeur convaincue. C’est ainsi que Saint-Just terrifiait Schneider. En même temps, la majorité de l’Évêché, composée surtout de pauvres et d’hommes violents, qui étaient bons, croyait en Cimourdain et le suivait. Il avait pour vicaire, ou pour aide de camp, comme on voudra, cet autre prêtre républicain, Danjou, que le peuple aimait pour sa haute taille et avait baptisé l’abbé Six-Pieds. Cimourdain eût mené où il eût voulu cet intrépide chef qu’on appelait le \emph{général la Pique}, et ce hardi Truchon, dit le Grand-Nicolas, qui avait voulu sauver madame de Lamballe, et qui lui avait donné le bras et fait enjamber les cadavres ; ce qui eût réussi sans la féroce plaisanterie du barbier Charlot.\par
La Commune surveillait la Convention, l’Évêché surveillait la Commune. Cimourdain, esprit droit et répugnant à l’intrigue, avait cassé plus d’un fil mystérieux dans la main de Pache, que Beurnonville appelait « l’homme noir ». Cimourdain, à l’Évêché, était de plain-pied avec tous. Il était consulté par Dobsent et Momoro. Il parlait espagnol à Gusman, italien à Pio, anglais à Arthur, flamand à Pereyra, allemand à l’Autrichien Proly, bâtard d’un prince. Il créait  l’entente entre ces discordances. De là une situation obscure et forte. Hébert le craignait.\par
Cimourdain avait, dans ces temps et dans ces groupes tragiques, la puissance des inexorables. C’était un impeccable qui se croit infaillible. Personne ne l’avait vu pleurer. Vertu inaccessible et glaciale. Il était l’effrayant homme juste.\par
Pas de milieu pour un prêtre dans la révolution. Un prêtre ne pouvait se donner à la prodigieuse aventure flagrante que pour les motifs les plus bas ou les plus hauts ; il fallait qu’il fût infâme ou qu’il fût sublime. Cimourdain était sublime, mais sublime dans l’isolement, dans l’escarpement, dans la lividité inhospitalière ; sublime dans un entourage de précipices. Les hautes montagnes ont cette virginité sinistre.\par
Cimourdain avait l’apparence d’un homme ordinaire, vêtu de vêtements quelconques, d’aspect pauvre. Jeune, il avait été tonsuré ; vieux, il était chauve. Le peu de cheveux qu’il avait étaient gris. Son front était large, et sur ce front il y avait pour l’observateur un signe. Cimourdain avait une façon de parler brusque, passionnée et solennelle, la voix brève, l’accent péremptoire, la bouche triste et amère, l’œil clair et profond, et sur tout le visage on ne sait quel air indigné.\par
Tel était Cimourdain.\par
Personne aujourd’hui ne sait son nom. L’histoire a de ces inconnus terribles.
 \subsubsection[{III. Un coin non trempé dans le styx}]{III \\
Un coin non trempé dans le styx}\phantomsection
\label{p2l1c3}
\noindent Un tel homme était-il un homme ? Le serviteur du genre humain pouvait-il avoir une affection ? N’était-il pas trop une âme pour être un cœur ? Cet embrassement énorme qui admettait tout et tous, pouvait-il se réserver à quelqu’un ? Cimourdain pouvait-il aimer ? Disons-le. Oui.\par
Étant jeune, et précepteur dans une maison presque princière, il avait eu un élève, fils et héritier de la maison, et il l’aimait. Aimer un enfant est si facile. Que ne pardonne-t-on pas à un enfant ? On lui pardonne d’être seigneur, d’être prince, d’être roi. L’innocence de l’âge fait oublier les crimes de la race ; la faiblesse de l’être fait oublier l’exagération du rang. Il est si petit qu’on lui pardonne d’être grand. L’esclave lui pardonne d’être le maître. Le vieillard nègre idolâtre le marmot blanc. Cimourdain avait pris en passion son élève. L’enfance a cela d’ineffable qu’on peut épuiser sur elle tous les amours. Tout ce qui pouvait aimer dans Cimourdain s’était abattu, pour ainsi dire, sur cet enfant ; ce doux être innocent était devenu une sorte de proie pour ce cœur condamné à  la solitude. Il l’aimait de toutes les tendresses à la fois, comme père, comme frère, comme ami, comme créateur. C’était son fils ; le fils, non de sa chair, mais de son esprit. Il n’était pas le père, et ce n’était pas son œuvre ; mais il était le maître, et c’était son chef-d’œuvre. De ce petit seigneur, il avait fait un homme. Qui sait ? un grand homme peut-être. Car tels sont les rêves. A l’insu de la famille, — a-t-on besoin de permission pour créer une intelligence, une volonté et une droiture ? — il avait communiqué au jeune vicomte, son élève, tout le progrès qu’il avait en lui ; il lui avait inoculé le virus redoutable de sa vertu ; il lui avait infusé dans les veines sa conviction, sa conscience, son idéal ; dans ce cerveau d’aristocrate, il avait versé l’âme du peuple.\par
L’esprit allaite, l’intelligence est une mamelle. Il y a analogie entre la nourrice qui donne son lait et le précepteur qui donne sa pensée. Quelquefois le précepteur est plus père que le père, de même que souvent la nourrice est plus mère que la mère.\par
Cette profonde paternité spirituelle liait Cimourdain à son élève. La seule vue de cet enfant l’attendrissait.\par
Ajoutons ceci : remplacer le père était facile, l’enfant n’en avait plus ; il était orphelin ; son père était mort, sa mère était morte ; il n’avait pour veiller sur lui qu’une grand’mère aveugle et un grand-oncle absent. La grand’mère mourut ; le grand-oncle, chef de la famille, homme d’épée et de grande seigneurie, pourvu de charges à la cour, fuyait le vieux donjon de famille, vivait à Versailles, allait aux armées, et  laissait l’orphelin seul dans le château solitaire. Le précepteur était donc le maître, dans toute l’acception du mot.\par
Ajoutons ceci encore : Cimourdain avait vu naître l’enfant qui avait été son élève. L’enfant, orphelin tout petit, avait eu une maladie grave. Cimourdain, en ce danger de mort, l’avait veillé jour et nuit ; c’est le médecin qui soigne, c’est le garde-malade qui sauve, et Cimourdain avait sauvé l’enfant. Non-seulement son élève lui avait dû l’éducation, l’instruction, la science ; mais il lui avait dû la convalescence et la santé ; non-seulement son élève lui devait de penser ; mais il lui devait de vivre. Ceux qui nous doivent tout, on les adore ; Cimourdain adorait cet enfant.\par
L’écart naturel de la vie s’était fait. L’éducation finie, Cimourdain avait dû quitter l’enfant devenu jeune homme. Avec quelle froide et inconsciente cruauté ces séparations-là se font ! Comme les familles congédient tranquillement le précepteur qui laisse sa pensée dans un enfant, et la nourrice qui y laisse ses entrailles ! Cimourdain, payé et mis dehors, était sorti du monde d’en haut et rentré dans le monde d’en bas ; la cloison entre les grands et les petits s’était refermée ; le jeune seigneur, officier de naissance et fait d’emblée capitaine, était parti pour une garnison quelconque ; l’humble précepteur, déjà au fond de son cœur prêtre insoumis, s’était hâté de redescendre dans cet obscur rez-de-chaussée de l’église qu’on appelait le bas clergé ; et Cimourdain avait perdu de vue son élève.\par
La révolution était venue ; le souvenir de cet être  dont il avait fait un homme avait continué de couver en lui, caché, mais non éteint, par l’immensité des choses publiques.\par
Modeler une statue et lui donner la vie, c’est beau ; modeler une intelligence et lui donner la vérité, c’est plus beau encore. Cimourdain était le Pygmalion d’une âme.\par
Un esprit peut avoir un enfant.\par
Cet élève, cet enfant, cet orphelin, était le seul être qu’il aimât sur la terre.\par
Mais, même dans une telle affection, un tel homme était-il vulnérable ?\par
On va le voir.\par
  \subsection[{Livre deuxième. Le cabaret de la rue du paon}]{Livre deuxième \\
Le cabaret de la rue du paon}\phantomsection
\label{p2l2}
\subsubsection[{I. Minos, Éaque et Rhadamante}]{I \\
Minos, Éaque et Rhadamante}\phantomsection
\label{p2l2c1}
\noindent Il y avait rue du Paon un cabaret qu’on appelait café. Ce café avait une arrière-chambre, aujourd’hui historique. C’était là que se rencontraient parfois, à peu près secrètement, des hommes tellement puissants et tellement surveillés qu’ils hésitaient à se parler en public. C’était là qu’avait été échangé, le 23 octobre 1792, un baiser fameux entre la Montagne et la Gironde. C’était là que Garat, bien qu’il n’en convienne pas dans ses \emph{Mémoires}, était venu aux renseignements dans cette nuit lugubre où, après avoir mis Clavière en sûreté rue de Beaune, il arrêta sa voiture sur le Pont-Royal pour écouter le tocsin.\par
 Le 28 juin 1793, trois hommes étaient réunis autour d’une table dans cette arrière-chambre. Leurs chaises ne se touchaient pas ; ils étaient assis chacun à un des côtés de la table, laissant vide le quatrième. Il était environ huit heures du soir ; il faisait jour encore dans la rue, mais il faisait nuit dans l’arrière-chambre, et un quinquet accroché au plafond, luxe d’alors, éclairait la table.\par
Le premier de ces trois hommes était pâle, jeune, grave, avec les lèvres minces et le regard froid. Il avait dans la joue un tic nerveux qui devait le gêner pour sourire. Il était poudré, ganté, brossé, boutonné. Son habit bleu clair ne faisait pas un pli. Il avait une culotte de nankin, des bas blancs, une haute cravate, un jabot plissé, des souliers à boucles d’argent. Les deux autres hommes étaient, l’un une espèce de géant, l’autre une espèce de nain. Le grand, débraillé dans un vaste habit de drap écarlate, le col nu dans une cravate dénouée tombant plus bas que le jabot, la veste ouverte avec des boutons arrachés, était botté de bottes à revers et avait les cheveux tout hérissés, quoiqu’on y vît un reste de coiffure et d’apprêt ; il y avait de la crinière dans sa perruque. Il avait la petite vérole sur la face, une ride de colère entre les sourcils, le pli de la bonté au coin de la bouche, les lèvres épaisses, les dents grandes, un poing de portefaix, l’œil éclatant. Le petit était un homme jaune qui, assis, semblait difforme ; il avait la tête renversée en arrière, les yeux injectés de sang, des plaques livides sur le visage, un mouchoir noué sur ses cheveux gras  et plats, pas de front, une bouche énorme et terrible. Il avait un pantalon à pied, de larges souliers, un gilet qui semblait avoir été de satin blanc, et par-dessus ce gilet une rouppe dans les plis de laquelle une ligne dure et droite laissait deviner un poignard.\par
Le premier de ces hommes s’appelait Robespierre, le second Danton, le troisième Marat.\par
Ils étaient seuls dans cette salle. Il y avait devant Danton un verre et une bouteille de vin couverte de poussière, rappelant la chope de bière de Luther, devant Marat une tasse de café, devant Robespierre des papiers.\par
Auprès des papiers on voyait un de ces lourds encriers de plomb, ronds et striés, que se rappellent ceux qui étaient écoliers au commencement de ce siècle. Une plume était jetée à côté de l’écritoire. Sur les papiers était posé un gros cachet de cuivre sur lequel on lisait \emph{Palloy fecit}, et qui figurait un petit modèle exact de la Bastille.\par
Une carte de France était étalée au milieu de la table.\par
A la porte et dehors se tenait le chien de garde de Marat, ce Laurent Basse, commissionnaire du numéro 18 de la rue des Cordeliers, qui, le 13 juillet, environ quinze jours après ce 28 juin, devait asséner un coup de chaise sur la tête d’une femme nommée Charlotte Corday, laquelle en ce moment-là était à Caen, songeant vaguement. Laurent Basse était le porteur d’épreuves de l’\emph{Ami du peuple}. Ce soir-là, amené par son maître au café de la rue du Paon, il avait la consigne de tenir fermée la salle où étaient Marat,  Danton et Robespierre, et de n’y laisser pénétrer personne, à moins que ce ne fût quelqu’un du comité de salut public, de la Commune ou de l’Évêché.\par
Robespierre ne voulait pas fermer la porte à Saint-Just, Danton ne voulait pas la fermer à Pache, Marat ne voulait pas la fermer à Gusman.\par
La conférence durait depuis longtemps déjà. Elle avait pour sujet les papiers étalés sur la table et dont Robespierre avait donné lecture. Les voix commençaient à s’élever. Quelque chose comme de la colère grondait entre ces trois hommes. Du dehors on entendait par moments des éclats de parole. A cette époque l’habitude des tribunes publiques semblait avoir créé le droit d’écouter. C’était le temps où l’expéditionnaire Fabricius Pâris regardait par le trou de la serrure ce que faisait le comité de salut public. Ce qui, soit dit en passant, ne fut pas inutile, car ce fut ce Pâris qui avertit Danton la nuit du 30 au 31 mars 1794. Laurent Basse avait appliqué son oreille contre la porte de l’arrière-salle où étaient Danton, Marat et Robespierre. Laurent Basse servait Marat, mais il était de l’Évêché.
 \subsubsection[{II. Magna testantur voce per umbras}]{II \\
Magna testantur voce per umbras}\phantomsection
\label{p2l2c2}
\noindent Danton venait de se lever ; il avait vivement reculé sa chaise.\par
— Écoutez, cria-t-il. Il n’y a qu’une urgence, la république en danger. Je ne connais qu’une chose, délivrer la France de l’ennemi. Pour cela tous les moyens sont bons. Tous ! tous ! tous ! Quand j’ai affaire à tous les périls, j’ai recours à toutes les ressources, et quand je crains tout, je brave tout. Ma pensée est une lionne. Pas de demi-mesures, pas de pruderie en révolution. Némésis n’est pas une bégueule. Soyons épouvantables, et utiles. Est-ce que l’éléphant regarde où il met sa patte ? Écrasons l’ennemi.\par
Robespierre répondit avec douceur :\par
— Je veux bien.\par
Et il ajouta :\par
— La question est de savoir où est l’ennemi.\par
— Il est dehors, et je l’ai chassé, dit Danton.\par
— Il est dedans, et je le surveille, dit Robespierre.\par
— Et je le chasserai encore, reprit Danton.\par
— On ne chasse pas l’ennemi du dedans.\par
— Qu’est-ce donc qu’on fait ?\par
 — On l’extermine.\par
— J’y consens, dit à son tour Danton.\par
Et il reprit :\par
— Je vous dis qu’il est dehors, Robespierre.\par
— Danton, je vous dis qu’il est dedans.\par
— Robespierre, il est à la frontière.\par
— Danton, il est en Vendée.\par
— Calmez-vous, dit une troisième voix, il est partout ; et vous êtes perdus.\par
C’était Marat qui parlait.\par
Robespierre regarda Marat et repartit tranquillement :\par
— Trêve aux généralités. Je précise. Voici des faits.\par
— Pédant ! grommela Marat.\par
Robespierre posa la main sur les papiers étalés devant lui et continua :\par
— Je viens de vous lire les dépêches de Prieur de la Marne. Je viens de vous communiquer les renseignements donnés par ce Gélambre. Danton, écoutez, la guerre étrangère n’est rien, la guerre civile est tout. La guerre étrangère, c’est une écorchure qu’on a au coude ; la guerre civile, c’est l’ulcère qui vous mange le foie. De tout ce que je viens de vous lire, il résulte ceci : la Vendée, jusqu’à ce jour éparse entre plusieurs chefs, est au moment de se concentrer. Elle va désormais avoir un capitaine unique...\par
— Un brigand central, murmura Danton.\par
— C’est, poursuivit Robespierre, l’homme débarqué près de Pontorson le 2 juin. Vous avez vu ce qu’il est. Remarquez que ce débarquement coïncide avec  l’arrestation des représentants en mission, Prieur de la Côte-d’Or et Romme, à Bayeux, par ce district traître du Calvados, le 2 juin, le même jour.\par
— Et leur translation au château de Caen, dit Danton.\par
Robespierre reprit :\par
— Je continue de résumer les dépêches. La guerre de forêt s’organise sur une vaste échelle. En même temps une descente anglaise se prépare ; Vendéens et Anglais, c’est Bretagne avec Bretagne. Les hurons du Finistère parlent la même langue que les topinambous de Cornouailles. J’ai mis sous vos yeux une lettre interceptée de Puisaye où il est dit que « vingt mille habits rouges distribués aux insurgés en feront lever cent mille ». Quand l’insurrection paysanne sera complète, la descente anglaise se fera. Voici le plan. Suivez-le sur la carte.\par
Robespierre posa le doigt sur la carte, et poursuivit :\par
— Les Anglais ont le choix du point de descente, de Cancale à Paimpol. Craig préférerait la baie de Saint-Brieuc, Cornwallis la baie de Saint-Cast. C’est un détail. La rive gauche de la Loire est gardée par l’armée vendéenne rebelle, et, quant aux vingt-huit lieues à découvert entre Ancenis et Pontorson, quarante paroisses normandes ont promis leur concours. La descente se fera sur trois points, Plérin, Iffiniac et Pléneuf ; de Plérin on ira à Saint-Brieuc, et de Pléneuf à Lamballe ; le deuxième jour on gagnera Dinan où il y a neuf cents prisonniers anglais, et l’on occupera  en même temps Saint-Jouan et Saint-Méen ; on y laissera de la cavalerie ; le troisième jour, deux colonnes se dirigeront l’une de Jouan sur Bédée, l’autre de Dinan sur Becherel qui est une forteresse naturelle, et où l’on établira deux batteries ; le quatrième jour, on est à Rennes. Rennes, c’est la clef de la Bretagne. Qui a Rennes a tout. Rennes prise, Châteauneuf et Saint-Malo tombent. Il y a à Rennes un million de cartouches et cinquante pièces d’artillerie de campagne...\par
— Qu’ils rafleraient, murmura Danton.\par
Robespierre continua :\par
— Je termine. De Rennes, trois colonnes se jetteront l’une sur Fougères, l’autre sur Vitré, l’autre sur Redon. Comme les ponts sont coupés, les ennemis se muniront, vous avez vu ce fait précisé, de pontons et de madriers, et ils auront des guides pour les points guéables à la cavalerie. De Fougères on rayonnera sur Avranches, de Redon sur Ancenis, de Vitré sur Laval. Nantes se rendra, Brest se rendra. Redon donne tout le cours de la Vilaine, Fougères donne la route de Normandie, Vitré donne la route de Paris. Dans quinze jours, on aura une armée de brigands de trois cent mille hommes, et toute la Bretagne sera au roi de France.\par
— C’est-à-dire au roi d’Angleterre, dit Danton.\par
— Non. Au roi de France.\par
Et Robespierre ajouta :\par
— Le roi de France est pire. Il faut quinze jours pour chasser l’étranger, et dix-huit cents ans pour éliminer la monarchie.\par
 Danton, qui s’était rassis, mit ses coudes sur la table et sa tête dans ses mains, rêveur.\par
— Vous voyez le péril, dit Robespierre. Vitré donne la route de Paris aux Anglais.\par
Danton redressa le front et abattit ses deux grosses mains crispées sur la carte, comme sur une enclume.\par
— Robespierre, est-ce que Verdun ne donnait pas la route de Paris aux Prussiens ?\par
— Eh bien ?\par
— Eh bien, on chassera les Anglais comme on a chassé les Prussiens.\par
Et Danton se leva de nouveau.\par
Robespierre posa sa main froide sur le poing fiévreux de Danton.\par
— Danton, la Champagne n’était pas pour les Prussiens, et la Bretagne est pour les Anglais. Reprendre Verdun, c’est de la guerre étrangère ; reprendre Vitré, c’est de la guerre civile.\par
Et Robespierre murmura avec un accent froid et profond :\par
— Sérieuse différence.\par
Il reprit :\par
— Rasseyez-vous, Danton, et regardez la carte au lieu de lui donner des coups de poing.\par
Mais Danton était tout à sa pensée.\par
— Voilà qui est fort ! s’écria-t-il, de voir la catastrophe à l’ouest quand elle est à l’est. Robespierre, je vous accorde que l’Angleterre se dresse sur l’Océan ; mais l’Espagne se dresse aux Pyrénées, mais  l’Italie se dresse aux Alpes, mais l’Allemagne se dresse sur le Rhin. Et le grand ours russe est au fond. Robespierre, le danger est un cercle et nous sommes dedans. A l’extérieur la coalition, à l’intérieur la trahison. Au midi Servant entre-bâille la porte de la France au roi d’Espagne, au nord Dumouriez passe à l’ennemi. Au reste il avait toujours moins menacé la Hollande que Paris. Nerwinde efface Jemmapes et Valmy. Le philosophe Rabaut Saint-Étienne, traître comme un protestant qu’il est, correspond avec le courtisan Montesquiou. L’armée est décimée. Pas un bataillon qui ait maintenant plus de quatre cents hommes ; le vaillant régiment de Deux-Ponts est réduit à cent cinquante hommes ; le camp de Pamars est livré ; il ne reste plus à Givet que cinq cents sacs de farine ; nous rétrogradons sur Landau ; Wurmser presse Kléber ; Mayence succombe vaillamment, Condé lâchement. Valenciennes aussi. Ce qui n’empêche pas Chancel qui défend Valenciennes et le vieux Féraud qui défend Condé d’être deux héros, aussi bien que Meunier qui défendait Mayence. Mais tous les autres trahissent. Dharville trahit à Aix-la-Chapelle, Mouton trahit à Bruxelles, Valence trahit à Bréda, Neuilly trahit à Limbourg, Miranda trahit à Maëstricht ; Stengel, traître, Lanoue, traître, Ligonnier, traître, Menou, traître, Dillon, traître ; monnaie hideuse de Dumouriez. Il faut des exemples. Les contre-marches de Custine me sont suspectes ; je soupçonne Custine de préférer la prise lucrative de Francfort à la prise utile de Coblentz. Francfort peut payer quatre millions  de contributions de guerre, soit. Qu’est-ce que cela à côté du nid des émigrés écrasé ? Trahison, dis-je. Meunier est mort le 13 juin. Voilà Kléber seul. En attendant, Brunswick grossit et avance. Il arbore le drapeau allemand sur toutes les places françaises qu’il prend. Le margrave de Brandebourg est aujourd’hui l’arbitre de l’Europe ; il empoche nos provinces ; il s’adjugera la Belgique, vous verrez ; on dirait que c’est pour Berlin que nous travaillons ; si cela continue, et si nous n’y mettons ordre, la révolution française se sera faite au profit de Postdam, elle aura eu pour unique résultat d’agrandir le petit état de Frédéric II, et nous aurons tué le roi de France pour le roi de Prusse.\par
Et Danton, terrible, éclata de rire.\par
Le rire de Danton fit sourire Marat.\par
— Vous avez chacun votre dada ; vous, Danton, la Prusse ; vous, Robespierre, la Vendée. Je vais préciser, moi aussi. Vous ne voyez pas le vrai péril ; le voici : les cafés et les tripots. Le café de Choiseul est jacobin, le café Patin est royaliste, le café du Rendez-vous attaque la garde nationale, le café de la Porte-Saint-Martin la défend, le café de la Régence est contre Brissot, le café Corazza est pour, le café Procope jure par Diderot, le café du Théâtre-Français jure par Voltaire, à la Rotonde on déchire les assignats, les cafés Saint-Marceau sont en fureur, le café Manouri agite la question des farines, au café de Foy tapages et gourmades, au Perron bourdonnement des frêlons de finances. Voilà ce qui est sérieux.\par
 Danton ne riait plus. Marat souriait toujours. Sourire de nain pire qu’un rire de colosse.\par
— Vous moquez-vous, Marat ? gronda Danton.\par
Marat eut ce mouvement de hanche convulsif, qui était célèbre. Son sourire s’était effacé.\par
— Ah ! je vous retrouve, citoyen Danton. C’est bien vous qui en pleine Convention m’avez appelé « l’individu Marat ». Écoutez. Je vous pardonne. Nous traversons un moment imbécile. Ah ! je me moque ! En effet, quel homme suis-je ? J’ai dénoncé Chazot, j’ai dénoncé Pétion, j’ai dénoncé Kersaint, j’ai dénoncé Moreton, j’ai dénoncé Dufriche-Valazé, j’ai dénoncé Ligonnier, j’ai dénoncé Menou, j’ai dénoncé Banneville, j’ai dénoncé Gensonné, j’ai dénoncé Biron, j’ai dénoncé Lidon et Chambon ; ai-je eu tort ? je flaire la trahison dans le traître, et je trouve utile de dénoncer le criminel avant le crime. J’ai l’habitude de dire la veille ce que vous autres vous dites le lendemain. Je suis l’homme qui a proposé à l’assemblée un plan complet de législation criminelle. Qu’ai-je fait jusqu’à présent ? J’ai demandé qu’on instruise les sections afin de les discipliner à la révolution, j’ai fait lever les scellés des trente-deux cartons, j’ai réclamé les diamants déposés dans les mains de Roland, j’ai prouvé que les brissotins avaient donné au comité de sûreté générale des mandats d’arrêt en blanc, j’ai signalé les omissions du rapport de Lindet sur les crimes de Capet, j’ai voté le supplice du tyran dans les vingt-quatre heures, j’ai défendu les bataillons le Mauconseil et le Républicain, j’ai empêché la lecture de la lettre de Narbonne et de  Malhouet, j’ai fait une motion pour les soldats blessés, j’ai fait supprimer la commission des six, j’ai pressenti dans l’affaire de Mons la trahison de Dumouriez, j’ai demandé qu’on prît cent mille parents d’émigrés comme otages pour les commissaires livrés à l’ennemi, j’ai proposé de déclarer traître tout représentant qui passerait les barrières, j’ai démasqué la faction rolandine dans les troubles de Marseille, j’ai insisté pour qu’on mît à prix la tête d’Égalité fils, j’ai défendu Bouchotte, j’ai voulu l’appel nominal pour chasser Isnard du fauteuil, j’ai fait déclarer que les Parisiens ont bien mérité de la patrie ; c’est pourquoi je suis traité de pantin par Louvet, le Finistère demande qu’on m’expulse, la ville de Loudun souhaite qu’on m’exile, la ville d’Amiens désire qu’on me mette une muselière, Cobourg veut qu’on m’arrête, et Lecointe-Puyraveau propose à la Convention de me décréter fou. Ah çà ! citoyen Danton, pourquoi m’avez-vous fait venir à votre conciliabule, si ce n’est pour avoir mon avis ? Est-ce que je vous demandais d’en être ? loin de là. Je n’ai aucun goût pour les tête-à-tête avec des contre-révolutionnaires tels que Robespierre et vous. Du reste, je devais m’y attendre, vous ne m’avez pas compris ; pas plus vous que Robespierre, pas plus Robespierre que vous. Il n’y a donc pas d’homme d’état ici ? Il faut donc vous faire épeler la politique, il faut donc vous mettre les points sur les \emph{i}. Ce que je vous ai dit voulait dire ceci : Vous vous trompez tous les deux. Le danger n’est ni à Londres, comme le croit Robespierre, ni à Berlin, comme le croit Danton ;  il est à Paris. Il est dans l’absence d’unité, dans le droit qu’a chacun de tirer de son côté, à commencer par vous deux, dans la mise en poussière des esprits, dans l’anarchie des volontés...\par
— L’anarchie ! interrompit Danton, qui la fait, si ce n’est vous ?\par
Marat ne s’arrêta pas.\par
— Robespierre, Danton, le danger est dans ce tas de cafés, dans ce tas de brelans, dans ce tas de clubs, club des Noirs, club des Fédérés, club des Dames, club des Impartiaux, qui date de Clermont-Tonnerre et qui a été le club monarchique de 1790, cercle social imaginé par le prêtre Claude Fauchet, club des Bonnets de laine fondé par le gazetier Prudhomme, \emph{et cœtera ;} sans compter votre club des Jacobins, Robespierre, et votre club des Cordeliers, Danton. Le danger est dans la famine, qui fait que le porte-sacs Blin a accroché à la lanterne de l’Hôtel-de-ville le boulanger du marché Palu, François Denis, et dans la justice, qui a pendu le porte-sacs Blin pour avoir pendu le boulanger Denis. Le danger est dans le papier-monnaie qu’on déprécie. Rue du Temple, un assignat de cent francs est tombé à terre, et un passant, un homme du peuple, a dit : \emph{Il ne vaut pas la peine d’être ramassé}. Les agioteurs et les accapareurs, voilà le danger. Arborer le drapeau noir à l’Hôtel-de-ville, la belle avance ! Vous arrêtez le baron de Trenck, cela ne suffit pas. Tordez-moi le cou à ce vieil intrigant de prison. Vous croyez vous tirer d’affaire parce que le président de la Convention pose une couronne civique  sur la tête de Labertèche, qui a reçu quarante et un coups de sabre à Jemmapes, et dont Chénier se fait le cornac ? Comédies et batelages. Ah ! vous ne regardez pas Paris ! Ah ! vous cherchez le danger loin, quand il est près ! A quoi vous sert votre police, Robespierre ? Car vous avez vos espions, Payan, à la Commune, Coffïnhal, au tribunal révolutionnaire, David, au comité de sûreté générale, Couthon, au comité de salut public. Vous voyez que je suis bien informé. Eh bien, sachez ceci : le danger est sur vos têtes, le danger est sous vos pieds ; on conspire, on conspire, on conspire ; les passants dans les rues s’entre-lisent les journaux et se font des signes de tête ; six mille hommes, sans cartes de civisme, émigrés rentrés, muscadins et mathevons, sont cachés dans les caves et dans les greniers et dans les galeries de bois du Palais-Royal ; on fait queue chez les boulangers ; les bonnes femmes, sur le pas des portes, joignent les mains et disent : Quand aura-t-on la paix ? Vous avez beau aller vous enfermer, pour être entre vous, dans la salle du conseil exécutif, on sait tout ce que vous y dites ; et la preuve, Robespierre, c’est que voici les paroles que vous avez dites hier soir à Saint-Just : « Barbaroux commence à prendre du ventre, cela va le gêner dans sa fuite. » Oui, le danger est partout, et surtout au centre, à Paris. Les ci-devant complotent, les patriotes vont pieds nus, les aristocrates arrêtés le 9 mars sont déjà relâchés, les chevaux de luxe qui devraient être attelés aux canons sur la frontière nous éclaboussent dans les rues, le pain de  quatre livres vaut trois francs douze sous, les théâtres jouent des pièces impures, et Robespierre fera guillotiner Danton.\par
— Ouiche ! dit Danton.\par
Robespierre regardait attentivement la carte.\par
— Ce qu’il faut, cria brusquement Marat, c’est un dictateur. Robespierre, vous savez que je veux un dictateur.\par
Robespierre releva la tête.\par
— Je sais, Marat, vous ou moi.\par
— Moi ou vous, dit Marat.\par
Danton grommela entre ses dents :\par
— La dictature, touchez-y !\par
Marat vit le froncement de sourcil de Danton.\par
— Tenez, reprit-il. Un dernier effort. Mettons-nous d’accord. La situation en vaut la peine. Ne nous sommes-nous déjà pas mis d’accord pour la journée du 31 mai ? La question d’ensemble est plus grave encore que le girondinisme qui est une question de détail. Il y a du vrai dans ce que vous dites ; mais le vrai, tout le vrai, le vrai vrai, c’est ce que je dis. Au midi, le fédéralisme ; à l’ouest, le royalisme ; à Paris, le duel de la Convention et de la Commune ; aux frontières, la reculade de Custine et la trahison de Dumouriez. Qu’est-ce que tout cela ? Le démembrement. Que nous faut-il ? L’unité. Là est le salut. Mais hâtons-nous. Il faut que Paris prenne le gouvernement de la révolution. Si nous perdons une heure, demain les Vendéens peuvent être à Orléans, et les Prussiens à Paris. Je vous accorde ceci, Danton, je vous concède  cela, Robespierre. Soit. Eh bien, la conclusion, c’est la dictature. Prenons la dictature. A nous trois nous représentons la révolution. Nous sommes les trois têtes de Cerbère. De ces trois têtes, l’une parle, c’est vous, Robespierre ; l’autre rugit, c’est vous, Danton...\par
— L’autre mord, dit Danton, c’est vous, Marat.\par
— Toutes trois mordent, dit Robespierre.\par
Il y eut un silence. Puis le dialogue, plein de secousses sombres, recommença.\par
— Écoutez, Marat, avant de s’épouser, il faut se connaître. Comment avez-vous su le mot que j’ai dit hier à Saint-Just ?\par
— Ceci me regarde, Robespierre.\par
— Marat !\par
— C’est mon devoir de m’éclairer, et c’est mon affaire de me renseigner.\par
— Marat !\par
— J’aime à savoir.\par
— Marat !\par
— Robespierre, je sais ce que vous dites à Saint-Just, comme je sais ce que Danton dit à Lacroix ; comme je sais ce qui se passe quai des Théatins, à l’hôtel de Labriffe, repaire où se rendent les nymphes de l’émigration ; comme je sais ce qui se passe dans la maison des Thilles, près Gonesse, qui est à Valmerange, l’ancien administrateur des postes, où allaient jadis Maury et Cazalès, où sont allés depuis Sieyès et Vergniaud, et où, maintenant, on va une fois par semaine.\par
 En prononçant cet \emph{on}, Marat regarda Danton.\par
Danton s’écria :\par
— Si j’avais deux liards de pouvoir, ce serait terrible.\par
Marat poursuivit :\par
— Je sais ce que vous dites, Robespierre, comme je sais ce qui se passait à la tour du Temple quand on y engraissait Louis XVI, si bien que, seulement dans le mois de septembre, le loup, la louve et les louveteaux ont mangé quatrevingt-six paniers de pêches. Pendant ce temps-là le peuple est affamé. Je sais cela, comme je sais que Roland a été caché dans un logis donnant sur une arrière-cour, rue de la Harpe ; comme je sais que six cents des piques du 14 juillet avaient été fabriquées par Faure, serrurier du duc d’Orléans ; comme je sais ce qu’on fait chez la Saint-Hilaire, maîtresse de Sillery ; les jours de bal, c’est le vieux Sillery qui frotte lui-même, avec de la craie, les parquets du salon jaune de la rue Neuve-des-Mathurins ; Buzot et Kersaint y dînaient. Saladin y a dîné le 27, et avec qui, Robespierre ? Avec votre ami Lasource.\par
— Verbiage, murmura Robespierre. Lasource n’est pas mon ami.\par
Et il ajouta, pensif :\par
— En attendant il y a à Londres dix-huit fabriques de faux assignats.\par
Marat continua d’une voix tranquille, mais avec un léger tremblement, qui était effrayant :\par
— Vous êtes la faction des importants. Oui, je  sais tout, malgré ce que Saint-Just appelle \emph{le silence d’état...}\par
Marat souligna ce mot par l’accent, regarda Robespierre, et poursuivit :\par
— Je sais ce qu’on dit à votre table les jours où Lebas invite David à venir manger la cuisine faite par sa promise, Élisabeth Duplay, votre future belle-sœur, Robespierre. Je suis l’œil énorme du peuple, et, du fond de ma cave, je regarde. Oui, je vois, oui, j’entends, oui, je sais. Les petites choses vous suffisent. Vous vous admirez. Robespierre se fait contempler par sa madame de Chalabre, la fille de ce marquis de Chalabre qui fit le whist avec Louis XV le soir de l’exécution de Damiens. Oui, on porte haut la tête. Saint-Just habite une cravate. Legendre est correct ; lévite neuve et gilet blanc, et un jabot, pour faire oublier son tablier. Robespierre s’imagine que l’histoire voudra savoir qu’il avait une redingote olive à la Constituante et un habit bleu-ciel à la Convention. Il a son portrait sur tous les murs de sa chambre...\par
Robespierre interrompit d’une voix plus calme encore que celle de Marat.\par
— Et vous, Marat, vous avez le vôtre dans tous les égouts.\par
Ils continuèrent sur un ton de causerie dont la lenteur accentuait la violence des répliques et des ripostes, et ajoutait on ne sait quelle ironie à la menace.\par
— Robespierre, vous avez qualifié ceux qui veulent le renversement des trônes, \emph{les Don Quichottes du genre humain}.\par
 — Et vous, Marat, après le 4 août, dans votre numéro 559 de \emph{l’Ami du Peuple}, ah ! j’ai retenu le chiffre, c’est utile, vous avez demandé qu’on rendît aux nobles leurs titres. Vous avez dit : \emph{Un duc est toujours un duc}.\par
— Robespierre, dans la séance du 7 décembre, vous avez défendu la femme Roland contre Viard.\par
— De même que mon frère vous a défendu, Marat, quand on vous a attaqué aux Jacobins. Qu’est-ce que cela prouve ? rien.\par
— Robespierre, on connaît le cabinet des Tuileries où vous avez dit à Garat : \emph{Je suis las de la révolution}.\par
— Marat, c’est ici, dans ce cabaret, que, le 29 octobre, vous avez embrassé Barbaroux.\par
— Robespierre, vous avez dit à Buzot : \emph{La république, qu’est-ce que cela ?}\par
— Marat, c’est dans ce cabaret que vous avez invité à déjeuner trois Marseillais par compagnie.\par
— Robespierre, vous vous faites escorter d’un fort de la halle armé d’un bâton.\par
— Et vous, Marat, la veille du 10 août, vous avez demandé à Buzot de vous aider à fuir à Marseille, déguisé en jockey.\par
— Pendant les justices de septembre, vous vous êtes caché, Robespierre.\par
— Et vous, Marat, vous vous êtes montré.\par
— Robespierre, vous avez jeté à terre le bonnet rouge.\par
— Oui, quand un traître l’arborait. Ce qui pare Dumouriez souille Robespierre.\par
 — Robespierre, vous avez refusé, pendant le passage des soldats de Chateauvieux, de couvrir d’un voile la tête de Louis XVI.\par
— J’ai fait mieux que lui voiler la tête, je la lui ai coupée.\par
Danton intervint, mais comme l’huile intervient dans le feu.\par
— Robespierre, Marat, dit-il, calmez-vous.\par
Marat n’aimait pas à être nommé le second. Il se retourna.\par
— De quoi se mêle Danton ? dit-il.\par
Danton bondit.\par
— De quoi je me mêle ? De ceci. Qu’il ne faut pas de fratricide ; qu’il ne faut pas de lutte entre deux hommes qui servent le peuple ; que c’est assez de la guerre étrangère, que c’est assez de la guerre civile, et que ce serait trop de la guerre domestique ; que c’est moi qui ai fait la révolution, et que je ne veux pas qu’on la défasse. Voilà de quoi je me mêle.\par
Marat répondit sans élever la voix.\par
— Mêlez-vous de rendre vos comptes.\par
— Mes comptes ! cria Danton. Allez les demander aux défilés de l’Argonne, à la Champagne délivrée, à la Belgique conquise, aux armées où j’ai été quatre fois déjà offrir ma poitrine à la mitraille ! allez les demander à la place de la Révolution, à l’échafaud du 21 janvier, au trône jeté à terre, à la guillotine, cette veuve...\par
Marat interrompit Danton.\par
 — La guillotine est une vierge ; on se couche sur elle, on ne la féconde pas.\par
— Qu’en savez-vous ? répliqua Danton, je la féconderais, moi !\par
— Nous verrons, dit Marat.\par
Et il sourit.\par
Danton vit ce sourire.\par
— Marat, cria-t-il, vous êtes l’homme caché, moi je suis l’homme du grand air et du grand jour. Je hais la vie reptile. Être cloporte ne me va pas. Vous habitez une cave ; moi j’habite la rue. Vous ne communiquez avec personne ; moi, quiconque passe peut me voir et me parler.\par
— Joli garçon, voulez-vous monter chez moi ? grommela Marat.\par
Et cessant de sourire, il reprit d’un accent péremptoire :\par
— Danton, rendez compte des trente-trois mille écus, argent sonnant, que Montmorin vous a payés au nom du roi, sous prétexte de vous indemniser de votre charge de procureur au Châtelet.\par
— J’étais du 14 juillet, dit Danton avec hauteur.\par
— Et le garde-meuble ? et les diamants de la couronne ?\par
— J’étais du 6 octobre.\par
— Et les vols de votre \emph{alter ego} Lacroix en Belgique ?\par
— J’étais du 20 juin.\par
— Et les prêts faits à la Montansier ?\par
— Je poussais le peuple au retour de Varennes.\par
 — Et la salle de l’Opéra qu’on bâtit avec de l’argent fourni par vous ?\par
— J’ai armé les sections de Paris.\par
— Et les cent mille livres de fonds secrets du ministère de la justice ?\par
— J’ai fait le 10 août.\par
— Et les deux millions de dépenses secrètes de l’Assemblée, dont vous avez pris le quart ?\par
— J’ai arrêté l’ennemi en marche et barré le passage aux rois coalisés.\par
— Prostitué ! dit Marat.\par
Danton se dressa, effrayant.\par
— Oui, cria-t-il, je suis une fille publique, j’ai vendu mon ventre, mais j’ai sauvé le monde.\par
Robespierre s’était remis à se ronger les ongles. Il ne pouvait, lui, ni rire, ni sourire. Le rire, éclair de Danton, et le sourire, piqûre de Marat, lui manquaient.\par
Danton reprit :\par
— Je suis comme l’océan ; j’ai mon flux et mon reflux ; à mer basse on voit mes bas-fonds, à mer haute on voit mes flots.\par
— Votre écume, dit Marat.\par
— Ma tempête, dit Danton.\par
En même temps que Danton, Marat s’était levé. Lui aussi éclata. La couleuvre devint subitement dragon.\par
— Ah ! cria-t-il, ah ! Robespierre ! ah ! Danton ! vous ne voulez pas m’écouter ! Eh bien, je vous le dis, vous êtes perdus. Votre politique aboutit à des impossibilités  d’aller plus loin ; vous n’avez plus d’issue ; et vous faites des choses qui ferment devant vous toutes les portes, excepté celle du tombeau.\par
— C’est notre grandeur, dit Danton.\par
Et il haussa les épaules.\par
Marat continua :\par
— Danton, prends garde. Vergniaud aussi a la bouche large et les lèvres épaisses et les sourcils en colère, Vergniaud aussi est grêlé comme Mirabeau et comme toi, cela n’a pas empêché le 31 mai. Ah ! tu hausses les épaules. Quelquefois hausser les épaules fait tomber la tête. Danton, je te le dis, ta grosse voix, ta cravate lâche, tes bottes molles, tes petits soupers, tes grandes poches, cela regarde Louisette.\par
Louisette était le nom d’amitié que Marat donnait à la guillotine.\par
Il poursuivit :\par
— Et quant à toi, Robespierre, tu es un modéré, mais cela ne te servira de rien. Va, poudre-toi, coiffe-toi, brosse-toi, fais le faraud, aie du linge, sois pincé, frisé, calamistré, tu n’en iras pas moins en place de Grève, lis la déclaration de Brunswick, tu n’en seras pas moins traité comme le régicide Damiens, et tu es tiré à quatre épingles en attendant que tu sois tiré à quatre chevaux.\par
— Écho de Coblentz ! dit Robespierre entre ses dents.\par
— Robespierre, je ne suis l’écho de rien, je suis le cri de tout. Ah ! vous êtes jeunes, vous. Quel âge as-tu, Danton ? trente-quatre ans. Quel âge as-tu,  Robespierre ? trente-trois ans. Eh bien, moi, j’ai toujours vécu, je suis la vieille souffrance humaine, j’ai six mille ans.\par
— C’est vrai, répliqua Danton, depuis six mille ans, Caïn s’est conservé dans la haine comme le crapaud dans la pierre, le bloc se casse, Caïn saute parmi les hommes, et c’est Marat.\par
— Danton ! cria Marat. Et une lueur livide apparut dans ses yeux.\par
— Eh bien quoi ? dit Danton.\par
Ainsi parlaient ces trois hommes formidables. Querelle de tonnerres.
 \subsubsection[{III. Tressaillement des fibres profondes}]{III \\
Tressaillement des fibres profondes}\phantomsection
\label{p2l2c3}
\noindent Le dialogue eut un répit ; ces titans rentrèrent un moment chacun dans sa pensée.\par
Les lions s’inquiètent des hydres. Robespierre était devenu très pâle et Danton très rouge. Tous deux avaient un frémissement. La prunelle fauve de Marat s’était éteinte ; le calme, un calme impérieux, s’était refait sur la face de cet homme, redouté des redoutables.\par
Danton se sentait vaincu, mais ne voulait pas se rendre. Il reprit :\par
— Marat parle très haut de dictature et d’unité, mais il n’a qu’une puissance, dissoudre.\par
Robespierre, desserrant ses lèvres étroites, ajouta :\par
— Moi, je suis de l’avis d’Anacharsis Cloots ; je dis : Ni Roland, ni Marat.\par
— Et moi, répondit Marat, je dis : Ni Danton, ni Robespierre.\par
Il les regarda tous deux fixement, et ajouta :\par
— Laissez-moi vous donner un conseil, Danton. Vous êtes amoureux, vous songez à vous remarier, ne vous mêlez plus de politique, soyez sage.\par
 Et, reculant d’un pas vers la porte pour sortir, il leur fit ce salut sinistre :\par
— Adieu, messieurs.\par
Danton et Robespierre eurent un frisson.\par
En ce moment une voix s’éleva au fond de la salle, et dit :\par
— Tu as tort, Marat.\par
Tous se retournèrent. Pendant l’explosion de Marat, et sans qu’ils s’en fussent aperçus, quelqu’un était entré par la porte du fond.\par
— C’est toi, citoyen Cimourdain, dit Marat. Bonjour.\par
C’était Cimourdain en effet.\par
— Je dis que tu as tort, Marat, reprit-il.\par
Marat verdit, ce qui était sa façon de pâlir.\par
Cimourdain ajouta :\par
— Tu es utile, mais Robespierre et Danton sont nécessaires. Pourquoi les menacer ? Union, union, citoyens ! le peuple veut qu’on soit uni.\par
Cette entrée fit un effet d’eau froide, et, comme l’arrivée d’un étranger dans une querelle de ménage, apaisa, sinon le fond, du moins la surface.\par
Cimourdain s’avança vers la table.\par
Danton et Robespierre le connaissaient. Ils avaient souvent remarqué dans les tribunes publiques de la Convention ce puissant homme obscur que le peuple saluait. Robespierre pourtant, formaliste, demanda :\par
— Citoyen, comment êtes-vous entré ?\par
— Il est de l’Évêché, répondit Marat d’une voix où l’on sentait on ne sait quelle soumission.\par
 Marat bravait la Convention, menait la Commune et craignait l’Évêché.\par
Ceci est une loi.\par
Mirabeau sent remuer à une profondeur inconnue Robespierre, Robespierre sent remuer Marat, Marat sent remuer Hébert, Hébert sent remuer Babeuf. Tant que les couches souterraines sont tranquilles, l’homme politique peut marcher ; mais sous le plus révolutionnaire il y a un sous-sol, et les plus hardis s’arrêtent inquiets quand ils sentent sous leurs pieds le mouvement qu’ils ont créé sur leur tête.\par
Savoir distinguer le mouvement qui vient des convoitises du mouvement qui vient des principes, combattre l’un et seconder l’autre, c’est là le génie et la vertu des grands révolutionnaires.\par
Danton vit plier Marat.\par
— Oh ! le citoyen Cimourdain n’est pas de trop, dit-il.\par
Et il tendit la main à Cimourdain.\par
Puis :\par
— Parbleu, dit-il, expliquons la situation au citoyen Cimourdain. Il vient à propos. Je représente la Montagne, Robespierre représente le comité de salut public, Marat représente la Commune, Cimourdain représente l’Évêché. Il va nous départager.\par
— Soit, dit Cimourdain, grave et simple. De quoi s’agit-il ?\par
— De la Vendée, répondit Robespierre.\par
— La Vendée ! dit Cimourdain.\par
Et il reprit :\par
 — C’est la grande menace. Si la révolution meurt, elle mourra par la Vendée. Une Vendée est plus redoutable que dix Allemagnes. Pour que la France vive, il faut tuer la Vendée.\par
Ces quelques mots lui gagnèrent Robespierre.\par
Robespierre pourtant fit cette question :\par
— N’êtes-vous pas un ancien prêtre ?\par
L’air prêtre n’échappait pas à Robespierre. Il reconnaissait hors de lui ce qu’il avait au dedans de lui.\par
Cimourdain répondit :\par
— Oui, citoyen.\par
— Qu’est-ce que cela fait ? s’écria Danton. Quand les prêtres sont bons, ils valent mieux que les autres. En temps de révolution, les prêtres se fondent en citoyens, comme les cloches en sous et en canons. Danjou est prêtre, Daunou est prêtre. Thomas Lindet est évêque d’Évreux. Robespierre, vous vous asseyez à la Convention coude à coude avec Massieu, évêque de Beauvais. Le grand-vicaire Vaugeois était du comité d’insurrection du 10 août. Chabot est capucin. C’est dom Gerle qui a fait le serment du Jeu de paume ; c’est l’abbé Audran qui a fait déclarer l’Assemblée nationale supérieure au roi ; c’est l’abbé Goutte qui a demandé à la Législative qu’on ôtât le dais du fauteuil de Louis XVI ; c’est l’abbé Grégoire qui a provoqué l’abolition de la royauté.\par
— Appuyé, ricana Marat, par l’histrion Collot-d’Herbois. A eux deux, ils ont fait la besogne ; le prêtre a renversé le trône, le comédien a jeté bas le roi.\par
 — Revenons à la Vendée, dit Robespierre.\par
— Eh bien, demanda Cimourdain, qu’y a-t-il ? qu’est-ce qu’elle fait, cette Vendée ?\par
Robespierre répondit :\par
— Ceci. Elle a un chef. Elle va devenir épouvantable.\par
— Qui est ce chef, citoyen Robespierre ?\par
— C’est un ci-devant marquis de Lantenac, qui s’intitule prince breton.\par
Cimourdain fit un mouvement.\par
— Je le connais, dit-il. J’ai été prêtre chez lui.\par
Il songea un moment, et reprit :\par
— C’était un homme à femmes avant d’être un homme de guerre.\par
— Comme Biron qui a été Lauzun, dit Danton.\par
Et Cimourdain, pensif, ajouta :\par
— Oui, c’est un ancien homme de plaisir. Il doit être terrible.\par
— Affreux, dit Robespierre. Il brûle les villages, achève les blessés, massacre les prisonniers, fusille les femmes.\par
— Les femmes ?\par
— Oui. Il a fait fusiller entre autres une mère de trois enfants. On ne sait ce que les enfants sont devenus. En outre, c’est un capitaine. Il sait la guerre.\par
— En effet, répondit Cimourdain. Il a fait la guerre de Hanovre, et les soldats disaient : Richelieu en dessus, Lantenac en dessous ; c’est Lantenac qui a été le vrai général. Parlez-en à Dussaulx, votre collègue.\par
 Robespierre resta un moment pensif, puis le dialogue reprit entre lui et Cimourdain.\par
— Eh bien, citoyen Cimourdain, cet homme-là est en Vendée.\par
— Depuis quand ?\par
— Depuis trois semaines.\par
— Il faut le mettre hors la loi.\par
— C’est fait.\par
— Il faut mettre sa tête à prix.\par
— C’est fait.\par
— Il faut offrir, à qui le prendra, beaucoup d’argent.\par
— C’est fait.\par
— Pas en assignats.\par
— C’est fait.\par
— En or.\par
— C’est fait.\par
— Et il faut le guillotiner.\par
— Ce sera fait.\par
— Par qui ?\par
— Par vous.\par
— Par moi ?\par
— Oui, vous serez délégué du comité de salut Public, avec pleins pouvoirs.\par
— J’accepte, dit Cimourdain.\par
Robespierre était rapide dans ses choix ; qualité d’homme d’état. Il prit dans le dossier qui était devant lui une feuille de papier blanc sur laquelle on lisait cet en-tête imprimé : R{\scshape épublique française, une et indivisible}. C{\scshape omité de salut public}.\par
 Cimourdain continua :\par
— Oui, j’accepte. Terrible contre terrible. Lantenac est féroce, je le serai. Guerre à mort avec cet homme. J’en délivrerai la république, s’il plaît à Dieu.\par
Il s’arrêta, puis reprit :\par
— Je suis prêtre ; c’est égal, je crois en Dieu.\par
— Dieu a vieilli, dit Danton.\par
— Je crois en Dieu, dit Cimourdain impassible.\par
D’un signe de tête, Robespierre, sinistre, approuva.\par
Cimourdain reprit :\par
— Près de qui serai-je délégué ?\par
Robespierre répondit :\par
— Près du commandant de la colonne expéditionnaire envoyée contre Lantenac. Seulement, je vous en préviens, c’est un noble.\par
Danton s’écria :\par
— Voilà encore de quoi je me moque. Un noble ? Eh bien, après ? Il en est du noble comme du prêtre. Quand il est bon, il est excellent. La noblesse est un préjugé ; mais il ne faut pas plus l’avoir dans un sens que dans l’autre, pas plus contre que pour. Robespierre, est-ce que Saint-Just n’est pas un noble ? Florelle de Saint-Just, parbleu ! Anacharsis Cloots est baron. Notre ami Charles Hesse, qui ne manque pas une séance des Cordeliers, est prince, et frère du landgrave régnant de Hesse-Rothenbourg. Montaut, l’intime de Marat, est marquis de Montaut. Il y a dans le tribunal révolutionnaire un juré qui est prêtre, Vilate, et un juré qui est noble, Leroy, marquis de Montflabert. Tous deux sont sûrs.\par
 — Et vous oubliez, ajouta Robespierre, le chef du jury révolutionnaire...\par
— Antonelle ?\par
— Qui est le marquis Antonelle, dit Robespierre.\par
Danton reprit :\par
— C’est un noble, Dampierre, qui vient de se faire tuer devant Condé pour la république, et c’est un noble, Beaurepaire, qui s’est brûlé la cervelle plutôt que d’ouvrir les portes de Verdun aux Prussiens.\par
— Ce qui n’empêche pas, grommela Marat, que le jour où Condorcet a dit : \emph{Les Gracques étaient des nobles,} Danton n’ait crié à Condorcet : \emph{Tous les nobles sont des traîtres, à commencer par Mirabeau et à finir par toi.}\par
La voix grave de Cimourdain s’éleva.\par
— Citoyen Danton, citoyen Robespierre, vous avez raison peut-être de vous confier, mais le peuple se défie, et il n’a pas tort de se défier. Quand c’est un prêtre qui est chargé de surveiller un noble, la responsabilité est double, et il faut que le prêtre soit inflexible.\par
— Certes, dit Robespierre.\par
Cimourdain ajouta :\par
— Et inexorable.\par
Robespierre reprit :\par
— C’est bien dit, citoyen Cimourdain. Vous aurez affaire à un jeune homme. Vous aurez de l’ascendant sur lui, ayant le double de son âge. Il faut le diriger, mais le ménager. Il paraît qu’il a des talents militaires, tous les rapports sont unanimes là-dessus. Il  fait partie d’un corps qu’on a détaché de l’armée du Rhin pour aller en Vendée. Il arrive de la frontière, où il a été admirable d’intelligence et de bravoure. Il mène supérieurement la colonne expéditionnaire. Depuis quinze jours, il tient en échec ce vieux marquis de Lantenac. Il le réprime et le chasse devant lui. Il finira par l’acculer à la mer et par l’y culbuter. Lantenac a la ruse d’un vieux général, et lui a l’audace d’un jeune capitaine. Ce jeune homme a déjà des ennemis et des envieux. L’adjudant-général Léchelle est jaloux de lui.\par
— Ce Léchelle, interrompit Danton, il veut être général en chef, il n’a pour lui qu’un calembour : \emph{Il faut Léchelle pour monter sur Charette.} En attendant, Charette le bat.\par
— Et il ne veut pas, poursuivit Robespierre, qu’un autre que lui batte Lantenac. Le malheur de la guerre de Vendée est dans ces rivalités-là. Des héros mal commandés, voilà nos soldats. Un simple capitaine de hussards, Chambon, entre dans Saumur avec un trompette en sonnant \emph{Ça ira ;} il prend Saumur ; il pourrait continuer et prendre Cholet, mais il n’a pas d’ordres, et il s’arrête. Il faut remanier tous les commandements de la Vendée. On éparpille les corps de garde, on disperse les forces ; une armée éparse est une armée paralysée ; c’est un bloc dont on fait de la poussière. Au camp de Paramé il n’y a plus que des tentes. Il y a entre Tréguier et Dinan cent petits postes inutiles avec lesquels on pourrait faire une division et couvrir tout le littoral. Léchelle, appuyé  par Parrein, dégarnit la côte nord sous prétexte de protéger la côte sud, et ouvre ainsi la France aux Anglais. Un demi-million de paysans soulevés, et une descente de l’Angleterre en France, tel est le plan de Lantenac. Le jeune commandant de la colonne expéditionnaire met l’épée aux reins à ce Lantenac et le presse et le bat, sans la permission de Léchelle ; or Léchelle est son chef ; aussi Léchelle le dénonce. Les avis sont partagés sur ce jeune homme. Léchelle veut le faire fusiller. Prieur de la Marne veut le faire adjudant-général.\par
— Ce jeune homme, dit Cimourdain, me semble avoir de grandes qualités.\par
— Mais il a un défaut !\par
L’interruption était de Marat.\par
— Lequel ? demanda Cimourdain.\par
— La clémence, dit Marat.\par
Et Marat poursuivit :\par
— C’est ferme au combat, et mou après. Ça donne dans l’indulgence, ça pardonne, ça fait grâce, ça protège les religieuses et les nonnes, ça sauve les femmes et les filles des aristocrates, ça relâche les prisonniers, ça met en liberté les prêtres.\par
— Grave faute, murmura Cimourdain.\par
— Crime, dit Marat.\par
— Quelquefois, dit Danton.\par
— Souvent, dit Robespierre.\par
— Presque toujours, reprit Marat.\par
— Quand on a affaire aux ennemis de la patrie, toujours, dit Cimourdain.\par
 Marat se tourna vers Cimourdain.\par
— Et que ferais-tu donc d’un chef républicain qui mettrait en liberté un chef royaliste ?\par
— Je serais de l’avis de Léchelle, je le ferais fusiller.\par
— Ou guillotiner, dit Marat.\par
— Au choix, dit Cimourdain.\par
Danton se mit à rire.\par
— J’aime autant l’un que l’autre.\par
— Tu es sûr d’avoir l’un ou l’autre, grommela Marat.\par
Et son regard, quittant Danton, revint sur Cimourdain.\par
— Ainsi, citoyen Cimourdain, si un chef républicain bronchait, tu lui ferais couper la tête ?\par
— Dans les vingt-quatre heures.\par
— Eh bien, repartit Marat, je suis de l’avis de Robespierre, il faut envoyer le citoyen Cimourdain comme commissaire délégué du comité de salut public près du commandant de la colonne expéditionnaire de l’armée des côtes. Comment s’appelle-t-il déjà, ce commandant ?\par
Robespierre répondit :\par
— C’est un ci-devant, un noble.\par
Et il se mit à feuilleter le dossier.\par
— Donnons au prêtre le noble à garder, dit Danton. Je me défie d’un prêtre qui est seul ; je me défie d’un noble qui est seul ; quand ils sont ensemble, je ne les crains pas ; l’un surveille l’autre, et ils vont.\par
L’indignation propre au sourcil de Cimourdain s’accentua ;  mais trouvant sans doute l’observation juste au fond, il ne se tourna point vers Danton, et il éleva sa voix sévère.\par
— Si le commandant républicain qui m’est confié fait un faux pas, peine de mort.\par
Robespierre, les yeux sur le dossier, dit :\par
— Voici le nom. Citoyen Cimourdain, le commandant sur qui vous aurez pleins pouvoirs est un ci-devant vicomte. Il s’appelle Gauvain.\par
Cimourdain pâlit.\par
— Gauvain ! s’écria-t-il.\par
Marat vit la pâleur de Cimourdain.\par
— Le vicomte Gauvain ! répéta Cimourdain.\par
— Oui, dit Robespierre.\par
— Eh bien ? dit Marat, l’œil fixé sur Cimourdain.\par
Il y eut un temps d’arrêt. Marat reprit :\par
— Citoyen Cimourdain, aux conditions indiquées par vous-même, acceptez-vous la mission de commissaire délégué près le commandant Gauvain ? Est-ce dit ?\par
— C’est dit, répondit Cimourdain.\par
Il était de plus en plus pâle.\par
Robespierre prit la plume qui était près de lui, écrivit de son écriture lente et correcte quatre lignes sur la feuille de papier portant en tête : C{\scshape omité de salut public}, signa, et passa la feuille et la plume à Danton ; Danton signa, et Marat, qui ne quittait pas des yeux la face livide de Cimourdain, signa après Danton.\par
Robespierre, reprenant la feuille, la data, et la remit à Cimourdain, qui lut :\par
 
\labelblock{an ii de la république}

\noindent « Pleins pouvoirs sont donnés au citoyen Cimourdain, commissaire délégué du comité de salut public près le citoyen Gauvain, commandant la colonne expéditionnaire de l’armée des côtes.\par

\byline{« R{\scshape obespierre}. — D{\scshape anton}. — M{\scshape arat}. »}
\noindent Et au-dessous des signatures :\par

\dateline{« 28 juin 1793. »}
\noindent Le calendrier révolutionnaire, dit calendrier civil, n’existait pas encore légalement à cette époque, et ne devait être adopté par la Convention, sur la proposition de Romme, que le 5 octobre 1793.\par
Pendant que Cimourdain lisait, Marat le regardait.\par
Marat dit à demi-voix, comme se parlant à lui-même :\par
— Il faudra faire préciser tout cela par un décret de la Convention ou par un arrêté spécial du comité de salut public. Il reste quelque chose à faire.\par
— Citoyen Cimourdain, demanda Robespierre, où demeurez-vous ?\par
— Cour du Commerce.\par
— Tiens, moi aussi, dit Danton, vous êtes mon voisin.\par
Robespierre reprit :\par
— Il n’y a pas un moment à perdre. Demain vous recevrez votre commission en règle, signée de tous  les membres du comité de salut public. Ceci est une confirmation de la commission, qui vous accréditera spécialement près des représentants en mission, Philippeaux, Prieur de la Marne, Lecointre, Alquier et les autres. Nous savons qui vous êtes. Vos pouvoirs sont illimités. Vous pouvez faire Gauvain général ou l’envoyer à l’échafaud. Vous aurez votre commission demain à trois heures. Quand partirez-vous ?\par
— A quatre heures, dit Cimourdain.\par
Et ils se séparèrent.\par
En rentrant chez lui, Marat prévint Simonne Évrard qu’il irait le lendemain à la Convention.\par
  \subsection[{Livre troisième. La convention}]{Livre troisième \\
La convention}\phantomsection
\label{p2l3}
\subsubsection[{I. La convention}]{I \\
La convention}\phantomsection
\label{p2l3c1}

\labelblock{i}

\noindent Nous approchons de la grande cime.\par
Voici la Convention.\par
Le regard devient fixe en présence de ce sommet.\par
Jamais rien de plus haut n’est apparu sur l’horizon des hommes.\par
Il y a l’Himalaya et il y a la Convention.\par
La Convention est peut-être le point culminant de l’histoire.\par
Du vivant de la Convention, car cela vit, une assemblée, on ne se rendait pas compte de ce qu’elle était. Ce qui échappait aux contemporains, c’était précisément sa grandeur ; on était trop effrayé pour être ébloui. Tout ce qui est grand a une horreur sacrée. Admirer les médiocres et les collines, c’est  aisé ; mais ce qui est trop haut, un génie aussi bien qu’une montagne, une assemblée aussi bien qu’un chef-d’œuvre, vus de trop près, épouvantent. Toute cime semble une exagération. Gravir fatigue. On s’essouffle aux escarpements, on glisse sur les pentes, on se blesse à des aspérités qui sont des beautés ; les torrents, en écumant, dénoncent les précipices, les nuages cachent les sommets ; l’ascension terrifie autant que la chute. De là plus d’effroi que d’admiration. On éprouve ce sentiment bizarre, l’aversion du grand. On voit les abîmes, on ne voit pas les sublimités ; on voit le monstre, on ne voit pas le prodige. Ainsi fut d’abord jugée la Convention. La Convention fut toisée par les myopes, elle, faite pour être contemplée par les aigles.\par
Aujourd’hui elle est en perspective, et elle dessine sur le ciel profond, dans un lointain serein et tragique, l’immense profil de la révolution française.\par

\labelblock{ii}

\noindent Le 14 juillet avait délivré.\par
Le 10 août avait foudroyé.\par
Le 21 septembre fonda.\par
Le 21 septembre, l’équinoxe, l’équilibre. \emph{Libra}. La balance. Ce fut, suivant la remarque de Romme, sous ce signe de l’Égalité et de la Justice que la république fut proclamée. Une constellation fit l’annonce.\par
La Convention est le premier avatar du peuple.  C’est par la Convention que s’ouvrit la grande page nouvelle et que l’avenir d’aujourd’hui commença.\par
A toute idée il faut une enveloppe visible, à tout principe il faut une habitation ; une église, c’est Dieu entre quatre murs, à tout dogme il faut un temple.\par
Quand la Convention fut, il y eut un premier problème à résoudre, loger la Convention.\par
On prit d’abord le Manège, puis les Tuileries. On y dressa un châssis, un décor, une grande grisaille peinte par David, des bancs symétriques, une tribune carrée, des pilastres parallèles, des socles pareils à des billots, de longues étraves rectilignes, des alvéoles rectangulaires où se pressait la multitude et qu’on appelait les tribunes publiques, un velarium romain, des draperies grecques, et dans ces angles droits et dans ces lignes droites on installa la Convention ; dans cette géométrie on mit la tempête. Sur la tribune le bonnet rouge était peint en gris. Les royalistes commencèrent par rire de ce bonnet rouge gris, de cette salle postiche, de ce monument de carton, de ce sanctuaire de papier mâché, de ce panthéon de boue et de crachat. Comme cela devait disparaître vite ! Les colonnes étaient en douves de tonneau, les voûtes étaient en volige, les bas-reliefs étaient en mastic, les entablements étaient en sapin, les statues étaient en plâtre, les marbres étaient en peinture, les murailles étaient en toile ; et dans ce provisoire la France a fait de l’éternel.\par
Les murailles de la salle du Manège, quand la Convention vint y tenir séance, étaient toutes couvertes  des affiches qui avaient pullulé dans Paris à l’époque du retour de Varennes. On lisait sur l’une : — \emph{Le roi rentre. Bâtonner qui l’applaudira, pendre qui l’insultera.} — Sur une autre : — \emph{Paix là. Chapeaux sur la tête. Il va passer devant ses juges.} — Sur une autre : — \emph{Le roi a couché la nation en joue. Il a fait long feu. A la nation de tirer maintenant}. — Sur une autre : — \emph{La Loi ! La Loi !} Ce fut entre ces murs-là que la Convention jugea Louis XVI.\par
Aux Tuileries, où la Convention vint siéger le 10 mai 1793, et qui s’appelèrent le Palais-National, la salle des séances occupait tout l’intervalle entre le pavillon de l’Horloge appelé pavillon-Unité et le pavillon Marsan appelé pavillon-Liberté. Le pavillon de Flore s’appelait pavillon-Égalité. C’est par le grand-escalier de Jean Bullant qu’on montait à la salle des séances. Sous le premier étage occupé par l’assemblée, tout le rez-de-chaussée du palais était une sorte de longue salle des gardes, encombrée des faisceaux et des lits de camp des troupes de toutes armes qui veillaient autour de la Convention. L’assemblée avait une garde d’honneur qu’on appelait « les grenadiers de la Convention. »\par
Un ruban tricolore séparait le château où était l’assemblée du jardin où le peuple allait et venait.\par

\labelblock{iii}

\noindent Ce qu’était la salle des séances, achevons de le dire. Tout intéresse de ce lieu terrible.\par
 Ce qui, en entrant, frappait d’abord le regard, c’était, entre deux larges fenêtres, une haute statue de la Liberté.\par
Quarante-deux mètres de longueur, dix mètres de largeur, onze mètres de hauteur, telles étaient les dimensions de ce qui avait été le théâtre du roi et de ce qui devint le théâtre de la révolution. L’élégante et magnifique salle bâtie par Vigarani pour les courtisans disparut sous la sauvage charpente qui en 93 dut subir le poids du peuple. Cette charpente, sur laquelle s’échafaudaient les tribunes publiques, avait, détail qui vaut la peine d’être noté, pour point d’appui unique un poteau. Ce poteau était d’un seul morceau, et avait dix mètres de portée. Peu de cariatides ont travaillé comme ce poteau ; il a soutenu pendant des années la rude poussée de la révolution. Il a porté l’acclamation, l’enthousiasme, l’injure, le bruit, le tumulte, l’immense chaos des colères, l’émeute. Il n’a pas fléchi. Après la Convention, il a vu le conseil des Anciens. Le 18 brumaire l’a relayé.\par
Percier alors remplaça le pilier de bois par des colonnes de marbre, qui ont moins duré.\par
L’idéal des architectes est parfois singulier ; l’architecte de la rue de Rivoli a eu pour idéal la trajectoire d’un boulet de canon, l’architecte de Carlsruhe a eu pour idéal un éventail ; un gigantesque tiroir de commode, tel semble avoir été l’idéal de l’architecte qui construisit la salle où la Convention vint siéger le 10 mai 1793 ; c’était long, haut et plat. A l’un des grands côtés du parallélogramme était adossé un  vaste demi-cirque ; c’était l’amphithéâtre des bancs des représentants, sans tables ni pupitres ; Garan-Coulon, qui écrivait beaucoup, écrivait sur son genou ; en face des bancs, la tribune ; devant la tribune, le buste de Lepelletier-Saint-Fargeau ; derrière la tribune, le fauteuil du président.\par
La tête du buste dépassait un peu le rebord de la tribune ; ce qui fit que, plus tard, on l’ôta de là.\par
L’amphithéâtre se composait de dix-neuf bancs demi-circulaires, étagés les uns derrière les autres ; des tronçons de bancs prolongeaient cet amphithéâtre dans les deux encoignures.\par
En bas, dans le fer à cheval au pied de la tribune, se tenaient les huissiers.\par
D’un côté de la tribune, dans un cadre de bois noir, était appliquée au mur une pancarte de neuf pieds de haut, portant, sur deux pages séparées par une sorte de sceptre, la déclaration des droits de l’homme ; de l’autre côté, il y avait une place vide qui plus tard fut occupée par un cadre pareil contenant la Constitution de l’an II, dont les deux pages étaient séparées par un glaive. Au-dessus de la tribune, au-dessus de la tête de l’orateur, frissonnaient, sortant d’une profonde loge à deux compartiments pleine de peuple, trois immenses drapeaux tricolores, presque horizontaux, appuyés à un autel sur lequel on lisait ce mot : {\scshape la loi}. Derrière cet autel, se dressait, comme la sentinelle de la parole libre, un énorme faisceau romain, haut comme une colonne. Des statues colossales, droites contre le mur, faisaient face aux représentants.  Le président avait à sa droite Lycurgue et à sa gauche Solon ; au-dessus de la Montagne il y avait Platon.\par
Ces statues avaient pour piédestaux de simples dés, posés sur une longue corniche saillante qui faisait le tour de la salle et séparait le peuple de l’assemblée. Les spectateurs s’accoudaient à cette corniche.\par
Le cadre de bois noir du placard des \emph{Droits de l’Homme} montait jusqu’à la corniche et entamait le dessin de l’entablement, effraction de la ligne droite qui faisait murmurer Chabot. — \emph{C’est laid}, disait-il à Vadier.\par
Sur les têtes des statues, alternaient des couronnes de chêne et de laurier.\par
Une draperie verte, où étaient peintes en vert plus foncé les mêmes couronnes, descendait à gros plis droits de la corniche de pourtour et tapissait tout le rez-de-chaussée de la salle occupée par l’assemblée. Au-dessus de cette draperie la muraille était blanche et froide. Dans cette muraille se creusaient, coupés comme à l’emporte-pièce, sans moulure ni rinceau, deux étages de tribunes publiques, les carrées en bas, les rondes en haut ; selon la règle, car Vitruve n’était pas détrôné, les archivoltes étaient superposées aux architraves. Il y avait dix tribunes sur chacun des grands côtés de la salle, et à chacune des deux extrémités deux loges démesurées ; en tout vingt-quatre. Là s’entassaient les foules.\par
Les spectateurs des tribunes inférieures débordaient  sur tous les plats-bords et se groupaient sur tous les reliefs de l’architecture. Une longue barre de fer, solidement scellée à hauteur d’appui, servait de garde-fou aux tribunes hautes, et garantissait les spectateurs contre la pression des cohues montant les escaliers. Une fois pourtant, un homme fut précipité dans l’assemblée, il tomba un peu sur Massieu, évêque de Beauvais, ne se tua pas, et dit : \emph{Tiens ! c’est donc bon à quelque chose, un évêque !}\par
La salle de la Convention pouvait contenir deux mille personnes, et, les jours d’insurrection, trois mille.\par
La Convention avait deux séances, une du jour, une du soir.\par
Le dossier du président était rond, à clous dorés. Sa table était contrebutée par quatre monstres ailés à un seul pied, qu’on eût dit sortis de l’apocalypse pour assister à la révolution. Ils semblaient avoir été dételés du char d’Ézéchiel pour venir traîner le tombereau de Sanson.\par
Sur la table du président il y avait une grosse sonnette, presque une cloche, un large encrier de cuivre, et un in-folio relié en parchemin qui était le livre des procès-verbaux.\par
Des têtes coupées, portées au bout d’une pique, se sont égouttées sur cette table.\par
On montait à la tribune par un degré de neuf marches. Ces marches étaient hautes, roides, et assez difficiles ; elles firent un jour trébucher Gensonné qui les gravissait. \emph{C’est un escalier d’échafaud !} dit-il. — \emph{Fais ton apprentissage}, lui cria Carrier.\par
 Là où le mur avait paru trop nu, dans les angles de la salle, l’architecte avait appliqué pour ornements des faisceaux, la hache en dehors.\par
A droite et à gauche de la tribune, des socles portaient deux candélabres de douze pieds de haut, ayant à leur sommet quatre paires de quinquets. Il y avait dans chaque loge publique un candélabre pareil. Sur les socles de ces candélabres étaient sculptés des ronds que le peuple appelait « colliers de guillotine ».\par
Les bancs de l’assemblée montaient presque jusqu’à la corniche des tribunes ; les représentants et le peuple pouvaient dialoguer.\par
Les vomitoires des tribunes se dégorgeaient dans un labyrinthe de corridors, plein parfois d’un bruit farouche.\par
La Convention encombrait le palais, et refluait jusque dans les hôtels voisins, l’hôtel de Longueville, l’hôtel de Coigny. C’est à l’hôtel de Coigny qu’après le 10 août, si l’on en croit une lettre de lord Bradford, on transporta le mobilier royal. Il fallut deux mois pour vider les Tuileries.\par
Les comités étaient logés aux environs de la salle ; au pavillon-Égalité, la législation, l’agriculture et le commerce ; au pavillon-Liberté, la marine, les colonies, les finances, les assignats, le salut public ; au pavillon-Unité, la guerre.\par
Le comité de sûreté générale communiquait directement avec le comité de salut public par un couloir obscur, éclairé nuit et jour d’un réverbère, où allaient  et venaient les espions de tous les partis. On y parlait bas.\par
La barre de la Convention a été plusieurs fois déplacée. Habituellement elle était à la droite du président.\par
Aux deux extrémités de la salle, les deux cloisons verticales qui fermaient du côté droit et du côté gauche les demi-cercles concentriques de l’amphithéâtre laissaient entre elles et le mur deux couloirs étroits et profonds sur lesquels s’ouvraient deux sombres portes carrées. On entrait et on sortait par là.\par
Les représentants entraient directement dans la salle par une porte donnant sur la terrasse des Feuillants.\par
Cette salle, peu éclairée le jour par de pâles fenêtres, mal éclairée quand venait le crépuscule par des flambeaux livides, avait on ne sait quoi de nocturne. Ce demi-éclairage s’ajoutait aux ténèbres du soir ; les séances aux lampes étaient lugubres. On ne se voyait pas ; d’un bout de la salle à l’autre, de la droite à la gauche, des groupes de faces vagues s’insultaient. On se rencontrait sans se reconnaître. Un jour Laignelot, courant à la tribune, se heurte, dans le couloir de descente, à quelqu’un. — Pardon, Robespierre, dit-il. — Pour qui me prends-tu ? répond une voix rauque. — Pardon, Marat, dit Laignelot.\par
En bas, à droite et à gauche du président, deux tribunes étaient réservées ; car, chose étrange, il y avait à la Convention des spectateurs privilégiés. Ces tribunes étaient les seules qui eussent une draperie.  Au milieu de l’architrave, deux glands d’or relevaient cette draperie. Les tribunes du peuple étaient nues.\par
Tout cet ensemble était violent, sauvage, régulier. Le correct dans le farouche ; c’est un peu toute la révolution. La salle de la Convention offrait le plus complet spécimen de ce que les artistes ont appelé depuis « l’architecture messidor ». C’était massif et grêle. Les bâtisseurs de ce temps-là prenaient le symétrique pour le beau. Le dernier mot de la Renaissance avait été dit sous Louis XV, et une réaction s’était faite. On avait poussé le noble jusqu’au fade, et la pureté jusqu’à l’ennui. La pruderie existe en architecture. Après les éblouissantes orgies de forme et de couleur du dix-huitième siècle, l’art s’était mis à la diète, et ne se permettait plus que la ligne droite. Ce genre de progrès aboutit à la laideur. L’art réduit au squelette, tel est le phénomène. C’est l’inconvénient de ces sortes de sagesses et d’abstinences ; le style est si sobre qu’il devient maigre.\par
En dehors de toute émotion politique, et à ne voir que l’architecture, un certain frisson se dégageait de cette salle. On se rappelait confusément l’ancien théâtre, les loges enguirlandées, le plafond d’azur et de pourpre, le lustre à facettes, les girandoles à reflets de diamants, les tentures gorge de pigeon, la profusion d’amours et de nymphes sur le rideau et sur les draperies, toute l’idylle royale et galante, peinte, sculptée et dorée, qui avait empli de son sourire ce lieu sévère, et l’on regardait partout autour de soi ces durs angles rectilignes, froids et tranchants  comme l’acier ; c’était quelque chose comme Boucher guillotiné par David.\par

\labelblock{iv}

\noindent Qui voyait l’assemblée ne songeait plus à la salle. Qui voyait le drame ne pensait plus au théâtre. Rien de plus difforme et de plus sublime. Un tas de héros, un troupeau de lâches. Des fauves sur une montagne, des reptiles dans un marais. Là fourmillaient, se coudoyaient, se provoquaient, se menaçaient, luttaient et vivaient tous ces combattants qui sont aujourd’hui des fantômes.\par
Dénombrement titanique.\par
A droite, la Gironde, légion de penseurs ; à gauche, la Montagne, groupe d’athlètes. D’un côté, Brissot, qui avait reçu les clefs de la Bastille ; Barbaroux, auquel obéissaient les Marseillais ; Kervélégan, qui avait sous la main le bataillon de Brest caserné au faubourg Saint-Marceau ; Gensonné, qui avait établi la suprématie des représentants sur les généraux ; le fatal Guadet, auquel une nuit, aux Tuileries, la reine avait montré le dauphin endormi ; Guadet baisa le front de l’enfant et fit tomber la tête du père ; Salles, le dénonciateur chimérique des intimités de la Montagne avec l’Autriche ; Sillery, le boiteux de la droite, comme Couthon était le cul-de-jatte de la gauche ; Lause-Duperret, qui, traité de \emph{scélérat} par un journaliste, l’invita à dîner en disant : « \emph{Je sais que} « scélérat  » \emph{veut simplement dire l’homme qui ne pense pas comme nous ;} » Rabaut-Saint-Étienne, qui avait commencé son almanach de 1790 par ce mot : \emph{La révolution est finie ;} Quinette, un de ceux qui précipitèrent Louis XVI ; le janséniste Camus, qui rédigeait la constitution civile du clergé, croyait aux miracles du diacre Pâris, et se prosternait toutes les nuits devant un christ de sept pieds de haut cloué au mur de sa chambre ; Fauchet, un prêtre qui, avec Camille Desmoulins, avait fait le 14 juillet ; Isnard, qui commit le crime de dire : \emph{Paris sera détruit}, au moment même où Brunswick disait : \emph{Paris sera brûlé ;} Jacob Dupont, le premier qui cria : \emph{Je suis athée}, et à qui Robespierre répondit : \emph{L’athéisme est aristocratique ;} Lanjuinais, dure, sagace et vaillante tête bretonne ; Ducos, l’Euryale de Boyer-Fonfrède ; Rebecqui, le Pylade de Barbaroux ; Rebecqui donnait sa démission parce qu’on n’avait pas encore guillotiné Robespierre ; Richaud, qui combattait la permanence des sections ; Lasource, qui avait émis cet apophthegme meurtrier : \emph{Malheur aux nations reconnaissantes !} et qui, au pied de l’échafaud, devait se contredire par cette fière parole jetée aux montagnards : \emph{Nous mourons parce que le peuple dort, et vous mourrez parce que le peuple se réveillera ;} Biroteau, qui fit décréter l’abolition de l’inviolabilité, fut ainsi, sans le savoir, le forgeron du couperet, et dressa l’échafaud pour lui-même ; Charles Villette, qui abrita sa conscience sous cette protestation : \emph{Je ne veux pas voter sous les couteaux ;} Louvet, l’auteur de \emph{Faublas}, qui devait finir libraire au Palais-Royal avec  Lodoïska au comptoir ; Mercier, l’auteur du \emph{Tableau de Paris}, qui s’écriait : \emph{Tous les rois ont senti sur leurs nuques le 21 janvier ;} Marec, qui avait pour souci « la faction des anciennes limites » ; le journaliste Carra qui, au pied de l’échafaud, dit au bourreau : \emph{Ça m’ennuie de mourir. J’aurais voulu voir la suite ;} Vigée, qui s’intitulait grenadier dans le deuxième bataillon de Mayenne-et-Loire, et qui, menacé par les tribunes publiques, s’écriait : \emph{Je demande qu’au premier murmure des tribunes, nous nous retirions tous, et marchions à Versailles, le sabre à la main !} Buzot, réservé à la mort de faim ; Valazé, promis à son propre poignard ; Condorcet, qui devait mourir à Bourg-la-Reine devenu Bourg-Égalité, dénoncé par l’Horace qu’il avait dans sa poche ; Pétion, dont la destinée était d’être adoré par la foule en 1792 et dévoré par les loups en 1794 ; vingt autres encore, Pontécoulant, Marboz, Lidon, Saint-Martin, Dussaulx, traducteur de Juvénal, qui avait fait la campagne de Hanovre ; Boilleau, Bertrand, Lesterp-Beauvais, Lesage, Gomaire, Gardien, Mainvieille, Duplantier, Lacaze, Antiboul, et en tête un Barnave qu’on appelait Vergniaud.\par
De l’autre côté, Antoine-Louis-Léon Florelle de Saint-Just, pâle, front bas, profil correct, œil mystérieux, tristesse profonde, vingt-trois ans ; Merlin de Thionville, que les Allemands appelaient Feuer-Teufel, « le diable de feu » ; Merlin de Douai, le coupable auteur de la loi des suspects ; Soubrany, que le peuple de Paris, au premier prairial, demanda pour général ; l’ancien curé Lebon, tenant un sabre de la main qui  avait jeté de l’eau bénite ; Billaud-Varennes, qui entrevoyait la magistrature de l’avenir : pas de juges, des arbitres ; Fabre d’Églantine, qui eut une trouvaille charmante, le calendrier républicain, comme Rouget de Lisle eut une inspiration sublime, la Marseillaise, mais l’un et l’autre sans récidive ; Manuel, le procureur de la Commune, qui avait dit : \emph{Un roi mort n’est pas un homme de moins ;} Goujon, qui était entré dans Tripstadt, dans Newstadt et dans Spire, et avait vu fuir l’armée prussienne ; Lacroix, avocat changé en général, fait chevalier de Saint-Louis six jours avant le 10 août ; Fréron-Thersite, fils de Fréron-Zoïle ; Ruhl, l’inexorable fouilleur de l’armoire de fer, prédestiné au grand suicide républicain, devant se tuer le jour où mourrait la république ; Fouché, âme de démon, face de cadavre ; Camboulas, l’ami du père Duchêne, lequel disait à Guillotin : \emph{Tu es du club des Feuillants, mais ta fille est du club des Jacobins} ; Jagot, qui à ceux qui plaignaient la nudité des prisonniers répondait : \emph{Une prison est un habit de pierre ;} Javogues, l’effrayant déterreur des tombeaux de Saint-Denis ; Osselin, proscripteur qui cachait chez lui une proscrite, madame Charry ; Bentabolle, qui, lorsqu’il présidait, faisait signe aux tribunes d’applaudir ou de huer ; le journaliste Robert, mari de mademoiselle Kéralio, laquelle écrivait : \emph{Ni Robespierre, ni Marat ne viennent chez moi ; Robespierre y viendra quand il voudra, Marat, jamais ;} Garan-Coulon, qui avait fièrement demandé, quand l’Espagne était intervenue dans le procès de Louis XVI, que l’assemblée ne daignât pas lire la  lettre d’un roi pour un roi ; Grégoire, évêque digne d’abord de la primitive église, mais qui plus tard sous l’empire effaça le républicain Grégoire par le comte Grégoire ; Amar, qui disait : \emph{Toute la terre condamne Louis XVI. A qui donc appeler du jugement ? Aux planètes ;} Rouyer, qui s’était opposé, le 21 janvier, à ce qu’on tirât le canon du Pont-Neuf, disant : \emph{Une tête de roi ne doit pas faire en tombant plus de bruit que la tête d’un autre homme ;} Chénier, frère d’André ; Vadier, un de ceux qui posaient un pistolet sur la tribune ; Tanis, qui disait à Momoro : \emph{Je veux que Marat et Robespierre s’embrassent à ma table chez moi. — Où demeures-tu ? — A Charenton. — Ailleurs m’eût étonné,} disait Momoro ; Legendre, qui fut le boucher de la révolution de France comme Pride avait été le boucher de la révolution d’Angleterre ; — \emph{Viens, que je t’assomme !} criait-il à Lanjuinais. Et Lanjuinais répondait : \emph{Fais d’abord décréter que je suis un bœuf ;} Collot d’Herbois, ce lugubre comédien, ayant sur la face l’antique masque aux deux bouches qui disent Oui et Non, approuvant par l’une ce qu’il blâmait par l’autre, flétrissant Carrier à Nantes et déifiant Châlier à Lyon, envoyant Robespierre à l’échafaud et Marat au Panthéon ; Génissieux, qui demandait la peine de mort contre quiconque aurait sur lui la médaille \emph{Louis XVI martyrisé ;} Léonard Bourdon, le maître d’école qui avait offert sa maison au vieillard du Mont-Jura ; Topsent, marin, Goupilleau, avocat, Laurent Lecointre, marchand, Duhem, médecin, Sergent, statuaire, David, peintre, Joseph Égalité, prince. D’autres encore ;  Lecointe-Puiraveau, qui demandait que Marat fût déclaré par décret « en état de démence » ; Robert Lindet, l’inquiétant créateur de cette pieuvre dont la tête était le comité de sûreté générale et qui couvrait la France de vingt et un mille bras qu’on appelait les comités révolutionnaires ; Lebœuf, sur qui Girey-Dupré, dans son \emph{Noël des faux patriotes}, avait fait ce vers :\par

Lebœuf vit Legendre et beugla.\\

\noindent Thomas Paine, américain, et clément ; Anacharsis Cloots, allemand, baron, millionnaire, athée, hébertiste, candide ; l’intègre Lebas, l’ami des Duplay ; Rovère, un des rares hommes qui sont méchants pour la méchanceté, car l’art pour l’art existe plus qu’on ne croit ; Charlier, qui voulait qu’on dît \emph{vous} aux aristocrates ; Tallien, élégiaque et féroce, qui fera le 9 thermidor par amour ; Cambacérès, procureur qui sera prince ; Carrier, procureur qui sera tigre ; Laplanche, qui s’écria un jour : \emph{Je demande la priorité pour le canon d’alarme ;} Thuriot, qui voulait le vote à haute voix des jurés du tribunal révolutionnaire ; Bourdon de l’Oise, qui provoquait en duel Chambon, dénonçait Paine, et était dénoncé par Hébert ; Fayau, qui proposait « l’envoi d’une armée incendiaire » dans la Vendée ; Tavaux, qui le 13 avril fut presque un médiateur entre la Gironde et la Montagne ; Vernier, qui demandait que les chefs girondins et les chefs montagnards allassent servir comme simples soldat ; Rewbell, qui s’enferma dans Mayence ; Bourbotte, qui  eut son cheval tué sous lui à la prise de Saumur ; Guimberteau, qui dirigea l’armée des Côtes de Cherbourg ; Jard-Panvilliers, qui dirigea l’armée des Côtes de la Rochelle ; Lecarpentier, qui dirigea l’escadre de Cancale ; Roberjot, qu’attendait le guet-apens de Rastadt ; Prieur de la Marne, qui portait dans les camps sa vieille contre-épaulette de chef d’escadron ; Levasseur de la Sarthe, qui, d’un mot, décidait Serrent, commandant du bataillon de Saint-Amand, à se faire tuer ; Reverchon, Maure, Bernard de Saintes, Charles Richard, Lequinio, et au sommet de ce groupe un Mirabeau qu’on appelait Danton.\par
En dehors de ces deux camps, et les tenant tous deux en respect, se dressait un homme, Robespierre.\par

\labelblock{v}

\noindent Au-dessous se courbaient l’épouvante, qui peut être noble, et la peur, qui est basse. Sous les passions, sous les héroïsmes, sous les dévouements, sous les rages, la morne cohue des anonymes. Les bas-fonds de l’assemblée s’appelaient la Plaine. Il y avait là tout ce qui flotte ; les hommes qui doutent, qui hésitent, qui reculent, qui ajournent, qui épient, chacun craignant quelqu’un. La Montagne, c’était une élite, la Gironde, c’était une élite ; la Plaine, c’était la foule. La Plaine se résumait et se condensait en Sieyès.\par
Sieyès, homme profond qui était devenu creux. Il s’était arrêté au tiers-état, et n’avait pu monter jusqu’au  peuple. De certains esprits sont faits pour rester à mi-côte. Sieyès appelait tigre Robespierre qui l’appelait taupe. Ce métaphysicien avait abouti, non à la sagesse, mais à la prudence. Il était courtisan et non serviteur de la révolution. Il prenait une pelle et allait, avec le peuple, travailler au Champ de Mars, attelé à la même charrette qu’Alexandre de Beauharnais. Il conseillait l’énergie dont il n’usait point. Il disait aux Girondins : \emph{Mettez le canon de votre parti.} Il y a les penseurs qui sont les lutteurs ; ceux-là étaient, comme Condorcet, avec Vergniaud, ou, comme Camille Desmoulins, avec Danton. Il y a les penseurs qui veulent vivre, ceux-ci étaient avec Sieyès.\par
Les cuves les plus généreuses ont leur lie. Au-dessous même de la Plaine, il y avait le Marais. Stagnation hideuse laissant voir les transparences de l’égoïsme. Là grelottait l’attente muette des trembleurs. Rien de plus misérable. Tous les opprobres, et aucune honte ; la colère latente ; la révolte sous la servitude. Ils étaient cyniquement effrayés ; ils avaient tous les courages de la lâcheté ; ils préféraient la Gironde et choisissaient la Montagne ; le dénoûment dépendait d’eux ; ils versaient du côté qui réussissait ; ils livraient Louis XVI à Vergniaud, Vergniaud à Danton, Danton à Robespierre, Robespierre à Tallien. Ils piloriaient Marat vivant et divinisaient Marat mort. Ils soutenaient tout jusqu’au jour où ils renversaient tout. Ils avaient l’instinct de la poussée décisive à donner à tout ce qui chancelle. A leurs yeux, comme ils s’étaient mis en service à la condition qu’on fût  solide, chanceler, c’était les trahir. Ils étaient le nombre, ils étaient la force, ils étaient la peur. De [{\corr là}] l’audace des turpitudes.\par
De là le 31 mai, le 11 germinal, le 9 thermidor ; tragédies nouées par les géants et dénouées par les nains.\par

\labelblock{vi}

\noindent A ces hommes pleins de passions étaient mêlés les hommes pleins de songes. L’utopie était là sous toutes ses formes, sous sa forme belliqueuse qui admettait l’échafaud, et sous sa forme innocente qui abolissait la peine de mort ; spectre du côté des trônes, ange du côté des peuples. En regard des esprits qui combattaient, il y avait les esprits qui couvaient. Les uns avaient dans la tête la guerre, les autres la paix ; un cerveau, Carnot, enfantait quatorze armées ; un autre cerveau, Jean Debry, méditait une fédération démocratique universelle. Parmi ces éloquences furieuses, parmi ces voix hurlantes et grondantes, il y avait des silences féconds. Lakanal se taisait, et combinait dans sa pensée l’éducation publique nationale ; Lanthenas se taisait, et créait les écoles primaires ; Revellière-Lepeaux se taisait, et rêvait l’élévation de la philosophie à la dignité de religion. D’autres s’occupaient de questions de détail, plus petites et plus pratiques. Guyton-Morveaux étudiait l’assainissement des hôpitaux,  Maire l’abolition des servitudes réelles, Jean-Bon-Saint-André la suppression de la prison pour dettes et de la contrainte par corps, Romme la proposition de Chappe, Duboë la mise en ordre des archives, Coren-Fustier la création du cabinet d’anatomie et du muséum d’histoire naturelle, Guyomard la navigation fluviale et le barrage de l’Escaut. L’art avait ses fanatiques et même ses monomanes ; le 21 janvier, pendant que la tête de la monarchie tombait sur la place de la Révolution, Bézard, représentant de l’Oise, allait voir un tableau de Rubens trouvé dans un galetas de la rue Saint-Lazare. Artistes, orateurs, prophètes, hommes-colosses comme Danton, hommes-enfants comme Cloots, gladiateurs et philosophes, tous allaient au même but, le progrès. Rien ne les déconcertait. La grandeur de la Convention fut de chercher la quantité de réel qui est dans ce que les hommes appellent l’impossible. A l’une de ses extrémités, Robespierre avait l’œil fixé sur le droit ; à l’autre extrémité, Condorcet avait l’œil fixé sur le devoir.\par
Condorcet était un homme de rêverie et de clarté ; Robespierre était un homme d’exécution ; et quelquefois, dans les crises finales des sociétés vieillies, exécution signifie extermination. Les révolutions ont deux versants, montée et descente, et portent étagées sur ces versants toutes les saisons, depuis la glace jusqu’aux fleurs. Chaque zone de ces versants produit les hommes qui conviennent à son climat, depuis ceux qui vivent dans le soleil jusqu’à ceux qui vivent dans la foudre.\par
 
\labelblock{vii}

\noindent On se montrait le repli du couloir de gauche où Robespierre avait dit bas à l’oreille de Garat, l’ami de Clavière, ce mot redoutable : \emph{Clavière a conspiré partout où il a respiré}. Dans ce même recoin, commode aux apartés et aux colères à demi-voix, Fabre d’Églantine avait querellé Romme et lui avait reproché de défigurer son calendrier par le changement de \emph{Fervidor} en \emph{Thermidor}. On se montrait l’angle où siégeaient, se touchant le coude, les sept représentants de la Haute-Garonne qui, appelés les premiers à prononcer leur verdict sur Louis XVI, avaient ainsi répondu l’un après l’autre : Mailhe : la mort. — Delmas : la mort. — Projean : la mort. — Calès : la mort. — Ayral : la mort. — Julien : la mort. — Desascy : la mort. Éternelle répercussion qui emplit toute l’histoire, et qui, depuis que la justice humaine existe, a toujours mis l’écho du sépulcre sur le mur du tribunal. On désignait du doigt, dans la tumultueuse mêlée des visages, tous ces hommes d’où était sorti le brouhaha des votes tragiques ; Paganel, qui avait dit : \emph{La mort. Un roi n’est utile que par sa mort ;} Millaud, qui avait dit : \emph{Aujourd’hui, si la mort n’existait pas, il faudrait l’inventer ;} le vieux Raffron du Trouillet, qui avait dit : \emph{La mort vite !} Goupilleau, qui avait crié : \emph{L’échafaud tout de suite. La lenteur aggrave la mort ;}  Sieyès, qui avait eu cette concision funèbre : \emph{La mort ; } Thuriot, qui avait rejeté l’appel au peuple proposé par Buzot : \emph{Quoi ! les assemblées primaires ! quoi ! quarante-quatre mille tribunaux ! Procès sans terme. La tête de Louis XVI aurait le temps de blanchir avant de tomber ;} Augustin-Bon Robespierre, qui, après son frère, s’était écrié : \emph{Je ne connais point l’humanité qui égorge les peuples et qui pardonne aux despotes. La mort ! Demander un sursis, c’est substituer à l’appel au peuple un appel aux tyrans ;} Foussedoire, le remplaçant de Bernardin de Saint-Pierre, qui avait dit : \emph{J’ai en horreur l’effusion du sang humain, mais le sang d’un roi n’est pas le sang d’un homme. La mort ;} Jean-Bon-Saint-André, qui avait dit : \emph{Pas de peuple libre sans le tyran mort ;} Lavicomterie, qui avait proclamé cette formule : \emph{Tant que le tyran respire, la liberté étouffe. La mort ;} Chateauneuf-Randon, qui avait jeté ce cri : \emph{La mort de Louis le Dernier !} Guyardin, qui avait émis ce vœu : \emph{Qu’on l’exécute Barrière-Renversée !} la Barrière-Renversée c’était la barrière du Trône ; Tellier, qui avait dit : \emph{Qu’on forge, pour tirer contre l’ennemi, un canon du calibre de la tête de Louis XVI}. Et les indulgents : Gentil, qui avait dit : \emph{Je vote la réclusion. Faire un Charles I\textsuperscript{er}, c’est faire un Cromwell ;} Bancal, qui avait dit : \emph{L’exil. Je veux voir le premier roi de l’univers condamné à faire un métier pour gagner sa vie ;} Albouys, qui avait dit : \emph{Le bannissement. Que ce spectre vivant aille errer autour des trônes ;} Zangiacomi, qui avait dit : \emph{La détention. Gardons Capet vivant comme épouvantail ;} Chaillon, qui avait dit : \emph{Qu’il vive. Je ne veux pas faire  un mort dont Rome fera un saint.} Pendant que ces sentences tombaient de ces lèvres sévères et, l’une après l’autre, se dispersaient dans l’histoire, dans les tribunes des femmes décolletées et parées comptaient les voix, une liste à la main, et piquaient des épingles sous chaque vote.\par
Où est entrée la tragédie, l’horreur et la pitié restent.\par
Voir la Convention, à quelque époque de son règne que ce fût, c’était revoir le jugement du dernier Capet ; la légende du 21 janvier semblait mêlée à tous ses actes ; la redoutable assemblée était pleine de ces haleines fatales qui avaient passé sur le vieux flambeau monarchique allumé depuis dix-huit siècles, et l’avaient éteint ; le décisif procès de tous les rois dans un roi était comme le point de départ de la grande guerre qu’elle faisait au passé ; quelle que fût la séance de la Convention à laquelle on assistât, on voyait s’y projeter l’ombre portée de l’échafaud de Louis XVI ; les spectateurs se racontaient les uns aux autres la démission de Kersaint, la démission de Roland, Duchâtel le député des Deux-Sèvres, qui se fit apporter malade sur son lit, et, mourant, vota la vie, ce qui fit rire Marat ; et l’on cherchait des yeux le représentant, oublié par l’histoire aujourd’hui, qui, après cette séance de trente-sept heures, tombé de lassitude et de sommeil sur son banc, et réveillé par l’huissier quand ce fut son tour de voter, entr’ouvrit les yeux, dit : \emph{La mort !} et se rendormit.\par
Au moment où ils condamnèrent à mort Louis XVI,  Robespierre avait encore dix-huit mois à vivre, Danton quinze mois, Vergniaud neuf mois, Marat cinq mois et trois semaines, Lepelletier-Saint-Fargeau un jour. Court et terrible souffle des bouches humaines !\par

\labelblock{viii}

\noindent Le peuple avait sur la Convention une fenêtre ouverte, les tribunes publiques, et, quand la fenêtre ne suffisait pas, il ouvrait la porte, et la rue entrait dans l’assemblée. Ces invasions de la foule dans ce sénat sont une des plus surprenantes visions de l’histoire. Habituellement, ces irruptions étaient cordiales. Le carrefour fraternisait avec la chaise curule. Mais c’est une cordialité redoutable que celle d’un peuple qui, un jour, en trois heures, avait pris les canons des Invalides et quarante mille fusils. A chaque instant, un défilé interrompait la séance ; c’étaient des députations admises à la barre, des pétitions, des hommages, des offrandes. La pique d’honneur du faubourg Saint-Antoine entrait, portée par des femmes. Des Anglais offraient vingt mille souliers aux pieds nus de nos soldats. « Le citoyen Arnoux, disait le \emph{Moniteur}, curé d’Aubignan, commandant du bataillon de la Drôme, demande à marcher aux frontières, et que sa cure lui soit conservée. » Les délégués des sections arrivaient apportant sur des brancards des plats, des patènes, des calices, des  ostensoirs, des monceaux d’or, d’argent et de vermeil, offerts à la patrie par cette multitude en haillons, et demandaient pour récompense la permission de danser la carmagnole devant la Convention. Chenard, Narbonne et Vallière venaient chanter des couplets en l’honneur de la Montagne. La section du Mont-Blanc apportait le buste de Lepelletier, et une femme posait un bonnet rouge sur la tête du président qui l’embrassait ; « les citoyennes de la section du Mail » jetaient des fleurs « aux législateurs » ; les « élèves de la patrie » venaient, musique en tête, remercier la Convention d’avoir « préparé la prospérité du siècle » ; les femmes de la section des Gardes-Françaises offraient des roses ; les femmes de la section des Champs-Élysées offraient une couronne de chêne ; les femmes de la section du Temple venaient à la barre jurer \emph{de ne s’unir qu’à de vrais républicains ;} la section de Molière présentait une médaille de Franklin qu’on suspendait, par décret, à la couronne de la statue de la Liberté ; les enfants-trouvés, déclarés enfants de la république, défilaient, revêtus de l’uniforme national ; les jeunes filles de la section de Quatrevingt-douze arrivaient en longues robes blanches, et le lendemain le \emph{Moniteur} contenait cette ligne : « Le président reçoit un bouquet des mains innocentes d’une jeune beauté. » Les orateurs saluaient les foules ; parfois ils les flattaient ; ils disaient à la multitude : — \emph{Tu es infaillible, tu es irréprochable, tu es sublime ;} — le peuple a un côté enfant, il aime ces sucreries. Quelquefois l’émeute traversait  l’assemblée, y entrait furieuse et sortait apaisée, comme le Rhône qui traverse le lac Léman, et qui est de fange en y entrant et d’azur en en sortant.\par
Parfois c’était moins pacifique, et Henriot faisait apporter devant la porte des Tuileries des grils à rougir les boulets.\par

\labelblock{ix}

\noindent En même temps qu’elle dégageait de la révolution, cette assemblée produisait de la civilisation. Fournaise, mais forge. Dans cette cuve où bouillonnait la terreur, le progrès fermentait. De ce chaos d’ombre et de cette tumultueuse fuite de nuages, sortaient d’immenses rayons de lumière parallèles aux lois éternelles. Rayons restés sur l’horizon, visibles à jamais dans le ciel des peuples, et qui sont l’un la justice, l’autre la tolérance, l’autre la bonté, l’autre la raison, l’autre la vérité, l’autre l’amour. La Convention promulguait ce grand axiome : \emph{La liberté du citoyen finit où la liberté d’un autre citoyen commence ;} ce qui résume en deux lignes toute la sociabilité humaine. Elle déclarait l’indigence sacrée ; elle déclarait l’infirmité sacrée dans l’aveugle et dans le sourd-muet devenus pupilles de l’état, la maternité sacrée dans la fille-mère qu’elle consolait et relevait, l’enfance sacrée dans l’orphelin qu’elle faisait adopter par la patrie, l’innocence sacrée dans l’accusé acquitté qu’elle indemnisait. Elle  flétrissait la traite des noirs ; elle abolissait l’esclavage. Elle proclamait la solidarité civique. Elle décrétait l’instruction gratuite. Elle organisait l’éducation nationale par l’école normale à Paris, l’école centrale au chef-lieu, et l’école primaire dans la commune. Elle créait les conservatoires et les musées. Elle décrétait l’unité de code, l’unité de poids et de mesures, et l’unité de calcul par le système décimal. Elle fondait les finances de la France, et à la longue banqueroute monarchique elle faisait succéder le crédit public. Elle donnait à la circulation le télégraphe, à la vieillesse les hospices dotés, à la maladie les hôpitaux purifiés, à l’enseignement l’école polytechnique, à la science le bureau des longitudes, à l’esprit humain l’institut. En même temps que nationale, elle était cosmopolite. Des onze mille deux cent dix décrets qui sont sortis de la Convention, un tiers a un but politique, les deux tiers ont un but humain. Elle déclarait la morale universelle base de la société et la conscience universelle base de la loi. Et tout cela, servitude abolie, fraternité proclamée, humanité protégée, conscience humaine rectifiée, loi du travail transformée en droit et d’onéreuse devenue secourable, richesse nationale consolidée, enfance éclairée et assistée, lettres et sciences propagées, lumière allumée sur tous les sommets, aide à toutes les misères, promulgation de tous les principes, la Convention le faisait, ayant dans les entrailles cette hydre, la Vendée, et sur les épaules ce tas de tigres, les rois.\par
 
\labelblock{x}

\noindent Lieu immense. Tous les types humains, inhumains et surhumains étaient là. Amas épique d’antagonismes. Guillotin évitant David, Bazire insultant Chabot, Guadet raillant Saint-Just, Vergniaud dédaignant Danton, Louvet attaquant Robespierre, Buzot dénonçant Égalité, Chambon flétrissant Pache, tous exécrant Marat. Et que de noms encore il faudrait enregistrer ! Armonville, dit Bonnet-Rouge, parce qu’il ne siégeait qu’en bonnet phrygien, ami de Robespierre, et voulant, « après Louis XVI, guillotiner Robespierre » par goût de l’équilibre ; Massieu, collègue et ménechme de ce bon Lamourette, évêque fait pour laisser son nom à un baiser ; Lehardy du Morbihan stigmatisant les prêtres de Bretagne ; Barère, l’homme des majorités, qui présidait quand Louis XVI parut à la barre, et qui était à Paméla ce que Louvet était à Lodoïska ; l’oratorien Daunou qui disait : \emph{Gagnons du temps ;} Dubois-Crancé, à l’oreille de qui se penchait Marat ; le marquis de Chateauneuf, Laclos, Hérault de Séchelles qui reculait devant Henriot criant : \emph{Canonniers, à vos pièces !} Julien, qui comparait la Montagne aux Thermopyles ; Gamon, qui voulait une tribune publique réservée uniquement aux femmes ; Laloy, qui décerna les honneurs de la séance à l’évêque Gobel venant à la Convention déposer la mitre et  coiffer le bonnet rouge ; Lecomte, qui s’écriait : \emph{C’est donc à qui se déprêtrisera !} Féraud, dont Boissy-d’Anglas saluera la tête, laissant à l’histoire cette question : — Boissy-d’Anglas a-t-il salué la tête, c’est-à-dire la victime, ou la pique, c’est-à-dire les assassins ? — Les deux frères Duprat, l’un montagnard, l’autre girondin, qui se haïssaient comme les deux frères Chénier.\par
Il s’est dit à cette tribune de ces vertigineuses paroles qui ont quelquefois, à l’insu même de celui qui les prononce, l’accent fatidique des révolutions, et à la suite desquelles les faits matériels paraissent avoir brusquement on ne sait quoi de mécontent et de passionné, comme s’ils avaient mal pris les choses qu’on vient d’entendre ; ce qui se passe semble courroucé de ce qui se dit ; les catastrophes surviennent, furieuses et comme exaspérées par les paroles des hommes. Ainsi une voix dans la montagne suffit pour détacher l’avalanche. Un mot de trop peut être suivi d’un écroulement. Si l’on n’avait pas parlé, cela ne serait pas arrivé. On dirait parfois que les événements sont irascibles.\par
C’est de cette façon, c’est par le hasard d’un mot d’orateur mal compris qu’est tombée la tête de madame Élisabeth.\par
A la Convention l’intempérance de langage était de droit.\par
Les menaces volaient et se croisaient dans la discussion comme les flammèches dans l’incendie. — P{\scshape étion} : Robespierre, venez au fait. — R{\scshape obespierre} :  Le fait, c’est vous, Pétion. J’y viendrai, et vous le verrez. — U{\scshape ne voix} : Mort à Marat ! — M{\scshape arat} : Le jour où Marat mourra, il n’y aura plus de Paris, et le jour où Paris périra, il n’y aura plus de république. — Billaud-Varennes se lève et dit : Nous voulons... — Barère l’interrompt : Tu parles comme un roi. — Un autre jour, P{\scshape hilippeaux} : Un membre a tiré l’épée contre moi. — A{\scshape udouin} : Président, rappelez à l’ordre l’assassin. — L{\scshape e} P{\scshape résident} : Attendez. — P{\scshape anis} : Président, je vous rappelle à l’ordre, moi. — On riait aussi, rudement. — L{\scshape ecointre} : Le curé du Chant-de-Bout se plaint de Fauchet, son évêque, qui lui défend de se marier. — U{\scshape ne voix} : Je ne vois pas pourquoi Fauchet, qui a des maîtresses, veut empêcher les autres d’avoir des épouses. — U{\scshape ne autre voix} : Prêtre, prends femme ! — Les tribunes se mêlaient à la conversation. Elles tutoyaient l’assemblée. Un jour le représentant Ruamps monte à la tribune. Il avait une « hanche » beaucoup plus grosse que l’autre. Un des spectateurs lui cria : — Tourne ça du côté de la droite, puisque tu as une « joue » à la David ! — Telles étaient les libertés que le peuple prenait avec la Convention. Une fois pourtant, dans le tumulte du 11 avril 1793, le président fit arrêter un interrupteur des tribunes.\par
Un jour, cette séance a eu pour témoin le vieux Buonarotti, Robespierre prend la parole et parle deux heures, regardant Danton, tantôt fixement, ce qui était grave, tantôt obliquement, ce qui était pire. Il foudroie à bout portant. Il termine par une explosion indignée, pleine de mots funèbres : — On connaît les  intrigants, on connaît les corrupteurs et les corrompus, on connaît les traîtres ; ils sont dans cette assemblée. Ils nous entendent, nous les voyons et nous ne les quittons pas des yeux. Qu’ils regardent au-dessus de leur tête, et ils y verront le glaive de la loi. Qu’ils regardent dans leur conscience, et ils y verront leur infamie. Qu’ils prennent garde à eux. — Et quand Robespierre a fini, Danton, la face au plafond, les yeux à demi fermés, un bras pendant par-dessus le dossier de son banc, se renverse en arrière, et on l’entend fredonner :\par


\begin{verse}
Cadet Roussel fait des discours\\
Qui ne sont pas longs quand ils sont courts.\\
\end{verse}

\noindent Les imprécations se donnaient la réplique. — Conspirateur ! — Assassin ! — Scélérat ! — Factieux ! — Modéré ! — On se dénonçait au buste de Brutus qui était là. Apostrophes, injures, défis. Regards furieux d’un côté à l’autre. Poings montrés, pistolets entrevus, poignards à demi tirés. Énorme flamboiement de la tribune. Quelques-uns parlaient comme s’ils étaient adossés à la guillotine. Les têtes ondulaient, épouvantées et terribles. Montagnards, girondins, feuillants, modérantistes, terroristes, jacobins, cordeliers ; dix-huit prêtres régicides.\par
Tous ces hommes ! tas de fumées poussées dans tous les sens.\par
 
\labelblock{xi}

\noindent Esprits en proie au vent.\par
Mais ce vent était un vent de prodige.\par
Être un membre de la Convention, c’était être une vague de l’océan. Et ceci était vrai des plus grands. La force d’impulsion venait d’en haut. Il y avait dans la Convention une volonté qui était celle de tous et n’était celle de personne. Cette volonté était une idée, idée indomptable et démesurée qui soufflait dans l’ombre du haut du ciel. Nous appelons cela la Révolution. Quand cette idée passait, elle abattait l’un et soulevait l’autre ; elle emportait celui-ci en écume et brisait celui-là aux écueils. Cette idée savait où elle allait, et poussait le gouffre devant elle. Imputer la révolution aux hommes, c’est imputer la marée aux flots.\par
La révolution est une action de l’inconnu. Appelez-la bonne action ou mauvaise action, selon que vous aspirez à l’avenir ou au passé, mais laissez-la à celui qui l’a faite. Elle semble l’œuvre en commun des grands événements et des grands individus mêlés, mais elle est en réalité la résultante des événements. Les événements dépensent, les hommes payent. Les événements dictent, les hommes signent. Le 14 juillet est signé Camille Desmoulins, le 10 août est signé Danton, le 2 septembre est signé Marat, le 21 septembre est  signé Grégoire, le 21 janvier est signé Robespierre ; mais Desmoulins, Danton, Marat, Grégoire et Robespierre ne sont que des greffiers. Le rédacteur énorme et sinistre de ces grandes pages a un nom, Dieu, et un masque, Destin. Robespierre croyait en Dieu. Certes !\par
La révolution est une forme du phénomène immanent qui nous presse de toutes parts et que nous appelons la Nécessité.\par
Devant cette mystérieuse complication de bienfaits et de souffrances se dresse le Pourquoi ? de l’histoire.\par
\emph{Parce que}. Cette réponse de celui qui ne sait rien est aussi la réponse de celui qui sait tout.\par
En présence de ces catastrophes climatériques qui dévastent et vivifient la civilisation, on hésite à juger le détail. Blâmer ou louer les hommes à cause du résultat, c’est presque comme si on louait ou blâmait les chiffres à cause du total. Ce qui doit passer passe, ce qui doit souffler souffle. La sérénité éternelle ne souffre pas de ces aquilons. Au-dessus des révolutions la vérité et la justice demeurent comme le ciel étoilé au-dessus des tempêtes.\par

\labelblock{xii}

\noindent Telle était cette Convention démesurée ; camp retranché du genre humain attaqué par toutes les ténèbres à la fois, feux nocturnes d’une armée d’idées  assiégées, immense bivouac d’esprits sur un versant d’abîme. Rien dans l’histoire n’est comparable à ce groupe, à la fois sénat et populace, conclave et carrefour, aréopage et place publique, tribunal et accusé.\par
La Convention a toujours ployé au vent ; mais ce vent sortait de la bouche du peuple et était le souffle de Dieu.\par
Et aujourd’hui, après quatrevingts ans écoulés, chaque fois que devant la pensée d’un homme, quel qu’il soit, historien ou philosophe, la Convention apparaît, cet homme s’arrête et médite. Impossible de ne pas être attentif à ce grand passage d’ombres.
 \subsubsection[{II. Marat dans la coulisse}]{II \\
Marat dans la coulisse}\phantomsection
\label{p2l3c2}
\noindent Comme il l’avait annoncé à Simonne Évrard, Marat, le lendemain de la rencontre de la rue du Paon, alla à la Convention.\par
Il y avait à la Convention un marquis maratiste, Louis de Montaut, celui qui plus tard offrit à la Convention une pendule décimale surmontée du buste de Marat.\par
Au moment où Marat entrait, Chabot venait de s’approcher de Montaut.\par
— Ci-devant..., dit-il.\par
Montaut leva les yeux.\par
— Pourquoi m’appelles-tu ci-devant ?\par
— Parce que tu l’es.\par
— Moi ?\par
— Puisque tu étais marquis.\par
— Jamais.\par
— Bah !\par
— Mon père était soldat, mon grand-père était tisserand.\par
— Qu’est-ce que tu nous chantes-là, Montaut ?\par
— Je ne m’appelle pas Montaut.\par
 — Comment donc t’appelles-tu ?\par
— Je m’appelle Maribon.\par
— Au fait, dit Chabot, cela m’est égal.\par
Et il ajouta entre ses dents :\par
— C’est à qui ne sera pas marquis.\par
Marat s’était arrêté dans le couloir de gauche et regardait Montaut et Chabot.\par
Toutes les fois que Marat entrait, il y avait une rumeur ; mais loin de lui. Autour de lui on se taisait. Marat n’y prenait pas garde. Il dédaignait le « coassement du marais ».\par
Dans la pénombre des bancs obscurs d’en bas, Conpé de l’Oise, Prunelle, Villars, évêque, qui plus tard fut membre de l’Académie française, Boutroue, Petit, Plaichard, Bonet, Thibaudeau, Valdruche, se le montraient du doigt.\par
— Tiens ! Marat !\par
— Il n’est donc pas malade ?\par
— Si, puisqu’il est en robe de chambre.\par
— En robe de chambre ?\par
— Pardieu oui !\par
— Il se permet tout !\par
— Il ose venir ainsi à la Convention !\par
— Puisqu’un jour il y est venu coiffé de lauriers, il peut bien y venir en robe de chambre !\par
— Face de cuivre et dents de vert-de-gris.\par
— Sa robe de chambre paraît neuve.\par
— En quoi est-elle ?\par
— En reps.\par
— Rayé.\par
 — Regardez donc les revers.\par
— Ils sont en peau.\par
— De tigre.\par
— Non, d’hermine.\par
— Fausse.\par
— Et il a des bas !\par
— C’est étrange.\par
— Et des souliers à boucles.\par
— D’argent !\par
— Voilà ce que les sabots de Camboulas ne lui pardonneront pas.\par
Sur d’autres bancs on affectait de ne pas voir Marat. On causait d’autre chose. Santhonax abordait Dussaulx.\par
— Vous savez, Dussaulx ?\par
— Quoi ?\par
— Le ci-devant comte de Brienne ?\par
— Qui était à la Force avec le ci-devant duc de Villeroy ?\par
— Oui.\par
— Je les ai connus tous les deux. Eh bien ?\par
— Ils avaient si grand’peur qu’ils saluaient tous les bonnets rouges de tous les guichetiers, et qu’un jour ils ont refusé de jouer une partie de piquet parce qu’on leur présentait un jeu de cartes à rois et à reines.\par
— Eh bien ?\par
— On les a guillotinés hier.\par
— Tous les deux ?\par
— Tous les deux.\par
 — En somme, comment avaient-ils été dans la prison ?\par
— Lâches.\par
— Et comment ont-ils été sur l’échafaud ?\par
— Intrépides.\par
Et Dussaulx jetait cette exclamation :\par
— Mourir est plus facile que vivre.\par
Barère était en train de lire un rapport ; il s’agissait de la Vendée. Neuf cents hommes du Morbihan étaient partis avec du canon pour secourir Nantes. Redon était menacé par les paysans. Paimbœuf était attaqué. Une station navale croisait à Maindrin pour empêcher les descentes. Depuis Ingrande jusqu’à Maure, toute la rive gauche de la Loire était hérissée de batteries royalistes. Trois mille paysans étaient maîtres de Pornic. Ils criaient : \emph{Vivent les Anglais !} Une lettre de Santerre à la Convention, que Barère lisait, se terminait ainsi : « Sept mille paysans ont attaqué Vannes. Nous les avons repoussés, et ils ont laissé dans nos mains quatre canons... »\par
— Et combien de prisonniers ? interrompit une voix.\par
Barère continua... — Post-scriptum de la lettre : « Nous n’avons pas de prisonniers, parce que nous n’en faisons plus\footnote{ \noindent \emph{Moniteur,} t. XIX, p. 81.
 }. »\par
Marat toujours immobile n’écoutait pas, il était comme absorbé par une préoccupation sévère.\par
Il tenait dans sa main et froissait entre ses doigts  un papier sur lequel quelqu’un qui l’eût déplié eût pu lire ces lignes, qui étaient de l’écriture de Momoro, et qui étaient probablement une réponse à une question posée par Marat :\par
« — Il n’y a rien à faire contre l’omnipotence des commissaires délégués, surtout contre les délégués du comité de salut public. Génissieux a eu beau dire dans la séance du 6 mai : « \emph{Chaque commissaire est plus qu’un roi}, » cela n’y fait rien. Ils ont pouvoir de vie et de mort. Massade à Angers, Trullard à Saint-Amand, Nyon près du général Marcé, Parrein à l’armée des Sables, Millières à l’armée de Niort, sont tout-puissants. Le club des Jacobins a été jusqu’à nommer Parrein général de brigade. Les circonstances absolvent tout. Un délégué du comité de salut public tient en échec un général en chef. »\par
Marat acheva de froisser le papier, le mit dans sa poche, et s’avança lentement vers Montaut et Chabot qui continuaient à causer et ne l’avaient pas vu entrer.\par
Chabot disait :\par
— Maribon ou Montaut, écoute ceci : je sors du comité de salut public.\par
— Et qu’y fait-on ?\par
— On y donne un noble à garder à un prêtre.\par
— Ah !\par
— Un noble comme toi...\par
— Je ne suis pas noble, dit Montaut.\par
— A un prêtre...\par
— Comme toi.\par
— Je ne suis pas prêtre, dit Chabot.\par
 Tous deux se mirent à rire.\par
— Précise l’anecdote, repartit Montaut.\par
— Voici ce que c’est. Un prêtre appelé Cimourdain est délégué avec pleins pouvoirs près d’un vicomte nommé Gauvain ; ce vicomte commande la colonne expéditionnaire de l’armée des Côtes. Il s’agit d’empêcher le noble de tricher et le prêtre de trahir.\par
— C’est bien simple, répondit Montaut. Il n’y a qu’à mettre la mort dans l’aventure.\par
— Je viens pour cela, dit Marat.\par
Ils levèrent la tête.\par
— Bonjour, Marat, dit Chabot, tu assistes rarement à nos séances.\par
— Mon médecin me commande les bains, répondit Marat.\par
— Il faut se défier des bains, reprit Chabot ; Sénèque est mort dans un bain.\par
Marat sourit :\par
— Chabot, il n’y a pas ici de Néron.\par
— Il y a toi, dit une voix rude.\par
C’était Danton qui passait et qui montait à son banc.\par
Marat ne se retourna pas.\par
Il pencha sa tête entre les deux visages de Montaut et de Chabot.\par
— Écoutez. Je viens pour une chose sérieuse. Il faut qu’un de nous trois propose aujourd’hui un projet de décret à la Convention.\par
— Pas moi, dit Montaut ; on ne m’écoute pas, je suis marquis.\par
 — Moi, dit Chabot, on ne m’écoute pas, je suis capucin.\par
— Et moi, dit Marat, on ne m’écoute pas, je suis Marat.\par
Il y eut entre eux un silence.\par
Marat préoccupé n’était pas aisé à interroger. Montaut pourtant hasarda une question.\par
— Marat, quel est le décret que tu désires ?\par
— Un décret qui punisse de mort tout chef militaire qui fait évader un rebelle prisonnier.\par
Chabot intervint.\par
— Ce décret existe. On a voté cela fin avril.\par
— Alors c’est comme s’il n’existait pas, dit Marat. Partout, dans toute la Vendée, c’est à qui fera évader les prisonniers, et l’asile est impuni.\par
— Marat, c’est que le décret est en désuétude.\par
— Chabot, il faut le remettre en vigueur.\par
— Sans doute.\par
— Et pour cela parler à la Convention.\par
— Marat, la Convention n’est pas nécessaire ; le comité de salut public suffit.\par
— Le but est atteint, ajouta Montaut, si le comité de salut public fait placarder le décret dans toutes les communes de la Vendée, et fait deux ou trois bons exemples.\par
— Sur les grandes têtes, reprit Chabot. Sur les généraux.\par
Marat grommela : — En effet, cela suffira.\par
— Marat, repartit Chabot, va toi-même dire cela au comité de salut public.\par
 Marat le regarda entre les deux yeux, ce qui n’était pas agréable, même pour Chabot.\par
— Chabot, dit-il, le comité de salut public, c’est chez Robespierre. Je ne vais pas chez Robespierre.\par
— J’irai, moi, dit Montaut.\par
— Bien, dit Marat.\par
Le lendemain était expédié dans toutes les directions un ordre du comité de salut public enjoignant d’afficher dans les villes et villages de Vendée et de faire exécuter strictement le décret portant peine de mort contre toute connivence dans les évasions de brigands et d’insurgés prisonniers.\par
Ce décret n’était qu’un premier pas. La Convention devait aller plus loin encore. Quelques mois après, le 11 brumaire an II (novembre 1793), à propos de Laval qui avait ouvert ses portes aux Vendéens fugitifs, elle décréta que toute ville qui donnerait asile aux rebelles serait démolie et détruite.\par
De leur côté, les princes de l’Europe, dans le manifeste du duc de Brunswick, inspiré par les émigrés et rédigé par le marquis de Linnon, intendant du duc d’Orléans, avaient déclaré que tout Français pris les armes à la main serait fusillé, et que, si un cheveu tombait de la tête du roi, Paris serait rasé.\par
Sauvagerie contre barbarie.\par
  \section[{Troisième partie. En Vendée}]{Troisième partie \\
En Vendée}\phantomsection
\label{p3}\renewcommand{\leftmark}{Troisième partie \\
En Vendée}

  \subsection[{Livre premier. La Vendée}]{Livre premier \\
La Vendée}\phantomsection
\label{p3l1}
\subsubsection[{I. Les forêts}]{I \\
Les forêts}\phantomsection
\label{p3l1c1}
\noindent Il y avait alors en Bretagne sept forêts horribles. La Vendée, c’est la révolte-prêtre. Cette révolte a eu pour auxiliaire la forêt. Les ténèbres s’entr’aident.\par
Les sept Forêts-Noires de Bretagne étaient la forêt de Fougères qui barre le passage entre Dol et Avranches ; la forêt de Princé qui a huit lieues de tour ; la forêt de Paimpont, pleine de ravines et de ruisseaux, presque inaccessible du côté de Baignon, avec une retraite facile sur Concornet qui était un bourg royaliste ; la forêt de Rennes d’où l’on entendait le tocsin des paroisses républicaines, toujours nombreuses près des villes ; c’est là que Puysaye perdit Focard ; la forêt de Machecoul qui avait Charette pour bête fauve ; la forêt de la Garnache qui était aux La Trémoille, aux Gauvain et aux Rohan ; la forêt de Brocéliande qui était aux fées.\par
 Un gentilhomme en Bretagne avait le titre de \emph{seigneur des Sept-Forêts}. C’était le vicomte de Fontenay, prince breton.\par
Car le prince breton existait, distinct du prince français. Les Rohan étaient princes bretons. Garnier de Saintes, dans son rapport à la Convention, 15 nivôse an II, qualifie ainsi le prince de Talmont : « Ce Capet des brigands, souverain du Maine et de la Normandie. »\par
L’histoire des forêts bretonnes, de 1792 à 1800, pourrait être faite à part, et elle se mêlerait à la vaste aventure de la Vendée comme une légende.\par
L’histoire a sa vérité, la légende a la sienne. La vérité légendaire est d’une autre nature que la vérité historique. La vérité légendaire, c’est l’invention ayant pour résultat la réalité. Du reste l’histoire et la légende ont le même but, peindre sous l’homme momentané l’homme éternel.\par
La Vendée ne peut être complètement expliquée que si la légende complète l’histoire ; il faut l’histoire pour l’ensemble et la légende pour le détail.\par
Disons que la Vendée en vaut la peine. La Vendée est un prodige.\par
Cette Guerre des Ignorants, si stupide et si splendide, abominable et magnifique, a désolé et enorgueilli la France. La Vendée est une plaie qui est une gloire.\par
A de certaines heures la société humaine a ses énigmes, énigmes qui pour les sages se résolvent en lumière et pour les ignorants en obscurité, en violence et en barbarie. Le philosophe hésite à accuser. Il tient compte du trouble que produisent les problèmes.  Les problèmes ne passent point sans jeter au-dessous d’eux une ombre comme les nuages.\par
Si l’on veut comprendre la Vendée, qu’on se figure cet antagonisme, d’un côté la révolution française, de l’autre le paysan breton. En face de ces événements incomparables, menace immense de tous les bienfaits à la fois, accès de colère de la civilisation, excès du progrès furieux, amélioration démesurée et inintelligible, qu’on place ce sauvage grave et singulier, cet homme à l’œil clair et aux longs cheveux, vivant de lait et de châtaignes, borné à son toit de chaume, à sa haie et à son fossé, distinguant chaque hameau du voisinage au son de la cloche, ne se servant de l’eau que pour boire, ayant sur le dos une veste de cuir avec des arabesques de soie, inculte et brodé, tatouant ses habits, comme ses ancêtres les Celtes avaient tatoué leurs visages, respectant son maître dans son bourreau, parlant une langue morte, ce qui est faire habiter une tombe à sa pensée, piquant ses bœufs, aiguisant sa faulx, sarclant son blé noir, pétrissant sa galette de sarrasin, vénérant sa charrue d’abord, sa grand’mère ensuite, croyant à la sainte Vierge et à la Dame blanche, dévot à l’autel et aussi à la haute pierre mystérieuse debout au milieu de la lande, laboureur dans la plaine, pêcheur sur la côte, braconnier dans le hallier, aimant ses rois, ses seigneurs, ses prêtres, ses poux ; pensif, immobile souvent des heures entières sur la grande grève déserte, sombre écouteur de la mer.\par
Et qu’on se demande si cet aveugle pouvait accepter cette clarté.
 \subsubsection[{II. Les hommes}]{II \\
Les hommes}\phantomsection
\label{p3l1c2}
\noindent Le paysan a deux points d’appui : le champ qui le nourrit, le bois qui le cache.\par
Ce qu’étaient les forêts bretonnes, on se le figurerait difficilement ; c’étaient des villes. Rien de plus sourd, de plus muet et de plus sauvage que ces inextricables enchevêtrements d’épines et de branchages ; ces vastes broussailles étaient des gîtes d’immobilité et de silence ; pas de solitude d’apparence plus morte et plus sépulcrale ; si l’on eût pu, subitement et d’un seul coup pareil à l’éclair, couper les arbres, on eût brusquement vu dans cette ombre un fourmillement d’hommes.\par
Des puits ronds et étroits, masqués au dehors par des couvercles de pierre et de branches, verticaux, puis horizontaux, s’élargissant sous terre en entonnoir, et aboutissant à des chambres ténébreuses, voilà ce que Cambyse trouva en Égypte et ce que Westermann trouva en Bretagne ; là c’était dans le désert, ici c’était dans la forêt ; dans les caves d’Égypte il y avait des morts, dans les caves de Bretagne il y avait des vivants. Une des plus sauvages clairières du bois de Misdon,  toute perforée de galeries et de cellules où allait et venait un peuple mystérieux, s’appelait « la Grande ville. » Une autre clairière, non moins déserte en dessus et non moins habitée en dessous, s’appelait « la Place royale ».\par
Cette vie souterraine était immémoriale en Bretagne. De tout temps l’homme y avait été en fuite devant l’homme. De là les tanières de reptiles creusées sous les racines des arbres. Cela datait des druides, et quelques-unes de ces cryptes étaient aussi anciennes que les dolmens. Les larves de la légende et les monstres de l’histoire, tout avait passé sur ce noir pays, Teutatès, César, Hoël, Néomène, Geoffroy d’Angleterre, Alain-gant-de-fer, Pierre Mauclerc, la maison française de Blois, la maison anglaise de Montfort, les rois et les ducs, les neuf barons de Bretagne, les juges des Grands-Jours, les comtes de Nantes querellant les comtes de Rennes, les routiers, les malandrins, les grandes compagnies, René II, vicomte de Rohan, les gouverneurs pour le roi, le « bon duc de Chaulnes » branchant les paysans sous les fenêtres de madame de Sévigné, au quinzième siècle les boucheries seigneuriales, au seizième et au dix-septième siècle les guerres de religion, au dix-huitième siècle les trente mille chiens dressés à chasser aux hommes ; sous ce piétinement effroyable le peuple avait pris le parti de disparaître. Tour à tour les troglodytes pour échapper aux Celtes, les Celtes pour échapper aux Romains, les Bretons pour échapper aux Normands, les huguenots pour échapper aux catholiques, les contrebandiers  pour échapper aux gabelous, s’étaient réfugiés d’abord dans les forêts, puis sous la terre. Ressource des bêtes. C’est là que la tyrannie réduit les nations. Depuis deux mille ans, le despotisme sous toutes ses espèces, la conquête, la féodalité, le fanatisme, le fisc, traquait cette misérable Bretagne éperdue ; sorte de battue inexorable qui ne cessait sous une forme que pour recommencer sous l’autre. Les hommes se terraient.\par
L’épouvante, qui est une sorte de colère, était toute prête dans les âmes, et les tanières étaient toutes prêtes dans les bois, quand la république française éclata. La Bretagne se révolta, se trouvant opprimée par cette délivrance de force. Méprise habituelle aux esclaves.
 \subsubsection[{III. Connivence des hommes et des forêts}]{III \\
Connivence des hommes et des forêts}\phantomsection
\label{p3l1c3}
\noindent Les tragiques forêts bretonnes reprirent leur vieux rôle et furent servantes et complices de cette rébellion, comme elles l’avaient été de toutes les autres.\par
Le sous-sol de telle forêt était une sorte de madrépore percé et traversé en tous sens par une voirie inconnue de sapes, de cellules et de galeries. Chacune de ces cellules aveugles abritait cinq ou six hommes. La difficulté était d’y respirer. On a de certains chiffres étranges qui font comprendre cette puissante organisation de la vaste émeute paysanne. En Ille-et-Vilaine, dans la forêt du Pertre, asile du prince de Talmont, on n’entendait pas un souffle, on ne trouvait pas une trace humaine, et il y avait six mille hommes avec Focard ; en Morbihan, dans la forêt de Meulac, on ne voyait personne, et il y avait huit mille hommes. Ces deux forêts, le Pertre et Meulac, ne comptent pourtant pas parmi les grandes forêts bretonnes. Si l’on marchait là-dessus, c’était terrible. Ces halliers hypocrites, pleins de combattants tapis dans une sorte de labyrinthe sous-jacent, étaient comme d’énormes éponges obscures d’où, sous la pression de  ce pied gigantesque, la révolution, jaillissait la guerre civile.\par
Des bataillons invisibles guettaient. Ces armées ignorées serpentaient sous les armées républicaines, sortaient de terre tout à coup et y rentraient, bondissaient innombrables et s’évanouissaient, douées d’ubiquité et de dispersion, avalanche puis poussière, colosses ayant le don du rapetissement, géants pour combattre, nains pour disparaître. Des jaguars ayant des mœurs de taupes.\par
Il n’y avait pas que les forêts, il y avait les bois. De même qu’au-dessous des cités il y a les villages, au-dessous des forêts il y avait les broussailles. Les forêts se reliaient entre elles par le dédale, partout épars, des bois. Les anciens châteaux qui étaient des forteresses, les hameaux qui étaient des camps, les fermes qui étaient des enclos faits d’embûches et de pièges, les métairies, ravinées de fossés et palissadées d’arbres, étaient les mailles de ce filet où se prirent les armées républicaines.\par
Cet ensemble était ce qu’on appelait le Bocage.\par
Il y avait le bois de Misdon, au centre duquel était un étang, et qui était à Jean Chouan ; il y avait le bois de Gennes qui était à Taillefer ; il y avait le bois de la Huisserie qui était à Gouge-le-Bruant ; le bois de la Charnie qui était à Courtillé-le-Bâtard, dit l’Apôtre saint Paul, chef du camp de la Vache-Noire ; le bois de Burgault qui était à cet énigmatique Monsieur Jacques, réservé à une fin mystérieuse dans le souterrain de Juvardeil ; il y avait le bois de Charreau  où Pimousse et Petit-Prince, attaqués par la garnison de Châteauneuf, allaient prendre à bras-le-corps dans les rangs républicains des grenadiers qu’ils rapportaient prisonniers ; le bois de la Heureuserie, témoin de la déroute du poste de Longue-Faye ; le bois de l’Aulne d’où l’on épiait la route entre Rennes et Laval ; le bois de la Gravelle qu’un prince de La Trémoille avait gagné en jouant à la boule ; le bois de Lorges dans les Côtes-du-Nord, où Charles de Boishardy régna après Bernard de Villeneuve ; le bois de Bagnard, près Fontenay, où Lescure offrit le combat à Chalbos qui, étant un contre cinq, l’accepta ; le bois de la Durondais que se disputèrent jadis Alain le Redru et Hérispoux, fils de Charles le Chauve ; le bois de Croqueloup, sur la lisière de cette lande où Coquereau tondait les prisonniers ; le bois de la Croix-Bataille qui assista aux insultes homériques de Jambe-d’Argent à Morière et de Morière à Jambe-d’Argent ; le bois de la Saudraie que nous avons vu fouiller par un bataillon de Paris. Bien d’autres encore.\par
Dans plusieurs de ces forêts et de ces bois, il n’y avait pas seulement des villages souterrains groupés autour du terrier du chef ; mais il y avait encore de véritables hameaux de huttes basses cachés sous les arbres, et si nombreux que parfois la forêt en était remplie. Souvent les fumées les trahissaient. Deux de ces hameaux du bois de Misdon sont restés célèbres, Lorrière, près de Létang, et, du côté de Saint-Ouen-les-Toits, le groupe de cabanes appelé la Rue-de-Bau.\par
Les femmes vivaient dans les huttes et les hommes  dans les cryptes. Ils utilisaient pour cette guerre les galeries des fées et les vieilles sapes celtiques. On apportait à manger aux hommes enfouis. Il y en eut qui, oubliés, moururent de faim. C’étaient d’ailleurs des maladroits qui n’avaient pas su rouvrir leurs puits. Habituellement le couvercle fait de mousse et de branches était si artistement façonné, qu’impossible à distinguer du dehors dans l’herbe, il était très facile à ouvrir et à fermer du dedans. Ces repaires étaient creusés avec soin. On allait jeter à quelque étang voisin la terre qu’on ôtait du puits. La paroi intérieure et le sol étaient tapissés de fougère et de mousse. Ils appelaient ce réduit « la loge ». On était bien là, à cela près qu’on était sans jour, sans feu, sans pain et sans air.\par
Remonter sans précaution parmi les vivants et se déterrer hors de propos était grave. On pouvait se trouver entre les jambes d’une armée en marche. Bois redoutables ; pièges à doubles trappes. Les bleus n’osaient entrer, les blancs n’osaient sortir.
 \subsubsection[{IV. Leur vie sous terre}]{IV \\
Leur vie sous terre}\phantomsection
\label{p3l1c4}
\noindent Les hommes dans ces caves de bêtes s’ennuyaient. La nuit, quelquefois, à tout risque, ils sortaient et s’en allaient danser sur la lande voisine. Ou bien ils priaient pour tuer le temps. \emph{Tout le jour}, dit Bourdoiseau, \emph{Jean Chouan nous faisait chapeletter}.\par
Il était presque impossible, la saison venue, d’empêcher ceux du Bas-Maine de sortir pour se rendre à la Fête de la Gerbe. Quelques-uns avaient des idées à eux. Denys, dit Tranche-Montagne, se déguisait en femme pour aller à la comédie à Laval ; puis il rentrait dans son trou.\par
Brusquement ils allaient se faire tuer, quittant le cachot pour le sépulcre.\par
Quelquefois ils soulevaient le couvercle de leur fosse, et ils écoutaient si l’on se battait au loin ; ils suivaient de l’oreille le combat. Le feu des républicains était régulier, le feu des royalistes était éparpillé ; ceci les guidait. Si les feux de peloton cessaient subitement, c’était signe que les royalistes avaient le dessous ; si les feux saccadés continuaient et s’enfonçaient à l’horizon, c’était signe qu’ils avaient le dessus.  Les blancs poursuivaient toujours ; les bleus jamais, ayant le pays contre eux.\par
Ces belligérants souterrains étaient admirablement renseignés. Rien de plus rapide que leurs communications, rien de plus mystérieux. Ils avaient rompu tous les ponts, ils avaient démonté toutes les charrettes, et ils trouvaient moyen de tout se dire et de s’avertir de tout. Des relais d’émissaires étaient établis de forêt à forêt, de village à village, de ferme à ferme, de chaumière à chaumière, de buisson à buisson.\par
Tel paysan qui avait l’air stupide passait portant des dépêches dans son bâton, qui était creux.\par
Un ancien constituant, Boétidoux, leur fournissait, pour aller et venir d’un bout à l’autre de la Bretagne, des passe-ports républicains nouveau modèle, avec les noms en blanc, dont ce traître avait des liasses. Il était impossible de les surprendre. \emph{Des secrets livrés}, dit Puysaye\footnote{ \noindent Tome II, page 35.
 }, \emph{à plus de quatre cent mille individus ont été religieusement gardés.}\par
Il semblait que ce quadrilatère fermé au sud par la ligne des Sables à Thouars, à l’est par la ligne de Thouars à Saumur et par la rivière de Thoué, au nord par la Loire et à l’ouest par l’Océan, eût un même appareil nerveux, et qu’un point de ce sol ne pût tressaillir sans que tout s’ébranlât. En un clin d’œil on était informé de Noirmoutier à Luçon, et le camp de la Loué savait ce que faisait le camp de la Croix- Morineau. On eût dit que les oiseaux s’en mêlaient. Hoche écrivait, 7 messidor, an III : \emph{On croirait qu’ils ont des télégraphes}.\par
C’étaient des clans, comme en Écosse. Chaque paroisse avait son capitaine. Cette guerre, mon père l’a faite, et j’en puis parler.
 \subsubsection[{V. Leur vie en guerre}]{V \\
Leur vie en guerre}\phantomsection
\label{p3l1c5}
\noindent Beaucoup n’avaient que des piques. Les bonnes carabines de chasse abondaient. Pas de plus adroits tireurs que les braconniers du Bocage et les contrebandiers du Loroux. C’étaient des combattants étranges, affreux et intrépides. Le décret de la levée de trois cent mille hommes avait fait sonner le tocsin dans six cents villages. Le pétillement de l’incendie éclata sur tous les points à la fois. Le Poitou et l’Anjou firent explosion le même jour. Disons qu’un premier grondement s’était fait entendre dès 1792, le 8 juillet, un mois avant le 10 août, sur la lande de Kerbader. Alain Redeler, aujourd’hui ignoré, fut le précurseur de La Rochejaquelein et de Jean Chouan. Les royalistes forçaient, sous peine de mort, tous les hommes valides à marcher. Ils réquisitionnaient les attelages, les chariots, les vivres. Tout de suite, Sapinaud eut trois mille soldats, Cathelineau dix mille, Stofflet vingt mille, et Charette fut maître de Noirmoutier. Le vicomte de Scépeaux remua le Haut-Anjou, le chevalier de Dieuzie l’Entre-Vilaine-et-Loire, Tristan-l’Hermite le Bas-Maine, le barbier Gaston la ville de Guéménée,  et l’abbé Bernier tout le reste. Pour soulever ces multitudes, peu de chose suffisait. On plaçait dans le tabernacle d’un curé assermenté, d’un \emph{prêtre jureur}, comme ils disaient, un gros chat noir qui sautait brusquement dehors pendant la messe. — \emph{C’est le diable !} criaient les paysans, et tout un canton s’insurgeait. Un souffle de feu sortait des confessionnaux. Pour assaillir les bleus et pour franchir les ravins, ils avaient leur long bâton de quinze pieds de long, \emph{la ferte}, arme de combat et de fuite. Au plus fort des mêlées, quand les paysans attaquaient les carrés républicains, s’ils rencontraient sur le champ de combat une croix ou une chapelle, tous tombaient à genoux et disaient leur prière sous la mitraille ; le rosaire fini, ceux qui restaient se relevaient et se ruaient sur l’ennemi. Quels géants, hélas ! Ils chargeaient leur fusil en courant ; c’était leur talent. On leur faisait accroire ce qu’on voulait ; les prêtres leur montraient d’autres prêtres dont ils avaient rougi le cou avec une ficelle serrée, et leur disaient : \emph{Ce sont des guillotinés ressuscités}. Ils avaient leurs accès de chevalerie ; ils honorèrent Fesque, un porte-drapeau républicain qui s’était fait sabrer sans lâcher son drapeau. Ces paysans raillaient ; ils appelaient les prêtres mariés républicains \emph{des sans-calottes devenus sans-culottes}. Ils commencèrent par avoir peur des canons ; puis ils se jetèrent dessus avec des bâtons, et ils en prirent. Ils prirent d’abord un beau canon de bronze qu’ils baptisèrent \emph{le Missionnaire ;} puis un autre qui datait des guerres catholiques et où étaient gravées les armes de Richelieu et  une figure de la Vierge ; ils l’appelèrent \emph{Marie-Jeanne}. Quand ils perdirent Fontenay, ils perdirent Marie-Jeanne, autour de laquelle tombèrent sans broncher six cents paysans ; puis ils reprirent Fontenay afin de reprendre Marie-Jeanne, et ils la ramenèrent sous le drapeau fleurdelysé en la couvrant de fleurs et en la faisant baiser aux femmes qui passaient. Mais deux canons, c’était peu. Stofflet avait pris Marie-Jeanne ; Cathelineau, jaloux, partit de Pin-en-Mauge, donna l’assaut à Jallais, et prit un troisième canon ; Forest attaqua Saint-Florent et en prit un quatrième. Deux autres capitaines, Chouppes et Saint-Pol, firent mieux ; ils figurèrent des canons par des troncs d’arbres coupés, et des canonniers par des mannequins, et avec cette artillerie, dont ils riaient vaillamment, ils firent reculer les bleus à Mareuil. C’était là leur grande époque. Plus tard, quand Chalbos mit en déroute La Marsonnière, les paysans laissèrent derrière eux sur le champ de bataille déshonoré trente-deux canons aux armes d’Angleterre. L’Angleterre alors payait les princes français, et l’on envoyait « des fonds à monseigneur, écrivait Nantiat le 10 mai 1794, parce qu’on a dit à M. Pitt que cela était décent ». Mélinet, dans un rapport du 31 mars, dit : « Le cri des rebelles est \emph{Vivent les Anglais !} » Les paysans s’attardaient à piller. Ces dévots étaient des voleurs. Les sauvages ont des vices. C’est par là que les prend plus tard la civilisation. Puysaye dit, tome II, page 187 : « J’ai préservé plusieurs fois le bourg de Plélan du pillage. » Et plus loin, page 434, il se prive d’entrer à Montfort :  « Je fis un circuit pour éviter le pillage des maisons des jacobins. » Ils détroussèrent Chollet ; ils mirent à sac Challans. Après avoir manqué Granville, ils pillèrent Ville-Dieu. Ils appelaient \emph{masse jacobine} ceux des campagnards qui s’étaient ralliés aux bleus, et ils les exterminaient plus que les autres. Ils aimaient le carnage comme des soldats, et le massacre comme des brigands. Fusiller les « patauds », c’est-à-dire les bourgeois, leur plaisait ; ils appelaient cela « se décarêmer ». A Fontenay, un de leurs prêtres, le curé Barbotin, abattit un vieillard d’un coup de sabre. A Saint-Germain-sur-Ille\footnote{ \noindent Puysaye, t. II, p. 35.
 }, un de leurs capitaines, gentilhomme, tua d’un coup de fusil le procureur de la commune et lui prit sa montre. A Machecoul, ils mirent les républicains en coupe réglée, à trente par jour ; cela dura cinq semaines ; chaque chaîne de trente s’appelait « le chapelet ». On adossait la chaîne à une fosse creusée et l’on fusillait ; les fusillés tombaient dans la fosse parfois vivants ; on les enterrait tout de même. Nous avons revu ces mœurs. Joubert, président du district, eut les poings sciés. Ils mettaient aux prisonniers bleus des menottes coupantes, forgées exprès. Ils les assommaient sur les places publiques en sonnant l’hallali. Charette, qui signait : \emph{Fraternité ; le chevalier Charette,} et qui avait pour coiffure, comme Marat, un mouchoir noué sur les sourcils, brûla la ville de Pornic et les habitants dans les maisons. Pendant ce temps-là, Carrier était épouvantable.  La terreur répliquait à la terreur. L’insurgé breton avait presque la figure de l’insurgé grec, veste courte, fusil en bandoulière, jambières, larges braies pareilles à la fustanelle ; le gars ressemblait au klephte. Henri de La Rochejaquelein, à vingt et un ans, partait pour cette guerre avec un bâton et une paire de pistolets. L’armée vendéenne comptait cent cinquante-quatre divisions. Ils faisaient des sièges en règle ; ils tinrent trois jours Bressuire bloquée. Dix mille paysans, un vendredi saint, canonnèrent la ville des Sables à boulets rouges. Il leur arriva de détruire en un seul jour quatorze cantonnements républicains, de Montigné à Courbeveilles. A Thouars, sur la haute muraille, on entendait ce dialogue superbe entre La Rochejaquelein et un gars : — Carle ! — Me voilà. — Tes épaules que je monte dessus. — Faites. — Ton fusil. — Prenez. — Et La Rochejaquelein sauta dans la ville, et l’on prit sans échelles ces tours qu’avait assiégées Duguesclin. Ils préféraient une cartouche à un louis d’or. Ils pleuraient quand ils perdaient de vue leur clocher. Fuir leur semblait simple ; alors les chefs criaient : \emph{Jetez vos sabots, gardez vos fusils !} Quand les munitions manquaient, ils disaient leur chapelet et allaient prendre de la poudre dans les caissons de l’artillerie républicaine ; plus tard d’Elbée en demanda aux Anglais. Quand l’ennemi approchait, s’ils avaient des blessés, ils les cachaient dans les grands blés ou dans les fougères vierges, et, l’affaire finie, venaient les reprendre. D’uniformes point. Leurs vêtements se délabraient. Paysans et gentilshommes s’habillaient  des premiers haillons venus. Roger Mouliniers portait un turban et un dolman pris au magasin de costumes du théâtre de la Flèche ; le chevalier de Beauvilliers avait une robe de procureur et un chapeau de femme par-dessus un bonnet de laine. Tous portaient l’écharpe et la ceinture blanche ; les grades se distinguaient par le nœud. Stofflet avait un nœud rouge ; La Rochejaquelein avait un nœud noir ; Wimpfen, demi-girondin, qui du reste ne sortit pas de Normandie, portait le brassard des carabots de Caen. Ils avaient dans leurs rangs des femmes, madame de Lescure, qui fut plus tard madame de La Rochejaquelein ; Thérèse de Mollien, maîtresse de La Rouarie, laquelle brûla la liste des chefs de paroisse ; madame de La Rochefoucauld, belle, jeune, le sabre à la main, ralliant les paysans au pied de la grosse tour du château du Puy-Rousseau, et cette Antoinette Adams, dite le chevalier Adams, si vaillante que, prise, on la fusilla, mais debout, par respect. Ce temps épique était cruel. On était des furieux. Madame de Lescure faisait exprès marcher son cheval sur les républicains gisant hors de combat ; \emph{morts}, dit-elle ; blessés, peut-être. Quelquefois les hommes trahirent, les femmes jamais. Mademoiselle Fleury, du Théâtre-Français, passa de La Rouarie à Marat, mais par amour. Les capitaines étaient souvent aussi ignorants que les soldats ; M. de Sapinaud ne savait pas l’orthographe, il écrivait : « nous \emph{orions} de notre \emph{cauté}. » Les chefs s’entre-haïssaient ; les capitaines du marais criaient : \emph{A bas ceux du pays haut !} Leur cavalerie était peu nombreuse  et difficile à former. Puysaye écrit : \emph{Tel homme qui me donne gaîment ses deux fils devient froid si je lui demande un de ses chevaux}. Fertes, fourches, faulx, fusils vieux et neufs, couteaux de braconnage, broches, gourdins ferrés et cloutés, c’étaient là leurs armes ; quelques-uns portaient en sautoir une croix faite de deux os de mort. Ils attaquaient à grands cris, surgissaient subitement de partout, des bois, des collines, des cépées, des chemins creux, s’égaillaient, c’est-à-dire faisaient le croissant, tuaient, exterminaient, foudroyaient, et se dissipaient. Quand ils traversaient un bourg républicain, ils coupaient l’arbre de la liberté, le brûlaient, et dansaient en rond autour du feu. Toutes leurs allures étaient nocturnes. Règle du Vendéen : être toujours inattendu. Ils faisaient quinze lieues en silence, sans courber une herbe sur leur passage. Le soir venu, après avoir fixé, entre chefs et en conseil de guerre, le lieu où le lendemain matin ils surprendraient les postes républicains, ils chargeaient leurs fusils, marmottaient leur prière, ôtaient leurs sabots, et filaient en longues colonnes, à travers les bois, pieds nus sur la bruyère et sur la mousse, sans un bruit, sans un mot, sans un souffle. Marche de chats dans les ténèbres.
 \subsubsection[{VI. L’âme de la terre passe dans l’homme}]{VI \\
L’âme de la terre passe dans l’homme}\phantomsection
\label{p3l1c6}
\noindent La Vendée insurgée ne peut être évaluée à moins de cinq cent mille hommes, femmes et enfants. Un demi-million de combattants, c’est le chiffre donné par Tuffin de La Rouarie.\par
Les fédéralistes aidaient ; la Vendée eut pour complice la Gironde. La Lozère envoyait au Bocage trente mille hommes. Huit départements se coalisaient, cinq en Bretagne, trois en Normandie. Évreux, qui fraternisait avec Caen, se faisait représenter dans la rébellion par Chaumont, son maire, et Gardembas, notable. Buzot, Gorsas et Barbaroux à Caen, Brissot à Moulins, Chassan à Lyon, Rabaut-Saint-Étienne à Nîmes, Meillan et Duchâtel en Bretagne, toutes ces bouches soufflaient sur la fournaise.\par
Il y a eu deux Vendées : la grande, qui faisait la guerre des forêts, la petite, qui faisait la guerre des buissons ; là est la nuance qui sépare Charette de Jean Chouan. La petite Vendée était naïve, la grande était corrompue ; la petite valait mieux. Charette fut fait marquis, lieutenant-général des armées du roi, et grand-croix de Saint-Louis ; Jean Chouan resta Jean Chouan. Charette confine au bandit, Jean Chouan au paladin.\par
Quant à ces chefs magnanimes, Bonchamps, Lescure,  La Rochejaquelein, ils se trompèrent. La grande armée catholique a été un effort insensé ; le désastre devait suivre. Se figure-t-on une tempête paysanne attaquant Paris, une coalition de villages assiégeant le Panthéon, une meute de noëls et d’oremus aboyant autour de \emph{la Marseillaise}, la cohue des sabots se ruant sur la légion des esprits ? Le Mans et Savenay châtièrent cette folie. Passer la Loire était impossible à la Vendée. Elle pouvait tout, excepté cette enjambée. La guerre civile ne conquiert point. Passer le Rhin complète César et augmente Napoléon ; passer la Loire tue La Rochejaquelein.\par
La vraie Vendée, c’est la Vendée chez elle ; là elle est plus qu’invulnérable, elle est insaisissable. Le Vendéen chez lui est contrebandier, laboureur, soldat, pâtre, braconnier, franc-tireur, chevrier, sonneur de cloches, paysan, espion, assassin, sacristain, bête des bois.\par
La Rochejaquelein n’est qu’Achille, Jean Chouan est Protée.\par
La Vendée a avorté. D’autres révoltes ont réussi, la Suisse par exemple. Il y a cette différence entre l’insurgé de montagne comme le Suisse et l’insurgé de forêt comme le Vendéen, que, presque toujours, fatale influence du milieu, l’un se bat pour un idéal, et l’autre pour des préjugés. L’un plane, l’autre rampe. L’un combat pour l’humanité, l’autre pour la solitude ; l’un veut la liberté, l’autre veut l’isolement ; l’un défend la commune, l’autre la paroisse. Communes ! communes ! criaient les héros de Morat. L’un a affaire aux précipices, l’autre aux fondrières ; l’un est l’homme  des torrents et des écumes, l’autre est l’homme des flaques stagnantes d’où sort la fièvre ; l’un a sur la tête l’azur, l’autre une broussaille ; l’un est sur une cime, l’autre est dans une ombre.\par
L’éducation n’est point la même, faite par les sommets ou par les bas-fonds.\par
La montagne est une citadelle, la forêt est une embuscade ; l’une inspire l’audace, l’autre le piège. L’antiquité plaçait les dieux sur les faîtes et les satyres dans les halliers. Le satyre c’est le sauvage ; demi-homme, demi-bête. Les pays libres ont des Apennins, des Alpes, des Pyrénées, un Olympe. Le Parnasse est un mont. Le mont Blanc était le colossal auxiliaire de Guillaume Tell ; au fond et au-dessus des immenses luttes des esprits contre la nuit qui emplissent les poëmes de l’Inde, on aperçoit l’Himalaya. La Grèce, l’Espagne, l’Italie, l’Helvétie, ont pour figure la montagne ; la Cimmérie, Germanie ou Bretagne, a le bois. La forêt est barbare.\par
La configuration du sol conseille à l’homme beaucoup d’actions. Elle est complice, plus qu’on ne croit. En présence de certains paysages féroces, on est tenté d’exonérer l’homme et d’incriminer la création ; on sent une sourde provocation de la nature ; le désert est parfois malsain à la conscience, surtout à la conscience peu éclairée ; la conscience peut être géante, cela fait Socrate et Jésus ; elle peut être naine, cela fait Atrée et Judas. La conscience petite est vite reptile ; les futaies crépusculaires, les ronces, les épines, les marais sous les branches, sont une fatale fréquentation  pour elle ; elle subit là la mystérieuse infiltration des persuasions mauvaises. Les illusions d’optique, les mirages inexpliqués, les effarements d’heure ou de lieu jettent l’homme dans cette sorte d’effroi, demi-religieux, demi-bestial, qui engendre, en temps ordinaires, la superstition, et dans les époques violentes, la brutalité. Les hallucinations tiennent la torche qui éclaire le chemin du meurtre. Il y a du vertige dans le brigand. La prodigieuse nature a un double sens qui éblouit les grands esprits et aveugle les âmes fauves. Quand l’homme est ignorant, quand le désert est visionnaire, l’obscurité de la solitude s’ajoute à l’obscurité de l’intelligence ; de là dans l’homme des ouvertures d’abîmes. De certains rochers, de certains ravins, de certains taillis, de certaines claires-voies farouches du soir à travers les arbres, poussent l’homme aux actions folles et atroces. On pourrait presque dire qu’il y a des lieux scélérats.\par
Que de choses tragiques a vues la sombre colline qui est entre Baignon et Plélan !\par
Les vastes horizons conduisent l’âme aux idées générales ; les horizons circonscrits engendrent les idées partielles ; ce qui condamne quelquefois de grands cœurs à être de petits esprits ; témoin Jean Chouan.\par
Les idées générales haïes par les idées partielles, c’est là la lutte même du progrès.\par
Pays, Patrie, ces deux mots résument toute la guerre de Vendée ; querelle de l’idée locale contre l’idée universelle. Paysans contre patriotes.
 \subsubsection[{VII. La Vendée a fini la Bretagne}]{VII \\
La Vendée a fini la Bretagne}\phantomsection
\label{p3l1c7}
\noindent La Bretagne est une vieille rebelle. Toutes les fois qu’elle s’était révoltée pendant deux mille ans, elle avait eu raison ; la dernière fois, elle a eu tort. Et pourtant au fond, contre la révolution comme contre la monarchie, contre les représentants en mission comme contre les gouverneurs ducs et pairs, contre la planche aux assignats comme contre la ferme des gabelles, quels que fussent les personnages combattant, Nicolas Rapin, François de La Noue, le capitaine Pluviaut et la dame de La Garnache, ou Stofflet, Coquereau et Lechandelier de Pierreville, sous M. de Rohan contre le roi et sous M. de La Rochejaquelein pour le roi, c’était toujours la même guerre que la Bretagne faisait, la guerre de l’esprit local contre l’esprit central.\par
Ces antiques provinces étaient un étang ; courir répugnait à cette eau dormante ; le vent qui soufflait ne les vivifiait pas, il les irritait. Finisterre ; c’était là que finissait la France, que le champ donné à l’homme se terminait et que la marche des générations s’arrêtait. Halte ! criait l’océan à la terre et la barbarie à  la civilisation. Toutes les fois que le centre, Paris, donne une impulsion, que cette impulsion vienne de la royauté ou de la république, qu’elle soit dans le sens du despotisme ou dans le sens de la liberté, c’est une nouveauté, et la Bretagne se hérisse. Laissez-nous tranquilles. Qu’est-ce qu’on nous veut ? Le Marais prend sa fourche, le Bocage prend sa carabine. Toutes nos tentatives, notre initiative en législation et en éducation, nos encyclopédies, nos philosophies, nos génies, nos gloires, viennent échouer devant le Houroux ; le tocsin de Bazouges menace la révolution française, la lande du Faou s’insurge contre nos orageuses places publiques, et la cloche du Haut-des-Prés déclare la guerre à la Tour du Louvre.\par
Surdité terrible.\par
L’insurrection vendéenne est un lugubre malentendu.\par
Échauffourée colossale, chicane de titans, rébellion démesurée, destinée à ne laisser à l’histoire qu’un mot, la Vendée, mot illustre et noir ; se suicidant pour des absents, dévouée à l’égoïsme, passant son temps à faire à la lâcheté l’offre d’une immense bravoure ; sans calcul, sans stratégie, sans tactique, sans plan, sans but, sans chef, sans responsabilité ; montrant à quel point la volonté peut être l’impuissance ; chevaleresque et sauvage ; l’absurdité en rut, bâtissant contre la lumière un garde-fou de ténèbres ; l’ignorance faisant à la vérité, à la justice, au droit, à la raison, à la délivrance, une longue résistance bête et superbe ; l’épouvante de huit années, le ravage de  quatorze départements, la dévastation des champs, l’écrasement des moissons, l’incendie des villages, la ruine des villes, le pillage des maisons, le massacre des femmes et des enfants, la torche dans les chaumes, l’épée dans les cœurs, l’effroi de la civilisation, l’espérance de M. Pitt ; telle fut cette guerre, essai inconscient de parricide.\par
En somme, en démontrant la nécessité de trouer dans tous les sens la vieille ombre bretonne et de percer cette broussaille de toutes les flèches de la lumière à la fois, la Vendée a servi le progrès. Les catastrophes ont une sombre façon d’arranger les choses.\par
  \subsection[{Livre deuxième. Les trois enfants}]{Livre deuxième \\
Les trois enfants}\phantomsection
\label{p3l2}
\subsubsection[{I. Plus quam civilia bella}]{I \\
Plus quam civilia bella}\phantomsection
\label{p3l2c1}
\noindent L’été de 1792 avait été très pluvieux ; l’été de 1793 fut très chaud. Par suite de la guerre civile, il n’y avait pour ainsi dire plus de chemins en Bretagne. On y voyageait pourtant, grâce à la beauté de l’été. La meilleure route est une terre sèche.\par
A la fin d’une sereine journée de juillet, une heure environ après le soleil couché, un homme à cheval, qui venait du côté d’Avranches, s’arrêta devant la petite auberge dite la Croix-Branchard, qui était à l’entrée de Pontorson, et dont l’enseigne portait cette inscription qu’on y lisait encore il y a quelques années : \emph{Bon cidre à dépoteyer}. Il avait fait chaud tout le jour, mais le vent commençait à souffler.\par
Ce voyageur était enveloppé d’un ample manteau qui couvrait la croupe de son cheval. Il portait un  large chapeau avec cocarde tricolore, ce qui n’était point sans hardiesse dans ce pays de haies et de coups de fusil où une cocarde était une cible. Le manteau noué au cou s’écartait pour laisser les bras libres, et dessous on pouvait entrevoir une ceinture tricolore et deux pommeaux de pistolets sortant de la ceinture. Un sabre qui pendait dépassait le manteau.\par
Au bruit du cheval qui s’arrêtait, la porte de l’auberge s’ouvrit, et l’aubergiste parut, une lanterne à la main. C’était l’heure intermédiaire ; il faisait jour sur la route et nuit dans la maison.\par
L’hôte regarda la cocarde.\par
— Citoyen, dit-il, vous arrêtez-vous ici ?\par
— Non.\par
— Où donc allez-vous ?\par
— A Dol.\par
— En ce cas, retournez à Avranches ou restez à Pontorson.\par
— Pourquoi ?\par
— Parce qu’on se bat à Dol.\par
— Ah ! dit le cavalier.\par
Et il reprit :\par
— Donnez l’avoine à mon cheval.\par
L’hôte apporta l’auge, y vida un sac d’avoine, et débrida le cheval qui se mit à souffler et à manger.\par
Le dialogue continua.\par
— Citoyen, est-ce un cheval de réquisition ?\par
— Non.\par
— Il est à vous ?\par
— Oui. Je l’ai acheté et payé.\par
 — D’où venez-vous ?\par
— De Paris.\par
— Pas directement ?\par
— Non.\par
— Je crois bien, les routes sont interceptées. Mais la poste marche encore.\par
— Jusqu’à Alençon. J’ai quitté la poste là.\par
— Ah ! il n’y aura bientôt plus de postes en France. Il n’y a plus de chevaux. Un cheval de trois cents francs se paye six cents francs, et les fourrages sont hors de prix. J’ai été maître de poste et me voilà gargotier. Sur treize cent treize maîtres de poste qu’il y avait, deux cents ont donné leur démission. Citoyen, vous avez voyagé d’après le nouveau tarif ?\par
— Du premier mai. Oui.\par
— Vingt sous par poste dans la voiture, douze sous dans le cabriolet, cinq sous dans le fourgon. C’est à Alençon que vous avez acheté ce cheval ?\par
— Oui.\par
— Vous avez marché aujourd’hui toute la journée ?\par
— Depuis l’aube.\par
— Et hier ?\par
— Et avant-hier.\par
— Je vois cela. Vous êtes venu par Domfront et Mortain.\par
— Et Avranches.\par
— Croyez-moi, reposez-vous, citoyen. Vous devez être fatigué, votre cheval l’est.\par
— Les chevaux ont droit à la fatigue, les hommes non.\par
 Le regard de l’hôte se fixa de nouveau sur le voyageur. C’était une figure grave, calme et sévère, encadrée de cheveux gris.\par
L’hôtelier jeta un coup d’œil sur la route qui était déserte à perte de vue, et dit :\par
— Et vous voyagez seul comme cela ?\par
— J’ai une escorte.\par
— Où ça ?\par
— Mon sabre et mes pistolets.\par
L’aubergiste alla chercher un seau d’eau et fit boire le cheval, et, pendant que le cheval buvait, l’hôte considérait le voyageur et se disait en lui-même : — C’est égal, il a l’air d’un prêtre.\par
Le cavalier reprit :\par
— Vous dites qu’on se bat à Dol ?\par
— Oui. Ça doit commencer dans ce moment-ci.\par
— Qui est-ce qui se bat ?\par
— Un ci-devant contre un ci-devant.\par
— Vous dites ?\par
— Je dis qu’un ci-devant qui est pour la république se bat contre un ci-devant qui est pour le roi.\par
— Mais il n’y a plus de roi.\par
— Il y a le petit. Et le curieux, c’est que les deux ci-devant sont deux parents.\par
Le cavalier écoutait attentivement. L’aubergiste poursuivit :\par
— L’un est jeune, l’autre est vieux. C’est le petit-neveu qui se bat contre le grand-oncle. L’oncle est royaliste, le neveu est patriote. L’oncle commande les blancs, le neveu commande les bleus. Ah ! ils ne se  feront pas quartier, allez. C’est une guerre à mort.\par
— A mort ?\par
— Oui, citoyen. Tenez, voulez-vous voir les politesses qu’ils se jettent à la tête ? Ceci est une affiche que le vieux trouve moyen de faire placarder partout, sur toutes les maisons et sur tous les arbres, et qu’il a fait coller jusque sur ma porte.\par
L’hôte approcha sa lanterne d’un carré de papier appliqué sur un des battants de sa porte, et, comme l’affiche était en très gros caractères, le cavalier, du haut de son cheval, put lire :\par
« — Le marquis de Lantenac a l’honneur d’informer son petit-neveu, monsieur le vicomte Gauvain, que, si monsieur le marquis a la bonne fortune de se saisir de sa personne, il fera bellement arquebuser monsieur le vicomte. »\par
— Et, poursuivit l’hôtelier, voici la réponse.\par
Il se retourna, et éclaira de sa lanterne une autre affiche placée en regard de la première sur l’autre battant de la porte. Le voyageur lut :\par
« — Gauvain prévient Lantenac que s’il le prend il le fera fusiller. »\par
— Hier, dit l’hôte, le premier placard a été collé sur ma porte, et ce matin le second. La réplique ne s’est pas fait attendre.\par
Le voyageur, à demi-voix, et comme se parlant à lui-même, prononça ces quelques mots, que l’aubergiste entendit sans trop les comprendre :\par
— Oui, c’est plus que la guerre dans la patrie, c’est la guerre dans la famille. Il le faut, et c’est bien.  Les grands rajeunissements des peuples sont à ce prix.\par
Et le voyageur portant la main à son chapeau, l’œil fixé sur la deuxième affiche, la salua.\par
L’hôte continua :\par
— Voyez-vous, citoyen, voici l’affaire. Dans les villes et dans les gros bourgs nous sommes pour la révolution, dans la campagne ils sont contre ; autant dire dans les villes on est français et dans les villages on est breton. C’est une guerre de bourgeois à paysans. Ils nous appellent patauds, nous les appelons rustauds. Les nobles et les prêtres sont avec eux.\par
— Pas tous, interrompit le cavalier.\par
— Sans doute, citoyen, puisque nous avons ici un vicomte contre un marquis.\par
Et il ajouta à part lui :\par
— Et que je crois bien que je parle à un prêtre.\par
Le cavalier continua :\par
— Et lequel des deux l’emporte ?\par
— Jusqu’à présent, le vicomte. Mais il a de la peine. Le vieux est rude. Ces gens-là, c’est la famille Gauvain, des nobles d’ici. C’est une famille à deux branches ; il y a la grande branche dont le chef s’appelle le marquis de Lantenac, et la petite branche dont le chef s’appelle le vicomte Gauvain. Aujourd’hui les deux branches se battent. Cela ne se voit pas chez les arbres, mais cela se voit chez les hommes. Ce marquis de Lantenac est tout-puissant en Bretagne ; pour les paysans, c’est un prince. Le jour de son débarquement,  il a eu tout de suite huit mille hommes ; en une semaine trois cents paroisses ont été soulevées. S’il avait pu prendre un coin de la côte, les Anglais débarquaient. Heureusement ce Gauvain s’est trouvé là, qui est son petit-neveu, drôle d’aventure. Il est commandant républicain, et il a rembarré son grand-oncle. Et puis le bonheur a voulu que ce Lantenac, en arrivant et en massacrant une masse de prisonniers, ait fait fusiller deux femmes dont une avait trois enfants qui étaient adoptés par un bataillon de Paris. Alors cela a fait un bataillon terrible. Il s’appelle le bataillon du Bonnet-Rouge. Il n’en reste pas beaucoup de ces parisiens-là, mais ce sont de furieuses bayonnettes. Ils ont été incorporés dans la colonne du commandant Gauvain. Rien ne leur résiste. Ils veulent venger les femmes et ravoir les enfants. On ne sait pas ce que le vieux en a fait, de ces petits. C’est ce qui enrage les grenadiers de Paris. Supposez que ces enfants n’y soient pas mêlés, cette guerre-là ne serait pas ce qu’elle est. Le vicomte est un bon et brave jeune homme. Mais le vieux est un effroyable marquis. Les paysans appellent ça la guerre de saint Michel contre Belzébuth. Vous savez peut-être que saint Michel est un ange du pays. Il a une montagne à lui au milieu de la mer dans la baie. Il passe pour avoir fait tomber le démon et pour l’avoir enterré sous une autre montagne qui est près d’ici, et qu’on appelle Tombelaine.\par
— Oui, murmura le cavalier, Tumba Beleni, la tombe de Belenus, de Belus, de Bel, de Bélial, de Belzébuth.\par
 — Je vois que vous êtes informé.\par
Et l’hôte se dit en aparté :\par
— Décidément, il sait le latin, c’est un prêtre.\par
Puis il reprit :\par
— Eh bien, citoyen, pour les paysans, c’est cette guerre-là qui recommence. Il va sans dire que pour eux saint Michel, c’est le général royaliste, et Belzébuth, c’est le commandant patriote ; mais s’il y a un diable, c’est bien Lantenac, et s’il y a un ange, c’est Gauvain. Vous ne prenez rien, citoyen ?\par
— J’ai ma gourde et un morceau de pain. Mais vous ne me dites pas ce qui se passe à Dol.\par
— Voici. Gauvain commande la colonne d’expédition de la côte. Le but de Lantenac était d’insurger tout, d’appuyer la Basse-Bretagne sur la Basse-Normandie, d’ouvrir la porte à Pitt, et de donner un coup d’épaule à la grande armée vendéenne avec vingt mille Anglais et deux cent mille paysans. Gauvain a coupé court à ce plan. Il tient la côte, et il repousse Lantenac dans l’intérieur et les Anglais dans la mer. Lantenac était ici, et il l’en a délogé ; il lui a repris le Pont-au-Beau ; il l’a chassé d’Avranches, il l’a chassé de Villedieu, il l’a empêché d’arriver à Granville. Il manœuvre pour le refouler dans la forêt de Fougères, et l’y cerner. Tout allait bien. Hier Gauvain était ici avec sa colonne. Tout à coup, alerte. Le vieux, qui est habile, a fait une pointe ; on apprend qu’il a marché sur Dol. S’il prend Dol, et s’il établit sur le Mont-Dol une batterie, car il a du canon, voilà un point de la côte où les Anglais peuvent aborder, et tout est perdu. C’est pourquoi,  comme il n’y avait pas une minute à perdre, Gauvain, qui est un homme de tête, n’a pris conseil que de lui-même, n’a pas demandé d’ordre et n’en a pas attendu, a sonné le boute-selle, attelé son artillerie, ramassé sa troupe, tiré son sabre, et voilà comment, pendant que Lantenac se jette sur Dol, Gauvain se jette sur Lantenac. C’est à Dol que ces deux fronts bretons vont se cogner. Ce sera un fier choc. Ils y sont maintenant.\par
— Combien de temps faut-il pour aller à Dol ?\par
— A une troupe qui a des charrois, au moins trois heures ; mais ils y sont.\par
Le voyageur prêta l’oreille et dit :\par
— En effet, il me semble que j’entends le canon.\par
L’hôte écouta.\par
— Oui, citoyen. Et la fusillade. On déchire de la toile. Vous devriez passer la nuit ici. Il n’y a rien de bon à attraper par là.\par
— Je ne puis m’arrêter. Je dois continuer ma route.\par
— Vous avez tort. Je ne connais pas vos affaires, mais le risque est grand, et, à moins qu’il ne s’agisse de ce que vous avez de plus cher au monde...\par
— C’est en effet de cela qu’il s’agit, répondit le cavalier.\par
— ... De quelque chose comme votre fils...\par
— A peu près, dit le cavalier.\par
L’aubergiste leva la tête et se dit à part soi :\par
— Ce citoyen me fait pourtant l’effet d’être un prêtre.\par
Puis, après réflexion :\par
 — Après ça, un prêtre, ça a des enfants.\par
— Rebridez mon cheval, dit le voyageur. Combien vous dois-je ?\par
Et il paya.\par
L’hôte rangea l’auge et le seau le long de son mur, et revint vers le voyageur.\par
— Puisque vous êtes décidé à partir, écoutez mon conseil. Il est clair que vous allez à Saint-Malo. Eh bien, n’allez pas par Dol. Il y a deux chemins, le chemin par Dol, et le chemin le long de la mer. L’un n’est guère plus court que l’autre. Le chemin le long de la mer va par Saint-Georges de Brehaigne, Cherrueix, et Hirel-le-Vivier. Vous laissez Dol au sud et Cancale au nord. Citoyen, au bout de la rue, vous allez trouver l’embranchement des deux routes ; celle de Dol est à gauche, celle de Saint-Georges de Brehaigne est à droite. Écoutez-moi bien, si vous allez par Dol, vous tombez dans le massacre. C’est pourquoi ne prenez pas à gauche, prenez à droite.\par
— Merci, dit le voyageur.\par
Et il piqua son cheval.\par
L’obscurité s’était faite, il s’enfonça dans la nuit.\par
L’aubergiste le perdit de vue.\par
Quand le voyageur fut au bout de la rue à l’embranchement des deux chemins, il entendit la voix de l’aubergiste qui lui criait de loin :\par
— Prenez à droite !\par
Il prit à gauche.
 \subsubsection[{II. Dol}]{II \\
Dol}\phantomsection
\label{p3l2c2}
\noindent Dol, ville espagnole de France en Bretagne, ainsi la qualifient les cartulaires, n’est pas une ville, c’est une rue. Grande vieille rue gothique, toute bordée à droite et à gauche de maisons à piliers, point alignées, qui font des caps et des coudes dans la rue, d’ailleurs très large. Le reste de la ville n’est qu’un réseau de ruelles se rattachant à cette grande rue diamétrale et y aboutissant comme des ruisseaux à une rivière. La ville, sans portes ni murailles, ouverte, dominée par le Mont-Dol, ne pourrait soutenir un siège ; mais la rue en peut soutenir un. Les promontoires de maisons, qu’on y voyait encore il y a cinquante ans, et les deux galeries sous piliers qui la bordent en faisaient un lieu de combat très solide et très résistant. Autant de maisons, autant de forteresses ; et il fallait enlever l’une après l’autre. La vieille halle était à peu près au milieu de la rue.\par
L’aubergiste de la Croix-Branchard avait dit vrai, une mêlée forcenée emplissait Dol au moment où il parlait. Un duel nocturne entre les blancs arrivés le matin et les bleus survenus le soir avait brusquement  éclaté dans la ville. Les forces étaient inégales, les blancs étaient six mille, les bleus étaient quinze cents, mais il y avait égalité d’acharnement. Chose remarquable, c’étaient les quinze cents qui avaient attaqué les six mille.\par
D’un côté une cohue, de l’autre une phalange. D’un côté six mille paysans, avec des cœurs-de-Jésus sur leurs vestes de cuir, des rubans blancs à leurs chapeaux ronds, des devises chrétiennes sur leurs brassards, des chapelets à leurs ceinturons, ayant plus de fourches que de sabres et des carabines sans bayonnettes, traînant des canons attelés de cordes, mal équipés, mal disciplinés, mal armés, mais frénétiques. De l’autre quinze cents soldats, avec le tricorne à cocarde tricolore, l’habit à grandes basques et à grands revers, le baudrier croisé, le briquet à poignée de cuivre et le fusil à longue bayonnette, dressés, alignés, dociles et farouches, sachant obéir en gens qui sauraient commander, volontaires eux aussi, mais volontaires de la patrie, en haillons du reste, et sans souliers ; pour la monarchie, des paysans paladins ; pour la révolution, des héros va-nu-pieds ; et chacune des deux troupes ayant pour âme son chef ; les royalistes un vieillard, les républicains un jeune homme. D’un côté Lantenac, de l’autre Gauvain.\par
La révolution, à côté des jeunes figures gigantesques, telles que Danton, Saint-Just et Robespierre, a les jeunes figures idéales, comme Hoche et Marceau. Gauvain était une de ces figures.\par
Gauvain avait trente ans, une encolure d’hercule,  l’œil sérieux d’un prophète et le rire d’un enfant. Il ne fumait pas, il ne buvait pas, il ne jurait pas. Il emportait à travers la guerre un nécessaire de toilette ; il avait grand soin de ses ongles, de ses dents, de ses cheveux qui étaient bruns et superbes ; et dans les haltes il secouait lui-même au vent son habit de capitaine qui était troué de balles et blanc de poussière. Toujours rué éperdument dans les mêlées, il n’avait jamais été blessé. Sa voix très douce avait à propos les éclats brusques du commandement. Il donnait l’exemple de coucher à terre, sous la bise, sous la pluie, dans la neige, roulé dans son manteau, et sa tête charmante posée sur une pierre. C’était une âme héroïque et innocente. Le sabre au poing le transfigurait. Il avait cet air efféminé qui dans la bataille est formidable.\par
Avec cela penseur et philosophe, un jeune sage ; Alcibiade pour qui le voyait, Socrate pour qui l’entendait.\par
Dans cette immense improvisation qui est la révolution française, ce jeune homme avait été tout de suite un chef de guerre.\par
Sa colonne, formée par lui, était, comme la légion romaine, une sorte de petite armée complète ; elle se composait d’infanterie et de cavalerie ; elle avait des éclaireurs, des pionniers, des sapeurs, des pontonniers ; et, de même que la légion romaine avait des catapultes, elle avait des canons. Trois pièces attelées faisaient la colonne forte en la laissant maniable.\par
 Lantenac aussi était un chef de guerre, pire encore. Il était à la fois plus réfléchi et plus hardi. Les vrais vieux héros ont plus de froideur que les jeunes parce qu’ils sont loin de l’aurore, et plus d’audace parce qu’ils sont près de la mort. Qu’ont-ils à perdre ? si peu de chose. De là les manœuvres téméraires, en même temps que savantes, de Lantenac. Mais en somme, et presque toujours, dans cet opiniâtre corps-à-corps du vieux et du jeune, Gauvain avait le dessus. C’était plutôt fortune qu’autre chose. Tous les bonheurs, même le bonheur terrible, font partie de la jeunesse. La victoire est un peu fille.\par
Lantenac était exaspéré contre Gauvain ; d’abord parce que Gauvain le battait, ensuite parce que c’était son parent. Quelle idée a-t-il d’être jacobin ? ce Gauvain ! ce polisson ! son héritier, car le marquis n’avait pas d’enfants, un petit-neveu, presque un petit-fils ! — \emph{Ah !} disait ce quasi grand-père, \emph{si je mets la main dessus, je le tue comme un chien !}\par
Du reste, la république avait raison de s’inquiéter de ce marquis de Lantenac. A peine débarqué, il faisait trembler. Son nom avait couru dans l’insurrection vendéenne comme une traînée de poudre, et Lantenac était tout de suite devenu centre. Dans une révolte de cette nature où tous se jalousent et où chacun a son buisson ou son ravin, quelqu’un de haut qui survient rallie les chefs épars égaux entre eux. Presque tous les capitaines des bois s’étaient joints à Lantenac, et, de près ou de loin, lui obéissaient.\par
Un seul l’avait quitté, c’était le premier qui s’était  joint à lui, Gavard. Pourquoi ? C’est que c’était un homme de confiance. Gavard avait eu tous les secrets et adopté tous les plans de l’ancien système de guerre civile que Lantenac venait supplanter et remplacer. On n’hérite pas d’un homme de confiance ; le soulier de La Rouarie n’avait pu chausser Lantenac. Gavard était allé rejoindre Bonchamp.\par
Lantenac, comme homme de guerre, était de l’école de Frédéric II ; il entendait combiner la grande guerre avec la petite. Il ne voulait ni d’une « masse confuse », comme la grosse armée catholique et royale, foule destinée à l’écrasement, ni d’un éparpillement dans les halliers et les taillis, bon pour harceler, impuissant pour terrasser. La guérilla ne conclut pas, ou conclut mal ; on commence par attaquer une république et l’on finit par détrousser une diligence. Lantenac ne comprenait cette guerre bretonne, ni toute en rase campagne comme La Rochejaquelein, ni toute dans la forêt comme Jean Chouan ; ni Vendée, ni Chouannerie ; il voulait la vraie guerre ; se servir du paysan, mais l’appuyer sur le soldat. Il voulait des bandes pour la stratégie et des régiments pour la tactique. Il trouvait excellentes pour l’attaque, l’embuscade et la surprise, ces armées de village, tout de suite assemblées, tout de suite dispersées ; mais il les sentait trop fluides ; elles étaient dans sa main comme de l’eau ; il voulait dans cette guerre flottante et diffuse créer un point solide ; il voulait ajouter à la sauvage armée des forêts une troupe régulière qui fût le pivot de manœuvre des paysans. Pensée profonde  et affreuse ; si elle eût réussi, la Vendée eût été inexpugnable.\par
Mais où trouver une troupe régulière ? où trouver des soldats ? où trouver des régiments ? où trouver une armée toute faite ? En Angleterre. De là l’idée fixe de Lantenac, faire débarquer les Anglais. Ainsi capitule la conscience des partis ; la cocarde blanche lui cachait l’habit rouge. Lantenac n’avait qu’une pensée : s’emparer d’un point du littoral, et le livrer à Pitt. C’est pourquoi, voyant Dol sans défense, il s’était jeté dessus, afin d’avoir par Dol le Mont-Dol, et par le Mont-Dol la côte.\par
Le lieu était bien choisi. Le canon du Mont-Dol balayerait d’un côté le Fresnois, de l’autre Saint-Brelade, tiendrait à distance la croisière de Cancale et ferait toute la plage libre à une descente, du Raz-sur-Couesnon à Saint-Mêloir-des-Ondes.\par
Pour faire réussir cette tentative décisive, Lantenac avait amené avec lui un peu plus de six mille hommes, ce qu’il avait de plus robuste dans les bandes dont il disposait, et toute son artillerie, dix couleuvrines de seize, une bâtarde de huit et une pièce de régiment de quatre livres de balles. Il entendait établir une forte batterie sur le Mont-Dol, d’après ce principe que mille coups tirés avec dix canons font plus de besogne que quinze cents coups tirés avec cinq canons.\par
Le succès semblait certain. On était six mille hommes. On n’avait à craindre, vers Avranches, que Gauvain et ses quinze cents hommes, et vers Dinan  que Léchelle. Léchelle, il est vrai, avait vingt-cinq mille hommes, mais il était à vingt lieues. Lantenac était donc rassuré, du côté de Léchelle, par la grande distance contre le grand nombre, et, du côté de Gauvain, par le petit nombre contre la petite distance. Ajoutons que Léchelle était imbécile, et que, plus tard, il fit écraser ses vingt-cinq mille hommes aux landes de la Croix-Bataille, échec qu’il paya de son suicide.\par
Lantenac avait donc une sécurité complète. Son entrée à Dol fut brusque et dure. Le marquis de Lantenac avait une rude renommée ; on le savait sans miséricorde. Aucune résistance ne fut essayée. Les habitants terrifiés se barricadèrent dans leurs maisons. Les six mille Vendéens s’installèrent dans la ville avec la confusion campagnarde, presque en champ de foire, sans fourriers, sans logis marqués, bivouaquant au hasard, faisant la cuisine en plein vent, s’éparpillant dans les églises, quittant les fusils pour les rosaires. Lantenac alla en hâte avec quelques officiers d’artillerie reconnaître le Mont-Dol, laissant la lieutenance à Gouge-le-Bruant, qu’il avait nommé sergent de bataille.\par
Ce Gouge-le-Bruant a laissé une vague trace dans l’histoire. Il avait deux surnoms, \emph{Brise-Bleu}, à cause de ses carnages de patriotes, et \emph{l’Imânus}, parce qu’il avait en lui on ne sait quoi d’inexprimablement horrible. \emph{Imânus}, dérivé d’\emph{immanis}, est un vieux mot bas-normand qui exprime la laideur surhumaine, et quasi divine, dans l’épouvante, le démon, le satyre, l’ogre.  Un ancien manuscrit dit : \emph{d’mes daeux iers j’vis l’imânus}. Les vieillards du Bocage ne savent plus aujourd’hui ce que c’est que Gouge-le-Bruant, ni ce que signifie Brise-Bleu ; mais ils connaissent confusément l’Imânus. L’Imânus est mêlé aux superstitions locales. On parle encore de l’Imânus à Trémorel et à Plumaugat, deux villages où Gouge-le-Bruant a laissé la marque de son pied sinistre. Dans la Vendée, les autres étaient les sauvages, Gouge-le-Bruant était le barbare. C’était une espèce de cacique, tatoué de croix-de-par-Dieu et de fleurs-de-lys ; il avait sur sa face la lueur hideuse, et presque surnaturelle, d’une âme à laquelle ne ressemblait aucune autre âme humaine. Il était infernalement brave dans le combat, ensuite atroce. C’était un cœur plein d’aboutissements tortueux, porté à tous les dévouements, enclin à toutes les fureurs. Raisonnait-il ? Oui, mais comme les serpents rampent ; en spirale. Il partait de l’héroïsme pour arriver à l’assassinat. Il était impossible de deviner d’où lui venaient ses résolutions, parfois grandioses à force d’être monstrueuses. Il était capable de tous les inattendus horribles. Il avait la férocité épique.\par
De là ce surnom difforme, \emph{l’Imânus}.\par
Le marquis de Lantenac avait confiance en sa cruauté.\par
Cruauté, c’était juste, l’Imânus y excellait ; mais en stratégie et en tactique il était moins supérieur, et peut-être le marquis avait-il tort d’en faire son sergent de bataille. Quoi qu’il en soit, il laissa derrière lui l’Imânus avec charge de le remplacer et de veiller à tout.\par
 Gouge-le-Bruant, homme plus guerrier que militaire, était plus propre à égorger un clan qu’à garder une ville. Pourtant il posa des grand’gardes.\par
Le soir venu, comme le marquis de Lantenac, après avoir reconnu l’emplacement de la batterie projetée, s’en retournait vers Dol, tout à coup, il entendit le canon. Il regarda. Une fumée rouge s’élevait de la grande rue. Il y avait surprise, irruption, assaut ; on se battait dans la ville.\par
Bien que difficile à étonner, il fut stupéfait. Il ne s’attendait à rien de pareil. Qui cela pouvait-il être ? Évidemment ce n’était pas Gauvain. On n’attaque pas à un contre quatre. Était-ce Léchelle ? Mais alors quelle marche forcée ! Léchelle était improbable, Gauvain impossible.\par
Lantenac poussa son cheval ; chemin faisant il rencontra des habitants qui s’enfuyaient ; il les questionna, ils étaient fous de peur. Ils criaient : Les bleus ! les bleus ! et quand il arriva, la situation était mauvaise.\par
Voici ce qui s’était passé.
 \subsubsection[{III. Petites armées et grandes batailles}]{III \\
Petites armées et grandes batailles}\phantomsection
\label{p3l2c3}
\noindent En arrivant à Dol, les paysans, on vient de le voir, s’étaient dispersés dans la ville, chacun faisant à sa guise, comme cela arrive quand « \emph{on obéit d’amitié} », c’était le mot des Vendéens. Genre d’obéissance qui fait des héros, mais non des troupiers. Ils avaient garé leur artillerie avec les bagages sous les voûtes de la vieille halle, et, las, buvant, mangeant, « chapelettant, » ils s’étaient couchés pêle-mêle en travers de la grande rue, plutôt encombrée que gardée. Comme la nuit tombait, la plupart s’endormirent, la tête sur leurs sacs, quelques-uns ayant leur femme à côté d’eux ; car souvent les paysannes suivaient les paysans ; en Vendée, les femmes grosses servaient d’espions. C’était une douce nuit de juillet ; les constellations resplendissaient dans le profond bleu noir du ciel. Tout ce bivouac, qui était plutôt une halte de caravane qu’un campement d’armée, se mit à sommeiller paisiblement. Tout à coup, à la lueur du crépuscule, ceux qui n’avaient pas encore fermé les yeux virent trois pièces de canon braquées à l’entrée de la grande rue.\par
C’était Gauvain. Il avait surpris les grand’gardes,  il était dans la ville, et il tenait avec sa colonne la tête de la rue.\par
Un paysan se dressa, cria : qui vive ? et lâcha son coup de fusil ; un coup de canon répliqua. Puis une mousqueterie furieuse éclata. Toute la cohue assoupie se leva en sursaut. Rude secousse. S’endormir sous les étoiles et se réveiller sous la mitraille.\par
Le premier moment fut terrible. Rien de tragique comme le fourmillement d’une foule foudroyée. Ils se jetèrent sur leurs armes. On criait, on courait, beaucoup tombaient. Les gars, assaillis, ne savaient plus ce qu’ils faisaient et s’arquebusaient les uns les autres. Il y avait des gens ahuris qui sortaient des maisons, qui y rentraient, qui sortaient encore, et qui erraient dans la bagarre, éperdus. Des familles s’appelaient. Combat lugubre, mêlé de femmes et d’enfants. Les balles sifflantes rayaient l’obscurité. La fusillade partait de tous les coins noirs. Tout était fumée et tumulte. L’enchevêtrement des fourgons et des charrois s’y ajoutait. Les chevaux ruaient. On marchait sur des blessés. On entendait à terre des hurlements. Horreur de ceux-ci, stupeur de ceux-là. Les soldats et les officiers se cherchaient. Au milieu de tout cela, de sombres indifférences. Une femme allaitait son nouveau-né, assise contre un pan de mur auquel était adossé son mari qui avait la jambe cassée et qui, pendant que son sang coulait, chargeait tranquillement sa carabine et tirait au hasard, tuant devant lui dans l’ombre. Des hommes à plat ventre tiraient à travers les roues des charrettes. Par moments  il s’élevait un hourvari de clameurs. La grosse voix du canon couvrait tout. C’était épouvantable.\par
Ce fut comme un abatis d’arbres ; tous tombaient les uns sur les autres. Gauvain, embusqué, mitraillait à coup sûr et perdait peu de monde.\par
Pourtant l’intrépide désordre des paysans finit par se mettre sur la défensive ; ils se replièrent sous la halle, vaste redoute obscure, forêt de piliers de pierre. Là ils reprirent pied ; tout ce qui ressemblait à un bois leur donnait confiance. L’Imânus suppléait de son mieux à l’absence de Lantenac. Ils avaient du canon, mais, au grand étonnement de Gauvain, ils ne s’en servaient point ; cela tenait à ce que, les officiers d’artillerie étant allés avec le marquis reconnaître le Mont-Dol, les gars ne savaient que faire des couleuvrines et des bâtardes ; mais ils criblaient de balles les bleus qui les canonnaient. Les paysans ripostaient par la mousqueterie à la mitraille. C’étaient eux maintenant qui étaient abrités. Ils avaient entassé les haquets, les tombereaux, les bagages, toutes les futailles de la vieille halle, et improvisé une haute barricade avec des claires-voies par où passaient leurs carabines. Par ces trous leur fusillade était meurtrière. Tout cela se fit vite. En un quart d’heure la halle eut un front imprenable.\par
Ceci devenait grave pour Gauvain. Cette halle brusquement transformée en citadelle, c’était l’inattendu. Les paysans étaient là, massés et solides. Gauvain avait réussi la surprise et manqué la déroute. Il avait mis pied à terre. Attentif, ayant son épée au poing sous ses bras croisés, debout dans la lueur  d’une torche qui éclairait sa batterie, il regardait toute cette ombre.\par
Sa haute taille dans cette clarté le faisait visible aux hommes de la barricade. Il était le point de mire, mais il n’y songeait pas.\par
Les volées de balles qu’envoyait la barricade s’abattaient autour de Gauvain, pensif.\par
Mais contre toutes ces carabines il avait du canon. Le boulet finit toujours par avoir raison. Qui a l’artillerie a la victoire. Sa batterie, bien servie, lui assurait la supériorité.\par
Subitement, un éclair jaillit de la halle pleine de ténèbres, on entendit comme un coup de foudre, et un boulet vint trouer une maison au-dessus de la tête de Gauvain.\par
La barricade répondait au canon par le canon.\par
Que se passait-il ? Il y avait du nouveau. L’artillerie maintenant n’était plus d’un seul côté.\par
Un second boulet suivit le premier et vint s’enfoncer dans le mur tout près de Gauvain. Un troisième boulet jeta à terre son chapeau.\par
Ces boulets étaient de gros calibre. C’était une pièce de seize qui tirait.\par
— On vous vise, commandant, crièrent les artilleurs.\par
Et ils éteignirent la torche. Gauvain, rêveur, ramassa son chapeau.\par
Quelqu’un en effet visait Gauvain, c’était Lantenac.\par
Le marquis venait d’arriver dans la barricade par le côté opposé.\par
L’Imânus avait couru à lui.\par
 — Monseigneur, nous sommes surpris.\par
— Par qui ?\par
— Je ne sais.\par
— La route de Dinan est-elle libre ?\par
— Je le crois.\par
— Il faut commencer la retraite.\par
— Elle commence. Beaucoup se sont déjà sauvés.\par
— Il ne faut pas se sauver ; il faut se retirer. Pourquoi ne vous servez-vous pas de l’artillerie ?\par
— On a perdu la tête, et puis les officiers n’étaient pas là.\par
— J’y vais.\par
— Monseigneur, j’ai dirigé sur Fougères le plus que j’ai pu des bagages, les femmes, tout l’inutile. Que faut-il faire des trois petits prisonniers ?\par
— Ah ! ces enfants ?\par
— Oui.\par
— Ils sont nos otages. Fais-les conduire à la Tourgue.\par
Cela dit, le marquis alla à la barricade. Le chef venu, tout changea de face. La barricade était mal faite pour l’artillerie, il n’y avait place que pour deux canons ; le marquis mit en batterie deux pièces de seize, auxquelles on fit des embrasures. Comme il était penché sur un des canons, observant la batterie ennemie par l’embrasure, il aperçut Gauvain.\par
— C’est lui ! cria-t-il.\par
Alors il prit lui-même l’écouvillon et le fouloir, chargea la pièce, fixa le fronton de mire, et pointa.\par
Trois fois il ajusta Gauvain, et le manqua. Le troisième coup ne réussit qu’à le décoiffer.\par
 — Maladroit ! murmura Lantenac. Un peu plus bas, j’avais la tête.\par
Brusquement la torche s’éteignit, et il n’eut plus devant lui que les ténèbres.\par
— Soit, dit-il.\par
Et se tournant vers les canonniers paysans, il cria :\par
— A mitraille !\par
Gauvain de son côté n’était pas moins sérieux. La situation s’aggravait. Une phase nouvelle du combat se dessinait. La barricade en était à le canonner. Qui sait si elle n’allait point passer de la défensive à l’offensive ? Il avait devant lui, en défalquant les morts et les fuyards, au moins cinq mille combattants, et il ne lui restait à lui que douze cents hommes maniables. Que deviendraient les républicains si l’ennemi s’apercevait de leur petit nombre ? Les rôles seraient intervertis. On était assaillant, on serait assailli. Que la barricade fît une sortie, tout pouvait être perdu.\par
Que faire ? Il ne fallait point songer à attaquer la barricade de front ; un coup de vive force était chimérique ; douze cents hommes ne débusquent pas cinq mille hommes. Brusquer était impossible, attendre était funeste. Il fallait en finir. Mais comment ?\par
Gauvain était du pays, il connaissait la ville ; il savait que la vieille halle, où les Vendéens s’étaient crénelés, était adossée à un dédale de ruelles étroites et tortueuses.\par
Il se tourna vers son lieutenant qui était ce vaillant capitaine Guéchamp, fameux plus tard pour avoir nettoyé la forêt de Concise où était né Jean Chouan,  et pour avoir, en barrant aux rebelles la chaussée de l’étang de la Chaîne, empêché la prise de Bourgneuf.\par
— Guéchamp, dit-il, je vous remets le commandement. Faites tout le feu que vous pourrez. Trouez la barricade à coups de canon. Occupez-moi tous ces gars-là.\par
— C’est compris, dit Guéchamp.\par
— Massez toute la colonne, armes chargées, et tenez-la prête à l’attaque.\par
Il ajouta quelques mots à l’oreille de Guéchamp.\par
— C’est entendu, dit Guéchamp.\par
Gauvain reprit :\par
— Tous nos tambours sont-ils sur pied ?\par
— Oui.\par
— Nous en avons neuf. Gardez-en deux, donnez-m’en sept.\par
Les sept tambours vinrent en silence se ranger devant Gauvain.\par
Alors Gauvain cria :\par
— A moi le bataillon du Bonnet-Rouge !\par
Douze hommes, dont un sergent, sortirent du gros de la troupe.\par
— Je demande tout le bataillon, dit Gauvain.\par
— Le voilà, répondit le sergent.\par
— Vous êtes douze !\par
— Nous restons douze.\par
— C’est bien, dit Gauvain.\par
Ce sergent était le bon et rude troupier Radoub, qui avait adopté au nom du bataillon les trois enfants rencontrés dans le bois de la Saudraie.\par
 Un demi-bataillon seulement, on s’en souvient, avait été exterminé à Herbe-en-Pail, et Radoub avait eu ce bon hasard de n’en point faire partie.\par
Un fourgon de fourrage était proche ; Gauvain le montra du doigt au sergent.\par
— Sergent, faites faire à vos hommes des liens de paille, et qu’on torde cette paille autour des fusils pour qu’on n’entende pas de bruit s’ils s’entre-choquent.\par
Une minute s’écoula, l’ordre fut exécuté, en silence et dans l’obscurité.\par
— C’est fait, dit le sergent.\par
— Soldats, ôtez vos souliers, reprit Gauvain.\par
— Nous n’en avons pas, dit le sergent.\par
Cela faisait, avec les sept tambours, dix-neuf hommes ; Gauvain était le vingtième.\par
Il cria :\par
— Sur une seule file. Suivez-moi. Les tambours derrière moi. Le bataillon ensuite. Sergent, vous commanderez le bataillon.\par
Il prit la tête de la colonne, et, pendant que la canonnade continuait des deux côtés, ces vingt hommes, glissant comme des ombres, s’enfoncèrent dans les ruelles désertes.\par
Ils marchèrent quelque temps de la sorte, serpentant le long des maisons. Tout semblait mort dans la ville ; les bourgeois s’étaient blottis dans les caves. Pas une porte qui ne fût barrée, pas un volet qui ne fût fermé. De lumière nulle part.\par
La grande rue faisait dans ce silence un fracas furieux ; le combat au canon continuait ; la batterie  républicaine et la barricade royaliste se crachaient toute leur mitraille avec rage.\par
Après vingt minutes de marche tortueuse, Gauvain, qui dans cette obscurité cheminait avec certitude, arriva à l’extrémité d’une ruelle d’où l’on rentrait dans la grande rue ; seulement on était de l’autre côté de la halle.\par
La position était tournée. De ce côté-ci il n’y avait pas de retranchement, ceci est l’éternelle imprudence des constructeurs de barricades, la halle était ouverte, et l’on pouvait entrer sous les piliers où étaient attelés quelques chariots de bagages prêts à partir. Gauvain et ses dix-neuf hommes avaient devant eux les cinq mille Vendéens, mais de dos et non de front.\par
Gauvain parla à voix basse au sergent ; on défit la paille nouée autour des fusils ; les douze grenadiers se postèrent en bataille derrière l’angle de la ruelle, et les sept tambours, la baguette haute, attendirent.\par
Les décharges d’artillerie étaient intermittentes. Tout à coup, dans un intervalle entre deux détonations, Gauvain leva son épée, et d’une voix qui, dans ce silence, sembla un éclat de clairon, il cria :\par
— Deux cents hommes par la droite, deux cents hommes par la gauche, tout le reste sur le centre !\par
Les douze coups de fusil partirent, et les sept tambours sonnèrent la charge.\par
Et Gauvain jeta le cri redoutable des bleus :\par
— A la bayonnette ! Fonçons !\par
L’effet fut inouï.\par
 Toute cette masse paysanne se sentit prise à revers, et s’imagina avoir une nouvelle armée dans le dos. En même temps, entendant le tambour, la colonne qui tenait le haut de la grande rue et que commandait Guéchamp s’ébranla, battant la charge de son côté, et se jeta au pas de course sur la barricade ; les paysans se virent entre deux feux ; la panique est un grossissement, dans la panique un coup de pistolet fait le bruit d’un coup de canon, toute clameur est fantôme, et l’aboiement d’un chien semble le rugissement d’un lion. Ajoutons que le paysan prend peur comme le chaume prend feu, et, aussi aisément qu’un feu de chaume devient incendie, une peur de paysan devient déroute. Ce fut une fuite inexprimable.\par
En quelques instants la halle fut vide, les gars terrifiés se désagrégèrent, rien à faire pour les officiers. L’imânus tua inutilement deux ou trois fuyards, on n’entendait que ce cri : \emph{Sauve qui peut !} et cette armée, à travers les rues de la ville comme à travers les trous d’un crible, se dispersa dans la campagne, avec une rapidité de nuée emportée par l’ouragan.\par
Les uns s’enfuirent vers Châteauneuf, les autres vers Plerguer, les autres vers Antrain.\par
Le marquis de Lantenac vit cette déroute. Il encloua de sa main les canons, puis il se retira, le dernier, lentement et froidement, et il dit : — Décidément les paysans ne tiennent pas. Il nous faut les Anglais.
 \subsubsection[{IV. C’est la seconde fois}]{IV \\
C’est la seconde fois}\phantomsection
\label{p3l2c4}
\noindent La victoire était complète.\par
Gauvain se tourna vers les hommes du bataillon du Bonnet-Rouge, et leur dit :\par
— Vous êtes douze, mais vous en valez mille.\par
Un mot du chef, c’était la croix d’honneur de ce temps-là.\par
Guéchamp, lancé par Gauvain hors de la ville, poursuivit les fuyards et en prit beaucoup.\par
On alluma des torches et l’on fouilla la ville.\par
Tout ce qui ne put s’évader se rendit. On illumina la grande rue avec des pots à feu. Elle était jonchée de morts et de blessés. La fin d’un combat s’arrache toujours, quelques groupes désespérés résistaient encore çà et là, on les cerna, et ils mirent bas les armes.\par
Gauvain avait remarqué dans le pêle-mêle effréné de la déroute un homme intrépide, espèce de faune agile et robuste, qui avait protégé la fuite des autres et ne s’était pas enfui. Ce paysan s’était magistralement servi de sa carabine, fusillant avec le canon, assommant avec la crosse, si bien qu’il l’avait cassée ;  maintenant il avait un pistolet dans un poing et un sabre dans l’autre. On n’osait l’approcher. Tout à coup Gauvain le vit qui chancelait et qui s’adossait à un pilier de la grande rue. Cet homme venait d’être blessé. Mais il avait toujours aux poings son sabre et son pistolet. Gauvain mit son épée sous son bras et alla à lui.\par
— Rends-toi, dit-il.\par
L’homme le regarda fixement. Son sang coulait sous ses vêtements d’une blessure qu’il avait, et faisait une mare à ses pieds.\par
— Tu es mon prisonnier, reprit Gauvain.\par
L’homme resta muet.\par
— Comment t’appelles-tu ?\par
L’homme dit :\par
— Je m’appelle Danse-à-l’Ombre.\par
— Tu es un vaillant, dit Gauvain.\par
Et il lui tendit la main.\par
L’homme répondit :\par
— Vive le roi !\par
Et ramassant ce qui lui restait de force, levant les deux bras à la fois, il tira au cœur de Gauvain un coup de pistolet et lui asséna sur la tête un coup de sabre.\par
Il fit cela avec une promptitude de tigre ; mais quelqu’un fut plus prompt encore. Ce fut un homme à cheval qui venait d’arriver et qui était là depuis quelques instants, sans qu’on eût fait attention à lui. Cet homme, voyant le Vendéen lever le sabre et le pistolet, se jeta entre lui et Gauvain. Sans cet homme, Gauvain était mort. Le cheval reçut le coup de pistolet, l’homme  reçut le coup de sabre, et tous deux tombèrent. Tout cela se fit le temps de jeter un cri.\par
Le Vendéen de son côté s’était affaissé sur le pavé.\par
Le coup de sabre avait frappé l’homme en plein visage ; il était à terre, évanoui. Le cheval était tué.\par
Gauvain s’approcha.\par
— Qui est cet homme ? dit-il.\par
Il le considéra. Le sang de la balafre inondait le blessé et lui faisait un masque rouge. Il était impossible de distinguer sa figure. On lui voyait des cheveux gris.\par
— Cet homme m’a sauvé la vie, poursuivit Gauvain. Quelqu’un d’ici le connaît-il ?\par
— Mon commandant, dit un soldat, cet homme est entré dans la ville tout à l’heure. Je l’ai vu arriver. Il venait par la route de Pontorson.\par
Le chirurgien-major de la colonne était accouru avec sa trousse. Le blessé était toujours sans connaissance. Le chirurgien l’examina et dit :\par
— Une simple balafre. Ce n’est rien. Cela se recoud. Dans huit jours il sera sur pied. C’est un beau coup de sabre.\par
Le blessé avait un manteau, une ceinture tricolore, des pistolets, un sabre. On le coucha sur une civière. On le déshabilla. On apporta un seau d’eau fraîche, le chirurgien lava la plaie. Le visage commença à apparaître. Gauvain le regardait avec une attention profonde.\par
— A-t-il des papiers sur lui ? demanda Gauvain.\par
 Le chirurgien tâta la poche de côté et en tira un portefeuille, qu’il tendit à Gauvain.\par
Cependant le blessé, ranimé par l’eau froide, revenait à lui. Ses paupières remuaient vaguement.\par
Gauvain fouillait le portefeuille ; il y trouva une feuille de papier pliée en quatre, il la déplia, il lut :\par
« Comité de salut public. Le citoyen Cimourdain... »\par
Il jeta un cri :\par
— Cimourdain !\par
Ce cri fit ouvrir les yeux au blessé.\par
Gauvain était éperdu.\par
— Cimourdain ! c’est vous ! C’est la seconde fois que vous me sauvez la vie.\par
Cimourdain regardait Gauvain. Un ineffable éclair de joie illuminait sa face sanglante.\par
Gauvain tomba à genoux devant le blessé en criant :\par
— Mon maître !\par
— Ton père, dit Cimourdain.
 \subsubsection[{V. La goutte d’eau froide}]{V \\
La goutte d’eau froide}\phantomsection
\label{p3l2c5}
\noindent Ils ne s’étaient pas vus depuis beaucoup d’années, mais leurs cœurs ne s’étaient jamais quittés ; ils se reconnurent comme s’ils s’étaient séparés la veille.\par
On avait improvisé une ambulance à l’hôtel de ville de Dol. On porta Cimourdain sur un lit dans une petite chambre contiguë à la grande salle commune aux blessés. Le chirurgien, qui avait recousu la balafre, mit fin aux épanchements entre ces deux hommes, et jugea qu’il fallait laisser dormir Cimourdain. Gauvain d’ailleurs était réclamé par ces mille soins qui sont les devoirs et les soucis de la victoire. Cimourdain resta seul ; mais il ne dormit pas ; il avait deux fièvres, la fièvre de sa blessure et la fièvre de sa joie.\par
Il ne dormit pas, et pourtant il ne lui semblait pas être éveillé. Était-ce possible ? son rêve était réalisé. Cimourdain était de ceux qui ne croient pas au quine, et il l’avait. Il retrouvait Gauvain. Il l’avait quitté enfant, il le retrouvait homme ; il le retrouvait grand, redoutable, intrépide. Il le retrouvait triomphant, et triomphant pour le peuple. Gauvain était en  Vendée le point d’appui de la révolution, et c’était lui, Cimourdain, qui avait fait cette colonne à la république. Ce victorieux était son élève. Ce qu’il voyait rayonner à travers cette jeune figure réservée peut-être au panthéon républicain, c’était sa pensée, à lui Cimourdain ; son disciple, l’enfant de son esprit, était dès à présent un héros et serait avant peu une gloire ; il semblait à Cimourdain qu’il revoyait sa propre âme faite Génie. Il venait de voir de ses yeux comment Gauvain faisait la guerre ; il était comme Chiron ayant vu combattre Achille. Rapport mystérieux entre le prêtre et le centaure ; car le prêtre n’est homme qu’à mi-corps.\par
Tous les hasards de cette aventure, mêlés à l’insomnie de sa blessure, emplissaient Cimourdain d’une sorte d’enivrement mystérieux. Une jeune destinée se levait, magnifique, et, ce qui ajoutait à sa joie profonde, il avait plein pouvoir sur cette destinée ; encore un succès comme celui qu’il venait de voir, et Cimourdain n’aurait qu’un mot à dire pour que la république confiât à Gauvain une armée. Rien n’éblouit comme l’étonnement de voir tout réussir. C’était le temps où chacun avait son rêve militaire ; chacun voulait faire un général ; Danton voulait faire Westermann, Marat voulait faire Rossignol, Hébert voulait faire Ronsin ; Robespierre voulait les défaire tous. Pourquoi pas Gauvain ? se disait Cimourdain ; et il songeait. L’illimité était devant lui ; il passait d’une hypothèse à l’autre ; tous les obstacles s’évanouissaient ; une fois qu’on a mis le pied sur cette échelle-là,  on ne s’arrête plus, c’est la montée infinie, on part de l’homme et l’on arrive à l’étoile. Un grand général n’est qu’un chef d’armées ; un grand capitaine est en même temps un chef d’idées ; Cimourdain rêvait Gauvain grand capitaine. Il lui semblait, car la rêverie va vite, voir Gauvain sur l’Océan, chassant les Anglais ; sur le Rhin, châtiant les rois du Nord ; aux Pyrénées, repoussant l’Espagne ; aux Alpes, faisant signe à Rome de se lever. Il y avait en Cimourdain deux hommes, un homme tendre et un homme sombre ; tous deux étaient contents ; car, l’inexorable étant son idéal, en même temps qu’il voyait Gauvain superbe, il le voyait terrible. Cimourdain pensait à tout ce qu’il fallait détruire avant de construire, et, certes, se disait-il, ce n’est pas l’heure des attendrissements. Gauvain sera « à la hauteur », mot du temps. Cimourdain se figurait Gauvain écrasant du pied les ténèbres, cuirassé de lumière, avec une lueur de météore au front, ouvrant les grandes ailes idéales de la justice, de la raison et du progrès, et une épée à la main ; ange, mais exterminateur.\par
Au plus fort de cette rêverie qui était presque une extase, il entendit, par la porte entr’ouverte, qu’on parlait dans la grande salle de l’ambulance, voisine de sa chambre ; il reconnut la voix de Gauvain ; cette voix, malgré les années d’absence, avait toujours été dans son oreille, et la voix de l’enfant se retrouve dans la voix de l’homme. Il écouta. Il y avait un bruit de pas. Des soldats disaient :\par
— Mon commandant, cet homme-ci est celui qui a  tiré sur vous. Pendant qu’on ne le voyait pas, il s’était traîné dans une cave. Nous l’avons trouvé. Le voilà.\par
Alors Cimourdain entendit ce dialogue entre Gauvain et l’homme :\par
— Tu es blessé ?\par
— Je me porte assez bien pour être fusillé.\par
— Mettez cet homme dans un lit. Pansez-le, soignez-le, guérissez-le.\par
— Je veux mourir.\par
— Tu vivras. Tu as voulu me tuer au nom du roi ; je te fais grâce au nom de la république.\par
Une ombre passa sur le front de Cimourdain. Il eut comme un réveil en sursaut, et il murmura avec une sorte d’accablement sinistre :\par
— En effet, c’est un clément.
 \subsubsection[{VI. Sein guéri, cœur saignant}]{VI \\
Sein guéri, cœur saignant}\phantomsection
\label{p3l2c6}
\noindent Une balafre se guérit vite ; mais il y avait quelque part quelqu’un de plus gravement blessé que Cimourdain. C’était la femme fusillée que le mendiant Tellmarch avait ramassée dans la grande mare de sang de la ferme d’Herbe-en-Pail.\par
Michelle Fléchard était plus en danger encore que Tellmarch ne l’avait cru ; au trou qu’elle avait au-dessus du sein correspondait un trou dans l’omoplate ; en même temps qu’une balle lui cassait la clavicule, une autre balle lui traversait l’épaule ; mais, comme le poumon n’avait pas été touché, elle put guérir. Tellmarch était un « philosophe », mot de paysans qui signifie un peu médecin, un peu chirurgien et un peu sorcier. Il soigna la blessée dans sa tanière de bête sur son grabat de varech, avec ces choses mystérieuses qu’on appelle des « simples », et, grâce à lui, elle vécut.\par
La clavicule se ressouda, les trous de la poitrine et de l’épaule se fermèrent ; après quelques semaines, la blessée fut convalescente.\par
 Un matin, elle put sortir du carnichot, appuyée sur Tellmarch ; elle alla s’asseoir sous les arbres au soleil. Tellmarch savait d’elle peu de chose, les plaies de poitrine exigent le silence, et, pendant la quasi-agonie qui avait précédé sa guérison, elle avait à peine dit quelques paroles. Quand elle voulait parler, Tellmarch la faisait taire ; mais elle avait une rêverie opiniâtre, et Tellmarch observait dans ses yeux une sombre allée et venue de pensées poignantes. Ce matin-là elle était forte, elle pouvait presque marcher seule ; une cure, c’est une paternité, et Tellmarch la regardait, heureux. Ce bon vieux homme se mit à sourire. Il lui parla.\par
— Eh bien, nous sommes debout. Nous n’avons plus de plaie.\par
— Qu’au cœur, dit-elle.\par
Et elle reprit :\par
— Alors vous ne savez pas du tout où ils sont ?\par
— Qui ça ? demanda Tellmarch.\par
— Mes enfants.\par
Cet « alors » exprimait tout un monde de pensées ; cela signifiait : « puisque vous ne m’en parlez pas, puisque depuis tant de jours vous êtes près de moi sans m’en ouvrir la bouche, puisque vous me faites taire chaque fois que je veux rompre le silence, puisque vous semblez craindre que je n’en parle, c’est que vous n’avez rien à m’en dire. » Souvent dans la fièvre, dans l’égarement, dans le délire, elle avait appelé ses enfants, et elle avait bien vu, car le délire fait ses remarques, que le vieux homme ne lui répondait pas.\par
 C’est qu’en effet Tellmarch ne savait que lui dire. Ce n’est pas aisé de parler à une mère de ses enfants perdus. Et puis, que savait-il ? rien. Il savait qu’une mère avait été fusillée, que cette mère avait été trouvée à terre par lui, que lorsqu’il l’avait ramassée, c’était à peu près un cadavre, que ce cadavre avait trois enfants, et que le marquis de Lantenac, après avoir fait fusiller la mère, avait emmené les enfants. Toutes ses informations s’arrêtaient là. Qu’est-ce que ces enfants étaient devenus ? Étaient-ils même encore vivants ? Il savait, pour s’en être informé, qu’il y avait deux garçons et une petite fille, à peine sevrée. Rien de plus. Il se faisait sur ce groupe infortuné une foule de questions, mais il n’y pouvait répondre. Les gens du pays qu’il avait interrogés s’étaient bornés à hocher la tête. M. de Lantenac était un homme dont on ne causait pas volontiers.\par
On ne parlait pas volontiers de Lantenac et on ne parlait pas volontiers à Tellmarch. Les paysans ont un genre de soupçon à eux. Ils n’aimaient pas Tellmarch. Tellmarch-le-Caimand était un homme inquiétant. Qu’avait-il à regarder toujours le ciel ? que faisait-il, et à quoi pensait-il dans ses longues heures d’immobilité ? Certes, il était étrange. Dans ce pays en pleine guerre, en pleine conflagration, en pleine combustion, où tous les hommes n’avaient qu’une affaire, la dévastation, et qu’un travail, le carnage, où c’était à qui brûlerait une maison, égorgerait une famille, massacrerait un poste, saccagerait un village, où l’on ne songeait qu’à se tendre des embuscades, qu’à s’attirer  dans des pièges, et qu’à s’entre-tuer les uns les autres, ce solitaire, absorbé dans la nature, comme submergé dans la paix immense des choses, cueillant des herbes et des plantes, uniquement occupé des fleurs, des oiseaux et des étoiles, était évidemment dangereux. Visiblement, il n’avait pas sa raison ; il ne s’embusquait derrière aucun buisson, il ne tirait aucun coup de fusil à personne. De là une certaine crainte autour de lui.\par
— Cet homme est fou, disaient les passants.\par
Tellmarch était plus qu’un homme isolé, c’était un homme évité.\par
On ne lui faisait point de questions, et on ne lui faisait guère de réponses. Il n’avait donc pu se renseigner autant qu’il l’aurait voulu. La guerre s’était répandue ailleurs, on était allé se battre plus loin, le marquis de Lantenac avait disparu de l’horizon, et dans l’état d’esprit où était Tellmarch, pour qu’il s’aperçût de la guerre, il fallait qu’elle mît le pied sur lui.\par
Après ce mot, — \emph{mes enfants}, — Tellmarch avait cessé de sourire, et la mère s’était mise à penser. Que se passait-il dans cette âme ? Elle était comme au fond d’un gouffre. Brusquement elle regarda Tellmarch, et cria de nouveau et presque avec un accent de colère :\par
— Mes enfants !\par
Tellmarch baissa la tête comme un coupable.\par
Il songeait à ce marquis de Lantenac qui certes ne pensait pas à lui, et qui, probablement, ne savait même plus qu’il existât. Il s’en rendait compte, il se disait : — Un seigneur, quand c’est dans le danger, ça vous  connaît ; quand c’est dehors, ça ne vous connaît plus.\par
Et il se demandait : — Mais alors pourquoi ai-je sauvé ce seigneur ?\par
Et il se répondait : — Parce que c’est un homme.\par
Il fut là-dessus quelque temps pensif, et il reprit en lui-même :\par
— En suis-je bien sûr ?\par
Et il se répéta son mot amer : — Si j’avais su !\par
Toute cette aventure l’accablait ; car dans ce qu’il avait fait il voyait une sorte d’énigme. Il méditait douloureusement. Une bonne action peut donc être une mauvaise action. Qui sauve le loup tue les brebis. Qui raccommode l’aile du vautour est responsable de sa griffe.\par
Il se sentait en effet coupable. La colère inconsciente de cette mère avait raison.\par
Pourtant, avoir sauvé cette mère le consolait d’avoir sauvé ce marquis.\par
Mais les enfants ?\par
La mère aussi songeait. Ces deux pensées se côtoyaient et, sans se le dire, se rencontraient peut-être, dans les ténèbres de la rêverie.\par
Cependant son regard, au fond duquel était la nuit, se fixa de nouveau sur Tellmarch.\par
— Ça ne peut pourtant pas se passer comme ça, dit-elle.\par
— Chut ! fit Tellmarch, et il mit le doigt sur sa bouche.\par
Elle poursuivit :\par
— Vous avez eu tort de me sauver, et je vous en  veux. J’aimerais mieux être morte, parce que je suis sûre que je les verrais. Je saurais où ils sont. Ils ne me verraient pas, mais je serais près d’eux. Une morte, ça doit pouvoir protéger.\par
Il lui prit le bras et lui tâta le pouls.\par
— Calmez-vous, vous vous redonnez la fièvre.\par
Elle lui demanda presque durement :\par
— Quand pourrai-je m’en aller ?\par
— Vous en aller ?\par
— Oui. Marcher.\par
— Jamais, si vous n’êtes pas raisonnable. Demain, si vous êtes sage.\par
— Qu’appelez-vous être sage ?\par
— Avoir confiance en Dieu.\par
— Dieu ! où m’a-t-il mis mes enfants ?\par
Elle était comme égarée. Sa voix devint très douce.\par
— Vous comprenez, lui dit-elle, je ne peux pas rester comme cela. Vous n’avez pas eu d’enfants, moi j’en ai eu. Cela fait une différence. On ne peut pas juger d’une chose quand on ne sait pas ce que c’est. Vous n’avez pas eu d’enfants, n’est-ce pas ?\par
— Non, répondit Tellmarch.\par
— Moi, je n’ai eu que ça. Sans mes enfants, est-ce que je suis ? Je voudrais qu’on m’expliquât pourquoi je n’ai pas mes enfants. Je sens bien qu’il se passe quelque chose, puisque je ne comprends pas. On a tué mon mari, on m’a fusillée, mais c’est égal, je ne comprends pas.\par
— Allons, dit Tellmarch, voilà que la fièvre vous reprend. Ne parlez plus.\par
 Elle le regarda, et se tut.\par
A partir de ce jour, elle ne parla plus.\par
Tellmarch fut obéi plus qu’il ne voulait. Elle passait de longues heures accroupie au pied du vieux arbre, stupéfaite. Elle songeait et se taisait. Le silence offre on ne sait quel abri aux âmes simples qui ont subi l’approfondissement sinistre de la douleur. Elle semblait renoncer à comprendre. A un certain degré le désespoir est inintelligible au désespéré.\par
Tellmarch l’examinait, ému. En présence de cette souffrance, ce vieux homme avait des pensées de femme. — Oh oui, se disait-il, ses lèvres ne parlent pas, mais ses yeux parlent, je vois bien ce qu’elle a, une idée fixe. Avoir été mère, et ne plus l’être ! avoir été nourrice, et ne plus l’être ! Elle ne peut pas se résigner. Elle pense à la toute petite qu’elle allaitait il n’y a pas longtemps. Elle y pense, elle y pense, elle y pense. Au fait, ce doit être si charmant de sentir une petite bouche rose qui vous tire votre âme de dedans le corps et qui avec votre vie à vous se fait une vie à elle !\par
Il se taisait de son côté, comprenant, devant un tel accablement, l’impuissance de la parole. Le silence d’une idée fixe est terrible. Et comment faire entendre raison à l’idée fixe d’une mère ? La maternité est sans issue ; on ne discute pas avec elle. Ce qui fait qu’une mère est sublime, c’est que c’est une espèce de bête. L’instinct maternel est divinement animal. La mère n’est plus femme, elle est femelle.\par
Les enfants sont des petits.\par
 De là dans la mère quelque chose d’inférieur et de supérieur au raisonnement. Une mère a un flair. L’immense volonté ténébreuse de la création est en elle, et la mène. Aveuglement plein de clairvoyance.\par
Tellmarch maintenant voulait faire parler cette malheureuse ; il n’y réussissait pas. Une fois, il lui dit :\par
— Par malheur, je suis vieux, et je ne marche plus. J’ai plus vite trouvé le bout de ma force que le bout de mon chemin. Après un quart d’heure, mes jambes refusent, et il faut que je m’arrête ; sans quoi je pourrais vous accompagner. Au fait, c’est peut-être un bien que je ne puisse pas. Je serais pour vous plus dangereux qu’utile ; on me tolère ici ; mais je suis suspect aux bleus comme paysan et aux paysans comme sorcier.\par
Il attendit ce qu’elle répondrait. Elle ne leva même pas les yeux.\par
Une idée fixe aboutit à la folie ou à l’héroïsme. Mais de quel héroïsme peut être capable une pauvre paysanne ? d’aucun. Elle peut être mère, et voilà tout. Chaque jour elle s’enfonçait davantage dans sa rêverie. Tellmarch l’observait.\par
Il chercha à l’occuper ; il lui apporta du fil, des aiguilles, un dé ; et en effet, ce qui fit plaisir au pauvre caimand, elle se mit à coudre ; elle songeait, mais elle travaillait, signe de santé ; les forces lui revenaient peu à peu ; elle raccommoda son linge, ses vêtements, ses souliers ; mais sa prunelle restait vitreuse. Tout en cousant elle chantait à demi-voix  des chansons obscures. Elle murmurait des noms, probablement des noms d’enfants, pas assez distinctement pour que Tellmarch les entendît. Elle s’interrompait et écoutait les oiseaux, comme s’ils avaient des nouvelles à lui donner. Elle regardait le temps qu’il faisait. Ses lèvres remuaient. Elle se parlait bas. Elle fit un sac, et elle le remplit de châtaignes. Un matin Tellmarch la vit qui se mettait en marche, l’œil fixé au hasard sur les profondeurs de la forêt.\par
— Où allez-vous ? lui demanda-t-il.\par
Elle répondit :\par
— Je vais les chercher.\par
Il n’essaya pas de la retenir.
 \subsubsection[{VII. Les deux pôles du vrai}]{VII \\
Les deux pôles du vrai}\phantomsection
\label{p3l2c7}
\noindent Au bout de quelques semaines pleines de tous les va-et-vient de la guerre civile, il n’était bruit dans le pays de Fougères que de deux hommes dont l’un était l’opposé de l’autre, et qui cependant faisaient la même œuvre, c’est-à-dire combattaient côte à côte le grand combat révolutionnaire.\par
Le sauvage duel vendéen continuait, mais la Vendée perdait du terrain. Dans l’Ille-et-Vilaine en particulier, grâce au jeune commandant qui, à Dol, avait si à propos riposté à l’audace des six mille royalistes par l’audace des quinze cents patriotes, l’insurrection était, sinon éteinte, du moins très amoindrie et très circonscrite. Plusieurs coups heureux avaient suivi celui-là, et de ces succès multipliés était née une situation nouvelle.\par
Les choses avaient changé de face, mais une singulière complication était survenue.\par
Dans toute cette partie de la Vendée, la république avait le dessus, ceci était hors de doute ; mais quelle république ? Dans le triomphe qui s’ébauchait, deux formes de la république étaient en présence, la république de la terreur et la république de la  clémence, l’une voulant vaincre par la rigueur et l’autre par la douceur. Laquelle prévaudrait ? Ces deux formes, la forme conciliante et la forme implacable, étaient représentées par deux hommes ayant chacun son influence et son autorité, l’un commandant militaire, l’autre délégué civil ; lequel de ces deux hommes l’emporterait ? De ces deux hommes, l’un, le délégué, avait de redoutables points d’appui ; il était arrivé apportant la menaçante consigne de la commune de Paris aux bataillons de Santerre : « \emph{Pas de grâce, pas de quartier !} » Il avait, pour tout soumettre à son autorité, le décret de la Convention portant « peine de mort contre quiconque mettrait en liberté et ferait évader un chef rebelle prisonnier », de pleins pouvoirs émanés du comité de salut public, et une injonction de lui obéir, à lui délégué, signée : R{\scshape obespierre}, D{\scshape anton}, M{\scshape arat}. L’autre, le soldat, n’avait pour lui que cette force, la pitié.\par
Il n’avait pour lui que son bras, qui battait les ennemis, et son cœur, qui leur faisait grâce. Vainqueur, il se croyait le droit d’épargner les vaincus.\par
De là un conflit latent, mais profond, entre ces deux hommes. Ils étaient tous les deux dans des nuages différents, tous les deux combattant la rébellion, et chacun ayant sa foudre à lui, l’un la victoire, l’autre la terreur.\par
Dans tout le Bocage on ne parlait que d’eux ; et, ce qui ajoutait à l’anxiété des regards fixés sur eux de toutes parts, c’est que ces deux hommes, si absolument opposés, étaient en même temps étroitement  unis. Ces deux antagonistes étaient deux amis. Jamais sympathie plus haute et plus profonde n’avait rapproché deux cœurs ; le farouche avait sauvé la vie au débonnaire, et il en avait la balafre au visage. Ces deux hommes incarnaient, l’un la mort, l’autre la vie ; l’un était le principe terrible, l’autre le principe pacifique, et ils s’aimaient. Problème étrange. Qu’on se figure Oreste miséricordieux et Pylade inclément. Qu’on se figure Arimane frère d’Ormus.\par
Ajoutons que celui des deux qu’on appelait « le féroce » était en même temps le plus fraternel des hommes ; il pansait les blessés, soignait les malades, passait ses jours et ses nuits dans les ambulances et les hôpitaux, s’attendrissait sur des enfants pieds nus, n’avait rien à lui, donnait tout aux pauvres. Quand on se battait, il y allait ; il marchait à la tête des colonnes et au plus fort du combat, armé, car il avait à sa ceinture un sabre et deux pistolets, et désarmé, car jamais on ne l’avait vu tirer son sabre et toucher à ses pistolets. Il affrontait les coups et n’en rendait pas. On disait qu’il avait été prêtre.\par
L’un de ces hommes était Gauvain, l’autre était Cimourdain.\par
L’amitié était entre les deux hommes, mais la haine était entre les deux principes ; c’était comme une âme coupée en deux, et partagée ; Gauvain, en effet, avait reçu une moitié de l’âme de Cimourdain, mais la moitié douce. Il semblait que Gauvain avait eu le rayon blanc et que Cimourdain avait gardé pour lui ce qu’on pourrait appeler le rayon noir. De là un  désaccord intime. Cette sourde guerre ne pouvait pas ne point éclater. Un matin la bataille commença.\par
Cimourdain dit à Gauvain :\par
— Où en sommes-nous ?\par
Gauvain répondit :\par
— Vous le savez aussi bien que moi. J’ai dispersé les bandes de Lantenac. Il n’a plus avec lui que quelques hommes. Le voilà acculé à la forêt de Fougères. Dans huit jours, il sera cerné.\par
— Et dans quinze jours ?\par
— Il sera pris.\par
— Et puis ?\par
— Vous avez lu mon affiche ?\par
— Oui. Eh bien ?\par
— Il sera fusillé.\par
— Encore de la clémence. Il faut qu’il soit guillotiné.\par
— Moi, dit Gauvain, je suis pour la mort militaire.\par
— Et moi, répliqua Cimourdain, pour la mort révolutionnaire.\par
Il regarda Gauvain en face et lui dit :\par
— Pourquoi as-tu fait mettre en liberté ces religieuses du couvent de Saint-Marc-le-Blanc ?\par
— Je ne fais pas la guerre aux femmes, répondit Gauvain.\par
— Ces femmes-là haïssent le peuple. Et pour la haine une femme vaut dix hommes. Pourquoi as-tu refusé d’envoyer au tribunal révolutionnaire tout ce troupeau de vieux prêtres fanatiques pris à Louvigné ?\par
— Je ne fais pas la guerre aux vieillards.\par
 — Un vieux prêtre est pire qu’un jeune. La rébellion est plus dangereuse, prêchée par les cheveux blancs. On a foi dans les rides. Pas de fausse pitié, Gauvain. Les régicides sont les libérateurs. Aie l’œil fixé sur la tour du Temple.\par
— La tour du Temple. J’en ferais sortir le dauphin. Je ne fais pas la guerre aux enfants.\par
L’œil de Cimourdain devint sévère.\par
— Gauvain, sache qu’il faut faire la guerre à la femme quand elle se nomme Marie-Antoinette, au vieillard quand il se nomme Pie VI, pape, et à l’enfant quand il se nomme Louis Capet.\par
— Mon maître, je ne suis pas un homme politique.\par
— Tâche de ne pas être un homme dangereux. Pourquoi, à l’attaque du poste de Cossé, quand le rebelle Jean Treton, acculé et perdu, s’est rué seul, le sabre au poing, contre toute ta colonne, as-tu crié : \emph{Ouvrez les rangs. Laissez passer !}\par
— Parce qu’on ne se met pas à quinze cents pour tuer un homme.\par
— Pourquoi, à la Cailleterie d’Astillé, quand tu as vu que tes soldats allaient tuer le Vendéen Joseph Bézier, qui était blessé, et qui se traînait, as-tu crié : \emph{Allez en avant ! J’en fais mon affaire !} et as-tu tiré ton coup de pistolet en l’air ?\par
— Parce qu’on ne tue pas un homme à terre.\par
— Et tu as eu tort. Tous deux sont aujourd’hui chefs de bande ; Joseph Bézier, c’est Moustache, et Jean Treton, c’est Jambe-d’Argent. En sauvant ces deux hommes, tu as donné deux ennemis à la république.\par
 — Certes, je voudrais lui faire des amis, et non lui donner des ennemis.\par
— Pourquoi, après ta victoire de Landéan, n’as-tu pas fait fusiller tes trois cents paysans prisonniers ?\par
— Parce que, Bonchamp ayant fait grâce aux prisonniers républicains, j’ai voulu qu’il fût dit que la république faisait grâce aux prisonniers royalistes.\par
— Mais alors, si tu prends Lantenac, tu lui feras grâce ?\par
— Non.\par
— Pourquoi ? Puisque tu as fait grâce aux trois cents paysans ?\par
— Les paysans sont des ignorants ; Lantenac sait ce qu’il fait.\par
— Mais Lantenac est ton parent.\par
— La France est la grande parente.\par
— Lantenac est un vieillard.\par
— Lantenac est un étranger. Lantenac n’a pas d’âge. Lantenac appelle les Anglais. Lantenac, c’est l’invasion. Lantenac est l’ennemi de la patrie. Le duel entre lui et moi ne peut finir que par sa mort ou par la mienne.\par
— Gauvain, souviens-toi de cette parole.\par
— Elle est dite.\par
Il y eut un silence, et tous deux se regardèrent.\par
Et Gauvain reprit :\par
— Ce sera une date sanglante que cette année 93 où nous sommes.\par
— Prends garde ! s’écria Cimourdain. Les devoirs terribles existent. N’accuse pas qui n’est point accusable. Depuis quand la maladie est-elle la faute du  médecin ? Oui, ce qui caractérise cette année énorme, c’est d’être sans pitié. Pourquoi ? parce qu’elle est la grande année révolutionnaire. Cette année où nous sommes incarne la révolution. La révolution a un ennemi, le vieux monde, et elle est sans pitié pour lui, de même que le chirurgien a un ennemi, la gangrène, et est sans pitié pour elle. La révolution extirpe la royauté dans le roi, l’aristocratie dans le noble, le despotisme dans le soldat, la superstition dans le prêtre, la barbarie dans le juge, en un mot, tout ce qui est la tyrannie, dans tout ce qui est le tyran. L’opération est effrayante, la révolution la fait d’une main sûre. Quant à la quantité de chair saine qu’elle sacrifie, demande à Boerhave ce qu’il en pense. Quelle tumeur à couper n’entraîne une perte de sang ? Quel incendie à éteindre n’exige la part du feu ? Ces nécessités redoutables sont la condition même du succès. Un chirurgien ressemble à un boucher ; un guérisseur peut faire l’effet d’un bourreau. La révolution se dévoue à son œuvre fatale. Elle mutile, mais elle sauve. Quoi ! vous lui demandez grâce pour le virus ! vous voulez qu’elle soit clémente pour ce qui est vénéneux ! Elle n’écoute pas. Elle tient le passé, elle l’achèvera. Elle fait à la civilisation une incision profonde d’où sortira la santé du genre humain. Vous souffrez ? sans doute. Combien de temps cela durera-t-il ? le temps de l’opération. Ensuite vous vivrez. La révolution ampute le monde. De là cette hémorrhagie, 93.\par
— Le chirurgien est calme, dit Gauvain, et les hommes que je vois sont violents.\par
 — La révolution, répliqua Cimourdain, veut pour l’aider des ouvriers farouches. Elle repousse toute main qui tremble. Elle n’a foi qu’aux inexorables. Danton, c’est le terrible, Robespierre, c’est l’inflexible, Saint-Just, c’est l’irréductible, Marat, c’est l’implacable. Prends-y garde, Gauvain. Ces noms-là sont nécessaires. Ils valent pour nous des armées. Ils terrifieront l’Europe.\par
— Et peut-être aussi l’avenir, dit Gauvain.\par
Il s’arrêta et repartit :\par
— Du reste, mon maître, vous faites erreur, je n’accuse personne. Selon moi, le vrai point de vue de la révolution, c’est l’irresponsabilité. Personne n’est innocent, personne n’est coupable. Louis XVI, c’est un mouton jeté parmi des lions. Il veut fuir, il veut se sauver, il cherche à se défendre ; il mordrait, s’il pouvait. Mais n’est pas lion qui veut. Sa velléité passe pour crime. Ce mouton en colère montre les dents. Le traître ! disent les lions. Et ils le mangent. Cela fait, ils se battent entre eux.\par
— Le mouton est une bête.\par
— Et les lions, que sont-ils ?\par
Cette réplique fit songer Cimourdain. Il releva la tête et dit :\par
— Ces lions-là sont des consciences. Ces lions-là sont des idées. Ces lions-là sont des principes.\par
— Ils font la terreur.\par
— Un jour, la révolution sera la justification de la terreur.\par
— Craignez que la terreur ne soit la calomnie de la révolution.\par
 Et Gauvain reprit :\par
— Liberté, Égalité, Fraternité, ce sont des dogmes de paix et d’harmonie. Pourquoi leur donner un aspect effrayant ? Que voulons-nous ? conquérir les peuples à la république universelle. Eh bien, ne leur faisons pas peur. A quoi bon l’intimidation ? Pas plus que les oiseaux, les peuples ne sont attirés par l’épouvantail. Il ne faut pas faire le mal pour faire le bien. On ne renverse pas le trône pour laisser l’échafaud debout. Mort aux rois, et vie aux nations. Abattons les couronnes, épargnons les têtes. La révolution, c’est la concorde, et non l’effroi. Les idées douces sont mal servies par les hommes incléments. Amnistie est pour moi le plus beau mot de la langue humaine. Je ne veux verser de sang qu’en risquant le mien. Du reste, je ne sais que combattre, et je ne suis qu’un soldat. Mais si l’on ne peut pardonner, cela ne vaut pas la peine de vaincre. Soyons pendant la bataille les ennemis de nos ennemis, et après la victoire leurs frères.\par
— Prends garde ! répéta Cimourdain pour la troisième fois. Gauvain, tu es pour moi plus que mon fils, prends garde !\par
Et il ajouta, pensif :\par
— Dans des temps comme les nôtres, la pitié peut être une des formes de la trahison.\par
En entendant parler ces deux hommes, on eût cru entendre le dialogue de l’épée et de la hache.
 \subsubsection[{VIII. Dolorosa}]{VIII \\
Dolorosa}\phantomsection
\label{p3l2c8}
\noindent Cependant la mère cherchait ses petits.\par
Elle allait devant elle. Comment vivait-elle ? Impossible de le dire. Elle ne le savait pas elle-même. Elle marcha des jours et des nuits ; elle mendia, elle mangea de l’herbe, elle coucha à terre, elle dormit en plein air, dans les broussailles, sous les étoiles, quelquefois sous la pluie et la bise.\par
Elle rôdait de village en village, de métairie en métairie, s’informant. Elle s’arrêtait aux seuils. Sa robe était en haillons. Quelquefois on l’accueillait, quelquefois on la chassait. Quand elle ne pouvait entrer dans les maisons, elle allait dans les bois.\par
Elle ne connaissait pas le pays, elle ignorait tout, excepté Siscoignard et la paroisse d’Azé, elle n’avait point d’itinéraire, elle revenait sur ses pas, recommençait une route déjà parcourue, faisait du chemin inutile. Elle suivait tantôt le pavé, tantôt l’ornière d’une charrette, tantôt les sentiers dans les taillis. A cette vie au hasard, elle avait usé ses misérables vêtements. Elle avait marché d’abord avec ses souliers, puis avec ses pieds nus, puis avec ses pieds sanglants.\par
Elle allait à travers la guerre, à travers les coups de fusil, sans rien entendre, sans rien voir, sans rien  éviter, cherchant ses enfants. Tout étant en révolte, il n’y avait plus de gendarmes, plus de maires, plus d’autorités. Elle n’avait affaire qu’aux passants.\par
Elle leur parlait. Elle demandait :\par
— Avez-vous vu quelque part trois petits enfants ? \par
Les passants levaient la tête.\par
— Deux garçons et une fille, disait-elle.\par
Elle continuait :\par
— René-Jean, Gros-Alain, Georgette ? Vous n’avez pas vu ça ?\par
Elle poursuivait :\par
— L’aîné a quatre ans et demi, la petite a vingt mois.\par
Elle ajoutait :\par
— Savez-vous où ils sont ? on me les a pris.\par
On la regardait et c’était tout.\par
Voyant qu’on ne la comprenait pas, elle disait :\par
— C’est qu’ils sont à moi. Voilà pourquoi.\par
Les gens passaient leur chemin. Alors elle s’arrêtait et ne disait plus rien, et se déchirait le sein avec les ongles.\par
Un jour pourtant un paysan l’écouta. Le bonhomme se mit à réfléchir.\par
— Attendez donc, dit-il. Trois enfants ?\par
— Oui.\par
— Deux garçons ?...\par
— Et une fille.\par
— C’est ça que vous cherchez ?\par
— Oui.\par
— J’ai ouï parler d’un seigneur qui avait pris trois petits enfants et qui les avait avec lui.\par
 — Où est cet homme ? cria-t-elle. Où sont-ils ?\par
Le paysan répondit :\par
— Allez à la Tourgue.\par
— Est-ce que c’est là que je trouverai mes enfants ?\par
— Peut-être bien que oui.\par
— Vous dites ?...\par
— La Tourgue.\par
— Qu’est-ce que c’est que la Tourgue ?\par
— C’est un endroit.\par
— Est-ce un village ? un château ? une métairie ?\par
— Je n’y suis jamais allé.\par
— Est-ce loin ?\par
— Ce n’est pas près.\par
— De quel côté ?\par
— Du côté de Fougères.\par
— Par où y va-t-on ?\par
— Vous êtes à Vantortes, dit le paysan, vous laisserez Ernée à gauche et Coxelles à droite, vous passerez par Lorchamp et vous traverserez le Leroux.\par
Et le paysan leva sa main vers l’occident.\par
— Toujours droit devant vous en allant du côté où le soleil se couche.\par
Avant que le paysan eût baissé son bras, elle était en marche.\par
Le paysan lui cria :\par
— Mais prenez garde. On se bat par là.\par
Elle ne se retourna point pour lui répondre, et continua d’aller en avant.
 \subsubsection[{IX. Une Bastille de province}]{IX \\
Une Bastille de province}\phantomsection
\label{p3l2c9}
\paragraph[{i La Tourgue.}]{\textsc{i} \\
La Tourgue.}\phantomsection
\label{p3l2c9p1}
\noindent Le voyageur qui, il y a quarante ans, entré dans la forêt de Fougères du côté de Laignelet, en ressortait du côté de Parigné, faisait, sur la lisière de cette profonde futaie, une rencontre sinistre. En débouchant du hallier, il avait brusquement devant lui la Tourgue.\par
Non la Tourgue vivante, mais la Tourgue morte. La Tourgue lézardée, sabordée, balafrée, démantelée. La ruine est à l’édifice ce que le fantôme est à l’homme. Pas de plus lugubre vision que la Tourgue. Ce qu’on avait sous les yeux, c’était une haute tour ronde, toute seule au coin du bois comme un malfaiteur. Cette tour, droite sur un bloc de roche à pic, avait presque l’aspect romain, tant elle était correcte et solide, et tant dans cette masse robuste l’idée de la puissance était mêlée à l’idée de la chute. Romaine, elle l’était même un peu, car elle était romane. Commencée au neuvième siècle,  elle avait été achevée au douzième, après la troisième croisade. Les impostes à oreillons de ses baies disaient son âge. On approchait, on gravissait l’escarpement, on apercevait une brèche, on se risquait à entrer, on était dedans, c’était vide. C’était quelque chose comme l’intérieur d’un clairon de pierre posé debout sur le sol. Du haut en bas, aucun diaphragme ; pas de toit, pas de plafonds, pas de planchers, des arrachements de voûtes et de cheminées, des embrasures à fauconneaux, à des hauteurs diverses, des cordons de corbeaux de granit et quelques poutres transversales marquant les étages ; sur les poutres les fientes des oiseaux de nuit, la muraille colossale, quinze pieds d’épaisseur à la base et douze au sommet, çà et là des crevasses et des trous qui avaient été des portes, par où l’on entrevoyait des escaliers dans l’intérieur ténébreux du mur. Le passant qui pénétrait là le soir entendait crier les hulottes, les tette-chèvres, les bihoreaux et les crapauds-volants, et voyait sous ses pieds des ronces, des pierres, des reptiles, et sur sa tête, à travers une rondeur noire qui était le haut de la tour et qui semblait la bouche d’un puits énorme, les étoiles.\par
C’était la tradition du pays qu’aux étages supérieurs de cette tour il y avait des portes secrètes faites, comme les portes des tombeaux des rois de Juda, d’une grosse pierre tournant sur pivot, s’ouvrant, puis se refermant, et s’effaçant dans la muraille ; mode architecturale rapportée des croisades avec l’ogive. Quand ces portes étaient closes, il était impossible de les retrouver, tant elles étaient bien mêlées aux autres  pierres du mur. On voit encore aujourd’hui de ces portes-là dans les mystérieuses cités de l’Anti-Liban, échappées au tremblement des douze villes sous Tibère.
\paragraph[{ii La Brèche.}]{\textsc{ii} \\
La Brèche.}\phantomsection
\label{p3l2c9p2}
\noindent La brèche par où l’on entrait dans la ruine était une trouée de mine. Pour un connaisseur, familier avec Errard, Sardi et Pagan, cette mine avait été savamment faite. La chambre à feu en bonnet de prêtre était proportionnée à la puissance du donjon qu’elle avait à éventrer. Elle avait dû contenir au moins deux quintaux de poudre. On y arrivait par un canal serpentant qui vaut mieux que le canal droit ; l’écroulement produit par la mine montrait à nu dans le déchirement de la pierre le saucisson, qui avait le diamètre voulu d’un œuf de poule. L’explosion avait fait à la muraille une blessure profonde par où les assiégeants avaient dû pouvoir entrer. Cette tour avait évidemment soutenu, à diverses époques, de vrais sièges en règle ; elle était criblée de mitrailles ; et ces mitrailles n’étaient pas toutes du même temps ; chaque projectile a sa façon de marquer un rempart, et tous avaient laissé à ce donjon leur balafre, depuis les boulets de pierre du quatorzième siècle jusqu’aux boulets de fer du dix-huitième.\par
 La brèche donnait entrée dans ce qui avait dû être le rez-de-chaussée. Vis-à-vis de la brèche, dans le mur de la tour, s’ouvrait le guichet d’une crypte taillée dans le roc et se prolongeant dans les fondations de la tour jusque sous la salle du rez-de-chaussée.\par
Cette crypte, aux trois quarts comblée, a été déblayée en 1835 par les soins de M. Auguste Le Prévost, l’antiquaire de Bernay.
\paragraph[{iii L’Oubliette.}]{\textsc{iii} \\
L’Oubliette.}\phantomsection
\label{p3l2c9p3}
\noindent Cette crypte était l’oubliette. Tout donjon avait la sienne. Cette crypte, comme beaucoup de caves pénales des mêmes époques, avait deux étages. Le premier étage, où l’on pénétrait par le guichet, était une chambre voûtée assez vaste, de plain-pied avec la salle du rez-de-chaussée. On voyait sur la paroi de cette chambre deux sillons parallèles et verticaux qui allaient d’un mur à l’autre en passant par la voûte où ils étaient profondément empreints, et qui donnaient l’idée de deux ornières. C’étaient deux ornières en effet. Ces deux sillons avaient été creusés par deux roues. Jadis, aux temps féodaux, c’était dans cette chambre que se faisait l’écartèlement, par un procédé moins tapageur que les quatre chevaux. Il y avait là deux roues, si fortes et si grandes qu’elles touchaient  les murs et la voûte. On attachait à chacune de ces roues un bras et une jambe du patient, puis on faisait tourner les deux roues en sens inverse, ce qui arrachait l’homme. Il fallait de l’effort ; de là les ornières creusées dans la pierre que les roues effleuraient. On peut voir encore aujourd’hui une chambre de ce genre à Vianden.\par
Au-dessous de cette chambre il y en avait une autre. C’était l’oubliette véritable. On n’y entrait point par une porte, on y pénétrait par un trou. Le patient, nu, était descendu, au moyen d’une corde sous les aisselles, dans la chambre d’en bas par un soupirail pratiqué au milieu du dallage de la chambre d’en haut. S’il s’obstinait à vivre, on lui jetait sa nourriture par ce trou. On voit encore aujourd’hui un trou de ce genre à Bouillon.\par
Par ce trou il venait du vent. La chambre d’en bas, creusée sous la salle du rez-de-chaussée, était plutôt un puits qu’une chambre. Elle aboutissait à de l’eau, et un souffle glacial l’emplissait. Ce vent, qui faisait mourir le prisonnier d’en bas, faisait vivre le prisonnier d’en haut. Il rendait la prison respirable. Le prisonnier d’en haut, à tâtons sous sa voûte, ne recevait d’air que par ce trou. Du reste, qui y entrait, ou qui y tombait, n’en sortait plus. C’était au prisonnier à s’en garer dans l’obscurité. Un faux pas pouvait du patient d’en haut faire le patient d’en bas. Cela le regardait. S’il tenait à la vie, ce trou était son danger ; s’il s’ennuyait, ce trou était sa ressource. L’étage supérieur était le cachot, l’étage inférieur était le tombeau.  Superposition ressemblante à la société d’alors.\par
C’est là ce que nos aïeux appelaient « un cul-de-basse-fosse ». La chose ayant disparu, le nom pour nous n’a plus de sens. Grâce à la révolution, nous entendons prononcer ces mots-là avec indifférence.\par
Du dehors de la tour, au-dessus de la brèche qui en était, il y a quarante ans, l’entrée unique, on apercevait une embrasure plus large que les autres meurtrières, à laquelle pendait un grillage de fer descellé et défoncé.
\paragraph[{iv Le Pont-châtelet.}]{\textsc{iv} \\
Le Pont-châtelet.}\phantomsection
\label{p3l2c9p4}
\noindent A cette tour, et du côté opposé à la brèche, se rattachait un pont de pierre de trois arches peu endommagées. Le pont avait porté un corps de logis dont il restait quelques tronçons. Ce corps de logis, où étaient visibles les marques d’un incendie, n’avait plus que sa charpente noircie, sorte d’ossature à travers laquelle passait le jour, et qui se dressait auprès de la tour, comme un squelette à côté d’un fantôme.\par
Cette ruine est aujourd’hui tout à fait démolie, et il n’en reste aucune trace. Ce qu’ont fait beaucoup de siècles et beaucoup de rois, il suffit d’un jour et d’un paysan pour le défaire.\par
\emph{La Tourgue}, abréviation paysanne, signifie la Tour-Gauvain, de même que \emph{la Jupelle} signifie la Jupellière,  et que ce nom d’un bossu chef de bande, \emph{Pinson-le-Tort}, signifie Pinson-le-Tortu.\par
La Tourgue, qui il y a quarante ans était une ruine et qui aujourd’hui est une ombre, était en 1793 une forteresse. C’était la vieille bastille des Gauvain, gardant à l’occident l’entrée de la forêt de Fougères, forêt qui, elle-même, est à peine un bois maintenant.\par
On avait construit cette citadelle sur un de ces gros blocs de schiste qui abondent entre Mayenne et Dinan, et qui sont partout épars parmi les halliers et les bruyères, comme si les titans s’étaient jeté là des pavés à la tête.\par
La tour était toute la forteresse ; sous la tour le rocher, au pied du rocher un de ces cours d’eau que le mois de janvier change en torrents et que le mois de juin met à sec.\par
Simplifiée à ce point, cette forteresse était, au moyen âge, à peu près imprenable. Le pont l’affaiblissait. Les Gauvain gothiques l’avaient bâtie sans pont. On y abordait par une de ces passerelles branlantes qu’un coup de hache suffisait à rompre. Tant que les Gauvain furent vicomtes, elle leur plut ainsi, et ils s’en contentèrent ; mais quand ils furent marquis, et quand ils quittèrent la caverne pour la cour, ils jetèrent trois arches sur le torrent, et ils se firent accessibles du côté de la plaine de même qu’ils s’étaient faits accessibles du côté du roi. Les marquis au dix-septième siècle, et les marquises au dix-huitième, ne tenaient plus à être imprenables. Copier Versailles remplaça ceci : continuer les aïeux.\par
 En face de la tour, du côté occidental, il y avait un plateau assez élevé allant aboutir aux plaines ; ce plateau venait presque toucher la tour, et n’en était séparé que par un ravin très creux où coulait le cours d’eau qui est un affluent du Couesnon. Le pont, trait d’union entre la forteresse et le plateau, fut fait haut sur piles ; et sur ces piles on construisit, comme à Chenonceaux, un édifice en style Mansard plus logeable que la tour. Mais les mœurs étaient encore très rudes ; les seigneurs gardèrent la coutume d’habiter les chambres du donjon pareilles à des cachots. Quant au bâtiment sur le pont, qui était une sorte de petit châtelet, on y pratiqua un long couloir qui servait d’entrée et qu’on appela la salle des gardes ; au-dessus de cette salle des gardes, qui était une sorte d’entre-sol, on mit une bibliothèque, au-dessus de la bibliothèque un grenier. De longues fenêtres à petites vitres en verre de Bohême, des pilastres entre les fenêtres, des médaillons sculptés dans le mur ; trois étages ; en bas des pertuisanes et des mousquets ; au milieu, des livres ; en haut, des sacs d’avoine ; tout cela était un peu sauvage et fort noble.\par
La tour à côté était farouche.\par
Elle dominait cette bâtisse coquette de toute sa hauteur lugubre. De la plate-forme on pouvait foudroyer le pont.\par
Les deux édifices, l’un abrupt, l’autre poli, se choquaient plus qu’ils ne s’accostaient. Les deux styles n’étaient point d’accord ; bien que deux demi-cercles semblent devoir être identiques, rien ne ressemble  moins à un plein-cintre roman qu’une archivolte classique. Cette tour digne des forêts était une étrange voisine pour ce pont digne de Versailles. Qu’on se figure Alain Barbe-Torte donnant le bras à Louis XIV. L’ensemble terrifiait. Des deux majestés mêlées sortait on ne sait quoi de féroce.\par
Au point de vue militaire, le pont, insistons-y, livrait presque la tour. Il l’embellissait et la désarmait ; en gagnant de l’ornement elle avait perdu de la force. Le pont la mettait de plain-pied avec le plateau. Toujours inexpugnable du côté de la forêt, elle était maintenant vulnérable du côté de la plaine. Autrefois elle commandait le plateau, à présent le plateau la commandait. Un ennemi installé là serait vite maître du pont. La bibliothèque et le grenier étaient pour l’assiégeant, et contre la forteresse. Une bibliothèque et un grenier se ressemblent en ceci que les livres et la paille sont du combustible. Pour un assiégeant qui utilise l’incendie, brûler Homère ou brûler une botte de foin, pourvu que cela brûle, c’est la même chose. Les Français l’ont prouvé aux Allemands en brûlant la bibliothèque de Heidelberg, et les Allemands l’ont prouvé aux Français en brûlant la bibliothèque de Strasbourg. Ce pont, ajouté à la Tourgue, était donc stratégiquement une faute ; mais au dix-septième siècle, sous Colbert et Louvois, les princes Gauvain, pas plus que les princes de Rohan ou les princes de La Trémoille, ne se croyaient désormais assiégeables. Pourtant les constructeurs du pont avaient pris quelques précautions. Premièrement, ils avaient prévu l’incendie ;  au-dessous des trois fenêtres du côté aval, ils avaient accroché transversalement, à des crampons qu’on voyait encore il y a un demi-siècle, une forte échelle de sauvetage ayant pour longueur la hauteur des deux premiers étages du pont, hauteur qui dépassait celle de trois étages ordinaires ; deuxièmement, ils avaient prévu l’assaut ; ils avaient isolé le pont de la tour au moyen d’une lourde et basse porte de fer ; cette porte était cintrée ; on la fermait avec une grosse clef qui était dans une cachette connue du maître seul, et, une fois fermée, cette porte pouvait défier le bélier, et presque braver le boulet.\par
Il fallait passer par le pont pour arriver à cette porte, et passer par cette porte pour pénétrer dans la tour. Pas d’autre entrée.
\paragraph[{v La Porte de fer.}]{\textsc{v} \\
La Porte de fer.}\phantomsection
\label{p3l2c9p5}
\noindent Le deuxième étage du châtelet du pont, surélevé à cause des piles, correspondait avec le deuxième étage de la tour ; c’est à cette hauteur que, pour plus de sûreté, avait été placée la porte de fer.\par
La porte de fer s’ouvrait du côté du pont sur la bibliothèque et du côté de la tour sur une grande salle voûtée avec pilier au centre. Cette salle, on vient de le dire, était le second étage du donjon. Elle était ronde comme la tour ; de longues meurtrières, donnant  sur la campagne, l’éclairaient. La muraille, toute sauvage, était nue, et rien n’en cachait les pierres, d’ailleurs très symétriquement ajustées. On arrivait à cette salle par un escalier en colimaçon pratiqué dans la muraille, chose toute simple quand les murs ont quinze pieds d’épaisseur. Au moyen âge, on prenait une ville rue par rue, une rue maison par maison, une maison chambre par chambre. On assiégeait une forteresse étage par étage. La Tourgue était sous ce rapport fort savamment disposée et très revêche et très difficile. On montait d’un étage à l’autre par un escalier en spirale d’un abord malaisé ; les portes étaient de biais et n’avaient pas hauteur d’homme, et il fallait baisser la tête pour y passer ; or, tête baissée c’est tête assommée ; et, à chaque porte, l’assiégé attendait l’assiégeant.\par
Il y avait au-dessous de la salle ronde à pilier deux chambres pareilles, qui étaient le premier étage et le rez-de-chaussée, et au-dessus trois ; sur ces six chambres superposées la tour se fermait par un couvercle de pierre qui était la plate-forme, et où l’on arrivait par une étroite guérite.\par
Les quinze pieds d’épaisseur de muraille qu’on avait dû percer pour y placer la porte de fer, et au milieu desquels elle était scellée, l’emboîtaient dans une longue voussure ; de sorte que la porte, quand elle était fermée, était, tant du côté de la tour que du côté du pont, sous un porche de six ou sept pieds de profondeur ; quand elle était ouverte, ces deux porches se confondaient et faisaient la voûte d’entrée.\par
 Sous le porche du côté du pont s’ouvrait dans l’épaisseur du mur le guichet bas d’une vis-de-Saint-Gilles qui menait au couloir du premier étage sous la bibliothèque ; c’était encore là une difficulté pour l’assiégeant. Le châtelet sur le pont n’offrait à son extrémité du côté du plateau qu’un mur à pic, et le pont était coupé là. Un pont-levis, appliqué contre une porte basse, le mettait en communication avec le plateau, et ce pont-levis, qui, à cause de la hauteur du plateau, ne s’abaissait jamais qu’en plan incliné, donnait dans le long couloir dit salle des gardes. Une fois maître de ce couloir, l’assiégeant, pour arriver à la porte de fer, était forcé d’enlever de vive force l’escalier en vis-de-Saint-Gilles qui montait au deuxième étage.
\paragraph[{vi La Bibliothèque.}]{\textsc{vi} \\
La Bibliothèque.}\phantomsection
\label{p3l2c9p6}
\noindent Quant à la bibliothèque, c’était une salle oblongue ayant la largeur et la longueur du pont, et une porte unique, la porte de fer. Une fausse porte battante, capitonnée de drap vert, et qu’il suffisait de pousser, masquait à l’intérieur la voussure d’entrée de la tour. Le mur de la bibliothèque était du haut en bas, et du plancher au plafond, revêtu d’armoires vitrées dans le beau goût de menuiserie du dix-septième siècle. Six grandes fenêtres, trois de chaque côté, une au-dessus  de chaque arche, éclairaient cette bibliothèque. Par ces fenêtres, du dehors et du haut du plateau, on en voyait l’intérieur. Dans les entre-deux de ces fenêtres se dressaient sur des gaines de chêne sculpté six bustes de marbre, Hermolaüs de Byzance, Athénée, grammairien naucratique, Suidas, Casaubon, Clovis, roi de France, et son chancelier Anachalus, lequel, du reste, n’était pas plus chancelier que Clovis n’était roi.\par
Il y avait dans cette bibliothèque des livres quelconques. Un est resté célèbre. C’était un vieil in-quarto avec estampes, portant pour titre en grosses lettres S{\scshape aint}-B{\scshape arthélemy}, et pour sous-titre \emph{Évangile selon saint Barthélemy, précédé d’une dissertation de Pantœnus, philosophe chrétien, sur la question de savoir si cet évangile doit être réputé apocryphe et si saint Barthélemy est le même que Nathanaël}. Ce livre, considéré comme exemplaire unique, était sur un pupitre au milieu de la bibliothèque. Au dernier siècle, on le venait voir par curiosité.
\paragraph[{vii Le Grenier.}]{\textsc{vii} \\
Le Grenier.}\phantomsection
\label{p3l2c9p7}
\noindent Quant au grenier, qui avait, comme la bibliothèque, la forme oblongue du pont, c’était simplement le dessous de la charpente du toit. Cela faisait une grande halle encombrée de paille et de foin, et éclairée par six mansardes. Pas d’autre ornement qu’une  figure de saint Barnabé sculptée sur la porte, et au-dessous ce vers :\par

\emph{Barnabus sanctus falcem jubet ire per herbam.}\\

\noindent Ainsi une haute et large tour, à six étages, percée çà et là de quelques meurtrières, ayant pour entrée et pour issue unique une porte de fer donnant sur un pont-châtelet fermé par un pont-levis ; derrière la tour, la forêt ; devant la tour, un plateau de bruyères, plus haut que le pont, plus bas que la tour ; sous le pont, entre la tour et le plateau, un ravin profond, étroit, plein de broussailles, torrent en hiver, ruisseau au printemps, fossé pierreux l’été, voilà ce que c’était que la Tour-Gauvain, dite la Tourgue.
 \paragraph[{X. Les otages}]{X \\
Les otages}\phantomsection
\label{p3l2c10}
\noindent Juillet s’écoula, août vint, un souffle héroïque et féroce passait sur la France, deux spectres venaient de traverser l’horizon, Marat un couteau au flanc, Charlotte Corday sans tête, tout devenait formidable. Quant à la Vendée, battue dans la grande stratégie, elle se réfugiait dans la petite, plus redoutable, nous l’avons dit ; cette guerre était maintenant une immense bataille déchiquetée dans les bois ; les désastres de la grosse armée, dite catholique et royale, commençaient ; un décret envoyait en Vendée l’armée de Mayence ; huit mille Vendéens étaient morts à Ancenis ; les Vendéens étaient repoussés de Nantes, débusqués de Montaigu, expulsés de Thouars, chassés de Noirmoutier, culbutés hors de Chollet, de Mortagne et de Saumur ; ils évacuaient Parthenay, ils abandonnaient Clisson ; ils lâchaient pied à Châtillon ; ils perdaient un drapeau à Saint-Hilaire ; ils étaient battus à Pornic, aux Sables, à Fontenay, à Doué, au Château-d’Eau, aux Ponts-de-Cé ; ils étaient en échec à Luçon, en retraite à la Châtaigneraye, en déroute à la Roche-sur-Yon ; mais, d’une part, ils menaçaient la Rochelle,  et d’autre part, dans les eaux de Guernesey, une flotte anglaise, aux ordres du général Craig, portant, mêlés aux meilleurs officiers de la marine française, plusieurs régiments anglais, n’attendait qu’un signal du marquis de Lantenac pour débarquer. Ce débarquement pouvait redonner la victoire à la révolte royaliste. Pitt était d’ailleurs un malfaiteur d’état ; dans la politique il y a la trahison de même que dans la panoplie il y a le poignard. Pitt poignardait notre pays et trahissait le sien ; c’est trahir son pays que de le déshonorer ; l’Angleterre, sous lui et par lui, faisait la guerre punique. Elle espionnait, fraudait, mentait. Braconnière et faussaire, rien ne lui répugnait ; elle descendait jusqu’aux minuties de la haine. Elle faisait accaparer le suif, qui coûtait cinq francs la livre ; on saisissait à Lille, sur un Anglais, une lettre de Prigent, agent de Pitt en Vendée, où on lisait ces lignes : « Je vous prie de ne pas épargner l’argent. Nous espérons que les assassinats se feront avec prudence. Les prêtres déguisés et les femmes sont les personnes les plus propres à cette opération. Envoyez soixante mille livres à Rouen et cinquante mille livres à Caen. » Cette lettre fut lue par Barère à la Convention le 1\textsuperscript{er} août. A ces perfidies ripostaient les sauvageries de Parrein et plus tard les atrocités de Carrier. Les républicains de Metz et les républicains du Midi demandaient à marcher contre les rebelles. Un décret ordonnait la formation de vingt-quatre compagnies de pionniers pour incendier les haies et les clôtures du Bocage. Crise inouïe. La guerre ne cessait sur un point  que pour recommencer sur l’autre. Pas de grâce ! pas de prisonniers ! était le cri des deux partis. L’histoire était pleine d’une ombre terrible.\par
Dans ce mois d’août la Tourgue était assiégée.\par
Un soir, pendant le lever des étoiles, dans le calme d’un crépuscule caniculaire, pas une feuille ne remuant dans la forêt, pas une herbe ne frissonnant dans la plaine, à travers le silence de la nuit tombante, un son de trompe se fit entendre. Ce son de trompe venait du haut de la tour.\par
A ce son de trompe répondit un son de clairon qui venait d’en bas.\par
Au haut de la tour il y avait un homme armé ; en bas, dans l’ombre, il y avait un camp.\par
On distinguait confusément dans l’obscurité autour de la Tour-Gauvain un fourmillement de formes noires. Ce fourmillement était un bivouac. Quelques feux commençaient à s’y allumer sous les arbres de la forêt et parmi les bruyères du plateau, et piquaient çà et là de points lumineux les ténèbres, comme si la terre voulait s’étoiler en même temps que le ciel. Sombres étoiles que celles de la guerre ! Le bivouac du côté du plateau se prolongeait jusqu’aux plaines et du côté de la forêt s’enfonçait dans le hallier. La Tourgue était bloquée.\par
L’étendue du bivouac des assiégeants indiquait une troupe nombreuse.\par
Le camp serrait la forteresse étroitement, et venait du côté de la tour jusqu’au rocher et du côté du pont jusqu’au ravin.\par
 Il y eut un deuxième bruit de trompe que suivit un deuxième coup de clairon.\par
Cette trompe interrogeait et ce clairon répondait.\par
Cette trompe, c’était la tour qui demandait au camp : Peut-on vous parler ? et ce clairon, c’était le camp qui répondait : Oui.\par
A cette époque, les Vendéens n’étant pas considérés par la Convention comme belligérants, et défense étant faite par décret d’échanger avec les « brigands » des parlementaires, on suppléait comme on pouvait aux communications que le droit des gens autorise dans la guerre ordinaire et interdit dans la guerre civile. De là, dans l’occasion, une certaine entente entre la trompe paysanne et le clairon militaire. Le premier appel n’était qu’une entrée en matière, le second appel posait la question : Voulez-vous écouter ? Si, à ce second appel, le clairon se taisait, refus ; si le clairon répondait, consentement. Cela signifiait : Trêve de quelques instants.\par
Le clairon ayant répondu au deuxième appel, l’homme qui était au haut de la tour parla, et l’on entendit ceci :\par
« — Hommes qui m’écoutez, je suis Gouge-le-Bruant, surnommé Brise-Bleu, parce que j’ai exterminé beaucoup des vôtres, et surnommé aussi l’Imânus, parce que j’en tuerai encore plus que je n’en ai tué ; j’ai eu le doigt coupé d’un coup de sabre sur le canon de mon fusil à l’attaque de Granville, et vous avez fait guillotiner à Laval mon père et ma mère, et  ma sœur Jacqueline, âgée de dix-huit ans. Voilà ce que je suis.\par
« Je vous parle au nom de monseigneur le marquis Gauvain de Lantenac, vicomte de Fontenay, prince breton, seigneur des sept forêts, mon maître.\par
« Sachez d’abord que monseigneur le marquis, avant de s’enfermer dans cette tour où vous le tenez bloqué, a distribué la guerre entre six chefs, ses lieutenants ; il a donné à Delière le pays entre la route de Brest et la route d’Ernée ; à Treton, le pays entre la Roë et Laval ; à Jacquet, dit Taillefer, la lisière du Haut-Maine ; à Gaulier, dit Grand-Pierre, Château-Gontier ; à Lecomte, Craon ; Fougères, à monsieur Dubois-Guy ; et toute la Mayenne à monsieur de Rochambeau ; de sorte que rien n’est fini pour vous par la prise de cette forteresse, et que, lors même que monseigneur le marquis mourrait, la Vendée de Dieu et du roi ne mourra pas.\par
« Ce que j’en dis, sachez cela, est pour vous avertir. Monseigneur est là, à mes côtés. Je suis la bouche par où passent ses paroles. Hommes qui nous assiégez, faites silence.\par
« Voici ce qu’il importe que vous entendiez :\par
« N’oubliez pas que la guerre que vous nous faites n’est point juste. Nous sommes des gens qui habitons notre pays, et nous combattons honnêtement, et nous sommes simples et purs sous la volonté de Dieu comme l’herbe sous la rosée. C’est la république qui nous a attaqués ; elle est venue nous troubler dans nos campagnes, et elle a brûlé nos maisons et  nos récoltes et mitraillé nos métairies, et nos femmes et nos enfants ont été obligés de s’enfuir pieds nus dans les bois pendant que la fauvette d’hiver chantait encore.\par
« Vous qui êtes ici et qui m’entendez, vous nous avez traqués dans la forêt, et vous nous cernez dans cette tour ; vous avez tué ou dispersé ceux qui s’étaient joints à nous ; vous avez du canon ; vous avez réuni à votre colonne les garnisons et postes de Mortain, de Barenton, de Teilleul, de Landivy, d’Évran, de Tinténiac et de Vitré, ce qui fait que vous êtes quatre mille cinq cents soldats qui nous attaquez ; et nous, nous sommes dix-neuf hommes qui nous défendons.\par
« Nous avons des vivres et des munitions.\par
« Vous avez réussi à pratiquer une mine et à faire sauter un morceau de notre rocher et un morceau de notre mur.\par
« Cela a fait un trou au pied de la tour, et ce trou est une brèche par laquelle vous pouvez entrer, bien qu’elle ne soit pas à ciel ouvert et que la tour, toujours forte et debout, fasse voûte au-dessus d’elle.\par
« Maintenant vous préparez l’assaut.\par
« Et nous, d’abord monseigneur le marquis, qui est prince de Bretagne et prieur séculier de l’abbaye de Sainte-Marie de Lantenac, où une messe de tous les jours a été fondée par la reine Jeanne, ensuite les autres défenseurs de la tour, dont est monsieur l’abbé Turmeau, en guerre Grand-Francœur, mon camarade Guinoiseau, qui est capitaine du Camp-Vert, mon camarade Chante-en-Hiver, qui est capitaine du camp  de l’Avoine, mon camarade la Musette, qui est capitaine du camp des Fourmis et moi, paysan, qui suis né au bourg de Daon, où coule le ruisseau Moriandre, nous tous, nous avons une chose à vous dire.\par
« Hommes qui êtes au bas de cette tour, écoutez.\par
« Nous avons en nos mains trois prisonniers, qui sont trois enfants. Ces enfants ont été adoptés par un de vos bataillons, et ils sont à vous. Nous vous offrons de vous rendre ces trois enfants.\par
« A une condition.\par
« C’est que nous aurons la sortie libre.\par
« Si vous refusez, écoutez bien, vous ne pouvez attaquer que de deux façons, par la brèche, du côté de la forêt, ou par le pont, du côté du plateau. Le bâtiment sur le pont a trois étages ; dans l’étage d’en bas, moi l’Imânus, moi qui vous parle, j’ai fait mettre six tonnes de goudron et cent fascines de bruyères sèches ; dans l’étage d’en haut, il y a de la paille ; dans l’étage du milieu, il y a des livres et des papiers ; la porte de fer qui communique du pont avec la tour est fermée, et monseigneur en a la clef sur lui ; moi, j’ai fait sous la porte un trou, et par ce trou passe une mèche soufrée dont un bout est dans une des tonnes de goudron et l’autre bout à la portée de ma main, dans l’intérieur de la tour ; j’y mettrai le feu quand bon me semblera. Si vous refusez de nous laisser sortir, les trois enfants seront placés dans le deuxième étage du pont, entre l’étage où aboutit la mèche soufrée et où est le goudron et l’étage où est la paille, et la porte de fer sera refermée sur eux. Si vous attaquez par le pont, ce sera  vous qui incendierez le bâtiment ; si vous attaquez par la brèche, ce sera nous ; si vous attaquez à la fois par la brèche et par le pont, le feu sera mis à la fois par vous et par nous ; et, dans tous les cas, les trois enfants périront.\par
« A présent, acceptez ou refusez.\par
« Si vous acceptez, nous sortons.\par
« Si vous refusez, les enfants meurent.\par
« J’ai dit. » — \par
L’homme qui parlait du haut de la tour se tut.\par
Une voix d’en bas cria :\par
— Nous refusons.\par
Cette voix était brève et sévère. Une autre voix moins dure, ferme pourtant, ajouta :\par
— Nous vous donnons vingt-quatre heures pour vous rendre à discrétion.\par
Il y eut un silence, et la même voix continua :\par
— Demain, à pareille heure, si vous n’êtes pas rendus, nous donnons l’assaut.\par
Et la première voix reprit :\par
— Et alors pas de quartier.\par
A cette voix farouche, une autre voix répondit du haut de la tour. On vit entre deux créneaux se pencher une haute silhouette dans laquelle on put, à la lueur des étoiles, reconnaître la redoutable figure du marquis de Lantenac, et cette figure, d’où un regard tombait dans l’ombre et semblait chercher quelqu’un, cria :\par
— Tiens, c’est toi, prêtre !\par
— Oui, c’est moi, traître ! répondit la rude voix d’en bas.
 \paragraph[{XI. Affreux comme l’antique}]{XI \\
Affreux comme l’antique}\phantomsection
\label{p3l2c11}
\noindent La voix implacable en effet était la voix de Cimourdain ; la voix plus jeune et moins absolue était celle de Gauvain.\par
Le marquis de Lantenac en reconnaissant l’abbé Cimourdain ne s’était pas trompé.\par
En peu de semaines, dans ce pays que la guerre civile faisait sanglant, Cimourdain, on le sait, était devenu fameux ; pas de notoriété plus lugubre que la sienne ; on disait : Marat à Paris, Châlier à Lyon, Cimourdain en Vendée. On flétrissait l’abbé Cimourdain de tout le respect qu’on avait eu pour lui autrefois ; c’est là l’effet de l’habit de prêtre retourné. Cimourdain faisait horreur. Les sévères sont des infortunés ; qui voit leurs actes les condamne, qui verrait leur conscience les absoudrait peut-être. Un Lycurgue qui n’est pas expliqué semble un Tibère. Quoi qu’il en fût, deux hommes, le marquis de Lantenac et l’abbé Cimourdain, étaient égaux dans la balance de haine ; la malédiction des royalistes sur Cimourdain faisait contrepoids à l’exécration des républicains pour Lantenac. Chacun de ces deux hommes était, pour le  camp opposé, le monstre ; à tel point qu’il se produisit ce fait singulier que, tandis que Prieur de la Marne à Granville mettait à prix la tête de Lantenac, Charette à Noirmoutier mettait à prix la tête de Cimourdain.\par
Disons-le, ces deux hommes, le marquis et le prêtre, étaient jusqu’à un certain point le même homme. Le masque de bronze de la guerre civile a deux profils, l’un tourné vers le passé, l’autre tourné vers l’avenir, mais aussi tragiques l’un que l’autre. Lantenac était le premier de ces profils, Cimourdain était le second ; seulement l’amer rictus de Lantenac était couvert d’ombre et de nuit, et sur le front fatal de Cimourdain il y avait une lueur d’aurore.\par
Cependant la Tourgue assiégée avait un répit.\par
Grâce à l’intervention de Gauvain, on vient de le voir, une sorte de trêve de vingt-quatre heures avait été convenue.\par
L’Imânus, du reste, était bien renseigné, et, par suite des réquisitions de Cimourdain, Gauvain avait maintenant sous ses ordres quatre mille cinq cents hommes, tant garde nationale que troupe de ligne, avec lesquels il cernait Lantenac dans la Tourgue, et il avait pu braquer contre la forteresse douze pièces de canon, six du côté de la tour sur la lisière de la forêt, en batterie enterrée, et six du côté du pont, sur le plateau en batterie haute. Il avait pu faire jouer la mine, et la brèche était ouverte au pied de la tour.\par
Ainsi, sitôt les vingt-quatre heures de trêve expirées, la lutte allait s’engager dans les conditions que voici :\par
 Sur le plateau et dans la forêt, on était quatre mille cinq cents.\par
Dans la tour, dix-neuf.\par
Les noms de ces dix-neuf assiégés peuvent être retrouvés par l’histoire dans les affiches de mise hors la loi. Nous les rencontrerons peut-être.\par
Pour commander à ces quatre mille cinq cents hommes qui étaient presque une armée, Cimourdain aurait voulu que Gauvain se laissât faire adjudant-général. Gauvain avait refusé, et avait dit : — Quand Lantenac sera pris, nous verrons. Je n’ai encore rien mérité.\par
Ces grands commandements avec d’humbles grades étaient d’ailleurs dans les mœurs républicaines. Bonaparte, plus tard, fut en même temps chef d’escadron d’artillerie et général en chef de l’armée d’Italie.\par
La Tour-Gauvain avait une destinée étrange ; un Gauvain l’attaquait, un Gauvain la défendait. De là une certaine réserve dans l’attaque, mais non dans la défense, car M. de Lantenac était de ceux qui ne ménagent rien, et d’ailleurs il avait surtout habité Versailles et n’avait aucune superstition pour la Tourgue, qu’il connaissait à peine. Il était venu s’y réfugier, n’ayant plus d’autre asile, voilà tout ; mais il l’eût démolie sans scrupule. Gauvain était plus respectueux.\par
Le point faible de la forteresse était le pont ; mais dans la bibliothèque, qui était sur le pont, il y avait les archives de la famille ; si l’assaut était donné là,  l’incendie du pont était inévitable ; il semblait à Gauvain que brûler les archives, c’était attaquer ses pères. La Tourgue était le manoir de famille des Gauvain ; c’est de cette tour que mouvaient tous leurs fiefs de Bretagne, de même que tous les fiefs de France mouvaient de la tour du Louvre ; les souvenirs domestiques des Gauvain étaient là ; lui-même, il y était né ; les fatalités tortueuses de la vie l’amenaient à attaquer, homme, cette muraille vénérable qui l’avait protégé enfant. Serait-il impie envers cette demeure jusqu’à la mettre en cendres ? Peut-être son propre berceau, à lui Gauvain, était-il dans quelque coin du grenier de la bibliothèque. Certaines réflexions sont des émotions. Gauvain, en présence de l’antique maison de famille, se sentait ému. C’est pourquoi il avait épargné le pont. Il s’était borné à rendre toute sortie ou toute évasion impossible par cette issue et à tenir le pont en respect par une batterie, et il avait choisi pour l’attaque le côté opposé. De là, la mine et la sape au pied de la tour.\par
Cimourdain l’avait laissé faire ; il se le reprochait, car son âpreté fronçait le sourcil devant toutes ces vieilleries gothiques, et il ne voulait pas plus l’indulgence pour les édifices que pour les hommes. Ménager un château, c’était un commencement de clémence. Or la clémence était le côté faible de Gauvain ; Cimourdain, on le sait, le surveillait et l’arrêtait sur cette pente, à ses yeux funeste. Pourtant lui-même, et en ne se l’avouant qu’avec une sorte de colère, il n’avait pas revu la Tourgue sans un secret tressaillement ;  il se sentait attendri devant cette salle studieuse où étaient les premiers livres qu’il eût fait lire à Gauvain ; il avait été curé du village voisin, Parigné ; il avait, lui, Cimourdain, habité les combles du châtelet du pont ; c’est dans la bibliothèque qu’il tenait entre ses genoux le petit Gauvain épelant l’alphabet ; c’est entre ces vieux quatre murs-là qu’il avait vu son élève bien-aimé, le fils de son âme, grandir comme homme et croître comme esprit. Cette bibliothèque, ce châtelet, ces murs pleins de ses bénédictions sur l’enfant, allait-il les foudroyer et les brûler ? Il leur faisait grâce. Non sans remords.\par
Il avait laissé Gauvain entamer le siège sur le point opposé. La Tourgue avait son côté sauvage, la tour, et son côté civilisé, la bibliothèque. Cimourdain avait permis à Gauvain de ne battre en brèche que le côté sauvage.\par
Du reste, attaquée par un Gauvain, défendue par un Gauvain, cette vieille demeure revenait, en pleine révolution française, à ses habitudes féodales. Les guerres entre parents sont toute l’histoire du moyen âge ; les Étéocles et les Polynices sont gothiques aussi bien que grecs, et Hamlet fait dans Elseneur ce qu’Oreste a fait dans Argos.
 \paragraph[{XII. Le sauvetage s’ébauche}]{XII \\
Le sauvetage s’ébauche}\phantomsection
\label{p3l2c12}
\noindent Toute la nuit se passa de part et d’autre en préparatifs.\par
Sitôt le sombre pourparler qu’on vient d’entendre terminé, le premier soin de Gauvain fut d’appeler son lieutenant.\par
Guéchamp, qu’il faut un peu connaître, était un homme de second plan, honnête, intrépide, médiocre, meilleur soldat que chef, rigoureusement intelligent jusqu’au point où c’est le devoir de ne plus comprendre, jamais attendri, inaccessible à la corruption, quelle qu’elle fût, aussi bien à la vénalité qui corrompt la conscience qu’à la pitié qui corrompt la justice. Il avait sur l’âme et sur le cœur ces deux abat-jour, la discipline et la consigne, comme un cheval a ses garde-vue sur les deux yeux, et il marchait devant lui dans l’espace que cela lui laissait libre. Son pas était droit, mais sa route était étroite.\par
Du reste, homme sûr ; rigide dans le commandement, exact dans l’obéissance.\par
Gauvain adressa vivement la parole à Guéchamp.\par
— Guéchamp, une échelle.\par
— Mon commandant, nous n’en avons pas.\par
 — Il faut en avoir une.\par
— Pour escalade ?\par
— Non. Pour sauvetage.\par
Guéchamp réfléchit et répondit :\par
— Je comprends. Mais pour ce que vous voulez, il la faut très haute.\par
— D’au moins trois étages.\par
— Oui, mon commandant, c’est à peu près la hauteur.\par
— Et il faut dépasser cette hauteur, car il faut être sûr de réussir.\par
— Sans doute.\par
— Comment se fait-il que vous n’ayez pas d’échelle ?\par
— Mon commandant, vous n’avez pas jugé à propos d’assiéger la Tourgue par le plateau, vous vous êtes contenté de la bloquer de ce côté-là ; vous avez voulu attaquer, non par le pont, mais par la tour. On ne s’est plus occupé que de la mine, et l’on a renoncé à l’escalade. C’est pourquoi nous n’avons pas d’échelles.\par
— Faites-en faire une sur-le-champ.\par
— Une échelle de trois étages ne s’improvise pas.\par
— Faites ajouter bout à bout plusieurs échelles courtes.\par
— Il faut en avoir.\par
— Trouvez-en.\par
— On n’en trouvera pas. Partout les paysans détruisent les échelles, de même qu’ils démontent les charrettes et qu’ils coupent les ponts.\par
— Ils veulent paralyser la république, c’est vrai.\par
 — Ils veulent que nous ne puissions ni traîner un charroi, ni passer une rivière, ni escalader un mur.\par
— Il me faut une échelle, pourtant.\par
— J’y songe, mon commandant, il y a à Javené, près de Fougères, une grande charpenterie. On peut en avoir une là.\par
— Il n’y a pas une minute à perdre.\par
— Quand voulez-vous avoir l’échelle ?\par
— Demain, à pareille heure, au plus tard.\par
— Je vais envoyer à Javené un exprès à franc-étrier. Il portera l’ordre de réquisition. Il y a à Javené un poste de cavalerie qui fournira l’escorte. L’échelle pourra être ici demain avant le coucher du soleil.\par
— C’est bien, cela suffira, dit Gauvain, faites vite. Allez.\par
Dix minutes après, Guéchamp revint et dit à Gauvain :\par
— Mon commandant, l’exprès est parti pour Javené.\par
Gauvain monta sur le plateau et demeura longtemps l’œil fixé sur le pont-châtelet qui était en travers du ravin. Le pignon du châtelet, sans autre baie que la basse entrée fermée par le pont-levis dressé, faisait face à l’escarpement du ravin. Pour arriver du plateau au pied des piles du pont, il fallait descendre le long de cet escarpement, ce qui n’était pas impossible, de broussaille en broussaille. Mais une fois dans le fossé, l’assaillant serait exposé à tous les projectiles pouvant pleuvoir des trois étages. Gauvain acheva de se convaincre qu’au point où le siège en était, la véritable attaque était par la brèche de la tour.\par
 Il prit toutes ses mesures pour qu’aucune fuite ne fût possible ; il compléta l’étroit blocus de la Tourgue ; il resserra les mailles de ses bataillons de façon que rien ne pût passer au travers. Gauvain et Cimourdain se partagèrent l’investissement de la forteresse ; Gauvain se réserva le côté de la forêt et donna à Cimourdain le côté du plateau. Il fut convenu que, tandis que Gauvain, secondé par Guéchamp, conduirait l’assaut par la sape, Cimourdain, toutes les mèches de la batterie haute allumées, observerait le pont et le ravin.
 \paragraph[{XIII. Ce que fait le marquis}]{XIII \\
Ce que fait le marquis}\phantomsection
\label{p3l2c13}
\noindent Pendant qu’au dehors tout s’apprêtait pour l’attaque, au dedans tout s’apprêtait pour la résistance.\par
Ce n’est pas sans une réelle analogie qu’une tour se nomme une douve, et l’on frappe quelquefois une tour d’un coup de mine comme une douve d’un coup de poinçon. La muraille se perce comme une bonde. C’est ce qui était arrivé à la Tourgue.\par
Le puissant coup de poinçon donné par deux ou trois quintaux de poudre avait troué de part en part le mur énorme. Ce trou partait du pied de la tour, traversait la muraille dans sa plus grande épaisseur et venait aboutir en arcade informe dans le rez-de-chaussée de la forteresse. Du dehors, les assiégeants, afin de rendre ce trou praticable à l’assaut, l’avaient élargi et façonné à coups de canon.\par
Le rez-de-chaussée où pénétrait cette brèche était une grande salle ronde toute nue, avec pilier central portant la clef de voûte. Cette salle, qui était la plus vaste de tout le donjon, n’avait pas moins de quarante pieds de diamètre. Chacun des étages de la tour se composait d’une chambre pareille, mais moins large, avec des logettes dans les embrasures des meurtrières. La salle du rez-de-chaussée n’avait pas de meurtrières,  pas de soupiraux, pas de lucarnes ; juste autant de jour et d’air qu’une tombe.\par
La porte des oubliettes, faite de plus de fer que de bois, était dans la salle du rez-de-chaussée. Une autre porte de cette salle ouvrait sur un escalier qui conduisait aux chambres supérieures. Tous les escaliers étaient pratiqués dans l’épaisseur du mur.\par
C’est dans cette salle basse que les assiégeants avaient chance d’arriver par la brèche qu’ils avaient faite. Cette salle prise, il leur restait la tour à prendre.\par
On n’avait jamais respiré dans cette salle basse. Nul n’y passait vingt-quatre heures sans être asphyxié. Maintenant, grâce à la brèche, on y pouvait vivre.\par
C’est pourquoi les assiégés ne fermèrent pas la brèche.\par
D’ailleurs, à quoi bon ? Le canon l’eût rouverte.\par
Ils piquèrent dans le mur une torchère de fer, y plantèrent une torche, et cela éclaira le rez-de-chaussée.\par
Maintenant, comment s’y défendre ?\par
Murer le trou était facile, mais inutile. Une retirade valait mieux. Une retirade, c’est un retranchement à angle rentrant, sorte de barricade chevronnée qui permet de faire converger les feux sur les assaillants, et qui, en laissant à l’extérieur la brèche ouverte, la bouche à l’intérieur. Les matériaux ne leur manquaient pas ; ils construisirent une retirade, avec fissures pour le passage des canons de fusil. L’angle de la retirade s’appuyait au pilier central ; les deux ailes touchaient le mur des deux côtés. Cela fait, on disposa dans les bons endroits des fougasses.\par
 Le marquis dirigeait tout. Inspirateur, ordonnateur, guide et maître, âme terrible.\par
Lantenac était de cette race d’hommes de guerre du dix-huitième siècle qui, à quatrevingts ans, sauvaient des villes. Il ressemblait à ce comte d’Alberg qui, presque centenaire, chassa de Riga le roi de Pologne.\par
— Courage, amis ! disait le marquis ; au commencement de ce siècle, en 1713, à Bender, Charles XII, enfermé dans une maison, a tenu tête, avec trois cents Suédois, à vingt mille Turcs.\par
On barricada les deux étages d’en bas, on fortifia les chambres, on crénela les alcôves, on contrebuta les portes avec des solives enfoncées à coups de maillet, qui faisaient comme des arcs-boutants ; seulement on dut laisser libre l’escalier en spirale qui communiquait à tous les étages, car il fallait pouvoir y circuler, et l’entraver pour l’assiégeant, c’eût été l’entraver pour l’assiégé. La défense des places a toujours ainsi un côté faible.\par
Le marquis, infatigable, robuste comme un jeune homme, soulevant des poutres, portant des pierres, donnait l’exemple, mettait la main à la besogne, commandait, aidait, fraternisait, riait avec ce clan féroce, toujours le seigneur pourtant, haut, familier, élégant, farouche.\par
Il ne fallait pas lui répliquer. Il disait : \emph{Si une moitié de vous se révoltait, je la ferais fusiller par l’autre, et je défendrais la place avec le reste}. Ces choses-là font qu’on adore un chef.
 \paragraph[{XIV. Ce que fait l’Imânus}]{XIV \\
Ce que fait l’Imânus}\phantomsection
\label{p3l2c14}
\noindent Pendant que le marquis s’occupait de la brèche et de la tour, l’Imânus s’occupait du pont. Dès le commencement du siège, l’échelle de sauvetage, suspendue transversalement en dehors et au-dessous des fenêtres du deuxième étage, avait été retirée par ordre du marquis, et placée par l’Imânus dans la salle de la bibliothèque. C’est peut-être à cette échelle-là que Gauvain voulait suppléer. Les fenêtres du premier étage entre-sol, dit salle des gardes, étaient défendues par une triple armature de barreaux de fer scellés dans la pierre, et l’on ne pouvait ni entrer ni sortir par là.\par
Il n’y avait point de barreaux aux fenêtres de la bibliothèque, mais elles étaient très hautes.\par
L’Imânus se fit accompagner de trois hommes, comme lui capables de tout et résolus à tout. Ces hommes étaient Hoisnard, dit Branche-d’Or, et les deux frères Pique-en-Bois. L’Imânus prit une lanterne sourde, ouvrit la porte de fer, et visita minutieusement les trois étages du châtelet du pont. Hoisnard Branche-d’Or était aussi implacable que l’Imânus, ayant eu un frère tué par les républicains.\par
 L’Imânus examina l’étage d’en haut, regorgeant de foin et de paille, et l’étage d’en bas, dans lequel il fit apporter quelques pots à feu, qu’il ajouta aux tonnes de goudron ; il fit mettre le tas de fascines de bruyères en contact avec les tonnes de goudron, et il s’assura du bon état de la mèche soufrée dont une extrémité était dans le pont et l’autre dans la tour. Il répandit sur le plancher, sous les tonnes et sous les fascines, une mare de goudron où il immergea le bout de la mèche soufrée ; puis il fit placer dans la salle de la bibliothèque, entre le rez-de-chaussée où était le goudron et le grenier où était la paille, les trois berceaux où étaient René-Jean, Gros-Alain et Georgette, plongés dans un profond sommeil. On apporta les berceaux très doucement pour ne point réveiller les petits.\par
C’étaient de simples petites crèches de campagne, sortes de corbeilles d’osier très basses qu’on pose à terre, ce qui permet à l’enfant de sortir du berceau seul et sans aide. Près de chaque berceau, l’Imânus fit placer une écuelle de soupe avec une cuiller de bois. L’échelle de sauvetage décrochée de ses crampons avait été déposée sur le plancher, contre le mur ; l’Imânus fit ranger les trois berceaux bout à bout le long de l’autre mur en regard de l’échelle. Puis, pensant que les courants d’air pouvaient être utiles, il ouvrit toutes grandes les six fenêtres de la bibliothèque. C’était une nuit d’été, bleue et tiède.\par
Il envoya les frères Pique-en-Bois ouvrir les fenêtres de l’étage inférieur et de l’étage supérieur. Il avait remarqué, sur la façade orientale de l’édifice, un grand  vieux lierre desséché, couleur d’amadou, qui couvrait tout un côté du pont du haut en bas et encadrait les fenêtres des trois étages. Il pensa que ce lierre ne nuirait pas. L’Imânus jeta partout un dernier coup d’œil ; après quoi, ces quatre hommes sortirent du châtelet et rentrèrent dans le donjon. L’Imânus referma la lourde porte de fer à double tour, considéra attentivement la serrure énorme et terrible, et examina, avec un signe de tête satisfait, la mèche soufrée qui passait par le trou pratiqué par lui, et était désormais la seule communication entre la tour et le pont. Cette mèche partait de la chambre ronde, passait sous la porte de fer, entrait sous la voussure, descendait l’escalier du rez-de-chaussée du pont, serpentait sur les degrés en spirale, rampait sur le plancher du couloir entre-sol, et allait aboutir à la mare de goudron sur le tas de fascines sèches. L’Imânus avait calculé qu’il fallait environ un quart d’heure pour que cette mèche, allumée dans l’intérieur de la tour, mît le feu à la mare de goudron sous la bibliothèque. Tous ces arrangements pris, et toutes ces inspections faites, il rapporta la clef de la porte de fer au marquis de Lantenac, qui la mit dans sa poche.\par
Il importait de surveiller tous les mouvements des assiégeants. L’Imânus alla se poster en vedette, sa trompe de bouvier à la ceinture, dans la guérite de la plate-forme, au haut de la tour. Tout en observant, un œil sur la forêt, un œil sur le plateau, il avait près de lui, dans l’embrasure de la lucarne de la guérite, une poire à poudre, un sac de toile plein de balles de calibre,  et de vieux journaux qu’il déchirait, et il faisait des cartouches.\par
Quand le soleil parut, il éclaira dans la forêt huit bataillons, le sabre au côté, la giberne au dos, la bayonnette au fusil, prêts à l’assaut ; sur le plateau, une batterie de canons, avec caissons, gargousses et boîtes à mitraille ; dans la forteresse, dix-neuf hommes chargeant des tromblons, des mousquets, des pistolets et des espingoles ; et dans les trois berceaux trois enfants endormis.
 \subsection[{Livre troisième. Le massacre de Saint-Barthelemy}]{Livre troisième \\
Le massacre de Saint-Barthelemy}\phantomsection
\label{p3l3}
\subsubsection[{I}]{I}\phantomsection
\label{p3l3c1}
\noindent Les enfants se réveillèrent.\par
Ce fut d’abord la petite.\par
Un réveil d’enfants, c’est une ouverture de fleurs ; il semble qu’un parfum sorte de ces fraîches âmes.\par
Georgette, celle de vingt mois, la dernière née des trois, qui tétait encore en mai, souleva sa petite tête, se dressa sur son séant, regarda ses pieds, et se mit à jaser.\par
Un rayon du matin était sur son berceau ; il eût été difficile de dire quel était le plus rose, du pied de Georgette ou de l’aurore.\par
Les deux autres dormaient encore ; c’est plus lourd, les hommes. Georgette, gaie et calme, jasait.\par
 René-Jean était brun, Gros-Alain était châtain, Georgette était blonde. Ces nuances des cheveux, d’accord dans l’enfance avec l’âge, peuvent changer plus tard. René-Jean avait l’air d’un petit Hercule ; il dormait sur le ventre, avec ses deux poings dans ses yeux. Gros-Alain avait les deux jambes hors de son petit lit.\par
Tous trois étaient en haillons ; les vêtements que leur avait donnés le bataillon du Bonnet-Rouge s’en étaient allés en loques ; ce qu’ils avaient sur eux n’était même pas une chemise ; les deux garçons étaient presque nus, Georgette était affublée d’une guenille qui avait été une jupe et qui n’était plus guère qu’une brassière. Qui avait soin de ces enfants ? on n’eût pu le dire. Pas de mère. Ces sauvages paysans combattants, qui les traînaient avec eux de forêt en forêt, leur donnaient leur part de soupe. Voilà tout. Les petits s’en tiraient comme ils pouvaient. Ils avaient tout le monde pour maître et personne pour père. Mais les haillons des enfants, c’est plein de lumière. Ils étaient charmants.\par
Georgette jasait.\par
Ce qu’un oiseau chante, un enfant le jase. C’est le même hymne. Hymne indistinct, balbutié, profond. L’enfant a de plus que l’oiseau la sombre destinée humaine devant lui. De là la tristesse des hommes qui écoutent, mêlée à la joie du petit qui chante. Le cantique le plus sublime qu’on puisse entendre sur la terre, c’est le bégaiement de l’âme humaine sur les lèvres de l’enfance. Ce chuchotement confus d’une  pensée qui n’est encore qu’un instinct contient on ne sait quel appel inconscient à la justice éternelle ; peut-être est-ce une protestation sur le seuil avant d’entrer ; protestation humble et poignante ; cette ignorance souriant à l’infini compromet toute la création dans le sort qui sera fait à l’être faible et désarmé. Le malheur, s’il arrive, sera un abus de confiance.\par
Le murmure de l’enfant, c’est plus et moins que la parole ; ce ne sont pas des notes, et c’est un chant ; ce ne sont pas des syllabes, et c’est un langage ; ce murmure a eu son commencement dans le ciel et n’aura pas sa fin sur la terre ; il est d’avant la naissance, et il continue ; c’est une suite. Ce bégaiement se compose de ce que l’enfant disait quand il était ange et de ce qu’il dira quand il sera homme ; le berceau a un Hier de même que la tombe a un Demain ; ce demain et cet hier amalgament dans ce gazouillement obscur leur double inconnu ; et rien ne prouve Dieu, l’éternité, la responsabilité, la dualité du destin, comme cette ombre formidable dans cette âme rose.\par
Ce que balbutiait Georgette ne l’attristait pas, car tout son doux visage était un sourire. Sa bouche souriait, ses yeux souriaient, les fossettes de ses joues souriaient. Il se dégageait de ce sourire une mystérieuse acceptation du matin. L’âme a foi dans le rayon. Le ciel était bleu, il faisait chaud, il faisait beau. La frêle créature, sans rien savoir, sans rien connaître, sans rien comprendre, mollement noyée dans la rêverie qui ne pense pas, se sentait en sûreté dans cette  nature, dans ces arbres honnêtes, dans cette verdure sincère, dans cette campagne pure et paisible, dans ces bruits de nids, de sources, de mouches, de feuilles, au-dessus desquels resplendissait l’immense innocence du soleil.\par
Après Georgette, René-Jean, l’aîné, le grand, qui avait quatre ans passés, se réveilla. Il se leva debout, enjamba virilement son berceau, aperçut son écuelle, trouva cela tout simple, s’assit par terre et commença à manger sa soupe.\par
La jaserie de Georgette n’avait pas éveillé Gros-Alain, mais au bruit de la cuiller dans l’écuelle il se retourna en sursaut, et ouvrit les yeux. Gros-Alain était celui de trois ans. Il vit son écuelle, il n’avait que le bras à étendre, il la prit, et, sans sortir de son lit, son écuelle sur ses genoux, sa cuiller au poing, il fit comme René-Jean, il se mit à manger.\par
Georgette ne les entendait pas, et les ondulations de sa voix semblaient moduler le bercement d’un rêve. Ses yeux grands ouverts regardaient en haut, et étaient divins ; quel que soit le plafond ou la voûte qu’un enfant a au-dessus de sa tête, ce qui se reflète dans ses yeux, c’est le ciel.\par
Quand René-Jean eut fini, il gratta avec la cuiller le fond de l’écuelle, soupira, et dit avec dignité : — J’ai mangé ma soupe.\par
Ceci tira Georgette de sa rêverie.\par
— Poupoupe, dit-elle.\par
Et voyant que René-Jean avait mangé et que Gros-Alain mangeait, elle prit l’écuelle de soupe qui était à  côté d’elle, et mangea, non sans porter sa cuiller beaucoup plus souvent à son oreille qu’à sa bouche.\par
De temps en temps elle renonçait à la civilisation et mangeait avec ses doigts.\par
Gros-Alain, après avoir, comme son frère, gratté le fond de l’écuelle, était allé le rejoindre et courait derrière lui.
\subsubsection[{II}]{II}\phantomsection
\label{p3l3c2}
\noindent Tout à coup on entendit au dehors, en bas, du côté de la forêt, un bruit de clairon, sorte de fanfare hautaine et sévère. A ce bruit de clairon répondit du haut de la tour un son de trompe.\par
Cette fois, c’était le clairon qui appelait et la trompe qui donnait la réplique.\par
Il y eut un deuxième coup de clairon que suivit un deuxième son de trompe.\par
Puis, de la lisière de la forêt, s’éleva une voix lointaine, mais précise, qui cria distinctement ceci :\par
— Brigands ! sommation. Si vous n’êtes pas rendus à discrétion au coucher du soleil, nous attaquons.\par
Une voix, qui ressemblait à un grondement, répondit, de la plate-forme de la tour :\par
— Attaquez.\par
La voix d’en bas reprit :\par
 — Un coup de canon sera tiré, comme dernier avertissement, une demi-heure avant l’assaut.\par
Et la voix d’en haut répéta :\par
— Attaquez.\par
Ces voix n’arrivaient pas jusqu’aux enfants, mais le clairon et la trompe portaient plus haut et plus loin, et Georgette, au premier coup de clairon, dressa le cou et cessa de manger ; au son de trompe, elle posa sa cuiller dans son écuelle ; au deuxième coup de clairon, elle leva le petit index de sa main droite, et, l’abaissant et le relevant tour à tour, marqua les cadences de la fanfare, que vint prolonger le deuxième son de trompe ; quand la trompe et le clairon se turent, elle demeura pensive le doigt en l’air et murmura à demi-voix : — Misique.\par
Nous pensons qu’elle voulait dire « musique ».\par
Les deux aînés, René-Jean et Gros-Alain, n’avaient pas fait attention à la trompe et au clairon ; ils étaient absorbés par autre chose ; un cloporte était en train de traverser la bibliothèque.\par
Gros-Alain l’aperçut et cria :\par
— Une bête.\par
René-Jean accourut.\par
Gros-Alain reprit :\par
— Ça pique.\par
— Ne lui fais pas de mal, dit René-Jean.\par
Et tous deux se mirent à regarder ce passant.\par
Cependant Georgette avait fini sa soupe ; elle chercha des yeux ses frères. René-Jean et Gros-Alain étaient dans l’embrasure d’une fenêtre, accroupis et  graves au-dessus du cloporte ; ils se touchaient du front et mêlaient leurs cheveux ; ils retenaient leur respiration, émerveillés, et considéraient la bête, qui s’était arrêtée et ne bougeait plus, peu contente de tant d’admiration.\par
Georgette, voyant ses frères en contemplation, voulut savoir ce que c’était. Il n’était pas aisé d’arriver jusqu’à eux, elle l’entreprit pourtant ; le trajet était hérissé de difficultés ; il y avait des choses par terre, des tabourets renversés, des tas de paperasses, des caisses d’emballage déclouées et vides, des bahuts, des monceaux quelconques autour desquels il fallait cheminer, tout un archipel d’écueils ; Georgette s’y hasarda. Elle commença par sortir de son berceau, premier travail ; puis elle s’engagea dans les récifs, serpenta dans les détroits, poussa un tabouret, rampa entre deux coffres, passa par-dessus une liasse de papiers, grimpant d’un côté, roulant de l’autre, montrant avec douceur sa pauvre petite nudité, et parvint ainsi à ce qu’un marin appellerait la mer libre, c’est-à-dire à un assez large espace de plancher qui n’était plus obstrué et où il n’y avait plus de périls ; alors elle s’élança, traversa cet espace qui était tout le diamètre de la salle, à quatre pattes, avec une vitesse de chat, et arriva près de la fenêtre ; là il y avait un obstacle redoutable ; la grande échelle gisante le long du mur venait aboutir à cette fenêtre, et l’extrémité de l’échelle dépassait un peu le coin de l’embrasure ; cela faisait entre Georgette et ses frères une sorte de cap à franchir ; elle s’arrêta et médita ; son monologue intérieur  terminé, elle prit son parti ; elle empoigna résolument de ses doigts roses un des échelons, lesquels étaient verticaux et non horizontaux, l’échelle étant couchée sur un de ses montants ; elle essaya de se lever sur ses pieds, et retomba ; elle recommença ; deux fois, elle échoua ; à la troisième fois elle réussit ; alors, droite et debout, s’appuyant successivement à chacun des échelons, elle se mit à marcher le long de l’échelle ; arrivée à l’extrémité, le point d’appui lui manquait, elle trébucha, mais saisissant de ses petites mains le bout du montant qui était énorme, elle se redressa, doubla le promontoire, regarda René-Jean et Gros-Alain, et rit.
\subsubsection[{III}]{III}\phantomsection
\label{p3l3c3}
\noindent En ce moment-là, René-Jean, satisfait du résultat de ses observations sur le cloporte, relevait la tête et disait :\par
— C’est une femelle.\par
Le rire de Georgette fit rire René-Jean, et le rire de René-Jean fit rire Gros-Alain.\par
Georgette opéra sa jonction avec ses frères, et cela fit un petit cénacle assis par terre.\par
Mais le cloporte avait disparu.\par
Il avait profité du rire de Georgette pour se fourrer dans un trou du plancher.\par
 D’autres événements suivirent le cloporte.\par
D’abord des hirondelles passèrent.\par
Leurs nids étaient probablement sous le rebord du toit. Elles vinrent voler tout près de la fenêtre, un peu inquiètes des enfants, décrivant de grands cercles dans l’air, et poussant leur doux cri du printemps. Cela fit lever les yeux aux trois enfants, et le cloporte fut oublié.\par
Georgette braqua son doigt sur les hirondelles et cria : — Cocos !\par
René-Jean la réprimanda.\par
— Mamoiselle, on ne dit pas des cocos, on dit des oseaux.\par
— Zozo, dit Georgette.\par
Et tous les trois regardèrent les hirondelles.\par
Puis une abeille entra.\par
Rien ne ressemble à une âme comme une abeille. Elle va de fleur en fleur comme une âme d’étoile en étoile, et elle rapporte le miel comme l’âme rapporte la lumière.\par
Celle-ci fit grand bruit en entrant, elle bourdonnait à voix haute, et elle avait l’air de dire : J’arrive, je viens de voir les roses, maintenant je viens voir les enfants. Qu’est-ce qui se passe ici ?\par
Une abeille, c’est une ménagère, et cela gronde en chantant.\par
Tant que l’abeille fut là, les trois petits ne la quittèrent pas des yeux.\par
L’abeille explora toute la bibliothèque, fureta les recoins, voleta ayant l’air d’être chez elle et dans une  ruche, et rôda, ailée et mélodieuse, d’armoire en armoire, regardant à travers les vitres les titres des livres, comme si elle eût été un esprit.\par
Sa visite faite, elle partit.\par
— Elle va dans sa maison, dit René-Jean.\par
— C’est une bête, dit Gros-Alain.\par
— Non, repartit René-Jean, c’est une mouche.\par
— Muche, dit Georgette.\par
Là-dessus, Gros-Alain, qui venait de trouver à terre une ficelle à l’extrémité de laquelle il y avait un nœud, prit entre son pouce et son index le bout opposé au nœud, fit de la ficelle une sorte de moulinet, et la regarda tourner avec une attention profonde.\par
De son côté, Georgette, redevenue quadrupède et ayant repris son va-et-vient capricieux sur le plancher, avait découvert un vénérable fauteuil de tapisserie mangé des vers dont le crin sortait par plusieurs trous. Elle s’était arrêtée à ce fauteuil. Elle élargissait les trous et tirait le crin avec recueillement.\par
Brusquement, elle leva un doigt, ce qui voulait dire : Écoutez.\par
Les deux frères tournèrent la tête.\par
Un fracas vague et lointain s’entendait au dehors ; c’était probablement le camp d’attaque qui exécutait quelque mouvement stratégique dans la forêt ; des chevaux hennissaient, des tambours battaient, des caissons roulaient, des chaînes s’entre-heurtaient, des sonneries militaires s’appelaient et se répondaient, confusion de bruits farouches qui en se mêlant devenaient  une sorte d’harmonie ; les enfants écoutaient, charmés.\par
— C’est le mondieu qui fait ça, dit René-Jean.
\subsubsection[{IV}]{IV}\phantomsection
\label{p3l3c4}
\noindent Le bruit cessa.\par
René-Jean était demeuré rêveur.\par
Comment les idées se décomposent-elles et se recomposent-elles dans ces petits cerveaux-là ? Quel est le remuement mystérieux de ces mémoires si troubles et si courtes encore ? Il se fit dans cette douce tête pensive un mélange du mondieu, de la prière, des mains jointes, d’on ne sait quel tendre sourire qu’on avait sur soi autrefois, et qu’on n’avait plus, et René-Jean chuchota à demi-voix : Maman.\par
— Maman, dit Gros-Alain.\par
— Mman, dit Georgette.\par
Et puis René-Jean se mit à sauter.\par
Ce que voyant, Gros-Alain sauta.\par
Gros-Alain reproduisait tous les mouvements et tous les gestes de René-Jean ; Georgette moins. Trois ans, cela copie quatre ans ; mais vingt mois, cela garde son indépendance.\par
Georgette resta assise, disant de temps en temps un mot. Georgette ne faisait pas de phrases. C’était une penseuse ; elle parlait par apophthegmes. Elle était monosyllabique.\par
 Au bout de quelque temps néanmoins, l’exemple la gagna, et elle finit par tâcher de faire comme ses frères, et ces trois petites paires de pieds nus se mirent à danser, à courir et à chanceler, dans la poussière du vieux parquet de chêne poli, sous le grave regard des bustes de marbre, auxquels Georgette jetait de temps en temps de côté un œil inquiet, en murmurant : — Les momommes !\par
Dans le langage de Georgette, un « momomme », c’était tout ce qui ressemblait à un homme et pourtant n’en était pas un. Les êtres n’apparaissent à l’enfant que mêlés aux fantômes.\par
Georgette, marchant moins qu’elle n’oscillait, suivait ses frères, mais plus volontiers à quatre pattes.\par
Subitement, René-Jean, s’étant approché d’une croisée, leva la tête, puis la baissa, et alla se réfugier derrière le coin du mur de l’embrasure de la fenêtre. Il venait d’apercevoir quelqu’un qui le regardait. C’était un soldat bleu du campement du plateau qui, profitant de la trêve et l’enfreignant peut-être un peu, s’était hasardé jusqu’à venir au bord de l’escarpement du ravin d’où l’on découvrait l’intérieur de la bibliothèque. Voyant René-Jean se réfugier, Gros-Alain se réfugia, il se blottit à côté de René-Jean, et Georgette vint se cacher derrière eux. Ils demeurèrent là en silence, immobiles, et Georgette mit son doigt sur ses lèvres. Au bout de quelques instants, René-Jean se risqua à avancer la tête ; le soldat y était encore. René-Jean rentra sa tête vivement ; et les trois petits n’osèrent plus  souffler. Cela dura assez longtemps. Enfin cette peur ennuya Georgette, elle eut de l’audace, elle regarda. Le soldat s’en était allé. Ils se remirent à courir et à jouer.\par
Gros-Alain, bien qu’imitateur et admirateur de René-Jean, avait une spécialité, les trouvailles. Son frère et sa sœur le virent tout à coup caracoler éperdument en tirant après lui un petit chariot à quatre roues qu’il avait déterré je ne sais où.\par
Cette voiture à poupée était là depuis des années dans la poussière, oubliée, faisant bon voisinage avec les livres des génies et les bustes des sages. C’était peut-être un des hochets avec lesquels avait joué Gauvain enfant.\par
Gros-Alain avait fait de sa ficelle un fouet qu’il faisait claquer ; il était très fier. Tels sont les inventeurs. Quand on ne découvre pas l’Amérique, on découvre une petite charrette. C’est toujours cela.\par
Mais il fallut partager. René-Jean voulut s’atteler à la voiture, et Georgette voulut monter dedans.\par
Elle essaya de s’y asseoir. René-Jean fut le cheval. Gros-Alain fut le cocher.\par
Mais le cocher ne savait pas son métier, le cheval le lui apprit.\par
René-Jean cria à Gros-Alain :\par
— Dis : Hu !\par
— Hu ! répéta Gros-Alain.\par
La voiture versa. Georgette roula. Cela crie, les anges. Georgette cria.\par
Puis elle eut une vague envie de pleurer.\par
 — Mamoiselle, dit René-Jean, vous êtes trop grande.\par
— J’ai grande, fit Georgette.\par
Et sa grandeur la consola de sa chute.\par
La corniche d’entablement au-dessous des fenêtres était fort large ; la poussière des champs envolée du plateau de bruyère avait fini par s’y amasser, les pluies avaient refait de la terre avec cette poussière, le vent y avait apporté des graines, si bien qu’une ronce avait profité de ce peu de terre pour pousser là. Cette ronce était de l’espèce vivace dite \emph{mûrier de renard.} On était en août, la ronce était couverte de mûres, et une branche de la ronce entrait par une fenêtre. Cette branche pendait presque jusqu’à terre.\par
Gros-Alain, après avoir découvert la ficelle, après avoir découvert la charrette, découvrit cette ronce. Il s’en approcha.\par
Il cueillit une mûre et la mangea.\par
— J’ai faim, dit René-Jean.\par
Et Georgette, galopant sur ses genoux et sur ses mains, arriva.\par
A eux trois ils pillèrent la branche et mangèrent toutes les mûres. Ils s’en grisèrent et s’en barbouillèrent, et, tout vermeils de cette pourpre de la ronce, ces trois petits séraphins finirent par être trois petits faunes, ce qui eût choqué Dante et charmé Virgile. Ils riaient aux éclats.\par
De temps en temps la ronce leur piquait les doigts. Rien pour rien.\par
Georgette tendit à René-Jean son doigt où perlait  une petite goutte de sang, et dit en montrant la ronce : Pique.\par
Gros-Alain, piqué aussi, regarda la ronce avec défiance, et dit :\par
— C’est une bête.\par
— Non, répondit René-Jean, c’est un bâton.\par
— Un bâton, c’est méchant, reprit Gros-Alain.\par
Georgette, cette fois encore, eut envie de pleurer, mais elle se mit à rire.
\subsubsection[{V}]{V}\phantomsection
\label{p3l3c5}
\noindent Cependant René-Jean, jaloux peut-être des découvertes de son frère cadet Gros-Alain, avait conçu un grand projet. Depuis quelque temps, tout en cueillant des mûres et en se piquant les doigts, ses yeux se tournaient fréquemment du côté du lutrin-pupitre monté sur pivot et isolé comme un monument au milieu de la bibliothèque. C’est sur ce lutrin que s’étalait le célèbre volume \emph{Saint-Barthélemy.}\par
C’était vraiment un in-quarto magnifique et mémorable. Ce \emph{Saint-Barthélemy} avait été publié à Cologne par le fameux éditeur de la Bible de 1682, Blœuw, en latin Cœsius. Il avait été fabriqué par des presses à boîtes et à nerfs de bœuf ; il était imprimé, non sur papier de Hollande, mais sur ce beau papier arabe, si admiré par Édrisi, qui est en soie et coton et toujours blanc ; la reliure était de cuir doré et les fermoirs  étaient d’argent ; les gardes étaient de ce parchemin que les parcheminiers de Paris faisaient serment d’acheter à la salle Saint-Mathurin « et point ailleurs ». Ce volume était plein de gravures sur bois et sur cuivre, et de figures géographiques de beaucoup de pays ; il était précédé d’une protestation des imprimeurs, papetiers et libraires contre l’édit de 1635 qui frappait d’un impôt « les cuirs, les bières, le pied fourché, le poisson de mer et le papier » ; et au verso du frontispice on lisait une dédicace adressée aux Gryphes, qui sont à Lyon ce que les Elzévirs sont à Amsterdam. De tout cela, il résultait un exemplaire illustre, presque aussi rare que l’\emph{Apostol} de Moscou.\par
Ce livre était beau ; c’est pourquoi René-Jean le regardait, trop peut-être. Le volume était précisément ouvert à une grande estampe représentant saint Barthélemy portant sa peau sur son bras. Cette estampe se voyait d’en bas. Quand toutes les mûres furent mangées, René-Jean la considéra avec un regard d’amour terrible, et Georgette, dont l’œil suivait la direction des yeux de son frère, aperçut l’estampe et dit : Gimage.\par
Ce mot sembla déterminer René-Jean. Alors, à la grande stupeur de Gros-Alain, il fit une chose extraordinaire.\par
Une grosse chaise de chêne était dans un angle de la bibliothèque ; René-Jean marcha à cette chaise, la saisit, et la traîna à lui tout seul jusqu’au pupitre. Puis, quand la chaise toucha le pupitre, il monta dessus et posa ses deux poings sur le livre.\par
 Parvenu à ce sommet, il sentit le besoin d’être magnifique ; il prit la « gimage » par le coin d’en haut et la déchira soigneusement ; cette déchirure de saint Barthélemy se fit de travers, mais ce ne fut pas la faute de René-Jean ; il laissa dans le livre tout le côté gauche avec un œil et un peu de l’auréole du vieil évangéliste apocryphe, et offrit à Georgette l’autre moitié du saint et toute sa peau. Georgette reçut le saint et dit : Momomme.\par
— Et moi ! cria Gros-Alain.\par
Il en est de la première page arrachée comme du premier sang versé. Cela décide le carnage.\par
René-Jean tourna le feuillet ; derrière le saint il y avait le commentateur, Pantœnus ; René-Jean décerna Pantœnus à Gros-Alain.\par
Cependant Georgette déchira son grand morceau en deux petits, puis les deux petits en quatre ; si bien que l’histoire pourrait dire que saint Barthélemy, après avoir été écorché en Arménie, fut écartelé en Bretagne.
\subsubsection[{VI}]{VI}\phantomsection
\label{p3l3c6}
\noindent L’écartèlement terminé, Georgette tendit la main à René-Jean et dit : Encore !\par
Après le saint et le commentateur venaient, portraits rébarbatifs, les glossateurs. Le premier en date était Gavantus ; René-Jean l’arracha, et mit dans la main de Georgette Gavantus.\par
 Tous les glossateurs de saint Barthélemy y passèrent.\par
Donner est une supériorité. René-Jean ne se réserva rien. Gros-Alain et Georgette le contemplaient ; cela lui suffisait ; il se contenta de l’admiration de son public.\par
René-Jean, inépuisable et magnanime, offrit à Gros-Alain Fabricio Pignatelli et à Georgette le père Stilting ; il offrit à Gros-Alain Alphonse Tostat et à Georgette \emph{Cornelius a Lapide ;} Gros-Alain eut Henri Hammond, et Georgette eut le père Roberti, augmenté d’une vue de la ville de Douai, où il naquit en 1619. Gros-Alain reçut la protestation des papetiers et Georgette obtint la dédicace aux Gryphes. Il y avait aussi des cartes. René-Jean les distribua. Il donna l’Éthiopie à Gros-Alain et la Lycaonie à Georgette. Cela fait, il jeta le livre à terre.\par
Ce fut un moment effrayant. Gros-Alain et Georgette virent, avec une extase mêlée d’épouvante, René-Jean froncer ses sourcils, roidir ses jarrets, crisper ses poings, et pousser hors du lutrin l’in-quarto massif. Un bouquin majestueux qui perd contenance, c’est tragique. Le lourd volume désarçonné pendit un moment, hésita, se balança, puis s’écroula, et, rompu, froissé, lacéré, déboîté dans sa reliure, disloqué dans ses fermoirs, s’aplatit lamentablement sur le plancher. Heureusement il ne tomba point sur eux.\par
Ils furent éblouis, point écrasés. Toutes les aventures des conquérants ne finissent pas aussi bien.\par
 Comme toutes les gloires, cela fit un grand bruit et un nuage de poussière.\par
Ayant terrassé le livre, René-Jean descendit de la chaise.\par
Il y eut un instant de silence et de terreur ; la victoire a ses effrois. Les trois enfants se prirent les mains et se tinrent à distance, considérant le vaste volume démantelé.\par
Mais, après un peu de rêverie, Gros-Alain s’approcha énergiquement et lui donna un coup de pied.\par
Ce fut fini. L’appétit de la destruction existe. René-Jean donna son coup de pied, Georgette donna son coup de pied, ce qui la fit tomber par terre, mais assise ; elle en profita pour se jeter sur Saint-Barthélemy ; tout prestige disparut ; René-Jean se précipita, Gros-Alain se rua, et joyeux, éperdus, triomphants, impitoyables, déchirant les estampes, balafrant les feuillets, arrachant les signets, égratignant la reliure, décollant le cuir doré, déclouant les clous des coins d’argent, cassant le parchemin, déchiquetant le texte auguste, travaillant des pieds, des mains, des ongles, des dents, roses, riants, féroces, les trois anges de proie s’abattirent sur l’évangéliste sans défense.\par
Ils anéantirent l’Arménie, la Judée, le Bénévent où sont les reliques du saint, Nathanaël qui est peut-être le même que Barthélemy, le pape Gélase qui déclara apocryphe l’évangile Barthélemy-Nathanaël, toutes les figures, toutes les cartes, et l’exécution inexorable du vieux livre les absorba tellement qu’une souris passa sans qu’ils y prissent garde.\par
 Ce fut une extermination.\par
Tailler en pièces l’histoire, la légende, la science, les miracles vrais ou faux, le latin d’église, les superstitions, les fanatismes, les mystères, déchirer toute une religion du haut en bas, c’est un travail pour trois géants, et même pour trois enfants ; les heures s’écoulèrent dans ce labeur, mais ils en vinrent à bout ; rien ne resta de Saint-Barthélemy.\par
Quand ce fut fini, quand la dernière page fut détachée, quand la dernière estampe fut par terre, quand il ne resta plus du livre que des tronçons de texte et d’images dans un squelette de reliure, René-Jean se dressa debout, regarda le plancher jonché de toutes ces feuilles éparses, et battit des mains.\par
Gros-Alain battit des mains.\par
Georgette prit à terre une de ces feuilles, se leva, s’appuya contre la fenêtre qui lui venait au menton, et se mit à déchiqueter par la croisée la grande page en petits morceaux.\par
Ce que voyant, René-Jean et Gros-Alain en firent autant. Ils ramassèrent et déchirèrent, ramassèrent encore et déchirèrent encore, par la croisée comme Georgette ; et, page à page, émietté par ces petits doigts acharnés, presque tout l’antique livre s’envola dans le vent. Georgette, pensive, regarda ces essaims de petits papiers blancs se disperser à tous les souffles de l’air, et dit :\par
— Papillons.\par
Et le massacre se termina par un évanouissement dans l’azur.\par
 \subsubsection[{VII}]{VII}\phantomsection
\label{p3l3c7}
\noindent Telle fut la deuxième mise à mort de saint Barthélemy qui avait été déjà une première fois martyr l’an 49 de Jésus-Christ.\par
Cependant le soir venait, la chaleur augmentait, la sieste était dans l’air, les yeux de Georgette devenaient vagues, René-Jean alla à son berceau, en tira le sac de paille qui lui tenait lieu de matelas, le traîna jusqu’à la fenêtre, s’allongea dessus et dit : — Couchons-nous.\par
Gros-Alain mit sa tête sur René-Jean, Georgette mit sa tête sur Gros-Alain, et les trois malfaiteurs s’endormirent.\par
Les souffles tièdes entraient par les fenêtres ouvertes ; des parfums de fleurs sauvages, envolés des ravins et des collines, erraient mêlés aux haleines du soir ; l’espace était calme et miséricordieux, tout rayonnait, tout s’apaisait, tout aimait tout ; le soleil donnait à la création cette caresse, la lumière ; on percevait par tous les pores l’harmonie qui se dégage de la douceur colossale des choses ; il y avait de la maternité dans l’infini ; la création est un prodige en plein épanouissement, elle complète son énormité par sa bonté ; il semblait que l’on sentît quelqu’un d’invisible prendre ces mystérieuses précautions qui dans le  redoutable conflit des êtres protègent les chétifs contre les forts ; en même temps, c’était beau ; la splendeur égalait la mansuétude. Le paysage, ineffablement assoupi, avait cette moire magnifique que font sur les prairies et sur les rivières les déplacements de l’ombre et de la clarté ; les fumées montaient vers les nuages, comme des rêveries vers des visions ; des vols d’oiseaux tourbillonnaient au-dessus de la Tourgue ; les hirondelles regardaient par les croisées, et avaient l’air de venir voir si les enfants dormaient bien. Ils étaient gracieusement groupés l’un sur l’autre, immobiles, demi-nus, dans des poses d’amours ; ils étaient adorables et purs, à eux trois ils n’avaient pas neuf ans, ils faisaient des songes de paradis qui se reflétaient sur leurs bouches en vagues sourires, Dieu leur parlait peut-être à l’oreille, ils étaient ceux que toutes les langues humaines appellent les faibles et les bénis, ils étaient les innocents vénérables ; tout faisait silence comme si le souffle de leurs douces poitrines était l’affaire de l’univers et était écouté de la création entière, les feuilles ne bruissaient pas, les herbes ne frissonnaient pas ; il semblait que le vaste monde étoilé retînt sa respiration pour ne point troubler ces trois humbles dormeurs angéliques, et rien n’était sublime comme l’immense respect de la nature autour de cette petitesse.\par
Le soleil allait se coucher et touchait presque à l’horizon. Tout à coup, dans cette paix profonde, éclata un éclair qui sortit de la forêt, puis un bruit farouche. On venait de tirer un coup de canon. Les échos s’emparèrent  de ce bruit et en firent un fracas. Le grondement prolongé de colline en colline fut monstrueux. Il réveilla Georgette.\par
Elle souleva un peu sa tête, dressa son petit doigt, écouta, et dit :\par
— Poum !\par
Le bruit cessa, et tout rentra dans le silence, Georgette remit sa tête sur Gros-Alain, et se rendormit.\par
  \subsection[{Livre quatrième. La mère}]{Livre quatrième \\
La mère}\phantomsection
\label{p3l4}
\subsubsection[{I. La mort passe}]{I \\
La mort passe}\phantomsection
\label{p3l4c1}
\noindent Ce soir-là, la mère, qu’on a vue cheminant presque au hasard, avait marché toute la journée. C’était, du reste, son histoire de tous les jours ; aller devant elle et ne jamais s’arrêter. Car ses sommeils d’accablement dans le premier coin venu n’étaient pas plus du repos que ce qu’elle mangeait çà et là, comme les oiseaux picorent, n’était de la nourriture. Elle mangeait et dormait juste autant qu’il fallait pour ne pas tomber morte.\par
C’était dans une grange abandonnée qu’elle avait passé la nuit précédente ; les guerres civiles font de ces masures-là ; elle avait trouvé dans un champ désert quatre murs, une porte ouverte, un peu de paille sous un reste de toit, et elle s’était couchée sur cette paille  et sous ce toit, sentant à travers la paille le glissement des rats et voyant à travers le toit le lever des astres. Elle avait dormi quelques heures ; puis s’était réveillée au milieu de la nuit, et remise en route afin de faire le plus de chemin possible avant la grande chaleur du jour. Pour qui voyage à pied l’été, minuit est plus clément que midi.\par
Elle suivait de son mieux l’itinéraire sommaire que lui avait indiqué le paysan de Vantortes ; elle allait le plus possible au couchant. Qui eût été près d’elle l’eût entendue dire sans cesse à demi-voix : — La Tourgue. — Avec les noms de ses trois enfants, elle ne savait plus guère que ce mot-là.\par
Tout en marchant, elle songeait. Elle pensait aux aventures qu’elle avait traversées ; elle pensait à tout ce qu’elle avait souffert, à tout ce qu’elle avait accepté ; aux rencontres, aux indignités, aux conditions faites, aux marchés proposés et subis, tantôt pour un asile, tantôt pour un morceau de pain, tantôt simplement pour obtenir qu’on lui montrât sa route. Une femme misérable est plus malheureuse qu’un homme misérable, parce qu’elle est instrument de plaisir. Affreuse marche errante ! Du reste, tout lui était bien égal pourvu qu’elle retrouvât ses enfants.\par
Sa première rencontre, ce jour-là, avait été un village sur la route ; l’aube paraissait à peine ; tout était encore baigné du sombre de la nuit ; pourtant quelques portes étaient déjà entre-bâillées dans la grande rue du village, et des têtes curieuses sortaient des fenêtres. Les habitants avaient l’agitation d’une ruche inquiétée.  Cela tenait à un bruit de roues et de ferrailles qu’on avait entendu.\par
Sur la place, devant l’église, un groupe ahuri, les yeux en l’air, regardait quelque chose descendre par la route vers le village du haut d’une colline. C’était un chariot à quatre roues traîné par cinq chevaux attelés de chaînes. Sur le chariot on distinguait un entassement qui ressemblait à un monceau de longues solives au milieu desquelles il y avait on ne sait quoi d’informe ; c’était recouvert d’une grande bâche, qui avait l’air d’un linceul. Dix hommes à cheval marchaient en avant du chariot et dix autres en arrière. Ces hommes avaient des chapeaux à trois cornes et l’on voyait se dresser au-dessus de leurs épaules des pointes qui paraissaient être des sabres nus. Tout ce cortège, avançant lentement, se découpait en vive noirceur sur l’horizon. Le chariot semblait noir, l’attelage semblait noir, les cavaliers semblaient noirs. Le matin blêmissait derrière.\par
Cela entra dans le village et se dirigea vers la place.\par
Il s’était fait un peu de jour pendant la descente de ce chariot, et l’on put voir distinctement ce cortège qui paraissait une marche d’ombres, car il n’en sortait pas une parole.\par
Les cavaliers étaient des gendarmes. Ils avaient en effet le sabre nu. La bâche était noire.\par
La misérable mère errante entra de son côté dans le village et s’approcha de l’attroupement des paysans au moment où arrivaient sur la place cette voiture et ces gendarmes. Dans l’attroupement, des voix chuchotaient des questions et des réponses.\par
 — Qu’est-ce que c’est que ça ?\par
— C’est la guillotine qui passe.\par
— D’où vient-elle ?\par
— De Fougères.\par
— Où va-t-elle ?\par
— Je ne sais pas. On dit qu’elle va à un château du côté de Parigné.\par
— A Parigné !\par
— Qu’elle aille où elle voudra, pourvu qu’elle ne s’arrête pas ici !\par
Cette grande charrette avec son chargement voilé d’une sorte de suaire, cet attelage, ces gendarmes, le bruit de ces chaînes, le silence de ces hommes, l’heure crépusculaire, tout cet ensemble était spectral.\par
Ce groupe traversa la place et sortit du village ; le village était dans un fond entre une montée et une descente ; au bout d’un quart d’heure, les paysans, restés là comme pétrifiés, virent reparaître la lugubre procession au sommet de la colline qui était à l’occident. Les ornières cahotaient les grosses roues, les chaînes de l’attelage grelottaient au vent du matin, les sabres brillaient ; le soleil se levait. La route tourna, tout disparut.\par
C’était le moment même où Georgette, dans la salle de la bibliothèque, se réveillait à côté de ses frères encore endormis, et disait bonjour à ses pieds roses.
 \subsubsection[{II. La mort parle}]{II \\
La mort parle}\phantomsection
\label{p3l4c2}
\noindent La mère avait regardé cette chose obscure passer, mais n’avait pas compris ni cherché à comprendre, ayant devant les yeux une autre vision, ses enfants perdus dans les ténèbres.\par
Elle sortit du village, elle aussi, peu après le cortège qui venait de défiler, et suivit la même route, à quelque distance en arrière de la deuxième escouade de gendarmes. Subitement le mot « guillotine » lui revint ; « guillotine, » pensa-t-elle ; cette sauvage, Michelle Fléchard, ne savait pas ce que c’était ; mais l’instinct avertit ; elle eut, sans pouvoir dire pourquoi, un frémissement, il lui sembla horrible de marcher derrière cela, et elle prit à gauche, quitta la route, et s’engagea sous des arbres qui étaient la forêt de Fougères.\par
Après avoir rôdé quelque temps, elle aperçut un clocher et des toits, c’était un des villages de la lisière du bois, elle y alla. Elle avait faim.\par
Ce village était un de ceux où les républicains avaient établi des postes militaires.\par
Elle pénétra jusqu’à la place de la mairie.\par
 Dans ce village-là aussi il y avait émoi et anxiété. Un rassemblement se pressait devant un perron de quelques marches qui était l’entrée de la mairie. Sur ce perron on apercevait un homme escorté de soldats qui tenait à la main un grand placard déployé. Cet homme avait à sa droite un tambour et à sa gauche un afficheur portant un pot à colle et un pinceau.\par
Sur le balcon au-dessus de la porte le maire était debout, ayant son écharpe tricolore mêlée à ses habits de paysan.\par
L’homme au placard était un crieur public.\par
Il avait son baudrier de tournée auquel était suspendue une petite sacoche, ce qui indiquait qu’il allait de village en village, et qu’il avait quelque chose à crier dans tout le pays.\par
Au moment où Michelle Fléchard approcha, il venait de déployer le placard, et il en commençait la lecture. Il dit d’une voix haute :\par
— « République française une et indivisible. »\par
Le tambour fit un roulement. Il y eut dans le rassemblement une sorte d’ondulation. Quelques-uns ôtèrent leurs bonnets ; d’autres renfoncèrent leurs chapeaux. Dans ce temps-là et dans ce pays-là, on pouvait presque reconnaître l’opinion à la coiffure ; les chapeaux étaient royalistes, les bonnets étaient républicains. Les murmures de voix confuses cessèrent, on écouta, le crieur lut :\par
— « ... En vertu des ordres à nous donnés et des pouvoirs à nous délégués par le comité de salut public... »\par
 Il y eut un deuxième roulement de tambour. Le crieur poursuivit :\par
— « ... Et en exécution du décret de la Convention nationale qui met hors la loi les rebelles pris les armes à la main, et qui frappe de la peine capitale quiconque leur donnera asile ou les fera évader... »\par
Un paysan demanda bas à son voisin :\par
— Qu’est-ce que c’est que ça, la peine capitale ?\par
Le voisin répondit :\par
— Je ne sais pas.\par
Le crieur agita le placard :\par
— « ... Vu l’article 17 de la loi du 30 avril qui donne tout pouvoir aux délégués et aux subdélégués contre les rebelles,\par
« Sont mis hors la loi... »\par
Il fit une pause et reprit :\par
— « ... Les individus désignés sous les noms et surnoms qui suivent... »\par
Tout l’attroupement prêta l’oreille.\par
La voix du crieur devint tonnante. Il dit :\par
— « ... Lantenac, brigand. »\par
— C’est monseigneur, murmura un paysan.\par
Et l’on entendit dans la foule ce chuchotement : C’est monseigneur.\par
Le crieur reprit :\par
— « ... Lantenac, ci-devant marquis, brigand. — L’Imânus, brigand... »\par
Deux paysans se regardèrent de côté.\par
— C’est Gouge-le-Bruant.\par
— Oui, c’est Brise-Bleu.\par
 Le crieur continuait de lire la liste :\par
— « Grand-Francœur, brigand... »\par
Le rassemblement murmura :\par
— C’est un prêtre.\par
— Oui, monsieur l’abbé Turmeau.\par
— Oui, quelque part, du côté du bois de la Chapelle, il est curé.\par
— Et brigand, dit un homme à bonnet.\par
Le crieur lut :\par
— « ... Boisnouveau, brigand. — Les deux frères Pique-en-bois, brigands. — Houzard, brigand... »\par
— C’est monsieur de Quélen, dit un paysan.\par
— « Panier, brigand... »\par
— C’est monsieur Sepher.\par
— « ... Place-Nette, brigand... »\par
— C’est monsieur Jamois.\par
Le crieur poursuivait sa lecture sans s’occuper de ces commentaires.\par
— « ... Guinoiseau, brigand. — Chatenay, dit Robi, brigand... »\par
Un paysan chuchota :\par
— Guinoiseau est le même que le Blond, Chatenay est de Saint-Ouen.\par
— « ... Hoisnard, brigand, » reprit le crieur.\par
Et l’on entendit dans la foule :\par
— Il est de Ruillé.\par
— Oui, c’est Branche-d’Or.\par
— Il a eu son frère tué à l’attaque de Pontorson.\par
— Oui, Hoisnard-Malonnière.\par
— Un beau jeune homme de dix-neuf ans.\par
 — Attention, dit le crieur. Voici la fin de la liste :\par
— « ... Belle-Vigne, brigand. — La Musette, brigand. — Sabre tout, brigand. — Brin-d’Amour, brigand... »\par
Un garçon poussa le coude d’une fille. La fille sourit.\par
Le crieur continua :\par
— « Chante-en-hiver, brigand. — Le Chat, brigand... »\par
Un paysan dit :\par
— C’est Moulard.\par
— « ... Tabouze, brigand... »\par
Un paysan dit :\par
— C’est Gauffre.\par
— Ils sont deux, les Gauffre, ajouta une femme.\par
— Tous des bons, grommela un gars.\par
Le crieur secoua l’affiche et le tambour battit un ban.\par
Le crieur reprit sa lecture :\par
— « ... Les susnommés, en quelque lieu qu’ils soient saisis, et après l’identité constatée, seront immédiatement mis à mort. »\par
Il y eut un mouvement.\par
Le crieur poursuivit :\par
— « ... Quiconque leur donnera asile ou aidera à leur évasion sera traduit en cour martiale, et mis à mort. Signé... »\par
Le silence devint profond.\par
— « ... Signé : le délégué du comité de salut public, C{\scshape imourdain}. »\par
 — Un prêtre, dit un paysan.\par
— L’ancien curé de Parigné, dit un autre.\par
Un bourgeois ajouta :\par
— Turmeau et Cimourdain. Un prêtre blanc et un prêtre bleu.\par
— Tous deux noirs, dit un autre bourgeois.\par
Le maire, qui était sur le balcon, souleva son chapeau et cria :\par
— Vive la république !\par
Un roulement de tambour annonça que le crieur n’avait pas fini. En effet il fit un signe de la main.\par
— Attention, dit-il. Voici les quatre dernières lignes de l’affiche du gouvernement. Elles sont signées du chef de la colonne d’expédition des Côtes-du-Nord, qui est le commandant Gauvain.\par
— Écoutez ! dirent les voix de la foule.\par
Et le crieur lut :\par
— « Sous peine de mort... »\par
Tous se turent.\par
— « ... Défense est faite, en exécution de l’ordre ci-dessus, de porter aide et secours aux dix-neuf rebelles susnommés qui sont à cette heure investis et cernés dans la Tourgue. »\par
— Hein ? dit une voix.\par
C’était une voix de femme. C’était la voix de la mère.
 \subsubsection[{III. Bourdonnement de paysans}]{III \\
Bourdonnement de paysans}\phantomsection
\label{p3l4c3}
\noindent Michelle Fléchard était mêlée à la foule. Elle n’avait rien écouté, mais ce qu’on n’écoute pas, on l’entend. Elle avait entendu ce mot, la Tourgue. Elle dressait la tête.\par
— Hein ? répéta-t-elle, la Tourgue ?\par
On la regarda. Elle avait l’air égaré. Elle était en haillons. Des voix murmurèrent : — Ça a l’air d’une brigande.\par
Une paysanne qui portait des galettes de sarrasin dans un panier s’approcha et lui dit tout bas :\par
— Taisez-vous.\par
Michelle Fléchard considéra cette femme avec stupeur. De nouveau elle ne comprenait plus. Ce nom, la Tourgue, avait passé comme un éclair, et la nuit se refaisait. Est-ce qu’elle n’avait pas le droit de s’informer ? Qu’est-ce qu’on avait donc à la regarder ainsi ?\par
Cependant le tambour avait battu un dernier ban, l’afficheur avait collé l’affiche, le maire était rentré dans la mairie, le crieur était parti pour quelque autre village, et l’attroupement se dispersait.\par
 Un groupe était resté devant l’affiche. Michelle Fléchard alla à ce groupe.\par
On commentait les noms des hommes mis hors la loi.\par
Il y avait des paysans et des bourgeois, c’est-à-dire des blancs et des bleus.\par
Un paysan disait :\par
— C’est égal, ils ne tiennent pas tout le monde. Dix-neuf, ça n’est que dix-neuf. Ils ne tiennent pas Priou, ils ne tiennent pas Benjamin Moulins, ils ne tiennent pas Goupil, de la paroisse d’Andouillé.\par
— Ni Lorieul, de Monjean, dit un autre.\par
D’autres ajoutèrent :\par
— Ni Brice-Denys.\par
— Ni François Dudouet.\par
— Oui, celui de Laval.\par
— Ni Huet, de Launey-Villiers.\par
— Ni Grégis.\par
— Ni Pilon.\par
— Ni Filleul.\par
— Ni Ménicent.\par
— Ni Guéharrée.\par
— Ni les trois frères Logerais.\par
— Ni monsieur Lechandelier de Pierreville.\par
— Imbéciles ! dit un vieux sévère à cheveux blancs. Ils ont tout, s’ils ont Lantenac.\par
— Ils ne l’ont pas encore, murmura un des jeunes.\par
Le vieillard répliqua :\par
— Lantenac pris, l’âme est prise. Lantenac mort, la Vendée est tuée.\par
 — Qu’est-ce que c’est donc que ce Lantenac ? demanda un bourgeois.\par
Un bourgeois répondit :\par
— C’est un ci-devant.\par
Et un autre reprit :\par
— C’est un de ceux qui fusillent les femmes.\par
Michelle Fléchard entendit, et dit :\par
— C’est vrai.\par
On se retourna.\par
Et elle ajouta :\par
— Puisqu’on m’a fusillée.\par
Le mot était singulier ; il fit l’effet d’une vivante qui se dit morte. On se mit à l’examiner, un peu de travers.\par
Elle était inquiétante à voir, en effet ; tressaillant de tout, effarée, frissonnante, ayant une anxiété fauve, et si effrayée qu’elle était effrayante. Il y a dans le désespoir de la femme on ne sait quoi de faible qui est terrible. On croit voir un être suspendu à l’extrémité du sort. Mais les paysans prennent la chose plus en gros. L’un d’eux grommela : — Ça pourrait bien être une espionne.\par
— Taisez-vous donc, et allez-vous-en, lui dit tout bas la bonne femme qui lui avait déjà parlé.\par
Michelle Fléchard répondit :\par
— Je ne fais pas de mal. Je cherche mes enfants.\par
La bonne femme regarda ceux qui regardaient Michelle Fléchard, se toucha le front du doigt en clignant de l’œil, et dit :\par
— C’est une innocente.\par
 Puis elle la prit à part, et lui donna une galette de sarrasin.\par
Michelle Fléchard, sans remercier, mordit avidement dans la galette.\par
— Oui, dirent les paysans, elle mange comme une bête. C’est une innocente.\par
Et le reste du rassemblement se dissipa. Tous s’en allèrent l’un après l’autre.\par
Quand Michelle Fléchard eut mangé, elle dit à la paysanne :\par
— C’est bon, j’ai mangé. Maintenant, la Tourgue ?\par
— Voilà que ça la reprend ! s’écria la paysanne.\par
— Il faut que j’aille à la Tourgue. Dites-moi le chemin de la Tourgue.\par
— Jamais ! dit la paysanne. Pour vous faire tuer, n’est-ce pas ? D’ailleurs, je ne sais pas. Ah çà, vous êtes donc vraiment folle. Écoutez, pauvre femme, vous avez l’air fatiguée. Voulez-vous vous reposer chez moi ?\par
— Je ne me repose pas, dit la mère.\par
— Elle a les pieds tout écorchés, murmura la paysanne.\par
Michelle Fléchard reprit :\par
— Puisque je vous dis qu’on m’a volé mes enfants. Une petite fille et deux petits garçons. Je viens du carnichot qui est dans la forêt. On peut parler de moi à Tellmarch-le-Caimand. Et puis à l’homme que j’ai rencontré dans le champ là-bas. C’est le caimand qui m’a guérie. Il paraît que j’avais quelque chose de cassé. Tout cela, ce sont des choses qui sont arrivées. Il y a encore le sergent Radoub. On peut lui parler. Il dira.  Puisque c’est lui qui nous a rencontrés dans un bois. Trois. Je vous dis trois enfants. Même que l’aîné s’appelle René-Jean. Je puis prouver tout cela. L’autre s’appelle Gros-Alain, et l’autre s’appelle Georgette. Mon mari est mort. On l’a tué. Il était métayer à Siscoignard. Vous avez l’air d’une bonne femme. Enseignez-moi mon chemin. Je ne suis pas une folle, je suis une mère. J’ai perdu mes enfants. Je les cherche. Voilà tout. Je ne sais pas au juste d’où je viens. J’ai dormi cette nuit-ci sur de la paille dans une grange. La Tourgue, voilà où je vais. Je ne suis pas une voleuse. Vous voyez bien que je dis la vérité. On devrait m’aider à retrouver mes enfants. Je ne suis pas du pays. J’ai été fusillée, mais je ne sais pas où.\par
La paysanne hocha la tête et dit :\par
— Écoutez, la passante. Dans des temps de révolution, il ne faut pas dire des choses qu’on ne comprend pas. Ça peut vous faire arrêter.\par
— Mais la Tourgue ! cria la mère. Madame, pour l’amour de l’enfant Jésus et de la sainte bonne Vierge du paradis, je vous en prie, madame, je vous en supplie, je vous en conjure, dites-moi par où l’on va pour aller à la Tourgue !\par
La paysanne se mit en colère.\par
— Je ne le sais pas ! et je le saurais que je ne le dirais pas ! Ce sont là de mauvais endroits. On ne va pas là.\par
— J’y vais pourtant, dit la mère.\par
Et elle se mit en route.\par
La paysanne la regarda s’éloigner, et grommela :\par
 — Il faut cependant qu’elle mange.\par
Elle courut après Michelle Fléchard et lui mit une galette de blé noir dans la main.\par
— Voilà pour votre souper.\par
Michelle Fléchard prit le pain de sarrasin, ne répondit pas, ne tourna pas la tête, et continua de marcher.\par
Elle sortit du village. Comme elle atteignait les dernières maisons elle rencontra trois petits enfants déguenillés et pieds nus, qui passaient. Elle s’approcha d’eux et dit :\par
— Ceux-ci, c’est deux filles et un garçon.\par
Et voyant qu’ils regardaient son pain, elle le leur donna.\par
Les enfants prirent le pain et eurent peur.\par
Elle s’enfonça dans la forêt.
 \subsubsection[{IV. Une méprise}]{IV \\
Une méprise}\phantomsection
\label{p3l4c4}
\noindent Cependant, ce jour-là même, avant que l’aube parût, dans l’obscurité indistincte de la forêt, il s’était passé, sur le tronçon de chemin qui va de Javené à Lécousse, ceci :\par
Tout est chemin creux dans le Bocage, et, entre toutes, la route de Javené à Parigné par Lécousse est très encaissée. De plus, tortueuse. C’est plutôt un ravin qu’un chemin. Cette route vient de Vitré et a eu l’honneur de cahoter le carrosse de madame de Sévigné. Elle est comme murée à droite et à gauche par les haies. Pas de lieu meilleur pour une embuscade.\par
Ce matin-là, une heure avant que Michelle Fléchard, sur un autre point de la forêt, arrivât dans ce premier village où elle avait eu la sépulcrale apparition de la charrette escortée de gendarmes, il y avait, dans les halliers que la route de Javené traverse au sortir du pont sur le Couesnon, un pêle-mêle d’hommes invisibles. Les branches cachaient tout. Ces hommes étaient des paysans, tous vêtus du grigo, sayon de poil que portaient les rois de Bretagne au sixième siècle et les paysans au dix-huitième. Ces hommes étaient armés,  les uns de fusils, les autres de cognées. Ceux qui avaient des cognées venaient de préparer dans une clairière une sorte de bûcher de fagots secs et de rondins auquel on n’avait plus qu’à mettre le feu. Ceux qui avaient des fusils étaient groupés des deux côtés du chemin dans une posture d’attente. Qui eût pu voir à travers les feuilles eût aperçu partout des doigts sur des détentes et des canons de carabine braqués dans les embrasures que font les entrecroisements des branchages. Ces gens étaient à l’affût. Tous les fusils convergeaient sur la route, que le point du jour blanchissait.\par
Dans ce crépuscule, des voix basses dialoguaient.\par
— Es-tu sûr de ça ?\par
— Dame, on le dit.\par
— Elle va passer ?\par
— On dit qu’elle est dans le pays.\par
— Il ne faut pas qu’elle en sorte.\par
— Il faut la brûler.\par
— Nous sommes trois villages venus pour cela.\par
— Oui, mais l’escorte ?\par
— On tuera l’escorte.\par
— Mais est-ce que c’est par cette route-ci qu’elle passe ?\par
— On le dit.\par
— C’est donc alors qu’elle viendrait de Vitré ?\par
— Pourquoi pas ?\par
— Mais c’est qu’on disait qu’elle venait de Fougères.\par
— Qu’elle vienne de Fougères ou de Vitré, elle vient du diable.\par
 — Oui.\par
— Et il faut qu’elle y retourne.\par
— Oui.\par
— C’est donc à Parigné qu’elle irait ?\par
— Il paraît.\par
— Elle n’ira pas.\par
— Non.\par
— Non, non, non !\par
— Attention.\par
Il devenait utile de se taire en effet, car il commençait à faire un peu jour.\par
Tout à coup les hommes embusqués retinrent leur respiration ; on entendait un bruit de roues et de chevaux. Ils regardèrent à travers les branches et distinguèrent confusément dans le chemin creux une longue charrette, une escorte à cheval, quelque chose sur la charrette ; cela venait à eux.\par
— La voilà ! dit celui qui paraissait le chef.\par
— Oui, dit un des guetteurs, avec l’escorte.\par
— Combien d’hommes d’escorte ?\par
— Douze.\par
— On disait qu’ils étaient vingt.\par
— Douze ou vingt, tuons tout.\par
— Attendons qu’ils soient en pleine portée.\par
Peu après, à un tournant du chemin, la charrette et l’escorte apparurent.\par
— Vive le roi ! cria le chef paysan.\par
Cent coups de fusil partirent à la fois.\par
Quand la fumée se dissipa, l’escorte aussi était  dissipée. Sept cavaliers étaient tombés, cinq s’étaient enfuis. Les paysans coururent à la charrette.\par
— Tiens, s’écria le chef, ce n’est pas la guillotine. C’est une échelle.\par
La charrette avait en effet pour tout chargement une longue échelle.\par
Les deux chevaux s’étaient abattus, blessés ; le charretier avait été tué, mais pas exprès.\par
— C’est égal, dit le chef, une échelle escortée est suspecte. Cela allait du côté de Parigné. C’était pour l’escalade de la Tourgue, bien sûr.\par
— Brûlons l’échelle ! crièrent les paysans.\par
Et ils brûlèrent l’échelle.\par
Quant à la funèbre charrette qu’ils attendaient, elle suivait une autre route, et elle était déjà à deux lieues plus loin, dans ce village où Michelle Fléchard la vit passer au soleil levant.
 \subsubsection[{V. Vox in deserto}]{V \\
Vox in deserto}\phantomsection
\label{p3l4c5}
\noindent Michelle Fléchard, en quittant les trois enfants auxquels elle avait donné son pain, s’était mise à marcher au hasard à travers le bois.\par
Puisqu’on ne voulait pas lui montrer son chemin, il fallait bien qu’elle le trouvât toute seule. Par instants, elle s’asseyait, et elle se relevait, et elle s’asseyait encore. Elle avait cette fatigue lugubre qu’on a d’abord dans les muscles, puis qui passe dans les os ; fatigue d’esclave. Elle était esclave en effet. Esclave de ses enfants perdus. Il fallait les retrouver. Chaque minute écoulée pouvait être leur perte ; qui a un tel devoir n’a plus de droit ; reprendre haleine lui était interdit. Mais elle était bien lasse. A ce degré d’épuisement, un pas de plus est une question. Le pourra-t-on faire ? Elle marchait depuis le matin ; elle n’avait plus rencontré de village, ni même de maison. Elle prit d’abord le sentier qu’il fallait, puis celui qu’il ne fallait pas, et elle finit par se perdre au milieu des branches pareilles les unes aux autres. Approchait-elle du but ? touchait-elle au terme de sa passion ? Elle était dans la voie douloureuse, et elle sentait l’accablement de la dernière  station. Allait-elle tomber sur la route et expirer là ? A un certain moment, avancer encore lui sembla impossible, le soleil déclinait, la forêt était obscure, les sentiers s’étaient effacés sous l’herbe, et elle ne sut plus que devenir. Elle n’avait plus que Dieu. Elle se mit à appeler, personne ne répondit.\par
Elle regarda autour d’elle, elle vit une claire-voie dans les branches, elle se dirigea de ce côté-là, et brusquement se trouva hors du bois.\par
Elle avait devant elle un vallon étroit comme une tranchée, au fond duquel coulait dans les pierres un clair filet d’eau. Elle s’aperçut alors qu’elle avait une soif ardente. Elle alla à cette eau, s’agenouilla, et but.\par
Elle profita de ce qu’elle était à genoux pour faire sa prière.\par
En se relevant, elle chercha à s’orienter.\par
Elle enjamba le ruisseau.\par
Au delà du petit vallon se prolongeait à perte de vue un vaste plateau couvert de broussailles courtes, qui, à partir du ruisseau, montait en plan incliné et emplissait tout l’horizon. La forêt était une solitude, ce plateau était un désert. Dans la forêt, derrière chaque buisson on pouvait rencontrer quelqu’un ; sur le plateau, aussi loin que le regard pouvait s’étendre, on ne voyait rien. Quelques oiseaux qui avaient l’air de fuir volaient dans les bruyères.\par
Alors, en présence de cet abandon immense, sentant fléchir ses genoux, et comme devenue insensée, la mère éperdue jeta à la solitude ce cri étrange : — Y a-t-il quelqu’un ici ?\par
 Et elle attendit la réponse.\par
On répondit.\par
Une voix sourde et profonde éclata ; cette voix venait du fond de l’horizon, elle se répercuta d’écho en écho ; cela ressemblait à un coup de tonnerre, à moins que ce ne fût un coup de canon ; et il semblait que cette voix répliquait à la question de la mère et qu’elle disait : Oui.\par
Puis le silence se fit.\par
La mère se dressa, ranimée ; il y avait quelqu’un ; il lui paraissait qu’elle avait maintenant à qui parler ; elle venait de boire et de prier ; les forces lui revenaient, elle se mit à gravir le plateau du côté où elle avait entendu l’énorme voix lointaine.\par
Tout à coup elle vit sortir de l’extrême horizon une haute tour. Cette tour était seule dans ce sauvage paysage ; un rayon du soleil couchant l’empourprait. Elle était à plus d’une lieue de distance. Derrière cette tour se perdait dans la brume une grande verdure diffuse qui était la forêt de Fougères.\par
Cette tour lui apparaissait sur le même point de l’horizon d’où était venu ce grondement qui lui avait semblé un appel. Était-ce cette tour qui avait fait ce bruit ?\par
Michelle Fléchard était arrivée sur le sommet du plateau ; elle n’avait plus devant elle que de la plaine.\par
Elle marcha vers la tour.
 \subsubsection[{VI. Situation}]{VI \\
Situation}\phantomsection
\label{p3l4c6}
\noindent Le moment était venu.\par
L’inexorable tenait l’impitoyable.\par
Cimourdain avait Lantenac dans sa main.\par
Le vieux royaliste rebelle était pris au gîte ; évidemment il ne pouvait échapper ; et Cimourdain entendait que le marquis fût décapité chez lui, sur place, sur ses terres, et en quelque sorte dans sa maison, afin que la demeure féodale vît tomber la tête de l’homme féodal, et que l’exemple fût mémorable.\par
C’est pourquoi il avait envoyé chercher à Fougères la guillotine. On vient de la voir en route.\par
Tuer Lantenac, c’était tuer la Vendée ; tuer la Vendée, c’était sauver la France. Cimourdain n’hésitait pas. Cet homme était à l’aise dans la férocité du devoir.\par
Le marquis semblait perdu ; de ce côté Cimourdain était tranquille, mais il était inquiet d’un autre côté. La lutte serait certainement affreuse ; Gauvain la dirigerait, et voudrait s’y mêler peut-être ; il y avait du soldat dans ce jeune chef ; il était homme à se jeter dans ce pugilat ; pourvu qu’il n’y fût pas tué ? Gauvain ! son enfant ! l’unique affection qu’il eût sur la terre ! Gauvain avait eu du bonheur jusque-là, mais le bonheur se lasse. Cimourdain tremblait. Sa destinée avait  cela d’étrange qu’il était entre deux Gauvain, l’un dont il voulait la mort, l’autre dont il voulait la vie.\par
Le coup de canon qui avait secoué Georgette dans son berceau et appelé la mère du fond des solitudes n’avait pas fait que cela. Soit hasard, soit intention du pointeur, le boulet, qui n’était pourtant qu’un boulet d’avertissement, avait frappé, crevé et arraché à demi l’armature de barreaux de fer qui masquait et fermait la grande meurtrière du premier étage de la tour. Les assiégés n’avaient pas eu le temps de réparer cette avarie.\par
Les assiégés s’étaient vantés ; ils avaient très peu de munitions. Leur situation, insistons-y, était plus critique encore que les assiégeants ne le supposaient. S’ils avaient eu assez de poudre, ils auraient fait sauter la Tourgue, eux et l’ennemi dedans ; c’était leur rêve ; mais toutes leurs réserves étaient épuisées. A peine avaient-ils trente coups à tirer par homme. Ils avaient beaucoup de fusils, d’espingoles et de pistolets, et peu de cartouches. Ils avaient chargé toutes les armes afin de pouvoir faire un feu continu ; mais combien de temps durerait ce feu ? Il fallait à la fois le nourrir et le ménager. Là était la difficulté. Heureusement — bonheur sinistre — la lutte serait surtout d’homme à homme, et à l’arme blanche ; au sabre et au poignard. On se colleterait plus qu’on ne se fusillerait. On se hacherait ; c’était là leur espérance.\par
L’intérieur de la tour semblait inexpugnable. Dans la salle basse où aboutissait le trou de brèche, était la retirade, cette barricade savamment construite par  Lantenac, qui obstruait l’entrée. En arrière de la retirade, une longue table était couverte d’armes chargées, tromblons, carabines et mousquetons, et de sabres, de haches et de poignards. N’ayant pu utiliser, pour faire sauter la tour, le cachot-crypte des oubliettes qui communiquait avec la salle basse, le marquis avait fait fermer la porte de ce caveau. Au-dessus de la salle basse était la chambre ronde du premier étage à laquelle on n’arrivait que par une vis-de-Saint-Gilles très étroite ; cette chambre, meublée, comme la salle basse, d’une table couverte d’armes toutes prêtes et sur lesquelles on n’avait qu’à mettre la main, était éclairée par la grande meurtrière dont un boulet venait de défoncer le grillage ; au-dessus de cette chambre, l’escalier en spirale montait à la chambre ronde du second étage où était la porte de fer donnant sur le pont-châtelet. Cette chambre du second s’appelait indistinctement \emph{la chambre de la porte de fer} ou \emph{la chambre des miroirs}, à cause de beaucoup de petits miroirs, accrochés à cru sur la pierre nue à de vieux clous rouillés, bizarre recherche mêlée à la sauvagerie. Les chambres d’en haut ne pouvant être utilement défendues, cette chambre des miroirs était ce que Manesson-Mallet, le législateur des places fortes, appelle « le dernier poste où les assiégés font leur capitulation ». Il s’agissait, nous l’avons dit déjà, d’empêcher les assiégeants d’arriver là.\par
Cette chambre ronde du second étage était éclairée par des meurtrières ; pourtant une torche y brûlait. Cette torche, plantée dans une torchère de fer pareille  à celle de la salle basse, avait été allumée par l’Imânus, qui avait placé tout à côté l’extrémité de la mèche soufrée. Soins horribles.\par
Au fond de la salle basse, sur un long tréteau, il y avait à manger, comme dans une caverne homérique ; de grands plats de riz, du fur, qui est une bouillie de blé noir, de la godnivelle, qui est un hachis de veau, des rondeaux de houichepote, pâte de farine et de fruits cuits à l’eau, de la badrée, des pots de cidre. Buvait et mangeait qui voulait.\par
Le coup de canon les mit tous en arrêt. On n’avait plus qu’une demi-heure devant soi.\par
L’Imânus, du haut de la tour, surveillait l’approche des assiégeants. Lantenac avait commandé de ne pas tirer et de les laisser arriver. Il avait dit :\par
— Ils sont quatre mille cinq cents. Tuer dehors est inutile. Ne tuez que dedans. Dedans, l’égalité se refait.\par
Et il avait ajouté en riant : — Égalité, Fraternité.\par
Il était convenu que lorsque l’ennemi commencerait son mouvement, l’Imânus, avec sa trompe, avertirait.\par
Tous, en silence, postés derrière la retirade, ou sur les marches des escaliers, attendaient, une main sur leur mousquet, l’autre sur leur rosaire.\par
La situation se précisait, et était ceci :\par
Pour les assaillants, une brèche à gravir, une barricade à forcer, trois salles superposées à prendre de haute lutte l’une après l’autre, deux escaliers tournants à emporter marche par marche, sous une nuée de mitraille ; pour les assiégés, mourir.
 \subsubsection[{VII. Préliminaires}]{VII \\
Préliminaires}\phantomsection
\label{p3l4c7}
\noindent Gauvain de son côté mettait en ordre l’attaque. Il donnait ses dernières instructions à Cimourdain, qui, on s’en souvient, devait, sans prendre part à l’action, garder le plateau, et à Guéchamp qui devait rester en observation avec le gros de l’armée dans le camp de la forêt. Il était entendu que ni la batterie basse du bois ni la batterie haute du plateau ne tireraient, à moins qu’il n’y eût sortie ou tentative d’évasion. Gauvain se réservait le commandement de la colonne de brèche. C’est là ce qui troublait Cimourdain.\par
Le soleil venait de se coucher.\par
Une tour en rase campagne ressemble à un navire en pleine mer. Elle doit être attaquée de la même façon. C’est plutôt un abordage qu’un assaut. Pas de canon. Rien d’inutile. A quoi bon canonner des murs de quinze pieds d’épaisseur ? Un trou dans le sabord, les uns qui le forcent, les autres qui le barrent, des haches, des couteaux, des pistolets, les poings et les dents. Telle est l’aventure.\par
Gauvain sentait qu’il n’y avait pas d’autre moyen d’enlever la Tourgue. Une attaque où l’on se voit le  blanc des yeux, rien de plus meurtrier. Il connaissait le redoutable intérieur de la tour, y ayant été enfant.\par
Il songeait profondément.\par
Cependant, à quelques pas de lui, son lieutenant, Guéchamp, une longue-vue à la main, examinait l’horizon du côté de Parigné. Tout à coup Guéchamp s’écria :\par
— Ah ! enfin !\par
Cette exclamation tira Gauvain de sa rêverie.\par
— Qu’y a-t-il, Guéchamp ?\par
— Mon commandant, il y a que voici l’échelle.\par
— L’échelle de sauvetage ?\par
— Oui.\par
— Comment ! nous ne l’avions pas encore ?\par
— Non, commandant. Et j’étais inquiet. L’exprès que j’avais envoyé à Javené était revenu.\par
— Je le sais.\par
— Il avait annoncé qu’il avait trouvé à la charpenterie de Javené l’échelle de la dimension voulue, qu’il l’avait réquisitionnée, qu’il avait fait mettre l’échelle sur une charrette, qu’il avait requis une escorte de douze cavaliers, et qu’il avait vu partir pour Parigné la charrette, l’escorte et l’échelle. Sur quoi il était revenu à franc étrier.\par
— Et nous avait fait ce rapport. Et il avait ajouté que la charrette, étant bien attelée et partie vers deux heures du matin, serait ici avant le coucher du soleil. Je sais tout cela. Eh bien ?\par
— Eh bien, mon commandant, le soleil vient de se coucher et la charrette qui apporte l’échelle n’est pas encore arrivée.\par
 — Est-ce possible ? Mais il faut pourtant que nous attaquions. L’heure est venue. Si nous tardions, les assiégés croiraient que nous reculons.\par
— Mon commandant, on peut attaquer.\par
— Mais l’échelle de sauvetage est nécessaire.\par
— Sans doute.\par
— Mais nous ne l’avons pas.\par
— Nous l’avons.\par
— Comment ?\par
— C’est ce qui m’a fait dire : Ah ! enfin ! La charrette n’arrivait pas ; j’ai pris ma longue-vue, et j’ai examiné la route de Parigné à la Tourgue, et, mon commandant, je suis content, la charrette est là-bas avec l’escorte. Elle descend une côte. Vous pouvez la voir.\par
Gauvain prit la longue-vue et regarda.\par
— En effet. La voici. Il ne fait plus assez de jour pour tout distinguer. Mais on voit l’escorte, c’est bien cela. Seulement l’escorte me paraît plus nombreuse que vous ne disiez, Guéchamp.\par
— Et à moi aussi.\par
— Ils sont à environ un quart de lieue.\par
— Mon commandant, l’échelle de sauvetage sera ici dans un quart d’heure.\par
— On peut attaquer.\par
C’était bien une charrette en effet qui arrivait, mais ce n’était pas celle qu’ils croyaient.\par
Gauvain, en se retournant, vit derrière lui le sergent Radoub, droit, les yeux baissés, dans l’attitude du salut militaire.\par
 — Qu’est-ce, sergent Radoub ?\par
— Citoyen commandant, nous, les hommes du bataillon du Bonnet-Rouge, nous avons une grâce à vous demander.\par
— Laquelle ?\par
— De nous faire tuer.\par
— Ah ! dit Gauvain.\par
— Voulez-vous avoir cette bonté ?\par
— Mais... c’est selon, dit Gauvain.\par
— Voici, mon commandant. Depuis l’affaire de Dol, vous nous ménagez. Nous sommes encore douze.\par
— Eh bien ?\par
— Ça nous humilie.\par
— Vous êtes la réserve.\par
— Nous aimons mieux être l’avant-garde.\par
— Mais j’ai besoin de vous pour décider le succès à la fin d’une action. Je vous conserve.\par
— Trop.\par
— C’est égal. Vous êtes dans la colonne. Vous marchez.\par
— Derrière. C’est le droit de Paris de marcher devant.\par
— J’y penserai, sergent Radoub.\par
— Pensez-y aujourd’hui, mon commandant. Voici une occasion. Il va y avoir un rude croc-en-jambe à donner ou à recevoir. Ce sera dru. La Tourgue brûlera les doigts de ceux qui y toucheront. Nous demandons la faveur d’en être.\par
Le sergent s’interrompit, se tordit la moustache, et reprit d’une voix altérée :\par
 — Et puis, voyez-vous, mon commandant, dans cette tour, il y a nos mômes. Nous avons là nos enfants, les enfants du bataillon, nos trois enfants. Cette affreuse face de Gribouille-mon-cul-te-baise, le nommé Brise-bleu, le nommé Imânus, ce Gouge-le-Bruand, ce Bouge-le-Gruand, ce Fouge-le-Truand, ce tonnerre de Dieu d’homme du diable, menace nos enfants. Nos enfants, nos mioches, mon commandant ! Quand tous les tremblements s’en mêleraient, nous ne voulons pas qu’il leur arrive malheur. Entendez-vous ça, autorité ? Nous ne le voulons pas. Tantôt, j’ai profité de ce qu’on ne se battait pas, et je suis monté sur le plateau, et je les ai regardés par une fenêtre ; oui, ils sont vraiment là, on peut les voir du bord du ravin, et je les ai vus, et je leur ai fait peur, à ces amours. Mon commandant, s’il tombe un seul cheveu de leurs petites caboches de chérubins, je le jure, mille noms de noms de tout ce qu’il y a de sacré, moi le sergent Radoub, je m’en prends à la carcasse du Père Éternel. Et voici ce que dit le bataillon : nous voulons que les mômes soient sauvés, ou être tous tués. C’est notre droit, ventraboumine ! oui, tous tués. Et maintenant, salut et respect.\par
Gauvain tendit la main à Radoub, et dit :\par
— Vous êtes des braves. Vous serez de la colonne d’attaque. Je vous partage en deux. Je mets six de vous à l’avant-garde, afin qu’on avance, et j’en mets six à l’arrière-garde, afin qu’on ne recule pas.\par
— Est-ce toujours moi qui commande les douze ?\par
— Certes.\par
 — Alors, mon commandant, merci. Car je suis de l’avant-garde.\par
Radoub refit le salut militaire et regagna le rang.\par
Gauvain tira sa montre, dit quelques mots à l’oreille de Guéchamp, et la colonne d’attaque commença à se former.
 \subsubsection[{VIII. Le verbe et le rugissement}]{VIII \\
Le verbe et le rugissement}\phantomsection
\label{p3l4c8}
\noindent Cependant Cimourdain, qui n’avait pas encore gagné son poste du plateau, et qui était à côté de Gauvain, s’approcha d’un clairon.\par
— Sonne à la trompe, lui dit-il.\par
Le clairon sonna, la trompe répondit.\par
Un son de clairon et un son de trompe s’échangèrent encore.\par
— Qu’est-ce que c’est ? demanda Gauvain à Guéchamp. Que veut Cimourdain ?\par
Cimourdain s’était avancé vers la tour, un mouchoir blanc à la main.\par
Il éleva la voix.\par
— Hommes qui êtes dans la tour, me connaissez-vous ?\par
Une voix, la voix de l’Imânus, répliqua du haut de la tour :\par
— Oui.\par
Les deux voix alors se parlèrent et se répondirent, et l’on entendit ceci :\par
— Je suis l’envoyé de la république.\par
— Tu es l’ancien curé de Parigné.\par
 — Je suis le délégué du comité de salut public.\par
— Tu es un prêtre.\par
— Je suis le représentant de la loi.\par
— Tu es un renégat.\par
— Je suis le commissaire de la révolution.\par
— Tu es un apostat.\par
— Je suis Cimourdain.\par
— Tu es le démon.\par
— Vous me connaissez ?\par
— Nous t’exécrons.\par
— Seriez-vous contents de me tenir en votre pouvoir ?\par
— Nous sommes ici dix-huit qui donnerions nos têtes pour avoir la tienne.\par
— Eh bien, je viens me livrer à vous.\par
On entendit au haut de la tour un éclat de rire sauvage et ce cri :\par
— Viens !\par
Il y avait dans le camp un profond silence d’attente.\par
Cimourdain reprit :\par
— A une condition.\par
— Laquelle ?\par
— Écoutez.\par
— Parle.\par
— Vous me haïssez ?\par
— Oui.\par
— Moi, je vous aime. Je suis votre frère.\par
La voix du haut de la tour répondit :\par
— Oui, Caïn.\par
 Cimourdain repartit avec une inflexion singulière, qui était à la fois haute et douce :\par
— Insultez, mais écoutez. Je viens ici en parlementaire. Oui, vous êtes mes frères. Vous êtes de pauvres hommes égarés. Je suis votre ami. Je suis la lumière et je parle à l’ignorance. La lumière contient toujours de la fraternité. D’ailleurs, est-ce que nous n’avons pas tous la même mère, la patrie ? Eh bien, écoutez-moi. Vous saurez plus tard, ou vos enfants sauront, ou les enfants de vos enfants sauront que tout ce qui se fait en ce moment se fait par l’accomplissement des lois d’en haut, et que ce qu’il y a dans la révolution, c’est Dieu. En attendant le moment où toutes les consciences, même les vôtres, comprendront, et où tous les fanatismes, même les nôtres, s’évanouiront, en attendant que cette grande clarté soit faite, personne n’aura-t-il pitié de vos ténèbres ? Je viens à vous, je vous offre ma tête ; je fais plus, je vous tends la main. Je vous demande la grâce de me perdre pour vous sauver. J’ai pleins pouvoirs, et ce que je dis je le puis. C’est un instant suprême ; je fais un dernier effort. Oui, celui qui vous parle est un citoyen, et dans ce citoyen, oui, il y a un prêtre. Le citoyen vous combat, mais le prêtre vous supplie. Écoutez-moi. Beaucoup d’entre vous ont des femmes et des enfants. Je prends la défense de vos enfants et de vos femmes. Je prends leur défense contre vous. O mes frères...\par
— Va, prêche ! ricana l’Imânus.\par
Cimourdain continua :\par
 — Mes frères, ne laissez pas sonner l’heure exécrable. On va ici s’entr’égorger. Beaucoup d’entre nous qui sommes ici devant vous ne verront pas le soleil de demain ; oui, beaucoup d’entre nous périront, et vous, vous tous, vous allez mourir. Faites-vous grâce à vous-mêmes. Pourquoi verser tout ce sang quand c’est inutile ? Pourquoi tuer tant d’hommes quand deux suffisent ?\par
— Deux ? dit l’Imânus.\par
— Oui. Deux.\par
— Qui ?\par
— Lantenac et moi.\par
Et Cimourdain éleva la voix :\par
— Deux hommes sont de trop, Lantenac pour nous, moi pour vous. Voici ce que je vous offre, et vous aurez tous la vie sauve : donnez-nous Lantenac, et prenez-moi. Lantenac sera guillotiné, et vous ferez de moi ce que vous voudrez.\par
— Prêtre, hurla l’Imânus, si nous t’avions, nous te brûlerions à petit feu.\par
— J’y consens, dit Cimourdain.\par
Et il reprit :\par
— Vous, les condamnés qui êtes dans cette tour, vous pouvez tous dans une heure être vivants et libres. Je vous apporte le salut. Acceptez-vous ?\par
L’Imânus éclata.\par
— Tu n’es pas seulement scélérat, tu es fou. Ah çà, pourquoi viens-tu nous déranger ? Qui est-ce qui te prie de venir nous parler ? Nous, livrer monseigneur ! Qu’est-ce que tu veux ?\par
 — Sa tête. Et je vous offre...\par
— Ta peau. Car nous t’écorcherions comme un chien, curé Cimourdain. Eh bien, non, ta peau ne vaut pas sa tête. Va-t’en.\par
— Cela va être horrible. Une dernière fois, réfléchissez.\par
La nuit venait pendant ces paroles sombres qu’on entendait au dedans de la tour comme au dehors. Le marquis de Lantenac se taisait et laissait faire. Les chefs ont de ces sinistres égoïsmes. C’est un des droits de la responsabilité.\par
L’Imânus jeta sa voix par-dessus Cimourdain, et cria :\par
— Hommes qui nous attaquez, nous vous avons dit nos propositions, elles sont faites, et nous n’avons rien à y changer. Acceptez-les, sinon, malheur ! Consentez-vous ? Nous vous rendrons les trois enfants qui sont là, et vous nous donnerez la sortie libre et la vie sauve, à tous.\par
— A tous, oui, répondit Cimourdain, excepté un.\par
— Lequel ?\par
— Lantenac.\par
— Monseigneur ! livrer monseigneur ! Jamais.\par
— Il nous faut Lantenac.\par
— Jamais.\par
— Nous ne pouvons traiter qu’à cette condition.\par
— Alors commencez.\par
Le silence se fit.\par
L’Imânus, après avoir sonné avec sa trompe le coup de signal, redescendit ; le marquis mit l’épée à la main ;  les dix-neuf assiégés se groupèrent en silence dans la salle basse, en arrière de la retirade, et se mirent à genoux ; ils entendaient le pas mesuré de la colonne d’attaque qui avançait vers la tour dans l’obscurité ; ce bruit se rapprochait ; tout à coup ils le sentirent tout près d’eux, à la bouche même de la brèche. Alors tous, agenouillés, épaulèrent à travers les fentes de la retirade leurs fusils et leurs espingoles, et l’un d’eux, Grand-Francœur, qui était le prêtre Turmeau, se leva, et, un sabre nu dans la main droite, un crucifix dans la main gauche, dit d’une voix grave :\par
— Au nom du Père, du Fils et du Saint-Esprit !\par
Tous firent feu à la fois, et la lutte s’engagea.
 \subsubsection[{IX. Titans contre géants}]{IX \\
Titans contre géants}\phantomsection
\label{p3l4c9}
\noindent Cela fut en effet épouvantable.\par
Ce corps à corps dépassa tout ce qu’on avait pu rêver.\par
Pour trouver quelque chose de pareil, il faudrait remonter aux grands duels d’Eschyle ou aux antiques tueries féodales ; à ces « \emph{attaques à armes courtes} » qui ont duré jusqu’au dix-septième siècle, quand on pénétrait dans les places fortes par les fausses brayes ; assauts tragiques, où, dit le vieux sergent de la province d’Alentejo, « les fourneaux ayant fait leur effet, les assiégeants s’avanceront portant des planches couvertes de lames de fer-blanc, armés de rondaches et de mantelets, et fournis de quantité de grenades, faisant abandonner les retranchements ou retirades à ceux de la place, et s’en rendront maîtres, poussant vigoureusement les assiégés. »\par
Le lieu d’attaque était horrible ; c’était une de ces brèches qu’on appelle en langue du métier \emph{brèches sous voûte,} c’est-à-dire, on se le rappelle, une crevasse traversant le mur de part en part et non une fracture évasée à ciel ouvert. La poudre avait agi comme une  vrille. L’effet de la mine avait été si violent que la tour avait été fendue par l’explosion à plus de quarante pieds au-dessus du fourneau, mais ce n’était qu’une lézarde, et la déchirure praticable qui servait de brèche et donnait entrée dans la salle basse ressemblait plutôt au coup de lance qui perce qu’au coup de hache qui entaille.\par
C’était une ponction au flanc de la tour, une longue fracture pénétrante, quelque chose comme un puits couché à terre, un couloir serpentant et montant comme un intestin à travers une muraille de quinze pieds d’épaisseur, on ne sait quel informe cylindre encombré d’obstacles, de pièges, d’explosions, où l’on se heurtait le front aux granits, les pieds aux gravats, les yeux aux ténèbres.\par
Les assaillants avaient devant eux ce porche noir, bouche de gouffre ayant pour mâchoires, en bas et en haut, toutes les pierres de la muraille déchiquetée ; une gueule de requin n’a pas plus de dents que cet arrachement effroyable. Il fallait entrer dans ce trou, et en sortir.\par
Dedans éclatait la mitraille, dehors se dressait la retirade. Dehors, c’est-à-dire dans la salle basse du rez-de-chaussée.\par
Les rencontres de sapeurs dans les galeries couvertes quand la contre-mine vient couper la mine, les boucheries à la hache sous les entre-ponts des vaisseaux qui s’abordent dans les batailles navales, ont seules cette férocité. Se battre au fond d’une fosse, c’est le dernier degré de l’horreur. Il est affreux de s’entretuer  avec un plafond sur la tête. Au moment où le premier flot des assiégeants entra, toute la retirade se couvrit d’éclairs, et ce fut quelque chose comme la foudre éclatant sous terre. Le tonnerre assaillant répliqua au tonnerre embusqué. Les détonations se ripostèrent ; le cri de Gauvain s’éleva : Fonçons ! Puis le cri de Lantenac : Faites ferme contre l’ennemi ! Puis le cri de l’imânus : A moi les Mainiaux ! Puis des cliquetis, sabres contre sabres, et, coup sur coup, d’effroyables décharges tuant tout. La torche accrochée au mur éclairait vaguement toute cette épouvante. Impossible de rien distinguer ; on était dans une noirceur rougeâtre ; qui entrait là était subitement sourd et aveugle, sourd du bruit, aveugle de la fumée. Les hommes mis hors de combat gisaient parmi les décombres, on marchait sur des cadavres, on écrasait des plaies, on broyait des membres cassés d’où sortaient des hurlements, on avait les pieds mordus par des mourants. Par instants, il y avait des silences plus hideux que le bruit. On se colletait, on entendait l’effrayant souffle des bouches, puis des grincements, des râles, des imprécations, et le tonnerre recommençait. Un ruisseau de sang sortait de la tour par la brèche, et se répandait dans l’ombre. Cette flaque sombre fumait dehors dans l’herbe.\par
On eût dit que c’était la tour elle-même qui saignait et que la géante était blessée.\par
Chose surprenante, cela ne faisait presque pas de bruit dehors. La nuit était très noire, et dans la plaine et dans la forêt il y avait autour de la forteresse attaquée  une sorte de paix funèbre. Dedans c’était l’enfer, dehors c’était le sépulcre. Ce choc d’hommes s’exterminant dans les ténèbres, ces mousqueteries, ces clameurs, ces rages, tout ce tumulte expirait sous la masse des murs et des voûtes, l’air manquait au bruit, et au carnage s’ajoutait l’étouffement. Hors de la tour cela s’entendait à peine. Les petits enfants dormaient pendant ce temps-là.\par
L’acharnement augmentait. La retirade tenait bon. Rien de plus malaisé à forcer que ce genre de barricade en chevron rentrant. Si les assiégés avaient contre eux le nombre, ils avaient pour eux la position. La colonne d’attaque perdait beaucoup de monde. Alignée et allongée dehors au pied de la tour, elle s’enfonçait lentement dans l’ouverture de la brèche, et se raccourcissait, comme une couleuvre qui entre dans son trou.\par
Gauvain, qui avait des imprudences de jeune chef, était dans la salle basse au plus fort de la mêlée, avec toute la mitraille autour de lui. Ajoutons qu’il avait la confiance de l’homme qui n’a jamais été blessé.\par
Comme il se retournait pour donner un ordre, une lueur de mousqueterie éclaira un visage tout près de lui.\par
— Cimourdain ! s’écria-t-il, qu’est-ce que vous venez faire ici ?\par
C’était Cimourdain en effet. Cimourdain répondit :\par
— Je viens être près de toi.\par
— Mais vous allez vous faire tuer !\par
— Hé bien, toi, qu’est-ce que tu fais donc ?\par
 — Mais je suis nécessaire ici. Vous pas.\par
— Puisque tu y es, il faut que j’y sois.\par
— Non, mon maître.\par
— Si, mon enfant !\par
Et Cimourdain resta près de Gauvain.\par
Les morts s’entassaient sur les pavés de la salle basse.\par
Bien que la retirade ne fût pas forcée encore, le nombre évidemment devait finir par vaincre. Les assaillants étaient à découvert et les assaillis étaient à l’abri, dix assiégeants tombaient contre un assiégé ; mais les assiégeants se renouvelaient. Les assiégeants croissaient et les assiégés décroissaient.\par
Les dix-neuf assiégés étaient tous derrière la retirade, l’attaque étant là. Ils avaient des morts et des blessés. Quinze tout au plus combattaient encore. Un des plus farouches, Chante-en-hiver, avait été affreusement mutilé. C’était un Breton trapu et crépu, de l’espèce petite et vivace. Il avait un œil crevé et la mâchoire brisée. Il pouvait encore marcher. Il se traîna dans l’escalier en spirale, et monta dans la chambre du premier étage, espérant pouvoir là prier et mourir.\par
Il s’était adossé au mur près de la meurtrière pour tâcher de respirer un peu.\par
En bas, la boucherie devant la retirade était de plus en plus horrible. Dans une intermittence, entre deux décharges, Cimourdain éleva la voix :\par
— Assiégés ! cria-t-il. Pourquoi faire couler le sang plus longtemps ? Vous êtes pris. Rendez-vous.  Songez que nous sommes quatre mille cinq cents contre dix-neuf, c’est-à-dire plus de deux cents contre un. Rendez-vous.\par
— Cessons ce marivaudage, répondit le marquis de Lantenac.\par
Et vingt balles ripostèrent à Cimourdain.\par
La retirade ne montait pas jusqu’à la voûte ; cela permettait aux assiégés de tirer par-dessus, mais cela permettait aux assiégeants de l’escalader.\par
— L’assaut à la retirade ! cria Gauvain. Y a-t-il quelqu’un de bonne volonté pour escalader la retirade ?\par
— Moi, dit le sergent Radoub.
 \subsubsection[{X. Radoub}]{X \\
Radoub}\phantomsection
\label{p3l4c10}
\noindent Ici les assaillants eurent une stupeur. Radoub était entré par le trou de brèche, à la tête de la colonne d’attaque, lui sixième, et, sur ces six hommes du bataillon parisien, quatre étaient déjà tombés. Après qu’il eut jeté ce cri : Moi ! on le vit, non avancer, mais reculer, et baissé, courbé, rampant presque entre les jambes des combattants, regagner l’ouverture de la brèche, et sortir. Était-ce une fuite ? Un tel homme fuir ? Qu’est-ce que cela voulait dire ?\par
Arrivé hors de la brèche, Radoub, encore aveuglé par la fumée, se frotta les yeux comme pour en ôter l’horreur et la nuit, et, à la lueur des étoiles, regarda la muraille de la tour. Il fit ce signe de tête satisfait qui veut dire : Je ne m’étais pas trompé.\par
Radoub avait remarqué que la lézarde profonde de l’explosion de la mine montait au-dessus de la brèche jusqu’à cette meurtrière du premier étage dont un boulet avait défoncé et disloqué l’armature de fer. Le réseau des barreaux rompus pendait à demi arraché, et un homme pouvait passer.\par
Un homme pouvait passer, mais un homme  pouvait-il monter ? Par la lézarde, oui ; à la condition d’être un chat.\par
C’est ce qu’était Radoub. Il était de cette race que Pindare appelle « les athlètes agiles ». On peut être vieux soldat et homme jeune ; Radoub, qui avait été garde-française, n’avait pas quarante ans. C’était un Hercule leste.\par
Radoub posa à terre son mousqueton, ôta sa buffleterie, quitta son habit et sa veste, et ne garda que ses deux pistolets qu’il mit dans la ceinture de son pantalon et son sabre nu qu’il prit entre ses dents. La crosse des deux pistolets passait au-dessus de sa ceinture.\par
Ainsi allégé de l’inutile, et suivi des yeux dans l’obscurité par tous ceux de la colonne d’attaque qui n’étaient pas encore entrés dans la brèche, il se mit à gravir les pierres de la lézarde du mur comme les marches d’un escalier. N’avoir pas de souliers lui fut utile ; rien ne grimpe comme un pied nu ; il crispait ses orteils dans les trous des pierres. Il se hissait avec ses poings et s’affermissait avec ses genoux. La montée était rude. C’était quelque chose comme une ascension le long des dents d’une scie. — Heureusement, pensait-il, qu’il n’y a personne dans la chambre du premier étage, car on ne me laisserait pas escalader ainsi.\par
Il n’avait pas moins de quarante pieds à gravir de cette façon. A mesure qu’il montait, un peu gêné par les pommeaux saillants de ses pistolets, la lézarde allait se rétrécissant, et l’ascension devenait de plus  en plus difficile. Le risque de la chute augmentait en même temps que la profondeur du précipice.\par
Enfin il parvint au rebord de la meurtrière ; il écarta le grillage tordu et descellé, il avait largement de quoi passer ; il se souleva d’un effort puissant, appuya son genou sur la corniche du rebord, saisit d’une main un tronçon de barreau à droite, de l’autre main un tronçon à gauche, et se dressa jusqu’à mi-corps devant l’embrasure de la meurtrière, le sabre aux dents, suspendu par ses deux poings sur l’abîme.\par
Il n’avait plus qu’une enjambée à faire pour sauter dans la salle du premier étage.\par
Mais une face apparut dans la meurtrière.\par
Radoub vit brusquement devant lui dans l’ombre quelque chose d’effroyable ; un œil crevé, une mâchoire fracassée, un masque sanglant.\par
Ce masque, qui n’avait plus qu’une prunelle, le regardait.\par
Ce masque avait deux mains ; ces deux mains sortirent de l’ombre et s’avancèrent vers Radoub ; l’une, d’une seule poignée, lui prit ses deux pistolets dans sa ceinture, l’autre lui ôta son sabre des dents.\par
Radoub était désarmé. Son genou glissait sur le plan incliné de la corniche, ses deux poings crispés aux tronçons du grillage suffisaient à peine à le soutenir, et il avait derrière lui quarante pieds de précipice.\par
Ce masque et ces mains, c’était Chante-en-hiver.\par
Chante-en-hiver, suffoqué par la fumée qui montait d’en bas, avait réussi à entrer dans l’embrasure de  la meurtrière, là l’air extérieur l’avait ranimé, la fraîcheur de la nuit avait figé son sang, et il avait repris un peu de force ; tout à coup il avait vu surgir au dehors devant l’ouverture le torse de Radoub ; alors, Radoub ayant les mains cramponnées aux barreaux et n’ayant que le choix de se laisser tomber ou de se laisser désarmer, Chante-en-hiver, épouvantable et tranquille, lui avait cueilli ses pistolets à sa ceinture et son sabre entre les dents.\par
Un duel inouï commença. Le duel du désarmé et du blessé.\par
Évidemment, le vainqueur c’était le mourant. Une balle suffisait pour jeter Radoub dans le gouffre béant sous ses pieds.\par
Par bonheur pour Radoub, Chante-en-hiver, ayant les deux pistolets dans une seule main, ne put en tirer un et fut forcé de se servir du sabre. Il porta un coup de pointe à l’épaule de Radoub. Ce coup de sabre blessa Radoub et le sauva.\par
Radoub, sans armes, mais ayant toute sa force, dédaigna sa blessure qui d’ailleurs n’avait pas entamé l’os, fit un soubresaut en avant, lâcha les barreaux et bondit dans l’embrasure.\par
Là il se trouva face à face avec Chante-en-hiver, qui avait jeté le sabre derrière lui et qui tenait les deux pistolets dans ses deux poings.\par
Chante-en-hiver, dressé sur ses genoux, ajusta Radoub presque à bout portant, mais son bras affaibli tremblait, et il ne tira pas tout de suite.\par
Radoub profita de ce répit pour éclater de rire.\par
 — Dis donc, cria-t-il, Vilain-à-voir ! est-ce que tu crois me faire peur avec ta gueule en bœuf à la mode ? Sapristi, comme on t’a délabré le minois !\par
Chante-en-hiver le visait.\par
Radoub continua :\par
— Ce n’est pas pour dire, mais tu as eu la gargoine joliment chiffonnée par la mitraille. Mon pauvre garçon, Bellone t’a fracassé la physionomie. Allons, allons, crache ton petit coup de pistolet, mon bonhomme.\par
Le coup partit et passa si près de la tête qu’il arracha à Radoub la moitié de l’oreille. Chante-en-hiver éleva l’autre bras armé du second pistolet, mais Radoub ne lui laissa pas le temps de viser.\par
— J’ai assez d’une oreille de moins, cria-t-il. Tu m’as blessé deux fois. A moi la belle !\par
Et il se rua sur Chante-en-hiver, lui rejeta le bras en l’air, fit partir le coup qui alla n’importe où, et lui saisit et lui mania sa mâchoire disloquée.\par
Chante-en-hiver poussa un rugissement et s’évanouit.\par
Radoub l’enjamba et le laissa dans l’embrasure.\par
— Maintenant que je t’ai fait savoir mon ultimatum, dit-il, ne bouge plus. Reste là, méchant traîne-à-terre. Tu penses bien que je ne vais pas à présent m’amuser à te massacrer. Rampe à ton aise sur le sol, concitoyen de mes savates. Meurs, c’est toujours ça de fait. C’est tout à l’heure que tu vas savoir que ton curé ne te disait que des bêtises. Va-t’en dans le grand mystère, paysan.\par
Et il sauta dans la salle du premier étage.\par
 — On n’y voit goutte, grommela-t-il.\par
Chante-en-hiver s’agitait convulsivement et hurlait à travers l’agonie. Radoub se retourna.\par
— Silence ! fais-moi le plaisir de te taire, citoyen sans le savoir. Je ne me mêle plus de ton affaire. Je méprise de t’achever. Fiche-moi la paix.\par
Et, inquiet, il fourra son poing dans ses cheveux, tout en considérant Chante-en-hiver.\par
— Ah çà, qu’est-ce que je vais faire ? c’est bon tout ça, mais me voilà désarmé. J’avais deux coups à tirer. Tu me les as gaspillés, animal ! Et avec ça une fumée qui vous fait aux yeux un mal de chien !\par
Et rencontrant son oreille déchirée :\par
— Aïe ! dit-il.\par
Et il reprit :\par
— Te voilà bien avancé de m’avoir confisqué une oreille ! Au fait, j’aime mieux avoir ça de moins qu’autre chose, ça n’est guère qu’un ornement. Tu m’as aussi égratigné à l’épaule, mais ce n’est rien. Expire, villageois, je te pardonne.\par
Il écouta. Le bruit dans la salle basse était effrayant. Le combat était plus forcené que jamais.\par
— Ça va bien en bas. C’est égal, ils gueulent vive le roi. Ils crèvent noblement.\par
Ses pieds cognèrent son sabre, à terre. Il le ramassa, et il dit à Chante-en-hiver qui ne bougeait plus et qui était peut-être mort :\par
— Vois-tu, homme des bois, pour ce que je voulais faire, mon sabre ou zut, c’est la même chose. Je le reprends par amitié. Mais il me fallait mes pistolets.  Que le diable t’emporte, sauvage ! Ah çà, qu’est-ce que je vais faire ? Je ne suis bon à rien ici.\par
Il avança dans la salle tâchant de voir et de s’orienter. Tout à coup, dans la pénombre, derrière le pilier du milieu, il aperçut une longue table, et sur cette table quelque chose qui brillait vaguement. Il tâta. C’étaient des tromblons, des pistolets, des carabines, une rangée d’armes à feu disposées en ordre et semblant n’attendre que des mains pour les saisir ; c’était la réserve de combat préparée par les assiégés pour la deuxième phase de l’assaut ; tout un arsenal.\par
— Un buffet ! s’écria Radoub.\par
Et il se jeta dessus, ébloui.\par
Alors il devint formidable.\par
La porte de l’escalier communiquant aux étages d’en haut et d’en bas était visible, toute grande ouverte, à côté de la table chargée d’armes. Radoub laissa tomber son sabre, prit dans ses deux mains deux pistolets à deux coups et les déchargea à la fois au hasard sous la porte dans la spirale de l’escalier, puis il saisit une espingole et la déchargea, puis il empoigna un tromblon gorgé de chevrotines et le déchargea. Le tromblon, vomissant quinze balles, sembla un coup de mitraille. Alors Radoub, reprenant haleine, cria d’une voix tonnante dans l’escalier : Vive Paris !\par
Et, s’emparant d’un deuxième tromblon plus gros que le premier, il le braqua sous la voûte tortueuse de la vis-de-Saint-Gilles, et attendit.\par
Le désarroi dans la salle basse fut indescriptible. Ces étonnements imprévus désagrégent la résistance.\par
 Deux des balles de la triple décharge de Radoub avaient porté ; l’une avait tué l’aîné des frères Pique-en-bois, l’autre avait tué Houzard, qui était M. de Quélen.\par
— Ils sont en haut ! cria le marquis.\par
Ce cri détermina l’abandon de la retirade, une volée d’oiseaux n’est pas plus vite en déroute, et ce fut à qui se précipiterait dans l’escalier. Le marquis encourageait cette fuite.\par
— Faites vite, disait-il. Le courage est d’échapper. Montons tous au deuxième étage ! Là nous recommencerons.\par
Il quitta la retirade le dernier.\par
Cette bravoure le sauva.\par
Radoub, embusqué au haut du premier étage de l’escalier, le doigt sur la détente du tromblon, guettait la déroute. Les premiers qui apparurent au tournant de la spirale reçurent la décharge en pleine face, et tombèrent foudroyés. Si le marquis en eût été, il était mort. Avant que Radoub eût eu le temps de saisir une nouvelle arme, les autres passèrent, le marquis après tous, et plus lent que les autres. Ils croyaient la chambre du premier pleine d’assiégeants, ils ne s’y arrêtèrent pas, et gagnèrent la salle du second étage, la chambre des miroirs. C’est là qu’était la porte de fer, c’est là qu’était la mèche soufrée, c’est là qu’il fallait capituler ou mourir.\par
Gauvain, aussi surpris qu’eux-mêmes des détonations de l’escalier et ne s’expliquant pas le secours qui lui arrivait, en avait profité sans chercher à  comprendre, avait sauté, lui et les siens, par-dessus la retirade, et avait poussé les assiégés l’épée aux reins jusqu’au premier étage.\par
Là il trouva Radoub.\par
Radoub commença par le salut militaire et dit :\par
— Une minute, mon commandant. C’est moi qui ai fait ça. Je me suis souvenu de Dol. J’ai fait comme vous. J’ai pris l’ennemi entre deux feux.\par
— Bon élève, dit Gauvain en souriant.\par
Quand on est un certain temps dans l’obscurité, les yeux finissent par se faire à l’ombre comme ceux des oiseaux de nuit ; Gauvain s’aperçut que Radoub était tout en sang.\par
— Mais tu es blessé, camarade !\par
— Ne faites pas attention, mon commandant. Qu’est-ce que c’est que ça, une oreille de plus ou de moins ? J’ai aussi un coup de sabre, je m’en fiche. Quand on casse un carreau, on s’y coupe toujours un peu. D’ailleurs il n’y a pas que de mon sang.\par
On fit une sorte de halte dans la salle du premier étage, conquise par Radoub. On apporta une lanterne. Cimourdain rejoignit Gauvain. Ils délibérèrent. Il y avait lieu à réfléchir en effet. Les assiégeants n’étaient pas dans le secret des assiégés ; ils ignoraient leur pénurie de munitions ; ils ne savaient pas que les défenseurs de la place étaient à court de poudre ; le deuxième étage était le dernier poste de résistance ; les assiégeants pouvaient croire l’escalier miné.\par
Ce qui était certain, c’est que l’ennemi ne pouvait échapper. Ceux qui n’étaient pas morts étaient là  comme sous clef. Lantenac était dans la souricière.\par
Avec cette certitude, on pouvait se donner un peu le temps de chercher le meilleur dénoûment possible. On avait déjà bien des morts. Il fallait tâcher de ne pas perdre trop de monde dans ce dernier assaut.\par
Le risque de cette suprême attaque serait grand. Il y aurait probablement un rude premier feu à essuyer.\par
Le combat était interrompu. Les assiégeants, maîtres du rez-de-chaussée et du premier étage, attendaient, pour continuer, le commandement du chef. Gauvain et Cimourdain tenaient conseil. Radoub assistait en silence à leur délibération.\par
Il hasarda un nouveau salut militaire, timide.\par
— Mon commandant ?\par
— Qu’est-ce, Radoub ?\par
— Ai-je droit à une petite récompense ?\par
— Certes. Demande ce que tu voudras.\par
— Je demande à monter le premier.\par
On ne pouvait le lui refuser. D’ailleurs il l’eût fait sans permission.
 \subsubsection[{XI. Les désespérés}]{XI \\
Les désespérés}\phantomsection
\label{p3l4c11}
\noindent Pendant qu’on délibérait au premier étage, on se barricadait au second. Le succès est une fureur, la défaite est une rage. Les deux étages allaient se heurter éperdument. Toucher à la victoire, c’est une ivresse. En bas il y avait l’espérance, qui serait la plus grande des forces humaines si le désespoir n’existait pas.\par
Le désespoir était en haut.\par
Un désespoir calme, froid, sinistre.\par
En arrivant à cette salle de refuge, au delà de laquelle il n’y avait rien pour eux, le premier soin des assiégés fut de barrer l’entrée. Fermer la porte était inutile, encombrer l’escalier valait mieux. En pareil cas, un obstacle à travers lequel on peut voir et combattre vaut mieux qu’une porte fermée.\par
La torche plantée dans la torchère du mur par l’Imânus près de la mèche soufrée les éclairait.\par
Il y avait dans cette salle du second un de ces gros et lourds coffres de chêne où l’on serrait les vêtements et le linge avant l’invention des meubles à tiroirs.\par
Ils traînèrent ce coffre et le dressèrent debout sous  la porte de l’escalier. Il s’y emboîtait solidement et bouchait l’entrée. Il ne laissait d’ouvert, près de la voûte, qu’un espace étroit, pouvant laisser passer un homme, excellent pour tuer les assaillants un à un. Il était douteux qu’on s’y risquât.\par
L’entrée obstruée leur donnait un répit.\par
Ils se comptèrent.\par
Les dix-neuf n’étaient plus que sept, dont l’Imânus. Excepté l’Imânus et le marquis, tous étaient blessés.\par
Les cinq, qui étaient blessés, mais très vivants, car, dans la chaleur du combat, toute blessure qui n’est pas mortelle vous laisse aller et venir, étaient Chatenay, dit Robi, Guinoiseau, Hoisnard Branche-d’Or, Brin-d’Amour et Grand-Francœur. Tout le reste était mort.\par
Ils n’avaient plus de munitions. Les gibernes étaient épuisées. Ils comptèrent les cartouches. Combien, à eux sept, avaient-ils de coups à tirer ? Quatre.\par
On était arrivé à ce moment où il n’y a plus qu’à tomber. On était acculé à l’escarpement, béant et terrible. Il était difficile d’être plus près du bord.\par
Cependant l’attaque venait de recommencer ; mais lente et d’autant plus sûre. On entendait les coups de crosse des assiégeants sondant l’escalier marche à marche.\par
Nul moyen de fuir. Par la bibliothèque ? Il y avait là sur le plateau six canons braqués, mèche allumée. Par les chambres d’en haut ? A quoi bon ? Elles aboutissaient à la plate-forme. Là on trouvait la ressource de se jeter du haut en bas de la tour.\par
Les sept survivants de cette bande épique se  voyaient inexorablement enfermés et saisis par cette épaisse muraille qui les protégeait et qui les livrait. Ils n’étaient pas encore pris ; mais ils étaient déjà prisonniers.\par
Le marquis éleva la voix :\par
— Mes amis, tout est fini.\par
Et après un silence il ajouta :\par
— Grand-Francœur redevient l’abbé Turmeau.\par
Tous s’agenouillèrent, le rosaire à la main. Les coups de crosse des assaillants se rapprochaient.\par
Grand-Francœur, tout sanglant d’une balle qui lui avait effleuré le crâne et arraché le cuir chevelu, dressa de la main droite son crucifix. Le marquis, sceptique au fond, mit un genou en terre.\par
— Que chacun, dit Grand-Francœur, confesse ses fautes à haute voix. Monseigneur, parlez.\par
Le marquis répondit :\par
— J’ai tué.\par
— J’ai tué, dit Hoisnard.\par
— J’ai tué, dit Guinoiseau.\par
— J’ai tué, dit Brin-d’Amour.\par
— J’ai tué, dit Châtenay.\par
— J’ai tué, dit l’Imânus.\par
Et Grand-Francœur reprit :\par
— Au nom de la Très-Sainte-Trinité, je vous absous. Que vos âmes aillent en paix.\par
— Ainsi soit-il, répondirent toutes les voix.\par
Le marquis se releva.\par
— Maintenant, dit-il, mourons.\par
— Et tuons, dit l’Imânus.\par
 Les coups de crosse commençaient à ébranler le coffre qui barrait la porte.\par
— Pensez à Dieu, dit le prêtre. La terre n’existe plus pour vous.\par
— Oui, reprit le marquis, nous sommes dans la tombe.\par
Tous courbèrent le front et se frappèrent la poitrine. Le marquis seul et le prêtre étaient debout. Les yeux étaient fixés à terre, le prêtre priait, les paysans priaient, le marquis songeait. Le coffre, battu comme par des marteaux, sonnait lugubrement.\par
En ce moment une voix vive et forte, éclatant brusquement derrière eux, cria :\par
— Je vous l’avais bien dit, monseigneur !\par
Toutes les têtes se retournèrent, stupéfaites.\par
Un trou venait de s’ouvrir dans le mur.\par
Une pierre, parfaitement rejointoyée avec les autres, mais non cimentée, et ayant un piton en haut et un piton en bas, venait de pivoter sur elle-même à la façon des tourniquets, et en tournant avait ouvert la muraille. La pierre ayant évolué sur son axe, l’ouverture était double et offrait deux passages, l’un à droite, l’autre à gauche, étroits, mais suffisants pour laisser passer un homme. Au delà de cette porte inattendue on apercevait les premières marches d’un escalier en spirale. Une face d’homme apparaissait à l’ouverture.\par
Le marquis reconnut Halmalo.
 \subsubsection[{XII. Sauveur}]{XII \\
Sauveur}\phantomsection
\label{p3l4c12}
\noindent — C’est toi, Halmalo !\par
— Moi, monseigneur. Vous voyez que les pierres qui tournent, cela existe, et qu’on peut sortir d’ici. J’arrive à temps, mais faites vite. Dans dix minutes, vous serez en pleine forêt.\par
— Dieu est grand, dit le prêtre.\par
— Sauvez-vous, monseigneur, crièrent toutes les voix.\par
— Vous tous d’abord, dit le marquis.\par
— Vous le premier, monseigneur, dit l’abbé Turmeau.\par
— Moi le dernier.\par
Et le marquis reprit d’une voix sévère :\par
— Pas de combat de générosité. Nous n’avons pas le temps d’être magnanimes. Vous êtes blessés. Je vous ordonne de vivre et de fuir. Vite ! et profitez de cette issue. Merci, Halmalo.\par
— Monsieur le marquis, dit l’abbé Turmeau, nous allons nous séparer ?\par
— En bas ; sans doute. On ne s’échappe jamais qu’un à un.\par
— Monseigneur nous assigne-t-il un rendez-vous ?\par
 — Oui. Une clairière dans la forêt. La Pierre-Gauvaine. Connaissez-vous l’endroit ?\par
— Nous le connaissons tous.\par
— J’y serai demain. A midi. Que tous ceux qui pourront marcher s’y trouvent.\par
— On y sera.\par
— Et nous recommencerons la guerre, dit le marquis.\par
Cependant Halmalo, en pesant sur la pierre tournante, venait de s’apercevoir qu’elle ne bougeait plus. L’ouverture ne pouvait plus se clore.\par
— Monseigneur, dit-il, dépêchons-nous. La pierre résiste à présent. J’ai pu ouvrir le passage, mais je ne pourrai le fermer.\par
La pierre, en effet, après une longue désuétude, était comme ankylosée dans sa charnière. Impossible désormais de lui imprimer un mouvement.\par
— Monseigneur, reprit Halmalo, j’espérais refermer le passage, et que les bleus, quand ils entreraient, ne trouveraient plus personne, et n’y comprendraient rien, et vous croiraient en allés en fumée. Mais voilà la pierre qui ne veut pas. L’ennemi verra la sortie ouverte et pourra poursuivre. Au moins ne perdons pas une minute. Vite, tous dans l’escalier.\par
L’Imânus posa la main sur l’épaule de Halmalo.\par
— Camarade, combien de temps faut-il pour qu’on sorte par cette passe et qu’on soit en sûreté dans la forêt ?\par
— Personne n’est blessé grièvement ? demanda Halmalo.\par
 Ils répondirent : — Personne.\par
— En ce cas, un quart d’heure suffit.\par
— Ainsi, repartit l’Imânus, si l’ennemi n’entrait ici que dans un quart d’heure ?...\par
— Il pourrait nous poursuivre, il ne nous atteindrait pas.\par
— Mais, dit le marquis, ils seront ici dans cinq minutes, ce vieux coffre n’est pas pour les gêner longtemps. Quelques coups de crosse en viendront à bout. Un quart d’heure ! qui est-ce qui les arrêtera un quart d’heure ?\par
— Moi, dit l’Imânus.\par
— Toi, Gouge-le-Bruant ?\par
— Moi, monseigneur. Écoutez. Sur six, vous êtes cinq blessés. Moi, je n’ai pas une égratignure.\par
— Ni moi, dit le marquis.\par
— Vous êtes le chef, monseigneur. Je suis le soldat. Le chef et le soldat, c’est deux.\par
— Je le sais, nous avons chacun un devoir différent.\par
— Non, monseigneur, nous avons, vous et moi, le même devoir, qui est de vous sauver.\par
L’Imânus se tourna vers ses camarades.\par
— Camarades, il s’agit de tenir en l’échec ennemi et de retarder la poursuite le plus possible. Écoutez. J’ai toute ma force, je n’ai pas perdu une goutte de sang ; n’étant pas blessé, je durerai plus longtemps qu’un autre. Partez tous. Laissez-moi vos armes. J’en ferai bon usage. Je me charge d’arrêter l’ennemi une bonne demi-heure. Combien y a-t-il de pistolets chargés ?\par
 — Quatre.\par
— Mettez-les là, à terre.\par
On fit ce qu’il voulait.\par
— C’est bien. Je reste. Ils trouveront à qui parler. Maintenant, vite, allez-vous-en.\par
Les situations à pic suppriment les remerciements. A peine prit-on le temps de lui serrer la main.\par
— A bientôt, lui dit le marquis.\par
— Non, monseigneur. J’espère que non. Pas à bientôt ; car je vais mourir.\par
Tous s’engagèrent l’un après l’autre dans l’étroit escalier, les blessés d’abord. Pendant qu’ils descendaient, le marquis prit le crayon de son carnet de poche, et écrivit quelques mots sur la pierre qui ne pouvait plus tourner et qui laissait le passage béant.\par
— Venez, monseigneur, il n’y a plus que vous, dit Halmalo.\par
Et Halmalo commença à descendre.\par
Le marquis le suivit.\par
L’Imânus resta seul.
 \subsubsection[{XIII. Bourreau}]{XIII \\
Bourreau}\phantomsection
\label{p3l4c13}
\noindent Les quatre pistolets chargés avaient été posés sur les dalles, car cette salle n’avait pas de plancher. L’Imânus en prit deux, un dans chaque main.\par
Il s’avança obliquement vers l’entrée de l’escalier que le coffre obstruait et masquait.\par
Les assaillants craignaient évidemment quelque surprise, une de ces explosions finales qui sont la catastrophe du vainqueur en même temps que celle du vaincu. Autant la première attaque avait été impétueuse, autant la dernière était lente et prudente. Ils n’avaient pas pu, ils n’avaient pas voulu peut-être enfoncer violemment le coffre ; ils en avaient démoli le fond à coups de crosse, et troué le couvercle à coups de bayonnette, et par ces trous ils tâchaient de voir dans la salle avant de se risquer à y pénétrer.\par
La lueur des lanternes dont ils éclairaient l’escalier passait à travers ces trous.\par
L’Imânus aperçut à un de ces trous une de ces prunelles qui regardaient. Il ajusta brusquement à ce trou le canon d’un de ses pistolets et pressa la détente. Le coup partit, et l’Imânus, joyeux, entendit  un cri horrible. La balle avait crevé l’œil et traversé la tête, et le soldat qui regardait venait de tomber dans l’escalier à la renverse.\par
Les assaillants avaient entamé assez largement le bas du couvercle en deux endroits, et y avaient pratiqué deux espèces de meurtrières ; l’Imânus profita de l’une de ces entailles, y passa le bras, et lâcha au hasard dans le tas des assiégeants son deuxième coup de pistolet. La balle ricocha probablement, car on entendit plusieurs cris, comme si trois ou quatre étaient tués ou blessés, et il se fit dans l’escalier un grand tumulte d’hommes qui lâchent pied et qui reculent.\par
L’Imânus jeta les deux pistolets qu’il venait de décharger, et prit les deux qui restaient ; puis, les deux pistolets à ses deux poings, il regarda par les trous du coffre.\par
Il constata le premier effet produit.\par
Les assaillants avaient redescendu l’escalier. Des mourants se tordaient sur les marches ; le tournant de la spirale ne laissait voir que trois ou quatre degrés.\par
L’Imânus attendit.\par
— C’est du temps de gagné, pensait-il.\par
Cependant il vit un homme, à plat ventre, monter en rampant les marches de l’escalier, et en même temps, plus bas, une tête de soldat apparut derrière le pilier central de la spirale. L’Imânus visa cette tête et tira. Il y eut un cri, le soldat tomba, et l’Imânus fit passer de sa main gauche dans sa main droite le dernier pistolet chargé qui lui restait.\par
En ce moment-là il sentit une affreuse douleur, et  ce fut lui qui, à son tour, jeta un hurlement. Un sabre lui fouillait les entrailles. Un poing, le poing de l’homme qui rampait, venait de passer à travers la deuxième meurtrière du bas du coffre, et ce poing avait plongé un sabre dans le ventre de l’Imânus.\par
La blessure était effroyable. Le ventre était fendu de part en part.\par
L’Imânus ne tomba pas. Il grinça des dents, et dit : C’est bon !\par
Puis, chancelant et se traînant, il recula jusqu’à la torche qui brûlait à côté de la porte de fer, il posa son pistolet à terre et empoigna la torche, et, soutenant de la main gauche ses intestins qui sortaient, de la main droite il abaissa la torche et mit le feu à la mèche soufrée.\par
Le feu prit, la mèche flamba. L’Imânus lâcha la torche, qui continua de brûler à terre, ressaisit son pistolet, et, tombé sur la dalle, mais se soulevant encore, attisa la mèche du peu de souffle qui lui restait.\par
La flamme courut, passa sous la porte de fer et gagna le pont-châtelet.\par
Alors, voyant son exécrable réussite, plus satisfait peut-être de son crime que de sa vertu, cet homme qui venait d’être un héros et qui n’était plus qu’un assassin, et qui allait mourir, sourit.\par
— Ils se souviendront de moi, murmura-t-il. Je venge, sur leurs petits, notre petit à nous, le roi qui est au Temple.
 \subsubsection[{XIV. L’Imanus aussi s’évade}]{XIV \\
L’Imanus aussi s’évade}\phantomsection
\label{p3l4c14}
\noindent En cet instant-là, un grand bruit se fit, le coffre violemment poussé s’effondra, et livra passage à un homme qui se rua dans la salle, le sabre à la main.\par
— C’est moi, Radoub. Qui en veut ? Ça m’ennuie d’attendre. Je me risque. C’est égal, je viens toujours d’en éventrer un. Maintenant je vous attaque tous. Qu’on me suive ou qu’on ne me suive pas, me voilà. Combien êtes-vous ?\par
C’était Radoub, en effet, et il était seul. Après le massacre que l’Imânus venait de faire dans l’escalier, Gauvain, redoutant quelque fougasse masquée, avait fait replier ses hommes et se concertait avec Cimourdain.\par
Radoub, le sabre à la main sur le seuil, dans cette obscurité où la torche presque éteinte jetait à peine une lueur, répéta sa question :\par
— Je suis un. Combien êtes-vous ?\par
N’entendant rien, il avança. Un de ces jets de clarté qu’exhalent par instants les foyers agonisants et qu’on pourrait appeler des sanglots de lumière, jaillit de la torche et illumina toute la salle.\par
Radoub avisa un des petits miroirs accrochés au  mur, s’en approcha, regarda sa face ensanglantée et son oreille pendante, et dit :\par
— Démantibulage hideux.\par
Puis il se retourna, stupéfait de voir la salle vide.\par
— Il n’y a personne ! s’écria-t-il. Zéro d’effectif.\par
Il aperçut la pierre qui avait tourné, l’ouverture et l’escalier.\par
— Ah ! je comprends. Clef des champs. Venez donc tous ! camarades, venez ! ils s’en sont allés. Ils ont filé, fusé, fouiné, fichu le camp. Cette cruche de vieille tour était fêlée. Voici le trou par où ils ont passé, canailles ! Comment veut-on qu’on vienne à bout de Pitt et Cobourg avec des farces comme ça ! C’est le bon Dieu du diable qui est venu à leur secours ! Il n’y a plus personne !\par
Un coup de pistolet partit, une balle lui effleura le coude et s’aplatit contre le mur.\par
— Mais si ! il y a quelqu’un. Qui est-ce qui a la bonté de me faire cette politesse ?\par
— Moi, dit une voix.\par
Radoub avança la tête et distingua dans le clair-obscur quelque chose qui était l’Imânus.\par
— Ah ! cria-t-il. J’en tiens un. Les autres se sont échappés, mais toi, tu n’échapperas pas.\par
— Crois-tu ? répondit l’Imânus.\par
Radoub fit un pas et s’arrêta.\par
— Hé, l’homme qui es par terre, qui es-tu ?\par
— Je suis celui qui est par terre et qui se moque de ceux qui sont debout.\par
— Qu’est-ce que tu as dans ta main droite ?\par
 — Un pistolet.\par
— Et dans ta main gauche ?\par
— Mes boyaux.\par
— Je te fais prisonnier.\par
— Je t’en défie.\par
Et l’Imânus, se penchant sur la mèche en combustion, soufflant son dernier soupir sur l’incendie, expira.\par
Quelques instants après, Gauvain et Cimourdain, et tous, étaient dans la salle. Tous virent l’ouverture. On fouilla les recoins, on sonda l’escalier ; il aboutissait à une sortie dans le ravin. On constata l’évasion. On secoua l’Imânus, il était mort. Gauvain, une lanterne à la main, examina la pierre qui avait donné issue aux assiégés ; il avait entendu parler de cette pierre tournante, mais lui aussi tenait cette légende pour une fable. Tout en considérant la pierre, il aperçut quelque chose qui était écrit au crayon ; il approcha la lanterne et lut ceci :\par
— \emph{Au revoir, monsieur le vicomte.} — \par

\byline{L{\scshape antenac}}
\noindent Guéchamp avait rejoint Gauvain. La poursuite était évidemment inutile, la fuite était consommée et complète, les évadés avaient pour eux tout le pays, le buisson, le ravin, le taillis, l’habitant ; ils étaient sans doute déjà bien loin ; nul moyen de les retrouver ; et la forêt de Fougères tout entière était une immense cachette. Que faire ? Tout était à recommencer. Gauvain et Guéchamp échangeaient leurs désappointements et leurs conjectures.\par
 Cimourdain écoutait, grave, sans dire une parole.\par
— A propos, Guéchamp, dit Gauvain, et l’échelle ?\par
— Commandant, elle n’est pas arrivée.\par
— Mais pourtant nous avons vu venir une voiture escortée par des gendarmes.\par
Guéchamp répondit :\par
— Elle n’apportait pas l’échelle.\par
— Qu’est-ce donc qu’elle apportait ?\par
— La guillotine, dit Cimourdain.
 \subsubsection[{XV. Ne pas mettre dans la même poche une montre et une clef}]{XV \\
Ne pas mettre dans la même poche une montre et une clef}\phantomsection
\label{p3l4c15}
\noindent Le marquis de Lantenac n’était pas si loin qu’ils le croyaient.\par
Il n’en était pas moins entièrement en sûreté et hors de leur atteinte.\par
Il avait suivi Halmalo.\par
L’escalier par où Halmalo et lui étaient descendus, à la suite des autres fugitifs, se terminait tout près du ravin et des arches du pont par un étroit couloir voûté. Ce couloir s’ouvrait sur une profonde fissure naturelle du sol qui d’un côté aboutissait au ravin, et de l’autre à la forêt. Cette fissure, absolument dérobée aux regards, serpentait sous des végétations impénétrables. Impossible de reprendre là un homme. Un évadé, une fois parvenu dans cette fissure, n’avait plus qu’à faire une fuite de couleuvre, et était introuvable. L’entrée du couloir secret de l’escalier était tellement obstruée de ronces que les constructeurs du passage souterrain avaient considéré comme inutile de la fermer autrement.\par
Le marquis n’avait plus maintenant qu’à s’en aller.  Il n’avait pas à s’inquiéter d’un déguisement. Depuis son arrivée en Bretagne, il n’avait pas quitté ses habits de paysan, se jugeant plus grand seigneur ainsi.\par
Il s’était borné à ôter son épée, dont il avait débouclé et jeté le ceinturon.\par
Quand Halmalo et le marquis débouchèrent du couloir dans la fissure, les cinq autres, Guinoiseau, Hoisnard Branche-d’Or, Brin-d’Amour, Chatenay et l’abbé Turmeau n’y étaient déjà plus.\par
— Ils n’ont pas été longtemps à prendre leur volée, dit Halmalo.\par
— Fais comme eux, dit le marquis.\par
— Monseigneur veut que je le quitte ?\par
— Sans doute. Je te l’ai dit déjà. On ne s’évade bien que seul. Où un passe, deux ne passent pas. Ensemble nous appellerions l’attention. Tu me ferais prendre et je te ferais prendre.\par
— Monseigneur connaît le pays ?\par
— Oui.\par
— Monseigneur maintient le rendez-vous à la Pierre-Gauvaine ?\par
— Demain. A midi.\par
— J’y serai. Nous y serons.\par
Halmalo s’interrompit.\par
— Ah ! monseigneur, quand je pense que nous avons été en pleine mer, que nous étions seuls, que je voulais vous tuer, que vous étiez mon seigneur, que vous pouviez me le dire, et que vous ne me l’avez pas dit ! Quel homme vous êtes !\par
Le marquis reprit :\par
 — L’Angleterre. Il n’y a plus d’autre ressource. Il faut que dans quinze jours les Anglais soient en France.\par
— J’aurai bien des comptes à rendre à monseigneur. J’ai fait ses commissions.\par
— Nous parlerons de tout cela demain.\par
— A demain, monseigneur.\par
— A propos, as-tu faim ?\par
— Peut-être, monseigneur. J’étais si pressé d’arriver que je ne sais pas si j’ai mangé aujourd’hui.\par
Le marquis tira de sa poche une tablette de chocolat, la cassa en deux, en donna une moitié à Halmalo et se mit à manger l’autre.\par
— Monseigneur, dit Halmalo, à votre droite, c’est le ravin ; à votre gauche, c’est la forêt.\par
— C’est bien. Laisse-moi. Va de ton côté.\par
Halmalo obéit. Il s’enfonça dans l’obscurité. On entendit un bruit de broussailles froissées, puis plus rien. Au bout de quelques secondes il eût été impossible de ressaisir sa trace. Cette terre du Bocage, hérissée et inextricable, était l’auxiliaire du fugitif. On ne disparaissait pas, on s’évanouissait. C’est cette facilité des dispersions rapides qui faisait hésiter nos armées devant cette Vendée toujours reculante, et devant ses combattants si formidablement fuyards.\par
Le marquis demeura immobile. Il était de ces hommes qui s’efforcent de ne rien éprouver ; mais il ne put se soustraire à l’émotion de respirer l’air libre après avoir respiré tant de sang et de carnage. Se sentir complètement sauvé après avoir été complètement  perdu ; après la tombe vue de si près, prendre possession de la pleine sécurité ; sortir de la mort et rentrer dans la vie, c’était là, même pour un homme comme Lantenac, une secousse ; et, bien qu’il en eût déjà traversé de pareilles, il ne put soustraire son âme imperturbable à un ébranlement de quelques instants. Il s’avoua à lui-même qu’il était content. Il dompta vite ce mouvement qui ressemblait presque à de la joie.\par
Il tira sa montre, et la fit sonner. Quelle heure était-il ?\par
A son grand étonnement, il n’était que dix heures. Quand on vient de subir une de ces péripéties de la vie humaine où tout a été mis en question, on est toujours stupéfait que des minutes si pleines ne soient pas plus longues que les autres. Le coup de canon d’avertissement avait été tiré un peu avant le coucher du soleil, et la Tourgue avait été abordée par la colonne d’attaque une demi-heure après, entre sept et huit heures, à la nuit tombante. Ainsi, ce colossal combat, commencé à huit heures, était fini à dix. Toute cette épopée avait duré cent vingt minutes. Quelquefois une rapidité d’éclair est mêlée aux catastrophes. Les événements ont des raccourcis surprenants.\par
En y réfléchissant, c’est le contraire qui eût pu étonner ; une résistance de deux heures d’un si petit nombre contre un si grand nombre était extraordinaire, et certes elle n’avait pas été courte, ni tout de suite finie, cette bataille de dix-neuf contre quatre mille.\par
 Cependant il était temps de s’en aller, Halmalo devait être loin, et le marquis jugea qu’il n’était pas nécessaire de rester là plus longtemps. Il remit sa montre dans sa veste ; non dans la même poche, car il venait de remarquer qu’elle y était en contact avec la clef de la porte de fer que lui avait rapportée l’Imânus, et que le verre de sa montre pouvait se briser contre cette clef ; et il se disposa à gagner à son tour la forêt.\par
Comme il allait prendre à gauche, il lui sembla qu’une sorte de rayon vague pénétrait jusqu’à lui.\par
Il se retourna, et, à travers les broussailles nettement découpées sur un fond rouge et devenues tout à coup visibles dans leurs moindres détails, il aperçut une grande lueur dans le ravin. Il y marcha, puis se ravisa, trouvant inutile de s’exposer à cette clarté, quelle qu’elle fût ; ce n’était pas son affaire après tout ; il reprit la direction que lui avait montrée Halmalo et fit quelques pas vers la forêt.\par
Tout à coup, profondément enfoui et caché sous les ronces, il entendit sur sa tête un cri terrible ; ce cri semblait partir du rebord même du plateau au-dessus du ravin. Le marquis leva les yeux et s’arrêta.\par
  \subsection[{Livre cinquième. In dæmone Deus}]{Livre cinquième \\
In dæmone Deus}\phantomsection
\label{p3l5}
\subsubsection[{I. Trouvés, mais perdus}]{I \\
Trouvés, mais perdus}\phantomsection
\label{p3l5c1}
\noindent Au moment où Michelle Fléchard avait aperçu la tour rougie par le soleil couchant, elle en était à plus d’une lieue. Elle qui pouvait à peine faire un pas, elle n’avait point hésité devant cette lieue à faire. Les femmes sont faibles, mais les mères sont fortes. Elle avait marché.\par
Le soleil s’était couché ; le crépuscule était venu, puis l’obscurité profonde ; elle avait entendu, marchant toujours, sonner au loin, à un clocher qu’on ne voyait pas, huit heures, puis neuf heures. Ce clocher était probablement celui de Parigné. De temps en temps elle s’arrêtait pour écouter des espèces de coups sourds qui étaient peut-être un des fracas vagues de la nuit.\par
 Elle avançait droit devant elle, cassant les ajoncs et les landes aiguës sous ses pieds sanglants. Elle était guidée par une faible clarté qui se dégageait du donjon lointain, le faisait saillir, et donnait dans l’ombre à cette tour un rayonnement mystérieux. Cette clarté devenait plus vive quand les coups devenaient plus distincts, puis elle s’effaçait.\par
Le vaste plateau où cheminait Michelle Fléchard n’était qu’herbe et bruyère, sans une maison ni un arbre ; il s’élevait insensiblement, et, à perte de vue, appuyait sa longue ligne droite et dure sur le sombre horizon étoilé. Ce qui la soutint dans cette montée, c’est qu’elle avait toujours la tour sous les yeux.\par
Elle la voyait grandir lentement.\par
Les détonations étouffées et les lueurs pâles qui sortaient de la tour avaient, nous venons de le dire, des intermittences ; elles s’interrompaient, puis reprenaient, proposant on ne sait quelle poignante énigme à la misérable mère en détresse.\par
Brusquement elles cessèrent ; tout s’éteignit, bruit et clarté ; il y eut un moment de plein silence, une sorte de paix lugubre se fit.\par
C’est en cet instant là que Michelle Fléchard arriva au bord du plateau.\par
Elle aperçut à ses pieds un ravin dont le fond se perdait dans une blême épaisseur de nuit ; à quelque distance, sur le haut du plateau, un enchevêtrement de roues, de talus et d’embrasures qui était une batterie de canons ; et, devant elle, confusément éclairé par les mèches allumées de la batterie, un énorme  édifice qui semblait bâti avec des ténèbres plus noires que toutes les autres ténèbres qui l’entouraient.\par
Cet édifice se composait d’un pont dont les arches plongeaient dans le ravin, et d’une sorte de château qui s’élevait sur le pont, et le château et le pont s’appuyaient à une haute rondeur obscure, qui était la tour vers laquelle cette mère avait marché de si loin.\par
On voyait des clartés aller et venir aux lucarnes de la tour, et, à une rumeur qui en sortait, on la devinait pleine d’une foule d’hommes dont quelques silhouettes débordaient en haut jusque sur la plate-forme.\par
Il y avait près de la batterie un campement dont Michelle Fléchard distinguait les vedettes ; mais, dans l’obscurité et dans les broussailles, elle n’en avait pas été aperçue.\par
Elle était parvenue au bord du plateau, si près du pont qu’il lui semblait presque qu’elle y pouvait toucher avec la main. La profondeur du ravin l’en séparait. Elle distinguait dans l’ombre les trois étages du château du pont.\par
Elle resta un temps quelconque, car les mesures du temps s’effaçaient dans son esprit, absorbée et muette devant ce ravin béant et cette bâtisse ténébreuse. Qu’était-ce que cela ? Que se passait-il là ? Était-ce la Tourgue ? Elle avait le vertige d’on ne sait quelle attente qui ressemblait à l’arrivée et au départ. Elle se demandait pourquoi elle était là.\par
Elle regardait, elle écoutait.\par
Subitement elle ne vit plus rien.\par
Un voile de fumée venait de monter entre elle et ce  qu’elle regardait. Une âcre cuisson lui fit fermer les yeux. A peine avait-elle clos les paupières qu’elles s’empourprèrent et devinrent lumineuses. Elle les rouvrit.\par
Ce n’était plus la nuit qu’elle avait devant elle, c’était le jour ; mais une espèce de jour funeste, le jour qui sort du feu. Elle avait sous les yeux un commencement d’incendie.\par
La fumée de noire était devenue écarlate, et une grande flamme était dedans ; cette flamme apparaissait, puis disparaissait, avec ces torsions farouches qu’ont les éclairs et les serpents.\par
Cette flamme sortait comme une langue de quelque chose qui ressemblait à une gueule et qui était une fenêtre pleine de feu. Cette fenêtre, grillée de barreaux de fer déjà rouges, était une des croisées de l’étage inférieur du château construit sur le pont. De tout l’édifice on n’apercevait que cette fenêtre. La fumée couvrait tout, même le plateau, et l’on ne distinguait que le bord du ravin, noir sur la flamme vermeille.\par
Michelle Fléchard, étonnée, regardait. La fumée est nuage, le nuage est rêve ; elle ne savait plus ce qu’elle voyait. Devait-elle fuir ? Devait-elle rester ? Elle se sentait presque hors du réel.\par
Un souffle de vent passa et fendit le rideau de fumée, et dans la déchirure la tragique bastille, soudainement démasquée, se dressa visible tout entière, donjon, pont, châtelet, éblouissante, horrible, avec la magnifique dorure de l’incendie, réverbéré sur elle de haut en bas. Michelle Fléchard put tout voir dans la netteté sinistre du feu.\par
 L’étage inférieur du château bâti sur le pont brûlait.\par
Au-dessus on distinguait les deux autres étages encore intacts, mais comme portés par une corbeille de flammes. Du rebord du plateau, où était Michelle Fléchard, on en voyait vaguement l’intérieur à travers des interpositions de feu et de fumée. Toutes les fenêtres étaient ouvertes.\par
Par les fenêtres du second étage, qui étaient très grandes, Michelle Fléchard apercevait, le long des murs, des armoires qui lui semblaient pleines de livres, et, devant une des croisées, à terre, dans la pénombre, un petit groupe confus, quelque chose qui avait l’aspect indistinct et amoncelé d’un nid ou d’une couvée, et qui lui faisait l’effet de remuer par moments.\par
Elle regardait cela.\par
Qu’était-ce que ce petit groupe d’ombre ?\par
A de certains instants, il lui venait à l’esprit que cela ressemblait à des formes vivantes ; elle avait la fièvre, elle n’avait pas mangé depuis le matin, elle avait marché sans relâche, elle était exténuée, elle se sentait dans une sorte d’hallucination dont elle se défiait instinctivement ; pourtant ses yeux de plus en plus fixes ne pouvaient se détacher de cet obscur entassement d’objets quelconques, inanimés probablement, et en apparence inertes, qui gisait là sur le parquet de cette salle superposée à l’incendie.\par
Tout à coup le feu, comme s’il avait une volonté, allongea d’en bas un de ses jets vers le grand lierre mort qui couvrait précisément cette façade que Michelle Fléchard regardait. On eût dit que la flamme  venait de découvrir ce réseau de branches sèches ; une étincelle s’en empara avidement, et se mit à monter le long des sarments avec l’agilité affreuse des traînées de poudre. En un clin d’œil la flamme atteignit le second étage. Alors, d’en haut, elle éclaira l’intérieur du premier. Une vive lueur mit subitement en relief trois petits êtres endormis.\par
C’était un tas charmant, bras et jambes mêlés, paupières fermées, blondes têtes souriantes.\par
La mère reconnut ses enfants.\par
Elle jeta un cri effrayant.\par
Ce cri de l’inexprimable angoisse n’est donné qu’aux mères. Rien n’est plus farouche et rien n’est plus touchant. Quand une femme le jette, on croit entendre une louve ; quand une louve le pousse, on croit entendre une femme.\par
Ce cri de Michelle Fléchard fut un hurlement. Hécube aboya, dit Homère.\par
C’était ce cri que le marquis de Lantenac venait d’entendre.\par
On a vu qu’il s’était arrêté.\par
Le marquis était entre l’issue du passage par où Halmalo l’avait fait échapper, et le ravin. A travers les broussailles entre-croisées sur lui, il vit le pont en flammes, la Tourgue rouge de la réverbération, et, par l’écartement de deux branches, il aperçut au-dessus de sa tête, de l’autre côté, sur le rebord du plateau, vis-à-vis du château brûlant et dans le plein jour de l’incendie, une figure hagarde et lamentable, une femme penchée sur le ravin.\par
 C’était de cette femme qu’était venu ce cri.\par
Cette figure, ce n’était plus Michelle Fléchard, c’était Gorgone. Les misérables sont les formidables. La paysanne s’était transfigurée en euménide. Cette villageoise quelconque, vulgaire, ignorante, inconsciente, venait de prendre brusquement les proportions épiques du désespoir. Les grandes douleurs sont une dilatation gigantesque de l’âme ; cette mère, c’était la maternité ; tout ce qui résume l’humanité est surhumain ; elle se dressait là, au bord de ce ravin, devant cet embrasement, devant ce crime, comme une puissance sépulcrale ; elle avait le cri de la bête et le geste de la déesse ; sa face, d’où tombaient des imprécations, semblait un masque de flamboiement. Rien de souverain comme l’éclair de ses yeux noyés de larmes ; son regard foudroyait l’incendie.\par
Le marquis écoutait. Cela tombait sur sa tête ; il entendait on ne sait quoi d’inarticulé et de déchirant, plutôt des sanglots que des paroles.\par
— Ah ! mon Dieu ! mes enfants ! ce sont mes enfants ! Au secours ! au feu ! au feu ! au feu ! Mais vous êtes donc des bandits ! Est-ce qu’il n’y a personne là ? Mais mes enfants vont brûler ! Ah ! voilà une chose ! Georgette ! mes enfants ! Gros-Alain, René-Jean ! Mais qu’est-ce que cela veut dire ? Qui donc a mis mes enfants là ? Ils dorment. Je suis folle ! C’est une chose impossible. Au secours !\par
Cependant un grand mouvement se faisait dans la Tourgue et sur le plateau. Tout le camp accourait autour du feu qui venait d’éclater. Les assiégeants,  après avoir eu affaire à la mitraille, avaient affaire à l’incendie. Gauvain, Cimourdain, Guéchamp donnaient des ordres. Que faire ? Il y avait à peine quelques seaux d’eau à puiser dans le maigre ruisseau du ravin. L’angoisse allait croissant. Tout le rebord du plateau était couvert de visages effarés qui regardaient.\par
Ce qu’on voyait était effroyable.\par
On regardait, et l’on n’y pouvait rien.\par
La flamme, par le lierre qui avait pris feu, avait gagné l’étage d’en haut. Là elle avait trouvé le grenier plein de paille et elle s’y était précipitée. Tout le grenier brûlait maintenant. La flamme dansait ; la joie de la flamme, chose lugubre. Il semblait qu’un souffle scélérat attisait ce bûcher. On eût dit que l’épouvantable Imânus tout entier était là changé en tourbillon d’étincelles, vivant de la vie meurtrière du feu, et que cette âme monstre s’était faite incendie. L’étage de la bibliothèque n’était pas encore atteint, la hauteur de son plafond et l’épaisseur de ses murs retardaient l’instant où il prendrait feu, mais cette minute fatale approchait ; il était léché par l’incendie du premier étage et caressé par celui du troisième. L’affreux baiser de la mort l’effleurait. En bas une cave de lave, en haut une voûte de braise ; qu’un trou se fît au plancher, c’était l’écroulement dans la cendre rouge ; qu’un trou se fît au plafond, c’était l’ensevelissement sous les charbons ardents. René-Jean, Gros-Alain et Georgette ne s’étaient pas encore réveillés, ils dormaient du sommeil profond et simple de l’enfance, et, à travers les plis de flamme et de fumée qui tour  à tour couvraient et découvraient les fenêtres, on les apercevait dans cette grotte de feu, au fond d’une lueur de météore, paisibles, gracieux, immobiles, comme trois enfants-Jésus confiants endormis dans un enfer ; et un tigre eût pleuré de voir ces roses dans cette fournaise et ces berceaux dans ce tombeau.\par
Cependant la mère se tordait les bras.\par
— Au feu ! je crie au feu ! on est donc des sourds qu’on ne vient pas ! on me brûle mes enfants ! arrivez donc, vous les hommes qui êtes là. Voilà des jours et des jours que je marche, et c’est comme ça que je les retrouve ! Au feu ! au secours ! Des anges ! dire que ce sont des anges ! Qu’est-ce qu’ils ont fait, ces innocents-là ? moi on m’a fusillée, eux on les brûle ! qui est-ce donc qui fait ces choses-là ? Au secours ! sauvez mes enfants ! est-ce que vous ne m’entendez pas ? une chienne, on aurait pitié d’une chienne ! Mes enfants ! mes enfants ! ils dorment ! Ah ! Georgette ! je vois son petit ventre à cet amour ! René-Jean ! Gros-Alain ! c’est comme cela qu’ils s’appellent. Vous voyez bien que je suis leur mère. Ce qui se passe dans ce temps-ci est abominable. J’ai marché des jours et des nuits. Même que j’ai parlé ce matin à une femme. Au secours ! au secours ! au feu ! On est donc des monstres ! c’est une horreur ! L’aîné n’a pas cinq ans, la petite n’a pas deux ans. Je vois leurs petites jambes nues. Ils dorment, bonne sainte Vierge ! la main du ciel me les rend et la main de l’enfer me les reprend. Dire que j’ai tant marché ! Mes enfants que j’ai nourris de mon lait ! moi qui me croyais malheureuse de ne pas les retrouver ! Ayez pitié de moi !  Je veux mes enfants, il me faut mes enfants ! C’est pourtant vrai qu’ils sont là dans le feu ! Voyez mes pauvres pieds comme ils sont tout en sang. Au secours ! Ce n’est pas possible qu’il y ait des hommes sur la terre et qu’on laisse ces pauvres petits mourir comme cela ! au secours ! à l’assassin ! Des choses comme on n’en voit pas de pareilles. Ah ! les brigands ! qu’est-ce que c’est que cette affreuse maison-là ? On me les a volés pour me les tuer ! Jésus misère ! je veux mes enfants. Oh ! je ne sais pas ce que je ferais ! Je ne veux pas qu’ils meurent ! au secours ! au secours ! au secours ! Oh ! s’ils devaient mourir comme cela, je tuerais Dieu !\par
En même temps que la supplication terrible de la mère, des voix s’élevaient sur le plateau et dans le ravin :\par
— Une échelle !\par
— On n’a pas d’échelle !\par
— De l’eau !\par
— On n’a pas d’eau !\par
— Là-haut, dans la tour, au second étage, il y a une porte.\par
— Elle est en fer.\par
— Enfoncez-la.\par
— On ne peut pas !\par
Et la mère redoublait ses appels désespérés :\par
— Au feu ! au secours ! Mais dépêchez-vous donc ! Alors, tuez-moi ! Mes enfants ! mes enfants ! Ah ! l’horrible feu ! qu’on les en ôte, ou qu’on m’y jette !\par
Dans les intervalles de ces clameurs on entendait le pétillement tranquille de l’incendie.\par
 Le marquis tâta sa poche et y toucha la clef de la porte de fer. Alors, se courbant sous la voûte par laquelle il s’était évadé, il rentra dans le passage d’où il venait de sortir.
 \subsubsection[{II. De la porte de pierre a la porte de fer}]{II \\
De la porte de pierre a la porte de fer}\phantomsection
\label{p3l5c2}
\noindent Toute une armée éperdue autour d’un sauvetage impossible, quatre mille hommes ne pouvant secourir trois enfants ; telle était la situation.\par
On n’avait pas d’échelle en effet ; l’échelle envoyée de Javené n’était pas arrivée ; l’embrasement s’élargissait comme un cratère qui s’ouvre ; essayer de l’éteindre avec le ruisseau du ravin presque à sec était dérisoire ; autant jeter un verre d’eau sur un volcan.\par
Cimourdain, Guéchamp et Radoub étaient descendus dans le ravin ; Gauvain était remonté dans la salle du deuxième étage de la Tourgue où étaient la pierre tournante, l’issue secrète, et la porte de fer de la bibliothèque. C’est là qu’avait été la mèche soufrée allumée par l’Imânus ; c’était de là que l’incendie était parti.\par
Gauvain avait amené avec lui vingt sapeurs. Enfoncer la porte de fer, il n’y avait plus que cette ressource. Elle était effroyablement bien fermée.\par
On commença par des coups de hache. Les haches cassèrent. Un sapeur dit :\par
 — L’acier est du verre sur ce fer-là.\par
La porte était en effet de fer battu, et faite de doubles lames boulonnées ayant chacune trois pouces d’épaisseur.\par
On prit des barres de fer et l’on essaya des pesées sous la porte. Les barres de fer cassèrent.\par
— Comme des allumettes, dit le sapeur.\par
Gauvain, sombre, murmura :\par
— Il n’y a qu’un boulet qui ouvrirait cette porte. Il faudrait pouvoir monter ici une pièce de canon.\par
— Et encore ! dit le sapeur.\par
Il y eut un moment d’accablement. Tous ces bras impuissants s’arrêtèrent. Muets, vaincus, consternés, ces hommes considéraient l’horrible porte inébranlable. Une réverbération rouge passait par-dessous. Derrière, l’incendie croissait.\par
L’affreux cadavre de l’Imânus était là, sinistre victorieux.\par
Encore quelques minutes peut-être, et tout allait s’effondrer.\par
Que faire ? Il n’y avait plus d’espérance.\par
Gauvain exaspéré s’écria, l’œil fixé sur la pierre tournante du mur et sur l’issue ouverte de l’évasion :\par
— C’est pourtant par là que le marquis de Lantenac s’en est allé !\par
— Et qu’il revient, dit une voix.\par
Et une tête blanche se dessina dans l’encadrement de pierre de l’issue secrète.\par
C’était le marquis.\par
 Depuis bien des années Gauvain ne l’avait pas vu de si près. Il recula.\par
Tous ceux qui étaient là restèrent dans l’attitude où ils étaient, pétrifiés.\par
Le marquis avait une grosse clef à la main, il refoula d’un regard altier quelques-uns des sapeurs qui étaient devant lui, marcha droit à la porte de fer, se courba sous la voûte, et mit la clef dans la serrure. La serrure grinça, la porte s’ouvrit, on vit un gouffre de flamme. Le marquis y entra.\par
Il y entra d’un pied ferme, la tête haute.\par
Tous le suivaient des yeux, frissonnants.\par
A peine le marquis eut-il fait quelques pas dans la salle incendiée que le parquet miné par le feu et ébranlé par son talon s’effondra derrière lui et mit entre lui et la porte un précipice. Le marquis ne tourna pas la tête et continua d’avancer. Il disparut dans la fumée.\par
On ne vit plus rien.\par
Avait-il pu aller plus loin ? Une nouvelle fondrière de feu s’était-elle ouverte sous lui ? N’avait-il réussi qu’à se perdre lui-même ? On ne pouvait rien dire. On n’avait devant soi qu’une muraille de fumée et de flamme. Le marquis était au delà, mort ou vivant.
 \subsubsection[{III. Ou l’on voit se réveiller les enfants qu’on a vus se rendormir}]{III \\
Ou l’on voit se réveiller les enfants qu’on a vus se rendormir}\phantomsection
\label{p3l5c3}
\noindent Cependant les enfants avaient fini par ouvrir les yeux.\par
L’incendie, qui n’était pas encore entré dans la salle de la bibliothèque, jetait au plafond un reflet rose. Les enfants ne connaissaient pas cette espèce d’aurore-là. Ils la regardèrent. Georgette la contempla.\par
Toutes les splendeurs de l’incendie se déployaient ; l’hydre noire et le dragon écarlate apparaissaient dans la fumée difforme, superbement sombre et vermeille. De longues flammèches s’envolaient au loin et rayaient l’ombre, et l’on eût dit des comètes combattantes, courant les unes après les autres. Le feu est une prodigalité ; les brasiers sont pleins d’écrins qu’ils sèment au vent ; ce n’est pas pour rien que le charbon est identique au diamant. Il s’était fait au mur du troisième étage des crevasses par où la braise versait dans le ravin des cascades de pierreries ; les tas de paille et d’avoine qui brûlaient dans le grenier commençaient à ruisseler par les fenêtres en avalanches  de poudre d’or, et les avoines devenaient des améthystes, et les brins de paille devenaient des escarboucles.\par
— Joli ! dit Georgette.\par
Ils s’étaient dressés tous les trois.\par
— Ah ! cria la mère, ils se réveillent !\par
René-Jean se leva, alors Gros-Alain se leva, alors Georgette se leva.\par
René-Jean étira ses bras, alla vers la croisée et dit : — J’ai chaud.\par
— Ai chaud, répéta Georgette.\par
La mère les appela.\par
— Mes enfants ! René ! Alain ! Georgette !\par
Les enfants regardaient autour d’eux. Ils cherchaient à comprendre. Où les hommes sont terrifiés, les enfants sont curieux. Qui s’étonne aisément s’effraye difficilement ; l’ignorance contient de l’intrépidité. Les enfants ont si peu droit à l’enfer que, s’ils le voyaient, ils l’admireraient.\par
La mère répéta :\par
— René ! Alain ! Georgette !\par
René-Jean tourna la tête ; cette voix le tira de sa distraction ; les enfants ont la mémoire courte, mais ils ont le souvenir rapide ; tout le passé est pour eux hier ; René-Jean vit sa mère, trouva cela tout simple, et, entouré comme il l’était de choses étranges, sentant un vague besoin d’appui, il cria :\par
— Maman !\par
— Maman ! dit Gros-Alain.\par
— M’man ! dit Georgette.\par
 Et elle tendit ses petits bras.\par
Et la mère hurla : — Mes enfants !\par
Tous les trois vinrent au bord de la fenêtre ; par bonheur, l’embrasement n’était pas de ce côté-là.\par
— J’ai trop chaud, dit René-Jean.\par
Il ajouta :\par
— Ça brûle.\par
Et il chercha des yeux sa mère.\par
— Viens donc, maman.\par
— Don, m’man, répéta Georgette.\par
La mère échevelée, déchirée, saignante, s’était laissée rouler de broussaille en broussaille dans le ravin. Cimourdain y était avec Guéchamp, aussi impuissants en bas que Gauvain en haut. Les soldats, désespérés d’être inutiles, fourmillaient autour d’eux. La chaleur était insupportable, personne ne la sentait. On considérait l’escarpement du pont, la hauteur des arches, l’élévation des étages, les fenêtres inaccessibles, et la nécessité d’agir vite. Trois étages à franchir. Nul moyen d’arriver là. Radoub, blessé, un coup de sabre à l’épaule, une oreille arrachée, ruisselant de sueur et de sang, était accouru ; il vit Michelle Fléchard. — Tiens, dit-il, la fusillée, vous êtes donc ressuscitée ! — Mes enfants ! dit la mère. — C’est juste, répondit Radoub ; nous n’avons pas le temps de nous occuper des revenants. Et il se mit à escalader le pont, essai inutile, il enfonça ses ongles dans la pierre, il grimpa quelques instants ; mais les assises étaient lisses, pas une cassure, pas un relief, la muraille était aussi correctement rejointoyée qu’une muraille neuve, et  Radoub retomba. L’incendie continuait, épouvantable ; on apercevait, dans l’encadrement de la croisée toute rouge, les trois têtes blondes. Radoub, alors, montra le poing au ciel, comme s’il y cherchait quelqu’un du regard, et dit : C’est donc ça une conduite, bon Dieu ! La mère embrassait à genoux les piles du pont en criant : Grâce !\par
De sourds craquements se mêlaient aux pétillements du brasier. Les vitres des armoires de la bibliothèque se fêlaient, et tombaient avec bruit. Il était évident que la charpente cédait. Aucune force humaine n’y pouvait rien. Encore un moment et tout allait s’abîmer. On n’attendait plus que la catastrophe. On entendait les petites voix répéter : Maman ! maman ! On était au paroxisme de l’effroi.\par
Tout à coup, à la fenêtre voisine de celle où étaient les enfants, sur le fond pourpre du flamboiement, une haute figure apparut.\par
Toutes les têtes se levèrent, tous les yeux devinrent fixes. Un homme était là-haut, un homme était dans la salle de la bibliothèque, un homme était dans la fournaise. Cette figure se découpait en noir sur la flamme, mais elle avait des cheveux blancs. On reconnut le marquis de Lantenac.\par
Il disparut, puis il reparut.\par
L’effrayant vieillard se dressa à la fenêtre maniant une énorme échelle. C’était l’échelle de sauvetage, déposée dans la bibliothèque, qu’il était allé chercher le long du mur et qu’il avait traînée jusqu’à la fenêtre. Il la saisit par une extrémité, et avec l’agilité  magistrale d’un athlète, il la fit glisser hors de la croisée sur le rebord de l’appui extérieur jusqu’au fond du ravin. Radoub, en bas, éperdu, tendit les mains, reçut l’échelle, la serra dans ses bras, et cria : — Vive la République !\par
Le marquis répondit : — Vive le Roi !\par
Et Radoub grommela : — Tu peux bien crier tout ce que tu voudras, et dire des bêtises si tu veux, tu es le bon Dieu.\par
L’échelle était posée ; la communication était établie entre la salle incendiée et la terre ; vingt hommes accoururent, Radoub en tête, et en un clin d’œil ils s’étagèrent du haut en bas, adossés aux échelons comme les maçons qui montent et qui descendent des pierres. Cela fit sur l’échelle de bois une échelle humaine. Radoub, au faîte de l’échelle, touchait à la fenêtre. Il était, lui, tourné vers l’incendie.\par
La petite armée, éparse dans les bruyères et sur les pentes, se pressait, bouleversée de toutes les émotions à la fois, sur le plateau, dans le ravin, sur la plate-forme de la tour.\par
Le marquis disparut encore, puis reparut, apportant un enfant.\par
Il y eut un immense battement de mains.\par
C’était le premier que le marquis avait saisi au hasard. C’était Gros-Alain.\par
Gros-Alain criait : — J’ai peur.\par
Le marquis donna Gros-Alain à Radoub, qui le passa derrière lui et au-dessous de lui à un soldat qui le passa à un autre, et, pendant que Gros-Alain, très  effrayé et criant, arrivait ainsi de bras en bras jusqu’au bas de l’échelle, le marquis, un moment absent, revint à la fenêtre avec René-Jean qui résistait et pleurait, et qui battit Radoub au moment où le marquis le passa au sergent.\par
Le marquis rentra dans la salle pleine de flammes. Georgette était restée seule. Il alla à elle. Elle sourit. Cet homme de granit sentit quelque chose d’humide lui venir aux yeux. Il demanda : — Comment t’appelles-tu ?\par
— Orgette, dit-elle.\par
Il la prit dans ses bras, elle souriait toujours, et au moment où il la remettait à Radoub, cette conscience si haute et si obscure eut l’éblouissement de l’innocence, le vieillard donna à l’enfant un baiser.\par
— C’est la petite môme ! dirent les soldats ; et Georgette, à son tour, descendit de bras en bras jusqu’à terre parmi des cris d’adoration. On battait des mains, on trépignait ; les vieux grenadiers sanglotaient, et elle leur souriait.\par
La mère était au pied de l’échelle, haletante, insensée, ivre de tout cet inattendu, jetée sans transition de l’enfer dans le paradis. L’excès de joie meurtrit le cœur à sa façon. Elle tendait les bras, elle reçut d’abord Gros-Alain, ensuite René-Jean, ensuite Georgette, elle les couvrit pêle-mêle de baisers, puis elle éclata de rire et tomba évanouie.\par
Un grand cri s’éleva :\par
— Tous sont sauvés !\par
Tous étaient sauvés en effet, excepté le vieillard.\par
 Mais personne n’y songeait, pas même lui peut-être.\par
Il resta quelques instants rêveur au bord de la fenêtre, comme s’il voulait laisser au gouffre de flamme le temps de prendre un parti. Puis, sans se hâter, lentement, fièrement, il enjamba l’appui de la croisée, et, sans se retourner, droit, debout, adossé aux échelons, ayant derrière lui l’incendie, faisant face au précipice, il se mit à descendre l’échelle en silence avec une majesté de fantôme. Ceux qui étaient sur l’échelle se précipitèrent en bas, tous les assistants tressaillirent, il se fit autour de cet homme qui arrivait d’en haut un recul d’horreur sacrée comme autour d’une vision. Lui, cependant, s’enfonçait gravement dans l’ombre qu’il avait devant lui ; pendant qu’ils reculaient, il s’approchait d’eux ; sa pâleur de marbre n’avait pas un pli, son regard de spectre n’avait pas un éclair ; à chaque pas qu’il faisait vers ces hommes dont les prunelles effarées se fixaient sur lui dans les ténèbres, il semblait plus grand, l’échelle tremblait et sonnait sous son pied lugubre, et l’on eût dit la statue du commandeur redescendant dans le sépulcre.\par
Quand le marquis fut en bas, quand il eut atteint le dernier échelon et posé son pied à terre, une main s’abattit sur son collet. Il se retourna.\par
— Je t’arrête, dit Cimourdain.\par
— Je t’approuve, dit Lantenac.\par
  \subsection[{Livre sixième. C’est après la victoire qu’a lieu le combat}]{Livre sixième \\
C’est après la victoire qu’a lieu le combat}\phantomsection
\label{p3l6}
\subsubsection[{I. Lantenac pris}]{I \\
Lantenac pris}\phantomsection
\label{p3l6c1}
\noindent C’était dans le sépulcre en effet que le marquis était redescendu.\par
On l’emmena.\par
La crypte-oubliette du rez-de-chaussée de la Tourgue fut immédiatement rouverte sous l’œil sévère de Cimourdain ; on y mit une lampe, une cruche d’eau et un pain de soldat, on y jeta une botte de paille, et, moins d’un quart d’heure après la minute où la main du prêtre avait saisi le marquis, la porte du cachot se refermait sur Lantenac.\par
Cela fait, Cimourdain alla trouver Gauvain ; en ce moment-là l’église lointaine de Parigné sonnait onze heures du soir ; Cimourdain dit à Gauvain :\par
 — Je vais convoquer la cour martiale. Tu n’en seras pas. Tu es Gauvain et Lantenac est Gauvain. Tu es trop proche parent pour être juge, et je blâme Égalité d’avoir jugé Capet. La cour martiale sera composée de trois juges, un officier, le capitaine Guéchamp, un sous-officier, le sergent Radoub, et moi, qui présiderai. Rien de tout cela ne te regarde plus. Nous nous conformerons au décret de la Convention ; nous nous bornerons à constater l’identité du ci-devant marquis de Lantenac. Demain la cour martiale, après-demain la guillotine. La Vendée est morte.\par
Gauvain ne répliqua pas une parole, et Cimourdain, préoccupé de la chose suprême qui lui restait à faire, le quitta. Cimourdain avait des heures à désigner et des emplacements à choisir. Il avait comme Lequinio à Granville, comme Tallien à Bordeaux, comme Châlier à Lyon, comme Saint-Just à Strasbourg, l’habitude, réputée de bon exemple, d’assister de sa personne aux exécutions ; le juge venait voir travailler le bourreau ; usage emprunté par la Terreur de 93 aux parlements de France et à l’inquisition d’Espagne.\par
Gauvain aussi était préoccupé.\par
Un vent froid soufflait de la forêt. Gauvain, laissant Guéchamp donner les ordres nécessaires, alla à sa tente qui était dans le pré de la lisière du bois, au pied de la Tourgue, et y prit son manteau à capuchon, dont il s’enveloppa. Ce manteau était bordé de ce simple galon qui, selon la mode républicaine sobre d’ornements, désignait le commandant en chef. Il se mit à marcher dans ce pré sanglant où l’assaut avait  commencé. Il était là seul. L’incendie continuait, désormais dédaigné ; Radoub était près des enfants et de la mère, presque aussi maternel qu’elle ; le châtelet du pont achevait de brûler, les sapeurs faisaient la part du feu, on creusait des fosses, on enterrait les morts, on pansait les blessés, on avait démoli la retirade, on désencombrait de cadavres les chambres et les escaliers, on nettoyait le lieu du carnage, on balayait le tas d’ordures terrible de la victoire, les soldats faisaient, avec la rapidité militaire, ce qu’on pourrait appeler le ménage de la bataille finie. Gauvain ne voyait rien de tout cela.\par
A peine jetait-il un regard, à travers sa rêverie, au poste de la brèche doublé sur l’ordre de Cimourdain.\par
Cette brèche, il la distinguait dans l’obscurité, à environ deux cents pas du coin de prairie où il s’était comme réfugié. Il voyait cette ouverture noire. C’était par là que l’attaque avait commencé, il y avait trois heures de cela ; c’était par là que lui Gauvain avait pénétré dans la tour ; c’était là le rez-de-chaussée où était la retirade ; c’était dans ce rez-de-chaussée que s’ouvrait la porte du cachot où était le marquis. Ce poste de la brèche gardait ce cachot.\par
En même temps que son regard apercevait vaguement cette brèche, son oreille entendait confusément revenir, comme un glas qui tinte, ces paroles : Demain la cour martiale, après-demain la guillotine.\par
L’incendie qu’on avait isolé et sur lequel les sapeurs lançaient toute l’eau qu’on avait pu se procurer, ne s’éteignait pas sans résistance et jetait des flammes  intermittentes ; on entendait par instants craquer les plafonds et se précipiter l’un sur l’autre les étages croulants ; alors des tourbillons d’étincelles s’envolaient comme d’une torche secouée, une clarté d’éclair faisait visible l’extrême horizon, et l’ombre de la Tourgue, subitement gigantesque, s’allongeait jusqu’à la forêt.\par
Gauvain allait et venait à pas lents dans cette ombre et devant la brèche de l’assaut. Par moments il croisait ses deux mains derrière sa tête recouverte de son capuchon de guerre. Il songeait.
 \subsubsection[{II. Gauvain pensif}]{II \\
Gauvain pensif}\phantomsection
\label{p3l6c2}
\noindent Sa rêverie était insondable.\par
Un changement à vue inouï venait de se faire.\par
Le marquis de Lantenac s’était transfiguré.\par
Gauvain avait été témoin de cette transfiguration.\par
Jamais il n’aurait cru que de telles choses pussent résulter d’une complication d’incidents, quels qu’ils fussent. Jamais il n’aurait, même en rêve, imaginé qu’il pût arriver rien de pareil. L’imprévu, cet on ne sait quoi de hautain qui joue avec l’homme, avait saisi Gauvain et le tenait. Gauvain avait devant lui l’impossible devenu réel, visible, palpable, inévitable, inexorable.\par
Que pensait-il de cela, lui, Gauvain ?\par
Il ne s’agissait pas de tergiverser ; il fallait conclure.\par
Une question lui était posée ; il ne pouvait prendre la fuite devant elle.\par
Posée par qui ?\par
Par les événements.\par
Et pas seulement par les événements.\par
Car lorsque les événements, qui sont variables,  nous font une question, la justice, qui est immuable, nous somme de répondre.\par
Derrière le nuage, qui nous jette son ombre, il y a l’étoile, qui nous jette sa clarté.\par
Nous ne pouvons pas plus nous soustraire à la clarté qu’à l’ombre.\par
Gauvain subissait un interrogatoire.\par
Il comparaissait devant quelqu’un.\par
Devant quelqu’un de redoutable.\par
Sa conscience.\par
Gauvain sentait tout vaciller en lui. Ses résolutions les plus solides, ses promesses les plus fermement faites, ses décisions les plus irrévocables, tout cela chancelait dans les profondeurs de sa volonté.\par
Il y a des tremblements d’âme.\par
Plus il réfléchissait à ce qu’il venait de voir, plus il était bouleversé.\par
Gauvain, républicain, croyait être, et était, dans l’absolu. Un absolu supérieur venait de se révéler.\par
Au-dessus de l’absolu révolutionnaire, il y a l’absolu humain.\par
Ce qui se passait ne pouvait être éludé ; le fait était grave ; Gauvain faisait partie de ce fait ; il en était ; il ne pouvait s’en retirer ; et, bien que Cimourdain lui eût dit : — « Cela ne te regarde plus, » — il sentait en lui quelque chose comme ce qu’éprouve l’arbre au moment où on l’arrache de sa racine.\par
Tout homme a une base ; un ébranlement à cette base cause un trouble profond ; Gauvain sentait ce trouble.\par
Il pressait sa tête dans ses deux mains, comme  pour en faire jaillir la vérité. Préciser une telle situation n’était pas facile ; simplifier le complexe, rien de plus malaisé ; il avait devant lui de redoutables chiffres dont il fallait faire le total ; faire l’addition de la destinée, quel vertige ! il l’essayait ; il tâchait de se rendre compte ; il s’efforçait de rassembler ses idées, de discipliner les résistances qu’il sentait en lui, et de récapituler les faits.\par
Il se les exposait à lui-même.\par
A qui n’est-il pas arrivé de se faire un rapport, et de s’interroger, dans une circonstance suprême, sur l’itinéraire à suivre, soit pour avancer, soit pour reculer ?\par
Gauvain venait d’assister à un prodige.\par
En même temps que le combat terrestre, il y avait eu un combat céleste.\par
Le combat du bien contre le mal.\par
Un cœur effrayant venait d’être vaincu.\par
Étant donné l’homme avec tout ce qui est mauvais en lui, la violence, l’erreur, l’aveuglement, l’opiniâtreté malsaine, l’orgueil, l’égoïsme, Gauvain venait de voir un miracle.\par
La victoire de l’humanité sur l’homme.\par
L’humanité avait vaincu l’inhumain.\par
Et par quel moyen ? de quelle façon ? comment avait-elle terrassé un colosse de colère et de haine ? quelles armes avait-elle employées ? quelle machine de guerre ? Le berceau.\par
Un éblouissement venait de passer sur Gauvain. En pleine guerre sociale, en pleine conflagration de toutes les inimitiés et de toutes les vengeances, au  moment le plus obscur et le plus furieux du tumulte, à l’heure où le crime donnait toute sa flamme et la haine toutes ses ténèbres, à cet instant des luttes où tout devient projectile, où la mêlée est si funèbre qu’on ne sait plus où est le juste, où est l’honnête, où est le vrai ; brusquement, l’Inconnu, l’avertisseur mystérieux des âmes, venait de faire resplendir, au-dessus des clartés et des noirceurs humaines, la grande lueur éternelle.\par
Au-dessus du sombre duel entre le faux et le relatif, dans les profondeurs, la face de la vérité avait tout à coup apparu.\par
Subitement la force des faibles était intervenue.\par
On avait vu trois pauvres êtres à peine nés, inconscients, abandonnés, orphelins, seuls, bégayants, souriants, ayant contre eux la guerre civile, le talion, l’affreuse logique des représailles, le meurtre, le carnage, le fratricide, la rage, la rancune, toutes les gorgones, triompher ; on avait vu l’avortement et la défaite d’un infâme incendie, chargé de commettre un crime ; on avait vu les préméditations atroces déconcertées et déjouées ; on avait vu l’antique férocité féodale, le vieux dédain inexorable, la prétendue expérience des nécessités de la guerre, la raison d’état, tous les arrogants partis-pris de la vieillesse farouche, s’évanouir devant le bleu regard de ceux qui n’ont pas vécu ; et c’est tout simple, car celui qui n’a pas vécu encore n’a pas fait le mal, il est la justice, il est la vérité, il est la blancheur, et les immenses anges du ciel sont dans les petits enfants.\par
 Spectacle utile ; conseil ; leçon. Les combattants frénétiques de la guerre sans merci avaient soudainement vu, en face de tous les forfaits, de tous les attentats, de tous les fanatismes, de l’assassinat, de la vengeance attisant les bûchers, de la mort arrivant une torche à la main, au-dessus de l’énorme légion des crimes, se dresser cette toute-puissance, l’innocence.\par
Et l’innocence avait vaincu.\par
Et l’on pouvait dire : Non, la guerre civile n’existe pas, la barbarie n’existe pas, la haine n’existe pas, le crime n’existe pas, les ténèbres n’existent pas ; pour dissiper ces spectres, il suffit de cette aurore, l’enfance.\par
Jamais, dans aucun combat, Satan n’avait été plus visible, ni Dieu.\par
Ce combat avait eu pour arène une conscience.\par
La conscience de Lantenac.\par
Maintenant il recommençait, plus acharné et plus décisif encore peut-être, dans une autre conscience.\par
La conscience de Gauvain.\par
Quel champ de bataille que l’homme !\par
Nous sommes livrés à ces dieux, à ces monstres, à ces géants, nos pensées.\par
Souvent ces belligérants terribles foulent aux pieds notre âme.\par
Gauvain méditait.\par
Le marquis de Lantenac, cerné, bloqué, condamné, mis hors la loi, serré, comme la bête dans le cirque, comme le clou dans la tenaille, enfermé dans son gîte devenu sa prison, étreint de toutes parts par une muraille de fer et de feu, était parvenu à se dérober. Il  avait fait ce miracle d’échapper. Il avait réussi ce chef-d’œuvre, le plus difficile de tous dans une telle guerre, la fuite. Il avait repris possession de la forêt pour s’y retrancher, du pays pour y combattre, de l’ombre pour y disparaître. Il était redevenu le redoutable allant et venant, l’errant sinistre, le capitaine des invisibles, le chef des hommes souterrains, le maître des bois. Gauvain avait la victoire, mais Lantenac avait la liberté. Lantenac désormais avait la sécurité, la course illimitée devant lui, le choix inépuisable des asiles. Il était insaisissable, introuvable, inaccessible. Le lion avait été pris au piège, et il en était sorti.\par
Eh bien, il y était rentré.\par
Le marquis de Lantenac avait volontairement, spontanément, de sa pleine préférence, quitté la forêt, l’ombre, la sécurité, la liberté, pour rentrer dans le plus effroyable péril, intrépidement, une première fois, Gauvain l’avait vu, en se précipitant dans l’incendie au risque de s’y engouffrer, une deuxième fois, en descendant cette échelle qui le rendait à ses ennemis, et qui, échelle de sauvetage pour les autres, était pour lui échelle de perdition.\par
Et pourquoi avait-il fait cela ?\par
Pour sauver trois enfants.\par
Et maintenant qu’allait-on faire de cet homme ?\par
Le guillotiner.\par
Ainsi, cet homme, pour trois enfants, les siens ? non ; de sa famille ? non ; de sa caste ? non ; pour trois petits pauvres, les premiers venus, des enfants  trouvés, des inconnus, des déguenillés, des va-nu-pieds, ce gentilhomme, ce prince, ce vieillard, sauvé, délivré, vainqueur, car l’évasion est un triomphe, avait tout risqué, tout compromis, tout remis en question, et, hautainement, en même temps qu’il rendait les enfants, il avait apporté sa tête, et cette tête, jusqu’alors terrible, maintenant auguste, il l’avait offerte.\par
Et qu’allait-on faire ?\par
L’accepter.\par
Le marquis de Lantenac avait eu le choix entre la vie d’autrui et la sienne ; dans cette option superbe, il avait choisi sa mort.\par
Et on allait la lui accorder.\par
On allait le tuer.\par
Quel salaire de l’héroïsme !\par
Répondre à un acte généreux par un acte sauvage !\par
Donner ce dessous à la révolution !\par
Quel rapetissement pour la république !\par
Tandis que l’homme des préjugés et des servitudes, subitement transformé, rentrait dans l’humanité, eux, les hommes de la délivrance et de l’affranchissement, ils resteraient dans la guerre civile, dans la routine du sang, dans le fratricide !\par
Et la haute loi divine de pardon, d’abnégation, de rédemption, de sacrifice, existerait pour les combattants de l’erreur, et n’existerait pas pour les soldats de la vérité !\par
Quoi ! ne pas lutter de magnanimité ! se résigner à cette défaite, étant les plus forts, d’être les plus  faibles, étant les victorieux, d’être les meurtriers, et de faire dire qu’il y a, du côté de la monarchie, ceux qui sauvent les enfants, et du côté de la république, ceux qui tuent les vieillards !\par
On verrait ce grand soldat, cet octogénaire puissant, ce combattant désarmé, volé plutôt que pris, saisi en pleine bonne action, garrotté avec sa permission, ayant encore au front la sueur d’un dévouement grandiose, monter les marches de l’échafaud comme on monte les degrés d’une apothéose ! Et l’on mettrait sous le couperet cette tête, autour de laquelle voleraient, suppliantes, les trois âmes des petits anges sauvés ! et, devant ce supplice infamant pour les bourreaux, on verrait le sourire sur la face de cet homme, et sur la face de la république la rougeur !\par
Et cela s’accomplirait en présence de Gauvain, chef !\par
Et, pouvant l’empêcher, il s’abstiendrait ! Et il se contenterait de ce congé altier, \emph{ — Cela ne te regarde plus !} — Et il ne se dirait point qu’en pareil cas, abdication, c’est complicité ! Et il ne s’apercevrait pas que, dans une action si énorme, entre celui qui fait et celui qui laisse faire, celui qui laisse faire est le pire, étant le lâche !\par
Mais cette mort, ne l’avait-il pas promise ? Lui, Gauvain, l’homme clément, n’avait-il pas déclaré que Lantenac faisait exception à la clémence, et qu’il livrerait Lantenac à Cimourdain ?\par
Cette tête, il la devait. Eh bien, il la payait. Voilà tout.\par
Mais était-ce bien la même tête ?\par
Jusqu’ici Gauvain n’avait vu dans Lantenac que le  combattant barbare, le fanatique de royauté et de féodalité, le massacreur de prisonniers, l’assassin déchaîné par la guerre, l’homme sanglant. Cet homme-là, il ne le craignait pas ; ce proscripteur, il le proscrirait ; cet implacable le trouverait implacable. Rien de plus simple, le chemin était tracé et lugubrement facile à suivre, tout était prévu, on tuera celui qui tue, on était dans la ligne droite de l’horreur. Inopinément, cette ligne droite s’était rompue, un tournant imprévu révélait un horizon nouveau, une métamorphose avait eu lieu. Un Lantenac inattendu entrait en scène. Un héros sortait du monstre ; plus qu’un héros, un homme. Plus qu’une âme, un cœur. Ce n’était plus un tueur que Gauvain avait devant lui, mais un sauveur. Gauvain était terrassé par un flot de clarté céleste. Lantenac venait de le frapper d’un coup de foudre de bonté.\par
Et Lantenac transfiguré ne transfigurerait pas Gauvain ! Quoi ! ce coup de lumière serait sans contre-coup ! L’homme du passé irait en avant, et l’homme de l’avenir en arrière ! L’homme des barbaries et des superstitions ouvrirait des ailes subites, et planerait, et regarderait ramper sous lui, dans de la fange et dans de la nuit, l’homme de l’idéal ! Gauvain resterait à plat ventre dans la vieille ornière féroce, tandis que Lantenac irait dans le sublime courir les aventures !\par
Autre chose encore.\par
Et la famille !\par
Ce sang qu’il allait répandre, — car le laisser verser, c’est le verser soi-même, — est-ce que ce n’était pas son sang, à lui Gauvain ? Son grand-père était  mort, mais son grand-oncle vivait ; et ce grand-oncle, c’était le marquis de Lantenac. Est-ce que celui des deux frères qui était dans le tombeau ne se dressait pas pour empêcher l’autre d’y entrer ? Est-ce qu’il n’ordonnait pas à son petit-fils de respecter désormais cette couronne de cheveux blancs, sœur de sa propre auréole ? Est-ce qu’il n’y avait pas là, entre Gauvain et Lantenac, le regard indigné d’un spectre ?\par
Est-ce donc que la révolution avait pour but de dénaturer l’homme ? Est-ce pour briser la famille, est-ce pour étouffer l’humanité, qu’elle était faite ? Loin de là. C’est pour affirmer ces réalités suprêmes, et non pour les nier, que 89 avait surgi. Renverser les bastilles, c’est délivrer l’humanité ; abolir la féodalité, c’est fonder la famille. L’auteur étant le point de départ de l’autorité, et l’autorité étant incluse dans l’auteur, il n’y a point d’autre autorité que la paternité ; de là la légitimité de la reine-abeille qui crée son peuple, et qui, étant mère, est reine ; de là l’absurdité du roi-homme, qui, n’étant pas le père, ne peut être le maître ; de là la suppression du roi ; de là la république. Qu’est-ce que tout cela ? C’est la famille, c’est l’humanité, c’est la révolution. La révolution, c’est l’avénement du peuple ; et, au fond, le Peuple, c’est l’Homme.\par
Il s’agissait maintenant de savoir si, quand Lantenac venait de rentrer dans l’humanité, Gauvain allait, lui, rentrer dans la famille.\par
Il s’agissait de savoir si l’oncle et le neveu allaient se rejoindre dans la lumière supérieure, ou bien si à un progrès de l’oncle répondrait un recul du neveu.\par
 La question, dans ce débat pathétique de Gauvain avec sa conscience, arrivait à se poser ainsi, et la solution semblait se dégager d’elle-même : sauver Lantenac.\par
Oui. Mais la France ?\par
Ici le vertigineux problème changeait de face brusquement.\par
Quoi ! la France était aux abois ! la France était livrée, ouverte, démantelée ! elle n’avait plus de fossé, l’Allemagne passait le Rhin ; elle n’avait plus de muraille, l’Italie enjambait les Alpes, et l’Espagne les Pyrénées. Il lui restait le grand abîme, l’Océan. Elle avait pour elle le gouffre. Elle pouvait s’y adosser, et, géante, appuyée à toute la mer, combattre toute la terre. Situation, après tout, inexpugnable. Eh bien non, cette situation allait lui manquer. Cet Océan n’était plus à elle. Dans cet Océan, il y avait l’Angleterre. L’Angleterre, il est vrai, ne savait comment passer. Eh bien, un homme allait lui jeter le pont, un homme allait lui tendre la main, un homme allait dire à Pitt, à Craig, à Cornwallis, à Dundas, aux pirates : venez ! un homme allait crier : Angleterre, prends la France ! Et cet homme était le marquis de Lantenac.\par
Cet homme, on le tenait. Après trois mois de chasse, de poursuite, d’acharnement, on l’avait enfin saisi. La main de la révolution venait de s’abattre sur le maudit ; le poing crispé de 93 avait pris le meurtrier royaliste au collet ; par un de ces effets de la préméditation mystérieuse qui se mêle d’en haut aux choses humaines, c’était dans son propre cachot de famille que ce parricide attendait maintenant son châtiment ;  l’homme féodal était dans l’oubliette féodale ; les pierres de son château se dressaient contre lui et se fermaient sur lui, et celui qui voulait livrer son pays était livré par sa maison. Dieu avait visiblement édifié tout cela. L’heure juste avait sonné ; la révolution avait fait prisonnier cet ennemi public ; il ne pouvait plus combattre, il ne pouvait plus lutter, il ne pouvait plus nuire ; dans cette Vendée où il y avait tant de bras, il était le seul cerveau ; lui fini, la guerre civile était finie ; on l’avait ; dénoûment tragique et heureux ; après tant de massacres et de carnages, il était là, l’homme qui avait tué, et c’était son tour de mourir.\par
Et il se trouverait quelqu’un pour le sauver !\par
Cimourdain, c’est-à-dire 93, tenait Lantenac, c’est-à-dire la monarchie, et il se trouverait quelqu’un pour ôter de cette serre de bronze cette proie ! Lantenac, l’homme en qui se concentrait cette gerbe de fléaux qu’on nomme le passé, le marquis de Lantenac était dans la tombe, la lourde porte éternelle s’était refermée sur lui, et quelqu’un viendrait, du dehors, tirer le verrou ! ce malfaiteur social était mort, et avec lui la révolte, la lutte fratricide, la guerre bestiale, et quelqu’un le ressusciterait !\par
Oh ! comme cette tête de mort rirait !\par
Comme ce spectre dirait : C’est bon, me voilà vivant, imbéciles !\par
Comme il se remettrait à son œuvre hideuse ! comme Lantenac se replongerait, implacable et joyeux, dans le gouffre de haine et de guerre ! comme on reverrait, dès le lendemain, les maisons brûlées, les  prisonniers massacrés, les blessés achevés, les femmes fusillées !\par
Et après tout, cette action qui fascinait Gauvain, Gauvain ne se l’exagérait-il pas ?\par
Trois enfants étaient perdus ; Lantenac les avait sauvés.\par
Mais qui donc les avait perdus ?\par
N’était-ce pas Lantenac ?\par
Qui avait mis ces berceaux dans cet incendie ?\par
N’était-ce pas l’Imânus ?\par
Qu’était-ce que l’Imânus ?\par
Le lieutenant du marquis.\par
Le responsable, c’est le chef.\par
Donc, l’incendiaire et l’assassin, c’était Lantenac.\par
Qu’avait-il donc fait de si admirable ?\par
Il n’avait point persisté. Rien de plus.\par
Après avoir construit le crime, il avait reculé devant, il s’était fait horreur à lui-même. Le cri de la mère avait réveillé en lui ce fond de vieille pitié humaine, sorte de dépôt de la vie universelle, qui est dans toutes les âmes, même les plus fatales. A ce cri, il était revenu sur ses pas. De la nuit où il s’enfonçait, il avait rétrogradé vers le jour. Après avoir fait le crime, il l’avait défait. Tout son mérite était ceci : n’avoir pas été un monstre jusqu’au bout.\par
Et pour si peu, lui rendre tout ! lui rendre l’espace, les champs, les plaines, l’air, le jour, lui rendre la forêt dont il userait pour le banditisme, lui rendre la liberté dont il userait pour la servitude, lui rendre la vie dont il userait pour la mort !\par
 Quant à essayer de s’entendre avec lui, quant à vouloir traiter avec cette âme altière, quant à lui proposer sa délivrance sous condition, quant à lui demander s’il consentirait, moyennant la vie sauve, à s’abstenir désormais de toute hostilité et de toute révolte ; quelle faute ce serait qu’une telle offre, quel avantage on lui donnerait, à quel dédain on se heurterait, comme il souffletterait la question par la réponse ! comme il dirait : Gardez les hontes pour vous ! tuez-moi !\par
Rien à faire en effet avec un tel homme, que le tuer ou le délivrer. Cet homme était à pic ; il était toujours prêt à s’envoler ou à se sacrifier ; il était à lui-même son aigle et son précipice. Ame étrange.\par
Le tuer ? quelle anxiété ! le délivrer ? quelle responsabilité !\par
Lantenac sauvé, tout serait à recommencer avec la Vendée, comme avec l’hydre tant que la tête n’est pas coupée. En un clin d’œil, et avec une course de météore, toute la flamme, éteinte par la disparition de cet homme, se rallumerait. Lantenac ne se reposerait pas tant qu’il n’aurait point réalisé ce plan exécrable, poser, comme un couvercle de tombe, la monarchie sur la république et l’Angleterre sur la France. Sauver Lantenac, c’était sacrifier la France ; la vie de Lantenac, c’était la mort d’une foule d’êtres innocents, hommes, femmes, enfants, repris par la guerre domestique ; c’était le débarquement des Anglais, le recul de la révolution, les villes saccagées, le peuple déchiré, la Bretagne sanglante, la proie rendue à la griffe. Et  Gauvain, au milieu de toutes sortes de lueurs incertaines et de clartés en sens contraires, voyait vaguement s’ébaucher dans sa rêverie et se poser devant lui ce problème : la mise en liberté du tigre.\par
Et puis, la question reparaissait sous son premier aspect ; la pierre de Sisyphe, qui n’est pas autre chose que la querelle de l’homme avec lui-même, retombait : Lantenac, était-ce donc le tigre ?\par
Peut-être l’avait-il été ; mais l’était-il encore ? Gauvain subissait ces spirales vertigineuses de l’esprit revenant sur lui-même, qui font la pensée pareille à la couleuvre. Décidément, même après examen, pouvait-on nier le dévouement de Lantenac, son abnégation stoïque, son désintéressement superbe ? Quoi ! en présence de toutes les gueules de la guerre civile ouvertes, attester l’humanité ! quoi ! dans le conflit des vérités inférieures, apporter la vérité supérieure ! quoi ! prouver qu’au-dessus des royautés, au-dessus des révolutions, au-dessus des questions terrestres, il y a l’immense attendrissement de l’âme humaine, la protection due aux faibles par les forts, le salut dû à ceux qui sont perdus par ceux qui sont sauvés, la paternité due à tous les enfants par tous les vieillards ! Prouver ces choses magnifiques, et les prouver par le don de sa tête ! Quoi ! être un général, et renoncer à la stratégie, à la bataille, à la revanche ! quoi ! être un royaliste, prendre une balance, mettre dans un plateau le roi de France, une monarchie de quinze siècles, les vieilles lois à rétablir, l’antique société à restaurer, et dans l’autre trois petits paysans quelconques,  et trouver le roi, le trône, le sceptre et les quinze siècles de monarchie légers, pesés à ce poids de trois innocences ! quoi ! tout cela ne serait rien ! quoi ! celui qui a fait cela resterait le tigre et devrait être traité en bête fauve ! non ! non ! non ! ce n’était pas un monstre l’homme qui venait d’illuminer de la clarté d’une action divine le précipice des guerres civiles ! Le porte-glaive s’était métamorphosé en porte-lumière. L’infernal Satan était redevenu le Lucifer céleste. Lantenac s’était racheté de toutes ses barbaries par un acte de sacrifice ; en se perdant matériellement il s’était sauvé moralement ; il s’était refait innocent ; il avait signé sa propre grâce. Est-ce que le droit de se pardonner à soi-même n’existe pas ? Désormais il était vénérable.\par
Lantenac venait d’être extraordinaire. C’était maintenant le tour de Gauvain.\par
Gauvain était chargé de lui donner la réplique.\par
La lutte des passions bonnes et des passions mauvaises faisait en ce moment sur le monde le chaos ; Lantenac, dominant ce chaos, venait d’en dégager l’humanité ; c’était à Gauvain maintenant d’en dégager la famille.\par
Qu’allait-il faire ?\par
Gauvain allait-il tromper la confiance de Dieu ?\par
Non. Et il balbutiait en lui-même : — Sauvons Lantenac.\par
Alors c’est bien. Va, fais les affaires des Anglais. Déserte. Passe à l’ennemi. Sauve Lantenac et trahis la France.\par
 Et il frémissait.\par
Ta solution n’en est pas une, ô songeur ! — Gauvain voyait dans l’ombre le sinistre sourire du sphinx.\par
Cette situation était une sorte de carrefour redoutable où les vérités combattantes venaient aboutir et se confronter, et où se regardaient fixement les trois idées suprêmes de l’homme, l’humanité, la famille, la patrie.\par
Chacune de ces voix prenait à son tour la parole, et chacune à son tour disait vrai. Comment choisir ? Chacune à son tour semblait trouver le joint de sagesse et de justice, et disait : Fais cela. Était-ce cela qu’il fallait faire ? Oui. Non. Le raisonnement disait une chose ; le sentiment en disait une autre ; les deux conseils étaient contraires. Le raisonnement n’est que la raison ; le sentiment est souvent la conscience ; l’un vient de l’homme, l’autre de plus haut.\par
C’est ce qui fait que le sentiment a moins de clarté et plus de puissance.\par
Quelle force pourtant dans la raison sévère !\par
Gauvain hésitait.\par
Perplexités farouches.\par
Deux abîmes s’ouvraient devant Gauvain. Perdre le marquis ? ou le sauver ? Il fallait se précipiter dans l’un ou dans l’autre.\par
Lequel de ces deux gouffres était le devoir ?
 \subsubsection[{III. Le capuchon du chef}]{III \\
Le capuchon du chef}\phantomsection
\label{p3l6c3}
\noindent C’est au devoir en effet qu’on avait affaire.\par
Ce devoir se dressait, sinistre devant Cimourdain, formidable devant Gauvain.\par
Simple devant l’un ; multiple, divers, tortueux, devant l’autre.\par
Minuit sonna, puis une heure du matin.\par
Gauvain s’était, sans s’en apercevoir, insensiblement rapproché de l’entrée de la brèche.\par
L’incendie ne jetait plus qu’une réverbération diffuse et s’éteignait.\par
Le plateau, de l’autre côté de la tour, en avait le reflet, et devenait visible par instants, puis s’éclipsait quand la fumée couvrait le feu. Cette lueur, ravivée par soubresauts et coupée d’obscurités subites, disproportionnait les objets et donnait aux sentinelles du camp des aspects de larves. Gauvain, à travers sa méditation, considérait vaguement ces effacements de  la fumée par le flamboiement et du flamboiement par la fumée. Ces apparitions et ces disparitions de la clarté devant ses yeux avaient on ne sait quelle analogie avec les apparitions et les disparitions de la vérité dans son esprit.\par
Soudain, entre deux tourbillons de fumée, une flammèche envolée du brasier décroissant éclaira vivement le sommet du plateau et fit jaillir la silhouette vermeille d’une charrette. Gauvain regarda cette charrette ; elle était entourée de cavaliers qui avaient des chapeaux de gendarme. Il lui sembla que c’était la charrette que la longue-vue de Guéchamp lui avait fait voir à l’horizon, quelques heures auparavant, au moment où le soleil se couchait. Des hommes étaient sur la charrette et avaient l’air occupés à la décharger. Ce qu’ils retiraient de la charrette paraissait pesant, et rendait par instants un son de ferraille ; il eût été difficile de dire ce que c’était ; cela ressemblait à des charpentes ; deux d’entre eux descendirent et posèrent à terre une caisse qui, à en juger par sa forme, devait contenir un objet triangulaire. La flammèche s’éteignit, tout rentra dans les ténèbres ; Gauvain, l’œil fixe, demeura pensif devant ce qu’il y avait là dans l’obscurité.\par
Des lanternes s’étaient allumées, on allait et venait sur le plateau, mais les formes qui se mouvaient étaient confuses, et d’ailleurs Gauvain d’en bas, et de l’autre côté du ravin, ne pouvait voir que ce qui était tout à fait sur le bord du plateau.\par
Des voix parlaient, mais on ne percevait pas les  paroles. Çà et là des chocs sonnaient sur du bois. On entendait aussi on ne sait quel grincement métallique pareil au bruit d’une faulx qu’on aiguise.\par
Deux heures sonnèrent.\par
Gauvain lentement, et comme quelqu’un qui ferait volontiers deux pas en avant et trois pas en arrière, se dirigea vers la brèche. A son approche, reconnaissant dans la pénombre le manteau et le capuchon galonné du commandant, la sentinelle présenta les armes. Gauvain pénétra dans la salle du rez-de-chaussée, transformée en corps de garde. Une lanterne était pendue à la voûte. Elle éclairait juste assez pour qu’on pût traverser la salle sans marcher sur les hommes du poste, gisant à terre sur de la paille, et la plupart endormis.\par
Ils étaient couchés là ; ils s’y étaient battus quelques heures auparavant ; la mitraille, éparse sous eux en grains de fer et de plomb, et mal balayée, les gênait un peu pour dormir ; mais ils étaient fatigués, et ils se reposaient. Cette salle avait été le lieu horrible ; là on avait attaqué ; là on avait rugi, hurlé, grincé, frappé, tué, expiré ; beaucoup des leurs étaient tombés morts sur ce pavé où ils se couchaient assoupis ; cette paille qui servait à leur sommeil buvait le sang de leurs camarades ; maintenant c’était fini, le sang était étanché, les sabres étaient essuyés, les morts étaient morts ; eux ils dormaient paisibles. Telle est la guerre. Et puis, demain, tout le monde aura le même sommeil.\par
A l’entrée de Gauvain, quelques-uns de ces  hommes assoupis se levèrent, entre autres l’officier qui commandait le poste. Gauvain lui désigna la porte du cachot.\par
— Ouvrez-moi, dit-il.\par
Les verrous furent tirés, la porte s’ouvrit.\par
Gauvain entra dans le cachot.\par
La porte se referma derrière lui.\par
  \subsection[{Livre septième. Féodalité et Révolution}]{Livre septième \\
Féodalité et Révolution}\phantomsection
\label{p3l7}
\subsubsection[{I. L’ancêtre}]{I \\
L’ancêtre}\phantomsection
\label{p3l7c1}
\noindent Une lampe était posée sur la dalle de la crypte, à côté du soupirail carré de l’oubliette.\par
On apercevait aussi sur la dalle la cruche pleine d’eau, le pain de munition et la botte de paille. La crypte étant taillée dans le roc, le prisonnier qui eût eu la fantaisie de mettre le feu à la paille eût perdu sa peine ; aucun risque d’incendie pour la prison, certitude d’asphyxie pour le prisonnier.\par
A l’instant où la porte tourna sur ses gonds, le marquis marchait dans son cachot ; va-et-vient machinal propre à tous les fauves mis en cage.\par
Au bruit que fit la porte en s’ouvrant puis en se refermant, il leva la tête, et la lampe qui était à terre entre Gauvain et le marquis éclaira ces deux hommes en plein visage.\par
 Ils se regardèrent, et ce regard était tel qu’il les fit tous deux immobiles.\par
Le marquis éclata de rire et s’écria :\par
— Bonjour, monsieur. Voilà pas mal d’années que je n’ai eu la bonne fortune de vous rencontrer. Vous me faites la grâce de venir me voir. Je vous remercie. Je ne demande pas mieux que de causer un peu. Je commençais à m’ennuyer. Vos amis perdent le temps ; des constatations d’identité, des cours martiales, c’est long toutes ces manières-là. J’irais plus vite en besogne. Je suis ici chez moi. Donnez-vous la peine d’entrer. Eh bien, qu’est-ce que vous dites de tout ce qui se passe ? C’est original, n’est-ce pas ? Il y avait une fois un roi et une reine ; le roi, c’était le roi ; la reine, c’était la France. On a tranché la tête au roi et marié la reine à Robespierre ; ce monsieur et cette dame ont eu une fille qu’on nomme la guillotine, et avec laquelle il paraît que je ferai connaissance demain matin. J’en serai charmé. Comme de vous voir. Venez-vous pour cela ? Avez-vous monté en grade ? Seriez-vous le bourreau ? Si c’est une simple visite d’amitié, j’en suis touché. Monsieur le vicomte, vous ne savez peut-être plus ce que c’est qu’un gentilhomme.Eh bien, en voilà un ; c’est moi. Regardez ça. C’est curieux ; ça croit en Dieu, ça croit à la tradition, ça croit à la famille, ça croit à ses aïeux, ça croit à l’exemple de son père, à la fidélité, à la loyauté, au devoir envers son prince, au respect des vieilles lois, à la vertu, à la justice ; et ça vous ferait fusiller avec plaisir. Ayez, je vous prie, la bonté de vous asseoir. Sur le pavé, c’est vrai ; car il  n’y a pas de fauteuil dans ce salon ; mais qui vit dans la boue peut s’asseoir par terre. Je ne dis pas cela pour vous offenser, car ce que nous appelons la boue, vous l’appelez la nation. Vous n’exigez sans doute pas que je crie Liberté, Égalité, Fraternité ? Ceci est une ancienne chambre de ma maison ; jadis les seigneurs y mettaient les manants ; maintenant les manants y mettent les seigneurs. Ces niaiseries-là se nomment une révolution. Il paraît qu’on me coupera le cou d’ici à trente-six heures. Je n’y vois pas d’inconvénient. Par exemple, si l’on était poli, on m’aurait envoyé ma tabatière, qui est là-haut dans la chambre des miroirs, où vous avez joué tout enfant et où je vous ai fait sauter sur mes genoux. Monsieur, je vais vous apprendre une chose, vous vous appelez Gauvain, et, chose bizarre, vous avez du sang noble dans les veines, pardieu, le même sang que le mien, et ce sang qui fait de moi un homme d’honneur fait de vous un gueusard. Telles sont les particularités. Vous me direz que ce n’est pas votre faute. Ni la mienne. Parbleu, on est un malfaiteur sans le savoir. Cela tient à l’air qu’on respire ; dans des temps comme les nôtres, on n’est pas responsable de ce qu’on fait, la révolution est coquine pour tout le monde, et tous vos grands criminels sont de grands innocents. Quelles buses ! A commencer par vous. Souffrez que je vous admire. Oui, j’admire un garçon tel que vous, qui, homme de qualité, bien situé dans l’état, ayant un grand sang à répandre pour les grandes causes, vicomte de cette Tour-Gauvain, prince de Bretagne, pouvant être duc par  droit et pair de France par héritage, ce qui est à peu près tout ce que peut désirer ici-bas un homme de bon sens, s’amuse, étant ce qu’il est, à être ce que vous êtes, si bien qu’il fait à ses ennemis l’effet d’un scélérat et à ses amis l’effet d’un imbécile. A propos, faites mes compliments à monsieur l’abbé Cimourdain.\par
Le marquis parlait à son aise, paisiblement, sans rien souligner, avec sa voix de bonne compagnie, avec son œil clair et tranquille, les deux mains dans ses goussets. Il s’interrompit, respira longuement, et reprit :\par
— Je ne vous cache pas que j’ai fait ce que j’ai pu pour vous tuer. Tel que vous me voyez, j’ai trois fois, moi-même, en personne, pointé un canon sur vous. Procédé discourtois, je l’avoue ; mais ce serait faire fond sur une mauvaise maxime que de s’imaginer qu’en guerre l’ennemi cherche à nous être agréable. Car nous sommes en guerre, monsieur mon neveu. Tout est à feu et à sang. C’est pourtant vrai qu’on a tué le roi. Joli siècle.\par
Il s’arrêta encore, puis poursuivit :\par
— Quand on pense que rien de tout cela ne serait arrivé si l’on avait pendu Voltaire et mis Rousseau aux galères ! Ah ! les gens d’esprit, quel fléau ! Ah çà, qu’est-ce que vous lui reprochez à cette monarchie ? C’est vrai, on envoyait l’abbé Pucelle à son abbaye de Corbigny, en lui laissant le choix de la voiture et tout le temps qu’il voudrait pour faire le chemin ; et quant à votre monsieur Titon, qui avait été, s’il vous plaît,  un fort débauché, et qui allait chez les filles avant d’aller aux miracles du diacre Pâris, on le transférait du château de Vincennes au château de Ham en Picardie, qui est, j’en conviens, un assez vilain endroit. Voilà les griefs ; je m’en souviens ; j’ai crié aussi dans mon temps ; j’ai été aussi bête que vous.\par
Le marquis tâta sa poche comme s’il y cherchait sa tabatière, et continua :\par
— Mais pas aussi méchant. On parlait pour parler. Il y avait aussi la mutinerie des enquêtes et requêtes ; et puis ces messieurs les philosophes sont venus, on a brûlé les écrits au lieu de brûler les auteurs, les cabales de la cour s’en sont mêlées, il y a eu tous ces benêts, Turgot, Quesnay, Malesherbes, les physiocrates, et cætera, et le grabuge a commencé. Tout est venu des écrivailleurs et des rimailleurs. L’encyclopédie ! Diderot ! d’Alembert ! Ah ! les méchants bélîtres ! Un homme bien né comme ce roi de Prusse, avoir donné là-dedans ! Moi, j’eusse supprimé tous les gratteurs de papier. Ah ! nous étions des justiciers, nous autres. On peut voir ici, sur le mur, la marque des roues d’écartèlement. Nous ne plaisantions pas. Non, non, point d’écrivassiers ! Tant qu’il y aura des Arouet, il y aura des Marat. Tant qu’il y aura des grimauds qui griffonnent, il y aura des gredins qui assassinent ; tant qu’il y aura de l’encre, il y aura de la noirceur ; tant que la patte de l’homme tiendra la plume de l’oie, les sottises frivoles engendreront les sottises atroces. Les livres font les crimes. Le mot chimère a deux sens, il signifie rêve,  et il signifie monstre. Comme on se paye de billevesées ! Qu’est-ce que vous nous chantez avec vos droits ? Droits de l’homme ! droits du peuple ! Cela est-il assez creux, assez stupide, assez imaginaire, assez vide de sens ! Moi, quand je dis : Havoise, sœur de Conan II, apporta le comté de Bretagne à Hoël, comte de Nantes et de Cornouailles, qui laissa le trône à Alain Fergant, oncle de Berthe, qui épousa Alain le Noir, seigneur de la Roche-sur-Yon, et en eut Conan le Petit, aïeul de Guy ou Gauvain de Thouars, notre ancêtre, je dis une chose claire, et voilà un droit. Mais vos drôles, vos marauds, vos croquants, qu’appellent-ils leurs droits ? Le déicide et le régicide. Si ce n’est pas hideux ! Ah ! les maroufles ! J’en suis fâché pour vous, monsieur ; mais vous êtes de ce fier sang de Bretagne ; vous et moi, nous avons Gauvain de Thouars pour grand-père ; nous avons encore pour aïeul ce grand duc de Montbazon qui fut pair de France et honoré du collier des ordres, qui attaqua le faubourg de Tours et fut blessé au combat d’Arques, et qui mourut grand-veneur de France en sa maison de Couzières en Touraine, âgé de quatrevingt-six ans. Je pourrais vous parler encore du duc de Laudunois, fils de la dame de la Garnache, de Claude de Lorraine, duc de Chevreuse, et de Henri de Lenoncourt, et de Françoise de Laval-Boisdauphin. Mais à quoi bon ? Monsieur a l’honneur d’être un idiot, et il tient à être l’égal de mon palefrenier. Sachez ceci, j’étais déjà un vieil homme que vous étiez encore un marmot. Je vous ai mouché, morveux, et je vous moucherais encore. En  grandissant, vous avez trouvé moyen de vous rapetisser. Depuis que nous ne nous sommes vus, nous sommes allés chacun de notre côté, moi du côté de l’honnêteté, vous du côté opposé. Ah ! je ne sais pas comment tout cela finira, mais messieurs vos amis sont de fiers misérables. Ah ! oui, c’est beau, j’en tombe d’accord, les progrès sont superbes, on a supprimé dans l’armée la peine de la chopine d’eau infligée trois jours consécutifs au soldat ivrogne ; on a le maximum, la Convention, l’évêque Gobel, monsieur Chaumette et monsieur Hébert, et l’on extermine en masse tout le passé, depuis la Bastille jusqu’à l’almanach. On remplace les saints par les légumes. Soit, messieurs les citoyens, soyez les maîtres, régnez, prenez vos aises, donnez-vous-en, ne vous gênez pas. Tout cela n’empêchera point que la religion ne soit la religion, que la royauté n’emplisse quinze cents ans de notre histoire, et que la vieille seigneurie française, même décapitée, ne soit plus haute que vous. Quant à vos chicanes sur le droit historique des races royales, nous en haussons les épaules. Chilpéric, au fond, n’était qu’un moine appelé Daniel ; ce fut Rainfroy qui inventa Chilpéric pour ennuyer Charles Martel ; nous savons ces choses-là aussi bien que vous. Ce n’est pas la question. La question est ceci : être un grand royaume ; être la vieille France ; être ce pays d’arrangement magnifique où l’on considère premièrement la personne sacrée des monarques, seigneurs absolus de l’état, puis les princes, puis les officiers de la couronne, pour les armes sur terre et sur mer,  pour l’artillerie, direction et surintendance des finances. Ensuite il y a la justice souveraine et subalterne, suivie du maniement des gabelles et recettes générales, et enfin la police du royaume dans ses trois ordres. Voilà qui était beau et noblement ordonné ; vous l’avez détruit. Vous avez détruit les provinces, comme de lamentables ignorants que vous êtes, sans même vous douter de ce que c’était que les provinces. Le génie de la France est composé du génie même du continent, et chacune des provinces de France représentait une vertu de l’Europe ; la franchise de l’Allemagne était en Picardie, la générosité de la Suède en Champagne, l’industrie de la Hollande en Bourgogne, l’activité de la Pologne en Languedoc, la gravité de l’Espagne en Gascogne, la sagesse de l’Italie en Provence, la subtilité de la Grèce en Normandie, la fidélité de la Suisse en Dauphiné. Vous ne saviez rien de tout cela ; vous avez cassé, brisé, fracassé, démoli, et vous avez été tranquillement des bêtes brutes. Ah ! vous ne voulez plus avoir de nobles ! Eh bien, vous n’en aurez plus. Faites-en votre deuil. Vous n’aurez plus de paladins, vous n’aurez plus de héros. Bonsoir les grandeurs anciennes. Trouvez-moi un d’Assas à présent ! Vous avez tous peur pour votre peau. Vous n’aurez plus les chevaliers de Fontenoy qui saluaient avant de tuer ; vous n’aurez plus les combattants en bas de soie du siège de Lérida ; vous n’aurez plus de ces fières journées militaires où les panaches passaient comme des météores ; vous êtes un peuple fini ; vous subirez ce viol, l’invasion ; si Alaric II revient, il ne  trouvera plus en face de lui Clovis ; si Abdérame revient, il ne trouvera plus en face de lui Charles Martel ; si les Saxons reviennent, ils ne trouveront plus devant eux Pépin ; vous n’aurez plus Agnadel, Rocroy, Lens, Staffarde, Nerwinde, Steinkerque, la Marsaille, Raucoux, Lawfeld, Mahon ; vous n’aurez plus Marignan avec François I\textsuperscript{er} ; vous n’aurez plus Bouvines avec Philippe-Auguste faisant prisonnier, d’une main, Renaud, comte de Boulogne, et, de l’autre, Ferrand, comte de Flandre. Vous aurez Azincourt, mais vous n’aurez plus pour s’y faire tuer, enveloppé de son drapeau, le sieur de Bacqueville, le grand porte-oriflamme ! Allez ! allez ! faites ! Soyez les hommes nouveaux. Devenez petits !\par
Le marquis fit un moment silence, et repartit :\par
— Mais laissez-nous grands. Tuez les rois, tuez les nobles, tuez les prêtres, abattez, ruinez, massacrez, foulez tout aux pieds, mettez les maximes antiques sous le talon de vos bottes, piétinez le trône, trépignez l’autel, écrasez Dieu, dansez dessus ! c’est votre affaire. Vous êtes des traîtres et des lâches, incapables de dévouement et de sacrifice. J’ai dit. Maintenant faites-moi guillotiner, monsieur le vicomte. J’ai l’honneur d’être votre très humble.\par
Et il ajouta :\par
— Ah ! je vous dis vos vérités ! Qu’est-ce que cela me fait ? je suis mort.\par
— Vous êtes libre, dit Gauvain.\par
Et Gauvain s’avança vers le marquis, défit son manteau de commandant, le lui jeta sur les épaules,  et lui rabattit le capuchon sur les yeux. Tous deux étaient de même taille.\par
— Eh bien, qu’est-ce que tu fais ? dit le marquis.\par
Gauvain éleva la voix et cria :\par
— Lieutenant, ouvrez-moi.\par
La porte s’ouvrit.\par
Gauvain cria :\par
— Vous aurez soin de refermer la porte derrière moi.\par
Et il poussa dehors le marquis stupéfait.\par
La salle basse, transformée en corps de garde, avait, on s’en souvient, pour tout éclairage, une lanterne de corne qui faisait tout voir trouble, et donnait plus de nuit que de jour. Dans cette lueur confuse, ceux des soldats qui ne dormaient pas virent marcher au milieu d’eux, se dirigeant vers la sortie, un homme de haute stature ayant le manteau et le capuchon galonné de commandant en chef ; ils firent le salut militaire, et l’homme passa.\par
Le marquis, lentement, traversa le corps de garde, traversa la brèche, non sans s’y heurter la tête plus d’une fois, et sortit.\par
La sentinelle, croyant voir Gauvain, lui présenta les armes.\par
Quand il fut dehors, ayant sous ses pieds l’herbe des champs, à deux cents pas la forêt, et devant lui l’espace, la nuit, la liberté, la vie, il s’arrêta et demeura un moment immobile comme un homme qui s’est laissé faire, qui a cédé à la surprise, et qui, ayant profité d’une porte ouverte, cherche s’il a bien ou mal  agi, hésite avant d’aller plus loin, et donne audience à une dernière pensée. Après quelques secondes de rêverie attentive, il leva sa main droite, fit claquer son médius contre son pouce et dit : Ma foi !\par
Et il s’en alla.\par
La porte du cachot s’était refermée. Gauvain était dedans.
 \subsubsection[{II. La cour martiale}]{II \\
La cour martiale}\phantomsection
\label{p3l7c2}
\noindent Tout alors dans les cours martiales était à peu près discrétionnaire. Dumas, à l’Assemblée législative, avait esquissé une ébauche de législation militaire, retravaillée plus tard par Talot au conseil des Cinq-Cents, mais le code définitif des conseils de guerre n’a été rédigé que sous l’empire. C’est de l’empire que date, par parenthèse, l’obligation imposée aux tribunaux militaires de ne recueillir les votes qu’en commençant par le grade inférieur. Sous la révolution, cette loi n’existait pas.\par
En 1793, le président d’un tribunal militaire était presque à lui seul tout le tribunal ; il choisissait les membres, classait l’ordre des grades, réglait le mode du vote ; il était le maître en même temps que le juge.\par
Cimourdain avait désigné, pour prétoire de la cour martiale, cette salle même du rez-de-chaussée où avait été la retirade et où était maintenant le corps de garde. Il tenait à tout abréger, le chemin de la prison au tribunal et le trajet du tribunal à l’échafaud.\par
A midi, conformément à ses ordres, la cour était en séance avec l’apparat que voici : trois chaises de  paille, une table de sapin, deux chandelles allumées, un tabouret devant la table.\par
Les chaises étaient pour les juges et le tabouret pour l’accusé. Aux deux bouts de la table il y avait deux autres tabourets, l’un pour le commissaire-auditeur qui était un fourrier, l’autre pour le greffier qui était un caporal.\par
Il y avait sur la table un bâton de cire rouge, le sceau de la république en cuivre, deux écritoires, des dossiers de papier blanc, et deux affiches imprimées, étalées toutes grandes ouvertes, contenant, l’une, la mise hors la loi, l’autre, le décret de la Convention.\par
La chaise du milieu était adossée à un faisceau de drapeaux tricolores ; dans ces temps de rude simplicité, un décor était vite posé, et il fallait peu de temps pour changer un corps de garde en cour de justice.\par
La chaise du milieu, destinée au président, faisait face à la porte du cachot.\par
Pour public, les soldats.\par
Deux gendarmes gardaient la sellette.\par
Cimourdain était assis sur la chaise du milieu, ayant à sa droite le capitaine Guéchamp, premier juge, et à sa gauche le sergent Radoub, deuxième juge.\par
Il avait sur la tête son chapeau à panache tricolore, à son côté son sabre, dans sa ceinture ses deux pistolets. Sa balafre, qui était d’un rouge vif, ajoutait à son air farouche.\par
Radoub avait fini par se faire panser. Il avait autour de la tête un mouchoir sur lequel s’élargissait lentement une plaque de sang.\par
 A midi, l’audience n’était pas encore ouverte, une estafette, dont on entendait dehors piaffer le cheval, était debout près de la table du tribunal. Cimourdain écrivait. Il écrivait ceci :\par
« Citoyens membres du comité de salut public,\par
« Lantenac est pris. Il sera exécuté demain. »\par
Il data et signa, plia et cacheta la dépêche, et la remit à l’estafette, qui partit.\par
Cela fait, Cimourdain dit d’une voix haute :\par
— Ouvrez le cachot.\par
Les deux gendarmes tirèrent les verrous, ouvrirent le cachot, et y entrèrent.\par
Cimourdain leva la tête, croisa les bras, regarda la porte, et cria :\par
— Amenez le prisonnier.\par
Un homme apparut entre les deux gendarmes, sous le cintre de la porte ouverte.\par
C’était Gauvain.\par
Cimourdain eut un tressaillement.\par
— Gauvain ! s’écria-t-il.\par
Et il reprit :\par
— Je demande le prisonnier.\par
— C’est moi, dit Gauvain.\par
— Toi ?\par
— Moi.\par
— Et Lantenac ?\par
— Il est libre.\par
— Libre !\par
— Oui.\par
— Évadé ?\par
 — Évadé.\par
Cimourdain balbutia avec un tremblement :\par
— En effet, ce château est à lui, il en connaît toutes les issues, l’oubliette communique peut-être à quelque sortie, j’aurais dû y songer, il aura trouvé moyen de s’enfuir, il n’aura eu besoin pour cela de l’aide de personne.\par
— Il a été aidé, dit Gauvain.\par
— A s’évader ?\par
— A s’évader.\par
— Qui l’a aidé ?\par
— Moi.\par
— Toi !\par
— Moi.\par
— Tu rêves !\par
— Je suis entré dans le cachot, j’étais seul avec le prisonnier, j’ai ôté mon manteau, je le lui ai mis sur le dos, je lui ai rabattu le capuchon sur le visage, il est sorti à ma place, et je suis resté à la sienne. Me voici.\par
— Tu n’as pas fait cela !\par
— Je l’ai fait.\par
— C’est impossible.\par
— C’est réel.\par
— Amenez-moi Lantenac !\par
— Il n’est plus ici. Les soldats, lui voyant le manteau de commandant, l’ont pris pour moi et l’ont laissé passer. Il faisait encore nuit.\par
— Tu es fou.\par
— Je dis ce qui est.\par
 Il y eut un silence. Cimourdain bégaya :\par
— Alors tu mérites...\par
— La mort, dit Gauvain.\par
Cimourdain était pâle comme une tête coupée. Il était immobile comme un homme sur qui vient de tomber la foudre. Il semblait ne plus respirer. Une grosse goutte de sueur perla sur son front.\par
Il raffermit sa voix et dit :\par
— Gendarmes, faites asseoir l’accusé.\par
Gauvain se plaça sur le tabouret.\par
Cimourdain reprit :\par
— Gendarmes, tirez vos sabres.\par
C’était la formule usitée quand l’accusé était sous le poids d’une sentence capitale.\par
Les gendarmes tirèrent leurs sabres.\par
La voix de Cimourdain avait repris son accent ordinaire.\par
— Accusé, dit-il, levez-vous.\par
Il ne tutoyait plus Gauvain.
 \subsubsection[{III. Les votes}]{III \\
Les votes}\phantomsection
\label{p3l7c3}
\noindent Gauvain se leva.\par
— Comment vous nommez-vous ? demanda Cimourdain.\par
Gauvain répondit :\par
— Gauvain.\par
Cimourdain continua l’interrogatoire.\par
— Qui êtes-vous ?\par
— Je suis commandant en chef de la colonne expéditionnaire des Côtes-du-Nord.\par
— Êtes-vous parent ou allié de l’homme évadé ?\par
— Je suis son petit-neveu.\par
— Vous connaissez le décret de la Convention ?\par
— J’en vois l’affiche sur votre table.\par
— Qu’avez-vous à dire sur ce décret ?\par
— Que je l’ai contresigné, que j’en ai ordonné l’exécution, et que c’est moi qui ai fait faire cette affiche au bas de laquelle est mon nom.\par
— Faites choix d’un défenseur.\par
— Je me défendrai moi-même.\par
— Vous avez la parole.\par
 Cimourdain était redevenu impassible. Seulement son impassibilité ressemblait moins au calme d’un homme qu’à la tranquillité d’un rocher.\par
Gauvain demeura un moment silencieux et comme recueilli.\par
Cimourdain reprit :\par
— Qu’avez-vous à dire pour votre défense ?\par
Gauvain leva lentement la tête, ne regarda personne, et répondit :\par
— Ceci : une chose m’a empêché d’en voir une autre ; une bonne action, vue de trop près, m’a caché cent actions criminelles ; d’un côté un vieillard, de l’autre des enfants, tout cela s’est mis entre moi et le devoir. J’ai oublié les villages incendiés, les champs ravagés, les prisonniers massacrés, les blessés achevés, les femmes fusillées, j’ai oublié la France livrée à l’Angleterre ; j’ai mis en liberté le meurtrier de la patrie. Je suis coupable. En parlant ainsi, je semble parler contre moi ; c’est une erreur. Je parle pour moi. Quand le coupable reconnaît sa faute, il sauve la seule chose qui vaille la peine d’être sauvée, l’honneur.\par
— Est-ce là, repartit Cimourdain, tout ce que vous avez à dire pour votre défense ?\par
— J’ajoute qu’étant le chef, je devais l’exemple, et qu’à votre tour, étant les juges, vous le devez.\par
— Quel exemple demandez-vous ?\par
— Ma mort.\par
— Vous la trouvez juste ?\par
— Et nécessaire.\par
 — Asseyez-vous.\par
Le fourrier, commissaire-auditeur, se leva et donna lecture, premièrement, de l’arrêté qui mettait hors la loi le ci-devant marquis de Lantenac ; deuxièmement, du décret de la Convention édictant la peine capitale contre quiconque favoriserait l’évasion d’un rebelle prisonnier. Il termina par les quelques lignes imprimées au bas de l’affiche du décret, intimant défense « de porter aide et secours » au rebelle susnommé « sous peine de mort », et signées : \emph{Le commandant en chef de la colonne expéditionnaire}, G{\scshape auvain}.\par
Ces lectures faites, le commissaire-auditeur se rassit.\par
Cimourdain croisa les bras et dit :\par
— Accusé, soyez attentif. Public, écoutez, regardez, et taisez-vous. Vous avez devant vous la loi. Il va être procédé au vote. La sentence sera rendue à la majorité simple. Chaque juge opinera à son tour, à haute voix, en présence de l’accusé, la justice n’ayant rien à cacher.\par
Cimourdain continua :\par
— La parole est au premier juge. Parlez, capitaine Guéchamp.\par
Le capitaine Guéchamp ne semblait voir ni Cimourdain, ni Gauvain. Ses paupières abaissées cachaient ses yeux immobiles fixés sur l’affiche du décret et la considérant comme on considérerait un gouffre. Il dit :\par
— La loi est formelle. Un juge est plus et moins qu’un homme ; il est moins qu’un homme, car il n’a  pas de cœur ; il est plus qu’un homme, car il a le glaive. L’an 414 de Rome, Manlius fit mourir son fils pour le crime d’avoir vaincu sans son ordre. La discipline violée voulait une expiation. Ici, c’est la loi qui a été violée, et la loi est plus haute encore que la discipline. Par suite d’un accès de pitié, la patrie est remise en danger. La pitié peut avoir les proportions d’un crime. Le commandant Gauvain a fait évader le rebelle Lantenac. Gauvain est coupable. Je vote la mort.\par
— Écrivez, greffier, dit Cimourdain.\par
Le greffier écrivit : « Capitaine Guéchamp : la mort. »\par
Gauvain éleva la voix.\par
— Guéchamp, dit-il, vous avez bien voté, et je vous remercie.\par
Cimourdain reprit :\par
— La parole est au deuxième juge. Parlez, sergent Radoub.\par
Radoub se leva, se tourna vers Gauvain et fit à l’accusé le salut militaire. Puis il s’écria :\par
— Si c’est ça, alors, guillotinez-moi. Car j’en donne ici ma nom de Dieu de parole d’honneur la plus sacrée, je voudrais avoir fait, d’abord ce qu’a fait le vieux, et ensuite ce qu’a fait mon commandant. Quand j’ai vu cet individu de quatrevingts ans se jeter dans le feu pour en tirer les trois mioches, j’ai dit : Bonhomme, tu es un brave homme ! et quand j’apprends que c’est mon commandant qui a sauvé ce vieux de votre bête de guillotine, mille noms de noms,  je dis : Mon commandant, vous devriez être mon général, et vous êtes un vrai homme, et moi, tonnerre ! je vous donnerais la croix de Saint-Louis, s’il y avait encore des croix, s’il y avait encore des saints, et s’il y avait encore des louis ! Ah çà ! est-ce qu’on va être des imbéciles à présent ? Si c’est pour des choses comme ça qu’on a gagné la bataille de Jemmapes, la bataille de Valmy, la bataille de Fleurus et la bataille de Wattignies, alors il faut le dire. Comment ! voilà le commandant Gauvain qui depuis quatre mois mène toutes ces bourriques de royalistes tambour battant, et qui sauve la république à coups de sabre, et qui a fait la chose de Dol où il fallait joliment de l’esprit, et, quand vous avez cet homme-là, vous tâchez de ne plus l’avoir ! et, au lieu d’en faire votre général, vous voulez lui couper le cou ! je dis que c’est à se jeter la tête la première par-dessus le parapet du Pont-Neuf, et que vous-même, citoyen Gauvain, mon commandant, si, au lieu d’être mon général, vous étiez mon caporal, je vous dirais que vous avez dit de fichues bêtises tout à l’heure. Le vieux a bien fait de sauver les enfants, vous avez bien fait de sauver le vieux, et si l’on guillotine les gens parce qu’ils ont fait de bonnes actions, alors va-t’en à tous les diables, je ne sais plus du tout de quoi il est question. Il n’y a plus de raison pour qu’on s’arrête. C’est pas vrai, n’est-ce pas, tout ça ? Je me pince pour savoir si je suis éveillé. Je ne comprends pas. Il fallait donc que le vieux laisse brûler les mômes tout vifs, il fallait donc que mon commandant laisse couper le cou au  vieux. Tenez, oui, guillotinez-moi. J’aime autant ça. Une supposition, les mioches seraient morts, le bataillon du Bonnet-Rouge était déshonoré. Est-ce que c’est ça qu’on voulait ? Alors mangeons-nous les uns les autres. Je me connais en politique aussi bien que vous qui êtes là, j’étais du club de la section des Piques. Sapristi ! nous nous abrutissons à la fin ! Je résume ma façon de voir. Je n’aime pas les choses qui ont l’inconvénient de faire qu’on ne sait plus du tout où on en est. Pourquoi diable nous faisons-nous tuer ? Pour qu’on nous tue notre chef ! Pas de ça, Lisette. Je veux mon chef ! Il me faut mon chef ! Je l’aime encore mieux aujourd’hui qu’hier. L’envoyer à la guillotine, mais vous me faites rire ! Tout ça, tout ça, nous n’en voulons pas. J’ai écouté. On dira tout ce qu’on voudra. D’abord, pas possible.\par
Et Radoub se rassit. Sa blessure s’était rouverte. Un filet de sang qui sortait du bandeau coulait le long de son cou, de l’endroit où avait été son oreille.\par
Cimourdain se tourna vers Radoub.\par
— Vous votez pour que l’accusé soit absous ?\par
— Je vote, dit Radoub, pour qu’on le fasse général.\par
— Je vous demande si vous votez pour qu’il soit acquitté.\par
— Je vote pour qu’on le fasse le premier de la république.\par
— Sergent Radoub, votez-vous pour que le commandant Gauvain soit acquitté, oui ou non ?\par
— Je vote pour qu’on me coupe la tête à sa place.\par
 — Acquittement, dit Cimourdain. Écrivez, greffier.\par
Le greffier écrivit : « Sergent Radoub : acquittement. »\par
Puis le greffier dit :\par
— Une voix pour la mort. Une voix pour l’acquittement. Partage.\par
C’était à Cimourdain de voter.\par
Il se leva. Il ôta son chapeau et le posa sur la table.\par
Il n’était plus pâle ni livide. Sa face était couleur de terre.\par
Tous ceux qui étaient là eussent été couchés dans des suaires que le silence n’eût pas été plus profond.\par
Cimourdain dit d’une voix grave, lente et ferme :\par
— Accusé Gauvain, la cause est entendue. Au nom de la république, la cour martiale, à la majorité de deux voix contre une...\par
Il s’interrompit, il eut comme un temps d’arrêt ; hésitait-il devant la mort ? hésitait-il devant la vie ? toutes les poitrines étaient haletantes. Cimourdain continua :\par
— ... Vous condamne à la peine de mort.\par
Son visage exprimait la torture du triomphe sinistre. Quand Jacob dans les ténèbres se fit bénir par l’ange qu’il avait terrassé, il devait avoir ce sourire effrayant.\par
Ce ne fut qu’une lueur, et cela passa. Cimourdain redevint de marbre, se rassit, remit son chapeau sur sa tête, et ajouta :\par
 — Gauvain, vous serez exécuté demain, au lever du soleil.\par
Gauvain se leva, salua et dit :\par
— Je remercie la cour.\par
— Emmenez le condamné, dit Cimourdain.\par
Cimourdain fit un signe, la porte du cachot se rouvrit, Gauvain y rentra, le cachot se referma. Les deux gendarmes restèrent en faction des deux côtés de la porte, le sabre nu.\par
On emporta Radoub, qui venait de tomber sans connaissance.
 \subsubsection[{IV. Après Cimourdain juge, Cimourdain maître}]{IV \\
Après Cimourdain juge, Cimourdain maître}\phantomsection
\label{p3l7c4}
\noindent Un camp, c’est un guêpier. En temps de révolution surtout. L’aiguillon civique, qui est dans le soldat, sort volontiers et vite, et ne se gêne pas pour piquer le chef après avoir chassé l’ennemi. La vaillante troupe qui avait pris la Tourgue eut des bourdonnements variés ; d’abord contre le commandant Gauvain quand on apprit l’évasion de Lantenac. Lorsqu’on vit Gauvain sortir du cachot où l’on croyait tenir Lantenac, ce fut comme une commotion électrique, et en moins d’une minute tout le corps fut informé. Un murmure éclata dans la petite armée. Ce premier murmure fut : — Ils sont en train de juger Gauvain. Mais c’est pour la frime. Fiez-vous donc aux ci-devant et aux calotins ! Nous venons de voir un vicomte qui sauve un marquis, et nous allons voir un prêtre qui absout un noble ! — Quand on sut la condamnation de Gauvain, il y eut un deuxième murmure :  — Voilà qui est fort ! notre chef, notre brave chef, notre jeune commandant, un héros ! C’est un vicomte, eh bien, il n’en a que plus de mérite à être républicain ! Comment ! lui, le libérateur de Pontorson, de Villedieu, de Pont-au-Beau ! le vainqueur de Dol et de la Tourgue ! celui par qui nous sommes invincibles ! celui qui est l’épée de la république dans la Vendée ! l’homme qui depuis cinq mois tient tête aux chouans et répare toutes les sottises de Léchelle et des autres ! ce Cimourdain ose le condamner à mort ! pourquoi ? parce qu’il a sauvé un vieillard qui avait sauvé trois enfants ! un prêtre tuer un soldat ! —\par
Ainsi grondait le camp victorieux et mécontent. Une sombre colère entourait Cimourdain. Quatre mille hommes contre un seul, il semble que ce soit une force ; ce n’en est pas une. Ces quatre mille hommes étaient une foule, et Cimourdain était une volonté. On savait que Cimourdain fronçait aisément le sourcil, et il n’en fallait pas davantage pour tenir l’armée en respect. Dans ces temps sévères, il suffisait que l’ombre du comité de salut public fût derrière un homme pour faire cet homme redoutable et pour faire aboutir l’imprécation au chuchotement et le chuchotement au silence. Avant comme après les murmures, Cimourdain restait l’arbitre du sort de Gauvain comme du sort de tous. On savait qu’il n’y avait rien à lui demander et qu’il n’obéirait qu’à sa conscience, voix surhumaine entendue de lui seul. Tout dépendait de lui. Ce qu’il avait fait comme juge martial, seul il pouvait le défaire comme délégué civil. Seul il pouvait faire grâce. Il avait pleins pouvoirs ;  d’un signe il pouvait mettre Gauvain en liberté ; il était le maître de la vie et de la mort ; il commandait à la guillotine. En ce moment tragique, il était l’homme suprême.\par
On ne pouvait qu’attendre.\par
La nuit vint.
 \subsubsection[{V. Le cachot}]{V \\
Le cachot}\phantomsection
\label{p3l7c5}
\noindent La salle de justice était redevenue corps de garde ; le poste était doublé comme la veille ; deux factionnaires gardaient la porte du cachot fermé.\par
Vers minuit, un homme, qui tenait une lanterne à la main, traversa le corps de garde, se fit reconnaître, et se fit ouvrir le cachot.\par
C’était Cimourdain.\par
Il entra, et la porte resta entr’ouverte derrière lui.\par
Le cachot était ténébreux et silencieux. Cimourdain fit un pas dans cette obscurité, posa la lanterne à terre, et s’arrêta. On entendait dans l’ombre la respiration égale d’un homme endormi. Cimourdain écouta, pensif, ce bruit paisible.\par
Gauvain était au fond du cachot, sur la botte de paille. C’était son souffle qu’on entendait. Il dormait profondément.\par
Cimourdain s’avança avec le moins de bruit possible, vint tout près et se mit à regarder Gauvain ; une mère regardant son nourrisson dormir n’aurait pas un plus tendre et plus inexprimable regard. Ce regard était plus fort peut-être que Cimourdain ; Cimourdain appuya, comme font quelquefois les enfants,  ses deux poings sur ses yeux, et demeura un moment immobile. Puis il s’agenouilla, souleva doucement la main de Gauvain, et posa ses lèvres dessus.\par
Gauvain fit un mouvement. Il ouvrit les yeux avec le vague étonnement du réveil en sursaut. La lanterne éclairait faiblement la cave. Il reconnut Cimourdain.\par
— Tiens, dit-il, c’est vous, mon maître.\par
Et il ajouta :\par
— Je rêvais que la mort me baisait la main.\par
Cimourdain eut cette secousse que nous donne parfois la brusque invasion d’un flot de pensées ; quelquefois ce flot est si haut et si orageux qu’il semble qu’il va éteindre l’âme. Rien ne sortit du profond cœur de Cimourdain. Il ne put dire que : — Gauvain !\par
Et tous deux se regardèrent ; Cimourdain avec des yeux pleins de ces flammes qui brûlent les larmes, Gauvain avec son plus doux sourire.\par
Gauvain se souleva sur son coude et dit :\par
— Cette balafre que je vois sur votre visage, c’est le coup de sabre que vous avez reçu pour moi. Hier encore vous étiez dans cette mêlée à côté de moi et à cause de moi. Si la providence ne vous avait pas mis près de mon berceau, où serais-je aujourd’hui ? dans les ténèbres. Si j’ai la notion du devoir, c’est de vous qu’elle me vient. J’étais né noué. Les préjugés sont des ligatures, vous m’avez ôté ces bandelettes, vous avez remis ma croissance en liberté, et de ce qui n’était déjà plus qu’une momie, vous avez refait un enfant. Dans l’avorton probable vous avez mis une conscience. Sans vous j’aurais grandi petit. J’existe par  vous. Je n’étais qu’un seigneur, vous avez fait de moi un citoyen ; je n’étais qu’un citoyen, vous avez fait de moi un esprit ; vous m’avez fait propre, comme homme, à la vie terrestre, et, comme âme, à la vie céleste. Vous m’avez donné, pour aller dans la réalité humaine, la clef de vérité, et, pour aller au delà, la clef de lumière. O mon maître, je vous remercie. C’est vous qui m’avez créé.\par
Cimourdain s’assit sur la paille à côté de Gauvain et lui dit :\par
— Je viens souper avec toi.\par
Gauvain rompit le pain noir, et le lui présenta. Cimourdain en prit un morceau ; puis Gauvain lui tendit la cruche d’eau.\par
— Bois le premier, dit Cimourdain.\par
Gauvain but et passa la cruche à Cimourdain qui but après lui. Gauvain n’avait bu qu’une gorgée ; Cimourdain but à longs traits.\par
Dans ce souper, Gauvain mangeait et Cimourdain buvait. Signe du calme de l’un et de la fièvre de l’autre.\par
On ne sait quelle sérénité terrible était dans ce cachot. Ces deux hommes causaient.\par
Gauvain disait :\par
— Les grandes choses s’ébauchent. Ce que la révolution fait en ce moment est mystérieux. Derrière l’œuvre visible il y a l’œuvre invisible. L’une cache l’autre. L’œuvre visible est farouche, l’œuvre invisible est sublime. En cet instant je distingue tout très nettement. C’est étrange et beau. Il a bien fallu se  servir des matériaux du passé. De là cet extraordinaire 93. Sous un échafaudage de barbarie se construit un temple de civilisation.\par
— Oui, répondit Cimourdain. De ce provisoire sortira le définitif. Le définitif, c’est-à-dire le droit et le devoir parallèles, l’impôt proportionnel et progressif, le service militaire obligatoire, le nivellement, aucune déviation, et, au-dessus de tous et de tout, cette ligne droite, la loi. La république de l’absolu.\par
— Je préfère, dit Gauvain, la république de l’idéal.\par
Il s’interrompit, puis continua :\par
— O mon maître, dans tout ce que vous venez de dire, où placez-vous le dévouement, le sacrifice, l’abnégation, l’entrelacement magnanime des bienveillances, l’amour ? Mettre tout en équilibre, c’est bien ; mettre tout en harmonie, c’est mieux. Au-dessus de la balance il y a la lyre. Votre république dose, mesure et règle l’homme ; la mienne l’emporte en plein azur. C’est la différence qu’il y a entre un théorème et un aigle.\par
— Tu te perds dans le nuage.\par
— Et vous dans le calcul.\par
— Il y a du rêve dans l’harmonie.\par
— Il y en a aussi dans l’algèbre.\par
— Je voudrais l’homme fait par Euclide.\par
— Et moi, dit Gauvain, je l’aimerais mieux fait par Homère.\par
Le sourire sévère de Cimourdain s’arrêta sur Gauvain comme pour tenir cette âme en arrêt.\par
 — Poésie. Défie-toi des poëtes.\par
— Oui, je connais ce mot. Défie-toi des souffles, défie-toi des rayons, défie-toi des parfums, défie-toi des fleurs, défie-toi des constellations.\par
— Rien de tout cela ne donne à manger.\par
— Qu’en savez-vous ? L’idée aussi est nourriture. Penser, c’est manger.\par
— Pas d’abstractions. La république c’est deux et deux font quatre. Quand j’ai donné à chacun ce qui lui revient...\par
— Il vous reste à donner à chacun ce qui ne lui revient pas.\par
— Qu’entends-tu par là ?\par
— J’entends l’immense concession réciproque que chacun doit à tous et que tous doivent à chacun, et qui est toute la vie sociale.\par
— Hors du droit strict, il n’y a rien.\par
— Il y a tout.\par
— Je ne vois que la justice.\par
— Moi, je regarde plus haut.\par
— Qu’y a-t-il donc au-dessus de la justice ?\par
— L’équité.\par
Par moments ils s’arrêtaient comme si des lueurs passaient.\par
Cimourdain reprit :\par
— Précise, je t’en défie.\par
— Soit. Vous voulez le service militaire obligatoire. Contre qui ? contre d’autres hommes. Moi, je ne veux pas de service militaire. Je veux la paix. Vous voulez les misérables secourus, moi je veux la  misère supprimée. Vous voulez l’impôt proportionnel. Je ne veux point d’impôt du tout. Je veux la dépense commune réduite à sa plus simple expression et payée par la plus-value sociale.\par
— Qu’entends-tu par là ?\par
— Ceci : d’abord supprimez les parasitismes ; le parasitisme du prêtre, le parasitisme du juge, le parasitisme du soldat. Ensuite, tirez parti de vos richesses ; vous jetez l’engrais à l’égout, jetez-le au sillon. Les trois quarts du sol sont en friche, défrichez la France, supprimez les vaines pâtures ; partagez les terres communales. Que tout homme ait une terre, et que toute terre ait un homme. Vous centuplerez le produit social. La France, à cette heure, ne donne à ses paysans que quatre jours de viande par an ; bien cultivée, elle nourrirait trois cents millions d’hommes, toute l’Europe. Utilisez la nature, cette immense auxiliaire dédaignée. Faites travailler pour vous tous les souffles de vent, toutes les chutes d’eau, toutes les effluves magnétiques. Le globe a un réseau veineux souterrain, il y a dans ce réseau une circulation prodigieuse d’eau, d’huile, de feu ; piquez la veine du globe, et faites jaillir cette eau pour vos fontaines, cette huile pour vos lampes, ce feu pour vos foyers. Réfléchissez au mouvement des vagues, au flux et reflux, au va-et-vient des marées. Qu’est-ce que l’océan ? une énorme force perdue. Comme la terre est bête ! ne pas employer l’océan !\par
— Te voilà en plein songe.\par
 — C’est-à-dire en pleine réalité.\par
Gauvain reprit :\par
— Et la femme ? qu’en faites-vous ?\par
Cimourdain répondit :\par
— Ce qu’elle est. La servante de l’homme.\par
— Oui. A une condition.\par
— Laquelle ?\par
— C’est que l’homme sera le serviteur de la femme.\par
— Y penses-tu ? s’écria Cimourdain, l’homme serviteur ! jamais. L’homme est maître. Je n’admets qu’une royauté, celle du foyer. L’homme chez lui est roi.\par
— Oui. A une condition.\par
— Laquelle ?\par
— C’est que la femme y sera reine.\par
— C’est-à-dire que tu veux pour l’homme et pour la femme...\par
— L’égalité.\par
— L’égalité ! y songes-tu ? les deux êtres sont divers.\par
— J’ai dit l’égalité. Je n’ai pas dit l’identité.\par
Il y eut encore une pause, comme une sorte de trêve entre ces deux esprits échangeant des éclairs. Cimourdain la rompit.\par
— Et l’enfant ! à qui le donnes-tu ?\par
— D’abord au père qui l’engendre, puis à la mère qui l’enfante, puis au maître qui l’élève, puis à la cité qui le virilise, puis à la patrie qui est la mère suprême, puis à l’humanité qui est la grande aïeule.\par
 — Tu ne parles pas de Dieu.\par
— Chacun de ces degrés, père, mère, maître, cité, patrie, humanité, est un des échelons de l’échelle qui monte à Dieu.\par
Cimourdain se taisait, Gauvain poursuivit :\par
— Quand on est au bout de l’échelle, on est arrivé à Dieu. Dieu s’ouvre ; on n’a plus qu’à entrer.\par
Cimourdain fit le geste d’un homme qui en rappelle un autre.\par
— Gauvain, reviens sur la terre. Nous voulons réaliser le possible.\par
— Commencez par ne pas le rendre impossible.\par
— Le possible se réalise toujours.\par
— Pas toujours. Si l’on rudoie l’utopie, on la tue. Rien n’est plus sans défense que l’œuf.\par
— Il faut pourtant saisir l’utopie, lui imposer le joug du réel, et l’encadrer dans le fait. L’idée abstraite doit se transformer en idée concrète ; ce qu’elle perd en beauté, elle le regagne en utilité ; elle est moindre, mais meilleure. Il faut que le droit entre dans la loi ; et, quand le droit s’est fait loi, il est absolu. C’est là ce que j’appelle le possible.\par
— Le possible est plus que cela.\par
— Ah ! te revoilà dans le rêve.\par
— Le possible est un oiseau mystérieux toujours planant au-dessus de l’homme.\par
— Il faut le prendre.\par
— Vivant.\par
Gauvain continua :\par
 — Ma pensée est : Toujours en avant. Si Dieu avait voulu que l’homme reculât, il lui aurait mis un œil derrière la tête. Regardons toujours du côté de l’aurore, de l’éclosion, de la naissance. Ce qui tombe encourage ce qui monte. Le craquement du vieil arbre est un appel à l’arbre nouveau. Chaque siècle fera son œuvre, aujourd’hui civique, demain humaine. Aujourd’hui la question du droit, demain la question du salaire. Salaire et droit, au fond c’est le même mot. L’homme ne vit pas pour n’être point payé ; Dieu en donnant la vie contracte une dette ; le droit, c’est le salaire inné ; le salaire, c’est le droit acquis.\par
Gauvain parlait avec le recueillement d’un prophète. Cimourdain écoutait. Les rôles étaient intervertis, et maintenant il semblait que c’était l’élève qui était le maître.\par
Cimourdain murmura :\par
— Tu vas vite.\par
— C’est que je suis peut-être un peu pressé, dit Gauvain en souriant.\par
Et il reprit :\par
— O mon maître, voici la différence entre nos deux utopies. Vous voulez la caserne obligatoire, moi, je veux l’école. Vous rêvez l’homme soldat, je rêve l’homme citoyen. Vous le voulez terrible, je le veux pensif. Vous fondez une république de glaives, je fonde...\par
Il s’interrompit :\par
— Je fonderais une république d’esprits.\par
Cimourdain regarda le pavé du cachot et dit :\par
 — Et en attendant que veux-tu ?\par
— Ce qui est.\par
— Tu absous donc le moment présent ?\par
— Oui.\par
— Pourquoi ?\par
— Parce que c’est une tempête. Une tempête sait toujours ce qu’elle fait. Pour un chêne foudroyé, que de forêts assainies ! La civilisation avait une peste, ce grand vent l’en délivre. Il ne choisit pas assez peut-être. Peut-il faire autrement ? Il est chargé d’un si rude balayage ! Devant l’horreur du miasme, je comprends la fureur du souffle.\par
Gauvain continua :\par
— D’ailleurs, que m’importe la tempête, si j’ai la boussole, et que me font les événements, si j’ai ma conscience !\par
Et il ajouta de cette voix basse qui est aussi la voix solennelle :\par
— Il y a quelqu’un qu’il faut toujours laisser faire.\par
— Qui ? demanda Cimourdain.\par
Gauvain leva le doigt au-dessus de sa tête. Cimourdain suivit du regard la direction de ce doigt levé, et, à travers la voûte du cachot, il lui sembla voir le ciel étoilé.\par
Ils se turent encore.\par
Cimourdain reprit :\par
— Société plus grande que nature. Je te le dis, ce n’est plus le possible, c’est le rêve.\par
— C’est le but. Autrement, à quoi bon la société ?  Restez dans la nature. Soyez les sauvages. Otaïti est un paradis. Seulement, dans ce paradis on ne pense pas. Mieux vaudrait encore un enfer intelligent qu’un paradis bête. Mais non, point d’enfer. Soyons la société humaine. Plus grande que nature. Oui. Si vous n’ajoutez rien à la nature, pourquoi sortir de la nature ? Alors, contentez-vous du travail comme la fourmi, et du miel comme l’abeille. Restez la bête ouvrière au lieu d’être l’intelligence reine. Si vous ajoutez quelque chose à la nature, vous serez nécessairement plus grand qu’elle ; ajouter, c’est augmenter ; augmenter, c’est grandir. La société, c’est la nature sublimée. Je veux tout ce qui manque aux ruches, tout ce qui manque aux fourmilières, les monuments, les arts, la poésie, les héros, les génies. Porter des fardeaux éternels, ce n’est pas la loi de l’homme. Non, non, non, plus de parias, plus d’esclaves, plus de forçats, plus de damnés ! je veux que chacun des attributs de l’homme soit un symbole de civilisation et un patron de progrès ; je veux la liberté devant l’esprit, l’égalité devant le cœur, la fraternité devant l’âme. Non ! plus de joug ! l’homme est fait, non pour traîner des chaînes, mais pour ouvrir des ailes. Plus d’homme reptile. Je veux la transfiguration de la larve en lépidoptère ; je veux que le ver de terre se change en une fleur vivante, et s’envole. Je veux...\par
Il s’arrêta. Son œil devint éclatant.\par
Ses lèvres remuaient. Il cessa de parler.\par
La porte était restée ouverte. Quelque chose des rumeurs du dehors pénétrait dans le cachot. On  entendait de vagues clairons, c’était probablement la diane ; puis des crosses de fusil sonnant à terre, c’étaient les sentinelles qu’on relevait ; puis, assez près de la tour, autant qu’on en pouvait juger dans l’obscurité, un mouvement pareil à un remuement de planches et de madriers, avec des bruits sourds et intermittents qui ressemblaient à des coups de marteaux.\par
Cimourdain, pâle, écoutait. Gauvain n’entendait pas.\par
Sa rêverie était de plus en plus profonde. Il semblait qu’il ne respirât plus, tant il était attentif à ce qu’il voyait sous la voûte visionnaire de son cerveau. Il avait de doux tressaillements. La clarté d’aurore qu’il avait dans la prunelle grandissait.\par
Un certain temps se passa ainsi. Cimourdain lui demanda :\par
— A quoi penses-tu ?\par
— A l’avenir, dit Gauvain.\par
Et il retomba dans sa méditation. Cimourdain se leva du lit de paille où ils étaient assis tous les deux. Gauvain ne s’en aperçut pas. Cimourdain, couvant du regard le jeune homme pensif, recula lentement jusqu’à la porte et sortit. Le cachot se referma.
 \subsubsection[{VI. Cependant le soleil se lève}]{VI \\
Cependant le soleil se lève}\phantomsection
\label{p3l7c6}
\noindent Le jour ne tarda pas à poindre à l’horizon.\par
En même temps que le jour, une chose étrange, immobile, surprenante, et que les oiseaux du ciel ne connaissaient pas, apparut sur le plateau de la Tourgue au-dessus de la forêt de Fougères.\par
Cela avait été mis là dans la nuit. C’était dressé plutôt que bâti. De loin sur l’horizon c’était une silhouette faite de lignes droites et dures, ayant l’aspect d’une lettre hébraïque ou d’un de ces hiéroglyphes d’Égypte qui faisaient partie de l’alphabet de l’antique énigme.\par
Au premier abord, l’idée que cette chose éveillait était l’idée de l’inutile. Elle était là parmi les bruyères en fleur. On se demandait à quoi cela pouvait servir. Puis on sentait venir un frisson. C’était une sorte de tréteau ayant pour pieds quatre poteaux. A un bout du tréteau, deux hautes solives, debout et droites, reliées à leur sommet par une traverse, élevaient et tenaient suspendu un triangle qui semblait noir sur l’azur du matin. A l’autre bout du tréteau, il  y avait une échelle. Entre les deux solives, en bas, au-dessous du triangle, on distinguait une sorte de panneau composé de deux sections mobiles qui, en s’ajustant l’une à l’autre, offraient au regard un trou rond à peu près de la dimension du cou d’un homme. La section supérieure du panneau glissait dans une rainure, de façon à pouvoir se hausser ou s’abaisser. Pour l’instant, les deux croissants qui en se rejoignant formaient le collier étaient écartés. On apercevait au pied des deux piliers portant le triangle une planche pouvant tourner sur charnières et ayant l’aspect d’une bascule. A côté de cette planche il y avait un panier long, et, entre les deux piliers, en avant, et à l’extrémité du tréteau, un panier carré. C’était peint en rouge. Tout était en bois, excepté le triangle qui était en fer. On sentait que cela avait été construit par des hommes, tant c’était laid, mesquin et petit ; et cela aurait mérité d’être apporté là par des génies, tant c’était formidable.\par
Cette bâtisse difforme, c’était la guillotine.\par
En face, à quelques pas, dans le ravin, il y avait un autre monstre, la Tourgue. Un monstre de pierre faisant pendant au monstre de bois. Et, disons-le, quand l’homme a touché au bois et à la pierre, le bois et la pierre ne sont plus ni bois ni pierre, et prennent quelque chose de l’homme. Un édifice est un dogme, une machine est une idée.\par
La Tourgue était cette résultante fatale du passé qui s’appelait la Bastille à Paris, la Tour de Londres en Angleterre, le Spielberg en Allemagne, l’Escurial  en Espagne, le Kremlin à Moscou, le château Saint-Ange à Rome.\par
Dans la Tourgue étaient condensés quinze cents ans, le moyen âge, le vasselage, la glèbe, la féodalité ; dans la guillotine une année, 93 ; et ces douze mois faisaient contre-poids à ces quinze siècles.\par
La Tourgue, c’était la monarchie ; la guillotine, c’était la révolution.\par
Confrontation tragique.\par
D’un côté, la dette ; de l’autre, l’échéance. D’un côté, l’inextricable complication gothique, le serf, le seigneur, l’esclave, le maître, la roture, la noblesse, le code multiple ramifié en coutumes, le juge et le prêtre coalisés, les ligatures innombrables, le fisc, les gabelles, la mainmorte, les capitations, les exceptions, les prérogatives, les préjugés, les fanatismes, le privilège royal de banqueroute, le sceptre, le trône, le bon plaisir, le droit divin ; de l’autre, cette chose simple, un couperet.\par
D’un côté, le nœud ; de l’autre, la hache.\par
La Tourgue avait été longtemps seule dans ce désert. Elle était là avec ses mâchicoulis d’où avaient ruisselé l’huile bouillante, la poix enflammée et le plomb fondu, avec ses oubliettes pavées d’ossements, avec sa chambre aux écartèlements, avec la tragédie énorme dont elle était remplie ; elle avait dominé de sa figure funeste cette forêt, elle avait eu dans cette ombre quinze siècles de tranquillité farouche, elle avait été dans ce pays l’unique puissance, l’unique respect et l’unique effroi ; elle avait régné ;  elle avait été, sans partage, la barbarie ; et tout à coup elle voyait se dresser devant elle, et contre elle, quelque chose, plus que quelque chose, — quelqu’un d’aussi horrible qu’elle, la guillotine.\par
La pierre semble quelquefois avoir des yeux étranges. Une statue observe, une tour guette, une façade d’édifice contemple. La Tourgue avait l’air d’examiner la guillotine.\par
Elle avait l’air de s’interroger.\par
Qu’était-ce que cela ?\par
Il semblait que cela était sorti de terre.\par
Et cela en était sorti en effet.\par
Dans la terre fatale avait germé l’arbre sinistre. De cette terre, arrosée de tant de sueurs, de tant de larmes, de tant de sang, de cette terre où avaient été creusées tant de fosses, tant de tombes, tant de cavernes, tant d’embûches, de cette terre où avaient pourri toutes les espèces de morts faits par toutes les espèces de tyrannies, de cette terre superposée à tant d’abîmes, et où avaient été enfouis tant de forfaits, semences affreuses, de cette terre profonde, était sortie, au jour marqué, cette inconnue, cette vengeresse, cette féroce machine porte-glaive, et 93 avait dit au vieux monde : — Me voilà.\par
Et la guillotine avait le droit de dire au donjon : — Je suis ta fille.\par
Et en même temps le donjon, car ces choses fatales vivent d’une vie obscure, se sentait tué par elle.\par
La Tourgue, devant la redoutable apparition, avait  on ne sait quoi d’effaré. On eût dit qu’elle avait peur. La monstrueuse masse de granit était majestueuse et infâme, cette planche avec son triangle était pire. La toute-puissante déchue avait l’horreur de la toute-puissante nouvelle. L’histoire criminelle considérait l’histoire justicière. La violence d’autrefois se comparait à la violence d’à-présent ; l’antique forteresse, l’antique prison, l’antique seigneurie, où avaient hurlé les patients démembrés, la construction de guerre et de meurtre, hors de service et hors de combat, violée, démantelée, découronnée, tas de pierres valant un tas de cendres, hideuse, magnifique et morte, toute pleine du vertige des siècles effrayants, regardait passer la terrible heure vivante. Hier frémissait devant Aujourd’hui, la vieille férocité constatait et subissait la nouvelle épouvante, ce qui n’était plus que le néant ouvrait des yeux d’ombre devant ce qui était la terreur, et le fantôme regardait le spectre.\par
La nature est impitoyable ; elle ne consent pas à retirer ses fleurs, ses musiques, ses parfums et ses rayons devant l’abomination humaine ; elle accable l’homme du contraste de la beauté divine avec la laideur sociale ; elle ne lui fait grâce ni d’une aile de papillon, ni d’un chant d’oiseau ; il faut qu’en plein meurtre, en pleine vengeance, en pleine barbarie, il subisse le regard des choses sacrées ; il ne peut se soustraire à l’immense reproche de la douceur universelle et à l’implacable sérénité de l’azur. Il faut que la difformité des lois humaines se montre toute nue au milieu de l’éblouissement éternel. L’homme brise et  broie, l’homme stérilise, l’homme tue ; l’été reste l’été, le lys reste le lys, l’astre reste l’astre.\par
Ce matin-là, jamais le ciel frais du jour levant n’avait été plus charmant. Un vent tiède remuait les bruyères, les vapeurs rampaient mollement dans les branchages, la forêt de Fougères, toute pénétrée de l’haleine qui sort des sources, fumait dans l’aube comme une vaste cassolette pleine d’encens ; le bleu du firmament, la blancheur des nuées, la claire transparence des eaux, la verdure, cette gamme harmonieuse qui va de l’aigue-marine à l’émeraude, les groupes d’arbres fraternels, les nappes d’herbes, les plaines profondes, tout avait cette pureté qui est l’éternel conseil de la nature à l’homme. Au milieu de tout cela s’étalait l’affreuse impudeur humaine ; au milieu de tout cela apparaissaient la forteresse et l’échafaud, la guerre et le supplice, les deux figures de l’âge sanguinaire et de la minute sanglante ; la chouette de la nuit du passé et la chauve-souris du crépuscule de l’avenir. En présence de la création fleurie, embaumée, aimante et charmante, le ciel splendide inondait d’aurore la Tourgue et la guillotine, et semblait dire aux hommes : Regardez ce que je fais et ce que vous faites.\par
Tels sont les formidables usages que le soleil fait de sa lumière.\par
Ce spectacle avait des spectateurs.\par
Les quatre mille hommes de la petite armée expéditionnaire étaient rangés en ordre de combat sur le plateau. Ils entouraient la guillotine de trois côtés, de  façon à tracer autour d’elle, en plan géométral, la figure d’un E ; la batterie placée au centre de la plus grande ligne faisait le cran de l’E. La machine rouge était comme enfermée dans ces trois fronts de bataille, sorte de muraille de soldats repliée des deux côtés jusqu’aux bords de l’escarpement du plateau ; le quatrième côté, le côté ouvert, était le ravin même, et regardait la Tourgue.\par
Cela faisait une place en carré long, au milieu de laquelle était l’échafaud. A mesure que le jour montait, l’ombre portée de la guillotine décroissait sur l’herbe.\par
Les artilleurs étaient à leurs pièces, mèches allumées.\par
Une douce fumée bleue s’élevait du ravin ; c’était l’incendie du pont qui achevait d’expirer.\par
Cette fumée estompait sans la voiler la Tourgue dont la haute plate-forme dominait tout l’horizon. Entre cette plate-forme et la guillotine il n’y avait que l’intervalle du ravin. De l’une à l’autre on pouvait se parler.\par
Sur cette plate-forme avaient été transportées la table du tribunal et la chaise ombragée de drapeaux tricolores. Le jour se levait derrière la Tourgue et faisait saillir en noir la masse de la forteresse et, à son sommet, sur la chaise du tribunal et sous le faisceau de drapeaux, la figure d’un homme assis, immobile et les bras croisés.\par
Cet homme était Cimourdain. Il avait, comme la veille, son costume de délégué civil, sur la tête le chapeau à panache tricolore, le sabre au côté et les pistolets à la ceinture.\par
 Il se taisait. Tous se taisaient. Les soldats avaient le fusil au pied et baissaient les yeux. Ils se touchaient du coude, mais ne se parlaient pas. Ils songeaient confusément à cette guerre, à tant de combats, aux fusillades des haies si vaillamment affrontées, aux nuées de paysans furieux chassés par leur souffle, aux citadelles prises, aux batailles gagnées, aux victoires, et il leur semblait maintenant que toute cette gloire leur tournait en honte. Une sombre attente serrait toutes les poitrines. On voyait sur l’estrade de la guillotine le bourreau qui allait et venait. La clarté grandissante du matin emplissait majestueusement le ciel.\par
Soudain on entendit ce bruit voilé que font les tambours couverts d’un crêpe. Ce roulement funèbre approcha ; les rangs s’ouvrirent, et un cortège entra dans le carré, et se dirigea vers l’échafaud.\par
D’abord les tambours noirs, puis une compagnie de grenadiers, l’arme basse, puis un peloton de gendarmes, le sabre nu, puis le condamné, — Gauvain.\par
Gauvain marchait librement. Il n’avait de cordes ni aux pieds ni aux mains. Il était en petit uniforme. Il avait son épée.\par
Derrière lui venait un autre peloton de gendarmes.\par
Gauvain avait encore sur le visage cette joie pensive qui l’avait illuminé au moment où il avait dit à Cimourdain : Je pense à l’avenir. Rien n’était ineffable et sublime comme ce sourire continué.\par
En arrivant sur le lieu triste, son premier regard fut pour le haut de la tour. Il dédaigna la guillotine.\par
Il savait que Cimourdain se ferait un devoir  d’assister à l’exécution. Il le chercha des yeux sur la plate-forme. Il l’y trouva.\par
Cimourdain était blême et froid. Ceux qui étaient près de lui n’entendaient pas son souffle.\par
Quand il aperçut Gauvain, il n’eut pas un tressaillement.\par
Gauvain cependant s’avançait vers l’échafaud.\par
Tout en marchant, il regardait Cimourdain et Cimourdain le regardait. Il semblait que Cimourdain s’appuyât sur ce regard.\par
Gauvain arriva au pied de l’échafaud. Il y monta. L’officier qui commandait les grenadiers l’y suivit. Il défit son épée et la remit à l’officier ; il ôta sa cravate et la remit au bourreau.\par
Il ressemblait à une vision. Jamais il n’avait apparu plus beau. Sa chevelure brune flottait au vent ; on ne coupait pas les cheveux alors. Son cou blanc faisait songer à une femme, et son œil héroïque et souverain faisait songer à un archange. Il était sur l’échafaud, rêveur. Ce lieu-là aussi est un sommet. Gauvain y était debout, superbe et tranquille. Le soleil, l’enveloppant, le mettait comme dans une gloire.\par
Il fallait pourtant lier le patient. Le bourreau vint, une corde à la main.\par
En ce moment-là, quand ils virent leur jeune capitaine si décidément engagé sous le couteau, les soldats n’y tinrent plus ; le cœur de ces gens de guerre éclata. On entendit cette chose énorme, le sanglot d’une armée. Une clameur s’éleva. Grâce ! grâce ! Quelques-uns tombèrent à genoux ; d’autres jetaient  leurs fusils et levaient les bras vers la plate-forme où était Cimourdain. Un grenadier cria en montrant la guillotine : — Reçoit-on des remplaçants pour ça ? Me voici. — Tous répétaient frénétiquement : Grâce ! grâce ! et des lions qui auraient entendu cela eussent été émus ou effrayés, car les larmes des soldats sont terribles.\par
Le bourreau s’arrêta, ne sachant plus que faire.\par
Alors une voix brève et basse, et que tous pourtant entendirent, tant elle était sinistre, cria du haut de la tour :\par
— Force à la loi !\par
On reconnut l’accent inexorable. Cimourdain avait parlé. L’armée frissonna.\par
Le bourreau n’hésita plus. Il s’approcha tenant sa corde.\par
— Attendez, dit Gauvain.\par
Il se tourna vers Cimourdain, lui fit, de sa main droite encore libre, un geste d’adieu, puis se laissa lier.\par
Quand il fut lié, il dit au bourreau :\par
— Pardon. Un moment encore.\par
Et il cria :\par
— Vive la République !\par
On le coucha sur la bascule, cette tête charmante et fière s’emboîta dans l’infâme collier, le bourreau lui releva doucement les cheveux, puis pressa le ressort, le triangle se détacha et glissa, lentement d’abord, puis rapidement ; on entendit un coup hideux...\par
 Au même instant on en entendit un autre. Au coup de hache répondit un coup de pistolet. Cimourdain venait de saisir un des pistolets qu’il avait à sa ceinture, et, au moment où la tête de Gauvain roulait dans le panier, Cimourdain se traversait le cœur d’une balle. Un flot de sang lui sortit de la bouche, il tomba mort.\par
Et ces deux âmes, sœurs tragiques, s’envolèrent ensemble, l’ombre de l’une mêlée à la lumière de l’autre.
 \section[{Notes du manuscrit original}]{Notes du manuscrit original}\phantomsection
\label{notes}\renewcommand{\leftmark}{Notes du manuscrit original}


\labelblock{Note I}

\noindent Sur la première page du Livre \emph{la Corvette Claymore :}\par
\bigbreak
\noindent La famille V.......existe peut-être encore. Elle est innocente de la honte de son aïeul. Pourquoi affliger cette famille ? Je mettrai, dans le livre publié, \emph{Gèlambre.}\par

\labelblock{Note II}

\noindent Dans le chapitre \emph{le Massacre de Saint-Barthélemy}, après la phrase : « Une souris passa sans qu’ils y prissent garde..., » la page que voici a été retranchée :\par

\begin{quoteblock}
 \noindent Il y eut des épisodes.\par
 Une estampe les charma et fut un instant épargnée. Elle représentait les vaches grasses et les vaches maigres.\par
   Ceci amena une déclaration de Georgette. Gros-Alain lui demanda :\par
 — Voudrais-tu avoir une vache ?\par
 Elle répondit : Boui.\par
 — Voudrais-tu la mener aux champs ?\par
 — Boui.\par
 — Avec un fouet ?\par
 — Boui.\par
 — En aurais-tu peur ?\par
 — Mais non, dit-elle.\par
 De lacération en lacération, ils arrivèrent à une autre estampe au bas de laquelle on lisait : — An de Rome 739. Consulat de Livius Drusus et de Calpurnius Piso. — L’estampe représentait une petite Vierge Marie âgée de quatre ans. Cette figure féminine fit réver René-Jean, et parut éveiller en lui de tendres souvenirs. Il apostropha Gros-Alain :\par
 — Hein, toi, tu n’as pas de bonne amie !\par
 Et il lui tira la langue.\par
 Gros-Alain, un peu confus, baissa la tête.
 \end{quoteblock}


\labelblock{Note III}

\noindent Aux marges du manuscrit, on trouve les dates et les notes suivantes :\par
\bigbreak
\noindent En tête de la première page :\par

\begin{quoteblock}
 \noindent Je commence ce livre aujourd’hui 16 décembre 1872. Je suis à Hauteville House. — V. H.
 \end{quoteblock}

\noindent Première partie, Livre II. Au milieu du chapitre \emph{9 = 380 :}\par

\begin{quoteblock}
 \noindent 1\textsuperscript{er} janvier 1873.
 \end{quoteblock}

 \noindent Même Livre. Fin du chapitre \emph{Utilité des gros caractères :}\par

\begin{quoteblock}
 \noindent 10 janvier. — La nouvelle arrive que Louis Bonaparte est mort.
 \end{quoteblock}

\noindent En tête de la première page de la seconde partie, \emph{A Paris :}\par

\begin{quoteblock}
 \noindent Aujourd’hui \emph{vingt et un janvier} 1873, je commence à écrire cette seconde partie du livre \emph{Quatrevingt-treize}.
 \end{quoteblock}

\noindent Vers le milieu du chapitre \emph{la Convention :}\par

\begin{quoteblock}
 \noindent 12 février. — Hier \emph{Marion de Lorme} a été reprise aux Français. Paul Meurice m’envoie ce télégramme :\par
 \bigbreak
 
\dateline{Paris, 11 février, 11 h. du soir.}
 \bigbreak
 \noindent \emph{Succès immense, devant un public dur, de gens du monde.}
 \end{quoteblock}

\noindent Au bas de la dernière page du même chapitre :\par

\begin{quoteblock}
 \noindent J’achève ces pages sur la Convention aujourd’hui 26 février, anniversaire de ma naissance. J’ai aujourd’hui soixante et onze ans.
 \end{quoteblock}

\noindent Au bas de la dernière page :\par

\begin{quoteblock}
 \noindent Je finis ce livre aujourd’hui 9 juin 1873, à Hauteville House, dans l’atelier d’en bas, à midi et demi.
 \end{quoteblock}

  


% at least one empty page at end (for booklet couv)
\ifbooklet
  \pagestyle{empty}
  \clearpage
  % 2 empty pages maybe needed for 4e cover
  \ifnum\modulo{\value{page}}{4}=0 \hbox{}\newpage\hbox{}\newpage\fi
  \ifnum\modulo{\value{page}}{4}=1 \hbox{}\newpage\hbox{}\newpage\fi


  \hbox{}\newpage
  \ifodd\value{page}\hbox{}\newpage\fi
  {\centering\color{rubric}\bfseries\noindent\large
    Hurlus ? Qu’est-ce.\par
    \bigskip
  }
  \noindent Des bouquinistes électroniques, pour du texte libre à participation libre,
  téléchargeable gratuitement sur \href{https://hurlus.fr}{\dotuline{hurlus.fr}}.\par
  \bigskip
  \noindent Cette brochure a été produite par des éditeurs bénévoles.
  Elle n’est pas faîte pour être possédée, mais pour être lue, et puis donnée.
  Que circule le texte !
  En page de garde, on peut ajouter une date, un lieu, un nom ; pour suivre le voyage des idées.
  \par

  Ce texte a été choisi parce qu’une personne l’a aimé,
  ou haï, elle a en tous cas pensé qu’il partipait à la formation de notre présent ;
  sans le souci de plaire, vendre, ou militer pour une cause.
  \par

  L’édition électronique est soigneuse, tant sur la technique
  que sur l’établissement du texte ; mais sans aucune prétention scolaire, au contraire.
  Le but est de s’adresser à tous, sans distinction de science ou de diplôme.
  Au plus direct ! (possible)
  \par

  Cet exemplaire en papier a été tiré sur une imprimante personnelle
   ou une photocopieuse. Tout le monde peut le faire.
  Il suffit de
  télécharger un fichier sur \href{https://hurlus.fr}{\dotuline{hurlus.fr}},
  d’imprimer, et agrafer ; puis de lire et donner.\par

  \bigskip

  \noindent PS : Les hurlus furent aussi des rebelles protestants qui cassaient les statues dans les églises catholiques. En 1566 démarra la révolte des gueux dans le pays de Lille. L’insurrection enflamma la région jusqu’à Anvers où les gueux de mer bloquèrent les bateaux espagnols.
  Ce fut une rare guerre de libération dont naquit un pays toujours libre : les Pays-Bas.
  En plat pays francophone, par contre, restèrent des bandes de huguenots, les hurlus, progressivement réprimés par la très catholique Espagne.
  Cette mémoire d’une défaite est éteinte, rallumons-la. Sortons les livres du culte universitaire, cherchons les idoles de l’époque, pour les briser.
\fi

\ifdev % autotext in dev mode
\fontname\font — \textsc{Les règles du jeu}\par
(\hyperref[utopie]{\underline{Lien}})\par
\noindent \initialiv{A}{lors là}\blindtext\par
\noindent \initialiv{À}{ la bonheur des dames}\blindtext\par
\noindent \initialiv{É}{tonnez-le}\blindtext\par
\noindent \initialiv{Q}{ualitativement}\blindtext\par
\noindent \initialiv{V}{aloriser}\blindtext\par
\Blindtext
\phantomsection
\label{utopie}
\Blinddocument
\fi
\end{document}
