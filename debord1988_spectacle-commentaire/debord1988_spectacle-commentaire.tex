%%%%%%%%%%%%%%%%%%%%%%%%%%%%%%%%%
% LaTeX model https://hurlus.fr %
%%%%%%%%%%%%%%%%%%%%%%%%%%%%%%%%%

% Needed before document class
\RequirePackage{pdftexcmds} % needed for tests expressions
\RequirePackage{fix-cm} % correct units

% Define mode
\def\mode{a4}

\newif\ifaiv % a4
\newif\ifav % a5
\newif\ifbooklet % booklet
\newif\ifcover % cover for booklet

\ifnum \strcmp{\mode}{cover}=0
  \covertrue
\else\ifnum \strcmp{\mode}{booklet}=0
  \booklettrue
\else\ifnum \strcmp{\mode}{a5}=0
  \avtrue
\else
  \aivtrue
\fi\fi\fi

\ifbooklet % do not enclose with {}
  \documentclass[french,twoside]{book} % ,notitlepage
  \usepackage[%
    papersize={105mm, 297mm},
    inner=12mm,
    outer=12mm,
    top=20mm,
    bottom=15mm,
    marginparsep=0pt,
  ]{geometry}
  \usepackage[fontsize=9.5pt]{scrextend} % for Roboto
\else\ifav
  \documentclass[french,twoside]{book} % ,notitlepage
  \usepackage[%
    a5paper,
    inner=25mm,
    outer=15mm,
    top=15mm,
    bottom=15mm,
    marginparsep=0pt,
  ]{geometry}
  \usepackage[fontsize=12pt]{scrextend}
\else% A4 2 cols
  \documentclass[twocolumn]{report}
  \usepackage[%
    a4paper,
    inner=15mm,
    outer=10mm,
    top=25mm,
    bottom=18mm,
    marginparsep=0pt,
  ]{geometry}
  \setlength{\columnsep}{20mm}
  \usepackage[fontsize=9.5pt]{scrextend}
\fi\fi

%%%%%%%%%%%%%%
% Alignments %
%%%%%%%%%%%%%%
% before teinte macros

\setlength{\arrayrulewidth}{0.2pt}
\setlength{\columnseprule}{\arrayrulewidth} % twocol
\setlength{\parskip}{0pt} % classical para with no margin
\setlength{\parindent}{1.5em}

%%%%%%%%%%
% Colors %
%%%%%%%%%%
% before Teinte macros

\usepackage[dvipsnames]{xcolor}
\definecolor{rubric}{HTML}{800000} % the tonic 0c71c3
\def\columnseprulecolor{\color{rubric}}
\colorlet{borderline}{rubric!30!} % definecolor need exact code
\definecolor{shadecolor}{gray}{0.95}
\definecolor{bghi}{gray}{0.5}

%%%%%%%%%%%%%%%%%
% Teinte macros %
%%%%%%%%%%%%%%%%%
%%%%%%%%%%%%%%%%%%%%%%%%%%%%%%%%%%%%%%%%%%%%%%%%%%%
% <TEI> generic (LaTeX names generated by Teinte) %
%%%%%%%%%%%%%%%%%%%%%%%%%%%%%%%%%%%%%%%%%%%%%%%%%%%
% This template is inserted in a specific design
% It is XeLaTeX and otf fonts

\makeatletter % <@@@


\usepackage{blindtext} % generate text for testing
\usepackage[strict]{changepage} % for modulo 4
\usepackage{contour} % rounding words
\usepackage[nodayofweek]{datetime}
% \usepackage{DejaVuSans} % seems buggy for sffont font for symbols
\usepackage{enumitem} % <list>
\usepackage{etoolbox} % patch commands
\usepackage{fancyvrb}
\usepackage{fancyhdr}
\usepackage{float}
\usepackage{fontspec} % XeLaTeX mandatory for fonts
\usepackage{footnote} % used to capture notes in minipage (ex: quote)
\usepackage{framed} % bordering correct with footnote hack
\usepackage{graphicx}
\usepackage{lettrine} % drop caps
\usepackage{lipsum} % generate text for testing
\usepackage[framemethod=tikz,]{mdframed} % maybe used for frame with footnotes inside
\usepackage{pdftexcmds} % needed for tests expressions
\usepackage{polyglossia} % non-break space french punct, bug Warning: "Failed to patch part"
\usepackage[%
  indentfirst=false,
  vskip=1em,
  noorphanfirst=true,
  noorphanafter=true,
  leftmargin=\parindent,
  rightmargin=0pt,
]{quoting}
\usepackage{ragged2e}
\usepackage{setspace} % \setstretch for <quote>
\usepackage{tabularx} % <table>
\usepackage[explicit]{titlesec} % wear titles, !NO implicit
\usepackage{tikz} % ornaments
\usepackage{tocloft} % styling tocs
\usepackage[fit]{truncate} % used im runing titles
\usepackage{unicode-math}
\usepackage[normalem]{ulem} % breakable \uline, normalem is absolutely necessary to keep \emph
\usepackage{verse} % <l>
\usepackage{xcolor} % named colors
\usepackage{xparse} % @ifundefined
\XeTeXdefaultencoding "iso-8859-1" % bad encoding of xstring
\usepackage{xstring} % string tests
\XeTeXdefaultencoding "utf-8"
\PassOptionsToPackage{hyphens}{url} % before hyperref, which load url package

% TOTEST
% \usepackage{hypcap} % links in caption ?
% \usepackage{marginnote}
% TESTED
% \usepackage{background} % doesn’t work with xetek
% \usepackage{bookmark} % prefers the hyperref hack \phantomsection
% \usepackage[color, leftbars]{changebar} % 2 cols doc, impossible to keep bar left
% \usepackage[utf8x]{inputenc} % inputenc package ignored with utf8 based engines
% \usepackage[sfdefault,medium]{inter} % no small caps
% \usepackage{firamath} % choose firasans instead, firamath unavailable in Ubuntu 21-04
% \usepackage{flushend} % bad for last notes, supposed flush end of columns
% \usepackage[stable]{footmisc} % BAD for complex notes https://texfaq.org/FAQ-ftnsect
% \usepackage{helvet} % not for XeLaTeX
% \usepackage{multicol} % not compatible with too much packages (longtable, framed, memoir…)
% \usepackage[default,oldstyle,scale=0.95]{opensans} % no small caps
% \usepackage{sectsty} % \chapterfont OBSOLETE
% \usepackage{soul} % \ul for underline, OBSOLETE with XeTeX
% \usepackage[breakable]{tcolorbox} % text styling gone, footnote hack not kept with breakable


% Metadata inserted by a program, from the TEI source, for title page and runing heads
\title{\textbf{ Commentaires sur la société du spectacle }}
\date{1988}
\author{Debord, Guy}
\def\elbibl{Debord, Guy. 1988. \emph{Commentaires sur la société du spectacle}}
\def\elsource{\href{http://rai.chez-alice.fr/html/debordcom.htm}{\dotuline{http://rai.chez-alice.fr/html/debordcom.htm}}\footnote{\href{http://rai.chez-alice.fr/html/debordcom.htm}{\url{http://rai.chez-alice.fr/html/debordcom.htm}}} }

% Default metas
\newcommand{\colorprovide}[2]{\@ifundefinedcolor{#1}{\colorlet{#1}{#2}}{}}
\colorprovide{rubric}{red}
\colorprovide{silver}{lightgray}
\@ifundefined{syms}{\newfontfamily\syms{DejaVu Sans}}{}
\newif\ifdev
\@ifundefined{elbibl}{% No meta defined, maybe dev mode
  \newcommand{\elbibl}{Titre court ?}
  \newcommand{\elbook}{Titre du livre source ?}
  \newcommand{\elabstract}{Résumé\par}
  \newcommand{\elurl}{http://oeuvres.github.io/elbook/2}
  \author{Éric Lœchien}
  \title{Un titre de test assez long pour vérifier le comportement d’une maquette}
  \date{1566}
  \devtrue
}{}
\let\eltitle\@title
\let\elauthor\@author
\let\eldate\@date


\defaultfontfeatures{
  % Mapping=tex-text, % no effect seen
  Scale=MatchLowercase,
  Ligatures={TeX,Common},
}


% generic typo commands
\newcommand{\astermono}{\medskip\centerline{\color{rubric}\large\selectfont{\syms ✻}}\medskip\par}%
\newcommand{\astertri}{\medskip\par\centerline{\color{rubric}\large\selectfont{\syms ✻\,✻\,✻}}\medskip\par}%
\newcommand{\asterism}{\bigskip\par\noindent\parbox{\linewidth}{\centering\color{rubric}\large{\syms ✻}\\{\syms ✻}\hskip 0.75em{\syms ✻}}\bigskip\par}%

% lists
\newlength{\listmod}
\setlength{\listmod}{\parindent}
\setlist{
  itemindent=!,
  listparindent=\listmod,
  labelsep=0.2\listmod,
  parsep=0pt,
  % topsep=0.2em, % default topsep is best
}
\setlist[itemize]{
  label=—,
  leftmargin=0pt,
  labelindent=1.2em,
  labelwidth=0pt,
}
\setlist[enumerate]{
  label={\bf\color{rubric}\arabic*.},
  labelindent=0.8\listmod,
  leftmargin=\listmod,
  labelwidth=0pt,
}
\newlist{listalpha}{enumerate}{1}
\setlist[listalpha]{
  label={\bf\color{rubric}\alph*.},
  leftmargin=0pt,
  labelindent=0.8\listmod,
  labelwidth=0pt,
}
\newcommand{\listhead}[1]{\hspace{-1\listmod}\emph{#1}}

\renewcommand{\hrulefill}{%
  \leavevmode\leaders\hrule height 0.2pt\hfill\kern\z@}

% General typo
\DeclareTextFontCommand{\textlarge}{\large}
\DeclareTextFontCommand{\textsmall}{\small}

% commands, inlines
\newcommand{\anchor}[1]{\Hy@raisedlink{\hypertarget{#1}{}}} % link to top of an anchor (not baseline)
\newcommand\abbr[1]{#1}
\newcommand{\autour}[1]{\tikz[baseline=(X.base)]\node [draw=rubric,thin,rectangle,inner sep=1.5pt, rounded corners=3pt] (X) {\color{rubric}#1};}
\newcommand\corr[1]{#1}
\newcommand{\ed}[1]{ {\color{silver}\sffamily\footnotesize (#1)} } % <milestone ed="1688"/>
\newcommand\expan[1]{#1}
\newcommand\foreign[1]{\emph{#1}}
\newcommand\gap[1]{#1}
\renewcommand{\LettrineFontHook}{\color{rubric}}
\newcommand{\initial}[2]{\lettrine[lines=2, loversize=0.3, lhang=0.3]{#1}{#2}}
\newcommand{\initialiv}[2]{%
  \let\oldLFH\LettrineFontHook
  % \renewcommand{\LettrineFontHook}{\color{rubric}\ttfamily}
  \IfSubStr{QJ’}{#1}{
    \lettrine[lines=4, lhang=0.2, loversize=-0.1, lraise=0.2]{\smash{#1}}{#2}
  }{\IfSubStr{É}{#1}{
    \lettrine[lines=4, lhang=0.2, loversize=-0, lraise=0]{\smash{#1}}{#2}
  }{\IfSubStr{ÀÂ}{#1}{
    \lettrine[lines=4, lhang=0.2, loversize=-0, lraise=0, slope=0.6em]{\smash{#1}}{#2}
  }{\IfSubStr{A}{#1}{
    \lettrine[lines=4, lhang=0.2, loversize=0.2, slope=0.6em]{\smash{#1}}{#2}
  }{\IfSubStr{V}{#1}{
    \lettrine[lines=4, lhang=0.2, loversize=0.2, slope=-0.5em]{\smash{#1}}{#2}
  }{
    \lettrine[lines=4, lhang=0.2, loversize=0.2]{\smash{#1}}{#2}
  }}}}}
  \let\LettrineFontHook\oldLFH
}
\newcommand{\labelchar}[1]{\textbf{\color{rubric} #1}}
\newcommand{\milestone}[1]{\autour{\footnotesize\color{rubric} #1}} % <milestone n="4"/>
\newcommand\name[1]{#1}
\newcommand\orig[1]{#1}
\newcommand\orgName[1]{#1}
\newcommand\persName[1]{#1}
\newcommand\placeName[1]{#1}
\newcommand{\pn}[1]{\IfSubStr{-—–¶}{#1}% <p n="3"/>
  {\noindent{\bfseries\color{rubric}   ¶  }}
  {{\footnotesize\autour{ #1}  }}}
\newcommand\reg{}
% \newcommand\ref{} % already defined
\newcommand\sic[1]{#1}
\newcommand\surname[1]{\textsc{#1}}
\newcommand\term[1]{\textbf{#1}}

\def\mednobreak{\ifdim\lastskip<\medskipamount
  \removelastskip\nopagebreak\medskip\fi}
\def\bignobreak{\ifdim\lastskip<\bigskipamount
  \removelastskip\nopagebreak\bigskip\fi}

% commands, blocks
\newcommand{\byline}[1]{\bigskip{\RaggedLeft{#1}\par}\bigskip}
\newcommand{\bibl}[1]{{\RaggedLeft{#1}\par\bigskip}}
\newcommand{\biblitem}[1]{{\noindent\hangindent=\parindent   #1\par}}
\newcommand{\dateline}[1]{\medskip{\RaggedLeft{#1}\par}\bigskip}
\newcommand{\labelblock}[1]{\medbreak{\noindent\color{rubric}\bfseries #1}\par\mednobreak}
\newcommand{\salute}[1]{\bigbreak{#1}\par\medbreak}
\newcommand{\signed}[1]{\bigbreak\filbreak{\raggedleft #1\par}\medskip}

% environments for blocks (some may become commands)
\newenvironment{borderbox}{}{} % framing content
\newenvironment{citbibl}{\ifvmode\hfill\fi}{\ifvmode\par\fi }
\newenvironment{docAuthor}{\ifvmode\vskip4pt\fontsize{16pt}{18pt}\selectfont\fi\itshape}{\ifvmode\par\fi }
\newenvironment{docDate}{}{\ifvmode\par\fi }
\newenvironment{docImprint}{\vskip6pt}{\ifvmode\par\fi }
\newenvironment{docTitle}{\vskip6pt\bfseries\fontsize{18pt}{22pt}\selectfont}{\par }
\newenvironment{msHead}{\vskip6pt}{\par}
\newenvironment{msItem}{\vskip6pt}{\par}
\newenvironment{titlePart}{}{\par }


% environments for block containers
\newenvironment{argument}{\itshape\parindent0pt}{\vskip1.5em}
\newenvironment{biblfree}{}{\ifvmode\par\fi }
\newenvironment{bibitemlist}[1]{%
  \list{\@biblabel{\@arabic\c@enumiv}}%
  {%
    \settowidth\labelwidth{\@biblabel{#1}}%
    \leftmargin\labelwidth
    \advance\leftmargin\labelsep
    \@openbib@code
    \usecounter{enumiv}%
    \let\p@enumiv\@empty
    \renewcommand\theenumiv{\@arabic\c@enumiv}%
  }
  \sloppy
  \clubpenalty4000
  \@clubpenalty \clubpenalty
  \widowpenalty4000%
  \sfcode`\.\@m
}%
{\def\@noitemerr
  {\@latex@warning{Empty `bibitemlist' environment}}%
\endlist}
\newenvironment{quoteblock}% may be used for ornaments
  {\begin{quoting}}
  {\end{quoting}}

% table () is preceded and finished by custom command
\newcommand{\tableopen}[1]{%
  \ifnum\strcmp{#1}{wide}=0{%
    \begin{center}
  }
  \else\ifnum\strcmp{#1}{long}=0{%
    \begin{center}
  }
  \else{%
    \begin{center}
  }
  \fi\fi
}
\newcommand{\tableclose}[1]{%
  \ifnum\strcmp{#1}{wide}=0{%
    \end{center}
  }
  \else\ifnum\strcmp{#1}{long}=0{%
    \end{center}
  }
  \else{%
    \end{center}
  }
  \fi\fi
}


% text structure
\newcommand\chapteropen{} % before chapter title
\newcommand\chaptercont{} % after title, argument, epigraph…
\newcommand\chapterclose{} % maybe useful for multicol settings
\setcounter{secnumdepth}{-2} % no counters for hierarchy titles
\setcounter{tocdepth}{5} % deep toc
\markright{\@title} % ???
\markboth{\@title}{\@author} % ???
\renewcommand\tableofcontents{\@starttoc{toc}}
% toclof format
% \renewcommand{\@tocrmarg}{0.1em} % Useless command?
% \renewcommand{\@pnumwidth}{0.5em} % {1.75em}
\renewcommand{\@cftmaketoctitle}{}
\setlength{\cftbeforesecskip}{\z@ \@plus.2\p@}
\renewcommand{\cftchapfont}{}
\renewcommand{\cftchapdotsep}{\cftdotsep}
\renewcommand{\cftchapleader}{\normalfont\cftdotfill{\cftchapdotsep}}
\renewcommand{\cftchappagefont}{\bfseries}
\setlength{\cftbeforechapskip}{0em \@plus\p@}
% \renewcommand{\cftsecfont}{\small\relax}
\renewcommand{\cftsecpagefont}{\normalfont}
% \renewcommand{\cftsubsecfont}{\small\relax}
\renewcommand{\cftsecdotsep}{\cftdotsep}
\renewcommand{\cftsecpagefont}{\normalfont}
\renewcommand{\cftsecleader}{\normalfont\cftdotfill{\cftsecdotsep}}
\setlength{\cftsecindent}{1em}
\setlength{\cftsubsecindent}{2em}
\setlength{\cftsubsubsecindent}{3em}
\setlength{\cftchapnumwidth}{1em}
\setlength{\cftsecnumwidth}{1em}
\setlength{\cftsubsecnumwidth}{1em}
\setlength{\cftsubsubsecnumwidth}{1em}

% footnotes
\newif\ifheading
\newcommand*{\fnmarkscale}{\ifheading 0.70 \else 1 \fi}
\renewcommand\footnoterule{\vspace*{0.3cm}\hrule height \arrayrulewidth width 3cm \vspace*{0.3cm}}
\setlength\footnotesep{1.5\footnotesep} % footnote separator
\renewcommand\@makefntext[1]{\parindent 1.5em \noindent \hb@xt@1.8em{\hss{\normalfont\@thefnmark . }}#1} % no superscipt in foot
\patchcmd{\@footnotetext}{\footnotesize}{\footnotesize\sffamily}{}{} % before scrextend, hyperref


%   see https://tex.stackexchange.com/a/34449/5049
\def\truncdiv#1#2{((#1-(#2-1)/2)/#2)}
\def\moduloop#1#2{(#1-\truncdiv{#1}{#2}*#2)}
\def\modulo#1#2{\number\numexpr\moduloop{#1}{#2}\relax}

% orphans and widows
\clubpenalty=9996
\widowpenalty=9999
\brokenpenalty=4991
\predisplaypenalty=10000
\postdisplaypenalty=1549
\displaywidowpenalty=1602
\hyphenpenalty=400
% Copied from Rahtz but not understood
\def\@pnumwidth{1.55em}
\def\@tocrmarg {2.55em}
\def\@dotsep{4.5}
\emergencystretch 3em
\hbadness=4000
\pretolerance=750
\tolerance=2000
\vbadness=4000
\def\Gin@extensions{.pdf,.png,.jpg,.mps,.tif}
% \renewcommand{\@cite}[1]{#1} % biblio

\usepackage{hyperref} % supposed to be the last one, :o) except for the ones to follow
\urlstyle{same} % after hyperref
\hypersetup{
  % pdftex, % no effect
  pdftitle={\elbibl},
  % pdfauthor={Your name here},
  % pdfsubject={Your subject here},
  % pdfkeywords={keyword1, keyword2},
  bookmarksnumbered=true,
  bookmarksopen=true,
  bookmarksopenlevel=1,
  pdfstartview=Fit,
  breaklinks=true, % avoid long links
  pdfpagemode=UseOutlines,    % pdf toc
  hyperfootnotes=true,
  colorlinks=false,
  pdfborder=0 0 0,
  % pdfpagelayout=TwoPageRight,
  % linktocpage=true, % NO, toc, link only on page no
}

\makeatother % /@@@>
%%%%%%%%%%%%%%
% </TEI> end %
%%%%%%%%%%%%%%


%%%%%%%%%%%%%
% footnotes %
%%%%%%%%%%%%%
\renewcommand{\thefootnote}{\bfseries\textcolor{rubric}{\arabic{footnote}}} % color for footnote marks

%%%%%%%%%
% Fonts %
%%%%%%%%%
\usepackage[]{roboto} % SmallCaps, Regular is a bit bold
% \linespread{0.90} % too compact, keep font natural
\newfontfamily\fontrun[]{Roboto Condensed Light} % condensed runing heads
\ifav
  \setmainfont[
    ItalicFont={Roboto Light Italic},
  ]{Roboto}
\else\ifbooklet
  \setmainfont[
    ItalicFont={Roboto Light Italic},
  ]{Roboto}
\else
\setmainfont[
  ItalicFont={Roboto Italic},
]{Roboto Light}
\fi\fi
\renewcommand{\LettrineFontHook}{\bfseries\color{rubric}}
% \renewenvironment{labelblock}{\begin{center}\bfseries\color{rubric}}{\end{center}}

%%%%%%%%
% MISC %
%%%%%%%%

\setdefaultlanguage[frenchpart=false]{french} % bug on part


\newenvironment{quotebar}{%
    \def\FrameCommand{{\color{rubric!10!}\vrule width 0.5em} \hspace{0.9em}}%
    \def\OuterFrameSep{\itemsep} % séparateur vertical
    \MakeFramed {\advance\hsize-\width \FrameRestore}
  }%
  {%
    \endMakeFramed
  }
\renewenvironment{quoteblock}% may be used for ornaments
  {%
    \savenotes
    \setstretch{0.9}
    \normalfont
    \begin{quotebar}
  }
  {%
    \end{quotebar}
    \spewnotes
  }


\renewcommand{\headrulewidth}{\arrayrulewidth}
\renewcommand{\headrule}{{\color{rubric}\hrule}}

% delicate tuning, image has produce line-height problems in title on 2 lines
\titleformat{name=\chapter} % command
  [display] % shape
  {\vspace{1.5em}\centering} % format
  {} % label
  {0pt} % separator between n
  {}
[{\color{rubric}\huge\textbf{#1}}\bigskip] % after code
% \titlespacing{command}{left spacing}{before spacing}{after spacing}[right]
\titlespacing*{\chapter}{0pt}{-2em}{0pt}[0pt]

\titleformat{name=\section}
  [block]{}{}{}{}
  [\vbox{\color{rubric}\large\raggedleft\textbf{#1}}]
\titlespacing{\section}{0pt}{0pt plus 4pt minus 2pt}{\baselineskip}

\titleformat{name=\subsection}
  [block]
  {}
  {} % \thesection
  {} % separator \arrayrulewidth
  {}
[\vbox{\large\textbf{#1}}]
% \titlespacing{\subsection}{0pt}{0pt plus 4pt minus 2pt}{\baselineskip}

\ifaiv
  \fancypagestyle{main}{%
    \fancyhf{}
    \setlength{\headheight}{1.5em}
    \fancyhead{} % reset head
    \fancyfoot{} % reset foot
    \fancyhead[L]{\truncate{0.45\headwidth}{\fontrun\elbibl}} % book ref
    \fancyhead[R]{\truncate{0.45\headwidth}{ \fontrun\nouppercase\leftmark}} % Chapter title
    \fancyhead[C]{\thepage}
  }
  \fancypagestyle{plain}{% apply to chapter
    \fancyhf{}% clear all header and footer fields
    \setlength{\headheight}{1.5em}
    \fancyhead[L]{\truncate{0.9\headwidth}{\fontrun\elbibl}}
    \fancyhead[R]{\thepage}
  }
\else
  \fancypagestyle{main}{%
    \fancyhf{}
    \setlength{\headheight}{1.5em}
    \fancyhead{} % reset head
    \fancyfoot{} % reset foot
    \fancyhead[RE]{\truncate{0.9\headwidth}{\fontrun\elbibl}} % book ref
    \fancyhead[LO]{\truncate{0.9\headwidth}{\fontrun\nouppercase\leftmark}} % Chapter title, \nouppercase needed
    \fancyhead[RO,LE]{\thepage}
  }
  \fancypagestyle{plain}{% apply to chapter
    \fancyhf{}% clear all header and footer fields
    \setlength{\headheight}{1.5em}
    \fancyhead[L]{\truncate{0.9\headwidth}{\fontrun\elbibl}}
    \fancyhead[R]{\thepage}
  }
\fi

\ifav % a5 only
  \titleclass{\section}{top}
\fi

\newcommand\chapo{{%
  \vspace*{-3em}
  \centering % no vskip ()
  {\Large\addfontfeature{LetterSpace=25}\bfseries{\elauthor}}\par
  \smallskip
  {\large\eldate}\par
  \bigskip
  {\Large\selectfont{\eltitle}}\par
  \bigskip
  {\color{rubric}\hline\par}
  \bigskip
  {\Large TEXTE LIBRE À PARTICPATION LIBRE\par}
  \centerline{\small\color{rubric} {hurlus.fr, tiré le \today}}\par
  \bigskip
}}

\newcommand\cover{{%
  \thispagestyle{empty}
  \centering
  {\LARGE\bfseries{\elauthor}}\par
  \bigskip
  {\Large\eldate}\par
  \bigskip
  \bigskip
  {\LARGE\selectfont{\eltitle}}\par
  \vfill\null
  {\color{rubric}\setlength{\arrayrulewidth}{2pt}\hline\par}
  \vfill\null
  {\Large TEXTE LIBRE À PARTICPATION LIBRE\par}
  \centerline{{\href{https://hurlus.fr}{\dotuline{hurlus.fr}}, tiré le \today}}\par
}}

\begin{document}
\pagestyle{empty}
\ifbooklet{
  \cover\newpage
  \thispagestyle{empty}\hbox{}\newpage
  \cover\newpage\noindent Les voyages de la brochure\par
  \bigskip
  \begin{tabularx}{\textwidth}{l|X|X}
    \textbf{Date} & \textbf{Lieu}& \textbf{Nom/pseudo} \\ \hline
    \rule{0pt}{25cm} &  &   \\
  \end{tabularx}
  \newpage
  \addtocounter{page}{-4}
}\fi

\thispagestyle{empty}
\ifaiv
  \twocolumn[\chapo]
\else
  \chapo
\fi
{\it\elabstract}
\bigskip
\makeatletter\@starttoc{toc}\makeatother % toc without new page
\bigskip

\pagestyle{main} % after style

  
\salute{À la mémoire de Gérard Lebovici, assassiné à Paris, le 5 mars 1984, dans un guet-apens resté mystérieux.}
\noindent « Quelque critiques que puissent être la situation et les circonstances où vous vous trouvez, ne désespérez de rien ; c’est dans les occasions où tout est à craindre, qu’il ne faut rien craindre ; c’est lorsqu’on est environné de tous les dangers, qu’il n’en faut redouter aucun ; c’est lorsqu’on est sans aucune ressource, qu’il faut compter sur toutes ; c’est lorsqu’on est surpris, qu’il faut surprendre l’ennemi lui-même. »\par

\bibl{Sun Zi (\emph{L’art de la guerre})}

\labelblock{I}

\noindent Ces \emph{Commentaires} sont assurés d’être promptement connus de cinquante ou soixante personnes ; autant dire beaucoup dans les jours que nous vivons, et quand on traite de questions si graves. Mais aussi c’est parce que j’ai, dans certains milieux, la réputation d’être un connaisseur. Il faut également considérer que, de cette élite qui va s’y intéresser, la moitié, ou un nombre qui s’en approche de très près, est composée de gens qui s’emploient à maintenir le système de domination spectaculaire, et l’autre moitié de gens qui s’obstineront à faire tout le contraire. Ayant ainsi à tenir compte de lecteurs très attentifs et diversement influents, je ne peux évidemment parler en toute liberté. Je dois surtout prendre garde à ne pas trop instruire n’importe qui.\par
Le malheur des temps m’obligera donc à écrire, encore une fois, d’une façon nouvelle. Certains éléments seront volontairement omis ; et le plan devra rester assez peu clair. On pourra y rencontrer, comme la signature même de l’époque, quelques leurres. A condition d’intercaler çà et là plusieurs autres pages, le sens total peut apparaître : ainsi, bien souvent, des articles secrets ont été ajoutés à ce que des traités stipulaient ouvertement, et de même il arrive que des agents chimiques ne révèlent une part inconnue de leurs propriétés que lorsqu’ils se trouvent associés à d’autres. Il n’y aura, d’ailleurs, dans ce bref ouvrage, que trop de choses qui seront, hélas, faciles à comprendre.\par

\labelblock{II}

\noindent En 1967, j’ai montré dans un livre, \emph{La Société du Spectacle}, ce que le spectacle moderne était déjà essentiellement : le règne autocratique de l’économie marchande ayant accédé à un statut de souveraineté irresponsable, et l’ensemble des nouvelles techniques de gouvernement qui accompagnent ce règne. Les troubles de 1968, qui se sont prolongés dans divers pays au cours des années suivantes, n’ayant en aucun lieu abattu l’organisation existante de la société, dont il sourd comme spontanément, le spectacle a donc continué partout de se renforcer, c’est-à-dire à la fois de s’étendre aux extrêmes par tous les côtés, et d’augmenter sa densité au centre. Il a même appris de nouveaux procédés défensifs, comme il arrive ordinairement aux pouvoirs attaqués. Quand j’ai commencé la critique de la société spectaculaire, on a surtout remarqué, vu le moment, le contenu révolutionnaire que l’on pouvait découvrir dans cette critique, et on l’a ressenti, naturellement, comme son élément le plus fâcheux. Quant à la chose même, on m’a parfois accusé de l’avoir inventée de toutes pièces, et toujours de m’être complu dans l’outrance en évaluant la profondeur et l’unité de ce spectacle et de son action réelle. Je dois convenir que les autres, après, faisant paraître de nouveaux livres autour du même sujet, ont parfaitement démontré que l’on pouvait éviter d’en dire tant. Ils n’ont eu qu’à remplacer l’ensemble et son mouvement par un seul détail statique de la surface du phénomène, l’originalité de chaque auteur se plaisant à le choisir différent, et par là d’autant moins inquiétant. Aucun n’a voulu altérer la modestie scientifique de son interprétation personnelle en y mêlant de téméraires jugements historiques.\par
Mais enfin la société du spectacle n’en a pas moins continué sa marche. Elle va vite car, en 1967, elle n’avait guère plus d’une quarantaine d’années derrière elle ; mais pleinement employées. Et de son propre mouvement, que personne ne prenait plus la peine d’étudier, elle a montré depuis, par d’étonnants exploits, que sa nature effective était bien ce que j’avais dit. Ce point établi n’a pas seulement une valeur académique ; parce qu’il est sans doute indispensable d’avoir reconnu l’unité et l’articulation de la force agissante qu’est le spectacle, pour être à partir de là capable de rechercher dans quelles directions cette force a pu se déplacer, étant ce qu’elle était. Ces questions sont d’un grand intérêt : c’est nécessairement dans de telles conditions que se jouera la suite du conflit dans la société. Puisque le spectacle, à ce jour, est assurément plus puissant qu’il l’était auparavant, que fait-il de cette puissance supplémentaire ? jusqu’où s’est-il avancé, où il n’était pas précédemment ? Quelles sont, en somme, ses \emph{lignes d’opérations} en ce moment ? Le sentiment vague qu’il s’agit d’une sorte d’invasion rapide, qui oblige les gens à mener une vie très différente, est désormais largement répandu ; mais on ressent cela plutôt comme une modification inexpliquée du climat ou d’un autre équilibre naturel, modification devant laquelle l’ignorance sait seulement qu’elle n’a rien à dire. De plus, beaucoup admettent que c’est une invasion civilisatrice, au demeurant inévitable, et ont même envie d’y collaborer. Ceux-là aiment mieux ne pas savoir à quoi sert précisément cette conquête, et comment elle chemine.\par
Je vais évoquer quelques conséquences pratiques, encore peu connues, qui résultent de ce déploiement rapide du spectacle durant les vingt dernières années. Je ne me propose, sur aucun aspect de la question, d’en venir à des polémiques, désormais trop faciles et trop inutiles ; pas davantage de convaincre. Les présents commentaires ne se soucient pas de moraliser. Ils n’envisagent pas ce qui est souhaitable, ou seulement préférable. Ils s’en tiendront à noter ce qui est.\par

\labelblock{III}

\noindent Maintenant que personne ne peut raisonnablement douter de l’existence et de la puissance du spectacle, on peut par contre douter qu’il soit raisonnable d’ajouter quelque chose sur une question que l’expérience a tranchée d’une manière aussi draconienne. \emph{Le Monde} du 19 septembre 1987 illustrait avec bonheur la formule « \emph{Ce qui existe, on n’a donc plus besoin d’en parler »}, véritable loi fondamentale de ces temps spectaculaires qui, à cet égard au moins, n’ont laissé en retard aucun pays : « Que la société contemporaine soit une société de spectacle, c’est une affaire entendue. Il faudra bientôt remarquer ceux qui ne se font pas remarquer. On ne compte plus les ouvrages décrivant un phénomène qui en vient à caractériser les nations industrielles sans épargner les pays en retard sur leur temps. Mais en notant cette cocasserie que les livres qui analysent, en général pour le déplorer, ce phénomène doivent, eux aussi, sacrifier au spectacle pour se faire connaître. » Il est vrai que cette critique spectaculaire du spectacle, venue tard et qui pour comble voudrait « se faire connaître » sur le même terrain, s’en tiendra forcément à des généralités vaines ou à d’hypocrites regrets ; comme aussi paraît vaine cette sagesse désabusée qui bouffonne dans un journal.\par
La discussion creuse sur le spectacle, c’est-à-dire sur ce que font les propriétaires du monde, est ainsi organisée \emph{par lui-même} : on insiste sur les grands moyens du spectacle, afin de ne rien dire de leur grand emploi. On préfère souvent l’appeler, plutôt que spectacle, le médiatique. Et par là, on veut désigner un simple instrument, une sorte de service public qui gérerait avec un impartial « professionnalisme » la nouvelle richesse de la communication de tous par \emph{mass media}, communication enfin parvenue à la pureté unilatérale, où se fait paisiblement admirer la décision déjà prise. Ce qui est communiqué, ce sont \emph{des ordres} ; et, fort harmonieusement, ceux qui les ont donnés sont également ceux qui diront ce qu’ils en pensent.\par
Le pouvoir du spectacle, qui est si essentiellement unitaire, centralisateur par la force même des choses, et parfaitement despotique dans son esprit, s’indigne assez souvent de voir se constituer, sous son règne, une politique-spectacle, une justice-spectacle, une médecine-spectacle, ou tant d’aussi surprenants « excès médiatiques ». Ainsi le spectacle ne serait rien d’autre que l’excès du médiatique, dont la nature, indiscutablement bonne puisqu’il sert à communiquer, est parfois portée aux excès. Assez fréquemment, les maîtres de la société se déclarent mal servis par leurs employés médiatiques ; plus souvent ils reprochent à la plèbe des spectateurs sa tendance à s’adonner sans retenue, et presque bestialement, aux plaisirs médiatiques. On dissimulera ainsi, derrière une multitude virtuellement infinie de prétendues divergences médiatiques, ce qui est tout au contraire le résultat d’une convergence spectaculaire voulue avec une remarquable ténacité. De même que la logique de la marchandise prime sur les diverses ambitions concurrentielles de tous les commerçants, ou que la logique de la guerre domine toujours les fréquentes modifications de l’armement, de même la logique sévère du spectacle commande partout la foisonnante diversité des extravagances médiatiques.\par
Le changement qui a le plus d’importance, dans tout ce qui s’est passé depuis vingt ans, réside dans la continuité même du spectacle. Cette importance ne tient pas au perfectionnement de son instrumentation médiatique, qui avait déjà auparavant atteint un stade de développement très avancé : c’est tout simplement que la domination spectaculaire ait pu élever une génération pliée à ses lois. Les conditions extraordinairement neuves dans lesquelles cette génération, dans l’ensemble, a effectivement vécu, constituent un résumé exact et suffisant de tout ce que désormais le spectacle empêche ; et aussi de tout ce qu’il permet.\par

\labelblock{IV}

\noindent Sur le plan simplement théorique, il ne me faudra ajouter à ce que j’avais formulé antérieurement qu’un détail, mais qui va loin. En 1967, je distinguais deux formes, successives et rivales, du pouvoir spectaculaire, la concentrée et la diffuse. L’une et l’autre planaient au-dessus de la société réelle, comme son but et son mensonge. La première, mettant en avant l’idéologie résumée autour d’une personnalité dictatoriale, avait accompagné la contre-révolution totalitaire, la nazie aussi bien que la stalinienne. L’autre, incitant les salariés à opérer librement leur choix entre une grande variété de marchandises nouvelles qui s’affrontaient, avait représenté cette américanisation du monde, qui effrayait par quelques aspects, mais aussi bien séduisait les pays où avaient pu se maintenir plus longtemps les conditions des démocraties bourgeoises de type traditionnel. Une troisième forme s’est constituée depuis, par la combinaison raisonnée des deux précédentes, et sur la base générale d’une victoire de celle qui s’était montrée la plus forte, la forme diffuse. Il s’agit du \emph{spectaculaire intégré}, qui désormais tend à s’imposer mondialement.\par
La place prédominante qu’ont tenue la Russie et l’Allemagne dans la formation du spectaculaire concentré, et les États-Unis dans celle du spectaculaire diffus, semble avoir appartenu à la France et à l’Italie au moment de la mise en place du spectaculaire intégré, par le jeu d’une série de facteurs historiques communs : rôle important des parti et syndicat staliniens dans la vie politique et intellectuelle, faible tradition démocratique, longue monopolisation du pouvoir par un seul parti de gouvernement, nécessité d’en finir avec une contestation révolutionnaire apparue par surprise.\par
Le spectaculaire intégré se manifeste à la fois comme concentré et comme diffus, et depuis cette unification fructueuse il a su employer plus grandement l’une et l’autre qualité. Leur mode d’application antérieur a beaucoup changé. À considérer le côté concentré, le centre directeur en est maintenant devenu occulte : on n’y place jamais plus un chef connu, ni une idéologie claire. Et à considérer le côté diffus, l’influence spectaculaire n’avait jamais marqué à ce point la presque totalité des conduites et des objets qui sont produits socialement. Car le sens final du spectaculaire intégré, c’est qu’il s’est intégré dans la réalité même à mesure qu’il en parlait ; et qu’il la reconstruisait comme il en parlait. De sorte que cette réalité maintenant ne se tient plus en face de lui comme quelque chose d’étranger. Quand le spectaculaire était concentré la plus grande part de la société périphérique lui échappait ; et quand il était diffus, une faible part ; aujourd’hui rien. Le spectacle s’est mélangé à toute réalité, en l’irradiant. Comme on pouvait facilement le prévoir en théorie, l’expérience pratique de l’accomplissement sans frein des volontés de la raison marchande aura montré vite et sans exceptions que le devenir-monde de la falsification était aussi un devenir-falsification du monde. Hormis un héritage encore important, mais destiné à se réduire toujours, de livres et de bâtiments anciens, qui du reste sont de plus en plus souvent sélectionnés et mis en perspective selon les convenances du spectacle, il n’existe plus rien, dans la culture et dans la nature, qui n’ait été transformé, et pollué, selon les moyens et les intérêts de l’industrie moderne. La génétique même est devenue pleinement accessible aux forces dominantes de la société.\par
Le gouvernement du spectacle, qui à présent détient tous les moyens de falsifier l’ensemble de la production aussi bien que de la perception, est maître absolu des souvenirs comme il est maître incontrôlé des projets qui façonnent le plus lointain avenir. Il règne seul partout ; \emph{il exécute ses jugements sommaires}.\par
C’est dans de telles conditions que l’on peut voir se déchaîner soudainement, avec une allégresse carnavalesque, une fin parodique de la division du travail ; d’autant mieux venue qu’elle coïncide avec le mouvement général de disparition de toute vraie compétence. Un financier va chanter, un avocat va se faire indicateur de police, un boulanger va exposer ses préférences littéraires, un acteur va gouverner, un cuisinier va philosopher sur les moments de cuisson comme jalons dans l’histoire universelle. Chacun peut surgir dans le spectacle afin de s’adonner publiquement, ou parfois pour s’être livré secrètement, à une activité complètement autre que la spécialité par laquelle il s’était d’abord fait connaître. Là où la possession d’un « statut médiatique » a pris une importance infiniment plus grande que la valeur de ce que l’on a été capable de faire réellement, il est normal que ce statut soit aisément transférable, et confère le droit de briller, de la même façon, n’importe où ailleurs. Le plus souvent, ces particules médiatiques accélérées poursuivent leur simple carrière dans l’admirable statutairement garanti. Mais il arrive que la transition médiatique fasse la \emph{couverture} entre beaucoup d’entreprises, officiellement indépendantes, mais en fait secrètement reliées par différents réseaux \emph{ad hoc}. De sorte que, parfois, la division sociale du travail, ainsi que la solidarité couramment prévisible de son emploi, reparaissent sous des formes tout à fait nouvelles : par exemple, on peut désormais publier un roman pour préparer un assassinat. Ces pittoresques exemples veulent dire aussi que l’on ne peut plus se fier à personne sur son métier.\par
Mais l’ambition la plus haute du spectaculaire intégré, c’est encore que les agents secrets deviennent des révolutionnaires, et que les révolutionnaires deviennent des agents secrets.\par

\labelblock{V}

\noindent La société modernisée jusqu’au stade du spectaculaire intégré se caractérise par l’effet combiné de cinq traits principaux, qui sont : le renouvellement technologique incessant ; la fusion économico-étatique ; le secret généralisé ; le faux sans réplique ; un présent perpétuel.\par
Le mouvement d’innovation technologique dure depuis longtemps, et il est constitutif de la société capitaliste, dite parfois industrielle ou post-industrielle. Mais depuis qu’il a pris sa plus récente accélération (au lendemain de la Deuxième Guerre mondiale), il renforce d’autant mieux l’autorité spectaculaire, puisque par lui chacun se découvre entièrement livré à l’ensemble des spécialistes, à leurs calculs et à leurs jugements toujours satisfaits sur ces calculs. La fusion économico-étatique est la tendance la plus manifeste de ce siècle ; et elle y est pour le moins devenue le moteur du développement économique le plus récent. L’alliance défensive et offensive conclue entre ces deux puissances, l’économie et l’État, leur a assuré les plus grands bénéfices communs, dans tous les domaines : on peut dire de chacune qu’elle possède l’autre ; il est absurde de les opposer, ou de distinguer leurs raisons et leurs déraisons. Cette union s’est aussi montrée extrêmement favorable au développement de la domination spectaculaire, qui précisément, dès sa formation, n’était pas autre chose. Les trois derniers traits sont les effets directs de cette domination, à son stade intégré.\par
Le secret généralisé se tient derrière le spectacle, comme le complément décisif de ce qu’il montre et, si l’on descend au fond des choses, comme sa plus importante opération.\par
Le seul fait d’être désormais sans réplique a donné au faux une qualité toute nouvelle. C’est (lu même coup le vrai qui a cessé d’exister presque partout, ou dans le meilleur cas s’est vu réduit à l’état d’une hypothèse qui ne peut jamais être démontrée. Le faux sans réplique a achevé de faire disparaître l’opinion publique, qui d’abord s’était trouvée incapable de se faire entendre ; puis, très vite par la suite, de seulement se former. Cela entraîne évidemment d’importantes conséquences dans la politique, les sciences appliquées, la justice, la connaissance artistique.\par
La construction d’un présent où la mode elle-même, de l’habillement aux chanteurs, s’est immobilisée, qui veut oublier le passé et qui ne donne plus l’impression de croire à un avenir, est obtenue par l’incessant passage circulaire de l’information, revenant à tout instant sur une liste très succincte des mêmes vétilles, annoncées passionnément comme d’importantes nouvelles ; alors que ne passent que rarement, et par brèves saccades, les nouvelles véritablement importantes, sur ce qui change effectivement. Elles concernent toujours la condamnation que ce monde semble avoir prononcée contre son existence, les étapes de son auto-destruction programmée.\par

\labelblock{VI}

\noindent La première intention de la domination spectaculaire était de faire disparaître la connaissance historique en général ; et d’abord presque toutes les informations et tous les commentaires raisonnables sur le plus récent passé. Une si flagrante évidence n’a pas besoin d’être expliquée. Le spectacle organise avec maîtrise l’ignorance de ce qui advient et, tout de suite après, l’oubli de ce qui a pu quand même en être connu. Le plus important est le plus caché. Rien, depuis vingt ans, n’a été recouvert de tant de mensonges commandés que l’histoire de mai 1968. D’utiles leçons ont pourtant été tirées de quelques études démystifiées sur ces journées et leurs origines ; mais c’est le secret de l’État.\par
En France, il y a déjà une dizaine d’années, un président de la République, oublié depuis mais flottant alors à la surface du spectacle, exprimait naïvement la joie qu’il ressentait, « sachant que nous vivrons désormais dans un monde sans mémoire, où, comme sur la surface de l’eau, l’image chasse indéfiniment l’image ». C’est en effet commode pour qui est aux affaires ; et sait y rester. La fin de l’histoire est un plaisant repos pour tout pouvoir présent. Elle lui garantit absolument le succès de l’ensemble de ses entreprises, ou du moins le bruit du succès.\par
Un pouvoir absolu supprime d’autant plus radicalement l’histoire qu’il a pour ce faire des intérêts ou des obligations plus impérieux, et surtout selon qu’il a trouvé de plus ou moins grandes facilités pratiques d’exécution. Qin Shi Huangdi a fait brûler les livres, mais il n’a pas réussi à les faire disparaître tous. Staline avait poussé plus loin la réalisation d’un tel projet dans notre siècle mais, malgré les complicités de toutes sortes qu’il a pu trouver hors des frontières de son empire, il restait une vaste zone du monde inaccessible à sa police, où l’on riait de ses impostures. Le spectaculaire intégré a fait mieux, avec de très nouveaux procédés, et en opérant cette fois mondialement. L’ineptie qui se fait respecter partout, il n’est plus permis d’en rire ; en tout cas il est devenu impossible de faire savoir qu’on en rit.\par
Le domaine de l’histoire était le mémorable, la totalité des événements dont les conséquences se manifesteraient longtemps. C’était inséparablement la connaissance qui devrait durer, et aiderait à comprendre, au moins partiellement, ce qu’il adviendrait de nouveau : « une acquisition pour toujours », dit Thucydide. Par là l’histoire était la \emph{mesure} d’une nouveauté véritable ; et qui vend la nouveauté a tout intérêt à faire disparaître le moyen de la mesurer. Quand l’important se fait socialement reconnaître comme ce qui est instantané, et va l’être encore l’instant d’après, autre et même, et que remplacera toujours une autre importance instantanée, on peut aussi bien dire que le moyen employé garantit une sorte d’éternité de cette non-importance, qui parle si haut.\par
Le précieux avantage que le spectacle a retiré de cette \emph{mise hors la loi} de l’histoire, d’avoir déjà condamné toute l’histoire récente à passer à la clandestinité, et d’avoir réussi à faire oublier très généralement l’esprit historique dans la société, c’est d’abord de couvrir sa propre histoire : le mouvement même de sa récente conquête du monde. Son pouvoir apparaît déjà familier, comme s’il avait depuis toujours été là. Tous les usurpateurs ont voulu faire oublier \emph{qu’ils viennent d’arriver}.\par

\labelblock{VII}

\noindent Avec la destruction de l’histoire, c’est l’événement contemporain lui-même qui s’éloigne aussitôt dans une distance fabuleuse, parmi ses récits invérifiables, ses statistiques incontrôlables, ses explications invraisemblables et ses raisonnements intenables. À toutes les sottises qui sont avancées spectaculairement, il n’y a Jamais que des médiatiques qui pourraient répondre, par quelques respectueuses rectifications ou remontrances, et encore en sont ils avares car, outre leur extrême ignorance, leur Solidarité, de métier et de cœur, avec l’autorité générale du spectacle, et avec la société qu’il exprime, leur fait un devoir, et aussi un plaisir, (le ne jamais s’écarter de cette autorité, dont la majesté ne doit pas être lésée. Il ne faut pas oublier que tout médiatique, et par salaire et par autres récompenses ou soultes, a toujours un maître, parfois plusieurs ; et que tout médiatique se sait remplaçable.\par
Tous les experts sont médiatiques-étatiques, et ne sont reconnus experts que par là. Tout expert sert son maître, car chacune des anciennes possibilités d’indépendance a été à peu près réduite à rien par les conditions d’organisation de la société présente. L’expert qui sert le mieux, c’est, bien sûr, l’expert qui ment. Ceux qui ont besoin de l’expert, ce sont, pour des motifs différents, le falsificateur et l’ignorant. Là où l’individu n’y reconnaît plus rien par lui-même, il sera formellement rassuré par l’expert. Il était auparavant normal qu’il y ait des experts de l’art des Étrusques ; et ils étaient toujours compétents, car l’art étrusque n’est pas sur le marché. Mais, par exemple, une époque qui trouve rentable de falsifier chimiquernent nombre de vins célèbres, ne pourra les vendre que si elle a formé des experts en vins qui entraîneront les \emph{caves} à aimer leurs nouveaux parfums, plus reconnaissables. Cervantès remarque que « sous un mauvais manteau, on trouve souvent un bon buveur ». Celui qui connaît le vin ignore souvent les règles de l’industrie nucléaire ; mais la domination spectaculaire estime que, puisqu’un expert s’est moqué de lui à propos d’industrie nucléaire, un autre expert pourra bien s’en moquer à propos du vin. Et on sait, par exemple, combien l’expert en météorologie médiatique, qui annonce les températures ou les pluies prévues pour les quarante-huit heures à venir, est tenu à beaucoup de réserves par l’obligation de maintenir des équilibres économiques, touristiques et régionaux, quand tant de gens circulent si souvent sur tant de routes, entre des lieux également désolés ; de sorte qu’il aura plutôt à réussir comme amuseur.\par
Un aspect de la disparition de toute connaissance historique objective se manifeste à propos de n’importe quelle réputation personnelle, qui est devenue malléable et rectifiable à volonté par ceux qui contrôlent toute l’information, celle que l’on recueille et aussi celle, bien différente, que l’on diffuse ; ils ont donc toute licence pour falsifier. Car une évidence historique dont on ne veut rien savoir dans le spectacle n’est plus une évidence. Là où personne n’a plus que la renommée qui lui a été attribuée comme une faveur par la bienveillance d’une Cour spectaculaire, la disgrâce peut suivre instantanément. Une notoriété anti-spectaculaire est devenue quelque chose d’extrêmement rare. Je suis moi-même l’un des derniers vivants à en posséder une ; à n’en avoir jamais eu d’autre. Mais c’est aussi devenu extraordinairement suspect. La société s’est officiellement proclamée spectaculaire. Être connu en dehors des relations spectaculaires, cela équivaut déjà à être connu comme ennemi de la société.\par
Il est permis de changer du tout au tout le passé de quelqu’un, de le modifier radicalement, de le recréer dans le style des procès de Moscou ; et sans qu’il soit même nécessaire de recourir aux lourdeurs d’un procès. On peut tuer à moindres frais. Les faux témoins, peut-être maladroits – mais quelle capacité de sentir cette maladresse pourrait-elle rester aux spectateurs qui seront témoins des exploits de ces faux témoins ? – et les faux documents, toujours excellents, ne peuvent manquer à ceux qui gouvernent le spectaculaire intégré, ou à leurs amis. Il n’est donc plus possible de croire, sur personne, rien de ce qui n’a pas été connu par soi-même, et directement. Mais, en fait, on n’a même plus très souvent besoin d’accuser faussement quelqu’un. Dès lors que l’on détient le mécanisme commandant la seule vérification sociale qui se fait pleinement et universellement reconnaître, on dit ce qu’on veut. Le mouvement de la démonstration spectaculaire se prouve simplement en marchant en rond : en revenant, en se répétant, en continuant d’affirmer sur l’unique terrain où réside désormais ce qui peut s’affirmer publiquement, et se faire croire, puisque c’est de cela seulement que tout le monde sera témoin. L’autorité spectaculaire peut également nier n’importe quoi, une fois, trois fois, et dire qu’elle n’en parlera plus, et parler d’autre chose ; sachant bien qu’elle ne risque plus aucune autre riposte sur son propre terrain, ni sur un autre. Car il n’existe plus d’agora, de communauté générale ; ni même de communautés restreintes à des corps intermédiaires ou à des institutions autonomes, à des salons ou des cafés, aux travailleurs d’une seule entreprise ; nulle place où le débat sur les vérités qui concernent ceux qui sont là puisse s’affranchir durablement de l’écrasante présence du discours médiatique, et des différentes forces organisées pour le relayer. Il n’existe plus maintenant de jugement, garanti relativement indépendant, de ceux qui constituaient le monde savant ; de ceux par exemple qui, autrefois, plaçaient leur fierté dans une capacité de vérification, permettant d’approcher ce qu’on appelait l’histoire impartiale des faits, de croire au moins qu’elle méritait d’être connue. Il n’y a même plus de vérité bibliographique incontestable, et les résumés informatisés des fichiers des bibliothèques nationales pourront en supprimer d’autant mieux les traces. On s’égarerait en pensant à ce que furent naguère des magistrats, des médecins, des historiens, et aux obligations impératives qu’ils se reconnaissaient, souvent, dans les limites de leurs compétences : \emph{les hommes ressemblent plus à leur temps qu’à leur père}.\par
Ce dont le spectacle peut cesser de parler pendant trois jours est comme ce qui n’existe pas. Car il parle alors de quelque chose d’autre, et c’est donc cela qui, dès lors, en somme, existe. Les conséquences pratiques, on le voit, en sont immenses.\par
On croyait savoir que l’histoire était apparue, en Grèce, avec la démocratie. On peut vérifier qu’elle disparaît du monde avec elle.\par
Il faut pourtant ajouter, à cette liste des triomphes du pouvoir, un résultat pour lui négatif : un État, dans la gestion duquel s’installe durablement un grand déficit de connaissances historiques, ne peut plus être conduit stratégiquement.\par

\labelblock{VIII}

\noindent La société qui s’annonce démocratique, quand elle est parvenue au stade du spectaculaire intégré, semble être admise partout comme étant la réalisation d’une \emph{perfection fragile}. De sorte qu’elle ne doit plus être exposée à des attaques, puisqu’elle est fragile ; et du reste n’est plus attaquable, puisque parfaite comme jamais société ne fut. C’est une société fragile parce qu’elle a grand mal à maîtriser sa dangereuse expansion technologique. Mais c’est une société parfaite pour être gouvernée ; et la preuve, c’est que tous ceux qui aspirent à gouverner veulent gouverner celle-là, par les mêmes procédés, et la maintenir presque exactement comme elle est. C’est la première fois, dans l’Europe contemporaine, qu’aucun parti ou fragment de parti n’essaie plus de seulement prétendre qu’il tenterait de changer quelque chose d’important. La marchandise ne peut plus être critiquée par personne : ni en tant que système général, ni même en tant que cette pacotille déterminée qu’il aura convenu aux chefs d’entreprises de mettre pour l’instant sur le marché. Partout où règne le spectacle, les seules forces organisées sont celles qui veulent le spectacle. Aucune ne peut donc plus être ennemie de ce qui existe, ni transgresser l’\emph{omertà} qui concerne tout. On en a fini avec cette inquiétante conception, qui avait dominé durant plus de deux cents ans, selon laquelle une société pouvait être critiquable et transformable, réformée ou révolutionnée. Et cela n’a pas été obtenu par l’apparition d’arguments nouveaux, mais tout simplement parce que les arguments sont devenus inutiles. À ce résultat, on mesurera, plutôt que le bonheur général, la force redoutable des réseaux de la tyrannie.\par
Jamais censure n’a été plus parfaite. Jamais l’opinion de ceux à qui l’on fait croire encore, dans quelques pays, qu’ils sont restés des citoyens libres, n’a été moins autorisée à se faire connaître, chaque fois qu’il s’agit d’un choix qui affectera leur vie réelle. Jamais il n’a été permis de leur mentir avec une si parfaite absence de conséquence. Le spectateur est seulement censé ignorer tout, ne mériter rien. Qui regarde toujours, pour savoir la suite, n’agira jamais : et tel doit bien être le spectateur. On entend citer fréquemment l’exception des États-Unis, où Nixon avait fini par pâtir un jour d’une série de dénégations trop cyniquement maladroites ; mais cette exception toute locale, qui avait quelques vieilles causes historiques, n’est manifestement plus vraie, puisque Reagan a pu faire récemment la même chose avec impunité. Tout ce qui n’est jamais sanctionné est véritablement permis. Il est donc archaïque de parler de scandale. On prête à un homme d’État italien de premier plan, ayant siégé simultanément dans le ministère et dans le gouvernement parallèle appelé P. 2, \emph{Potere Due}, un mot qui résume le plus profondément la période où, un peu après l’Italie et les États-Unis, est entré le monde entier : « Il y avait des scandales, mais il n’y en a plus. »\par
Dans \emph{Le 18 Brumaire de Louis Bonaparte}, Marx décrivait le rôle envahissant de l’Etat dans la France du second Empire, riche alors d’un demi-million de fonctionnaires : « Tout devint ainsi un objet de l’activité gouvernementale, depuis le pont, la maison d’école, la propriété communale d’un village jusqu’aux chemins de fer, aux propriétés nationales et aux universités provinciales. » La fameuse question du financement des partis politiques se posait déjà à l’époque, puisque Marx note que « les partis qui, à tour de rôle, luttaient pour la suprématie, voyaient dans la prise de possession de cet édifice énorme la proie principale du vainqueur ». Voilà qui sonne tout de même un peu bucolique et, comme on dit, dépassé, puisque les spéculations de l’État d’aujourd’hui concernent plutôt les villes nouvelles et les autoroutes, la circulation souterraine et la production d’énergie électro-nucléaire, la recherche pétrolière et les ordinateurs, l’administration des banques et les centres socio-culturels, les modifications du « paysage audiovisuel » et les exportations clandestines d’armes, la promotion immobilière et l’industrie pharmaceutique, l’agro-alimentaire et la gestion des hôpitaux, les crédits militaires et les fonds secrets du département, à toute heure grandissant, qui doit gérer les nombreux services de protection de la société. Et pourtant Marx est malheureusement resté trop longtemps actuel, qui évoque dans le même livre ce gouvernement « qui ne prend pas la nuit des décisions qu’il veut exécuter dans la journée, mais décide le jour et exécute la nuit ».\par

\labelblock{IX}

\noindent Cette démocratie si parfaite fabrique elle-même son inconcevable ennemi, le terrorisme. Elle veut, en effet, \emph{être jugée sur ses ennemis plutôt que sur ses résultats}. L’histoire du terrorisme est écrite par l’État ; elle est donc éducative. Les populations spectatrices ne peuvent certes pas tout savoir du terrorisme, mais elles peuvent toujours en savoir assez pour être persuadées que, par rapport à ce terrorisme, tout le reste devra leur sembler plutôt acceptable, en tout cas plus rationnel et plus démocratique.\par
La modernisation de la répression a fini par mettre au point, d’abord dans l’expérience pilote de l’Italie sous le nom de « repentis », des \emph{accusateurs professionnels} assermentés ; ce qu’à leur première apparition au XVII\textsuperscript{e} siècle, lors des troubles de la Fronde, on avait appelé des « témoins à brevet ». Ce progrès spectaculaire de la justice a peuplé les prisons italiennes de plusieurs milliers de condamnés qui expient une guerre civile qui n’a pas eu lieu, une sorte de vaste insurrection armée qui par hasard n’a jamais vu venir son heure, un putschisme tissé de l’étoffe dont sont faits les rêves.\par
On peut remarquer que l’interprétation des mystères du terrorisme paraît avoir introduit une symétrie entre des opinions contradictoires ; comme s’il s’agissait de deux écoles philosophiques professant des constructions métaphysiques absolument antagonistes. Certains ne verraient dans le terrorisme rien de plus que quelques évidentes manipulations par des services secrets ; d’autres estimeraient qu’au contraire il ne faut reprocher aux terroristes que leur manque total de sens historique. L’emploi d’un peu de logique historique permettrait de conclure assez vite qu’il n’y a rien de contradictoire à considérer que des gens qui manquent de tout sens historique peuvent également être manipulés ; et même encore plus facilement que d’autres. Il est aussi plus facile d’amener à « se repentir » quelqu’un à qui l’on peut montrer que l’on savait tout, d’avance, de ce qu’il a cru faire librement. C’est un effet inévitable des formes organisationnelles clandestines de type militaire, qu’il suffit d’infiltrer peu de gens en certains points du réseau pour en faire marcher, et tomber, beaucoup. La critique, dans ces questions d’évaluation des luttes armées, doit analyser quelquefois une de ces opérations en particulier, sans se laisser égarer par la ressemblance générale que toutes auraient éventuellement revêtue. On devrait d’ailleurs s’attendre, comme logiquement probable, à ce que les services de protection de l’État pensent à utiliser tous les avantages qu’ils rencontrent sur le terrain du spectacle, lequel justement a été de longue date organisé pour cela ; c’est au contraire la difficulté de s’en aviser qui est étonnante, et ne sonne pas juste.\par
L’intérêt actuel de la justice répressive dans ce domaine consiste bien sûr à généraliser au plus vite. L’important dans cette sorte de marchandise, c’est l’emballage, ou l’étiquette : les barres de codage. Tout ennemi de la démocratie spectaculaire en vaut un autre, comme se valent toutes les démocraties spectaculaires. Ainsi, il ne peut plus y avoir de droit d’asile pour les terroristes, et même si l’on ne leur reproche pas de l’avoir été, ils vont certainement le devenir, et l’extradition s’impose. En novembre 1978, sur le cas de Gabor Winter, jeune ouvrier typographe accusé principalement, par le gouvernement de la République Fédérale Allemande, d’avoir rédigé quelques tracts révolutionnaires, Mlle Nicole Pradain, représentant du ministère public devant la Chambre d’accusation de la Cour d’appel de Paris, a vite démontré que « les motivations politiques », seule cause de refus d’extradition prévue par la convention franco-allemande du 29 novembre 1951, ne pouvaient être invoquées : « Gabor Winter n’est pas un délinquant politique, mais social. Il refuse les contraintes sociales. Un vrai délinquant politique n’a pas de sentiment de rejet devant la société. Il s’attaque aux structures politiques et non, comme Gabor Winter, aux structures sociales. » La notion du délit politique respectable ne s’est vue reconnaître en Europe qu’à partir du moment où la bourgeoisie avait attaqué avec succès les structures sociales antérieurement établies. La qualité de délit politique ne pouvait se disjoindre des diverses intentions de la critique sociale. C’était vrai pour Blanqui, Varlin, Durruti. On affecte donc maintenant de vouloir garder, comme un luxe peu coûteux, un délit purement politique, que personne sans doute n’aura plus jamais l’occasion de commettre, puisque personne ne s’intéresse plus au sujet ; hormis les professionnels de la politique eux-mêmes, dont les délits ne sont presque jamais poursuivis, et ne s’appellent pas non plus politiques. Tous les délits et les crimes sont effectivement sociaux. Mais de tous les crimes sociaux, aucun ne devra être regardé comme pire que l’impertinente prétention de vouloir encore changer quelque chose dans cette société, qui pense qu’elle n’a été jusqu’ici que trop patiente et trop bonne ; mais qui \emph{ne veut plus être blâmée}.\par

\labelblock{X}

\noindent La dissolution de la logique a été poursuivie, selon les intérêts fondamentaux du nouveau système de domination, par différents moyens qui ont opéré en se prêtant toujours un soutien réciproque. Plusieurs de ces moyens tiennent à l’instrumentation technique qu’a expérimentée et popularisée le spectacle ; mais quelques-uns sont plutôt liés à la psychologie de masse de la soumission.\par
Sur le plan des techniques, quand l’image construite et choisie par \emph{quelqu’un d’autre} est devenue le principal rapport de l’individu au monde qu’auparavant il regardait par lui-même, de chaque endroit où il pouvait aller, on n’ignore évidemment pas que l’image va supporter tout ; parce qu’à l’intérieur d’une même image on peut juxtaposer sans contradiction n’importe quoi. Le flux des images emporte tout, et c’est également quelqu’un d’autre qui gouverne à son gré ce résumé simplifié du monde sensible ; qui choisit où ira ce courant, et aussi le rythme de ce qui devra s’y manifester, comme perpétuelle surprise arbitraire, ne voulant laisser nul temps à la réflexion, et tout à fait indépendamment de ce que le spectateur peut en comprendre ou en penser. Dans cette expérience concrète de la soumission permanente, se trouve la racine psychologique de l’adhésion si générale à ce qui est là ; qui en vient à lui reconnaître \emph{ipso facto} une valeur suffisante. Le discours spectaculaire tait évidemment, outre ce qui est proprement secret, tout ce qui ne lui convient pas. Il isole toujours, de ce qu’il montre, l’entourage, le passé, les intentions, les conséquences. Il est donc totalement illogique. Puisque personne ne peut plus le contredire, le spectacle a le droit de se contredire lui-même, de rectifier son passé. La hautaine attitude de ses serviteurs quand ils ont à faire savoir une version nouvelle, et peut-être plus mensongère encore, de certains faits, est de rectifier rudement l’ignorance et les mauvaises interprétations attribuées à leur public, alors qu’ils sont ceux-là mêmes qui s’empressaient la veille de répandre cette erreur, avec leur assurance coutumière. Ainsi, l’enseignement du spectacle et l’ignorance des spectateurs passent indûment pour des facteurs antagoniques alors qu’ils naissent l’un de l’autre. Le langage binaire de l’ordinateur est également une irrésistible incitation à admettre dans chaque instant, sans réserve, ce qui a été programmé comme l’a bien voulu quelqu’un d’autre, et qui se fait passer pour la source intemporelle d’une logique supérieure, impartiale et totale. Quel gain de vitesse, et de vocabulaire, pour juger de tout ! Politique ? Social ? Il faut choisir. Ce qui est l’un ne peut être l’autre. Mon choix s’impose. On nous siffle, et l’on sait pour qui sont ces structures. Il n’est donc pas surprenant que, dès l’enfance, les écoliers aillent facilement commencer, et avec enthousiasme, par le Savoir Absolu de l’informatique : tandis qu’ils ignorent toujours davantage la lecture, qui exige un véritable jugement à toutes les lignes ; et qui seule aussi peu donner accès à la vaste expérience humaine anté-spectaculaire. Car la conversation est presque morte, et bientôt le seront beaucoup de ceux qui savaient parler.\par
Sur le plan des moyens de la pensée des populations contemporaines, la première cause de la décadence tient clairement au fait que tout discours montré dans le spectacle ne laisse aucune place à la réponse ; et la logique ne s’était socialement formée que dans le dialogue. Mais aussi, quand s’est répandu le respect de ce qui parle dans le spectacle, qui est censé être important, riche, prestigieux, qui \emph{est l’autorité même}, la tendance se répand aussi parmi les spectateurs de vouloir être aussi illogiques que le spectacle, pour afficher un reflet individuel de cette autorité. Enfin, la logique n’est pas facile, et personne n’a souhaité la leur enseigner. Aucun drogué n’étudie la logique ; parce qu’il n’en a plus besoin, et parce qu’il n’en a plus la possibilité. Cette paresse du spectateur est aussi celle de n’importe quel cadre intellectuel, du spécialiste vite formé, qui essaiera dans tous les cas de cacher les étroites limites de ses connaissances par la répétition dogmatique de quelque argument d’autorité illogique.\par

\labelblock{XI}

\noindent On croit généralement que ceux qui ont montré la plus grande incapacité en matière de logique sont précisément ceux qui se sont proclamés révolutionnaires. Ce reproche injustifié vient d’une époque antérieure, où presque tout le monde pensait avec un minimum de logique, à l’éclatante exception des crétins et des militants ; et chez ceux-ci la mauvaise foi souvent s’y mêlait, voulue parce que crue efficace. Mais il n’est pas possible aujourd’hui de négliger le fait que l’usage intensif du spectacle a, comme il fallait s’y attendre, rendu idéologue la majorité des contemporains, quoique seulement par saccades et par fragments. Le manque de logique, c’est-à-dire la perte de la possibilité de reconnaître instantanément ce qui est important et ce qui est mineur ou hors de la question ; ce qui est incompatible ou inversement pourrait bien être complémentaire ; tout ce qu’implique telle conséquence et ce que, du même coup, elle interdit ; cette maladie a été volontairement injectée à haute dose dans la population par \emph{les anesthésistes-réanimateurs} du spectacle. Les contestataires n’ont été d’aucune manière plus irrationnels que les gens soumis. C’est seulement que, chez eux, cette irrationalité générale se voit plus intensément, parce qu’en affichant leur projet, ils ont essayé de mener une opération pratique ; ne serait-ce que lire certains textes en montrant qu’ils en comprennent le sens. Ils se sont donné diverses obligations de dominer la logique, et jusqu’à la stratégie, qui est très exactement le champ complet du déploiement de la logique dialectique des conflits ; alors que, tout comme les autres, ils sont même fort dépourvus de la simple capacité de se guider sur les vieux instruments imparfaits de la logique formelle. On n’en doute pas à propos d’eux ; alors que l’on n’y pense guère à propos des autres.\par
L’individu que cette pensée spectaculaire appauvrie a marqué en profondeur, et \emph{plus que tout autre élément de sa formation}, se place ainsi d’entrée de jeu au service de l’ordre établi, alors que son intention subjective a pu être complètement contraire à ce résultat. Il suivra pour l’essentiel le langage du spectacle, car c’est le seul qui lui est familier : celui dans lequel on lui a appris à parler. Il voudra sans doute se montrer ennemi de sa rhétorique ; mais il emploiera sa syntaxe. C’est un des points les plus importants de la réussite obtenue par la domination spectaculaire.\par
La disparition si rapide du vocabulaire préexistant n’est qu’un moment de cette opération. Elle la sert.\par

\labelblock{XII}

\noindent L’effacement de la personnalité accompagne fatalement les conditions de l’existence concrètement soumise aux normes spectaculaires, et ainsi toujours plus séparée des possibilités de connaître des expériences qui soient authentiques, et par là de découvrir ses préférences individuelles. L’individu, paradoxalement, devra se renier en permanence, s’il tient à être un peu considéré dans une telle société. Cette existence postule en effet une fidélité toujours changeante, une suite d’adhésions constamment décevantes à des produits fallacieux. Il s’agit de courir vite derrière l’inflation des signes dépréciés de la vie. La drogue aide à se conformer à cette organisation des choses ; la folie aide à la fuir.\par
Dans toutes sortes d’affaires de cette société, où la \emph{distribution} des biens s’est centralisée de telle manière qu’elle est devenue maîtresse, à la fois d’une façon notoire et d’une façon secrète, de la définition même de ce que pourra être le bien, il arrive que l’on attribue à certaines personnes des qualités, ou des connaissances, ou quelquefois même des vices, parfaitement imaginaires, pour expliquer par de telles causes le développement satisfaisant de certaines entreprises ; et cela à seule fin de cacher, ou du moins de dissimuler autant que possible, la fonction de diverses \emph{ententes qui décident de tout}.\par
Cependant, malgré ses fréquentes intentions, et ses lourds moyens, de mettre en lumière la pleine dimension de nombreuses personnalités supposées remarquables, la société actuelle, et pas seulement par tout ce qui a remplacé aujourd’hui les arts, ou par les discours à ce propos, montre beaucoup plus souvent le contraire : l’incapacité complète se heurte à une autre incapacité comparable ; elles s’affolent, et c’est à qui se mettra en déroute avant l’autre. Il arrive qu’un avocat, oubliant qu’il ne figure dans un procès que pour y être l’homme d’une cause, se laisse sincèrement influencer par un raisonnement de l’avocat adverse ; et même alors que ce raisonnement a pu être tout aussi peu rigoureux que le sien propre. Il arrive aussi qu’un suspect, innocent, avoue momentanément ce crime qu’il n’a pas commis ; pour la seule raison qu’il avait été impressionné \emph{par la logique} de l’hypothèse d’un délateur qui voulait le croire coupable (cas du docteur Archambeau, à Poitiers, en 1984).\par
McLuhan lui-même, le premier apologiste du spectacle, qui paraissait l’imbécile le plus convaincu de son siècle, a changé d’avis en découvrant enfin, en 1976, que « la pression des \emph{mass media} pousse vers l’irrationnel », et qu’il deviendrait urgent d’en modérer l’emploi. Le penseur de Toronto avait auparavant passé plusieurs décennies à s’émerveiller des multiples libertés qu’apportait ce « village planétaire » si instantanément accessible à tous sans fatigue. Les villages, contrairement aux villes, ont toujours été dominés par le conformisme, l’isolement, la surveillance mesquine, l’ennui, les ragots toujours répétés sur quelques mêmes familles. Et c’est bien ainsi que se présente désormais la vulgarité de la planète spectaculaire, où il n’est plus possible de distinguer la dynastie des Grimaldi-Monaco, ou des Bourbons-Franco, de celle qui avait remplacé les Stuart. Pourtant d’ingrats disciples essaient aujourd’hui de faire oublier McLuhan, et de rajeunir ses premières trouvailles, visant à leur tour une carrière dans l’éloge médiatique de toutes ces nouvelles libertés qui seraient à « choisir » aléatoirement dans l’éphémère. Et probablement ils se renieront plus vite que leur inspirateur.\par

\labelblock{XIII}

\noindent Le spectacle ne cache pas que quelques dangers environnent l’ordre merveilleux qu’il a établi. La pollution des océans et la destruction des forêts équatoriales menacent le renouvellement de l’oxygène de la Terre ; sa couche d’ozone résiste mal au progrès industriel ; les radiations d’origine nucléaire s’accumulent irréversiblement. Le spectacle conclut seulement que c’est sans importance. Il ne veut discuter que sur les dates et les doses. Et en ceci seulement, il parvient à rassurer ; ce qu’un esprit pré-spectaculaire aurait tenu pour impossible.\par
Les méthodes de la démocratie spectaculaire sont d’une grande souplesse, contrairement à la simple brutalité du diktat totalitaire. On peut garder le nom quand la chose a été secrètement changée (de la bière, du bœuf, un philosophe). On peut aussi bien changer le nom quand la chose a été secrètement continuée : par exemple en Angleterre l’usine de retraitement des déchets nucléaires de Windscale a été amenée à faire appeler sa localité Sellafield afin de mieux égarer les soupçons, après un désastreux incendie en 1957, mais ce retraitement toponymique n’a pas empêché l’augmentation de la mortalité par cancer et leucémie dans ses alentours. Le gouvernement anglais, on l’apprend démocratiquement trente ans plus tard, avait alors décidé de garder secret un rapport sur la catastrophe qu’il jugeait, et non sans raison, de nature à ébranler la confiance que le public accordait au nucléaire.\par
Les pratiques nucléaires, militaires ou civiles, nécessitent une dose de secret plus forte que partout ailleurs ; où comme on sait il en faut déjà beaucoup. Pour faciliter la vie, c’est-à-dire les mensonges, des savants élus par les maîtres de ce système, on a découvert l’utilité de changer aussi les mesures, de les varier selon un plus grand nombre de points de vue, les raffiner, afin de pouvoir jongler, selon les cas, avec plusieurs de ces chiffres difficilement convertibles. C’est ainsi que l’on peut disposer, pour évaluer la radioactivité, des unités de mesure suivantes : le curie, le becquerel, le röntgen, le rad, alias centigray, le rem, sans oublier le facile millirad et le sivert, qui n’est autre qu’une pièce de 100 rems. Cela évoque le souvenir des subdivisions de la monnaie anglaise, dont les étrangers ne maîtrisaient pas vite la complexité, au temps où Sellafield s’appelait encore Windscale.\par
On conçoit la rigueur et la précision qu’auraient pu atteindre, au \textsc{XIX}\textsuperscript{e} siècle, l’histoire des guerres et, par conséquent, les théoriciens de la stratégie si, afin de ne pas donner d’informations trop confidentielles aux commentateurs neutres ou aux historiens ennemis, on s’en était habituellement tenu à rendre compte d’une campagne en ces termes : « La phase préliminaire comporte une série d’engagements où, de notre côté, une solide avant-garde, constituée par quatre généraux et les unités placées sous leur commandement, se heurte à un corps ennemi comptant 13 000 baïonnettes. Dans la phase ultérieure se développe une bataille rangée, longuement disputée, où s’est portée la totalité de notre armée, avec ses 290 canons et sa cavalerie forte de 18 000 sabres ; tandis que l’adversaire lui a opposé des troupes qui n’alignaient pas moins de 3 600 lieutenants d’infanterie, quarante capitaines de hussard et vingt-quatre de cuirassiers. Après des alternances d’échecs et de succès de part et d’autre, la bataille peut être considérée finalement comme indécise. Nos pertes, plutôt au-dessous du chiffre moyen que l’on constate habituellement dans des combats d’une durée et d’une intensité comparables, sont sensiblement supérieures à celles des Grecs à Marathon, mais restent inférieures à celles des Prussiens à Iéna. » D’après cet exemple, il n’est pas impossible à un spécialiste de se faire une idée vague des forces engagées. Mais la conduite des opérations est assurée de rester au-dessus de tout jugement.\par
En juin 1987, Pierre Bacher, directeur adjoint de l’équipement à l’EDF, a exposé la dernière doctrine de la sécurité des centrales nucléaires. En les dotant de vannes et de filtres, il devient beaucoup plus facile d’éviter les catastrophes majeures, la fissuration ou l’explosion de l’enceinte, qui toucheraient l’ensemble d’une « région ». C’est ce que l’on obtient à trop vouloir confiner. Il vaut mieux, chaque fois que la machine fait mine de s’emballer, décompresser doucement, en arrosant un étroit voisinage de quelques kilomètres, voisinage qui sera chaque fois très différemment et aléatoirement prolongé par le caprice des vents. Il révèle que, dans les deux années précédentes, les discrets essais menés à Cadarache, dans la Drôme, « ont concrètement montré que les rejets – essentiellement des gaz – ne dépassent pas quelques pour mille, au pire un pour cent de la radioactivité régnant dans l’enceinte ». Ce pire reste donc très modéré : un pour cent. Auparavant on était sûrs qu’il n’y avait aucun risque, sauf dans le cas d’accident, logiquement impossible. Les premières années d’expérience ont changé ce raisonnement ainsi : puisque l’accident est toujours possible, ce qu’il faut éviter, c’est qu’il atteigne un seuil catastrophique, et c’est aisé. Il suffit de contaminer coup par coup avec modération. Qui ne sent qu’il est infiniment plus sain de se borner pendant quelques années à boire 140 centilitres de vodka par jour, au lieu de commencer tout de suite à s’enivrer comme des Polonais ?\par
Il est assurément dommage que la société humaine rencontre de si brûlants problèmes au moment où il est devenu matériellement impossible de faire entendre la moindre objection au discours marchand ; au moment où la domination, justement parce qu’elle est abritée par le spectacle de toute réponse à ses décisions et justifications fragmentaires ou délirantes, \emph{croit qu’elle n’a plus besoin de penser} ; et véritablement ne sait plus penser. Aussi ferme que soit le démocrate, ne préférerait-il pas qu’on lui ait choisi des maîtres plus intelligents ?\par
À la conférence internationale d’experts tenue à Genève en décembre 1986, il était tout simplement question d’une interdiction mondiale de la production de chloro-fluorocarbone, le gaz qui fait disparaître depuis peu, mais à très vive allure, la mince couche d’ozone qui protégeait cette planète – on s’en souviendra contre les effets nocifs du rayonnement cosmique. Daniel Verilhe, représentant de la filiale de produits chimiques d’Elf-Aquitaine, et siégeant à ce titre dans une délégation française fermement opposée à cette interdiction, faisait une remarque pleine de sens : « Il faut bien trois ans pour mettre au point d’éventuels substituts et les coûts peuvent être multipliés par quatre. » On sait que cette fugitive couche d’ozone, à une telle altitude, n’appartient à personne, et n’a aucune valeur marchande. Le stratège industriel a donc pu faire mesurer à ses contradicteurs toute leur inexplicable insouciance économique, par ce rappel à la réalité : « Il est très hasardeux de baser une stratégie industrielle sur des impératifs en matière d’environnement. »\par
Ceux qui avaient, il y a déjà bien longtemps, commencé à critiquer l’économie politique en la définissant comme « le reniement achevé de l’homme », ne s’étaient pas trompés. On la reconnaîtra à ce trait.\par

\labelblock{XIV}

\noindent On entend dire que la science est maintenant soumise à des impératifs de rentabilité économique ; cela a toujours été vrai. Ce qui est nouveau, c’est que l’économie en soit venue à faire ouvertement la guerre aux humains ; non plus seulement aux possibilités de leur vie, mais aussi à celles de leur survie. C’est alors que la pensée scientifique a choisi, contre une grande part de son propre passé anti-esclavagiste, de servir la domination spectaculaire. La science possédait, avant d’en venir là, une autonomie relative. Elle savait donc penser sa parcelle de réalité ; et ainsi elle avait pu immensément contribuer à augmenter les moyens de l’économie. Quand l’économie toute-puissante est devenue folle, \emph{et les temps spectaculaires ne sont rien d’autre}, elle a supprimé les dernières traces de l’autonomie scientifique, inséparablement sur le plan méthodologique et sur le plan des conditions pratiques de l’activité des « chercheurs ». On ne demande plus à la science de comprendre le monde, ou d’y améliorer quelque chose. On lui demande de justifier instantanément tout ce qui se fait. Aussi stupide sur ce terrain que sur tous les autres, qu’elle exploite avec la plus ruineuse irréflexion, la domination spectaculaire a fait abattre l’arbre gigantesque de la connaissance scientifique à seule fin de s’y faire tailler une matraque. Pour obéir à cette ultime demande sociale d’une justification manifestement impossible, il vaut mieux ne plus trop savoir penser, mais être au contraire assez bien exercé aux commodités du discours spectaculaire. Et c’est en effet dans cette carrière qu’a lestement trouvé sa plus récente spécialisation, avec beaucoup de bonne volonté, la science prostituée de ces jours méprisables.\par
La science de la justification mensongère était naturellement apparue dès les premiers symptômes de la décadence de la société bourgeoise, avec la prolifération cancéreuse des pseudosciences dites « de l’homme » ; mais par exemple la médecine moderne avait pu, un temps, se faire passer pour utile, et ceux qui avaient vaincu la variole ou la lèpre étaient autres que ceux qui ont bassement capitulé devant les radiations nucléaires ou la chimie agroalimentaire. On remarque vite que la médecine aujourd’hui n’a, bien sûr, plus le droit de défendre la santé de la population contre l’environnemment pathogène, car ce serait s’opposer à l’État, ou seulement à l’industrie pharmaceutique.\par
Mais ce n’est pas seulement par cela qu’elle est obligée de taire, que l’activité scientifique présente avoue ce qu’elle est devenue. C’est aussi par ce que, très souvent, elle a la simplicité de dire. Annonçant en novembre 1985, après une expérimentation on de huit jours sur quatre malades, qu’ils avaient peut-être découvert un remède efficace contre le SIDA, les professeurs Even et Andrieu, de l’hôpital Laënnec, soulevèrent deux jours après, les malades étant morts, quelques reserves de la part de plusieurs médecins, moins a avancés ou peut-être jaloux, pour leur façon assez précipitée de courir faire enregistrer ce qu’ici n’était qu’une trompeuse apparence de victoire ; quelques heures avant l’écroulement. Et ceux-là s’en défendirent sans se troubler, en affirmant qu’« après tout, mieux vaut de faux espoirs que pas d’espoir du tout ». Ils étaient mêmes trop ignorants pour reconnaître que cet : argument, à lui seul, était un complet reniement de l’esprit scientifique ; et qu’il avait historiquement toujours servi à couvrir les profitables rêveries des charlatans et des sorciers, dans les temps où on ne leur confiait pas la direction des hôpitaux.\par
Quand la science officielle en vient à être conduite de la sorte, comme tout le reste du spectacle social qui, sous une présentation matériellement modernisée et enrichie, n’a fait que reprendre les très anciennes techniques des tréteaux forains – \emph{illusionnistes, aboyeurs et barons} –, on ne peut être surpris de voir quelle grande autorité reprennent parallèlement, un peu partout, les mages et les sectes, le zen emballé sous vide ou la théologie des Mormons. L’ignorance, qui a bien servi les puissances établies, a été en surplus toujours exploitée par d’ingénieuses entreprises qui se tenaient en marge des lois. Quel moment plus favorable que celui où l’analphabétisme a tant progressé ? Mais cette réalité est niée à son tour par une autre démonstration de sorcellerie. L’UNESCO, lors de sa fondation, avait adopté une définition scientifique, très précise, de l’analphabétisme qu’elle se donnait pour tâche de combattre dans les pays arriérés. Quand on a vu revenir inopinément le même fait, mais cette fois du côté des pays dits avancés, comme un autre, attendant Grouchy, vit surgir Blûcher dans sa bataille, il a suffi de faire donner la Garde des experts ; et ils ont vite enlevé la formule d’un seul assaut irrésistible, en remplaçant \emph{le terme} analphabétisme par celui d’illettrisme : comme un « faux patriotique » peut paraître opportunément pour soutenir une bonne cause nationale. Et pour fonder sur le roc, entre pédagogues, la pertinence du néologisme, on fait vite passer une nouvelle définition, comme si elle était admise depuis toujours, et selon laquelle, tandis que l’analphabète était, on sait, celui qui n’avait jamais appris à lire, l’illettré au sens moderne est, tout au contraire, celui qui a appris la lecture (et l’a même \emph{mieux apprise} qu’avant, peuvent du coup témoigner froidement les plus doués des théoriciens et historiens officiels de la pédagogie), mais qui l’a par hasard \emph{aussitôt oubliée}. Cette surprenante explication risquerait d’être moins apaisante qu’inquiétante, si elle n’avait l’art d’éviter, en parlant à côté et comme si elle ne la voyait pas, la première conséquence qui serait venue à l’esprit de tous dans des époques plus scientifiques : à savoir que ce dernier phénomène mériterait lui-même d’être expliqué, et combattu, puisqu’il n’avait jamais pu être observé, ni même imaginé, où que ce soit, avant les récents progrès de la pensée avariée ; quand la décadence de l’explication accompagne d’un pas égal la décadence de la pratique.\par

\labelblock{XV}

\noindent Il y a plus de cent ans, le \emph{Nouveau Dictionnaire des Synonymes français} d’A.-L. Sardou définissait les nuances qu’il faut saisir entre : \emph{fallacieux, trompeur, imposteur, séducteur, insidieux, captieux} ; et qui ensemble constituent aujourd’hui une sorte de palette des couleurs qui conviennent à un portrait de la société du spectacle. Il n’appartenait pas à son temps, ni à son expérience de spécialiste, d’exposer aussi clairement les sens voisins, mais très différents, des périls que doit normalement s’attendre à affronter tout groupe qui s’adonne à la subversion, et suivant par exemple cette gradation : \emph{égaré, provoqué, infiltré, manipulé, usurpé, retourné}. Ces nuances considérables ne sont jamais apparues, en tout cas, aux doctrinaires de la « lutte armée ».\par
« \emph{Fallacieux}, du latin \emph{fallaciosus}, habile ou habitué à tromper, plein de fourberie : la terminaison de cet adjectif équivaut au superlatif de \emph{trompeur}. Ce qui trompe ou induit à erreur de quelque manière que ce soit, est \emph{trompeur} : ce qui est fait pour tromper, abuser, jeter dans l’erreur par un dessein formé de tromper avec l’artifice et l’appareil imposant le plus propre pour abuser, est \emph{fallacieux, Trompeur} est un mot générique et vague ; tous les genres de signes et d’apparences incertaines sont \emph{trompeurs} : fallacieux désigne la fausseté, la fourberie, l’imposture étudiée ; des discours, des protestations, des raisonnements sophistiques, sont \emph{fallacieux}. Ce mot a des rapports avec ceux d’\emph{imposteur}, de \emph{séducteur}, d’\emph{insidieux}, de \emph{captieux}, mais sans équivalent. \emph{Imposteur} désigne tous les genres de fausses apparences, ou de trames concertées pour abuser ou pour nuire ; l’hypocrisie, par exemple, la calomnie, etc. \emph{Séducteur} exprime l’action propre de s’emparer de quelqu’un, de l’égarer par des moyens adroits et insinuants. \emph{Insidieux} ne marque que l’action de tendre adroitement des pièges et d’y faire tomber. \emph{Captieux} se borne à l’action subtile de surprendre quelqu’un et de le faire tomber dans l’erreur. \emph{Fallacieux} rassemble la plupart de ces caractères. »\par

\labelblock{XVI}

\noindent Le concept, encore jeune, de \emph{désinformation} a été récemment importé de Russie, avec beaucoup d’autres inventions utiles à la gestion des États modernes. Il est toujours hautement employé par un pouvoir, ou corollairement par des gens qui détiennent un fragment d’autorité économique ou politique, pour maintenir ce qui est établi ; et toujours en attribuant à cet emploi une fonction \emph{contre-offensive}. Ce qui peut s’opposer à une seule vérité officielle doit être forcément une désinformation émanant de puissances hostiles, ou au moins de rivaux, et elle aurait été intentionnellement faussée par la malveillance. La désinformation ne serait pas la simple négation d’un fait qui convient aux autorités, ou la simple affirmation d’un fait qui ne leur convient pas : on appelle cela psychose. Contrairement au pur mensonge, la désinformation, et voilà en quoi le concept est intéressant pour les défenseurs de la société dominante, doit fatalement contenir une certaine part de vérité, mais délibérément manipulée par un habile ennemi. Le pouvoir qui parle de désinformation ne croit pas être lui-même absolument sans défauts, mais il sait qu’il pourra attribuer à toute critique précise cette excessive insignifiance qui est dans la nature de la désinformation ; et que de la sorte il n’aura jamais à convenir d’un défaut particulier.\par
En somme, la désinformation serait le mauvais usage de la vérité. Qui la lance est coupable, et qui la croit, imbécile. Mais qui serait donc l’habile ennemi ? Ici, ce ne peut pas être le terrorisme, qui ne risque de « désinformer » personne, puisqu’il est chargé de représenter ontologiquement \emph{l’erreur} la plus balourde et la moins admissible. Grâce à son étymologie, et aux souvenirs contemporains des affrontements limités qui, vers le milieu du siècle, opposèrent brièvement l’Est et l’Ouest, spectaculaire concentré et spectaculaire diffus, aujourd’hui encore le capitalisme du spectaculaire intégré fait semblant de croire que le capitalisme de bureaucratie totalitaire – présenté même parfois comme la base arrière ou l’inspiration des terroristes – reste son ennemi essentiel, comme aussi bien l’autre dira la même chose du premier ; malgré les preuves innombrables de leur alliance et solidarité profondes. En fait tous les pouvoirs qui sont installés, en dépit de quelques réelles rivalités locales, et sans vouloir le dire jamais, pensent continuellement ce qu’avait su rappeler un jour, du côté de la subversion et sans grand succès sur l’instant, un des rares internationalistes allemands après qu’eût commencé la guerre de 1914 : « L’ennemi principal est dans notre pays. » La désinformation est finalement l’équivalent de ce que représentaient, dans le discours de la guerre sociale du \textsc{XIX}\textsuperscript{e} siècle, « les mauvaises passions ». C’est tout ce qui est obscur et risquerait de vouloir s’opposer à l’extraordinaire bonheur dont cette société, on le sait bien, fait bénéficier ceux qui lui ont fait confiance ; bonheur qui ne saurait être trop payé par différents risques ou déboires insignifiants. Et tous ceux qui \emph{voient} ce bonheur dans le spectacle admettent qu’il n’y a pas à lésiner sur son coût ; tandis que les autres désinforment.\par
L’autre avantage que l’on trouve à dénoncer, en l’expliquant ainsi, une désinformation bien particulière, c’est qu’en conséquence le discours global du spectacle ne saurait être soupçonné d’en contenir, puisqu’il peut désigner, avec la plus scientifique assurance, le terrain où se reconnaît la seule désinformation : c’est tout ce qu’on peut dire et qui ne lui plaira pas.\par
C’est sans doute par erreur – à moins plutôt que ce ne soit un leurre délibéré – qu’a été agité récemment en France le projet d’attribuer officiellement une sorte de label à du médiatique « garanti sans désinformation » : ceci blessait quelques professionnels des \emph{media}, qui voudraient encore croire, ou plus modestement faire croire, qu’ils ne sont pas effectivement censurés dès à présent. Mais surtout le concept de désinformation n’a évidemment pas à être employé \emph{défensivement}, et encore moins dans une défensive statique, en garnissant une Muraille de Chine, une ligne Maginot, qui devrait couvrir absolument un espace censé être interdit à la désinformation. Il faut qu’il y ait de la désinformation, et qu’elle reste fluide, pouvant passer partout. Là où le discours spectaculaire n’est pas attaqué, il serait stupide de le défendre ; et ce concept s’userait extrêmement vite à le défendre, contre l’évidence, sur des points qui doivent au contraire éviter de mobiliser l’attention. De plus, les autorités n’ont aucun besoin réel de garantir qu’une information précise ne contiendrait pas de désinformation. Et elles n’en ont pas les moyens : elles ne sont pas si respectées, et ne feraient qu’attirer la suspicion sur l’information en cause. Le concept de désinformation n’est bon que dans la contre-attaque. Il faut le maintenir en deuxième ligne, puis le jeter instantanément en avant pour repousser toute vérité qui viendrait à surgir.\par
Si parfois une sorte de désinformation désordonnée risque d’apparaître, au service de quelques intérêts particuliers passagèrement en conflit, et d’être crue elle aussi, devenant incontrôlable et s’opposant par là au travail d’ensemble d’une désinformation moins irresponsable, ce n’est pas qu’il y ait lieu de craindre que dans celle-là ne se trouvent engagés d’autres manipulateurs plus experts ou plus subtils : c’est simplement parce que la désinformation se déploie maintenant \emph{dans un monde où il n n’y a Plus de place pour aucune vérification}.\par
Le concept confusionniste de désinformation est mis en vedette pour réfuter instantanément, par le seul bruit de son nom, toute critique que n’auraient pas suffi à faire disparaître les diverses agences de l’organisation du silence. Par exemple, on pourrait dire un jour, si cela paraissait souhaitable, que cet écrit est une entreprise de désinformation sur le spectacle ; ou bien, c’est la même chose, de désinformation au détriment de la démocratie.\par
Contrairement à ce qu’affirme son concept spectaculaire inversé, la pratique de la désinformation ne peut que servir l’Étel ici et maintenant, sous sa conduite directe, ou à l’initiative de ceux qui défendent les mêmes valeurs. En fait, la désinformation réside dans toute l’information existante ; et comme son caractère principal. On ne la nomme que là où il faut maintenir, par l’intimidation, la passivité. Là où la désinformation est \emph{nommée}, elle n’existe pas. Là où elle existe, on ne la nomme pas.\par
Quand il y avait encore des idéologies qui s’affrontaient, qui se proclamaient pour ou contre tel aspect connu de la réalité, il y avait des fanatiques, et des menteurs, mais pas de « désinformateurs ».\par
Quand il n’est plus permis, par le respect du consensus spectaculaire, ou au moins par une volonté de gloriole spectaculaire, de dire vraiment ce à quoi l’on s’oppose, ou aussi bien ce que l’on approuve dans toutes ses conséquences ; mais où l’on rencontre souvent l’obligation de dissimuler un côté que l’on considère, pour quelque raison, comme dangereux dans ce que l’on est censé admettre, alors on pratique la désinformation ; comme par étourderie, ou comme par oubli, ou par \emph{prétendu} faux raisonnement. Et par exemple, sur le terrain de la contestation après 1968, les récupérateurs incapables qui furent appelés « pro-situs » ont été \emph{les premiers désinformateurs}, parce qu’ils dissimulaient autant que possible les manifestations pratiques à travers lesquelles s’était affirmée la critique qu’ils se flattaient d’adopter ; et, point gênés d’en affaiblir l’expression, ils ne citaient jamais rien ni personne, pour avoir l’air d’avoir eux-mêmes trouvé quelque chose.\par

\labelblock{XVII}

\noindent Renversant une formule fameuse de Hegel, je notais déjà en 1967 que « dans le monde \emph{réellement renversé}, le vrai est un moment du faux ». Les années passées depuis lors ont montré les progrès de ce principe dans chaque domaine particulier, sans exception.\par
Ainsi, dans une époque où ne peut plus exister d’art contemporain, il devient difficile de juger des arts classiques. Ici comme ailleurs, l’ignorance n’est produite que pour être exploitée. En même temps que se perdent ensemble le sens de l’histoire et le goût, on organise des réseaux de falsification. Il suffit de tenir les experts et les commissaires-priseurs, et c’est assez facile, pour tout faire passer puisque dans les affaires de cette nature, comme finalement dans les autres, c’est la vente qui authentifie toute valeur. Après, ce sont les collectionneurs ou les musées, notamment américains, qui, gorgés de faux, auront intérêt à en maintenir la bonne réputation, tout comme le Fonds Monétaire International maintient la fiction de la valeur positive des immenses dettes de cent nations.\par
Le faux forme le goût, et soutient le faux, en faisant sciemment disparaître la possibilité de référence à l’authentique. On \emph{refait} même le vrai, dès que c’est possible, pour le faire ressembler au faux. Les Américains, étant les plus riches et les plus modernes, ont été les principales dupes de ce commerce du faux en art. Et ce sont justement les mêmes qui financent les travaux de restauration de Versailles ou de la Chapelle Sixtine. C’est pourquoi les fresques de Michel-Ange devront prendre des couleurs ravivées de bande dessinée, et les meubles authentiques de Versailles acquérir ce vif éclat de la dorure qui les fera ressembler beaucoup au faux mobilier d’époque Louis XIV importé à grands frais au Texas.\par
Le jugement de Feuerbach, sur le fait que son temps préférait « l’image à la chose, la copie à l’original, la représentation à la réalité », a été entièrement confirmé par le siècle du spectacle, et cela dans plusieurs domaines où le \textsc{XIX}\textsuperscript{e} siècle avait voulu rester à l’écart de ce qui était déjà sa nature profonde : la production industrielle capitaliste. C’est ainsi que la bourgeoisie avait beaucoup répandu l’esprit rigoureux du musée, de l’objet original, de la critique historique exacte, du document authentique. Mais aujourd’hui, c’est partout que le factice a tendance à remplacer le vrai. À ce point, c’est très opportunément que la pollution due à la circulation des automobiles oblige à remplacer par des répliques en plastique les chevaux de Marly ou les statues romanes du portail de Saint-Trophime. Tout sera en somme plus beau qu’avant, pour être photographié par des touristes.\par
Le point culminant est sans doute atteint par le risible faux bureaucratique chinois des grandes statues de la vaste \emph{armée industrielle} du Premier Empereur, que tant d’hommes d’État en voyage ont été conviés à admirer \emph{in situ}. Cela prouve donc, puisque l’on a pu se moquer d’eux si cruellement, qu’aucun ne disposait, dans la masse de tous leurs conseillers, d’un seul individu qui connaisse l’histoire de l’art, en Chine ou hors de Chine. On sait que leur instruction a été tout autre : « L’ordinateur de Votre Excellence n’en a pas été informé. » Cette constatation que, pour la première fois, on peut gouverner sans avoir aucune connaissance artistique ni aucun sens de l’authentique ou de l’impossible, pourrait à elle seule suffire à conjecturer que tous ces naïfs jobards de l’économie et de l’administration vont probablement conduire le monde à quelque grande catastrophe ; si leur pratique effective ne l’avait pas déjà montré.\par

\labelblock{XVIII}

\noindent Notre société est bâtie sur le secret, depuis les « sociétés-écrans » qui mettent à l’abri de toute lumière les biens concentrés des possédants jusqu’au « secret-défense » qui couvre aujourd’hui un immense domaine de pleine liberté extrajudiciaire de l’État ; depuis les secrets, souvent effrayants, de la \emph{fabrication pauvre}, qui sont cachés derrière la publicité, jusqu’aux projections des variantes de l’avenir extrapolé, sur lesquelles la domination lit seule le cheminement le plus probable de ce qu’elle affirme n’avoir aucune sorte d’existence, tout en calculant les réponses qu’elle y apportera mystérieusement. On peut faire à ce propos quelques observations.\par
Il y a toujours un plus grand nombre de lieux, dans les grandes villes comme dans quelques espaces réservés de la campagne, qui sont inaccessibles, c’est-à-dire gardés et protégés de tout regard ; qui sont mis hors de portée de la curiosité innocente, et fortement abrités de l’espionnage. Sans être tous proprement militaires, ils sont sur ce modèle placés au-delà de tout risque de contrôle par des passants ou des habitants ; ou même par la police, qui a vu depuis longtemps ses fonctions ramenées aux seules surveillance et répression de la délinquance la plus commune. Et c’est ainsi qu’en Italie, lorsque Aldo Moro était prisonnier de \emph{Potere Due}, il n’a pas été détenu dans un bâtiment plus ou moins introuvable, mais simplement dans un bâtiment impénétrable.\par
Il y a toujours un plus grand nombre d’hommes formés pour agir dans le secret ; instruits et exercés à ne faire que cela. Ce sont des détachements spéciaux d’hommes armés d’archives réservées, c’est-à-dire d’observations et d’analyses secrètes. Et d’autres sont armés de diverses techniques pour l’exploitation et la manipulation de ces affaires secrètes. Enfin, quand il s’agit de leurs branches « Action », ils peuvent également être équipés d’autres capacités de simplification des problèmes étudiés.\par
Tandis que les moyens attribués à ces hommes spécialisés dans la surveillance et l’influence deviennent plus grands, ils rencontrent aussi des circonstances générales qui leur sont chaque année plus favorables. Quand par exemple les nouvelles conditions de la société du spectaculaire intégré ont forcé sa critique à rester réellement clandestine, non parce qu’elle se cache mais \emph{puisqu’elle est cachée} par la pesante mise en scène de la pensée du divertissement, ceux qui sont pourtant chargés de surveiller cette critique, et au besoin de la démentir, peuvent finalement employer contre elle les recours traditionnels dans le milieu de la clandestinité : provocation, infiltrations, et diverses formes d’élimination de la critique authentique au profit d’une fausse qui aura pu être mise en place à cet effet. L’incertitude grandit, à tout propos, quand l’imposture générale du spectacle s’enrichit d’une possibilité de recours à mille impostures particulières. Un crime inexpliqué peut aussi être dit suicide, en prison comme ailleurs ; et la dissolution de la logique permet des enquêtes et des procès qui décollent verticalement dans le déraisonnable, et qui sont fréquemment faussés dès l’origine par d’extravagantes autopsies, que pratiquent de singuliers experts.\par
Depuis longtemps, on s’est habitué partout à voir exécuter sommairement toutes sortes de gens. Les terroristes connus, ou considérés comme tels, sont combattus ouvertement d’une manière terroriste. Le Mossad va tuer au loin Abou jihad, ou les SAS anglais des Irlandais, ou la police parallèle du « GAL » des Basques. Ceux que l’on fait tuer par de supposés terroristes ne sont pas eux-mêmes choisis sans raison ; mais il est généralement impossible d’être assuré de connaître ces raisons. On peut savoir que la gare de Bologne a sauté pour que l’Italie continue d’être bien gouvernée ; et ce que sont les « Escadrons de la mort » au Brésil ; et que la Mafia peut incendier un hôtel aux ÉtatsUnis pour appuyer un \emph{racket}. Mais comment savoir à quoi ont pu servir, au fond, les « tueurs fous du Brabant » ? Il est difficile d’appliquer le principe \emph{Cui prodest} ? dans un monde où tant d’intérêts agissants sont si bien cachés. De sorte que, sous le spectaculaire intégré, on vit et on meurt au point de confluence d’un très grand nombre de mystères.\par
Des rumeurs médiatiques-policières prennent à l’instant, ou au pire après avoir été répétées trois ou quatre fois, le poids indiscuté de preuves historiques séculaires. Selon l’autorité légendaire du spectacle du jour, d’étranges personnages éliminés dans le silence reparaissent comme survivants fictifs, dont le retour pourra toujours être évoqué ou supputé, et \emph{prouvé} par le plus simple on-dit des spécialistes. Ils sont quelque part entre l’Achéron et le Léthé, ces morts qui n’ont pas été régulièrement enterrés par le spectacle, ils sont censés dormir en attendant qu’on veuille les réveiller, tous, le terroriste redescendu des collines et le pirate revenu de la mer ; et le voleur qui n’a plus besoin de voler.\par
L’incertitude est ainsi organisée partout. La protection de la domination procède très souvent par \emph{fausses attaques}, dont le traitement médiatique fera perdre de vue la véritable opération : tel le bizarre coup de force de Tejero et de ses gardes civils aux Cortès en 1981, dont l’échec devait cacher un autre \emph{pronunciamiento} plus moderne, c’est-à-dire masqué, qui a réussi. Également voyant, l’échec d’un sabotage par les services spéciaux français, en 1985, en NouvelleZélande, a été parfois considéré comme un stratagème, peut-être destiné à détourner l’attention des nombreux nouveaux emplois de ces services, en faisant croire à leur caricaturale maladresse dans le choix des objectifs comme dans les modalités de l’exécution. Et plus assurément il a été presque partout estimé que les recherches géologiques d’un gisement pétrolier dans le sous-sol \emph{de la ville de Paris}, qui ont été bruyamment menées à l’automne 1986, n’avaient pas d’autre intention sérieuse que celle de mesurer le point qu’avait pu atteindre la capacité d’hébétude et de soumission des habitants ; en leur montrant une prétendue recherche si parfaitement démentielle sur le plan économique.\par
Le pouvoir est devenu si mystérieux qu’après l’affaire des ventes illégales d’armes à l’Iran par la présidence des États-Unis, on a pu se demander qui commandait vraiment aux États-Unis, la plus forte puissance du monde dit démocratique ?\par
Et donc qui diable peut commander le monde démocratique ? Plus profondément, dans ce monde officiellement si plein de respect pour toutes les nécessités économiques, personne ne sait jamais ce que coûte véritablement n’importe quelle chose produite : en effet, la part la plus importante du coût réel \emph{n’est jamais calculée ; et le reste est tenu secret}.\par

\labelblock{XIX}

\noindent Le général Noriega s’est fait un instant connaître mondialement au début de l’année 1988. Il était dictateur sans titre du Panama, pays sans armée, où il commandait la Garde Nationale. Car le Panama n’est pas vraiment un État souverain : il a été creusé pour son canal, et non l’inverse. Le dollar est sa monnaie, et la véritable armée qui y stationne est pareillement étrangère. Noriega avait donc fait toute sa carrière, ici parfaitement identique à celle de Jaruzelski en Pologne, comme général-policier, au service de l’occupant. Il était importateur de drogue aux États-Unis, car le Panama ne rapporte pas assez, et il exportait en Suisse ses capitaux « panaméens ». Il avait travaillé avec la CIA contre Cuba et, pour avoir la couverture adéquate à ses activités économiques, il avait aussi dénoncé aux autorités américaines, si obsédées par ce problème, un certain nombre de ses rivaux dans l’importation. Son principal conseiller en matière de sécurité, qui donnait de lajalousie à Washington, était le meilleur sur le marché, Michael Harari, ancien officier du Mossad, le service secret d’Israël. Quand les Américains ont voulu se défaire du personnage, parce que certains de leurs tribunaux l’avaient imprudemment condamné, Noriega s’est déclaré prêt à se défendre pendant mille ans, par patriotisme panaméen, à la fois contre son peuple en révolte et contre l’étranger ; il a reçu aussitôt l’approbation publique des dictateurs bureaucratiques plus austères de Cuba et du Nicaragua, au nom de l’anti-impérialisme.\par
Loin d’être une étrangeté étroitement panaméenne, ce général Noriega, qui \emph{vend tout et simule tout} dans un monde qui partout fait de même, était, de part en part, comme sorte d’homme d’une sorte d’État, comme sorte de général, comme capitaliste, parfaitement représentatif du spectaculaire intégré ; et des réussites qu’il autorise dans les directions les plus variées de sa politique intérieure et internationale. C’est un modèle du \emph{prince de notre temps} ; et parmi ceux qui se destinent à venir et à rester au pouvoir où que ce puisse être, les plus capables lui ressemblent beaucoup. Ce n’est pas le Panama qui produit de telles merveilles, c’est cette époque.\par

\labelblock{XX}

\noindent Pour tout service de renseignements, sur ce point en accord avec la juste théorie clausewitzienne de la guerre, un \emph{savoir} doit devenir un \emph{Pouvoir}. De là ce service tire à présent son prestige, son espèce de poésie spéciale. Tandis que l’intelligence a été si absolument chassée du spectacle, qui ne permet pas d’agir et ne dit pas grand-chose de vrai sur l’action des autres, elle semble presque s’être réfugiée parmi ceux qui analysent des réalités, et agissent secrètement sur des réalités. Récemment, des révélations que Margaret Thatcher a tout fait pour étouffer, mais en vain, les authentifiant de la sorte, ont montré qu’en Angleterre ces services avaient déjà été capables d’amener la chute d’un ministère dont ils jugeaient la politique dangereuse. Le mépris général que suscite le spectacle redonne ainsi, pour de nouvelles raisons, une attirance à ce qui a pu être appelé, au temps de Kipling, « le grand jeu ».\par
La « conception policière de l’histoire » était au \textsc{XIX}\textsuperscript{e} siècle une explication réactionnaire, et ridicule, alors que tant de puissants mouvements sociaux agitaient les masses. Les pseudocontestataires d’aujourd’hui savent bien cela, par ouï-dire ou par quelques livres, et croient que cette conclusion est restée vraie pour l’éternité ; ils ne veulent jamais voir la pratique réelle de leur temps ; parce qu’elle est trop triste pour leurs froides espérances. L’État ne l’ignore pas, et en joue.\par
Au moment où presque tous les aspects de la vie politique internationale, et un nombre grandissant de ceux qui comptent dans la politique intérieure, sont conduits et montrés dans le style des services secrets, avec leurres, désinformation, double explication – celle qui \emph{peut} en cacher une autre, ou seulement en avoir l’air le spectacle se borne à faire connaître le monde fatigant de l’incompréhensible obligatoire, une ennuyeuse série de romans policiers privés de vie et où toujours manque la conclusion. C’est là que la mise en scène réaliste d’un combat de nègres, la nuit, dans un tunnel, doit passer pour un ressort dramatique suffisant.\par
L’imbécillité croit que tout est clair, quand la télévision a montré une belle image, et l’a commentée d’un hardi mensonge. La demi-élite se contente de savoir que presque tout est obscur, ambivalent, « monté » en fonction de codes inconnus. Une élite plus fermée voudrait savoir le vrai, très malaisé à distinguer clairement dans chaque cas singulier, malgré toutes les données réservées et les confidences dont elle peut disposer. C’est pourquoi elle aimerait connaître la méthode de la vérité, quoique chez elle cet amour reste généralement malheureux.\par

\labelblock{XXI}

\noindent Le secret domine ce monde, et d’abord comme secret de la domination. Selon le spectacle, le secret ne serait qu’une nécessaire exception à la règle de l’information abondamment offerte sur toute la surface de la société, de même que la domination, dans ce « monde libre » du spectaculaire intégré, se serait réduite à n’être qu’un Département exécutif au service de la démocratie. Mais personne ne croit vraiment le spectacle. Comment les spectateurs acceptent-ils l’existence du secret qui, à lui seul, garantit qu’ils ne pourraient gérer un monde dont ils ignorent les principales réalités, si par extraordinaire on leur demandait vraiment leur avis sur la manière de s’y prendre ? C’est un fait que le secret n’apparaît à presque personne dans sa pureté inaccessible, et dans sa généralité fonctionnelle. Tous admettent qu’il y ait inévitablement une petite zone de secret réservée à des spécialistes ; et pour la généralité des choses, beaucoup croient \emph{être dans le secret}.\par
La Boétie a montré, dans le \emph{Discours sur la servitude volontaire}, comment le pouvoir d’un tyran doit rencontrer de nombreux appuis parmi les cercles concentriques des individus qui y trouvent, ou croient y trouver, leur avantage. Et de même beaucoup de gens, parmi les politiques ou médiatiques qui sont flattés qu’on ne puisse les soupçonner d’être des \emph{irresponsables} connaissent beaucoup de choses par relations et par confidences. Celui qui est content d’être dans la confidence n’est guère porté à la critiquer : ni donc à remarquer que, dans toutes les confidences, la part principale de réalité lui sera toujours cachée. Il connaît, par la bienveillante protection des tricheurs, un peu plus de cartes, mais qui peuvent être fausses ; et jamais la méthode qui dirige et explique le jeu. Il s’identifie donc tout de suite aux manipulateurs, et méprise l’ignorance qu’au fond il partage. Car les bribes d’information que l’on offre à ces familiers de la tyrannie mensongère sont normalement infectées de mensonge, incontrôlables, manipulées. Elles font plaisir pourtant à ceux qui y accèdent, car ils se sentent supérieurs à tous ceux qui ne savent rien. Elles ne valent du reste que pour faire mieux approuver la domination, et jamais pour la comprendre effectivement. Elles constituent le privilège des \emph{spectateurs de première classe} : ceux qui ont la sottise de croire qu’ils peuvent comprendre quelque chose, non en se servant de ce qu’on leur cache, mais \emph{en croyant ce qu’on leur révèle} !\par
La domination est lucide au moins en ceci qu’elle attend de sa propre gestion, libre et sans entraves, un assez grand nombre de catastrophes de première grandeur pour très bientôt ; et cela tant sur les terrains écologiques, chimique par exemple, que sur les terrains économiques, bancaire par exemple. Elle s’est mise, depuis quelque temps déjà, en situation de traiter ces malheurs exceptionnels autrement que par le maniement habituel de la douce désinformation.\par

\labelblock{XXII}

\noindent Quant aux assassinats, en nombre croissant depuis plus de deux décennies, qui sont restés entièrement inexpliqués – car si l’on a parfois sacrifié quelque comparse, jamais il n’a été question de remonter aux commanditaires – leur caractère de production en série a sa marque : les mensonges patents, et changeants, des déclarations officielles ; Kennedy, Aldo Moro, Olaf Palme, des ministres ou financiers, un ou deux papes, d’autres qui valaient mieux qu’eux. Ce syndrome d’une maladie sociale récemment acquise s’est vite répandu un peu partout, comme si, à partir des premiers cas observés, il \emph{descendait} des sommets des États, sphère traditionnelle de ce genre d’attentats, et comme si, en même temps, il \emph{remontait} des bas-fonds, autre lieu traditionnel des trafics illégaux et protections, où s’est toujours déroulé ce genre de guerre, entre professionnels. Ces pratiques tendent à se rencontrer au \emph{milieu} de toutes les affaires de la société, comme si en effet l’État ne dédaignait pas de s’y mêler, et la Mafia parvenait à s’y élever ; une sorte de jonction s’opérant par là.\par
On a tout entendu dire pour tenter d’expliquer accidentellement ce nouveau genre de mystères : incompétence des polices, sottise des juges d’instruction, inopportunes révélations de la presse, crise de croissance des services secrets, malveillance des témoins, grève catégorielle des délateurs. Edgar Poe pourtant avait déjà trouvé la direction certaine de la vérité, par son célèbre raisonnement du \emph{Double assassinat dans la rue Morgue} :\par
« Il me semble que le mystère est considéré comme insoluble, par la raison même qui devrait le faire regarder comme facile à résoudre – je veux parler du caractère excessif sous lequel il apparaît… Dans des investigations du genre de celle qui nous occupe, il ne faut pas tant se demander comment les choses se sont passées, qu’étudier en quoi elles se distinguent de tout ce qui est arrivé jusqu’à présent. »\par

\labelblock{XXIII}

\noindent En janvier 1988, la Mafia colombienne de la drogue publiait un communiqué destiné à rectifier l’opinion du public sur sa prétendue existence. La plus grande exigence d’une Mafia, où qu’elle puisse être constituée, est naturellement d’établir qu’elle n’existe pas, ou qu’elle a été victime de calomnies peu scientifiques ; et c’est son premier point de ressemblance avec le capitalisme. Mais en la circonstance, cette Mafia irritée d’être seule mise en vedette, est allée jusqu’à évoquer les autres groupements qui voudraient se faire oublier en la prenant abusivement comme bouc émissaire. Elle déclare : « Nous n’appartenons pas, nous, à la mafia bureaucratique et politicienne, ni à celle des banquiers et des financiers, ni à celle des millionnaires, ni à la mafia des grands contrats frauduleux, à celle des monopoles ou à celle du pétrole, ni à celle des grands moyens de communication. »\par
On peut sans doute estimer que les auteurs de cette déclaration ont intérêt à déverser, tout comme les autres, leurs propres pratiques dans le vaste fleuve des eaux troubles de la criminalité, et des illégalités plus banales, qui arrose dans toute son étendue la société actuelle ; mais aussi il est juste de convenir que voilà des gens qui savent mieux que d’autres, par profession, de quoi ils parlent. La Mafia vient partout au mieux sur le sol de la société moderne. Elle est en croissance aussi rapide que les autres produits du travail par lequel la société du spectaculaire intégré façonne son monde. La Mafia grandit avec les immenses progrès des ordinateurs et de l’alimentation industrielle, de la complète reconstruction urbaine et du bidonville, des services spéciaux et de l’analphabétisme.\par

\labelblock{XXIV}

\noindent La Mafia n’était qu’un archaïsme transplanté, quand elle commençait à se manifester au début du siècle aux États-Unis, avec l’immigration de travailleurs siciliens ; comme au même instant apparaissaient sur la côte ouest des guerres de gangs entre les sociétés secrètes chinoises. Fondée sur l’obscurantisme et la misère, la Mafia ne pouvait alors même pas s’implanter dans l’Italie du Nord. Elle semblait condamnée à s’effacer partout devant l’État moderne. C’était une forme de crime organisé qui ne pouvait prospérer que sur la « protection » de minorités attardées, en dehors du monde des villes, là où ne pouvait pas pénétrer le contrôle d’une police rationnelle et des lois de la bourgeoisie. La tactique défensive de la Mafia ne pouvait jamais être que la suppression des témoignages, pour neutraliser la police et la justice, et faire régner dans sa sphère d’activité le secret qui lui est nécessaire. Elle a par la suite trouvé un champ nouveau dans le \emph{nouvel obscurantisme} de la société du spectaculaire diffus, puis intégré : avec la victoire totale du secret, la démission générale des citoyens, la perte complète de la logique, et les progrès de la vénalité et de la lâcheté universelles, toutes les conditions favorables furent réunies pour qu’elle devînt une puissance moderne, et offensive.\par
La Prohibition américaine – grand exemple des prétentions des États du siècle au contrôle autoritaire de tout, et des résultats qui en découlent – a laissé au crime organisé, pendant plus d’une décennie, la gestion du commerce de l’alcool. La Mafia, à partir de là enrichie et exercée, s’est liée à la politique électorale, aux affaires, au développement du marché des tueurs professionnels, à certains détails de la politique internationale. Ainsi, elle fut favorisée par le gouvernement de Washington pendant la Deuxième Guerre mondiale, pour aider à l’invasion de la Sicile. L’alcool redevenu légal a été remplacé par les stupéfiants, qui ont alors constitué la marchandise-vedette des consommations illégales. Puis elle a pris une importance considérable dans l’immobilier, les banques, la grande politique et les grandes affaires de l’État, puis les industries du spectacle : télévision, cinéma, édition. C’est aussi vrai déjà, aux États-Unis en tout cas, pour l’industrie même du disque, comme partout où la publicité d’un produit dépend d’un nombre assez concentré de gens. On peut donc facilement faire pression sur eux, en les achetant ou en les intimidant, puisque l’on dispose évidemment de bien assez de capitaux, ou d’hommes de main qui ne peuvent être reconnus ni punis. En corrompant les \emph{disc-jokeys}, on décide donc de ce qui devra être le succès, parmi des marchandises si également misérables.\par
C’est sans doute en Italie que la Mafia, au retour de ses expériences et conquêtes américaines, a acquis la plus grande force : depuis l’époque de son compromis historique avec le gouvernement parallèle, elle s’est trouvée en situation de faire tuer des juges d’instruction ou des chefs de police : pratique qu’elle avait pu inaugurer dans sa participation aux montages du « terrorisme » politique. Dans des conditions relativement indépendantes, l’évolution similaire de l’équivalent japonais de la Mafia prouve bien l’unité de l’époque.\par
On se trompe chaque fois que l’on veut expliquer quelque chose en opposant la Mafia à l’État : ils ne sont jamais en rivalité. La théorie vérifie avec facilité ce que toutes les rumeurs de la vie pratique avaient trop facilement montré.\par
La Mafia n’est pas étrangère dans ce monde ; elle y est parfaitement chez elle. Au moment du spectaculaire intégré, elle règne en fait comme le \emph{modèle} de toutes les entreprises commerciales avancées.\par

\labelblock{XXV}

\noindent Avec les nouvelles conditions qui prédominent actuellement dans la société écrasée sous le \emph{talon de fer} du spectacle, on sait que, par exemple, un assassinat politique se trouve placé dans une autre lumière ; en quelque sorte tamisée. Il y a partout beaucoup plus de fous qu’autrefois, mais ce qui est infiniment plus commode, c’est que l’on peut en parler \emph{follement}. Et ce n’est pas une quelconque terreur régnante qui imposerait de telles explications médiatiques. Au contraire, c’est l’existence paisible de telles explications qui doit causer de la terreur.\par
Quand en 1914, la guerre étant imminente, Villain assassina jaurès, personne n’a douté que Villain, individu sans doute assez peu équilibré, avait cru devoir tuer Jaurès parce que celui-ci paraissait, aux yeux d’extrémistes de la droite patriotique qui avaient profondément influencé Villain, quelqu’un qui serait certainement nuisible pour la défense du pays. Ces extrémistes avaient seulement sous-estimé l’immense force du consentement patriotique dans le parti socialiste, qui devait le pousser instantanément à « l’union sacrée » ; que Jaurès fût assassiné ou qu’au contraire on lui laissât l’occasion de tenir ferme sur sa position internationaliste en refusant la guerre. Aujourd’hui, en présence d’un tel événement, des journalistes-policiers, experts notoires en « faits de société » et en « terrorisme », diraient tout de suite que Villain était bien connu pour avoir à plusieurs reprises esquissé des tentatives de meurtre, la pulsion visant chaque fois des hommes, qui pouvaient professer des opinions politiques très diverses, mais qui tous avaient par hasard une ressemblance physique ou vestimentaire avec Jaurès. Des psychiatres l’attesteraient, et les media, rien qu’en attestant qu’ils l’ont dit, attesteraient par le fait même leur compétence et leur impartialité d’experts incomparablement autorisés. Puis l’enquête policière officielle pourrait établir dès le lendemain que l’on vient de découvrir plusieurs personnes honorables qui sont prêtes à témoigner du fait que ce même Villain, s’estimant un jour mal servi à la « Chope du Croissant », avait, en leur présence, abondamment menacé de se venger prochainement du cafétier, en abattant devant tout le monde, et sur place, un de ses meilleurs clients.\par
Ce n’est pas dire que, dans le passé, la vérité s’imposait souvent et tout de suite ; puisque Villain a été finalement acquitté par la justice française. Il n’a été fusillé qu’en 1936, quand éclata la révolution espagnole, car il avait commis l’imprudence de résider aux îles Baléares.\par

\labelblock{XXVI}

\noindent C’est parce que les nouvelles conditions d’un maniement profitable des affaires économiques, au moment où l’État détient une part hégémonique dans l’orientation de la production et où la demande pour toutes les marchandises dépend étroitement de la centralisation réalisée dans l’information-incitation spectaculaire, à laquelle devront aussi s’adapter les formes de la distribution, l’exigent impérativement que l’on voit se constituer partout des réseaux d’influence ou des sociétés secrètes. Ce n’est donc qu’un produit naturel du mouvement de concentration des capitaux, de la production de la distribution. Ce qui, en cette matière, ne s’étend pas, doit disparaître ; et aucune entreprise ne peut s’étendre qu’avec les valeurs, les techniques, les moyens, de ce que sont aujourd’hui l’industrie, le spectacle, l’État. C’est, en dernière analyse, le développement particulier qui a été choisi par l’économie de notre époque, qui en vient à imposer partout \emph{la formation de nouveaux liens personnels de dépendance et de protection.}\par
C’est justement en ce point que réside la profonde vérité de cette formule, si bien comprise dans l’Italie entière, qu’emploie la Mafia sicilienne : « Quand on a de l’argent et des amis, on se rit de la justice. » Dans le spectaculaire intégré, \emph{les lois dorment} ; parce qu’elles n’avaient pas été faites pour les nouvelles techniques de production, et parce qu’elles sont tournées dans la distribution par des ententes d’un type nouveau. Ce que pense, ou ce que préfère, le public, n’a plus d’importance. Voilà ce qui est caché par le spectacle de tant de sondages d’opinions, d’élections, de restructurations modernisantes. Quels que soient les gagnants, \emph{le moins bon sera enlevé} par l’aimable clientèle : puisque ce sera exactement ce qui aura été produit pour elle.\par
On ne parle à tout instant d’« État de droit » que depuis le moment où l’État moderne dit démocratique a généralement cessé d’en être un : ce n’est point par hasard que l’expression n’a été popularisée que peu après 1970, et d’abord justement en Italie. En plusieurs domaines, on fait même des lois précisément \emph{afin qu’elles soient tournées}, par ceux-là qui justement en auront tous les moyens. L’illégalité en certaines circonstances, par exemple autour du commerce mondial de toutes sortes d’armements, et plus souvent concernant des produits de la plus haute technologie, n’est qu’une sorte de force d’appoint de l’opération économique ; qui s’en trouvera d’autant plus rentable. Aujourd’hui, beaucoup d’affaires sont nécessairement \emph{malhonnêtes comme le siècle}, et non comme l’étaient autrefois celles que pratiquaient, par séries clairement délimitées, des gens qui avaient choisi les voies de la malhonnêteté.\par
À mesure que croissent les réseaux de promotion-contrôle pour jalonner et tenir des secteurs exploitables du marché, s’accroît aussi le nombre de services personnels qui ne peuvent être refusés à ceux qui sont au courant, et qui n’ont pas davantage refusé leur aide ; et ce ne sont pas toujours des policiers ou des gardiens des intérêts ou de la sécurité de l’Etat. Les complicités fonctionnelles communiquent au loin, et très longtemps, car leurs réseaux disposent de tous les moyens d’imposer ces sentiments de reconnaissance ou de fidélité qui, malheureusement, ont toujours été si rares dans l’activité libre des temps bourgeois.\par
On apprend toujours quelque chose de son adversaire. Il faut croire que les gens de l’État ont été amenés, eux aussi, à lire les remarques du jeune Lukàcs sur les concepts de légalité et d’illégalité ; au moment où ils ont eu à traiter le passage éphémère d’une nouvelle génération du négatif – Homère a dit qu’« une génération d’hommes passe aussi vite qu’une génération de feuilles ». Les gens de l’État, dès lors, ont pu cesser comme nous de s’embarrasser de n’importe quelle sorte d’idéologie sur cette question ; et il est vrai que les pratiques de la société spectaculaire ne favorisaient plus du tout des illusions idéologiques de ce genre. À propos de nous tous finalement, on pourra conclure que ce qui nous a empêché souvent de nous enfermer dans une seule activité illégale, c’est que nous en avons eu plusieurs.\par

\labelblock{XXVII}

\noindent Thucydide, au livre VIII, chapitre 66, de \emph{La Guerre du Péloponnèse} dit, à propos des opérations d’une autre conspiration oligarchique, quelque chose qui a beaucoup de parenté avec la situation où nous nous trouvons :\par
« Qui plus est, ceux qui y prenaient la parole étaient du complot et les discours qu’ils prononçaient avaient été soumis au préalable à l’examen de leurs amis. Aucune opposition ne se manifestait parmi le reste des citoyens, qu’effrayait le nombre des conjurés. Lorsque quelqu’un essayait malgré tout de les contredire, on trouvait aussitôt un moyen commode de le faire mourir. Les meurtriers n’étaient pas recherchés et aucune poursuite n’était engagée contre ceux qu’on soupçonnait. Le peuple ne réagissait pas et les gens étaient tellement terrorisés qu’ils s’estimaient heureux, même en restant muets, d’échapper aux violences. Croyant les conjurés bien plus nombreux qu’ils n’étaient, ils avaient le sentiment d’une impuissance complète. La ville était trop grande et ils ne se connaissaient pas assez les uns les autres, pour qu’il leur fût possible de découvrir ce qu’il en était vraiment. Dans ces conditions, si indigné qu’on fût, on ne pouvait confier ses griefs à personne. On devait donc renoncer à engager une action contre les coupables, car il eût fallu pour cela s’adresser soit à un inconnu, soit à une personne de connaissance en qui on n’avait pas confiance. Dans le parti démocratique, les relations personnelles étaient partout empreintes de méfiance et l’on se demandait toujours si celui auquel on avait affaire n’était pas de connivence avec les conjurés. Il y avait en effet parmi ces derniers des hommes dont on n’aurait jamais cru qu’ils se rallieraient à l’oligarchie. »\par
Si l’histoire doit nous revenir après cette éclipse, ce qui dépend de facteurs encore en lutte et donc d’un aboutissement que nul ne saurait exclure avec certitude, ces \emph{Commentaires} pourront servir à écrire un jour l’histoire du spectacle ; sans doute le plus important événement qui se soit produit dans ce siècle ; et aussi celui que l’on s’est le moins aventuré à expliquer. En des circonstances différentes, je crois que j’aurais pu me considérer comme grandement satisfait de mon premier travail sur ce sujet, et laisser à d’autres le soin de regarder la suite. Mais, dans le moment où nous sommes, il m’a semblé que personne d’autre ne le ferait.\par

\labelblock{XXVIII}

\noindent Des réseaux de promotion-contrôle, on glisse insensiblement aux réseaux de surveillance-désinformation. Autrefois, on ne conspirait jamais que contre un ordre établi. Aujourd’hui, \emph{conspirer en sa faveur} est un nouveau métier en grand développement. Sous la domination spectaculaire, on conspire pour la maintenir, et pour assurer ce qu’elle seule pourra appeler sa bonne marche. Cette conspiration fait partie de son fonctionnement même.\par
On a déjà commencé à mettre en place quelques moyens d’une sorte de guerre civile préventive, adaptés à différentes projections de l’avenir calculé. Ce sont des « organisations spécifiques », chargées d’intervenir sur quelques points selon les besoins du spectaculaire intégré. On a ainsi prévu, pour la pire des éventualités, une tactique dite par plaisanterie « des Trois Cultures », en évocation d’une place de Mexico à l’été de 1968, mais cette fois sans prendre de gants, et qui du reste devrait être appliquée avant le jour de la révolte. Et en dehors de cas si extrêmes, il n’est pas nécessaire, pour être un bon moyen de gouvernement, que l’assassinat inexpliqué touche beaucoup de monde ou revienne assez fréquemment : le seul fait que l’on sache que sa possibilité existe, complique tout de suite les calculs en un très grand, nombre de domaines. Il n’a pas non plus besoin d’être intelligemment sélectif, \emph{ad hominem}. L’emploi du procédé d’une manière purement aléatoire serait peut-être plus productif.\par
On s’est mis aussi en situation de faire composer des fragments d’une critique sociale \emph{d’élevage}, qui ne sera plus confiée à des universitaires ou des médiatiques, qu’il vaut mieux désormais tenir éloignés des menteries trop traditionnelles en ce débat ; mais critique meilleure, lancée et exploitée d’une façon nouvelle, maniée par une autre espèce de professionnels, mieux formés. Il commence à paraître, d’une manière assez confidentielle, des textes lucides, anonymes ou signés par des inconnus – tactique d’ailleurs facilitée par la concentration des connaissances de tous sur les bouffons du spectacle ; laquelle a fait que les gens inconnus paraissent justement les plus estimables, non seulement sur des sujets qui ne sont jamais abordés dans le spectacle, mais encore avec des arguments dont lajustesse est rendue plus frappante par l’espèce d’originalité, calculable, qui leur vient du fait de n’être en somme \emph{jamais employés, quoiqu’ils soient assez évidents}. Cette pratique peut servir au moins de premier degré d’initiation pour recruter des esprits un peu éveillés, à qui l’on dira plus tard, s’ils semblent convenables, une plus grande dose de la suite possible. Et ce qui sera, pour certains, le premier pas d’une carrière, sera pour d’autres – moins bien classés – le premier degré du piège dans lequel on les prendra.\par
Dans certains cas, il s’agit de créer, sur des questions qui risqueraient de devenir brûlantes, une autre pseudo-opinion critique ; et entre les deux opinions qui surgiraient ainsi, l’une et l’autre étrangères aux miséreuses conventions spectaculaires, le jugement ingénu pourra indéfiniment osciller, et la discussion pour les peser sera relancée chaque fois qu’il conviendra. Plus souvent, il s’agit d’un discours général sur ce qui est médiatiquement caché, et ce discours pourra être fort critique, et sur quelques points manifestement intelligent, mais en restant curieusement décentré. Les thèmes et les mots ont été sélectionnés facticement, à l’aide d’ordinateurs informés en pensée critique. Il y a dans ces textes quelques absences, assez peu visibles, mais tout de même remarquables : le point de fuite de la perspective y est toujours anormalement absent. Ils ressemblent au \emph{fac-simile} d’une arme célèbre, où manque seulement le percuteur. C’est nécessairement une \emph{critique latérale}, qui voit plusieurs choses avec beaucoup de franchise et de justesse, mais en se plaçant de côté. Ceci non parce qu’elle affecterait une quelconque impartialité, car il lui faut au contraire avoir l’air de blâmer beaucoup, mais sans jamais sembler ressentir le besoin de laisser paraître quelle est \emph{sa cause} ; donc de dire, même implicitement, d’où elle vient et vers quoi elle voudrait aller.\par
À cette sorte de fausse critique contre-journalistique, peut se joindre la pratique organisée de la \emph{rumeur}, dont on sait qu’elle est originairement une sorte de rançon sauvage de l’information spectaculaire, puisque tout le monde ressent au moins vaguement un caractère trompeur dans celle-ci, et donc le peu de confiance qu’elle mérite. La rumeur a été à l’origine superstitieuse, naïve, auto-intoxiquée. Mais, plus récemment, la surveillance a commencé à mettre en place dans la population des gens susceptibles de lancer, au premier signal, les rumeurs qui pourront lui convenir. Ici, on s’est décidé à appliquer dans la pratique les observations d’une théorie formulée il y a près de trente ans, et dont l’origine se trouvait dans la sociologie américaine de la publicité : la théorie des individus qu’on a pu appeler des « locomotives », c’est-à-dire que d’autres dans leur entourage vont être portés à suivre et imiter ; mais en passant cette fois du spontané à l’exercé. On a aussi dégagé à présent les moyens budgétaires, ou extra-budgétaires, d’entretenir beaucoup de supplétifs ; à côté des précédents spécialistes, universitaires et médiatiques, sociologues ou policiers, du passé récent. Croire que s’appliquent encore mécaniquement quelques modèles connus dans le passé, est aussi égarant que l’ignorance générale du passé. « Rome n’est plus dans Rome », et la Mafia n’est plus la pègre. Et les services de surveillance et désinformation ressemblent aussi peu au travail des policiers et indicateurs d’autrefois – par exemple aux roussins et mouchards du second Empire – que les services spéciaux actuels, dans tous les pays, ressemblent peu aux activités des officiers du Deuxième Bureau de l’état-major de l’Armée en 1914.\par
Depuis que l’art est mort, on sait qu’il est devenu extrêmement facile de déguiser des policiers en artistes. Quand les dernières imitations d’un néo-dadaïsme retourné sont autorisées à pontifier glorieusement dans le médiatique, et donc aussi bien à modifier un peu le décor des palais officiels, comme les fous des rois de la pacotille, on voit que d’un même mouvement une couverture culturelle se trouve garantie à tous les agents ou supplétifs des réseaux d’influence de l’État. On ouvre des pseudo-musées vides, ou des pseudo-centres de recherche sur l’œuvre complète d’un personnage inexistant, aussi vite que l’on fait la réputation de journalistes-policiers, ou d’historiens-policiers, ou de romanciers-policiers. Arthur Cravan voyait sans doute venir ce monde quand il écrivait dans \emph{Maintenant} : « Dans la rue on ne verra bientôt plus que des artistes, et on aura toutes les peines du monde à y découvrir un homme. » Tel est bien le sens de cette forme ra eunie d’une ancienne boutade des voyous de 9 Paris : « Salut, les artistes ! Tant pis si je me trompe. »\par
Les choses en étant arrivées à être ce qu’elles sont, on peut voir quelques auteurs collectifs employés par l’édition la plus moderne, c’est-à-dire celle qui s’est donné la meilleure diffusion commerciale. L’authenticité de leurs pseudonymes n’étant assurée que par les journaux, ils se les repassent, collaborent, se remplacent, engagent de nouveaux cerveaux artificiels. Ils se sont chargés d’exprimer le style de vie et de pensée de l’époque, non en vertu de leur personnalité, mais sur ordres. Ceux qui croient qu’ils sont véritablement des entrepreneurs littéraires individuels, indépendants, peuvent donc en arriver à assurer savamment que, maintenant, Ducasse s’est fâché avec le comte de Lautréamont ; que Dumas n’est pas Macquet, et qu’il ne faut surtout pas confondre Erckmann avec Chatrian ; que Censier et Daubenton ne se parlent plus. Il serait mieux de dire que ce genre d’auteurs modernes a voulu suivre Rimbaud, au moins en ceci que « je est un autre ».\par
Les services secrets étaient appelés par toute l’histoire de la société spectaculaire à y jouer le rôle de plaque tournante centrale ; car en eux se concentrent au plus fort degré les caractéristiques et les moyens d’exécution d’une semblable société. Ils sont aussi toujours davantage chargés d’arbitrer les intérêts généraux de cette société, quoique sous leur modeste titre de « services ». Il ne s’agit pas d’abus, puisqu’ils expriment fidèlement les mœurs ordinaires du siècle du spectacle. Et c’est ainsi que surveillants et surveillés fuient sur un océan sans bords. Le spectacle a fait triompher le secret, et il devra être toujours plus dans les mains des spécialistes du secret qui, bien entendu, ne sont pas tous des fonctionnaires en venant à s’autonomiser, à différents degrés, du contrôle de l’État ; qui ne sont pas tous des fonctionnaires.\par

\labelblock{XXIX}

\noindent Une loi générale du fonctionnement du spectaculaire intégré, tout au moins pour ceux qui en gèrent la conduite, c’est que, dans ce cadre, \emph{tout ce que l’on Peut faire doit être fait}. C’est dire que tout nouvel instrument doit être employé, quoi qu’il en coûte. L’outillage nouveau devient partout le but et le moteur du système ; et sera seul à pouvoir modifier notablement sa marche, chaque fois que son emploi s’est imposé sans autre réflexion. Les propriétaires de la société, en effet, veulent avant tout maintenir un certain « rapport social entre des personnes », mais il leur faut aussi y poursuivre le renouvellement technologique incessant ; car telle a été une des obligations qu’ils ont acceptées avec leur héritage. Cette loi s’applique donc également aux services qui protègent la domination. L’instrument que l’on a mis au point doit être employé, et son emploi renforcera les conditions mêmes qui favorisaient cet emploi. C’est ainsi que les procédés d’urgence deviennent procédures de toujours.\par
La cohérence de la société du spectacle a, d’une certaine manière, donné raison aux révolutionnaires, puisqu’il est devenu clair que l’on ne peut y réformer le plus pauvre détail sans défaire l’ensemble. Mais, en même temps, cette cohérence a supprimé toute tendance révolutionnaire organisée en supprimant les terrains sociaux où elle avait pu plus ou moins bien s’exprimer : du syndicalisme aux journaux, de la ville aux livres. D’un même mouvement, on a pu mettre en lumière l’incompétence et l’irréflexion dont cette tendance était tout naturellement porteuse. Et sur le plan individuel, la cohérence qui règne est fort capable d’éliminer, ou d’acheter, certaines exceptions éventuelles.\par

\labelblock{XXX}

\noindent La surveillance pourrait être beaucoup plus dangereuse si elle n’avait été poussée, sur le chemin du contrôle absolu de tous, jusqu’à un point où elle rencontre des difficultés venues de ses propres progrès. Il y a contradiction entre la masse des informations relevées sur un nombre croissant d’individus, et le temps et l’intelligence disponibles pour les analyser ; ou tout simplement leur intérêt possible. L’abondance de la matière oblige à la résumer à chaque étage : beaucoup en disparaît, et le restant est encore trop long pour être lu. La conduite de la surveillance et de la manipulation n’est pas unifiée. Partout en effet, on lutte pour le partage des profits ; et donc aussi pour le développement prioritaire de telle ou telle virtualité de la société existante, au détriment de toutes ses autres virtualités qui cependant, et pourvu qu’elles soient de la même farine, sont tenues pour également respectables.\par
\emph{On lutte aussi par jeu}. Chaque officier traitant est porté à survaloriser ses agents, et aussi les adversaires dont il s’occupe. Chaque pays, sans faire mention des nombreuses alliances supra-nationales, possède à présent un nombre indéterminé de services de police ou contre-espionnage, et de services secrets, étatiques ou para-étatiques. Il existe aussi beaucoup de compagnies privées qui s’occupent de surveillance, protection, renseignement. Les grandes firmes multinationales ont naturellement leurs propres services ; mais également des entreprises nationalisées, même de dimension modeste, qui n’en mènent pas moins leur politique indépendante, sur le plan national et quelquefois international. On peut voir un groupement industriel nucléaire s’opposer à un groupement pétrolier, bien qu’ils soient l’un et l’autre la propriété du même État et, ce qui est plus, qu’ils soient dialectiquement unis l’un à l’autre par leur attachement à maintenir élevé le cours du pétrole sur le marché mondial. Chaque service de sécurité d’une industrie particulière combat le sabotage chez lui, et au besoin l’organise chez le rival : qui place de grands intérêts dans un tunnel sous-marin est favorable à l’insécurité des ferry-boats et peut soudoyer des journaux en difficulté pour la leur faire sentir à la première occasion, et sans trop longue réflexion ; et qui concurrence Sandoz est indifférent aux nappes phréatiques de la vallée du Rhin. On surveille secrètement ce qui est secret. De sorte que chacun de ces organismes, confédérés avec beaucoup de souplesse autour de ceux qui sont en charge de la \emph{raison d’État}, aspire pour son propre compte à une espèce d’hégémonie privée de sens. Car le sens s’est perdu avec le centre connaissable.\par
La société moderne qui, jusqu’en 1968, allait de succès en succès, et s’était persuadée qu’elle était aimée, a dû renoncer depuis lors à ces rêves ; elle préfère être redoutée. Elle sait bien que « son air d’innocence ne reviendra plus ».\par
Ainsi, mille complots en faveur de l’ordre établi s’enchevêtrent et se combattent un peu partout, avec l’imbrication toujours plus poussée des réseaux et des questions ou actions secrètes ; et leur processus d’intégration rapide à chaque branche de l’économie, la politique, la culture. La teneur du mélange en observateurs, en désinformateurs, en affaires spéciales, augmente continuellement dans toutes les zones de la vie sociale. Le complot général étant devenu si dense qu’il s’étale presque au grand jour, chacune de ses branches peut commencer a gêner ou inquiéter l’autre, car tous ces conspirateurs professionnels en arrivent à s’observer sans savoir exactement pourquoi, ou se rencontrent par hasard, sans pouvoir se reconnaître avec assurance. Qui veut observer qui ? Pour le compte de qui, apparemment ? Mais en réalité ? Les véritables influences restent cachées, et les intentions ultimes ne peuvent qu’être assez difficilement soupçonnées, presque jamais comprises. De sorte que personne ne peut dire qu’il n’est pas leurré ou manipulé, mais ce n’est qu’à de rares instants que le manipulateur lui-même peut savoir s’il a été gagnant. Et d’ailleurs, se trouver du côté gagnant de la manipulation ne veut pas dire que l’on avait choisi avec justesse la perspective stratégique. C’est ainsi que des succès tactiques peuvent enliser de grandes forces sur de mauvaises voies.\par
Dans un même réseau, poursuivant apparemment une même fin, ceux qui ne constituent qu’une partie du réseau sont obligés d’ignorer toutes les hypothèses et conclusions des autres parties, et surtout de leur noyau dirigeant. Le fait assez notoire que tous les renseignements sur n’importe quel sujet observé peuvent aussi bien être complètement imaginaires, ou gravement faussés, ou interprétés très inadéquatement, complique et rend peu sûrs, dans une vaste mesure, les calculs des inquisiteurs ; car ce qui est suffisant pour faire condamner quelqu’un n’est pas aussi sûr quand il s’agit de le connaître ou de l’utiliser. Puisque les sources d’information sont rivales, les falsifications le sont aussi.\par
C’est à partir de telles conditions de son exercice que l’on peut parler d’une tendance à la rentabilité décroissante du contrôle, à mesure qu’il s’approche de la totalité de l’espace social, et qu’il augmente conséquemment son personnel et ses moyens. Car ici chaque moyen aspire, et travaille, à devenir une fin. La surveillance se surveille elle-même et complote contre elle-même.\par
Enfin sa principale contradiction actuelle, c’est qu’elle surveille, infiltre, influence, \emph{un parti absent} : celui qui est censé vouloir la subversion de l’ordre social. Mais où le voit-on à l’œuvre ? Car, certes, jamais les conditions n’ont été partout si gravement révolutionnaires, mais il n’y a que les gouvernements qui le pensent. La négation a été si parfaitement privée de sa pensée, qu’elle est depuis longtemps dispersée. De ce fait, elle n’est plus que menace vague, mais pourtant très inquiétante, et la surveillance a été à son tour privée du meilleur champ de son activité. Cette force de surveillance et d’intervention est justement conduite par les nécessités présentes qui commandent les conditions de son engagement, à se porter sur le terrain même de la menace pour la combattre \emph{par avance}. C’est pourquoi la surveillance aura intérêt à organiser elle-même des pôles de négation qu’elle informera en dehors des moyens discrédités du spectacle, afin d’influencer, non plus cette fois des terroristes, mais des théories.\par

\labelblock{XXXI}

\noindent Baltasar Graciàn, grand connaisseur du temps historique, dit avec beaucoup de pertinence, dans \emph{L’Homme de cour} : « Soit l’action, soit le discours, tout doit être mesuré au temps. Il faut vouloir quand on le peut ; car ni la saison, ni le temps n’attendent personne. »\par
Mais Omar Khàyyàm moins optimiste : « Pour parler clairement et sans paraboles, – Nous sommes les pièces du jeu que joue le Ciel-, On s’amuse avec nous sur l’échiquier de l’Être, – Et puis nous retournons, un par un, dans la boîte du Néant. »\par

\labelblock{XXXII}

\noindent La Révolution française entraîna de grands changements dans l’art de la guerre. C’est après cette expérience que Clausewitz put établir la distinction selon laquelle la tactique était l’emploi des forces dans le combat, pour y obtenir la victoire, tandis que la stratégie était l’emploi des victoires afin d’atteindre les buts de la guerre. L’Europe fut subjuguée, tout de suite et pour une longue période, par les résultats. Mais la théorie n’en a été établie que plus tard, et inégalement développée. On comprit d’abord les caractères positifs amenés directement par une profonde transformation sociale : l’enthousiasme, la mobilité qui vivait sur le pays en se rendant relativement indépendante des magasins et convois, la multiplication des effectifs. Ces éléments pratiques se trouvèrent un jour équilibrés par l’entrée en action, du côté adverse, d’éléments similaires : les armées françaises se heurtèrent en Espagne à un autre enthousiasme populaire ; dans l’espace russe à un pays sur lequel elles ne purent vivre ; après le soulèvement de l’Allemagne à des effectifs très supérieurs. Cependant l’effet de rupture, dans la nouvelle tactique française, qui fut la base simple sur laquelle Bonaparte fonda sa stratégie – celle-ci consistait à employer les victoires \emph{par avance}, comme acquises à crédit : à concevoir dès le départ la manœuvre et ses diverses variantes en tant que conséquences d’une victoire qui n’était pas encore obtenue mais le serait assurément au premier choc –, découlait aussi de l’abandon forcé d’idées fausses. Cette tactique avait été brusquement obligée de s’affranchir de ces idées fausses, en même temps qu’elle trouvait, par le jeu concomitant des autres innovations citées, les moyens d’un tel affranchissement. Les soldats français, de récente levée, étaient incapables de combattre en ligne, c’est-à-dire de rester dans leur rang et d’exécuter les feux à commandements. Ils vont alors se déployer en tirailleurs et pratiquer le feu à volonté en marchant sur l’ennemi. Or, le feu à volonté se trouvait justement être le seul efficace, celui qui opérait réellement la destruction par le fusil, la plus décisive à cette époque dans l’affrontement des armées. Cependant la pensée militaire s’était universellement refusée à une telle conclusion dans le siècle qui finissait, et la discussion de cette question a pu encore se prolonger pendant près d’un autre siècle, malgré les exemples constants de la pratique des combats, et les progrès incessants dans la portée et la vitesse de tir du fusil.\par
Semblablement, la mise en place de la domination spectaculaire est une transformation sociale si profonde qu’elle a radicalement changé l’art de gouverner. Cette simplification, qui a si vite porté de tels fruits dans la pratique, n’a pas encore été pleinement comprise théoriquement. De vieux préjugés partout démentis, des précautions devenues inutiles, et jusqu’à des traces de scrupules d’autres temps, entravent encore un peu dans la pensée d’assez nombreux gouvernants cette compréhension, que toute la pratique établit et confirme chaque jour. Non seulement on fait croire aux assujettis qu’ils sont encore, pour l’essentiel, dans un monde que l’on a fait disparaître, mais les gouvernants eux-mêmes souffrent parfois de l’inconséquence de s’y croire encore par quelques côtés. Il leur arrive de penser à une part de ce qu’ils ont supprimé, comme si c’était demeuré une réalité, et qui devrait rester présente dans leurs calculs. Ce retard ne se prolongera pas beaucoup. Qui a pu en faire tant sans peine ira forcément plus loin. On ne doit pas croire que puissent se maintenir durablement, comme un archaïsme, dans les environs du pouvoir réel, ceux qui n’auraient pas assez vite compris toute la plasticité des nouvelles règles de leur jeu, et son espèce de grandeur barbare. Le destin du spectacle n’est certainement pas de finir en despotisme éclairé.\par
Il faut conclure qu’une relève est imminente et inéluctable dans la caste cooptée qui gère la domination, et notamment dirige la protection de cette domination. En une telle matière, la nouveauté, bien sûr, ne sera jamais exposée sur la scène du spectacle. Elle apparaît seulement comme la foudre, qu’on ne reconnaît qu’à ses coups. Cette relève, qui va décisivement parachever l’œuvre des temps spectaculaires, s’opère discrètement, et quoique concernant des gens déjà installés tous dans la sphère même du pouvoir, conspirativement. Elle sélectionnera ceux qui y prendront part sur cette exigence principale : qu’ils sachent clairement de quels obstacles ils sont délivrés, et de quoi ils sont capables.\par

\labelblock{XXXIII}

\noindent Le même Sardou dit aussi : « \emph{Vainement} est relatif au sujet ; \emph{en vain} est relatif à l’objet ; \emph{inutilement}, c’est sans utilité pour personne. On a travaillé \emph{vainement} lorsqu’on l’a fait sans succès, de sorte que l’on a perdu son temps et sa peine : on a travaillé \emph{en vain} lorsqu’on l’a fait sans atteindre le but qu’on se proposait, à cause de la défectuosité de l’ouvrage. Si je ne puis venir à bout de faire ma besogne, je travaille \emph{vainement} ; je perds inutilement mon temps et ma peine. Si ma besogne faite n’a pas l’effet que j’en attendais, si je n’ai pas atteint mon but, j’ai travaillé \emph{en vain} ; c’est-à-dire que j’ai fait une chose inutile…\par
On dit aussi que quelqu’un a travaillé \emph{vainement}, lorsqu’il n’est pas récompensé de son travail, ou que ce travail n’est pas agréé ; car dans ce cas le travailleur a perdu son temps et sa peine, sans préjuger aucunement la valeur de son travail, qui peut d’ailleurs être fort bon. »\par

\dateline{(Paris, février-avril 1988.)}
 


% at least one empty page at end (for booklet couv)
\ifbooklet
  \pagestyle{empty}
  \clearpage
  % 2 empty pages maybe needed for 4e cover
  \ifnum\modulo{\value{page}}{4}=0 \hbox{}\newpage\hbox{}\newpage\fi
  \ifnum\modulo{\value{page}}{4}=1 \hbox{}\newpage\hbox{}\newpage\fi


  \hbox{}\newpage
  \ifodd\value{page}\hbox{}\newpage\fi
  {\centering\color{rubric}\bfseries\noindent\large
    Hurlus ? Qu’est-ce.\par
    \bigskip
  }
  \noindent Des bouquinistes électroniques, pour du texte libre à participation libre,
  téléchargeable gratuitement sur \href{https://hurlus.fr}{\dotuline{hurlus.fr}}.\par
  \bigskip
  \noindent Cette brochure a été produite par des éditeurs bénévoles.
  Elle n’est pas faîte pour être possédée, mais pour être lue, et puis donnée.
  Que circule le texte !
  En page de garde, on peut ajouter une date, un lieu, un nom ; pour suivre le voyage des idées.
  \par

  Ce texte a été choisi parce qu’une personne l’a aimé,
  ou haï, elle a en tous cas pensé qu’il partipait à la formation de notre présent ;
  sans le souci de plaire, vendre, ou militer pour une cause.
  \par

  L’édition électronique est soigneuse, tant sur la technique
  que sur l’établissement du texte ; mais sans aucune prétention scolaire, au contraire.
  Le but est de s’adresser à tous, sans distinction de science ou de diplôme.
  Au plus direct ! (possible)
  \par

  Cet exemplaire en papier a été tiré sur une imprimante personnelle
   ou une photocopieuse. Tout le monde peut le faire.
  Il suffit de
  télécharger un fichier sur \href{https://hurlus.fr}{\dotuline{hurlus.fr}},
  d’imprimer, et agrafer ; puis de lire et donner.\par

  \bigskip

  \noindent PS : Les hurlus furent aussi des rebelles protestants qui cassaient les statues dans les églises catholiques. En 1566 démarra la révolte des gueux dans le pays de Lille. L’insurrection enflamma la région jusqu’à Anvers où les gueux de mer bloquèrent les bateaux espagnols.
  Ce fut une rare guerre de libération dont naquit un pays toujours libre : les Pays-Bas.
  En plat pays francophone, par contre, restèrent des bandes de huguenots, les hurlus, progressivement réprimés par la très catholique Espagne.
  Cette mémoire d’une défaite est éteinte, rallumons-la. Sortons les livres du culte universitaire, cherchons les idoles de l’époque, pour les briser.
\fi

\ifdev % autotext in dev mode
\fontname\font — \textsc{Les règles du jeu}\par
(\hyperref[utopie]{\underline{Lien}})\par
\noindent \initialiv{A}{lors là}\blindtext\par
\noindent \initialiv{À}{ la bonheur des dames}\blindtext\par
\noindent \initialiv{É}{tonnez-le}\blindtext\par
\noindent \initialiv{Q}{ualitativement}\blindtext\par
\noindent \initialiv{V}{aloriser}\blindtext\par
\Blindtext
\phantomsection
\label{utopie}
\Blinddocument
\fi
\end{document}
