%%%%%%%%%%%%%%%%%%%%%%%%%%%%%%%%%
% LaTeX model https://hurlus.fr %
%%%%%%%%%%%%%%%%%%%%%%%%%%%%%%%%%

% Needed before document class
\RequirePackage{pdftexcmds} % needed for tests expressions
\RequirePackage{fix-cm} % correct units

% Define mode
\def\mode{a4}

\newif\ifaiv % a4
\newif\ifav % a5
\newif\ifbooklet % booklet
\newif\ifcover % cover for booklet

\ifnum \strcmp{\mode}{cover}=0
  \covertrue
\else\ifnum \strcmp{\mode}{booklet}=0
  \booklettrue
\else\ifnum \strcmp{\mode}{a5}=0
  \avtrue
\else
  \aivtrue
\fi\fi\fi

\ifbooklet % do not enclose with {}
  \documentclass[french,twoside]{book} % ,notitlepage
  \usepackage[%
    papersize={105mm, 297mm},
    inner=12mm,
    outer=12mm,
    top=20mm,
    bottom=15mm,
    marginparsep=0pt,
  ]{geometry}
  \usepackage[fontsize=9.5pt]{scrextend} % for Roboto
\else\ifav
  \documentclass[french,twoside]{book} % ,notitlepage
  \usepackage[%
    a5paper,
    inner=25mm,
    outer=15mm,
    top=15mm,
    bottom=15mm,
    marginparsep=0pt,
  ]{geometry}
  \usepackage[fontsize=12pt]{scrextend}
\else% A4 2 cols
  \documentclass[twocolumn]{report}
  \usepackage[%
    a4paper,
    inner=15mm,
    outer=10mm,
    top=25mm,
    bottom=18mm,
    marginparsep=0pt,
  ]{geometry}
  \setlength{\columnsep}{20mm}
  \usepackage[fontsize=9.5pt]{scrextend}
\fi\fi

%%%%%%%%%%%%%%
% Alignments %
%%%%%%%%%%%%%%
% before teinte macros

\setlength{\arrayrulewidth}{0.2pt}
\setlength{\columnseprule}{\arrayrulewidth} % twocol
\setlength{\parskip}{0pt} % classical para with no margin
\setlength{\parindent}{1.5em}

%%%%%%%%%%
% Colors %
%%%%%%%%%%
% before Teinte macros

\usepackage[dvipsnames]{xcolor}
\definecolor{rubric}{HTML}{800000} % the tonic 0c71c3
\def\columnseprulecolor{\color{rubric}}
\colorlet{borderline}{rubric!30!} % definecolor need exact code
\definecolor{shadecolor}{gray}{0.95}
\definecolor{bghi}{gray}{0.5}

%%%%%%%%%%%%%%%%%
% Teinte macros %
%%%%%%%%%%%%%%%%%
%%%%%%%%%%%%%%%%%%%%%%%%%%%%%%%%%%%%%%%%%%%%%%%%%%%
% <TEI> generic (LaTeX names generated by Teinte) %
%%%%%%%%%%%%%%%%%%%%%%%%%%%%%%%%%%%%%%%%%%%%%%%%%%%
% This template is inserted in a specific design
% It is XeLaTeX and otf fonts

\makeatletter % <@@@


\usepackage{blindtext} % generate text for testing
\usepackage[strict]{changepage} % for modulo 4
\usepackage{contour} % rounding words
\usepackage[nodayofweek]{datetime}
% \usepackage{DejaVuSans} % seems buggy for sffont font for symbols
\usepackage{enumitem} % <list>
\usepackage{etoolbox} % patch commands
\usepackage{fancyvrb}
\usepackage{fancyhdr}
\usepackage{float}
\usepackage{fontspec} % XeLaTeX mandatory for fonts
\usepackage{footnote} % used to capture notes in minipage (ex: quote)
\usepackage{framed} % bordering correct with footnote hack
\usepackage{graphicx}
\usepackage{lettrine} % drop caps
\usepackage{lipsum} % generate text for testing
\usepackage[framemethod=tikz,]{mdframed} % maybe used for frame with footnotes inside
\usepackage{pdftexcmds} % needed for tests expressions
\usepackage{polyglossia} % non-break space french punct, bug Warning: "Failed to patch part"
\usepackage[%
  indentfirst=false,
  vskip=1em,
  noorphanfirst=true,
  noorphanafter=true,
  leftmargin=\parindent,
  rightmargin=0pt,
]{quoting}
\usepackage{ragged2e}
\usepackage{setspace} % \setstretch for <quote>
\usepackage{tabularx} % <table>
\usepackage[explicit]{titlesec} % wear titles, !NO implicit
\usepackage{tikz} % ornaments
\usepackage{tocloft} % styling tocs
\usepackage[fit]{truncate} % used im runing titles
\usepackage{unicode-math}
\usepackage[normalem]{ulem} % breakable \uline, normalem is absolutely necessary to keep \emph
\usepackage{verse} % <l>
\usepackage{xcolor} % named colors
\usepackage{xparse} % @ifundefined
\XeTeXdefaultencoding "iso-8859-1" % bad encoding of xstring
\usepackage{xstring} % string tests
\XeTeXdefaultencoding "utf-8"
\PassOptionsToPackage{hyphens}{url} % before hyperref, which load url package

% TOTEST
% \usepackage{hypcap} % links in caption ?
% \usepackage{marginnote}
% TESTED
% \usepackage{background} % doesn’t work with xetek
% \usepackage{bookmark} % prefers the hyperref hack \phantomsection
% \usepackage[color, leftbars]{changebar} % 2 cols doc, impossible to keep bar left
% \usepackage[utf8x]{inputenc} % inputenc package ignored with utf8 based engines
% \usepackage[sfdefault,medium]{inter} % no small caps
% \usepackage{firamath} % choose firasans instead, firamath unavailable in Ubuntu 21-04
% \usepackage{flushend} % bad for last notes, supposed flush end of columns
% \usepackage[stable]{footmisc} % BAD for complex notes https://texfaq.org/FAQ-ftnsect
% \usepackage{helvet} % not for XeLaTeX
% \usepackage{multicol} % not compatible with too much packages (longtable, framed, memoir…)
% \usepackage[default,oldstyle,scale=0.95]{opensans} % no small caps
% \usepackage{sectsty} % \chapterfont OBSOLETE
% \usepackage{soul} % \ul for underline, OBSOLETE with XeTeX
% \usepackage[breakable]{tcolorbox} % text styling gone, footnote hack not kept with breakable


% Metadata inserted by a program, from the TEI source, for title page and runing heads
\title{\textbf{ Apologie pour l’histoire ou métier d’historien }}
\date{1941}
\author{Bloch, Marc}
\def\elbibl{Bloch, Marc. 1941. \emph{Apologie pour l’histoire ou métier d’historien}}
\def\elsource{\{source\}}

% Default metas
\newcommand{\colorprovide}[2]{\@ifundefinedcolor{#1}{\colorlet{#1}{#2}}{}}
\colorprovide{rubric}{red}
\colorprovide{silver}{lightgray}
\@ifundefined{syms}{\newfontfamily\syms{DejaVu Sans}}{}
\newif\ifdev
\@ifundefined{elbibl}{% No meta defined, maybe dev mode
  \newcommand{\elbibl}{Titre court ?}
  \newcommand{\elbook}{Titre du livre source ?}
  \newcommand{\elabstract}{Résumé\par}
  \newcommand{\elurl}{http://oeuvres.github.io/elbook/2}
  \author{Éric Lœchien}
  \title{Un titre de test assez long pour vérifier le comportement d’une maquette}
  \date{1566}
  \devtrue
}{}
\let\eltitle\@title
\let\elauthor\@author
\let\eldate\@date


\defaultfontfeatures{
  % Mapping=tex-text, % no effect seen
  Scale=MatchLowercase,
  Ligatures={TeX,Common},
}


% generic typo commands
\newcommand{\astermono}{\medskip\centerline{\color{rubric}\large\selectfont{\syms ✻}}\medskip\par}%
\newcommand{\astertri}{\medskip\par\centerline{\color{rubric}\large\selectfont{\syms ✻\,✻\,✻}}\medskip\par}%
\newcommand{\asterism}{\bigskip\par\noindent\parbox{\linewidth}{\centering\color{rubric}\large{\syms ✻}\\{\syms ✻}\hskip 0.75em{\syms ✻}}\bigskip\par}%

% lists
\newlength{\listmod}
\setlength{\listmod}{\parindent}
\setlist{
  itemindent=!,
  listparindent=\listmod,
  labelsep=0.2\listmod,
  parsep=0pt,
  % topsep=0.2em, % default topsep is best
}
\setlist[itemize]{
  label=—,
  leftmargin=0pt,
  labelindent=1.2em,
  labelwidth=0pt,
}
\setlist[enumerate]{
  label={\bf\color{rubric}\arabic*.},
  labelindent=0.8\listmod,
  leftmargin=\listmod,
  labelwidth=0pt,
}
\newlist{listalpha}{enumerate}{1}
\setlist[listalpha]{
  label={\bf\color{rubric}\alph*.},
  leftmargin=0pt,
  labelindent=0.8\listmod,
  labelwidth=0pt,
}
\newcommand{\listhead}[1]{\hspace{-1\listmod}\emph{#1}}

\renewcommand{\hrulefill}{%
  \leavevmode\leaders\hrule height 0.2pt\hfill\kern\z@}

% General typo
\DeclareTextFontCommand{\textlarge}{\large}
\DeclareTextFontCommand{\textsmall}{\small}

% commands, inlines
\newcommand{\anchor}[1]{\Hy@raisedlink{\hypertarget{#1}{}}} % link to top of an anchor (not baseline)
\newcommand\abbr[1]{#1}
\newcommand{\autour}[1]{\tikz[baseline=(X.base)]\node [draw=rubric,thin,rectangle,inner sep=1.5pt, rounded corners=3pt] (X) {\color{rubric}#1};}
\newcommand\corr[1]{#1}
\newcommand{\ed}[1]{ {\color{silver}\sffamily\footnotesize (#1)} } % <milestone ed="1688"/>
\newcommand\expan[1]{#1}
\newcommand\foreign[1]{\emph{#1}}
\newcommand\gap[1]{#1}
\renewcommand{\LettrineFontHook}{\color{rubric}}
\newcommand{\initial}[2]{\lettrine[lines=2, loversize=0.3, lhang=0.3]{#1}{#2}}
\newcommand{\initialiv}[2]{%
  \let\oldLFH\LettrineFontHook
  % \renewcommand{\LettrineFontHook}{\color{rubric}\ttfamily}
  \IfSubStr{QJ’}{#1}{
    \lettrine[lines=4, lhang=0.2, loversize=-0.1, lraise=0.2]{\smash{#1}}{#2}
  }{\IfSubStr{É}{#1}{
    \lettrine[lines=4, lhang=0.2, loversize=-0, lraise=0]{\smash{#1}}{#2}
  }{\IfSubStr{ÀÂ}{#1}{
    \lettrine[lines=4, lhang=0.2, loversize=-0, lraise=0, slope=0.6em]{\smash{#1}}{#2}
  }{\IfSubStr{A}{#1}{
    \lettrine[lines=4, lhang=0.2, loversize=0.2, slope=0.6em]{\smash{#1}}{#2}
  }{\IfSubStr{V}{#1}{
    \lettrine[lines=4, lhang=0.2, loversize=0.2, slope=-0.5em]{\smash{#1}}{#2}
  }{
    \lettrine[lines=4, lhang=0.2, loversize=0.2]{\smash{#1}}{#2}
  }}}}}
  \let\LettrineFontHook\oldLFH
}
\newcommand{\labelchar}[1]{\textbf{\color{rubric} #1}}
\newcommand{\milestone}[1]{\autour{\footnotesize\color{rubric} #1}} % <milestone n="4"/>
\newcommand\name[1]{#1}
\newcommand\orig[1]{#1}
\newcommand\orgName[1]{#1}
\newcommand\persName[1]{#1}
\newcommand\placeName[1]{#1}
\newcommand{\pn}[1]{\IfSubStr{-—–¶}{#1}% <p n="3"/>
  {\noindent{\bfseries\color{rubric}   ¶  }}
  {{\footnotesize\autour{ #1}  }}}
\newcommand\reg{}
% \newcommand\ref{} % already defined
\newcommand\sic[1]{#1}
\newcommand\surname[1]{\textsc{#1}}
\newcommand\term[1]{\textbf{#1}}

\def\mednobreak{\ifdim\lastskip<\medskipamount
  \removelastskip\nopagebreak\medskip\fi}
\def\bignobreak{\ifdim\lastskip<\bigskipamount
  \removelastskip\nopagebreak\bigskip\fi}

% commands, blocks
\newcommand{\byline}[1]{\bigskip{\RaggedLeft{#1}\par}\bigskip}
\newcommand{\bibl}[1]{{\RaggedLeft{#1}\par\bigskip}}
\newcommand{\biblitem}[1]{{\noindent\hangindent=\parindent   #1\par}}
\newcommand{\dateline}[1]{\medskip{\RaggedLeft{#1}\par}\bigskip}
\newcommand{\labelblock}[1]{\medbreak{\noindent\color{rubric}\bfseries #1}\par\mednobreak}
\newcommand{\salute}[1]{\bigbreak{#1}\par\medbreak}
\newcommand{\signed}[1]{\bigbreak\filbreak{\raggedleft #1\par}\medskip}

% environments for blocks (some may become commands)
\newenvironment{borderbox}{}{} % framing content
\newenvironment{citbibl}{\ifvmode\hfill\fi}{\ifvmode\par\fi }
\newenvironment{docAuthor}{\ifvmode\vskip4pt\fontsize{16pt}{18pt}\selectfont\fi\itshape}{\ifvmode\par\fi }
\newenvironment{docDate}{}{\ifvmode\par\fi }
\newenvironment{docImprint}{\vskip6pt}{\ifvmode\par\fi }
\newenvironment{docTitle}{\vskip6pt\bfseries\fontsize{18pt}{22pt}\selectfont}{\par }
\newenvironment{msHead}{\vskip6pt}{\par}
\newenvironment{msItem}{\vskip6pt}{\par}
\newenvironment{titlePart}{}{\par }


% environments for block containers
\newenvironment{argument}{\itshape\parindent0pt}{\vskip1.5em}
\newenvironment{biblfree}{}{\ifvmode\par\fi }
\newenvironment{bibitemlist}[1]{%
  \list{\@biblabel{\@arabic\c@enumiv}}%
  {%
    \settowidth\labelwidth{\@biblabel{#1}}%
    \leftmargin\labelwidth
    \advance\leftmargin\labelsep
    \@openbib@code
    \usecounter{enumiv}%
    \let\p@enumiv\@empty
    \renewcommand\theenumiv{\@arabic\c@enumiv}%
  }
  \sloppy
  \clubpenalty4000
  \@clubpenalty \clubpenalty
  \widowpenalty4000%
  \sfcode`\.\@m
}%
{\def\@noitemerr
  {\@latex@warning{Empty `bibitemlist' environment}}%
\endlist}
\newenvironment{quoteblock}% may be used for ornaments
  {\begin{quoting}}
  {\end{quoting}}

% table () is preceded and finished by custom command
\newcommand{\tableopen}[1]{%
  \ifnum\strcmp{#1}{wide}=0{%
    \begin{center}
  }
  \else\ifnum\strcmp{#1}{long}=0{%
    \begin{center}
  }
  \else{%
    \begin{center}
  }
  \fi\fi
}
\newcommand{\tableclose}[1]{%
  \ifnum\strcmp{#1}{wide}=0{%
    \end{center}
  }
  \else\ifnum\strcmp{#1}{long}=0{%
    \end{center}
  }
  \else{%
    \end{center}
  }
  \fi\fi
}


% text structure
\newcommand\chapteropen{} % before chapter title
\newcommand\chaptercont{} % after title, argument, epigraph…
\newcommand\chapterclose{} % maybe useful for multicol settings
\setcounter{secnumdepth}{-2} % no counters for hierarchy titles
\setcounter{tocdepth}{5} % deep toc
\markright{\@title} % ???
\markboth{\@title}{\@author} % ???
\renewcommand\tableofcontents{\@starttoc{toc}}
% toclof format
% \renewcommand{\@tocrmarg}{0.1em} % Useless command?
% \renewcommand{\@pnumwidth}{0.5em} % {1.75em}
\renewcommand{\@cftmaketoctitle}{}
\setlength{\cftbeforesecskip}{\z@ \@plus.2\p@}
\renewcommand{\cftchapfont}{}
\renewcommand{\cftchapdotsep}{\cftdotsep}
\renewcommand{\cftchapleader}{\normalfont\cftdotfill{\cftchapdotsep}}
\renewcommand{\cftchappagefont}{\bfseries}
\setlength{\cftbeforechapskip}{0em \@plus\p@}
% \renewcommand{\cftsecfont}{\small\relax}
\renewcommand{\cftsecpagefont}{\normalfont}
% \renewcommand{\cftsubsecfont}{\small\relax}
\renewcommand{\cftsecdotsep}{\cftdotsep}
\renewcommand{\cftsecpagefont}{\normalfont}
\renewcommand{\cftsecleader}{\normalfont\cftdotfill{\cftsecdotsep}}
\setlength{\cftsecindent}{1em}
\setlength{\cftsubsecindent}{2em}
\setlength{\cftsubsubsecindent}{3em}
\setlength{\cftchapnumwidth}{1em}
\setlength{\cftsecnumwidth}{1em}
\setlength{\cftsubsecnumwidth}{1em}
\setlength{\cftsubsubsecnumwidth}{1em}

% footnotes
\newif\ifheading
\newcommand*{\fnmarkscale}{\ifheading 0.70 \else 1 \fi}
\renewcommand\footnoterule{\vspace*{0.3cm}\hrule height \arrayrulewidth width 3cm \vspace*{0.3cm}}
\setlength\footnotesep{1.5\footnotesep} % footnote separator
\renewcommand\@makefntext[1]{\parindent 1.5em \noindent \hb@xt@1.8em{\hss{\normalfont\@thefnmark . }}#1} % no superscipt in foot
\patchcmd{\@footnotetext}{\footnotesize}{\footnotesize\sffamily}{}{} % before scrextend, hyperref


%   see https://tex.stackexchange.com/a/34449/5049
\def\truncdiv#1#2{((#1-(#2-1)/2)/#2)}
\def\moduloop#1#2{(#1-\truncdiv{#1}{#2}*#2)}
\def\modulo#1#2{\number\numexpr\moduloop{#1}{#2}\relax}

% orphans and widows
\clubpenalty=9996
\widowpenalty=9999
\brokenpenalty=4991
\predisplaypenalty=10000
\postdisplaypenalty=1549
\displaywidowpenalty=1602
\hyphenpenalty=400
% Copied from Rahtz but not understood
\def\@pnumwidth{1.55em}
\def\@tocrmarg {2.55em}
\def\@dotsep{4.5}
\emergencystretch 3em
\hbadness=4000
\pretolerance=750
\tolerance=2000
\vbadness=4000
\def\Gin@extensions{.pdf,.png,.jpg,.mps,.tif}
% \renewcommand{\@cite}[1]{#1} % biblio

\usepackage{hyperref} % supposed to be the last one, :o) except for the ones to follow
\urlstyle{same} % after hyperref
\hypersetup{
  % pdftex, % no effect
  pdftitle={\elbibl},
  % pdfauthor={Your name here},
  % pdfsubject={Your subject here},
  % pdfkeywords={keyword1, keyword2},
  bookmarksnumbered=true,
  bookmarksopen=true,
  bookmarksopenlevel=1,
  pdfstartview=Fit,
  breaklinks=true, % avoid long links
  pdfpagemode=UseOutlines,    % pdf toc
  hyperfootnotes=true,
  colorlinks=false,
  pdfborder=0 0 0,
  % pdfpagelayout=TwoPageRight,
  % linktocpage=true, % NO, toc, link only on page no
}

\makeatother % /@@@>
%%%%%%%%%%%%%%
% </TEI> end %
%%%%%%%%%%%%%%


%%%%%%%%%%%%%
% footnotes %
%%%%%%%%%%%%%
\renewcommand{\thefootnote}{\bfseries\textcolor{rubric}{\arabic{footnote}}} % color for footnote marks

%%%%%%%%%
% Fonts %
%%%%%%%%%
\usepackage[]{roboto} % SmallCaps, Regular is a bit bold
% \linespread{0.90} % too compact, keep font natural
\newfontfamily\fontrun[]{Roboto Condensed Light} % condensed runing heads
\ifav
  \setmainfont[
    ItalicFont={Roboto Light Italic},
  ]{Roboto}
\else\ifbooklet
  \setmainfont[
    ItalicFont={Roboto Light Italic},
  ]{Roboto}
\else
\setmainfont[
  ItalicFont={Roboto Italic},
]{Roboto Light}
\fi\fi
\renewcommand{\LettrineFontHook}{\bfseries\color{rubric}}
% \renewenvironment{labelblock}{\begin{center}\bfseries\color{rubric}}{\end{center}}

%%%%%%%%
% MISC %
%%%%%%%%

\setdefaultlanguage[frenchpart=false]{french} % bug on part


\newenvironment{quotebar}{%
    \def\FrameCommand{{\color{rubric!10!}\vrule width 0.5em} \hspace{0.9em}}%
    \def\OuterFrameSep{\itemsep} % séparateur vertical
    \MakeFramed {\advance\hsize-\width \FrameRestore}
  }%
  {%
    \endMakeFramed
  }
\renewenvironment{quoteblock}% may be used for ornaments
  {%
    \savenotes
    \setstretch{0.9}
    \normalfont
    \begin{quotebar}
  }
  {%
    \end{quotebar}
    \spewnotes
  }


\renewcommand{\headrulewidth}{\arrayrulewidth}
\renewcommand{\headrule}{{\color{rubric}\hrule}}

% delicate tuning, image has produce line-height problems in title on 2 lines
\titleformat{name=\chapter} % command
  [display] % shape
  {\vspace{1.5em}\centering} % format
  {} % label
  {0pt} % separator between n
  {}
[{\color{rubric}\huge\textbf{#1}}\bigskip] % after code
% \titlespacing{command}{left spacing}{before spacing}{after spacing}[right]
\titlespacing*{\chapter}{0pt}{-2em}{0pt}[0pt]

\titleformat{name=\section}
  [block]{}{}{}{}
  [\vbox{\color{rubric}\large\raggedleft\textbf{#1}}]
\titlespacing{\section}{0pt}{0pt plus 4pt minus 2pt}{\baselineskip}

\titleformat{name=\subsection}
  [block]
  {}
  {} % \thesection
  {} % separator \arrayrulewidth
  {}
[\vbox{\large\textbf{#1}}]
% \titlespacing{\subsection}{0pt}{0pt plus 4pt minus 2pt}{\baselineskip}

\ifaiv
  \fancypagestyle{main}{%
    \fancyhf{}
    \setlength{\headheight}{1.5em}
    \fancyhead{} % reset head
    \fancyfoot{} % reset foot
    \fancyhead[L]{\truncate{0.45\headwidth}{\fontrun\elbibl}} % book ref
    \fancyhead[R]{\truncate{0.45\headwidth}{ \fontrun\nouppercase\leftmark}} % Chapter title
    \fancyhead[C]{\thepage}
  }
  \fancypagestyle{plain}{% apply to chapter
    \fancyhf{}% clear all header and footer fields
    \setlength{\headheight}{1.5em}
    \fancyhead[L]{\truncate{0.9\headwidth}{\fontrun\elbibl}}
    \fancyhead[R]{\thepage}
  }
\else
  \fancypagestyle{main}{%
    \fancyhf{}
    \setlength{\headheight}{1.5em}
    \fancyhead{} % reset head
    \fancyfoot{} % reset foot
    \fancyhead[RE]{\truncate{0.9\headwidth}{\fontrun\elbibl}} % book ref
    \fancyhead[LO]{\truncate{0.9\headwidth}{\fontrun\nouppercase\leftmark}} % Chapter title, \nouppercase needed
    \fancyhead[RO,LE]{\thepage}
  }
  \fancypagestyle{plain}{% apply to chapter
    \fancyhf{}% clear all header and footer fields
    \setlength{\headheight}{1.5em}
    \fancyhead[L]{\truncate{0.9\headwidth}{\fontrun\elbibl}}
    \fancyhead[R]{\thepage}
  }
\fi

\ifav % a5 only
  \titleclass{\section}{top}
\fi

\newcommand\chapo{{%
  \vspace*{-3em}
  \centering % no vskip ()
  {\Large\addfontfeature{LetterSpace=25}\bfseries{\elauthor}}\par
  \smallskip
  {\large\eldate}\par
  \bigskip
  {\Large\selectfont{\eltitle}}\par
  \bigskip
  {\color{rubric}\hline\par}
  \bigskip
  {\Large TEXTE LIBRE À PARTICPATION LIBRE\par}
  \centerline{\small\color{rubric} {hurlus.fr, tiré le \today}}\par
  \bigskip
}}

\newcommand\cover{{%
  \thispagestyle{empty}
  \centering
  {\LARGE\bfseries{\elauthor}}\par
  \bigskip
  {\Large\eldate}\par
  \bigskip
  \bigskip
  {\LARGE\selectfont{\eltitle}}\par
  \vfill\null
  {\color{rubric}\setlength{\arrayrulewidth}{2pt}\hline\par}
  \vfill\null
  {\Large TEXTE LIBRE À PARTICPATION LIBRE\par}
  \centerline{{\href{https://hurlus.fr}{\dotuline{hurlus.fr}}, tiré le \today}}\par
}}

\begin{document}
\pagestyle{empty}
\ifbooklet{
  \cover\newpage
  \thispagestyle{empty}\hbox{}\newpage
  \cover\newpage\noindent Les voyages de la brochure\par
  \bigskip
  \begin{tabularx}{\textwidth}{l|X|X}
    \textbf{Date} & \textbf{Lieu}& \textbf{Nom/pseudo} \\ \hline
    \rule{0pt}{25cm} &  &   \\
  \end{tabularx}
  \newpage
  \addtocounter{page}{-4}
}\fi

\thispagestyle{empty}
\ifaiv
  \twocolumn[\chapo]
\else
  \chapo
\fi
{\it\elabstract}
\bigskip
\makeatletter\@starttoc{toc}\makeatother % toc without new page
\bigskip

\pagestyle{main} % after style

  \section[{À Lucien Febvre. En manière de dédicace}]{À Lucien Febvre \\
En manière de dédicace}\renewcommand{\leftmark}{À Lucien Febvre \\
En manière de dédicace}

\noindent \emph{Si ce livre doit, un jour, être publié ; si, de simple antidote auquel, parmi les pires douleurs et les pires anxiétés, personnelles et collectives, je demande aujourd’hui un peu d’équilibre de l’âme, il se change jamais en un vrai livre, offert pour être lu : un autre nom que le vôtre, cher ami, sera alors inscrit sur la feuille de garde. Vous le sentez, il le fallait, ce nom‑là, à cette place : seul rappel permis à une tendresse trop profonde et trop sacrée pour souffrir même d’être dite. Vous aussi cependant, comment me résignerais‑je à ne vous voir paraître seulement qu’au hasard de quelques références ? Longuement nous avons combattu de concert, pour une histoire plus large et plus humaine. La tâche commune, au moment où j’écris, subit bien des menaces. Non par notre faute. Nous sommes les vaincus provisoires d’un injuste destin. Le temps viendra, j’en suis sûr, où notre collaboration pourra vraiment re­prendre, publique comme par le passé et, comme par le passé, libre. En attendant, c’est dans ces pages toutes pleines de votre présence que, de mon côté, elle se poursuivra. Elle y gardera le rythme, qui fut toujours le sien, d’un accord fondamental, vivifié, en surface, par le profitable jeu de nos affectueuses discussions. Parmi les idées que je me propose de soutenir, plus d’une assurément me vient tout droit de vous. De beaucoup d’autres, je ne saurais décider, en toute conscience, si elles sont de vous, de moi, ou de nous deux. Vous approuverez, je m’en flatte, souvent. Vous me gour­manderez quelquefois. Et tout cela fera entre nous un lien de plus.}\par

\dateline{Fougères (Creuse) \\
le 10 mai 1941.}
\section[{Introduction}]{Introduction}\renewcommand{\leftmark}{Introduction}

\noindent  \phantomsection
\label{pIX} « Papa, explique‑moi donc à quoi sert l’histoire. » Ainsi un jeune garçon qui me touche de près interrogeait, il y a peu d’années, un père historien. Du livre qu’on va lire, j’aimerais pouvoir dire qu’il est ma réponse. Car je n’imagine pas, pour un écrivain, de plus belle louange que de savoir parler, du même ton, aux doctes et aux écoliers. Mais une simplicité si haute est le privilège de quelques rares élus. Du moins cette question d’un enfant – dont, sur le moment, je n’ai peut être pas trop bien réussi à satisfaire la soif de savoir – volontiers je la retiendrai ici comme épi­graphe. D’aucuns en jugeront, sans doute, la formule naïve. Elle me semble au contraire parfaitement pertinente \footnote{En quoi je me trouve m’opposer dès le début, et sans l’avoir cherché à l’\emph{In­troduction aux Études Historiques} de Langlois et Seignobos. Le passage qu’on vient de lire était écrit depuis longtemps déjà quand m’est tombée sous les yeux, dans l’\emph{Avertissement} de cet ouvrage (p. XII), une liste de « questions oiseuses ». J’y vois figurer textuellement, celle‑ci « À quoi sert l’histoire ? » – Sans doute en va‑t‑il de ce problème comme de presque tous ceux qui concernent les raisons d’être de nos actes et de nos pensées : les esprits qui leur demeurent par nature indifférents – ou ont volontairement décidé de se rendre tels – comprennent toujours difficilement que d’autres esprits y trouvent le sujet de réflexions pas­sionnantes. Cependant, puisque l’occasion m’en est ainsi offerte, mieux vaut je crois, fixer dès maintenant ma position vis‑à‑vis d’un livre justement notoire – que le mien d’ailleurs, construit sur un autre plan et, dans certaines de ses parties, beaucoup moins développé, ne prétend nullement remplacer. J’ai été l’élève de ses deux auteurs et, spécialement, de M. Seignobos. Ils m’ont donné, l’un et l’autre, de précieuses marques de leur bienveillance. Mon éducation pre­mière a dû beaucoup à leur enseignement et à leur œuvre. Mais ils ne nous ont pas seulement appris, tous deux, que l’historien a pour premier devoir d’être sincère ; ils ne dissimulaient pas davantage que le progrès même de nos études est fait de la contradiction nécessaire entre les générations de travailleurs. Je resterai donc fidèle à leurs leçons en les critiquant, là où je le jugerai utile, très librement ; comme je souhaite qu’un jour mes élèves, à leur tour, me critiquent.}. Le problème qu’elle pose, avec l’embarrassante droiture de cet âge implacable, n’est rien de moins que celui de la légitimité de l’histoire.\par
Voilà donc l’historien appelé à rendre ses comptes. Il ne s’y hasardera qu’avec un peu de tremblement intérieur : quel artisan, vieilli dans le métier, s’est jamais demandé, sans un pincement de cœur, s’il a fait de sa vie un sage emploi ? Mais le débat dépasse, de beaucoup, les petits scrupules d’une morale corporative. Notre civilisation occidentale tout entière y est intéressée.\par
Car, à la différence d’autres types de culture, elle a toujours beaucoup attendu de sa mémoire. Tout l’y portait : l’héritage chrétien comme l’héri­tage antique. Les Grecs et les Latins, nos premiers maîtres, étaient des peuples historiographes. Le christianisme est une religion d’historiens. D’autres systèmes religieux ont pu fonder leurs croyances et leurs rites sur une mythologie à peu près extérieure au temps humain. Pour Livres Sacrés, les chrétiens ont des livres d’histoire, et leurs liturgies commé­morent, avec les épisodes de la vie terrestre d’un Dieu, les fastes de l’Église et des saints. Historique, le christianisme l’est encore d’une autre façon, peut‑être plus profonde : placée entre la Chute et le Jugement, la destinée de l’humanité figure, à ses yeux, une longue aventure, dont chaque destin, chaque « pèlerinage » individuel présente, à son tour, le reflet ; c’est dans la durée, partant dans l’histoire, qu’axe central de toute méditation chré­tienne se déroule le grand drame du Péché et de la Rédemption. Notre  \phantomsection
\label{pX} art, nos monuments littéraires sont pleins des échos du passé ; nos hommes d’action ont incessamment à la bouche ses leçons, réelles ou prétendues. Sans doute conviendrait‑il de marquer, entre les psychologies de groupes, plus d’une nuance. Cournot l’a observé il y a longtemps ; éternellement enclins à reconstruire le monde sur les lignes de la raison, les Français, dans leur masse, vivent leurs souvenirs collectifs beaucoup moins intensé­ment que les Allemands, par exemple \footnote{\emph{Fragment de cette note sur feuille volante : le début est perdu} : [… comme l’a montré] Lucien Febvre, c’est l’histoire elle‑même qui, interrogée sur la ligne que le développement de l’humanité n’a cessé de suivre, se charge de leur infliger le plus flagrant démenti. Non seulement chaque science, prise à part, trouve dans les transfuges des secteurs voisins, les artisans souvent les meilleurs de ses succès. Pasteur, qui renouvela la biologie, n’était pas un biologiste – et, de son vivant, on le lui fit bien voir ; tout de même que Durkheim et Vidal de la Blache qui ont laissé sur les études historiques du début du XX\textsuperscript{ᵉ} siècle une marque incom­parablement plus profonde que celle de n’importe quel spécialiste, étant, le pre­mier, un philosophe passé à la sociologie, le second un géographe, ne se rangeaient ni l’un ni l’autre parmi les historiens à brevet.}. Sans doute aussi, les civilisations peuvent changer. Il n’est pas inconcevable, en soi, que la nôtre ne se détourne un jour de l’histoire. Les historiens feront sagement d’y réfléchir. L’histoire mal entendue pourrait bien, si l’on n’y prenait garde, risquer d’entraîner finalement dans son discrédit l’histoire mieux comprise. Mais si nous devions jamais en arriver là, ce serait au prix d’une profonde rupture avec nos plus constantes traditions intellectuelles.\par
Pour l’instant, nous n’en sommes, à ce sujet, qu’au stade de l’examen de conscience. Chaque fois que nos strictes sociétés, en perpétuelle crise de croissance, se prennent à douter d’elles‑mêmes, on les voit se demander si elles ont eu raison d’interroger leur passé ou si elles l’ont bien interrogé. Lisez ce qui s’écrivait avant la guerre, ce qui peut encore s’écrire aujour­d’hui : parmi les inquiétudes diffuses du temps présent, vous entendrez, presque immanquablement, cette inquiétude mêler sa voix aux autres. En plein drame, il m’a été donné d’en saisir l’écho tout spontané. C’était en juin 1940, le jour même, si je me souviens bien, de l’entrée des Allemands à Paris. Dans le jardin normand où notre état-major, privé de troupes, traînait son oisiveté, nous remâchions les causes du désastre : « Faut‑il croire que l’histoire nous ait trompés ? », murmura l’un de nous. Ainsi l’angoisse de l’homme fait rejoignait, avec un accent plus amer, la simple curiosité du jouvenceau. Il faut répondre à l’une et à l’autre.\par
Encore, cependant, convient‑il de savoir ce que veut dire ce mot « servir ». Mais avant de l’examiner, que j’ajoute encore un mot d’excuse. Les circonstances de ma vie présente, l’impossibilité où je suis d’atteindre aucune grande bibliothèque, la perte de mes propres livres, font que je dois me fier beaucoup à mes notes et à mon acquis. Les lectures complé­mentaires, les vérifications qu’appelleraient les lois mêmes du métier dont je me propose de décrire les pratiques me demeurent trop souvent interdites. Me sera‑t‑il donné un jour de combler ces lacunes ? Jamais entièrement, je le crains. Je ne puis, là‑dessus, que solliciter l’indulgence, – je dirais « plaider coupable », si ce n’était prendre sur moi, plus qu’il n’est légitime, les fautes de la destinée.\par

\astermono

\noindent Certes, même si l’histoire devait être jugée incapable d’autres services, il resterait à faire valoir, en sa faveur, qu’elle est distrayante. Ou, pour  \phantomsection
\label{pXI} être plus exact – car chacun cherche ses distractions où il lui plaît – qu’elle paraît telle, incontestablement, à un grand nombre d’hommes. Personnellement, d’aussi loin que je me souvienne, elle m’a toujours beau­coup diverti. Comme tous les historiens, je pense. Sans quoi, pour quelles raisons auraient‑ils choisi ce métier ? Aux yeux de quiconque n’est point un sot, en trois lettres, toutes les sciences sont intéressantes. Mais chaque savant n’en trouve guère qu’une dont la pratique l’amuse. La découvrir, pour s’y consacrer, est proprement ce qu’on nomme vocation.\par
En soi, d’ailleurs, cet indéniable attrait de l’histoire mérite déjà d’ar­rêter la réflexion.\par
Comme germe d’abord et comme aiguillon, son rôle a été et demeure capital. Avant le désir de connaissance, le simple goût ; avant l’œuvre de science, pleinement consciente de ses fins, l’instinct qui y conduit : l’évolution de notre comportement intellectuel abonde en filiations de cette sorte. Il n’est pas jusqu’à la physique dont les premiers pas ne doivent beaucoup aux vieux « cabinets de curiosités ». Nous avons vu, de même, les petites joies de l’antiquaille figurer au berceau de plus d’une orientation d’études qui, peu à peu, s’est chargée de sérieux. Telles la genèse de l’archéologie et, plus près de nous, du folklore. Les lecteurs d’Alexandre Dumas ne sont peut‑être que des historiens en puissance, auxquels manque seulement d’avoir été dressés à se donner un plaisir plus pur et, à mon gré, plus aigu ; celui de la couleur vraie.\par
Que, d’autre part, ce charme soit bien loin de s’éteindre, une fois l’en­quête méthodique abordée, avec ses nécessaires austérités ; qu’alors au contraire – tous les véritables historiens peuvent en témoigner – il gagne encore en vivacité et en plénitude ; il n’y a rien là, en un sens, qui ne vaille pour n’importe quel travail de l’esprit. L’histoire, pourtant, on n’en saurait douter, a ses jouissances esthétiques propres, qui ne ressemblent à celles d’aucune autre discipline. C’est que le spectacle des activités humaines, qui forme son objet particulier, est, plus que tout autre, fait pour séduire l’imagination des hommes. Surtout lorsque, grâce à leur éloignement dans le temps ou l’espace, leur déploiement se pare des subtiles séductions de l’étrange. Le grand Leibniz lui-même nous en a laissé l’aveu : quand, des abstraites spéculations de la mathématique ou de la théodicée, il passait au déchiffrement des vieilles chartes ou les vieilles chroniques de l’Allemagne impériale, il éprouvait, tout comme nous, cette « volupté d’apprendre des choses singulières ». Gardons‑nous de retirer à notre science sa part de poésie. Gardons‑nous surtout, comme j’en ai surpris le sentiment chez certains, d’en rougir. Ce serait une éton­nante sottise de croire que, pour exercer sur la sensibilité un si puissant appel, elle doive être moins capable de satisfaire aussi notre intelligence.\par

\astermono

\noindent  \phantomsection
\label{pXII} Si l’histoire, néanmoins, vers laquelle nous porte ainsi un attrait pres­que universellement ressenti, n’avait que lui pour se justifier ; si elle n’était, en somme, qu’un aimable passe‑temps, comme le bridge ou la pêche à la ligne, vaudrait‑elle toute la peine que nous prenons pour l’écrire ? Pour l’écrire, j’entends, honnêtement, véridiquement et en allant, autant que faire se peut, vers les ressorts cachés ; par suite difficilement. Le jeu, a écrit André Gide, a cessé aujourd’hui de nous être permis : fût‑ce, ajoutait‑il, ceux de l’intelligence. Cela était dit en 1938. En 1942, où j’écris à mon tour, combien le propos se charge‑t‑il encore d’un sens plus lourd ! À coup sûr, dans un monde qui vient d’aborder la chimie de l’atome et commence seulement à sonder le secret des espaces stellaires, dans notre pauvre monde qui, justement fier de sa science, n’arrive pour­tant pas à se créer un peu de bonheur, les longues minuties de l’érudition historique, fort capables de dévorer toute une vie, mériteraient d’être condamnées comme un gaspillage de forces absurde au point d’être cri­minel, si elles ne devaient aboutir qu’à enrober d’un peu de vérité un de nos délassements. Ou il faudra déconseiller la pratique de l’histoire à tous les esprits susceptibles de mieux s’employer ailleurs ; ou c’est comme connaissance que l’histoire aura à prouver sa bonne conscience.\par
Mais ici une nouvelle question se pose : qu’est‑ce, au juste, qui fait la légitimité d’un effort intellectuel ?\par
Personne, j’imagine, n’oserait plus dire aujourd’hui, avec les positi­vistes de stricte observance, que la valeur d’une recherche se mesure, en tout et pour tout, à son aptitude à servir l’action. L’expérience ne nous a pas seulement appris qu’il est impossible de décider à l’avance si les spéculations en apparence les plus désintéressées ne se révéleront pas, un jour, étonnement secourables à la pratique. Ce serait infliger à l’humanité une étrange mutilation que de lui refuser le droit de chercher, en dehors de tout souci de bien-être, l’apaisement de ses faims intellec­tuelles. L’histoire dût‑elle être éternellement indifférente à l’homo \emph{faber} ou \emph{politicus} qu’il lui suffirait, pour sa défense, d’être reconnue comme nécessaire au plein épanouissement de l’\emph{homo sapiens.} Cependant, même ainsi bornée, la question n’est pas, pour cela, d’emblée résolue.\par
Car la nature de notre entendement le porte beaucoup moins à vouloir savoir qu’à vouloir comprendre. D’où il résulte que les seules sciences authentiques sont, à son gré, celles qui réussissent à établir entre les phénomènes des liaisons explicatives. Le reste n’est, selon l’expression de Malebranche, que « polymathie ». Or la polymathie peut bien faire figure de distraction ou de manie, pas plus aujourd’hui qu’au temps de Malebranche, elle ne saurait passer pour une des bonnes œuvres de l’intel­ligence. Indépendamment même de toute éventualité d’application à la conduite, l’histoire n’aura donc le droit de revendiquer sa place parmi  \phantomsection
\label{pXIII} les connaissances vraiment dignes d’effort, seulement dans la mesure où, au lieu d’une simple énumération, sans liens et quasiment sans limite, elle nous promettra un classement rationnel et une progressive intelli­gibilité.\par
Il n’est point niable, pourtant, qu’une science nous paraîtra toujours avoir quelque chose d’incomplet si elle ne doit pas, tôt ou tard, nous aider à mieux vivre. Comment, en particulier, n’éprouverions‑nous pas ce sentiment avec beaucoup de force envers l’histoire, d’autant plus clairement destinée, croirait‑on, à travailler au profit de l’homme qu’elle a l’homme même et ses actes pour matière ? En fait, un vieux penchant, auquel on supposera, au moins, une valeur d’instinct, nous incline à lui demander les moyens de guider notre action ; par suite, à nous indigner contre elle, comme le soldat vaincu dont je rappelais le propos, si, d’aven­ture, elle semble manifester son impuissance à les fournir. Le problème de l’utilité de l’histoire, au sens étroit, au sens « pragmatique » du mot utile, ne se confond pas avec celui de sa légitimité, proprement intel­lectuelle. Il ne peut, d’ailleurs, venir qu’en second : pour agir raison­nablement, ne faut‑il pas d’abord comprendre ? Mais, sous peine de ne répondre qu’à demi aux suggestions les plus impérieuses de sens commun, ce problème‑là non plus ne saurait être éludé.\par

\astermono

\noindent À ces questions, certains, parmi nos conseillers ou qui voudraient l’être, ont déjà répondu. Ç’a été pour rabrouer nos espérances. Les plus indulgents ont dit : l’histoire est sans profit comme sans solidité. D’autres, dont la sévérité ne s’embarrasse pas de demi-mesures : elle est pernicieuse. « Le produit le plus dangereux que la chimie de l’intellect ait élaboré » : ainsi a prononcé l’un d’eux et non des moins notoires. Ces condamnations ont un redoutable attrait : elles justifient, d’avance, l’ignorance. Heureu­sement pour ce qui subsiste encore chez nous de curiosité d’esprit, elles ne sont peut‑être pas sans appel.\par
Mais, si le débat doit être reconsidéré, il importe que ce soit sur des données plus sûres.\par
Car il est une précaution dont les détracteurs ordinaires de l’histoire ne semblent pas s’être avisés. Leur parole ne manque ni d’éloquence, ni d’esprit. Mais ils ont, pour la plupart, omis de s’informer exactement de ce dont ils parlent. L’image qu’ils se font de nos études n’a pas été prise dans l’atelier. Elle sent l’oratoire Académie plutôt que le cabinet de travail. Elle est surtout périmée. En sorte qu’il se pourrait que tant de verve se soit, au bout du compte, dépensée à n’exorciser qu’un fan­tasme. Notre effort, ici, doit être bien différent. Les méthodes dont nous chercherons à peser le degré de certitude seront celles dont use, réellement,  \phantomsection
\label{pXIV} la recherche, jusque dans l’humble et délicat détail de ses techniques. Nos problèmes seront les problèmes mêmes qu’à l’Historien impose, quo­tidiennement, sa matière. En un mot, on voudrait, avant tout, dire com­ment et pourquoi un historien pratique son métier. Affaire au lecteur de décider, ensuite, si ce métier mérite d’être exercé.\par
Faisons‑y bien attention, pourtant. Ce n’est qu’en apparence que, même ainsi comprise et limitée, la tâche peut passer pour simple. Elle le serait, peut‑être, si nous nous trouvions en présence d’un de ces arts d’application dont on a rendu un compte suffisant lorsqu’on en a énuméré, les uns après les autres, les tours de main, longuement éprouvés. Mais l’histoire n’est pas l’horlogerie ou l’ébénisterie. Elle est un effort vers le mieux connaître : par suite une chose en mouvement. Se borner à décrire une science telle qu’elle se fait sera toujours la trahir un peu. Il est encore plus important de dire comment elle espère réussir progres­sivement à se faire. Or, de la part de l’analyste, une pareille entreprise exige forcément une assez large dose de choix personnel. Toute science, en effet, est, à chacune de ses étapes, constamment traversée par des tendances divergentes, qu’il n’est guère possible de départager sans une sorte d’anticipation sur l’avenir. On ne compte pas reculer ici devant cette nécessité. En matière intellectuelle, pas plus qu’en aucune autre, l’horreur des responsabilités n’est un sentiment bien recommandable. Cependant, il n’était qu’honnête d’avertir le lecteur.\par
Aussi bien les difficultés auxquelles se heurte inévitablement toute étude des méthodes varient‑elles beaucoup selon le point que chaque discipline se trouve avoir momentanément atteint sur la courbe, toujours un peu saccadée, de son développement. Il y a cinquante ans, quand Newton régnait encore en maître, il était, j’imagine, singulièrement plus aisé qu’aujourd’hui de construire, avec une rigueur d’épure, un exposé de la mécanique. Mais l’histoire en est encore à une phase bien plus favo­rable aux certitudes.\par
Car l’histoire n’est pas seulement une science en marche. C’est aussi une science dans l’enfance : comme toutes celles qui, pour objet ont l’esprit humain, ce tard‑venu dans le champ de la connaissance ration­nelle. Ou, pour mieux dire, vieille sous la forme embryonnaire du récit, longtemps encombrée de fictions, plus longtemps encore attachée aux événements les plus immédiatement saisissables, elle est, comme entre­prise raisonnée d’analyse, toute jeune. Elle peine à pénétrer, enfin, au-­dessous des faits de surface ; à rejeter, après les séductions de la légende ou de la rhétorique, les poisons, aujourd’hui plus dangereux, de la routine érudite et de l’empirisme déguisé en sens commun. Elle n’a pas encore dépassé, sur quelques‑uns des problèmes essentiels de sa méthode, les premiers tâtonnements. Et c’est pourquoi Fustel de Coulanges et, déjà avant lui, Bayle n’avaient sans doute pas tout à fait tort qui la disaient « la plus difficile de toutes les sciences ».\par

\astermono

\noindent  \phantomsection
\label{pXV} Est‑ce une illusion, cependant ? Si incertaine que demeure, sur tant de points, notre route, nous sommes, me semble‑t‑il, à l’heure présente mieux placés que nos prédécesseurs immédiats pour y voir un peu clair.\par
Les générations qui sont venues juste avant la nôtre, dans les dernières décades du XIX\textsuperscript{ᵉ} siècle et jusqu’aux premières années du XX\textsuperscript{ᵉ}, ont vécu comme hallucinées par une image très rigide, une image vraiment contienne des sciences du monde physique. Étendant à l’ensemble des acquisitions de l’esprit ce schéma prestigieux, il leur semblait donc ne pouvoir exister de connaissance authentique qui ne dût aboutir, par des démonstrations d’emblée irréfutables, à des certitudes formulées sous l’aspect de lois impérieusement universelles. C’était là une opinion à peu près unanime. Mais, appliquée aux études historiques, elle donna nais­sance, selon les tempéraments, à deux tendances opposées.\par
Les uns crurent possible, en effet, d’instituer une science de l’évolution humaine, qui se conformât à cet idéal en quelque sorte pan‑scientifique et ils travaillèrent de leur mieux à l’établir : quitte, d’ailleurs, à prendre leur parti de laisser finalement en dehors des atteintes de cette connais­sance des hommes beaucoup de réalités très humaines, mais qui leur paraissaient désespérément rebelles à un savoir rationnel. Ce résidu, c’était ce qu’ils appelaient, dédaigneusement, l’événement ; c’était aussi une bonne part de la vie la plus intimement individuelle. Telle fut, en somme, la position de l’école sociologique fondée par Durkheim. Du moins, si l’on ne tient pas compte des assouplissements qu’à la première raideur des principes nous vîmes peu à peu apportés par des hommes trop intel­ligents pour ne pas subir, fût‑ce malgré eux, la pression des choses. À ce grand effort, nos études doivent beaucoup. Il nous a appris à analyser plus en profondeur, à serrer de plus près les problèmes, à penser, oserais‑je dire, à moins bon marché. Il n’en sera parlé ici qu’avec infiniment de reconnaissance et de respect. S’il semble aujourd’hui dépassé, c’est pour tous les mouvements intellectuels, tôt ou tard, la rançon de leur fécondité.\par
D’autres chercheurs, cependant, prirent, au même moment, une attitude bien différente. Ne réussissant pas à insérer l’histoire dans les cadres du légalisme physique, particulièrement préoccupés, au surplus, en raison de leur éducation première, par les difficultés, les doutes, les fréquents recommencements de la critique documentaire, ils puisèrent dans ces constatations, avant tout, une leçon d’humilité désabusée. La discipline à laquelle ils vouaient leurs talents ne leur parut, au bout de compte capable ni dans le présent de conclusions bien assurées, ni dans le futur de beaucoup de perspectives de progrès. Ils inclinèrent à voir en elle plutôt qu’une connaissance vraiment scientifique, une sorte de jeu esthétique ou, au moins, d’exercice d’hygiène favorable à la santé de l’esprit. On les a nommés, parfois, « historiens historisants » : sobriquet injurieux  \phantomsection
\label{pXVI} à notre corporation, puisqu’il semble faire tenir l’essence de l’histoire dans la négation même de ses possibilités. Pour ma part, je leur trouverais volontiers, dans le moment de la pensée française auquel ils se rattachent, un signe de ralliement plus expressif.\par
L’aimable et fuyant Sylvestre Bonnard, si l’on s’en tient aux dates que le livre fixe à son activité, est un anachronisme : tout comme ces saints antiques que les écrivains du Moyen Âge peignaient, naïvement, sous les couleurs de leur propre temps. Sylvestre Bonnard (pour peu qu’on veuille bien supposer, un instant, à cette ombre inventée une exis­tence selon la chair) le « vrai » Sylvestre Bonnard, né sous le Premier Empire – la génération des grands historiens romantiques, l’eût encore compté parmi les siens : il en aurait partagé les enthousiasmes touchants et féconds, la foi un peu candide dans l’avenir de la « philosophie » de l’histoire. Négligeons l’époque à laquelle il est censé avoir appartenu et rendons‑le à celle qui vit écrire sa vie imaginaire : il méritera de figurer comme le patron, comme le saint corporatif de tout un groupe d’historiens, qui furent à peu près les contemporains intellectuels de son biographe : travailleurs profondément honnêtes, mais de souffle un peu court et dont on croirait parfois que, pareils aux enfants dont les pères se sont trop amusés, ils portaient dans leurs os la fatigue des grandes orgies historiques du romantisme ; disposés à se faire assez petits devant leurs confrères du laboratoire ; plus désireux, en somme, de nous conseiller la prudence que l’élan. Leur devise, serait‑il trop malicieux de la chercher dans ce mot étonnant, échappé un jour à l’homme d’intelligence si vive que fut pourtant mon cher maître Charles Seignobos : « Il est très utile de se poser des questions, \emph{mais très dangereux d’y répondre. »} ? Ce n’est pas là, assurément, le propos d’un fanfaron. Mais si les physiciens n’avaient fait davantage profession d’intrépidité, où en serait la physique ?\par
Or notre atmosphère mentale n’est plus la même. La théorie cinétique des gaz, la mécanique einsteinienne, la théorie des quanta ont profon­dément altéré l’idée qu’hier encore chacun se formait de la science. Elles ne l’ont pas amoindrie. Mais elles l’ont assouplie. Au certain, elles ont substitué, sur beaucoup de points, l’infiniment probable ; au rigoureu­sement mesurable, la notion de l’éternelle relativité de la mesure. Leur action s’est fait sentir même sur les esprits innombrables – je dois, hélas ! me ranger parmi eux – auxquels les faiblesses de leur intelligence ou de leur éducation interdisent de suivre, autrement que de très loin et en quelque sorte par reflet, cette grande métamorphose. Nous sommes donc, désormais, beaucoup mieux préparés à admettre que, pour ne pas s’avérer capables de démonstrations euclidiennes ou d’immuables lois de répéti­tion, une connaissance puisse, néanmoins, prétendre au nom de scienti­fique. Nous acceptons beaucoup plus aisément de faire de la certitude et de l’universalisme une question de degré. Nous ne nous sentons plus l’obligation de chercher à imposer à tous les objets du savoir un modèle  \phantomsection
\label{pXVII} intellectuel uniforme, emprunté aux sciences de la nature physique ; puisque, là même, ce gabarit a cessé de s’appliquer tout entier. Nous ne savons pas encore très bien ce que seront un jour les sciences de l’homme. Nous savons que pour être – tout en continuant, cela va de soi, d’obéir aux règles fondamentales de la raison – elles n’auront pas besoin de renoncer à leur originalité, ni d’en avoir honte.\par
J’aimerais que, parmi les historiens de profession, les jeunes, en parti­culier, s’habituassent à réfléchir sur ces hésitations, ces perpétuels « re­pentirs » de notre métier. Ce sera pour eux la plus sûre manière, de se préparer, par un choix délibéré, à conduire raisonnablement leur effort. Je souhaiterais surtout les voir venir, de plus en plus nombreux, à cette histoire à la fois élargie et poussée en profondeur, dont nous sommes plusieurs – nous‑mêmes, chaque jour moins rares – à concevoir le dessein. Si mon livre peut les y aider, j’aurai le sentiment qu’il n’aura pas été absolument inutile. Il y a en lui, je l’avoue, une part de programme.\par
Mais je n’écris pas uniquement ni même, surtout, pour l’usage intérieur de l’atelier. Aux simples curieux, non plus, je n’ai pas pensé qu’il fallût rien cacher des irrésolutions de notre science. Elles sont notre excuse. Mieux encore : elles font la fraîcheur de nos études. Nous n’avons pas seulement le droit de réclamer, en faveur de l’histoire, l’indulgence qui est due à tons les commencements. L’inachevé, s’il tend perpétuellement à se dépasser, a, pour tout esprit un peu ardent, une séduction qui vaut bien celle de la plus parfaite réussite. Le bon laboureur, – a dit, ou à peu près, Péguy, – aime le labour et les semailles autant que les moissons.\par

\astermono

\noindent Il convient que ces quelques mots d’introduction s’achèvent par une confession personnelle. Chaque science, prise isolément, ne figure jamais qu’un fragment de l’universel mouvement vers la connaissance. J’ai déjà eu l’occasion d’en donner un exemple plus haut : pour bien entendre et apprécier ses procédés d’investigation, fût‑ce en apparence les plus par­ticuliers, il serait indispensable de savoir les relier, d’un trait parfaitement sûr, à l’ensemble des tendances qui se manifestent, au même montent, dans les autres ordres de discipline. Or cette étude des méthodes pour elles‑mêmes constitue, à sa façon, une spécialité, dont les techniciens se nomment philosophes. C’est un titre auquel il m’est interdit de pré­tendre. À cette lacune de ma formation première, l’essai que voici perdra sans doute beaucoup, en précision de langage comme en largeur d’hori­zon. Je ne puis le présenter que pour ce qu’il est : le mémento d’un artisan, qui a toujours aimé à méditer sur sa tâche quotidienne, le carnet d’un compagnon, qui a longuement manié la toise et le niveau, sans pour cela se croire mathématicien.
\section[{Chapitre premier. L’histoire, les hommes et le temps}]{Chapitre premier. \\
L’histoire, les hommes et le temps}\renewcommand{\leftmark}{Chapitre premier. \\
L’histoire, les hommes et le temps}

\subsection[{I. Le choix de l’historien}]{I. Le choix de l’historien}
\noindent  \phantomsection
\label{p1} Le mot d’histoire est un très vieux mot, si vieux qu’on s’en est parfois lassé. Rarement, il est vrai, on est allé jusqu’à vouloir le rayer entièrement du vocabulaire. Les sociologues de l’école durckheimienne eux‑mêmes lui font place. Mais c’est pour le reléguer dans un pauvre petit coin des sciences de l’homme : sorte d’oubliettes où, réservant à la sociologie tout ce qui leur paraît susceptible d’analyse rationnelle, ils précipitent les faits humains jugés, à la fois, les plus superficiels et les plus fortuits.\par
Nous lui garderons ici, au contraire, sa signification la plus large. Il n’interdit, à l’avance, aucune direction d’enquête, qu’elle doive se tourner de préférence vers l’individu ou la société, vers la description des crises momentanées ou la poursuite des éléments les plus durables ; il ne renferme en lui-même aucun credo – il n’engage, selon son étymologie première, à rien d’autre qu’à la « recherche ». Assurément, depuis qu’il est apparu, voici plus de deux millénaires, sur les lèvres des hommes, il a beaucoup changé de contenu. C’est le sort, dans le langage, de tous les termes vrai­ment vivants. Si les sciences devaient, à chacune de leurs conquêtes, se chercher une appellation nouvelle – au royaume des académies que de baptêmes, et de pertes de temps !\par
À demeurer paisiblement fidèle à son glorieux nom hellène, notre histoire ne sera point, pour autant, tout à fait celle qu’écrivait Hécatée de Milet ; pas plus que la physique de Lord Kelvin ou de Langevin n’est celle d’Aristote. Qu’est‑elle cependant ?\par
En tête de ce livre, centré autour des problèmes réels de la recherche, il n’y aurait aucun intérêt à dresser une longue et raide définition. Quel travailleur sérieux s’est jamais embarrassé de pareils articles de foi ? Leur méticuleuse précision ne laisse pas seulement échapper le meilleur de tout élan intellectuel : entendez, ce qu’il y a en lui de simples velléités  \phantomsection
\label{p2} d’élan vers un savoir encore mal déterminé, de puissance d’extension. Leur pire danger est de ne définir si soigneusement que pour mieux déli­miter. « Ce sujet », dit le Gardien des Dieux termes, ou cette façon de le traiter, voilà sans doute qui peut séduire. Mais prends garde, ô éphèbe : ce n’est pas de l’Histoire. » Sommes‑nous donc une jurande de l’ancien temps pour codifier les tâches permises aux gens du métier ? et sans doute, la liste une fois close, en réserver l’exercice à nos maîtres patentés \footnote{Le Français anti-historien : Cournot, \emph{Souvenirs}, p. 43, au sujet de l’absence de tout sentiment royaliste à la fin de l’Empire : « … Pour l’explication du fait singulier qui nous occupe, je crois qu’il faut aussi tenir compte du peu de popu­larité de notre histoire et du faible développement qu’a pris chez nous, dans les classes inférieures, par suite de causes qu’il serait trop long d’analyser, le sen­timent de la tradition historique. »} ? Les physiciens et les chimistes sont plus sages – que nul, à ma connais­sance, n’a jamais vu se quereller sur les droits respectifs de la physique, de la chimie, de la chimie physique ou – à supposer que ce terme existe – de la physique chimique.\par
Il n’en est pas moins vrai que, face à l’immense et confuse réalité, l’historien est nécessairement amené à y découper le point d’application particulier de ses outils ; par suite, à faire en elle un choix qui, de toute évidence, ne sera pas le même que celui du biologiste par exemple ; qui sera proprement un choix d’historien. Ceci est un authentique pro­blème d’action. Il nous suivra tout le long de notre étude.
\subsection[{II. L’histoire et les hommes}]{II. L’histoire et les hommes}
\noindent On a dit quelquefois : « l’Histoire est la science du passé ». C’est à mon sens mal parler.\par
Car d’abord, l’idée même que le passé, en tant que tel, puisse être objet de science est absurde. Des phénomènes qui n’ont d’autre caractère com­mun que de ne pas avoir été nos contemporains, comment sans décantage préalable, en ferait‑on la matière d’une connaissance rationnelle ? Imagine-­t‑on, en pendant, une science totale de l’Univers dans son état présent ?\par
Sans doute, aux origines de l’historiographie, les vieux annalistes ne s’embarrassaient guère de ces scrupules. Ils racontaient, pêle‑mêle, des événements dont le seul lien était de s’être produits vers le même moment : les éclipses, les chutes de grêle, l’apparition d’étonnants météores avec les batailles, les traités, les morts des héros et des rois. Mais dans cette première mémoire de l’humanité, confuse comme une perception de petit enfant, un effort soutenu d’analyse a, peu à peu, opéré le classement nécessaire. Il est vrai : le langage, foncièrement traditionaliste, garde volontiers le nom d’histoire à toute étude d’un changement dans la durée… L’habitude est sans danger, parce qu’elle ne trompe personne. Il y a en ce sens, une histoire du système solaire, puisque les astres qui le com­posent n’ont pas toujours été tels que nous les voyons. Elle est du ressort de l’astronomie. Il y a une histoire des éruptions volcaniques qui est, j’en suis sûr, du plus vif intérêt pour la physique du globe. Elle n’appar­tient pas à l’histoire des historiens.\par
 \phantomsection
\label{p3} Ou du moins, elle ne lui appartient que dans la mesure où, peut ‑être, ses observations, par quelque biais, se trouveraient rejoindre les préoc­cupations spécifiques de notre histoire à nous. Comment s’établit donc, en pratique, le partage des tâches ? Un exemple le fera mieux saisir, sans doute, que beaucoup de discours.\par

\astermono

\noindent Au X\textsuperscript{ᵉ} siècle de notre ère, un golfe profond, le Zwin, endentait la côte flamande. Puis, il s’ensabla. À quelle section de la connaissance porter l’étude de ce phénomène ? D’emblée, chacun désignera la géologie. Méca­nisme de l’alluvionnement, rôle des courants marins, changements, peut-être, dans le niveau des océans : n’a‑t‑elle pas été créée et mise au monde pour traiter de tout cela ? Assurément. À y regarder de près, pourtant, les choses ne sont pas tout à fait aussi simples.\par
S’agit‑il, d’abord, de scruter les origines de la transformation ? Voici déjà notre géologue contraint de se poser des questions qui ne sont plus strictement de son obédience. Car, sans doute, le colmatage fut‑il, au moins, favorisé par des constructions de digues, des détournements de chenaux, des dessèchements : autant d’actes de l’homme, nés de besoins collectifs, et que, seule, une certaine structure sociale rendit possibles.\par
À l’autre bout de la chaîne, nouveau problème : celui des conséquences. À peu de distance du fond du golfe, une ville s’élevait. C’était Bruges. Elle communiquait avec lui par un bref trajet de rivière. Par les eaux du Zwin, elle recevait ou expédiait la plus grande part des marchandises qui faisaient d’elle, toutes proportions gardées, le Londres ou le New-York de ce temps. Vinrent, chaque jour plus sensibles, les progrès du comble­ment. Bruges eut beau, à mesure que reculait la surface inondée, pousser plus loin vers l’embouchure ses avants‑ports : ses quais peu à peu s’endor­mirent. Certes, telle ne fut point, à beaucoup près, la cause unique de son déclin. Le physique agit‑il jamais sur le social sans que son action soit préparée, aidée ou permise par d’autres facteurs qui, eux, viennent déjà de l’homme ? Mais, dans le train des ondes causales, cette cause-là compte du moins, on n’en saurait douter, parmi les plus efficaces.\par
Or, l’œuvre d’une société, remodelant selon ses besoins le sol sur lequel elle vit, est, chacun le sent d’instinct, un fait éminemment « historique ». De même, les vicissitudes d’un puissant foyer d’échanges ; par un exemple bien caractéristique de la topographie du savoir, voilà donc, d’une part, un point de chevauchement, où l’alliance de deux disciplines se révèle indispensable à toute tentative d’explication ; de l’autre, un point de passage où, lorsqu’il a été rendu compte d’un phénomène et que ses effets seuls, désormais, sont en balance, il est en quelque sorte définitivement cédé par une discipline à une autre. Que s’est‑il produit, chaque fois, qui ait semblé appeler impérieusement l’intervention de l’histoire ? C’est que l’humain a fait son apparition.\par
 \phantomsection
\label{p4} Il y a longtemps, en effet, que nos grands aînés, un Michelet, un Fustel de Coulanges nous avaient appris à le reconnaître : l’objet de l’histoire est par nature l’homme \footnote{Fustel de Coulanges, Leçon d’ouverture de 1862, dans \emph{Revue de Synthèse historique}, t. II, 1901, p. 243 ; Michelet, cours de l’École Normale, 1829, cité par G. Monod, \emph{La Vie et la Pensée de Jules Michelet}, t. I, p. 127 : « Nous nous occuperons à la fois de l’étude de l’homme individuel, et ce sera la philosophie – et de l’étude de l’homme social, et ce sera l’histoire. » – Il convient d’ajouter que Fustel, plus tard, a dit dans une formule plus serrée et plus pleine dont le développement qu’on vient de lire ne fait guère, en somme, que donner un com­mentaire : « L’histoire n’est pas l’accumulation des événements de toute nature qui se sont produits dans le passé. Elle est la science des sociétés humaines. » – Mais c’est peut‑être réduire à l’excès, dans l’histoire, la part de l’individu ; l’homme en société et les sociétés ne sont pas deux notions exactement équivalentes.}. Disons mieux : les hommes. Plutôt que le singu­lier, favorable à l’abstraction, le pluriel, qui est le mode grammatical de la relativité, convient à une science du divers. Derrière les traits sensibles du paysage, les outils ou les machines, derrière les écrits en apparence les plus glacés et les institutions en apparence les plus complètement détachées de ceux qui les ont établies, ce sont les hommes que l’histoire veut saisir \footnote{« Pas l’homme encore une fois, jamais l’homme. Les sociétés humaines, les groupes organisés », Lucien Febvre, \emph{La Terre et l’évolution humaine}, p. 201.}. Qui n’y parvient pas, ne sera jamais, au mieux, qu’un manœuvre de l’érudition. Le bon historien, lui, ressemble à l’ogre de la légende. Là où il flaire la chair humaine, il sait que là est son gibier.\par

\astermono

\noindent Du caractère de l’histoire comme connaissance des hommes découle sa position particulière vis‑à‑vis du problème de l’expression. Est‑elle « science » ou « art » ? Là‑dessus nos arrière-grands‑pères, aux environs de 1800, aimaient à disserter gravement. Plus tard, vers les années 1890, baignées dans une atmosphère de positivisme un peu rudimentaire, on put voir des spécialistes de la méthode s’indigner que, dans les travaux historiques, le public attachât une importance, à leur gré excessive, à ce qu’ils appelaient « la forme ». Art contre science, forme contre fond : autant de querelles bonnes à remiser dans les sacs à procès de la scolas­tique !\par
Il n’y a pas moins de beauté dans une exacte équation que dans une phrase juste. Mais chaque science a son esthétique de langage, qui lui est propre. Les faits humains sont, par essence, des phénomènes très délicats, dont beaucoup échappent à la mesure mathématique. Pour bien les traduire, par suite pour bien les pénétrer (car comprend‑on jamais parfaitement ce qu’on ne sait dire ?), une grande finesse de langage, une juste couleur dans le ton verbal sont nécessaires. Là où calculer est impos­sible, suggérer s’impose. Entre l’expression des réalités du monde physique et celle des réalités de l’esprit humain, le contraste est, en somme, le même qu’entre la tâche de l’ouvrier fraiseur et celle du luthier : tous deux tra­vaillent au millimètre ; mais le fraiseur use d’instruments mécaniques de précision ; le luthier se guide, avant tout, sur la sensibilité de l’oreille et des doigts. Il ne serait bon ni que le fraiseur se contentât de l’empirisme du luthier, ni que le luthier prétendit singer le fraiseur. Niera‑t‑on qu’il n’y ait, comme de la main, un tact des mots ?
\subsection[{III. Le temps historique}]{III. Le temps historique}
\noindent « Science des hommes », avons‑nous dit. C’est encore beaucoup trop vague. Il faut ajouter : « des hommes dans le temps ». L’historien ne pense  \phantomsection
\label{p5} pas seulement « humain ». L’atmosphère où sa pensée respire naturelle­ment est la catégorie de la durée.\par
Certes, on imagine difficilement qu’une science, quelle qu’elle soit, puisse faire abstraction du temps. Cependant, pour beaucoup d’entr’elles, qui, par convention, le morcellent en fragments artificiellement homogènes, il ne représente guère plus qu’une mesure. Réalité concrète et vivante rendue à l’irréversibilité de son élan, le temps de l’histoire, au contraire, est le plasma même où baignent les phénomènes et comme le lieu de leur intelligibilité. Le nombre de secondes, d’années ou de siècle qu’un corps radioactif exige pour se muer en d’autres corps est, pour l’atomistique, une donnée fondamentale. Mais que telle ou telle de ces métamorphoses ait eu lieu il y a mille ans, hier ou aujourd’hui ou qu’elle doive se produire demain, cette considération intéresserait sans doute le géologue, parce que la géologie est, à sa façon, une discipline historique, elle laisse le physicien parfaitement froid. Aucun historien, en revanche, ne se satisfera de constater que César mit huit ans pour conquérir la Gaule ; qu’il fallut quinze ans à Luther pour que de l’orthodoxe novice d’Erfurt sortît le réformateur de Wittemberg. Il lui importe encore bien davantage d’assigner à la Conquête de la Gaule son exacte place chrono­logique dans les vicissitudes des sociétés européennes ; et, sans nier le moins du monde ce qu’une crise d’âme comme celle du frère Martin a pu contenir d’éternel, il ne croira en rendre un juste compte qu’après en avoir fixé avec précision le moment sur la courbe des destinées à la fois de l’homme, qui en fut le héros et de la civilisation qu’elle eut pour climat.\par
Or, ce temps véritable est, par nature, un continu. Il est aussi per­pétuel changement. De l’antithèse de ces deux attributs viennent les grands problèmes de la recherche historique. Celui-ci avant tout autre, qui met en cause jusqu’à la raison d’être de nos travaux. Soit deux périodes successives découpées dans la suite ininterrompue des âges. Dans quelle mesure le lien qu’établit entre elles le flux de la durée l’emportant, ou non, sur la dissemblance née de cette durée même – devra‑t‑on tenir la connaissance de la plus ancienne pour nécessaire ou superflue à l’intel­ligence de la plus récente ?
\subsection[{IV. L’idole des origines}]{IV. L’idole des origines}
\noindent Il n’est jamais mauvais de commencer par un \emph{mea culpa}. Naturellement chère à des hommes qui font, du passé leur principal sujet de recherche, l’explication du plus proche par le plus lointain a parfois dominé nos études jusqu’à l’hypnose. Sous la forme la plus caractéristique, cette idole de la tribu des historiens a un nom : c’est la hantise des origines. Dans le développement de la pensée historique, elle a eu aussi son moment de faveur particulière.\par
 \phantomsection
\label{p6} C’est Renan, je crois, qui a écrit un jour (je cite de mémoire : donc, j’en ai peur, inexactement) : « Dans toutes les choses humaines, les origines avant tout sont dignes d’étude. » Et Sainte‑Beuve avant lui : « J’épie et note avec curiosité ce qui commence. » L’idée est bien de leur temps. Le mot d’origines aussi. Aux \emph{Origines du Christianisme} ont répondu un peu plus tard les \emph{Origines de la France contemporaine.} Sans compter les épigones. Mais le mot est inquiétant, parce qu’il est équivoque.\par
Signifie‑t‑il simplement « commencements » ? Il sera à peu près clair. Sous réserve, cependant, que pour la plupart des réalités historiques, la notion même de ce point initial demeure singulièrement fuyante. Affaire de définition, sans doute. D’une définition que, malheureusement, on oublie trop aisément de donner.\par
Par origines, entendra‑t‑on au contraire les causes ? Il n’y aura alors plus d’autres difficultés que celles qui, constamment (et plus encore, sans doute, dans les sciences de l’homme) sont, par nature, inhérentes aux recherches causales.\par
Mais entre les deux sens s’établit, fréquemment, une contamination d’autant plus redoutable qu’elle n’est pas, en général, très clairement sentie. Dans le vocabulaire courant, les origines sont un commencement qui explique. Pis encore : qui suffit à expliquer. Là est l’ambiguïté, là est le danger.\par

\astermono

\noindent Il y aurait une recherche à entreprendre, des plus intéressantes sur cette obsession embryogénique si marquée dans toute préoccupation d’exégètes. « Je ne comprends pas votre émoi, avouait Barrès à un prêtre qui avait perdu la foi. Les discussions d’une poignée de savants autour de quelques mots hébreux, qu’ont‑elles à voir avec ma sensibilité ? Il suffit de l’atmosphère des églises. » Et Maurras, à son tour : « Que me font les évangiles de quatre juifs obscurs ? (« obscurs » veut dire, j’imagine, plé­béiens ; car à Mathieu, Marc, Luc et Jean, il semble difficile de ne pas reconnaître, au moins, une certaine notoriété littéraire). Ces plaisantins nous la baillent belle et Pascal ni Bossuet n’auraient assurément parlé ainsi. Sans doute peut‑on concevoir une expérience religieuse qui ne doive rien à l’histoire. Au pur déiste, une illumination intérieure suffit pour croire en Dieu. Non pour croire au Dieu des chrétiens. Car le chris­tianisme, je l’ai déjà rappelé, est par essence une religion historique : entendez, dont les dogmes primordiaux reposent sur des événements. Relisez votre \emph{Credo} : « Je crois en Jésus-Christ… qui fut crucifié sous Ponce Pilate… et ressuscita d’entre les morts le 3\textsuperscript{ᵉ} jour. » Là, les commen­cements de la foi sont aussi ses fondements.\par
Or, par une contagion sans doute inévitable, ces préoccupations qui, dans une certaine forme d’analyse religieuse, pouvaient avoir leur raison  \phantomsection
\label{p7} d’être, s’étendirent à d’autres champs de recherche, où leur légitimité était beaucoup plus contestable. Là aussi une histoire, centrée sur les naissances, fut mise au service de l’appréciation des valeurs. En scrutant les « origines » de la France de son temps, que se proposait Taine, sinon de dénoncer l’erreur d’une politique issue, à son gré, d’une fausse philo­sophie de l’homme ? Qu’il s’agît des invasions germaniques ou de la conquête normande de l’Angleterre, le passé ne fut employé si activement à expliquer le présent que dans le dessein de mieux le justifier ou le con­damner. En sorte qu’en bien des cas le démon des origines fut peut‑être seulement un avatar de cet autre satanique ennemi de la véritable histoire : la manie du jugement.\par

\astermono

\noindent Revenons cependant aux études chrétiennes. Autre chose est, pour l’inquiète conscience qui se cherche, une règle de fixer son attitude vis‑à‑vis de la religion catholique, telle qu’elle se définit quotidiennement dans nos églises ; autre chose, pour l’historien, d’expliquer, comme un fait d’obser­vation, le catholicisme du présent. Indispensable, cela va de soi, à une juste intelligence des phénomènes religieux actuels, la connaissance de leurs commencements ne suffit pas à les expliquer. Afin de simplifier le problème, renonçons même à nous demander jusqu’à quel point, sous un nom qui n’a point changé, la foi, dans sa substance, est réellement demeurée toute immuable. Si intacte qu’on suppose une tradition, il restera toujours à donner les raisons de son maintien. Raisons humaines, s’entend ; l’hypothèse d’une action providentielle échapperait à la science. La question, en un mot, n’est plus de savoir si Jésus fut crucifié, puis ressuscité. Ce qu’il s’agit désormais de comprendre, c’est comment il se fait que tant d’hommes autour de nous croient à la Crucifixion et à la Résurrection. Or la fidélité à une croyance n’est, de toute évidence, qu’un des aspects de la vie générale du groupe où ce caractère se manifeste. Elle se place comme un nœud où s’emmêlent une foule de traits convergents, soit de structure sociale, soit de mentalité. Elle pose, en un mot, tout un pro­blème de climat humain. Le chêne naît du gland. Mais chêne il devient et demeure seulement s’il rencontre des conditions de milieu favorables, lesquelles ne relèvent plus de l’embryologie.\par

\astermono

\noindent L’histoire religieuse n’a été citée ici qu’à titre d’exemple. À quelque activité humaine que l’étude s’attache, la même erreur guette les cher­cheurs d’origine : de confondre une filiation avec une explication.\par
C’est déjà, en somme l’illusion des vieux étymologistes, qui pensaient avoir tout dit quand, en regard du sens actuel, ils mettaient le plus ancien  \phantomsection
\label{p8} sens connu ; quand ils avaient prouvé, je suppose, que « bureau » a désigné, primitivement, une étoffe ou « timbre » un tambour. Comme si le gros problème n’était pas de savoir comment et pourquoi le glissement s’est opéré. Comme si, surtout, autant que son propre passé, un mot quelconque n’avait pas son rôle fixé, dans la langue, par l’état contemporain du vocabulaire : lequel, à son tour, est commandé par les conditions sociales du moment. « Bureaux », dans bureaux de ministère, veut dire une bureaucratie. Lorsque je demande des « timbres » au guichet de la poste, l’emploi que je fais ainsi du terme a exigé pour s’établir, avec l’organisation lentement élaborée d’un service postal, la transformation technique décisive pour l’avenir des échanges entre pensées humaines, qui, à l’impression d’un cachet, substitua naguère l’apposition d’une vignette gommée. Il a été rendu possible seulement parce que, spécialisées par métiers, les diffé­rentes acceptions du vieux nom se sont aujourd’hui à tel point écartées l’une de l’autre qu’aucune confusion ne risque de se produire entre le timbre que je vais coller sur mon enveloppe et celui, par exemple, dont le marchand de musique me vantera la pureté dans ses instruments.\par
« Origines du régime féodal », dit‑on. Où les chercher ? D’aucuns ont répondu « à Rome ». D’autres « en Germanie ». Les raisons de ces mirages sont évidentes. Ici ou là certains usages existaient en effet – relations de clientèle, compagnonnage guerrier, rôle de la tenure comme salaire des services – que les générations postérieures, contemporaines, en Europe, des âges dits féodaux, devaient continuer. Non, d’ailleurs, sans les modifier beaucoup. Des deux parts, surtout, des mots étaient employés – « bienfait » (\emph{beneficium}) chez les Latins, « fief » chez les Germains – dont ces générations persisteront à se servir, tout en leur conférant, peu à peu et sans s’en rendre compte, un contenu presque entièrement nouveau. Car, au grand désespoir des historiens, les hommes n’ont pas coutume, chaque fois qu’ils changent de mœurs, de changer de vocabulaire. Ce sont là, certainement, des constatations pleines d’intérêt. Croira‑t‑on qu’elles épuisent le problème des causes. La féodalité européenne, dans ses insti­tutions caractéristiques, ne fut par un archaïque tissu de survivances. Durant une certaine phase de notre passé, elle naquit de toute une ambiance sociale.\par
M. Seignobos a dit quelque part : « Je crois que les idées révolutionnaires du XVIII\textsuperscript{e} siècle… proviennent des idées anglaises du XVII\textsuperscript{ᵉ}. » Enten­dait‑il signifier par là qu’ayant lu certains écrits anglais du siècle précédent ou subissant indirectement leur influence, les publicistes français de l’époque des lumières en adoptèrent les principes politiques ? On pourra lui donner raison. À supposer du moins que dans les formules étrangères nos philosophes n’aient vraiment rien versé à leur tour d’original, comme substance intellectuelle ou comme tonalité de sentiment. Mais, même ainsi réduite, non sans beaucoup d’arbitraire, à un fait d’emprunt, l’histoire de ce mouvement de pensée sera loin d’être complètement éclaircie. Car  \phantomsection
\label{p9} le problème subsistera toujours de savoir pourquoi la transmission s’opéra à la date indiquée : ni plus tôt, ni plus tard. Une contagion suppose deux choses : des générations de microbes et, à l’instant où le mal prend, un « terrain ».\par
Jamais, en un mot, un phénomène historique ne s’explique pleinement en dehors de l’étude de son moment. Cela est vrai de toutes les étapes de l’évolution. De celle où nous vivons comme des autres. Le proverbe arabe l’a dit avant nous : « Les hommes ressemblent plus à leur temps qu’à leurs pères. » Pour avoir oublié cette sagesse orientale, l’étude du passé s’est parfois discréditée.
\subsection[{V. Des limites de l’actuel et de l’inactuel}]{V. Des limites de l’actuel et de l’inactuel}
\noindent Cependant pour ne pas expliquer tout le présent, faut‑il croire que le passé soit inutile à son explication ? Le singulier est que la question, aujourd’hui, puisse se poser.\par
Jusqu’à une époque très proche de nous, en effet, elle a paru, presque unanimement, résolue d’avance. « Celui qui voudra s’en tenir au présent, à l’actuel, ne comprendra pas l’actuel », écrivait, au siècle dernier, Michelet, en tête de ce beau livre du \emph{Peuple}, tout plein pourtant des fièvres du moment. Et déjà Leibniz rangeait parmi les bienfaits qu’il attendait de l’histoire « les origines des choses présentes trouvées dans les choses passées ; car, ajoutait‑il, une réalité ne se comprend jamais mieux que par ses causes » \footnote{Préface aux \emph{Accessiones Historicae} (1700), Opéra, éd. Dutens, t. IV 2, p. 53 : « Tria sunt quae expetimus in Historia : primum, voluptatem nos cendi res sin­gulares ; deinde, utilia in primis vitae praecepta ; ac denique origines praesentium a praeteritis repetitas, cum omnia optime ex causis nos cantur. »}.\par
Mais, depuis Leibniz, depuis Michelet, un grand fait s’est produit : les révolutions successives des techniques ont démesurément élargi l’in­tervalle psychologique entre les générations. Non sans quelque raison, peut‑être, l’homme de l’âge de l’électricité ou de l’avion se sent très loin de ses ancêtres. Volontiers il en conclut, plus imprudemment, qu’il a cessé d’être déterminé par eux. Ajoutez le tour moderniste inné à toute mentalité d’ingénieur. Pour mettre en marche et réparer une dynamo, est‑il nécessaire d’avoir pénétré les idées du vieux Volta sur le galvanisme ? Par une analogie, sans nul doute boîteuse, mais qui s’impose spontanément à plus d’une intelligence soumise à la machine, on pensera de même que, pour comprendre les grands problèmes humains de l’heure et tenter de les résoudre, il ne sert à rien d’en avoir analysé les antécédents. Pris eux aussi, sans bien s’en rendre compte, dans cette atmosphère moderniste, comment les historiens n’auraient‑ils point le sentiment que, dans leur domaine également, la frontière qui sépare le récent de l’ancien ne se déplace pas d’un mouvement moins constant ? Le régime de la monnaie stable et de l’étalon or qui, hier, figurait dans tous les manuels d’économie politique, comme la norme même de l’actualité, pour l’éco­nomiste d’aujourd’hui, est‑ce encore du présent ou de l’histoire qui déjà  \phantomsection
\label{p10} sent fortement le moisi ! – Derrière ces paralogismes, cependant, il est facile de découvrir un faisceau d’idées moins inconsistantes et dont la simpli­cité, au moins apparente, a séduit certains esprits.\par

\astermono

\noindent Dans le vaste écoulement des temps, on croit pouvoir mettre à part une phase de faible étendue. Relativement peu distante de nous, à son point de départ, elle recouvre à son aboutissement les jours mêmes que nous vivons. En elle, rien, ni les caractères les plus marquants de l’état social ou politique, ni l’outillage matériel, ni la tonalité générale de la civilisation ne présentent, semble‑t‑il, de différences profondes avec le monde où nous avons nos habitudes. Elle paraît, en un mot, affectée par rapport à nous d’un coefficient très fort de contemporanéité ». D’où l’honneur, ou la tare, de ne pas être confondue avec le reste du passé. « Depuis 1830, ce n’est plus de l’histoire », nous disait un de nos profes­seurs de lycée, qui était très vieux quand j’étais très jeune, « c’est de la politique ». On ne dirait plus aujourd’hui : « depuis 1830 » – les Trois Glorieuses, à leur tour, ont pris de l’âge – ni « c’est de la politique ». Plutôt d’un ton respectueux : « de la sociologie » ; ou, avec moins de considération : « du journalisme ». Beaucoup cependant répéteraient vo­lontiers : depuis 1914 ou 1940, ce n’est plus de l’histoire. Sans d’ailleurs très bien s’entendre sur les motifs de cet ostracisme.\par
Certains, estimant que les faits les plus voisins de nous sont par là même rebelles à toute étude vraiment sereine, souhaitent seulement épargner à la chaste Clio de trop brûlants contacts. Ainsi pensait, j’imagine, mon vieux maître. C’est, assurément, nous prêter une faible maîtrise de nos nerfs. C’est aussi oublier que, dès que les résonances sentimentales entrent en jeu, la limite entre l’actuel et l’inactuel est loin de se régler nécessairement sur la mesure mathématique d’un intervalle de temps. Avait‑il si tort, mon brave proviseur qui, dans le lycée languedocien où je fis mes premières armes de professeur, m’avertissait de sa grosse voix de capitaine d’enseignement : « Ici, le dix‑neuvième siècle, ce n’est pas bien dangereux. Mais quand vous toucherez aux guerres de religion, soyez très prudent. » En vérité, qui, une fois devant sa table de travail, n’a pas la force de soustraire son cerveau aux virus du moment sera fort capable d’en laisser filtrer les toxines jusque dans un commentaire de l’Iliade ou du Ramayana.\par
D’autres savants, au contraire, jugent avec raison le présent humain parfaitement susceptible de connaissance scientifique. Mais c’est pour en réserver l’étude à des disciplines bien distinctes de celle qui a le passé pour objet. Ils analysent, par exemple, ils prétendent comprendre l’éco­nomie contemporaine à l’aide d’observations bornées, dans le temps, à quelques décades. En un mot, ils considèrent l’époque où ils vivent comme  \phantomsection
\label{p11} séparée de celles qui l’ont précédée par de trop vifs contrastes pour ne point porter en elle‑même sa propre explication. Telle est aussi l’attitude instinctive de beaucoup de simples curieux. L’histoire des périodes un peu lointaines ne les séduit que comme un inoffensif luxe de l’esprit. D’un côté une poignée d’antiquaires occupés, par macabre dilection, à démailloter les dieux morts ; de l’autre, sociologues, économistes, publi­cistes : les seuls explorateurs du vivant…
\subsection[{VI. Comprendre le présent par le passé}]{VI. Comprendre le présent par le passé}
\noindent À y regarder de près, le privilège d’auto‑intelligibilité ainsi reconnu au présent s’appuie sur une suite d’étranges postulats.\par
Il suppose d’abord que les conditions humaines ont subi, dans l’intervalle d’une ou deux générations, un changement non seulement très rapide, mais aussi total : en sorte qu’aucune institution un peu ancienne, aucune manière de se conduire traditionnelle n’auraient échappé aux révolutions du laboratoire ou de l’usine. C’est oublier la force d’inertie propre à tant de créations sociales.\par

\astermono

\noindent L’homme passe son temps à monter des mécanismes, dont il demeure ensuite le prisonnier plus ou moins volontaire. Quel observateur parcou­rant nos campagnes du Nord n’y a été frappé par l’étrange dessin des champs ? En dépit des atténuations que les vicissitudes de la propriété ont, au cours des âges, apporté au schéma primitif, le spectacle de ces lanières qui, démesurément étroites et allongées, découpent le sol arable en un nombre prodigieux de parcelles, garde encore aujourd’hui de quoi confondre l’agronome. Le gaspillage d’efforts qu’entraîne une pareille disposition, les gênes qu’elle impose aux exploitants ne sont guère contes­tables. Comment l’expliquer ? Par le Code Civil et ses inévitables effets, ont répondu des publicistes trop pressés. Modifiez donc, ajoutaient‑ils, nos lois sur l’héritage ; et vous supprimerez tout le mal. S’ils avaient mieux su l’histoire, s’ils avaient aussi mieux interrogé une mentalité paysanne formée par des siècles d’empirisme, ils auraient jugé le remède moins facile. En fait, cette armature remonte à des origines si reculées que pas un savant, jusqu’ici, n’est parvenu à en rendre un compte satis­faisant ; les défricheurs de l’âge des dolmens y sont probablement pour davantage que les légistes du Premier Empire. L’erreur sur la cause se prolongeant donc ici, comme il arrive presque nécessairement, en faute de thérapeutique, l’ignorance du passé ne se borne pas à nuire à la connaissance du présent ; elle compromet, dans le présent, l’action même.\par
 \phantomsection
\label{p12} Il y a plus. Pour qu’une société, quelle qu’elle fût, pût être déterminée tout entière par le moment immédiatement antérieur à celui qu’elle vit, il ne lui suffirait pas d’une structure si parfaitement adaptable au chan­gement qu’elle serait véritablement désossée ; il faudrait encore que les échanges entre les générations s’opérassent seulement, si j’ose dire, à la file indienne – les enfants n’ayant de contact avec leurs ancêtres que par intermédiaire des pères.\par
Or, cela n’est pas vrai, même des communications purement orales. Regardez, par exemple, nos villages. Parce que les conditions du travail y tiennent pendant presque toute la journée le père et la mère éloignés des jeunes enfants, ceux‑ci sont élevés surtout par les grands parents. À chaque nouvelle formation d’esprit un pas en arrière se fait donc qui, par dessus la génération éminemment porteuse de changements, relie les cerveaux les plus malléables aux plus cristallisés. De là vient, avant tout, n’en doutons pas, le traditionalisme inhérent à tant de sociétés paysannes. Le cas est particulièrement net. Il n’est pas unique. L’anta­gonisme naturel aux groupes d’âge s’exerçant, principalement, entre groupes limitrophes, plus d’une jeunesse a dû aux leçons des vieillards au moins autant qu’à celles des hommes mûrs.\par

\astermono

\noindent À plus forte raison, l’écrit facilite‑t‑il grandement, entre des géné­rations parfois très écartées, ces transferts de pensée qui font, au propre, la continuité d’une civilisation. Luther, Calvin, Loyola : des hommes d’autrefois, sans doute, des hommes du seizième siècle, que l’historien occupé à les comprendre et faire comprendre, aura pour premier devoir de replacer dans leur milieu, baignés par l’atmosphère mentale de leur temps, face à des problèmes de conscience qui ne sont plus exactement les nôtres. Osera‑t‑on pourtant dire qu’à la juste compréhension du monde actuel l’intelligence de la Réforme protestante ou de la Réforme catholique, éloignées de nous par un espace plusieurs fois centenaire, n’importe pas davantage que celle de beaucoup d’autres mouvements d’idée ou de sensibilité, plus proches, assurément, dans le temps, mais plus éphémères ?\par
L’erreur, en somme, est claire et sans doute pour la détruire suffit‑il de la formuler. On se représente le courant de l’évolution humaine comme fait d’une suite de brèves et profondes saccades, dont chacune ne durerait que l’espace de quelques vies. L’observation prouve, au contraire, que, dans cet immense continu, les grands ébranlements sont parfaitement capables de se propager des molécules les plus lointaines jusqu’aux plus proches. Que dirait‑on d’un géo‑physicien qui, se contentant de dénom­brer les myriamètres, estimerait l’action de la lune sur notre globe beau­coup plus considérable que celle du soleil ? Pas plus dans la durée que dans le ciel, l’efficacité d’une force ne se mesure exclusivement à la distance.\par
\bigbreak
\noindent  \phantomsection
\label{p13} Parmi les choses passées, enfin, celles mêmes – croyances disparues sans laisser la moindre trace, formes sociales avortées, techniques mortes – qui ont, semble‑t‑il, cessé de commander le présent, les tiendra‑t‑on pour inutiles à son intelligence ? Ce serait oublier qu’il n’est pas de con­naissance véritable sans un certain clavier de comparaison. À condition, il est vrai, que le rapprochement porte sur des réalités à la fois diverses et pourtant apparentées. On ne nierait guère que ce ne soit ici le cas.\par
Certes, nous n’estimons plus aujourd’hui que, comme l’écrivait Ma­chiavel, comme le pensaient Hume ou Bonald, il y ait dans le temps « au moins quelque chose d’immuable : c’est l’homme ». Nous avons appris que l’homme aussi a beaucoup changé : dans son esprit et, sans doute, jusque dans les plus délicats mécanismes de son corps. Comment en serait‑il autrement ? Son atmosphère mentale s’est profondément transformée ; son hygiène, son alimentation, non moins. Il faut bien, cependant, qu’il existe dans l’humaine nature et dans les sociétés humaines un fonds permanent. Sans quoi les noms mêmes d’homme et de société ne vou­draient rien dire. Ces hommes donc, croirons‑nous les comprendre si nous ne les étudions que dans leurs réactions devant les circonstances particulières à un moment ? Même sur ce qu’ils sont à ce moment‑là, l’expérience sera insuffisante. Beaucoup de virtualités provisoirement peu apparentes, mais qui, à chaque instant, peuvent se réveiller, beaucoup de moteurs, plus ou moins inconscients, des attitudes individuelles ou collectives demeureront dans l’ombre. Une expérience unique est toujours impuissante à discriminer ses propres facteurs ; par suite à fournir sa propre interprétation.
\subsection[{VII. Comprendre le passé par le présent}]{VII. Comprendre le passé par le présent}
\noindent Aussi bien cette solidarité des âges a‑t‑elle tant de force qu’entre eux les liens d’intelligibilité sont véritablement à double sens. L’incompré­hension du présent naît fatalement de l’ignorance du passé. Mais il n’est peut‑être pas moins vain de s’épuiser à comprendre le passé, si l’on ne sait rien du présent. J’ai déjà ailleurs rappelé l’anecdote : j’accompagnais à Stockholm, Henri Pirenne ; à peine arrivés, il me dit : « Qu’allons‑nous voir d’abord ? Il paraît qu’il y a un Hôtel de Ville tout neuf. Commen­çons par lui ». Puis, comme s’il voulait prévenir un étonnement, il ajouta : « Si j’étais un antiquaire, je n’aurais d’yeux que pour les vieilles choses. Mais je suis un historien. C’est pourquoi j’aime la vie ». Cette faculté d’appréhension du vivant, voilà bien, en effet, la qualité maîtresse de l’historien. Ne nous laissons pas tromper par certaines froideurs de style. Les plus grands parmi nous l’ont tous possédée : Fustel ou Maitland à leur façon, qui était plus austère, non moins que Michelet. Et peut‑être est‑elle, en son principe, un don des fées, que nul ne saurait prétendre acquérir s’il ne l’a trouvé en son berceau. Elle n’en a pas moins besoin d’être  \phantomsection
\label{p14} constamment exercée et développée. Comment ? sinon ainsi que Pirenne lui-même en donnait l’exemple, par un contact perpétuel avec l’aujourd’hui.\par
Car le frémissement de vie humaine, qu’il faudra tout un dur effort d’imagination pour restituer aux vieux textes, est ici directement per­ceptible à nos sens. J’avais lu bien des fois, j’avais souvent raconté des récits de guerre et de batailles. Connaissais‑je vraiment, au sens plein du verbe connaître, connaissais‑je par le dedans avant d’en avoir éprouvé moi-même l’atroce nausée, ce que sont pour une armée l’encerclement, pour un peuple la défaite ? Avant d’avoir moi-même, durant l’été et l’au­tomne 1918, respiré l’allégresse de la victoire – (en attendant, je l’espère bien, d’en regonfler une seconde fois mes poumons : mais le parfum, hélas ! ne sera plus tout à fait le même) – savais‑je vraiment ce qu’enferme ce beau mot ? À la vérité, consciemment ou non, c’est toujours à nos expériences quotidiennes que, pour les nuancer, là où il se doit, de teintes nouvelles, nous empruntons en dernière analyse les éléments qui nous servent à reconstituer le passé : les noms mêmes dont nous usons afin de caractériser les états d’âmes disparus, les formes sociales évanouies, quel sens auraient‑ils pour nous si nous n’avions d’abord vu vivre des hommes ? À cette imprégnation instinctive, mieux vaut cent fois substi­tuer une observation volontaire et contrôlée. Un grand mathématicien ne sera pas moins grand, je suppose, pour avoir traversé les yeux clos le monde où il vit. Mais l’érudit qui n’a le goût de regarder autour de lui ni les hommes, ni les choses, ni les événements, il méritera peut‑être, comme disait Pirenne, le nom d’un utile antiquaire. Il fera sagement de renoncer à celui d’historien.\par

\astermono

\noindent Au surplus, l’éducation de la sensibilité historique n’est pas toujours seule en cause. Il arrive que, dans une ligne donnée, la connaissance du présent importe plus directement encore à l’intelligence du passé.\par
L’erreur, en effet, serait grave de croire que l’ordre adopté par les historiens dans leurs enquêtes doive nécessairement se modeler sur celui des événements. Quitte à restituer ensuite à l’histoire son mouvement véritable, ils ont souvent profit à commencer par la lire, comme disait Maitland, « à rebours ». Car la démarche naturelle de toute recherche est d’aller du mieux ou du moins mal connu au plus obscur. Sans doute, il s’en faut de beaucoup que, la lumière des documents se fasse régulièrement plus vive à mesure qu’on descend le fil des âges. Nous sommes incomparablement moins bien renseignés sur le X\textsuperscript{ᵉ} siècle de notre ère, par exemple, que sur l’époque de César ou d’Auguste. Dans la majorité des cas, les périodes les plus proches n’en coïncident pas moins avec les zones de clarté relative. Ajoutez qu’à procéder mécaniquement de l’arrière à l’avant,  \phantomsection
\label{p15} on court toujours le risque de perdre son temps à pourchasser les débuts ou les causes de phénomènes qui, à l’expérience, se révèleront peut‑être imaginaires. Pour avoir omis de pratiquer, quand et où elle s’imposait, une méthode prudemment régressive, les plus illustres d’entre nous se sont par­fois abandonnés à d’étranges erreurs. Fustel de Coulanges s’est penché sur les « origines » d’institutions féodales dont il ne se formait, je le crains, qu’une image assez confuse et sur les prémices d’un servage que, mal instruit par des descriptions de seconde main, il concevait sous des couleurs tout à fait fausses.\par
Or, moins exceptionnellement sans doute qu’on ne le pense, il arrive qu’afin d’atteindre le jour, ce soit jusqu’au présent qu’il faille poursuivre. Dans quelques‑uns de ses caractères fondamentaux, notre paysage rural, on le sait déjà, date d’époques extrêmement lointaines. Mais, pour inter­préter les rares documents qui nous permettent de pénétrer cette brumeuse genèse, pour poser correctement les problèmes, pour en avoir même l’idée, une première condition a dû être remplie : observer, analyser le paysage d’aujourd’hui. Car lui seul donnait les perspectives d’ensemble, dont il était indispensable de partir. Non certes qu’il puisse s’agir, ayant immo­bilisé une fois pour toutes cette image, de l’imposer telle quelle, à chaque étape du passé, successivement rencontrée de l’aval à l’amont. Ici comme ailleurs, c’est un changement que l’historien veut saisir. Mais, dans le film qu’il considère, seule la dernière pellicule est intacte. Pour reconstituer les traits brisés des autres, force a été de dérouler, d’abord, la bobine en sens inverse des prises de vues.\par

\astermono

\noindent Il n’y a donc qu’une science des hommes dans le temps et qui sans cesse a besoin d’unir l’étude des morts à celle des vivants. Comment l’appeler ? J’ai déjà dit pourquoi l’antique nom d’histoire me paraît le plus compréhensif, le moins exclusif ; le plus chargé aussi des émouvants souvenirs d’un effort beaucoup plus que séculaire : partant, le meilleur. En proposant ainsi de l’étendre, contrairement à certains préjugés, d’ailleurs bien moins vieux que lui, jusqu’à la connaissance du présent, on ne pour­suit – faut‑il s’en défendre ? – aucune revendication corporative. La vie est trop brève, les connaissances sont trop longues à acquérir pour permettre, même au plus beau génie, une expérience totale de l’humanité. Le monde actuel aura toujours ses spécialistes, comme l’âge de pierre ou l’égyptologie. Aux uns comme aux autres, on demande simplement de se souvenir que les recherches historiques ne souffrent pas d’autarcie. Isolé, aucun d’eux ne comprendra jamais rien qu’à demi, fût‑ce à son propre champ d’études ; et la seule histoire véritable, qui ne peut se faire que par entr’aide, est l’histoire universelle.\par
 \phantomsection
\label{p16} Une science, cependant, ne se définit pas uniquement par son objet. Ses limites peuvent être fixées, tout autant, par la nature propre de ses méthodes. Reste donc à nous demander si, selon qu’on se rapproche ou s’éloigne du moment présent, les techniques mêmes de l’enquête ne de­vraient pas être tenues pour foncièrement différentes. C’est poser le problème de l’observation historique.
\section[{Chapitre II. L’observation historique}]{Chapitre II. \\
L’observation historique}\renewcommand{\leftmark}{Chapitre II. \\
L’observation historique}

\subsection[{I. Caractères généraux de l’observation historique}]{I. Caractères généraux de l’observation historique}
\noindent  \phantomsection
\label{p17} Plaçons‑nous résolument, pour commencer, dans l’étude du passé.\par
Les caractères les plus apparents de l’information historique entendue dans ce sens restreint et usuel du terme ont été maintes fois décrits. Les faits qu’il étudie, l’historien, nous dit‑on est, par définition, dans l’impos­sibilité absolue de les constater lui-même. Aucun égyptologue n’a vu Ramsès. Aucun spécialiste des guerres napoléoniennes n’a entendu le canon d’Austerlitz. Des âges qui nous ont précédés, nous ne saurions donc parler que d’après témoins. Nous sommes, à leur égard, dans la situation du juge d’instruction qui s’efforce de reconstituer un crime auquel il n’a point assisté ; du physicien qui, retenu à la chambre par la grippe, ne connaîtrait les résultats de ses expériences que grâce aux rap­ports d’un garçon de laboratoire. En un mot, par contraste avec la con­naissance du présent, celle du passé serait nécessairement « indirecte ».\par
Qu’il y ait, dans ces remarques, une part de vérité, nul ne songera à le nier. Elles demandent cependant à être sensiblement nuancées.\par

\astermono

\noindent Un chef d’armées, je suppose, vient de remporter une victoire. Sur le champ, il entreprend d’en écrire de sa main le récit. Il a conçu le plan de la bataille. Il l’a dirigée. Grâce à la médiocre étendue du terrain (car décidés à mettre tous les atouts dans notre jeu, nous imaginons une ren­contre de l’ancien temps, ramassée dans un étroit espace), il a pu voir la mêlée presque tout entière se dérouler sous ses yeux. N’en doutons point pourtant : sur plus d’un épisode essentiel, force lui sera de s’en  \phantomsection
\label{p18} remettre au rapport de ses lieutenants. En quoi, d’ailleurs, il ne fera que se conformer, devenu narrateur, à la conduite même qu’il avait tenue quelques heures plus tôt, dans l’action. Pour régler alors, de moment en moment, sur les vicissitudes du combat les mouvements de ses troupes, quelles informations croira‑t‑on qui l’auront le mieux servi : les images plus ou moins confusément entrevues à travers sa lorgnette, ou les comptes rendus qu’apportaient, bride abattue, estafettes ou aides de camp ? Rarement le conducteur d’hommes se suffit d’être à lui-même son propre témoin. Cependant, jusque dans une hypothèse aussi favorable, que reste-­t‑il donc de cette fameuse observation directe, privilège prétendu de l’étude du présent ?\par
C’est qu’en vérité elle n’est presque jamais qu’un leurre : aussitôt, du moins, que l’horizon de l’observateur s’élargit un peu. Tout recueil de choses vues est fait pour une bonne moitié de choses vues par autrui. Économiste, j’étudie le mouvement des échanges ce mois‑ci, cette semaine-­ci ; c’est à l’aide de statistiques que je n’ai pas personnellement dressées. Explorateur de l’extrême pointe de l’actuel, je m’attache à sonder l’opinion publique sur les grands problèmes de l’heure ; je pose des questions ; je note, collationne et dénombre des réponses. Que me fournissent‑elles, sinon, plus ou moins gauchement exprimée, l’image que mes interlocuteurs se forment de ce qu’ils croient penser eux‑mêmes ou celle qu’ils souhaitent me présenter de leurs pensées. Ils sont les sujets de mon expérience. Mais alors qu’un physiologiste, qui dissèque un cobaye, aperçoit de ses propres yeux la lésion ou l’anomalie cherchées, je ne connais l’état d’âme de mes « hommes de la rue » qu’à travers le tableau qu’ils acceptent eux‑mêmes de m’en fournir. Parce que, dans l’immense tissu d’événements, de gestes et de paroles dont se compose le destin d’un groupe humain, l’individu ne perçoit jamais qu’un petit coin, étroitement borné par ses sens et sa faculté d’attention ; parce qu’en outre, il ne possède jamais la conscience immédiate que de ses propres états mentaux : toute connaissance de l’humanité, quel qu’en soit, dans le temps, le point d’application, puisera toujours dans les témoignages d’autrui une grande part de sa substance. L’enquêteur du présent n’est guère là‑dessus beaucoup mieux partagé que l’historien du passé.\par

\astermono

\noindent Mais, il y a plus. L’observation du passé, même d’un passé très reculé, est‑il sûr qu’elle soit toujours à ce point « indirecte » ?\par
On voit très bien pour quelles raisons l’impression de cet éloignement entre l’objet de la connaissance et le chercheur s’est imposé avec tant de force à beaucoup de théoriciens de l’histoire. C’est qu’ils pensaient avant tout à une histoire d’événements, voire d’épisodes : je veux dire qui, à tort ou à raison (le moment n’est pas venu de l’examiner), attache une  \phantomsection
\label{p19} extrême importance à retracer exactement les actes, propos ou attitudes de quelques personnages, groupés dans une scène de durée relativement courte où se ramassent, comme dans la tragédie classique, toutes les forces de crise du moment : journée révolutionnaire, combat, entrevue diplo­matique. On a raconté que, le 2 septembre 1792, la tête de la princesse de Lamballe avait été promenée au bout d’une pique sous les fenêtres de la famille royale. Est‑ce vrai ? Est‑ce faux ? M. Pierre Caron, qui a écrit sur les Massacres un livre d’une admirable probité, n’ose se prononcer. S’il lui avait été donné de contempler, lui-même, depuis une des tours du Temple, l’affreux cortège, il saurait assurément à quoi s’en tenir. À supposer, du moins, qu’ayant gardé, comme on peut le croire, dans ces circonstances, tout son sang‑froid de savant, il eût, en outre, par une juste méfiance de sa mémoire, pris soin de noter sur le champ ses obser­vations. En pareil cas, sans nul doute, l’historien se sent, par rapport au bon témoin d’un fait présent, dans une position un peu humiliante. Il est comme à la queue d’une colonne où les avis se transmettent depuis la tête, de rang en rang. Ce n’est pas une très bonne place pour être sûre­ment renseigné. J’ai vu, naguère, durant une relève nocturne, passer ainsi, le long de la file, le cri : « Attention ! Trous d’obus à gauche ». Le dernier homme le reçut sous la forme : « Allez à gauche », fit un pas de ce côté et s’effondra.\par
Il est cependant d’autres éventualités. Dans les murs de certaines citadelles syriennes, élevées quelques millénaires avant Jésus‑Christ, les archéologues ont retrouvé de nos jours, prises en plein bloquage, des poteries pleines de squelettes d’enfants. Comme on ne saurait raisonna­blement supposer que ces ossements soient venus là par hasard, nous sommes de toute évidence en face de restes de sacrifices humains, accomplis au moment même de la construction et liés à celle‑ci. Sur les croyances qui s’expriment par ces rites, force nous sera sans doute de nous en remettre à des témoignages du temps, s’il en existe, ou de procéder par analogie à l’aide d’autres témoignages. Une foi que nous ne partageons pas, com­ment donc la connaîtrions‑nous sinon à travers les dires d’autrui ? C’est le cas, il faut le répéter, de tous les phénomènes de conscience dès qu’ils nous sont étrangers. Quant au fait même du sacrifice, par contre, notre position est bien différente. Certes, nous ne le saisissons pas, à proprement parler, d’une prise absolument immédiate ; pas plus que le géologue, l’ammonite dont il découvre le fossile. Pas plus que le physicien, le mou­vement moléculaire dont il décèle les effets dans le mouvement brownien. Mais le raisonnement très simple qui, en excluant toute autre possibilité d’explication, nous permet de passer de l’objet véritablement constaté au fait dont cet objet apporte la preuve – ce travail d’interprétation rudimentaire très voisin, en somme, des opérations mentales instinctives sans lesquelles aucune sensation ne deviendrait perception – il n’est rien chez lui qui, entre la chose et nous, ait exigé l’interposition d’un  \phantomsection
\label{p20} autre observateur. Les spécialistes de la méthode ont généralement entendu par connaissance indirecte celle qui n’atteint l’esprit de l’historien que par le canal d’esprits humains différents. Le terme n’est peut‑être pas très bien choisi ; il se borne à indiquer la présence d’un intermédiaire ; on ne voit pas pourquoi ce chaînon serait nécessairement de nature humaine. Acceptons cependant, sans disputer sur les mots, l’usage commun. En ce sens, notre connaissance des immolations murales, dans l’antique Syrie, n’a assurément rien d’indirect.\par
Or, beaucoup d’autres vestiges du passé nous offrent un accès également tout de plain‑pied. Tel est le cas, dans leur presque totalité, de l’immense masse des témoignages non écrits, celui même d’un bon nombre d’écrits. Si les théoriciens les plus connus de nos méthodes n’avaient pas manifesté envers les techniques propres de l’archéologie une aussi étonnante et superbe indifférence, s’ils n’avaient pas été, dans l’ordre documentaire, obsédés par le récit autant que dans l’ordre des faits par l’événement, sans doute les aurait‑on vus moins prompts à nous rejeter vers une obser­vation éternellement dépendante. Dans les tombes royales d’Our, en Chaldée, on a trouvé des grains de colliers faits d’amazonite. Comme les gisements les plus proches de cette pierre se placent au cœur de l’Inde ou dans les alentours du lac Baïkal, la conclusion a semblé s’imposer que, dès le troisième millénaire avant notre ère, les cités du Bas Euphrate entretenaient des relations d’échange avec des terres extrêmement loin­taines. L’induction pourra paraître bonne ou fragile. Quelque jugement qu’on porte sur elle, c’est indéniablement une induction du type le plus classique ; elle se fonde sur la constatation d’un fait et la parole d’autrui n’y intervient en rien. Mais les documents matériels ne sont pas, à beau­coup près, les seuls à posséder ce privilège de pouvoir être ainsi appréhendés de première main. Autant que le silex taillé jadis par l’artisan des âges de pierre, un trait de langage, une règle de droit incorporée dans un texte, un rite fixé par un livre des cérémonies ou représenté sur une stèle sont des réalités que nous saisissons nous‑mêmes et que nous exploitons par un effort d’intelligence strictement personnel. Aucun autre cerveau humain n’a besoin d’y être appelé comme truchement. Il n’est point vrai, pour reprendre la comparaison de tout à l’heure, que l’historien en soit néces­sairement réduit à ne savoir ce qui se passe dans son laboratoire que par les comptes rendus d’un étranger. Il n’arrive jamais qu’après l’expérience terminée. Mais, si les circonstances le favorisent, l’expérience aura laissé des résidus qu’il ne lui sera pas impossible de percevoir de ses propres yeux.\par

\astermono

\noindent C’est donc en d’autres termes, à la fois moins ambigus et plus compré­hensifs, qu’il convient de définir les indiscutables particularités de l’obser­vation historique.\par
 \phantomsection
\label{p21} Pour premier caractère, la connaissance de tous les faits humains dans le passé, de la plupart d’entre eux dans le présent, a d’être, selon l’heureuse expression de François Simiand, une connaissance par traces. Qu’il s’agisse des ossements murés dans les remparts de la Syrie, d’un mot dont la forme ou l’emploi révèle une coutume, du récit écrit par le témoin d’une scène ancienne ou récente, qu’entendons‑nous en effet par \emph{documenta} sinon une « trace », c’est‑à‑dire la marque, perceptible aux sens, qu’a laissée un phénomène en lui-même impossible à saisir ? Peu importe que l’objet originel se trouve par nature inaccessible à la sensation, comme l’atome dont la trajectoire est rendue visible dans le tube de Crookes – ou qu’il soit seulement devenu tel aujourd’hui par l’effet du temps, comme la fougère, pourrie depuis des millénaires, dont l’empreinte subsiste sur le bloc de houille ou comme les solennités tombées dans une longue désuétude que l’on voit peintes et commentées sur les murs des temples égyp­tiens. Dans les deux cas, le procédé de reconstitution est le même et toutes les sciences en offrent de multiples exemples.\par
Mais, de ce qu’un grand nombre de chercheurs de toute catégorie se trouvent ainsi contraints de ne saisir certains phénomènes centraux qu’à travers d’autres phénomènes qui en sont dérivés, il ne résulte pas entre eux, à beaucoup près, une parfaite égalité de moyens. Il se peut que, comme le physicien, ils aient le pouvoir de provoquer eux‑mêmes l’apparition de ces traces. Il se peut, au contraire, qu’ils soient réduits à l’attendre du caprice de forces sur lesquelles ils ne possèdent pas la moindre influence. Dans l’un ou l’autre cas, leur position sera, de toute évidence, extrêmement différente. Qu’en est‑il des observateurs des faits humains ? Ici, les ques­tions de date reprennent leurs droits.\par

\astermono

\noindent Que tous les faits humains un peu complexes échappent à la possibilité d’une reproduction ou d’une orientation volontaires, la chose semble aller de soi et nous aurons, d’ailleurs, à y revenir plus tard. Certes, depuis les plus élémentaires mesures de sensation jusqu’aux textes les plus raffinés de l’intelligence ou de l’émotivité, il existe une expérimentation psycho­logique. Mais elle ne s’applique, en somme, qu’à l’individu. La psychologie collective lui est à peu près entièrement rebelle. On ne pourrait guère – on n’oserait guère, à supposer qu’on le pût – susciter délibérément une panique ou un mouvement d’enthousiasme religieux. Cependant lorsque les phénomènes étudiés appartiennent au présent ou au tout proche passé, l’observateur – si incapable soit‑il de les forcer à se répéter ou d’en infléchir, à son gré, le déroulement – ne se trouve pas également désarmé vis‑à‑vis de leurs traces. Il peut, littéralement, appeler certaines d’entre elles à l’existence. Ce sont les rapports des témoins.\par
 \phantomsection
\label{p22} Le 5 décembre 1805, l’expérience d’Austerlitz n’était pas plus qu’au­jourd’hui susceptible de recommencement. Qu’avait fait pourtant dans la bataille tel ou tel régiment ? Napoléon, s’il a voulu, quelques heures après que le feu eût cessé, se renseigner là‑dessus – deux mots lui ont suffi pour qu’un des officiers lui adressât un compte rendu. Aucune relation de cette sorte, publique ou privée, n’a‑t‑elle au contraire jamais été établie ? Celles qui ont été écrites se sont‑elles perdues ? Nous aurons beau nous poser à notre tour la même question, elle risquera fort de demeurer éter­nellement sans réponse, avec beaucoup d’autres beaucoup plus impor­tantes. Quel historien n’a rêvé de pouvoir, comme Ulysse, nourrir les ombres de sang pour les interroger ? Mais les miracles de la \emph{Nekuia} ne sont plus de saison et nous n’avons d’autre machine à remonter le temps que celle qui fonctionne dans notre cerveau, avec des matériaux fournis par les générations passées.\par
Sans doute, il ne faudrait point exagérer non plus les privilèges de l’étude du présent. Imaginons que tous les officiers, que tous les hommes du régiment aient péri ; ou, plus simplement, que parmi les survivants il ne se soit plus trouvé de témoins dont la mémoire, dont les facultés d’attention. fussent dignes de créance. Napoléon n’aura pas été mieux loti que nous. Quiconque a pris part, fût‑ce dans le rôle le plus humble, à quelque grande action, le sait bien ; il arrive qu’un épisode parfois capital soit, au bout de peu d’heures, impossible à préciser. Ajoutez que toutes les traces ne se prêtent pas avec la même docilité à cette évocation à retardement. Si les douanes ont négligé d’enregistrer chaque jour, en novembre 1942, l’entrée et la sortie des marchandises, je n’aurai prati­quement aucun moyen, en décembre, d’apprécier le commerce extérieur du mois précédent. En un mot, de l’enquête sur le lointain à l’enquête sur le tout proche, la différence est une fois de plus seulement de degré. Elle n’atteint pas le fond des méthodes. Elle n’en est pas moins impor­tante pour cela, et il convient d’en tirer les conséquences.\par

\astermono

\noindent Le passé est, par définition, un donné que rien ne modifiera plus. Mais la connaissance du passé est une chose en progrès, qui sans cesse se transforme et se perfectionne. À qui en douterait, il suffirait de rap­peler ce qui, depuis un peu plus d’un siècle, s’est fait sous nos yeux. D’im­menses pans d’humanité sont sortis des brumes. L’Égypte et la Chaldée ont secoué leurs linceuls. Les villes mortes de l’Asie Centrale ont révélé leurs langues que nul ne savait plus parler, et leurs religions dès longtemps éteintes. Une civilisation tout entière ignorée vient de se lever du tom­beau, sur les bords de l’Indus. Ce n’est pas tout et l’ingéniosité des cher­cheurs à fouiller plus avant les bibliothèques, à ouvrir dans les vieux sols de nouvelles tranchées ne travaille pas seule, ni peut‑être le plus  \phantomsection
\label{p23} efficacement, à enrichir l’image des temps accomplis. Des procédés d’in­vestigation jusque là inconnus ont eux aussi surgi. Nous savons mieux que nos prédécesseurs interroger les langues sur les mœurs, les outils sur l’ouvrier. Nous avons appris surtout à descendre plus profondément dans l’analyse des faits sociaux. L’étude des croyances et rites populaires développe à peine ses premières perspectives. L’histoire de l’économie dont Cournot, naguère, énumérant les divers aspects de la recherche historique, n’avait pas même l’idée, commence seulement de se constituer. Tout cela est certain. Tout cela nous permet les plus vastes espoirs. Non des espoirs illimités. Ce sentiment de progression véritablement indéfinie que donne une science comme la chimie, capable de créer jusqu’à son propre objet, nous est refusé. C’est que les explorateurs du passé ne sont pas des hommes tout à fait libres. Le passé est leur tyran. Il leur interdit de rien connaître de lui qu’il ne leur ait lui même livré, sciemment ou non. Nous n’établirons jamais une statistique des prix à l’époque méro­vingienne, car aucun document n’a enregistré ces prix en nombre suffisant. Nous ne pénétrerons jamais aussi bien la mentalité des hommes du XI\textsuperscript{ᵉ} siècle européen, par exemple, que nous pouvons le faire pour les contem­porains de Pascal ou de Voltaire ; parce que nous n’avons d’eux ni lettres privées, ni confessions ; parce que nous n’avons sur quelques‑uns d’entre eux que de mauvaises biographies en style convenu. À cause de cette lacune, toute une partie de notre histoire affecte nécessairement l’allure, un peu exsangue, d’un monde sans individus. Ne nous plaignons pas trop. Dans cette étroite soumission à un inflexible destin nous ne sommes pas – nous, pauvres adeptes souvent raillés des jeunes sciences de l’homme – plus mal partagés que beaucoup de nos confrères, voués à des disciplines plus vieilles et plus sûres d’elles. Tel est le sort commun de toutes les études que leur mission appelle à scruter des phénomènes révolus – et le préhistorien n’est pas, faute d’écrits, plus incapable de restituer les litur­gies de l’âge de pierre que le paléontologue, je suppose, les glandes à sécrétion interne du plésiosaure, dont seul le squelette subsiste. Il est toujours désagréable de dire : « je ne sais pas », « je ne peux pas savoir ». Il ne faut le dire qu’après avoir énergiquement, désespérément cherché. Mais il y a des moments où le plus impérieux devoir du savant est, ayant tout tenté, de se résigner à l’ignorance et de l’avouer honnêtement.
\subsection[{II. Les témoignages}]{II. Les témoignages}
\noindent « Hérodote de Thourioi expose ici ses recherches, afin que les choses faites par les hommes ne s’oublient pas avec le temps et que de grandes et merveilleuses actions, accomplies tant par les Grecs que par les Bar­bares, ne perdent point leur éclat. » Ainsi commence le plus ancien livre d’histoire qui, dans le monde occidental, soit, autrement que par  \phantomsection
\label{p24} fragments, venu jusqu’à nous. À côté de lui, plaçons, par exemple, un de ces guides du voyage d’au-delà que les Égyptiens, au temps des Pha­raons, glissaient dans les tombeaux. Nous aurons, face à face, les types mêmes des deux grandes classes entre lesquelles se répartit la masse im­mensément variée des documents mis par le passé à la disposition des historiens. Les témoignages du premier groupe sont volontaires. Les autres, non.\par
Quand, en effet, nous lisons, pour nous informer, Hérodote ou Froissart, les \emph{Mémoires} du Maréchal Joffre ou les comptes rendus, d’ailleurs parfai­tement contradictoires, que les journaux allemands et britanniques donnent, ces jours‑ci, de l’attaque d’un convoi en Méditerranée – que faisons‑nous, sinon nous conformer exactement à ce que les auteurs de ces écrits attendaient de nous ? Au contraire, les formules des papyrus des morts n’étaient destinées qu’à être récitées par l’âme en péril et enten­dues des dieux seuls ; l’homme des palafittes, qui, dans le lac voisin où l’archéologue les remue aujourd’hui, jetait les débris de sa cuisine, ne voulait qu’épargner une souillure à sa hutte ; la bulle d’exemption ponti­ficale n’était si précautionneusement conservée dans les coffres du Monas­tère qu’afin d’être, le moment venu, brandie sous les yeux d’un évêque importun. À tous ces soins, le souci d’instruire l’opinion, soit des contem­porains, soit des historiens futurs n’avait aucune part ; et lorsque le médiéviste dans les archives feuillette, en l’an de grâce 1942, la corres­pondance commerciale des Cedames de Lucques, il se rend coupable d’une indiscrétion que les Cedames de nos jours, s’il prenait les mêmes libertés avec leur copie‑lettres, qualifieraient durement.\par
Or les sources narratives – pour employer dans son français un peu baroque l’expression consacrée – c’est‑à‑dire les récits délibérément voués à l’information des lecteurs, n’ont pas cessé assurément de prêter au chercheur un secours précieux. Entre autres avantages, elles sont ordi­nairement les seules à fournir un encadrement chronologique tant soit peu suivi. Que le préhistorien – que l’historien de l’Inde – ne donnerait‑il pas pour disposer d’un Hérodote ? On n’en saurait douter : c’est dans la seconde catégorie de témoignages, c’est dans les témoins malgré eux que la recherche historique, au cours de ses progrès, a été amenée à mettre de plus en plus sa confiance. Comparez l’histoire romaine telle que l’écri­vaient Rollin ou même Niebuhr avec celle que n’importe quel précis met aujourd’hui sous nos yeux : la première qui de Tite Live, Suétone ou Florus tirait le plus clair de sa substance, la seconde bâtie pour une large par à-coups d’inscriptions, de papyrus, de monnaies. Des morceaux entiers du passé n’ont pu être reconstitués qu’ainsi : toute la préhistoire, presque toute l’histoire économique, presque toute l’histoire des structures so­ciales. Dans le présent même qui de nous, plutôt que tous les journaux de 1938 ou 1939, ne préférerait tenir en mains quelques pièces secrètes de chancellerie, quelques rapports confidentiels de chefs militaires ?\par

\astermono

\noindent  \phantomsection
\label{p25} Ce n’est pas que les documents de ce genre soient plus que d’autres exempts d’erreur ou de mensonge. Les fausses bulles ne manquent point et pas plus que toutes les relations d’ambassadeurs, toutes les lettres d’affaire ne disent la vérité. Mais la déformation ici, à supposer qu’elle existe, n’a du moins pas été conçue spécialement à l’intention de la posté­rité. Surtout ces indices que, sans préméditation, le passé laisse tomber le long de sa route, ne nous permettent pas seulement de suppléer aux récits lorsque ceux‑ci font défaut, ou de les contrôler, si la véracité en est suspecte. Ils écartent de nos études un danger plus mortel que l’igno­rance ou l’inexactitude : celui d’une irrémédiable sclérose. Sans leur secours en effet, ne verrait‑on pas inévitablement l’historien, chaque fois qu’il se penche sur des générations disparues, devenir aussitôt le prison­nier des préjugés, des fausses prudences, des myopies dont la vision de ces générations mêmes avait souffert – le médiéviste, par exemple, n’accorder qu’une faible importance au mouvement communal, sous pré­texte que les écrivains du Moyen Âge n’en entretenaient pas volontiers leur public, ou dédaigner les grands élans de la vie religieuse pour la belle raison qu’ils occupent, dans la littérature narrative du temps, une place beaucoup plus mince que les guerres des barons ; l’histoire, en un mot – pour reprendre une antithèse chère à Michelet – se faire moins l’explo­ratrice de plus en plus hardie des âges révolus que l’éternelle et immobile élève de leurs « chroniques ».\par
Aussi bien, jusque dans les témoignages les plus résolument volontaires, ce que le texte nous dit expressément a cessé aujourd’hui d’être l’objet préféré de notre attention. Nous nous attachons ordinairement avec bien plus d’ardeur à ce qu’il nous laisse entendre, sans avoir souhaité le dire. Chez Saint‑Simon, que découvrons‑nous de plus instructif ? Ses infor­mations, souvent controuvées, sur les événements du règne ? Ou l’éton­nante lumière que les \emph{Mémoires} nous ouvrent sur la mentalité d’un grand seigneur à la cour du Roi Soleil ? Parmi les vies des saints du haut Moyen Âge, les trois quarts au moins sont incapables de rien nous apprendre de solide sur les pieux personnages dont elles prétendent retracer le destin. Interrogeons‑les, au contraire, sur les façons de vivre ou de penser parti­culières aux époques où elles furent écrites, toutes choses que l’hagiographe n’avait pas le moindre désir de nous exposer : nous les trouverons d’un prix inestimable. Dans notre inévitable subordination envers le passé nous nous sommes donc affranchis du moins en ceci que, condamnés toujours à le connaître exclusivement par ses traces, nous parvenons toutefois à en savoir sur lui beaucoup plus long qu’il n’avait lui-même cru bon de nous en faire connaître. C’est, à bien le prendre, une grande revanche de l’intelligence sur le donné.\par

\astermono

\noindent  \phantomsection
\label{p26} Mais du moment que nous ne sommes plus résignés à enregistrer pure­ment et simplement les propos de nos témoins, du moment que nous entendons les forcer à parler, fût‑ce contre leur gré – un questionnaire plus que jamais s’impose. Telle est, en effet, la première nécessité de toute recherche historique bien conduite.\par
Beaucoup de personnes et même, semble‑t‑il, certains auteurs de manuels se font de la marche de notre travail une image étonnamment candide. Au commencement, diraient‑elles volontiers, sont les documents. L’histo­rien les rassemble, les lit, s’efforce d’en peser l’authenticité et la véracité. Après quoi, et après quoi seulement, il les met en œuvre. Il n’y a qu’un malheur : aucun historien, jamais, n’a procédé ainsi. Même lorsque d’aven­ture il s’imagine le faire.\par
Car les textes, ou les documents archéologiques, fût‑ce les plus clairs en apparence et les plus complaisants, ne parlent que lorsqu’on sait les interroger. Avant Boucher de Perthes, les silex abondaient, comme de nos jours, dans les alluvions de la Somme. Mais l’interrogateur manquait et il n’y avait pas de préhistoire. Vieux médiéviste, j’avoue ne connaître guère de lecture plus attrayante qu’un cartulaire. C’est que je sais à peu près quoi lui demander. Un recueil d’inscriptions romaines, en revanche, me dit peu. Je sais tant bien que mal les lire, non les solliciter. En d’autres termes, toute recherche historique suppose, dès ses premiers pas, que l’enquête ait déjà une direction. Au commencement est l’esprit. Jamais, dans aucune science, l’observation passive n’a rien donné de fécond. À supposer, d’ailleurs, qu’elle soit possible.\par
Ne nous y laissons pas tromper en effet. Il arrive, sans doute, que le questionnaire demeure purement instinctif. Il est là cependant. Sans que le travailleur en ait conscience, les articles lui en sont dictés par les affir­mations ou les hésitations que ses expériences antérieures ont obscuré­ment inscrites dans son cerveau, par la tradition, par le sens commun, c’est‑à‑dire, trop souvent, par les préjugés communs. On n’est jamais aussi réceptif qu’on ne le croit. Il n’y a pas de pire conseil à donner à un débutant que celui d’attendre ainsi, dans une attitude d’apparente sou­mission, l’inspiration du document. Par là plus d’une recherche de bonne volonté a été vouée à l’échec ou à l’insignifiance.\par
Naturellement il le faut, ce choix raisonné des questions, extrêmement souple, susceptible de se charger chemin faisant d’une multitude d’articles nouveaux, ouvert à toutes les surprises – tel cependant qu’il puisse, dès l’abord, servir d’aimant aux limailles du document. L’itinéraire que l’explorateur établit, au départ, il sait bien d’avance qu’il ne le suivra pas de point en point. À ne pas en avoir, cependant, il risquerait d’errer éternellement à l’aventure.\par

\astermono

\noindent  \phantomsection
\label{p27} La diversité des témoignages historiques est presque infinie. Tout ce que l’homme dit ou écrit, tout ce qu’il fabrique, tout ce qu’il touche peut et doit renseigner sur lui. Il est curieux de constater combien les personnes étrangères à notre travail jaugent imparfaitement l’étendue de ces possibilités. C’est qu’elles continuent de s’attacher à une idée surannée de notre science : celle du temps où l’on ne savait guère lire que les témoignages volontaires. Reprochant à l’ » histoire traditionnelle » de laisser dans l’ombre des « phénomènes considérables », pourtant « plus gros de conséquences, plus capables de modifier la vie prochaine que tous les événements politiques », Paul Valéry proposait pour exemple « la conquête de la terre » par l’électricité. Sur quoi, on l’applaudira des deux mains. Il est malheureusement trop exact que cet immense sujet n’a encore donné lieu à aucun travail sérieux. Mais quand, entraîné en quelque sorte par l’excès même de sa sévérité à justifier la faute qu’il vient de dénoncer, P. Valéry ajoute que ces phénomènes « échappent » nécessairement à l’historien – car, poursuit‑il, « aucun document ne les mentionne expressément » – l’accusation cette fois, en passant du savant à la science, se trompe d’adresse. Qui croira que les entreprises d’électri­cité n’aient pas leurs archives, leurs états de consommation, leurs cartes d’extension des réseaux ? Les historiens, dites‑vous, ont jusqu’ici négligé d’interroger ces documents. Ils ont grand tort assurément : à moins, toutefois, que la responsabilité n’en incombe aux gardiens peut‑être trop jaloux de tant de beaux trésors. Prenez donc patience. L’histoire n’est pas encore telle qu’elle devrait être. Ce n’est pas une raison pour faire porter à l’histoire telle qu’elle peut s’écrire le poids d’erreurs qui n’ap­partiennent qu’à l’histoire mal comprise.\par
De ce caractère merveilleusement disparate de nos matériaux naît cependant une difficulté : assez grave en vérité pour compter parmi les trois ou quatre grands paradoxes du métier d’historien.\par
L’illusion serait grande d’imaginer qu’à chaque problème historique réponde un type unique de documents, spécialisé dans cet emploi. Plus la recherche, au contraire, s’efforce d’atteindre les faits profonds, moins il lui est permis d’espérer la lumière autrement que des rayons conver­gents de témoignages très divers dans leur nature. Quel historien des religions voudrait se contenter de compulser des traités de théologie ou des recueils d’hymnes ? Il le sait bien : sur les croyances et les sensibilités mortes, les images peintes ou sculptées aux murs des sanctuaires, la disposition et le mobilier des tombes ont au moins aussi long à lui dire que beaucoup d’écrits. Autant que du dépouillement des chroniques ou des chartes, notre connaissance des invasions germaniques dépend de l’archéologie funéraire et de l’étude des noms de lieux. À mesure qu’on se rapproche de notre temps, ces exigences se font sans doute différentes.\par
 \phantomsection
\label{p28} Elles ne deviennent pas pour cela moins impérieuses. Pour comprendre les sociétés d’aujourd’hui, croira‑t‑on qu’il suffise de se plonger dans la lecture des débats parlementaires ou des pièces de chancellerie ? Ne faut‑il pas encore savoir interpréter un bilan de banque : texte, pour le profane, plus hermétique que beaucoup de hiéroglyphes ? L’historien d’une époque où la machine est reine, acceptera‑t‑on qu’il ignore comment sont constituées et se sont modifiées les machines ?\par
Or, si presque tout problème humain important appelle ainsi le manie­ment de témoignages de types opposés – c’est au contraire, de toute nécessité, par type de témoignage que se distinguent les techniques érudites. L’apprentissage de chacune d’elles est long ; leur pleine possession veut une pratique plus longue encore et quasiment constante. Un bien petit nombre de travailleurs, par exemple, peuvent se vanter d’être également bien préparés à lire et critiquer une charte médiévale, à interpréter cor­rectement les noms de lieux (qui sont, avant tout, des faits de langage), à dater sans erreur les vestiges de l’habitat préhistorique, celte, gallo­-romain ; à analyser les associations végétales d’un pré, d’un guéret, d’une lande. Sans tout cela pourtant, comment prétendre décrire l’histoire de l’occupation du sol ? Peu de sciences, je crois, sont contraintes d’user, simultanément, de tant d’outils dissemblables. C’est que les faits humains sont entre tous complexes : C’est que l’homme se place à la pointe extrême de la nature.\par
Il est bon, à mon sens, il est indispensable que l’historien possède au moins une teinture de toutes les principales techniques de son métier. Fût‑ce seulement afin de savoir mesurer à l’avance la force de l’outil et les difficultés de son maniement. La liste des « disciplines auxiliaires » dont nous proposons l’enseignement à nos débutants est beaucoup trop courte. Des hommes qui, la moitié du temps, ne pourront atteindre les objets de leurs études qu’à travers les mots, par quel absurde paralogisme leur permet‑on, entre autres lacunes, d’ignorer les acquisitions fondamen­tales de la linguistique ?\par
Cependant, quelque variété de connaissances qu’on veuille bien prêter aux chercheurs les mieux armés, elles trouveront, toujours et ordinaire­ment très vite, leurs limites. Point d’autre remède alors que de substituer à la multiplicité des compétences chez un même homme, une alliance des techniques pratiquées par des érudits différents, mais toutes tendues vers l’élucidation d’un thème unique. Cette méthode suppose le consen­tement au travail par équipes. Elle exige aussi la définition préalable, par accord commun, de quelques grands problèmes dominants. Ce sont des réussites dont nous nous trouvons encore beaucoup trop loin. Elles commanderont pourtant, dans une large mesure, n’en doutons pas, l’avenir de notre science.
\subsection[{III. La transmission des témoignages}]{III. La transmission des témoignages}
\noindent  \phantomsection
\label{p29} C’est une des tâches les plus difficiles de l’historien que de rassembler les documents dont il estime avoir besoin. Il ne saurait guère y parvenir sans l’aide de guides divers : inventaires d’archives ou de bibliothèques, catalogues de musées, répertoires bibliographiques de toute sorte. On voit parfois des pédants à la cavalière s’étonner du temps sacrifié et par quelques érudits à composer de pareils ouvrages, et par tous les travailleurs à en apprendre l’existence et le maniement. Comme si, grâce aux heures ainsi dépensées à des besognes qui, pour n’être pas sans un certain attrait caché, manquent assurément d’éclat romanesque – le plus affreux gas­pillage d’énergie ne se trouvait pas finalement épargné. Passionné à juste titre par l’histoire du culte des saints, supposez que j’ignore la \emph{Bibliotheca Hagiographica Latina} des Pères Bollandistes : vous imaginerez difficile­ment, si vous n’êtes pas spécialiste, la somme d’efforts stupidement inutiles que cette lacune de mon équipement ne manquera pas de me coûter. Ce qu’il convient de regretter, en vérité, ce n’est pas que nous puissions déjà mettre sur les rayons de nos bibliothèques une quantité notable de ces instruments (dont l’énumération, matière par matière, appartient aux livres spéciaux d’orientation). C’est qu’ils ne soient pas encore assez nombreux, surtout pour les époques les moins éloignées de nous ; que leur établissement, en France notamment, n’obéisse que par exception à un plan d’ensemble rationnellement conçu ; que leur remise à jour, enfin, soit trop souvent abandonnée aux caprices des individus ou à la parcimonie mal informée de quelques maisons d’édition. Le tome premier des admirables \emph{Sources de l’Histoire de France}, que nous devons à Émile Molinier, n’a pas été réédité depuis sa première apparition, en 1901. Ce simple fait vaut un acte d’accusation. L’outil, certes, ne fait pas la science. Mais une société qui prétend respecter les sciences ne devrait pas se désintéresser de leurs outils. Sans doute serait‑elle sage aussi de ne pas trop s’en remettre pour cela à des corps académiques, que leur recrutement, favorable à la prééminence de l’âge et propice aux bons élèves, ne dispose pas particulièrement à l’esprit d’entreprise. Notre École de Guerre et nos états‑majors ne sont pas seuls, chez nous, à avoir conservé au temps de l’automobile la mentalité du char à bœufs.\par
Cependant si bien faits, si abondants que puissent être ces poteaux indicateurs, ils ne seraient que de peu de secours à un travailleur qui n’aurait pas, d’avance, quelque idée du terrain à explorer. En dépit de ce que semblent parfois imaginer les débutants, les documents ne sur­gissent pas, ici ou là, par l’effet d’on ne sait quel mystérieux décret des Dieux. Leur présence ou leur absence, dans tel fonds d’archives, dans telle bibliothèque, dans tel sol, relèvent de causes humaines qui n’échap­pent nullement à l’analyse, et les problèmes que pose leur transmission,  \phantomsection
\label{p30} loin d’avoir seulement la portée d’exercices de techniciens, touchent eux‑mêmes au plus intime de la vie du passé, car ce qui se trouve ainsi mis en jeu n’est rien de moins que le passage du souvenir à travers les générations. En tête des ouvrages historiques du genre sérieux, l’auteur place généralement une liste des cotes d’archives qu’il a compulsées, des recueils dont il a fait usage. Cela est fort bien. Mais cela n’est pas assez. Tout livre d’histoire digne de ce nom devrait comporter un chapitre, ou si l’on préfère, insérée aux points tournants du développement, une suite de paragraphes qui s’intitulerait à peu près : « Comment puis‑je savoir ce que je vais dire ? » Je suis persuadé qu’à prendre connaissance de ces confessions, même les lecteurs qui ne sont pas du métier éprouveraient un vrai plaisir intellectuel. Le spectacle de la recherche, avec ses succès et ses traverses, est rarement ennuyeux. C’est le tout fait qui répand la glace et l’ennui.\par

\astermono

\noindent Il m’arrive de recevoir la visite de travailleurs qui désirent écrire l’his­toire de leur village. Régulièrement, je leur tiens les propos suivants, que je simplifie seulement un peu, afin d’éviter les détails d’érudition qui seraient ici hors de saison. « Les communautés paysannes n’ont pos­sédé d’archives que rarement et tardivement. Les seigneuries, au con­traire, étant des entreprises relativement bien organisées et douées de continuité, ont généralement conservé de bonne heure leurs dossiers. Pour toute la période antérieure à 1789, et spécialement pour les époques les plus anciennes, les principaux documents, dont vous pouvez espérer vous servir, seront donc de provenance seigneuriale. D’où il résulte à son tour que la première question à laquelle vous aurez à répondre et dont presque tout dépendra va être celle‑ci : en 1789, quel était le seigneur du village ? » (En fait, l’existence simultanée de plusieurs seigneurs, entre lesquels le village aurait été partagé, n’est nullement invraisemblable ; mais, pour faire court, on laissera de côté cette supposition.) « Trois éven­tualités sont concevables. La seigneurie peut avoir appartenu à une église ; à un laïque qui, sous la Révolution, émigra ; à un laïque encore, mais qui au contraire n’émigra jamais. Le premier cas est, de beaucoup, le plus favorable. Le fonds n’a pas seulement chance d’avoir été mieux tenu et depuis plus longtemps. Il a certainement été confisqué, dès 1790, en même temps que les terres, par application, de la Constitution Civile du Clergé. Porté alors dans quelque dépôt publié, on peut raisonnablement espérer qu’il continue aujourd’hui d’y figurer, à peu près intact, à la disposition des érudits. L’hypothèse de l’émigré mérite encore une assez bonne note. Là aussi il a dû y avoir saisie et transfert ; tout au plus le risque d’une destruction volontaire, comme vestige d’un régime honni, semblera‑t‑il, un peu plus à redouter. Reste la dernière possibilité. Elle  \phantomsection
\label{p31} serait infiniment fâcheuse. Les « ci-devant », en effet, du moment qu’ils ne quittaient pas la France ni ne tombaient, de quelque autre façon, sous le coup des lois de Salut Public, n’étaient nullement frappés dans leurs biens. Ils perdaient, sans doute, leurs droits seigneuriaux puisque ceux‑ci avaient été universellement abolis. Ils conservaient l’ensemble de leurs propriétés personnelles ; par suite, leurs dossiers d’affaires. N’ayant donc jamais été réclamées par l’État, les pièces que nous cherchons auront, cette fois, simplement subi le sort commun, durant les XIX\textsuperscript{ᵉ} et XX\textsuperscript{ᵉ} siècles, à tous les papiers de famille. À supposer qu’elles n’aient pas été égarées, mangées par les rats, ou dispersées, au gré des ventes et héritages, à travers les greniers de trois ou quatre maisons de campagne différentes, rien n’obligera leur détenteur actuel à vous les communiquer. »\par
J’ai cité cet exemple parce qu’il me paraît tout à fait typique des condi­tions qui fréquemment déterminent et limitent la documentation. Il ne sera pas sans intérêt d’en analyser, de plus près, les enseignements.\par

\astermono

\noindent Le rôle qu’on vient de voir jouer par les confiscations révolutionnaires est celui d’une déité souvent propice au chercheur : la catastrophe. D’in­nombrables municipes romains se sont transformées en banales petites villes italiennes, où l’archéologue retrouve péniblement quelques vestiges de l’antiquité ; seule l’éruption du Vésuve a préservé Pompéi.\par
Certes, il s’en faut de beaucoup que les grands désastres de l’humanité aient toujours servi l’histoire. Avec les manuscrits littéraires et historio­graphiques par monceaux, les inestimables dossiers de la bureaucratie impériale romaine ont sombré dans le trouble des Invasions. Sous nos yeux, les deux guerres mondiales ont rayé d’un sol chargé de gloire, monu­ments et dépôts d’archives. Jamais plus nous ne pourrons feuilleter les lettres des vieux marchands d’Ypres et j’ai vu brûler, durant la déroute, les carnets d’ordres d’une Armée.\par
Cependant à son tour, la paisible continuité d’une vie sociale sans poussées de fièvre se montre beaucoup moins favorable qu’on ne le croit parfois à la transmission du souvenir. Ce sont les révolutions qui forcent les portes des armoires de fer et contraignent les ministres à la fuite, avant qu’ils n’aient trouvé le temps de brûler leurs notes secrètes. Dans les anciennes archives judiciaires, les fonds de faillites nous livrent aujour­d’hui les papiers d’entreprises qui, s’il leur avait été donné de mener jusqu’au bout une existence fructueuse et honorée, n’eussent pas manqué de vouer finalement au pilon le contenu de leurs cartonniers. Grâce à l’admirable permanence des institutions monastiques, l’abbaye de Saint­-Denis conservait encore, en 1789, les diplômes qui lui avaient été octroyés, plus de mille ans auparavant, par les rois mérovingiens. Mais c’est aux Archives Nationales que nous les lisons aujourd’hui. Si la communauté  \phantomsection
\label{p32} des moines dyonisiens avait survécu à la Révolution, est‑il sûr qu’elle nous permettrait de fouiller dans ses coffres ? Pas plus, peut‑être, que la compagnie de Jésus n’ouvre au profane l’accès de ses collections, faute desquelles tant de problèmes de l’histoire moderne demeureront toujours désespérément obscurs, ou que la Banque de France n’invite les spécia­listes du Premier Empire à compulser ses registres, même les plus poudreux, tant la mentalité de l’initié est inhérente à toutes les corporations. Voilà où l’historien du présent se trouve nettement à son désavantage : il est presque totalement privé de ces confidences involontaires. Pour compen­sation, il dispose, il est vrai, des indiscrétions que lui chuchotent à l’oreille ses amis. Le renseignement, hélas, s’y distingue mal du racontar. Un brave cataclysme ferait souvent mieux notre affaire.\par
Du moins, en sera‑t‑il ainsi tant que, renonçant à s’en remettre de ce soin à leurs propres tragédies, les sociétés consentiront enfin à organiser rationnellement, avec leur mémoire, leur connaissance d’elle‑mêmes. Elles n’y réussiront qu’en s’attaquant corps à corps aux deux principaux res­ponsables de l’oubli ou de l’ignorance : la négligence, qui égare les docu­ments ; et plus dangereuse encore, la passion du secret – secret diplo­matique, secret des affaires, secret des familles – qui les cache ou le détruit. Il est naturel que le notaire ait le devoir de ne pas révéler les opérations de son client. Mais qu’il lui soit permis d’envelopper d’un aussi impénétrable mystère les contrats passés par les clients de son bisaïeul – alors que, d’autre part, rien ne lui interdit sérieusement de laisser ces pièces s’en aller en poussière – nos lois, là‑dessus, fleurent vraiment le moisi. Quant aux motifs qui engagent la plupart des grandes entreprises à refuser de rendre publiques les statistiques les plus indispensables à une saine conduite de l’économie nationale, ils sont rarement dignes de respect. Notre civilisation aura accompli un immense progrès le jour où la dissimulation érigée en méthode d’action et presque en bourgeoise vertu cèdera la place au goût du renseignement : c’est‑à‑dire nécessai­rement des échanges de renseignements.\par

\astermono

\noindent Revenons cependant à notre village. Les circonstances qui, dans ce cas précis, décident de la perte ou de la conservation, de l’accessibilité ou de l’inaccessibilité des témoignages ont leur origine dans des forces historiques de caractère général. Elles ne présentent aucun trait qui ne soit parfaitement intelligible ; mais elles sont dépourvues de tout rapport logique avec l’objet de l’enquête dont l’issue se trouve, pourtant, placée sous leur dépendance. Car on ne voit évidemment pas pourquoi l’étude d’une petite communauté rurale, au moyen âge, par exemple, serait plus ou moins instructive selon que, quelques siècles plus tard, son maître du moment s’avisa ou non d’aller grossir les rassemblements de Coblenz.\par
 \phantomsection
\label{p33} Rien de plus fréquent que ce désaccord. Si nous connaissons l’Égypte romaine infiniment mieux que la Gaule, au même temps, ce n’est pas que nous portions aux Égyptiens un intérêt plus vif qu’aux gallo‑romains : la sécheresse, les sables et les rites funéraires de la momification ont pré­servé là‑bas les écrits que le climat de l’Occident et ses usages vouaient, au contraire, à une rapide destruction. Entre les causes qui font le succès ou l’échec de la poursuite des documents et les motifs qui nous rendent ces documents désirables, il n’y a ordinairement rien de commun : tel est l’élément irrationnel, impossible à éliminer, qui donne à nos recherches un peu de ce tragique intérieur où tant d’œuvres de l’esprit trouvent peut‑être, avec leurs limites, une des raisons secrètes de leur destruction.\par
Encore, dans l’exemple cité, le sort des documents, village par village, devient‑il une fois le fait crucial connu, à peu près prévisible. Tel n’est pas toujours le cas. Le résultat final tient parfois à la rencontre d’un si grand nombre de chaînes causales pleinement indépendantes les unes des autres que toute prévision s’avère impraticable. Je sais que quatre incen­dies successifs, puis un pillage ont dévasté les archives de l’antique abbaye de Saint Benoît‑sur‑Loire. Comment, abordant ce fonds, devinerais‑je à l’avance quels types de pièces ces ravages ont de préférence épargnés ? Ce qu’on a appelé la migration des manuscrits offre un sujet d’études du plus haut intérêt ; les passages d’une œuvre littéraire à travers les biblio­thèques, l’exécution des copies, le soin ou la négligence des bibliothécaires et des copistes sont autant de traits par où s’expriment, au vif, les vicis­situdes de la culture et le variable jeu de ses grands courants. Mais l’érudit le mieux informé aurait‑il pu annoncer, avant la découverte, que le manuscrit unique de la Germanie de Tacite avait échoué, au XVI\textsuperscript{ᵉ} siècle, au monastère de Hersfeld ? En un mot, il y a au fond de presque chaque enquête documentaire, un résidu d’inopiné et, par suite, de risqué. Un travailleur, que j’ai quelque raison de très bien connaître, m’a raconté qu’à Dunkerque, comme il attendait sans marquer trop d’impatience, sur la côte bombardée, un incertain embarquement, un de ses camarades lui dit, avec une mine d’étonnement : « c’est singulier, vous n’avez pas l’air de détester l’aventure ! » Mon ami aurait pu répondre qu’en dépit du préjugé courant l’habitude de la recherche n’est nullement défavorable en effet, à une acceptation assez aisée du pari avec la destinée.\par
Entre la connaissance du passé humain et celle du présent, nous deman­dions‑nous tout à l’heure, existe‑t‑il une opposition de techniques ? La réponse vient d’être donnée. Certes l’explorateur de l’actuel et celui des époques lointaines ont chacun leur façon particulière de manier l’outil. Chacun aussi, selon le cas, possède l’avantage. Le premier touche la vie d’une prise plus immédiatement sensible ; le second, dans ses fouilles, dispose de moyens qui sont souvent refusés au premier. Ainsi la dissection du cadavre, en découvrant au biologiste bien des secrets que l’étude du vivant lui aurait laissé ignorer, se tait sur beaucoup d’autres, dont le  \phantomsection
\label{p34} corps vivant seul détient la révélation. Mais, à quelque âge de l’humanité que le chercheur s’adresse, les méthodes de l’observation qui se font presque uniformément sur traces, demeurent fondamentalement les mêmes. Pareilles également, nous allons le voir, sont les règles critiques auxquelles l’observation, pour être féconde, se doit d’obéir.
\section[{Chapitre III. La critique}]{Chapitre III. \\
La critique}\renewcommand{\leftmark}{Chapitre III. \\
La critique}

\subsection[{I. Esquisse d’une histoire de la méthode critique}]{I. Esquisse d’une histoire de la méthode critique}
\noindent  \phantomsection
\label{p35} Que les témoins ne doivent pas être forcément crus sur parole, les plus naïfs des policiers le savent bien. Quitte, du reste, à ne pas toujours tirer de cette connaissance théorique le parti qu’il faudrait. De même, il y a beau temps qu’on s’est avisé de ne pas accepter aveuglément tous les témoignages historiques. Une expérience, vieille presque comme l’huma­nité, nous l’a appris : plus d’un texte se donne pour d’une autre époque ou d’une autre provenance qu’il ne l’est réellement ; tous les récits ne sont pas véridiques et les traces matérielles, elles aussi, peuvent être truquées. Au Moyen Âge, devant l’abondance même des faux, le doute fut souvent comme un réflexe naturel de défense. « Avec de l’encre, n’im­porte qui peut écrire n’importe quoi », s’écriait au XI\textsuperscript{ᵉ} siècle un hobereau lorrain, en procès contre des moines qui s’armaient contre lui de preuves documentaires. La Donation de Constantin – cette étonnante élucubra­tion qu’un clerc romain du VIII\textsuperscript{ᵉ} siècle mit sous le nom du premier César chrétien – fut, trois siècles plus tard, contestée dans l’entourage du très pieux Empereur Othon III. Les fausses reliques ont été pourchassées presque depuis qu’il y eut des reliques.\par
Cependant le scepticisme de principe n’est pas une attitude intellec­tuelle plus estimable ni plus féconde que la crédulité, avec laquelle d’ail­leurs il se combine aisément dans beaucoup d’esprits un peu simples. J’ai connu, pendant l’autre guerre, un brave vétérinaire qui, non sans quelque apparence de raison, refusait systématiquement toute créance aux nouvelles des journaux. Mais un compagnon de hasard déversait‑il dans son oreille attentive les plus abracadabrants bobards ? Mon homme les buvait comme petit lait.\par
De même la critique de simple bon sens qui a été longtemps la seule  \phantomsection
\label{p36} pratiquée, qui, d’aventure, séduit encore certains esprits, ne pouvait mener bien loin. Qu’est‑ce, en effet, le plus souvent que ce prétendu bon sens ? Rien d’autre qu’un composé de postulats irraisonnés et d’expé­riences hâtivement généralisées. S’agit‑il du monde physique ? Il a nié les antipodes. Il nie l’univers einsteinien. Il a fait traiter de fable le récit d’Hérodote rapportant qu’en tournant autour de l’Afrique, les naviga­teurs voyaient un jour le point où le soleil se lève passer de leur droite à leur gauche. S’agit‑il d’actes humains ? Le pis est que les observations élevées ainsi à l’éternel sont forcément empruntées à un moment très court de la durée : la nôtre. Là a résidé le principal vice de la critique voltairienne, par ailleurs si souvent si pénétrante. Non seulement les bizarreries individuelles sont de tous les temps ; plus d’un état d’âmes jadis commun nous paraît bizarre, parce que nous ne le partageons plus. Le « bon sens », semble‑t‑il, interdirait d’accepter que l’empereur Othon I\textsuperscript{ᵉʳ} ait pu souscrire, en faveur des papes, des concessions territoriales inap­plicables, qui démentaient ses actes antérieurs et dont ceux qui suivirent ne devaient tenir aucun compte. Il faut bien, croire, cependant, qu’il n’avait pas l’esprit bâti tout à fait comme nous – que, plus précisément, on mettait de son temps, entre l’écrit et l’action, une distance dont l’éten­due nous surprend – puisque le privilège est incontestablement authen­tique.\par
Le vrai progrès est venu le jour où le doute s’est fait, comme disait Volney, « examinateur » ; où des règles objectives, en d’autres termes, ont été peu à peu élaborées qui, entre le mensonge et la vérité, permettent d’établir un tri. Le jésuite Papebroeck, auquel la lecture des \emph{Vies de Saints} avait inspiré une incoercible méfiance envers l’héritage du haut Moyen Âge tout entier, tenait pour faux tous les diplômes mérovingiens, conservés dans les monastères. Non, répondit, en substance Mabillon, il y a incontestablement des diplômes forgés de toutes pièces, remaniés ou interpolés. Il en est aussi d’authentiques, et voici comment il est possible de les distinguer les uns des autres. Cette année-là – 1681, l’année de la publication du \emph{De Re Diplomatica}, une grande date en vérité dans l’histoire de l’esprit humain – la critique des documents d’archives fut définitivement fondée.\par

\astermono

\noindent Tel fut bien, d’ailleurs, de toute façon, dans l’histoire de la méthode critique, le moment décisif. L’humanisme de l’âge précédent avait eu ses velléités et ses intuitions. Il n’était pas allé plus loin. Rien de plus caracté­ristique qu’un passage des Essais. Montaigne y justifie Tacite d’avoir rapporté des prodiges. Affaire, dit‑il, aux théologiens et aux philosophes de discuter les « communes créances ». Les historiens n’ont qu’à les « réci­ter » comme leurs sources les leur donnent. « Qu’ils nous rendent l’histoire  \phantomsection
\label{p37} plus selon qu’ils reçoivent que selon qu’ils estiment ». En d’autres termes, une critique philosophique appuyée sur une certaine conception de l’ordre naturel ou divin, est parfaitement légitime ; et l’on entend de reste que Montaigne ne prend pas à son compte les miracles de Vespasien : non plus que beaucoup d’autres. Mais de l’examen, spécifiquement historique, d’un témoignage en tant que tel, il ne saisit visiblement pas bien comment la pratique en serait possible. La doctrine de recherches s’élabora seule­ment au cours de ce XVII\textsuperscript{ᵉ} siècle dont on ne place pas toujours la vraie grandeur là où il faudrait, et nommément, vers sa seconde moitié.\par
Les hommes de ce temps eux‑mêmes en ont eu conscience. C’était un lieu commun, entre 1680 et 1690, que de dénoncer comme une mode du moment le « pyrrhonisme de l’histoire ». « On dit », écrit Michel Levassor commentant ce terme, « que la droiture de l’esprit consiste à ne pas croire légèrement et à savoir douter en plusieurs rencontres. » Le mot même de critique, qui n’avait guère désigné jusque-là qu’un jugement de goût, passe alors au sens presque nouveau d’épreuve de véracité. On ne le risque d’abord qu’en s’excusant. Car « il n’est pas tout à fait du bel usage » : entendez qu’il a encore une saveur technique. Il gagne cependant de plus en plus. Bossuet le tient prudemment à distance. Quand il parle de « nos auteurs critiques », on devine le haussement d’épaules. Mais Richard Simon l’inscrit dans le titre de presque tous ses ouvrages. Les plus avisés ne s’y méprennent d’ailleurs pas. Ce que ce nom annonce, c’est bien la découverte d’une méthode d’application presque universelle. La critique, cette « espèce de lambeau qui nous éclaire et nous conduit dans les routes obscures de l’antiquité, en nous faisant distinguer le vrai du faux », ainsi s’exprime Ellies du Pin. Et Bayle, plus nettement encore : « M. Simon a répandu dans cette nouvelle Réponse plusieurs règles de critique qui peuvent servir non seulement pour entendre l’Écriture, mais aussi pour lire avec fruit bien d’autres ouvrages. »\par
Or confrontons quelques dates de naissance : Papebroeck (qui, s’il se trompa sur les chartes, n’en a pas moins sa place au premier rang, parmi les fondateurs de la critique appliquée à l’historiographie), 1628 ; Mabillon, 1632 ; Richard Simon (dont les travaux dominent les débuts de l’exégèse biblique), 1638. Ajoutez, en dehors de la cohorte des érudits proprement dits, Spinoza – le Spinoza du \emph{Traité Théologico‑politique}, ce pur chef‑d’œuvre de critique – philologique et historique : 1632 encore. Au sens le plus juste du mot, c’est une génération dont les contours se dessinent ainsi devant nous, avec une étonnante netteté. Mais il faut préciser davantage. C’est, très exactement, la génération qui vit le jour vers le moment où paraissait le \emph{Discours de la Méthode.}\par
Ne disons pas : une génération de cartésiens. Mabillon, pour ne parler que de lui, était un moine dévot, orthodoxe avec simplicité et qui nous a laissé, comme dernier écrit, un traité de \emph{La Mort Chrétienne.} On doutera qu’il ait connu de bien près la nouvelle philosophie, alors si suspecte à  \phantomsection
\label{p38} tant de pieuses gens ; plus encore que, s’il en eût d’aventure quelques lueurs, il y ait trouvé beaucoup de sujets d’approbation. D’autre part – quoi que semblent suggérer quelques pages, peut‑être trop célèbres, de Claude Bernard – les vérités d’évidence, de caractère mathématique, auxquelles le doute méthodique, chez Descartes, a pour mission de frayer le chemin – présentent peu de traits communs avec les probabilités de plus en plus approchées que la critique historique, comme les sciences du laboratoire, se satisfait de dégager. Mais, pour qu’une philosophie imprègne tout un âge, il n’est nécessaire ni qu’elle agisse exactement selon sa lettre, ni que la plupart des esprits en subissent les effets autrement que par une sorte d’osmose, souvent à demi-inconsciente. Comme la « science » cartésienne, la critique du témoignage historique fait table rase de la créance. Comme la science cartésienne encore, elle ne procède à cet implacable renversement de tous les étais anciens qu’afin de parvenir par là à de nouvelles certitudes (ou à de grandes probabilités), désormais dûment éprouvées. En d’autres termes, l’idée qui l’inspire suppose un retournement presque total des conceptions anciennes du doute. Que ses morsures parussent une souffrance ou qu’on trouvât en lui, au contraire, je ne sais quelle noble douceur, il n’avait guère été considéré jusque-là que comme une attitude mentale purement négative, comme une simple absence. On estime, désormais, que rationnellement conduit, il peut de­venir un instrument de connaissance. C’est une idée dont l’apparition se place à un moment très précis de l’histoire de la pensée.\par
Dès lors, les règles essentielles de la méthode critique étaient en somme fixées. Leur portée générale échappait si peu, qu’au XVIII\textsuperscript{e} siècle, entre les sujets les plus fréquemment proposés par l’Université de Paris au concours d’agrégation des philosophes, on voit figurer celui-ci, qui rend un son curieusement moderne : « du témoignage des hommes sur les faits historiques ». Ce n’est pas assurément que les générations suivantes n’aient apporté à l’outil bien des perfectionnements. Surtout, elles en ont beaucoup généralisé l’emploi et considérablement étendu les appli­cations.\par

\astermono

\noindent Longtemps, les techniques de la critique furent pratiquées, au moins d’une manière suivie, à peu près exclusivement par une poignée d’érudits, d’exégètes et de curieux. Les écrivains attachés à composer des ouvrages historiques d’une certaine envolée ne se souciaient guère de se familiariser avec ces recettes de laboratoire, à leur gré beaucoup trop minutieuses – et c’est à peine s’ils consentaient à tenir compte de leurs résultats. Or, il n’est jamais bon que, selon le mot de Humboldt, les chimistes crai­gnent « de se mouiller les mains ». Pour l’histoire, le danger d’un pareil schisme entre la préparation et la mise en œuvre est à double face. Il  \phantomsection
\label{p39} atteint d’abord et cruellement, les grands essais d’interprétation. Ceux‑ci ne manquent pas seulement, par là, au devoir primordial de la véracité patiemment cherchée ; privés, en outre, de ce perpétuel renouvellement, de cette surprise toujours renaissante que la lutte avec le document est seule à procurer, il leur devient impossible d’échapper à une oscillation sans trêve entre quelques thèmes stéréotypés qu’impose la routine. Mais le travail technique lui-même ne souffre pas moins. N’étant plus guidé d’en haut, il risque de s’accrocher indéfiniment à des problèmes insigni­fiants ou mal posés. Il n’est pas de pire gaspillage que celui de l’érudition, quand elle tourne à vide, ni de superbe plus mal placée que l’orgueil de l’outil qui se prend pour une fin en soi.\par
Contre ces périls, le consciencieux effort du XIX\textsuperscript{ᵉ} siècle a vaillamment lutté. L’école allemande, Renan, Fustel de Coulanges ont rendu à l’éru­dition son rang intellectuel. L’historien a été ramené à l’établi. La partie, cependant, est‑elle tout à fait gagnée ? Il y aurait beaucoup d’optimisme à le croire. Trop souvent le travail de recherches continue de s’en aller cahin‑caha, sans choix raisonné de ses points d’application. Surtout, le besoin critique n’a pas encore réussi à conquérir pleinement cette opinion des « honnêtes gens » (au sens ancien du terme) dont l’assentiment, nécessaire sans doute à l’hygiène morale de toute science, est plus particulièrement indispensable à la nôtre. Ayant les hommes pour objet d’étude­ comment, si les hommes manquent à nous comprendre, n’aurions‑nous pas le sentiment de n’accomplir qu’à demi notre mission ?\par

\astermono

\noindent Peut‑être, d’ailleurs, ne l’avons‑nous point, en réalité, parfaitement remplie. L’ésotérisme rébarbatif où les meilleurs parfois d’entre nous persistent à s’enfermer ; dans notre production de lecture courante, la prépondérance du triste manuel, que l’obsession d’un enseignement mal conçu substitue à la véritable synthèse ; la pudeur singulière qui, aussitôt sortis de l’atelier, semble nous interdire de mettre sous les yeux des pro­fanes les nobles tâtonnements de nos méthodes : toutes ces mauvaises habitudes, nées de l’accumulation de préjugés contradictoires, compro­mettent une cause pourtant belle. Elles conspirent à livrer, sans défense, la masse des lecteurs aux faux brillants d’une histoire prétendue, dont l’absence de sérieux, le pittoresque de pacotille, les parti pris politiques pensent se racheter par une immodeste assurance : Maurras, Bainville ou Plekhanov affirment, là où Fustel de Coulanges ou Pirenne auraient douté. Entre l’enquête historique, telle qu’elle se fait ou aspire à se faire, et le public qui lit, un malentendu incontestablement subsiste. Pour mettre en jeu des deux parts d’assez plaisants travers, la grande querelle des notes n’est pas le moins significatif de ces symptômes.\par
 \phantomsection
\label{p40} Les marges inférieures des pages exercent sur beaucoup d’érudits une attraction qui touche au vertige. Il est sûrement absurde d’en encombrer les blancs, comme ils le font, de renvois bibliographiques qu’une liste, dressée en tête du volume, eût, pour la plupart, épargnés ; ou pis encore, d’y relé­guer, par pure paresse, de longs développements dont la place était mar­quée dans le corps même de l’exposé : en sorte que le plus utile de ces ouvrages, c’est souvent à la cave qu’il le faut chercher. Mais lorsque certains lecteurs se plaignent que la moindre ligne, faisant cavalier seul au bas du texte, leur brouille la cervelle, lorsque certains éditeurs préten­dent que leurs chalands, sans doute moins hypersensibles en réalité qu’ils ne veulent bien les peindre, souffrent le martyre à la vue de toute feuille ainsi déshonorée, ces délicats prouvent simplement leur imperméabilité aux plus élémentaires préceptes d’une morale de l’intelligence. Car, hors des libres jeux de la fantaisie, une affirmation n’a le droit de se produire qu’à la condition de pouvoir être vérifiée ; et pour un historien, s’il emploie un document, en indiquer le plus brièvement possible la provenance, c’est‑à‑dire le moyen de le retrouver, équivaut sans plus à se soumettre à une règle universelle de probité. Empoisonnée de dogmes et de mythes, notre opinion, même la moins ennemie des lumières, a perdu jusqu’au goût du contrôle. Le jour où, ayant pris soin d’abord de ne pas la rebuter par un oiseux pédantisme, nous aurons réussi à la persuader de mesurer la valeur d’une connaissance sur son empressement à tendre le cou d’avance à la réfutation, les forces de la raison remporteront une de leurs plus éclatantes victoires. C’est à la préparer que travaillent nos humbles notes, nos petites références tatillonnes que moquent aujourd’hui, sans les comprendre, tant de beaux esprits.\par

\astermono

\noindent Les documents que maniaient les premiers érudits étaient, le plus souvent, des écrits qui se présentaient eux‑mêmes ou que l’on présentait, traditionnellement, comme d’un auteur ou d’un temps donnés et qui racontaient délibérément tels ou tels événements. Disaient‑ils vrai ? Les livres qualifiés de mosaïques étaient‑ils réellement de Moïse – et de Clovis les diplômes qui portent son nom ? Que valaient les récits de l’\emph{Exode} ou ceux des \emph{Vies} de saints ? Tel était le problème. Mais, à mesure que l’histoire a été conduite à faire des témoignages involontaires un emploi de plus en plus fréquent, elle a cessé de pouvoir se borner à peser les affirmations explicites des documents. Il lui a fallu aussi leur extorquer les renseignements qu’ils n’entendaient pas fournir.\par
Or, les règles critiques, qui avaient fait leur preuve dans le premier cas, se montrèrent également efficaces dans le second. J’ai, sous les yeux, un lot de chartes médiévales. Certaines sont datées, d’autres, non. Là  \phantomsection
\label{p41} où l’indication figure, il faudra la vérifier : car l’expérience prouve qu’elle peut être mensongère. Manque‑t‑elle ? Il importe de la rétablir. Dans les deux cas, les mêmes moyens serviront. Par l’écriture (s’il s’agit d’un original), par l’état de la latinité, par les institutions auxquelles il est fait allusion et l’allure générale du dispositif, un acte, je suppose, répond aux usages facilement connaissables des notaires français aux environs de l’an mil. S’il se donne comme d’époque mérovingienne, voilà la fraude dénoncée. Est‑il sans date ? La voilà approximativement fixée. De même l’archéologue, qu’il se propose de classer par âges et par civilisations des outils préhistoriques ou de dépister de fausses antiquités, examine, rap­proche, distingue les formes ou les procédés de fabrication, selon des règles, des deux parts, foncièrement semblables.\par
L’historien n’est pas, il est de moins en moins ce juge d’instruction un peu grincheux dont certains manuels d’initiation, si l’on n’y prenait garde, imposeraient aisément la désobligeante image. Il n’est pas devenu, sans doute, crédule. Il sait que ses témoins peuvent se tromper où mentir. Mais avant tout, il se préoccupe de les faire parler, pour les comprendre. Ce n’est pas un des moins beaux traits de la méthode critique que d’avoir réussi, sans rien modifier de ses premiers principes, à continuer de guider la recherche dans cet élargissement.\par
Il y aurait cependant mauvais gré à le nier : le mauvais témoignage n’a pas été seulement l’excitant qui a provoqué les premiers efforts d’une technique de vérité. Il reste le cas simple d’où celle‑ci, pour développer ses analyses, doit nécessairement partir.\par
\bigbreak
\subsection[{II. À la poursuite du mensonge et de l’erreur}]{II. À la poursuite du mensonge et de l’erreur}
\noindent De tous les poisons capables de vicier un témoignage, le plus virulent est l’imposture.\par
Celle‑ci, à son tour, peut prendre deux formes. C’est d’abord la trom­perie sur l’auteur et la date : le faux au sens juridique du mot. Toutes les lettres publiées sous la signature de Marie‑Antoinette n’ont pas été écrites par elle ; il s’en trouve qui ont été fabriquées au XIX\textsuperscript{ᵉ} siècle. Vendue au Louvre comme antiquité scitho‑grecque du III\textsuperscript{ᵉ} siècle avant notre ère, la tiare dite de Saïtaphernès avait été ciselée en 1895 à Odessa. Vient ensuite la tromperie sur le fond. César, dans ses \emph{Commentaires}, dont la paternité ne saurait lui être contestée, a sciemment beaucoup déformé, beaucoup omis. La statue qu’on montre à Saint‑Denis comme représentant Philippe le Hardi est bien la figure funéraire de ce roi, telle qu’elle fut exécutée après sa mort ; mais tout indique que le sculpteur  \phantomsection
\label{p42} se borna à reproduire un modèle de convention, qui n’a d’un portrait que le nom.\par
Or, ces deux aspects du mensonge soulèvent des problèmes bien dis­tincts dont les solutions ne se commandent pas l’une l’autre.\par
Assurément, la plupart des écrits mis sous un nom supposé mentent aussi par le contenu. Les protocoles des Sages de Sion, outre qu’ils ne sont pas des Sages de Sion, s’écartent dans leur substance de la vérité autant qu’il est possible. Un soi-disant diplôme de Charlemagne se révèle-­t‑il, à l’examen, comme forgé deux ou trois siècles plus tard ? Il y a tout à parier que les générosités dont il attribue l’honneur à l’Empereur ont été également inventées. Cela même, cependant, ne saurait être admis d’avance. Car certains actes ont été fabriqués à la seule fin de répéter les dispositions de pièces parfaitement authentiques qui avaient été perdues. Exceptionnellement, un faux peut dire vrai.\par
Il devrait être superflu de rappeler qu’inversement les témoignages les plus insoupçonnables dans leur provenance avouée ne sont pas pour autant, de toute nécessité, des témoignages véridiques. Mais avant d’ac­cepter une pièce comme authentique, les érudits se donnent tant de mal pour la peser dans leurs balances qu’ils n’ont pas toujours ensuite le stoïcisme d’en critiquer les affirmations. Le doute, en particulier, hésite volontiers devant les écrits qui se présentent à l’abri de garanties juri­diques impressionnantes : actes du pouvoir ou contrats privés, pour peu que ces derniers aient été solennellement validés. Ni les uns ni les autres ne sont pourtant dignes de beaucoup de respect. Le 21 avril 1834, avant le procès des Sociétés secrètes, Thiers écrivait au Préfet du Bas‑Rhin : « Je vous recommande d’apporter le plus grand soin à fournir votre contri­bution de documents pour la grande procédure qui va s’instruire… Ce qu’il importe de bien éclaircir, c’est la correspondance de tous les anar­chistes ; c’est l’intime connexion des événements de Paris, Lyon, Stras­bourg ; c’est en un mot, l’existence d’un vaste complot embrassant la France entière. » Voilà incontestablement une documentation officielle bien préparée. Quant au mirage des chartes dûment scellées, dûment datées, la moindre expérience du présent suffit à le dissiper. Nul ne l’ignore : les actes notariés les plus régulièrement établis fourmillent d’inexactitudes volontaires, et j’ai souvenir d’avoir naguère antidaté, par ordre, ma signature au bas d’un procès‑verbal commandé par une des grandes administrations de l’État. Nos pères n’étaient pas là‑dessus plus délicats. « Donné tel jour, en tel lieu », lit‑on au bas des diplômes royaux. Mais consultez les comptes de voyage du souverain. Vous y verrez plus d’une fois qu’au jour dit, il séjournait, en fait, à plusieurs lieues de là. D’innom­brables actes d’affranchissement de serfs que nul ne songeait, sans folie, à arguer de faux, s’affirment accordés par charité pure, alors que nous pouvons mettre en face d’eux la facture de la liberté.\par

\astermono

\noindent  \phantomsection
\label{p43} Mais constater la tromperie ne suffit point. Il faut aussi en découvrir les motifs. Ne serait‑ce d’abord que pour la mieux dépister. Tant qu’un doute pourra subsister sur ses origines, il demeure en elle quelque chose de rebelle à l’analyse ; partant, de seulement à demi-prouvé. Surtout un mensonge, en tant que tel, est à sa façon un témoignage. Prouver, sans plus, que le célèbre diplôme de Charlemagne pour l’église d’Aix‑la‑Chapelle n’est pas authentique – c’est s’épargner une erreur ; ce n’est pas acquérir une connaissance. Réussissons‑nous, au contraire, à déterminer que le faux fut composé dans l’entourage de Frédéric Barberousse ? qu’il y eut pour raison d’être de servir les grands rêves impériaux ? Une vue nouvelle s’ouvre sur de vastes perspectives historiques. Voilà donc la critique amenée à chercher, derrière l’imposture, l’imposteur ; c’est‑à‑dire, confor­mément à la devise même de l’histoire, l’homme.\par
Il serait puéril de prétendre énumérer, dans leur infinie variété, les raisons qui peuvent amener à mentir. Mais les historiens, naturellement portés à intellectualiser à l’excès l’humanité, feront sagement de se sou­venir que toutes ces raisons ne sont pas raisonnables. Chez certains êtres, le mensonge (bien qu’associé généralement lui-même à un complexe de vanité et de refoulement) devient presque, selon la terminologie d’André Gide, un « acte gratuit ». Le savant allemand qui peina à rédiger, en fort bon grec, l’histoire orientale dont il attribua la paternité au fictif San­choniathon, se serait aisément acquis, à moindres frais, une estimable réputation d’helléniste. Fils d’un membre de l’Institut, appelé lui-même plus tard à siéger dans cette honorable Compagnie, François Lenormant entra dans la carrière à dix‑sept ans, en mystifiant son propre père par la fausse découverte des inscriptions de La Chapelle Saint‑Eloi, entière­ment fabriquées de ses mains ; déjà vieux et chargé de dignités, son dernier coup de maître fut, dit‑on, de publier comme originaux de Grèce quelques banales antiquités préhistoriques qu’il avait simplement ramassées dans la campagne française.\par
Or, aussi bien que des individus, il a existé des époques mythomanes. Telles, vers la fin du XVIII\textsuperscript{e} siècle et le début du XIX\textsuperscript{ᵉ}, les générations préromantiques ou romantiques. Poèmes pseudo‑celtiques mis sous le nom d’Ossian ; épopées, ballades, que Chatterton crut écrire en vieil anglais : poésies prétendument médiévales de Clothilde de Surville ; chants bretons imaginés par Villemarqué ; chants soi-disant traduits du croate par Mérimée ; chants héroïques tchèques du manuscrit de Kravoli-Dvor – et j’en passe : c’est d’un bout à l’autre de l’Europe, durant ces quelques décades, comme une vaste symphonie de fraudes. Le Moyen Âge, surtout du VIII\textsuperscript{ᵉ} au XII\textsuperscript{ᵉ} siècle, présente un autre exemple de cette épidémie collective. Certes la plupart des faux diplômes, des faux décrets ponti­ficaux, des faux capitulaires, alors forgés en si grand nombre, le furent  \phantomsection
\label{p44} par intérêt. Assurer à une église un bien contesté, appuyer l’autorité du Siège Romain, défendre les moines contre l’Évêque, les évêques contre les métropolitains, le Pape contre les souverains temporels, l’empereur contre le Pape, les faussaires ne voyaient pas plus loin. Le fait caracté­ristique n’en demeure pas moins qu’à ces tromperies des personnages d’une piété et, souvent, d’une vertu incontestées ne craignaient pas de prêter la main. Visiblement, elles n’offusquaient guère la moralité com­mune. Quant au plagiat, il semblait universellement, en ce temps, l’acte le plus innocent du monde : l’annaliste, l’hagiographe s’appropriaient sans remords, par passages entiers, les écrits d’auteurs plus anciens. Rien de moins « futuriste » cependant, que ces deux sociétés, par ailleurs de type si différent. À sa foi, comme à son droit, le Moyen Âge ne connaissait d’autre fondement que la leçon des ancêtres. Le romantisme souhaitait s’abreuver à la source vive du primitif autant que du populaire. Ainsi les périodes les plus attachées à la tradition ont été aussi celles qui prirent avec son exact héritage le plus de libertés. Comme si, par une singulière revanche d’un irrésistible besoin de création, à force de vénérer le passé on était naturellement conduit à l’inventer.\par

\astermono

\noindent Au mois de juillet 1857, le mathématicien Michel Chasles communiqua à l’Académie des Sciences tout un lot de lettres inédites de Pascal, que lui avait vendues son fournisseur habituel, l’illustre faussaire Vrain‑Lucas. Il en ressortait que l’auteur des \emph{Provinciales} avait, avant Newton, formulé le principe de l’attraction universelle. Un savant anglais s’étonna. Com­ment expliquer, disait‑il en substance, que ces textes fassent état de mesures astronomiques effectuées bien des années après la mort de Pascal et dont Newton lui-même n’eut connaissance qu’une fois publiés les premiers éditoriaux de son ouvrage ? Vrain‑Lucas n’était pas homme à s’embarrasser pour si peu. Il se remit à son établi ; et bientôt, réarmé par ses soins, Chasles put produire de nouveaux autographes. Pour signa­taire, ils avaient cette fois Galilée ; pour destinataire, Pascal. Ainsi l’énigme était éclaircie : l’illustre astronome avait fourni les observations ; Pascal, les calculs. Le tout, des deux parts, secrètement. Il est vrai : Pascal, à la mort de Galilée, n’avait que dix‑huit ans. Mais quoi ? ce n’était qu’une raison de plus d’admirer la précocité de son génie.\par
Voilà bien cependant, remarqua l’infatigable objecteur, une autre étrangeté : dans une de ces lettres, datée de 1641, on voit Galilée se plaindre de n’écrire qu’au prix de beaucoup de fatigue pour ses yeux. Or, ne savons­nous pas que depuis la fin de l’année 1637, il était en réalité complètement aveugle ? Pardon, répliqua peu après le bon Chasles ; à cette cécité chacun, je l’accorde, a cru jusqu’ici. Bien à tort. Car surgie à point nommé pour confondre la commune erreur, je puis maintenant verser aux débats une pièce décisive. Un autre savant italien le faisait connaître à Pascal, le \textsubscript{p.45 2} décembre 1641 : à cette date, Galilée, dont la vue sans doute faiblissait depuis plusieurs années, venait tout juste de la perdre entièrement…\par
Tous les imposteurs, assurément, n’ont pas déployé autant de fécondité que Vrain‑Lucas ; ni toutes les dupes la candeur de sa lamentable victime ; mais que l’insulte au vrai soit un engrenage, que tout mensonge en entraîne presque forcément, à sa suite, beaucoup d’autres, appelés à se prêter en apparence du moins, un mutuel appui – l’expérience de la vie l’enseigne et celle de l’histoire le confirme. C’est pourquoi tant de faux célèbres se présentent par grappes. Faux privilèges du siège de Canterbury, faux privilèges du duché d’Autriche – souscrits par tant de grands souverains, de Jules César à Frédéric Barberousse – faux, en arbre généalogique, de l’affaire Dreyfus : on croirait (et je n’ai voulu citer que quelques exem­ples) voir un foisonnement de colonies microbiennes. La fraude, par nature, enfante la fraude.\par

\astermono

\noindent Il est enfin, une forme plus insidieuse, de la tromperie. Au lieu de la contre‑vérité brutale, pleine et, si je puis dire, franche, c’est le sour­nois remaniement ; interpolations dans des chartes authentiques ; dans la narration, broderies, sur un fond grossièrement véridique de détails inventés. On interpole généralement par intérêt. On brode souvent pour orner. Les ravages qu’une fallacieuse esthétique exerça sur l’historio­graphie antique ou médiévale ont été souvent dénoncés. Leur part n’est peut‑être pas beaucoup moindre dans notre presse. Fût‑ce aux dépens de la véracité, le plus modeste nouvelliste campe volontiers ses person­nages selon les conventions d’une rhétorique dont l’âge n’a point usé les prestiges et dans nos bureaux de rédaction, Aristote et Quintilien comp­tent plus de disciples qu’on ne le croit communément.\par
Certaines conditions techniques même semblent favoriser ces défor­mations. Quand fut condamné, en 1917, l’espion Bolo, un quotidien dit‑on publia, dès le 6 avril, le récit de l’exécution. D’abord fixée, en effet, à cette date, elle n’eut lieu réellement que onze jours plus tard. Le journaliste avait établi son « papier » d’avance ; persuadé que l’événement suivrait au jour prévu, il estima inutile de vérifier. Je ne sais ce que vaut l’anecdote. Certainement des fautes aussi lourdes sont exception­nelles. Mais que, pour aller plus vite – car il faut avant tout livrer la copie à temps – les reportages de scènes attendues soient parfois préparés avant l’heure, la supposition n’a rien d’invraisemblable. Presque toujours, soyons‑en convaincus, le canevas, après observation, sera modifié, s’il y a lieu sur tous les points importants ; on doute, par contre, que beaucoup de retouches soient apportées aux traits accessoires, jugés nécessaires à la couleur et que personne ne songera à contrôler. Du moins, c’est ce qu’un profane croit entrevoir. On souhaiterait qu’un homme du métier nous apportât là‑dessus de sincères lumières. Le journal malheureusement  \phantomsection
\label{p46} n’a pas encore trouvé son Mabillon. Ce qui est sûr, c’est que l’obéissance à un code, un peu désuet, de bienséance littéraire, le respect d’une psy­chologie stéréotypée, la rage du pittoresque ne sont pas prés de perdre leur place dans la galerie des fauteurs de mensonges.\par

\astermono

\noindent De la feinte pure et simple à l’erreur entièrement involontaire, il est bien des degrés. Ne serait‑ce qu’en raison de la facile métamorphose par où la bourde la plus sincère se mue, l’occasion aidant, en menterie. In­venter suppose un effort auquel répugne la paresse d’esprit commune à la plupart des hommes. Combien il est plus commode d’accepter complai­samment une illusion, à son origine spontanée, qui flatte l’intérêt du moment ?\par
Voyez l’épisode célèbre de « l’avion de Nuremberg ». Encore que le point n’ait jamais été parfaitement éclairci, il semble bien qu’un avion commercial français survola la ville peu de jours avant la déclaration de guerre. Il est probable qu’on le prit pour un appareil militaire. Il n’est pas invraisemblable que, dans une population déjà en proie aux fantômes de la mêlée prochaine, le bruit se soit alors répandu de bombes, ça et là jetées. Il est sûr pourtant qu’il n’en fut point lancé ; que les gouvernants de l’Empire allemand possédaient tous les moyens de réduire ce faux bruit à néant ; que, par suite, en l’accueillant, sans contrôle, pour en faire un motif de guerre, ils ont proprement menti. Mais sans rien imaginer, ni même, peut‑être, sans avoir d’abord une conscience très claire de leur imposture. L’absurde rumeur fut crue parce qu’il était utile de la croire. De tous les types du mensonge, celui qu’on se fait à soi-même ne compte point parmi les moins fréquents et le mot de sincérité recouvre un concept un peu gros, qui ne saurait être manié sans y introduire beaucoup de nuances.\par

\astermono

\noindent Il n’en est pas moins vrai que beaucoup de témoins se trompent en toute bonne foi. Voilà donc le moment venu pour l’historien de mettre à profit les précieux résultats dont, depuis quelques décades, l’observation sur le vivant a armé une discipline presque nouvelle : la psychologie du témoignage. En tant qu’elles intéressent nos études, ces acquisitions semblent être, pour l’essentiel, les suivantes.\par
À en croire Guillaume de Saint‑Thierry, son disciple et ami saint Bernard fut un jour très surpris d’apprendre que la chapelle où, jeune moine, il suivait quotidiennement les offices divins, s’ouvrait au chevet par trois fenêtres ; il s’était toujours imaginé qu’elle n’en avait qu’une. Sur ce trait, l’hagiographe à son tour s’étonne et admire : quel parfait serviteur  \phantomsection
\label{p47} de Dieu un pareil détachement des choses de la terre ne présageait‑il point ? Sans doute, Bernard paraît bien avoir été d’une distraction peu commune ; si du moins, il est vrai, comme on le raconte aussi, qu’il lui arriva plus tard de côtoyer toute une journée durant le Léman sans y prendre garde. De nombreuses épreuves cependant l’attestent : pour se tromper grossièrement sur les réalités qui devraient, semble‑t‑il, nous être les mieux connues, point n’est besoin de compter parmi les princes de la mystique. Les étudiants du professeur Claparède, à Genève, se sont montrés au cours d’expériences célèbres, aussi incapables de décrire cor­rectement le vestibule de leur Université que le Docteur « à la parole de miel », l’église de son monastère. La vérité est que, dans la plupart des cerveaux, le monde environnant ne trouve que de médiocres appareils enregistreurs. Ajoutez que les témoignages n’étant, au propre, que l’expres­sion de souvenirs, les erreurs premières de la perception risquent toujours de s’y compliquer d’erreurs de mémoire, de cette coulante, de cette » escou­lourjante » mémoire, que dénonçait déjà un de nos vieux juristes.\par
Chez certains esprits, l’inexactitude prend des allures véritablement pathologiques – serait‑il trop irrévérencieux de proposer, pour cette psychose, le nom de « maladie de Lamartine » ? – Chacun le sait, ces personnes‑là ne sont pas ordinairement les moins promptes à affirmer. Mais, s’il est ainsi des témoins plus ou moins suspects et sûrs, l’expérience prouve qu’il ne s’en rencontre point dont les dires soient également dignes de foi sur tous les sujets et en toutes circonstances. Au sens absolu, le bon témoin n’existe pas ; il n’y a que de bons ou mauvais témoignages. Deux ordres de cause, principalement, altèrent, jusque chez l’homme le mieux doué, la véracité des images cérébrales. Les unes tiennent à l’état momentané de l’observateur : ce sont la fatigue, par exemple, ou l’émo­tion. D’autres, au degré de son attention. À peu d’exceptions près, on ne voit, on n’entend bien que ce qu’on s’attachait à percevoir. Un médecin se rend au chevet d’un malade ; je le croirai plus volontiers sur l’aspect de son patient, dont il a examiné avec soin le comportement, que sur les meubles de la chambre, sur laquelle il n’a probablement jeté que des regards distraits. C’est pourquoi en dépit d’un préjugé assez commun, les objets les plus familiers – comme, pour saint Bernard, la chapelle de Cîteaux – comptent à l’ordinaire parmi ceux dont il est le plus difficile d’obtenir une juste description : car la familiarité amène presque néces­sairement l’indifférence.\par
Or, beaucoup d’événements historiques n’ont pu être observés que dans des moments de violent trouble émotif, ou par des témoins dont l’atten­tion, soit sollicitée trop tard, s’il y avait surprise, soit retenue par les soucis dé l’action immédiate, était incapable de se porter avec assez de force sur les traits auxquels l’historien avec raison attribuerait aujour­d’hui un intérêt prépondérant. Certains cas sont célèbres. Le premier coup de feu qui, le 25 février 1848, devant l’Hôtel des Affaires Étrangères  \phantomsection
\label{p48} déclencha l’émeute – d’où devait sortir, à son tour, la Révolution – fut‑il tiré de la troupe ? ou de la foule ? Nous ne le saurons vraisembla­blement jamais. Comment, d’autre part, chez les chroniqueurs, prendre désormais au sérieux les grands morceaux descriptifs, les peintures minu­tieuses des costumes, des gestes, des cérémonies, des épisodes guerriers, par quelle routine obstinée conserver la moindre illusion sur la véracité de tout ce bric à brac, dont se repaissait le menu fretin des historiens romantiques, alors qu’autour de nous pas un témoin n’est en mesure de retenir correctement, dans leur intégralité, les détails sur lesquels on a si naïvement interrogé les vieux auteurs ? Au mieux, ces tableaux nous donnent le décor des actions, tel que, du temps de l’écrivain, on s’imaginait qu’il devait être. Cela est extrêmement instructif ; ce n’est pas le genre de renseignements que les amateurs de pittoresque demandent générale­ment à leurs sources.\par
Il convient de voir cependant à quelles conclusions ces remarques, pessimistes peut‑être, seulement en apparence, engagent dorénavant nos études. Elles n’atteignent pas la structure élémentaire du passé. Le mot de Bayle demeure toujours juste. « Jamais on n’objectera rien qui vaille contre cette vérité que César a battu Pompée et que, dans quelque sorte de principe qu’on veuille passer en disputant, on ne trouvera guère de choses plus inébranlables que cette proposition : \emph{César et Pompée ont existé et n’ont pas été une simple modification de l’âme de ceux qui ont écrit leur vie. »} Il est vrai : s’il ne devait subsister comme assuré, que quelques faits de ce type dépourvus d’explication, l’histoire se réduirait à une suite de notations grossières, sans grande valeur intellectuelle. Par bon­heur, tel n’est point le cas. Les seules causes que la psychologie du témoi­gnage frappe ainsi d’une fréquente incertitude sont les antécédents tout à fait immédiats. Un grand événement peut se comparer à une explosion. Dans quelles conditions, exactement, se produisit le dernier choc molé­culaire, indispensable à la détente des gaz ? Force sera souvent de nous résigner à l’ignorer. Cela est regrettable sans doute, mais les chimistes sont‑ils toujours beaucoup mieux placés ? Cela n’empêche point que la composition du mélange détonant ne reste parfaitement susceptible d’ana­lyse. La Révolution de 1848 – ce mouvement si clairement déterminé dont, par une étrange aberration, certains historiens ont cru pouvoir faire le type même de l’événement fortuit, de nombreux facteurs, très divers et très actifs que, dès l’heure, un Tocqueville a su entrevoir, l’avaient de longue date préparée. La fusillade du Boulevard des Capucines fut‑elle autre chose que l’ultime petite étincelle ?\par
Aussi bien, nous le verrons, des causes prochaines ne se dérobent pas seulement trop souvent à l’observation de nos répondants ; par suite, à la nôtre. En elles‑mêmes, elles constituent aussi dans l’histoire la part privilégiée de l’imprévisible, du « hasard ». Nous pouvons nous consoler, sans trop de peine, que les infirmités du témoignage les dissimulent  \phantomsection
\label{p49} ordinairement aux plus subtils de nos instruments. Même mieux connues, leur rencontre avec les grandes chaînes causales de l’évolution représenterait le résidu de mensonges que notre science ne parviendra jamais à éliminer, qu’elle n’a pas le droit de prétendre éliminer. Quant aux res­sorts intimes des destinées humaines, aux vicissitudes de la mentalité ou de la sensibilité, des techniques, de la structure sociale ou économique, les témoins que nous interrogeons là‑dessus ne sont guère sujets aux fra­gilités de la perception momentanée. Par un heureux accord, que Voltaire avait déjà entrevu, ce qu’il y a en histoire de plus profond pourrait être aussi ce qu’il y a de plus sûr.\par

\astermono

\noindent Éminemment variable d’individu à individu, la faculté d’observation n’est pas, non plus, une constante sociale. Certaines époques s’en sont trouvées plus que d’autres dépourvues. Si médiocre, par exemple, que demeure aujourd’hui, chez la plupart des hommes, l’appréciation des nombres, elle n’est plus aussi universellement fautive que parmi les anna­listes médiévaux ; notre perception, comme notre civilisation, s’est imprégnée de mathématiques. Cependant, si les erreurs du témoignage n’étaient déterminées, en dernière analyse, que par les faiblesses des sens ou de l’attention, l’historien n’aurait guère, en somme, qu’à en abandonner l’étude au psychologue. Mais par-delà ces petits accidents cérébraux, d’une nature assez commune, beaucoup d’entre elles remontent à des causes autrement significatives d’une atmosphère sociale particulière. C’est pourquoi elles prennent souvent, à leur tour, comme le mensonge, une valeur documentaire.\par
Au mois de septembre 1917, le régiment d’infanterie auquel j’appar­tenais tenait les tranchées du Chemin des Dames, au nord de la petite ville de Braisne. Dans un coup de main, nous fîmes un prisonnier. C’était un réserviste, négociant de son métier et originaire de Brême, sur la Weser. Peu après, une curieuse histoire nous vint de l’arrière des lignes. « L’espionnage allemand », disaient à peu près ces camarades bien informés, « quelle merveille ! On enlève un de leurs petits postes au cœur de la France. Qu’y surprend‑on ? Un commerçant établi durant la paix à quelques kilomètres de là : à Braisne ». Le coq-à-l’âne paraît clair. Gardons‑nous cependant d’en rendre un compte trop simple. En accusera‑t‑on sans plus une erreur de l’ouïe ? Ce serait, en tout état de cause, s’exprimer assez inexactement. Car, plutôt que mal entendu, le nom véritable avait été sans doute mal compris : généralement inconnu, il n’accrochait pas l’attention. Par une pente naturelle de l’esprit, on crut saisir à sa place un nom familier. Mais il y a plus. Dans ce premier travail d’interprétation, un second, également inconscient, se trouvait déjà impliqué. L’image trop souvent véridique des ruses allemandes avait été popularisée par des  \phantomsection
\label{p50} innombrables récits : elle flattait au vif la sensibilité romanesque des foules. La substitution de Braisne à Brême s’harmonisait trop bien avec cette hantise pour ne pas s’imposer, en quelque sorte, spontanément.\par
Or, tel est le cas d’un grand nombre de déformations du témoignage. L’erreur presque toujours est orientée d’avance. Surtout, elle ne se répand, elle ne prend vie qu’à la condition de s’accorder avec les parti pris de l’opinion commune ; elle devient alors comme le miroir où la conscience collective contemple ses propres traits. Beaucoup de maisons belges pré­sentent, sur leurs façades, d’étroites ouvertures, destinées à faciliter aux recrépisseurs le placement de leurs échafaudages ; dans ces innocents artifices de maçons, les soldats allemands, en 1914, n’auraient jamais songé à voir autant de meurtrières, préparées pour les francs‑tireurs, si leur imagination n’avait été hallucinée de longue date par la crainte des guérillas. Les nuages n’ont point changé de forme depuis le Moyen Âge. Nous n’y apercevons plus cependant ni croix ni épée miraculeuses. La queue de la comète qu’observa le grand Ambroise Paré n’était vraisem­blablement guère différente de celles qui balaient parfois nos ciels. Il crut pourtant y découvrir toute une panoplie d’armes étranges. L’obéissance au préjugé universel avait triomphé de l’habituelle exactitude de son regard ; et son témoignage, comme tant d’autres, renseigne, non sur ce qu’il vit en réalité, mais sur ce que, de son temps, on estimait naturel de voir.\par

\astermono

\noindent Cependant, pour que l’erreur d’un témoin devienne celle de beaucoup d’hommes, pour qu’une mauvaise observation se métamorphose en un faux bruit, il faut aussi que l’état de la société favorise cette diffusion. Tous les types sociaux ne lui sont pas, à beaucoup près, également pro­pices. Là-dessus, les extraordinaires troubles de la vie collective que nos générations ont vécus constituent autant d’admirables expériences. Celles du moment présent, à dire vrai, sont trop proches de nous pour souffrir encore une exacte analyse. La guerre de 1914‑1918 permet davantage le recul.\par
Chacun sait combien ces quatre années se sont montrées fécondes en fausses nouvelles. Notamment, chez les combattants. C’est dans la société très particulière des tranchées que leur formation semble la plus intéres­sante à étudier.\par
Le rôle de la propagande et de la censure fut, à sa façon, considérable. Mais exactement inverse de ce que les créateurs de ces institutions atten­daient d’elles. Comme l’a fort bien dit un humoriste : « l’opinion prévalait aux tranchées que tout pouvait être vrai à l’exception de ce qu’on laissait imprimer ». On ne croyait pas aux journaux ; aux lettres, guère plus ; car, outre qu’elles arrivaient irrégulièrement, elles passaient pour très  \phantomsection
\label{p51} surveillées. D’où un renouveau prodigieux de la tradition orale, mère antique des légendes et des mythes. Par un coup hardi, que n’eût jamais osé rêver le plus audacieux des expérimentateurs, les gouvernements, abolissant les siècles écoulés, ramenaient le soldat du front aux moyens d’information et à l’état d’esprit des vieux âges, avant le journal, avant la feuille de nouvelles, avant le livre.\par
Ce n’était pas, ordinairement, sur la ligne de feu que les rumeurs prenaient naissance. Pour cela, les petits groupes y étaient trop isolés les uns des autres. Le soldat n’avait point le droit de se déplacer sans ordre ; il ne l’eût fait, d’ailleurs, le plus souvent qu’au péril de sa vie. Par mo­ments, circulaient des voyageurs intermittents : agents de liaison, télé­phonistes réparant leurs lignes, observateurs d’artillerie. Ces personnages considérables frayaient peu avec le simple troupier. Mais il y avait aussi des communications périodiques, beaucoup plus importantes. Elles étaient imposées par le souci de la nourriture. L’agora de ce petit monde des abris et des postes de guet, ce furent les cuisines. Là, une ou deux fois par jour, les ravitailleurs venus des divers points du secteur, se retrou­vaient et bavardaient entre eux ou avec les cuisiniers. Ceux-ci savaient beaucoup, car placés au carrefour de toutes les unités, ils avaient en outre le rare privilège de pouvoir quotidiennement échanger quelques mots avec les conducteurs du train régimentaire, hommes fortunés qui cantonnaient au voisinage des états-majors. Ainsi, pour un instant, autour des feux en plein vent ou des foyers des roulantes, se nouaient, entre des milieux singulièrement dissemblables, des liens précaires. Puis les corvées s’ébranlaient par les pistes et les boyaux, et ramenaient jusqu’au plus avant du front, avec leurs marmites, les renseignements vrais ou faux, presque toujours déformés en tout cas, et prêts là-bas pour une nouvelle élaboration. Sur les plans directeurs, un peu en arrière des traits enlacés qui dessinaient les premières positions, on aurait pu ombrer de hachures une bande continue : c’eût été la zone de formation des légendes.\par
Or, l’histoire a connu plus d’une société régie, en gros, par des condi­tions analogues ; à cette différence près, qu’au lieu d’être l’effet passager d’une crise tout exceptionnelle, elles y représentaient la trame normale de la vie. Là aussi, la transmission orale était presque la seule efficace. Là aussi, entre les éléments très fragmentés, les liaisons s’opéraient pres­que exclusivement par des intermédiaires spécialisés, ou en des points de jonction définis. Colporteurs, jongleurs, pèlerins, mendiants tenaient la place du petit peuple errant des boyaux. Les rencontres régulières se produisaient dans les marchés ou à l’occasion des fêtes religieuses. Ainsi par exemple, durant le haut Moyen Âge. Faites à coup d’interrogatoires avec les passants pour informateurs, les chroniques monastiques ressem­blent beaucoup aux mémentos qu’auraient pu tenir, s’ils en avaient eu le goût, nos caporaux d’ordinaire. Ces sociétés‑là ont toujours été, pour les fausses nouvelles, un excellent bouillon de culture. Des relations  \phantomsection
\label{p52} fréquentes entre les hommes rendent aisée la comparaison entre les divers récits. Elles excitent le sens critique. Au contraire, on croit fortement le narrateur qui, à longs intervalles, apporte par des chemins difficiles les rumeurs lointaines.
\subsection[{III. Essai d’une logique de la méthode critique}]{III. Essai d’une logique de la méthode critique}
\noindent La critique du témoignage, qui travaille sur des réalités psychiques, demeurera toujours un art de finesse. Il n’existe point pour elle de livre de recettes. Mais c’est aussi un art rationnel qui repose sur la pratique méthodique de quelques grandes opérations de l’esprit. Elle a, en un mot, sa dialectique propre, qu’il convient de chercher à dégager.\par

\astermono

\noindent Supposons que, d’une civilisation disparue, un seul objet subsiste ; qu’en outre, les conditions de sa découverte interdisent de le mettre en rapport même avec des traces étrangères à l’homme, telles que des sédi­mentations géologiques (car, dans cette recherche des liaisons, la nature inanimée peut avoir sa part). Il sera tout à fait impossible de dater ce vestige unique ni de se prononcer sur son authenticité. On ne rétablit, en effet, jamais une date, on ne contrôle et, en somme, on n’interprète jamais un document que par insertion dans une série chronologique ou un ensemble synchrone. C’est en rapprochant les diplômes mérovingiens tantôt entre eux, tantôt avec d’autres textes différents, d’époque ou de nature, que Mabillon a fondé la diplomatique ; c’est de la confrontation des récits évangéliques qu’est née l’exégèse. À la base de presque toute critique s’inscrit un travail de comparaison.\par
Mais les résultats de cette comparaison n’ont rien d’automatique. Elle aboutit, nécessairement, à déceler tantôt des ressemblances, tantôt des différences. Or, selon les cas, l’accord d’un témoignage avec les témoi­gnages voisins peut imposer des conclusions exactement inverses.\par

\astermono

\noindent Il faut considérer d’abord le cas élémentaire du récit. Dans ses \emph{Mé­moires}, qui ont fait battre tant de jeunes cœurs, Marbot raconte, avec une grande abondance de détails, un trait de bravoure dont il se donne pour le héros : à l’en croire, il aurait, dans le nuit du 7 au 8 mai 1809, traversé en barque les flots démontés du Danube, alors en pleine crue, pour enlever sur l’autre bord quelques prisonniers autrichiens. Comment vérifier l’anecdote ? En appelant à la rescousse d’autres témoignages. Nous possédons les ordres, les carnets de marche, les comptes rendus des armées en présence ; ils attestent que, durant la fameuse nuit, le corps  \phantomsection
\label{p53} autrichien, dont Marbot prétend avoir trouvé les bivouacs sur la rive gauche, occupait encore la rive opposée. De la Correspondance même de Napoléon, il ressort, par ailleurs, que le 8 mai, les hautes eaux n’avaient pas commencé. Enfin on a retrouvé une demande de promotion établie le 30 juin 1809, par Marbot en personne. Parmi les titres qu’il y invoque il ne souffle mot de son soi-disant exploit du mois précédent. D’un côté, voilà donc les \emph{Mémoires}, de l’autre, tout un lot de textes qui les démentent. Il convient de départager ces irréconciliables témoins. Quelle alternative jugera‑t‑on la plus vraisemblable ? Que sur le moment même, les états-­majors, l’Empereur lui-même, se soient trompés (à moins que, Dieu sait pourquoi, ils aient sciemment altéré la réalité) ; que le Marbot de 1809, en mal d’avancement, ait péché par folle modestie : ou que, beaucoup plus tard, le vieux guerrier, dont les hâbleries sont, par ailleurs, notoires, ait donné un nouveau croc-en-jambe à la vérité ? Personne assurément n’hésitera : les \emph{Mémoires}, une fois de plus, ont menti.\par
Ici donc, la constatation d’un désaccord a ruiné un des témoignages opposés. Il fallait que l’un d’eux succombât. Ainsi l’exigeait le plus uni­versel des postulats logiques : qu’un événement puisse à la fois être et ne pas être, le principe de contradiction l’interdit impitoyablement. Il se rencontre de par le monde des érudits dont la bienveillance s’épuise à découvrir entre des affirmations antagonistes, un moyen terme : c’est imiter le marmot qui, interrogé sur le carré de 2, comme l’un de ses voi­sins lui soufflait « 4 », et l’autre « 8 », crut tomber juste en répondant « 6 ».\par
Restait ensuite à faire choix du témoignage rejeté et de celui qui devait subsister. Une analyse psychologique en a décidé : chez les témoins, tout, à tour, on a pesé les raisons présumées de la véracité, du mensonge ou de l’erreur. Il s’est trouvé, en l’espèce, que cette appréciation portait un caractère d’évidence presque absolue. Elle ne manquera pas de se montrer, en d’autres circonstances, affectée d’un beaucoup plus fort coefficient d’incertitude. Des conclusions qui se fondent sur un délicat dosage de motifs supposent, de l’infiniment probable au tout juste vraisemblable, une longue dégradation.\par

\astermono

\noindent Mais voici maintenant des exemples d’un autre type.\par
Une charte, qui se dit du XII\textsuperscript{ᵉ} siècle, est écrite sur papier, alors que tous les originaux de cette époque jusqu’ici retrouvés sont sur parchemin ; la forme des lettres y apparaît très différente du dessin qu’on observe sur les autres documents de même date ; la langue abonde en mots et en tours de style étrangers à leur usage unanime. Ou bien, la taille d’un outil prétendument paléolithique révèle des procédés de fabrication employés, à notre connaissance, seulement en des temps beaucoup plus  \phantomsection
\label{p54} proches de nous. Nous conclurons que la charte, que l’outil sont des faux. Comme précédemment, le désaccord condamne. Mais pour des raisons d’une nature très différente.\par
L’idée qui, cette fois, guide l’argumentation est que, dans une même génération d’une même société, il règne une similitude de coutumes et de techniques trop forte pour permettre à aucun individu de s’éloigner sensiblement de la pratique commune. Nous tenons pour assuré qu’un Français du temps de Louis VII traçait ses jambages à peu près comme ses contemporains \footnote{J’ai, dans ma jeunesse, entendu un très illustre érudit qui fut directeur de l’École des Chartes, nous dire assez fièrement : « Je date sans erreur l’écriture d’un manuscrit à une vingtaine d’années près. » Il n’oubliait qu’une chose : beau­coup d’hommes, de scribes, vivent plus de quarante ans – et, si les écritures parfois se modifient en vieillissant, c’est rarement pour s’adapter aux nouvelles écritures ambiantes. Il a dû y avoir, aux environs de 1200, des scribes qui, sexa­génaires, écrivaient encore comme on leur avait appris à le faire vers 1150. En fait, l’histoire de l’écriture retarde, étrangement, sur celle du langage. Elle attend son Diez – ou son Meillet.} ; qu’il s’exprimait à peu près dans les mêmes termes ; qu’il se servait des mêmes matières ; que, si un ouvrier des tribus magda­léniennes, pour découper ses pointes d’os, avait disposé d’une scie mécanique, ses camarades en auraient usé comme lui. Le postulat, en résumé ­est là d’ordre sociologique. Confirmées, sans nul doute, dans leur valeur générale par une constante expérience de l’humanité, les notions d’endos­mose collective, de pression du nombre, d’impérieuse imitation sur les­quelles il repose, se confondent au total avec le concept même de civi­lisation.\par

\astermono

\noindent Il ne faut pas cependant que la ressemblance soit trop forte. Elle ces­serait alors de déposer en faveur du témoignage. Elle en prononcerait au contraire la condamnation.\par
Quiconque a pris part à la bataille de Waterloo a su que Napoléon y fut vaincu. Le témoin, trop original, qui nierait la défaite, nous le tien­drions pour un faux témoin. Par ailleurs, que Napoléon ait été vaincu à Waterloo, nous consentons qu’il n’y ait pas, en français, beaucoup de manières distinctes de le dire, pour peu qu’on se borne à cette simple et grossière constatation. Mais deux témoins, ou soi-disant tels, décrivent-­ils la bataille exactement dans le même langage ? Ou, fût‑ce au prix d’une certaine diversité d’expression, exactement avec les mêmes détails ? On conclura sans hésiter que l’un des deux a copié l’autre ou qu’ils copièrent tous deux un modèle commun. Notre raison refuse, en effet, d’admettre que, placés nécessairement en des points différents de l’espace et doués de facultés d’attention inégales, deux observateurs aient pu noter, trait pour trait, les mêmes épisodes : que, parmi les innombrables mots de la langue française, deux écrivains travaillant indépendamment l’un de l’autre se soient fortuitement trouvés faire choix des mêmes termes, pareillement assemblés, pour raconter les mêmes choses. Si les deux récits se donnent pour pris directement à la réalité, il faut donc que l’un d’eux, au moins, mente.\par
Considérez encore, sur deux monuments antiques, sculptées de part et d’autre dans la pierre, ces deux scènes guerrières. Elles se rapportent à des campagnes différentes. Elles sont représentées pourtant sous des  \phantomsection
\label{p55} traits presque pareils. L’archéologue dira : l’un des deux artistes cer­tainement a plagié l’autre, à moins qu’ils ne se soient tous deux contentés de reproduire un poncif d’école. Peu importe que les combats aient été séparés seulement par un court intervalle ; qu’ils aient opposé peut‑être des adversaires pris dans les mêmes peuples – Égyptiens contre Hittites, Assur contre Élam. Nous nous révoltons contre l’idée que, dans l’immense variété des attitudes humaines, deux actions distinctes, à des moments divers, aient pu voir se renouveler exactement les mêmes gestes. Comme témoignage sur les fastes militaires qu’elle feint de retracer, l’une des deux images au moins – sinon les deux – est proprement un faux.\par
Ainsi la critique se meut entre ces deux extrêmes : la similitude qui justifie et celle qui discrédite. C’est que le hasard des rencontres a ses limites et que l’accord social est de mailles, à tout prendre, assez lâche. En d’autres termes, nous estimons qu’il y a dans l’univers et dans la société assez d’uniformité pour exclure l’éventualité d’écarts trop marqués. Mais cette uniformité, telle que nous nous la représentons, se tient à des caractères très généraux. Elle suppose, pensons‑nous, en quelque sorte elle englobe, aussitôt qu’on pénètre plus avant dans le réel, un nombre de combinaisons possibles trop proche de l’infini pour que leur répétition spontanée soit concevable : il y faut un acte volontaire d’imitation. Si bien qu’au bout du compte, la critique du témoignage s’appuie sur une instinctive métaphysique du semblable et du dissemblable, de l’un et du multiple.\par

\astermono

\noindent Reste, lorsque l’hypothèse de la copie s’est ainsi imposée, à fixer les directions d’influence. Dans chaque couple, les deux documents ont‑ils puisé à une source commune ? À supposer que l’un d’eux, au contraire, soit original, auquel reconnaître ce titre ? Parfois la réponse sera fournie par des critères extérieurs, tels que, par exemple, les dates relatives, s’il est possible de les établir. À défaut de ce secours, l’analyse psycholo­gique, s’aidant des caractères internes de l’objet ou le texte, reprendra ses droits.\par
Il va de soi qu’elle ne comporte pas de règles mécaniques. Faut‑il, par exemple, poser en principe, comme certains érudits semblent le faire, que les remanieurs vont constamment multipliant les inventions nou­velles ; en sorte que le texte plus sobre et le moins invraisemblable aurait toujours chance d’être le plus ancien ? Cela est vrai quelquefois. D’ins­cription en inscription, on voit les chiffres des ennemis tombés sous les coups d’un roi d’Assyrie s’enfler démesurément. Mais il arrive aussi que la raison se rebelle. La plus fabuleuse des Passions de saint Georges est la première en date ; par la suite, reprenant le vieux récit, les rédacteurs  \phantomsection
\label{p56} successifs en ont sacrifié d’abord tel trait, puis tel autre, dont l’intempé­rante fantaisie les choquait. Il y a bien des façons différentes d’imiter. Elles varient selon l’individu, parfois selon des modes communes à une génération. Pas plus qu’aucune autre attitude mentale, elles ne sauraient se présupposer, sous prétexte qu’elles nous paraîtraient « naturelles ».\par
Heureusement, les plagiaires se trahissent souvent par leurs mala­dresses. Quand ils ne comprennent pas leur modèle, leurs contre‑sens dénoncent la fraude. Cherchent‑ils à déguiser leurs emprunts ? La gau­cherie de leurs stratagèmes les perd. J’ai connu un lycéen qui, durant une composition, l’œil fixé sur le devoir de son voisin, en transcrivait soigneusement les phrases à rebours. Avec beaucoup d’esprit de suite, il muait les sujets en attributs et l’actif au passif. Il ne réussit qu’à fournir à son professeur un excellent exemple de critique historique.\par
Démasquer une imitation, c’est, là où nous croyons d’abord avoir affaire à deux ou plusieurs témoins, n’en plus laisser subsister qu’un. Deux contemporains de Marbot, le comte de Ségur et le général Pelet, ont donné du prétendu passage du Danube un récit analogue au sien. Mais Ségur venait après Pelet. Il l’a lu. Il n’a guère fait que le copier. Quant à Pelet, il a beau avoir écrit avant Marbot ; il était son ami, il l’avait sans nul doute souvent entendu évoquer ses fictives prouesses – car l’infatigable vantard se préparait volontiers, en dupant ses familiers, à mystifier la postérité. Marbot reste donc bien notre unique garant, puisque ses cautions apparentes n’ont parlé que d’après lui. Lorsque Tite‑Live reproduit Polybe, fût‑ce en l’ornant, c’est Polybe qui est notre seule autorité. Lorsque Éginhard, sous couleur de nous peindre Charle­magne, démarque le portrait d’Auguste par Suétone, il n’y a plus, au sens propre, de témoin du tout.\par
Il arrive enfin que, derrière le soi-disant témoin, un souffleur se cache qui ne voulait point se nommer. Étudiant le procès des Templiers, Robert Lea a observé que, lorsque deux accusés appartenant à deux maisons différentes étaient interrogés par le même inquisiteur, on les voyait inva­riablement avouer les mêmes atrocités et les mêmes blasphèmes. Venus de la même maison, étaient‑ils, au contraire, interrogés par des inquisiteurs différents ? Les aveux cessent de concorder. La conclusion évi­dente est que le juge dictait les réponses. C’est un trait dont les annales judiciaires fourniraient, j’imagine, d’autres exemples.\par

\astermono

\noindent Nulle part, sans doute, le rôle tenu dans le raisonnement critique, par ce qu’on pourrait appeler le principe de ressemblance limitée, n’apparaît sous un jour plus curieux qu’avec une des applications les plus neuves de la méthode : la critique statistique.\par
J’étudie, je suppose, l’histoire des prix entre deux dates déterminées,  \phantomsection
\label{p57} dans une société bien liée, que parcourent des courants d’échanges actifs. Après moi, un second travailleur, puis un troisième, entreprennent la même recherche, mais à l’aide d’éléments qui, différents des miens, dif­fèrent également entre eux : autres livres de comptes, autres mercuriales. Chacun de notre côté, nous établissons nos moyennes annuelles, nos nombres‑indices à partir d’une base commune, nos graphiques. Les trois courbes se recouvrent à peu près. On en conclura que chacune d’elles fournit du mouvement une image sommairement exacte. Pourquoi ?\par
La raison n’est pas seulement que dans un milieu économique homogène les grandes fluctuations des prix devaient nécessairement obéir à un rythme sensiblement uniforme. Cette considération suffirait sans doute à frapper de suspicion des courbes brutalement divergentes ; non à nous assurer que, parmi tous les tracés possibles, celui que les trois graphiques s’accordent à donner soit, parce qu’ils s’y accordent, forcément le vrai. Trois pesées, avec des balances pareillement faussées, fourniront le même chiffre, et ce chiffe sera faux. Tout le raisonnement repose ici sur une analyse du mécanisme des erreurs. De ces erreurs de détail, aucune des trois listes de prix ne saurait être tenue pour exempte. En matière de statistique, elles sont à peu près inévitables. Supposons même éliminées les fautes personnelles du chercheur (sans parler de méprises plus gros­sières : qui de nous osera se dire sûr de n’avoir jamais achoppé dans l’affreux dédale des anciennes mesures ?) si merveilleusement attentif qu’on imagine l’érudit, il restera toujours les pièges tendus par les documents eux-mêmes : certains prix ont pu être, par étourderie ou mauvaise foi, inexactement inscrits ; d’autres seront exceptionnels (prix « d’amis » par exemple, ou inversement prix de dupes) par là fort propres à troubler les moyennes ; les mercuriales qui enregistraient les cours moyens pra­tiqués sur les marchés n’auront pas toujours été dressées avec un soin parfait. Mais sur un grand nombre de prix, ces erreurs se compensent. Car il serait hautement invraisemblable qu’elles fussent toujours allées dans le même sens. Si donc la concordance des résultats, obtenus à l’aide de données différentes, les confirme les uns par les autres, c’est qu’à la base la concordance dans les négligences, les menues tromperies, les menues complaisances nous paraît, à juste titre, inconcevable. Ce qu’il y a d’irréductiblement divers dans les témoins a amené à conclure que leur accord final ne peut venir que d’une réalité dont l’unité foncière était, dans ce cas, hors de doute.\par

\astermono

\noindent Les réactifs de l’épreuve du témoignage ne sont pas faits pour être maniés brutalement. Presque tous les principes rationnels, presque toutes  \phantomsection
\label{p58} les expériences qui la guident, trouvent, pour peu qu’on les pousse à fond, leurs limites dans des principes ou des expériences contraires. Comme toute logique qui se respecte, la critique historique a ses antinomies, au moins apparentes.\par
Pour qu’un témoignage soit reconnu authentique, la méthode, on l’a vu, exige qu’il présente une certaine similitude avec les témoignages voisins. À appliquer, cependant, ce précepte à la lettre, que deviendrait la découverte ? Car qui dit découverte, dit surprise, et dissemblance. Une science qui se bornerait à constater que tout se passe toujours comme on l’attendait, la pratique n’en serait guère profitable, ni amusante. On n’a pas retrouvé jusqu’ici de charte rédigée en français (au lieu de l’être comme précédemment en latin) qui soit antérieure à l’année 1204. Ima­ginons que demain un chercheur produise une charte française datée de 1180. Conclura‑t‑on que le document est faux ? ou que nos connaissances étaient insuffisantes ?\par
Non seulement, d’ailleurs, l’impression d’une contradiction entre un témoignage nouveau et son entourage risque de n’avoir d’autre origine qu’une temporaire infirmité de notre savoir. Mais il arrive que le désaccord soit authentiquement dans les choses. L’uniformité sociale n’a pas tant de force que certains individus ou petits groupes ne puissent y échapper. Sous prétexte que Pascal n’écrivait pas comme Arnauld, que Cézanne ne peignait pas comme Bouguereau, refusera‑t‑on d’admettre les dates reconnues des \emph{Provinciales} ou de la « Montagne Sainte-Victoire » ? Arguera­-t‑on de faux les plus anciens outils de bronze pour la raison que la plupart des gisements du même temps ne nous donnent encore que des outils de pierre ?\par
Ces fausses conclusions n’ont rien d’imaginaire et la liste serait longue des faits que la routine érudite a d’abord niés parce qu’ils étaient sur­prenants : depuis la zoolâtrie égyptienne, dont Voltaire s’égayait si fort, jusqu’aux vestiges romains de l’ère tertiaire. À y mieux regarder cepen­dant, le paradoxe méthodologique n’est que de surface. Le raisonnement de ressemblance ne perd pas ses droits. Il importe seulement qu’une plus exacte analyse discerne les écarts possibles, les points de similitude néces­saires.\par
Car toute originalité individuelle a ses bornes. Le style de Pascal n’ap­partient qu’à lui ; mais sa grammaire et le fonds de son vocabulaire sont de son temps. Par l’emploi qu’elle fait d’une langue inusitée, notre charte supposée de 1180 aura beau différer des autres chartes de même date jusqu’ici connues. Pour qu’elle soit jugée recevable, il faudra que son français se conforme, en gros, à l’état du langage attesté, à cette époque, par les textes littéraires ; que les institutions mentionnées cor­respondent à celles du moment.\par
Aussi bien la comparaison critique bien entendue ne se satisfait pas  \phantomsection
\label{p59} de rapprocher les témoignages sur un même plan de la durée. Un phéno­mène humain est toujours le maillon d’une série qui traverse les âtres. Le jour où un nouveau Vrain‑Lucas, jetant sur la table de l’Académie une poignée d’autographes, prétendra nous prouver que Pascal inventa, avant Einstein, la relativité généralisée, tenons‑nous pour assurés d’a­vance que les pièces seront fausses. Ce n’est pas que Pascal fût incapable de trouver ce que ne trouvaient pas ses contemporains. Mais la théorie de la relativité prend son point de départ dans un long développement antérieur de spéculations mathématiques. Si grand fût‑il, aucun homme ne pouvait, par la seule force de son génie, suppléer à ce travail des géné­rations. Lorsque, par contre, devant les premières découvertes de pein­tures paléolithiques, on vit certains savants en contester l’authenticité ou la date, sous prétexte qu’un pareil art ne saurait avoir fleuri, puis s’évanouir, ces sceptiques raisonnaient mal : il y a des chaînes qui se brisent et les civilisations sont mortelles.\par

\astermono

\noindent Quand on lit, écrit en substance le Père Delehaye, que l’Église célèbre le même jour la fête de deux de ses serviteurs morts, tous deux, en Italie ; que la conversion de l’un et de l’autre fut amenée par la lecture de la Vie des Saints ; qu’ils fondèrent chacun un ordre religieux sous le même vocable ; que ces deux ordres, enfin, furent supprimés par deux papes homonymes, il n’est personne qui ne soit tenté de s’écrier qu’un seul individu, dédoublé par erreur, a été inscrit au martyrologe sous deux noms divers. Il est bien vrai, pourtant, que pareillement conquis à la vie religieuse par l’exemple de pieuses biographies, saint Jean Colombini établit l’ordre des Jésuates et Ignace de Loyola celui des Jésuites ; qu’ils moururent tous les deux un 31 juillet, le premier près de Sienne en 1364, le second à Rome en 1556 ; que les Jésuates furent dissous par le Pape Clément IX et la Compagnie de Jésus par Clément XIV. L’exemple est piquant. Il n’est sans doute pas unique. Si jamais un cataclysme ne laisse subsister de l’œuvre philosophique de ces derniers siècles que quelques maigres linéaments, combien de scrupules de conscience ne préparent pas aux érudits de l’avenir l’existence de deux penseurs qui, anglais l’un et l’autre et porteurs tous deux du nom de Bacon, s’accordèrent à faire dans leurs doctrines une grande part à la connaissance expérimentale ? M. Païs a condamné comme légendaires beaucoup d’anciennes traditions romaines pour la seule raison, ou peu s’en faut, qu’on y voit ainsi repasser les mêmes noms, associés à des épisodes assez semblables. N’en déplaise à la critique du plagiat, dont l’âme est la négation des répétitions spon­tanées d’événements ou de mots, la coïncidence est une de ces bizarreries qui ne se laisse pas éliminer de l’histoire.\par
 \phantomsection
\label{p60} Mais il ne saurait suffire de reconnaître en gros la possibilité de ren­contres fortuites. Réduite à cette simple constatation, la critique balan­cerait éternellement entre le pour et le contre. Pour que le doute devienne instrument de connaissance, il faut que, dans chaque cas particulier, puisse être pesé avec quelque exactitude le degré de vraisemblance de la combinaison. Ici, la recherche historique, comme tant d’autres dis­ciplines de l’esprit, croise sa route avec la grande voie royale de la théorie des probabilités.\par

\astermono

\noindent Évaluer la probabilité d’un événement, c’est mesurer les chances qu’il a de se produire. Cela posé, est‑il légitime de parler de la possibilité d’un fait passé ? Au sens absolu, évidemment non. L’avenir seul est aléatoire. Le passé est un donné qui ne laisse plus de place au possible. Avant le coup de dé, la probabilité pour que n’importe quelle face apparût était de un sur six ; une fois le cornet vidé, le problème s’évanouit. Il se peut que nous hésitions plus tard, si ce jour‑là le trois ou bien le cinq était sorti. L’incertitude est alors en nous, dans notre mémoire ou celle de nos témoins. Elle n’est pas dans les choses.\par
À bien l’analyser, pourtant, l’usage que la recherche historique fait de la notion du probable n’a rien de contradictoire. L’historien qui s’in­terroge sur la probabilité d’un événement écoulé, que tente‑t‑il, en effet, sinon de se transporter par un mouvement hardi de l’esprit, avant cet événement même pour en jauger les chances, telles qu’elles se présentaient à la veille de son accomplissement ? La probabilité reste donc bien dans l’avenir. Mais la ligne du présent ayant été, en quelque sorte, imaginaire­ment reculée, c’est un avenir d’autrefois bâti avec un morceau de ce qui, pour nous, est actuellement le passé. Si le fait a incontestablement eu lieu, ces spéculations n’ont guère la valeur que de jeux métaphysiques. Quelle était la probabilité pour que Napoléon naquît ? pour qu’Adolphe Hitler, soldat de 1914, échappât aux balles françaises ? Il n’est pas interdit de se divertir à ces questions. À condition de ne les prendre que, pour ce qu’elles sont réellement : de simples artifices de langage, destinés à mettre en lumière, dans la marche de l’humanité, la part de la contingence et de l’imprévisible. Elles n’ont rien à voir avec la critique du témoignage. L’existence même du fait, au contraire, semble‑t‑elle incertaine ? Doutons-­nous, par exemple, qu’un auteur, sans avoir copié un récit étranger, puise se trouver en répéter spontanément beaucoup d’épisodes et beaucoup de mots ; que le hasard seul ou je ne sais quelle harmonie divinement pré­établie suffisent à expliquer, des \emph{Protocoles des Sages de Sion} aux pam­phlets d’un obscur polémiste du Second Empire, une si frappante res­semblance ? Selon que la coïncidence, avant que le récit ne fût composé,  \phantomsection
\label{p61} devait paraître affectée d’un plus ou moins fort coefficient de probabilité, nous en admettrons aujourd’hui ou nous en rejetterons la vraisemblance.\par
Les mathématiques du hasard, cependant, reposent sur une fiction. Entre tous les cas possibles, elles postulent, au départ, l’impartialité des conditions : une cause particulière qui, d’avance, favoriserait l’un ou l’autre, serait, dans le calcul, comme un corps étranger. Le dé des théori­ciens est un cube parfaitement équilibré ; si, sous une de ses faces, on glissait un grain de plomb, les chances des joueurs cesseraient d’être égales. Mais, en critique du témoignage, presque tous les dés sont pipés. Car des éléments humains très délicats interviennent constamment pour faire pencher la balance vers une éventualité privilégiée.\par
Une discipline historique, à vrai dire, fait exception. C’est la linguis­tique, ou du moins celle de ses branches qui s’attache à établir les parentés entre les langues. Très différente par sa portée des opérations proprement critiques, cette recherche n’en a pas moins avec beaucoup d’entre elles, comme trait commun, de s’efforcer de découvrir des filiations. Or les conditions sur lesquelles elle raisonne sont exceptionnellement proches de la convention primordiale d’égalité, familière à la théorie du hasard. Elle doit cette prérogative aux particularités mêmes des phénomènes du langage. Non seulement, en effet, le nombre immense des combinaisons possibles entre les sons réduit à une valeur infime la probabilité de leur répétition fortuite, en grande quantité, dans des parlers différents. Chose beaucoup plus importante encore : quelques rares harmonies imitatives mises à part, les significations attribuées à ces combinaisons sont tout à fait arbitraires. Que les associations vocales très voisines \emph{tu} ou \emph{tou} (\emph{tu} prononcé à la française ou à la latine) servent à noter la deuxième per­sonne, de toute évidence aucune liaison d’images préalable ne l’impose. Si donc on constate qu’elles ont ce rôle, à la fois, en français, en italien, en espagnol et en roumain ; si l’on observe, en même temps, entre ces langues, une foule d’autres correspondances, également irrationnelles : la seule explication sensée sera que le français, l’italien, l’espagnol et le roumain ont une origine commune. Parce que les divers possibles étaient, humainement, indifférents, un calcul de chances presque pur a emporté la décision.\par
Mais il s’en faut de beaucoup que cette simplicité soit ordinaire.\par
Plusieurs diplômes d’un souverain médiéval, traitant d’affaires différentes, reproduisent les mêmes mots et les mêmes tournures. C’est donc, affirment les fanatiques de la « critique des styles », qu’un même notaire les a rédigés. D’accord, si le hasard seul se trouvait en cause. Mais tel n’est point le cas. Chaque société et, plus encore, chaque petit groupe professionnel a ses habitudes de langage. Il ne suffisait donc pas de dénombrer les points de similitude. Encore eût‑il fallu distinguer, parmi eux, le rare de l’usuel. Seules, les expressions vraiment exceptionnelles peuvent dénoncer un auteur : à supposer, bien entendu, que les répétitions en soient  \phantomsection
\label{p62} assez nombreuses. L’erreur est ici d’attribuer à tous les éléments du discours un poids égal : comme si les variables coefficients de préférence sociale dont chacun d’eux se trouve affecté n’étaient pas les grains de plomb qui contrarient l’équivalence des chances.\par
Toute une école d’érudits s’est attachée, depuis le début du XIX\textsuperscript{ᵉ} siècle, à étudier la transmission des textes littéraires. Le principe est simple : soit trois manuscrits d’un même ouvrage : B, C, et D. On constate qu’ils présentent tous trois les mêmes leçons, évidemment erronées (c’est la méthode des fautes, la plus ancienne, celle de Lachmann). Ou bien, plus généralement, on y relève les mêmes leçons, bonnes ou mauvaises, mais différentes pour la plupart de celles des autres manuscrits (c’est le recen­sement intégral des variantes, préconisé par Dom Quentin). On décidera qu’ils sont « apparentés ». Entendez, selon les cas, ou qu’ils ont été copiés les uns sur les autres, selon un ordre qui reste à déterminer, ou qu’ils remontent tous par des filiations particulières à un modèle commun. Il est bien certain, en effet, qu’une rencontre aussi soutenue ne saurait être fortuite. Cependant deux observations, dont on s’est avisé assez récemment, ont contraint la critique textuelle à abandonner beaucoup de la rigueur, quasi-mécanique, de ses premières conclusions.\par
Les copistes corrigeaient parfois leur modèle. Alors même qu’ils tra­vaillaient indépendamment l’un de l’autre, des habitudes d’esprit com­munes ont dû, assez souvent, leur suggérer des conclusions pareilles. Térence emploie quelque part le mot \emph{raptio} qui est excessivement rare. Ne le comprenant pas, deux scribes l’ont remplacé par \emph{ratio}, qui fait contresens, mais leur était familier. Avaient‑ils besoin, pour cela, de se concerter ou de s’imiter ? Voilà donc un genre de fautes qui, sur la « géné­alogie » des manuscrits, est bien impuissant à rien nous apprendre. Il y a plus. Pourquoi le copiste n’aurait‑il jamais utilisé qu’un modèle unique ? Il ne lui était pas interdit, quand il le pouvait, de confronter plusieurs exemplaires, afin de choisir, de son mieux, parmi leurs variantes. Le cas a certainement été très exceptionnel au moyen âge, dont les bibliothèques étaient pauvres ; beaucoup plus fréquent, en revanche, selon toute appa­rence, dans l’antiquité. Sur les beaux arbres de Jephté, qu’il est d’usage de dresser au seuil des éditions critiques, quelle place assigner à ces inces­tueux produits de plusieurs traditions différentes ? Au jeu des coïncidences, la volonté de l’individu, comme la pression des forces collectives, triche avec le hasard.\par
Ainsi, comme l’avait déjà vu avec Volney la philosophie du XVIII\textsuperscript{e} siècle, la plupart des problèmes de la critique historique sont bien des problèmes de probabilité ; mais tels que le calcul le plus subtil doit s’avouer incapable de les résoudre. Ce n’est pas seulement que les données y sont d’une extraordinaire complexité. En elles‑mêmes, elles demeurent le plus souvent rebelles à toute traduction mathématique. Comment chiffrer, par exemple, la faveur particulière accordée par une société à un mot ou à  \phantomsection
\label{p63} un usage ? Nous ne nous déchargerons pas de nos difficultés sur l’art de Fermat, de Laplace et d’Émile Borel. Du moins, puisqu’il se place en quelque sorte à la limite inaccessible de notre logique, pouvons‑nous lui demander de nous aider, de haut, à mieux analyser nos raisonnements et à les mieux conduire.\par

\astermono

\noindent Quand on n’a pas soi-même pratiqué les érudits, on se rend mal compte combien ils répugnent, d’ordinaire, à accepter l’innocence d’une coïn­cidence. Parce que deux expressions semblables se retrouvent dans la loi salique et dans un édit de Clovis, n’a‑t‑on pas vu un honorable savant allemand affirmer que la Loi devait être de ce prince ? Laissons la banalité des mots, de part et d’autre employés. Une simple teinture de la théorie mathématique aurait suffi à prévenir le faux pas. Lorsque le hasard joue librement, la probabilité d’une rencontre unique ou d’un petit nombre de rencontres est rarement de l’ordre de l’impossible. Peu importe qu’elles nous paraissent étonnantes ; les surprises du sens commun sont rarement des impressions de beaucoup de valeur.\par
On peut s’amuser à calculer la probabilité du coup de hasard qui, dans deux années différentes, fixe au même jour du même mois les morts de deux personnages tout à fait distincts. Elle est de 1/365/2 \footnote{À supposer que les chances de mortalité pour chacun des jours de l’année soit égales. Ce qui n’est pas exact (il y a une courbe annuelle de la mortalité) ; mais peut, sans inconvénients, être postulé ici.}. Admettons maintenant (malgré l’absurdité du postulat) comme certain d’avance que les fondations de Jean Colombini et d’Ignace de Loyola dussent être supprimées par l’Église romaine. L’examen des listes pontificales permet d’établir que la probabilité pour l’abolition par deux papes du même nom était de 11/13. La probabilité combinée à la fois d’une même date de jour et de mois pour les morts, et de deux papes homonymes comme auteurs des condamnations, se place entre 1/10\textsuperscript{3} et 1/10\textsuperscript{6} \footnote{Depuis la mort de Jean Colombini jusqu’à nos jours, 65 papes ont gouverné l’Église (y compris la double et triple série du temps du Grand Schisme) ; 38 se sont succédé depuis la mort d’Ignace. La première liste offre 55 Homonymes avec la seconde, où ces mêmes noms sont répétés exactement 38 fois (les papes ayant, comme l’on sait, coutume de reprendre des noms déjà honorés par l’usage). La probabilité pour que les Jésuites fussent supprimés par un de ces papes homonymes était donc de 55/65 ou 11/13 ; pour les Jésuites, elle montait à 38/38 ou 1 ; autrement dit, elle devenait certitude. La probabilité combinée est de 11/13 x 1 ou 11/13. Enfin 1/365° ou 1/133.225 x 11/13 donne 11/1731.925, soit un peu plus de 1/157.447. Pour être tout à fait exact, il faudrait tenir compte des durées respectives des pontificats. Mais la nature de ce divertissement mathématique, dont l’unique objet est de mettre en lumière un ordre de grandeur, m’a paru autoriser à sim­plifier les calculs.}. Un parieur, sans doute, ne s’en contenterait pas. Mais les sciences de la nature ne considèrent comme proches de l’irréalisable, à l’échelle terrestre, que les  \phantomsection
\label{p64} possibilités de l’ordre de 10/15. On est, on le voit, loin de compte. À bon droit, comme en témoigne l’exemple sûrement attesté des deux saints.\par
Ce sont seulement les concordances accumulées dont la probabilité devient pratiquement négligeable : car en vertu d’un théorème bien connu, les probabilités des cas élémentaires se multiplient alors entre elles, pour donner la probabilité de la combinaison et, les probabilités étant des fractions, leur produit est par définition inférieur à ses compo­sants. L’exemple est célèbre, en linguistique, du mot \emph{bad} qui, en anglais comme en persan, veut dire « mauvais », sans que le terme anglais et le terme persan aient le moins du monde une origine commune. Qui, sur cette correspondance unique, prétendrait fonder une filiation pêcherait contre la loi tutélaire de toute critique des coïncidences : seuls les grands nombres y ont droit de cité.\par
Les concordances ou discordances massives sont faites d’une multitude de cas particuliers. Au total, les influences accidentelles se détruisent. Considérons‑nous, au contraire, chaque élément indépendamment des autres ? L’action de ces variables ne peut plus être éliminée. Même si les dés ont été truqués, le coup isolé demeurera toujours plus difficile à prévoir que l’issue de la partie ; par suite, une fois joué, sujet à une beaucoup plus grande diversité d’explications. C’est pourquoi, à mesure qu’elle pénètre plus avant dans le détail, les vraisemblances de la critique vont en se dégradant. Il n’est, dans l’\emph{Orestie}, telle que nous la lisons aujourd’hui, presque aucun mot pris à part que nous soyons sûrs de lire comme Eschyle l’avait écrit. N’en doutons pas, néanmoins : dans son ensemble, notre \emph{Orestie} est bien celle d’Eschyle. Il y a plus de certitude dans le tout que dans ses composants.\par
Dans quelle mesure cependant, nous est‑il permis de prononcer ce grand mot de certitude ? La critique des chartes ne saurait atteindre à la cer­titude « métaphysique », avouait déjà Mabillon. Il n’avait pas tort. C’est seulement par simplification que nous substituons quelquefois à un langage de probabilité un langage d’évidence. Mais, nous le savons aujourd’hui, mieux qu’au temps de Mabillon, cette convention ne nous est point par­ticulière. Il n’est pas, au sens absolu du terme, « impossible » que la  \phantomsection
\label{p65} \emph{Donation de Constantin} ne soit authentique que ; la \emph{Germanie} de Tacite – selon la lubie de quelques érudits – ne soit un faux. Dans le même sens, il n’est pas « impossible » non plus qu’en frappant au hasard le clavier d’une machine à écrire, un singe ne se trouve fortuitement reconstituer, lettre par lettre, la \emph{Donation} ou la \emph{Germanie.} « L’événement physiquement impossible », a dit Cournot, « n’est autre chose que l’événement dont la probabilité est infiniment petite. » En bornant sa part d’assurance à doser le probable et l’improbable, la critique historique ne se distingue de la plupart des autres sciences du réel que par un échelonnement des degrés sans doute plus nuancé.\par

\astermono

\noindent Mesure‑t‑on toujours avec exactitude le gain immense que fut l’avènement d’une méthode rationnelle de critique, appliquée au témoignage humain ? Gain, j’entends non seulement pour la connaissance historique, pour la connaissance tout court.\par
Naguère, à moins qu’on n’eût à l’avance des raisons bien fortes pour en soupçonner de mensonge les témoins ou les narrateurs, tout fait affirmé était, les trois quarts du temps, un fait accepté. Ne disons pas : il y a de cela très longtemps. Lucien Febvre l’a, pour la Renaissance, excellemment montré : on ne pensait pas, on n’agissait pas autrement à des époques assez voisines de nous pour que leurs œuvres maîtresses nous demeurent encore une vivante nourriture. Ne disons pas : telle était, naturellement, l’attitude de cette foule crédule dont, jusqu’aux jours où nous sommes, la masse pesante, mêlée hélas ! de plus d’un demi-savant, menace cons­tamment d’entraîner nos fragiles civilisations vers d’affreux abîmes d’igno­rances ou de folies. Les plus fermes intelligences n’échappaient pas alors, elles ne pouvaient pas échapper au préjugé commun. Racontait‑on qu’une pluie de sang était tombée ? C’est donc qu’il y a des pluies de sang. Mon­taigne lisait‑il dans ses chers Anciens telle ou telle baliverne sur le pays dont les habitants naissent sans tête ou sur la force prodigieuse du petit poisson rémora ? Il les inscrivait sans sourciller parmi les arguments de sa dialectique ; si capable qu’il fût de démonter ingénieusement le méca­nisme d’un faux bruit, les idées reçues le trouvaient beaucoup plus méfiant que les faits soi-disant attestés. Ainsi régnait, selon le mythe rabelaisien, le vieillard Ouï‑Dire. Sur le monde physique comme sur le monde des hommes. Sur le monde physique peut‑être plus encore que sur celui des hommes. Car, instruit par une expérience plus directe, on doutait plutôt d’un événement humain que d’un météore ou d’un prétendu accident de la vie organique. Votre philosophie répugnait‑elle au miracle ? Ou votre religion aux miracles des autres religions ? Il vous fallait vous efforcer  \phantomsection
\label{p66} péniblement de découvrir à ces surprenantes manifestations des causes soi-disant intelligibles qui, en fait, actions démoniaques ou occultes influx, continuaient d’adhérer à un système d’idées ou d’images complètement étranger à ce que nous appellerions aujourd’hui pensée scientifique. Nier la manifestation elle-même, une pareille audace ne venait guère à l’esprit. Coryphée de cette école padouane si étrangère au surnaturel chrétien, Pomponazzi ne croyait pas que des rois, fussent‑ils oints du chrême de la sainte Ampoule, pussent, parce qu’ils étaient rois, guérir les malades en les touchant. Il ne contestait pourtant point les guérisons. Il en rendait compte par une propriété physiologique, qu’il concevait héréditaire : le glorieux privilège de la fonction sacrée était ramené aux vertus curatives d’une salive dynastique.\par
Or, si notre image de l’univers a pu être, aujourd’hui, nettoyée de tant de fictifs prodiges confirmés cependant, semblait‑il, par l’accord des générations, nous le devons assurément, avant tout, à la notion lentement dégagée d’un ordre naturel, que commandent d’immuables lois. Mais cette notion même n’a pu s’établir si solidement, les observations qui semblaient la contredire n’ont pu être éliminées que grâce au patient travail d’une expérience poursuivie sur l’homme même en tant que témoin. Nous sommes désormais capables à la fois de déceler et d’expliquer les imperfections du témoignage. Nous avons acquis le droit de ne pas le croire toujours, parce que nous savons mieux que par le passé quand et pourquoi il ne doit pas être cru. Et c’est ainsi que les sciences ont réussi à rejeter le poids mort de beaucoup de faux problèmes.\par
Mais la connaissance pure n’est pas ici, plus qu’ailleurs, détachée de la conduite.\par
Richard Simon, dont le nom, dans la génération de nos fondateurs, a sa place au premier rang, ne nous a pas laissé seulement d’admirables leçons d’exégèse. On le vit, un jour, employer l’acuité de son intelligence à sauver quelques innocents, poursuivis par la stupide accusation du crime rituel. La rencontre n’avait rien d’arbitraire. Des deux parts, le besoin de propreté intellectuelle était le même. Un même instrument, chaque fois, permettait de le satisfaire. Amenée constamment à se guider sur les rapports d’autrui, l’action n’est pas moins intéressée que la re­cherche à en peser l’exactitude. Elle ne dispose pas, pour cela, de moyens différents. Disons mieux : ses moyens sont ceux que l’érudition avait d’abord forgés. Dans l’art de diriger utilement le doute, la pratique judi­ciaire n’a fait qu’emboîter le pas, non sans retard, aux Bollandistes et aux Bénédictins. Et les psychologues eux-mêmes ne se sont avisés de trouver dans le témoignage, directement observé et provoqué, un objet de science que longtemps après que la trouble mémoire du passé avait commencé d’être soumise à une épreuve raisonnée. En notre époque, plus  \phantomsection
\label{p67} que jamais exposée aux toxines du mensonge et du faux bruit, quel scan­dale que la méthode critique manque à figurer, fût‑ce dans le plus petit coin des programmes d’enseignements : car elle a cessé de n’être que l’humble auxiliaire de quelques travaux d’atelier. Elle voit s’ouvrir, désormais, devant elle des horizons beaucoup plus vastes ; et l’histoire a le droit de compter parmi ses gloires les plus sûres d’avoir ainsi, en éla­borant sa technique, ouvert aux hommes une route nouvelle vers le vrai et, par suite, le juste.
\section[{Chapitre IV. L’analyse historique}]{Chapitre IV. \\
L’analyse historique}\renewcommand{\leftmark}{Chapitre IV. \\
L’analyse historique}

\subsection[{I. Juger ou comprendre ?}]{I. Juger ou comprendre ?}
\noindent La formule du vieux Ranke est célèbre : l’historien ne se propose rien d’autre que de décrire les choses « telles qu’elles se sont passées, \emph{wie es eigentlich gewesen ».} Hérodote l’avait dit avant lui : « raconter ce qui fut, \emph{ton eonta »}. Le savant, l’historien, en d’autres termes, est invité à s’effacer devant les faits. Comme beaucoup de maximes, celle‑là n’a peut‑être dû sa fortune qu’à son ambiguïté. On y peut lire, modestement, un conseil de probité : tel était, on n’en saurait douter, le sens de Ranke. Mais aussi un conseil de passivité. En sorte que voilà, du même coup, soulevés deux problèmes : celui de l’impartialité historique, celui de l’histoire comme tentative de reproduction ou comme tentative d’analyse.\par

\astermono

\noindent  \phantomsection
\label{p69} Mais y a‑t‑il donc un problème de l’impartialité ? Il ne se pose que parce que le mot, à son tour, est équivoque.\par
Il existe deux façons d’être impartial : celle du savant et celle du juge. Elles ont une racine commune, qui est l’honnête soumission à la vérité. Le savant enregistre, bien mieux, il provoque l’expérience qui, peut‑être, renversera ses plus chères théories. Quel que soit le vœu secret de son cœur, le bon juge interroge les témoins sans autre souci que de connaître les faits, tels qu’ils furent. Cela est, des deux côtés, une obligation de conscience qui ne se discute point.\par
Un moment vient cependant, où les chemins se séparent. Quand le savant a observé et expliqué, sa tâche est finie. An juge, il reste encore à rendre sa sentence. Imposant silence à tout penchant personnel, la prononce‑t‑il selon la loi ? Il s’estimera impartial. Il le sera, en effet, au sens des juges. Non au sens des savants. Car on ne saurait condamner ou absoudre sans prendre parti pour une table des valeurs qui ne relève  \phantomsection
\label{p70} plus d’aucune science positive. Qu’un homme en ait tué un autre est un fait, éminemment susceptible de preuve. Mais châtier le meurtrier sup­pose qu’on tient le meurtre pour coupable : ce qui n’est, à tout prendre, qu’une opinion sur laquelle toutes les civilisations ne sont pas tombées d’accord.\par
Or longtemps l’historien a passé pour une manière de juge des Enfers, chargé de distribuer aux héros morts l’éloge ou le blâme. Il faut croire que cette attitude répond à un instinct puissamment enraciné. Car tous les maîtres qui ont eu à corriger des travaux d’étudiants savent combien ces jeunes gens se laissent difficilement dissuader de jouer, du haut de leurs pupitres, les Minos ou les Osiris. C’est plus que jamais le mot de Pascal : « Tout le monde fait le dieu en jugeant : cela est bon ou mauvais. » On oublie qu’un jugement de valeur n’a de raison d’être que comme la préparation d’un acte et de sens seulement par rapport à un système de références morales, délibérément accepté. Dans la vie quotidienne, les besoins de la conduite nous imposent cet étiquetage, ordinairement assez sommaire. Là où nous ne pouvons plus rien, là où les idéaux com­munément reçus diffèrent profondément des nôtres, il n’est plus qu’un embarras. Pour séparer, dans la troupe de nos pères, les justes des damnés, sommes‑nous donc si sûrs de nous‑mêmes et de notre temps ? Élevant à l’absolu les critères, tout relatifs, d’un individu, d’un parti ou d’une génération, quelle plaisanterie d’en infliger les normes à la façon dont Sylla gouverna Rome ou Richelieu les États du roi Très Chrétien ! Comme d’ailleurs rien n’est plus variable, par nature, que de pareils arrêts, soumis à toutes les fluctuations de la conscience collective ou du caprice personnel, l’histoire, en permettant trop souvent au palmarès de prendre le pas sur le carnet d’expériences, s’est gratuitement donné l’air de la plus incer­taine des disciplines ; aux creux réquisitoires succèdent autant de vaines réhabilitations. Robespierristes, anti-robespierristes, nous vous crions grâce : par pitié, dites‑nous, simplement, quel fut Robespierre.\par
Encore, si le jugement ne faisait que suivre l’explication, le lecteur en serait quitte pour sauter la page. Par malheur, à force de juger, on finit, presque fatalement, par perdre jusqu’au goût d’expliquer. Les passions du passé mêlant leurs reflets aux partis pris du présent, l’humaine réalité n’est plus qu’un tableau en blanc et en noir. Montaigne nous en avait déjà averti : « depuis que le jugement pend d’un côté, on ne peut se garder de contourner et tordre la narration à ce biais ». Aussi bien, pour pénétrer une conscience étrangère que sépare de nous l’intervalle des générations, il faut presque dépouiller son propre moi. Pour lui dire son fait, il suffit de rester soi-même. L’effort est assurément moins rude. Combien il est plus facile d’écrire pour ou contre Luther que de scruter son âme ; de croire le pape Grégoire VII sur l’Empereur Henri IV ou Henri IV sur  \phantomsection
\label{p71} Grégoire VII que de débrouiller les raisons profondes d’un des plus grands drames de la civilisation occidentale ! Voyez encore, hors du plan indi­viduel, la question des biens nationaux. Rompant avec la législation antérieure, le gouvernement révolutionnaire résolut de les vendre par parcelles et sans enchères. C’était sans conteste compromettre gravement les intérêts du Trésor. Contre cette politique, certains érudits, de nos jours, se sont véhémentement élevés. Quel courage, si siégeant à la Con­vention, ils avaient osé y parler de ce ton ! Loin de la guillotine, cette violence sans péril amuse. Mieux eût valu chercher ce que voulaient, réellement, les hommes de l’an III. Ils souhaitaient, avant tout, favoriser l’acquisition de la terre par le petit peuple des campagnes ; à l’équilibre du budget, ils préféraient le soulagement des paysans pauvres, garant de leur fidélité à l’ordre nouveau. Avaient‑ils tort ou raison ? Là‑dessus, que m’importe la décision attardée d’un historien ? Nous lui demandions seulement de ne pas s’hypnotiser sur son propre choix au point de ne plus concevoir qu’un autre, jadis, eût été possible. La leçon du dévelop­pement intellectuel de l’humanité est pourtant claire : les sciences sont toujours montrées d’autant plus fécondes et, par suite, d’autant plus serviables, finalement, à la pratique – qu’elles abandonnaient plus délibérément le vieil anthropocentrisme du bien et du mal. On rirait aujourd’hui d’un chimiste qui mettrait à part les méchants gaz, comme le chlore, les bons comme l’oxygène. Mais, si la chimie à ses débuts avait adopté ce classement, elle aurait fortement risqué de s’y enliser, au grand détriment de la connaissance des corps.\par

\astermono

\noindent Gardons‑nous cependant de trop presser l’analogie. La nomenclature d’une science des hommes aura toujours ses traits particuliers. Celle des sciences du monde physique exclut le finalisme. Les mots de succès ou d’échec, de maladresse ou d’habileté ne sauraient y tenir, au mieux, que le rôle de fictions toujours grosses de dangers. Ils appartiennent au con­traire, au vocabulaire normal de l’histoire. Car l’histoire a affaire à des êtres capables, par nature, de fins consciemment poursuivies.\par
On peut admettre qu’un chef d’armées, qui engage une bataille, s’efforce ordinairement de la gagner. S’il la perd, les forces étant, de part et d’autre, approximativement égales, il sera parfaitement légitime de dire qu’il a mal manœuvré. Cet accident lui était‑il habituel ? On ne sortira pas du plus scrupuleux jugement de fait en observant que ce n’était sans doute pas un bien bon stratège. Soit encore une mutation monétaire, dont l’objet, était, je le suppose, de favoriser les débiteurs aux dépens des créanciers. La qualifier d’excellente ou de déplorable serait prendre parti en faveur  \phantomsection
\label{p72} d’un des deux groupes ; – par suite, transporter arbitrairement, dans le passé, une notion toute subjective du bien public. Mais imaginons que, d’aventure, l’opération destinée à alléger le poids des dettes ait abouti pratiquement – cela s’est vu – à un résultat opposé. « Elle échoua », disons‑nous, sans rien faire par là que de constater honnêtement une réalité. L’acte manqué est un des éléments essentiels de l’évolution humaine. Comme de toute psychologie.\par
Il y a plus. Notre général a‑t‑il, par hasard, conduit volontairement ses troupes à la défaite ? On n’hésitera pas à avancer qu’il a trahi : parce que tout bonnement c’est ainsi que la chose s’appelle. Il y aurait de la part de l’histoire une délicatesse un peu pédante à repousser le secours du simple et droit lexique de l’usage commun. Restera ensuite à rechercher ce que la morale commune du temps ou du groupe pensait d’un pareil acte.. La trahison peut être, à sa façon, un conformisme : témoin les condottieres de l’ancienne Italie.\par
Un mot, pour tout dire, domine et illumine nos études : « comprendre ». Ne disons pas que le bon historien est étranger aux passions ; il a du moins celle‑là. Mot, ne nous le dissimulons pas, lourd de difficultés ; mais aussi d’espoirs. Mot surtout chargé d’amitié. Jusque dans l’action, nous jugeons beaucoup trop. Il est si commode de crier « au poteau » ! Nous ne comprenons jamais assez. Qui diffère de nous – étranger, adver­saire politique – passe, presque nécessairement, pour un méchant. Même pour conduire les inévitables luttes, un peu plus d’intelligence des âmes serait nécessaire ; à plus forte raison pour les éviter, quand il en est temps encore. L’histoire, à condition de renoncer elle‑même à ses faux airs d’ar­change, doit nous aider à guérir ce travers. Elle est une vaste expérience des variétés humaines, une longue rencontre des hommes. La vie, comme la science, a tout à gagner à ce que cette rencontre soit fraternelle.
\subsection[{II. De la diversité des faits humains à l’unité des consciences}]{II. De la diversité des faits humains à l’unité des consciences}
\noindent Comprendre, cependant, n’a rien d’une attitude de passivité. Pour faire une science, il faudra toujours deux choses : une matière, mais aussi un homme. La réalité humaine, comme celles du monde physique, est énorme et bigarrée. Une simple photographie, à supposer même que l’idée de cette reproduction mécaniquement intégrale eût un sens, serait illisible. Dira‑t‑on qu’entre le passé et nous les documents interposent déjà un premier filtre ? Sans doute, ils éliminent souvent à tort et à travers. Presque jamais, par contre, ils n’organisent conformément aux besoins d’un entendement qui veut connaître. Comme tout savant, comme tout cerveau qui simplement perçoit, l’historien choisit et trie. En un mot, il analyse. Et d’abord, il découvre, pour les rapprocher, les semblables.\par

\astermono

\noindent  \phantomsection
\label{p73} J’ai sous les yeux une inscription funéraire romaine : texte d’un seul bloc, établi dans un seul dessein. Rien de plus varié cependant que les témoignages qui, pêle‑mêle, y attendent le coup de baguette de l’érudit.\par
Nous attachons‑nous aux faits de langage ? Les mots, la syntaxe diront l’état du latin, tel qu’en ce temps et en ce lieu on s’efforçait de l’écrire et, par transparence à travers cette langue demi-savante, nous laisserons entrevoir le parler de tous les jours. Notre prédilection, au contraire, va‑t‑elle à l’étude des croyances ? Nous sommes en plein cœur des espoirs d’outre‑tombe. Au système politique ? Un nom d’empereur, une date de magistrature nous combleront d’aise. À l’économie ? L’épitaphe peut-être révèlera un métier ignoré. Et j’en passe. Au lieu d’un document isolé, considérons maintenant, connu par des documents nombreux et divers, un moment quelconque dans le déroulement d’une civilisation. Des hommes qui vivaient alors, il n’en était aucun qui ne participât presque simulta­nément à de multiples manifestations de la vitalité humaine ; qui ne parlât et ne se fît entendre de ses voisins ; qui n’eût ses dieux ; qui ne fût producteur, trafiquant ou simple consommateur ; qui, faute de tenir un rôle dans les événements politiques, n’en subît du moins les contre­coups. Toutes ces activités différentes, osera‑t‑on les retracer, sans choix ni regroupement, dans l’enchevêtrement même où nous les présentent chaque document ou chaque vie, individuelle ou collective ? Ce serait sacrifier la clarté non à l’ordre véritable du réel – qui est fait de naturelles affinités et de liaisons profondes – mais à l’ordre purement apparent du synchronisme. Un carnet d’expériences ne se confond pas avec le journal, minute par minute, de ce qui se passe dans le laboratoire.\par
Aussi bien, quand, dans le cours de l’évolution humaine, nous croyons discerner entre certains phénomènes ce que nous appelons une parenté, qu’entendons‑nous par là, sinon, que chaque type d’institutions, de croyances, de pratiques ou même d’événements, ainsi distingué, nous paraît exprimer une tendance particulière, et jusqu’à un certain point, stable, de l’individu ou de la société ? Niera‑t‑on, par exemple, qu’à travers tous les contrastes il n’y ait entre les émotions religieuses quelque chose de commun ? Il en résulte nécessairement qu’on comprendra tou­jours mieux un fait humain, quel qu’il soit, si on possède déjà l’intelligence d’autres faits de même sorte. L’usage que le premier âge féodal faisait de la monnaie, comme étalon des valeurs beaucoup plutôt que comme moyen de paiement, différait profondément des normes fixées par l’éco­nomie occidentale des environs de 1850 ; entre le régime monétaire du milieu du dix‑neuvième siècle et le nôtre, les contrastes à leur tour ne sont guère moins vifs. Un érudit pourtant, qui n’aurait rencontré la  \phantomsection
\label{p74} monnaie que vers l’an mil, je ne pense pas qu’il parviendrait aisément à saisir les originalités mêmes de son emploi à cette date. C’est ce qui justifie certaines spécialisations, en quelque sorte, verticales : dans le sens, cela va de soi, infiniment modeste, où les spécialisations sont jamais légitimes, c’est‑à‑dire comme remèdes contre le manque d’étendue de notre esprit et la brièveté de nos destins.\par
Il y a plus. À négliger d’ordonner rationnellement une matière qui nous est livrée toute brute, on n’aboutirait, en fin de compte, qu’à nier le temps ; par suite, l’histoire même. Car ce stade du latin, saurons‑nous le comprendre si nous le détachons du développement antérieur de la langue ? Cette structure de la propriété, ces croyances n’étaient pas, assurément, des commencements absolus. Dans la mesure où leur détermi­nation s’opère du plus ancien au plus récent, les phénomènes humains se com­mandent avant tout par chaînes de phénomènes semblables. Les classer par genres, c’est donc mettre à nu des lignes de force d’une efficacité capitale.\par
Mais, s’écrieront certains, les distinctions que vous établissez ainsi, en tranchant à travers la vie même, elles ne sont que dans votre intel­ligence, elles ne sont pas dans la réalité, où tout s’emmêle. Vous usez donc d’ » abstraction ». D’accord. Pourquoi avoir peur des mots ? Aucune science ne saurait se dispenser d’abstraction. Pas plus, d’ailleurs, que d’imagination. Il est significatif, soit dit en passant, que les mêmes esprits qui prétendent bannir la première manifestent généralement envers la seconde une égale mauvaise humeur. C’est, des deux parts, le même positivisme mal compris. Les sciences de l’homme ne font pas exception. En quoi la fonction chlorophylienne est‑elle plus « réelle », au sens de l’extrême réalisme, que la fonction économique ? Seules les classifications qui se reposeraient sur de fausses similitudes seraient funestes. Affaire à l’historien d’éprouver sans cesse les siennes, pour mieux prendre cons­cience de leurs raisons d’être et, s’il y a lieu, les réviser. Dans leur com­mun effort pour cerner le réel, elles peuvent d’ailleurs partir de points de vue très différents.\par
Voici, par exemple, l’ » histoire du droit ». L’enseignement et le manuel, qui sont d’admirables instruments de sclérose, ont vulgarisé le nom. Que recouvre‑t‑il cependant ? Une règle de droit est une norme sociale, expli­citement impérative ; sanctionnée, en outre, par une autorité capable d’en imposer le respect à l’aide d’un système précis de contraintes et de peines. Pratiquement, de pareils préceptes peuvent régir les activités les plus différentes. Jamais ils ne sont seuls à les commander : nous obéis­sons constamment, dans notre conduite journalière, à des codes moraux, professionnels, mondains, souvent autrement impérieux que le Code tout court. Les frontières de celui-ci oscillent d’ailleurs sans cesse ; et, pour y être ou non insérée, une obligation socialement reconnue, si elle peut en recevoir plus ou moins de force ou de clarté, ne change évidemment pas de nature. Le droit, au sens strict du mot, est donc l’enveloppe  \phantomsection
\label{p75} formelle de réalités en elles‑mêmes beaucoup trop variées pour fournir avec profit l’objet d’une étude unique ; et il n’épuise aucune d’elles. La famille, je suppose – qu’il s’agisse de la petite famille matrimoniale d’aujourd’hui, en état de perpétuelles systoles et diastoles, ou du grand lignage médiéval, cette collectivité cimentée par un si tenace réseau de sentiments et d’in­térêts – suffira‑t‑il jamais, pour en pénétrer vraiment la vie, d’énumérer les uns après les autres les articles de n’importe quel droit familial ? On semble parfois l’avoir cru : avec quels décevants résultats, l’impuissance où nous demeurons encore aujourd’hui de retracer l’intime évolution de la famille française le dénonce assez.\par
Pourtant, il y a bien, dans la notion du fait juridique comme distinct des autres, quelque chose d’exact. C’est qu’au moins dans beaucoup de sociétés, l’application et, dans une large mesure, l’élaboration même des règles de droit ont été l’œuvre propre d’un groupe d’hommes relativement spécialisé et, dans ce rôle (que ses membres pouvaient, cela va de soi, combiner avec d’autres fonctions sociales), suffisamment autonome pour posséder ses traditions propres et, souvent, jusqu’à la pratique d’une méthode de raisonnement particulière. L’histoire du droit, en somme, pourrait bien n’avoir d’existence séparée que comme l’histoire des juristes : ce qui n’est pas, pour une branche d’une science des hommes, une si mauvaise façon d’exister. Entendue en ce sens, elle jette sur des phéno­mènes très divers, mais soumis à une action humaine commune, des lueurs, dans leur champ nécessairement limité, très révélatrices.\par
Un tout autre genre de division est représenté par la discipline qu’on s’est habitué à nommer « géographie humaine ». Ici, l’angle, de visée n’est pas demandé à l’action d’une mentalité de groupe (comme c’est le cas, sans qu’elle s’en doute toujours, pour l’histoire du droit). Il n’est pas pris non plus, à l’exemple de l’histoire religieuse ou de l’histoire écono­mique, dans la nature spécifique d’un fait humain : croyances, émotions, effusions du cœur, espoirs et tremblements qu’inspire l’image de forces transcendantes à l’humanité ; efforts pour satisfaire et organiser les besoins matériels. L’enquête se centre sur un type de liaisons communes à un grand nombre de phénomènes sociaux. L’ « anthropogéographie » étudie les sociétés dans leurs relations avec le milieu physique : échanges à double. sens, cela va de soi, où l’homme sans cesse agit sur les choses en même temps que celles‑ci sur lui. Dans ce cas encore, on n’a donc rien de plus ni rien de moins qu’une perspective, dont la légitimité se prouve par sa fécondité, mais que d’autres perspectives devront compléter. Tel est bien, en effet, en tout ordre de recherche, le rôle de l’analyse. La science ne décompose le réel qu’afin de mieux l’observer, grâce à un jeu de feux croisés dont les rayons constamment se combinent et s’interpénè­trent. Le danger commence seulement quand chaque projecteur prétend à lui seul tout voir ; quand chaque canton du savoir se prend pour une patrie.\par
 \phantomsection
\label{p76} Une fois de plus cependant, méfions‑nous de postuler entre les sciences de la nature et une science des hommes, je ne sais quel parallélisme faus­sement géométrique. Dans la vue que j’ai de ma fenêtre, chaque savant prend son bien, sans trop s’occuper de l’ensemble. Le physicien explique le bleu du ciel ; le chimiste, l’eau du ruisseau ; le botaniste, l’herbe. Le soin de recomposer le paysage tel qu’il m’apparaît et m’émeut, ils le laissent à l’art, si le peintre ou le poète veulent bien s’en charger. C’est que le paysage, comme unité, existe seulement dans ma conscience. Or, le propre de la méthode scientifique, comme ces formes du savoir la pra­tiquent et, par leur succès, la justifient, est d’abandonner délibérément le contemplateur, pour ne plus vouloir connaître que les objets contemplés. Les liens que notre esprit tisse entre les choses leur paraissent arbitraires ; elles les brisent, de parti pris, pour rétablir une diversité à leur gré plus authentique. Déjà cependant, le monde organique pose à ses analystes des problèmes singulièrement plus délicats. Le biologiste peut bien, pour plus de commodité, étudier à part la respiration, la digestion, les fonctions motrices : il n’ignore pas que, par dessus tout cela, il y a l’individu dont il doit rendre compte. Mais les difficultés de l’histoire sont encore d’une autre essence. Car pour matière, elle a précisément, en dernier ressort, des consciences humaines. Les rapports qui se nouent à travers celles‑ci, les contaminations, voire les confusions dont elles sont le terrain cons­tituent, à ses yeux, la réalité même.\par
Or, \emph{homo religiosus, homo œconomicus, homo politicus}, toute cette kyrielle d’hommes en us dont on pourrait, à plaisir, allonger la liste, le péril serait grave de les prendre pour autre chose que ce qu’ils sont en vérité : des fantômes commodes, à condition de ne pas devenir encombrants. Le seul être de chair et d’os est l’homme, sans plus, qui réunit à la fois tout cela.\par
Certes les consciences ont leurs cloisons intérieures, que certains d’entre nous se montrent particulièrement habiles à élever. Gustave Lenôtre s’étonnait inlassablement de trouver parmi les Terroristes tant d’excel­lents pères de famille. Même si nos grands révolutionnaires avaient été les authentiques buveurs de sang dont la peinture chatouillait si agréa­blement un public douillettement embourgeoisé, cette stupeur n’en per­sisterait pas moins à trahir une psychologie assez courte. Que d’hommes mènent, sur trois ou quatre plans différents, plusieurs vies qu’ils souhaitent distinctes et parviennent quelquefois à maintenir telles ?\par
De là, cependant, à nier l’unité foncière du moi et les constantes inter­pénétrations de ses diverses attitudes, il y a loin. Étaient‑ils l’un pour l’autre deux étrangers, Pascal mathématicien et Pascal chrétien ? Ne croisaient‑ils jamais leurs chemins, le docte médecin François Rabelais et maître Alcofribas, de pantagruélique mémoire ? Lors même que les rôles alternativement tenus par l’acteur unique semblent s’opposer aussi brutalement que les personnages stéréotypés d’un mélodrame, il se peut qu’à y bien regarder cette antithèse soit seulement le masque d’une  \phantomsection
\label{p77} solidarité plus profonde. On s’est gaussé de l’élégiaque Florian qui, paraît‑il, battait ses maîtresses. Peut‑être ne répandait‑il dans ses vers tant de douceur que pour mieux se consoler de ne pas réussir à en mettre davan­tage dans sa conduite. Quand le marchand médiéval après avoir, à lon­gueur de journée, violé les commandements de l’Église sur l’usure et le juste prix allait s’agenouiller benoîtement devant l’image de Notre‑Dame, puis, au soir de sa vie, accumulait les pieuses et aumônières fondations ; quand le grand manufacturier des « temps difficiles » bâtissait des hôpi­taux avec l’argent épargné sur les misérables salaires d’enfants en gue­nilles, cherchaient‑ils seulement, l’un et l’autre, comme on le dit d’ordi­naire, à contracter contre les foudres célestes une assez basse assurance, ou bien, par ces explosions de foi ou de charité, ne satisfaisaient‑ils pas aussi, sans trop se l’exprimer, les secrets besoins du cœur, que la dure pratique quotidienne les avait condamnés à refouler ? Il est des contra­dictions qui ressemblent fort à des évasions.\par
Passe‑t‑on des individus à la société ? Comme celle‑ci, de quelque façon qu’on la considère, ne saurait être après tout autre chose, ne disons pas qu’une somme (ce serait, sans doute, trop peu dire), du moins qu’un produit de consciences individuelles, on ne s’étonnera pas d’y retrouver le même jeu de perpétuelles interactions. C’est un fait certain que, du XII\textsuperscript{ᵉ} siècle à la Réforme au moins, les communautés de tisserands constituèrent un des terrains privilégiés des hérésies. Voilà assurément une belle matière pour une fiche d’histoire religieuse. Rangeons donc soigneusement ce bout de carton dans son tiroir. Dans les casiers voisins étiquetés, cette fois, « histoire économique », précipitons une seconde moisson de notes. Croirons‑nous en avoir terminé, par là, avec ces remu­antes petites sociétés de la navette ? Il nous restera encore à les expliquer, puisqu’un de leurs traits fondamentaux fut non de faire coexister le religieux avec l’économique, mais de les entrelacer. Frappé par « cette sorte de certitude, de sécurité, d’assiette morale », dont quelques géné­rations venues immédiatement avant la nôtre semblent avoir joui avec une si étonnante plénitude, Lucien Febvre en découvre, par dessus tout, deux raisons : l’empire sur les intelligences du système cosmologique de Laplace et « l’anormale fixité » du régime monétaire. Point de faits humains, de nature en apparence plus opposée que ceux‑là. Ils collaborèrent pourtant à donner à l’attitude mentale d’un groupe sa tonalité entre toutes caractéristique.\par
Sans doute, pas plus qu’au sein de n’importe quelle conscience person­nelle, ces rapports à l’échelle collective ne sont simples. On n’oserait plus écrire aujourd’hui, tout uniment, que la littérature est « l’expression de la société ». Du moins ne l’est‑elle nullement au sens où un miroir « ex­prime » l’objet reflété. Elle peut traduire des réactions de défense aussi bien qu’un accord. Elle charrie, presque inévitablement, un grand nombre de thèmes hérités, de mécanismes formels appris dans l’atelier,  \phantomsection
\label{p78} d’anciennes conventions esthétiques, qui sont autant de causes de retarde­ment. « À la même date », écrit avec sagacité H. Focillon, « le politique, l’économique, l’artistique n’occupent pas – [je préférerais, « n’occupent pas forcément »] – la même position sur leurs courbes respectives ». Mais c’est de ces décalages, précisément, que la vie sociale tient son rythme presque toujours heurté. De même, chez la plupart des individus, les diverses âmes, pour parler le langage pluraliste de l’antique psychologie, ont rarement un âge identique : combien d’hommes mûrs conservent encore des coins d’enfance !\par
Michelet expliquait, en 1837, à Sainte‑Beuve : « Si je n’avais fait entrer dans la narration que l’histoire politique, si je n’avais point tenu compte des éléments divers de l’histoire (religion, droit, géographie, littérature, art, etc.) mon allure eût été tout autre. \emph{Mais il fallait un grand mou­vement vital, parce que tous ces éléments divers gravitaient ensemble dans l’unité du récit.} » En 1800, Fustel de Coulanges, à son tour, disait à ses auditeurs de la Sorbonne : « Supposez cent spécialistes se partageant, par lots, le passé de la France ! croyez‑vous qu’à la fin ils aient fait l’his­toire de la France ? J’en doute beaucoup. Il leur manquera au moins le lien des faits : \emph{or, ce lien aussi est une vérité historique. »} « Mouvement vital », « lien » : l’opposition des images est significative. Michelet pensait, sentait sous les espèces de l’organique ; fils d’un âge auquel l’univers newtonien semblait donner le modèle achevé de la science, Fustel recevait ses métaphores de l’espace. Leur accord fondamental n’en rend qu’un son plus plein. Ces deux grands historiens étaient trop grands pour l’ignorer : pas plus qu’un individu, une civilisation n’a rien d’un jeu de patience, mécaniquement assemblé ; la connaissance des fragments, étudiés suc­cessivement, chacun pour soi, ne procurera jamais celle du tout ; elle ne procurera même pas celle des fragments eux‑mêmes.\par
Mais le travail de recomposition ne saurait venir qu’après l’analyse. Disons mieux : il n’est que le prolongement de l’analyse, comme sa raison d’être. Dans l’image primitive, contemplée plutôt qu’observée, comment eût‑on discerné des liaisons, puisque rien n’était distinct ? Leur réseau délicat ne pouvait apparaître qu’une fois les faits d’abord classés par lignées spécifiques. Aussi bien, pour demeurer fidèle à la vie dans le cons­tant entrecroisement de ses actions et réactions, il n’est nullement néces­saire de prétendre l’embrasser tout entière, par une effort ordinairement trop vaste pour les possibilités d’un seul savant. Rien de plus légitime, rien souvent de plus salutaire que de centrer l’étude d’une société sur un de ses aspects particuliers, ou, mieux encore, sur un des problèmes précis que soulève tel ou tel de ces aspects : croyances, économie, structure des classes ou des groupes, crises politiques… Par ce choix raisonné, les problèmes ne seront pas seulement, à l’ordinaire, plus fermement posés : il n’est pas jusqu’aux faits de contact et d’échange qui ne ressortiront avec plus de clarté. À condition, simplement, de vouloir les découvrir.  \phantomsection
\label{p79} Ces grands marchands de l’Europe de la Renaissance, vendeurs de draps ou d’épices, accapareurs de cuivre, de mercure ou d’alun, banquiers des empereurs et des rois, souhaitez‑vous les connaître vraiment, dans leur marchandise même ? Souvenez‑vous qu’ils se faisaient peindre par Hol­bein, qu’ils lisaient Érasme ou Luther. L’attitude du vassal médiéval envers son seigneur, il faudra pour la comprendre vous informer aussi de son attitude envers son Dieu. L’historien ne sort jamais du temps ; mais par une oscillation nécessaire, que déjà le débat sur les origines nous a mise sous les yeux, il y considère tantôt les grandes ondes de phé­nomènes apparentés qui traversent, de part en part, la durée, tantôt le moment humain où ces courants se resserrent dans le nœud puissant des consciences.
\subsection[{III. La nomenclature}]{III. La nomenclature}
\noindent Ce serait pourtant peu de chose que de se borner à discerner dans un homme ou une société les principaux aspects de leur activité. À l’in­térieur de chacun de ces grands groupes de faits, un nouveau et plus délicat effort d’analyse est nécessaire. Il faut distinguer les diverses insti­tutions qui composent un système politique, les diverses croyances, pra­tiques, émotions dont une religion est faite. Il faut, dans chacune de ces pièces et dans les ensembles même, caractériser les traits qui tantôt les rapprochent, tantôt les écartent des réalités de même ordre… Problème de classement inséparable, à l’expérience, du problème fondamental de la nomenclature.\par
Car toute analyse veut d’abord, comme outil, un langage approprié ; un langage capable de dessiner avec précision les contours des faits, tout en conservant la souplesse nécessaire pour s’adapter progressivement aux découvertes, un langage surtout sans flottements ni équivoques. Or c’est là où le bât nous blesse, nous autres historiens. Un écrivain d’esprit aigu, qui ne nous aime guère, l’a bien vu : « Ce moment capital des défi­nitions et des conventions nettes et spéciales qui viennent remplacer les significations d’origine confuse et statistique n’est pas arrivé pour l’his­toire. » Ainsi parle M. Paul Valéry. Mais il est vrai que cette heure d’exac­titude n’est pas encore arrivée, est‑il impossible qu’elle sonne un jour ? Et, d’abord, pourquoi se montre‑t‑elle si lente à sonner ?\par

\astermono

\noindent La chimie s’est forgé son matériel de signes. Voire ses mots : « gaz » est, si je ne me trompe, un des rares vocables authentiquement inventés que possède la langue française. C’est que la chimie avait le grand avan­tage de s’adresser à des réalités incapables, par nature, de se nommer elles‑mêmes. Le langage de la perception confuse, qu’elle a rejeté, n’était  \phantomsection
\label{p80} pas moins extérieur aux choses et, en ce sens, moins arbitraire que celui de l’observation classée et contrôlée, qu’elle lui a substitué : qu’on dise vitriol ou acide sulfurique, le corps n’y est jamais pour rien. Il en va tout autrement d’une science de l’humanité. Pour donner des noms à leurs actes, à leurs croyances et aux divers aspects de leur vie de société, les hommes n’ont pas attendu de les voir devenir l’objet d’une recherche désintéressée. Son vocabulaire, l’histoire le reçoit donc, pour la plus grande part, de la matière même de son étude. Elle l’accepte, déjà fatigué et déformé par un long emploi ; ambigu d’ailleurs, souvent dès l’origine, comme tout système d’expressions qui n’est pas issu de l’effort sévère­ment concerté des techniciens.\par
Le pis est que ces emprunts mêmes manquent d’unité. Les documents tendent à imposer leur nomenclature ; l’historien, s’il les écoute, écrit sous la dictée d’une époque chaque fois différente. Mais il pense d’autre part, naturellement, selon les catégories de son propre temps ; par suite, avec les mots de celui-ci. Quand nous parlons de patriciens, un contem­porain du vieux Caton nous eût compris ; l’auteur par contre, qui évoque le rôle de la « bourgeoisie » dans les crises de l’Empire romain, comment traduirait‑il en latin le nom ou l’idée ? Ainsi deux orientations distinctes se partagent, presque nécessairement, le langage de l’histoire. Voyons‑les tour à tour.\par

\astermono

\noindent Reproduire ou calquer la terminologie du passé peut paraître, au premier abord, une démarche assez sûre. Elle se heurte, pourtant, dans l’application à de multiples difficultés.\par
C’est d’abord que les changements des choses sont loin d’entraîner toujours des changements parallèles, dans leurs noms. Telle est la suite naturelle du caractère traditionaliste inhérent à tout langage, comme de la faiblesse d’invention dont souffrent la plupart des hommes.\par
L’observation vaut même pour l’outillage, sujet pourtant à des modi­fications ordinairement assez tranchées. Quand mon voisin me dit : « Je sors en voiture », dois‑je comprendre qu’il parle d’un véhicule à cheval ou d’une automobile ? Seule l’expérience que je puis avoir, à l’avance, de sa remise ou de son garage me permettra de répondre. \emph{Aratrum} dési­gnait, en principe, l’instrument de labour sans roues ; \emph{carruca}, celui qui en était pourvu. Comme, cependant, le premier apparut avant le second, serai-je assuré, si je rencontre dans un texte le vieux mot, qu’il n’a pas été simplement maintenu à un nouvel outil ? Inversement, Mathieu de Dombasle a appelé « charrue » l’instrument qu’il avait imaginé et qui, privé de roues, était, au vrai, un araire.\par
 \phantomsection
\label{p81} Combien, toutefois, cet attachement au nom hérité n’apparaît‑il pas plus fort dès qu’on considère des réalités d’un ordre moins matériel ? C’est que les transformations, en pareil cas, s’opèrent presque toujours trop lentement pour être perceptibles aux hommes mêmes qu’elles affectent. Ils n’éprouvent pas le besoin de changer l’étiquette parce que le change­ment du contenu leur échappe. Le mot latin \emph{servus}, qui a donné en fran­çais \emph{serf}, a traversé les siècles. Mais au prix de tant d’altérations succes­sives dans la condition ainsi désignée qu’entre le \emph{servus} de l’ancienne Rome et le serf de la France de saint Louis, les contrastes l’emportaient de beaucoup sur les ressemblances. Aussi les historiens ont‑ils générale­ment pris le parti de réserver « serf » au Moyen Âge. S’agit‑il de l’anti­quité ? Ils parlent « d’esclaves ». Autrement dit, au décalque ils pré­fèrent, en l’occurrence, l’équivalent. Non sans sacrifier à l’exactitude intrinsèque du langage un peu de l’harmonie de ses couleurs ; car le terme, qu’ils transplantent ainsi dans un entourage romain, naquit seulement aux environs de l’an mil, sur les marchés de chair humaine où les captifs slaves semblaient fournir le modèle même d’une entière sujétion, devenue tout à fait étrangère aux serfs indigènes de l’Occident. L’artifice est com­mode, tant qu’on s’en tient aux extrêmes. Dans l’intervalle, où faudra‑t‑il que, devant le serf, l’esclave s’efface ? C’est l’éternel sophisme du tas de blé. De toutes façons, nous voilà donc contraints, pour rendre justice aux faits eux‑mêmes, de substituer à leur langage une nomenclature, sinon proprement inventée, du moins remaniée et décalée.\par
Réciproquement, il arrive aussi que les noms varient, dans le temps ou dans l’espace, indépendamment de toute variation dans les choses.\par
Quelquefois, ce sont les causes particulières à l’évolution du langage qui entraînent l’effacement du mot, sans que l’objet ou l’acte soit le moins du monde touché. Car les faits linguistiques ont leur coefficient propre de résistance ou de ductilité. Constatant la disparition dans les langues romanes du verbe latin \emph{emere}, et son remplacement par d’autres verbes d’origines très différentes – « acheter », « \emph{comprar} », etc. – un érudit, naguère, a cru pouvoir en tirer les conclusions les plus étendues, les plus ingénieuses sur les transformations qui, dans les sociétés héritières de Rome, auraient affecté le régime des échanges. Que ne s’était‑il de­mandé si ce fait indiscutable pouvait être traité comme un fait isolé ! Rien n’a été plus commun, au contraire, dans les parlers issus du latin, que la chute des mots trop courts ; l’anémie des syllabes atones les avait peu à peu rendus presque indistincts. Le phénomène est d’ordre stricte­ment phonétique, et l’erreur amuse d’avoir pris une aventure de la pro­nonciation pour un trait de civilisation économique.\par
Ailleurs ce sont les conditions sociales qui s’opposent à l’établissement ou au maintien d’un vocabulaire uniforme. Dans des sociétés très mor­celées, comme celles de notre Moyen Âge, il était fréquent que des insti­tutions foncièrement identiques fussent, selon les lieux, désignées par des  \phantomsection
\label{p82} termes très différents. De nos jours encore, les parlers ruraux s’écartent beaucoup entre eux, jusque dans les notations des objets les plus communs et des coutumes les plus universelles. Dans la province du Centre, où j’écris ces lignes, on appelle « village » ce qui, dans le Nord, serait dénommé « hameau ». Le village du Nord est ici un « bourg ». Ces divergences ver­bales présentent, en elles‑mêmes, des faits très dignes d’attention. À y conformer, cependant, sa propre terminologie, l’historien ne compro­mettrait pas seulement l’intelligibilité de son discours ; il s’interdirait jusqu’au travail de classement, qui figure au premier rang de ses devoirs.\par

\astermono

\noindent Notre science ne dispose pas, comme les mathématiques ou la chimie, d’un système de symboles détaché de toute langue nationale. L’historien parle uniquement avec des mots ; donc, avec ceux de son pays. Se trouve-­t‑il en présence de réalités qui s’exprimèrent dans une langue étrangère ? Force lui est de traduire. À cela, point d’obstacles sérieux, tant que les mots se rapportent à des choses ou à des actions banales : cette monnaie courante du vocabulaire s’échange aisément au pair. Aussitôt, par contre, qu’apparaissent des institutions, des croyances, des coutumes, qui parti­cipent plus profondément à la vie propre d’une société, la transposition dans une autre langue, faite à l’image d’une société différente, devient une entreprise grosse de périls. Car choisir l’équivalent, c’est postuler une ressemblance.\par
Nous résignerons‑nous donc, en désespoir de cause, à conserver, quitte à l’expliquer, le terme originel ? Assurément, il le faudra bien quelquefois. Quand on vit, en 1919, la Constitution de Weimar maintenir à l’État allemand son vieux nom de \emph{Reich} : « Étrange\emph{ République}, s’exclamèrent chez nous certains publicistes – ne la voilà‑t‑il pas qui persiste à se dire \emph{Empire »} ? La vérité n’est pas seulement que \emph{Reich} n’évoque nullement par lui-même l’idée d’un empereur ; associé aux images d’une histoire politique perpétuellement oscillante entre le particularisme et l’unité, le mot rend un son beaucoup trop spécifiquement allemand pour souffrir, dans une langue où se reflète un tout autre passé national, la moindre tentative de traduction.\par
Cette reproduction mécanique, cependant, véritable solution de moindre effort, comment la généraliser ? Laissons même tout souci de propreté de langage : il serait pourtant fâcheux, avouons‑le, de voir les historiens encombrant leurs propos de vocables étrangers, imiter ces auteurs de romans rustiques qui, à force de patoiser, glissent à un jargon où les champs ne se reconnaîtraient pas mieux que la ville. En renonçant à tout essai d’équivalence, c’est souvent à la réalité même que l’on ferait tort. Un usage qui remonte, je crois, au dix‑huitième siècle, veut que \emph{serf} en français, ou des mots de sens voisin dans les autres langues  \phantomsection
\label{p83} occidentales, soient employés à désigner le \emph{chriépostnoï} de l’ancienne Russie tsariste. Un rapprochement plus malencontreux pouvait difficilement être imaginé. Là‑bas, un régime d’attache à la glèbe, peu à peu transformé en un véritable esclavage ; chez nous, une forme de dépendance person­nelle qui, malgré sa rigueur, était très loin de traiter l’homme comme une chose dépourvue de tous droits : le prétendu servage russe n’avait à peu près rien de commun avec notre servage médiéval. Cependant, dire tout bonnement « \emph{chriépostnoï} » ne nous avancerait guère. Car il a existé en Roumanie, en Hongrie, en Pologne et jusque dans l’Allemagne orien­tale, des types de sujétion paysanne étroitement apparentés à celui qui s’établit en Russie. Faudra‑t‑il, tout à tour, parler roumain, hongrois, polonais, allemand ou russe ? Une fois de plus, l’essentiel échapperait, qui est de restituer les liaisons profondes des faits, en les exprimant par une juste nomenclature.\par
L’étiquette a été mal choisie. Une étiquette commune, surimposée par conséquent aux noms nationaux, au lieu de les copier, n’en demeure pas moins nécessaire. Là encore, la passivité est interdite.\par

\astermono

\noindent De nombreuses sociétés ont pratiqué ce qu’on peut appeler un bilin­guisme hiérarchique. Deux langues s’affrontaient, l’une populaire, l’autre savante. Ce qui se pensait et se disait couramment dans la première s’écrivait, exclusivement ou de préférence, dans la seconde. Ainsi, l’Abys­sinie, du XI\textsuperscript{ᵉ} au XVII\textsuperscript{ᵉ} siècle, écrivit le guèze, parla l’amharique. Ainsi les Évangiles ont rapporté en grec, qui était alors la grande langue de culture de l’Orient, des propos qu’il faut supposer échangés en araméen. Ainsi, plus près de nous, le Moyen Âge, pendant longtemps, ne s’administra, ne se raconta lui-même qu’en latin. Héritées de civilisations mortes ou empruntées à des civilisations étrangères, ces langues de lettrés, de prêtres et de notaires devaient nécessairement exprimer beaucoup de réalités pour lesquelles elles n’étaient originellement point faites. Elles n’y par­venaient qu’à l’aide de tout un système de transpositions, d’une inévi­table gaucherie.\par
Or c’est par ses écrits que – témoignages matériels exceptés – nous connaissons une société. Celles où triompha un pareil dualisme de langage ne nous apparaissent donc, dans beaucoup de leurs traits principaux, qu’à travers un voile d’à peu près. Parfois même, un écran supplémentaire s’interpose. Le grand cadastre de l’Angleterre que fit établir Guillaume le Conquérant, le fameux « livre du Jugement » (\emph{Domesday Book)}, fut l’œuvre de clercs normands ou manceaux. Ils ne décrivirent pas seulement en latin des institutions spécifiquement anglaises ; ils les avaient d’abord repensées en français. Lorsqu’il se heurte à ces nomenclatures par substi­tution de termes, l’historien n’a d’autre ressource que de refaire, à rebours,  \phantomsection
\label{p84} le travail. Si les correspondances ont été commodément choisies et surtout appliquées avec suite, la tâche sera relativement aisée. On n’aura pas beaucoup de peine à reconnaître, derrière les « consuls » des chroniqueurs, les comtes de la réalité. – Il se rencontre, malheureusement, des cas moins favorables. Qu’était le \emph{colonus} de nos chartes des XI\textsuperscript{ᵉ} et XII\textsuperscript{ᵉ} siècles ? Question dépourvue de sens. Sans héritier, en effet, dans la langue vulgaire, parce qu’il avait cessé de rien évoquer de vivant, le mot ne représentait qu’un artifice de traduction, employé par les notaires pour désigner tour à tour, en beau latin classique, des conditions juridiques ou économiques très diverses.\par
Aussi bien, cette opposition de deux langues forcément différentes ne figure en vérité que le cas‑limite de contrastes communs à toutes les sociétés. Jusque dans les nations les plus unifiées, comme la nôtre, chaque petite collectivité professionnelle, chaque groupe caractérisé par la culture ou la fortune possède son système d’expression particulier. Or, tous les groupes n’écrivent pas ou n’écrivent pas autant, ou n’ont pas autant de chances de faire passer leurs écrits à la postérité. Chacun le sait : il est rare que le procès‑verbal d’un interrogatoire judiciaire reproduise littéralement les paroles prononcées ; le greffier, presque spontanément, ordonne, clarifie, rétablit la syntaxe, émonde les mots jugés trop vul­gaires. Les civilisations du passé ont eu aussi leurs greffiers ; chroniqueurs, juristes surtout. Ce sont eux dont la voix, avant tout autre, nous est parvenue. Gardons‑nous d’oublier que les mots dont ils usaient, les clas­sifications qu’ils proposaient par ces mots, étaient le résultat d’une élabo­ration savante, souvent exagérément influencée par la tradition. Quel étonnement, peut‑être si, au lieu de peiner sur la terminologie embrouillée (et probablement artificielle) des censiers et des capitulaires carolingiens, nous pouvions, promenant nos pas dans un village de ce temps, écouter les paysans nommant entre eux leurs conditions ou les seigneurs celles de leurs sujets ? Sans doute, cette description de la pratique quoti­dienne par elle‑même ne nous donnerait pas, non plus, toute la vie ; car les tentatives d’expression et, par suite, d’interprétation, qui viennent des doctes ou des hommes de loi constituent, elles aussi, des forces concrètement agissantes ; ce serait, du moins, atteindre une fibre pro­fonde. Quel enseignement si – le dieu fût‑il d’hier ou d’aujourd’hui – nous réussissions à surprendre sur les lèvres des humbles leur véritable prière ! À supposer, cependant, qu’ils aient su, eux‑mêmes, traduire sans les mutiler les élans de leur cœur.\par
Car là est, en dernier ressort, le grand obstacle. Rien n’est plus difficile à un homme que de s’exprimer lui-même. Mais nous n’éprouvons guère moins de peine à trouver, pour les fluides réalités sociales qui sont la trame de notre existence, des noms exempts à la fois d’ambiguïté et de fausse rigueur. Les termes les plus usuels ne sont jamais que des appro­ximations. Même les termes de foi, qu’on imaginerait volontiers de sens  \phantomsection
\label{p85} strict. Scrutant la carte religieuse de la France, voyez combien de dis­tinctions nuancées un savant, comme M. Le Bras, est aujourd’hui contraint de substituer à cette trop simple étiquette : « catholique » ? Il y a là de quoi faire réfléchir les historiens qui, du haut de leur croyance (parfois, et plus souvent peut‑être, de leur incroyance), tranchent roidement du catholicisme d’un Érasme. D’autres réalités, très vivantes, ont manqué à rencontrer les mots qu’il fallait. Un ouvrier, de nos jours, parle aisément de sa conscience de classe : fût‑elle, d’aventure, assez faible. Je ne crois pas que ce sentiment de solidarité raisonnée et armée se soit jamais mani­festé avec plus de force ni de clarté que parmi les manouvriers de nos campagnes du Nord, vers la fin de l’Ancien Régime ; diverses pétitions, certains cahiers de 1789 nous en ont conservé de poignants échos. Le sentiment, cependant, ne pouvait alors se nommer, parce qu’il n’avait pas encore de nom.\par

\astermono

\noindent Pour tout résumer d’un mot, le vocabulaire des documents n’est, à sa façon, rien d’autre qu’un témoignage. Précieux, sans doute, entre tous ; mais, comme tous les témoignages, imparfait ; donc, sujet à critique. Chaque terme important, chaque tour de style caractéristique devient un véritable élément de connaissance – mais seulement une fois confronté avec son entourage ; replacé dans l’usage de l’époque, du milieu ou de l’auteur ; défendu surtout, lorsqu’il a longuement survécu, contre le danger toujours présent du contre‑sens par anachronisme. L’onction royale était volontiers, au XII\textsuperscript{ᵉ} siècle, traitée de sacrement ; propos gros de signification assurément – dépourvu cependant, à cette date, de la valeur singulièrement plus forte que lui attribuerait, aujourd’hui, une théologie raidie dans ses définitions et, par suite, dans son lexique. L’avè­nement du nom est toujours un grand fait, même si la chose avait précédé ; car il marque l’époque décisive de la prise de conscience. Quel pas, le jour où les adeptes d’une foi nouvelle se dirent eux‑mêmes chrétiens ! Certains de nos aînés, comme Fustel de Coulanges, nous ont donné d’admi­rables modèles de cette étude des sens, de cette « sémantique historique ». Depuis leur temps, les progrès de la linguistique ont encore aiguisé l’outil. Puissent les jeunes chercheurs ne pas se lasser de le manier et surtout en porter l’emploi jusque dans les époques les plus proches de nous, qui sont, à, cet égard, de beaucoup les moins bien explorées.\par
Certes, si incomplète que soit généralement l’adhérence, les noms tiennent, malgré tout, aux réalités d’une prise beaucoup trop forte pour permettre jamais de décrire une société sans qu’un large emploi soit fait de ses mots, dûment expliqués et interprétés. Nous n’imiterons pas les éternels traducteurs du Moyen Âge. Nous parlerons de comtes quand il s’agira de comtes, de consuls si Rome est en scène. Un grand progrès  \phantomsection
\label{p86} a été accompli dans l’intelligence des religions helléniques lorsque, sur les lèvres des érudits, Jupiter s’est vu définitivement détrôné par Zeus. Mais ceci touche surtout le détail des institutions, de l’outillage ou des croyances. Estimer que la nomenclature des documents puisse suffire entièrement à fixer la nôtre reviendrait, en somme, à admettre qu’ils nous apportent l’analyse toute prête. L’histoire, en ce cas, n’aurait plus grand’chose à faire. Heureusement, pour notre plaisir, il n’en est rien. C’est pourquoi nous sommes contraints de chercher ailleurs nos grands cadres de classement.\par

\astermono

\noindent Pour les fournir, tout un lexique déjà s’offre à nous, dont la généralité se veut supérieure aux résonances d’aucune époque particulière. Élaboré, sans dessein préétabli, par les retouches successives de plusieurs géné­rations d’historiens, il réunit des éléments de date et de provenance très diverses. « Féodal ». « féodalité », termes de basoche, tirés du Palais dès le XVIII\textsuperscript{e} siècle par Boulainvilliers, puis par Montesquieu – pour devenir les étiquettes, assez gauches, d’un type de structure sociale lui-même assez mal défini. « Capital », mot d’usurier et de comptable, dont les économistes de bonne heure étendirent beaucoup la signification. « Capi­taliste », lointain débris du jargon des spéculateurs, dans les premières bourses européennes. Mais « capitalisme », qui tient aujourd’hui, dans nos classiques, une place bien plus considérable, est tout jeune : il porte sa désinence comme une marque d’origine (\emph{Kapitalismus}). « Révolution » a échangé, pour un sens très humain, ses anciennes associations astrolo­giques ; dans le ciel, c’était, c’est encore, un mouvement régulier et qui sans cesse revient sur lui-même ; sur terre, désormais, une brusque crise toute tendue vers l’avant. « Prolétaire » se costume à l’antique, comme les hommes de 89 qui, à la suite de Rousseau, firent sa fortune : mais Marx, après Babeuf, y a pour toujours mis sa griffe. L’Amérique même a donné « totem » et l’Océanie « tabou » : emprunts d’ethnographes, devant lesquels hésite encore le classicisme de certains historiens…\par
Ni cette variété d’origines, ni ces déviations de sens ne sont une gêne. Un mot vaut beaucoup moins par son étymologie que par l’usage qui en est fait. Si capitalisme, même dans ses applications les plus larges, est loin de s’étendre à tous les régimes économiques où le capital des prêteurs d’argent joua un rôle ; si féodal sert couramment à caractériser des so­ciétés dont le fief ne fut certes pas le trait le plus significatif : il n’y a rien là qui contredise l’universelle pratique de toutes les sciences, obligées (du moment qu’elles ne se contentent pas de purs symboles algébriques) à puiser dans le vocabulaire mêlé de la vie quotidienne. Se scandalisera‑t‑on si le physicien persiste à nommer atome, c’est‑à‑dire indivisible, l’objet de ses plus audacieuses dissections ?\par
 \phantomsection
\label{p87} Autrement redoutables sont les effluves émotives dont tant de ces mots nous arrivent chargés. Les puissances du sentiment favorisent rare­ment la précision, dans le langage.\par
L’usage, jusque chez les historiens, tend à embrouiller, de la façon la plus fâcheuse, les deux expressions de « régime féodal » et de « régime seigneurial ». C’est arbitrairement assimiler, au réseau de liens de dépen­dance caractéristique d’une aristocratie guerrière, un type de sujétion paysanne qui, très différent de nature, avait, en outre, pris naissance beaucoup plus tôt, dura plus longtemps et fut, à travers le monde, beau­coup plus répandu.\par
Le quiproquo remonte au XVIII\textsuperscript{e} siècle. La vassalité et le fief conti­nuaient alors d’exister, mais à l’état de simples formes juridiques, depuis plusieurs siècles à peu près vides de substance. Issue de ce même passé, la seigneurie, au contraire, demeurait bien vivante. Dans cet héritage, les écrivains politiques ne surent pas faire de distinctions. Ce n’était pas seulement qu’ils comprenaient mal. Pour la plupart, ils ne le considé­raient pas froidement. Ils en détestaient à la fois les archaïsmes et, plus encore, ce qu’il s’obstinait à contenir de forces oppressives. Une com­mune condamnation enveloppait le tout. Puis la Révolution abolit, simul­tanément et sous un nom unique, avec les institutions proprement féodales, la seigneurie. Il n’en subsista plus qu’un souvenir, mais tenace et que l’image des luttes des derniers temps colorait de teintes vives. La confusion désormais était acquise. Née de la passion, elle res­tait toute prête à s’étendre encore, sous l’effet. de passions nouvelles. Aujourd’hui même, quand nous évoquons à tort et à travers les « féo­dalités » industrielles ou bancaires, est‑ce tout à fait calmement. Il y a toujours, là-derrière, un reflet de brûlements de châteaux, durant l’ardent été de 89.\par
Or, tel est malheureusement le sort de beaucoup de nos mots. Ils con­tinuent à vivre à côté de nous, d’une trouble vie de place publique. Ce n’est pas un historien dont les harangues nous somment aujourd’hui d’identifier capitalisme et communisme. De signes souvent variables, selon les milieux ou les moments, ces coefficients d’affectivité n’engen­drent que plus d’équivoques. Devant le nom de révolution, les ultras de 1815 se voilaient la face. Ceux de 1940 en camouflent leur coup d’État.\par

\astermono

\noindent Supposons, cependant, notre vocabulaire définitivement rendu à l’impas­sibilité. Les plus intellectuelles des langues ont aussi leurs pièges. Certes, on n’éprouve pas ici la moindre tentation de rééditer les « plaisanteries nominalistes » dont François Simiand s’étonnait naguère, avec raison, de voir réserver aux sciences de l’homme « le singulier privilège ». De quel  \phantomsection
\label{p88} droit nous refuser les facilités de langage, indispensables à toute con­naissance rationnelle ? Parlons‑nous, par exemple, de machinisme ? Ce n’est nullement créer une entité. Sous un nom expressif, c’est grouper des faits, concrets à souhait et dont la similitude que le nom a proprement pour objet de signifier est aussi une réalité. En soi, ces rubriques n’ont donc rien que de légitime. Leur vrai danger vient de leur commodité même. Mal choisi ou trop mécaniquement appliqué, le symbole (qui n’était là que pour aider à l’analyse) finit par dispenser d’analyser. Par là, il fomente l’anachronisme : entre tous les péchés, au regard d’une science du temps, le plus impardonnable.\par
Les sociétés médiévales distinguaient deux grandes conditions humaines : il y avait des hommes libres, d’autres qui passaient pour ne l’être point. Mais la notion de liberté est de celles que chaque époque remanie à son gré. Certains historiens ont donc jugé de nos jours qu’au sens prétendument normal du mot, c’est‑à‑dire au leur, les non‑libres du Moyen Âge avaient été mal nommés. Ce n’étaient, disent‑ils, que des « demi-libres ». Mot inventé sans aucun appui dans les textes, cet intrus, en tout état de cause, serait encombrant. Il n’est malheureusement pas que cela. Par une conséquence à peu près inévitable, la fausse rigueur qu’il donnait au langage a paru rendre superflue toute recherche vraiment approfondie sur la frontière de la liberté et de la servitude, telle que ces civilisations en concevaient l’image : limite souvent incertaine, variable même selon les parti pris du moment ou de groupe, mais dont un des caractères essentiels fut, justement, de n’avoir jamais souffert cette zone marginale que suggère, avec une malencontreuse insistance, le nom de demi-liberté Une nomenclature imposée au passé aboutira toujours à la défor­mer, si elle a pour dessein ou seulement pour résultat de ramener ses ca­tégories aux nôtres, haussées pour l’occasion jusqu’à l’éternel. Ces éti­quettes‑là, il n’y a envers elles d’autre attitude raisonnable que de les éliminer.\par
Capitalisme a été un mot utile. Il le redeviendra, sans doute, quand on aura réussi à le laver de toutes les équivoques dont, à mesure qu’il passait plus avant dans l’usage courant, il s’est de plus en plus chargé. Pour l’instant, transporté sans précaution à travers les civilisations les plus diverses, il aboutit, presque fatalement, à en masquer les originalités. « Capitaliste », le régime économique du XVI\textsuperscript{ᵉ} siècle ? Il se peut. Considérez cependant cette sorte d’universelle découverte du gain d’argent, filtrant alors du haut en bas de la société, agrippant le marchand ou le notaire de village aussi bien que le gros banquier d’Augsbourg ou de Lyon ; voyez l’accent mis sur le prêt ou la spéculation commerciale beaucoup plus tôt que sur l’organisation de la production : dans sa contexture humaine, qu’il était donc différent, ce « capitalisme » de la Renaissance, du système bien plus hiérarchisé, du système manufacturier, du système saint‑simonien de l’ère de la révolution industrielle ! Qui, à son tour…\par
 \phantomsection
\label{p89}Aussi bien, une remarque très simple suffirait à mettre on garde. Le capitalisme, non plus d’une époque déterminée, mais le capitalisme en soi, le capitalisme avec un grand C, à quelle date en fixer l’apparition ? dans l’Italie du XII\textsuperscript{ᵉ} siècle ? dans la Flandre du XIII\textsuperscript{ᵉ} ? au temps des Fugger et de la Bourse d’Anvers ? au XVIII\textsuperscript{e} siècle, voire au XIX\textsuperscript{ᵉ} ? Autant d’historiens, autant d’actes de naissance. Presque aussi nombreux, en vérité, que ceux de cette Bourgeoisie dont les manuels scolaires fêtent l’accession au pouvoir, selon les périodes successivement proposées aux méditations de nos marmots, tantôt sous Philippe le Bel, tantôt sous Louis XIV, à moins que ce ne soit en 1789 ou en 1830… Peut‑être, après tout, n’était‑ce pas exactement la même bourgeoisie ? Pas plus que le même capitalisme…\par

\astermono

\noindent Et voilà, je crois, où nous touchons le fond des choses. On se souvient de la jolie phrase de Fontenelle : Leibniz, disait‑il, « pose des définitions exactes, \emph{qui le privent de l’agréable liberté d’abuser des termes dans les occasions »}. Agréable, je ne sais ; périlleuse certainement. C’est une liberté qui ne nous est que trop familière. L’historien définit rarement. Il pourrait, en effet, juger ce soin superflu, s’il puisait dans un usage lui-même de sens strict. Comme tel n’est pas le cas, il n’a, jusque dans l’emploi de ses mots‑clefs, guère d’autre guide que son instinct personnel. Il étend, restreint, déforme despotiquement les significations – sans avertir le lecteur ; sans toujours s’en bien rendre compte lui-même. Que de « féoda­lités », de par le monde, depuis la Chine jusqu’à la Grèce des Achéens aux belles cnémides ? Pour la plupart, elles ne se ressemblent guère. C’est que chaque historien, ou peu s’en faut, comprend le nom à sa guise.\par
Définissons‑nous, cependant, par aventure ? Le plus souvent, c’est, chacun pour nous. Rien de plus significatif que le cas d’un analyste de l’économie aussi pénétrant que John Meynard Keynes. Il n’est presque aucun de ses livres où on ne le voie d’abord, s’emparant de termes, par exception, assez bien fixés, leur décréter des sens tout neufs : changeant, parfois, d’ouvrage à ouvrage ; volontairement éloignés, en tout cas, de la commune pratique. Curieux travers des sciences de l’homme qui, d’avoir longtemps figuré parmi les « belles‑lettres », semblent garder quelque chose de l’impénitent individualisme de l’art ! Conçoit‑on un chimiste disant : « il faut, pour former une molécule d’eau, deux corps : l’un fournit deux atomes, l’autre un seul : dans mon vocabulaire, c’est le premier qui s’appellera oxygène et le second hydrogène » ? Si bien définis qu’on les suppose, des langages d’historiens, côte à côte alignés, ne feront jamais le langage de l’histoire.\par
 \phantomsection
\label{p90} À dire vrai, des efforts mieux concertés ont été, çà et là, tentés ; par des groupes de spécialistes que la jeunesse relative de leurs disciplines semble mettre à l’abri des pires routines corporatives (linguistes, ethno­graphes, géographes) ; pour l’histoire tout entière, par le Centre de Syn­thèse, Toujours à l’affût des services à rendre et des exemples à donner. On doit beaucoup en attendre. Mais moins encore, peut‑être, que des progrès d’une diffuse bonne volonté. Un jour viendra, sans doute, où une série d’ententes permettront de préciser la nomenclature, puis, d’étape en étape, de l’affiner. Alors même, l’initiative du chercheur conservera traditionnellement, les articulations de son récit : quand, du moins, elle ne se contentait pas, se faisant annales, de boitiller de millésime en millé­sime.\par
L’une détruisant l’autre, les dominations des peuples conquérants traçaient les grandes époques. La mémoire collective du Moyen Âge presque tout entier vécut ainsi du mythe biblique des Quatre Empires assyrien, perse, grec, romain. Moule incommode, pourtant, s’il en fut. Il ne contraignait pas seulement, par soumission au texte sacré, de pro­longer jusqu’au présent le mirage d’une fictive unité romaine. Par un paradoxe étrange dans une société de chrétiens – comme il doit l’être aujourd’hui aux yeux de tout historien – la Passion semblait, dans la marche de l’humanité, un relais moins notable que les victoires d’illustres ravageurs de provinces. Quant aux divisions plus petites, la succession des monarques, dans chaque nation, leur donnait leurs limites.\par
Ces habitudes se sont prouvées merveilleusement tenaces. Fidèle miroir de l’école française, aux environs de 1900, l’\emph{Histoire de France} avance encore en achoppant de règne en règne ; à chaque mort de prince, racontée avec le détail qui s’attache aux grands événements, elle marque une halte. N’est‑il plus de rois ? Les systèmes de gouvernement, par bonheur, sont, eux aussi, mortels : leurs révolutions servent donc de jalons. Plus près de nous, c’est par « prépondérances » nationales – équivalents édul­corés des Empires d’autrefois – qu’une importante collection de manuels segmente volontiers le cours de l’histoire moderne. Espagnole, française, anglaise, ces hégémonies sont – est‑il besoin de le dire – de nature diplomatique et militaire. Le reste se range comme il peut.\par
Voilà beau temps cependant, que le XVIII\textsuperscript{e} siècle avait fait entendre sa protestation. « Il semble, écrivait Voltaire, que depuis quatorze cents ans, il n’y ait eu dans les Gaules que des rois, des ministres et des généraux. » Peu à peu, des divisions nouvelles apparurent donc, qui, ­étrangères à l’obsession impérialiste ou monarchique, entendaient se régler sur des phénomènes plus profonds. « Féodalité », comme nom d’une période autant que d’un système social et politique, date, on l’a vu, de ce temps. Mais, entre toutes, les destinées du mot de « Moyen Âge » sont instructives.\par

\astermono

\noindent  \phantomsection
\label{p91} Par son origine lointaine, il était lui-même médiéval. Il appartenait au vocabulaire de ce prophétisme à demi-hérétique qui, depuis le XIII\textsuperscript{ᵉ} siècle surtout, avait séduit tant d’âmes inquiètes. L’Incarnation avait mis fin à l’Ancienne Loi. Elle n’avait pas établi le Royaume de Dieu. Tendu vers l’espérance de ce jour béni, le temps présent n’était donc qu’un âge intermédiaire, un \emph{medium aevum.} Puis, dès les premiers huma­nistes, semble‑t‑il, auxquels cette langue mystique demeurait familière, l’image fut détournée vers des réalités plus profanes. En un sens, le règne de l’Esprit était venu. C’était cette « restauration » des lettres et de la pensée, dont la conscience, chez les meilleurs, se faisait alors si vive : témoin Rabelais, témoin Ronsard. L’ » âge moyen » était clos qui, entre la féconde Antiquité et sa nouvelle Révélation, n’avait figuré, lui aussi, qu’une longue attente. Ainsi entendue, l’expression, pendant plusieurs générations, vécut obscurément, bornée sans doute à quelques cercles érudits. Ce fut, croit‑on, tout à la fin du XVII\textsuperscript{ᵉ} siècle qu’un Allemand, un modeste faiseur de manuels, Christophe Keller, imagina, dans un ouvrage d’histoire générale, d’étiqueter « Moyen Âge » toute la période, beaucoup plus que millénaire, qui va des Invasions à la Renaissance. L’usage introduit, on ne sait trop par quels chenaux, prit définitivement droit de cité dans l’historiographie européenne et, notamment, française, vers le temps de Guizot et de Michelet.\par
Voltaire l’avait ignoré. « Vous voulez enfin surmonter le dégoût que vous cause l’\emph{Histoire moderne, depuis la décadence de l’Empire romain »} : on a reconnu la première phrase de l’\emph{Essai sur les Mœurs.} N’en doutons point pourtant : c’est bien l’esprit de l’Essai qui, si puissant sur les géné­rations suivantes, fit le succès de « Moyen Âge ». Comme d’ailleurs, de son pendant presque nécessaire, Renaissance. Courant de longue date dans le vocabulaire de l’histoire du goût, mais comme nom commun et avec l’adjonction obligée d’un complément (« la renaissance des arts ou des lettres sous Léon X, ou sous François 1\textsuperscript{ᵉʳ} », disait‑on), il ne conquit guère avant Michelet, en même temps que la majuscule, l’honneur de servir à lui tout seul de signe à la période entière. Des deux parts, l’idée est la même. Naguère, les batailles, la politique des cours, l’ascension ou la chute des grandes dynasties. fournissaient le cadre. Sous leurs ban­nières, l’art, la littérature, les sciences s’ordonnaient tant bien que mal. Il faudra désormais inverser. Aux époques de l’humanité, ce sont les manifestations les plus raffinées de l’esprit humain qui, par leurs variables progrès, donnent le ton. Point d’idée qui, plus clairement que celle‑là, ne porte la griffe voltairienne.\par
Mais une grave faiblesse viciait ces classifications : le trait distinctif était, en même temps, un jugement. « L’Europe, comprimée entre la tyrannie sacerdotale et le despotisme militaire, attend dans le sang et  \phantomsection
\label{p92} dans les larmes le moment où de nouvelles lumières lui permettront de renaître à la liberté, à l’humanité et aux vertus. » Ainsi Condorcet décrivait l’époque à laquelle un unanime consentement allait bientôt consacrer le nom de Moyen Âge. Du moment que nous ne croyons plus à cette « nuit », que nous avons renoncé à peindre comme un désert uniformément stérile des siècles qui, dans le domaine des inventions techniques, de l’art, du sentiment, de la réflexion religieuse, furent si riches, qui ont vu le premier essor de l’expansion économique européenne, qui nous ont, enfin, donné nos patries, quelle raison pourrait encore subsister pour confondre, sous une rubrique fallacieusement commune, la Gaule de Clovis et la France de Philippe le Bel, Alcuin avec saint Thomas ou Occam, le style animalier des bijoux « barbares » et les statues de Chartres, les petites villes res­serrées des temps carolingiens et les rayonnantes bourgeoisies de Gênes, de Bruges ou de Lubeck. Le Moyen Âge, en vérité, ne vit plus que d’une humble vie pédagogique : contestable commodité de programmes ; éti­quette, surtout, de techniques érudites, dont le champ, d’ailleurs, se trouve assez mal délimité par les dates traditionnelles. Le médiéviste est l’homme qui sait lire de vieilles écritures, critiquer une charte, comprendre le vieux français. C’est quelque chose, sans doute. Pas assez assurément, pour satisfaire, dans la recherche des divisions exactes, une science du réel.\par

\astermono

\noindent Dans le désarroi de nos classifications chronologiques, une mode s’est glissée, assez récente, je crois, d’autant plus envahissante, en tout cas, qu’elle est moins raisonnée. Volontiers, nous comptons par siècles.\par
Longtemps étranger à tout dénombrement exact d’années, le mot, lui aussi, avait originairement ses résonances mystiques : accents de Qua­trième Églogue ou de \emph{Dies Irae.} Peut‑être ne s’étaient‑elles pas tout à fait amorties au temps où, sans grand souci de précision numérique, l’histoire s’attardait, avec complaisance, sur le « siècle de Périclès », sur celu­i de « Louis XIV ». Mais notre langage s’est fait plus sévèrement mathéma­ticien. Nous ne nommons plus les siècles d’après leurs héros. Nous les numérotons à la file, bien sagement, de cent ans en cent ans, depuis un point de départ une fois pour toutes fixé à l’an un de notre ère. L’art du XIII\textsuperscript{ᵉ} siècle, la philosophie du XVIII\textsuperscript{e}, le « stupide XIX\textsuperscript{ᵉ} » : ces figures au masque arithmétique hantent les pages de nos livres. Qui de nous se vantera d’avoir toujours échappé aux séductions de leur apparente com­modité ?\par
Par malheur, aucune loi de l’histoire n’impose que les années dont le millésime se terminent par le chiffre 1 coïncident avec les points critiques de l’évolution humaine. D’où d’étranges fléchissements de sens. « Il est bien connu que le dix‑huitième siècle commence en 1715 et s’achève en \textsubscript{p.93 1}789. » J’ai lu cette phrase naguère dans une copie d’étudiant. Candeur ? Ou malice ? Je ne sais. C’était en tout cas assez bien mettre à nu cer­taines bizarreries de l’usage. Mais, s’il s’agit du XVIII\textsuperscript{e} siècle philoso­phique, on pourrait sans doute encore mieux dire qu’il débuta fort avant 1701 : l’\emph{Histoire des Oracles} parut en 1687 et le \emph{Dictionnaire} de Bayle en 1697. Le pis est que le nom, comme toujours, entraînant avec lui l’idée, ces fausses étiquettes finissent par tromper sur la marchandise. Les médié­vistes parlent de la « renaissance du douzième siècle ». Grand mouvement intellectuel, assurément. À l’inscrire cependant, sous cette rubrique, on se laisse trop aisément aller à oublier qu’il débuta, en réalité, vers 1060, et certaines liaisons essentielles échappent. En un mot, nous nous donnons l’air de distribuer, selon un rigoureux rythme pendulaire, arbitrairement choisi, des réalités auxquelles cette régularité est tout à fait étrangère. C’est une gageure. Nous la tenons naturellement fort mal. Il faut chercher mieux.\par

\astermono

\noindent Tant qu’on s’en tient à étudier, dans le temps, des chaînes de phéno­mènes apparentés, le problème, en somme, est simple. C’est à ces phéno­mènes mêmes qu’il convient de demander leurs propres périodes. Une histoire religieuse du règne de Philippe‑Auguste ? Une histoire écono­mique du règne de Louis XV ? Pourquoi pas : « Journal de ce qui s’est passé, dans mon laboratoire sous la deuxième présidence de Grévy », par Louis Pasteur ? Ou, inversement, « Histoire diplomatique de l’Europe, depuis Newton jusqu’à Einstein »\par
Sans doute, on voit bien par où des divisions tirées uniformément de la suite des empires, des rois ou des régimes politiques ont pu séduire. Elles n’avaient pas seulement pour elles le prestige qu’une longue tradition attache à l’exercice du pouvoir, « à ces actions, disait Machiavel, qui ont l’air de grandeur propre aux actes du gouvernement ou de l’État ». Un avènement, une révolution ont leur place fixée, dans la durée, à une année, voire à un jour près. Or, l’érudit aime comme on dit, à « dater finement ». Il y trouve, avec l’apaisement d’une instinctive horreur du vague, une grande commodité de conscience. Il souhaite avoir tout lu, tout compulsé de ce qui concerne son sujet. Combien sera‑t‑il plus à l’aise si, devant chaque dossier d’archives, il peut, calendrier en mains, faire le partage : avant, pendant, après !\par
Gardons‑nous, pourtant, de sacrifier à l’idole de la fausse exactitude. La coupure la plus exacte n’est pas forcément celle qui fait appel à l’unité de temps la plus petite – auquel cas, il faudrait préférer, non seulement l’année à la décade, mais aussi la seconde au jour – c’est la mieux adaptée à la nature des choses. Or chaque type de phénomènes a son épaisseur  \phantomsection
\label{p94} de mesure particulière et, pour ainsi dire, sa décimale spécifique. Les transformations de la structure sociale, de l’économie, des croyances, du comportement mental ne sauraient, sans déformation, se plier à un chro­nométrage trop serré. Lorsque j’écris qu’une modification extrêmement profonde de l’économie occidentale, marquée à la fois par les premières importations massives de blés exotiques et par le premier grand rayon­nement des industries allemande et américaine, se produisit entre 1875 et 1885 environ, j’use de la seule approximation qu’autorise ce genre de faits. Une date soi-disant plus précise trahirait la vérité. De même, en statistique, une moyenne décennale n’est pas, en soi, plus grossière qu’une moyenne annuelle ou hebdomadaire. Simplement, elle exprime un autre aspect de la réalité.\par
Il n’est d’ailleurs nullement impossible, \emph{a priori}, qu’à l’expérience, les phases naturelles de phénomènes d’ordre en apparence très différent, ne se trouvent se recouvrir. Est‑il exact que l’avènement du Second Empire introduisit une période nouvelle dans l’économie française ? Sombart avait‑il raison d’identifier l’essor du capitalisme avec celui de l’esprit protestant ? M. Thierry Maulnier voit‑il juste en découvrant dans la démocratie « l’expression politique » de ce même capitalisme (pas tout à fait le même, en réalité, je le crains) ? Nous n’avons pas le droit de rejeter de parti pris ces coïncidences, si douteuses qu’elles puissent nous sembler. Mais elle n’apparaîtront, s’il y a lieu, qu’à une condi­tion : de ne pas avoir été postulées à l’avance. Certainement, les marées sont en rapport avec les lunaisons. Pour le savoir, cependant, il a fallu d’abord déterminer, à part, les époques du flux et celles de la lune.\par

\astermono

\noindent Envisageant, au contraire, l’évolution sociale dans son intégralité, s’agit‑il d’en caractériser les étapes successives ? C’est un problème de note dominante. On ne peut ici que suggérer les voies où la classification semble devoir s’engager. L’histoire, ne l’oublions pas, est encore une science en travail.\par
Les hommes qui sont nés dans une même ambiance sociale, à des dates voisines, subissent nécessairement, en particulier dans leur période de formation, des influences analogues. L’expérience prouve que leur com­portement présente, par rapport aux groupes sensiblement plus vieux ou plus jeunes, des traits distinctifs ordinairement fort nets. Cela, jusque dans leurs désaccords, qui peuvent être des plus aigus. Se passionner pour un même débat, fût‑ce en sens opposé, c’est encore se ressembler. Cette communauté d’empreinte, venant d’une communauté d’âge, fait une génération.\par
 \phantomsection
\label{p95} Une société, à vrai dire, est rarement une. Elle se décompose en milieux différents. Dans chacun d’eux, les générations ne se recouvrent pas tou­jours : les forces qui agissent sur un jeune ouvrier s’exercent‑elles fatalement, au moins avec une intensité égale, sur le jeune paysan ? Ajou­tez, même dans les civilisations les mieux liées, la lenteur de propagation de certains courants. « On était romantique, en province, durant mon ado­lescence, alors que Paris avait cessé de l’être, me disait mon père, né à Strasbourg en 1848. Souvent d’ailleurs, comme dans ce cas, l’opposition se réduit surtout à un décalage. Quand donc nous parlons de telle ou telle génération française, par exemple, nous évoquons une image complexe et non, parfois, sans discordance – mais dont il est naturel de retenir avant tout les éléments vraiment directeurs.\par
Quant à la périodicité des générations, il va de soi qu’en dépit des rêveries pythagoriciennes de certains auteurs, elle n’a rien de régulier. Selon la cadence plus ou moins vive du mouvement social, les limites se resserrent ou s’écartent. Il y a, en histoire, des générations longues et des générations courtes. Seule l’observation permet de saisir les points où la courbe change d’orientation. J’ai appartenu à une École où les dates d’entrée facilitent les repères. De bonne heure, je me suis reconnu, à beaucoup d’égards, plus proche des promotions qui m’avaient précédé que de celles qui me suivirent presque immédiatement. Nous nous placions, mes camarades et moi, à la pointe dernière de ce qu’on peut appeler, je crois, la génération de l’Affaire Dreyfus. L’expérience de la vie n’a pas démenti cette impression.\par
Il arrive enfin, forcément, que les générations s’interpénètrent. Car les individus ne réagissent pas toujours pareillement aux mêmes influences. Parmi nos enfants, il est, dès aujourd’hui, assez aisé de discerner, en gros, selon les âges, la génération de la guerre de celle qui sera, seulement, celle de d’après‑guerre. À une réserve près, toutefois : dans les âges qui ne sont pas encore l’adolescence presque mûre, et ont pourtant dépassé la petite enfance, la sensibilité aux événements du présent varie beaucoup avec les tempéraments personnels ; les plus précoces seront vraiment « de la guerre » ; les autres demeureront sur le bord opposé.\par
La notion de génération est donc très souple, comme tout concept qui s’efforce d’exprimer, sans les déformer, les choses de l’homme. Mais elle répond aussi à des réalités que nous sentons très concrètes. Depuis long­temps, on l’a vu utilisée, comme d’instinct, par des disciplines que leur nature conduisait à se refuser, avant toutes autres, aux vieilles divisions par règnes ou par gouvernements : telle l’histoire de la pensée, ou celle des forces artistiques. Elle semble destinée à fournir, de plus en plus, à une analyse raisonnée des vicissitudes humaines, son premier jalonnement.\par
Mais une génération ne représente qu’une phase relativement courte. Les phases plus longues se nomment civilisations.\par
 \phantomsection
\label{p96} Grâce à Lucien Febvre, nous connaissons bien l’histoire du mot, insé­parable, cela va de soi, de celle de l’idée. Il ne s’est dégagé que lentement du jugement de valeur. Plus exactement, une dissociation s’est produite. Nous parlons encore (quoique avec moins d’assurance, hélas ! que nos aînés) de la civilisation en soi qui est un idéal, et de la difficile ascension de l’humanité vers ses nobles douceurs ; mais aussi des civilisations, au pluriel, qui sont simplement des réalités. Nous admettons, désormais, qu’il y ait, si j’ose dire, des civilisations de non‑civilisés. C’est que nous avons reconnu que, dans une société, quelle qu’elle soit, tout se lie et se commande mutuellement : la structure politique et sociale, l’économie, les croyances, les manifestations les plus élémentaires comme les plus subtiles de la mentalité. Ce complexe – « au sein duquel », écrivait déjà Guizot, « tous les éléments de la vie d’un peuple, toutes les forces de son existence. viennent se réunir » – comment l’appeler ? Créé par le XVIII\textsuperscript{e} siècle pour exprimer un bien absolu, le nom de civilisation, à mesure que les sciences de l’homme devenaient plus relativistes, s’est plié, naturellement, sans perdre son sens ancien, à ce nouveau sens de fait. Il y garde seulement, de ce qui fut autrefois sa signification unique, comme une résonance de sympathie humaine, doit le prix n’est pas négligeable.\par
Les oppositions entre civilisations apparaissent clairement dès que, dans l’espace, le contraste se relève d’exotisme : contestera‑t‑on qu’il n’y ait aujourd’hui une civilisation chinoise ? Ni qu’elle ne diffère grandement de l’européenne ? – Mais, sur les mêmes lieux aussi, l’accent majeur du complexe social peut se modifier, plus ou moins lentement ou brusquement. Quand la transformation s’est opérée, nous disons qu’une civilisation succède à une autre. Parfois, il y a secousse venue du dehors et accom­pagnée, ordinairement, de l’insertion d’éléments humains nouveaux : ainsi entre l’Empire romain et les sociétés du haut Moyen Âge. Parfois, au contraire, simple changement intérieur : la civilisation de la Renais­sance, par exemple, dont nous avons si largement hérité, chacun néanmoins s’accordera à penser qu’elle n’est plus la nôtre. Ces tonalités diverses sont difficiles à exprimer, sans doute. Elles ne sauraient l’être par des étiquettes trop sommaires. La commodité des mots en isme (\emph{Typismus, Konven­tionalismus)} a ruiné l’essai de description évolutive, pourtant intelligent, tenté naguère par Karl Lamprecht, dans son \emph{Histoire d’Allemagne.} C’était déjà l’erreur de Taine, chez qui nous étonne si fort aujourd’hui l’espèce de réalité personnelle confinée à la « conception dominatrice ». Cependant, que certains efforts aient pu échouer ne justifie pas le renoncement. Affaire à la recherche d’introduire dans ses distinctions une justesse et une finesse croissantes.\par

\astermono

\noindent Le temps humain, en résumé, demeurera toujours rebelle à l’implacable uniformité comme au sectionnement rigide du temps de l’horloge. Il lui faut des mesures accordées à la variabilité de son rythme et qui, pour limites, acceptent souvent, parce que la réalité le veut ainsi, de ne con­naître que des zones marginales. C’est seulement au prix de cette plasticité que l’histoire peut espérer adapter, selon le mot de Bergson, ses classi­fications aux « lignes mêmes du réel » : ce qui est, proprement, la fin dernière de toute science.
\section[{Chapitre V}]{Chapitre V}\renewcommand{\leftmark}{Chapitre V}

\noindent  \phantomsection
\label{p99} En vain, le positivisme a prétendu éliminer de la science l’idée de cause. Bon gré mal gré, tout physicien, tout biologiste pense par « pourquoi » et « parce que ». À cette commune loi de l’esprit, les historiens ne sauraient échapper. Les uns, comme Michelet, enchaînent dans un grand « mouve­ment vital », plutôt qu’ils n’expliquent en forme logique ; d’autres étalent leur appareil d’inductions et d’hypothèses ; partout le lien génétique est présent. Mais, de ce que l’établissement de rapports de cause à effet constitue ainsi un besoin instinctif de notre entendement, il ne s’ensuit pas que leur recherche puisse être abandonnée à l’instinct. Si la méta­physique de la causalité est ici hors de notre horizon, l’emploi de la relation causale, comme outil de la connaissance historique, exige incontestable­ment une prise de conscience critique.\par

\astermono

\noindent Un homme, je suppose, marche sur un sentier de montagne. Il trébuche et tombe dans un précipice. Il a fallu, pour que cet accident arrivât, la réunion d’un grand nombre d’éléments déterminants. Tels, entre autres : l’existence de la pesanteur, la présence d’un relief, résultant lui-même de longues vicissitudes géologiques ; le tracé d’un chemin, destiné, par exemple, à relier un village à ses pâturages d’été. Il sera donc parfai­tement légitime de dire que, si les lois de la mécanique céleste étaient différentes, si l’évolution de la Terre avait été autre, si l’économie alpestre ne se fondait pas sur la transhumance saisonnière, la chute n’aurait pas eu lieu. Demande‑t‑on cependant quelle en fut la cause ? Chacun répon­dra : le faux pas. Ce n’est point que cet antécédent‑là fût plus nécessaire à l’événement. Beaucoup d’autres l’étaient au même degré. Mais, entre tous, il se distingue par plusieurs caractères très frappants : il est venu le dernier ; il était le moins permanent, le plus exceptionnel dans l’ordre  \phantomsection
\label{p100} général du monde ; enfin, en raison même de cette moindre généralité, son intervention semble celle qui eût pu le plus facilement être évitée. Pour ces raisons, il paraît lié à l’effet d’une prise plus directe et nous n’échappons guère au sentiment qu’il l’ait seul véritablement produit. Aux yeux du sens commun qui, en parlant de cause, a toujours peine à se dépouiller d’un certain anthropomorphisme, ce composant de la dernière minute, ce composant particulier et inopiné fait un peu figure de l’artiste, qui donne forme à une matière plastique déjà toute préparée.\par
Le raisonnement historique, dans sa pratique courante, ne procède pas autrement. Les antécédents les plus constants et les plus généraux, si nécessaires soient‑ils, demeurent simplement sous‑entendus. Quel historien militaire songera à ranger parmi les raisons d’une victoire la gravitation, qui rend compte des trajectoires des obus, ou les dispositions physiolo­giques du corps humain, sans lesquelles les projectiles ne feraient pas d’atteintes mortelles ? Les antécédents déjà plus particuliers, mais doués encore, d’une certaine permanence, forment ce qu’on est convenu d’appeler les conditions. Le plus spécial, celui qui, dans le faisceau des forces géné­ratrices, représente, en quelque sorte l’élément différentiel, reçoit de pré­férence le nom de cause. On dira, par exemple, que l’inflation du temps de Law fut la cause de la hausse globale des prix. L’existence d’un milieu économique français, déjà homogène et bien lié, sera seulement une condition. Car ces facilités de circulation qui, en répandant les billets de toutes parts, permirent seule la hausse avaient précédé l’inflation et lui survécurent.\par

\astermono

\noindent Qu’il réside, dans cette discrimination, un principe fécond de recherche, on n’en saurait douter. À quoi bon s’appesantir sur des antécédents quasiment universels ? Ils sont communs à trop de phénomènes pour mériter de figurer dans la généalogie d’aucun d’eux en particulier. Je sais bien, d’avance, qu’il n’y aurait pas d’incendie si l’air ne contenait de l’oxygène ; ce qui m’intéresse, ce qui appelle et justifie un effort de découverte, c’est de déterminer comment le feu a pris. Les lois des tra­jectoires valent pour la défaite comme pour la victoire ; elles les expliquent toutes deux ; elles sont donc inutiles à l’explication propre de l’une ou de l’autre.\par
Mais on ne saurait sans danger élever à l’absolu un classement hiérar­chique qui n’est, au vrai, qu’une commodité de l’esprit, La réalité nous présente une quantité presque infinie de lignes de force qui, toutes, con­vergent vers un même phénomène. Le choix que nous faisons parmi elles peut bien se fonder sur des caractères, en pratique, très dignes d’attention ; ce n’est jamais qu’un choix. Il y a, notamment, beaucoup d’arbitraire dans l’idée d’une cause par excellence, opposée aux simples « conditions ».  \phantomsection
\label{p101} Simiand lui-même, si épris de rigueur et qui avait d’abord tenté (vaine­ment, je crois) des définitions plus strictes, semble avoir fini par recon­naître le caractère tout relatif de cette distinction. « Une épidémie », écrit‑il, « aura comme cause, pour le médecin, la propagation d’un microbe et, comme condition, la malpropreté, la mauvaise santé, engendrées par le paupérisme ; pour le sociologue et le philanthrope, le paupérisme sera la cause et les facteurs biologiques la condition. » C’est admettre, de bonne foi, la subordination de la perspective à l’angle propre de l’enquête.\par
Prenons‑y garde, d’ailleurs : la superstition de la cause unique, en histoire, n’est trop souvent que la forme insidieuse de la recherche du respon­sable : partant, du jugement de valeur. « À qui la faute, ou le mérite ? » dit le juge. Le savant se contente de demander « pourquoi » ? et il accepte que la réponse ne soit pas simple. Préjugé du sens commun, postulat de logicien ou tic de magistrat instructeur, le monisme de la cause ne serait pour l’explication historique qu’un embarras. Elle cherche des trains d’ondes causales et ne s’effraie pas, puisque la vie les montre ainsi, de les trouver multiples.\par

\astermono

\noindent Les faits historiques sont, par essence, des faits psychologiques. C’est donc dans d’autres faits psychologiques qu’ils trouvent normalement leurs antécédents. Sans doute, les destinées humaines s’insèrent dans le monde physique et en subissent le poids. Là même, pourtant, où l’intru­sion de ces forces extérieures semble la plus brutale, leur action ne s’exerce qu’orientée par l’homme et son esprit. Le virus de la Peste Noire fut la cause première du dépeuplement de l’Europe. Mais l’épidémie ne se pro­pagea si rapidement qu’en raison de certaines conditions sociales – donc, dans leur nature profonde, mentales – et ses effets moraux s’expli­quent seulement par les prédispositions particulières de la sensibilité collective.\par
Cependant, il n’est pas de psychologie que de la conscience claire. À lire certains livres d’histoire, on croirait l’humanité composée unique­ment de volontés logiciennes, pour qui leurs raisons d’agir n’auraient jamais le moindre secret. Face à l’état actuel des recherches sur la vie mentale et ses obscures profondeurs, c’est une preuve de plus de l’éternelle difficulté que les sciences éprouvent de rester exactement contempo­raines les unes des autres. C’est aussi répéter, en l’amplifiant, l’erreur, si souvent dénoncée pourtant, de la vieille théorie économique. Son \emph{homo œconomicus} n’était pas une ombre vaine seulement parce qu’on le sup­posait exclusivement occupé de ses intérêts : la pire illusion consistait à imaginer qu’il pût se faire de ces intérêts une idée si nette. « Il n’y a rien de plus rare qu’un dessein », disait déjà Napoléon. La lourde atmos­phère morale où nous sommes en ce moment plongés, estimera‑t‑on  \phantomsection
\label{p102} qu’elle marque en nous seulement l’homme des décisions raisonnées ? On fausserait gravement le problème des causes, en histoire, si on le réduisait, toujours et partout, à un problème de motifs.\par

\astermono

\noindent Quelle curieuse antinomie, d’ailleurs, dans les attitudes successives de tant d’historiens ! S’agit‑il de s’assurer si un acte humain a vraiment eu lieu ? Ils ne sauraient porter dans cette recherche assez de scrupules. Passent‑ils aux raisons de cet acte ? La moindre apparence les satisfait : fondée à l’ordinaire sur un de ces apophtegmes de banale psychologie, qui ne sont ni plus ni moins vrais que leurs contraires.\par
Deux critiques de formation philosophique, Georges Simmel en Alle­magne, François Simiand en France, se sont divertis à mettre à nu quelques­unes de ces pétitions de principe. Les Hébertistes, écrit un historien allemand, s’accordèrent d’abord parfaitement avec Robespierre, parce qu’il se pliait à tous leurs désirs ; puis ils s’écartèrent de lui, parce qu’ils le jugeaient trop puissant. C’est, observe en substance Simmel, sous­-entendre les deux propositions suivantes : un bienfait provoque la recon­naissance ; on n’aime pas à être dominé. Or ces deux propositions ne sont pas forcément fausses, sans doute. Mais ni forcément justes non plus. Car ne pourrait‑on soutenir, avec une égale vraisemblance, qu’une soumission trop prompte aux volontés d’un parti excite chez lui plus de mépris pour cette faiblesse que de gratitude ; et n’a‑t‑on jamais vu, d’autre part, un dictateur, par la crainte qu’inspire sa puissance, étouffer jusqu’à la moindre velléité de résistance ? – Un scolastique disait de l’autorité qu’elle a « un nez de cire, qui se plie indifféremment à gauche ou à droite ». Ainsi des prétendues vérités psychologiques du sens commun.\par
L’erreur, au fond, est analogue à celle dont s’inspirait le pseudo‑déter­minisme géographique, aujourd’hui définitivement ruiné. Que ce soit en présence d’un phénomène du monde physique ou d’un fait social, les réactions humaines n’ont rien d’un mouvement d’horlogerie, toujours déclenché dans le même sens. Le désert, quoi qu’en ait dit Renan, n’est pas nécessairement « monothéiste », parce que les peuples qui le hantent n’apportent pas tous à ses spectacles la même âme. Le petit nombre des points d’eau entraînerait, en tous lieux, le groupement de l’habitat rural et leur abondance sa dispersion seulement s’il était vrai que les campagnards fissent, obligatoirement, passer avant toute autre préoc­cupation la proximité des sources, des puits ou des mares. Il arrive, en réalité, qu’ils préfèrent se rassembler, par souci de sécurité ou d’entr’aide, voire par simple humeur grégaire, là même où tout coin de terre a sa fontaine ; ou bien qu’inversement (comme dans certaines régions de la Sardaigne), chacun établissant sa demeure au centre de son menu domaine, ils acceptent pour prix de cet égaillement qui leur tient à cœur,  \phantomsection
\label{p103} de longs cheminements vers l’eau rare. Dans la nature, l’homme n’est‑il point, par excellence, la grande variable ?\par

\astermono

\noindent Ne nous y trompons point cependant. La faute n’est pas, en pareil cas, dans l’explication elle‑même. Elle réside tout entière dans son a-prio­risme. Bien que les exemples, jusqu’ici, n’en paraissent guère fréquents, il se peut que, dans des conditions sociales données, la répartition des ressources en eau décide, avant toute autre cause, de l’habitat. Ce qui est sûr, c’est qu’elle n’en décide pas nécessairement. Il n’est nullement impossible que les Hébertistes aient vraiment obéi aux motifs que leur prêtait leur historien. Le tort a été de considérer cette hypothèse, d’avance, comme acquise. Il fallait la prouver. Puis, une fois fournie cette preuve – qu’on n’a pas le droit de tenir, de parti pris, pour impraticable – il restait encore, creusant plus avant l’analyse, à se demander pourquoi, de toutes les attitudes psychologiques concevables, celles‑là s’imposèrent, au groupe. Car, du moment qu’une réaction de l’intelligence ou de la sensibilité ne va jamais de soi, elle exige à son tour, si elle se produit, qu’on s’efforce d’en découvrir les raisons. Pour tout dire d’un mot, les causes, en histoire pas plus qu’ailleurs, ne se postulent pas. Elles se cher­chent…
 


% at least one empty page at end (for booklet couv)
\ifbooklet
  \pagestyle{empty}
  \clearpage
  % 2 empty pages maybe needed for 4e cover
  \ifnum\modulo{\value{page}}{4}=0 \hbox{}\newpage\hbox{}\newpage\fi
  \ifnum\modulo{\value{page}}{4}=1 \hbox{}\newpage\hbox{}\newpage\fi


  \hbox{}\newpage
  \ifodd\value{page}\hbox{}\newpage\fi
  {\centering\color{rubric}\bfseries\noindent\large
    Hurlus ? Qu’est-ce.\par
    \bigskip
  }
  \noindent Des bouquinistes électroniques, pour du texte libre à participation libre,
  téléchargeable gratuitement sur \href{https://hurlus.fr}{\dotuline{hurlus.fr}}.\par
  \bigskip
  \noindent Cette brochure a été produite par des éditeurs bénévoles.
  Elle n’est pas faîte pour être possédée, mais pour être lue, et puis donnée.
  Que circule le texte !
  En page de garde, on peut ajouter une date, un lieu, un nom ; pour suivre le voyage des idées.
  \par

  Ce texte a été choisi parce qu’une personne l’a aimé,
  ou haï, elle a en tous cas pensé qu’il partipait à la formation de notre présent ;
  sans le souci de plaire, vendre, ou militer pour une cause.
  \par

  L’édition électronique est soigneuse, tant sur la technique
  que sur l’établissement du texte ; mais sans aucune prétention scolaire, au contraire.
  Le but est de s’adresser à tous, sans distinction de science ou de diplôme.
  Au plus direct ! (possible)
  \par

  Cet exemplaire en papier a été tiré sur une imprimante personnelle
   ou une photocopieuse. Tout le monde peut le faire.
  Il suffit de
  télécharger un fichier sur \href{https://hurlus.fr}{\dotuline{hurlus.fr}},
  d’imprimer, et agrafer ; puis de lire et donner.\par

  \bigskip

  \noindent PS : Les hurlus furent aussi des rebelles protestants qui cassaient les statues dans les églises catholiques. En 1566 démarra la révolte des gueux dans le pays de Lille. L’insurrection enflamma la région jusqu’à Anvers où les gueux de mer bloquèrent les bateaux espagnols.
  Ce fut une rare guerre de libération dont naquit un pays toujours libre : les Pays-Bas.
  En plat pays francophone, par contre, restèrent des bandes de huguenots, les hurlus, progressivement réprimés par la très catholique Espagne.
  Cette mémoire d’une défaite est éteinte, rallumons-la. Sortons les livres du culte universitaire, cherchons les idoles de l’époque, pour les briser.
\fi

\ifdev % autotext in dev mode
\fontname\font — \textsc{Les règles du jeu}\par
(\hyperref[utopie]{\underline{Lien}})\par
\noindent \initialiv{A}{lors là}\blindtext\par
\noindent \initialiv{À}{ la bonheur des dames}\blindtext\par
\noindent \initialiv{É}{tonnez-le}\blindtext\par
\noindent \initialiv{Q}{ualitativement}\blindtext\par
\noindent \initialiv{V}{aloriser}\blindtext\par
\Blindtext
\phantomsection
\label{utopie}
\Blinddocument
\fi
\end{document}
