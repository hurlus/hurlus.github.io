%%%%%%%%%%%%%%%%%%%%%%%%%%%%%%%%%
% LaTeX model https://hurlus.fr %
%%%%%%%%%%%%%%%%%%%%%%%%%%%%%%%%%

% Needed before document class
\RequirePackage{pdftexcmds} % needed for tests expressions
\RequirePackage{fix-cm} % correct units

% Define mode
\def\mode{a4}

\newif\ifaiv % a4
\newif\ifav % a5
\newif\ifbooklet % booklet
\newif\ifcover % cover for booklet

\ifnum \strcmp{\mode}{cover}=0
  \covertrue
\else\ifnum \strcmp{\mode}{booklet}=0
  \booklettrue
\else\ifnum \strcmp{\mode}{a5}=0
  \avtrue
\else
  \aivtrue
\fi\fi\fi

\ifbooklet % do not enclose with {}
  \documentclass[french,twoside]{book} % ,notitlepage
  \usepackage[%
    papersize={105mm, 297mm},
    inner=12mm,
    outer=12mm,
    top=20mm,
    bottom=15mm,
    marginparsep=0pt,
  ]{geometry}
  \usepackage[fontsize=9.5pt]{scrextend} % for Roboto
\else\ifav
  \documentclass[french,twoside]{book} % ,notitlepage
  \usepackage[%
    a5paper,
    inner=25mm,
    outer=15mm,
    top=15mm,
    bottom=15mm,
    marginparsep=0pt,
  ]{geometry}
  \usepackage[fontsize=12pt]{scrextend}
\else% A4 2 cols
  \documentclass[twocolumn]{report}
  \usepackage[%
    a4paper,
    inner=15mm,
    outer=10mm,
    top=25mm,
    bottom=18mm,
    marginparsep=0pt,
  ]{geometry}
  \setlength{\columnsep}{20mm}
  \usepackage[fontsize=9.5pt]{scrextend}
\fi\fi

%%%%%%%%%%%%%%
% Alignments %
%%%%%%%%%%%%%%
% before teinte macros

\setlength{\arrayrulewidth}{0.2pt}
\setlength{\columnseprule}{\arrayrulewidth} % twocol
\setlength{\parskip}{0pt} % classical para with no margin
\setlength{\parindent}{1.5em}

%%%%%%%%%%
% Colors %
%%%%%%%%%%
% before Teinte macros

\usepackage[dvipsnames]{xcolor}
\definecolor{rubric}{HTML}{800000} % the tonic 0c71c3
\def\columnseprulecolor{\color{rubric}}
\colorlet{borderline}{rubric!30!} % definecolor need exact code
\definecolor{shadecolor}{gray}{0.95}
\definecolor{bghi}{gray}{0.5}

%%%%%%%%%%%%%%%%%
% Teinte macros %
%%%%%%%%%%%%%%%%%
%%%%%%%%%%%%%%%%%%%%%%%%%%%%%%%%%%%%%%%%%%%%%%%%%%%
% <TEI> generic (LaTeX names generated by Teinte) %
%%%%%%%%%%%%%%%%%%%%%%%%%%%%%%%%%%%%%%%%%%%%%%%%%%%
% This template is inserted in a specific design
% It is XeLaTeX and otf fonts

\makeatletter % <@@@


\usepackage{blindtext} % generate text for testing
\usepackage[strict]{changepage} % for modulo 4
\usepackage{contour} % rounding words
\usepackage[nodayofweek]{datetime}
% \usepackage{DejaVuSans} % seems buggy for sffont font for symbols
\usepackage{enumitem} % <list>
\usepackage{etoolbox} % patch commands
\usepackage{fancyvrb}
\usepackage{fancyhdr}
\usepackage{float}
\usepackage{fontspec} % XeLaTeX mandatory for fonts
\usepackage{footnote} % used to capture notes in minipage (ex: quote)
\usepackage{framed} % bordering correct with footnote hack
\usepackage{graphicx}
\usepackage{lettrine} % drop caps
\usepackage{lipsum} % generate text for testing
\usepackage[framemethod=tikz,]{mdframed} % maybe used for frame with footnotes inside
\usepackage{pdftexcmds} % needed for tests expressions
\usepackage{polyglossia} % non-break space french punct, bug Warning: "Failed to patch part"
\usepackage[%
  indentfirst=false,
  vskip=1em,
  noorphanfirst=true,
  noorphanafter=true,
  leftmargin=\parindent,
  rightmargin=0pt,
]{quoting}
\usepackage{ragged2e}
\usepackage{setspace} % \setstretch for <quote>
\usepackage{tabularx} % <table>
\usepackage[explicit]{titlesec} % wear titles, !NO implicit
\usepackage{tikz} % ornaments
\usepackage{tocloft} % styling tocs
\usepackage[fit]{truncate} % used im runing titles
\usepackage{unicode-math}
\usepackage[normalem]{ulem} % breakable \uline, normalem is absolutely necessary to keep \emph
\usepackage{verse} % <l>
\usepackage{xcolor} % named colors
\usepackage{xparse} % @ifundefined
\XeTeXdefaultencoding "iso-8859-1" % bad encoding of xstring
\usepackage{xstring} % string tests
\XeTeXdefaultencoding "utf-8"
\PassOptionsToPackage{hyphens}{url} % before hyperref, which load url package

% TOTEST
% \usepackage{hypcap} % links in caption ?
% \usepackage{marginnote}
% TESTED
% \usepackage{background} % doesn’t work with xetek
% \usepackage{bookmark} % prefers the hyperref hack \phantomsection
% \usepackage[color, leftbars]{changebar} % 2 cols doc, impossible to keep bar left
% \usepackage[utf8x]{inputenc} % inputenc package ignored with utf8 based engines
% \usepackage[sfdefault,medium]{inter} % no small caps
% \usepackage{firamath} % choose firasans instead, firamath unavailable in Ubuntu 21-04
% \usepackage{flushend} % bad for last notes, supposed flush end of columns
% \usepackage[stable]{footmisc} % BAD for complex notes https://texfaq.org/FAQ-ftnsect
% \usepackage{helvet} % not for XeLaTeX
% \usepackage{multicol} % not compatible with too much packages (longtable, framed, memoir…)
% \usepackage[default,oldstyle,scale=0.95]{opensans} % no small caps
% \usepackage{sectsty} % \chapterfont OBSOLETE
% \usepackage{soul} % \ul for underline, OBSOLETE with XeTeX
% \usepackage[breakable]{tcolorbox} % text styling gone, footnote hack not kept with breakable


% Metadata inserted by a program, from the TEI source, for title page and runing heads
\title{\textbf{ Manuscrits de 1844 }}
\date{1844}
\author{Marx, Karl}
\def\elbibl{Marx, Karl. 1844. \emph{Manuscrits de 1844}}
\def\elsource{ \href{https://www.marxists.org/francais/marx/works/1844/00/km18440000/index.htm}{\dotuline{marxists.org}}\footnote{\href{https://www.marxists.org/francais/marx/works/1844/00/km18440000/index.htm}{\url{https://www.marxists.org/francais/marx/works/1844/00/km18440000/index.htm}}} }

% Default metas
\newcommand{\colorprovide}[2]{\@ifundefinedcolor{#1}{\colorlet{#1}{#2}}{}}
\colorprovide{rubric}{red}
\colorprovide{silver}{lightgray}
\@ifundefined{syms}{\newfontfamily\syms{DejaVu Sans}}{}
\newif\ifdev
\@ifundefined{elbibl}{% No meta defined, maybe dev mode
  \newcommand{\elbibl}{Titre court ?}
  \newcommand{\elbook}{Titre du livre source ?}
  \newcommand{\elabstract}{Résumé\par}
  \newcommand{\elurl}{http://oeuvres.github.io/elbook/2}
  \author{Éric Lœchien}
  \title{Un titre de test assez long pour vérifier le comportement d’une maquette}
  \date{1566}
  \devtrue
}{}
\let\eltitle\@title
\let\elauthor\@author
\let\eldate\@date


\defaultfontfeatures{
  % Mapping=tex-text, % no effect seen
  Scale=MatchLowercase,
  Ligatures={TeX,Common},
}


% generic typo commands
\newcommand{\astermono}{\medskip\centerline{\color{rubric}\large\selectfont{\syms ✻}}\medskip\par}%
\newcommand{\astertri}{\medskip\par\centerline{\color{rubric}\large\selectfont{\syms ✻\,✻\,✻}}\medskip\par}%
\newcommand{\asterism}{\bigskip\par\noindent\parbox{\linewidth}{\centering\color{rubric}\large{\syms ✻}\\{\syms ✻}\hskip 0.75em{\syms ✻}}\bigskip\par}%

% lists
\newlength{\listmod}
\setlength{\listmod}{\parindent}
\setlist{
  itemindent=!,
  listparindent=\listmod,
  labelsep=0.2\listmod,
  parsep=0pt,
  % topsep=0.2em, % default topsep is best
}
\setlist[itemize]{
  label=—,
  leftmargin=0pt,
  labelindent=1.2em,
  labelwidth=0pt,
}
\setlist[enumerate]{
  label={\bf\color{rubric}\arabic*.},
  labelindent=0.8\listmod,
  leftmargin=\listmod,
  labelwidth=0pt,
}
\newlist{listalpha}{enumerate}{1}
\setlist[listalpha]{
  label={\bf\color{rubric}\alph*.},
  leftmargin=0pt,
  labelindent=0.8\listmod,
  labelwidth=0pt,
}
\newcommand{\listhead}[1]{\hspace{-1\listmod}\emph{#1}}

\renewcommand{\hrulefill}{%
  \leavevmode\leaders\hrule height 0.2pt\hfill\kern\z@}

% General typo
\DeclareTextFontCommand{\textlarge}{\large}
\DeclareTextFontCommand{\textsmall}{\small}

% commands, inlines
\newcommand{\anchor}[1]{\Hy@raisedlink{\hypertarget{#1}{}}} % link to top of an anchor (not baseline)
\newcommand\abbr[1]{#1}
\newcommand{\autour}[1]{\tikz[baseline=(X.base)]\node [draw=rubric,thin,rectangle,inner sep=1.5pt, rounded corners=3pt] (X) {\color{rubric}#1};}
\newcommand\corr[1]{#1}
\newcommand{\ed}[1]{ {\color{silver}\sffamily\footnotesize (#1)} } % <milestone ed="1688"/>
\newcommand\expan[1]{#1}
\newcommand\foreign[1]{\emph{#1}}
\newcommand\gap[1]{#1}
\renewcommand{\LettrineFontHook}{\color{rubric}}
\newcommand{\initial}[2]{\lettrine[lines=2, loversize=0.3, lhang=0.3]{#1}{#2}}
\newcommand{\initialiv}[2]{%
  \let\oldLFH\LettrineFontHook
  % \renewcommand{\LettrineFontHook}{\color{rubric}\ttfamily}
  \IfSubStr{QJ’}{#1}{
    \lettrine[lines=4, lhang=0.2, loversize=-0.1, lraise=0.2]{\smash{#1}}{#2}
  }{\IfSubStr{É}{#1}{
    \lettrine[lines=4, lhang=0.2, loversize=-0, lraise=0]{\smash{#1}}{#2}
  }{\IfSubStr{ÀÂ}{#1}{
    \lettrine[lines=4, lhang=0.2, loversize=-0, lraise=0, slope=0.6em]{\smash{#1}}{#2}
  }{\IfSubStr{A}{#1}{
    \lettrine[lines=4, lhang=0.2, loversize=0.2, slope=0.6em]{\smash{#1}}{#2}
  }{\IfSubStr{V}{#1}{
    \lettrine[lines=4, lhang=0.2, loversize=0.2, slope=-0.5em]{\smash{#1}}{#2}
  }{
    \lettrine[lines=4, lhang=0.2, loversize=0.2]{\smash{#1}}{#2}
  }}}}}
  \let\LettrineFontHook\oldLFH
}
\newcommand{\labelchar}[1]{\textbf{\color{rubric} #1}}
\newcommand{\milestone}[1]{\autour{\footnotesize\color{rubric} #1}} % <milestone n="4"/>
\newcommand\name[1]{#1}
\newcommand\orig[1]{#1}
\newcommand\orgName[1]{#1}
\newcommand\persName[1]{#1}
\newcommand\placeName[1]{#1}
\newcommand{\pn}[1]{\IfSubStr{-—–¶}{#1}% <p n="3"/>
  {\noindent{\bfseries\color{rubric}   ¶  }}
  {{\footnotesize\autour{ #1}  }}}
\newcommand\reg{}
% \newcommand\ref{} % already defined
\newcommand\sic[1]{#1}
\newcommand\surname[1]{\textsc{#1}}
\newcommand\term[1]{\textbf{#1}}

\def\mednobreak{\ifdim\lastskip<\medskipamount
  \removelastskip\nopagebreak\medskip\fi}
\def\bignobreak{\ifdim\lastskip<\bigskipamount
  \removelastskip\nopagebreak\bigskip\fi}

% commands, blocks
\newcommand{\byline}[1]{\bigskip{\RaggedLeft{#1}\par}\bigskip}
\newcommand{\bibl}[1]{{\RaggedLeft{#1}\par\bigskip}}
\newcommand{\biblitem}[1]{{\noindent\hangindent=\parindent   #1\par}}
\newcommand{\dateline}[1]{\medskip{\RaggedLeft{#1}\par}\bigskip}
\newcommand{\labelblock}[1]{\medbreak{\noindent\color{rubric}\bfseries #1}\par\mednobreak}
\newcommand{\salute}[1]{\bigbreak{#1}\par\medbreak}
\newcommand{\signed}[1]{\bigbreak\filbreak{\raggedleft #1\par}\medskip}

% environments for blocks (some may become commands)
\newenvironment{borderbox}{}{} % framing content
\newenvironment{citbibl}{\ifvmode\hfill\fi}{\ifvmode\par\fi }
\newenvironment{docAuthor}{\ifvmode\vskip4pt\fontsize{16pt}{18pt}\selectfont\fi\itshape}{\ifvmode\par\fi }
\newenvironment{docDate}{}{\ifvmode\par\fi }
\newenvironment{docImprint}{\vskip6pt}{\ifvmode\par\fi }
\newenvironment{docTitle}{\vskip6pt\bfseries\fontsize{18pt}{22pt}\selectfont}{\par }
\newenvironment{msHead}{\vskip6pt}{\par}
\newenvironment{msItem}{\vskip6pt}{\par}
\newenvironment{titlePart}{}{\par }


% environments for block containers
\newenvironment{argument}{\itshape\parindent0pt}{\vskip1.5em}
\newenvironment{biblfree}{}{\ifvmode\par\fi }
\newenvironment{bibitemlist}[1]{%
  \list{\@biblabel{\@arabic\c@enumiv}}%
  {%
    \settowidth\labelwidth{\@biblabel{#1}}%
    \leftmargin\labelwidth
    \advance\leftmargin\labelsep
    \@openbib@code
    \usecounter{enumiv}%
    \let\p@enumiv\@empty
    \renewcommand\theenumiv{\@arabic\c@enumiv}%
  }
  \sloppy
  \clubpenalty4000
  \@clubpenalty \clubpenalty
  \widowpenalty4000%
  \sfcode`\.\@m
}%
{\def\@noitemerr
  {\@latex@warning{Empty `bibitemlist' environment}}%
\endlist}
\newenvironment{quoteblock}% may be used for ornaments
  {\begin{quoting}}
  {\end{quoting}}

% table () is preceded and finished by custom command
\newcommand{\tableopen}[1]{%
  \ifnum\strcmp{#1}{wide}=0{%
    \begin{center}
  }
  \else\ifnum\strcmp{#1}{long}=0{%
    \begin{center}
  }
  \else{%
    \begin{center}
  }
  \fi\fi
}
\newcommand{\tableclose}[1]{%
  \ifnum\strcmp{#1}{wide}=0{%
    \end{center}
  }
  \else\ifnum\strcmp{#1}{long}=0{%
    \end{center}
  }
  \else{%
    \end{center}
  }
  \fi\fi
}


% text structure
\newcommand\chapteropen{} % before chapter title
\newcommand\chaptercont{} % after title, argument, epigraph…
\newcommand\chapterclose{} % maybe useful for multicol settings
\setcounter{secnumdepth}{-2} % no counters for hierarchy titles
\setcounter{tocdepth}{5} % deep toc
\markright{\@title} % ???
\markboth{\@title}{\@author} % ???
\renewcommand\tableofcontents{\@starttoc{toc}}
% toclof format
% \renewcommand{\@tocrmarg}{0.1em} % Useless command?
% \renewcommand{\@pnumwidth}{0.5em} % {1.75em}
\renewcommand{\@cftmaketoctitle}{}
\setlength{\cftbeforesecskip}{\z@ \@plus.2\p@}
\renewcommand{\cftchapfont}{}
\renewcommand{\cftchapdotsep}{\cftdotsep}
\renewcommand{\cftchapleader}{\normalfont\cftdotfill{\cftchapdotsep}}
\renewcommand{\cftchappagefont}{\bfseries}
\setlength{\cftbeforechapskip}{0em \@plus\p@}
% \renewcommand{\cftsecfont}{\small\relax}
\renewcommand{\cftsecpagefont}{\normalfont}
% \renewcommand{\cftsubsecfont}{\small\relax}
\renewcommand{\cftsecdotsep}{\cftdotsep}
\renewcommand{\cftsecpagefont}{\normalfont}
\renewcommand{\cftsecleader}{\normalfont\cftdotfill{\cftsecdotsep}}
\setlength{\cftsecindent}{1em}
\setlength{\cftsubsecindent}{2em}
\setlength{\cftsubsubsecindent}{3em}
\setlength{\cftchapnumwidth}{1em}
\setlength{\cftsecnumwidth}{1em}
\setlength{\cftsubsecnumwidth}{1em}
\setlength{\cftsubsubsecnumwidth}{1em}

% footnotes
\newif\ifheading
\newcommand*{\fnmarkscale}{\ifheading 0.70 \else 1 \fi}
\renewcommand\footnoterule{\vspace*{0.3cm}\hrule height \arrayrulewidth width 3cm \vspace*{0.3cm}}
\setlength\footnotesep{1.5\footnotesep} % footnote separator
\renewcommand\@makefntext[1]{\parindent 1.5em \noindent \hb@xt@1.8em{\hss{\normalfont\@thefnmark . }}#1} % no superscipt in foot
\patchcmd{\@footnotetext}{\footnotesize}{\footnotesize\sffamily}{}{} % before scrextend, hyperref


%   see https://tex.stackexchange.com/a/34449/5049
\def\truncdiv#1#2{((#1-(#2-1)/2)/#2)}
\def\moduloop#1#2{(#1-\truncdiv{#1}{#2}*#2)}
\def\modulo#1#2{\number\numexpr\moduloop{#1}{#2}\relax}

% orphans and widows
\clubpenalty=9996
\widowpenalty=9999
\brokenpenalty=4991
\predisplaypenalty=10000
\postdisplaypenalty=1549
\displaywidowpenalty=1602
\hyphenpenalty=400
% Copied from Rahtz but not understood
\def\@pnumwidth{1.55em}
\def\@tocrmarg {2.55em}
\def\@dotsep{4.5}
\emergencystretch 3em
\hbadness=4000
\pretolerance=750
\tolerance=2000
\vbadness=4000
\def\Gin@extensions{.pdf,.png,.jpg,.mps,.tif}
% \renewcommand{\@cite}[1]{#1} % biblio

\usepackage{hyperref} % supposed to be the last one, :o) except for the ones to follow
\urlstyle{same} % after hyperref
\hypersetup{
  % pdftex, % no effect
  pdftitle={\elbibl},
  % pdfauthor={Your name here},
  % pdfsubject={Your subject here},
  % pdfkeywords={keyword1, keyword2},
  bookmarksnumbered=true,
  bookmarksopen=true,
  bookmarksopenlevel=1,
  pdfstartview=Fit,
  breaklinks=true, % avoid long links
  pdfpagemode=UseOutlines,    % pdf toc
  hyperfootnotes=true,
  colorlinks=false,
  pdfborder=0 0 0,
  % pdfpagelayout=TwoPageRight,
  % linktocpage=true, % NO, toc, link only on page no
}

\makeatother % /@@@>
%%%%%%%%%%%%%%
% </TEI> end %
%%%%%%%%%%%%%%


%%%%%%%%%%%%%
% footnotes %
%%%%%%%%%%%%%
\renewcommand{\thefootnote}{\bfseries\textcolor{rubric}{\arabic{footnote}}} % color for footnote marks

%%%%%%%%%
% Fonts %
%%%%%%%%%
\usepackage[]{roboto} % SmallCaps, Regular is a bit bold
% \linespread{0.90} % too compact, keep font natural
\newfontfamily\fontrun[]{Roboto Condensed Light} % condensed runing heads
\ifav
  \setmainfont[
    ItalicFont={Roboto Light Italic},
  ]{Roboto}
\else\ifbooklet
  \setmainfont[
    ItalicFont={Roboto Light Italic},
  ]{Roboto}
\else
\setmainfont[
  ItalicFont={Roboto Italic},
]{Roboto Light}
\fi\fi
\renewcommand{\LettrineFontHook}{\bfseries\color{rubric}}
% \renewenvironment{labelblock}{\begin{center}\bfseries\color{rubric}}{\end{center}}

%%%%%%%%
% MISC %
%%%%%%%%

\setdefaultlanguage[frenchpart=false]{french} % bug on part


\newenvironment{quotebar}{%
    \def\FrameCommand{{\color{rubric!10!}\vrule width 0.5em} \hspace{0.9em}}%
    \def\OuterFrameSep{\itemsep} % séparateur vertical
    \MakeFramed {\advance\hsize-\width \FrameRestore}
  }%
  {%
    \endMakeFramed
  }
\renewenvironment{quoteblock}% may be used for ornaments
  {%
    \savenotes
    \setstretch{0.9}
    \normalfont
    \begin{quotebar}
  }
  {%
    \end{quotebar}
    \spewnotes
  }


\renewcommand{\headrulewidth}{\arrayrulewidth}
\renewcommand{\headrule}{{\color{rubric}\hrule}}

% delicate tuning, image has produce line-height problems in title on 2 lines
\titleformat{name=\chapter} % command
  [display] % shape
  {\vspace{1.5em}\centering} % format
  {} % label
  {0pt} % separator between n
  {}
[{\color{rubric}\huge\textbf{#1}}\bigskip] % after code
% \titlespacing{command}{left spacing}{before spacing}{after spacing}[right]
\titlespacing*{\chapter}{0pt}{-2em}{0pt}[0pt]

\titleformat{name=\section}
  [block]{}{}{}{}
  [\vbox{\color{rubric}\large\raggedleft\textbf{#1}}]
\titlespacing{\section}{0pt}{0pt plus 4pt minus 2pt}{\baselineskip}

\titleformat{name=\subsection}
  [block]
  {}
  {} % \thesection
  {} % separator \arrayrulewidth
  {}
[\vbox{\large\textbf{#1}}]
% \titlespacing{\subsection}{0pt}{0pt plus 4pt minus 2pt}{\baselineskip}

\ifaiv
  \fancypagestyle{main}{%
    \fancyhf{}
    \setlength{\headheight}{1.5em}
    \fancyhead{} % reset head
    \fancyfoot{} % reset foot
    \fancyhead[L]{\truncate{0.45\headwidth}{\fontrun\elbibl}} % book ref
    \fancyhead[R]{\truncate{0.45\headwidth}{ \fontrun\nouppercase\leftmark}} % Chapter title
    \fancyhead[C]{\thepage}
  }
  \fancypagestyle{plain}{% apply to chapter
    \fancyhf{}% clear all header and footer fields
    \setlength{\headheight}{1.5em}
    \fancyhead[L]{\truncate{0.9\headwidth}{\fontrun\elbibl}}
    \fancyhead[R]{\thepage}
  }
\else
  \fancypagestyle{main}{%
    \fancyhf{}
    \setlength{\headheight}{1.5em}
    \fancyhead{} % reset head
    \fancyfoot{} % reset foot
    \fancyhead[RE]{\truncate{0.9\headwidth}{\fontrun\elbibl}} % book ref
    \fancyhead[LO]{\truncate{0.9\headwidth}{\fontrun\nouppercase\leftmark}} % Chapter title, \nouppercase needed
    \fancyhead[RO,LE]{\thepage}
  }
  \fancypagestyle{plain}{% apply to chapter
    \fancyhf{}% clear all header and footer fields
    \setlength{\headheight}{1.5em}
    \fancyhead[L]{\truncate{0.9\headwidth}{\fontrun\elbibl}}
    \fancyhead[R]{\thepage}
  }
\fi

\ifav % a5 only
  \titleclass{\section}{top}
\fi

\newcommand\chapo{{%
  \vspace*{-3em}
  \centering % no vskip ()
  {\Large\addfontfeature{LetterSpace=25}\bfseries{\elauthor}}\par
  \smallskip
  {\large\eldate}\par
  \bigskip
  {\Large\selectfont{\eltitle}}\par
  \bigskip
  {\color{rubric}\hline\par}
  \bigskip
  {\Large TEXTE LIBRE À PARTICPATION LIBRE\par}
  \centerline{\small\color{rubric} {hurlus.fr, tiré le \today}}\par
  \bigskip
}}

\newcommand\cover{{%
  \thispagestyle{empty}
  \centering
  {\LARGE\bfseries{\elauthor}}\par
  \bigskip
  {\Large\eldate}\par
  \bigskip
  \bigskip
  {\LARGE\selectfont{\eltitle}}\par
  \vfill\null
  {\color{rubric}\setlength{\arrayrulewidth}{2pt}\hline\par}
  \vfill\null
  {\Large TEXTE LIBRE À PARTICPATION LIBRE\par}
  \centerline{{\href{https://hurlus.fr}{\dotuline{hurlus.fr}}, tiré le \today}}\par
}}

\begin{document}
\pagestyle{empty}
\ifbooklet{
  \cover\newpage
  \thispagestyle{empty}\hbox{}\newpage
  \cover\newpage\noindent Les voyages de la brochure\par
  \bigskip
  \begin{tabularx}{\textwidth}{l|X|X}
    \textbf{Date} & \textbf{Lieu}& \textbf{Nom/pseudo} \\ \hline
    \rule{0pt}{25cm} &  &   \\
  \end{tabularx}
  \newpage
  \addtocounter{page}{-4}
}\fi

\thispagestyle{empty}
\ifaiv
  \twocolumn[\chapo]
\else
  \chapo
\fi
{\it\elabstract}
\bigskip
\makeatletter\@starttoc{toc}\makeatother % toc without new page
\bigskip

\pagestyle{main} % after style

  \frontmatter \section[{Note du traducteur}]{Note du traducteur}\renewcommand{\leftmark}{Note du traducteur}

\noindent Notre traduction a été établie d’après le texte publié en 1932 dans le 3° volume de l’édition MEGA. Ce texte présente encore des erreurs de lecture, corrigées en partie dans celui publié à Berlin pour une part dans \emph{Die Heilige Familie} (1953) et pour une part dans \emph{Kleine ökonomische Schriften} (1955). L’Institut du Marxisme-Léninisme à Moscou nous a transmis au printemps 1961 toute une série de corrections, ce pourquoi nous lui exprimons ici nos remerciements. Notre traduction repose donc sur la version allemande la plus récente. Nous avons également consulté le texte russe publié en 1956 dans le volume : MARX i ENGELS : \emph{Iz rannikh proïzvedennii}, ainsi que la traduction anglaise parue en 1959.\par
Nous avons adopté la présentation de l’édition MEGA, c’est-à-dire que nous avons indiqué en chiffres romains gras entre crochets la numérotation des pages mêmes des manuscrits. Cela permettra au lecteur de rétablir s’il le désire l’ordre de la rédaction. De même, nous avons signalé par des < > les passages barrés par Marx d’un trait au crayon\footnote{Nous avons utilisé le soulignement pour cette édition électronique (\emph{MIA})}.\par
Pour les auteurs cités, nous avons repris les traductions françaises que Marx avait lui-même lues. Parfois nous avons rétabli le texte intégral en mettant entre [] les passages non repris. Ailleurs, nous avons indiqué en note les divergences entre l’original et la citation. Nous avons aussi été amenés à présenter comme citation des passages qui ne sont pas donnés comme tels dans le texte, mais que Marx emprunte littéralement à ses lectures.\par
La traduction a posé de nombreux problèmes. Marx emploie des notions qui ne nous sont plus très familières aujourd’hui ou utilise le vocabulaire de Feuerbach ou de Hegel. De ce fait, le même terme est souvent employé dans des acceptions différentes. Nous avons donc lorsque cela s’imposait, expliqué en note les raisons de notre choix. Notre traduction voudrait être un essai pour rendre intelligible un texte souvent obscur. Cela signifie que nous avons été souvent obligés d’opter en faveur de tel ou tel sens. Nous espérons l’avoir fait en toute honnêteté et en respectant la pensée de Marx. Mais nous ne saurions prétendre à l’infaillibilité.\par

\byline{E. B.}
\mainmatter \section[{Préface}]{Préface}\renewcommand{\leftmark}{Préface}

\bigbreak
\noindent J’ai annoncé dans les \emph{Annales franco-allemandes} la critique de la science du droit et de la science politique sous la forme d’une critique de la Philosophie du Droit de Hegel \footnote{Marx fait ici allusion à son article para dans les Annales franco-allemandes : “Contribution à la critique de la Philosophie du Droit de Hegel. Introduction.”}. Tandis que j’élaborais le manuscrit pour l’impression \footnote{Il est probable que Marx pense ici à la Contribution à la Critique de la Philosophie du Droit de Hegel qu’il rédigea au cours de l’été 1843, mais qui ne fut publiée qu’en 1927.}, il apparut qu’il était tout à fait inopportun de mêler la critique qui n’avait pour objet que la philosophie spéculative \footnote{Par philosophie spéculative (il emploie aussi dans le même sens le terme “spéculation”), Marx entend la philosophie de Hegel.} à celle des diverses matières elles-mêmes, et que ce mélange entravait l’exposé et en gênait l’intelligence. En outre, la richesse et la diversité des sujets à traiter n’auraient permis de les condenser en un seul ouvrage que sous forme d’aphorismes, et un tel procédé d’exposition aurait revêtu l’apparence d’une systématisation arbitraire. C’est pourquoi je donnerai successivement, sous forme de brochures séparées, la critique du droit, de la morale, de la politique, etc., et pour terminer, je tâcherai de rétablir, dans un travail particulier, l’enchaînement de l’ensemble, le rapport des diverses parties entre elles, et je ferai pour finir la critique de la façon dont la philosophie spéculative a travaillé sur ces matériaux \footnote{Ce plan ne fut jamais réalisé, mais La Sainte Famille et L’Idéologie allemande peuvent être considérées comme autant de contributions à la critique de la philosophie de Hegel.}. C’est pourquoi il ne sera traité, dans le présent ouvrage, des liens de l’économie politique avec l’État, le droit, la morale, la vie civile, etc., que pour autant que l’économie politique touche elle-même à ces sujets ex-professo.\par
Pour le lecteur familiarisé avec l’économie politique, je n’ai pas besoin de l’assurer dès l’abord que mes résultats sont le produit d’une analyse tout à fait empirique, qui se fonde sur une étude critique consciencieuse de l’économie politique \footnote{Marx a dépouillé à Paris toute une série d’ouvrages économiques. Ses notes et extraits ont été publiés dans MEGA I, tome 3, pp. 437-583.}.\par
\footnote{Les parties rayées par Marx d’un trait vertical dans le manuscrit sont ici soulignées.}Par contre, au critique ignare qui cherche à masquer sa complète ignorance et sa pauvreté de pensée en jetant à la tête du critique positif la formule “phraséologie utopique” ou des phrases creuses comme “La critique absolument pure, absolument décisive, absolument critique”, la “société qui n’est pas seulement juridique mais sociale, totalement sociale”, la “masse massive et compacte”, les “porte-parole qui se font les interprètes de la masse massive”, il reste encore à ce critique à fournir d’abord la preuve qu’en dehors de ses affaires de famille théologiques, il a aussi son mot à dire dans les affaires séculières.\footnote{Marx parle ici de Bruno Bauer qui éditait l’Algemeine Literatur Zeitung (Charlottenburg 1844). Les formules citées sont tirées d’articles de Bauer dans le cahier 1 et le cahier 8. Ce journal et le groupe de la critique critique feront l’objet d’une polémique plus approfondie dans La Sainte Famille.}.\par
Il va de soi qu’outre les socialistes français et anglais, j’ai aussi utilisé des travaux socialistes allemands. Toutefois, les travaux allemands substantiels et originaux dans cet ordre de science se réduisent – en dehors des ouvrages de Weitling \footnote{Wilhelm Weitling, ouvrier tailleur, fut un des premiers Allemande à annoncer l’émancipation du prolétariat. Il avait publié en 1838 : L’Humanité telle qu’elle est et telle qu’elle devrait être, en 1842 Les Garanties de l’harmonie et de la liberté et en 1843, L’Évangile d’un pauvre pécheur.} – aux articles de Hess publiés dans les 21 Feuilles \footnote{Les Einundzwanzig Bogen aus der Schweiz édités à Zurich en 1843, par Georg Herwegh, contenaient trois articles de M. Hess : “Socialisme et Communisme”, “La Liberté une et entière”, “Philosophie de l’action”.} et à l’ “Esquisse d’une Critique de l’économie politique” d’Engels dans les Annales franco-allemandes \footnote{C’est le fameux article d’Engels dont on dit communément qu’il éveilla chez Marx la curiosité de l’économie politique.} dans lesquelles j’ai également ébauché d’une manière très générale les premiers éléments de la présente étude.\par
Tout autant qu’à ces auteurs, qui ont traité de manière critique d’économie politique, la critique positive en général, donc aussi la critique positive allemande de l’économie politique, doit son véritable fondement aux découvertes de Feuerbach ; contre sa Philosophie de l’Avenir \footnote{Ludwig FEUERBACH : Grundsätze der Philosophie der Zukunft, Zürich und Winterthur 1843.} et ses “Thèses pour la Réforme de la Philosophie” dans les Anekdota \footnote{Anekdota sur neuesten deutschen Philosophie und Publizistik. ZürichWinterthur 1843. Ce recueil édité par Ruge contenait tous les articles refusés par la censure à la rédaction des Annales allemandes. Parmi eux figuraient les “Vorläufige Thesen zur Reform der Philosophie” de Feuerbach, qui présentaient, sous forme d’aphorismes, les principales idées développées ensuite dans la Philosophie de l’Avenir.} – bien qu’on les utilise tacitement – l’envie mesquine des uns et la colère réelle des autres semblent avoir organisé une véritable conspiration du silence.\par
C’est seulement de Feuerbach que date la critique humaniste et naturaliste positive. Moins il est tapageur, plus l’effet des œuvres de Feuerbach est sûr, profond, ample et durable, et ce sont, depuis la Phénoménologie et la Logique \footnote{La Phénoménologie de l’Esprit avait paru en 1807, La Science de la Logique en 1812.} de Hegel, les seuls écrits où soit contenue une révolution théorique réelle.\par
Quant au dernier chapitre du présent ouvrage, l’analyse critique de la dialectique de Hegel et de sa philosophie en général, je l’ai tenu, à l’opposé des théologiens critiques \footnote{Marx fait ici allusion aux collaborateurs de Bruno Bauer à l’Ailgemeine Literatur Zeitung, qui groupait les éléments idéalistes de la gauche hégélienne.} de notre époque, pour absolument nécessaire, car ce genre de travail n’a pas été fait – ce qui est un manque de sérieux inévitable, car même critique, le théologien reste théologien ; donc, ou bien il doit partir de postulats déterminés de la philosophie comme d’une autorité, ou bien si, au cours de la critique, et du fait des découvertes d’autrui, il lui vient des doutes sur ses postulats philosophiques, il les abandonne lâchement et sans justification, il en fait abstraction, il ne manifeste plus que d’une manière négative, dénuée de conscience et sophistique son asservissement à ceux-ci et le dépit qu’il éprouve de cette sujétion.\par
[Il] ne l’exprime que d’une façon négative et dénuée de conscience, soit qu’il renouvelle constamment l’assurance de la pureté de sa propre critique, soit que, afin de détourner l’œil de l’observateur et son œil propre du nécessaire règlement de comptes de la critique avec son origine – la dialectique de Hegel et la philosophie allemande en général –, de cette nécessité pour la critique moderne de s’élever au-dessus de sa propre étroitesse et de sa nature primitive, il cherche plutôt à donner l’illusion qu’en dehors d’elle-même, la critique n’aurait plus affaire qu’à une forme bornée de la critique – disons celle du XVIII° siècle – et à l’esprit borné de la \emph{masse.} Enfin, lorsque sont faites des découvertes – comme \emph{celles de Feuerbach – sur} la nature de ses propres postulats philosophiques, ou bien le théologien critique se donne l’apparence de les avoir \emph{lui-même} réalisées, et qui plus est il le fait en lançant, sous la forme de \emph{mots d’ordre}, sans pouvoir les élaborer, les résultats de ces découvertes à la tête des écrivains encore prisonniers de la philosophie. Ou bien il sait même se donner la conscience de son élévation au-dessus de ces découvertes, non pas peut-être en s’efforçant ou en étant capable de rétablir le juste rapport entre des éléments de la \emph{dialectique} de Hegel qu’il regrette de ne pas trouver dans cette critique [de Feuerbach] ou dont on ne lui a pas encore offert la jouissance critique, mais en les mettant mystérieusement en avant, contre cette critique de la dialectique hégélienne, d’une manière déguisée, sournoise et sceptique, sous la forme particulière qui lui est propre, ainsi par exemple la catégorie de la preuve médiate contre celle de la vérité positive qui a son origine en elle-même. Le critique théologique trouve en effet tout naturel que, du côté philosophique, tout soit \emph{à faire}, pour qu’il puisse \emph{se montrer bavard} sur la pureté, sur le caractère décisif, sur toute la critique critique, et il se donne l’impression d’être le vrai \emph{triomphateur de la philosophie, s’il} a par hasard le \emph{sentiment} qu’un élément de Hegel manque chez Feuerbach, car notre critique théologique, bien qu’il pratique l’idolâtrie spiritualiste de la “\emph{Conscience de soi}” et de l’ “\emph{Esprit}”, ne dépasse pas le sentiment pour s’élever à la conscience.\par
A bien y regarder, la critique \emph{théologique –} bien qu’au début du mouvement elle ait été un véritable moment du progrès – n’est en dernière analyse rien d’autre que la pointe et la conséquence logique poussées jusqu’à leur \emph{caricature théologique} de la vieille \emph{transcendance} de la \emph{philosophie} et en particulier de \emph{Hegel.} A une autre occasion, je montrerai dans le détail cette justice intéressante de l’histoire, cette Némésis historique, qui destine maintenant la théologie, qui fut toujours le coin pourri de la philosophie, à représenter aussi en soi la décomposition négative de la philosophie – c’est-à-dire son processus de putréfaction.\par
Par contre, dans quelle mesure les découvertes de \emph{Feuerbach} sur l’essence de la philosophie rendent toujours nécessaire – tout au moins pour leur servir de \emph{preuve –} une explication critique avec la dialectique philosophique, cela ressortira de ce que je vais exposer.\par
\bigbreak
\section[{Premier manuscrit}]{Premier manuscrit\protect\footnotemark }\renewcommand{\leftmark}{Premier manuscrit}

\footnotetext{Ce premier manuscrit se compose d’une liasse de 9 feuilles in-folio (soit 36 pages) réunies par Marx en cahier et paginées en chiffres romains. Chaque page est divisée par deux traits verticaux en trois colonnes qui portent les titres : Salaire, Profit du capital, Rente foncière. Ces titres, qui se retrouvent à chaque page, laissent à penser que Marx a conçu la division de son manuscrit en trois parties à peu près égales et qu’il a titré les colonnes préalablement à la rédaction. Mais à partir de la page XXII, titres et division en colonnes perdent toute signification. Le texte est écrit à la suite et il a été intitulé conformément à son contenu : Travail aliéné. \emph{Le premier manuscrit s’interrompt à la page XXVII.}}
\subsection[{Salaire}]{Salaire}
\noindent [I] Le \emph{salaire} est déterminé par la lutte ouverte entre capitaliste et ouvrier. Nécessité de la victoire pour le capitaliste. Le capitaliste peut vivre plus longtemps sans l’ouvrier, que l’ouvrier sans le capitaliste. Union entre capitalistes habituelle et efficace, celle entre ouvriers interdite et pleine de conséquences fâcheuses pour eux. En outre, le propriétaire foncier et le capitaliste peuvent ajouter à leurs revenus des avantages industriels ; l’ouvrier ne peut ajouter à son revenu industriel ni rente foncière, ni intérêts de capitaux. C’est pourquoi la concurrence est si grande entre les ouvriers. C’est donc pour l’ouvrier seul que la séparation du capital, de la propriété foncière et du travail est une séparation nécessaire, essentielle et nuisible. Le capital et la propriété foncière peuvent ne pas rester dans les limites de cette abstraction, mais le travail de l’ouvrier ne peut en sortir.\par
\emph{Donc, pour l’ouvrier, la séparation du capital, de la rente foncière et du travail est mortelle.}\par
Le taux minimum et le seul nécessaire pour le salaire est la subsistance de l’ouvrier pendant le travail, et l’excédent nécessaire pour pouvoir nourrir une famille et pour que la race des ouvriers ne s’éteigne pas. Le salaire ordinaire est, d’après Smith, le plus bas qui soit compatible avec la \emph{simple humanité} \footnote{A. SMITH : Recherches sur la nature et les causes de la richesse des nations. Traduit par Germain Garnier, Paris 1802, tome I, p. 138. Les deux derniers mots sont en français chez Marx.}, c’est-à-dire avec une existence de bête.\par
\emph{La demande d’hommes règle nécessairement la production des hommes comme de toute autre marchandise} \footnote{Loc. cit., I, p. 162.}. Si l’offre est plus grande que la demande, une partie des ouvriers tombe dans la mendicité ou la mort par inanition. L’existence de l’ouvrier est donc réduite à la condition d’existence de toute autre marchandise. L’ouvrier est devenu une marchandise et c’est une chance pour lui quand il arrive à se placer. Et la demande, dont dépend la vie de l’ouvrier, dépend de l’humeur des riches et des capitalistes. Si [la] quantité de l’offre [dépasse] \footnote{Restitué d’après le sens. Le manuscrit est ici taché d’encre.} la demande, un des éléments consti[tuant] \footnote{Restitué d’après le sens. Le manuscrit est ici taché d’encre.} le prix (profit, rente foncière, salaire) sera payé au-dessous du \emph{prix}, [une partie de] \footnote{Restitué d’après le sens. Le manuscrit est ici taché d’encre.} ces déterminations se soustrait donc à cette utilisation et ainsi le prix du marché gravite [autour] \footnote{Restitué d’après le sens. Le manuscrit est ici taché d’encre.} de son centre, le prix naturel \footnote{Il faut noter ici que Marx adopte, comme d’ailleurs par la suite, la terminologie et les définitions des économistes dont il ne fait que résumer et commenter la pensée dans ces premiers chapitres.}. Mais 1º à un niveau élevé de la division du travail, c’est l’ouvrier pour lequel il est le plus difficile de donner une orientation différente à son travail, 2º c’est lui le premier touché par ce préjudice, étant donné son rapport de subordination au capitaliste.\par
\emph{Du fait que le prix du marché gravite autour du prix naturel, c’est donc l’ouvrier qui perd le plus et qui perd nécessairement.} Et précisément la possibilité qu’a le capitaliste de donner une autre orientation à son capital a pour conséquence ou bien de priver de pain \foreign{l’ouvrier}\footnote{\textbf{\emph{Gras-italique}} : en français dans le texte.} limité à une branche d’activité déterminée, ou de le forcer à se soumettre à toutes les exigences de ce capitaliste.\par
[II] Les fluctuations contingentes et soudaines du prix du marché affectent moins la rente foncière que. la partie du prix qui se résout en profit et en salaires, mais elles affectent moins le profit que le salaire. Pour un salaire qui monte, il y en a la plupart du temps un qui reste \emph{stationnaire} et un qui \emph{baisse.}\par
\emph{L’ouvrier ne gagne pas nécessairement lorsque le capitaliste gagne, mais il perd nécessairement avec lui.} Ainsi l’ouvrier ne gagne pas, lorsque, en vertu du secret de fabrication ou du secret commercial, en vertu des monopoles ou de la situation favorable de sa propriété, le capitaliste maintient le prix du marché au-dessus du prix naturel.\par
En outre : les \emph{prix du travail sont beaucoup plus constants que les prix des moyens de subsistance.} Souvent ils sont en rapport inverse. Dans une année de vie chère, le salaire est diminué à cause de la réduction de la demande, augmenté à cause de la hausse des moyens de subsistance. Donc compensé. En tout cas, une quantité d’ouvriers privés de pain. Dans les années de bon marché, salaire élevé par l’élévation de la demande, diminué à cause des prix des moyens de subsistance. Donc compensé.\par
Autre désavantage de l’ouvrier :\par
\emph{Les prix du travail des différentes sortes d’ouvriers sont beaucoup plus variés que les gains des diverses branches dans lesquelles le capital s’investit.} Dans le travail, toute la diversité naturelle, intellectuelle et sociale de l’activité individuelle apparaît et elle est payée différemment, tandis que le capital inerte marche toujours du même pas et est indifférent à l’activité individuelle \emph{réelle.}\par
D’une manière générale, il faut remarquer que là où l’ouvrier et le capitaliste souffrent également, l’ouvrier souffre dans son existence, le capitaliste dans le profit de son veau d’or inerte.\par
L’ouvrier n’a pas seulement à lutter pour ses moyens de subsistance physiques, il doit aussi lutter pour gagner du travail, c’est-à-dire pour la possibilité, pour les moyens de réaliser son activité.\par
Prenons les trois états principaux dans lesquels peut se trouver la société et considérons la situation de l’ouvrier en elle.\par
1º Si la richesse de la société décline, c’est l’ouvrier qui souffre le plus, car : quoique la classe ouvrière ne puisse pas gagner autant que celle des propriétaires dans l’état de prospérité de la société, \emph{aucune ne souffre aussi} cruellement de \emph{son déclin que la classe des ouvriers} \footnote{SMITH : loc. cit., tome II, p. 162.}.\par
[III] 2º Prenons maintenant une société dans laquelle la richesse progresse. Cet état est le seul favorable à l’ouvrier. Là intervient la concurrence entre les capitalistes. La demande d’ouvriers dépasse l’offre. Mais :\par
\emph{D’une part}, l’augmentation du salaire entraîne \emph{l’excès de travail} parmi les ouvriers. Plus ils veulent gagner, plus ils doivent sacrifier leur temps et, se dessaisissant entièrement de toute liberté, accomplir un travail d’esclave au service de la cupidité. Ce faisant, ils abrègent ainsi le temps qu’ils ont à vivre. Ce raccourcissement de la durée de leur vit est une circonstance favorable pour la classe ouvrière dans son ensemble, parce qu’elle rend sans cesse nécessaire un apport nouveau. Cette classe doit toujours sacrifier une partie d’elle-même pour ne pas périr dans son ensemble.\par
\emph{En outre : Quand} une société se trouve-t-elle en état d’enrichissement croissant ? Quand les capitaux et les revenue d’un pays augmentent. Mais ceci est possible seulement\par
a) si beaucoup de travail est amoncelé, car le capital est du travail accumulé ; donc si une partie toujours plus grande de ses produits est enlevée des mains de l’ouvrier, si son propre travail s’oppose à lui de plus en plus en tant que propriété d’autrui et si ses moyens. d’existence et d’activité sont de plus en plus concentrés dans la main du capitaliste.\par
b) L’accumulation du capital accroît la division du travail. La division du travail accroît le nombre des ouvriers ; inversement, le nombre des ouvriers augmente la division du travail, tout comme la division du travail augmente l’accumulation des capitaux. Du fait de cette division du travail d’une part et de l’accumulation des capitaux d’autre part, l’ouvrier dépend de plus en plus purement du travail, et d’un travail déterminé, très unilatéral, mécanique. Donc, de même qu’il est ravalé intellectuellement et physiquement au rang de machine et que d’homme il est transformé en une activité abstraite et en un ventre, de même il dépend de plus en plus de toutes les fluctuations du prix du marché, de l’utilisation des capitaux et de l’humeur des riches. L’accroissement de la classe d’hommes [IV] qui n’ont que leur travail augmente tout autant la concurrence des ouvriers, donc abaisse leur prix. C’est dans le régime des fabriques que cette situation de l’ouvrier atteint son point culminant.\par
c) Dans une société dans laquelle la prospérité augmente, seule les plus riches peuvent encore vivre de l’intérêt de l’argent. Tous les autres doivent soit investir leur capital dans une entreprise, soit le jeter dans le commerce. Par suite, la concurrence entre les capitaux s’accroît donc, la concentration des capitaux devient plus grande, les grands capitalistes ruinent les petits et une partie des anciens capitalistes tombe dans la classe des ouvriers qui, du fait de cet apport, subit pour une part une nouvelle compression du salaire et tombe dans une dépendance plus grande encore des quelques grands capitalistes ; du fait que le nombre des capitalistes a diminué, leur concurrence dans la recherche des ouvriers n’existe à peu près plus, et du fait que le nombre des ouvriers a augmenté, leur concurrence entre eux est devenue d’autant plus grande, plus contraire à la nature et plus violente. Une partie de la classe ouvrière tombe donc tout aussi nécessairement dans l’état de mendicité ou de famine, qu’une partie des capitalistes moyens tombe dans la classe ouvrière.\par
Donc, même dans l’état de la société qui est le plus favorable à l’ouvrier, la conséquence nécessaire pour celui-ci est l’excès de travail et la mort précoce, le ravalement au rang de machine, d’esclave du capital qui s’accumule dangereusement en face de lui, le renouveau de la concurrence, la mort d’inanition ou la mendicité d’une partie des ouvriers.\par
[V] La hausse du salaire excite chez l’ouvrier la soif d’enrichissement du capitaliste, mais il ne peut la satisfaire qu’en sacrifiant son esprit et son corps. La hausse du salaire suppose l’accumulation du capital et l’entraîne ; elle oppose donc, de plus en plus étrangers l’un à l’autre, le produit du travail et l’ouvrier. De même la division du travail accroît de plus en plus l’étroitesse et la dépendance de l’ouvrier, tout comme elle entraîne la concurrence non seulement des hommes, mais même des machines. Comme l’ouvrier est tombé au rang de machine, la machine peut s’opposer à lui et lui faire concurrence. Enfin, de même que l’accumulation du capital augmente la quantité de l’industrie, donc des ouvriers, la même quantité d’industrie produit, du fait de cette accumulation, \emph{une plus grande quantité d’ouvrage}, laquelle se transforme en surproduction et a pour résultat final soit de priver de leur pain une grande partie des ouvriers, soit de réduire leur salaire au minimum le plus misérable.\par
Telles sont les conséquences d’un état social qui est le plus favorable à l’ouvrier, à savoir l’état de la richesse \emph{croissante et progressive.}\par
Mais enfin cet état de croissance doit finir par atteindre son point culminant. Quelle est alors la situation de l’ouvrier ?\par

\begin{quoteblock}
 \noindent 3º Dans un pays qui aurait atteint le dernier degré possible de sa richesse, le salaire et l’intérêt du capital seraient tous deux très bas. La concurrence entre les ouvriers pour obtenir de l’occupation serait nécessairement telle que les salaires y seraient réduits à ce qui est purement suffisant pour maintenir le même nombre d’ouvriers, et le pays étant déjà pleinement peuplé, ce nombre ne pourrait jamais augmenter \footnote{SMITH : loc. cit., tome I, p. 193. Marx condense ici Adam Smith. Voici le texte intégral : “Dans un pays qui aurait atteint le dernier degré de richesse auquel la nature de son sol et de son climat et sa situation à l’égard des autres pays peuvent lui permettre d’atteindre, qui par conséquent ne pourrait parvenir au-delà, et qui n’irait pas en rétrogradant, les salaires du travail et les profits des capitaux seraient probablement très bas tous les deux. Dans un pays aussi pleinement peuplé que le comporte la proportion de gens que peut nourrir son territoire ou que peut employer son capital, la concurrence, pour obtenir de l’occupation, serait nécessairement telle que les salaires y seraient réduits à ce qui est purement suffisant pour maintenir le même nombre d’ouvriers, et le pays étant déjà pleinement peuplé, ce nombre ne pourrait jamais augmenter.”}.
 \end{quoteblock}

\noindent Le + devrait mourir.\par
Donc, dans l’état de déclin de la société, progression de la misère de l’ouvrier, dans l’état de prospérité croissante, complication de la misère, à l’état de prospérité parfaite, misère stationnaire.\par
[VI] Mais comme, d’après Smith, une société “ne peut sûrement pas être réputée dans le bonheur et la prospérité quand la très majeure partie de ses membres \footnote{Ibid., tome I, p. 160.}” souffre, que l’état le plus riche de la société entraîne cette souffrance de la majorité et que l’économie politique (la société de l’intérêt privé en général) mène à cet état de richesse extrême, le \emph{malheur} de la société est donc le but de l’économie politique.\par
Quant au rapport entre ouvrier et capitaliste, il faut encore remarquer que l’élévation du salaire est plus que compensée pour le capitaliste par la diminution de la quantité de temps de travail et que la hausse du salaire et celle de l’intérêt du capital agissent sur le prix des marchandises comme l’intérêt simple et l’intérêt composé \footnote{Ibid. tome I, p. 201.}.\par
Il nous dit qu’à l’origine, et par conception même, “le \emph{produit entier} du travail appartient à l’ouvrier” \footnote{Ibid.: tome I, p. 129}. Mais il nous dit en même temps qu’en réalité, c’est la partie la plus petite et strictement indispensable du produit qui revient à l’ouvrier ; juste ce qui est nécessaire, non pas pour qu’il existe en tant qu’homme, mais pour qu’il existe en tant qu’ouvrier ; non pas pour qu’il perpétue l’humanité, mais pour qu’il perpétue la classe esclave des ouvriers.\par
L’économiste nous dit que tout s’achète avec du travail et que le capital n’est que du travail accumulé. Mais il nous dit en même temps que l’ouvrier, loin de pouvoir tout acheter, est obligé de se vendre lui-même et de vendre sa qualité d’homme.\par
Tandis que la rente foncière de ce paresseux de propriétaire foncier s’élève la plupart du temps au tiers du produit de la terre et que le profit de l’industrieux capitaliste atteint même le double de l’intérêt de l’argent, le surplus, ce que l’ouvrier gagne au meilleur cas, comporte juste assez pour que de ses quatre enfants, deux soient condamnés à avoir faim et à mourir. [VII] Tandis que, d’après les économistes, le travail est la seule chose par laquelle l’homme augmente la valeur des produits de la nature, tandis que le travail est sa propriété active, d’après la même économie politique le propriétaire foncier et le capitaliste qui, parce que propriétaire foncier et capitaliste, ne sont que des dieux privilégiés et oisifs, sont partout supérieurs à l’ouvrier et lui prescrivent des lois.\par
Tandis que d’après les économistes, le travail est le seul prix immuable des choses, rien n’est plus contingent que le prix du travail, rien n’est soumis à de plus grandes fluctuations.\par
Tandis que la division du travail augmente la force productive du travail, la richesse et le raffinement de la société, elle appauvrit l’ouvrier jusqu’à en faire une machine. Tandis que le travail entraîne l’accumulation des capitaux et par suite la prospérité croissante de la société, il fait de plus en plus dépendre l’ouvrier du capitaliste, le place dans une concurrence accrue, le pousse dans le rythme effréné de la surproduction, à laquelle fait suite un marasme tout aussi profond.\par
Tandis que d’après les économistes, l’intérêt de l’ouvrier ne s’oppose jamais à l’intérêt de la société, la société s’oppose toujours et nécessairement à l’intérêt de l’ouvrier.\par
D’après les économistes, l’intérêt de l’ouvrier ne s’oppose jamais à celui de la société : 1º parce que l’élévation du salaire est plus que compensée par la diminution de la quantité de temps de travail, en plus des autres conséquences exposées plus haut, et 2º parce que, rapporté à la société, tout le produit brut est produit net et que le net n’a de sens que rapporté à l’individu privé.\par
Mais que le travail lui-même, non seulement dans les conditions présentes, mais en général dans la mesure où son but est le simple accroissement de la richesse, je dis que le travail lui-même soit nuisible et funeste, cela résulte, sans que l’économiste le sache, de ses propres développements.\par

\astertri

\noindent De par leurs concepts mêmes, la rente foncière et le gain capitaliste sont des retenues que subit le salaire. Mais en réalité le salaire est une retenue que la terre et le capital font tenir à l’ouvrier, une concession du produit du travail à l’ouvrier, au travail.\par
C’est dans l’état de déclin de la société que l’ouvrier souffre le plus. Il doit le poids spécifique de la pression qu’il subit à sa situation d’ouvrier, mais il doit la pression en général à la situation de la société.\par
Mais dans l’état progressif de la société, la ruine et l’appauvrissement de l’ouvrier sont le produit de son travail et de la richesse qu’il crée. Misère qui résulte donc de l’essence du travail actuel.\par
L’état le plus prospère de la société, idéal qui n’est jamais atteint qu’approximativement et qui est tout au moins le but de l’économie politique comme de la société bourgeoise, signifie la misère stationnaire pour les ouvriers.\par
Il va de soi que l’économie politique ne considère le prolétaire, c’est-à-dire celui qui, sans capital ni rente foncière, vit uniquement du travail et d’un travail unilatéral et abstrait, que comme ouvrier. Elle peut donc établir en principe que, tout comme n’importe quel cheval, il doit gagner assez pour pouvoir travailler. Elle ne le considère pas dans le temps où il ne travaille pas, en tant qu’homme, mais elle en laisse le soin à la justice criminelle, aux médecins, à la religion, aux tableaux statistiques, à la politique et au prévôt des mendiants.\par
Élevons-nous maintenant au-dessus du niveau de l’économie politique et cherchons, d’après ce qui précède et qui a été donné presque dans les termes mêmes des \foreign{économistes} \footnote{La plupart des développements qui ont précédé sont, en effet, le résumé des idées exprimées par A. Smith, quand ils n’en reprennent pas exactement les termes.}, à répondre à deux questions.\par
1º Quel sens prend dans le développement de l’humanité cette réduction de la plus grande partie des hommes au travail abstrait ?\par
2º Quelle faute commettent les \foreign{réformateurs en détail} qui, ou bien veulent élever le salaire et améliorer ainsi la situation de la classe ouvrière, ou bien considèrent comme Proudhon l’égalité du salaire comme le but de la révolution sociale \footnote{Dans son premier mémoire : Qu’est-ce \emph{que la propriété ? (Paris} 1840), Proudhon soutient que “En tant qu’associés les travailleurs sont égaux, et il implique contradiction que l’un soit payé plus que l’autre” (p. 99).} ?\par
Le travail n’apparaît, en économie politique, que sous la forme de l’activité en vue d’un gain.\par

\begin{quoteblock}
 \noindent [VIII] On peut affirmer que des occupations qui supposent des dispositions spécifiques ou une formation plus longue sont dans l’ensemble devenues d’un meilleur rapport ; tandis que le salaire relatif pour une activité mécanique uniforme à laquelle n’importe qui peut être facilement et rapidement formé, a baissé à mesure que la concurrence augmentait, et il devait nécessairement baisser. Et c’est précisément ce genre de travail qui, dans l’état d’organisation actuelle de celui-ci, est encore de loin le plus fréquent. Si donc un ouvrier de la première catégorie gagne maintenant sept fois Plus et un autre de la deuxième autant qu’il y a, disons cinquante ans, tous deux gagnent certes en moyenne quatre fois plus. Mais si, dans un pays, la première catégorie de travail occupe 1 000 ouvriers et la seconde un million d’hommes, 999 000 ne s’en trouvent pas mieux qu’il y a cinquante ans, et ils s’en trouvent plus mal si, en même temps, les prix des denrées de première nécessité ont monté. Et c’est avec ce genre de calculs de moyennes superficielles u'on veut se leurrer sur la classe la plus nombreuse de la population. En outre, la grandeur du salaire n’est qu’un facteur dans l’appréciation du revenu de l’ouvrier \footnote{Chez SCHULZ : du revenu du travail.}, car pour mesurer ce dernier,. il est encore essentiel de considérer la durée assurée de celui-ci, ce dont toutefois il ne peut absolument être question dans l’anarchie de ce qu’on appelle la libre concurrence, avec ses fluctuations et ses à-coups qui se reproduisent sans cesse. Enfin, il faut encore tenir compte du temps de travail habituel, auparavant et maintenant. Or, pour les ouvriers anglais de l’industrie cotonnière, depuis vingt-cinq ans, c’est-à-dire précisément depuis l’introduction des machines économisant le travail, celui-ci a été élevé, par la soif de gain des entrepreneurs, [IX] jusqu’à douze et seize heures par jour et l’augmentation dans un pays et dans une branche de l’industrie devait plus ou moins se faire sentir ailleurs aussi, car partout encore l’exploitation absolue des pauvres par les riches est un droit reconnu \footnote{Die Bewegung der Produktion. Eine geschichtlich-statistische Abhandlung von Wilhelm SCHULZ. Zürich und Winterthur 1843.}. (SCHULZ : Mouvement de la production, p. 65.)\par
 Mais même s’il était aussi vrai qu’il est faux que le revenu moyen de toutes les classes de la société a augmenté, les différences et les écarts relatifs du revenu peuvent cependant avoir grandi et, par suite, les contrastes de la richesse et de la pauvreté se manifester avec plus de force. Car du fait précisément que la production globale augmente et dans la mesure même où cela se produit, les besoins, les désirs et les appétits augmentent aussi et la pauvreté relative peut donc augmenter, tandis que la pauvreté absolue diminue. Le Samoyède n’est pas pauvre avec son huile de baleine et ses poissons rances, parce que, dans sa société fermée, tous ont les mêmes besoins. Mais dans un État qui va de l’avant et qui, au cours d’une dizaine d’années par exemple, a augmenté sa production totale d’un tiers par rapport à la société \footnote{Chez SCHULZ : la population.}, l’ouvrier qui gagne autant au début et à la fin des dix ans n’est pas resté aussi prospère, mais s’est appauvri d’un tiers.” (Ibid., pp. 65-66).
 \end{quoteblock}

\noindent Mais l’économie politique ne connaît l’ouvrier que comme bête de travail, comme un animal réduit aux besoins vitaux les plus stricts.\par

\begin{quoteblock}
 \noindent Pour qu’un peuple puisse se développer plus librement au point de vue intellectuel, il ne doit plus subir l’esclavage de ses besoins physiques, ne plus être le serf de son corps. Il doit donc lui rester avant tout du temps pour pouvoir créer intellectuellement et goûter es joies de l’esprit. Les progrès réalisés dans l’organisme du travail gagnent ce temps. Avec les forces motrices nouvelles et l’amélioration des machines, un seul ouvrier dans les fabriques de coton n’exécute-t-il pas souvent l’ouvrage de 100, voire de 250 à 350 ouvriers d’autrefois ? Conséquences semblables dans toutes les branches de la production, parce que les forces extérieures de la nature sont de plus en plus \footnote{Marx résume ici la phrase de Schulz : “On peut noter des résultats semblables dans toutes les branches de la production, même s’ils n’ont pas la même extension ; comme conséquences nécessaires du fait que les forces extérieures ont été de plus en plus…”} contraintes [XI à participer au travail humain. Si, pour satisfaire une certaine quantité de besoins matériels, il fallait autrefois une dépense de temps et de force humaine qui, par la suite, a été réduite de moitié, la marge de temps nécessaire à la création et à la jouissance intellectuelle a été du même coup augmentée d’autant, sans que le bien-être physique en ait souffert. \footnote{Chez Schulz, cette phrase que Marx n’a pas reprise – “Et ainsi, il nous faut reconnaître qu’avec les progrès de la production matérielle, les nations se conquièrent simultanément un monde nouveau de l’esprit.”} Mais même de la répartition du butin que nous gagnons sur le vieux Chronos lui-même dans son propre domaine, c’est encore le jeu de dés du hasard aveugle et injuste qui décide. On a calculé en France qu’au niveau actuel de la production, un temps moyen de travail de cinq heures par jour, réparti sur tous ceux qui sont aptes au travail, suffirait pour satisfaire tous les intérêts matériels de la société… Sans tenir compte des économies \footnote{Chez Schulz : “Quoi qu’il en soit de ce mouvement, il est du moins certain que, sans tenir compte…”} de temps réalisées par le perfectionnement des machines, la durée du travail d’esclave dans les fabriques n’a fait qu’augmenter pour une grande partie de la population (Ibid., pp. 67-68).\par
 Le passage du travail manuel complexe [au travail mécanique] suppose sa décomposition en ses opérations simples ; or, ce n’est au début qu’une partie des opérations revenant uniformément qui incombera aux machines, tandis que l’autre écherra aux hommes. D’après la nature même de la chose et d’après Ie résultat concordant des expériences, une telle activité continûment uniforme est aussi néfaste pour l’esprit que pour le corps ; et ainsi, dans cette union du machinisme avec la simple division du travail entre des mains plus nombreuses apparaissent nécessairement aussi tous les désavantages de cette dernière. Ces désavantages se manifestent entre autres dans l’accroissement de la mortalité des ouvriers [XI] de fabriques \footnote{Cette phrase est en réalité le début d’une note de bas de page chez Schulz. La phrase suivante est la suite du texte.}… Cette grande distinction entre la mesure dans laquelle les hommes travaillent à l’aide de machines et celle où ils travaillent en tant que machines, on n’en a pas… tenu compte \footnote{Chez SCHULZ : “on n’en a pas toujours tenu compte.”} (Ibid., p. 69).\par
 Mais pour l’avenir de la vie des peuples, les forces naturelles privées de raison qui agissent dans les machines seront nos esclaves et nos serves. (Ibid., p. 74.)\par
 Dans les filatures anglaises, on occupe seulement 158 818 hommes et 196 818 femmes. Pour 100 ouvriers dans les fabriques de coton du comté de Lancaster, il y a 103 ouvrières et, en Écosse, il y en a même 209. Dans les fabriques anglaises de chanvre de Leeds, on comptait pour 100 ouvriers hommes 147 femmes. A Druden, et sur la côte orientale de l’Écosse, on en comptait même 280. Dans les fabriques de soierie anglaises, beaucoup d’ouvrières ; dans les fabriques de lainage qui demandent une plus grande force de travail, plus d’hommes \footnote{Chez SCHULZ : “Dans les fabriques de soierie anglaises se trouvent également beaucoup d’ouvrières ; tandis que dans les fabriques de lainage, qui demandent une plus grande force physique, plus d’hommes sont employés.”}… Même dans les fabriques de coton d’Amérique du Nord, il n’y avait, en 1833, pas moins de 38 927 femmes occupées pour 18 593 hommes. Du fait des transformations survenues dans l’organisme du travail, un champ plus vaste d’activité en vue du gain est donc échu au sexe féminin… Les femmes [dans] une position économique plus indépendante… les deux sexes devenus plus proches dans leurs rapports sociaux \footnote{Chez SCHULZ : “Mais si, de ce fait, c’est en conséquence du développement progressif de l’industrie que les femmes gagnent une position économique plus indépendante, nous voyons comment en conséquence les deux sexes se rapprochent dans leurs rapports sociaux.”}. (Ibid., pp. 71-72).\par
 Dans les filatures anglaises marchant à la vapeur et à la force hydraulique travaillaient, en 1835 : 20.558 enfants entre 8 et 12 ans ; 35 867 entre 12 et 13 ans et enfin 108.208 entre 13 et 18 ans… Certes, les progrès ultérieurs de la mécanique, en enlevant de plus en plus aux hommes toutes les occupations uniformes, tendent à éliminer [XII] peu à peu cette anomalie. Mais à ces progrès assez rapides eux-mêmes s’oppose précisément encore le fait que les capitalistes peuvent s’approprier les forces des classes inférieures jusqu’à l’enfance de la manière la plus facile et à meilleur compte pour les employer à la place des auxiliaires mécaniques et pour en abuser. (Schulz : Mouv. de la production, pp. 70-71).\par
 Appel de Lord Brougham aux ouvriers : “Devenez capitalistes !” \footnote{Chez SCHULZ : “Mais dans les circonstances actuelles l’appel de Lord \emph{Brougham} aux ouvriers : “Devenez capitalistes” apparaît nécessairement comme une amère raillerie.”}… “Le mal c’est que des millions d’hommes ne peuvent gagner chichement leurs moyens de vivre que par un travail astreignant, qui les mine physiquement et qui les étiole moralement et intellectuellement ; qu’ils doivent même tenir pour une chance le malheur d’avoir trouvé un tel travail.” (Ibid., p. 60).\par
 “Pour vivre donc, les non-propriétaires sont obligés de se mettre, directement ou indirectement, au service des propriétaires, c’est-à-dire sous leur dépendance.” (PECQUEUR : Théorie nouvelle d’économie sociale, etc., p. 409) \footnote{ \noindent En français chez Marx.\par
 C. PECQUEUR :\emph{ Théorie nouvelle d’économie sociale et politique ou étude sur l’organisation des sociétés.} Paris 1842. Les citations de Pecqueur sont en français dans le texte de Marx.
}.\par
 Domestiques – gages, ouvriers – salaires \footnote{Chez PECQUEUR salaire.}, employés – traitement ou émoluments (Ibid., pp. 409-410).\par
 “Louer son travail”, “prêter son travail à l’intérêt” 3, “travailler à la place d’autrui”.\par
 “Louer la matière du travail”, “prêter la matière du travail à l’intérêt” \footnote{Chez PECQUEUR à intérêt.}, “faire travailler autrui à sa place” (Ibid., p. 411).\par
 [XIII] Cette constitution économique condamne les hommes à des métiers tellement abjects, à une dégradation tellement désolante et amère, que la sauvagerie apparaît, en comparaison, comme une royale condition (l.c., pp. 417-418). La prostitution de la chair non-propriétaire sous toutes les formes. (p. 421 sq.) Chiffonniers.
 \end{quoteblock}

\noindent \emph{Ch. Loudon} \footnote{Charles LOUDON \emph{Solution du problème de la population et de la subsistance, soumise à un médecin dans une série de lettres.} Paris 1842.}, dans son ouvrage : \emph{Solution du problème de la population}, etc. (Paris 1842), estime le nombre des prostituées en Angleterre à 60 000 ou 70 000. Le nombre des \emph{femmes d’une vertu douteuse} serait tout aussi grand. (p. \emph{228.)}\par

\begin{quoteblock}
 \noindent La moyenne de vie de ces infortunées créatures sur le pavé, après qu’elles sont entrées dans la carrière du vice, est d’environ six ou sept ans. De manière que, pour maintenir le nombre de 60 000 à 70 000 prostituées, il doit y avoir, dans les trois royaumes, au moins 8.000 à 9 000 femmes qui se vouent à cet infâme métier chaque année, ou environ 24 \footnote{Dans le manuscrit, Marx copie par erreur 80. Toute la citation est recopiée en français.} nouvelles victimes par jour, ce qui est la moyenne d’une par heure ; et conséquent, si la même proportion a lieu sur toute la du globe, il doit y avoir constamment un million et demi de ces malheureuses. (Ibid., p. 229.)\par
 La population des misérables croît avec leur misère, et… c’est à la limite extrême du dénuement que les êtres humains se pressent en plus grand nombre pour se disputer le droit de souffrir… En 1821 \footnote{A partir d’ici tout le passage cité se trouve en note chez Buret.}, la population de l’Irlande était de 6 millions 801.827. En 1831, elle s’était élevée à 7.764.010 ; c’est 14 \% d’augmentation en dix ans. Dans le Leinster, province où il y a le plus d’aisance, la population n’a augmenté que de 8 \%, tandis que, dans le Connaught, province la plus misérable, l’augmentation s’est élevée à 21 \% (Extrait des Enquêtes publiées en Angleterre sur l’Irlande, Vienne 1840). BURET : De la misère, etc., tome I, pp. [36]-37 \footnote{Eugène BURET : De la misère des classes laborieuses en Angleterre et en France. 2 vol. Paris 1840.}.\par
 L’économie politique considère le travail abstraitement comme une chose ; le travail est une marchandise ; si le prix en est élevé, c’est que la marchandise est très demandée ; si, au contraire, il est très bas, c’est qu’elle est très offerte ; comme marchandise, le travail doit de plus en plus baisser de prix ; soit la concurrence entre capitalistes et ouvriers soit la concurrence entre ouvriers y oblige \footnote{Ibid., p. 42-43. Les phrases en italique sont reproduites en français par Marx. La dernière phrase résume l’argumentation de Buret.}.\par
 … La population ouvrière, marchande de travail, est forcément réduite à la plus faible part du produit… la théorie du travail marchandise est-elle autre chose qu’une théorie de servitude déguisée ? (l.c., p. 43). Pourquoi donc n’avoir vu dans le travail qu’une valeur d’échange ? (Ibid., p. 44) Les grands ateliers achètent de préférence le travail des femmes et des enfants qui coûte moins que celui des hommes. (l.c.) Le travailleur n’est point, vis-à-vis de celui qui l’emploie, dans la position d’un libre vendeur… le capitaliste est toujours libre d’employer le travail, et l’ouvrier est toujours forcé de le vendre. La valeur du travail est complètement détruite, s’il n’est pas vendu à chaque instant. Le travail n’est susceptible, ni d’accumulation, ni même d’épargne, à la différence des véritables [marchandises]. [XIV] Le travail c’est la vie, et si la vie ne s’échange pas chaque jour contre des aliments, elle souffre et périt bientôt. Pour que la vie de l’homme soit une marchandise, il faut donc admettre l’esclavage \footnote{Cette citation est en français dans le manuscrit.}. (I.c., pp. 49-50.)\par
 Si donc le travail est une marchandise, il est une marchandise douée des propriétés les plus funestes. Mais, même d’après les principes d’économie politique, il ne l’est pas, car il n’est pas le libre résultat d’un libre marché \footnote{La phrase en français chez Marx. Chez BURET : le résultat alun libre marché.}. Le régime économique actuel abaisse à la fois et le prix et la rémunération du travail, il perfectionne. l’ouvrier et dégrade l’homme. (l.c., pp. 52-53.) L’industrie est devenue une guerre et le commerce un jeu. (l.c., p. 62.)\par
 Les machines à travailler le coton (en Angleterre) représentent à elles seules 84 millions d’artisans \footnote{Ibid., p. 193, note. Le début de la citation en français chez Marx.}.
 \end{quoteblock}

\bigbreak

\begin{quoteblock}
 \noindent L’industrie se trouvait jusqu’ici dans l’état de la guerre de conquête. Elle a prodigué la vie des hommes qui composaient son armée avec autant d’indifférence que les grands conquérants. Son but était la possession de la richesse, et non le bonheur des hommes. (BURET, I.c., p. 20.)\par
 Ces intérêts (c’est-à-dire économiques), librement abandonnée à eux-mêmes… doivent nécessairement entrer en conflit ; ils n’ont d’autre arbitre que la guerre, et les décisions de la guerre donnent aux uns la défaite et la mort, pour donner aux autres la victoire… C’est dans le conflit des forces opposées que la science cherche l’ordre et l’équilibre : la guerre perpétuelle est selon elle le seul moyen d’obtenir la paix ; cette guerre s’appelle la concurrence. (I.c., p. 23.)\par
 La guerre industrielle demande, pour être conduite avec succès, des armées nombreuses qu’elle puisse entasser sur le même point et décimer largement. Et ce n’est ni par dévouement, ni par devoir, que les soldats de cette armée supportent les fatigues qu’on leur impose ; c’est uniquement pour échapper à la dure nécessité de la faim. Ils n’ont ni affection, ni reconnaissance pour leurs chefs ; les chefs ne tiennent à leurs inférieurs par aucun sentiment de bienveillance ; ils ne les connaissent pas comme hommes, mais seulement comme des instruments de production qui doivent rapporter le plus possible \footnote{Chez BURET : beaucoup.} en dépensant le moins possible. Ces populations de travailleurs de plus en plus pressées n’ont pas même la sécurité d’être toujours employées ; l’industrie qui les a convoquées ne les fait vivre que quand elle a besoin d’elles, et, sitôt qu’elle peut s’en passer, elle les abandonne sans le moindre souci ; et les ouvriers \footnote{Ici chez BURET : mis à la réforme.}… sont forcés d’offrir leur personne et leur force pour le prix qu’on veut bien leur accorder. Plus le travail qu’on leur donne est long, pénible et fastidieux, moins ils sont rétribués ; on en voit qui, avec seize heures par jour d’efforts continus, achètent à peine le droit de ne pas mourir (l.c., pp. [68]-69).\par
 [XV] Nous avons la conviction… partagée… par les commissaires chargés de l’enquête sur la condition des tisserands à la main, que les grandes villes industrielles perdraient, en peu de temps, leur population de travailleurs, si elles ne recevaient à chaque instant, des campagnes voisines, des recrues continuelles d’hommes sains, de sang nouveau (l.c., p. 362).
 \end{quoteblock}

\subsection[{Profit du Capital}]{Profit du Capital}
\subsubsection[{1º Le Capital}]{1º Le Capital}
\noindent 1º Sur quoi repose le \emph{capital}, c’est-à-dire la propriété privée des produits du travail d’autrui ?\par

\begin{quoteblock}
 \noindent En supposant même que le capital ne soit le fruit d’aucune spoliation, il faut encore le concours de la législation pour en consacrer l’hérédité. (SAY, tome I, p. 136. Nota) \footnote{Jean-Baptiste SAY : Traité d’Économie politique, 3ᵉ édition, 2 vol. Paris 1817. Nous donnons ici le texte de J.-B. Say. Marx ajoute après “spoliation” : et de la fraude. Il traduit la fin de la phrase par “pour consacrer l’héritage”.}.
 \end{quoteblock}

\noindent Comment devient-on propriétaire de fonds productifs ? Comment devient-on propriétaire des produits qui sont créés à l’aide de ces fonds ?\par
Grâce au \emph{droit positif} (SAY, tome II, p. 4) \footnote{Voici le texte de Say résumé par Marx : “Comment est-on propriétaire de ces fonds productifs ? et par suite comment est-on propriétaire de produits qui peuvent en sortir ? Ici le droit positif est venu ajouter sa sanction an droit naturel.”}.\par
Qu’acquiert-on avec le capital, en héritant d’une grande fortune, par exemple ?\par

\begin{quoteblock}
 \noindent Celui qui acquiert une grande fortune par héritage \footnote{Chez SMITH : “Mais celui qui acquiert une grande fortune ou qui l’a par héritage…”}, n’acquiert par là nécessairement aucun pouvoir politique […] Le genre de pouvoir que cette possession lui transmet immédiatement et directement, c’est le pouvoir d’acheter ; c’est un droit de commandement sur tout le travail d’autrui ou sur tout le produit de ce travail existant alors au marché (SMITH, tome I, p. 61).
 \end{quoteblock}

\noindent Le capital est donc le pouvoir de gouverner le travail et ses produits. Le capitaliste possède ce pouvoir, non pas en raison de ses qualités personnelles ou humaines, mais dans la mesure où il est propriétaire du capital. Son pouvoir, c’est le pouvoir d’achat de son capital, auquel rien ne peut résister.\par
Nous verrons plus loin, d’abord comment le capitaliste exerce son pouvoir de gouvernement sur le travail au moyen du capital, puis le pouvoir de gouvernement du capital sur le capitaliste lui-même.\par
Qu’est-ce que le capital ?\par

\begin{quoteblock}
 \noindent Une certaine quantité de travail amassé \footnote{Souligné par Marx.} et mis en réserve (SMITH, tome II, p. 312).
 \end{quoteblock}

\noindent Le capital est du travail amassé.\par
2º \emph{Fonds, stock}.\par

\begin{quoteblock}
 \noindent signifie tout amas [quelconque] des produits de la terre ou du travail des manufactures. Il ne prend le nom de capital que lorsqu’il rapporte à son propriétaire un revenu ou. profit [quelconque] \footnote{Le mot “quelconque” entre [] figure chez Smith et n’est pas repris par Marx.} (SMITH, tome II, p. 191, note 1).
 \end{quoteblock}

\subsubsection[{2º Le profit du Capital}]{2º Le profit du Capital}

\begin{quoteblock}
 \noindent Le profit ou gain du capital est tout à fait différent du salaire. Cette différence apparaît d’une double manière. D’une part, les gains du capital “se règlent en entier sur la valeur du capital employé”, quoique le travail d’inspection et de direction puisse être le même pour des capitaux différents. A cela s’ajoute que, dans de grandes fabriques, “tout le travail de ce genre est confié à un principal commis” dont le traitement “ne garde jamais de proportion réglée avec [II] le capital dont il surveille la régie.” Quoique ici le travail du propriétaire se réduise à peu près à rien, “il n’en compte pas moins que ses profits seront en proportion réglée avec son capital” (SMITH, tome I., pp. 97-99).
 \end{quoteblock}

\noindent Pourquoi le capitaliste réclame-t-il cette proportion entre gain et capital ?\par

\begin{quoteblock}
 \noindent Il n’aurait pas d’intérêt \footnote{Souligné par Marx.} à employer ces ouvriers s’il n’attendait pas de la vente de leur ouvrage quelque chose de plus que ce qu’il fallait pour remplacer ses fonds avancés pour le salaire et il n’aurait pas d’intérêt à employer une grosse somme de fonds plutôt qu’une petite, si ses profits ne gardaient pas quelque proportion avec l’étendue des fonds employés (tome I, p. 97)
 \end{quoteblock}

\noindent Le capitaliste tire donc un gain : primo, des salaires, secundo, des matières premières avancées.\par
Or quel est le rapport du gain au capital ?\par

\begin{quoteblock}
 \noindent Nous avons déjà observé qu’il était difficile de déterminer quel est le taux moyen des salaires du travail en un lieu et dans un temps déterminés \footnote{Chez SMITH particuliers.}… Mais ceci \footnote{Chez SMITH ceci même.} ne peut guère s’obtenir à l’égard des profits de capitaux […]. Ce profit se ressent, non seulement de chaque variation qui survient dans le prix des marchandises sur lesquelles il commerce, mais encore de la bonne ou mauvaise fortune de ses rivaux et de ses pratiques, et de mille autres accidents auxquels les marchandises sont exposées, soit dans leur transfert par terre ou par mer, soit même quand on les tient en magasin. Il varie donc non seulement d’une année à l’autre, mais même d’un jour à l’autre et presque d’heure en heure (SMITH, tome I, pp. 179-180). Mais quoiqu’il soit peut-être impossible de déterminer avec quelque précision quels sont ou quels ont été les profits moyens des capitaux, […] cependant on peut s’en faire quelque idée d’après l’intérêt de l’argent \footnote{Souligné par Marx}. Partout où on pourra faire beaucoup de profits par le moyen de l’argent, on donnera communément beaucoup pour avoir la faculté de s’en servir ; et on donnera en général moins quand il n’y aura que peu de profits à faire par son moyen (SMITH, tome I, pp. [180]-181). La proportion que le taux ordinaire de l’intérêt […] doit garder avec le taux ordinaire du profit net varie nécessairement selon que le profit hausse ou baisse. Dans la Grande-Bretagne, on porte au double de l’intérêt ce que les commerçants appellent un profit honnête, modéré, raisonnable. Toutes expressions qui […] ne signifient autre chose qu’un profit commun et d’usage (SMITH, tome I, p. 198).
 \end{quoteblock}

\noindent Quel est le taux le plus bas du profit ? Quel est le plus haut ?\par

\begin{quoteblock}
 \noindent Le taux le plus bas des profits ordinaires des capitaux doit toujours être quelque chose au-delà de \footnote{ ..} ce qu’il faut, pour compenser les pertes accidentelles auxquelles est exposé chaque emploi de capital. Il n’y a que ce surplus qui constitue vraiment le profit ou le bénéfice net. Il en va de même pour le taux le plus bas de l’intérêt. (SMITH, tome I, p. 196.)\par
 [III] Le taux le plus élevé auquel puissent monter les profits ordinaires est celui qui, dans la plus grande partie des marchandises, emporte la totalité de ce qui devrait aller à la rente de la terre \footnote{Souligné par Marx.} et laisse seulement ce qui est nécessaire \footnote{Souligné par Marx.} pour salarier le travail […] ait taux le plus bas \footnote{Souligné par Marx.} auquel le travail puisse jamais être payé […]. Il faut toujours que, de manière ou d’autre, l’ouvrier ait été nourri pendant le temps que l’ouvrage l’a employé \footnote{Chez Marx : aussi longtemps qu’il est employé à un ouvrage.} ; mais il peut très bien se faire que le propriétaire de la terre n’ait pas eu de rente. Exemple : au Bengale, les gens de la Compagnie de Commerce des Indes. (SMITH, tome I, pp. 197-198.)
 \end{quoteblock}

\noindent Outre tous les avantages d’une concurrence réduite que le capitaliste est en droit d’exploiter dans ce cas, il peut d’une manière honnête maintenir le prix du marché au-dessus du prix naturel.\par
D’une part par le secret commercial.\par

\begin{quoteblock}
 \noindent Si le marché est à une grande distance de ceux qui le fournissent : notamment en tenant secrets les changements de prix, en élevant celui-ci au-dessus de l’état naturel \footnote{SMITH, I, p. 121.}. Ce secret a en effet pour résultat que d’autres capitalistes ne jettent pas également leur capital dans cette branche.\par
 Ensuite par le secret de fabrication, qui permet au capitaliste de livrer, avec des frais de production moindres, sa marchandise au même prix, ou même à des prix plus bas que ses concurrents, avec plus de profit. (La tromperie par maintien du secret n’est pas immorale. Commerce de la Bourse.) – En outre, là où la production est liée à une localité déterminée (comme par exemple un vin précieux) et où la demande effective ne peut jamais être satisfaite. Enfin par der, monopoles d’individus ou de compagnies. Le prix de monopole est aussi élevé que possible \footnote{Chez SMITH : “Le prix \emph{de monopole} est, à tous les moments, le plus haut qu’il soit possible de retirer.”}. (SMITH, tome I, pp. 120-124.)\par
 Autres causes éventuelles qui peuvent élever le profit du capital : l’acquisition de territoires nouveaux ou de nouvelles branches de commerce augmente souvent, même dans un pays riche, le profit des capitaux parce qu’elle retire aux anciennes branches commerciales une partie des capitaux, diminue la concurrence, fait approvisionner le marché avec moins de marchandises, dont les prix montent alors ; les négociants de ces branches peuvent alors payer l’argent prêté à un taux plus élevé (SMITH, tome I, p. 190) \footnote{Chez Smith : “L’acquisition d’un nouveau territoire ou de quelques nouvelles branches de commerce peut quelquefois élever les profits des capitaux, et avec eux l’intérêt de l’argent, même dans un pays qui fait des progrès rapides vers l’opulence… Une partie de ce qui était auparavant employé dans d’autres commerces en est nécessairement retirée pour être versée dans ces affaires nouvelles qui sont plus profitables ; ainsi, dans toutes ces anciennes branches de commerce, la concurrence devient moindre qu’auparavant. Le marché vient à être moins complètement fourni de plusieurs différentes sortes de marchandises. Le prix de celles-ci hausse nécessairement plus ou moins, et rend un plus gros profit à ceux qui en trafiquent ; ce qui les met dans le cas de payer un intérêt plus fort des prêts qu’on leur fait.”}.\par
 À mesure qu’une marchandise particulière vient à être plus manufacturée, cette partie du prix qui se résout en salaires et en profits devient plus grande à proportion de la partie qui se résout en rente. Dans les progrès que fait la main-d’œuvre sur cette marchandise, non seulement le nombre des profits augmente, mais chaque profit subséquent est plus grand que le précédent parce que le capital d’où [IV] il procède est nécessairement toujours plus grand. Le capital qui met en œuvre les tisserands, par exemple, est nécessairement plus grand que celui qui fait travailler les fileurs, parce que non seulement il remplace ce dernier capital avec ses profits, mais il paie encore en outre les salaires des tisserands ; et […] il faut toujours que les profits gardent une sorte de proportion avec le capital (tome I, pp. 102-103).
 \end{quoteblock}

\noindent Donc, le progrès que le travail humain fait sur le produit naturel, qu’il a transformé en produit de la nature travaillé, n’augmente pas le salaire, mais soit le nombre de capitaux qui font du profit, soit le rapport aux précédents de tout capital subséquent.\par
Nous reviendrons plus loin sur le profit que le capitaliste tire de la division du travail.\par
Il tire un double profit, premièrement de la division du travail, deuxièmement en général du progrès que le travail humain fait sur le produit naturel. Plus est grande la participation humaine à une marchandise, plus est grand le profit du capital inerte.\par

\begin{quoteblock}
 \noindent Dans une seule et même société, le taux moyen des profits du capital est beaucoup plus proche d’un même niveau que le salaire des diverses espèces de travail (tome I, p. 228) \footnote{Chez SMITH : “… dans une même société ou canton, le taux moyen des profits ordinaires dans les différents emplois de capitaux se trouvera bien plus proche du même niveau, que celui des salaires pécuniaires des diverses espèces de travail…”}. Dans les divers emplois de capitaux, le taux ordinaire du profit varie plus ou moins suivant le plus ou moins de certitude des rentrées. Le taux \footnote{Chez SMITH “Le taux ordinaire.”} du profit s’élève toujours plus ou moins avec le risque. Il ne paraît pas pourtant qu’il s’élève à proportion du risque, ou de manière à le compenser parfaitement. [Ibid. pp. 226-227).
 \end{quoteblock}

\noindent Il va de soi que les profits du capital augmentent aussi avec l’allégement ou le prix de revient moindre des moyens de circulation (par exemple l’argent-papier).
\subsubsection[{3º La domination du Capital sur le travail et les motifs d capitaliste}]{3º La domination du Capital sur le travail et les motifs d capitaliste}

\begin{quoteblock}
 \noindent Le seul motif qui détermine le possesseur d’un capital à l’employer plutôt dans l’agriculture ou dans les manufactures, ou dans quelque branche particulière de commerce en gros ou en détail, C’est le point de vue \footnote{Chez SMITH “la vue.”} de son propre profit. Il n’entre jamais dans sa pensée de calculer combien chacun de ces différents genres d’emplois mettra de travail productif \footnote{Souligné par Marx.} en activité ou [VI ajoutera de valeur au produit annuel des terres et du travail de son pays (SMITH, tome II, pp. 400-401).\par
 L’emploi de capital le plus avantageux pour le capitaliste est celui qui, à sûreté égale, lui rapporte le plus gros profit ; mais cet emploi peut ne pas être le plus avantageux pour la société. […] Tous les capitaux employés à tirer parti des forces productives de la nature sont les plus avantageusement employés (SAY, tome II, pp. 130-131).\par
 Les opérations les plus importantes du travail sont réglées et dirigées d’après les plans et les spéculations de ceux qui emploient les capitaux ; et le but qu’ils se proposent dans tous ces plans et ces spéculations, c’est le profit. Donc \footnote{Chez SMITH : Or.}, le taux du profit ne hausse point, comme la rente et les salaires, avec la prospérité de la société, et ne tombe pas, comme eux, avec sa décadence. Au contraire, ce taux est naturellement bas dans les pays riches, et haut dans les pays pauvres ; et jamais il n’est si haut que dans ceux qui se précipitent le plus rapidement vers leur ruine. L’intérêt de cette […] classe n’a donc pas la même liaison que celui des deux autres, avec l’intérêt général de la société… L’intérêt particulier de ceux qui exercent une branche particulière de commerce ou de manufacture, est toujours, à quelques égards, différent et même contraire à celui du publie. L’intérêt du marchand est toujours d’agrandir le marché et de restreindre la concurrence des vendeurs… C’est là une classe de gens dont l’intérêt ne saurait jamais être exactement le même que l’intérêt de la société, qui ont, en général, intérêt à tromper le publie et à le surcharger (SMITH, tome II, pp. 163-165).
 \end{quoteblock}

\subsubsection[{4º L’accumulation des capitaux et la concurrence entre les capitalistes}]{4º L’accumulation des capitaux et la concurrence entre les capitalistes}

\begin{quoteblock}
 \noindent L’accroissement des capitaux qui fait hausser les salaires, tend à abaisser les profits des capitalistes par la concurrence entre eux (SMITH, tome I, p. 179).\par
 Quand, par exemple, le capital nécessaire au commerce d’épicerie d’une ville se trouve partagé entre deux épiciers différents, la concurrence fera que chacun d’eux vendra à meilleur marché que si le capital eut été dans les mains d’un seul ; et s’il est divisé entre vingt [VI] la concurrence en sera précisément d’autant plus active, et il y aura aussi d’autant moins de chances qu’ils puissent se concerter entre eux pour hausser le prix de leurs marchandises (SMITH, tome II, pp. 372-373).
 \end{quoteblock}

\noindent Comme nous savons déjà que les prix de monopole sont aussi élevée que possible, que l’intérêt des capitalistes même du point de vue de l’économie politique commune est opposé à la société, que l’augmentation du profit du capital agit sur le prix de la marchandise comme l’intérêt composé (SMITH, tome I, pp. 199-201) \footnote{Chez Smith : “La hausse des salaires opère en haussant le prix d’une marchandise, comme opère l’intérêt simple dans l’accumulation d’une dette. La hausse des profits opère comme l’intérêt composé.”}, \emph{la concurrence} est le seul remède contre les capitalistes qui, d’après les données de l’économie politique, agisse d’une façon aussi bienfaisante sur l’élévation du salaire que sur le bon marché des marchandises au profit du public des consommateurs.\par
Mais la concurrence n’est possible que si les capitaux augmentent, et qui plus est en de nombreuses mains. La naissance de capitaux nombreux n’est possible que par accumulation multilatérale, étant donné que le capital en général ne naît que par accumulation, et l’accumulation multilatérale se convertit nécessairement en accumulation unilatérale. La concurrence entre les capitaux augmente l’accumulation des capitaux. L’accumulation qui, sous le régime de la propriété privée, est \emph{concentration} du capital en peu de mains, est, d’une manière générale, une conséquence nécessaire, si les capitaux sont abandonnés à leur cours naturel, et c’est seulement la concurrence qui ouvre vraiment la voie à cette destination naturelle du capital.\par
On nous a dit que le profit du capital est proportionnel à sa grandeur. Abstraction faite tout d’abord de la concurrence intentionnelle, un grand capital s’accumule donc, relativement à sa grandeur, plus vite qu’un petit capital.\par
[VIII] En conséquence, même abstraction faite de la concurrence, l’accumulation du grand capital est beaucoup plus rapide que celle du petit. Mais poursuivons-en la marche.\par
À mesure que les capitaux augmentent, du fait de la concurrence, leurs profits diminuent. Donc le petit capitaliste est le premier à souffrir.\par
L’augmentation des capitaux et un grand nombre de capitaux supposent en outre la progression de la richesse du pays.\par

\begin{quoteblock}
 \noindent Dans un pays qui est parvenu au comble de sa mesure de richesse, […] comme le taux ordinaire du profit net y sera très petit, il s’ensuivra que le taux de l’intérêt ordinaire que ce profit pourra suffire à payer, sera trop bas pour qu’il soit possible, à d’autres qu’aux gens riches, de vivre de l’intérêt de leur argent. Tous les gens de fortune bornée ou médiocre seront obligés de diriger par leurs mains l’emploi de leurs capitaux. Il faudra absolument que. tout homme à peu près soit dans les affaires ou intéressé dans quelque genre de Commerce (SMITH, tome I, pp. [196]-197).
 \end{quoteblock}

\noindent Cette situation est la situation préférée de l’économie politique.\par

\begin{quoteblock}
 \noindent C’est […] la proportion existante entre la somme des capitaux et celle des revenus qui détermine partout la proportion dans laquelle se trouveront l’industrie et la fainéantise ; partout où les capitaux l’emportent, c’est l’industrie qui domine ; partout où ce sont les revenus, la fainéantise prévaut (SMITH, tome II, p. 325).
 \end{quoteblock}

\noindent Qu’en est-il donc de l’utilisation du capital dans cette concurrence accrue ?\par

\begin{quoteblock}
 \noindent À mesure que les capitaux se multiplient la quantité des \foreign{fonds à prêter à intérêt} devient successivement plus grande. A mesure que la quantité des fonds à prêter à intérêt vient à augmenter, l’intérêt […] va nécessairement en diminuant, non seulement en vertu de ces causes générales qui font que le prix de marché de toutes choses diminue à mesure que la quantité de ces choses augmente, mais encore en vertu d’autres causes qui sont particulières à ce cas-ci. A mesure que les capitaux se multiplient dans un pays \footnote{Souligné par Marx.}, le profit qu’on peut faire en les employant diminue nécessairement ; il devient successivement de plus en plus difficile de trouver dans ce pays une manière profitable d’employer un nouveau capital. En conséquence, il s’élève une concurrence entre les différents capitaux, le possesseur d’un capital faisant tous ses efforts pour s’emparer de l’emploi qui se trouve occupe par un autre. Mais le plus souvent, il ne peut espérer débusquer de son emploi cet autre capital, sinon par des offres de traiter à de meilleures conditions. Il se trouve obligé non seulement de vendre la chose meilleur marché, mais encore, pour trouver occasion de la vendre, il est quel quelquefois aussi obligé de l’acheter plus cher. Le fonds destiné à l’entretien du travail productif grossissant de jour en jour, la demande qu’on fait de ce travail devient aussi de jour en jour plus grande : les ouvriers trouvent aisément de l’emploi, [IX] mais les possesseurs de capitaux ont de la difficulté à trouver des ouvriers à employer. La concurrence des capitalistes fait hausser les salaires du travail et fait baisser les profits (SMITH, tome II, pp. 358-359).
 \end{quoteblock}

\noindent Le petit capitaliste a donc le choix : 1º ou bien de manger son capital, puisqu’il ne peut plus vivre des intérêts, donc de cesser d’être capitaliste. Ou bien 2º d’ouvrir lui-même une affaire, de vendre sa marchandise moins cher et d’acheter plus cher que le capitaliste plus riche, et de payer un salaire élevé ; donc, comme le prix du marché est déjà très bas du fait qu’on suppose une haute concurrence, de se ruiner. Par contre, si le grand capitaliste veut débusquer le petit, il a vis-à-vis de lui tous les avantages que le capitaliste a, en tant que capitaliste, vis-à-vis de l’ouvrier. Les profits moindres sont compensés pour lui par la masse plus grande de son capital et il peut même supporter des pertes momentanées, jusqu’à ce que le capitaliste plus petit soit ruiné et qu’il se voit délivré de cette concurrence. Ainsi, il accumule à son propre profit les gains du petit capitaliste.\par
En outre : le grand capitaliste achète toujours meilleur marché que le petit, puisqu’il achète par quantités plus grandes. Il peut donc sans dommage vendre meilleur marché.\par
Mais si la chute du taux de l’argent transforme les capitalistes moyens de rentiers en homme d’affaires, inversement l’augmentation des capitaux investis dans les affaires et la diminution du profit qui en résulte ont pour conséquence la chute du taux de l’argent.\par

\begin{quoteblock}
 \noindent Du fait que le bénéfice que l’on peut tirer de l’usage d’un capital diminue, le prix que l’on peut payer pour l’usage de ce capital diminue nécessairement (SMITH, tome II, p. 359) \footnote{Chez SMITH : “Or lorsque le bénéfice qu’on peut retirer de l’usage d’un capital se trouve ainsi pour ainsi dire rogné à la fois par les deux bouts, il faut bien nécessairement que le prix qu’on peut payer pour l’usage de ce capital diminue en même temps que ce bénéfice.”}.
 \end{quoteblock}

\noindent À mesure de l’augmentation des richesses, de l’industrie et de la population, l’intérêt de l’argent, donc le profit des capitaux diminue, mais les capitaux eux mêmes n’en augmentent pas moins ; ils continuent même à augmenter bien plus vite encore qu’auparavant, [malgré la diminution des profits]… Un gros capital, quoique avec de petits profits, augmente en général plus promptement qu’un petit capital avec de gros profits. L’argent fait l’argent, dit le proverbe (tome I, p. 189).\par
Si donc à ce grand capital s’opposent maintenant de petits capitaux avec de petits profits, comme c’est le cas dans l’état de forte concurrence de notre hypothèse, il les écrase entièrement.\par
Dans cette concurrence, la baisse générale de la qualité des marchandises, la falsification, la contrefaçon, l’empoisonnement général tel qu’on le voit dans les grandes villes, sont alors les conséquences nécessaires.\par
[X] Une circonstance importante dans la concurrence des capitaux grands et petits est en outre le rapport du capital fixe \footnote{En français dans le texte. Marx adopte ici la définition du capital fixe et du capital circulant que donne A. Smith. \emph{Il} en fera plus tard la critique dans le livre II du \emph{Capital}, au chapitre X (Cf. Le Capital. Éditions Sociales, tome IV, pp. 176-198). Smith appelle capital circulant ce que Marx appellera capital de circulation. Quant au capital fixe, il serait selon Smith générateur de profit. L’économiste anglais distingue deux manières de placer son capital ; ce qui n’est pas une distinction scientifique.} au capital circulant.\par

\begin{quoteblock}
 \noindent Le capital circulant est un capital qui est utilisé pour produire des moyens de subsistance, pour la manufacture ou le commerce. Le capital employé de cette manière ne peut rendre à son maître de revenu ou de profit tant qu’il reste en sa possession ou tant qu’il continue à rester sous la même forme […]. Il sort continuellement de ses mains sous une forme, pour y rentrer sous une autre, et ce n’est qu’au moyen de cette circulation ou de ces échanges successifs qu’il peut lui rendre quelque profit. Le capital fixe se compose du capital employé à améliorer des terres ou à acheter des machines utiles et des instruments de métier ou d’autres choses semblables (SMITH, [tome II], pp. 197-198).\par
 Toute épargne dans la dépense d’entretien du capital fixe est une bonification du revenu net [de la société]. La totalité du capital de l’entrepreneur d’un ouvrage quelconque est nécessairement partagée entre son capital fixe et son capital circulant. Tant que son capital total reste le même, plus l’une des deux parts est petite, plus l’autre sera nécessairement grande. C’est le capital circulant qui fournit les matières et les salaires du travail et qui met l’industrie en activité. Ainsi toute épargne [dans la dépense d’entretien) du capital fixe, qui ne diminue pas dans le travail la puissance productive, doit augmenter le fonds (SMITH, tome II, p. 226) \footnote{Nous donnons cette citation dans les termes mêmes d’Adam Smith. Nous avons mis entre [] les parties que Marx n’a pas reprises.}.
 \end{quoteblock}

\noindent On voit, dès l’abord, que le rapport entre \foreign{capital fixe} et \foreign{capital circulant} est bien plus favorable au grand capitaliste qu’au petit. Un très grand banquier n’a besoin que d’une quantité infinie de capital fixe de plus qu’un très petit. Leur capital fixe se limite à leur bureau. Les instruments d’un grand propriétaire foncier n’augmentent pas en proportion de la grandeur de sa propriété. De même, le crédit qu’un grand capitaliste a sur un petit l’avantage de posséder est une économie d’autant plus grande de capital fixe, c’est-à-dire de l’argent qu’il doit toujours avoir prêt. Enfin il va de soi que, là où le travail industriel a atteint un haut degré de développement, où donc presque tout le travail à la main s’est transformé en travail d’usine, tout son capital ne suffit pas au petit capitaliste pour posséder seulement le \foreign{capital fixe} nécessaire. \emph{On sait que les travaux de la grande culture n’occupent habituellement qu’un petit nombre de bra}s \footnote{Cette phrase en français a été rajoutée par Marx.}.\par
En général, dans l’accumulation des grands capitaux, il se produit aussi une concentration et une simplification relatives du \foreign{capital fixe}\emph{ par} rapport aux petits capitalistes. Le grand capitaliste introduit pour lui un type [XII d’organisation des instruments du travail.\par

\begin{quoteblock}
 \noindent De même, dans le domaine de l’industrie, toute manufacture et toute fabrique est déjà l’union assez large d’une assez grande fortune matérielle avec des facultés intellectuelles et des habiletés techniques nombreuses et variées dans un but commun de production… Là où la législation maintient de vastes propriétés foncières, l’excédent d’une population croissante se presse vers les industries et c’est donc, comme en Grande-Bretagne, le champ de l’industrie sur lequel s’accumule principalement la masse la plus grande des prolétaires. Mais là où la législation autorise le partage continu de la terre, on voit, comme en France, augmenter le nombre des petits propriétaires endettés qui sont jetés, par la progression du morcellement continuel, dans la classe des indigents et des mécontents. Si enfin ce morcellement et ce surcroît de dettes sont poussés à un niveau plus élevé, la grande propriété absorbe à nouveau la petite, comme la grande industrie anéantit la petite ; et comme de grands ensembles de biens fonciers se reconstituent, la masse des ouvriers sans biens qui n’est pas strictement indispensable à la culture du sol est de nouveau poussée vers l’industrie (SCHULZ, Mouvement de la production, pp. [58]-59).\par
 La nature des marchandises de même sorte change du fait des modifications dans le mode de production et en particulier de l’utilisation des machines. Ce n’est qu’en écartant la force humaine qu’il est devenu possible de filer, à l’aide d’une livre de coton d’une valeur de 3 shillings 8 pence, 350 écheveaux d’une longueur de 167 milles anglais, c’est-à-dire 36 milles allemands, et d’une valeur commerciale de 25 guinées (Ibid., p. 62).\par
 En moyenne les prix des cotonnades ont baissé en Angleterre depuis 45 ans des 11/12ᵉ et, d’après les calculs de Marshall, la même quantité de produits fabriqués pour laquelle on payait en 1814 16 shillings est livrée maintenant pour 1 shilling 10 pence. Le bon marché plus grand des produits industriels a augmenté et la consommation à l’intérieur, et le marché à l’étranger ; et à cela est lié le fait qu’en Grande-Bretagne, non seulement le nombre des ouvriers en coton n’a pas diminué après l’introduction des machines, mais qu’il est passé de 40 000 à 1 million 1/2. [XII] En ce qui concerne maintenant le gain des entrepreneurs et ouvriers industriels, du fait de la concurrence croissante entre propriétaires de fabriques, le profit de ceux-ci a nécessairement diminué relativement à la quantité de produits qu’ils livrent. Entre 1820 et 1833, le bénéfice brut du fabricant à Manchester est tombé pour une pièce de calicot de 4 shillings 1 1/3 pence à 1 shilling 9 pence. Mais, pour recouvrer cette perte, le volume de la fabrication a été augmenté d’autant. La conséquence en est… que, dans diverses branches de l’industrie, apparaît par moments une surproduction ; qu’il se produit des banqueroutes nombreuses qui ont pour effet, à l’intérieur de la classe des capitalistes et des patrons du travail, un flottement et une fluctuation peu rassurants de la propriété, ce qui rejette dans le prolétariat une partie de ceux qui ont été économiquement ruinés ; que souvent et brutalement un arrêt ou une diminution du travail devient nécessaire, dont la classe des salariés ressent toujours amèrement le préjudice (Ibid., p. 63).\par
 Louer son travail, c’est commencer son esclavage louer la matière du travail, c’est constituer sa liberté… Le travail est l’homme \footnote{Chez PECQUEUR : c’est l’homme.}, la matière au contraire n’est rien de l’homme. (PECQUEUR : Théorie sociale, etc., pp. 411-412) \footnote{Toutes les citations de Pecqueur qui suivent sont en français dans le manuscrit.}.\par
 L’élément matière, qui ne peut rien pour la création de la richesse sans l’autre élément travail, reçoit la 'vertu magique d’être fécond pour eux comme s’ils y avaient mis, de leur propre fait, cet indispensable élément (Ibid., l.c.).\par
 En supposant que le travail quotidien d’un ouvrier lui rapporte en moyenne 400 fr. par an, et que cette somme suffise à chaque adulte pour vivre d’une vie grossière, tout propriétaire de 2 000 fr. de rente, de fermage, de loyer, etc., force donc indirectement cinq hommes à travailler pour lui ; 100 000 fr. de rente représentent le travail de deux cent cinquante hommes, et 1 000 000 le travail de 2 500 individus (donc 300 millions (Louis-Philippe) le travail de 750 000 ouvriers) \footnote{Cette parenthèse est en allemand. C’est une addition de Marx à la citation de Pecqueur.} (Ibid., pp. 412-413).\par
 Les propriétaires ont reçu de la loi des hommes le droit d’user et d’abuser, c’est-à-dire de faire ce qu’ils veulent de la matière de tout travail… ils [ne] sont nullement obligés par la loi de fournir à propos et toujours du travail aux non-propriétaires, ni de leur payer un salaire toujours suffisant, etc. (l.c., p. 413). Liberté entière quant à la nature, à la quantité, à la qualité, à l’opportunité de la production, à l’usage, à la consommation des richesses, à la disposition de la matière de tout travail. Chacun est libre d’échanger sa chose comme il l’entend, sans autre considération que son propre intérêt d’individu (l.c., p. 413).\par
 La concurrence n’exprime pas autre chose que l’échange facultatif, qui lui-même est la conséquence prochaine et logique du droit individuel d’user et d’abuser des instruments de toute production. Ces trois moments économiques, lesquels n’en font qu’un : le droit d’user et d’abuser, la liberté d’échange et la concurrence arbitraire, entraînent les conséquences suivantes : chacun produit ce lu il veut, comme il veut, quand il veut, où il veut ; produit bien ou produit mal, trop ou pas assez, trop tôt ou trop tard, trop cher ou à trop bas prix ; chacun ignore s’il vendra, à qui il vendra \footnote{Chez PECQUEUR, “à qui il vendra” vient en dernier.}, comment il vendra, quand il vendra, où il vendra ; et il en est de même quant aux achats. [XIII] Le producteur ignore les besoins et les ressources, les demandes et les offres. Il vend quand il veut, quand il peut, où il veut, à qui il veut, au prix qu’il veut. Et il achète de même. En tout cela, il est toujours le jouet du hasard, l’esclave de la loi du plus fort, du moins pressé, du plus riche… Tandis que, sur un point, il y a disette d’une richesse, sur l’autre il y a trop-plein et gaspillage. Tandis qu’un producteur vend beaucoup ou très cher, et à bénéfice énorme, l’autre ne vend rien ou vend à perte… L’offre ignore la demande, et la demande ignore l’offre. Vous produisez sur la foi d’un goût, d’une mode qui se manifeste dans le public des consommateurs ; mais déjà, lorsque vous êtes prêts à livrer la marchandise, la fantaisie a passé et s’est fixée sur un autre genre de produit… conséquences infaillibles, la permanence et l’universalisation es banqueroutes ; les mécomptes, les ruines subites et les fortunes improvisées ; les crises commerciales, les chômages, les encombrements ou les disettes périodiques ; l’instabilité et l’avilissement des salaires et des profits ; la déperdition ou le gaspillage énorme de richesses, de temps et d’efforts, dans l’arène d’une concurrence acharnée (l.c., pp. 414-416).
 \end{quoteblock}

\noindent \emph{Ricardo}, dans son livre \footnote{David RICARDO :\emph{ Des principes de l’économie politique et de l’impôt.} Traduit de l’anglais par F.-S. Constancio. 2ᵉ édition, 2 vol. Paris 1835.} (La rente foncière) : Les nations ne sont que des ateliers de production. L’homme est une machine à consommer et à produire ; la vie humaine est un capital ; les lois économiques régissent aveuglément le monde. Pour Ricardo, les hommes ne sont rien, le produit est tout. Dans le 26° chapitre \footnote{Ibid., Chapitre XXVI : Du revenu brut et du revenu net.} de la traduction française, il est dit \footnote{Marx a copié ici le texte de la traduction française.} :\par

\begin{quoteblock}
 \noindent Il serait tout à fait indifférent pour une personne qui, sur un capital de 20.000 £, ferait 2.000 £ par an de profits, que son capital employât cent hommes ou mille… L’intérêt réel d’une nation n’est-il pas le même ? Pourvu que son revenu net et réel, et que ses fermages et profits soient les mêmes, qu’importe qu’elle se compose de dix ou de douze millions d’individus ? (tome Il, pp. 194-195). En vérité, dit M. de Sismondi \footnote{J.-C.-L. SIMONDE DE SISMONDI : Nouveaux principes d’économie politique. 2 vol. Paris 1819. Le passage cité se trouve dans une note dirigée contre Ricardo ; les phrases précédant la citation sont : “Quoi donc ! la richesse est tout, les hommes ne sont absolument rien ? Quoi ! la richesse elle-même n’est quelque chose que par rapport aux impôts ?…” Tout ce paragraphe est repris de BURET, l.c., tome I, pp. 6-7.} (tome II, p. 331), il ne reste plus qu’à désirer que le roi, demeuré tout seul dans l’île, en tournant constamment une manivelle, fasse accomplir, par des automates, tout l’ouvrage de l’Angleterre.\par
 Le maître, qui achète le travail de l’ouvrier à un prix si bas qu’il suffit à peine aux besoins les plus pressants, n’est responsable ni de l’insuffisance des salaires, ni de la trop longue durée du travail : il subit lui-même la loi qu’il impose… ce n’est pas tant des hommes que vient la misère, que de la puissance des choses ([BURET], l.c., p. 82) \footnote{Toute cette citation est en français dans le texte de Marx.}.\par
 Il y a beaucoup d’endroits dans la Grande-Bretagne où les habitants n’ont pas de capitaux suffisants pour cultiver et améliorer leurs terres. La laine des provinces du midi de l’Écosse vient, en grande partie, faire un long voyage par terre sur de fort mauvaises routes pour être manufacturée dans le Comté d’York, faute de capital pour être manufacturée sur les lieux. Il y a, en Angleterre, plusieurs petites villes de fabriques, dont les habitants manquent de capitaux suffisants pour transporter le produit de leur propre industrie à ces marchés éloignés où ils trouvent des demandes et des consommateurs. Si on y voit quelques marchands, ce ne sont [XIV] proprement que les agents de marchands plus riches qui résident dans quelques-unes des grandes villes commerçantes. (Smith, tome II, pp. 381-382). Pour augmenter la valeur du produit annuel de la terre et du travail, il n’y a pas d’autres moyens que d’augmenter, quant au nombre, les ouvriers productifs, ou d’augmenter, quant à la puissance, la faculté productive des ouvriers \footnote{ \noindent Souligné par Marx.\par
 Cette dernière citation en français dans le texte de Marx.
}, précédemment employés… Dans l’un et dans l’autre cas, il faut presque toujours un surcroît de capital (SMITH, tome II, p. 338) .\par
 Puis donc que, dans la nature des choses, l’accumulation d’un capital est un préalable nécessaire à la division du travail, le travail ne peut recevoir de subdivisions ultérieures qu’à proportion que les capitaux se sont préalablement accumulés de plus en plus. A mesure que le travail vient à se subdiviser, la quantité de matières qu’un même nombre de personnes peut mettre en œuvre augmente dans une grande proportion ; et comme la tâche de chaque ouvrier se trouve successivement réduite à un plus grand degré de simplicité, il arrive qu’on invente une foule de nouvelles machines pour faciliter et abréger ces tâches. A mesure donc, que la division du travail va en s’étendant, il faut, pour qu’un même nombre d’ouvriers soit constamment occupé, qu’on accumule d’avance une égale provision de vivres et une provision de matières et d’outils plus forte que celle qui aurait été nécessaire dans un état de choses moins avancé. Or, le nombre des ouvriers augmente en général dans chaque branche d’ouvrage, en même temps qu’y augmente la division du travail, ou plutôt c’est l’augmentation de leur nombre qui les met à portée de se classer et de se subdiviser de cette manière (SMITH, tome II, pp. 193-194).\par
 De même que le travail ne peut acquérir cette grande extension de puissance productive sans une accumulation préalable des capitaux, de même l’accumulation des capitaux amène naturellement cette extension. Le capitaliste veut en effet par son capital produire la quantité la plus grande possible d’ouvrage. Il tâche donc à la fois d’établir entre ses ouvriers la distribution de travail la plus convenable et de les fournir des meilleures machines qu’il puisse imaginer ou qu’il soit à même de se procurer. Ses moyens pour réussir dans ces deux objets [XV] sont proportionnés en général à l’étendue de son capital ou au nombre de gens que ce capital peut tenir occupés. Ainsi, non seulement la quantité d’industrie augmente dans un pays à mesure de l’accroissement du capital \footnote{Souligné par Marx.} qui la met en activité, mais encore, par une suite de cet accroissement, la même quantité d’industrie produit une beaucoup plus grande quantité d’ouvrage (SMITH, l.c., pp. 194-195).
 \end{quoteblock}

\noindent Donc \emph{surproduction.}\par

\begin{quoteblock}
 \noindent Combinaisons plus vastes des forces productives… dans l’industrie et le commerce par la réunion de forces humaines et de forces naturelles plus nombreuses et plus diverses, en vue d’entreprises à plus grande échelle. Çà et là aussi… liaison déjà plus étroite des branches principales de la production entre elles. Ainsi de grands fabricants chercheront en même temps à acquérir de grandes propriétés foncières pour au moins ne pas être obligés d’acquérir d’abord de troisième main une partie des matières premières nécessaires à leur industrie ; ou bien ils mettront en liaison avec leurs entreprises industrielles un commerce, non seulement pour la vente de leurs propres produits, mais aussi pour l’achat de produits d’autre sorte et pour la vente de ceux-ci à leurs ouvriers. En Angleterre, où certains patrons de fabriques sont quelquefois à la tête de 10.000 à 12.000 ouvriers… de telles réunions de branches de production différentes sous la direction d’une seule intelligence directrice, de tels petits États ou provinces dans l’État ne sont pas rares. Ainsi récemment les propriétaires de mines de Birmingham prennent à leur compte tout le processus de fabrication du fer, qui se répartissait autrefois entre différents entrepreneurs et différents propriétaires. Cf. le district minier de Birmingham, Deutsche Viertelj [ahresschrift] 3,1838 \footnote{ \noindent Deutsche Vierteljahresschrift, Stuttgart und Tübingen 1838 (l. Jg.) Helft 3 p. 47 sq. : Der Bergmnänische Distrikt zwischen Birmingham und Wolverhampton, von A.-V. TRESKOW.\par
 Depuis “Ainsi récemment…” ce passage est en note dans le livre de SCHULZ.
}. Enfin, nous voyons, dans les grandes entreprises par actions devenues si nombreuses, de vastes combinaisons des forces financières, de nombreux participants avec les connaissances et l’expérience scientifiques et techniques d’autres personnes auxquelles est confiée l’exécution du travail. Par là, possibilité pour les capitalistes d’utiliser leurs économies d’une manière plus diverse et aussi simultanément dans la production agricole, industrielle et commerciale, ce qui élargit en même temps le cercle de leurs intérêts, [XVI] adoucit et fond ensemble les oppositions entre les intérêts de l’agriculture, de l’industrie et du commerce. Mais même cette possibilité accrue de rendre le capital producteur de la manière la plus diverse doit augmenter l’opposition entre les classes aisées et les classes sans moyens (SCHULZ, l.c., pp. 40-41).
 \end{quoteblock}

\noindent Énorme profit que les propriétaires d’immeubles tirent de la misère, Le loyer 1 est inversement proportionnel à la misère industrielle.\par
De même, tantièmes tirés des vices des prolétaires ruinés. (Prostitution, ivrognerie, \emph{prêteur sur gages}.)\par
L’accumulation des capitaux augmente et leur concurrence diminue du fait que le capital et la propriété foncière se trouvent en une seule main, et aussi parce que le capital, de par son ampleur, a la possibilité de combiner des branches de production différentes.\par
Indifférence à l’égard des hommes. Les vingt billets de la loterie de Smith \footnote{ \noindent Marx pense ici au passage suivant d’A. SMITH (I.c., tome I, p. 216)\par
 “Dans une loterie parfaitement égale, ceux qui tirent les billets gagnants doivent gagner tout ce qui est perdu par ceux qui tirent les billets blancs. Dans une profession \emph{où il} y en a vingt qui échouent contre un qui réussit, cet un doit gagner tout ce qui aurait pu être gagné par les vingt malheureux.”
}.\par
\foreign{Revenu net et brut} de Say.
\subsection[{Rente foncière}]{Rente foncière}

\begin{quoteblock}
 \noindent [I] Le droit des propriétaires fonciers tire son origine de la spoliation (SAY [l.c.] tome I, p. 136 note). Les propriétaires fonciers, comme tous les autres hommes, aiment à recueillir où ils n’ont pas semé et ils demandent une rente même pour le produit naturel de la terre (SMITH, tome I, p. 99).\par
 On pourrait se figurer que la rente foncière n’est souvent autre chose qu’un profit […] du capital que le propriétaire a employé à l’amélioration de la terre… Il y a des circonstances où la rente pourrait être regardée comme telle en partie… mais le propriétaire exige : 1º une rente même pour la terre non-améliorée, et ce qu’on pourrait supposer être intérêt ou profit des dépenses d’amélioration, n’est, en général, qu’une addition à cette rente primitive ; 2º d’ailleurs ces améliorations ne sont pas toujours faites avec les fonds du propriétaire, mais quelquefois avec ceux du fermier ; cependant, quand il s’agit de renouveler le bail, le propriétaire exige ordinairement la même augmentation de rente, que si toutes ces améliorations eussent été faites de ses propres fonds ; 3º il exige quelquefois une rente pour ce qui est tout à fait incapable d’être amélioré par la main des hommes (SMITH, tome I, pp. 300-301).
 \end{quoteblock}

\noindent Smith donne comme exemple de ce dernier cas, la salicorne, espèce de plante marine qui donne, quand elle est brûlée, un sel alkali dont on se sert pour faire du verre, du savon, etc. Elle pousse en Grande-Bretagne, particulièrement en différents lieux d’Écosse, mais seulement sur des rochers situés au-dessous de la haute marée, qui sont deux fois par jour couverts par les eaux de la mer, et dont le produit, par conséquent, n’a jamais été augmenté par l’industrie des hommes. Cependant, le propriétaire d’une terre où pousse ce genre de plante en exige une rente, tout aussi bien que de ses terres à blé. Dans le voisinage des îles de Shetland, la mer est extraordinairement abondante en poisson… Une grande partie des habitants [II] vivent de la pêche.\par

\begin{quoteblock}
 \noindent Mais pour tirer parti du produit de la mer, il faut avoir une habitation sur la terre voisine. La rente du propriétaire est en proportion non de ce lue le fermier peut faire avec la terre, mais de ce qu’il peut faire avec la terre et la mer ensemble (SMITH, tome I, pp. 301-302).\par
 On peut considérer cette rente comme le produit de cette puissance de la nature, dont le propriétaire prête l’usage au fermier. Ce produit est plus ou moins grand selon qu’on suppose à cette puissance plus ou moins d’étendue, ou, en d’autres termes, selon qu’on suppose à la terre plus ou moins de fertilité naturelle ou artificielle. C’est l’œuvre de la nature qui reste après qu’on a fait la déduction ou la balance de tout ce qu’on peut regarder comme l’œuvre de l’homme (SMITH, tome Il, pp. 377-378).\par
 \foreign{La rente de la terre}, considérée comme le prix payé pour l’usage de la terre, est donc naturellement un \foreign{prix de monopole}. Elle n’est nullement en proportion de ce que le propriétaire peut avoir placé sur sa terre en améliorations, ou de ce qu’il lui suffirait de prendre pour ne pas perdre, mais bien de ce que le fermier peut suffire à donner sans perdre (SMITH, tome I, p. 302).\par
 Des trois classes primitives \footnote{ \noindent Souligné par Marx.\par
 Ce mot, qui résume une phrase précédente, est une addition de Marx. Il avait d’ailleurs écrit par inadvertance dans son manuscrit : productives.
}, c’est la seule (les propriétaires de terre) à laquelle son revenu ne coûte ni travail, ni souci, mais à laquelle il vient pour ainsi dire de lui-même, et sans qu’elle y apporte aucun dessein \footnote{Dans le manuscrit, Marx écrit “Einsicht” (jugement) pour “Absicht” (dessein).}, ni plan quelconque (SMITH, tome II, p. 161).
 \end{quoteblock}

\noindent On nous a déjà dit que la quantité de la rente foncière dépend de la \emph{fertilité} proportionnelle du sol.\par
Un autre facteur de sa détermination est la \emph{situation.}\par

\begin{quoteblock}
 \noindent La rente varie selon la \foreign{fertilité} de la terre quel que soit son produit et selon sa \foreign{situation}, quelle que soit sa fertilité (SMITH, tome I, p. 306).\par
 En supposant des terres, des mines et des pêcheries d’une égale fécondité, le produit qu’elles rendront sera en proportion de l’étendue des capitaux qu’on emploiera à leur culture et exploitation, et de la manière plus [MI ou moins convenable dont ces capitaux seront appliqués. En supposant des capitaux égaux et également bien appliqués, ce produit sera en proportion de la fécondité naturelle des terres, des mines et des pêcheries ([SMITH], tome II, p. 210).
 \end{quoteblock}

\noindent Ces phrases de Smith sont importantes parce que, à frais de production et à étendue égaux, elles réduisent la rente foncière à la fertilité plus ou moins grande de la terre. Elles montrent donc nettement le renversement des notions en économie politique, laquelle transforme la fertilité de la terre en une qualité du propriétaire foncier.\par
Mais considérons maintenant la rente foncière sous la forme qu’elle prend dans le commerce réel des hommes.\par
La rente foncière est fixée par \emph{la lutte entre fermier et propriétaire foncier.} Partout, en économie, nous trouvons l’opposition ouverte des intérêts, la lutte, la guerre, reconnues comme le fondement de l’organisation sociale.\par
Voyons maintenant quels sont les rapports de propriétaires à fermiers.\par

\begin{quoteblock}
 \noindent Le propriétaire, lors de la stipulation des clauses du bail, tâche, autant qu’il peut, de ne pas laisser [au fermier] dans le produit une portion plus forte que ce qu’il faut pour remplacer le capital qui fournit la semence, paie le travail, achète et entretient les bestiaux et autres instruments de labourage, et pour lui donner en outre les profits ordinaires que rendent les autres fermes dans le canton. Cette portion est évidemment la plus petite dont le fermier puisse se contenter sans être en perte et le propriétaire est rarement d’avis de lui en laisser davantage. Tout ce qui reste du produit ou de son prix […] au-delà de cette portion, quel que puisse être ce reste, le propriétaire tâche de se le réserver comme rente de sa terre ; ce qui est évidemment la plus forte rente que le fermier puisse suffire à payer, dans l’état actuel [IV] de la terre […]. Ce surplus peut toujours être regardé comme la rente naturelle de la terre ou la rente moyennant laquelle on peut naturellement penser que sont louées la plupart des terres. (SMITH, tome I, pp. 299-300).\par
 Les propriétaires. terriens, dit Say, exercent une espèce de monopole envers les fermiers. La demande de leur denrée, qui est le terrain, peut s’étendre sans cesse ; mais la quantité de leur denrée ne s’étend que jusqu’à un certain point… Le marché qui se conclut entre le propriétaire et le fermier, est toujours aussi avantageux qu’il peut l’être pour le premier… Outre cet avantage que le propriétaire tient de la nature des choses, il en tire un autre de sa position, qui d’ordinaire lui donne sur le fermier l’ascendant d’une fortune plus grande, et quelquefois celui du crédit et des places ; mais le premier de ces avantages suffit pour qu’il soit toujours à même de profiter seul des circonstances favorables au profit de la terre. L’ouverture d’un canal, un chemin, les progrès de la population et dé l’aisance d’un canton élèvent toujours le prix des fermages… Le fermier lui-même peut certes \footnote{Ce mot est une addition de Marx.} améliorer le fonds à ses frais ; mais c’est un capital dont il ne tire les intérêts que pendant la durée de son bail, et qui, à l’expiration de ce bail, ne pouvant être emporté \footnote{Ces quatre derniers mots ne figurent pas dans le manuscrit de Marx.}, demeure au propriétaire ; dès ce moment, celui-ci en retire les intérêts sans en avoir fait les avances, car le loyer s’élève en proportion (SAY, tome II, pp. 142-143).\par
 La rente, considérée comme le prix payé pour l’usage de la terre, est naturellement le prix le plus haut que le fermier soit en état de payer, dans les circonstances où se trouve la terre pour – le moment (SMITH, tome I, p. 299).\par
 La rente d’un bien à la surface de la terre, monte communément à ce qu’on suppose être le tiers du produit total, et c’est pour l’ordinaire une rente fixe et indépendante des variations accidentelles [V] de la récolte (SMITH, tome I, p. 151). C’est rarement moins du quart […] du produit total (Ibid., tome Il, p. 378) \footnote{Chez SMITH : “C’est rarement moins. du quart et souvent plus du tiers du produit total.”}.
 \end{quoteblock}

\noindent \emph{La rente foncière} ne peut pas être payée pour toutes les marchandises. Par exemple, dans beaucoup de régions, on ne paie pas de rente foncière pour les pierres.\par

\begin{quoteblock}
 \noindent On ne peut porter ordinairement au marché que ces parties seulement du produit de la terre dont le prix ordinaire est suffisant pour remplacer le capital qu’il faut employer pour les y porter, et les profits ordinaires de ce capital. Si le prix ordinaire est plus que suffisant, le surplus en ira naturellement à la rente de la terre. S’il n’est juste que suffisant, la marchandise pourra bien être portée au marché, mais elle ne peut fournir à payer une rente au propriétaire. Le prix sera-t-il ou ne sera-t-il pas plus que suffisant ? C’est ce qui dépend de la demande (SMITH, tome I, pp. 302-303).
 \end{quoteblock}

\noindent La rente entre \footnote{Souligné par Marx.} dans la composition du prix des marchandise, d’une autre manière que n’y entrent les salaires et les profits. Le taux haut ou bas des salaires et des profits est la cause du haut ou bas prix des marchandises : le taux haut ou bas de la rente est l’effet du prix (SMITH, tome I, p. 303).\par
Parmi les produits qui toujours rapportent une rente foncière, on compte la nourriture.\par

\begin{quoteblock}
 \noindent Les hommes, comme toutes les autres espèces animales se multipliant naturellement en proportion des moyens de leur subsistance, il y a toujours plus ou moins demande de nourriture. Toujours la nourriture pourra acheter […] [VI] une quantité plus ou moins grande de travail et toujours il se trouvera quelqu’un disposé à faire quelque chose pour la gagner. A la vérité, ce qu’elle peut acheter de travail n’est pas toujours égal \footnote{Les passages entre crochets n’ont pas été repris par Marx.} à ce qu’elle pourrait en faire subsister, si elle était distribuée de la manière la plus économique, et cela à cause des forts salaires qui sont quelquefois donnés au travail. Mais elle peut toujours acheter autant de travail qu’elle peut en faire subsister, au taux auquel ce genre de travail subsiste communément dans le pays. Or la terre, dans presque toutes les situations possibles, produit plus de nourriture que ce qu’il faut pour faire subsister tout le travail qui concourt à mettre cette nourriture au marché […] Te surplus de cette nourriture est aussi toujours plus que suffisant pour remplacer avec profit le capital qui fait mouvoir ce travail. Ainsi il reste toujours quelque chose pour donner une rente au propriétaire (SMITH, tome I, pp. 305-306). Non seulement c’est de la nourriture que la rente tire sa première origine, mais encore si quelqu’autre partie du produit de la terre vient aussi par. la suite à rapporter une rente, elle doit cette addition de valeur à l’accroissement de puissance qu’a acquis le travail pour produire la nourriture, au moyen de la culture et de l’amélioration de la terre (SMITH, tome I, p. 345). La nourriture de l’homme [parait être le seul des produits de la terre qui] fournisse toujours [et nécessairement] de quoi payer une rente quelconque au propriétaire  (tome I, p. 337). Les pays ne peuplent pas en proportion du nombre que leur produit peut vêtir et loger, mais en raison de celui que ce produit peut nourrir (SMITH, tome I, p. 342).\par
 Les deux plus grands besoins de l’homme après la nourriture sont le vêtement, le logement, le chauffage. Ils rapportent la plupart du temps une rente foncière, mais pas toujours obligatoirement (Ibid., tome I, pp. 337-338) \footnote{Voici les termes de Smith : “Les deux plus grands besoins de l’homme après la nourriture sont le vêtement et le logement. Ils peuvent quelquefois en rapporter une et quelquefois ne le peuvent pas, selon les circonstances”.}.
 \end{quoteblock}

\noindent [VIII] Voyons maintenant comment le propriétaire foncier exploite tous les avantages de la société.\par
1º La rente foncière augmente avec la population (SMITH, tome I, p. 335).\par
2º Say nous a déjà dit comment la rente foncière augmente avec les chemins de fer, etc., avec l’amélioration de la sécurité et la multiplication des moyens de communications.\par
3º\par

\begin{quoteblock}
 \noindent Toute amélioration qui se fait dans l’état de la société, tend, d’une manière directe ou indirecte \footnote{Souligné par Marx.}, à faire monter la rente réelle de la terre, à augmenter la richesse réelle du propriétaire, c’est-à-dire son pouvoir d’acheter le travail d’autrui ou le produit du travail d’autrui… L’extension de l’amélioration des terres et de la culture y tend d’une manière directe. La part du propriétaire dans le produit augmente nécessairement a mesure que le produit augmente. La hausse qui survient dans le prix réel de ces sortes de produits bruts, […] la hausse, par exemple, du prix du bétail tend aussi à élever, d’une manière directe, la rente du propriétaire et dans une proportion encore plus forte. Non seulement la valeur réelle de la part du propriétaire, le pouvoir réel que cette part lui donne sur le travail d’autrui, augmentent avec la valeur réelle du produit, mais encore la proportion de cette part, relativement au produit total, augmente aussi avec cette valeur. Ce produit, après avoir haussé dans son prix réel, n’exige pas plus de travail pour être recueilli […] et pour suffire à remplacer le capital qui fait mouvoir ce travail, avec les profits ordinaires de ce capital. La portion restante du produit, tu est la part du propriétaire, sera donc le, relativement au tout, qu’elle ne l’était auparavant (SMITH, tome Il, pp. 157-159).
 \end{quoteblock}

\noindent [IX] L’accroissement de la demande de produits bruts et par conséquent l’élévation de la valeur peut résulter, en partie, de l’augmentation de la population et de l’augmentation de ses besoins. Mais toute invention nouvelle, toute utilisation nouvelle que fait la manufacture d’une matière première qu’on n’avait pas encore ou peu utilisée auparavant, augmente la rente foncière. Ainsi, par exemple, la rente des mines de charbon a monté énormément avec les chemins de fer, les bateaux à vapeur, etc.\par
Outre cet avantage que le propriétaire foncier tire de la manufacture, des inventions, du travail, nous en verrons immédiatement un autre encore.\par
4º\par

\begin{quoteblock}
 \noindent Ces sortes d’améliorations dans la puissance productive du travail, qui tendent directement à réduire le prix réel des ouvrages de manufacture, tendent indirectement à élever la rente réelle de la terre. C’est contre du produit manufacturé que le propriétaire échange cette partie de – son produit brut, qui excède sa consommation personnelle, ou […] le prix de cette partie. Tout ce qui réduit le prix réel de ce premier genre de produit, élève le prix réel du second ; une même quantité de ce produit brut répond dès lors à une plus grande quantité de ce produit manufacturé, et le propriétaire se trouve à portée d’acheter une plus grande quantité des choses de commodité, d’ornement ou de luxe qu’il désire se procurer (SMITH, tome II, p. 159).
 \end{quoteblock}

\noindent Mais, si du fait que le propriétaire foncier exploite tous les avantages de la société, Smith [XI conclut (tome II, p. 161) que l’intérêt du propriétaire est toujours identique à celui de la société, c’est une stupidité. En économie politique, sous le régime de la propriété privée, l’intérêt que quelqu’un peut porter à la société est en proportion exactement inverse de l’intérêt que la société peut lui porter, de même que l’intérêt que l’usurier porte au dissipateur n’est absolument pas identique à l’intérêt de ce dernier.\par
Nous ne mentionnerons qu’en passant la soif de monopole du propriétaire foncier à l’égard de la propriété foncière des pays étrangers, dont datent par exemple les lois sur les blés \footnote{Marx fait ici allusion aux lois anglaises sur le blé de 1815. Il écrira plus tard dans Le Capital, livre III (tome VIII, p. 18) : “Elles instituaient une taxe sur le pain, qui, de l’aveu des législateurs, fut imposée au pays pour assurer aux propriétaires fonciers oisifs la pérennité de leurs rentes qui s’étaient anormalement accrues pendant les guerres contre les Jacobins.”}. De même, nous passerons ici sous silence le servage moyenâgeux, l’esclavage aux colonies, la misère des journaliers à la campagne en Grande-Bretagne. Tenons-nous en aux thèses de l’économie politique elle-même.\par
1º Dire que le propriétaire foncier est intéressé au bien de la société, c’est dire, d’après les principes de l’économie, qu’il est intéressé à la progression de sa population, de sa production artistique, à l’augmentation de ses besoins, en un mot à la croissance de la richesse ; et d’après ce que nous avons vu jusqu’ici, cette croissance va de pair avec la croissance de la misère et de l’esclavage. La liaison entre l’accroissement du loyer et celui de la misère est un exemple de l’intérêt que le propriétaire foncier porte à la société, car avec le loyer, la rente foncière, l’intérêt du sol sur lequel est bâtie la maison augmente.\par
2º D’après les économistes eux-mêmes, l’intérêt du propriétaire foncier est le contraire direct de celui du fermier ; donc déjà d’une partie importante de la société.\par
[XI] 3º Comme le propriétaire foncier peut exiger d’autant plus de rente [du] fermier que le fermier paie moins de salaire et comme le fermier rabaisse d’autant plus le salaire que le propriétaire exige plus de rente foncière, l’intérêt du propriétaire est tout aussi opposé à l’intérêt des travailleurs agricoles que celui des patrons de manufactures l’est à celui de leurs ouvriers. Il rabaisse également le salaire a un minimum.\par
4º Comme la baisse réelle du prix des produits manufacturés élève la rente de la terre, le propriétaire foncier a un intérêt direct à l’abaissement du salaire des ouvriers de manufacture, à la concurrence entre capitalistes, à la surproduction, à toute la misère qu’engendre la manufacture.\par
5º Si donc l’intérêt du propriétaire foncier, bien loin d’être identique à l’intérêt de la société, est le contraire direct de l’intérêt des fermiers, des travailleurs agricoles, des ouvriers des manufactures et des capitalistes, l’intérêt d’un propriétaire n’est même pas identique à celui de l’autre du fait de la concurrence que nous allons maintenant considérer.\par
Déjà, d’une manière générale, la grande propriété foncière est à la petite, comme le grand capital l’est au petit. Mais il s’y ajoute encore des circonstances spéciales qui amènent d’une façon obligatoire l’accumulation de la grande propriété et l’absorption de la petite par celle-ci.\par
[XII] 1º Nulle part le nombre relatif des ouvriers et des instruments ne diminue plus avec la grandeur du fonds que dans la propriété foncière. De même nulle part la possibilité de l’exploitation sous toutes les formes, l’économie des frais de production et la division habile du travail n’augmentent plus avec la grandeur du fonds que dans la propriété foncière. Si petit que soit un champ, les instruments de travail qu’il exige comme la charrue, la scie, etc., ont une certaine limite au-dessous de laquelle on ne peut plus descendre, tandis que la petitesse de la propriété peut descendre beaucoup au-dessous de cette limite.\par
2º La grande propriété foncière accumule à son profit les intérêts que le capital du fermier a appliqués à l’amélioration du sol. La petite propriété foncière doit utiliser son propre capital. Tout ce profit est donc perdu pour elle.\par
3º Alors que toute amélioration sociale sert la grande propriété foncière, elle nuit à la petite, parce qu’elle exige d’elle toujours plus d’argent liquide.\par
4º Deux lois importantes pour cette concurrence sont encore à considérer :\par

\begin{quoteblock}
 \noindent a) La rente des terres cultivées pour produire la nourriture des hommes règle la rente de la plupart des autres terres cultivées (SMITH, tome I, p. 331).
 \end{quoteblock}

\noindent Les moyens de subsistance comme le bétail, etc. ne peuvent, en dernière analyse, être produits que par la grande propriété. C’est donc elle qui règle la rente des autres terres et elle peut la réduire à un minimum.\par
Le petit propriétaire foncier qui travaille lui-même se trouve alors, vis-à-vis du grand propriétaire, dans le rapport d’un artisan qui possède son \emph{propre} instrument vis-à-vis du patron de fabrique. La petite propriété est devenue un simple instrument de travail. [XVI] La rente foncière disparaît entièrement pour le petit propriétaire, il lui reste tout au plus l’intérêt de son capital et son salaire ; car la concurrence peut amener la rente foncière à n’être plus que l’intérêt du capital que le propriétaire n’a pas lui-même investi.\par
b) Nous avons d’ailleurs vu déjà que, à fertilité égale et à habileté égale d’exploitation des terres, mines et pêcheries, le produit est en proportion de l’extension des capitaux. Donc victoire de la grande propriété foncière. De même, à égalité des capitaux en proportion de la fertilité. Donc à égalité de capitaux, c’est le propriétaire du sol le plus fertile qui gagne.\par

\begin{quoteblock}
 \noindent c) On peut dire d’une mine, en général, qu’elle est féconde ou qu’elle est stérile selon que la quantité de minerai que peut en tirer une certaine quantité de travail est plus ou moins grande que celle qu’une même quantité de travail tirerait de la plupart des autres mines de la même espèce (SMITH, tome I, pp. 345-346). Le prix de la mine de charbon la plus féconde règle le prix du charbon pour toutes les autres mines de son voisinage. Le propriétaire et l’entrepreneur trouvent tous deux qu’ils pourront se faire l’un une plus forte rente, l’autre un plus gros profit, en vendant quelque chose au-dessous de tous leurs voisins. Les voisins sont bientôt obligés de vendre au même prix, quoiqu’ils soient moins en état d’y suffire et quoique ce prix aille toujours en diminuant et leur enlève même quelquefois toute leur rente et tout leur profit. Quelques exploitations se trouvent alors entièrement abandonnées, d’autres ne rapportent plus de rente et ne peuvent plus être continuées que par le propriétaire de la mine (SMITH, tome I, p. 350). Après la découverte des mines du Pérou, les mines d’argent d’Europe furent pour la plupart abandonnées… La même chose arriva à l’égard des mines de Cuba et de Saint-Domingue, et même à l’égard des anciennes mines du Pérou, après la découverte de celles du Potosi (tome I, p. 353).
 \end{quoteblock}

\noindent Tout ce que Smith dit ici des mines est plus ou moins valable de la propriété foncière en général.\par

\begin{quoteblock}
 \noindent d) Il est à remarquer que partout le prix courant des terres dépend du taux courant de l’intérêt… Si la rente de la terre tombait au-dessous de l’intérêt de l’argent d’une différence plus forte, personne ne voudrait acheter de terres, ce qui réduirait bientôt leur prix courant. Au contraire, si les avantages faisaient beaucoup plus que compenser la différence, tout le monde voudrait acheter des terres, ce qui en relèverait encore bientôt le prix courant ([SMITH], tome II, pp. 367-368).
 \end{quoteblock}

\noindent Il résulte de ce rapport entre la rente foncière et le taux de l’argent que la rente foncière doit tomber de plus en plus, de sorte qu’en fin de compte, il n’y aura plus que les gens les plus riches qui pourront vivre de la rente foncière. Donc concurrence toujours plus grande entre les propriétaires fonciers qui n’afferment pas. Ruine d’une partie d’entre eux. – Nouvelle accumulation de la grande propriété foncière.\par
[XVII] Cette concurrence a en outre comme conséquence qu’une grande partie de la propriété foncière tombe entre les mains des capitalistes et que les capitalistes deviennent ainsi en même temps propriétaires fonciers, de même que, somme toute, les petits propriétaires fonciers ne sont déjà plus que des capitalistes. De même, une partie de la grande propriété foncière devient en même temps industrielle.\par
La conséquence dernière est donc la résolution de la différence entre capitaliste et propriétaire foncier, de sorte que, dans l’ensemble, il n’y a plus que deux classes de la population : la classe ouvrière et la classe des capitalistes. Cette mise dans le commerce de la propriété foncière, cette transformation de la propriété foncière en marchandise est la dernière chute de l’ancienne aristocratie et le dernier achèvement de l’aristocratie de l’argent.\par
1º Nous ne partageons pas les larmes sentimentales que le romantisme verse à ce sujet. Il confond l’infamie qu’il y a à trafiquer de la terre avec la logique tout à fait rationnelle, souhaitable et nécessaire dans le cadre de la propriété privée, que comporte la mise dans le commerce de la propriété privée de la terre. Premièrement, la propriété foncière féodale est déjà, par nature, de la terre dont on a trafiqué, qui est aliénée à l’homme et qui, par conséquent, l’affronte en la personne de quelques grands seigneurs.\par
Déjà la propriété féodale comporte la domination de la terre sur les hommes en tant que puissance qui leur est étrangère. Le serf est l’accessoire de la terre. De même le majorataire, le fils aîné appartient à la terre. C’est elle qui le reçoit en héritage. D’une manière générale, le règne de la propriété privée commence avec la propriété foncière, elle en est le fondement. Mais dans la propriété foncière féodale, le seigneur apparaît tout au moins comme le roi de la propriété. De même il existe encore l’apparence d’un rapport plus intime que celui de la simple richesse matérielle entre le possesseur et la terre. La terre s’individualise avec son maître, eue a son rang, elle est baronnie ou comtat avec lui, elle a ses privilèges, sa juridiction, ses relations politiques, etc. Elle apparaît comme le corps non-organique de son maître. D’où le proverbe : “\foreign{nulle terre sans maître}” qui exprime la soudure entre la seigneurie et la propriété foncière. De même le règne de la propriété foncière n’apparaît pas directement comme le règne du simple capital. Ses ressortissante sont plutôt, vis-à-vis d’elle, comme vis-à-vis de leur patrie. C’est un type étroit de nationalité.\par
[XVIII] De même la propriété foncière féodale donne son nom à son maître, comme un royaume le donne à son roi. L’histoire de sa famille, l’histoire de sa maison, etc., tout cela individualise pour lui la propriété foncière et en fait formellement sa maison, en fait une personne. De même ceux qui cultivent sa propriété foncière n’ont pas la situation de journaliers salariés, mais ou bien ils sont eux-mêmes sa propriété comme les serfs, ou bien ils sont vis-à-vis de lui dans un rapport d’allégeance, de sujétion et d’obligation. Sa situation vis-à-vis d’eux est donc directement politique mais elle a également un côté sentimental. Les mœurs, le caractère, etc. changent d’une terre à l’autre et semblent ne faire qu’un avec la parcelle, tandis que plus tard ce n’est plus que la bourse de l’homme qui le lie à la terre, et non son caractère ou son individualité. Enfin, il ne cherche pas à tirer le plus grand avantage possible de sa propriété foncière. Au contraire, il consomme ce qui est sur place et laisse tranquillement le soin de procurer le nécessaire au serf et au fermier. C’est la condition noble de la propriété foncière qui donne à son maître une auréole romantique.\par
Il est nécessaire que cette apparence soit supprimée ; que la propriété foncière, racine de la propriété privée, soit entraînée tout entière dans le mouvement de celle-ci et devienne une marchandise ; que la suprématie du propriétaire apparaisse comme la pure suprématie de la propriété privée, du capital, dépouillée de toute teinture politique ; que le rapport de propriétaire à ouvrier se réduise au rapport économique d’exploiteur à exploité ; que tout rapport personnel du propriétaire à sa propriété cesse et que celle-ci devienne seulement la richesse matérielle concrète ; que le mariage de l’intérêt prenne la place du mariage d’honneur avec la terre et que la terre soit tout autant ramenée à une valeur commerciale que l’homme. Il est nécessaire que ce qui est la racine de la propriété foncière, la cupidité sordide, apparaisse aussi sous sa forme cynique. Il est nécessaire que le monopole immobile se convertisse en monopole mobile et harcelé, en concurrence ; que la jouissance oisive de la sueur de sang d’autrui se transforme en l’affairement du commerce qu’on en fait. Il est enfin nécessaire que, sous la forme de capital, la propriété manifeste dans cette concurrence sa domination tant sur la classe ouvrière que sur les propriétaires eux mêmes, du fait que les lois de mouvement du capital les ruinent ou les élèvent. Alors, à la place de l’adage moyenâgeux : “\foreign{nulle terre sans seigneur}” , apparaîtra le proverbe moderne : “\foreign{l’argent n’a pas de maître}”, où s’exprime toute la domination de la matière inerte sur les hommes.\par
[XIX] 2° Quant à la querelle de la division ou de la non-division de la propriété foncière, il faut faire les remarques suivantes.\par
La \emph{division de la propriété} nie le \emph{grand monopole} de la propriété foncière, elle l’abolit, mais seulement en \emph{le généralisant.} Elle ne supprime pas le fondement du monopole, la propriété privée. Eue s’en prend à l’existence du monopole, mais non à son essence. Il s’ensuit qu’elle tombe sous le coup des lois de la propriété privée. La division de la propriété foncière correspond en effet au mouvement de la concurrence sur le terrain industriel. Outre les désavantages économiques de cette division des instruments et de cet isolement du travail de chacun (qu’il faut bien distinguer de la division du travail : le travail n’est pas réparti entre beaucoup d’individus, mais le même travail est fait chacun pour soi, c’est une multiplication du même travail), ce morcellement, comme ailleurs la concurrence, se convertit à nouveau nécessairement en accumulation.\par
Où donc se produit la division de la propriété foncière, il ne reste rien d’autre à faire que de revenir au monopole sous une forme encore plus odieuse ou de nier, d’abolir la division même de la propriété. Mais cela ne veut pas dire retour à la propriété féodale, mais au contraire abolition de la propriété privée du sol en général. La première abolition du monopole est toujours sa généralisation, l’extension de son existence. L’abolition du monopole qui a atteint son existence la plus large et la plus vaste possible est sa destruction complète. L’association appliquée au sol partage, au point de vue économique, les avantages de là grande propriété foncière et elle est la première à réaliser la tendance primitive de la division, c’est-à-dire l’égalité, de même qu’elle restaure, d’une manière rationnelle et non plus par la médiation de la servitude, de la domination et d’une absurde mystique de la propriété, le rapport sentimental de l’homme à la terre : en effet, la terre cesse d’être un objet de trafic et, par le travail et la jouissance libre, elle redevient une propriété vraie et personnelle de l’homme. Un grand avantage de la division est que la masse, qui ne peut plus se résoudre à la servitude, périt ici de la propriété d’une autre manière que [celle] de l’industrie.\par
\bigbreak
\noindent Quant à la grande propriété foncière, ses défenseurs ont toujours identifié d’une manière sophistique les avantages économiques qu’offre l’agriculture à grande échelle avec la grande propriété terrienne, comme si ce n’était pas l’abolition de la propriété qui commençait précisément à donner à ces avantages soit leur [XX] extension maximum, soit leur utilité sociale. De même, ils ont attaqué l’esprit mercantile de la petite propriété foncière comme si la grande propriété, même déjà sous sa forme féodale, n’incluait pas le trafic d’une façon latente. Pour ne rien dire de la forme anglaise moderne où s’allient le féodalisme du propriétaire et l’esprit mercantile et l’industrie du fermier.\par
De même que la grande propriété foncière peut retourner à la division de la propriété le reproche de monopole que celle-ci lui fait, car la division est aussi fondée sur le monopole de la propriété privée, de même la division de la propriété foncière peut retourner à la grande propriété le reproche de division, car là aussi celle-ci règne, mais sous une forme rigide, figée. En général, la propriété privée repose bien sur la division. D’ailleurs, de même que la division de la propriété foncière ramène à la grande propriété sous la forme de richesse capitaliste, de même la propriété féodale doit nécessairement aller jusqu’à la division ou tout au moins tomber entre les mains des capitalistes, quoi qu’elle fasse.\par
Car la grande propriété foncière, comme en Angleterre, pousse la majorité écrasante de la population dans les bras de l’industrie et réduit ses propres ouvriers à la misère complète. Elle engendre et accroît donc la force de ses ennemis, la capital, l’industrie, en jetant des pauvres et toute une activité du pays dans l’autre camp. Elle rend la majorité du pays industrielle, en fait donc l’adversaire de la grande propriété foncière. Si l’industrie a atteint une grande puissance, comme c’est aujourd’hui le cas en Angleterre, elle arrache peu à peu à la grande propriété ses monopoles par rapport à [ceux] de l’étranger et les jette dans la concurrence avec la propriété foncière de l’étranger. Sous le règne de l’industrie, la propriété foncière ne pouvait, en effet, assurer sa grandeur féodale que par des monopoles vis-à-vis de l’étranger pour se mettre ainsi à l’abri des lois générales du commerce qui sont contraires à sa nature féodale. Une fois jetée dans la concurrence, elle en suit les lois comme toute autre marchandise qui y est soumise. Elle se plie aux mêmes fluctuations, augmentations ou diminutions, passages d’une main à l’autre, et aucune loi ne peut plus la maintenir dans quelques mains prédestinées. [XXI] La conséquence directe est l’éparpillement en de nombreuses mains ; en tout cas, elle tombe au pouvoir des capitaux industriels.\par
Enfin, la grande propriété foncière, qui s’est ainsi maintenue par la force et qui a engendré auprès d’elle une industrie redoutable, conduit plus rapidement encore à la crise que la division de la propriété foncière, auprès de laquelle la puissance de l’industrie reste toujours de second ordre.\par
La grande propriété foncière a, comme nous le voyons en Angleterre, déjà perdu son caractère féodal et pris un caractère individuel dans la mesure où elle veut faire le plus d’argent possible. Elle [donne] au propriétaire la rente foncière la plus forte possible, au fermier le profit de son capital le plus grand possible. Les ouvriers agricoles sont donc déjà réduits au minimum et, à l’intérieur de la propriété foncière, la classe des fermiers représente déjà la puissance de l’industrie et du capital. Du fait de la concurrence avec l’étranger, la rente foncière cesse pour la plus grande part de pouvoir constituer un revenu indépendant. Une grande partie des propriétaires fonciers prend nécessairement la place des fermiers qui, de cette manière, tombent dans le prolétariat. D’autre part, beaucoup de fermiers s’empareront aussi de la propriété foncière ; car les grands propriétaires qui, avec leurs revenus faciles, se sont en majorité adonnés à la dissipation et la plupart du temps sont également impropres à diriger l’agriculture à grande échelle, ne possèdent pour une part ni le capital, ni les capacités nécessaires pour exploiter le soi. Donc une partie d’entre eux est entièrement ruinée. Enfin, le salaire réduit déjà à un minimum doit être réduit plus encore pour faire face à la concurrence. Cela conduit alors nécessairement à la révolution.\par
Il fallait que la propriété foncière se développât de chacune des deux manières pour connaître en l’une et en l’autre son déclin nécessaire, de même que l’industrie devait aussi se ruiner sous la forme du monopole et sous celle de la concurrence pour apprendre à croire en l’homme.
\subsection[{Le travail aliéné}]{Le travail aliéné}
\noindent [XXII] Nous sommes partis des prémisses de l’économie politique. Nous avons accepté son langage et ses lois. Nous avons supposé la propriété privée, la séparation du travail, du capital et de la terre, ainsi que celle du salaire, du profit capitaliste et de la rente foncière, tout comme la division du travail, la concurrence, la notion de valeur d’échange, etc. En partant de l’économie politique elle-même, en utilisant ses propres termes, nous avons montré que l’ouvrier est ravalé au rang de marchandise, et de la marchandise la plus misérable, que la misère de l’ouvrier est en raison inverse de la puissance et de la grandeur de sa production \footnote{C’est-à-dire que plus il produit, plus sa misère est grande.}, que le résultat nécessaire de la concurrence est l’accumulation du capital en un petit nombre de mains, donc la restauration encore plus redoutable du monopole ; qu’enfin la distinction entre capitaliste et propriétaire foncier, comme celle entre paysan et ouvrier de manufacture, disparaît et que toute la société doit se diviser en deux classes, celle des propriétaires et celle des ouvriers non propriétaires.\par
L’économie politique part du fait de la propriété privée. Elle ne nous l’explique pas. Elle exprime le processus matériel que décrit en réalité la propriété privée, en formules générales et abstraites, qui ont ensuite pour elle valeur de lois. Elle ne comprend \footnote{Begreift, c’est-à-dire : elle ne saisit pas ces lois dans leur concept.} pas ces lois, c’est-à-dire qu’elle ne montre pas comment elles résultent de l’essence de la propriété privée. L’économie politique ne nous fournit aucune explication sur la raison de la séparation du travail et du capital, du capital et de la terre. Quand elle détermine par exemple le rapport du salaire au profit du capital, ce qui est pour elle la raison dernière, c’est l’intérêt des capitalistes c’est-à-dire qu’elle suppose donné ce qui doit être le résultat de son développement. De même la concurrence intervient partout. Elle est expliquée par des circonstances extérieures. Dans quelle mesure ces circonstances extérieures, apparemment contingentes, ne sont que l’expression d’un développement nécessaire, l’économie politique ne nous l’apprend pas. Nous avons vu comment l’échange lui-même lui apparaît comme un fait du hasard. Les seuls mobiles qu’elle mette en mouvement sont la soif \emph{de richesses} et la \emph{guerre entre convoitises}, la \emph{concurrence.}\par
C’est précisément parce que l’économie ne comprend pas l’enchaînement du mouvement que, par exemple, la doctrine de la concurrence a pu s’opposer à nouveau à celle du monopole, la doctrine de la liberté industrielle à celle de la corporation, la doctrine de la division de la propriété foncière à celle de la grande propriété terrienne, car la concurrence, la liberté industrielle, la division de la propriété foncière n’étaient développées et comprises que comme des conséquences contingentes, intentionnelles, arrachées de force, et non pas nécessaires, inéluctables et naturelles du monopole, de la corporation et de la propriété féodale.\par
Nous avons donc maintenant à comprendre l’enchaînement essentiel qui lie la propriété privée, la soif de richesses, la séparation du travail, du capital et de la propriété, celle de l’échange et de la concurrence, de la valeur et de la dépréciation de l’homme, du monopole et de la concurrence, etc., bref le lien de toute cette \emph{aliénation} \footnote{Marx emploie ici le terme Entfremdung. Mais il utilise aussi, avec une fréquence presque égale, celui de Entäusserung. Étymologiquement, le mot Entfremdung insiste plus sur l’idée d’étranger tandis que Entdässerung marque plus l’idée de dépossession. Nous avons pour notre part renoncé à tenir compte d’une nuance que Marx n’a pas faite puisqu’il emploie indifféremment les deux termes. Hegel ne faisait pas non plus la différence et il nous a semblé inutile de recourir au procédé de M. Hippolyte qui a créé, dans sa traduction de la Phénoménologie, le mot extranéation. Là ou Marx, pour insister, utilise successivement les deux termes, nous avons traduit l’un des deux par dessaisissement. Quand Marx utilise l’adjectif entfremdet, nous avons traduit, lorsque c’était possible, par rendu étranger. Mais le terme aliéné n’a pas été réservé uniquement pour rendre entäussert.} avec le système de \emph{l’argent.}\par
Ne faisons pas comme l’économiste qui, lorsqu’il veut expliquer quelque chose, se place dans un état originel fabriqué de toutes pièces. Ce genre d’état originel n’explique rien. Il ne fait que repousser la question dans une grisaille lointaine et nébuleuse. Il suppose donné dans la forme du fait, de l’événement, ce qu’il veut en déduire, c’est-à-dire le rapport”nécessaire entre deux choses, par exemple entre la division du travail et l’échange. Ainsi le théologien explique l’origine du mal par le péché originel, c’est-à-dire suppose comme un fait, sous la forme historique, ce qu’il doit lui-même expliquer.\par
Nous partons d’un fait économique \emph{actuel.}\par
L’ouvrier devient d’autant plus pauvre qu’il produit plus de richesse, que sa production croît en puissance et en volume. L’ouvrier devient une marchandise d’autant plus vile qu’il crée plus de marchandises. La \emph{dépréciation} du monde des hommes augmente en raison directe de la \emph{mise en valeur} du monde des choses. Le travail ne produit pas que des marchandises ; il se produit lui-même et produit l’ouvrier en tant que \emph{marchandise}, et cela dans la mesure où il produit des marchandises en général.\par
Ce fait n’exprime rien d’autre que ceci : l’objet que le travail produit, son produit, l’affronte comme un \emph{être étranger}, comme une \emph{puissance indépendante} du producteur. Le produit du travail est le travail qui s’est fixé, concrétisé dans un objet, il est \emph{l’objectivation du travail.} L’actualisation du travail est son objectivation. Au stade de l’économie, cette actualisation du travail apparaît comme la \emph{perte} pour l’ouvrier \emph{de sa réalité}, l’objectivation comme la \emph{perte de l’objet ou l’asservissement} à celui-ci, l’appropriation comme \emph{l’aliénation}, le \emph{dessaisissement.}\par
La réalisation du travail se révèle être à tel point une perte de réalité que l’ouvrier perd sa réalité jusqu’à en mourir de faim. L’objectivation se révèle à tel point être la perte de l’objet, que l’ouvrier est spolié non seulement des objets les plus nécessaires à la vie, mais encore des objets du travail. Oui, le travail lui-même devient un objet dont il ne peut s’emparer qu’en faisant le plus grand effort et avec les interruptions les plus irrégulières. L’appropriation de l’objet se révèle à tel point être une aliénation que plus l’ouvrier produit d’objets, moins il peut posséder et plus il tombe sous la domination de son produit, le capital.\par
Toutes ces conséquences se trouvent dans cette détermination l’ouvrier est à l’égard du \emph{produit de son travail} dam le même rapport qu’à l’égard d’un objet \emph{étranger.} Car ceci est évident par hypothèse : plus l’ouvrier s’extériorise dans son travail, plus le monde étranger, objectif, qu’il crée en face de lui, devient puissant, plus il s’appauvrit lui-même et plus son monde intérieur devient pauvre, moins il possède en propre. Il en va de même dans la religion. Plus l’homme met de choses en Dieu, moins il en garde en lui-même. L’ouvrier met sa vie dans l’objet. Mais alors celle-ci ne lui appartient plus, elle appartient à l’objet. Donc plus cette activité est grande, plus l’ouvrier est sans objet \footnote{L’expression allemande est “gegenstandslos”.}. Il n’est pas ce qu’est le produit de son travail. Donc plus ce produit est grand, moins il est lui-même. L’aliénation de l’ouvrier dans son produit signifie non seulement que son travail devient un objet, une existence \emph{extérieure, mais} que son travail existe en \emph{dehors de} lui, indépendamment de lui, étranger à lui, et devient une puissance autonome vis-à-vis de lui, que la vie qu’il a prêtée à l’objet s’oppose à lui, hostile et étrangère.\par
[XXIII] Examinons maintenant de plus près \emph{l’objectivation, la} production de l’ouvrier et, en elle, l’aliénation, la \emph{perte de} l’objet, de son produit.\par
L’ouvrier ne peut rien créer sans la nature, sans le \emph{monde exté}rieur \emph{sensible. Elle} est la matière dans laquelle son travail se réalise, au sein de laquelle il s’exerce, à partir de laquelle et au moyen de laquelle il produit.\par
Mais, de même que la nature offre au travail les \emph{moyens de sub}sistance, dans ce sens que le travail ne peut pas vivre sans objets sur lesquels il s’exerce, de même elle fournit aussi d’autre part les moyens \emph{de subsistance} au sens restreint, c’est-à-dire les moyens de subsistance physique de l’ouvrier lui-même.\par
Donc, plus l’ouvrier s’approprie par son travail le monde extérieur, la nature sensible, plus il se soustrait de moyens \emph{de subsis}tance sous ce double point de vue : que, premièrement, le monde extérieur sensible cesse de plus en plus d’être un objet appartenant à son travail, un moyen \emph{de subsistance de} son travail ; et que, deuxièmement, il cesse de plus en plus d’être un \emph{moyen de subsistance au} sens immédiat, un moyen pour la subsistance physique de l’ouvrier.\par
De ce double point de vue, l’ouvrier devient donc un esclave de son objet : premièrement, il reçoit un objet de travail, c’est-à-dire du travail, et, deuxièmement, il reçoit des moyens \emph{de subsistance.} Donc, dans le sens qu’il lui doit la possibilité d’exister premièrement en tant qu’ouvrier et deuxièmement en tant que sujet \emph{physique. Le} comble de cette servitude est que seule sa qualité d’ouvrier lui permet de se conserver encore en tant que sujet \emph{physique, et} que ce n’est plus qu’en tant que \emph{sujet physique} \footnote{Le travail, oui est pour l’homme manifestation de sa personnalité, n’est \emph{plus} pour l’ouvrier que le moyen de subsister. Il ne peut se conserver \emph{cri} tant que sujet physique qu’en qualité d’ouvrier, et non en qualité d’homme ayant directement accès aux moyens de subsistance que lui offre la nature.} qu’il est \emph{ouvrier.}\par
(L’aliénation de l’ouvrier dans son objet s’exprime selon les lois de l’économie de la façon suivante : plus l’ouvrier produit, moins il a à consommer ; plus il crée de valeurs, plus il se déprécie et voit diminuer sa dignité ; plus son produit a de forme, plus l’ouvrier est difforme ; plus son objet est civilisé, plus l’ouvrier est barbare ; plus le travail est puissant, plus l’ouvrier est impuissant ; plus le travail s’est rempli d’esprit, plus l’ouvrier a été privé d’esprit et est devenu esclave de la nature.)\par
\emph{L’économie politique cache} l’aliénation dans \emph{l’essence} du travail \footnote{Pour Marx l’essence du travail c’est qu’il est une activité spécifique de l’homme, une manifestation de sa personnalité, l’objectivation de celle-ci. L’économie politique ne considère pas le travail dans son rapport à l’homme, mais seulement sous sa forme aliénée : dans la mesure où il est producteur de valeur, et que d’extériorisation des “forces essentielles” de l’homme il s’est transformé en activité en vue d’un gain.} par \emph{le} fait \emph{qu’elle} ne considère pas \emph{le} rapport \emph{direct entre} l’ouvrier (le travail) et la production. Certes, le travail produit des merveilles pour les riches, mais il produit le dénuement pour l’ouvrier. Il produit des palais, mais des tanières pour l’ouvrier. Il produit la beauté, mais l’étiolement pour l’ouvrier. Il remplace le travail par des machines, mais il rejette une partie des ouvriers dans un travail barbare et fait de l’autre partie des machines. Il produit l’esprit, mais il produit l’imbécillité, le crétinisme pour l’ouvrier.\par
\emph{Le rapport} immédiat du travail à ses produits est le rapport de l’ouvrier aux \emph{objets de} sa production. Le rapport de l’homme qui a de la fortune aux objets de la production et à la production elle-même n’est qu’une conséquence de ce premier rapport. Et il le confirme. Nous examinerons cet autre aspect plus tard.\par
Si donc nous posons la question : Quel est le rapport essentiel du travail, nous posons la question du rapport de l’ouvrier à la production.\par
Nous n’avons considéré jusqu’ici l’aliénation, le dessaisissement de l’ouvrier que sous un seul aspect, celui de son rapport aux produits \emph{de son} travail. Mais l’aliénation n’apparaît pas seulement dans le résultat, mais dans l’acte \emph{de} la production, à l’intérieur de l’activité productive elle-même. Comment l’ouvrier pourrait-il affronter en étranger le produit de son activité, si, dans l’acte de la production même, il ne devenait pas étranger à lui-même : le produit n’est, en fait, que le résumé de l’activité, de la production. Si donc le produit du travail est l’aliénation, la production elle-même doit être l’aliénation en acte, l’aliénation de l’activité, l’activité de l’aliénation. L’aliénation de l’objet du travail n’est que le résumé de l’aliénation, du dessaisissement, dans l’activité du travail elle-même.\par
Or, en quoi consiste l’aliénation du travail ?\par
D’abord, dans le fait que le travail est extérieur à l’ouvrier, c’est-à-dire qu’il n’appartient pas à son essence, que donc, dans son travail, celui-ci ne s’affirme pas mais se nie, ne se sent pas à l’aise, mais malheureux, ne déploie pas une libre activité physique et intellectuelle, mais mortifie son corps et ruine son esprit. En conséquence, l’ouvrier n’a le sentiment d’être auprès de lui-même \footnote{\emph{Bei sich}, c’est-à-dire libéré des déterminations extérieures à son être} qu’en dehors du travail et, dans le travail, il se sent en dehors de soi. Il est comme chez lui. quand il ne travaille pas et, quand il travaille, il ne se sent pas chez lui. Son travail n’est donc pas volontaire, mais contraint, c’est du travail forcé. Il n’est donc pas la satisfaction d’un besoin, mais seulement un moyen de satisfaire des besoins en dehors du travail. Le caractère étranger du travail apparaît nettement dans le fait que, dès qu’il n’existe pas de contrainte physique ou autre, le travail est fui comme la peste. Le travail extérieur, le travail dans lequel l’homme s’aliène, est un travail de sacrifice de soi, de mortification. Enfin, le caractère extérieur à l’ouvrier du travail apparaît dans le fait qu’il n’est pas son bien propre, mais celui d’un autre, qu’il ne lui appartient pas, que dans le travail l’ouvrier ne s’appartient pas lui-même, mais appartient à un autre. De même que, dans la religion, l’activité propre de l’imagination humaine, du cerveau humain et du cœur humain, agit sur l’individu indépendamment de lui, c’est-à-dire comme une activité étrangère divine ou diabolique, de même l’activité de l’ouvrier n’est pas son activité propre. Elle appartient à un autre, elle est la perte de soi-même.\par
On en vient donc à ce résultat que l’homme (l’ouvrier) ne se sent plus librement actif que dans ses fonctions animales, manger, boire et procréer, tout au plus encore dans l’habitation, qu’animal. Le bestial devient l’humain et l’humain devient le bestial.\par
Manger, boire et procréer, etc., sont certes aussi des fonctions authentiquement humaines. Mais, séparées abstraitement du reste du champ des activités humaines et devenues ainsi la fin dernière et unique, elles sont bestiales.\par
Nous avons considéré l’acte d’aliénation de l’activité humaine pratique, le travail, sous deux aspects : Premièrement, le rapport de l’ouvrier au produit du travail en tant qu’objet étranger et ayant barre sur lui. Ce rapport est en même temps le rapport au monde extérieur sensible, aux objets de la nature, monde qui s’oppose à lui d’une manière étrangère et hostile. Deuxièmement, le rapport du travail à l’acte de production à l’intérieur du travail. Ce rapport est le rapport de l’ouvrier à sa propre activité en tant qu’activité étrangère qui ne lui appartient pas, c’est l’activité qui est passivité, la force qui est impuissance, la procréation qui est castration, l’énergie physique et intellectuelle propre de l’ouvrier, sa vie personnelle – car qu’est-ce que la vie sinon l’activité – qui est activité dirigée contre lui-même, indépendante de lui, ne lui appartenant pas. L’aliénation de soi comme, plus haut, l’aliénation de la chose.\par
[XXIV] Or, nous avons encore à tirer des deux précédentes, une troisième détermination du travail aliéné.\par
L’homme est un être générique \footnote{Cette expression, courante dans la philosophie de l’époque, ne nous est plus guère familière aujourd’hui. Dans \emph{l’Encyclopédie (§ 177)}, Hegel définit le genre \emph{(die Gattung)} comme “l’Universel concret”. Il dit aussi \emph{(§ 367)} qu’il “constitue une unité simple étant en soi avec la singularité du sujet, dont il est substance concrète”. Dire que l’homme est un être générique, c’est donc dire que l’homme s’élève au-dessus de son individualité subjective, qu’il reconnaît en lui l’universel objectif et se dépasse ainsi en tant qu’être fini. Autrement dit, il est individuellement le représentant de l’Homme.}. Non seulement parce que, sur le plan pratique et théorique, il fait du genre, tant du sien propre que de celui des autres choses, son objet, mais encore – et ceci n’est qu’une autre façon d’exprimer la même chose – parce qu’il se comporte vis-à-vis de lui-même comme vis-à-vis du genre actuel vivant, parce qu’il se comporte vis-à-vis de lui-même comme vis-à-vis d’un être universel, donc libre.\par
La vie générique tant chez l’homme que chez l’animal consiste d’abord, au point de vue physique, dans le fait – que l’homme (comme l’animal) vit de la nature non-organique, et plus l’homme est universel par rapport à l’animal, plus est universel le champ de la nature non-organique dont il vit. De même que les plantes, les animaux, les pierres, l’air, la lumière, etc., constituent du point de vue théorique une partie de la conscience humaine, soit en tant qu’objets des sciences de la nature, soit en tant qu’objets de l’art – qu’ils constituent sa nature intellectuelle non-organique, qu’ils sont des moyens de subsistance intellectuelle que l’homme doit d’abord apprêter pour en jouir et les digérer – de même ils constituent aussi au point de vue pratique une partie de la vie humaine et de l’activité humaine. Physiquement, l’homme ne vit que de ces produits naturels, qu’ils apparaissent sous forme de nourriture, de chauffage, de vêtements, d’habitation, etc. L’universalité de l’homme apparaît en pratique précisément dans l’universalité qui fait de la nature entière son corps non-organique, aussi bien dans la mesure où, premièrement, elle est un moyen de subsistance immédiat que dans celle où, [deuxièmement], elle est la matière, l’objet et l’outil de son activité vitale. La nature, c’est-à-dire la nature qui n’est pas elle-même le corps humain, est le corps non-organique de l’homme. L’homme vit de la nature signifie : la nature est son corps avec lequel il doit maintenir un processus constant pour ne pas mourir. Dire que la vie physique et intellectuelle de l’homme est indissolublement liée à la nature ne signifie pas autre chose sinon que la nature est indissolublement liée avec elle-même, car l’homme est une partie de la nature.\par
Tandis que le travail aliéné rend étrangers à l’homme 1º la nature, 2º lui-même, sa propre fonction active, son activité vitale, il rend étranger à l’homme le genre : il fait pour lui de la vie générique le moyen de la vie individuelle. Premièrement, il rend étrangères la vie générique et la vie individuelle, et deuxièmement il fait de cette dernière, réduite à l’abstraction, le but de la première, qui est également prise sous sa forme abstraite et aliénée.\par
Car, premièrement, le travail, l’activité vitale, la vie productive n’apparaissent eux-mêmes à l’homme que comme un moyen de satisfaire un besoin, le besoin de conservation de l’existence physique. Mais la vie productive est la vie générique. C’est la vie engendrant la vie. Le mode d’activité vitale renferme tout le caractère d’une espèce \footnote{Species}, son caractère générique, et l’activité libre, consciente, est le caractère générique de l’homme. La vie elle-même n’apparaît que comme moyen de subsistance.\par
L’animal s’identifie directement avec son activité vitale. Il ne se distingue pas d’elle. Il est cette activité. L’homme fait de son activité vitale elle-même l’objet de sa volonté et de sa conscience. Il a une activité vitale consciente. Ce n’est pas une détermination avec laquelle il se confond directement. L’activité vitale consciente distingue directement l’homme de l’activité vitale de l’animal. C’est précisément par là, et par là seulement, qu’il est un être générique \footnote{ \noindent La citation suivante de Feuerbach (L’Essence du christianisme, Introduction), illustre bien la parenté des positions respectives de Marx et de Feuerbach et ce qui les distingue – “Quelle est donc cette différence essentielle qui distingue l’homme de l’animal ? A cette question, la plus simple et la plus générale des réponses, mais aussi la plus populaire est : c’est la conscience. Mais la conscience au sens strict ; car la conscience qui désigne le sentiment de soi, le pouvoir de distinguer les objets sensibles, de percevoir et même de juger les choses extérieures sur des indices déterminés tombant sous le sens, cette conscience ne peut être refusée aux animaux. La conscience entendue dans le sens le plus strict n’existe que pour un être qui a pour objet sa propre espèce et sa propre essence… Être doué de conscience, c’est être capable de science. La science est la \emph{conscience} des espèces… Or seul un être qui a pour objet sa propre espèce, sa propre essence, est susceptible de prendre pour objet, dans leur signification essentielle, des choses et des êtres autres que lui.\par
 C’est pourquoi l’animal n’a qu’une vie simple et l’homme une vie double chez l’animal la vie intérieure se confond avec la vie extérieure, l’homme, au contraire, possède une vie intérieure et une vie extérieure.” (Ludwig FEUERBACH : Manifestes philosophiques. Traduction de Louis Althusser, Paris 1960, pp. 57-58.)
}. Ou bien il est seulement un être conscient, autrement dit sa vie propre est pour lui un objet, précisément parce qu’il est un être générique. C’est pour cela seulement que son activité est activité libre. Le travail aliéné renverse le rapport de telle façon que l’homme, du fait qu’il est un être conscient, ne fait précisément de son activité vitale, de son essence qu’un moyen de son existence.\par
Par la production pratique d’un monde objectif, l’élaboration de la nature non-organique, l’homme fait ses preuves en tant qu’être générique conscient, c’est-à-dire en tant qu’être qui se comporte à l’égard du genre comme à l’égard de sa propre essence, ou à l’égard de soi, comme être générique. Certes, l’animal aussi produit. Il se construit un nid, des habitations, comme l’abeille, le castor, la fourmi, etc. Mais il produit seulement ce dont il a immédiatement besoin pour lui ou pour son petit ; il produit d’une façon unilatérale, tandis que l’homme produit d’une façon universelle ; il ne produit que sous l’empire du besoin physique immédiat, tandis que l’homme produit même libéré du besoin physique et ne produit vraiment que lorsqu’il en est libéré ; l’animal ne se produit que lui-même, tandis que l’homme reproduit toute la nature ; le produit de l’animal fait directement partie de son corps physique, tandis que l’homme affronte librement son produit. L’animal ne façonne qu’à la mesure et selon les besoins de l’espèce à laquelle il appartient, tandis que l’homme sait produire à la mesure de toute espèce et sait appliquer partout à l’objet sa nature inhérente ; l’homme façonne donc aussi d’après les lois de la beauté.\par
C’est précisément dans le fait d’élaborer le monde objectif que l’homme commence donc à faire réellement ses preuves d’être \emph{générique. Cette} production est sa vie générique active. Grâce à cette production, la nature apparaît comme son œuvre et sa réalité. L’objet du travail est donc \emph{l’objectivation de la} vie géné\emph{rique de l’homme : car} celui-ci ne se double pas lui-même d’une façon seulement intellectuelle, comme c’est le cas dans la conscience, mais activement, réellement, et il se contemple donc lui-même dans un monde qu’il a créé. Donc, tandis que le travail aliéné arrache à l’homme l’objet de sa production, il lui arrache sa vie générique, sa véritable objectivité générique, et il transforme l’avantage que l’homme a sur l’animal en ce désavantage que son corps non-organique, la nature, lui est dérobé.\par
De même, en dégradant au rang de moyen l’activité propre, la libre activité, le travail aliéné fait de la vie générique de l’homme le moyen de son existence physique.\par
La conscience que l’homme a de son genre se transforme donc du fait de l’aliénation de telle façon que la vie générique devient pour lui un moyen.\par
Donc le travail aliéné conduit aux résultats suivants :\par
3º \emph{L’être générique de} l’homme, aussi bien la nature que ses facultés intellectuelles génériques, sont transformées en un être qui lui est étranger, en moyen de son existence \emph{individuelle. Il} rend étranger à l’homme son propre corps, comme la nature en dehors de lui, comme son essence spirituelle, son essence \emph{humaine.}\par
4º Une conséquence immédiate du fait que l’homme est rendu étranger au produit de son travail, à son activité vitale, à son être générique, est celle-ci : l’homme est rendu étranger à l’homme. Lorsque l’homme est en face de lui-même, c’est l’autre qui lui fait face \footnote{On trouve chez Feuerbach : “Sans objet l’homme n’est rien… Or l’objet auquel un sujet se rapporte par essence et nécessité n’est rien d’autre que l’essence propre de ce sujet, mais objectivée.” (Ibid., p. 61.)}. Ce qui est vrai du rapport de l’homme à son travail, au produit de son travail et à lui-même, est vrai du rapport de l’homme à l’autre ainsi qu’au travail et à l’objet du travail de l’autre.\par
D’une manière générale, la proposition que son être générique est rendu étranger à l’homme, signifie qu’un homme est rendu étranger à l’autre comme chacun d’eux est rendu étranger a l’essence humaine.\par
L’aliénation de l’homme, et en général tout rapport dans lequel l’homme se trouve avec lui-même, ne s’actualise, ne s’exprime que dans le rapport où l’homme se trouve avec les autres hommes.\par
Donc, dans le rapport du travail aliéné, chaque homme considère autrui selon la mesure et selon le rapport dans lequel il se trouve lui-même en tant qu’ouvrier.\par
[XXV] Nous sommes partis d’un fait économique, l’aliénation de l’ouvrier et de sa production. Nous avons exprimé le concept de ce fait : le travail rendu étranger, aliéné. Nous avons analysé ce concept, donc analysé seulement un fait économique.\par
Voyons maintenant comment le concept du travail rendu étranger, aliéné, doit s’exprimer et se représenter dans la réalité.\par
Si le produit du travail m’est étranger, m’affronte comme puissance étrangère, à qui appartient-il alors ?\par
Si ma propre activité ne m’appartient pas, si elle est une activité étrangère, de commande, à 'qui appartient-elle alors ?\par
À un être autre que moi.\par
Qui est cet être ?\par
Les Dieux ? Certes, dans les premiers temps, la production principale, comme par exemple la construction des temples, etc., en Égypte, aux Indes, au Mexique, apparaît tout autant au service des Dieux que le produit en appartient aux Dieux. Mais les Dieux seuls n’ont jamais été maîtres du travail. Tout aussi peu la nature. Et quelle contradiction serait-ce aussi que, à mesure que l’homme se soumet la nature plus entièrement par son travail, que les miracles des Dieux sont rendus plus superflus par les miracles de l’industrie, l’homme doive pour l’amour de ces puissances renoncer a la joie de produire et à la jouissance du produit.\par
L’être étranger auquel appartient le travail et le produit du travail, au service duquel se trouve le travail et à la jouissance duquel sert le produit du travail, ne peut être que l’homme lui-même.\par
Si le produit du travail n’appartient pas à l’ouvrier, s’il est une puissance étrangère en face de lui, cela n’est possible que parce qu’il appartient à un autre homme en dehors de l’ouvrier. Si son activité lui est un tourment, elle doit être la jouissance d’un autre et la joie de vivre pour un autre. Ce ne sont pas les dieux, ce n’est pas la nature, qui peuvent être cette puissance étrangère sur l’homme, c’est seulement l’homme lui-même.\par
Réfléchissons encore à la proposition précédente : le rapport de l’homme à lui-même n’est objectif, réel, pour lui que par son rapport à l’autre. Si donc il se comporte à l’égard du produit de son travail, de son travail objectivé, comme à l’égard d’un objet étranger, hostile, puissant, indépendant de lui, il est à son égard dans un tel rapport qu’un autre homme qui lui est étranger, hostile, puissant, indépendant de lui, est le maître de cet objet. S’il se comporte à l’égard de sa propre activité comme à l’égard d’une activité non-libre, il se comporte vis-à-vis d’elle comme vis-à-vis de l’activité au service d’un autre homme, sous sa domination, sa contrainte et son joug.\par
Toute aliénation de soi de l’homme à l’égard de soi-même et de la nature apparaît dans le rapport avec d’autres hommes, distincts de lui, dans lequel il se place lui-même et place la nature. C’est pourquoi l’aliénation religieuse de soi apparaît nécessairement dans le rapport du laïque au prêtre ou, comme il s’agit ici du monde intellectuel, à un médiateur, etc. Dans le monde réel pratique, l’aliénation de soi ne peut apparaître que par le rapport réel pratique à l’égard d’autres hommes. Le moyen grâce auquel s’opère l’aliénation est lui-même un moyen pratique. Par le travail aliéné, l’homme n’engendre donc pas seulement son rapport avec l’objet et l’acte de production en tant que puissances étrangères et qui lui sont hostiles ; il engendre aussi le rapport dans lequel d’autres hommes se trouvent à l’égard de sa production et de son produit et le rapport dans lequel il se trouve avec ces autres hommes. De même qu’il fait de sa propre production sa propre privation de réalité, sa punition, et de son propre produit une perte, un produit qui ne lui appartient pas, de même il crée la domination de celui qui ne produit pas sur la production et sur le produit. De même qu’il se rend étrangère sa propre activité, de même il attribue en propre à l’étranger l’activité qui ne lui est pas propre.\par
Nous n’avons considéré jusqu’ici le rapport que du point de vue de l’ouvrier et nous l’examinerons par la suite aussi du point de vue du non-ouvrier.\par
Donc, par l’intermédiaire du travail devenu étranger, aliéné, l’ouvrier engendre le rapport à ce travail d’un homme qui y est étranger et se trouve placé en dehors de lui. Le rapport de l’ouvrier à l’égard du travail engendre le rapport du capitaliste, du maître du travail, quel que soit le nom qu’on lui donne, à l’égard de celui-ci. La propriété privée est donc le produit, le résultat, la conséquence nécessaire du travail aliéné, du rapport extérieur de l’ouvrier à la nature et à lui-même.\par
La propriété privée résulte donc par analyse du concept de travail aliéné, c’est-à-dire d’homme aliéné, de travail devenu étranger, de vie devenue étrangère, d’homme devenu étranger.\par
Nous avons certes tiré le concept de travail aliéné (de vie aliénée) de l’économie politique comme le résultat du mouvement de la propriété privée. Mais de l’analyse de ce concept, il ressort que, si la propriété privée apparaît comme la raison, la cause du travail aliéné, elle est bien plutôt une conséquence de celui-ci, de même que les dieux à l’origine ne sont pas la cause, mais l’effet de l’aberration de l’entendement humain. Plus tard, ce rapport se change en action réciproque.\par
Ce n’est qu’au point culminant du développement de la propriété privée que ce mystère qui lui est propre reparaît de nouveau, à savoir d’une part qu’elle est le produit du travail aliéné et d’autre part qu’elle est le moyen par lequel le travail s’aliène, qu’elle est la réalisation de cette aliénation.\par
Ce développement éclaire aussitôt diverses collisions non encore résolues.\par
1. L’économie politique part du travail comme de l’âme proprement dite de la production et pourtant elle ne donne rien au travail et tout à la propriété privée. Proudhon a, en partant de cette contradiction, conclu en faveur du travail contre la propriété privée. Mais nous voyons que cette apparente contradiction est la contradiction du travail aliéné avec lui-même et que l’économie politique n’a exprimé que les lois du travail aliéné.\par
Nous voyons par conséquent que le salaire et la propriété privée sont identiques : car le salaire, dans lequel le produit, l’objet du travail, rémunère le travail lui-même, n’est qu’une conséquence nécessaire de l’aliénation du travail, et dans le salaire le travail n’apparaît pas non plus comme le but en soi, mais comme le serviteur du salaire. Nous développerons ceci plus tard et nous n’en tirons plus pour l’instant que quelques [XXVI] conséquences.\par
Un relèvement du salaire par la force (abstraction faite de toutes les autres difficultés, abstraction faite de ce que, étant une anomalie, il ne pourrait être également maintenu que par la force) ne serait donc rien d’autre qu’une meilleure rétribution des esclaves et n’aurait conquis ni pour l’ouvrier ni pour le travail leur destination et leur dignité humaines.\par
L’égalité \emph{du} salaire elle-même, telle que la revendique Proudhon, ne fait que transformer le rapport de l’ouvrier actuel à son travail en le rapport de tous les hommes au travail. La société est alors conçue comme un capitaliste abstrait.\par
Le salaire est une conséquence directe du travail aliéné et le travail aliéné est la cause directe de la propriété privée. En conséquence la disparition d’un des termes entraîne aussi celle de l’autre.\par
2. De ce rapport du travail aliéné à la propriété privée, il résulte en outre que l’émancipation de la société de la propriété privée, etc., de la servitude, s’exprime sous la forme politique de l’émancipation des ouvriers, non pas comme s’il s’agissait seulement de leur émancipation, mais parce que celle-ci implique l’émancipation universelle de l’homme ; or celle-ci y est incluse parce que tout l’asservissement de l’homme est impliqué dans le rapport de l’ouvrier à la production et que tous les rapports de servitude ne sont que des variantes et des conséquences de ce rapport.\par
De même que du concept de travail aliéné, rendu étranger, nous avons tiré par analyse le concept de propriété privée, de même à l’aide de ces deux facteurs, on peut exposer toutes les catégories de l’économie et, dans chaque catégorie, comme par exemple le trafic, la concurrence, le capital, l’argent, nous ne retrouverons qu’une expression déterminée et développée de ces premières bases.\par
Toutefois, avant de considérer ces formes, cherchons à résoudre deux problèmes :\par
\bigbreak
\noindent 1° Déterminer l’essence générale de la propriété privée telle qu’elle apparaît comme résultat du travail aliéné dans son rapport à la propriété véritablement humaine et sociale.\par
2° Nous avons admis comme un fait l’aliénation du travail, son dessaisissement de soi, et nous avons analysé ce fait. Comment, demandons-nous maintenant, l’homme en vient-il à aliéner son travail, à le rendre étranger ? Comment cette aliénation est-elle fondée dans l’essence du développement humain ? Nous avons déjà fait un grand pas dans la solution de ce problème en transformant la question de l’origine de la propriété privée en celle du rapport du travail aliéné à la marche du développement de l’humanité. Car lorsqu’on parle de la propriété privée, on pense avoir affaire à une chose extérieure à l’homme. Et lorsqu’on parle du travail, on a directement affaire à l’homme lui-même. Cette nouvelle façon de poser la question implique déjà sa solution \footnote{Pour Marx, à ce stade de la formation de sa pensée, ces conclusions sont particulièrement importantes. L’aliénation du travail est un stade nécessaire du développement humain, mais elle a une origine dans l’histoire. La propriété privée est issue de l’aliénation du travail, elle est donc elle aussi historique. Cela signifie qu’elles sont toutes deux des phases du développement de l’humanité qui seront un jour dépassées.}.\par
A propos du point 1. \emph{Essence générale} de la propriété privée et son rapport à la propriété vraiment \emph{humaine.}\par
Le travail aliéné s’est résolu pour nous en deux éléments qui se conditionnent réciproquement ou qui ne sont que des expressions différentes d’un seul et même rapport. L’appropriation apparaît comme aliénation, dessaisissement, et le dessaisissement comme appropriation, l’aliénation comme la vraie accession au droit de cité \footnote{Dans la mesure où l’homme a cherché à s’approprier la nature, il est tombé dans l’aliénation. Cette aliénation, origine de la propriété privée, a été appropriation. L’homme en s’aliénant a développé la richesse de sa nature, de son inonde et il en est au stade où il peut réintégrer de plein droit ce monde qui, pour l’instant, lui est étranger.}.\par
Nous avons considéré l’un des aspects, le travail aliéné par rapport à l’ouvrier lui-même, c’est-à-dire le rapport du travail aliéné à soi-même. Nous avons trouvé comme produit, comme résultat nécessaire de ce rapport, le rapport de propriété \emph{du} non-ouvrier à l’ouvrier et au travail. La propriété privée, expression matérielle résumée du travail aliéné, embrasse les deux rapports, le rapport de l’ouvrier au travail et au Produit de son travail ainsi qu’au non-ouvrier, et le rapport du non-ouvrier à l’ouvrier et au produit du travail de celui-ci.\par
Or, si nous avons vu que, par rapport à l’ouvrier qui s’approprie la nature par le travail, l’appropriation apparaît comme aliénation, l’activité propre comme activité pour un autre et comme activité d’un autre, le processus vital comme sacrifice de la vie, la production de l’objet comme perte de l’objet au profit d’une puissance étrangère, d’un homme étranger, considérons maintenant le rapport avec l’ouvrier, le travail et son objet, de cet homme étranger au travail et à l’ouvrier.\par
Il convient d’abord de remarquer que ce qui apparaît chez l’ouvrier comme activité de dessaisissement, d’aliénation, apparaît chez le non-ouvrier comme état de dessaisissement, d’aliénation \footnote{L’ouvrier, le producteur, s’aliène par son activité sa nature d’homme qui lui devient étrangère. Le non-ouvrier par contre. le capitaliste, qui ne travaille, ne produit pas, est de ce fait même étranger à la nature de l’homme qui est précisément de produire.}.\par
Deuxièmement, que le comportement pratique réel de l’ouvrier dans la production et par rapport à son produit (comme état d’âme) apparaît chez le non-ouvrier qui lui fait face comme comportement théorique.\par
[XXVII] Troisièmement, le non-ouvrier fait contre l’ouvrier tout ce que l’ouvrier fait contre lui-même, mais il ne fait pas à l’égard de soi-même ce qu’il fait contre l’ouvrier.\par
Considérons en détails ces trois rapports.
\section[{Second manuscrit }]{Second manuscrit \protect\footnotemark }\renewcommand{\leftmark}{Second manuscrit }

\footnotetext{Seules les quatre dernières pages du manuscrit, paginées XXXX-XLIII, sont parvenues jusqu’à nous. Les 39 premiers feuillets, qui constituaient probablement la partie la plus importante de l’ouvrage, ont disparu.}
\subsection[{[Opposition du Capital et du Travail. Propriété foncière et Capital]}]{[Opposition du Capital et du Travail. Propriété foncière et Capital]}
\noindent [XXXX] constitue les intérêts de son capital \footnote{Il s’agit très probablement du salaire que Marx considère dans ce passage comme un intérêt de ce capital vivant qu’est l’ouvrier.}. En la personne de l’ouvrier se réalise donc subjectivement le fait que le capital est l’homme qui s’est complètement perdu lui-même, comme dans le capital se réalise objectivement le fait que le travail est l’homme qui s’est complètement perdu lui-même. Mais l’ouvrier a le malheur d’être un capital vivant, qui a \emph{donc} des besoins, et qui, à chaque instant où il ne travaille pas, perd ses intérêts et de ce fait son existence. En tant que capital, la valeur de l’ouvrier monte selon l’offre et la demande et même physiquement on a connu son existence, sa vie, et on la connaît comme une offre de marchandise analogue à celle de toute autre marchandise. L’ouvrier produit le capital, le capital le produit ; il se produit donc lui-même, et l’homme, en, tant qu’ouvrier, en tant que marchandise, est le produit de l’ensemble du mouvement. Pour l’homme qui n’est plus qu’ouvrier – et en tant qu’ouvrier –, ses qualités d’homme ne sont là que dans la mesure où elles sont là pour le capital qui lui est étranger. Mais comme le capital et l’homme sont étrangers l’un à l’autre, donc sont dans un rapport indifférent, extérieur et contingent, ce caractère étranger doit aussi apparaître comme réel. Donc, dès que le capital s’avise – idée nécessaire ou arbitraire – de ne plus être pour l’ouvrier, celui-ci n’existe plus pour lui-même, il n’a pas de travail, donc pas de salaire, et comme il n’a pas d’existence en tant qu’homme mais en tant qu’ouvrier, il peut se faire enterrer, mourir de faim, etc. L’ouvrier n’existe en tant qu’ouvrier que dès qu’il existe pour soi en tant que capital et il n’existe en tant que capital que dès qu’un capital existe pour lui. L’existence du capital est son existence, sa vie, et celui-ci détermine le contenu de sa vie d’une manière qui lui est indifférente. L’économie politique ne connaît donc pas l’ouvrier non-occupé, l’homme du travail, dans la mesure où il se trouve en dehors de cette sphère des rapports de travail. Le coquin, l’escroc, le mendiant, le travailleur qui chôme, qui meurt de faim, qui est misérable et criminel, sont des figures qui n’existent pas pour elle, mais seulement pour d’autres yeux, pour ceux du médecin, du juge, du fossoyeur et du prévôt des mendiants, etc. ; ils sont des fantômes hors de son domaine. Les besoins de l’ouvrier ne sont donc pour elle que le besoin de l’entretenir \emph{pendant le} travail, et de l’entretenir seulement de façon à empêcher que la race des ouvriers ne s’éteigne. Le salaire a donc tout à fait la même signification que l’entretien, le maintien en ordre de marche de tout autre instrument productif, que la consommation du capital en général, dont celui-ci a besoin pour se reproduire avec intérêts, que l’huile que l’on met sur les rouages pour les maintenir en mouvement. Le salaire fait donc partie des frais nécessaires du capital et du capitaliste et ne doit pas dépasser les limites de cette nécessité. C’était donc une attitude tout à fait conséquente que celle des patrons de fabriques anglais qui, avant l’Amendment Bill de 1834 \footnote{Marx : fait très certainement allusion ici à la \emph{New Poor Law} votée en \emph{1834} par le Parlement britannique. Cette loi célèbre, qui créa les \emph{work}houses, modifiait la loi sur le paupérisme qui datait de \emph{1601, 43ᵉ} année du règne d’Elisabeth. C’est sans doute pourquoi il emploie l’expression impropre d’Amendment Bill qui signifie proposition d’amendement.}, déduisaient de son salaire les aumônes publiques que l’ouvrier recevait par l’intermédiaire de la taxe des pauvres et les considéraient comme une partie intégrante de celui-ci.\par
La production ne produit pas l’homme seulement en tant que \emph{marchandise, que} marchandise humaine, l’homme défini comme marchandise, elle le produit, conformément à cette définition, comme un être \emph{déshumanisé aussi} bien intellectuellement que physiquement – immoralité, dégénérescence, abrutissement des ouvriers et des capitalistes. Son produit est la marchandise douée de conscience de soi et d’activité propre… la marchandise humaine…\par
Le grand progrès de Ricardo, Mill, etc., sur Smith et Say, c’est qu’ils déclarent l’existence de l’homme – la productivité humaine plus ou moins grande de la marchandise – indifférente et même nuisible. Le but véritable de la production ne serait pas le nombre des ouvriers qu’un capital entretient, mais la quantité des intérêts qu’il rapporte, la somme des économies annuelles. Ce fut également un grand progrès tout à fait logique de [XLI] l’économie anglaise moderne que – tout en faisant du travail le principe unique de l’économie – elle ait expliqué aussi avec une clarté complète que le salaire et les intérêts du capital sont en raison inverse l’un de l’autre et que, en règle générale, le capitaliste ne pouvait gagner qu’en comprimant le salaire et réciproquement. Ce n’est pas l’exploitation du consommateur, mais le fait pour le capitaliste et l’ouvrier de chercher à s’exploiter réciproquement qui, selon elle, est le rapport normal.\par
Le rapport de la propriété privée implique, d’une façon latente, le rapport de la propriété privée en tant que travail, ainsi que le rapport de celle-ci en tant que capital et la relation réciproque de l’un à l’autre. C’est, d’une part, lit production de l’activité humaine en tant que travail, c’est-à-dire en tant qu’activité tout à fait étrangère à elle-même, à l’homme et à la nature, donc à la conscience et à la manifestation de la vie, l’existence abstraite de l’homme conçu seulement en tant que travailleur, qui peut donc chaque jour être précipité de son néant rempli dans le néant absolu, dans sa non-existence sociale et par conséquent réelle. C’est d’autre part la production de l’objet de l’activité humaine en tant que capital où toute détermination naturelle et sociale de l’objet est \emph{effacée, où} la propriété privée a perdu sa qualité naturelle et sociale (donc a perdu toutes les illusions politiques et mondaines et n’est plus mêlée à aucune situation \emph{apparemment} humaine), où aussi le même capital reste le même dans l’existence naturelle et sociale la plus diverse, où il est tout à fait indifférent a son contenu réel. Cette opposition poussée à son comble constitue nécessairement l’expression dernière, le sommet et la fin de tout le rapport de la propriété privée.\par
En conséquence, c’est encore un haut fait de l’économie anglaise moderne d’avoir défini la rente foncière comme la différence entre les intérêts du sol le plus mauvais affecté à la culture et ceux de la meilleure terre cultivée, d’avoir montré les illusions romantiques du propriétaire foncier – son importance soi-disant sociale et l’identité de son intérêt avec celui de la société, identité qu’Adam Smith affirme encore après les physiocrates – et d’avoir anticipé et préparé le mouvement de la réalité qui transformera le propriétaire foncier en un capitaliste tout à fait ordinaire et prosaïque, simplifiera l’opposition entre capital et travail, la portera à son comble et précipitera ainsi sa suppression. La terre en tant que terre, la rente foncière en tant que rente foncière y ont perdu leur distinction de caste et sont devenues le capital et l’intérêt, qui ne disent rien ou plutôt qui ne parlent qu’argent.\par
La différence entre capital et terre, profit et rente foncière, comme la différence entre eux et le salaire, la différence entre industrie, agriculture, propriété immobilière et mobilière est encore une différence \emph{historique qui} n’est pas fondée sur l’essence même de la chose, un moment qui s’est cristallisé de la naissance et de la formation de l’opposition entre capital et travail. Dans l’industrie, etc., par contraste avec la propriété immobilière, ne s’expriment que la façon de naître et l’opposition dans laquelle l’industrie s’est développée par rapport à l’agriculture. En tant qu’espèce particulière du travail, en tant que différence essentielle importante et embrassant la vie, cette différence ne subsiste que tant que l’industrie (la vie citadine) se constitue face à la propriété rurale (la vie féodale noble) et porte encore en elle le caractère féodal de son contraire dans la forme du monopole, de la jurande, de la guilde, de la corporation, etc. ; à l’intérieur de ces déterminations, le travail a encore un sens apparemment social, il signifie encore la communauté réelle et n’est pas encore devenu indifférent à son contenu, il n’est pas complètement passé à l’Être-pour-soi \footnote{Hegel définit l’Être-pour-soi (Fürsichsein) comme le “retour infini en soi”, la négation de l’Être-autre. L’Être-pour-soi s’abstrait lui-même de tout ce qui n’est pas lui. Hegel parle dans la Phénoménologie de “cette pure abstraction de l’Être-pour-soi”.}, c’est-à-dire à l’abstraction de tout autre être et il n’est donc pas non plus devenu encore le capital affranchi \footnote{En allemand : freigelassen.}.\par
[XLII] Mais le développement nécessaire du travail est l’industrie affranchie, constituée pour elle-même comme industrie, et le capital affranchi. La puissance de l’industrie sur son contraire apparaît aussitôt dans la naissance de l’agriculture en tant qu’industrie réelle, taudis qu’auparavant la propriété foncière laissait l’essentiel du travail au sol et à l’esclave de ce sol à l’aide duquel il se cultivait lui-même. Avec la transformation de l’esclave en ouvrier libre, c’est-à-dire en mercenaire, le seigneur foncier en soi est transformé en un maître d’industrie, en un capitaliste, transformation qui a lieu tout d’abord par le moyen terme du fermier. Mais le fermier est le représentant, le mystère révélé du propriétaire foncier ; ce n’est que par lui qu’il existe économiquement, qu’il existe en tant que propriétaire privé – car la rente de sa terre n’existe que par la concurrence des fermiers. Donc, sous la forme du fermier, le propriétaire foncier s’est déjà essentiellement transformé en capitaliste ordinaire. Et ceci doit aussi s’accomplir dans la réalité, le capitaliste pratiquant l’agriculture – c’est-à-dire le fermier – doit devenir propriétaire foncier ou inversement. Le trafic industriel du fermier est celui du propriétaire foncier, car l’Être du premier pose l’Être du second.\par
Mais ils se souviennent de leurs origines contraires, de leur naissance – le propriétaire foncier connaît le capitaliste comme son esclave présomptueux et affranchi d’hier qui s’est enrichi, et il se voit menacé par lui en tant \emph{que capitaliste –} le capitaliste connaît le propriétaire foncier comme le maître oisif, cruel et égoïste d’hier. Il sait que celui-ci lui porte préjudice en tant que capitaliste, bien qu’il doive à l’industrie toute sa signification sociale actuelle, ses biens et ses plaisirs, il voit en lui le contraire de l’industrie libre et du capital \emph{libre, indépendant} de toute détermination naturelle. Cette opposition est pleine d’amertume et les deux parties se disent réciproquement leurs vérités. On n’a qu’à lire les attaques de la propriété immobilière contre la propriété mobilière et inversement pour se faire un tableau suggestif de leur manque de dignité réciproque. Le propriétaire foncier met l’accent sur la noblesse de naissance de sa propriété, les souvenirs féodaux, les réminiscences, la poésie du souvenir, sa nature enthousiaste, son importance politique, etc., et, dans le langage de l’économie, cela s’exprime ainsi : l’agriculture est seule productive. En même temps il décrit son adversaire comme un \emph{coquin d’argent} sans honneur, sans principes, sans poésie, sans substance, sans rien ; un rusé, faisant commerce de tout, dénigrant tout, trompant, avide et vénal ; un homme porté à la rébellion, qui n’a ni esprit ni cœur, qui est devenu étranger à la communauté et en fait trafic, un usurier, un entremetteur, un esclave, souple, habile à faire le beau, et à berner, un homme sec, qui est à l’origine de la concurrence et par suite du paupérisme et du crime, un homme qui provoque, nourrit et flatte la dissolution de tous les liens sociaux. (Voir entre autres le physiocrate Bergasse que Camille Desmoulins fustige déjà dans son journal : Les \emph{Révolutions de France et de} Brabant \footnote{Les Révolutions de France et de Brabant, par Camille DESMOULINS. Second trimestre, contenant mars, avril et mai. Paris an 1ᵉʳ. No 16, p. 139 sq. ; No 26, p. 520 sq. Cet hebdomadaire, qui parut de novembre 1789 à juillet 1791, était essentiellement une série de pamphlets.}, voir von Vincke, Lancizolle, Haller, Léo, Kosegarten \footnote{ \noindent Voir le théologien bouffi d’orgueil de la vieille école hégélienne, Funke [Die aus der unbeschränkten Teilbarkeit des Grundeigentums hervorgehenden Nachteile, nachgewiesen von G.L.W. Funke. Hamburg und Gotha, 1839, p. 56.], qui, d’après Léo [Studien und Skizzen zu einer Naturlehre des Staates. Halle 1833 1. Abt., p. 102.], racontait les larmes aux yeux comment, lors de l’abolition du servage, un esclave avait refusé de cesser d’être une propriété noble. Voir aussi les Fantaisies patriotiques de Justus Moeser [Justus MOESER : Patriotische Phantasien. Berlin 1775-1778.] qui se distinguent en ceci qu’elles n’abandonnent pas un instant l’horizon borné, bon papa, petit-bourgeois, “pot-au-feu”, ordinaire du philistin, et qu’elles sont pourtant de pures fantaisies. C’est cette contradiction qui les a rendues si attrayantes pour l’âme allemande. (Note de Marx.)
 } et voir surtout Sismondi).\par
La propriété mobilière de son côté montre les merveilles de l’industrie et du mouvement. Elle est l’enfant de l’époque moderne et sa fille légitime ; elle plaint son adversaire comme un esprit faible qui n’est pas éclairé sur sa propre nature (et c’est tout à fait juste), qui voudrait remplacer le capital moral et le travail libre par la violence brutale et immorale et le servage. Elle le décrit comme un Don Quichotte qui, sous l’apparence de la droiture, de l’honnêteté, de l’intérêt général, de la permanence, cache son impossibilité à se mouvoir, son désir cupide du plaisir, l’égocentrisme, l’intérêt particulier, la mauvaise intention. Elle déclare qu’il est un monopoliste rusé ; ses réminiscences, sa poésie, son enthousiasme elle les estompe sous une énumération historique et sarcastique de l’abjection, de la cruauté, de l’avilissement, de la prostitution, de l’infamie, de l’anarchie, de la révolte, dont les châteaux romantiques étaient les officines.\par
[XLIII] La propriété mobilière aurait donné aux peuples la liberté politique, délié les liens de la société civile, réuni les mondes entre eux, créé le commerce ami de l’homme, la morale pure, la culture pleine d’agrément ; au lieu de ses besoins grossiers, elle aurait donné au peuple des besoins civilisés et les moyens de les satisfaire, tandis que le propriétaire foncier – cet accapareur de blé, oisif et seulement gênant – hausserait les prix des moyens de subsistance élémentaire du peuple, obligeant par là le capitaliste à élever le salaire sans pouvoir élever la puissance de production ; il mettrait ainsi obstacle au revenu annuel de la nation, à l’accumulation des capitaux, donc à la possibilité de procurer du travail au peuple et de la richesse au pays pour, en fin de compte, les supprimer complètement ; il amènerait un déclin général et exploiterait en usurier tous les avantages de la civilisation moderne sans faire la moindre chose pour elle et même sans rien céder de ses préjugés féodaux. Enfin, – lui chez qui l’agriculture et la terre elle-même n’existent que comme une source d’argent qu’il a reçue en cadeau, – il n’aurait qu’à regarder son fermier et il devrait dire s’il n’est pas un honnête coquin roué et plein d’imagination qui, dans son cœur et dans la réalité, appartient depuis longtemps à l’industrie libre et au commerce aimable, quoiqu’il y répugne tant et qu’il fasse grand état de souvenirs historiques et de fins morales ou politiques. Tout ce qu’il alléguerait réellement en sa faveur ne serait vrai que pour l’agriculteur (le capitaliste et les journaliers), dont l’ennemi serait bien plutôt le propriétaire foncier ; il apporterait donc des preuves contre lui-même. Sans capital, la propriété foncière serait de la matière inerte et sans valeur. La victoire du capital, victoire digne de la civilisation, serait précisément d’avoir, à la place de la chose morte, découvert et créé le travail humain comme source de la richesse. (CL Paul-Louis Courier, Saint-Simon, Ganilh, Ricardo, Mill, Mac Culloch, Destutt de Tracy et Michel Chevalier.)\par
Du cours réel du développement (à insérer ici) résulte la victoire nécessaire du capitaliste, c’est-à-dire de la propriété privée développée sur la propriété bâtarde non-développée, sur le propriétaire foncier ; de même qu’en général le mouvement doit triompher de l’immobilité, la bassesse ouverte et consciente doit triompher de la bassesse cachée et inconsciente, la cupidité du goût du plaisir, l’égoïsme éclairé, franchement effréné et habile de l’égoïsme superstitieux local, prudent, bonasse, paresseux et fantaisiste. Tout comme l’argent doit triompher de toute autre forme de propriété privée.\par
Les États qui ont quelque soupçon du danger de l’industrie libre achevée, de la morale pure achevée et du commerce \emph{philanthro}pique achevé essaient – mais tout à fait en vain – d’arrêter la capitalisation de la propriété foncière.\par
La propriété foncière, à la différence du capital, est la propriété privée, le capital entaché encore de préjugés locaux et politiques, le capital encore non-achevé qui ne s’est pas encore dégagé entièrement de son enchevêtrement avec le monde pour arriver à lui-même. Au cours de son développement universel, il doit arriver à son expression abstraite, c’est-à-dire pure.\par
Le rapport de la propriété privée est travail, capital et la relation de l’un à l’autre.\par
Le mouvement que ces éléments ont a parcourir est\par
Premièrement : Unité immédiate ou médiate de l’un et de l’autre.\par
Le capital et le travail d’abord encore réunis, puis sans doute séparés et aliénés, mais se haussant et se stimulant réciproquement en tant que conditions positives.\par
[Deuxièmement] : Opposition de l’un et de l’autre.\par
Ils s’excluent réciproquement ; l’ouvrier connaît le capitaliste comme sa non-existence et inversement ; chacun cherche à arracher à l’autre son existence.\par
[Troisièmement] : Opposition de chacun à soi-même. Capital travail accumulé = travail. En tant que travail, se décompose en soi et en ses intérêts comme ceux-ci se décomposent à leur tour en intérêts et en profit. Sacrifice intégral du capitaliste. Il tombe dans la classe ouvrière comme l’ouvrier – mais d’une façon seulement exceptionnelle – devient capitaliste. Travail en tant qu’élément du capital, en tant que ses frais. Donc, le salaire est un sacrifice du capital.\par
Le travail se décompose en soi et en salaire. L’ouvrier lui-même est un capital, une marchandise.\par
Opposition réciproque hostile .
\section[{Troisième manuscrit}]{Troisième manuscrit\protect\footnotemark }\renewcommand{\leftmark}{Troisième manuscrit}

\footnotetext{Le troisième manuscrit est un cahier composé de 17 feuilles in-folio pliées en deux, soit 68 pages, que Marx a paginées lui-même. Toutefois, après la page XXI, Marx écrit XXIII et, après XXIV, il numérote XXVI. Les 23 dernières pages sont vides. Le manuscrit commence par deux addendas à un texte perdu qui constituent les deux premiers chapitres. Au cours de la page XI, immédiatement à la suite de développements économiques, commence la critique de la philosophie de Hegel, entrecoupée de nouvelles considérations économiques. Tout ce qui concernait la philosophie de Hegel a été regroupé en un chapitre, tandis que les parties économiques sont données d’abord sous forme de chapitres séparés. Enfin, à la page XXXIX, commence la préface qui figure maintenant en tête du volume.}
\subsection[{[Propriété privée et travail. Points de vue des mercantilistes, des physiocrates, d’Adam smith, de Ricardo et de son école.]}]{[Propriété privée et travail. Points de vue des mercantilistes, des physiocrates, d’Adam smith, de Ricardo et de son école.]}
\noindent [I] A propos de la page XXXVI.\par
L’essence subjective de la propriété privée, la propriété privée, comme activité étant pour soi, comme sujet, comme personne, est le travail. On comprend donc parfaitement que seule l’économie politique, qui a reconnu le travail pour principe – Adam Smith –, qui ne connaissait donc plus la propriété privée seulement comme un état en dehors de l’homme, que cette économie politique doit être considérée d’une part comme un produit de l’énergie et du mouvement réels de la propriété privée \footnote{Elle est le mouvement indépendant de la propriété privée devenu pour soi dans la conscience, l’industrie moderne en tant que sujet autonome. \emph{(Note de Marx.)}}, comme un produit de l’industrie moderne, et que, d’autre part, elle a accéléré, célébré l’énergie et le développement de cette industrie et en a fait une puissance de la conscience. C’est donc comme des fétichistes, des catholiques qu’apparaissent aux yeux de cette économie politique éclairée, qui a découvert l’essence subjective de la richesse – dans les limites de la propriété privée – les partisans du système monétaire et du mercantilisme qui connaissent la propriété privée comme une essence seulement objective pour l’homme. Engels a donc eu raison d’appeler \emph{Adam Smith le Luther de l’économie politique} \footnote{Esquisse d’une critique de l’économie politique. Cf. MEGA, I, tome II, p. 383.}. De même que Luther reconnaissait la \emph{religion}, la foi comme l’essence du \emph{monde} réel et s’opposait donc au paganisme catholique, de même qu’il abolissait la religiosité \emph{extérieure} en faisant de la religiosité l’essence \emph{intérieure} de l’homme, de même qu’il niait les prêtres existant en dehors du laïque, parce qu’il transférait le prêtre dans le cœur du laïque, de même la richesse qui se trouve en dehors de l’homme et indépendante de lui – qui ne peut donc être conservée et affirmée que d’une manière extérieure – est abolie ; en d’autres termes cette \emph{objectivité extérieure absurde} qui est la sienne est supprimée du fait que la propriété privée s’incorpore dans l’homme lui-même et que celui-ci est reconnu comme son essence ; mais, en conséquence, il est lui-même placé dans la détermination de la propriété privée, comme chez Luther il était placé dans celle de la religion. Sous couleur de reconnaître l’homme, l’économie politique, dont le principe est le travail, ne fait donc au contraire qu’accomplir avec conséquence le reniement de l’homme, car il n’est plus lui-même dans un rapport de tension externe avec l’essence extérieure de la propriété privée, mais il est devenu lui-même cette essence tendue de la propriété privée. Ce qui était autrefois \emph{l’être-extérieur-à-soi}, l’aliénation réelle de l’homme, n’est devenu que l’acte d’aliénation, l’aliénation de soi. Si donc cette économie politique débute en paraissant reconnaître l’homme, son indépendance, son activité propre, etc., et si, quand elle transfère la propriété privée dans l’essence même de l’homme, elle ne peut plus être conditionnée par les \emph{déterminations} locales, nationales, etc. de la \emph{propriété privée} en tant \emph{qu’essence existant en dehors d’elle ;} si donc elle développe une énergie \emph{cosmopolite}, universelle, qui renverse toute barrière et tout lien pour se poser elle-même à la place comme la \emph{seule} politique, la \emph{seule} universalité, la \emph{seule} barrière et le \emph{seul} lien, il faudra en continuant à se développer qu’elle rejette cette \emph{hypocrisie} et apparaisse dans tout son \emph{cynisme} ; et eue le fait – sans se soucier de toutes les contradictions apparentes où l’entraîne cette doctrine – en développant \emph{le travail} d’une façon beaucoup plus \emph{exclusive}, donc plus \emph{nette} et plus \emph{conséquente}, comme \emph{l’essence} unique de la \emph{richesse ; à} l’opposé de cette conception primitive, elle démontre au contraire que les conséquences de cette doctrine sont \emph{hostiles} à l’homme et elle donne, en fin de compte, le coup de grâce à la dernière existence \emph{individuelle}, naturelle, indépendante du mouvement du travail, de la propriété privée et à la source de la richesse – la \emph{rente foncière – cette} expression de la propriété féodale qui est déjà devenue tout à fait économique et qui est donc incapable de résister à l’économie (école de \emph{Ricardo).} Non seulement le \emph{cynisme} de l’économie politique grandit relativement de Smith en passant par Say pour aboutir à Ricardo, Mill, etc., dans la mesure où les conséquences de \emph{l’industrie} apparaissent aux derniers nommés plus développées et plus remplies de contradictions, mais encore, sur le plan positif, ceux-ci vont toujours et consciemment plus loin que celui qui les a précédés dans l’aliénation par rapport à l’homme, et ceci \emph{seulement} parce que leur science se développe avec plus de conséquence et de vérité. Du fait qu’ils font de la propriété privée sous sa forme active le sujet, que du même coup ils font donc de l’homme l’essence (de cet homme qu’ils réduisent à un monstre) \footnote{Marx emploie ici l’expression Unwesen. Le terme est la négation de Wesen qui signifie à la fois essence et être. Nous choisissons de traduire par monstre, ce qu’implique la pensée de Marx, mais qui nous oblige à renoncer à la violente opposition Wesen-Unwesen si caractéristique de son style. Nous ne pensons pas devoir retenir la traduction “quelque chose d’inessentiel” adoptée par l’édition anglaise.}, la contradiction de la réalité correspond pleinement à l’essence emplie de contradictions qu’ils ont reconnue pour principe. La \emph{réalité [II]} déchirée de \emph{l’industrie, loin} de le réfuter, confirme leur principe \emph{déchiré en} soi. Leur principe est en effet le principe de ce déchirement.\par
La doctrine physiocratique du \emph{docteur Quesnay} constitue le passage du mercantilisme à Adam Smith. La \emph{physiocratie} est directement la décomposition \emph{économique} de la propriété féodale, mais elle est de ce fait tout aussi immédiatement la \emph{transformation économique}, la restauration de celle-ci, à ceci près que son langage n’est plus maintenant féodal, mais économique. Toute richesse se résout en \emph{terre} et en \emph{agriculture.} La terre n’est pas encore le \emph{capital}, elle en est encore un mode d’existence \emph{particulier, qui} doit être valable dans sa particularité naturelle et à \emph{cause} d’elle ; mais la terre est cependant un \emph{élément} naturel, général, tandis que le mercantilisme ne reconnaissait que le métal précieux comme existence de la richesse. L’objet de la richesse, sa matière, a donc aussitôt reçu son universalité la plus haute dans le cadre des limites naturelles – dans la mesure où, en tant que nature, elle est aussi la richesse immédiatement objective. Et la terre n’est pour l’homme que par le travail, l’agriculture. Donc l’essence subjective de la richesse est déjà transférée dans le travail. Mais en même temps l’agriculture est le \emph{seul} travail \emph{productif.} Donc, le travail n’est pas encore saisi dans son universalité et son abstraction ; il est encore lié à un \emph{élément naturel} particulier, à \emph{sa matière, il} n’est donc encore reconnu que sous un \emph{mode d’existence particulier déterminé par la nature. Il} est donc seulement une aliénation \emph{déterminée, particulière de l’homme}, de même que son produit n’est encore conçu que comme une richesse déterminée – qui échoit plus encore à la nature qu’à lui-même. La terre est encore reconnue ici comme existence naturelle, indépendante de l’homme, et ne l’est pas encore comme capital, c’est-à-dire comme un moment du travail lui-même. C’est plutôt le travail qui apparaît comme son moment. Mais du fait que le fétichisme de la vieille richesse extérieure existant seulement comme objet est réduit à un élément naturel très simple et que son essence est déjà reconnue d’une manière particulière, si elle ne l’est que partiellement, dans son existence subjective, le progrès nécessaire sera que l’essence générale de la richesse sera reconnue et que, par conséquent, le \emph{travail}, dans son absolu achevé, c’est-à-dire son abstraction, sera érigé en \emph{principe. Il} sera démontré à la physiocratie que \emph{l’agriculture}, du point de vue économique, donc le seul fondé en droit, n’est différente d’aucune autre indus. trie ; que donc ce n’est pas un travail \emph{déterminé}, une extériorisation particulière du travail, hé à un élément particulier, mais \emph{le travail en général} qui est l’essence de la richesse.\par
La physiocratie nie la richesse \emph{particulière} extérieure seulement objective, en déclarant que le travail en est l’essence. Mais tout d’abord le travail n’est pour elle que l’essence \emph{subjective} de la propriété foncière (elle part de l’espèce de propriété qui apparaît historiquement comme l’espèce dominante et reconnue) ; elle fait seulement de la propriété foncière \emph{l’homme aliéné.} Elle abolit son caractère féodal en déclarant que \emph{l’industrie} (l’agriculture) est son essence ; mais elle a une attitude négative à l’égard du monde de l’industrie, elle reconnaît la féodalité en déclarant que \emph{l’agriculture} est la \emph{seule} industrie.\par
Il est évident que dès que l’on saisit \emph{l’essence subjective} de l’industrie qui se constitue en opposition avec la propriété privée, c’est-à-dire comme industrie, cette essence implique ce contraire qui lui est propre. Car de même que l’industrie englobe la propriété foncière abolie, de même son essence \emph{subjective} englobe également l’essence subjective \emph{de celle-ci.}\par
De même que la propriété foncière est la première forme de la propriété privée, que l’industrie ne l’affronte tout d’abord historiquement que comme une espèce particulière de propriété – elle est plutôt l’esclave affranchi de la propriété foncière –, de même ce processus se répète lorsque l’on saisit d’une manière scientifique l’essence \emph{subjective} de la propriété privée, le \emph{travail ;} et celui-ci n’apparaît d’abord que comme \emph{travail agricole}, mais il est ensuite reconnu comme \emph{travail} en général.\par
[III] Toute richesse s’est transformée en richesse \emph{industrielle}, en richesse du \emph{travail}, et l’industrie est le travail achevé, comme le \emph{régime de fabrique} est l’essence développée de \emph{l’industrie}, c’est-à-dire du travail, et \emph{le capital industriel} la forme objective achevée de la propriété privée.\par
Nous voyons comment la propriété privée peut achever maintenant seulement sa domination sur l’homme et, sous sa forme la plus universelle, devenir une puissance historique mondiale.
\subsection[{[Propriété privée et communisme, stades de développement des conceptions communistes. Le communisme grossier et égalitaire. Le communisme en tant que socialisme.]}]{[Propriété privée et communisme, stades de développement des conceptions communistes. Le communisme grossier et égalitaire. Le communisme en tant que socialisme.]}
\noindent A propos de la page XXXIX \footnote{Selon toute vraisemblance, Marx se réfère ici à la page XXXIX du second manuscrit dont seules les quatre dernières pages (XL à XLIII) nous sont parvenues.}.\par
Mais l’opposition entre la \emph{non-propriété et} la propriété est une opposition encore indifférente, qui n’est pas saisie dans sa relation active, dans son rapport \emph{interne, qui} n’est pas encore saisie comme \emph{contradiction, tant} qu’elle n’est pas comprise comme l’opposition du travail et du capital. Même sans le mouvement développé de la propriété privée dans la Rome antique, en Turquie, etc., cette opposition peut s’exprimer sous la première forme. Ainsi elle n’apparaît pas encore comme posée par la propriété privée elle-même. Mais le travail, essence subjective de la propriété privée comme exclusion de la propriété, et le capital, le travail objectif comme exclusion du travail, c’est la propriété privée, forme de cette opposition poussée jusqu’à la contradiction, donc forme énergique qui pousse à la solution de cette contradiction.\par
A propos de la même page. La suppression de l’aliénation de soi suit la même voie que l’aliénation de soi. Tout d’abord la propriété privée n’est considérée que sous son côté objectif – avec cependant le travail pour essence. Sa forme d’existence est donc le capital, qui doit être supprimé “en tant que tel” (Proudhon \footnote{“Tout capital accumulé étant une propriété sociale, nul ne peut en avoir la propriété exclusive.” (PROUDHON, I.c., p. 96.)}). Ou bien le mode particulier du travail, le travail nivelé, morcelé et par suite non libre, est saisi comme la source de la nocivité de la propriété privée et de son existence aliénée à l’homme – Fourier, qui, tout comme les physiocrates, conçoit aussi à son tour le travail agricole tout au moins comme le travail par excellence, tandis que chez Saint-Simon, au contraire, l’essentiel est le travail industriel en tant que tel et qu’il réclame de surcroît la domination exclusive des industriels et l’amélioration de la situation des ouvriers. Le communisme, enfin, est l’expression positive de la propriété privée abolie, et en premier lieu la propriété privée générale. En saisissant ce rapport dans son universalité, le communisme\par
1. n’est sous sa première forme qu’une généralisation et un achèvement de ce rapport ; en tant que rapport achevé, il apparaît sous un double aspect : d’une part la domination de la propriété matérielle est si grande vis-à-vis de lui qu’il veut anéantir tout ce qui n’est pas susceptible d’être possédé par tous comme propriété privée ; il veut faire de force abstraction du talent, etc. La possession physique directe est pour lui l’unique but de la vie et de l’existence ; la catégorie d’ouvrier n’est pas supprimée, mais étendue à tous les hommes ; le rapport de la propriété privée reste le rapport de la communauté au monde des choses. Enfin, ce mouvement qui consiste à opposer à la propriété privée la propriété privée générale s’exprime sous cette forme bestiale qu’au mariage (qui est certes une forme de la propriété privée exclusive) on oppose la communauté des femmes, dans laquelle la femme devient donc une propriété collective et commune. On peut dire que cette idée de la communauté des femmes constitue le secret révélé de ce communisme encore très grossier et très irréfléchi. De même que la femme passe du mariage à la prostitution générale \footnote{La prostitution n’est qu’une expression particulière de la prostitution générale de l’ouvrier et comme la prostitution est un rapport où entrent non seulement le prostitué mais aussi celui qui prostitue – dont l’abjection est plus grande encore – le capitaliste, etc., tombe aussi dans cette catégorie. (Note de Marx.)}, de même tout le monde de la richesse, c’est-à-dire de l’essence objective de l’homme, passe du rapport du mariage exclusif avec le propriétaire privé à celui de la prostitution universelle avec la communauté. Ce communisme – en niant partout la personnalité de l’homme – n’est précisément que l’expression conséquente de la propriété privée, qui est cette négation. L’envie générale et qui se constitue comme puissance est la forme dissimulée que prend la soif de richesse et sous laquelle elle ne fait que se satisfaire d’une autre manière. L’idée de toute propriété privée en tant que telle est tournée tout au moins contre la propriété privée plus riche, sous forme d’envie et de goût de l’égalisation, de sorte que ces derniers constituent me l’essence de la concurrence. Le communisme grossier n’est que l’achèvement de cette envie et de ce nivellement en partant de la représentation d’un minimum. Il a une mesure précise, limitée. A quel point cette abolition de la propriété privée est peu une appropriation réelle, la preuve en est précisément faite par la négation abstraite de tout le monde de la culture et de la civilisation, par le retour à la simplicité [IV] contraire à la nature de l’homme pauvre et sans besoin, qui non seulement n’a pas dépassé le stade de la propriété privée, mais qui n’y est même pas encore parvenu.\par
Cette communauté ne signifie que communauté du travail et égalité du salaire que paie le capital collectif, la communauté en tant que capitaliste général. Les deux aspects du rapport sont élevés à une généralité figurée, le travail devient la détermination dans laquelle chacun est placé, le capital l’universalité et la puissance reconnues de la communauté.\par
Dans le rapport à l’égard de la femme, proie et servante de la volupté collective, s’exprime l’infinie dégradation dans laquelle l’homme existe pour soi-même, car le secret de ce rapport trouve son expression non-équivoque, décisive, manifeste, dévoilée dans le.rapport de l’homme à la femme et dans la manière dont est saisi le rapport générique \footnote{Voir la note vers le fin du Premier Manuscrit rattachée au texte suivant : “L’homme est un être générique”.} naturel et immédiat. Le rapport immédiat, naturel, nécessaire de l’homme à l’homme est le rapport de l’homme à la femme. Dans ce rapport générique naturel, le rapport de l’homme à la nature est immédiatement son rapport à l’homme, de même que le rapport à l’homme est directement son rapport à la nature, sa propre détermination naturelle. Dans ce rapport apparaît donc de façon sensible, réduite à un fait concret la mesure dans laquelle, pour l’homme, l’essence humaine est devenue la nature, ou celle dans laquelle la nature est devenue l’essence humaine de l’homme. En partant de ce rapport, on peut donc juger tout le niveau de culture de l’homme. Du caractère de ce rapport résulte la mesure dans laquelle l’homme est devenu pour lui-même être générique, homme, et s’est saisi comme tel ; le rapport de l’homme à la femme est le rapport le plus naturel de l’homme à l’homme. En celui-ci apparaît donc dans quelle mesure le comportement naturel de l’homme est devenu humain ou dans quelle mesure l’essence humaine est devenue pour lui l’essence naturelle, dans quelle mesure sa nature humaine est devenue pour lui la nature. Dans ce rapport apparaît aussi dans quelle mesure le besoin de l’homme est devenu un besoin humain, donc dans quelle mesure l’homme autre en tant qu’homme est devenu pour lui un besoin, dans quelle mesure, dans son existence la plus individuelle, il est en même temps un être social.\par
La première abolition positive de la propriété privée, le communisme grossier, n’est donc qu’une forme sous laquelle apparaît l’ignominie de la propriété privée qui veut se poser comme la communauté Positive.\par
2. Le communisme a) encore de nature politique, démocratique ou despotique ;\par
b) avec suppression de l’État, mais en même temps encore inachevé et restant sous l’emprise de la propriété privée, c’est-à-dire de l’aliénation de l’homme. Sous ces deux formes, le communisme se connaît déjà comme réintégration ou retour de l’homme en soi, comme abolition de l’aliénation humaine de soi ; mais du fait qu’il n’a pas encore saisi l’essence positive de la propriété privée et qu’il a tout aussi peu compris la nature humaine du besoin, il est encore entravé et contaminé par la propriété privée. Il a certes saisi son concept, mais non encore son essence.\par
3. Le communisme, abolition positive de la propriété privée (elle-même aliénation humaine de soi) et par conséquent appropriation réelle de l’essence humaine par l’homme et pour l’homme ; donc retour total de l’homme pour soi en tant qu’homme social, c’est-à-dire humain, retour conscient et qui s’est opéré en conservant toute la richesse du développement antérieur. Ce communisme en tant que naturalisme \footnote{Il ne s’agit ici ni de naturalisme au sens littéraire, ni du retour à la nature. Marx veut dire que l’homme a retrouvé sa propre nature, qu’il peut développer librement ses forces essentielles sans que l’aliénation pervertisse les effets de cette manifestation de soi, fasse du monde des objets un monde hostile au lieu du prolongement de son être, et finalement aboutisse à la négation de sa nature d’homme.} achevé = humanisme, en tant qu’humanisme achevé = naturalisme ; il est la vraie solution de l’antagonisme entre l’homme et la nature, entre l’homme et l’homme, la vraie solution de la lutte entre existence et essence, entre objectivation et affirmation de soi, entre liberté et nécessité, entre individu et genre. Il est l’énigme résolue de l’histoire et il se connaît comme cette solution.\par
[V] Le mouvement entier de l’histoire est donc, d’une part, l’acte de procréation réel de ce communisme – l’acte de naissance de son existence empirique – et, d’autre part, il est pour sa conscience pensante, le mouvement \emph{compris et connu} de son \emph{devenir.} Par contre, cet autre communisme encore non achevé cherche pour lui une preuve historique dans des formations historiques isolées qui s’opposent à la propriété privée, il cherche une preuve dans ce qui existe, en détachant des moments pris à part du mouvement (Cabet, Villegardelle, etc., ont en particulier enfourché ce dada) et en les fixant pour prouver que, au point de vue historique, il est pur sang ; par là il fait précisément apparaître que la partie incomparablement la plus grande de ce mouvement contredit ses affirmations et que s’il a jamais existé, son Être passé réfute précisément sa prétention à l’essence.\par
Si tout le mouvement révolutionnaire trouve sa base tant empirique que théorique dans le mouvement de la propriété privée, de l’économie, on en comprend aisément la nécessité.\par
Cette propriété privée matérielle, immédiatement sensible, est l’expression matérielle sensible de la vie humaine aliénée. Son mouvement – la production et la consommation – est la révélation sensible du mouvement de toute la production passée, c’est-à-dire qu’il est la réalisation ou la réalité de l’homme. La religion, la famille, l’État, le droit, la morale, la science, l’art, etc., ne sont que des modes particuliers de la production et tombent sous sa loi générale. L’abolition positive de la propriété privée, l’appropriation de la vie humaine, signifie donc la suppression positive de toute aliénation, par conséquent le retour de l’homme hors de la religion, de la famille, de l’État, etc., à son existence humaine, c’est-à-dire sociale. L’aliénation religieuse en tant que telle ne se passe que dans le domaine de la conscience, du for intérieur de l’homme, mais l’aliénation économique est celle de la vie réelle – sa sup. pression embrasse donc l’un et l’autre aspects. Il est évident que chez les différents peuples le mouvement prend sa première origine selon que la véritable vie reconnue du peuple se déroule plus dans la conscience ou dans le monde extérieur, qu’elle est plus la vie idéale ou réelle. Le communisme commence immédiatement (Owen) avec l’athéisme. L’athéisme est au début encore bien loin d’être le communisme, de même que cet athéisme est plutôt encore une abstraction. La philanthropie de l’athéisme n’est donc au début qu’une philanthropie philosophique abstraite, celle du communisme est immédiatement réelle et directement tendue vers l’action (Wirkung).\par
Nous avons vu \footnote{Marx se réfère sans doute ici à un développement qui se trouvait dans le manuscrit perdu.} comment dans l’hypothèse de la propriété privée positivement abolie, l’homme produit l’homme, se produit soi-même et produit l’autre homme ; comment l’objet, qui est le produit de l’activité immédiate de son individualité, est en même temps se propre existence pour l’autre homme, l’existence de celui-ci et l’existence de ce dernier pour lui. Mais, de même, le matériel du travail aussi bien que l’homme en tant que sujet sont tout autant le résultat que le point de départ du mouvement (et la nécessité historique de la propriété privée réside précisément dans le fait qu’ils doivent être ce point de départ). Donc le caractère social est le caractère général de tout le mouvement ; de même que la société \footnote{Marx entend ici par société la société vraie, celle où les hommes ne s’opposeront plus et qui naîtra de l’abolition positive de la propriété privée.} elle-même produit l’homme en tant qu’homme, elle est produite par lui. L’activité et la jouissance tant par leur contenu que par leur genre d’origine sont sociales ; elles sont activité sociale et jouissance sociale. L’essence humaine de la nature n’est là que pour l’homme social ; car c’est seulement dans la société que la nature est pour lui comme lien avec l’homme, comme existence de lui-même pour l’autre et de l’autre pour lui, ainsi que comme élément vital de la réalité humaine ; ce n’est que là qu’elle est pour lui le fondement de sa propre existence humaine. Ce West que là que son existence naturelle est pour lui son existence humaine et que la nature est devenue pour lui l’homme. Donc, la société est l’achèvement de l’unité essentielle de l’homme avec la nature, la vraie résurrection de la nature, le naturalisme accompli de l’homme et l’humanisme accompli de la nature.\par
[VI] L’activité sociale et la jouissance sociale n’existent nullement sous la seule forme d’une activité immédiatement collective et d’une jouissance immédiatement collective, bien que l’activité collective et la jouissance collective, c’est-à-dire l’activité et la jouissance qui s’expriment et se vérifient directement en société réelle avec d’autres hommes, se rencontrent partout où cette expression immédiate de la sociabilité est fondée dans l’essence de leur contenu et appropriée à la nature de celui-ci.\par
Mais même si mon activité est scientifique, etc., et que je puisse rarement m’y livrer en communauté directe avec d’autres, je suis social parce que j’agis en tant qu’homme. Non seulement le matériel de mon activité – comme le langage lui-même grâce auquel le penseur exerce la sienne – m’est donné comme produit social, mais ma propre existence est activité sociale ; l’est en conséquence ce que je fais de moi, ce que je fais de moi pour la société et avec la conscience de moi en tant qu’être social.\par
Ma conscience universelle n’est que la forme \emph{théorique} de \emph{ce} dont la communauté \emph{réelle}, l’organisation sociale est la forme \emph{vivante}, tandis que de nos jours la conscience universelle est une abstraction de la vie réelle et, à ce titre, s’oppose à elle en ennemie. Donc L’acti\emph{vité} de ma conscience universelle – en tant que telle – est aussi mon existence \emph{théorique} en tant qu’être social.\par
Il faut surtout éviter de fixer de nouveau la “société” comme une abstraction en face de l’individu. L’individu \emph{est l’être social.} La manifestation de sa vie – même si elle n’apparaît pas sous la forme immédiate d’une manifestation collective de la vie, accomplie avec d’autres et en même temps qu’eux – est donc une manifestation et une affirmation de \emph{la vie sociale.} La vie individuelle et la vie générique de l’homme ne sont pas \emph{différentes}, malgré que – et ceci nécessairement – le mode d’existence de la vie individuelle soit un mode \emph{plus particulier ou plus général} de la vie générique ou que la vie du genre soit une vie individuelle plus \emph{particulière ou plus générale.}\par
En tant que \emph{conscience générique} l’homme affirme sa \emph{vie sociale} réelle et ne fait que répéter dans la pensée son existence réelle ; de même qu’inversement l’être générique s’affirme dans la conscience générique et qu’il est pour soi, dans son universalité, en tant qu’être pensant.\par
L’homme – à quelque degré \emph{qu’il} soit donc un individu \emph{particulier} et sa particularité en fait précisément un individu et un être social \emph{individuel} réel – est donc tout autant la \emph{totalité}, la totalité idéale, l’existence subjective pour soi de la société pensée et sentie, que dans la réalité il existe soit comme contemplation et jouissance réelle de l’existence sociale soit comme totalité de manifestations humaines de la vie.\par
La pensée et l’Être sont donc certes \emph{distincts}, mais en même temps ils forment ensemble une \emph{unité.}\par
La \emph{mort} apparaît comme une dure victoire du genre sur l’individu \emph{déterminé} et semble contredire leur unité ; mais l’individu déterminé n’est qu’un \emph{être générique déterminé}, et à ce titre mortel.\par
4. De même \emph{que} la propriété privée n’est que l’expression sensible du fait que l’homme devient à la fois \emph{objectif pour lui.} même et en même temps au contraire un objet étranger pour lui-même et non-humain, que la manifestation de sa vie est l’aliénation de sa vie, que sa réalisation est sa privation de réalité, une réalité \emph{étrangère}, de même l’abolition positive de la propriété privée, c’est-à-dire l’appropriation \emph{sensible} pour les hommes et par les hommes de la vie et de l’être humains, des hommes \emph{objectifs}, des œuvres humaines, ne doit pas être saisie seulement dans le sens de la \emph{jouissance immédiate}, exclusive, dans le sens de la possession, de \emph{l’avoir.} L’homme s’approprie son être universel d’une manière universelle, donc en tant qu’homme total. Chacun de ses rapports \emph{humains} avec le monde, la vue, l’ouïe, l’odorat, \emph{le} goût, le toucher, la pensée, la contemplation, le sentiment, la volonté, l’activité, l’amour, bref tous les organes de son individualité, comme les organes qui, dans leur forme, sont immédiatement des organes sociaux, [VII] sont dans leur comportement \emph{objectif ou} dans leur \emph{rapport à l’objet} l’appropriation de celui-ci, l’appropriation de la réalité \emph{humaine ;} leur rapport à l’objet est la \emph{manifestation de la réalité humaine} \footnote{Elle est donc tout aussi multiple que le sont les déterminations essentielles et les activités de l’homme. (Note de Marx.)}\emph{ ; c’est l’activité} humaine et la \emph{souffrance} humaine car, comprise au sens humain, la souffrance est une jouissance que l’homme a de soi.\par
La propriété privée nous a rendus si sots et si bornés qu’un objet n’est \emph{nôtre} que lorsque nous l’avons, qu’ [il] existe donc pour nous comme capital ou qu’il est immédiatement possédé, mangé, bu, porté sur notre corps, habité par nous, etc., bref qu’il est \emph{utilisé} par nous, bien que la propriété privée ne saisisse à son tour toutes ces réalisations directes de la possession elle-même que comme \emph{des moyens de subsistance}, et la vie, à laquelle elles servent de moyens, est la vie de la \emph{propriété privée}, le travail et la capitalisation.\par
A la place de tous les sens physiques et intellectuels est donc apparue la simple aliénation de \emph{tous} ces sens, le sens de \emph{l’avoir.} L’être humain devait être réduit à cette pauvreté absolue, afin d’engendrer sa richesse intérieure en partant de lui-même. (Sur la catégorie de \emph{l’Avoir cf. Hess} dans les 21 Feuilles \footnote{ \noindent Marx fait sans doute allusion ici au passage suivant de l’article de Hess intitulé : “Philosophie de l’action” dans les 21 Feuilles :\par
 “La propriété matérielle est l’être pour soi de l’esprit devenu idée fixe. Comme il ne saisit pas par la pensée le travail, la manifestation extérieure de soi par le travail, comme son acte libre, comme sa vie propre, mais qu’il le saisit comme quelque chose de matériellement différent il doit aussi le garder pour lui, pour ne pas se perdre dans l’infinité, pour arriver à son être-pour-soi. Mais la propriété cesse d’être pour l’esprit ce qu’elle doit être, à savoir son être pour soi, si ce qui est saisi et maintenu à deux mains comme l’être pour soi de l’esprit, ce n’est pas l’acte dans la création, mais le résultat, la chose créée – si c’est l’ombre, la représentation de l’esprit qui est saisie comme son concept, bref si c’est son être autre qui est saisi comme son être-pour-soi. C’est précisément la soif d’être, c’est-à-dire la soif de subsister comme individualité déterminée, comme moi borné, comme être fini qui conduit à la soif d’avoir. Ce sont à leur tour la négation de toute détermination, le moi abstrait et le communisme abstrait, la conséquence de la “chose en soi” vide, du criticisme et de la révolution, du devoir insatisfait qui ont conduit à l’être et à l’avoir.” (Moses Hess : S\emph{ozialistische Aufsätze}, édités par Zlocisti, Bertin 1921, P. 58-59).\par
 Marx a lui-même traité de cette catégorie de l’avoir dans \emph{La} Sainte Famille. Cf. MEGA, I, 3, p. 212.
}.)\par
L’abolition de la propriété privée est donc \emph{l’émancipation} totale de tous les sens et de toutes les qualités humaines ; mais elle est cette émancipation précisément parce que ces sens et ces qualités sont devenus \emph{humains}, tant subjectivement qu’objectivement. L’œil est devenu l’œil \emph{humain} de la même façon que son \emph{objet} est devenu un objet social, \emph{humain}, venant de l’homme et destiné à l’homme. Les \emph{sens} sont donc devenus directement dans leur praxis des \emph{théoriciens.} Ils se rapportent à \emph{la chose} pour la chose, mais la chose elle-même cet un rapport \emph{humain objectif} à elle-même et à l’homme \footnote{Je ne puis me rapporter humainement à la chose que si la chose se rapporte humainement à l’homme. \emph{(Note de Marx.)}} et inversement. Le besoin on la jouissance ont perdu de ce fait leur nature \emph{égoïste} et la nature a perdu sa simple utilité, car l’utilité est devenue l’utilité \emph{humaine.}\par
De même les sens et la jouissance des autres hommes sont devenus mon appropriation à \emph{moi.} En dehors de ces organes immédiats se constituent donc des organes \emph{sociaux sous} la \emph{forme} de la société ; ainsi, par exemple, l’activité directement en société avec d’autres, etc. est devenue un organe de la \emph{manifestation} de \emph{ma vie} et un mode d’appropriation de la vie \emph{humaine.}\par
Il va de soi que l’œil \emph{humain} jouit autrement que l’œil grossier non-humain ; l’oreille humaine autrement que l’oreille grossière, etc.\par
Ainsi que nous l’avons vu, l’homme ne se perd pas dans son objet à la seule condition que celui-ci devienne pour lui objet \emph{humain} ou homme objectif. Cela n’est possible que lorsque l’objet devient pour lui un objet \emph{social}, que s’il devient lui-même pour soi un être social, comme la société devient pour lui être dans cet objet.\par
Donc, d’une part, à mesure que partout dans la société la réalité objective devient pour l’homme la réalité des forces humaines essentielles, la réalité humaine et par conséquent la réalité de ses \emph{propres} forces essentielles, tous les \emph{objets} deviennent pour lui \emph{l’objectivation} de lui-même, les objets qui confirment et réalisent son individualité, \emph{ses} objets, c’est-à-dire qu’il devient \emph{lui-même} objet. \emph{De quelle manière ils} deviennent siens, cela dépend de la \emph{nature} de \emph{l’objet} et de la nature de la \emph{force essentielle qui} correspond à \emph{celle-ci ; car} c’est précisément la \emph{détermination} de ce rapport qui constitue le mode particulier, \emph{réel}, d’affirmation. Pour l’œil un objet est perçu autrement que pour \emph{l’oreille} et l’objet de]'œil \emph{est} un autre que celui de \emph{l’oreille.} La particularité de chaque force essentielle est précisément son \emph{essence particulière}, donc aussi le mode particulier de son objectivation, de son \emph{Être objectif, réel}, vivant. Non seulement dans la pensée (VIII] mais avec \emph{tous} les sens, l’homme s’affirme donc dans le monde objectif \footnote{On lit dans le premier chapitre de L’Essence du christianisme de Feuerbach, “C’est donc au contact de son objet que l’homme devient conscient de lui-même : la conscience de l’objet est la conscience de soi de l’homme. C’est à son objet que tu connais l’homme ; c’est en lui que l’apparaît son essence : l’objet est son essence révélée, son moi vrai et objectif. Et loin d’être vrai des seuls objets spirituels, ceci s’applique aussi et même aux objets sensibles. Parce qu’ils sont ses objets, et selon le sens où ils le sont, les objets les plus éloignés de l’homme sont, eux aussi, des révélations de l’essence humaine.” (loc. cit., p. 62).}.\par
\bigbreak
\noindent D’autre part, en prenant les choses subjectivement c’est d’abord la musique qui éveille le sens musical de l’homme pour l’oreille qui n’est pas musicienne, la musique la plus belle n’a \emph{aucun} sens \footnote{Feuerbach : “Si tu n’as sens ni sentiment musical, tu ne percevras rien de plus dans la plus belle des musiques que dans le vent qui siffle à tes oreilles ou dans le torrent qui mugit à tes pieds.” (Ibidem, p. 66.)} [n’] est [pas] un objet, car mon objet ne peut être que la confirmation d’une de mes forces essentielles, il ne peut donc être pour moi que tel que ma force essentielle est pour soi en tant que faculté subjective, car le sens d’un objet pour moi (il n’a de signification que pour un sens qui lui correspond) s’étend exactement aussi loin que s’étend \emph{mon} sens \footnote{Feuerbach : “Ton être s’étend aussi loin que ta vue, et inversement.” (Ibid.)}. Voilà pourquoi les \emph{sens} de l’homme social sont \emph{autres} que ceux de l’homme non-social ; c’est seulement grâce à la richesse déployée objectivement de l’essence humaine que la richesse de la faculté subjective de sentir \emph{de l’homme} est tout d’abord soit développée, soit produite, qu’une oreille devient musicienne, qu’un œil perçoit la beauté de la forme, bref que les \emph{sens} deviennent capables de jouissance humaine, deviennent des sens qui s’affirment comme des forces essentielles \emph{de l’homme.} Car \emph{non} seulement les cinq sens, mais aussi les sens dits spirituels, les sens pratiques (volonté, amour, etc.), en un mot le sens humain, l’humanité des sens, ne se forment que grâce à l’existence de leur objet, à la nature humanisée. La formation des cinq sens est le travail de toute l’histoire passée.\par
Le sens qui est encore prisonnier du besoin pratique grossier n’a qu’une signification limitée.\par
Pour l’homme qui meurt de faim, la forme humaine de l’aliment n’existe pas, mais seulement son existence abstraite en tant qu’aliment ; il pourrait tout aussi bien se trouver sous sa forme la plus grossière et on ne peut dire en quoi cette activité nutritive se distinguerait de l’activité nutritive animale. L’homme qui est dans le souci et le besoin n’a pas de sens pour le plus beau spectacle ; celui qui fait commerce de minéraux ne voit que la valeur mercantile, mais non la beauté ou la nature propre du minéral ; il n’a pas le sens minéralogique. Donc l’objectivation de l’essence humaine, tant au point de vue théorique que pratique, est nécessaire aussi bien pour rendre humain le sens de l’homme que pour créer le sens humain qui correspond à toute la richesse de l’essence de l’homme et de la nature.\par
De même que par le mouvement de la propriété privée et de sa richesse comme de sa misère – de la richesse et de la misère matérielles et spirituelles – la société qui prend naissance trouve tout le matériel nécessaire à cette formation, de même la société constituée produit comme sa réalité constante l’homme avec toute cette richesse de son être, l’homme riche, l’homme doué de sens universels et profondément développés.\par
On voit comment le subjectivisme et l’objectivisme, le spiritualisme et le matérialisme, l’activité et la passivité ne perdent leur opposition, et par suite leur existence en tant que contraires de ce genre, que dans l’état de société ;\par
on voit comment la solution des oppositions théoriques elles-mêmes n’est possible que d’une manière pratique, par l’énergie pratique des hommes, et que leur solution n’est donc aucunement la tâche de la seule connaissance, mais une tâche vitale réelle que la philosophie n’a pu résoudre parce qu’elle l’a précisément conçue comme une tâche seulement théorique…\par
On voit comment l’histoire de l’industrie et l’existence objective constituée de l’industrie sont le livre ouvert des forces humaines essentielles, la psychologie de l’homme concrètement présente, que jusqu’à présent on ne concevait pas dans sa connexion avec l’essence de l’homme, mais toujours uniquement du point de vue de quelque relation extérieure d’utilité, parce que – comme on se mouvait à l’intérieur de l’aliénation – on ne pouvait concevoir, comme réalité de ses forces essentielles et comme activité générique humaine, que l’existence universelle de l’homme, la religion, ou l’histoire dans son essence abstraite universelle (politique, art, littérature, etc.). [IX] Dans l’industrie matérielle courante (- on peut tout aussi bien la concevoir comme une partie du mouvement général en question, que l’on peut concevoir ce mouvement lui-même comme une partie particulière de l’industrie, puisque toute activité humaine a été jusqu’ici travail, donc industrie, activité aliénée à soi-même -), nous avons devant nous, sous forme d’objets concrets, étrangers, utiles, sous la forme de l’aliénation, les forces essentielles de l’homme objectivées. Une psychologie pour laquelle reste fermé ce livre, c’est-à-dire précisément la partie la plus concrètement présente, la plus accessible de l’histoire, ne peut devenir une science réelle et vraiment riche de contenu.\par
Que penser somme toute d’une science qui en se donnant de grands airs fait abstraction de cette grande partie du travail humain et qui n’a pas le sentiment de ses lacunes tant que toute cette richesse déployée de l’activité humaine ne lui dit rien, sinon peut-être ce que l’on peut dire d’un mot : “besoin”, “besoin vulgaire” ?\par
Les sciences de la nature ont déployé une énorme activité et ont fait leur un matériel qui va grandissant. Cependant, la philosophie leur est restée tout aussi étrangère qu’elles sont restées étrangères à la philosophie. Leur union momentanée n’était qu’une illusion de l’imagination \footnote{Marx pense ici à la philosophie de la nature de Hegel, sur laquelle il reviendra d’ailleurs dans le dernier chapitre.}. La volonté était là, mais les capacités manquaient. Les historiens eux-mêmes ne se réfèrent aux sciences de la nature qu’en passant, comme à un moment du développement des lumières, d’utilité, qu’illustrent quelques grandes découvertes. Mais par le moyen de l’industrie, les sciences de la nature sont intervenues d’autant plus pratiquement dans la vie humaine et l’ont transformée et ont préparé l’émancipation humaine, bien qu’elles aient dû parachever directement la déshumanisation. \emph{L’industrie} est le rapport historique \emph{réel} de la nature, et par suite des sciences de la nature, avec l’homme ; si donc on la saisit comme une révélation \emph{exotérique} des \emph{forces essentielles} de l’homme, on comprend aussi l’essence \emph{humaine} de la nature ou l’essence naturelle de l’homme ; en conséquence les sciences de la nature perdront leur orientation abstraitement matérielle ou plutôt idéaliste et deviendront la base de la science \emph{humaine}, comme elles sont déjà devenues – quoique sous une forme aliénée – la base de la vie réellement humaine ; dire qu’il y a \emph{une} base pour la vie et une autre pour la \emph{science} est de prime abord un mensonge.\par
La nature en devenir dans l’histoire humaine – acte de naissance de la société humaine – est la naturelle \emph{réelle} de l’homme, donc la nature telle que l’industrie la fait, quoique sous une forme \emph{aliénée}, est la nature \emph{anthropologique} véritable.\par
Le \emph{monde sensible (cf.} Feuerbach) doit être la base de toute science \footnote{Le terme de \emph{Sinnlichkeit} que nous traduisons ici par \emph{monde sensible} est employé chez Feuerbach dans des sens différents. Nous ne pensons pas cependant qu’il s’agisse ici de la sensibilité \emph{(sense-perception)} comme l’entend la traduction anglaise. S’opposant à la philosophie spéculative qui va “de l’abstrait au concret, de l’idéal au réel” et ne parvient jamais qu’à “la réalisation de ses propres abstractions”, Feuerbach réclame que la philosophie prenne pour point de départ le réel. Il écrit dans les \emph{Thèses provisoires pour la Réforme de la philosophie} (No 65) : “Toutes les sciences doivent se fonder sur la \emph{nature.} Tant qu’elle n’a pas trouvé sa \emph{base naturelle}, une théorie n’est qu’une \emph{hypothèse.” (loc. cil.}, p. 125).}. Ce n’est que s’il part de celle-ci sous la double forme et de la conscience \emph{sensible} et du besoin \emph{concret –} donc si la science part de la nature – qu’elle est science \emph{réelle}, L’histoire entière a servi à préparer (à développer) \footnote{Dans le manuscrit de Marx, les deux termes \emph{(Vorbereitungs – Entwicklungs-)} sont écrite l’un au-dessus de l’autre.} la transformation de \emph{“l’homme” en} objet de la conscience \emph{sensible} et du besoin de “l’homme en tant qu’homme” en besoin [naturel concret]. L’histoire elle-même est une partie \emph{réelle} de \emph{l’histoire de la nature}, de la transformation de la nature en homme. Les sciences de la nature comprendront plus tard aussi bien la science de l’homme, que la science de l’homme englobera les sciences de la nature : il y aura une \emph{seule science.}\par
[X] \emph{L’homme} est l’objet immédiat des sciences de la nature \footnote{Tout ce développement repose sur l’idée que c’est l’objet d’un être qui révèle son essence. Dans les \emph{Principes de la philosophie de l’avenir}, Feuerbach écrit (No 7) : “Or c’est à son \emph{objet} qu’on reconnaît la \emph{nature} d’un être ; l’objet auquel se rapporte nécessairement un être n’est rien d’autre que la \emph{révélation} de son essence.” (loc. \emph{cit., pp.} 132-133). Il ajoute plus loin : “Seule des êtres de même rang sont objets les uns pour les autres, et ils le sont tels qu’ils sont \emph{en soi.” (p}. 134.)}; car la \emph{nature sensible} immédiate pour l’homme est directement le monde sensible humain (expression identique) ; elle est immédiatement l’homme \emph{autre} qui existe concrètement pour lui ; car son propre monde sensible n’est que grâce à \emph{l’autre} homme monde sensible humain pour lui-même. Mais la \emph{nature} est l’objet immédiat de la \emph{science de l’homme.} Le premier objet de l’homme – l’homme – est nature, monde sensible, et les forces essentielles particulières et concrètes de l’homme, ne trouvant leur réalisation objective que dans les objets \emph{naturels}, ne peuvent parvenir à la connaissance de soi que dans la science de la nature en général. L’élément de la pensée elle-même, l’élément de la manifestation vitale de la pensée, le \emph{langage} est de nature concrète. La réalité \emph{sociale} de la nature et les sciences naturelles \emph{humaines ou} les \emph{sciences naturelles de l’homme} sont des expressions identiques.\par
On voit comment \emph{l’homme riche} et le besoin \emph{humain} riche prennent la place de la \emph{richesse} et de la \emph{misère} de l’économie politique. L’homme \emph{riche} est en même temps l’homme qui \emph{a besoin} d’une totalité de manifestation vitale humaine. L’homme chez qui sa propre réalisation existe comme nécessité intérieure, comme \emph{besoin.} Non seulement la \emph{richesse}, mais aussi la \emph{pauvreté} de l’homme reçoivent également – sous le socialisme – une signification \emph{humaine} et par conséquent sociale. Elle est le lien passif qui fait ressentir aux hommes comme un besoin la richesse la plus grande, \emph{l’autre} homme. La dénomination de l’essence objective en moi, l’explosion sensible de mon activité essentielle est la \emph{passion}, qui devient par là \emph{l’activité} de mon être \footnote{ \noindent On peut rapprocher de ce passage la thèse provisoire suivante de Feuerbach (no 43) : “Sans liberté, temps, ni souffrance, il n’est non plus ni qualité, ni énergie, ni esprit, ni flamme, ni amour. Seul l’être nécessiteux est l’être nécessaire. Une existence sans besoin est une existence superflue. Celui qui est dépourvu de tout besoin en général n’éprouve pas non plus le besoin d’exister. Qu’il soit ou ne soit pas, c’est tout un, tout un pour lui, tout un pour autrui. Un être sans souffrance est un être \emph{sans fondement. Seul} mérite d’exister celui qui peut souffrir. Seul l’être douloureux est un être divin. Un être sans affection est un être sans être. Un être sans affection n’est rien d’autre qu’un être sans sensibilité, sans matière.” (loc. cit., p. 115.)\par
 On mesurera mieux l’écart entre la pensée de Marx et celle de Feuerbach.
}.\par
5º Un \emph{être} ne commence à se tenir pour indépendant que dès qu’il est son propre maître, et il n’est son propre maître que lorsqu’il doit son \emph{existence à} soi-même. Un homme qui vit de la grâce d’un autre se considère comme un être dépendant. Mais je vis entièrement de la grâce d’un autre, si non seulement je lui dois l’entretien de ma vie, mais encore si en outre il a \emph{créé} ma \emph{vie, s’il} en est la \emph{source}, et ma vie a nécessairement un semblable fondement en dehors d’elle si elle n’est pas ma propre création. C’est pourquoi la \emph{création} est une idée très difficile à chasser de la conscience populaire. Le fait que la nature et l’homme sont par eux-mêmes lui est \emph{incompréhensible}, parce qu’il contredit toutes les \emph{évidences} de la vie pratique.\par
La création de la terre a été puissamment ébranlée par la géognosie, c’est-à-dire par la science qui représente la formation du globe, le devenir de la terre, comme un processus, un auto-engendrement. La génération spontanée est la seule réfutation pratique de la théorie de la création.\par
Or, il est certes facile de dire à l’individu isolé ce qu’Aristote dit déjà : “Tu es engendré par ton père et ta mère, c’est donc l’accouplement de deux hommes, c’est donc un acte générique des hommes qui a produit en toi l’homme. Tu vois donc que même physiquement l’homme doit sa vie à l’homme. Tu ne dois par conséquent pas garder la vue fixée sur un aspect seulement, sur la progression à l’infini à propos de laquelle tu continues à poser des questions : qui a engendré mon père, qui a engendré son grand-père ?…, etc. Tu dois aussi garder la vue fixée sur le mouvement cyclique qui est concrètement visible dans cette progression et qui fait que l’homme dans la procréation se répète lui-même, donc que l’homme reste toujours sujet. Mais tu répondras : si je t’accorde ce mouvement cyclique, accorde-moi la progression qui me fait remonter de plus en plus haut jusqu’à ce que je pose la question : qui a engendré le premier homme et la nature en général ? Je ne puis que te répondre : ta question est elle-même un produit de l’abstraction. Demande-toi comment tu en arrives à cette question ; demande-toi si ta question n’est pas posée en partant d’un point de vue auquel je ne puis répondre parce qu’il est absurde ? Demande-toi si cette progression existe en tant que telle pour une pensée raisonnable ? Si tu poses la question de la création de la nature et de l’homme, tu fais donc abstraction de l’homme et de la nature. Tu les poses comme n’existant pas et tu veux pourtant que je te démontre qu’ils existent. Je te dis alors : abandonne ton abstraction et tu abandonneras aussi ta question, ou bien si tu veux t’en tenir à ton abstraction, sois conséquent, et si, bien que tu penses l’homme et la nature comme n’étant pas [XII tu penses tout de même, alors pense-toi toi-même comme n’étant pas, puisqu’aussi bien tu es nature et homme. Ne pense pas, ne m’interroge pas, car dès que tu penses et que tu m’interroges, ta façon de faire abstraction de l’être de la nature et de l’homme n’a aucun sens. Ou bien es-tu à ce point égoïste que tu poses tout comme néant et que tu veuilles être toi-même ?\par
Tu peux me répliquer : je ne veux pas poser le néant de la nature, etc. ; je te pose la question de l’acte de sa naissance comme j’interroge l’anatomiste sur les formations osseuses, etc.\par
Mais, pour l’homme socialiste, tout ce qu’on appelle l’histoire universelle n’est rien d’autre que l’engendrement de l’homme par le travail humain, que le devenir de la nature pour l’homme ; il a donc la preuve évidente et irréfutable de son engendrement par lui-même, du processus de sa naissance. Si la réalité essentielle de l’homme et de la nature, si l’homme qui est pour l’homme l’existence, de la nature et la nature qui est pour l’homme l’existence de l’homme sont devenus un fait, quelque chose de concret, d’évident, la question d’un être étranger, d’un être placé au-dessus de la nature et de l’homme est devenue pratiquement impossible – cette question impliquant l’aveu de l’inessentialité de la nature et de l’homme. L’athéisme, dans la mesure où il nie cette, chose secondaire, n’a plus de sens, car l’athéisme est une négation de Dieu et par cette négation il pose l’existence de l’homme ; mais le socialisme en tant que socialisme n’a plus besoin de ce moyen terme. Il part de la conscience théoriquement et pratiquement sensible de l’homme et de la nature comme de l’essence. Il est la conscience de soi positive de l’homme, qui n’est plus par le moyen terme de l’abolition de la religion, comme la vie réelle est la réalité positive de l’homme qui n’est plus par le moyen terme de l’abolition de la propriété privée, le communisme. Le communisme pose le positif comme négation de la négation, il est donc le moment réel de l’émancipation et de la reprise de soi de l’homme, le moment nécessaire pour le développement à venir de l’histoire. Le communisme est la forme nécessaire et le principe énergétique du futur prochain, mais le communisme n’est pas en tant que tel le but du développement humain, – la forme de la société humaine.
\subsection[{[Signification des besoins humains dans le régime de la propriété privée et sous le socialisme. Différence entre la richesse dissipatrice et la richesse industrielle, division du travail dans la société bourgeoise.]}]{[Signification des besoins humains dans le régime de la propriété privée et sous le socialisme. Différence entre la richesse dissipatrice et la richesse industrielle, division du travail dans la société bourgeoise.]}
\noindent [XIV] 7º Nous avons vu quelle signification prend sous le socialisme la richesse des besoins humains et, par suite, quelle signification prennent un nouveau mode de production et un nouvel objet de la production : c’est une manifestation nouvelle de la force essentielle de l’homme et un enrichissement nouveau de l’essence \emph{humaine.} Dans le cadre de la propriété privée, les choses prennent une signification inverse. Tout homme s’applique à créer pour l’autre un besoin nouveau pour le contraindre à un nouveau sacrifice, le placer dans une nouvelle dépendance et le pousser à un nouveau mode de jouissance et, par suite, de ruine économique. Chacun cherche à créer une force essentielle étrangère dominant les autres hommes pour y trouver la satisfaction de son propre besoin égoïste. Avec la masse des objets augmente donc l’empire des êtres étrangers auquel l’homme est soumis et tout produit nouveau renforce encore la tromperie réciproque et le pillage mutuel. L’homme devient d’autant plus pauvre en tant qu’homme, il a d’autant plus besoin d’argent pour se rendre maître de l’être hostile, et la puissance de son argent tombe exactement en raison inverse du volume de la production, c’est-à-dire que son indigence augmente à mesure que croît la puissance de l’argent. – Le besoin d’argent est donc le vrai besoin produit par l’économie politique et l’unique besoin qu’elle produit. La quantité de l’argent devient de plus en plus l’unique et puissante propriété de celui-ci ; de même qu’il réduit tout être à son abstraction, il se réduit lui-même dans son propre mouvement à un être quantitatif. L’absence de mesure et la démesure deviennent sa véritable mesure.\par
— Sur le plan subjectif même cela se manifeste d’une part en ceci, que l’extension des produits et des besoins devient l’esclave inventif et toujours en train de calculer d’appétits inhumains, raffinés, contre nature et imaginaires – la propriété privée ne sait pas transformer le besoin grossier en besoin humain ; son idéalisme est l’imagination, l’arbitraire, le caprice et un eunuque ne flatte pas avec plus de bassesse son despote et ne cherche pas à exciter ses facultés émoussées de jouissance pour capter une faveur avec des moyens plus infâmes que l’eunuque industriel, le producteur, pour capter les pièces blanches et tirer les picaillons de la poche de son voisin très chrétiennement aimé. – (Tout produit est un appât avec lequel on tâche d’attirer à soi l’être de l’autre, son argent ; tout besoin réel ou possible est une faiblesse qui attirera la mouche dans la glu ; – exploitation universelle de l’essence sociale de l’homme, de même que chacune de ses imperfections est un lien avec le ciel, un côté par lequel son cœur est accessible au prêtre ; tout besoin est une occasion pour s’approcher du voisin avec l’air le plus aimable et lui dire : cher ami, je te donnerai ce qui t’est nécessaire ; mais tu connais la condition sine qua non ; tu sais de quelle encre tu dois signer le pacte qui te lie à moi ; je t’étrille en te procurant une jouissance). L’eunuque industriel se plie aux caprices les plus infâmes de l’homme, joue l’entremetteur entre son besoin et lui, excite en lui des appétits morbides, guette chacune de ses faiblesses pour lui demander ensuite le salaire de ces bons offices.\par
— Cette aliénation apparaît d’autre part en produisant, d’un côté, le raffinement des besoins et des moyens de les satisfaire, de l’autre le retour à une sauvagerie bestiale, la simplicité complète, grossière et abstraite du besoin ; ou plutôt elle ne fait que s’engendrer à nouveau elle-même avec sa signification opposée. Même le besoin de grand air cesse d’être un besoin pour l’ouvrier ; l’homme retourne à sa tanière, mais elle est maintenant empestée par le souffle pestilentiel et méphitique de la civilisation et il ne l’habite plus que d’une façon précaire, comme une puissance étrangère qui peut chaque jour se dérober à lui, dont il peut chaque jour être [XV]. expulsé s’il ne paie pas. Cette maison de mort, il faut qu’il la paie. La maison de lumière, que, dans Eschyle, Prométhée désigne comme l’un des plus grands cadeaux qui lui ait permis de transformer le sauvage en homme, cesse d’être pour l’ouvrier. La lumière, l’air, etc., ou la propreté animale la plus élémentaire cessent d’être un besoin pour l’homme. La saleté, cette stagnation, cette putréfaction de l’homme, ce cloaque (au sens littéral) de la civilisation devient son \emph{élément de vie.} L’incurie complète et \emph{contre nature}, la nature putride devient \emph{l’élément de sa vie. Aucun} de ses sens n’existe plus, non seulement sous son aspect humain, mais aussi sous son aspect \emph{inhumain}, c’est-à-dire pire qu’animal. On voit revenir les \emph{modes} (et \emph{instruments)} les plus grossiers du travail humain : la \emph{meule} \footnote{Pour punir \emph{les} esclaves romains, on les condamnait à faire tourner la meule d’un moulin.} des esclaves romains est devenue le mode de production, le mode d’existence pour beaucoup d’ouvriers anglais. Il n’est pas assez que l’homme n’ait pas de besoins humains, même les besoins \emph{animaux} cessent. L’Irlandais ne connaît plus que le besoin de \emph{manger}, et, qui plus est, seulement de \emph{manger des pommes de terre}, et même des \emph{pommes de terre à cochon}, celle de la pire espèce. Mais l’Angleterre et la France ont déjà dans chaque ville industrielle une \emph{petite} Irlande. Le sauvage, l’animal ressentent pourtant le besoin de la chasse, du mouvement, etc., de la société. – La simplification de la machine, du travail est utilisée pour transformer en ouvrier l’homme qui en est encore au stade de la formation, l’homme qui n’est encore absolument pas développé – l’enfant –, tandis que l’ouvrier est devenu un enfant laissé à l’abandon. La machine s’adapte à la \emph{faiblesse} de l’homme pour transformer l’homme \emph{faible} en machine. -\par
De quelle manière l’augmentation des besoins et des moyens de les satisfaire engendre-t-elle l’absence de besoins et de moyens ? L’économiste (et le capitaliste : en général nous parlons toujours des hommes d’affaires \emph{empiriques} lorsque nous recourons aux économistes… qui sont leur mea culpa et leur existence \emph{scientifiques)} le prouve ainsi : 1º il réduit le besoin de l’ouvrier à l’entretien le plus indispensable et le plus misérable de la vie physique et son activité au mouvement mécanique le plus abstrait, et dit en conséquence l’homme n’a pas d’autre besoin ni d’activité, ni de jouissance car \emph{même} cette vie-là, il la proclame vie et existence \emph{humaines} 2º\emph{ il calcule} la vie (l’existence) la plus \emph{indigente} possible comme norme et, qui plus est, comme norme universelle : universelle parce que valable pour la masse des hommes ; il fait de l’ouvrier un être privé de sens et de besoins, comme il fait de son activité une pure abstraction de toute activité ; tout \emph{luxe} de l’ouvrier lui apparaît donc condamnable et tout ce qui dépasse le besoin le plus abstrait – fût-ce comme jouissance passive ou manifestation d’activité – lui semble un luxe. L’économie politique, cette science de la \emph{richesse}, est donc en même temps la science du renoncement, des privations, de \emph{l’épargne}, et elle en arrive réellement à \emph{épargner} à l’homme même le \emph{besoin d’air} pur ou \emph{de mouvement} physique. Cette science de la merveilleuse industrie est aussi la science de \emph{l’ascétisme} et son véritable idéal est l’avare \emph{ascétique}, mais \emph{usurier}, et l’esclave \emph{ascétique}, mais \emph{producteur.} Son idéal moral est \emph{l’ouvrier} qui porte à la Caisse d’Épargne une partie de son salaire et, pour cette lubie favorite qui est la sienne, elle a même trouvé un \emph{art} servile. On a porté cela avec beaucoup de sentiment au théâtre. Elle est donc – malgré son aspect profane et voluptueux – une science morale réelle, la plus morale des sciences. Le renoncement à soi-même, le renoncement à la vie et à tous les besoins humains est sa thèse principale. Moins tu manges, tu bois, tu achètes des livres, moins tu vas au théâtre, au bal, au cabaret, moins tu penses, tu aimes, tu fais de la théorie, moins tu chantes, tu parles, tu fais de l’escrime, etc., plus tu \emph{épargnes, plus} tu \emph{augmentes} ton trésor que ne mangeront ni les mites ni la poussière, ton \emph{capital.} Moins tu \emph{es}, moins tu manifestes ta vie, plus tu \emph{possèdes, plus} ta vie \emph{aliénée} grandit, plus tu accumules de ton être aliéné. Tout [XVI] ce que l’économiste te prend de vie et d’humanité, il te le remplace en \emph{argent} et en \emph{richesse} et tout ce que tu ne peux pas, ton argent le peut : il peut manger, boire, aller au bal, au théâtre ; il connaît l’art, l’érudition, les curiosités historiques, la puissance politique ; il peut voyager ; il \emph{peut} t’attribuer tout cela ; il peut acheter tout cela ; il est la vraie \emph{capacité.} Mais lui qui est tout cela, il n’a d’autre possibilité que de se créer lui-même, de s’acheter lui-même, car tout le reste est son valet et si je possède l’homme, je possède aussi le valet et je n’ai pas besoin de son valet. Toutes les passions et toute activité doivent donc sombrer dans la soif \emph{de richesse.} L’ouvrier doit avoir juste assez pour vouloir vivre et ne doit vouloir vivre que pour posséder.\par
Certes il s’élève maintenant une controverse sur le terrain économique. Les uns (Lauderdale, Malthus, etc.) recommandent le \emph{luxe} et maudissent l’épargne ; les autres (Say, Ricardo, etc.) recommandent l’épargne et maudissent le luxe. Mais les premiers avouent qu’ils veulent le luxe pour produire le \emph{travail} (c’est-à-dire l’épargne absolue) ; les autres avouent qu’ils recommandent l’épargne pour, produire la \emph{richesse}, c’est-à-dire le luxe. Les premiers ont l’illusion \emph{romantique} que ce n’est pas la seule soif du gain qui doit déterminer la consommation des riches et ils contredisent leurs propres lois en donnant directement la \emph{prodigalité} comme moyen d’enrichissement ; et les autres leur démontrent en conséquence, avec beaucoup de gravité et un grand luxe de détails que, par la prodigalité, je diminue mon \emph{avoir} et ne l’augmente pas ; les seconds commettent l’hypocrisie de ne pas avouer que la production est précisément déterminée par le caprice et l’inspiration ; ils oublient les “besoins raffinés”, ils oublient que sans consommation on ne produirait pas ; ils oublient que la production ne peut devenir que plus universelle et plus luxueuse par la concurrence ; ils oublient que l’usage détermine pour eux la valeur de la chose et que la mode détermine l’usage. Ils souhaitent ne voir produire que de l’ “utile”, mais ils oublient qu’à force de produire de l’utile, la production produit un excès de population inutile. Les uns et les autres oublient que le gaspillage et l’épargne, le luxe et le dénuement, la richesse et la pauvreté s’équivalent.\par
Et non seulement tu dois être économe de tes sens immédiats comme le manger, etc., mais tu dois aussi t’épargner de prendre part aux intérêts généraux, d’avoir pitié, confiance, etc., si tu veux te conformer aux enseignements de l’économie, si tu ne veux pas périr d’illusions.\par
Tout ce qui t’appartient, tu dois le rendre vénal, c’est-à-dire utile. Si je demande à l’économiste : est-ce que j’obéis aux lois économiques si je tire de l’argent de l’abandon, de la vente de mon corps à la volupté d’autrui (en France les ouvriers d’usines appellent la prostitution de leurs femmes et de leurs filles l’heure de travail supplémentaire, ce qui est littéralement exact), ou bien est-ce que je n’agis pas conformément à l’économie lorsque je vends mon ami aux Marocains (et la vente directe des hommes sous la forme du commerce des recrues, etc., a lieu dans tous les pays civilisés). celui-ci me répond : tu n’agis pas à rencontre de mes lois ; mais prends garde à ce que disent mes cousines, la morale et la religion ; ma morale et ma religion économiques n’ont rien à t’objecter, mais… Mais qui dois-je plutôt croire alors de l’économie politique ou de la morale ? La morale de l’économie politique est le gain, le travail et l’épargne, la sobriété… mais l’économie politique me promet de satisfaire mes besoins. L’économie politique de la morale est la richesse en bonne conscience, en vertu, etc., mais comment puis-je être vertueux si je ne suis pas, comment puis-je avoir une bonne conscience si je ne sais rien ? Tout ceci est fondé dans l’essence de l’aliénation : chaque sphère m’applique une norme différente et contraire, la morale m’en applique une et l’économie une autre, car chacune est une aliénation déterminée de l’homme et chacune [XVII] retient une sphère particulière de l’activité essentielle aliénée, chacune est dans un rapport d’aliénation à l’autre aliénation. Ainsi M. Michel Chevalier reproche à Ricardo de faire abstraction de a morale. Mais Ricardo laisse l’économie parler son propre langage. Si celui-ci n’est pas moral, Ricardo n’y peut rien. M. Chevalier fait abstraction de l’économie dans la mesure où il moralise, mais il fait nécessairement et réellement abstraction de la morale dans la mesure où il fait de l’économie politique. La relation de l’économie à la morale, si par ailleurs elle West pas arbitraire, contingente, et par suite sans fondement et sans caractère scientifique, si on n’en fait pas état pour la frime, mais qu’on la considère comme essentielle, ne peut être que la relation des lois économiques à la morale : si celle-ci n’apparaît pas, ou plutôt que le contraire se produit, en quoi Ricardo en est-il responsable ? D’ailleurs l’opposition entre l’économie et la morale West qu’une apparence et s’il y a une opposition, ce n’en est pas une. L’économie politique ne fait qu’exprimer à sa manière les lois morales.\par
L’absence de besoins comme principe de l’économie se manifeste de la façon la plus éclatante dans sa théorie de la population. Il y a trop d’hommes. Même l’existence des hommes est un pur luxe et si l’ouvrier est “moral” (Mill propose des félicitations publiques pour ceux qui se montrent abstinents au point de vue sexuel, et un blâme public pour ceux qui pèchent contre cette stérilité [idéale] du mariage \footnote{J. MILL :\emph{ Éléments d’économie politique}, traduction Parisot, Paris 1823, p. 10 sq.}… N’est-ce pas moral, n’est-ce pas la doctrine de l’ascétisme ?), il sera économe sur le plan de la génération. La production de l’homme apparaît comme une calamité publique.\par
Le sens qu’a la production en ce qui concerne les riches apparaît ouvertement dans le sens qu’elle a pour les pauvres ; par rapport à ceux qui sont en haut, il s’exprime toujours d’une manière subtile, déguisée, ambiguë, il est l’apparence, par rapport à ceux qui sont en bas, il s’exprime d’une manière grossière, directe, sincère, il est l’essence. Le besoin grossier de l’ouvrier est une source bien plus grande de profit que le besoin raffiné du riche. Les sous-sols de Londres rapportent à leurs loueurs plus que les palais, c’est-à-dire que par rapport au propriétaire, ils sont une \emph{richesse plus grande}, donc pour parler comme l’économiste une plus grande richesse \emph{sociale.}\par
Et tout comme l’industrie spécule sur le raffinement des besoins, elle spécule sur leur \emph{grossièreté}, mais sur leur grossièreté provoquée artificiellement. La véritable joie que procurent ces besoins grossiers consiste donc à \emph{s’étourdir}, elle est donc cette satisfaction \emph{apparente} du besoin, cette civilisation à \emph{l’intérieur} de la grossière barbarie du besoin. Les estaminets anglais sont par conséquent des illustrations symboliques de la propriété privée. Leur luxe montre le véritable rapport à l’homme du luxe et de la richesse industriels. Ils sont donc aussi avec raison les seules réjouissances dominicales du peuple qui soient tout au moins traitée\$ avec douceur par la police anglaise.\par
Nous avons déjà vu comment l’économiste pose de façon variée l’unité du travail et du capital. 1º Le capital est du travail accumulé ; 2º La détermination du capital à l’intérieur de la production, soit la reproduction du capital avec profit, soit le capital comme matière première (matière du travail), soit comme instrument travaillant lui-même (la machine est le capital qui est posé immédiatement comme identique avec le travail), est le travail productif ; 3º L’ouvrier est un capital ; 4º Le salaire fait partie des frais du capital ; 5º En ce qui concerne l’ouvrier, le travail est la reproduction de son capital vital ; 6º En ce qui concerne le capitaliste, il est un facteur d’activité de son capital ; enfin 7º L’économiste suppose l’unité primitive de l’un et de l’autre, comme l’unité du capitaliste et de l’ouvrier ; c’est l’état primitif paradisiaque. Comme ces deux aspects qu’incarnent deux personnes [XIX] se sautent à la gorge l’un de l’autre, cela est pour l’économiste un événement contingent et par suite qui ne peut s’expliquer que de l’extérieur (cf. Mill) \footnote{Ibid., p. 59 sq.}.\par
Les nations qui sont encore aveuglées par l’éclat sensible des métaux précieux et qui sont donc encore des fétichistes de l’argent métal – ne sont pas encore les nations d’argent achevées. Opposition entre la France et l’Angleterre. – Combien la solution des énigmes théoriques est une tâche de la praxis et se fait par son entremise, combien la praxis vraie est la condition d’une théorie réelle et positive apparaît par exemple à propos du fétichisme. La conscience sensible du fétichiste est différente de celle du grec, parce que son existence sensible est aussi différente. L’hostilité abstraite entre sensibilité et esprit est nécessaire tant que le sens de l’homme pour la nature, le sens humain de la nature, donc aussi le sens naturel de l’homme n’est pas encore produit par le travail propre de l’homme. -\par
L’égalité n’est rien d’autre que le moi = moi de l’allemand traduit en français, c’est-à-dire dans le langage politique. L’égalité comme raison du communisme est son fondement politique et la même chose se passe lorsque l’Allemand se donne le fondement du communisme en concevant l’homme comme conscience de soi universelle. Il va de soi que l’abolition de l’aliénation part toujours de la forme de l’aliénation qui est la puissance dominante, en Allemagne la conscience de soi, en France l’égalité à cause de la politique, en Angleterre le besoin réel matériel pratique qui ne se mesure qu’à soi-même. C’est de là qu’il faut partir pour critiquer et apprécier Proudhon.\footnote{Marx, bien qu’à la même époque il reconnaisse expressément les mérites de Proudhon, esquisse ici une critique fondamentale de sa théorie qui repose essentiellement sur la notion d’égalité.}\par
Si nous caractérisons encore le communisme lui-même – parce qu’il est la négation de la négation, l’appropriation de l’essence humaine qui a pour moyen terme avec elle-même la négation de la propriété privée parce qu’il ne pose donc pas encore le positif de façon vraie, en partant de lui-même, mais en partant au contraire de la propriété privée \footnote{Le coin gauche de la page du manuscrit est déchiré. Il subsiste seulement les fins de lignes, ce qui interdit à peu près toute reconstitution du texte. Nous reproduisons ce qu’il en reste en utilisant les derniers travaux de l’Institut du Marxisme-Léninisme à Moscou.}, -\par
… de la… ainsi à la manière vieille allemande – à la manière de la Phénoménologie de Hegel…\par
… soit maintenant liquidé comme un mouvement dépassé et qu’on…\par
… et que l’on puisse se tranquilliser parce que dans sa conscience…\par
… de l’essence humaine seulement par là réelle…\par
… abolition de sa pensée tout comme avant…\par
comme demeurent donc avec lui l’aliénation réelle de la vie humaine et une aliénation d’autant plus grande que l’on en a plus conscience en tant que telle – peut être réalisé (e), elle (il) ne peut donc se réaliser que par le communisme rais en œuvre.\par
Pour abolir l’idée de la propriété privée, le communisme pensé suffit entièrement. Pour abolir la propriété privée réelle, il faut une action communiste réelle. L’histoire l’apportera et ce mouvement, dont nous savons déjà en pensée qu’il s’abolit lui-même, passera dans la réalité par un processus très rude et très étendu. Mais nous devons considérer comme un progrès réel que, de prime abord, nous ayons acquis une conscience tant de la limitation que du but du mouvement historique, et une conscience qui le dépasse. -\par
Lorsque les ouvriers communistes se réunissent, c’est d’abord la doctrine, la propagande, etc., qui est leur but. Mais en même temps ils s’approprient par là un besoin nouveau, le besoin de la société, et ce qui semble être le moyen est devenu le but. On peut observer les plus brillants résultats de ce mouvement pratique, lorsque l’on voit réunis des ouvriers socialistes français. Fumer, boire, manger, etc., ne sont plus là comme des prétextes à réunion ou des moyens d’union. L’assemblée, l’association la conversation qui à son tour a la société pour but leur suffisent, la fraternité humaine n’est pas chez eux une phrase vide, mais une vérité, et la noblesse de l’humanité brille sur ces figures endurcies par le travail.\par
[XX] Si l’économie politique affirme que la demande et l’offre se couvrent toujours l’une l’autre, elle oublie aussitôt que, d’après ses propres affirmations, l’offre en hommes (théorie de la Population) dépasse toujours la demande, que le résultat essentiel de toute la production – l’existence de l’homme – fait donc apparaître de la façon la plus éclatante la disproportion entre la demande et l’offre. -\par
quel point l’argent, qui à l’origine est moyen, est la puissance vraie et le but unique, – combien en général le moyen qui fait de moi un être, qui fait mien l’être objectif, étranger, est un but en soi… on peut le voix à la façon dont la propriété foncière, là où la terre est la source de la vie, dont le cheval et l’épée, là où ils sont les vrais moyens de subsistance, sont aussi reconnus comme les vraies puissantes politiques de la vie. Au Moyen Age une classe est émancipée dès qu’elle a le droit de porter l’épée. Dans les Populations nomades, le cheval est ce qui fait de moi un homme libre, un participant à la communauté,-\par
Nous avons dit plus haut \footnote{Cf. p. 143.} que l’homme retourne à sa tanière, etc., mais la retrouve sous une forme aliénée et hostile. Le sauvage dans sa caverne – cet élément de la nature qui s’offre spontanément à lui pour qu’il en jouisse et qu’il y trouve abri – ne se sent pas plus étranger, nu plus exactement tout aussi a l’aise que le poisson dans l’eau. Mais la cave où loge le pauvre est quelque chose d’hostile, elle est “un domicile qui contient en soi une puissance étrangère, qui ne se donne à lui que dans la mesure où il lui donne sa sueur”, 'il ne peut considérer comme sa propre maison, – où il pourrait enfin dire : ici je suis chez où il se trouve plutôt dans la maison d’un autre, dans la maison d’un étranger qui chaque jour le guette et l’expulse s’il ne paie pas le loyer. De même au point de vue de la qualité, il connaît son logement comme le contraire du logement humain situé dom l’au-delà, au ciel de la richesse.\par
L’aliénation apparaît tout autant dans le fait que mes moyens de subsistance appartiennent à un autre, que ce qui est mon désir est la possession inaccessible d’ait autre, que dans le fait que toute chose est elle-même autre qu’elle-même, que mon activité est autre chose, qu’enfin – et ceci est vrai aussi pour le capitalisme – c’est somme toute la puissance inhumaine qui règne.\par
Définition de la richesse inactive, dissipatrice adonnée seulement à la jouissance : d’une part, celui qui en jouit se conduit, certes, comme un individu seulement éphémère, se passant des lubies inconsistantes, et il considère également le travail d’esclave d’autrui, la sueur de sang de l’homme, comme la proie de son désir ; c’est pourquoi il connaît l’homme lui-même, donc se connaît lui-même, comme un être sacrifié et nul (cependant son mépris des hommes apparaît comme superbe, comme gaspillage de tout ce qui peut prolonger cent vies humaines ou bien comme l’illusion infâme que sa prodigalité effrénée et sa consommation impétueuse et improductive conditionnent le travail et par suite la subsistance d’autrui) ; la réalisation des forces essentielles de l’homme, il ne la connaît que comme la réalisation de sa monstruosité, de son caprice et de ses lubies arbitraires et bizarres. Mais cette richesse-là, d’autre part, connaît la richesse comme un simple moyen et comme une chose qui mérite elle est donc à la fois son esclave et son. maître, à la fois généreuse et abjecte, capricieuse, infatuée, orgueilleuse et raffinée, cultivée, spirituelle ; elle n’a pas encore fait l’expérience de la richesse comme d’une puissance totalement étrangère qui la domine ; elle voit bien plutôt en elle sa propre puissance et [ce n’est pas] la richesse, mais la jouissance (qui est pour elle] \footnote{La page est déchirée. Il manque trois ou quatre lignes.}… fin dernière. Cette… [XXI] et à l’illusion brillante, aveuglée par l’apparence sensible, l’essence de la richesse, s’oppose l’industriel travailleur, sobre, pensant selon l’économie, prosaïque – qui est éclairé sur l’essence même de la richesse – et tout en procurant à la soif de jouissance du dissipateur un champ plus vaste, en ne lui disant que de belles flatteries par ses productions, – ses produits sont précisément tout autant de bas complimenta aux appétits de celui-ci, – il sait s’approprier pour lui-même de la 4eule manière utile la puissance qui échappe à l’autre. Si donc la richesse industrielle apparaît tout d’abord comme le résultat de la richesse dissipatrice, fantaisiste, – le mouvement de la première la supplante aussi activement, par un mouvement qui lui est propre. La baisse du taux de l’intérêt est, en effet, une conséquence et un résultat nécessaire du mouvement industriel. Les moyens du dissipateur vivant de ses rentes diminuent donc chaque jour, exactement en raison inverse de l’augmentation des moyens de jouissance et de leurs pièges. Il doit donc ou bien manger lui-même son capital, donc périr, ou bien se transformer lui-même en capitaliste industriel… D’autre part, la rente foncière monte certes directement d’une façon continue grâce à la marche du mouvement industriel, mais – nous l’avons déjà vu – il vient nécessairement un moment où la propriété foncière doit tomber comme toute autre propriété dans la catégorie du capital qui se reproduit avec profit – et, qui plus est, c’est là le résultat de ce même mouvement industriel. Donc, le propriétaire foncier dissipateur doit, lui aussi, ou bien manger son capital, donc périr… ou bien devenir lui-même le fermier de sa propre terre – l’industriel pratiquant l’agriculture.\par
La diminution de l’intérêt de l’argent – que Proudhon considère comme la suppression du capital et comme la tendance à la socialisation du capital – n’est donc bien plutôt qu’un symptôme direct de la victoire complète du capital qui travaille sur la richesse dissipatrice, c’est-à-dire la transformation de toute propriété privée en capital industriel – la victoire complète de la propriété privée sur toutes ses qualités encore humaines en apparence et l’assujettissement total du propriétaire privé à l’essence de la propriété privée, – le travail. Certes le capitaliste industriel jouit lui aussi. Il ne revient nullement à la simplicité contre nature du besoin, niais sa jouissance n’est que chose secondaire, récréation, subordonnée à la production, et elle est avec cela jouissance calculée, donc même conforme à l’économie, car il l’ajoute aux frais du capital et elle ne doit donc lui coûter que ce qu’il faut pour que ce qu’il a dissipé pour elle soit remplacé avec profit par la reproduction du capital. La jouissance est donc subordonnée au capital, l’individu qui jouit est subordonné à celui qui capitalise, tandis qu’autrefois c’était le contraire. La diminution de l’intérêt n’est donc un symptôme de l’abolition du capital que dans la mesure où elle est un symptôme de sa domination en voie d’accomplissement, donc de l’aliénation qui s’achève et se hâte vers sa suppression. C’est somme toute l’unique manière dont ce qui existe confirme son contraire.\par
La querelle des économistes à propos du luxe et de l’épargne n’est par conséquent que la querelle de l’économie politique arrivée à une notion claire de l’essence de la richesse avec celle qui est encore entachée de souvenirs romantiques et anti-industriels. Mais les deux parties ne savent pas ramener l’objet de leur querelle à son expression simple et par suite n’arrivent pas à venir à bout l’une de l’autre.\par
[XXXIV] La rente foncière fut en outre renversée parce que rente foncière – car à l’opposé de l’argument des physiocrates qui faisaient du propriétaire foncier le seul vrai producteur, l’économie politique moderne a démontré au contraire qu’il était en tant que propriétaire foncier le seul rentier tout à fait improductif. L’agriculture serait l’affaire du capitaliste qui donnerait cet emploi à son capital s’il avait à en attendre le profit habituel. Le principe posé par les physiocrates – que la propriété foncière étant la seule propriété productrice devrait seule payer l’impôt d’État, donc aussi seule l’accorder et prendre part à la gestion de l’État – se change donc en la définition inverse : l’impôt sur la rente foncière est le seul impôt sur un revenu improductif et par suite le seul qui ne soit pas nuisible pour la production nationale. Il est évident que, selon cette conception, le privilège politique des propriétaires fonciers ne résulte plus non plus de ce qu’ils portent le poids principal de l’impôt. -\par
Tout ce que Proudhon saisit comme le mouvement du travail contre le capital n’est que le mouvement du travail dans sa détermination de capital, de capital industriel, contre le capital qui ne se consomme pas en tant que capital, c’est-à-dire d’une façon industrielle. Et ce mouvement suit sa voie victorieuse, c’est-à-dire la voie de la victoire du capital industriel. – On voit donc que ce West qu’une fois le travail saisi comme essence de la propriété privée que le mouvement de l’économie peut être lui aussi percé à jour en tant que tel dans sa détermination réelle.\par
La société – telle qu’elle apparaît à l’économiste – est la société bourgeoise dans laquelle chaque individu est un ensemble de besoins et n’est là que pour l’autre, comme l’autre [XXXV] n’est là que pour lui dans la mesure où ils deviennent l’un pour l’autre un moyen. L’économiste – aussi bien que la politique dans ses droits de l’homme – réduit tout à l’homme, c’est-à-dire à l’individu qu’il dépouille de toute détermination pour le retenir comme capitaliste ou comme ouvrier.\par
La division du travail est l’expression économique du caractère social du travail dans le cadre de l’aliénation. Ou bien, comme le travail n’est qu’une expression de l’activité de l’homme dans le cadre de l’aliénation, l’expression de la manifestation de la vie comme aliénation de la vie, la \emph{division du travail} n’est elle-même pas autre chose que le fait de poser, d’une manière \emph{devenue étrangère, aliénée}, l’activité humaine comme une \emph{activité générique réelle}, ou comme \emph{l’activité de l’homme en tant qu’être générique.}\par
Sur l’essence de \emph{la division du travail –} qui devait naturellement être conçue comme un facteur essentiel de la production de la richesse dès l’instant où \emph{le travail} était reconnu comme \emph{l’essence de la propriété privée – c’est-à-dire} sur cette forme \emph{devenue étrangère et aliénée de l’activité humaine en tant qu’activité générique}, les économistes sont très obscurs et se contredisent.\par
\emph{Adam Smith} \footnote{Cette citation, tirée de la \emph{Recherche sur la nature et les causes de la richesse des Nations}, est donnée dans le texte d’Adam Smith. Les passages entre [] n’ont pas été repris par Marx.} :\par

\begin{quoteblock}
 \noindent Cette division du travail, [de laquelle découlent tant d’avantages,] ne doit pas être regardée, dans son origine, comme l’effet d’une sagesse humaine… elle est la conséquence nécessaire, quoique lente et graduelle, de… ce penchant à trafiquer, à faire des trocs et des échanges d’une chose pour une autre. [Il n’est pas de mon sujet d’examiner si] ce penchant est [un de ces premiers principes de la nature humaine… ou bien,] comme il paraît plus probable, [s’il est] une conséquence nécessaire de l’usage du raisonnement et de la parole. Il est commun à tous les hommes, et on ne l’aperçoit dans aucune autre espèce d’animaux \footnote{\emph{Ibid.}, tome I, p. 29.}… Dans presque toutes les autres espèces d’animaux, chaque individu, quand il est parvenu à sa pleine croissance, est tout à fait indépendant… [Mais] l’homme a presque continuellement besoin du secours de ses semblables, et c’est en vain qu’il l’attendrait de leur seule bienveillance. Il sera bien plus sûr de son fait en s’adressant à leur intérêt personnel, et en leur persuadant qu’il y va de leur propre avantage de faire ce qu’il souhaite d’eux… Nous ne nous adressons pas à leur humanités, mais à leur égoïsme \footnote{Souligné par Marx.} ; et ce n’est jamais de nos besoins que nous leur parlons, c’est toujours de leur avantage \footnote{\emph{Ibid.}, tome I, pp. 30-31. Le dernier mot est souligné par Marx.}. Comme c’est ainsi par traité, par troc et par achat que nous obtenons des autres la plupart de ces bons offices qui nous sont mutuellement nécessaires, c’est cette même disposition à trafiquer \footnote{Souligné par Marx.} qui a, dans l’origine, donné lieu à la division du travail. Par exemple, dans une tribu de chasseurs ou de bergers, un particulier fait des arcs et des flèches avec plus de célérité et d’adresse qu’un autre. Il troque souvent avec ses compagnons ces sortes d’ouvrages contre du bétail ou du gibier, et il s’aperçoit bientôt que par ce moyen il peut se procurer plus de bétail et de gibier que s’il se mettait lui-même en campagne pour en avoir. Par calcul d’intérêt donc, il fait sa principale affaire de fabriquer des arcs et des flèches… Dans la réalité, la différence des talents naturels \footnote{Ibid., tome I, pp. 32-33.} entre les individus … n’est pas tant la cause que l’effet de la division du travail… Sans la disposition des hommes à trafiquer et à échanger, chacun aurait été obligé de se procurer à soi-même toutes les nécessités et commodités de la vie. Chacun aurait eu la même tâche à remplir et le même ouvrage à faire, et il n’y aurait pas eu lieu à cette grande différence d’occupations, qui seule peut donner naissance à une grande différence de talents. Comme c’est ce penchant à troquer qui donne lieu à cette diversité de talents, si remarquable entre hommes de différentes professions, c’est aussi ce même penchant qui rend cette diversité utile. Beaucoup de races d’animaux, qu’on reconnaît pour être de la même espèce, ont reçu de la nature des signes distinctifs, quant à leurs dispositions, beau. coup plus remarquables que ceux qu’on pourrait observer entre les hommes, antérieurement à l’effet des habitudes et de l’éducation. Par nature, un philosophe n’est pas de moitié aussi différent d’un portefaix, en talent et en intelligence, qu’un mâtin l’est d’un lévrier, un lévrier d’un épagneul, et celui-ci d’un chien de berger. Toutefois, ces différentes races d’animaux, quoique de même espèce, ne sont presque d’aucune utilité les unes pour les autres. Le mâtin ne peut pas ajouter aux avantages [XXXVI] de sa force en s’aidant de la légèreté du lévrier… Les effets de ces différents talents ou degrés d’intelligence, faute d’une faculté ou d’un penchant au commerce ou à l’échange, ne peuvent être mis en commun, et ne peuvent le moins du monde contribuer à l’avantage \footnote{Souligné par Marx.} ou à la commodité commune de l’espèce. Chaque animal est toujours obligé de s’entretenir et de se défendre lui-même à part et indépendamment des autres, et il ne peut retirer la moindre utilité de cette variété de talents que la nature a répartis entre ses pareils. Parmi les hommes, au contraire, les talents les plus disparates sont utiles les uns aux autres, parce que les différents produits de chacune de leurs diverses sortes d’industrie respective, au moyen de ce penchant universel à troquer et à commercer se trouvent mis, pour ainsi dire, en une masse commune où chaque homme peut aller acheter, suivant ses besoins, une portion quelconque du produit de l’industrie des autres. Puisque c’est la faculté d’échanger qui donne lieu à la division du travail, l’accroissement de cette division \footnote{Ibid., tome I, pp. 34-37.} doit par conséquent toujours être limité par l’étendue de la faculté d’échanger, ou, en d’autres termes, par l’étendue du marché. Si le marché est très petit, personne ne sera encourage à s’adonner entièrement à une seule occupation, faute de pouvoir trouver à échanger tout ce surplus du produit de son travail qui excédera sa propre consommation, contre un pareil surplus du produit du travail d’autrui qu’il voudrait se procurer…” . Dans l’état avancé : “Ainsi chaque homme subsiste d’échanges ou devient une espèce de marchand et la société elle-même est proprement une société commerçante. (Cf. Destutt de Tracy : la société est… une série continuelle d’échanges, le commerce est toute la société) \footnote{DESTUTT de Tracy : Éléments d’idéologie, IV° et V° parties : Traité de la volonté et de ses effets, Paris, 1826, p. 68, 78.}… L’accumulation des capitaux augmente avec la division du travail et réciproquement \footnote{Ibid., p. 46.}.
 \end{quoteblock}

\noindent Voilà pour \emph{Adam Smith.}\par

\begin{quoteblock}
 \noindent Si chaque famille produisait la totalité des objets de sa consommation, la société pourrait marcher ainsi, quoi qu’il ne s’y fît aucune espèce d’échanges ; je sais que, sans être fondamentaux, ils sont indispensables dans l’état avancé de nos sociétés \footnote{SAY : Traité d’économie politique. 3ᵉ édition. Paris 1817, tome I, p. 300}. On peut dire que la séparation des travaux est un habile emploi des forces de l’homme, qu’elle accroît en conséquence les produits de la société, c’est-à-dire sa puissance et ses jouissances, mais qu’elle ôte quelque chose à la capacité de chaque homme pris individuellement. La production ne peut avoir lieu sans échange \footnote{Ibid., tome I, p. 76.}.
 \end{quoteblock}

\noindent Ainsi parle J.-B. \emph{Say.}\par

\begin{quoteblock}
 \noindent Les forces inhérentes à l’homme sont : son intelligence et son aptitude physique au travail. Celles qui dérivent de l’état de société consistent – dans la faculté de diviser et de répartir parmi les hommes les divers travaux… et dans a faculté d’échanger les services mutuels et les produits qui constituent ces moyens… Les motifs pour lesquels il consent à vouer ses services à autrui… sont l’égoïsme, – l’homme exige… une récompense pour les services rendus à autrui \footnote{SKARBEK : Théorie des richesses sociales, suivie d’une bibliographie de l’économie politique. T. I-II, Paris 1829, tome I, pp. 25-27.}… L’existence du droit de propriété exclusive est donc indispensable pour que l’échange puisse s’établir parmi les hommes \footnote{Ibid., tome I, p. 75.}… Influence réciproque de la division d’industrie sur l’échange et de l’échange sur cette division \footnote{Ibid., tome I, p. 121. Cette citation est le titre du chapitre V.}.
 \end{quoteblock}

\noindent C’est ce que dit \emph{Skarbek.}\par
Mill représente l’échange développé, le \emph{commerce}, comme une \emph{conséquence} de la \emph{division du travail.}\par

\begin{quoteblock}
 \noindent L’action de l’homme peut être ramenée à de très simples éléments. Il ne peut, en effet, rien faire de plus que de produire du mouvement ; il peut mouvoir les choses pour les approcher [XXXVII] ou les éloigner les unes des autres ; les propriétés de la matière font tout le reste… \footnote{J. MILL : Éléments d’économie politique. Traduit par J.-T. Parisot. Paris, 1823, p. 7.}. Dans l’emploi du travail et des machines, on trouve souvent que les effets peuvent être augmentés… en séparant toutes les opérations qui ont une tendance à se contrarier, et en réunissant toutes celles qui peuvent, de quelque manière que ce soit, se faciliter les unes les autres. Comme en général les hommes ne peuvent exécuter beaucoup d’opérations différentes avec la même vitesse et la même dextérité qu’ils parviennent, par l’habitude, à en exécuter un petit nombre, il est toujours avantageux de limiter autant que possible le nombre d’opérations confiées à chaque individu. Pour diviser le travail et distribuer les forces des hommes et des machines de la manière la plus avantageuse, il est nécessaire, dans une foule de cas, d’opérer sur une grande échelle, ou en d’autres termes de produire les richesses par grandes masses. C’est cet avantage qui donne naissance aux grandes manufactures. Un petit nombre de ces manufactures placées dans les positions les plus convenables, approvisionnent quelquefois non pas un seul, mais plusieurs pays, de la quantité qu’on y désire de l’objet qu’elles produisent \footnote{Ibid., pp. 11-12.}.
 \end{quoteblock}

\noindent Voilà ce que dit Mill.\par
\bigbreak
\noindent Mais toute l’économie moderne s’accorde sur le fait que division du travail et richesse de la production, division du travail et accumulation du capital se conditionnent réciproquement, ainsi que sur le fait que la propriété privée \emph{affranchie}, laissée à elle-même, peut seule produire la division du travail la plus utile et la plus vaste.\par
On peut résumer ainsi le développement \emph{dAdam Smith : la division} du travail donne au travail une capacité infinie de production. Elle est fondée sur la \emph{disposition à l’échange} et au \emph{trafic}, disposition spécifiquement humaine qui n’est vraisemblablement pas fortuite, mais conditionnée par l’usage de la raison et du langage. Le mobile de celui qui pratique l’échange n’est pas \emph{l’humanité}, mais \emph{l’égoïsme.} La diversité des talents humains est plutôt l’effet que la cause de la division du travail, c’est-à-dire de l’échange. C’est aussi ce dernier seulement qui rend utile cette diversité. Les qualités particulières des diverses races d’une espèce animale sont par nature plus fortement marquées que la diversité des dons et de l’activité humaine. Mais comme les animaux ne peuvent pas \emph{échanger}, la propriété différente d’un animal de la même espèce mais de race différente ne sert à aucun individu animal. Les animaux ne peuvent pas additionner les qualités différentes de leur espèce ; ils ne peuvent en rien contribuer à l’avantage ou à la commodité \emph{communes} de leur espèce. Il en va différemment pour \emph{l’homme} chez qui les talents et les modes d’activité les plus disparates sont utiles les uns aux autres \emph{parce qu’ils} peuvent rassembler leurs \emph{divers} produits en une masse commune où chacun peut acheter. De même que la division du travail naît de la disposition à l’échange, elle grandit, elle est limitée par l’étendue de l’échange, du marché. Dans l’état avancé, chaque homme est commerçant, la société est une société de commerce. Say considère l’échange comme fortuit et non fondamental. La société pourrait subsister sans lui. Il devient indispensable dans l’état avancé de la société. Pourtant la production ne peut avoir lieu sans lui. La division du travail est un moyen commode et utile, une habile utilisation des forces humaines pour la richesse sociale, mais elle diminue la faculté de chaque homme pris individuellement. Cette dernière remarque est un progrès de Say.\par
Skarbek distingue les forces individuelles inhérentes à l’homme, l’intelligence et la disposition physique au travail, des forces dérivées de la société, l’échange et la division du travail qui se conditionnent réciproquement. Mais la condition nécessaire de l’échange est la propriété privée. Skarbek exprime ici, sous une forme objective, ce que Smith, Say, Ricardo, etc., disent lorsqu’ils font de l’égoïsme, de l’intérêt privé, le fondement de l’échange, ou du trafic la forme essentielle et adéquate de l’échange.\par
Mill représente le commerce comme la conséquence de la division du travail. L’activité humaine se réduit pour lui à un mouvement mécanique. La division du travail et l’utilisation des machines font progresse ; la richesse de la production. On doit confier à chaque homme un cercle aussi réduit que possible d’opérations. De leur côté, la division du travail et l’utilisation des machines conditionnent la production de la richesse en masse, donc du produit. C’est le fondement des grandes manufactures.\par
\bigbreak
\noindent [XXXVIII] L’examen de la division du travail et de l’échange est du plus haut intérêt, parce qu’ils sont l’expression visiblement aliénée de l’activité et de la force essentielle de l’homme en tant qu’activité et force essentielle génériques.\par
Dire que la division du travail et l’échange reposent sur la propriété privée n’est pas autre chose qu’affirmer que le travail est l’essence de la propriété privée, affirmation que l’économiste ne peut pas prouver et que nous allons prouver pour lui. Dans le fait précisément que division du travail et échange sont des formes de la propriété privée, repose la double preuve que, d’une part, la vie humaine avait besoin de la propriété privée pour se réaliser, et que, d’autre part, elle a maintenant besoin de l’abolition de la propriété privée.\par
Division du travail et échange sont les deux phénomènes qui font que l’économiste tire vanité du caractère social de sa science et que, inconsciemment, il exprime d’une seule haleine la contradiction de sa science, la fondation de la société par l’intérêt privé asocial.\par
Les aspects que nous avons à examiner sont les suivants d’une part \emph{la disposition à l’échange – dont} le motif est trouvé dans l’égoïsme – est considérée comme la raison ou l’effet en retour de la division du travail. Say estime que l’échange n’est pas \emph{fondamental} pour l’essence de la société. La richesse, la production est expliquée par la division du travail et l’échange. On admet que la division du travail provoque l’appauvrissement et la dégradation de l’activité individuelle. L’échange et la division du travail sont reconnus comme les producteurs de la grande \emph{diversité des talents humains}, diversité qui retrouve son \emph{utilité} grâce au premier. Skarbek divise les formes de production ou les forces essentielles productives de l’homme en deux parts, 1º les forces individuelles qui lui sont inhérentes, son intelligence et la faculté ou la disposition spéciale au travail ; 2º celles qui sont \emph{dérivées} de la société, – non de l’individu réel, – la division du travail et l’échange. En outre la division du travail est limitée par le \emph{marché. –} Le travail humain est un simple mouvement méca\emph{nique ;} l’essentiel est fait par les propriétés matérielles des objets. Il faut attribuer à un individu le moins d’opérations possible. Séparation du travail et concentration du capital, insignifiance de la production individuelle et production de la richesse en masse. – Intelligence de la propriété privée libre dans la division du travail \footnote{Ici s’interrompt la partie du troisième manuscrit qui est une sorte d’appendice à la page XXXIX du second manuscrit. Seule la partie gauche de la page XXXVIII est écrite, la partie droite est vierge. Vient ensuite la préface (placée en tête du volume) sur les pages XXXIX et XL, et le passage sur l’argent (pp. XLI-XLIII) que nous abordons maintenant.}.
\subsection[{[Pouvoir de l’argent dans la société bourgeoise]}]{[Pouvoir de l’argent dans la société bourgeoise]}
\noindent \emph{[XLI] Si les} sensations, les passions, etc. de l’homme ne sont pas seulement des déterminations anthropologiques au sens [étroit] \footnote{Le mot est illisible}, mais sont vraiment des affirmations \emph{ontologiques} essentielles (naturelles) – et si elles ne s’affirment réellement que par le fait que leur \emph{objet} est \emph{sensible} pour elles, il est évident 1º que le mode de leur affirmation n’est absolument pas un seul et même mode, mais qu’au contraire, la façon distincte dont elles s’affirment constitue le caractère propre de leur existence, de leur vie ; la façon dont l’objet existe pour elles constitue le caractère propre de chaque \emph{jouissance} spécifique ; 2º là où l’affirmation sensible est suppression directe de l’objet sous sa forme indépendante (manger, boire, façonnage de l’objet, etc.), c’est l’affirmation de l’objet ; 3º dans la mesure où l’homme est \emph{humain}, où donc sa sensation, etc., aussi est \emph{humaine}, l’affirmation de l’objet par un autre est également sa propre jouissance ; 4º ce n’est que par l’industrie développée, c’est-à-dire par le moyen terme de la propriété privée, que l’essence ontologique de la passion humaine atteint et sa totalité et son humanité ; la science de l’homme est donc elle-même un produit de la manifestation pratique de soi par l’homme ; 5º le sens de la propriété privée – détachée de son aliénation – est \emph{l’existence des objets essentiels} pour l’homme tant comme objets de jouissance que comme objets d’activité.\par
\emph{L’argent} en possédant la \emph{qualité} de tout acheter, en possédant la qualité de s’approprier tous les objets est donc \emph{l’objet} comme possession éminente. L’universalité de sa \emph{qualité} est la toute-puissance de son essence. Il passe donc pour tout-puissant… L’argent est \emph{l’entremetteur} entre le besoin et l’objet, entre la vie et le moyen de subsistance de l’homme. Mais \emph{ce qui} sert de moyen terme à \emph{ma} vie, sert aussi de moyen terme à l’existence des autres hommes pour moi. C’est pour moi \emph{l’autre} homme.\par

\begin{quoteblock}
 \noindent Que diantre ! il est clair que tes mains et les pieds\par
 Et ta tête et ton c… sont à toi ;\par
 Mais tout ce dont je jouis allégrement\par
 En est-ce donc moins à moi ?\par
 Si je puis payer six étalons,\par
 Leurs forces ne sont-elles pas miennes ?\par
 Je mène bon grain et suis un gros monsieur,\par
 Tout comme si j’avais vingt-quatre pattes.\par
 GOETHE : Faust (Méphistophélès)\footnote{Faust, 1ʳᵉ partie. Traduction Lichtenberger. Paris 1932, tome I, p. 58.}
 \end{quoteblock}

\noindent Shakespeare dans \emph{Timon d’Athènes} \footnote{SHAKESPEARE : Les Tragédies. Nouvelle traduction par Pierre Messiaen, Paris 1941. “La vie de Timon d’Athènes”, Acte IV, Scène 3, p. 1035 sq.} :\par

\begin{quoteblock}
 \noindent De l’or ! De l’or jaune, étincelant, précieux ! Non, dieux du ciel, je ne suis pas un soupirant frivole… Ce peu d’or suffirait à rendre blanc le noir, beau le laid, juste l’injuste, noble l’infâme, jeune le vieux, vaillant le lâche… Cet or écartera de vos autels vos prêtres et vos serviteurs ; il arrachera l’oreiller de dessous la tête des mourants ; cet esclave jaune garantira et rompra les serments, bénira les maudits, fera adorer la lèpre livide, donnera aux voleurs place, titre, hommage et louange sur le banc des sénateurs ; c’est lui qui pousse à se remarier la veuve éplorée. Celle qui ferait lever la gorge à un hôpital de plaies hideuses, l’or l’embaume, la parfume, en fait de nouveau un jour d’avril. Allons, métal maudit, putain commune à toute l’humanité, toi qui mets la discorde parmi la foule des nations…
 \end{quoteblock}

\noindent Et plus loin \footnote{Ibid., p. 1046.} :\par

\begin{quoteblock}
 \noindent O toi, doux régicide, cher agent de divorce entre le fils et le père, brillant profanateur du lit le plus pur d’Hymen, vaillant Mars, séducteur toujours jeune, frais, délicat et aimé, toi dont la splendeur fait fondre la neige sacrée qui couvre le giron de Diane, toi dieu visible,\& qui soudes ensemble les incompatibles \footnote{Souligné par Marx.} et les fais se baiser, toi qui parles par toutes les bouches [XLII] et dans tous les sens, pierre de touche des cœurs, traite en rebelle l’humanité, ton esclave, et par ta vertu jette-la en des querelles qui la détruisent \footnote{Souligné par Marx.}, afin que les bêtes aient l’empire du monde.
 \end{quoteblock}

\noindent Shakespeare décrit parfaitement l’essence de \emph{l’argent.} Pour le comprendre, commençons d’abord par expliquer le passage de Gœthe :\par
Ce qui grâce à \emph{l’argent} est pour moi, ce que je peux payer, c’est-à-dire ce que l’argent peut acheter, \emph{je} le suis moi-même, moi le possesseur de l’argent. Ma force est tout aussi grande qu’est la force de l’argent. Les qualités de l’argent sont mes qualités et mes forces essentielles – à moi son possesseur. Ce que je suis et ce que je \emph{peux} n’est donc nullement déterminé par mon individualité. \emph{Je suis} laid, mais je peux m’acheter \emph{la plus belle} femme. Donc je ne suis pas \emph{laid}, car l’effet de la \emph{laideur}, sa force repoussante, est anéanti par l’argent. De par mon individualité, je suis perclus, mais l’argent me procure vingt-quatre pattes ; je ne suis donc pas perclus ; je suis un homme mauvais, malhonnête, sans conscience, sans esprit, mais l’argent est vénéré, donc aussi son possesseur, l’argent est le bien suprême, donc son possesseur est bon, l’argent m’évite en outre la peine d’être malhonnête ; on me présume donc honnête ; je suis \emph{sans esprit}, mais l’argent est \emph{l’esprit réel} de toutes choses, comment son possesseur pourrait-il ne pas avoir d’esprit ? De plus, il peut acheter les gens spirituels et celui qui possède la puissance sur les gens d’esprit n’est-il pas plus spirituel que l’homme d’esprit ? Moi qui par l’argent peux \emph{tout} ce à quoi aspire un cœur humain, est-ce que je ne possède pas tous les pouvoirs humaine ? Donc mon argent ne transforme-t-il pas toutes mes impuissances en leur contraire ?\par
\emph{Si l’argent} est le lien qui me lie à la vie \emph{humaine}, qui lie à moi la société et qui me lie à la nature et à l’homme, l’argent n’est-il pas le lien de tous les \emph{liens} ? Ne peut-il pas dénouer et nouer tous les liens ? N’est-il non plus de ce fait le moyen universel de séparation ? Il est la vraie \emph{monnaie divisionnaire}, comme le vrai \emph{moyen d’union}, la force \emph{chimique} [universelle] \footnote{Un coin de la page est déchiré.} de la société.\par
Shakespeare souligne surtout deux propriétés de l’argent :\par
1º Il est la divinité visible, la transformation de toutes les qualités humaines et naturelles en leur contraire, la confusion et la perversion universelle des choses ; il fait fraterniser es impossibilités.\par
2º Il est la courtisane universelle, l’entremetteur universel des hommes et des peuples.\par
La perversion et la confusion de toutes les qualités humaines et naturelles, la fraternisation des impossibilités – la force divine – de l’argent sont impliquées dans son essence in tant qu’essence générique aliénée, aliénante et s’aliénant, des hommes. Il est la puissance aliénée de l’humanité.\par
Ce que je ne puis en tant qu’homme, donc ce que ne peuvent toutes mes forces essentielles d’individu, je le puis grâce à l’argent. L’argent fait donc de chacune de ces forces essentielles ce qu’elle n’est pas en soi ; c’est-à-dire qu’il en fait éon contraire.\par
Si j’ai envie d’un aliment ou si je veux prendre la chaise de poste, puisque je ne suis pas assez fort pour faire la route à pied, l’argent me procure l’aliment et la chaise de poste, c’est-à-dire qu’il transforme mes vœux d’êtres de la représentation qu’ils étaient, il les transfère de leur existence pensée, figurée, voulue, dans leur existence sensible, réelle ; il les fait passer de la représentation à la vie, de l’être figuré à l’être réel. Jouant ce rôle de moyen terme, l’[argent] est la force vraiment créatrice.\par
La \emph{demande existe} bien aussi pour celui qui n’a pas d’argent, mais sa demande est un pur être de la représentation qui sur moi, sur un tiers, sur les autres [XLIII] n’a pas d’effet, n’a pas d’existence, donc reste pour moi-même irréel, sans \emph{objet. La} différence entre la demande effective, basée sur l’argent, et la demande sans effet, basée sur mon besoin, ma passion, mon désir, etc., est la différence entre l’Être et la Pensée, entre la simple représentation existant en moi et la représentation telle qu’elle est pour moi en dehors de moi en tant \emph{qu’objet réel}.\par
Si je n’ai pas d’argent pour voyager, je n’ai pas de \emph{besoin}, c’est-à-dire de besoin réel et se réalisant de voyager. Si j’ai la vocation d’étudier mais que je n’ai pas l’argent pour le faire, je n’ai pas \emph{de} vocation d’étudier, c’est-à-dire pas de vocation active, véritable. Par contre, si je n’ai réellement pas de vocation d’étudier, mais que j’en ai la volonté et l’argent, j’ai par-dessus le marché une vocation effective. L’argent, – moyen et pouvoir universels, extérieurs, qui ne viennent pas de l’homme en tant qu’homme et de la société humaine en tant que société, – moyen et pouvoir de convertir la représentation en réalité et la réalité en simple représentation, transforme tout aussi bien les forces essentielles réelles et \emph{naturelles de l’homme en} représentation purement abstraite et par suite en imperfections, en chimères douloureuses, que d’autre part il transforme les \emph{imperfections et} chimères réelles, les forces essentielles réellement impuissantes qui n’existent que dans l’imagination de l’individu, en forces essentielles réelles et en pouvoir. Déjà d’après cette définition, il est donc la perversion générale des \emph{individualités, qui} les change en leur contraire et leur donne des qualités qui contredisent leurs qualités propres.\par
Il apparaît alors aussi comme cette puissance de perversion contre l’individu et contre les liens sociaux, etc., qui prétendent être des essences pour soi. Il transforme la fidélité en infidélité, l’amour en haine, la haine en amour, la vertu en vice, le vice en vertu, le valet en maître, le maître en valet, le crétinisme en intelligence, l’intelligence en crétinisme.\par
Comme l’argent, qui est le concept existant et se manifestant de la valeur, confond et échange toutes choses, il est la confusion a la permutation universelles de toutes choses, donc le monde à l’envers, la confusion et la permutation de toutes les qualités naturelles et humaines.\par
Qui peut acheter le courage est courageux, même s’il est lâche. Comme l’argent ne s’échange pas contre une qualité déterminée, contre une chose déterminée, contre des forces essentielles de l’homme, mais contre tout le monde objectif de l’homme et de la nature, il échange donc – du point de vue de son possesseur – toute qualité contre toute autre – et aussi sa qualité et son objet contraires ; il est la fraternisation des impossibilités. Il oblige à s’embrasser ce qui se contredit.\par
Si tu supposes l’homme en tant \emph{qu’homme et} son rapport au monde comme un rapport humain, tu ne peux échanger que l’amour contre l’amour, la confiance contre la confiance, etc. Si tu veux jouir de l’art, il faut que tu sois un homme ayant une culture artistique ; si tu veux exercer de l’influence sur d’autres hommes, il faut que tu sois un homme qui ait une action réellement animatrice et stimulante sur les autres hommes. Chacun de tes rapports à l’homme – et à la nature – doit être une manifestation déter\emph{minée}, répondant à l’objet de ta volonté, de ta vie \emph{individuelle réelle. Si} tu aimes sans provoquer d’amour réciproque, c’est-à-dire si ton amour, en tant qu’amour, ne provoque pas l’amour réciproque, si par ta \emph{manifestation vitale} en tant qu’homme aimant tu ne te transformes pas en \emph{homme aimé}, ton amour est impuissant et c’est un malheur.
\subsection[{[Critique de la dialectique de Hegel et de sa philosophie en général]}]{[Critique de la dialectique de Hegel et de sa philosophie en général]}
\noindent 6. Voici peut-être le moment et le lieu \footnote{Dans le manuscrit de Marx, ce passage vient immédiatement à la suite de ce qui est dans notre édition le chapitre : Propriété et communisme… (pp. 84-99). Dans sa Préface, Marx qualifie cette “analyse critique de la dialectique de Hegel et de sa philosophie en général” de “dernier chapitre”. Elle figure donc comme tel dans notre édition.} où, pour expliquer et justifier les idées développées, il conviendrait de donner quelques indications et sur la dialectique de Hegel en général et, en particulier, sur son exposé dans la \emph{Phénoménologie} et dans la \emph{Logique}, enfin sur le rapport du mouvement critique moderne à Hegel.\par
La critique allemande moderne s’occupa tellement du contenu du monde ancien, bien qu’empêtrée dans son sujet, elle se développa avec une telle force qu’il en résulta un manque complet d’attitude critique à l’égard de la méthode de la critique et une inconscience totale à l’égard de la question apparemment \emph{formelle}, mais réellement \emph{essentielle : Où} en sommes-nous avec \emph{la dialectique} de Hegel ? L’inconscience – au sujet des rapports de la critique moderne à la philosophie de Hegel en général et à la dialectique en particulier – était si grande que des critiques comme \emph{Strauss} \footnote{Il s’agit de David Friedrich STRAUSS dont le livre : Das Leben Jesu parut en 1835.} et \emph{Bruno Butter}, le premier totalement, le second dans ses \emph{Synoptiques} \footnote{Bruno BAUER : Kritik der evangelischen Geschichte der Synoptiker. Bd I-II, Leipzig 1841. Bd. III, Braunschwig 1842.} (où en opposition avec Strauss il remplace par la “conscience de soi” de l’homme abstrait la substance de la “nature abstraite”) et même encore dans \emph{Le Christianisme dévoilé} \footnote{Bruno BAUER : Das entdeckte Christentum. Eine Erinnerung an dos achtzehnte Jahrhundert und ein Beitrag sur Krisis des neunzehnten. Zurich. Winterthur 1843.} furent encore, virtuellement du moins, entièrement empêtrés dans la logique de Hegel. Ainsi, par exemple, nous lisons dans \emph{Le Christianisme} dévoilé : “Comme si la conscience de soi, en posant le monde, en posant la différence, et en se produisant elle-même dam ce qu’elle produit, car elle supprime à nouveau la différence entre ce qu’elle engendre et elle-même, car elle n’est elle-même que dans l’acte d’engendrer et dans son propre mouvement – comme si cette conscience de soi n’avait pas son but dans ce mouvement, etc.” \footnote{Ibid., p. 113.}. Ou encore : “Ils (les matérialistes français) n’ont encore pu comprendre que le mouvement de l’Univers n’est devenu réellement pour soi qu’en tant que mouvement de la conscience de soi et a atteint avec celui-ci l’unité avec lui-même \footnote{Ibid., p. 114 sq.}.” Ces expressions ne différent même pas par le vocabulaire de la conception hégélienne, mais au contraire la répètent littéralement.\par
[XII] Combien, en se livrant à la critique (BAUER \emph{Les Synoptiques)}, ces gens avaient peu conscience de leurs rapports avec la dialectique hégélienne, combien peu cette conscience est née, même une fois accompli l’acte de critique matérielle, Bauer le montre lorsque dans sa \emph{Bonne Cause de la Libertés} \footnote{Bruno BAUER : Die gute Sache der Freiheit und meine eigene Angelegenheit. Zurich und Winterthur 1842. Le passage auquel Marx fait allusion (p. 193 sq.) se rapporte en fait non à Gruppe, mais à Marheinecke.} il écarte la question indiscrète de M. Gruppe : “Qu’en est-il de la Logique” en le renvoyant aux critiques à venir.\par
Mais même maintenant, après que \emph{Feuerbach – tant} dans ses \emph{Thèses} \footnote{Voir la Préface des Manuscrits de 1844.}, dans les \emph{Anekdota}, que d’une manière détaillée dans la \emph{Philosophie de l’avenir} \footnote{Voir la Préface des Manuscrits de 1844.}- a renversé radicalement la vieille dialectique et la vieille philosophie, après que par contre cette fameuse critique, incapable d’accomplir cet acte, mais l’ayant vu accompli, [a'] est proclamée critique pure, décisive, absolue, qui y voit clair en elle-même, après que dans son orgueil spiritualiste elle a ramené tout le mouvement de l’histoire au rapport du reste du monde – qui en face d’elle tombe dam la catégorie de la “masse” \footnote{Marx fait ici allusion aux articles parus dans l’AlIgemeine Literatur Zeitung de Bruno BAUER (Charlottenburg 1844). Il reprendra sa critique d’une manière détaillée dans La Sainte Famille.} – avec elle-même et qu’elle a résolu toutes les oppositions dogmatiques en \emph{la seule} opposition dogmatique entre sa propre sagesse et la sottise du monde. entre le Christ critique et l’humanité en tant que “foule”, après avoir fait, jour après jour et heure après heure, la preuve de sa propre excellence en démontrant l’indigence d’esprit de la masse, après avoir enfin annoncé le jugement dernier critique en déclarant que le jour approchait où toute l’humanité décadente se rassemblerait en face d’elle, séparée par elle en groupes dont chacun se verrait attribuer son certificat d’indigence \footnote{Ce dernier membre de phrase résume le paragraphe final d’un article de HIRZEL dans l’AlIgemeine Literatur Zeitung (cahier 5, p. 15) dont voici le texte : “Lorsque enfin tout le monde s’alliera contre elle (la critique), – et le temps n’en est pas loin,:- quand tout le monde décadent se rassemblera autour d’elle pour le dernier assaut, alors le courage de la critique et sa signification auront trouvé la plus grande approbation. Nous ne sommes pas inquiets sur le résultat. Tout aboutira à ceci : nous réglerons nos comptes avec les groupes individuels et nous établirons un certificat général d’indigence à ces chevaleresques ennemis.”}, après avoir fait imprimer son élévation au-dessus des sentiments humains, ainsi qu’au-dessus du monde, sur lequel, trônant dans une sublime solitude, elle laisse seulement retentir de temps à autre du haut de ses lèvres sarcastiques le rire des Dieux de l’Olympe, – après toutes ces réjouissantes gesticulations de l’idéalisme (des “jeunes hégéliens”) qui agonise sous la forme de la critique, celui-ci n’a même pas fait la plus lointaine allusion à la nécessité d’avoir une explication critique avec sa mère, la dialectique de Hegel, il n’a même [rien] su indiquer sur son attitude critique à l’égard de la dialectique de Feuerbach. Voilà un comportement complètement dénué de critique vis-à-vis de soi-même.\par
Feuerbach est le seul qui ait eu une attitude sérieuse, critique, envers la dialectique hégélienne et qui ait fait de véritables découvertes dans ce domaine ; il est en somme le vrai vainqueur de l’ancienne philosophie. La grandeur de ce qu’il a accompli et la simplicité discrète avec laquelle Feuerbach la livre au monde font un contraste surprenant avec l’attitude inverse des autres.\par
La grande action de Feuerbach est : 1º d’avoir démontré que la philosophie n’est rien d’autre que la religion mise sous forme d’idées et développée par la pensée \footnote{Principes de la \emph{philosophie de} l’avenir, § 5 : “L’essence de la philosophie spéculative n’est rien d’autre que l’essence de Dieu rationalisée, réalisée et actualisée. La philosophie spéculative est la religion vraie, consé\emph{quente et} rationnelle.” (loc. cit., p. 129).}; qu’elle n’est qu’une autre forme et un autre mode d’existence de l’aliénation de l’homme ; donc qu’elle est tout aussi condamnable.\par
2º d’avoir fondé le vrai matérialisme et la science réelle en faisant également du rapport social “de nomme à l’homme” le principe de base de la théorie \footnote{Ibid., § 41 – “La communauté de l’homme avec l’homme est le principe et le critère premiers de la vérité et de l’universalité.” (p. 185). § 59 : “L’homme pour soi ne possède en lui l’essence de, l’homme ni au titre d’être moral, ni au titre d’être pensant. L’essence de l’homme n’est contenue que dans la communauté, dans l’unité de l’homme avec l’homme, unité qui ne repose que sur la réalité de la distinction du moi et du toi.” (p. 198).};\par
3º en opposant à la négation de la négation qui prétend être le positif absolu, le positif fondé positivement sur lui-même et reposant sur lui-même \footnote{Ibid., § 38 : “La vérité qui se médiatise est la vérité encore entachée de son contraire. On commence par le contraire, mais ensuite on le supprime. Mais s’il faut le supprimer et le nier, pourquoi commencer par lui, au lieu de commencer immédiatement par sa négation ?… Pourquoi donc ne pas commencer tout de suite par le concret ? Pourquoi donc ce qui doit sa certitude et sa garantie à soi-même ne serait-il pas supérieur à ce qui doit sa certitude à la nullité de son contraire ?” (pp. 182-183).}.\par
Voici comment Feuerbach explique la dialectique de Hegel – (et il fonde ainsi le point de départ du positif, de la certitude sensible) - :\par
Hegel part de l’aliénation (en termes de Logique : de l’infini, de l’universel abstrait) de la substance, de l’abstraction absolue et immobile – c’est-à-dire en langage populaire il part de la religion et de la théologie.\par
Deuxièmement : il abolit l’Infini ; il pose le Réel, le sensible, le concret, le fini, le particulier (la philosophie, abolition de la religion et de la théologie).\par
Troisièmement : il abolit à son tour le positif ; il rétablit l’abstraction, l’infini. Rétablissement de la religion et de la théologie.\par
Pour Feuerbach la négation de la négation n’est donc que la contradiction de la philosophie avec elle-même, la philosophie qui affirme la théologie (transcendance, etc.) après l’avoir niée, donc l’affirme en opposition avec elle-même \footnote{Voir sur ce point le § 21 des Principes de la Philosophie de l’avenir. Feuerbach y écrit notamment : “Le secret de la dialectique hégélienne ne consiste en définitive qu’à nier la théologie au nom de la philosophie, pour nier ensuite à son tour la philosophie au nom de la théologie. C’est la théologie qui est le commencement et la fin ; au milieu se tient la philosophie qui nie la première position ; mais c’est la théologie qui est la négation de la négation.” (loc. cit., pp. 158/9).}.\par
L’affirmation positive ou l’affirmation et la confirmation de soi, qui est impliquée dans la négation de la négation, est conçue comme n’étant pas encore sûre d’elle-même, donc affectée de son contraire, doutant d’elle-même, donc ayant besoin de preuve, comme ne se prouvant pas elle-même par son existence, comme inavouée, [XIII] et il lui oppose donc directement et sans médiation l’affirmation positive fondée sur elle-même de la certitude sensible \footnote{ \noindent Feuerbach conçoit encore la négation de la négation, le concept concret, comme la Pensée qui se dépasse elle-même dans la pensée et qui, en tant que pensée, veut être immédiatement intuition, nature, réalité (1), (note de Marx).\par
 Marx se réfère ici aux observations de Feuerbach dans les Principes de la \emph{philosophie de} l’avenir. Il dit au § 29 : “La pensée empiétant sur son contraire… est la pensée franchissant ses limites naturelles. La pensée empiète sur son contraire veut dire – la pensée revendique pour elle, non ce qui appartient à la pensée, mais ce qui appartient à l’être. Or c’est la singularité et \emph{l’individualité qui} appartiennent à l’être, et l’universalité à la pensée La pensée… fait de la négation de l’universalité… un moment de la pensée. C’est ainsi que la pensée “abstraite” ou le concept abstrait, qui laisse l’être hors de lui, devient concept concret.” (loc. cit., p. 170). Et au § 30 il dit : “Hegel est un penseur qui renchérit sur lui-même dans la pensée – il veut saisir la chose elle-même, mais dans la pensée de la chose ; il veut être hors de la pensée, mais au sein de la pensée même : d’où la difficulté de concevoir le concept concret.” (Ibid., p. 175).
}.\par
Mais, en considérant la négation de la négation – sous l’aspect positif qu’elle implique comme le seul positif véritable – sous l’aspect négatif qu’elle implique comme le seul acte véritable et comme l’acte de manifestation de soi de tout être, Hegel n’a trouvé que l’expression abstraite, logique, spéculative du mouvement de l’histoire qui n’est pas encore l’histoire réelle de l’homme en tant que sujet donné d’avance, mais qui est seulement l’acte d’engendrement, l’histoire de la naissance de l’homme. – Nous expliquerons et la forme abstraite de ce mouvement chez Hegel et la différence qui lui est propre et l’oppose à la critique moderne, au même processus dans L’Essence du Christianisme de Feuerbach, ou plutôt nous expliquerons la forme critique de ce mouvement qui n’est pas encore critique chez Hegel.\par
Jetons un coup d’œil sur le système de Hegel. Il faut commencer par la Phénoménologie, source véritable et secret de la philosophie de Hegel.
\subsection[{Phénoménologie}]{Phénoménologie\protect\footnotemark }
\footnotetext{Marx reprend ici la table des matières de la Phénoménologie. Il donne textuellement (et parfois avec de légères additions) toute la partie A. Pour les parties B, C et D, il ne cite que les têtes de chapitres. Nous avons adopté ici le texte et la terminologie de la traduction de M. J. Hyppolite (2 vol. Paris, Aubier, 1939).}
\subsubsection[{A. – La Conscience de soi}]{A. – La Conscience de soi}
\noindent I. Conscience. a) Certitude sensible ou le ceci et ma visée du ceci. b) La perception ou la chose avec ses propriétés et l’illusion. c) Force et entendement, phénomène et monde supra-sensible.\par
II. Conscience de soi. La vérité de la certitude de soi-même. a) Indépendance et dépendance de la conscience de soi, domination et servitude. b) Liberté de la conscience de soi. Stoïcisme, scepticisme, la conscience malheureuse.\par
III. Raison. Certitude et vérité de la raison. a) Raison observante ; observation de la nature et de la conscience de soi. b) Actualisation de la conscience de soi rationnelle par sa propre activité. Le plaisir et la nécessité. La loi du cœur et le délire de la présomption. La vertu et le cours du monde. c) L’individualité qui se sait elle-même réelle en soi et pour soi-même. Le règne animal de l’esprit et la tromperie ou la chose même. La raison législatrice. La raison examinant les lois.
\subsubsection[{B. – L’Esprit}]{B. – L’Esprit}
\noindent I. L’esprit vrai ; l’ordre éthique\par
II. L’esprit devenu étranger à soi-même, la culture\par
III. L’esprit certain de soi-même : la moralité.
\subsubsection[{C. – La Religion Religion naturelle. Religion esthétique. Religion révélée.}]{C. – La Religion Religion naturelle. Religion esthétique. Religion révélée.}
\subsubsection[{D. – Le Savoir absolu}]{D. – Le Savoir absolu}
\noindent L’Encyclopédie \footnote{G.W.F. HEGEL :\emph{ Enzyklopädie der} philosophischen Wissenschaften \emph{im} Grundrisse. Cet ouvrage comprend trois parties : 1. La logique ; Il. La philosophie de la nature ; III. La philosophie de l’esprit.} de Hegel commençant par la logique, par la pure pensée spéculative et finissant par le savoir absolu, par l’esprit philosophique ou absolu, c’est-à-dire surhumain et abstrait, conscient de lui-même, se saisissant lui-même, elle n’est dans sa totalité pas autre chose que le déploiement de l’esprit philosophique, son objectivation de soi ; l’esprit philosophique n’est pas autre chose que l’esprit du monde aliéné qui se saisit lui-même mentalement, c’est-à-dire abstraitement, sans sortir de son aliénation de soi. – La logique c’est l’argent de l’esprit, la valeur pensée, spéculative, de l’homme et de la nature – son essence devenue complètement indifférente à toute détermination réelle et pour cela même irréelle – c’est la pensée aliénée, qui fait donc abstraction de la nature et des hommes réels : la pensée abstraite. L’extériorité de cette pensée abstraite… la nature telle qu’elle est pour cette pensée abstraite. Elle est extérieure à l’esprit, elle est sa perte de lui-même ; et il la saisit aussi extérieurement comme une pensée abstraite, comme la pensée abstraite aliénée – enfin l’esprit, cette pensée qui revient à sa propre source, qui sous la forme de l’esprit anthropologique, phénoménologique, psychologique, moral, artistique, religieux, n’estime toujours pas qu’elle est pour soi jusqu’à ce qu’elle se trouve enfin elle-même comme savoir absolu, et par conséquent comme esprit absolu, c’est-à-dire abstrait, jusqu’à ce qu’elle se rapporte à elle-même et reçoive l’existence consciente qui lui convient. Car son existence réelle est l’abstraction.\par
Double erreur chez Hegel.\par
La première apparaît le plus clairement dans la Phénoménologie, source originelle de la philosophie de Hegel. Quand par exemple il a appréhendé la richesse, la puissance de l’État, etc., comme des essences devenues étrangères à l’être humain, il ne les prend que dans leur forme abstraite… Elles sont des êtres pensés – donc seulement une aliénation de la pensée philosophique pure, c’est-à-dire abstraite. C’est pourquoi tout le mouvement se termine par le savoir absolu. Ce dont ces objets sont l’aliénation et qu’ils affrontent en prétendant à la réalité, c’est précisément la pensée abstraite. Le philosophe – lui-même forme abstraite de l’homme aliéné – se donne pour la mesure du monde aliéné. C’est pourquoi toute l’histoire de l’aliénation et toute la reprise de cette aliénation ne sont pas autre chose que l’histoire de la production de la pensée abstraite, c’est-à-dire absolue, [XVII] de la pensée logique spéculative. L’aliénation qui constitue donc l’intérêt proprement dit de ce dessaisissement et de sa suppression est, à l’intérieur de la pensée elle-même, l’opposition de l’En Soi et du Pour Soi, de la conscience et de la conscience de soi, de l’objet et du sujet, c’est-à-dire l’opposition de la pensée abstraite et de la réalité sensible ou du sensible réel. Toutes les autres oppositions et leurs mouvements ne sont que l’apparence, l’enveloppe, la forme exotérique de ces oppositions, les seules intéressantes, qui constituent le sens des autres, les oppositions profanes. Ce qui passe pour l’essence posée et à supprimer de l’aliénation, ce n’est pas que l’être humain s’objective de façon inhumaine, en opposition à lui-même, mais qu’il s’objective en se différenciant de la pensée abstraite et en opposition à elle.\par
[XVIII] Par conséquent l’appropriation des forces essentielles de l’homme, devenues des objets, et des objets étrangers, est en premier lieu une appropriation qui se passe dan\& la conscience, dans la pensée pure, c’est-à-dire dans l’abstraction, elle est l’appropriation de ces objets en qualité de pensées et de mouvements de pensée ; c’est pourquoi déjà dans la Phénoménologie – malgré son aspect tout à fait négatif et critique et malgré la critique qu’elle contient et qui souvent anticipe largement le développement ultérieur – on voit déjà à l’état latent, existant en germe, en puissance, et comme mystère, le positivisme non critique et l’idéalisme pareillement non critique des productions ultérieures de Hegel – cette décomposition et restauration philosophiques de la réalité empirique existante. Deuxièmement. La revendication du retour à l’homme du monde objectif, – par exemple reconnaître que la conscience sensible n’est pas une conscience abstraitement sensible, mais une conscience humainement sensible, que la religion, la richesse, etc., ne sont que la réalité aliénée de l’objectivation humaine, des forces essentielles humaines devenues œuvres et qu’elles ne sont donc que la voie qui mène à la réalité humaine véritable, – cette appropriation ou l’intelligence de ce processus apparaît donc chez Hegel de telle façon que le monde sensible, la religion, le pouvoir de l’État, etc., sont des essences spirituelles – car seul l’esprit est l’essence véritable de l’homme et la forme vraie de l’esprit est l’esprit pensant, l’esprit logique spéculatif. Le caractère humain de la nature et de la nature engendrée par l’histoire, des produits de l’homme, apparaît en ceci qu’ils sont \emph{produits} de l’esprit abstrait et donc, dans cette mesure, des moments de \emph{l’esprit}, des \emph{êtres pensés.} C’est pourquoi \emph{la Phénoménologie} est la critique cachée, encore obscure pour elle-même et mystifiante ; mais dans la mesure où elle retient \emph{l’aliénation} de l’homme, – bien que l’homme n’y apparaisse que sous la forme de l’esprit, – on trouve cachés en elle \emph{tous} les éléments de la critique, et ceux-ci sont déjà souvent \emph{préparés et élaborés} d’une manière qui dépasse de beaucoup le point de vue hégélien. La “conscience malheureuse”, la “conscience honnête”, la lutte de la “conscience noble et de la conscience vile”, etc., chacune de ces sections contient – bien qu’encore sous une forme aliénée – les éléments de la \emph{critique} de domaines entiers comme la religion, l’État, la vie civile, etc. Et de même que l’essence, l’objet est toujours pour lui essence pensée, de même le sujet est toujours conscience ou conscience de soi, ou plus exactement l’objet n’apparaît que comme conscience abstraite et l’homme comme conscience de soi. C’est pourquoi les différentes formes de l’aliénation qui apparaissent dans la Phénoménologie ne sont que des formes variées de la conscience et de la conscience de soi. De même que la conscience abstraite – forme sous laquelle on appréhende l’objet – n’est en soi qu’un moment de différenciation de la conscience de soi, – de même on obtient comme résultat du mouvement l’identité de la conscience de soi et de la conscience, le savoir absolu, le mouvement de la pensée abstraite qui ne se fait plus en direction de l’extérieur, mais seulement au-dedans d’elle-même, c’est-à-dire qu’on obtient pour résultat la dialectique de la pensée pure.\par
[XXIII] La grandeur de la Phénoménologie de Hegel et de son résultat final – la dialectique de la négativité comme principe moteur et créateur – consiste donc, d’une part, en ceci, que Hegel saisit la production de l’homme par lui-même comme un processus, l’objectivation comme désobjectivation, comme aliénation et suppression de cette aliénation ; en ceci donc qu’il saisit l’essence du travail et conçoit l’homme objectif, véritable parce que réel, comme le résultat de son propre travail. Le rapport réel actif de l’homme à lui-même en tant qu’être générique ou la manifestation de soi comme être générique réel, c’est-à-dire comme être humain, n’est possible que parce que l’homme extériorise réellement par la création toutes ses forces génériques – ce qui ne peut à son tour être que par le fait de l’action d’ensemble des hommes, comme résultat de l’histoire, – qu’il se comporte vis-à-vis d’elles comme vis-à-vis d’objets, ce qui à son tour n’est d’abord possible que sous la forme de l’aliénation.\par
Nous allons maintenant exposer dans le détail l’étroitesse et la limitation de Hegel en étudiant le dernier chapitre de la Phénoménologie, le savoir absolu – chapitre qui contient à la fois l’esprit condensé de la Phénoménologie, son rapport à la dialectique spéculative, et également la conscience que Hegel a de l’un et de l’autre et de leurs rapports réciproques.\par
Provisoirement nous ne dirons plus pour anticiper que ceci : Hegel se place du point de vue de l’économie politique moderne. Il appréhende le travail comme l’essence, comme l’essence avérée de l’homme ; il voit seulement le côté positif du travail et non son côté négatif. Le travail est le devenir pour soi de l’homme à l’intérieur de l’aliénation ou en tant qu’homme aliéné. Le seul travail que connaisse et reconnaisse Hegel est le travail abstrait de l’esprit. Ce qui, en somme, constitue donc l’essence de la philosophie, l’aliénation de l’homme qui a la connaissance de soi, ou la science aliénée qui se pense elle-même, Hegel le saisit comme l’essence du travail et c’est pourquoi il peut, face à la philosophie antérieure, rassembler ses divers moments et présenter sa philosophie comme la Philosophie. Ce que les autres philosophes ont fait, – appréhender divers moments de la nature et de la vie humaine comme des moments de la conscience de soi et, qui plus est, de la conscience de soi abstraite, – Hegel le connaît comme l’action de la philosophie. C’est pourquoi sa science est absolue.\par
Passons maintenant à notre sujet.\par
Le Savoir absolu. Dernier chapitre de la Phénoménologie.\par
L’idée essentielle est que l’objet de la conscience n’est rien d’autre que la conscience de soi ou que l’objet n’est que la conscience de soi objectivée, la conscience de soi en tant qu’objet. (Poser l’homme conscience de soi.)\par
Il faut donc dépasser \emph{l’objet} de la conscience. L’objectivité en tant que telle est un rapport aliéné de l’homme, un rapport qui ne correspond pas à l’essence humaine, à la conscience de soi. La réappropriation de l’essence objective de l’homme, engendrée comme étrangère, dans la détermination de l’aliénation, ne signifie donc pas seulement la suppression de l’aliénation, mais aussi de l’objectivité ; c’est-à-dire donc que l’homme est un être non-objectif, spiritualiste.\par
Voici comment Hegel décrit le mouvement de dépassement de l’objet de la conscience :\par
L’objet n’apparaît pas seulement (et c’est, d’après Hegel, la conception unilatérale – qui n’appréhende donc qu’un des côtés – de ce mouvement) comme retournant dans le Soi \footnote{Dans la préface de La Phénoménologie, Hegel écrit : “Dans son com. portement négatif.. la pensée ratiocinante est elle-même le Soi dans lequel le contenu retourne ; par contre, dans sa-connaissance positive, le Soi est un sujet représenté auquel le contenu se rapporte comme accident et prédicat. Ce sujet constitue la base à laquelle le contenu est attaché, base sur laquelle le mouvement va et vient. Il en est tout autrement dans le cas de la pensée concevante. Puisque le concept est le Soi propre de l’objet qui se présente comme son devenir, le Soi n’est pas un sujet en repos supportant passivement les accidents, mais il est le concept se mouvant soi-même et reprenant en soi-même ses déterminations.” (trad. Hyppolite, tome I, p. 52).}. L’homme est posé comme égal au Soi. Mais le Soi n’est que l’homme saisi abstraitement et engendré par abstraction. L’homme est de la nature du Soi \footnote{Marx emploie ici le terme “selbstisch”. Le suffixe “isch” marque à la fois l’origine et la qualité. Nous avançons la traduction : de la nature du soi.}. Son œil, son oreille, etc., sont de la nature du Soi ; chacune de ses forces essentielles a en lui la qualité du Soi \footnote{Marx dit : Selbstigheit, qu’il faudrait traduire par la Soi-ité.}. Mais de ce fait il est maintenant tout à fait faux de dire – la conscience de soi à des yeux, des oreilles, des forces essentielles. C’est plutôt la conscience de soi qui est une qualité de la nature humaine, de l’œil humain, etc., et non la nature humaine qui est une qualité de [XXIV] la conscience de soi.\par
Le Soi abstrait et fixé pour soi est l’homme en tant qu’égoïste abstrait, l’égoïsme élevé à sa pure abstraction, à la pensée. (Nous y reviendrons.)\par
Pour Hegel, l’essence humaine, l’homme, égale la conscience de soi. Par conséquent toute aliénation de l’essence humaine n’est rien qu’aliénation de la conscience de soi. L’aliénation de la conscience de soi n’est pas l’expression, qui se réfléchit dans la pensée et le savoir, de l’aliénation réelle de l’essence humaine. Au contraire, l’aliénation réelle, apparaissant concrètement, n’est d’après son essence cachée la plus intime – et ramenée au jour seulement par la philosophie – rien d’autre que la manifestation de l’aliénation de l’essence humaine réelle, de l’aliénation de la conscience de soi. C’est pourquoi la science qui conçoit cela s’appelle la \emph{Phénoménologie}. Toute réappropriation de l’essence objective aliénée apparaît donc comme une intégration dans la conscience de soi ; l’homme qui se rend maître de son essence n’est que la conscience de soi qui se rend maîtresse de l’essence objective. Le retour de l’objet dans le Soi est donc la réappropriation de l’objet.\par
Exprimé \emph{d’une manière} universelle, le dépassement de l’objet de la conscience consiste en ceci :\par
— 1º L’objet en tant que tel se présente à la conscience sur le point de disparaître ; 2º c’est l’aliénation de la conscience de soi qui pose la choséité ; 3º cette aliénation a une signification non seulement négative, mais positive ; 4º elle ne l’a pas seulement pour nous ou en soi, mais encore pour elle-même ; 5º pour elle \footnote{C’est-à-dire pour la conscience de soi.}, le négatif de l’objet ou l’autosuppression de celui-ci a une signification positive, en d’autres termes la conscience de soi sait cette nullité de l’objet parce qu’elle s’aliène elle-même, car dans cette aliénation elle se pose soi-même comme objet, ou, en vertu de l’unité indivisible de l’Être-pour-soi, elle pose l’objet comme soi-même. 6º D’autre part cela implique en même temps cet autre moment qu’elle a et supprimé et repris en elle-même cette aliénation et cette objectivité et qu’elle est donc dans son être autre en tant que tel près de soi-même. 7º Tel est le mouvement de la conscience et elle est donc la totalité de ses moments. 8º Elle doit de même se rapporter à l’objet selon la totalité de ses déterminations et l’avoir ainsi appréhendé selon chacune d’entre elles. Cette totalité de ses déterminations élève en soi l’objet à l’essence spirituelle et, pour la conscience, il devient cela en vérité par l’appréhension de chacune de ses déterminations singulières comme le Soi ou par le comportement spirituel envers elles déjà mentionné \footnote{Marx a reproduit ici à peu près textuellement un passage du premier paragraphe (Le contenu simple du Soi qui se prouve comme l’être) du chapitre : “Le Savoir absolu”. (cf. traduction Hyppolite, tome II, pp. 293-294).}.\par
À propos de 1º. Le fait que l’objet en tant que tel se présente à la conscience sur le point de disparaître est le retour mentionné ci-dessus de l’objet dans le Soi.\par
À propos de 2º. \emph{L’aliénation de} la conscience de soi pose la choséité. Comme l’homme = la conscience de soi, son être objectif aliéné ou la choséité – (ce qui est objet pour lui, et n’est véritablement objet pour lui que ce qui est pour lui objet essentiel, ce qui est donc son être \emph{objectif. Comme} ce n’est pas l’homme réel en tant que tel, que ce n’est donc pas la nature non plus qui devient sujet, – l’homme n’est pas autre chose que la nature humaine,- mais seulement l’abstraction de l’homme, la conscience de soi, la choséité ne peut être que la conscience de soi aliénée) égale la conscience de soi aliénée, et la choséité est posée par cette aliénation. Il est tout à fait naturel qu’un être vivant, naturel, doué et pourvu de forces essentielles objectives, c’est-à-dire matérielles, ait des \emph{objets réels et naturels} de son être, et aussi que son aliénation de soi pose un monde objectif \emph{réel}, mais se présentant sous la forme de \emph{l’extériorité}, n’appartenant donc pas à son essence et le dominant. Il n’y a là rien d’incompréhensible ni d’énigmatique. C’est le contraire qui le serait. Mais il est tout aussi évident qu’une \emph{conscience de soi} ne peut poser, par son aliénation, que la \emph{choséité}, c’est-à-dire seulement une chose elle-même abstraite, une chose de l’abstraction, et non pas une chose réelle. Il est [XXVI] en outre évident que la choséité n’est donc absolument rien d’indépendant, d’essentiel par rapport à la conscience de soi, mais n’est qu’une simple création, quelque chose qu’elle a posé, et que le posé, au lieu de s’affirmer lui-même, n’est qu’une affirmation de l’acte de poser qui cristallise pour un instant son énergie sous la forme du produit et qui en apparence – mais pour un instant seulement – lui confère le rôle d’un être indépendant, réel.\par
Quand l’homme réel, en chair et en os, campé sur la terre solide et bien ronde, l’homme qui aspire et expire toutes les forces de la nature, pose ses forces essentielles objectives réelles par son aliénation comme des objets étrangers, ce n’est pas le fait de poser qui est sujet ; c’est la subjectivité de forces essentielles objectives, dont l’action doit donc être également objective. L’être objectif agit d’une manière objective et il n’agirait pas objectivement si l’objectivité n’était pas incluse dans la détermination de son essence. Il ne crée, il ne pose que des objets, parce qu’il est posé lui-même par des objets, parce qu’à l’origine il est Nature. Donc, dans l’acte de poser, il ne tombe pas de son “activité pure” dans une création de l’objet, mais son produit \emph{objectif ne} fait que confirmer son activité \emph{objective}, son activité d’être objectif naturel.\par
Nous voyons ici que le naturalisme conséquent, ou humanisme, se distingue aussi bien de l’idéalisme que du matérialisme et qu’il est en même temps leur vérité qui les unit. Nous voyons en même temps que seul le naturalisme est capable de comprendre l’acte de l’histoire universelle.\par
L’homme est immédiatement être de la nature. En qualité d’être naturel, et d’être naturel vivant, il est d’une part pourvu de forces naturelles, de forces vitales ; il est un être naturel actif ; ces forces existent en lui sous la forme de dispositions et de capacités, sous la forme d’inclinations. D’autre part, en qualité d’être naturel, en chair et en os, sensible, objectif, il est, pareillement aux animaux et aux plantes, un être passif, dépendant et limité ; c’est-à-dire que les objets de ses inclinations existent en dehors de lui, en tant qu’objets indépendants de lui ; mais ces objets sont objets de ses besoins ; ce sont des objets indispensables, essentiels pour la mise en jeu et la confirmation de ses forces essentielles. Dire que l’homme est un être en chair et en os, doué de forces naturelles, vivant, réel, sensible, objectif, c’est dire qu’il a pour objet de son être, de la manifestation de sa vie, des objets réels, sensibles, et qu’il ne peut manifester sa vie qu’à l’aide d’objets réels, sensibles \footnote{Feuerbach écrit dans les Principes \emph{de} la \emph{philosophie} de l’avenir : “… car seul un être \emph{sensible} a besoin pour exister de choses extérieures à lui. J’ai besoin d’air pour respirer, d’eau pour boire, de lumière pour voir, de substances végétales et animales pour manger ; mais je n’ai besoin de rien, du moins immédiatement, pour penser. Un être qui respire est impensable sans l’air, un être qui voit, impensable sans la lumière, mais l’être pensant, je puis le penser à part, pour soi. L’être qui respire se rapporte nécessairement à un être extérieur à lui. Son objet essentiel, qui le fait ce \emph{qu’il} ut, est extérieur à lui ; l’être pensant, lui, se rapporte à lui-même : il est son propre objet, il a son essence en lui-même, il est par lui-même ce qu’il est.” (loc. cit., p. 131).}. Être objectif, naturel, sensible, c’est la même chose qu’avoir en dehors de soi objet, nature, sens ou qu’être soi-même objet, nature, sens pour un tien. La faim est un besoin naturel ; c’est pourquoi, pour la satisfaire, pour la calmer, il lui faut une nature, un \emph{objet en} dehors d’elle. La faim c’est le besoin avoué qu’a mon corps d’un \emph{objet qui} se trouve en dehors de lui, qui est nécessaire pour le compléter et manifester son être. Le soleil est \emph{l’objet} de la plante, un objet qui lui est indispensable et qui confirme sa vie ; de même, la plante est l’objet du soleil en tant qu’elle manifeste la force vivifiante du soleil, la force essentielle objective du soleil \footnote{Dans l’Introduction à L’Essence du Christianisme, Feuerbach écrit : “Or l’objet auquel un sujet se rapporte par essence et par nécessité n’est rien d’autre que l’essence propre de ce sujet, mais objectivée.” (loc. cit., p. 61. Voir aussi à ce sujet la note 3, p. 96).}.\par
Un être qui n’a pas sa nature en dehors de lui n’est pas un être naturel, il ne participe pas à l’être de la nature. Un être qui n’a aucun objet en dehors de lui n’est pas un être objectif. Un être qui n’est pas lui-même objet pour un troisième être n’a aucun être pour \emph{objet, c’est-à-dire} ne se comporte pas de manière objective, son être n’est pas objectif.\par
[XXVII] Un être non-objectif est un non-être \footnote{Nous traduisons le terme de Unwesen par non-être. Mais ce mot signifie aussi monstre, absurdité. (voir note 1, p. 81.)} (Unwesen).\par
Supposez un être qui n’est pas objet lui-même et qui n’a pas d’objet. Un tel être serait, premièrement, être unique ; en dehors de lui il n’y aurait aucun être, il existerait seul et dans sa solitude. Car dès que des objets existent en dehors de moi, dès que je ne suis pas seul, je suis un autre, une autre réalité que l’objet en dehors de moi. Donc, pour ce troisième objet, je suis une autre réalité que lui, c’est-à-dire que je suis son objet. Un être qui n’est pas l’objet d’un autre être suppose donc qu’il n’existe aucun être objectif. Dès que j’ai un objet, cet objet m’a comme objet. Mais un être non \emph{objectif, c’est} un être non réel, non sensible, mais seulement pensé, c’est-à-dire seulement imaginé, un être d’abstraction. Être doué \emph{de} sens, c’est-à-dire être réel, c’est être objet des sens, objet sensible, donc avoir en dehors de soi des objets sensibles, der, objets de ses sens. Avoir des sens signifie souffrir \footnote{Nous donnons ici au “leidend sein” employé par Marx son sens fort, alors que nous avons précédemment traduit par passif. Mais il va introduire l’idée d’être passionné, et à l’origine de la passion il y a un manque, une souffrance que l’homme cherche à compenser.}.\par
C’est pourquoi l’homme, en tant qu’être objectif sensible, est un être qui souffre et comme il est un être qui ressent sa Souffrance, il est un être passionné. La passion est la force essentielle de l’homme qui tend énergiquement vers son objet \footnote{FEUERBACH : Thèses provisoires § 43 : “Sans limite, temps, ni souffrance, il n’est non plus ni qualité, ni énergie, ni esprit, ni flamme, ni amour. Seul l’être nécessiteux est l’être nécessaire. Une existence sans besoin est une existence superflue… Un être sans souffrance est un être sans fondement. Seul mérite d’exister celui qui peut souffrir. Seul l’être douloureux est un être divin. Un dire sans affection est un dire sans être.” (loc. cit., p. 115).}.\par
Mais l’homme n’est pas seulement un être naturel, il est aussi un être naturel humain ; c’est-à-dire un être existant pour soi, donc un être générique, qui doit se confirmer et se manifester en tant que tel dans son être et dans son savoir. Donc, ni les objets humains ne sont objets naturels tels qu’ils s’offrent immédiatement, ni le sens humain tel qu’il est immédiatement, objectivement, n’est \emph{la} sensibilité humaine, l’objectivité humaine. Ni la nature – au sens objectif – ni la nature au sens subjectif n’existent immédiatement d’une manière adéquate à l’être humain. Et de même que tout ce qui est naturel doit naître, de même l’homme a aussi son acte de naissance, l’histoire, mais elle est pour lui une histoire connue et par suite, en tant qu’acte de naissance, elle est un acte de naissance qui se supprime consciemment lui-même. L’histoire est la véritable histoire naturelle de l’homme – (y revenir).\par
Troisièmement, comme le fait de poser la choséité n’est lui-même qu’une apparence, un acte qui contredit l’essence de l’activité pure, il doit à son tour être supprimé, la choséité doit être niée.\par
Sur les points 3, 4, 5, 6 : 3º Cette aliénation de la conscience a une signification non seulement négative, mais aussi positive et 4º elle a cette signification positive non seulement pour nous ou en Soi, mais aussi pour elle-même, pour la conscience. 5º Pour elle \footnote{Pour “la conscience de soi”.} le négatif de l’objet ou l’auto-suppression de celui-ci a une signification positive (ou elle sait la nullité de l’objet) parce qu’elle s’aliène elle-même, car dans cette aliénation elle se sait objet, ou elle sait l’objet comme elle-même, en vertu de l’unité indivisible de l’Être-pour-Soi. 6º D’autre part, cela implique en même temps cet autre moment qu’elle a et supprimé et repris en elle. même cette aliénation et cette objectivité et qu’elle est donc, dans son être autre en tant que tel près de soi-même.\par
Ainsi que nous l’avons vu, l’appropriation de l’être objectif aliéné, ou la suppression de l’objectivité dans la détermination de l’aliénation – laquelle va nécessairement du caractère étranger indifférent jusqu’à l’aliénation hostile réelle – signifie en même temps, ou même principalement, pour Hegel, la suppression de \emph{l’objectivité, parce} que ce n’est pas le caractère déterminé de l’objet, mais son caractère \emph{objectif, qui} est pour la conscience de soi l’incongruité et l’aliénation. L’objet est donc un négatif, quelque chose qui se supprime soi-même, une nullité. Cette nullité de l’objet a pour la conscience un sens non seulement négatif, mais un sens positif, car cette nullité de l’objet est précisément l’auto-confirmation de la non-objectivité de celui-ci, de [XXVIII] son abstraction. Pour la conscience elle-même, la nullité de l’objet a une signification positive parce qu’elle connaît cette nullité, l’être objectif comme son aliénation de soi, qu’elle sait qu’il n’existe que par cette aliénation de soi…\par
La façon dont la conscience existe et dont les choses existent pour elle est le savoir. Le savoir est son acte unique. C’est pourquoi quelque chose existe pour la conscience dans la mesure où elle connaît ce \emph{quelque chose}. Savoir est son seul comportement objectif. – Or la conscience sait la nullité de l’objet, c’est-à-dire que l’objet ne se distingue pas d’elle, elle sait le non-être de l’objet pour elle – parce qu’elle sait que l’objet est son aliénation de soi, c’est-à-dire elle se connaît elle-même – le savoir comme objet – parce que l’objet n’est que l’apparence d’un objet, je ne sais quel mirage, mais par son essence il n’est rien d’autre que le savoir lui-même qui s’oppose à soi-même et qui s’est donc opposé une nullité, quelque chose qui n’a point d’objectivité en dehors du savoir ; en d’autres termes, le savoir sait qu’en tant qu’il se rapporte à un objet, il est seulement en \emph{dehors de} roi, qu’il s’aliène ; que \emph{lui-même} ne fait que s’apparaître comme objet, ou bien que ce qui \emph{lui} apparaît comme objet n’est que lui-même.\par
D’autre part, dit Hegel, cela implique en même temps cet autre moment : que la conscience de soi a et supprimé et repris en elle même cette aliénation et cette objectivité et qu’elle est donc \emph{dans son être-autre en tant que tel près de soi-même.}\par
Dans ce raisonnement, nous trouvons rassemblées toutes les illusions de la spéculation.\par
\emph{Premièrement.} La conscience, la conscience de soi se trouve – \emph{dans son être-autre en tant que tel près de soi-même.} Elle se trouve donc – ou si nous faisons abstraction de l’abstraction hégélienne et que nous remplaçons la conscience de soi par la conscience de soi de l’homme, – elle se trouve donc \emph{auprès de soi dans son être-autre en tant que tel.} Cela implique d’une part que la conscience – le savoir – en tant que savoir, – la pensée en tant que pensée, – prétend être immédiatement l’autre de soi-même, prétend être le monde sensible, la réalité, la vie. C’est la pensée qui renchérit sur elle-même dans la pensée (Feuerbach). Cet aspect est impliqué ici dans la mesure où la conscience en tant que conscience seulement ne se scandalise pas de l’objectivité aliénée, mais de \emph{l’objectivité en tant que telle.}\par
\emph{Deuxièmement}, cela implique que pour autant que l’homme conscient de soi a reconnu comme aliénation de soi et a supprimé le monde spirituel, – ou l’existence spirituelle universelle de son monde, – il réaffirme pourtant ce monde sous cette forme aliénée, le donne pour son existence véritable, le restaure, prétend que l’homme se trouve \emph{auprès de soi dans son être-autre en tant que tel.} Et ainsi, après avoir supprimé, par exemple, la religion, après avoir reconnu en elle un produit de l’aliénation de soi, il trouve cependant sa confirmation dans \emph{la religion en tant que religion.} C’est là que gît la racine du \emph{faux} positivisme de Hegel et de son criticisme qui n’est qu’apparent ; ce que Feuerbach appelle poser, nier et rétablir la religion et la théologie \footnote{Ibid., § 21 : “La contradiction de la philosophie moderne, du panthéisme en particulier, qui nie la théologie du point de vue de la théologie, ou transforme à nouveau en théologie la négation de la théologie : cette contradiction est particulièrement caractéristique de la philosophie hégélienne.” (p. 156). “Ainsi dès le principe suprême de la philosophie de Hegel, nous trouvons le principe et le résultat de sa philosophie de la religion, savoir que la philosophie, loin de supprimer les dogmes de la théologie, se contente de les rétablir à partir de la négation du rationalisme, et de les médiatiser. Le secret de la dialectique hégélienne ne consiste en définitive qu’à nier la théologie au nom de la philosophie, pour nier ensuite à son tour la philosophie au nom de la théologie. C’est la théologie qui est le commencement et la fin ; au milieu se tient la philosophie, qui nie la première position ; mais c’est la théologie qui est la négation de la négation.” (pp. 158-159).}, mais qu’on peut saisir d’une manière plus universelle. Donc la raison se trouve auprès de soi dans la déraison en tant que déraison. L’homme qui a reconnu que dans le droit, dans la politique, etc., il mène une vie aliénée, mène dans cette vie aliénée en tant que telle sa vie humaine véritable. L’affirmation de soi, la confirmation de soi en \emph{contradiction} avec soi-même, tant avec le savoir qu’avec l’essence de l’objet, c’est le vrai \emph{savoir} et la vraie \emph{vie.}\par
Ainsi, il ne peut même plus être question de concessions faites par Hegel à la religion, à l’État, etc., car ce mensonge est le mensonge de son principe même.\par
[XXIX] Si je \emph{sais} que la religion est la conscience de soi \emph{aliénée} de l’homme, je sais donc que dans la religion en tant que telle, ce n’est pas ma conscience de soi, mais ma conscience de soi aliénée qui trouve sa confirmation. Donc je sais alors que ma conscience de soi qui relève d’elle-même, de son essence, s’affirme non dans la \emph{religion}, mais au contraire dans la religion \emph{anéantie, abolie.}\par
C’est pourquoi chez Hegel la négation de la négation n’est pas la confirmation de l’essence véritable, précisément par la négation de l’essence apparente, mais la confirmation de l’essence apparente ou de l’essence aliénée à soi dans sa négation, ou encore la négation de cette essence apparente en tant qu’essence objective, résidant en dehors de l’homme et indépendante de lui, et sa transformation en sujet.\par
C’est un rôle propre que joue donc le dépassement \footnote{Nous avons traduit jusqu’ici le mot Aufhebung par suppression, abolition. Mais, dans le passage qui suit, Marx examine en particulier la notion hégélienne d’Aujhebung, qui est chez Hegel à la fois suppression et conservation. Dans la Logique (1ᵉʳ Livre, ire partie, chapitre I, Remarque), il écrit : “Aufheben a dans le langage ce double sens : le mot signifie quelque chose comme conserver, garder, et en même temps quelque chose comme faire cesser, mettre fin. Le fait de conserver lui-même implique déjà ce côté négatif, pour la garder, on soustrait la chose à son immédiateté et par suite à un être-là ouvert aux influences extérieures. Ainsi ce qui est supprimé est en même temps quelque chose de conservé, qui a seulement perdu son immédiateté, mais n’est pas pour autant anéanti.” Nous utiliserons donc dans ce sens le terme de dépassement.} \emph{(Aufhebung)} dans lequel sont liées la négation et la conservation, l’affirmation.\par
Ainsi par exemple dans la \emph{Philosophie du Droit} de Hegel, le \emph{droit privé} dépassé égale moralité, la moralité dépassée égale \emph{famille}, la famille dépassée égale \emph{société civile}, la société civile dépassée égale \emph{État}, l’État dépassé égale \emph{histoire universelle} \footnote{Marx donne ici l’enchaînement des principaux concepts de la philosophie du droit de Hegel, concepts qui constituent les principales parties du livre.}. Dans la \emph{réalité}, le droit privé, la morale, la famille, la société civile, l’État, etc., demeurent, mais ils sont devenus des moments, des existences et des modes d’être de l’homme, qui n’ont pas de valeur pris à part, qui se dissolvent et s’engendrent l’un l’autre. Moments du mouvement.\par
Dans leur existence réelle, leur essence mobile est cachée. Celle-ci n’apparaît, ne se révèle que dans la pensée, la philosophie, et c’est pourquoi ma véritable existence religieuse est mon existence dans la philosophie de la religion, ma véritable existence politique est mon existence dans la philosophie du droit, ma véritable existence naturelle est mon existence dans la philosophie de la nature, ma véritable existence artistique est mon existence dans la philosophie de l’art, ma véritable existence humaine est mon existence philosophique. De même, la véritable existence de la religion, de l’État, de la nature, de l’art, c’est la philosophie de la religion, la philosophie de la nature, la philosophie de l’État, la philosophie de l’art. Mais si seule la philosophie de la religion, etc., est pour moi la véritable existence de la religion, je ne suis aussi véritablement religieux qu’en tant que philosophe de la religion, ce qui me fait nier la religiosité réelle et l’homme réellement religieux. Mais en même temps je les confirme aussi, soit à l’intérieur de ma propre existence, soit à l’intérieur de celle d’autrui que je leur oppose, car celle-ci n’est que leur expression philosophique ; soit dans leur forme primitive propre, car ils ont pour moi la valeur de l’Être-autre seulement apparent, d’allégories, de figures cachées sous des envelopper, sensibles de leur propre existence vraie, c’est-à-dire de mon existence philosophique.\par
De même que la qualité dépassée égale quantité, la quantité dépassée égale mesure, la mesure dépassée égale essence, l’essence dépassée égale phénomène, le phénomène dépassé égale réalité, la réalité dépassée égale concept, le concept dépassé égale objectivité, l’objectivité dépassée égale idée absolue, ridée absolue dépassée égale nature, la nature dépassée égale esprit subjectif, l’esprit subjectif dépassé égale esprit moral, objectif, l’esprit moral dépassé égale art, l’art dépassé égale religion, la religion dépassée égale savoir absolu \footnote{Marx donne ici l’enchaînement des concepts tel qu’il résulte de la division et \emph{du} plan de \emph{l’encyclopédie.}}.\par
D’une part ce dépassement est un dépassement de l’être pensé, donc la propriété privée pensée se dépasse dans l’idée de la morale. Et comme la pensée s’imagine qu’elle est immédiatement l’autre de soi-même, qu’elle est la réalité sensible, comme par conséquent son action a pour elle valeur d’action réelle sensible, ce dépassement par la pensée, qui laisse en réalité son objet intact, croit l’avoir réellement surmonté ; d’autre part, comme cet objet est devenu pour elle un moment de la pensée, dans sa réalité il a donc aussi pour elle valeur d’auto-confirmation d’elle-même, de la conscience de soi, de l’abstraction.\par
[XXX] D’un côté cette existence que Hegel dépasse en la transférant dans la philosophie n’est donc pas la religion, l’état, la nature réelle, mais la religion déjà en qualité d’objet du savoir, la dogmatique, et de même la jurisprudence, la science politique et la science de la nature. D’un côté, il est donc en opposition et avec l’être réel et avec la science immédiate non-philosophique ou les concepts non-philosophiques de cet être. Par suite, il contredit les concepts courants.\par
D’autre part, l’homme religieux, etc., peut trouver chez Hegel sa confirmation finale.\par
Considérons maintenant les moments positifs de la dialectique de Hegel – à l’intérieur de la détermination de l’aliénation.\par

\begin{listalpha}[itemsep=0pt,]
\item Le dépassement, mouvement objectif reprenant en lui l’aliénation. –
\end{listalpha}

\noindent C’est, exprimée à l’intérieur de l’aliénation, ridée de l’appropriation de l’essence objective par la suppression de son aliénation. C’est la compréhension aliénée de l’objectivation réelle de l’homme, de l’appropriation réelle de son essence objective par l’anéantissement de la détermination aliénée du monde objectif, par sa suppression dans son existence aliénée, – de même que l’athéisme, suppression de Dieu, est le devenir de l’humanisme théorique, que le communisme, abolition de la propriété privée, est la revendication de la vie réelle de l’homme comme sa propriété, le devenir de l’humanisme pratique ; en d’autres termes, l’athéisme est l’humanisme ramené à lui-même par le moyen terme de la suppression de la religion, le communisme est l’humanisme ramené à lui-même par celui de l’abolition de la propriété privée. Ce n’est que par la suppression de ce moyen terme – qui est toutefois une condition préalable nécessaire – que naît l’humanisme qui part positivement de lui-même, l’humanisme positif.\par
Mais l’athéisme et le communisme ne sont pas une fuite, une abstraction, une perte du monde objectif engendré par l’homme, une perte de ses forces essentielles qui ont pris une forme objective. Ils ne sont pas une pauvreté qui retourne à la simplicité contre nature et non encore développée. Ils sont bien plutôt, pour la première fois, le devenir réel, la réalisation devenue réelle pour l’homme de son essence, et de son essence en tant qu’essence réelle.\par
En considérant le sens positif de la négation rapportée à elle-même – bien qu’à nouveau d’une manière aliénée – Hegel saisit donc l’aliénation de soi, l’aliénation de l’essence, la perte d’objectivité et de réalité de l’homme comme la prise de possession de soi, la manifestation de l’essence, l’objectivation, la réalisation.\par
Bref il saisit – à l’intérieur de l’abstraction – le travail comme l’acte d’engendrement de l’homme par lui-même, le rapport à soi-même comme à un être étranger et la manifestation de soi en tant qu’être étranger comme la conscience générique et la vie générique en devenir.\par
b) Mais chez Hegel – abstraction faite, ou plutôt comme conséquence, de la perversion que nous avons déjà décrite – cet acte apparaît d’une part comme un acte seulement formel, parce qu’abstrait, car l’être humain lui-même n’a de valeur que comme être pensant abstrait, comme conscience de soi ; et deuxièmement, parce que la conception en est formelle et abstraite, la suppression de l’aliénation se change en confirmation de l’aliénation. Autrement dit, pour Hegel, ce mouvement d’engendrement de soi, d’objectivation de soi, en tant qu’aliénation et dessaisissement de soi, est la manifestation absolue de la vie humaine, et par conséquent la dernière, celle qui est son propre but et qui est apaisée en elle-même, qui est parvenue à son essence.\par
Sous sa forme [XXXI] abstraite, en tant que dialectique, ce mouvement passe donc pour la vie véritablement humaine, et comme elle est tout de même une abstraction, une aliénation de la vie humaine, elle passe pour le processus divin, mais pour le processus divin de l’homme – processus par lequel passe son essence différente de lui, abstraite, pure, absolue.\par
Troisièmement : Il faut que ce processus ait un agent, un sujet mais ce sujet n’apparaît que comme résultat ; c’est pourquoi ce résultat, le sujet qui se connaît lui-même comme la conscience de soi absolue, est Dieu, l’Esprit absolu, l’Idée qui se connaît et se manifeste. L’homme réel et la nature réelle deviennent de simples prédicats, des symboles de cet homme irréel caché et de cette nature irréelle \footnote{Feuerbach écrit dans les Thèses provisoires (§ 51) – “Chez Hegel la pensée est l’être ; la pensée est le sujet, l’être est le prédicat. La Logique est la pensée dans l’élément de la pensée, ou la pensée qui se pense elle-même, la pensée comme sujet sans prédicat ou la pensée qui est à la fois sujet et son propre prédicat.” (loc. cit., p. 120).}. Sujet et prédicat sont donc dans un rapport d’inversion absolue à l’égard l’un de l’autre ; c’est le sujet-objet mystique ou la subjectivité qui déborde l’objet, le sujet absolu en tant que processus (le sujet s’aliène, revient à lui-même du fond de cette aliénation, mais la reprend en même temps en lui-même) et le sujet en tant que ce processus ; c’est le mouvement circulaire pur, incessant, en soi-même.\par
Premier point. Conception formelle et abstraite de l’acte d’auto-engendrement et d’auto-objectivation de l’homme.\par
L’objet devenu étranger, la réalité essentielle aliénée de l’homme – puisque Hegel pose l’homme égale la conscience de soi – ne sont rien que conscience, que l’idée de l’aliénation, l’expression abstraite, et par conséquent vide et irréelle de celle-ci, la négation. La suppression de l’aliénation n’est donc également rien qu’une suppression abstraite et vide de cette abstraction vide, la négation de la négation. L’activité substantielle, vivante, sensible, concrète de l’objectivation de soi devient donc sa pure abstraction, la négativité absolue, abstraction qui, à son tour, est fixée comme telle et qui est pensée comme une activité indépendante, comme l’activité à l’état pur. Or, comme la dite négativité n’est pas autre chose que la forme abstraite et vide de cet acte vivant, réel, son contenu ne peut être aussi qu’un contenu formel, produit en faisant abstraction de tout contenu. C’est pourquoi ce sont les formes générales abstraites de l’abstraction, propres à tout contenu et par suite aussi bien indifférentes à tout contenu que valables pour chacun d’eux, ce sont les formes de la pensée, les catégories logiques, détachées de l’esprit réel et de la nature réelle. (Nous développerons plus loin le contenu logique de la négativité absolue.)\par
Ce que Hegel a réalisé ici de positif, – dans sa Logique spéculative – c’est d’avoir fait des concepts déterminés, des formes universelles fixes de la pensée, dans leur indépendance à l’égard de la nature et de l’esprit, le résultat nécessaire de l’aliénation générale de l’être humain, donc aussi de la pensée de l’homme, et de les avoir en conséquence présentés et groupés comme des moments du processus d’abstraction. Par exemple, l’être dépassé est l’essence, l’essence dépassée est le concept, le concept dépassé… l’Idée absolue. Mais qu’est-ce que l’Idée absolue ? Elle se dépasse elle-même à son tour, si elle ne veut pas repasser depuis le début par tout l’acte d’abstraction et se contenter d’être une totalité d’abstractions ou l’abstraction qui se saisit elle-même. Mais l’abstraction qui se saisit elle-même comme abstraction se connaît comme n’étant rien ; elle doit s’abandonner elle-même, abandonner l’abstraction, et ainsi elle arrive auprès d’un être qui est son contraire direct, la \emph{Nature.} La Logique tout entière est donc la preuve que la pensée abstraite n’est rien pour elle-même, pas plus que l’Idée absolue, que seule la \emph{nature} est quelque chose.\par
[XXXII] L’Idée absolue, l’Idée \emph{abstraite}, qui \emph{“considérés} selon son unité avec elle-même est contemplation \footnote{Anschauung. Nous traduisons par contemplation, au sens d’intuition, de vue directe.}” (HEGEL : Encyclopédie, 3ᵉ édit., p. 222), qui “dans la vérité absolue d’elle-même \emph{se résout} à faire sortir librement d’elle le moment de sa particularité ou de la première détermination et de l’être-autre, \emph{l’idée immédiate} en tant que son reflet, à \emph{se faire sortir librement d’elle-même en tant que nature”} \footnote{HEGEL : Encyclopédie, 3ᵉ édit., p. 222 / § 244/. (Note de Marx.)}\emph{, toute} cette Idée qui se comporte d’une façon si étrange et si baroque et à propos de laquelle les hégéliens se sont terriblement cassé la tête, n’est absolument rien d’autre que \emph{l’abstraction}, c’est-à-dire le penseur abstrait. Instruite par l’expérience et éclairée sur sa vérité, elle se résout, sous de multiples conditions – fausses et encore abstraites elles-mêmes – à \emph{renoncer à elle} et à poser son être-autre, le particulier, le déterminé, à la place de son être-auprès-de-soi, de son non-être, de son universalité et de son indétermination ; elle se résout \emph{à faire sortir librement d’elle-même la nature}, qu’elle ne cachait en elle que comme abstraction, comme idée, c’est-à-dire à abandonner l’abstraction et à regarder enfin la nature qu’elle \emph{a fait sortir} d’elle. L’Idée abstraite, qui devient immédiatement \emph{contemplation}, n’est pas autre chose que la pensée abstraite qui renonce à elle-même et se résout à la \emph{contemplation.} Tout ce passage de la \emph{Logique} à la Philo\emph{sophie de la Nature} n’est pas autre chose que le passage – si difficile à réaliser pour le penseur abstrait et par suite décrit par lui de manière si extravagante – de \emph{l’abstraction à la contemplation.} Le sentiment \emph{mystique}, qui pousse le philosophe à quitter la pensée abstraite pour la contemplation, est \emph{l’ennui}, la nostalgie d’un contenu.\par
(L’homme devenu étranger à soi-même est aussi le penseur devenu étranger à son \emph{essence}, c’est-à-dire à l’essence naturelle et humaine. C’est pourquoi ses idées sont des esprits figés qui résident en dehors de la nature et de l’homme. Dans sa \emph{Logique}, Hegel a rassemblé et enfermé tous ces esprits figés et a considéré chacun d’eux, d’abord comme négation, c’est-à-dire comme \emph{aliénation} de la pensée de \emph{l’homme, puis} comme négation de la négation, c’est-à-dire comme suppression de cette aliénation, comme manifestation \emph{réelle} de la pensée humaine ; mais – comme il est encore lui-même prisonnier de l’aliénation – cette négation de la négation est soit le rétablissement de ces esprits figés dans leur aliénation, soit le fait de s’arrêter au dernier acte, – de se rapporter à soi-même dans l’aliénation qui est l’existence vraie de ces esprits figés \footnote{C’est-à-dire que. Hegel remplace ces abstractions figées par l’acte tournant en cercle en lui-même de l’abstraction ; en cela il a évidemment le mérite d’avoir montré la source de tous ces concepts inadéquats qui, d’après leur date d’origine, sont propres à divers philosophes, de les avoir rassemblés et d’avoir créé comme objet de la critique au lieu d’une abstraction déterminée l’abstraction complète, dans toute son extension (nous verrons plus loin pourquoi Hegel sépare la pensée du sujet ; mais dès maintenant il est évident que si l’homme n’est pas, la manifestation de son essence ne peut pas être humaine non plus ; donc la pensée ne pouvait pas non plus être conçue comme la manifestation de l’essence de l’homme en tant qu’il est un sujet humain et naturel, doué d’yeux, d’oreilles, etc., vivant dans la société, le monde et la nature). \emph{(Note de Marx.)}} ; soit encore dans la mesure où cette abstraction se saisit elle-même et ressent un ennui infini de soi-même, l’abandon de la pensée abstraite qui se meut, seulement dans la Pensée, qui n’a ni œil, ni dent, ni oreille, ni rien, apparaît chez Hegel comme la décision de reconnaître la \emph{nature} en tant qu’essence et de se consacrer à la contemplation.)\par
[XXXIII] Mais même la \emph{nature}, prise abstraitement, isolée, fixée dans la séparation de l’homme, n’est \emph{rien} pour lui. Il est évident que le penseur abstrait qui s’est résolu à la contemplation la contemple abstraitement. Comme la nature était enfermée par le penseur dans sa propre personne qui lui était encore cachée et énigmatique, sous forme d’idée absolue, de chose pensée, il a en vérité, en la libérant de soi, fait seulement sortir de lui cette \emph{nature abstraite}, cette \emph{pure abstraction} de la Nature – avec maintenant ce sens qu’elle est l’Être-autre de la pensée, qu’elle est la nature réelle contemplée, distincte de la pensée abstraite. Ou bien, pour parler un langage humain, dans sa contemplation de la nature, le penseur abstrait apprend que les êtres, que dans la dialectique divine il pensait créer à partir du néant, de l’abstraction pure comme de purs produits du travail de la pensée tournant en rond en elle-même et ne regardant nulle part au dehors dans la réalité, ne sont rien d’autre que des abstractions de déterminations naturelles. La nature entière ne fait donc que répéter pour lui, sous une forme sensible extérieure, les abstractions de la \emph{Logique. Il} l’analyse, et analyse à nouveau ces abstractions. Sa contemplation de la nature n’est donc que l’acte qui confirme son abstraction de la contemplation de la nature, le processus d’engendrement de son abstraction qu’il répète consciemment. Par exemple le temps est identique a la négativité qui se rapporte à elle-même (p. 238, l.c.) \footnote{Voici le texte de Hegel auquel Marx fait allusion : “La négativité qui se rapporte à l’espace en tant que point et développe en lui ses déterminations en tant que ligne et surface est pourtant dans la sphère de l’être extérieur à lui-même, également pour soi, et pose ses déterminations dans le pour soi de la négativité, mais en même temps dans la sphère de l’être extérieur à soi, y apparaissant comme indifférence vis-à-vis de la juxtaposition tranquille. Ainsi posée pour soi elle est le temps.” (Ibid., § 254).}. Au devenir supprimé en tant qu’existence correspond – sous sa forme naturelle – le mouvement supprime en tant que matière. La lumière est… la forme naturelle… de la réflexion en soi. Le corps en tant que lune et comète… est la forme naturelle de… l’opposition qui, d’après la Logique, est d’une part le positif reposant sur lui-même, d’autre part le négatif reposant sur lui-même. La terre est la forme naturelle du fond logique, en tant qu’unité négative de l’opposition, etc.\par
La nature en tant que nature, c’est-à-dire dans la mesure où elle se distingue encore concrètement de ce sens secret qui est caché en elle, la nature, séparée et distincte de ces abstractions, est le néant, un néant qui se vérifie comme néant, elle n’a pas de sens, ou elle n’a que le sens de son extériorité qui doit être supprimée.\par

\begin{quoteblock}
 \noindent Le point de vue de la téléologie finie implique la supposition juste que la nature ne renferme pas en elle la fin absolue (p. 225) \footnote{Ibid., 245.}.
 \end{quoteblock}

\noindent Son but est la confirmation de l’abstraction.\par

\begin{quoteblock}
 \noindent La nature s’est révélée comme l’idée dans la forme de l’Être-autre. Comme l’Idée est ainsi le négatif d’elle-même, autrement dit comme elle est extérieure à elle-même, la nature n’est pas extérieure seulement relativement à cette idée, mais l’extériorité constitue la détermination dans laquelle elle est comme nature (p. 227) \footnote{Ibid., 247.}.
 \end{quoteblock}

\noindent L’extériorité ne doit pas être comprise ici comme le \emph{monde sensible} qui s’extériorise et s’est ouvert à la lumière, à l’homme doué de sens. Il faut la prendre ici au sens de l’aliénation, d’une faute, d’une infirmité qui ne doit pas être. Car la vérité reste toujours l’Idée. La nature n’est que la forme de son Être-autre. Et comme la pensée abstraite est l’essence, ce qui lui est extérieur n’est, par son essence, que quelque chose d’extérieur. Le penseur abstrait reconnaît en même temps que le monde sensible est l’essence de la nature, l’extériorité en opposition avec la pensée qui tourne en rond en elle-même. Mais en même temps il exprime cette opposition de telle sorte que cette extériorité de la nature, son opposition à la pensée est son \emph{défaut, et} que, dans la mesure où elle se distingue de l’abstraction, elle est un être imparfait.\par
[XXXIV] Un être qui n’est pas seulement imparfait pour moi, a mes yeux, mais qui l’est en soi, a en dehors de lui quelque chose qui lui manque. C’est-à-dire que son essence est quelque chose d’autre que lui-même. C’est pourquoi la nature doit se supprimer elle-même pour le penseur abstrait, car elle est déjà posée par lui comme un être supprimé en puissance.\par

\begin{quoteblock}
 \noindent L’Esprit a pour nous, comme présupposition, la nature : il est sa vérité et par là le premier absolu. Dans cette vérité la nature a disparu et l’Esprit s’est révélé comme l’Idée qui a atteint son Être-pour-soi dont le concept est à la fois le sujet et l’objet. Cette identité est négativité, absolue, car dans la nature le concept a son objectivité extérieure achevée, mais il a supprimé cette aliénation qui est sienne et il est en elle devenu identique avec soi. Aussi est-il cette identité seulement en tant que revenu de la nature vers soi-même (p. 392) \footnote{Ibid., § 381.}.\par
 La manifestation, qui comme idée abstraite est passage immédiat, devenir de la nature, est en tant que manifestation de l’Esprit qui est libre, le fait de poser la nature comme son monde ; position qui, en tant que réflexion, est en même temps présupposition du monde comme nature indépendante. La manifestation dans le concept est création de la nature comme être de celui-ci, dans lequel il se donne la confirmation et la vérité de sa liberté… L’Absolu est l’Esprit, telle est la plus haute définition de l’Absolu \footnote{Ibid., § 384.}.
 \end{quoteblock}

 


% at least one empty page at end (for booklet couv)
\ifbooklet
  \pagestyle{empty}
  \clearpage
  % 2 empty pages maybe needed for 4e cover
  \ifnum\modulo{\value{page}}{4}=0 \hbox{}\newpage\hbox{}\newpage\fi
  \ifnum\modulo{\value{page}}{4}=1 \hbox{}\newpage\hbox{}\newpage\fi


  \hbox{}\newpage
  \ifodd\value{page}\hbox{}\newpage\fi
  {\centering\color{rubric}\bfseries\noindent\large
    Hurlus ? Qu’est-ce.\par
    \bigskip
  }
  \noindent Des bouquinistes électroniques, pour du texte libre à participation libre,
  téléchargeable gratuitement sur \href{https://hurlus.fr}{\dotuline{hurlus.fr}}.\par
  \bigskip
  \noindent Cette brochure a été produite par des éditeurs bénévoles.
  Elle n’est pas faîte pour être possédée, mais pour être lue, et puis donnée.
  Que circule le texte !
  En page de garde, on peut ajouter une date, un lieu, un nom ; pour suivre le voyage des idées.
  \par

  Ce texte a été choisi parce qu’une personne l’a aimé,
  ou haï, elle a en tous cas pensé qu’il partipait à la formation de notre présent ;
  sans le souci de plaire, vendre, ou militer pour une cause.
  \par

  L’édition électronique est soigneuse, tant sur la technique
  que sur l’établissement du texte ; mais sans aucune prétention scolaire, au contraire.
  Le but est de s’adresser à tous, sans distinction de science ou de diplôme.
  Au plus direct ! (possible)
  \par

  Cet exemplaire en papier a été tiré sur une imprimante personnelle
   ou une photocopieuse. Tout le monde peut le faire.
  Il suffit de
  télécharger un fichier sur \href{https://hurlus.fr}{\dotuline{hurlus.fr}},
  d’imprimer, et agrafer ; puis de lire et donner.\par

  \bigskip

  \noindent PS : Les hurlus furent aussi des rebelles protestants qui cassaient les statues dans les églises catholiques. En 1566 démarra la révolte des gueux dans le pays de Lille. L’insurrection enflamma la région jusqu’à Anvers où les gueux de mer bloquèrent les bateaux espagnols.
  Ce fut une rare guerre de libération dont naquit un pays toujours libre : les Pays-Bas.
  En plat pays francophone, par contre, restèrent des bandes de huguenots, les hurlus, progressivement réprimés par la très catholique Espagne.
  Cette mémoire d’une défaite est éteinte, rallumons-la. Sortons les livres du culte universitaire, cherchons les idoles de l’époque, pour les briser.
\fi

\ifdev % autotext in dev mode
\fontname\font — \textsc{Les règles du jeu}\par
(\hyperref[utopie]{\underline{Lien}})\par
\noindent \initialiv{A}{lors là}\blindtext\par
\noindent \initialiv{À}{ la bonheur des dames}\blindtext\par
\noindent \initialiv{É}{tonnez-le}\blindtext\par
\noindent \initialiv{Q}{ualitativement}\blindtext\par
\noindent \initialiv{V}{aloriser}\blindtext\par
\Blindtext
\phantomsection
\label{utopie}
\Blinddocument
\fi
\end{document}
