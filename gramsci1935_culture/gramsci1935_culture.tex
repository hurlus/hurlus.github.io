%%%%%%%%%%%%%%%%%%%%%%%%%%%%%%%%%
% LaTeX model https://hurlus.fr %
%%%%%%%%%%%%%%%%%%%%%%%%%%%%%%%%%

% Needed before document class
\RequirePackage{pdftexcmds} % needed for tests expressions
\RequirePackage{fix-cm} % correct units

% Define mode
\def\mode{a4}

\newif\ifaiv % a4
\newif\ifav % a5
\newif\ifbooklet % booklet
\newif\ifcover % cover for booklet

\ifnum \strcmp{\mode}{cover}=0
  \covertrue
\else\ifnum \strcmp{\mode}{booklet}=0
  \booklettrue
\else\ifnum \strcmp{\mode}{a5}=0
  \avtrue
\else
  \aivtrue
\fi\fi\fi

\ifbooklet % do not enclose with {}
  \documentclass[french,twoside]{book} % ,notitlepage
  \usepackage[%
    papersize={105mm, 297mm},
    inner=12mm,
    outer=12mm,
    top=20mm,
    bottom=15mm,
    marginparsep=0pt,
  ]{geometry}
  \usepackage[fontsize=9.5pt]{scrextend} % for Roboto
\else\ifav
  \documentclass[french,twoside]{book} % ,notitlepage
  \usepackage[%
    a5paper,
    inner=25mm,
    outer=15mm,
    top=15mm,
    bottom=15mm,
    marginparsep=0pt,
  ]{geometry}
  \usepackage[fontsize=12pt]{scrextend}
\else% A4 2 cols
  \documentclass[twocolumn]{report}
  \usepackage[%
    a4paper,
    inner=15mm,
    outer=10mm,
    top=25mm,
    bottom=18mm,
    marginparsep=0pt,
  ]{geometry}
  \setlength{\columnsep}{20mm}
  \usepackage[fontsize=9.5pt]{scrextend}
\fi\fi

%%%%%%%%%%%%%%
% Alignments %
%%%%%%%%%%%%%%
% before teinte macros

\setlength{\arrayrulewidth}{0.2pt}
\setlength{\columnseprule}{\arrayrulewidth} % twocol
\setlength{\parskip}{0pt} % classical para with no margin
\setlength{\parindent}{1.5em}

%%%%%%%%%%
% Colors %
%%%%%%%%%%
% before Teinte macros

\usepackage[dvipsnames]{xcolor}
\definecolor{rubric}{HTML}{800000} % the tonic 0c71c3
\def\columnseprulecolor{\color{rubric}}
\colorlet{borderline}{rubric!30!} % definecolor need exact code
\definecolor{shadecolor}{gray}{0.95}
\definecolor{bghi}{gray}{0.5}

%%%%%%%%%%%%%%%%%
% Teinte macros %
%%%%%%%%%%%%%%%%%
%%%%%%%%%%%%%%%%%%%%%%%%%%%%%%%%%%%%%%%%%%%%%%%%%%%
% <TEI> generic (LaTeX names generated by Teinte) %
%%%%%%%%%%%%%%%%%%%%%%%%%%%%%%%%%%%%%%%%%%%%%%%%%%%
% This template is inserted in a specific design
% It is XeLaTeX and otf fonts

\makeatletter % <@@@


\usepackage{blindtext} % generate text for testing
\usepackage[strict]{changepage} % for modulo 4
\usepackage{contour} % rounding words
\usepackage[nodayofweek]{datetime}
% \usepackage{DejaVuSans} % seems buggy for sffont font for symbols
\usepackage{enumitem} % <list>
\usepackage{etoolbox} % patch commands
\usepackage{fancyvrb}
\usepackage{fancyhdr}
\usepackage{float}
\usepackage{fontspec} % XeLaTeX mandatory for fonts
\usepackage{footnote} % used to capture notes in minipage (ex: quote)
\usepackage{framed} % bordering correct with footnote hack
\usepackage{graphicx}
\usepackage{lettrine} % drop caps
\usepackage{lipsum} % generate text for testing
\usepackage[framemethod=tikz,]{mdframed} % maybe used for frame with footnotes inside
\usepackage{pdftexcmds} % needed for tests expressions
\usepackage{polyglossia} % non-break space french punct, bug Warning: "Failed to patch part"
\usepackage[%
  indentfirst=false,
  vskip=1em,
  noorphanfirst=true,
  noorphanafter=true,
  leftmargin=\parindent,
  rightmargin=0pt,
]{quoting}
\usepackage{ragged2e}
\usepackage{setspace} % \setstretch for <quote>
\usepackage{tabularx} % <table>
\usepackage[explicit]{titlesec} % wear titles, !NO implicit
\usepackage{tikz} % ornaments
\usepackage{tocloft} % styling tocs
\usepackage[fit]{truncate} % used im runing titles
\usepackage{unicode-math}
\usepackage[normalem]{ulem} % breakable \uline, normalem is absolutely necessary to keep \emph
\usepackage{verse} % <l>
\usepackage{xcolor} % named colors
\usepackage{xparse} % @ifundefined
\XeTeXdefaultencoding "iso-8859-1" % bad encoding of xstring
\usepackage{xstring} % string tests
\XeTeXdefaultencoding "utf-8"
\PassOptionsToPackage{hyphens}{url} % before hyperref, which load url package

% TOTEST
% \usepackage{hypcap} % links in caption ?
% \usepackage{marginnote}
% TESTED
% \usepackage{background} % doesn’t work with xetek
% \usepackage{bookmark} % prefers the hyperref hack \phantomsection
% \usepackage[color, leftbars]{changebar} % 2 cols doc, impossible to keep bar left
% \usepackage[utf8x]{inputenc} % inputenc package ignored with utf8 based engines
% \usepackage[sfdefault,medium]{inter} % no small caps
% \usepackage{firamath} % choose firasans instead, firamath unavailable in Ubuntu 21-04
% \usepackage{flushend} % bad for last notes, supposed flush end of columns
% \usepackage[stable]{footmisc} % BAD for complex notes https://texfaq.org/FAQ-ftnsect
% \usepackage{helvet} % not for XeLaTeX
% \usepackage{multicol} % not compatible with too much packages (longtable, framed, memoir…)
% \usepackage[default,oldstyle,scale=0.95]{opensans} % no small caps
% \usepackage{sectsty} % \chapterfont OBSOLETE
% \usepackage{soul} % \ul for underline, OBSOLETE with XeTeX
% \usepackage[breakable]{tcolorbox} % text styling gone, footnote hack not kept with breakable


% Metadata inserted by a program, from the TEI source, for title page and runing heads
\title{\textbf{ Problèmes de civilisation et de culture }}
\date{1935}
\author{Gramsci, Antonio}
\def\elbibl{Gramsci, Antonio. 1935. \emph{Problèmes de civilisation et de culture}}
\def\elsource{https://www.marxists.org/francais/gramsci/intell/index.htm}

% Default metas
\newcommand{\colorprovide}[2]{\@ifundefinedcolor{#1}{\colorlet{#1}{#2}}{}}
\colorprovide{rubric}{red}
\colorprovide{silver}{lightgray}
\@ifundefined{syms}{\newfontfamily\syms{DejaVu Sans}}{}
\newif\ifdev
\@ifundefined{elbibl}{% No meta defined, maybe dev mode
  \newcommand{\elbibl}{Titre court ?}
  \newcommand{\elbook}{Titre du livre source ?}
  \newcommand{\elabstract}{Résumé\par}
  \newcommand{\elurl}{http://oeuvres.github.io/elbook/2}
  \author{Éric Lœchien}
  \title{Un titre de test assez long pour vérifier le comportement d’une maquette}
  \date{1566}
  \devtrue
}{}
\let\eltitle\@title
\let\elauthor\@author
\let\eldate\@date


\defaultfontfeatures{
  % Mapping=tex-text, % no effect seen
  Scale=MatchLowercase,
  Ligatures={TeX,Common},
}


% generic typo commands
\newcommand{\astermono}{\medskip\centerline{\color{rubric}\large\selectfont{\syms ✻}}\medskip\par}%
\newcommand{\astertri}{\medskip\par\centerline{\color{rubric}\large\selectfont{\syms ✻\,✻\,✻}}\medskip\par}%
\newcommand{\asterism}{\bigskip\par\noindent\parbox{\linewidth}{\centering\color{rubric}\large{\syms ✻}\\{\syms ✻}\hskip 0.75em{\syms ✻}}\bigskip\par}%

% lists
\newlength{\listmod}
\setlength{\listmod}{\parindent}
\setlist{
  itemindent=!,
  listparindent=\listmod,
  labelsep=0.2\listmod,
  parsep=0pt,
  % topsep=0.2em, % default topsep is best
}
\setlist[itemize]{
  label=—,
  leftmargin=0pt,
  labelindent=1.2em,
  labelwidth=0pt,
}
\setlist[enumerate]{
  label={\bf\color{rubric}\arabic*.},
  labelindent=0.8\listmod,
  leftmargin=\listmod,
  labelwidth=0pt,
}
\newlist{listalpha}{enumerate}{1}
\setlist[listalpha]{
  label={\bf\color{rubric}\alph*.},
  leftmargin=0pt,
  labelindent=0.8\listmod,
  labelwidth=0pt,
}
\newcommand{\listhead}[1]{\hspace{-1\listmod}\emph{#1}}

\renewcommand{\hrulefill}{%
  \leavevmode\leaders\hrule height 0.2pt\hfill\kern\z@}

% General typo
\DeclareTextFontCommand{\textlarge}{\large}
\DeclareTextFontCommand{\textsmall}{\small}

% commands, inlines
\newcommand{\anchor}[1]{\Hy@raisedlink{\hypertarget{#1}{}}} % link to top of an anchor (not baseline)
\newcommand\abbr[1]{#1}
\newcommand{\autour}[1]{\tikz[baseline=(X.base)]\node [draw=rubric,thin,rectangle,inner sep=1.5pt, rounded corners=3pt] (X) {\color{rubric}#1};}
\newcommand\corr[1]{#1}
\newcommand{\ed}[1]{ {\color{silver}\sffamily\footnotesize (#1)} } % <milestone ed="1688"/>
\newcommand\expan[1]{#1}
\newcommand\foreign[1]{\emph{#1}}
\newcommand\gap[1]{#1}
\renewcommand{\LettrineFontHook}{\color{rubric}}
\newcommand{\initial}[2]{\lettrine[lines=2, loversize=0.3, lhang=0.3]{#1}{#2}}
\newcommand{\initialiv}[2]{%
  \let\oldLFH\LettrineFontHook
  % \renewcommand{\LettrineFontHook}{\color{rubric}\ttfamily}
  \IfSubStr{QJ’}{#1}{
    \lettrine[lines=4, lhang=0.2, loversize=-0.1, lraise=0.2]{\smash{#1}}{#2}
  }{\IfSubStr{É}{#1}{
    \lettrine[lines=4, lhang=0.2, loversize=-0, lraise=0]{\smash{#1}}{#2}
  }{\IfSubStr{ÀÂ}{#1}{
    \lettrine[lines=4, lhang=0.2, loversize=-0, lraise=0, slope=0.6em]{\smash{#1}}{#2}
  }{\IfSubStr{A}{#1}{
    \lettrine[lines=4, lhang=0.2, loversize=0.2, slope=0.6em]{\smash{#1}}{#2}
  }{\IfSubStr{V}{#1}{
    \lettrine[lines=4, lhang=0.2, loversize=0.2, slope=-0.5em]{\smash{#1}}{#2}
  }{
    \lettrine[lines=4, lhang=0.2, loversize=0.2]{\smash{#1}}{#2}
  }}}}}
  \let\LettrineFontHook\oldLFH
}
\newcommand{\labelchar}[1]{\textbf{\color{rubric} #1}}
\newcommand{\milestone}[1]{\autour{\footnotesize\color{rubric} #1}} % <milestone n="4"/>
\newcommand\name[1]{#1}
\newcommand\orig[1]{#1}
\newcommand\orgName[1]{#1}
\newcommand\persName[1]{#1}
\newcommand\placeName[1]{#1}
\newcommand{\pn}[1]{\IfSubStr{-—–¶}{#1}% <p n="3"/>
  {\noindent{\bfseries\color{rubric}   ¶  }}
  {{\footnotesize\autour{ #1}  }}}
\newcommand\reg{}
% \newcommand\ref{} % already defined
\newcommand\sic[1]{#1}
\newcommand\surname[1]{\textsc{#1}}
\newcommand\term[1]{\textbf{#1}}

\def\mednobreak{\ifdim\lastskip<\medskipamount
  \removelastskip\nopagebreak\medskip\fi}
\def\bignobreak{\ifdim\lastskip<\bigskipamount
  \removelastskip\nopagebreak\bigskip\fi}

% commands, blocks
\newcommand{\byline}[1]{\bigskip{\RaggedLeft{#1}\par}\bigskip}
\newcommand{\bibl}[1]{{\RaggedLeft{#1}\par\bigskip}}
\newcommand{\biblitem}[1]{{\noindent\hangindent=\parindent   #1\par}}
\newcommand{\dateline}[1]{\medskip{\RaggedLeft{#1}\par}\bigskip}
\newcommand{\labelblock}[1]{\medbreak{\noindent\color{rubric}\bfseries #1}\par\mednobreak}
\newcommand{\salute}[1]{\bigbreak{#1}\par\medbreak}
\newcommand{\signed}[1]{\bigbreak\filbreak{\raggedleft #1\par}\medskip}

% environments for blocks (some may become commands)
\newenvironment{borderbox}{}{} % framing content
\newenvironment{citbibl}{\ifvmode\hfill\fi}{\ifvmode\par\fi }
\newenvironment{docAuthor}{\ifvmode\vskip4pt\fontsize{16pt}{18pt}\selectfont\fi\itshape}{\ifvmode\par\fi }
\newenvironment{docDate}{}{\ifvmode\par\fi }
\newenvironment{docImprint}{\vskip6pt}{\ifvmode\par\fi }
\newenvironment{docTitle}{\vskip6pt\bfseries\fontsize{18pt}{22pt}\selectfont}{\par }
\newenvironment{msHead}{\vskip6pt}{\par}
\newenvironment{msItem}{\vskip6pt}{\par}
\newenvironment{titlePart}{}{\par }


% environments for block containers
\newenvironment{argument}{\itshape\parindent0pt}{\vskip1.5em}
\newenvironment{biblfree}{}{\ifvmode\par\fi }
\newenvironment{bibitemlist}[1]{%
  \list{\@biblabel{\@arabic\c@enumiv}}%
  {%
    \settowidth\labelwidth{\@biblabel{#1}}%
    \leftmargin\labelwidth
    \advance\leftmargin\labelsep
    \@openbib@code
    \usecounter{enumiv}%
    \let\p@enumiv\@empty
    \renewcommand\theenumiv{\@arabic\c@enumiv}%
  }
  \sloppy
  \clubpenalty4000
  \@clubpenalty \clubpenalty
  \widowpenalty4000%
  \sfcode`\.\@m
}%
{\def\@noitemerr
  {\@latex@warning{Empty `bibitemlist' environment}}%
\endlist}
\newenvironment{quoteblock}% may be used for ornaments
  {\begin{quoting}}
  {\end{quoting}}

% table () is preceded and finished by custom command
\newcommand{\tableopen}[1]{%
  \ifnum\strcmp{#1}{wide}=0{%
    \begin{center}
  }
  \else\ifnum\strcmp{#1}{long}=0{%
    \begin{center}
  }
  \else{%
    \begin{center}
  }
  \fi\fi
}
\newcommand{\tableclose}[1]{%
  \ifnum\strcmp{#1}{wide}=0{%
    \end{center}
  }
  \else\ifnum\strcmp{#1}{long}=0{%
    \end{center}
  }
  \else{%
    \end{center}
  }
  \fi\fi
}


% text structure
\newcommand\chapteropen{} % before chapter title
\newcommand\chaptercont{} % after title, argument, epigraph…
\newcommand\chapterclose{} % maybe useful for multicol settings
\setcounter{secnumdepth}{-2} % no counters for hierarchy titles
\setcounter{tocdepth}{5} % deep toc
\markright{\@title} % ???
\markboth{\@title}{\@author} % ???
\renewcommand\tableofcontents{\@starttoc{toc}}
% toclof format
% \renewcommand{\@tocrmarg}{0.1em} % Useless command?
% \renewcommand{\@pnumwidth}{0.5em} % {1.75em}
\renewcommand{\@cftmaketoctitle}{}
\setlength{\cftbeforesecskip}{\z@ \@plus.2\p@}
\renewcommand{\cftchapfont}{}
\renewcommand{\cftchapdotsep}{\cftdotsep}
\renewcommand{\cftchapleader}{\normalfont\cftdotfill{\cftchapdotsep}}
\renewcommand{\cftchappagefont}{\bfseries}
\setlength{\cftbeforechapskip}{0em \@plus\p@}
% \renewcommand{\cftsecfont}{\small\relax}
\renewcommand{\cftsecpagefont}{\normalfont}
% \renewcommand{\cftsubsecfont}{\small\relax}
\renewcommand{\cftsecdotsep}{\cftdotsep}
\renewcommand{\cftsecpagefont}{\normalfont}
\renewcommand{\cftsecleader}{\normalfont\cftdotfill{\cftsecdotsep}}
\setlength{\cftsecindent}{1em}
\setlength{\cftsubsecindent}{2em}
\setlength{\cftsubsubsecindent}{3em}
\setlength{\cftchapnumwidth}{1em}
\setlength{\cftsecnumwidth}{1em}
\setlength{\cftsubsecnumwidth}{1em}
\setlength{\cftsubsubsecnumwidth}{1em}

% footnotes
\newif\ifheading
\newcommand*{\fnmarkscale}{\ifheading 0.70 \else 1 \fi}
\renewcommand\footnoterule{\vspace*{0.3cm}\hrule height \arrayrulewidth width 3cm \vspace*{0.3cm}}
\setlength\footnotesep{1.5\footnotesep} % footnote separator
\renewcommand\@makefntext[1]{\parindent 1.5em \noindent \hb@xt@1.8em{\hss{\normalfont\@thefnmark . }}#1} % no superscipt in foot
\patchcmd{\@footnotetext}{\footnotesize}{\footnotesize\sffamily}{}{} % before scrextend, hyperref


%   see https://tex.stackexchange.com/a/34449/5049
\def\truncdiv#1#2{((#1-(#2-1)/2)/#2)}
\def\moduloop#1#2{(#1-\truncdiv{#1}{#2}*#2)}
\def\modulo#1#2{\number\numexpr\moduloop{#1}{#2}\relax}

% orphans and widows
\clubpenalty=9996
\widowpenalty=9999
\brokenpenalty=4991
\predisplaypenalty=10000
\postdisplaypenalty=1549
\displaywidowpenalty=1602
\hyphenpenalty=400
% Copied from Rahtz but not understood
\def\@pnumwidth{1.55em}
\def\@tocrmarg {2.55em}
\def\@dotsep{4.5}
\emergencystretch 3em
\hbadness=4000
\pretolerance=750
\tolerance=2000
\vbadness=4000
\def\Gin@extensions{.pdf,.png,.jpg,.mps,.tif}
% \renewcommand{\@cite}[1]{#1} % biblio

\usepackage{hyperref} % supposed to be the last one, :o) except for the ones to follow
\urlstyle{same} % after hyperref
\hypersetup{
  % pdftex, % no effect
  pdftitle={\elbibl},
  % pdfauthor={Your name here},
  % pdfsubject={Your subject here},
  % pdfkeywords={keyword1, keyword2},
  bookmarksnumbered=true,
  bookmarksopen=true,
  bookmarksopenlevel=1,
  pdfstartview=Fit,
  breaklinks=true, % avoid long links
  pdfpagemode=UseOutlines,    % pdf toc
  hyperfootnotes=true,
  colorlinks=false,
  pdfborder=0 0 0,
  % pdfpagelayout=TwoPageRight,
  % linktocpage=true, % NO, toc, link only on page no
}

\makeatother % /@@@>
%%%%%%%%%%%%%%
% </TEI> end %
%%%%%%%%%%%%%%


%%%%%%%%%%%%%
% footnotes %
%%%%%%%%%%%%%
\renewcommand{\thefootnote}{\bfseries\textcolor{rubric}{\arabic{footnote}}} % color for footnote marks

%%%%%%%%%
% Fonts %
%%%%%%%%%
\usepackage[]{roboto} % SmallCaps, Regular is a bit bold
% \linespread{0.90} % too compact, keep font natural
\newfontfamily\fontrun[]{Roboto Condensed Light} % condensed runing heads
\ifav
  \setmainfont[
    ItalicFont={Roboto Light Italic},
  ]{Roboto}
\else\ifbooklet
  \setmainfont[
    ItalicFont={Roboto Light Italic},
  ]{Roboto}
\else
\setmainfont[
  ItalicFont={Roboto Italic},
]{Roboto Light}
\fi\fi
\renewcommand{\LettrineFontHook}{\bfseries\color{rubric}}
% \renewenvironment{labelblock}{\begin{center}\bfseries\color{rubric}}{\end{center}}

%%%%%%%%
% MISC %
%%%%%%%%

\setdefaultlanguage[frenchpart=false]{french} % bug on part


\newenvironment{quotebar}{%
    \def\FrameCommand{{\color{rubric!10!}\vrule width 0.5em} \hspace{0.9em}}%
    \def\OuterFrameSep{\itemsep} % séparateur vertical
    \MakeFramed {\advance\hsize-\width \FrameRestore}
  }%
  {%
    \endMakeFramed
  }
\renewenvironment{quoteblock}% may be used for ornaments
  {%
    \savenotes
    \setstretch{0.9}
    \normalfont
    \begin{quotebar}
  }
  {%
    \end{quotebar}
    \spewnotes
  }


\renewcommand{\headrulewidth}{\arrayrulewidth}
\renewcommand{\headrule}{{\color{rubric}\hrule}}

% delicate tuning, image has produce line-height problems in title on 2 lines
\titleformat{name=\chapter} % command
  [display] % shape
  {\vspace{1.5em}\centering} % format
  {} % label
  {0pt} % separator between n
  {}
[{\color{rubric}\huge\textbf{#1}}\bigskip] % after code
% \titlespacing{command}{left spacing}{before spacing}{after spacing}[right]
\titlespacing*{\chapter}{0pt}{-2em}{0pt}[0pt]

\titleformat{name=\section}
  [block]{}{}{}{}
  [\vbox{\color{rubric}\large\raggedleft\textbf{#1}}]
\titlespacing{\section}{0pt}{0pt plus 4pt minus 2pt}{\baselineskip}

\titleformat{name=\subsection}
  [block]
  {}
  {} % \thesection
  {} % separator \arrayrulewidth
  {}
[\vbox{\large\textbf{#1}}]
% \titlespacing{\subsection}{0pt}{0pt plus 4pt minus 2pt}{\baselineskip}

\ifaiv
  \fancypagestyle{main}{%
    \fancyhf{}
    \setlength{\headheight}{1.5em}
    \fancyhead{} % reset head
    \fancyfoot{} % reset foot
    \fancyhead[L]{\truncate{0.45\headwidth}{\fontrun\elbibl}} % book ref
    \fancyhead[R]{\truncate{0.45\headwidth}{ \fontrun\nouppercase\leftmark}} % Chapter title
    \fancyhead[C]{\thepage}
  }
  \fancypagestyle{plain}{% apply to chapter
    \fancyhf{}% clear all header and footer fields
    \setlength{\headheight}{1.5em}
    \fancyhead[L]{\truncate{0.9\headwidth}{\fontrun\elbibl}}
    \fancyhead[R]{\thepage}
  }
\else
  \fancypagestyle{main}{%
    \fancyhf{}
    \setlength{\headheight}{1.5em}
    \fancyhead{} % reset head
    \fancyfoot{} % reset foot
    \fancyhead[RE]{\truncate{0.9\headwidth}{\fontrun\elbibl}} % book ref
    \fancyhead[LO]{\truncate{0.9\headwidth}{\fontrun\nouppercase\leftmark}} % Chapter title, \nouppercase needed
    \fancyhead[RO,LE]{\thepage}
  }
  \fancypagestyle{plain}{% apply to chapter
    \fancyhf{}% clear all header and footer fields
    \setlength{\headheight}{1.5em}
    \fancyhead[L]{\truncate{0.9\headwidth}{\fontrun\elbibl}}
    \fancyhead[R]{\thepage}
  }
\fi

\ifav % a5 only
  \titleclass{\section}{top}
\fi

\newcommand\chapo{{%
  \vspace*{-3em}
  \centering % no vskip ()
  {\Large\addfontfeature{LetterSpace=25}\bfseries{\elauthor}}\par
  \smallskip
  {\large\eldate}\par
  \bigskip
  {\Large\selectfont{\eltitle}}\par
  \bigskip
  {\color{rubric}\hline\par}
  \bigskip
  {\Large TEXTE LIBRE À PARTICPATION LIBRE\par}
  \centerline{\small\color{rubric} {hurlus.fr, tiré le \today}}\par
  \bigskip
}}

\newcommand\cover{{%
  \thispagestyle{empty}
  \centering
  {\LARGE\bfseries{\elauthor}}\par
  \bigskip
  {\Large\eldate}\par
  \bigskip
  \bigskip
  {\LARGE\selectfont{\eltitle}}\par
  \vfill\null
  {\color{rubric}\setlength{\arrayrulewidth}{2pt}\hline\par}
  \vfill\null
  {\Large TEXTE LIBRE À PARTICPATION LIBRE\par}
  \centerline{{\href{https://hurlus.fr}{\dotuline{hurlus.fr}}, tiré le \today}}\par
}}

\begin{document}
\pagestyle{empty}
\ifbooklet{
  \cover\newpage
  \thispagestyle{empty}\hbox{}\newpage
  \cover\newpage\noindent Les voyages de la brochure\par
  \bigskip
  \begin{tabularx}{\textwidth}{l|X|X}
    \textbf{Date} & \textbf{Lieu}& \textbf{Nom/pseudo} \\ \hline
    \rule{0pt}{25cm} &  &   \\
  \end{tabularx}
  \newpage
  \addtocounter{page}{-4}
}\fi

\thispagestyle{empty}
\ifaiv
  \twocolumn[\chapo]
\else
  \chapo
\fi
{\it\elabstract}
\bigskip
\makeatletter\@starttoc{toc}\makeatother % toc without new page
\bigskip

\pagestyle{main} % after style

  
\chapteropen
\chapter[{La formation des intellectuels}]{La formation des intellectuels}\renewcommand{\leftmark}{La formation des intellectuels}


\chaptercont
\noindent  Les intellectuels constituent-ils un groupe social autonome et indépendant, ou bien chaque groupe social a-t-il sa propre catégorie spécialisée d’intellectuels ? Le problème est complexe, étant donné les formes diverses qu’a prises jusqu’ici le processus historique réel de la formation des différentes catégories d’intellectuels.\par
Les plus importantes de ces formes sont au nombre de deux :\par

\begin{enumerate}[itemsep=\baselineskip,]
\item  Chaque groupe social, naissant sur le terrain originel d’une fonction essentielle dans le monde de la production économique, crée en même temps que lui, organiquement, une ou plusieurs couches d’intellectuels qui lui donnent son homogénéité et la conscience de sa propre fonction, non seulement dans le domaine économique, mais aussi dans le domaine politique et social : le chef d’entreprise capitaliste crée avec lui le technicien de l’industrie, le savant de l’économie politique, l’organisateur d’une nouvelle culture, d’un nouveau droit, etc., etc. Il faut remarquer que le chef d’entreprise représente une élaboration sociale supérieure, déjà caractérisée par une certaine capacité de direction et de technique (c’est-à-dire une capacité intellectuelle) : il doit avoir une certaine capacité technique, en dehors de la sphère bien délimitée de son activité et de son initiative, au moins dans les autres domaines les plus proches de la production économique (il doit être un organisateur de masses d’hommes ; il doit organiser la « confiance » que les épargnants ont dans son entreprise, les acheteurs dans sa marchandise, etc).\par
 Sinon tous les chefs d’entreprise, du moins une élite d’entre eux doivent être capables d’être des organisateurs de la société en général, dans l’ensemble de l’organisme complexe de ses services, jusqu’à l’organisme d’Etat, car il leur est nécessaire de créer les conditions les plus favorables à l’expansion de leur propre classe - ou bien ils doivent du moins posséder la capacité de choisir leurs « commis » (employés spécialisés) auxquels ils pourront confier cette activité organisatrice des rapports généraux de l’entreprise avec l’extérieur. On peut observer que les intellectuels « organiques » que chaque nouvelle classe crée avec elle et qu’elle élabore au cours de son développement progressif, sont la plupart du temps des « spécialisations » de certains aspects partiels de l’activité primitive du nouveau type social auquel la nouvelle classe a donné naissance\footnote{Il faut examiner dans cette rubrique le livre \emph{: Elementi di scienza politica} de Mosca (nouvelle édition augmentée de 1923). Ce que Mosca appelle la « classe politique » n’est autre que la catégorie intellectuelle du groupe social dominant : le concept de « classe politique » de Mosca est à rapprocher du concept d’élite chez Pareto, qui est une autre tentative pour interpréter le phénomène historique des intellectuels et leur fonction dans la vie de l’Etat et de la société. Le livre de Mosca est un énorme fatras à caractère sociologique et positiviste, avec, en plus, un esprit tendancieux de politique immédiate qui le rend moins indigeste et plus vivant du point de vue littéraire. \emph{(Note de Gramsci.)}}.\par
 Les seigneurs de l’époque féodale eux aussi étaient les détenteurs d’une certaine capacité technique, dans le domaine militaire, et c’est justement à partir du moment où l’aristocratie perd le monopole de la compétence technico-militaire, que commence la crise du féodalisme. Mais la formation des intellectuels dans le monde féodal et dans le monde classique précédent est un problème qu’il faut examiner à part : cette formation, cette élaboration suivent des voies et prennent des formes qu’il faut étudier de façon concrète. Ainsi l’on peut remarquer que la masse des paysans, bien qu’elle exerce une fonction essentielle dans le monde de la production, ne crée pas des intellectuels qui lui soient propres, « organiques », et n’ « assimile » aucune couche d’intellectuels « traditionnels », bien que d’autres groupes sociaux tirent un grand nombre de leurs intellectuels de la masse paysanne, et qu’une grande partie des intellectuels traditionnels soient d’origine paysanne.
 
\item  Mais chaque groupe social « essentiel\footnote{Les groupes sociaux « essentiels » sont ceux qui ont été, ou sont, du point de vue historique, en mesure d’assumer le pouvoir et de prendre la direction des autres classes : tels sont, par exemple, la bourgeoisie et le prolétariat. – Voir « Clergé et intellectuels » : « Cette lutte a revêtu des caractères différents aux différentes époques. Dans la phase moderne, c’est la lutte pour l’hégémonie dans le domaine de l’éducation populaire, c’est le trait le plus caractéristique auquel tous les autres sont subordonnés. Il s’agit par conséquent de la lutte entre deux catégories d’intellectuels, lutte pour subordonner le clergé, comme catégorie type d’intellectuels, aux directives de l’Etat, c’est-à-dire de la classe dominante (liberté de l’enseignement - organisations de jeunes -organisations féminines – organisations professionnelles). » (Int., pp. 39-40).} « au moment où il émerge à la surface de l’histoire, venant de la précédente structure économique dont il exprime un de ses développements, a trouvé, du moins dans l’histoire telle qu’elle s’est déroulée jusqu’à ce jour, des catégories d’intellectuels qui existaient avant lui et qui, de plus, apparaissaient comme les représentants d’une continuité historique que n’avaient même pas interrompue les changements les plus compliqués et les plus radicaux des formes sociales et politiques.\par
 La plus typique de ces catégories intellectuelles est celle des ecclésiastiques, qui monopolisèrent pendant longtemps (tout au long d’une phase historique qui est même caractérisée en partie par ce monopole) certains services importants : l’idéologie religieuse, c’est-à-dire la philosophie et la science de l’époque, avec l’école, l’instruction, la morale, la justice, la bienfaisance, l’assistance, etc. La catégorie des ecclésiastiques peut être considérée comme la catégorie intellectuelle organiquement liée à l’aristocratie foncière : elle était assimilée juridiquement à l’aristocratie, avec laquelle elle partageait l’exercice de la propriété féodale de la terre et l’usage des privilèges d’Etat liés à la propriété\footnote{Pour une catégorie de ces intellectuels, la plus importante peut-être après celle des « ecclésiastiques » par le prestige qu’elle a eu et la fonction sociale qu’elle a remplie dans les sociétés primitives - la catégorie des \emph{médecins}, au sens large du terme, c’est-à-dire de tous ceux qui « luttent » ou paraissent lutter contre la mort et les maladies - il faudra voir la \emph{Storia della medicina [Histoire de la médecine]} de Arturo CASTIGLIONI. Rappeler qu’il y a eu connexion entre religion et médecine, et qu’elle continue à exister dans certaines zones : hôpitaux entre les mains de religieux pour certaines fonctions d’organisation, sans compter le fait que là ou le médecin apparaît, le prêtre aussi se montre (exorcismes, assistance sous des formes variées, etc.). Nombreux furent les grands personnages religieux qui ont été aussi représentés comme de grands « thérapeutes » : l’idée du miracle, jusqu’à la résurrection des morts. Pour les rois également, dura longtemps la croyance qu’ils pouvaient guérir par l’imposition des mains, etc. \emph{(Note de Gramsci.)}.}. Mais ce monopole des superstructures de la part des ecclésiastiques\footnote{De là est venu le sens général d’ « intellectuel », ou de « spécialiste » qu’à pris le mot « clerc » dans de nombreuses langues d’origine néo-latine, ou fortement influencées, à travers le latin d’église, par les langues néo-latines, avec son corrélatif : « laïque » au sens de profane, denon-spécialiste. \emph{(Note de Gramsci.)}} n’a pas été exercé sans luttes et sans restrictions, aussi a-t-on vu naître, sous diverses formes (à rechercher et étudier de façon concrète d’autres catégories, favorisées et développées par le renforcement du pouvoir central du monarque, jusqu’à l’absolutisme. Ainsi s’est formée peu à peu l’aristocratie de robe, avec ses privilèges particuliers, une couche d’administrateurs, etc., savants, théoriciens, philosophes non ecclésiastiques, etc.\par
 Comme ces diverses catégories d’intellectuels traditionnels éprouvent, avec un « esprit de corps » le sentiment de leur continuité historique ininterrompue et de leur qualification, ils se situent eux-mêmes comme autonomes et indépendants du groupe social dominant. Cette auto-position n’est pas sans conséquences de grande portée dans le domaine idéologique et politique : toute la philosophie idéaliste peut se rattacher facilement à cette position prise par le complexe social des intellectuels et l’on peut définir l’expression de cette utopie sociale qui fait que les intellectuels se croient « indépendants » autonomes, dotés de caractères qui leur sont propres, etc.\par
 Il faut noter cependant que si le Pape et la haute hiérarchie de l’Eglise se croient davantage liés au Christ et aux apôtres que ne le sont les sénateurs Agnelli et Benni\footnote{Agnelli et Benni furent tous deux sénateurs et grands représentants du capitalisme italien : Agnelli était l’un des principaux actionnaires de la Fiat, Benni de la Montecatini.}, il n’en est pas de même pour Gentile et pour Croce ; par exemple : Croce particulièrement, se sent fortement lié à Aristote et à Platon, mais il ne se cache pas, par contre, d’être lié aux sénateurs Agnelli et Benni, et c’est précisément là qu’il faut chercher le caractère le plus important de la philosophie de Croce\footnote{Croce a démenti qu’il ait jamais connu Agnelli et Benni. Il est évident qu’ici Gramsci fait allusion non à des liens physiques ou matériels, mais au fait que Croce aurait traduit, sur le terrain de la culture, les exigences économiques et politiques du grand capital italien, dans une phase déterminée de son développement.}.
 
\end{enumerate}

\noindent Quelles sont les limites « maxima » pour l’acception du terme d’ « intellectuel » ? Peut-on trouver un critère unitaire pour caractériser également toutes les activités intellectuelles, diverses et disparates, et en même temps pour distinguer celles-ci, et de façon essentielle, des autres groupements sociaux ? L'erreur de méthode la plus répandue me paraît être à avoir recherché ce critère de distinction dans ce qui est intrinsèque aux activités intellectuelles et non pas dans l’ensemble du système ! de rapports dans lequel ces activités (et par conséquent les groupes qui les personnifient) viennent à se trouver au sein du complexe général des rapports sociaux. En réalité l’ouvrier ou le prolétaire, par exemple, n’est pas spécifiquement caractérisé par son travail manuel ou à caractère instrumental mais par ce travail effectué dans des conditions déterminées et dans des rapports sociaux déterminés (sans compter qu’il n’existe pas de travail purement physique, et que l’expression elle-même de Taylor de « \emph{gorille apprivoisé} « est une métaphore pour indiquer une limite dans une certaine direction : dans n’importe quel travail physique, même le plus mécanique et le plus dégradé, il existe un minimum de qualification technique, c’est-à-dire un minimum d’activité intellectuelle créatrice). Et l’on a déjà observé que le chef d’entreprise, de par sa fonction elle-même, doit posséder, en une certaine mesure, un certain nombre de qualifications de caractère intellectuel, bien que son personnage social ne soit pas déterminé par elles, mais par les rapports sociaux généraux qui caractérisent précisément la position du patron dans l’industrie.\par
C'est pourquoi l’on pourrait dire que tous les hommes sont des intellectuels ; mais tous les hommes n’exercent pas dans la société la fonction d’intellectuel\footnote{De même il peut arriver à un certain moment à tout le monde de faire frire deux œufs ou de repriser un accroc à sa veste sans qu’on puisse dire pour autant que tout le monde est cuisinier ou tailleur. (\emph{Note de Gramsci}.)}.\par
Lorsque l’on distingue intellectuels et non-intellectuels, on ne se réfère en réalité qu’à la fonction sociale immédiate de la catégorie professionnelle des intellectuels, c’est-à-dire que l’on tient compte de la direction dans laquelle s’exerce le poids le plus fort de l’activité professionnelle spécifique : dans l’élaboration intellectuelle ou dans l’effort musculaire et nerveux. Cela signifie que, si l’on peut parler d’intellectuels, on ne peut pas parler de non-intellectuels, car les non-intellectuels n’existent pas. Mais le rapport lui-même entre l’effort d’élaboration intellectuel-cérébral et l’effort musculaire nerveux n’est pas toujours égal, aussi a-t-on divers degrés de l’activité intellectuelle spécifique. Il n’existe pas d’activité humaine dont on puisse exclure toute intervention intellectuelle, on ne peut séparer \emph{l’homo faber} de \emph{l’homo sapiens}\footnote{Expressions latines. Mot à mot : l’homme-artisan et l’homme-connaissant, pour désigner le travail manuel et l’activité intellectuelle.}. Chaque homme, enfin, en dehors de sa profession, exerce une quelconque activité intellectuelle, il est un « philosophe », un artiste, un homme de goût, il participe à une conception du monde, il a une ligne de conduite morale consciente, donc il contribue à soutenir ou à modifier une conception du monde, c’est-à-dire à faire naître de nouveaux modes de penser.\par
Le problème de la création d’une nouvelle couche d’intellectuels consiste donc à développer de façon critique l’activité intellectuelle qui existe chez chacun à un certain degré de développement, en modifiant son rapport avec l’effort musculaire-nerveux en vue d’un nouvel équilibre, et en obtenant que l’effort musculaire nerveux lui-même, en tant qu’élément d’une activité pratique générale qui renouvelle perpétuellement le monde physique et social, devienne le fondement d’une nouvelle et totale conception du monde. Le type traditionnel, le type de l’intellectuel est fourni par l’homme de lettres, le philosophe, l’artiste. Aussi les journalistes, qui se considèrent comme des hommes de lettres, des philosophes, des artistes, pensent aussi qu’ils sont les « vrais » intellectuels. Dans le monde moderne, l’éducation technique, étroitement liée au travail industriel même le plus primitif et le plus déprécié, doit former la base du nouveau type d’intellectuel.\par
C'est sur cette base qu’a travaillé \emph{L'Ordine nuovo} hebdomadaire pour développer certaines formes du nouvel intellectualisme et pour établir les nouvelles façons de le concevoir, et ce n’a pas été une des moindres raisons de son succès, parce qu’une telle façon de poser le problème correspondait à des aspirations latentes et était conforme au développement des formes réelles de la vie. La façon d’être du nouvel intellectuel ne peut plus consister dans l’éloquence, agent moteur extérieur et momentané des sentiments et des passions, mais dans le fait qu’il se mêle activement à la vie pratique, comme constructeur, organisateur, « persuadeur permanent » parce qu’il n’est plus un simple orateur - et qu’il est toutefois supérieur à l’esprit mathématique abstrait ; de la technique-travail il parvient à la technique-science et à la conception humaniste historique, sans laquelle on reste un « spécialiste » et l’on ne devient pas un « dirigeant » (spécialiste +politique).\par
Ainsi se forment historiquement des catégories spécialisées par l’exercice de la fonction intellectuelle, elles se forment en connexion avec tous les groupes sociaux, mais spécialement avec les groupes sociaux les plus importants et subissent une élaboration plus étendue et plus complexe en étroit rapport avec le groupe social dominant. Un des traits caractéristiques les plus importants de chaque groupe qui cherche à atteindre le pouvoir est la lutte qu’il mène pour assimiler et conquérir « idéologiquement » les intellectuels traditionnels, assimilation et conquête qui sont d’autant plus rapides et efficaces que ce groupe donné élabore davantage, en même temps, ses intellectuels organiques.\par
L'énorme développement qu’ont pris l’activité et l’organisation scolaires (au sens large) dans les sociétés surgies du monde médiéval, montre quelle importance ont prise, dans le monde moderne, les catégories et les fonctions intellectuelles : de même que l’on a cherché a approfondir et à élargir l’ « intellectualité » de chaque individu, on a aussi cherché à multiplier les spécialisations et à les affiner. Cela apparaît dans les organismes scolaires de divers degrés, jusqu’à ceux qui sont destinés à promouvoir ce qu’on appelle la « haute culture », dans tous les domaines de la science et de la technique.\par
L'école est l’instrument qui sert à former les intellectuels à différents degrés. La complexité de la fonction intellectuelle dans les divers Etats peut se mesurer objectivement à la quantité d’écoles spécialisées qu’ils possèdent, et à leur hiérarchisation : plus l’ « aire » scolaire est étendue, plus les « degrés » « verticaux » de l’école sont nombreux, et plus le monde culturel, la civilisation des divers Etats est complexe. On peut trouver un terme de comparaison dans la sphère de la technique industrielle : l’industrialisation d’un pays se mesure à son équipement dans le domaine de la construction des machines qui servent elles-mêmes à construire d’autres machines, et dans celui de la fabrication d’instruments toujours plus précis pour construire des machines et des instruments pour construire ces machines, etc. Le pays qui est le mieux équipé pour fabriquer des instruments pour les laboratoires des savants, et des instruments pour vérifier ces instruments, peut être considéré comme ayant l’organisation la plus complexe dans le domaine technico-industriel, comme étant le plus civilisé, etc. Il en est de même dans la préparation des intellectuels et dans les écoles consacrées à cette préparation ; on peut assimiler les écoles à des instituts de haute culture. Même dans ce domaine, on ne peut isoler la quantité de la qualité. A la spécialisation technico-culturelle la plus raffinée ne peut pas ne pas correspondre la plus grande extension possible de l’instruction primaire et la plus grande sollicitude pour ouvrir les degrés intermédiaires au plus grand nombre. Naturellement cette nécessité de créer la plus large base possible pour sélectionner et former les plus hautes qualifications intellectuelles - c’est-à-dire pour donner à la culture et à la technique supérieure une structure démocratique - n’est pas sans inconvénients : on crée ainsi la possibilité de vastes crises de chômage dans les couches intellectuelles moyennes, comme cela se produit en fait dans toutes les sociétés modernes.\par
Il faut remarquer que, dans la réalité concrète, la formation de couches intellectuelles ne se produit pas sur un terrain démocratique abstrait, mais selon des processus historiques traditionnels très concrets. Il s’est formé des couches sociales qui, traditionnellement, « produisent » des intellectuels et ce sont ces mêmes couches qui d’habitude se sont spécialisées dans « l’épargne », c’est-à-dire la petite et moyenne bourgeoisie terrienne et certaines couches de la petite et moyenne bourgeoisie des villes. La distribution différente des divers types d’écoles (classiques et professionnelles) sur le territoire « économique », et les aspirations différentes des diverses catégories de ces couches sociales déterminent la production des diverses branches de spécialisation intellectuelle, ou leur donnent leur forme. Ainsi en Italie la bourgeoisie rurale produit surtout des fonctionnaires d’Etat et des gens de professions libérales, tandis que la bourgeoisie citadine produit des techniciens pour l’industrie : c’est pourquoi l’Italie septentrionale produit surtout des techniciens alors que l’Italie méridionale alimente plus spécialement les corps des fonctionnaires et des professions libérales.\par
Le rapport entre les intellectuels et le monde de la production n’est pas immédiat, comme cela se produit pour les groupes sociaux fondamentaux, mais il est « médiat », à des degrés divers, par l’intermédiaire de toute la trame sociale, du complexe des superstructures, dont précisément les intellectuels sont les « fonctionnaires ». On pourrait mesurer le caractère « organique » des diverses couches d’intellectuels, leur liaison plus ou moins étroite avec un groupe social fondamental en établissant une échelle des fonctions et des superstructures de bas en haut (à partir de la base structurelle). On peut, pour le moment, établir deux grands « étages » dans les superstructures, celui que l’on peut appeler l’étage de la « société civile » \footnote{Voir pp. 270-271, 469, 576.}, c’est-à-dire de l’ensemble des organismes vulgairement dits « privés », et celui de la « société politique » ou de l’Etat ; ils correspondent à la fonction d’ « hégémonie » que le groupe dominant exerce sur toute la société, et à la fonction de « domination directe » ou de commandement qui s’exprime dans l’Etat et dans le gouvernement « juridique ». Ce sont là précisément des fonctions d’organisation et de connexion. Les intellectuels sont les « commis » du groupe dominant pour l’exercice des fonctions subalternes de l’hégémonie sociale et du gouvernement politique, c’est-à-dire :\par

\begin{enumerate}[itemsep=0pt,]
\item de l’accord « spontané » donné par les grandes masses de la population à l’orientation imprimée à la vie sociale par le groupe fondamental dominant, accord qui naît « historiquement » du prestige qu’a le groupe dominant (et de la confiance qu’il inspire) du fait de sa fonction dans le monde de la production ; 
\item de l’appareil de coercition d’Etat qui assure « légalement » la discipline des groupes qui refusent leur « accord » tant actif que passif ; mais cet appareil est constitué pour l’ensemble de la société en prévision des moments de crise dans le commandement et dans la direction, lorsque l’accord spontané vient à faire défaut.
\end{enumerate}

\noindent Cette façon de poser le problème a pour résultat une très grande extension du concept d’intellectuel, mais c’est la seule grande façon d’arriver à une approximation concrète de la réalité. Cette façon de poser le problème se heurte à des idées préconçues de caste : il est vrai que la fonction organisatrice de l’hégémonie sociale et de la domination d’Etat donne lieu à une certaine division du travail et par conséquent à toute une échelle de qualifications dont certaines ne remplissent plus aucun rôle de direction et d’organisation : dans l’appareil de direction sociale et gouvernementale il existe toute une série d’emplois de caractère manuel et instrumental (fonction de pure exécution et non d’initiative, d’agents et non d’officiers ou de fonctionnaires). Mais il faut évidemment faire cette distinction, comme il faudra en faire d’autres. En effet, même du point de vue intrinsèque, il faut distinguer dans l’activité intellectuelle différents degrés qui, à certains moments d’opposition extrême, donnent une véritable différence qualitative : à l’échelon le plus élevé il faudra placer les créateurs des diverses sciences, de la philosophie, de l’art, etc. ; au plus bas, les plus humbles « administrateurs » et divulgateurs de la richesse intellectuelle déjà existante, traditionnelle, accumulée\footnote{Dans ce cas aussi l’organisation militaire offre un modèle de cette gradation complexe : officiers subalternes, officiers supérieurs, état-major, sans oublier les différents grades de la troupe, dont l’importance réelle est plus grande qu’on ne pense d’ordinaire. Il est intéressant de remarquer que tous ces éléments se sentent solidaires, et même que les couches inférieures montrent un esprit de corps plus visible, et en tirent un « orgueil » qui les expose souvent à l’ironie et à la moquerie. \emph{(Note de Gramsci}.)}.\par
 Dans le monde moderne, la catégorie des intellectuels, ainsi entendue, s’est développée d’une façon prodigieuse. Le système social démocratique bureaucratique a créé des masses imposantes, pas toutes justifiées par les nécessités sociales de la production, même si elles sont justifiées par les nécessités politiques du groupe fondamental, dominant. D'où la conception de Loria\footnote{Cette conception de « travailleur improductif » est exposée notamment dans le \emph{Cours d’économie politique} de Loria, publié en 1909, et réédité plusieurs fois. Selon Loria, les « travailleurs improductifs » seraient « les poètes, les philosophes, les écrivains de tous genres, les médecins, les avocats, les professeurs, etc. » ; ils seraient en opposition avec les « propriétaires » (les capitalistes) parce que les propriétaires voudraient accroître leur nombre afin de moins payer leurs services, alors que leur intérêt voudrait le contraire. C'est une des nombreuses extravagances de Loria.} du « travailleur » improductif (mais improductif par référence à qui et à quel mode de production ?) qui pourrait se justifier si l’on tient compte que ces masses exploitent leur situation pour se faire attribuer des portions énormes du revenu national. La formation de masse a standardisé les individus, tant dans leur qualification individuelle que dans leur psychologie, en déterminant l’apparition des mêmes phénomènes que dans toutes les masses standardisées : concurrence qui crée la nécessité d’organisations professionnelles de défense, chômage, surproduction de diplômés, émigration, etc. (Int., pp. 3-10).\par
{\raggedleft \noindent [1930-1932]}
\chapterclose


\chapteropen
\chapter[{L'organisation de la culture}]{L'organisation de la culture}\renewcommand{\leftmark}{L'organisation de la culture}


\chaptercont
\section[{ L'organisation de l’école et de la culture}]{ L'organisation de l’école et de la culture}
\noindent On peut observer en général que, dans la civilisation moderne, toutes les activités pratiques sont devenues si complexes et les sciences se sont tellement imbriquées dans la vie que chaque activité pratique tend à créer une école pour ses propres dirigeants et spécialistes, et par suite à créer un groupe d’intellectuels du niveau le plus élevé, destinés à enseigner dans ces écoles. Ainsi, à côté du type d’école qu’on pourrait appeler « humaniste » (c’est le type traditionnel le plus ancien, qui visait à développer en chaque individu humain la culture générale encore indifférenciée, le pouvoir fondamental de penser et de savoir se diriger dans la vie), on a créé tout un système d’écoles particulières de différents niveaux, pour des branches professionnelles entières ou pour des professions déjà spécialisées et caractérisées avec précision. On peut même dire que la crise scolaire qui sévit aujourd’hui est justement liée au fait que ce processus de différenciation et de particularisation se produit dans le chaos, sans principes clairs et précis, sans un plan bien étudié et consciemment établi : la crise du programme et de l’organisation scolaire, autrement dit de l’orientation générale d’une politique de formation des cadres intellectuels modernes, est en grande partie un aspect et une complication de la crise organique plus globale et plus générale.\par
La division fondamentale de l’école en classique et professionnelle était un schéma rationnel : l’école professionnelle pour les classes exécutantes, l’école classique pour les classes dominantes et les intellectuels. Le développement de la base industrielle, tant en ville qu’à la campagne, suscitait un besoin croissant du nouveau type d’intellectuel urbain ; à côté de l’école classique se développa l’école technique (professionnelle mais non manuelle), ce qui mit en question le principe même de l’orientation concrète de la culture générale, de l’orientation humaniste de la culture générale fondée sur la tradition gréco-romaine. Cette orientation une fois mise en question, on peut dire qu’elle est liquidée ; car sa capacité formatrice se fondait en grande partie sur le prestige général et traditionnellement indiscuté d’une forme déterminée de civilisation.\par
La tendance actuelle est d’abolir tout type d’école « désintéressée » (non immédiatement intéressée) et formatrice, quitte à en laisser subsister un modèle réduit pour une petite élite de messieurs et de dames qui n’ont pas de souci de se préparer un avenir professionnel. La tendance est de répandre toujours davantage les écoles professionnelles spécialisées dans lesquelles la destinée de l’élève et son activité future sont prédéterminées. La crise aura une solution qui, rationnellement, devrait aller dans ce sens : école initiale unique de culture générale, humaniste, formatrice, qui trouverait un juste équilibre entre le développement de l’aptitude au travail manuel (technique, industriel) et le développement de l’aptitude au travail intellectuel. De ce type d’école unique, à travers des expériences répétées d’orientation professionnelle, on passera à l’une des écoles spécialisées ou au travail productif.\par
Il faut garder présente à l’esprit la tendance qui s’accentue : chaque activité pratique tend à se créer sa propre école spécialisée, comme chaque activité intellectuelle tend à se créer ses propres cercles de culture. Cercles qui jouent le rôle d’institutions postscolaires spécialisées dans l’organisation des conditions permettant à chacun de se tenir au courant des progrès réalisés dans sa propre branche scientifique.\par
On peut aussi observer que les organismes délibérants tendent toujours davantage à distinguer dans leur activité deux aspects « organiques » : l’activité délibérative, qui leur est essentielle, et l’activité technico-culturelle consistant dans l’examen préalable par des experts et dans l’analyse scientifique préalable des problèmes qui doivent donner lieu à décision. Cette activité a déjà créé tout un corps bureaucratique de structure nouvelle : en plus des bureaux spécialisés où le personnel compétent prépare le matériel technique pour les organismes délibérants, se crée un second corps de fonctionnaires plus ou moins « bénévoles » et désintéressés, choisis tour à tour dans l’industrie, la banque, la finance. C'est là un des mécanismes à travers lesquels la bureaucratie de carrière avait fini par contrôler les régimes démocratiques et les parlements ; à présent le mécanisme s’étend organiquement et absorbe dans son cercle les grands spécialistes de l’activité pratique privée, qui contrôle ainsi et les régimes et les bureaucraties. Il s’agit là d’un développement organique nécessaire qui tend à intégrer le personnel spécialisé dans la technique politique avec le personnel spécialisé dans les questions concrètes d’administration des activités pratiques essentielles des grandes et complexes sociétés nationales modernes : donc, toute tentative pour exorciser du dehors ces tendances ne produit d’autre résultat que sermons moralisateurs et gémissements rhétoriques.\par
La question se pose de modifier la préparation du personnel technique politique, en complétant sa culture selon les nécessités nouvelles, et d’élaborer de nouveaux types de fonctionnaires spécialisés capables de compléter collégialement l’activité délibérante. Le type traditionnel du « dirigeant » politique, préparé seulement aux activités juridico-formelles, devient anachronique et représente un danger pour la vie de l’Etat. Le dirigeant doit avoir ce minimum de culture générale technique qui lui permette, sinon de « créer » de façon autonome la solution juste, du moins de savoir arbitrer entre les solutions explorées par les experts et choisir alors celle qui est juste du point de vue « synthétique » de la technique politique.\par
Un type de collège délibérant qui cherche à s’incorporer la compétence technique nécessaire pour œuvrer à des fins réalistes a été décrit ailleurs : il s’agit de ce qui se passe dans certaines rédactions de revues, qui fonctionnent en même temps comme rédactions et comme cercles de culture. Le cercle critique collégialement et contribue ainsi à élaborer le travail de chaque rédacteur dont l’activité est organisée selon un plan et une division du travail rationnellement prévue. A travers la discussion et la critique collégiale (faite de suggestions, de conseils, d’indications méthodologiques, critique constructive et orientée vers l’éducation réciproque) qui permettent à chacun de fonctionner en spécialiste dans son domaine pour compléter la compétence collective, on réussit en réalité à élever le niveau ou la capacité du mieux préparé ; ce qui n’assure pas seulement à la revue une collaboration toujours plus choisie et organique, mais crée en outre les conditions pour que naisse un groupe homogène d’intellectuels prêts à produire une activité « de librairie » (non seulement de publications occasionnelles et d’essais partiels, mais de travaux organiques d’ensemble).\par
Sans aucun doute, dans cette sorte d’activité collective, chaque travail produit de nouvelles capacités et possibilités de travail, puisqu’il crée des conditions de travail toujours plus organiques : fichiers, dépouillements bibliographiques, collection d’œuvres spécialisées fondamentales, etc. Cela demande une lutte rigoureuse contre les habitudes de dilettantisme, d’improvisation, les solutions « rhétoriques » et déclamatoires. En particulier le travail doit être fait par écrit, de même que doivent être écrites les critiques, en notes concises et succinctes ; ce qu’on peut obtenir en distribuant à temps le matériel, etc. Ecrire les notes et les critiques est un principe didactique rendu nécessaire parce qu’il faut combattre les habitudes de prolixité, de déclamation et de paralogisme créées par la rhétorique. Ce type de travail intellectuel est nécessaire pour faire acquérir aux autodidactes la discipline des études que procure une scolarité régulière, pour tayloriser le travail intellectuel. Est utile dans le même sens le principe des « anciens de Sainte Zita » dont parle De Sanctis dans ses souvenirs sur l’école napolitaine de Basilio Puoti : c’est-à-dire une certaine « stratification » des capacités et aptitudes et la formation de groupes de niveaux sous la direction des plus expérimentés et des plus avancés, pour qu’ils accélèrent la préparation des plus retardés et des moins formés.\par
Un point important dans l’étude de l’organisation pratique de l’école unitaire concerne le cours de la scolarité dans ses divers niveaux conformes à l’âge des élèves, à leur développement intellectuel et moral et aux fins que l’école elle-même veut atteindre. L'école unitaire ou de formation humaniste (ce terme d’humanisme entendu au sens large et non seulement dans son sens traditionnel) ou de culture générale, devrait se proposer d’insérer les jeunes dans l’activité sociale après les avoir conduits à un certain niveau de maturité et de capacité pour la création intellectuelle et pratique, et d’autonomie dans l’orientation et l’initiative. La fixation de l’âge scolaire obligatoire dépend des conditions économiques générales, car celles-ci peuvent contraindre à demander aux jeunes et aux enfants un certain apport productif immédiat. L'école unitaire exige que l’Etat puisse assumer les dépenses qui sont aujourd’hui à la charge des familles pour l’entretien des élèves, c’est-à-dire qu’il transforme de fond en comble le budget du ministère de l’Education nationale, en l’étendant de façon inouïe et en le compliquant : toute la fonction d’éducation et de formation des nouvelles générations cesse d’être privée pour devenir publique, car ainsi seulement elle peut englober toutes les générations sans divisions de groupes ou de castes. Mais cette transformation de l’activité scolaire demande un développement inouï de l’organisation pratique de l’école, c’est-à-dire des bâtiments, du matériel scientifique, du corps enseignant, etc.\par
En particulier le corps enseignant devrait être plus nombreux, car l’efficacité de l’école est d’autant plus grande et intense que le rapport entre maître et élèves est plus petit, ce qui renvoie à d’autres problèmes dont la solution n’est ni facile ni rapide. Même la question des bâtiments n’est pas simple, parce que ce type d’école devrait être un collège avec dortoirs, réfectoires, bibliothèques spécialisées, salles adaptées aux travaux de séminaires, etc. C'est pourquoi, au début, ce nouveau type d’école devra être et ne pourra être que réservé à des groupes restreints, à des jeunes choisis par concours ou désignés, sous leur responsabilité, par des institutions appropriées.\par
L'école unitaire devrait correspondre à la période représentée aujourd’hui par les écoles élémentaires et moyennes, réorganisées non seulement pour le contenu et la méthode d’enseignement, mais aussi pour la disposition des différents niveaux de la scolarité. Le premier degré élémentaire ne devrait pas dépasser trois ou quatre années et, à côté de l’enseignement des premières notions « instrumentales » de l’instruction - lire, écrire, compter, géographie, histoire -, il devrait développer spécialement le domaine aujourd’hui négligé des « droits et devoirs » ; c’est-à-dire les premières notions de l’Etat et de la Société, entant qu’éléments primordiaux d’une nouvelle conception du monde qui entre en lutte avec les conceptions données parles divers milieux sociaux traditionnels, conceptions qu’on peut appeler folkloriques. Le problème didactique à résoudre est de tempérer et féconder l’orientation dogmatique qui ne peut pas ne pas être propre à ces premières années. Le reste du cursus ne devrait pas durer plus de six ans, de sorte qu’à quinze-seize ans, on devrait pouvoir avoir franchi tous les degrés de l’école unitaire.\par
On peut objecter qu’un tel cursus est trop fatigant par sa rapidité, si l’on veut atteindre effectivement les résultats que l’actuelle organisation de l’école classique se propose mais n’atteint pas. On peut dire pourtant que le complexe de la nouvelle organisation devra contenir en lui-même les éléments généraux qui font qu’aujourd’hui, pour une partie des élèves au moins, le cursus est au contraire trop lent. Quels sont ces éléments ? Dans une série de familles, en particulier celles des couches intellectuelles, les enfants trouvent dans la vie familiale une préparation, un prolongement et un complément de la vie scolaire ; ils absorbent, comme on dit, dans « l’air » quantité de notions et d’attitudes qui facilitent la scolarité proprement dite : ils connaissent déjà et développent la connaissance de la langue littéraire, c’est-à-dire le moyen d’expression et de connaissance, techniquement supérieur aux moyens possédés par la population scolaire moyenne de six à douze ans. C'est ainsi que les élèves de la ville, par le seul fait de vivre en ville, ont absorbé dès avant six ans quantité de notions et d’attitudes qui rendent la scolarité plus facile, plus profitable et plus rapide. Dans l’organisation interne de l’école unitaire doivent être créées au moins les principales de ces conditions, outre le fait, qui est à supposer, que parallèlement à l’école unitaire se développerait un réseau de jardins d’enfants et autres institutions dans lesquelles, même avant l’âge scolaire, les petits enfants seraient habitués à une certaine discipline collective et pourraient acquérir des notions et des habitudes préscolaires. En fait, l’école unitaire devrait être organisée comme un collège avec une vie collective diurne et nocturne, libérée des formes actuelles de discipline hypocrite et mécanique, et l’étude devrait se faire collectivement, avec l’aide des maîtres et des meilleurs élèves, même pendant les heures de travail dit individuel, etc.\par
Le problème fondamental se pose pour la phase du cursus actuel représenté \emph{aujourd’hui} par le lycée, phase qui \emph{aujourd’hui} ne se différencie en rien, comme type d’enseignement, des classes précédentes ; sinon par la supposition abstraite d’une plus grande maturité intellectuelle et morale de l’élève, conforme à son âge plus avancé et à l’expérience précédemment accumulée.\par
En fait, entre le lycée et l’université -c’est-à-dire entre l’école proprement dite et la vie -il y a aujourd’hui un saut, une véritable solution de continuité, et non un passage rationnel de la quantité (âge) à la qualité (maturité intellectuelle et morale). De l’enseignement presque purement dogmatique, dans lequel la mémoire joue un grand rôle, on passe à la phase créatrice ou au travail autonome et indépendant ; de l’école avec discipline d’étude imposée et contrôlée de façon autoritaire, on passe à une phase d’étude ou de travail professionnel où l’autodiscipline intellectuelle et l’autonomie morale sont théoriquement illimitées. Et cela arrive tout de suite après la crise de la puberté, quand la fougue des passions instinctives et élémentaires n’a pas encore fini de lutter avec les freins du caractère et de la conscience morale en formation. De plus, en Italie, où dans les universités le principe du travail de « séminaire » n’est pas répandu, le passage est encore plus brusque et mécanique.\par
 Il en résulte que, dans l’école unitaire, la phase ultime doit être conçue et organisée comme la phase décisive où l’on tend à créer les valeurs fondamentales de l’ « humanisme », l’auto-discipline intellectuelle et l’autonomie morale nécessaires pour la spécialisation ultérieure, qu’elle soit de caractère scientifique (études universitaires) ou de caractère immédiatement pratico-productif (industrie, bureaucratie, organisation des échanges, etc.). L'étude et l’apprentissage des méthodes créatrices dans la vie doivent commencer dans cette ultime phase de l’école, ne doivent plus être un monopole de l’université ni être laissés au hasard de la vie pratique : cette phase de la scolarité doit déjà contribuer à développer dans les individus l’élément de la responsabilité autonome, doit être une école créatrice. Il convient de distinguer entre école créatrice et école active, même sous la forme que lui donne la méthode Dalton. Toute l’école unitaire est école active, même s’il faut poser des limites aux idéologies libertaires dans ce domaine et revendiquer avec une certaine énergie le devoir pour les générations adultes, c’est-à-dire pour l’Etat, de « conformer » les nouvelles générations. On en est encore à la phase romantique de l’école active, phase dans laquelle les éléments de lutte contre l’école mécanique et jésuitique se sont dilatés de façon malsaine, pour des motifs conflictuels et polémiques : il convient d’entrer dans la phase « classique », rationnelle, de trouver dans les buts à atteindre la source naturelle pour élaborer les méthodes et les formes.\par
L'école créatrice est le couronnement de l’école active : dans la première phase on tend à discipliner, donc aussi à niveler, à obtenir une certaine espèce de « conformisme » qu’on peut appeler « dynamique » ; dans la phase créatrice, sur la base déjà acquise de la « collectivisation » du type social, on tend à l’expansion de la personnalité, devenue autonome et responsable, mais avec une conscience morale et sociale solide et homogène. Ainsi, école créatrice ne veut pas dire école d’ « inventeurs et découvreurs » ; il s’agit d’une phase et d’une méthode de recherche et de connaissance, et non d’un « programme » prédéterminé avec obligation à l’originalité et à l’innovation à tout prix. Il s’agit d’un apprentissage qui a lieu spécialement par un effort spontané et autonome du disciple, le maître exerçant seulement une fonction de guide amical comme cela se passe ou devrait se passer à l’université. Découvrir par soi-même, sans suggestion ni aide extérieure, c’est création, même si la vérité n’est pas neuve, et cela montre qu’on possède la méthode ; cela indique qu’en tout cas on est entré dans une phase de maturité intellectuelle permettant de découvrir des vérités nouvelles. C'est pourquoi dans cette phase l’activité scolaire fondamentale se déroulera dans les séminaires, dans les bibliothèques, dans les laboratoires expérimentaux ; c’est dans cette phase qu’on recueillera les indications organiques pour l’orientation professionnelle.\par
L'avènement de l’école unitaire signifie le début de nouveaux rapports entre travail intellectuel et travail industriel non seulement à l’école, mais dans toute la vie sociale. Le principe unitaire se reflètera donc dans tous les organismes de culture, en les transformant et en leur donnant un nouveau contenu (Int., pp. 97-103).\par
{\raggedleft \noindent [1930]}
\section[{ Problème de la nouvelle fonction que pourront remplir les universités et les académies}]{ Problème de la nouvelle fonction que pourront remplir les universités et les académies}
\noindent Aujourd’hui, ces deux institutions sont indépendantes l’une de l’autre et les Académies sont le symbole, qu’on raille souvent avec raison, de la séparation qui existe entre la haute culture et la vie, entre les intellectuels et le peuple. (D'où un certain succès que connurent les futuristes dans leur première période de \emph{Sturm und Drang}\footnote{Sturm und Drang (Tumulte et assaut) : Mouvement littéraire qui, dans l’Allemagne du XVIII° siècle, précéda le romantisme} anti-académique, anti-traditionaliste, etc.).\par
Dans un nouvel équilibre de rapports entre la vie et la culture, entre le travail intellectuel et le travail industriel, les Académies devraient devenir l’organisation culturelle (de systématisation, d’expansion et de création intellectuelle) des éléments qui après l’école unitaire passeront au travail professionnel, et un terrain de rencontre entre eux et les universitaires. Les éléments sociaux employés dans un travail professionnel ne doivent pas sombrer dans la passivité intellectuelle ; ils doivent avoir à leur disposition (par une initiative collective et non particulière, comme fonction sociale organique reconnue de nécessité et d’utilité publiques) des instituts spécialisés dans toutes les branches de la recherche et du travail scientifiques. Ils pourront y collaborer et y trouveront tout ce qui sera nécessaire pour toute forme d’activité culturelle qu’ils voudront entreprendre.\par
L'organisation académique devra être refondue et vivifiée de fond en comble. Il y aura une centralisation territoriale de compétences et de spécialisations : des centres nationaux auxquels s’adjoindront les grandes institutions existantes, des sections régionales et provinciales et des cercles locaux, urbains et ruraux. Les sections correspondent aux compétences culturelles et scientifiques ; elles seront toutes représentées dans les centres supérieurs et seulement en partie dans les cercles locaux. Unifier les différents types d’organisation culturelle existants : Académies, Instituts de culture, cercles de philologie, etc. Intégrer le travail académique traditionnel (qui s’emploie surtout à systématiser le savoir passé ou qui cherche à fixer la moyenne de la pensée nationale pour guider l’activité intellectuelle) à des activités liées à la vie collective, au monde de la production et du travail. On contrôlera les conférences industrielles, l’activité de l’organisation scientifique du travail, les cabinets expérimentaux des usines, etc. On construira un mécanisme pour sélectionner et faire progresser les capacités individuelles de la masse du peuple, qui aujourd’hui sont sacrifiées et s’égarent en erreurs et en tentatives sans issue. Chaque cercle local devra nécessairement comporter sa section de sciences morales et politiques et au fur et à mesure, organiser les autres sections spéciales pour discuter des aspects techniques des problèmes industriels agraires, d’organisation et de rationalisation du travail, à l’usine, aux champs 'et dans les bureaux, etc. Des congrès périodiques de divers degrés feront connaître les plus capables.\par
Il serait utile d’avoir le catalogue complet des Académies et des autres organisations culturelles qui existent aujourd’hui ainsi que des sujets qui sont de préférence traités dans leurs travaux et publiés dans leurs « Annales » ; il s’agit là en grande partie de cimetières de la culture, pourtant ces institutions ont aussi une fonction dans la psychologie des classes dirigeantes. \\
La collaboration de ces organismes avec les universités devrait être étroite ainsi qu’avec toutes les écoles supérieures spécialisées, de tout genre (militaires, navales, etc.). Le but est d’obtenir une centralisation et une impulsion de la culture nationale qui seraient supérieures à celles obtenues par l’Eglise catholique\footnote{Ce schéma d’organisation du travail culturel selon les principes généraux de l’école unitaire, devrait être développé dans \emph{toutes} ses parties avec soin et servir de guide pour constituer tout centre de culture même le plus élémentaire et le plus primitif ; il devrait être conçu comme un embryon et une molécule de tout l’ensemble de la structure. Même les initiatives que l’on sait transitoires et expérimentales devraient être conçues comme capables d’être absorbées dans le schéma général et en même temps comme éléments vitaux qui tendent à créer tout le schéma. Etudier avec attention l’organisation et le développement du Rotary-Club. \emph{(Note de Gramsci.)}} (Int., pp. 103-105).\par
{\raggedleft \noindent [1930]}
\section[{Pour la recherche du principe éducatif}]{Pour la recherche du principe éducatif}
\noindent La coupure déterminée par la réforme Gentile\footnote{Cf - p. 308.]} entre l’école élémentaire et moyenne d’une part, et l’école supérieure d’autre part. Avant la réforme une semblable coupure n’existait de façon très marquée qu’entre l’école professionnelle d’une part et les écoles moyennes et supérieures d’autre part ; l’école élémentaire était placée dans une sorte de limbe par certains de ses caractères particuliers.\par
Dans les écoles élémentaires deux éléments se prêtaient à l’éducation et à la formation des enfants : les premières notions des sciences naturelles et les notions des droits et devoirs du citoyen. Les notions scientifiques devaient servir à introduire l’enfant dans la \emph{societas rerum}\footnote{\emph{Societas rerum} (La société des choses) : c’est-à-dire la nature, par opposition, à la société humaine.}, les droits et devoirs dans la vie de l’Etat et dans la société civile. Les notions scientifiques entraient en lutte contre la conception magique du monde et de la nature que l’enfant absorbe dans un milieu imprégné de folklore, comme les notions de droits et devoirs entraient en lutte contre les tendances à la barbarie individualiste et particulariste, qui est elle aussi un aspect du folklore. L'école, par son enseignement, lutte contre le folklore, contre toutes les sédimentations traditionnelles de conceptions du monde pour répandre une conception plus moderne dont les éléments primitifs et fondamentaux sont fournis par l’apprentissage : des lois de la nature comme chose objective et rebelle, à quoi il faut s’adapter pour les dominer ; des lois de la société civile et de l’Etat qui sont produites par une activité humaine, qui sont établies par l’homme et que l’homme peut changer en vue de son développement collectif ; la loi civile et d’Etat dispose les hommes de la façon historiquement la plus apte à dominer les lois de la nature, c’est-à-dire à faciliter leur travail : manière propre à l’homme de participer activement à la vie de la nature pour la transformer et la socialiser toujours davantage, en profondeur et en extension. On peut donc dire que le principe éducatif qui fondait les écoles élémentaires était le concept de travail, qui ne peut se réaliser dans toute sa puissance d’expansion et de productivité sans une connaissance exacte et réaliste des lois de la nature et sans un ordre légal qui règle organiquement les rapports des hommes entre eux, ordre qui doit être respecté par convention spontanée et non seulement parce qu’il est imposé de l’extérieur, par nécessité qu’on reconnaît et se propose à soi-même comme liberté et non par pure coercition. Le concept et le fait du travail (de l’activité théorico-pratique) est le principe éducatif immanent à l’école élémentaire puisque c’est par le travail que l’ordre social et étatique (droits et devoirs) est introduit dans l’ordre naturel et identifié à lui. Le concept de l’équilibre entre ordre social et ordre naturel sur la base du travail, de l’activité théorico-pratique de l’homme, crée les premiers éléments d’une intuition du monde libérée de toute magie et sorcellerie, et fournit le point d’appui pour le développement ultérieur d’une conception historique, dialectique du monde, pour comprendre le mouvement et le devenir, pour évaluer la somme d’efforts et de sacrifices que le présent a coûté au passé et que l’avenir coûte au présent, pour concevoir l’actualité comme synthèse du passé, de toutes les générations passées, se projetant dans le futur. C'est là le fondement de l’école élémentaire ; qu’il ait porté tous ses fruits, que dans le corps enseignant ait été présente la conscience de sa tâche et du contenu philosophique de sa tâche, c’est une autre question, liée à la critique du niveau de conscience civique de toute la nation, dont le corps enseignant n’était qu’une expression, et encore appauvrie, certes pas une avant-garde.\par
Il n’est pas tout à fait exact que l’instruction ne soit pas en même temps éducation. avoir trop insisté sur cette distinction a été une grave erreur de la pédagogie idéaliste, et l’on en voit déjà les effets dans l’école réorganisée par cette pédagogie. Pour que l’instruction ne fût pas en même temps éducation, il faudrait que le disciple fût une pure passivité, un « récipient mécanique » de notions abstraites, chose absurde et d’ailleurs « abstraitement » niée par les tenants de la pure éducativité, précisément, contre la pure instruction mécanique. Le « certain » devient « vrai » \footnote{On peut être certain d’une chose sans qu’elle soit vraie pour cela. Inversement une chose peut être vraie à notre insu. On peut même reconnaître sa vérité extérieurement, du bout des lèvres, sans en être « certain », intimement persuadé. Cette distinction entre la certitude (subjective) et la vérité (objective) joue un grand rôle dans la philosophie de Hegel.} dans la conscience de l’enfant. Mais la conscience de l’enfant n’est pas quelque chose d’ « individuel » (et encore moins d’individualisé), elle est le reflet de la fraction de la société civile à laquelle l’enfant participe, des rapports sociaux tels qu’ils se nouent dans la famille, le voisinage, le village, etc. La conscience individuelle de la très grande majorité des enfants reflète des rapports civils et culturels divers et s’opposant à ceux qui sont représentés par les programmes scolaires : le « certain » d’une culture avancée devient « vrai » dans le cadre d’une culture fossilisée et anachronique, il n’y a pas d’unité entre l’école et la vie, c’est pourquoi il n’y a pas d’unité entre l’instruction et l’éducation. On peut donc dire qu’à l’école le lien instruction-éducation ne peut être représenté que par le travail vivant du maître, dans la mesure où le maître est conscient des contradictions entre le type de société et de culture qu’il représente et le type de société et de culture représenté par les élèves, conscient de sa tâche qui consiste à accélérer et discipliner chez l’enfant une formation conforme au type supérieur en lutte avec le type inférieur. Si le corps enseignant est déficient et si l’on brise le lien instruction-éducation pour résoudre le problème de l’enseignement schématiquement, sur le papier, en exaltant l’éducativité, l’œuvre du maître en deviendra encore plus médiocre : on aura une école rhétorique, sans sérieux, parce qu’il y manquera la matérialité physique du certain, et le vrai sera une vérité en paroles, rhétorique justement.\par
La dégénérescence est encore plus visible à l’école moyenne pour les cours de littérature et de philosophie. Avant, les élèves se constituaient au moins un certain « bagage » ou « équipement » (comme on voudra) de notions concrètes maintenant que le maître doit être surtout un philosophe et un esthète, l’élève néglige les notions concrètes et « se remplit la tête » de formules et de mots qui, la plupart du temps, n’ont pas de sens pour lui et sont tout de suite oubliés. La lutte contre la vieille école était juste, mais la réforme n’était pas si simple qu’on le croyait, il ne s’agissait pas de schémas programmatiques, mais d’hommes, et non des hommes qui sont directement des maîtres, mais de tout le complexe social dont les hommes sont l’expression. En réalité un enseignant médiocre peut réussir à rendre les élèves plus instruits, il ne réussira pas à les rendre plus cultivés, il accomplira avec scrupule et conscience bureaucratique la partie mécanique de l’enseignement et l’élève, si c’est un cerveau actif, organisera pour son propre compte, et avec l’aide de son milieu social, le « bagage » accumulé. Avec les nouveaux programmes, qui coïncident avec un abaissement général du niveau du corps enseignant, il n’y aura plus du tout de « bagage » à organiser. Les nouveaux programmes auraient dû abolir complètement les examens ; passer un examen, aujourd’hui, doit être terriblement plus un « jeu de hasard » qu’autrefois. Une date est toujours une date, quel que soit l’examinateur, et une « définition » est toujours une définition, mais un jugement, une analyse esthétique ou philosophique ?\par
L'efficacité éducative de la vieille école moyenne italienne, telle que l’avait organisée la vieille loi Casati, n’était pas à chercher (ou à nier) dans la volonté expresse d’être ou non école éducatrice, mais dans le fait que son organisation et ses programmes étaient l’expression d’un mode traditionnel de vie intellectuelle et morale, d’un climat culturel répandu dans toute la société italienne par de très anciennes traditions. Un tel climat et un tel mode de vie sont entrés en agonie, et l’école s’est détachée de la vie : c’est ce qui a déterminé la crise de l’école. Critiquer les programmes et l’organisation disciplinaire de l’école, cela signifie moins que rien si l’on ne tient pas compte de telles conditions. Ceci nous ramène à la participation réellement active de l’élève à l’école, participation qui ne peut exister que si l’école est liée à la vie. Quant aux nouveaux programmes, plus ils affirment et théorisent l’activité du disciple et sa collaboration active au travail de l’enseignant, plus ils sont prévus comme si le disciple était une pure passivité.\par
Dans la vieille école l’étude grammaticale des langues latine et grecque, jointe à l’étude des littératures et des histoires politiques respectives, était un principe éducatif dans la mesure où l’idéal humaniste, qui s’incarne dans Athènes et Rome, était répandu dans toute la société, était un élément essentiel de la vie et de la culture nationales. Même le caractère mécanique de l’étude grammaticale était vivifié par la perspective culturelle. Les notions particulières n’étaient pas apprises en vue d’un but immédiat pratico-professionnel : le but apparaissait désintéressé parce que l’intérêt était le développement intérieur de la personnalité, la formation du caractère à travers l’absorption et l’assimilation de tout le passé culturel de la civilisation européenne moderne. On n’apprenait pas le latin et le- grec pour les parler, pour devenir employé d’hôtel, interprète, correspondant commercial. On les apprenait pour connaître la civilisation des deux peuples, présupposé nécessaire à la civilisation moderne, c’est-à-dire pour être soi-même et se connaître soi-même en pleine conscience. Les langues latine et grecque étaient apprises selon la grammaire, mécaniquement, mais il y a beaucoup d’injustice et d’impropriété dans l’accusation de mécanisme et d’aridité. On a affaire à de jeunes enfants auxquels il importe de faire acquérir certaines habitudes de diligence, d’exactitude, de bonne tenue même physique, de concentration psychique sur des sujets déterminés, habitudes qu’on ne peut acquérir sans répétition mécanique d’actes disciplinés et méthodiques. Un savant de quarante ans serait-il capable de rester seize heures de suite assis à son bureau s’il n’avait dès l’enfance été contraint, par coercition mécanique, d’adopter les habitudes psycho-physiques appropriées ? Si l’on veut sélectionner de grands hommes de science, c’est encore par là qu’il faut commencer, et c’est sur tout le domaine scolaire qu’il faut faire pression pour réussir à faire émerger ces milliers ou ces centaines, ou ne serait-ce que ces douzaines de savants de grand talent, dont toute civilisation a besoin (même si l’on peut faire de grands progrès dans ce domaine, à l’aide des crédits scientifiques adéquats, sans revenir aux méthodes scolaires des jésuites).\par
On apprend le latin (ou mieux, on étudie le latin), on l’analyse jusqu’à ses subdivisions les plus élémentaires, on l’analyse comme une chose morte, c’est vrai, mais toute analyse faite par un enfant ne peut porter que sur des choses mortes ; d’autre part, il ne faut pas oublier que là où cette étude est faite sous cette forme, la vie des Romains est un mythe qui, dans une certaine mesure, a déjà intéressé l’enfant et l’intéresse, si bien que dans ce qui est mort est présente une plus grande vie. Et puis, la langue est morte, est analysée comme une chose inerte, comme un cadavre sur la table de dissection, mais elle revit continuellement dans les exemples, dans les narrations. Pourrait-on étudier de la même façon l’italien ? Impossible ; aucune langue vivante ne pourrait être étudiée comme le latin, cela serait et semblerait absurde. Aucun enfant ne connaît le latin quand il en commence l’étude par une telle méthode analytique. Une langue vivante pourrait être connue et il suffirait qu’un seul enfant la connaisse pour rompre le charme : tous iraient à l’école Berlitz, tout de suite. Le latin (le grec aussi) se présente à l’imagination comme un mythe, même pour l’enseignant. On n’étudie pas le latin pour apprendre le latin ; depuis longtemps, en vertu d’une tradition culturelle-scolaire dont on pourrait rechercher l’origine et le développement, on étudie le latin comme élément d’un programme scolaire idéal, élément qui résume et satisfait toute une série d’exigences pédagogiques et psychologiques ; on l’étudie pour habituer les enfants à étudier d’une façon déterminée, à analyser un corps historique qu’on peut traiter comme un cadavre constamment rappelé à la vie ; pour les habituer à raisonner, à abstraire schématiquement tout en étant capables de redescendre de l’abstraction à la vie réelle immédiate, pour voir dans chaque fait ou chaque donnée ce qu’il a de général et ce qu’il a de particulier, le concept et l’individu. Et la constante comparaison entre le latin et la langue qu’on parle, que ne signifie-t-elle pas du point de vue éducatif ? La distinction et l’identification des mots et des concepts, toute la logique formelle avec les contradictions des opposés et l’analyse des différents, avec le mouvement historique de l’ensemble linguistique qui se modifie dans le temps, qui a un devenir et n’est pas seulement une entité statique. Pendant les huit ans de gymnase-lycée\footnote{Gymnase : Nom donné en Italie à des établissements scolaires comportant les premières classes du second degré. Quelque chose comme nos C.E.S.}. on étudie toute la langue historiquement réelle, après l’avoir vue photographiée dans un instant abstrait sous forme de grammaire : on l’étudie depuis Ennius (et même depuis les termes des fragments des Douze Tables)jusqu’à Phèdre et aux auteurs chrétiens ; un processus historique est analysé de sa naissance à sa mort dans le temps, mort apparente puisqu’on sait que l’italien, auquel le latin est continuellement confronté, est du latin moderne. On étudie la grammaire d’une certaine époque, une abstraction, le vocabulaire d’une période déterminée, mais on étudie (par comparaison) la grammaire et le vocabulaire de chaque auteur déterminé, et la signification de chaque terme dans chaque « période » (stylistique) déterminée, on découvre ainsi que la grammaire et le vocabulaire de Phèdre ne sont pas ceux de Cicéron, ni ceux de Plaute ou de Lactance et Tertullien, qu’un même assemblage de sons n’a pas la même signification à différentes époques, chez différents écrivains. On compare continuellement le latin et l’italien ; mais chaque mot est un concept, une image dont la coloration varie selon les temps et les personnes dans chacune des deux langues comparées. On étudie l’histoire littéraire des livres écrits dans cette langue, l’histoire politique, les hauts faits des hommes qui ont parlé cette langue. Tout ce complexe organique détermine l’éducation du jeune homme, du fait qu’il a parcouru, ne serait-ce que matériellement, cet itinéraire avec ces étapes, etc. Il s’est plongé dans l’histoire, il a acquis une intuition historiciste du monde et de la vie, qui devient une seconde nature, presque une spontanéité, parce qu’elle n’a pas été inculquée de façon pédantesque, par une « volonté » extrinsèquement éducative. Cette étude éduquait sans en avoir la volonté expressément déclarée, avec le minimum d’intervention c éducatrice » de l’enseignant : elle éduquait parce qu’elle instruisait. Des expériences logiques, artistiques, psychologiques étaient faites sans « y réfléchir », sans se regarder continuellement dans la glace, et surtout était faite une grande expérience « synthétique », philosophique, de développement historico-réel. Cela ne veut pas dire (et le penser serait stupide) que le latin et le grec, comme tels, aient des vertus intrinsèquement thaumaturgiques dans le domaine éducatif. C'est toute la tradition culturelle, vivante aussi et surtout hors de l’école, qui, dans un milieu donné, produit de telles conséquences. On voit d’ailleurs comment, une fois changée la traditionnelle intuition de la culture, l’école est entrée en crise, et est entrée en crise l’étude du latin et du grec.\par
Il faudra remplacer le latin et le grec comme point d’appui de l’école formatrice et on les remplacera, mais il ne sera pas facile de disposer la nouvelle matière ou la nouvelle série de matières dans un ordre didactique qui donne des résultats équivalents pour l’éducation et la formation générale de la personnalité, depuis l’enfance jusqu’au seuil du choix professionnel. En effet, dans cette période les études ou la majeure partie des études doivent être désintéressées (ou apparaître telles à ceux qui apprennent), autrement dit ne pas avoir de buts pratiques immédiats ou trop immédiats, elles doivent être formatrices même si elles sont « instructives », c’est-à-dire riches de notions concrètes. Dans l’école actuelle, la crise profonde de la tradition culturelle, de la conception de la vie et de l’homme entraîne un processus de dégénérescence progressive : les écoles de type professionnel, c’est-à-dire préoccupées de satisfaire des intérêts pratiques immédiats, prennent l’avantage sur l’école formatrice, immédiatement désintéressée. L'aspect le plus paradoxal, c’est que ce nouveau type d’école paraît démocratique et est prôné comme tel, alors qu’elle est au contraire destinée non seulement à perpétuer les différences sociales, mais à les cristalliser à la chinoise\footnote{Allusion au système du mandarinat dans l’ancienne Chine.}.\par
L'école traditionnelle a été oligarchique parce que destinée à la nouvelle génération des groupes dirigeants, destinée à son tour à devenir dirigeante : mais elle n’était pas oligarchique par son mode d’enseignement. Ce n’est pas l’acquisition de capacités directives, ce n’est pas la tendance à former des hommes supérieurs qui donne son empreinte sociale à un type d’école. L'empreinte sociale est donnée par le fait que chaque groupe social a son propre type d’école, destiné à perpétuer dans ces couches une fonction traditionnelle déterminée, de direction ou d’exécution. Si l’on veut mettre en pièces cette trame, il convient donc ne de pas multiplier et graduer les types d’écoles professionnelles, mais de créer un type unique d’école préparatoire (élémentaire-moyenne) qui conduise le jeune homme jusqu’au seuil du choix professionnel, et le forme entre temps comme personne capable de penser, d’étudier, de diriger, ou de contrôler ceux qui dirigent.\par
La multiplication des types d’écoles professionnelles tend donc à pérenniser les différences traditionnelles. Mais comme, dans ces différences, elle tend à susciter des stratifications internes, voilà qu’elle donne l’impression d’avoir une tendance démocratique. Manœuvre et ouvrier qualifié, par exemple, paysan et géomètre ou petit agronome, etc. Mais la tendance démocratique, intrinsèquement, ne peut seulement signifier qu’un manœuvre devienne ouvrier qualifié ; elle signifie que tout « citoyen » peut devenir « gouvernant », et que la société le place, fût-ce « abstraitement » dans les conditions générales qui lui permettent de le devenir : la démocratie politique tend à faire coïncider gouvernants et gouvernés (en ce sens que le gouvernement doit avoir le consentement des gouvernés) en assurant à tout gouverné l’apprentissage gratuit de la capacité et de la préparation technique générale nécessaire à cet effet. Mais le type d’école qui se développe comme école pour le peuple ne tend même plus à maintenir l’illusion, puisqu’elle s’organise toujours davantage de manière à restreindre la base de la couche gouvernante techniquement préparée, dans un climat politique et social qui limite encore l’ « initiative privée » visant à donner cette capacité et cette préparation technico-politique, de sorte qu’on revient en réalité aux divisions en ordres « juridiquement » fixés et cristallisés, au lieu de dépasser les divisions en groupes : la multiplication des écoles professionnelles toujours plus spécialisées dès le début des études est une des manifestations les plus éclatantes de cette tendance.\par
A propos du dogmatisme et du criticisme-historicisme. à l’école élémentaire et moyenne, il est à noter que la nouvelle pédagogie a voulu battre en brèche le dogmatisme précisément dans le domaine de l’instruction, de l’acquisition des notions concrètes, c’est-à-dire précisément dans le domaine où un certain dogmatisme est pratiquement inévitable et ne peut être réabsorbé et dissout que dans le cycle entier du cours des études (on ne peut enseigner la grammaire historique dans les écoles élémentaires et au gymnase) ; mais il lui faut après cela voir introduire le dogmatisme par excellence dans le domaine de la pensée religieuse et voir décrire implicitement toute l’histoire de la philosophie comme une succession de folies et de délires. Dans l’enseignement de la philosophie, le nouveau cours pédagogique (au moins pour ces élèves, et ils sont l’immense majorité, qui ne reçoivent pas d’aide intellectuelle hors de l’école, en famille ou dans l’entourage familial, et doivent se former uniquement avec les indications reçues en classe) appauvrit l’enseignement et en rabaisse le niveau, pratiquement, bien que rationnellement il paraisse très beau, d’une très grande beauté utopique. La philosophie descriptive traditionnelle, renforcée par un cours d’histoire de la philosophie et par la lecture d’un certain nombre de philosophes, semble pratiquement la meilleure chose. La philosophie qui décrit et définit sera une abstraction dogmatique, comme la grammaire et la mathématique, mais c’est là une nécessité pédagogique et didactique. 1 = 1 est une abstraction, mais personne n’est conduit pour autant à penser que 1 mouche est égale à 1éléphant. Même les règles de la logique formelle sont des abstractions du même genre, elles sont comme la grammaire de la pensée normale, et pourtant, il faut les étudier car elles ne sont pas quelque chose d’inné mais doivent être acquises par le travail et la réflexion. Le nouveau cours présuppose que la logique formelle est quelque chose qu’on possède déjà quand on pense, mais n’explique pas comment on doit l’acquérir, si bien que pratiquement c’est comme si on la supposait innée. La logique formelle est comme la grammaire : elle est assimilée de façon « vivante » même si l’apprentissage nécessaire a été schématique et abstrait, car le disciple n’est pas un disque de phonographe, n’est par un récipient passivement mécanique, même si la liturgie conventionnelle des examens lui donne quelquefois cette apparence. Le rapport de ces schèmes éducatifs avec l’esprit enfantin est toujours actif et créateur, comme est actif et créateur le rapport entre l’ouvrier et ses instruments de travail ; un calibre est lui aussi un ensemble d’abstractions, et pourtant on ne produit pas d’objets réels sans calibrage, objets réels qui sont des rapports sociaux et contiennent implicitement des idées.\par
L'enfant qui s’escrime avec les \emph{barbara, baralipton}\footnote{Ces termes bizarres (dont Molière s’est moqué, dans \emph{le Bourgeois Gentilhomme}) n’ont aucune signification par eux-mêmes. Dans l’ancienne logique formelle, les lettres qui les composent correspondaient à une sorte de code permettant de repérer les différents types de raisonnement.} se fatigue certes, et il faut faire en sorte qu’il se fatigue autant qu’il est nécessaire et pas plus, mais il n’est pas moins certain qu’il devra toujours se fatiguer pour apprendre à se contraindre à des privations et limitations de mouvement physique, autrement dit se soumettre à un apprentissage psycho-physique. Il faut persuader beaucoup de gens que l’étude est elle aussi un métier, et très fatigant, avec son apprentissage spécial qui n’est pas seulement intellectuel, mais aussi musculaire-nerveux : c’est un processus d’adaptation, une habitude acquise avec effort, ennui et même souffrance. La participation de plus larges masses à l’école moyenne porte en elle la tendance à relâcher la discipline de l’étude, à demander des « facilités ». Beaucoup pensent carrément que les difficultés sont artificielles, parce qu’ils sont habitués à considérer comme travail et fatigue le seul travail manuel. La question est complexe. Certes l’enfant d’une famille traditionnelle d’intellectuels vient plus facilement à bout du processus d’adaptation psycho-physique ; dès la première fois qu’il entre en classe, il est avantagé sur plusieurs points par rapport à ses camarades, il a une orientation déjà acquise grâce aux habitudes familiales : il concentre plus facilement son attention parce qu’il a l’habitude d’une bonne tenue physique, etc. De la même façon, le fils d’un ouvrier de la ville souffre moins, quand il entre à l’usine, qu’un garçon de la campagne ou qu’un jeune paysan déjà formé pour la vie rurale. Même le régime alimentaire a une importance, etc. Voilà pourquoi beaucoup de gens du peuple pensent que dans la difficulté des études il doit y avoir un « truc » à leur désavantage (quand ils ne pensent pas être stupides par nature) : ils voient le monsieur (et pour beaucoup à la campagne surtout, monsieur veut dire intellectuel) accomplir avec souplesse et apparente facilité le travail qui coûte des larmes de sang à leurs enfants, et ils pensent qu’il doit y avoir un « truc ». Dans une situation nouvelle, ces problèmes peuvent devenir très ardus, et il faudra résister à la tendance à rendre facile ce qui ne peut l’être sans être dénaturé. Si l’on veut créer une nouvelle couche d’intellectuels, jusqu’aux plus grandes spécialisations, à partir d’un groupe social qui n’en a pas développé par tradition les attitudes, il faudra surmonter des difficultés inouïes.
\section[{ Quelques principes de la pédagogie moderne}]{ Quelques principes de la pédagogie moderne}
\noindent Chercher l’origine historique exacte de quelques principes de la pédagogie moderne : l’école active ou la collaboration amicale du maître et de l’élève ; l’école ouverte ; la nécessité de laisser libre cours au développement des facultés spontanées de l’écolier, sous la surveillance mais non sous le contrôle voyant du maître. La Suisse a apporté une grande contribution à la pédagogie moderne (Pestalozzi, etc.) à travers la tradition genevoise de Rousseau ; en réalité, cette pédagogie est une forme confuse de philosophie liée à une série de règles empiriques. On n’a pas tenu compte du fait que les idées de Rousseau sont une réaction violente contre l’école et contre les méthodes pédagogiques des jésuites et en tant que telles représentent un progrès ; mais il s’est formé ensuite une espèce d’église qui a paralysé les études pédagogiques et adonné lieu à de curieuses involutions (dans les doctrines de Gentile et de Lombardo-Radice). La « spontanéité » est une de ces involutions : on se représente presque le cerveau de l’enfant comme une pelote que le maître aide à dévider. En réalité, chaque génération éduque la nouvelle génération, c’est-à-dire la forme ; l’éducation est une lutte contre les instincts liés aux fonctions biologiques élémentaires, une lutte contre la nature pour la dominer et créer l’homme « actuel » dans son époque. On ne tient pas compte du f ait que l’enfant, dès qu’il commence à « voir et toucher » y peu de jours peut-être après la naissance, accumule des sensations et des images qui se multiplient et deviennent complexes au moment de l’apprentissage du langage. La « spontanéité », si on l’analyse, devient de plus en plus problématique.\footnote{Voir : Lettre à Julia du 30 décembre 1930 ; lettre à Tania du 7 mars 1932 ; lettre à Julia du 14 novembre 1931 pour le concept d’ « inclinaison infantile » (Gallimard 1971, pp. 373-374).} De plus, l’ « école », c’est-à-dire l’activité éducative directe, n’est qu’une partie de la vie de l’élève qui entre en contact aussi bien avec la société humaine qu’avec la \emph{societas rerum}, et se forme des critères à partir de ces sources « extra-scolaires » beaucoup plus importantes qu’on ne croit communément. L'école unique, intellectuelle et manuelle a aussi l’avantage de mettre l’enfant en contact en même temps avec l’histoire humaine et avec l’histoire des « choses » sous le contrôle du maître (Int., pp. 115-116).\par
{\raggedleft \noindent [1929-1930]}
\section[{ Pédagogie mécaniste et idéaliste}]{ Pédagogie mécaniste et idéaliste}
\noindent Pour élaborer un essai complet sur Antonio Labriola, il faut avoir à l’esprit, outre ses écrits qui sont rares et souvent seulement allusifs ou extrêmement synthétiques, les éléments et les fragments de conversation rapportés par ses amis et ses disciples (Labriola a laissé le souvenir d’un exceptionnel « causeur »). On peut recueillir çà et là dans les livres de B. Croce bon hombre de ces éléments et fragments. Ainsi, dans les \emph{Conversations critiques} (seconde série), édition italienne, pp. 60-61 : « Comment feriez-vous pour éduquer moralement un Papou ? » demanda un de nos élèves, voici plusieurs années, au professeur Labriola au cours d’une de ses leçons de pédagogie, en objection à l’efficacité de la pédagogie. « Provisoirement, répondit avec une âpreté digne de Vico et de Hegel le professeur hébartien, provisoirement j’en ferais un esclave ; ce serait la pédagogie adaptée à ce cas, quitte à voir si, pour ses petits et arrière-petits-fils, on pourra commencer à mettre en œuvre notre pédagogie\footnote{Croce, en accord avec Labriola, commente : « Le problème est là tout entier : non pas récuser le concept de culture, mais le définir exactement et trouver un mode approprié et concret de diffusion de la culture. Et ce mode peut être quelquefois le \emph{Odi profanum vulgus} (Je hais le vulgaire profane), et consister à repousser violemment les gens du seuil du temple de la science en les contraignant à rester dehors aussi longtemps qu’ils ne s’en rendent pas dignes. » (Conversations \emph{critiques}, série 11, Bari, Laterza, 1950, p. 61)}. « Il faut rapprocher cette réponse de Labriola de l’interview qu’il a donnée sur la question coloniale (Lybie) vers 1903 et rapportée dans le livre \emph{Divers écrits de philosophie et de politique}\footnote{Le \emph{Giornale d’Italia} avait demandé à Labriola « ce qu’il pensait, en tant que socialiste, d’une action de l’Italie à Tripoli, du point de vue de l’opportunité nationale et des intérêts du prolétariat ». Les journaux socialistes officiels avaient en effet, comme le rappelle le chapeau de la rédaction de l’interview, « repris leur vieille opposition à la politique d’expansion ». Labriola se dit favorable à l’entreprise et donna, dans la première partie de sa réponse, une justification « historiciste » générale de sa position. Voici le passage : « Les intérêts des socialistes ne peuvent pas être opposés aux intérêts nationaux, et ils doivent même les encourager sous toutes les formes. Les Etats d’Europe - je reprends ici des idées et des formules que j’ai exprimées d’autres fois - sont en devenir continuel et complexe, en ce qu’ils ambitionnent, conquièrent, assujettissent et exploitent dans tout le reste du monde. L'Italie ne peut pas se soustraire à ce développement des Etats qui porte avec lui un développement des peuples. Si elle le faisait, elle se soustrairait en réalité à la circulation de la vie moderne et resterait \emph{arriérée} en Europe. Le mouvement expansionniste des nations a ses raisons profondes dans la concurrence économique. « L'économie et la politique ne sont pas séparables à volonté et artificiellement. La lutte entre les Etats pour ce que l’on appelle la sphère d’influences et le rayon d’action, naît de la structure intime des Etats eux-mêmes et elle est le plus souvent la condition de leur progrès et le moyen de se donner consistance. Il n’est pas possible, dans les conditions actuelles effectives des Etats, que la concurrence cède le pas à une justice désincarnée et sans moyens de contrainte : je dis cela à ceux qui s’imaginent que l’on peut constituer \emph{l’arbitre international} pour concilier les querelles entre les nations et les forces automatiques et élémentaires de l’histoire ; je dis cela aujourd’hui à ceux qui, par aversion pour certaines formes de contrainte que les Etats sont obligés d’utiliser, préfèrent renoncer au relatif progrès qui naît de ce que l’on prend une part active à la concurrence impétueuse propre à notre temps. » (« Tripoli, le socialisme et l’expansion coloniale », tirée d’une interview au \emph{Giornale d’Italia} du 13 avril 1902, publiée dans \emph{Ecrits variés édités et inédits de philosophie politique} rassemblés et publiés par B. Croce, Bari, Laterza, 1906, pp. 433-434). - Labriola ajoutait ensuite que l’Italie, arrivée tardivement en Afrique pour y prendre « une position prédominante », aurait dû se contenter de la Tripolitaine qu’il fallait donc occuper avant que les autres ne le fassent, comme cela s’était produit pour l’Egypte et la Tunisie. Il considérait la question de Tripoli « comme le premier essai de notre première apparition libre dans la politique mondiale » et comme un choix politique qui, s’il était bien poursuivi, pouvait permettre de surmonter les maux chroniques du pays, l’assujettissement au capital étranger et l’émigration, en ouvrant la possibilité d’ « une politique de la population » et d’une politique économique capables de « faire de la masse de nos émigrants, des Italiens à même de peupler une nouvelle patrie ».}. Il faut aussi la rapprocher de la façon de penser de Gentile en ce qui concerne l’enseignement religieux dans les écoles primaires. Il semble que l’on ait affaire à un pseudo-historicisme, à un mécanisme assez empirique et très voisin du plus vulgaire évolutionnisme. On pourrait rappeler ce que dit Bertrando Spaventa à propos de ceux qui voudraient constamment maintenir les hommes au berceau (c’est-à-dire dans le moment de l’autorité qui éduque bien à la liberté les peuples primitifs) et qui pensent toute la vie (des autres) comme un berceau\footnote{Hegel avait affirmé que la servitude était le berceau de la liberté. Pour Hegel comme pour Machiavel, la « nouvelle souveraineté » (c’est-à-dire la période dictatoriale qui caractérise le début de chaque nouveau type d’Etat) et la servitude qui s’y rattache, ne se justifient que comme éducation et discipline de l’homme qui n’est pas encore libre. Aussi B. Spaventa (\emph{Principes d’Ethique}, Appendice, Naples, 1904) commente opportunément : « Mais le berceau n’est pas la vie. Quelques-uns nous voudraient constamment au berceau. » Un exemple typique de ce berceau que devient toute la vie est donné par le protectionnisme douanier, qui est toujours préconisé et justifié à titre de « berceau » mais tend à devenir un berceau éternel. \emph{(Note de Gramsci.)}}. Il me semble qu’il faut historiquement poser le problème d’une autre façon : une nation ou un groupe social parvenus à un degré supérieur de civilisation ne peuvent-ils pas (donc ne doivent-ils pas) « accélérer » le processus d’éducation des peuples et des groupes sociaux plus attardés, en universalisant et en traduisant de manière adaptée leur nouvelle expérience ? Ainsi, lorsque les Anglais enrôlent des recrues parmi les peuples primitifs qui n’ont jamais vu de fusils modernes, ils ne les instruisent pas au maniement de l’arc, du boomerang ou de la sarbacane, mais précisément à celui du fusil, bien que les normes d’instruction soient nécessairement adaptées à la « mentalité » de ce peuple primitif déterminé. La forme de pensée impliquée par la réponse de Labriola ne paraît donc ni dialectique ni progressiste mais plutôt mécanique et rétrograde, tout comme la pensée « pédagogique » religieuse de Gentile qui n’est rien d’autre qu’un dérivé du concept de la « religion bonne pour le peuple » (peuple = enfant = phase primitive de la pensée à laquelle correspond la religion, etc), à savoir la renonciation (tendancieuse) à éduquer le peuple. Le mécanisme implicite de la pensée de Labriola apparaît de façon encore plus évidente dans son interview sur la question coloniale. En effet, il se peut très bien qu’il soit « nécessaire de réduire les Papous en esclavage » pour les éduquer, mais il est tout aussi nécessaire d’affirmer que cette nécessité n’est que contingente car on se trouve dans des conditions déterminées, c’est-à-dire que cette nécessité est « historique » et non pas absolue : il est nécessaire au contraire de lutter à ce sujet, et cette lutte est justement la condition de la libération de l’esclavage et de l’éducation par la pédagogie moderne des petits et arrière-petits-fils des Papous. Que certains affirment nettement que la servitude des Papous n’est qu’une nécessité du moment et qu’ils se rebellent contre cette nécessité, est aussi un fait philosophico-historique :\par

\begin{enumerate}[itemsep=0pt,]
\item parce que cela contribuera à réduire la période d’esclavage au laps de temps nécessaire ; 
\item parce que cela amènera les Papous à réfléchir sur eux-mêmes, à s’autoéduquer, dans la mesure où ils comprendront qu’ils sont appuyés par des hommes d’une civilisation supérieure ; 
\item parce que cette résistance seule démontre que l’on est réellement dans une phase supérieure de civilisation et de pensée, etc.
\end{enumerate}

\noindent L'historicisme de Labriola et de Gentile est d’un genre très inférieur : c’est l’historicisme des juristes pour lesquels le knout n’est pas un knout quand il est un knout e historique ». Il s’agit d’ailleurs d’une façon de penser très nébuleuse et très confuse. Qu'une présentation « dogmatique » des notions scientifiques ou qu’une mythologie soit nécessaire dans les écoles élémentaires, ne signifie pas que le dogme doit être religieux et la mythologie, telle mythologie déterminée. Qu'un groupe social ou un peuple arriérés aient besoin d’une discipline extérieure coercitive pour être civilement éduqués ne signifie pas qu’ils doivent être réduits en esclavage ; à moins qu’on ne pense que toute coercition de l’Etat est esclavage. Il y a aussi une coercition de type militaire même pour le travail que l’on peut appliquer même à la classe dominante et qui n’est pas un « esclavage » mais l’expression adaptée de la pédagogie moderne destinée à éduquer un élément primitif (qui est bien sûr primitif mais au voisinage d’éléments déjà mûrs, alors que la servitude est organiquement l’expression de conditions universellement primitives). Spaventa, qui se plaçait du point de vue de la bourgeoisie libérale contre les « sophismes » historicistes des classes réactionnaires, ! exprimait, sous forme sarcastique, une conception bien plus progressiste et dialectique que celle de Labriola et de Gentile (Int., pp. 116-118).\par
{\raggedleft \noindent [1932-1933]}
\chapterclose


\chapteropen
\chapter[{Problèmes de critique littéraire}]{Problèmes de critique littéraire}\renewcommand{\leftmark}{Problèmes de critique littéraire}


\chaptercont
\section[{ L'art et la lutte pour une nouvelle civilisation}]{ L'art et la lutte pour une nouvelle civilisation}
\noindent Les rapports de nature artistique montrent, surtout dans la philosophie de la praxis, la naïveté prétentieuse des perroquets qui croient posséder, dans quelques petites formules stéréotypées, les clefs qui ouvrent toutes les portes (ces clefs s’appellent exactement des « rossignols »).\par
Deux écrivains peuvent représenter (exprimer) le même moment historico-social, mais l’un peut être un artiste et l’autre un simple scribouillard. Prétendre épuiser le problème en se bornant à décrire ce que les deux écrivains représentent ou expriment du point de vue social, c’est-à-dire en résumant, plus ou moins bien, les caractéristiques d’un moment historico-social déterminé, cela signifie qu’on n’a même pas effleuré le problème artistique. Tout cela peut être utile et nécessaire, et cela l’est même certainement, mais dans un autre domaine : dans celui de la critique politique, de la critique des murs, dans la lutte pour détruire et surmonter certains. courants de sentiments et de croyances, certaines attitudes envers la vie et le monde ; ce n’est pas de la' critique et de l’histoire de l’art, et cela ne peut être présenté comme tel, sous peine de tomber dans la confusion, de faire rétrograder ou d’immobiliser les concepts. scientifiques, c’est-à-dire précisément de ne pas poursuivre les fins propres à la lutte culturelle.\par
Un certain moment historico-social n’est jamais homogène, il est même riche en contradictions. Il acquiert une « personnalité », il est un « moment » du déroulement de l’histoire, par le fait qu’une certaine activité de la vie y domine les autres, il représente une « pointe » historique : mais cela suppose auparavant une hiérarchie, une opposition, une lutte. L'écrivain qui représente cette activité dominante, cette « pointe » historique, devrait représenter ce moment donné ; mais comment juger ceux qui représentent les autres activités, les autres éléments ? Ne sont-ils pas « représentatifs » eux aussi ? Et n’est-il pas, lui aussi, représentatif de ce « moment » celui qui en exprime les éléments « réactionnaires » et anachroniques ? Ou bien faudra-t-il considérer comme représentatif celui qui exprimera toutes les forces et tous les éléments en opposition et en lutte, c’est-à-dire celui qui représente les contradictions de l’ensemble historico-social ?\par
On peut aussi penser qu’une critique de la civilisation littéraire, une lutte pour créer une nouvelle culture, puisse être artistique dans ce sens qu’un art nouveau naîtra de la nouvelle culture, mais cela nous apparaît comme un sophisme. De toute façon, c’est peut-être en partant de telles suppositions que l’on peut mieux comprendre le rapport De Sanctis-Croce et les polémiques sur le contenu et la forme. La critique de De Sanctis est une critique militante, et non pas de façon t froide », esthétique ; elle est la critique d’une période de luttes culturelles, d’oppositions entre des conceptions de la vie antagonistes. Les analyses du contenu, la critique de ! la « structure » des œuvres c’est-à-dire de la cohérence logique et historique-actuelle des masses de sentiments représentés de f acon artistique, tout cela est lié à cette lutte culturelle : c’est justement en cela que consiste, semble-t-il, la profonde humanité et l’humanisme de De Sanctis, qui rendent si sympathique, aujourd’hui encore, le critique lui-même. On aime sentir en lui la ferveur passionnée de l’homme de parti, qui a de solides convictions morales et politiques et qui ne les cache pas, qui ne tente même pas de les cacher. Croce arrive à distinguer les divers aspects du critique qui, chez De Sanctis, étaient organiquement unis, fondus ensemble. Chez Croce revivent les mêmes motifs culturels que chez De Sanctis, mais dans leur période d’expansion et de triomphe ; il continue à lutter, mais pour un raffinement de la culture (d’une certaine culture), non pour son droit à la vie : la passion et la ferveur romantiques se sont accordées dans une sérénité supérieure et dans une indulgence pleine de bonhomie. Mais même chez Croce cette position n’est pas permanente : elle est suivie d’une phase où la sérénité et l’indulgence se fêlent et où l’on voit affleurer l’acrimonie et une colère difficilement contenue : phase défensive, non plus agressive et fervente, et par conséquent qui ne peut être comparée avec l’attitude correspondante de De Sanctis.\par
En somme, le type de critique littéraire propre à la philosophie de la praxis nous est offert par De Sanctis, non par Croce ou par -tout autre critique (et surtout pas par Carducci) : elle doit fondre la lutte pour une nouvelle culture, c’est-à-dire pour un nouvel humanisme, la critique des murs, des sentiments et des conceptions du monde, avec la critique esthétique ou purement artistique, dans une ferveur passionnée, même sous la forme du sarcasme… (L.V.N., pp. 6-7.) [1934]\par
Qu'il faille parler, pour être exact, de lutte pour une « nouvelle culture », et non pour un « art nouveau » (au sens immédiat du terme), cela paraît évident. Peut-être ne peut-on même pas dire, pour être exact, que l’on lutte pour un nouveau contenu de l’art, car celui-ci ne peut être pensé de façon abstraite, séparé de la forme. Lutter pour un art nouveau voudrait dire lutter pour créer de nouveaux artistes individuels, ce qui est absurde, car on ne peut créer artificiellement des artistes. Il faut parler de lutte pour une nouvelle culture, c’est-à-dire pour une nouvelle vie morale, qui ne peut pas ne pas être intimement liée à une nouvelle intuition de la vie, jusqu’à ce qu’elle devienne une nouvelle façon de sentir et de voir la réalité, et par conséquent un monde lié dans sa nature profonde avec les « artistes possibles », avec les « œuvres d’art possibles ».\par
Le fait qu’on ne peut créer artificiellement des artistes individuels ne signifie donc pas que le nouveau monde culturel pour lequel on lutte, en suscitant passions et chaleur humaine, suscite nécessairement de « nouveaux artistes » ; c’est-à-dire on ne peut affirmer que Durand ou Dupont deviendront des artistes, mais on peut affirmer que, du mouvement même naîtront de nouveaux artistes. Un nouveau groupe social qui entre dans la vie de l’histoire en aspirant à l’hégémonie, avec une assurance, une confiance en lui-même qu’il n’avait pas auparavant ne peut pas ne pas faire naître en son sein des personnalités qui, auparavant, n’auraient pas eu ! la force nécessaire pour s’exprimer totalement dans un certain sens.\par
Ainsi on ne peut dire qu’il se formera un nouveau « souffle poétique », selon l’expression qui fut à la mode il y a quelques années. Le « souffle poétique » n’est qu’une métaphore pour exprimer l’ensemble des artistes qui se sont déjà formés ou révélés ou du moins le processus de formation et de révélation déjà commencé et déjà consolidé… (L.V.N., pp. 9-10.) [1934]
\section[{L'art éducateur}]{L'art éducateur}

\begin{quoteblock}
 \noindent « L'art est éducateur en tant qu’art, mais non en tant qu’ » art éducateur « , parce que dans ce cas il n’est rien et que le néant ne peut éduquer. Certes, il semble que nous soyons tous d’accord pour désirer un art qui ressemble à celui du Risorgimento et non, par exemple, à celui de ! la période d’annunzienne ; mais, à la vérité, si l’on considère bien ce désir, il n’y a pas en lui le désir d’un art de préférence à un autre, mais bien le désir d’une réalité morale de préférence à une autre. De la même façon, quelqu’un qui désire qu’un miroir reflète une belle personne plutôt qu’une laide, ne souhaite pas avoir un miroir différent de celui qui est devant lui, mais une personne différente\footnote{Croce \emph{: Cultura e vita morale}, pp. 169-170, chap. : « Fede e programmi » (« La foi et les programmes. » ) de 1911. (\emph{Note de Gramsci.})}. »\par
 « Lorsqu’une œuvre poétique ou un cycle d’œuvres poétiques s’est formé, il est impossible de continuer ce cycle par l’étude, l’imitation et les variations autour de ces œuvres : par ce moyen on n’obtient que ce qu’on appelle l’école poétique, le servum pecus des épigones\footnote{\emph{Servum pecus} : expression latine du poète Horace pour désigner les imitateurs en littérature (mot à mot : le troupeau des esclaves). \emph{Epigones} : dans la mythologie grecque, les épigones étaient les fils des sept chefs qui périrent en assiégeant Thèbes. Par extension le mot désigne les successeurs, les descendants (dans un sens quelque peu ironique)}. La poésie n’engendre pas la poésie ; la parthénogénèse\footnote{\emph{Parthénogénèse : mode} de reproduction de certains animaux par des oeufs non fécondés (des mots grecs : \emph{Parthenos}, vierge, et \emph{Genesis}, engendrement).} n’a pas lieu ; il faut l’intervention de l’élément mâle, de ce qui est réel, passionnel, pratique, moral. Les plus grands critiques de la poésie conseillent, dans ces cas là, de ne pas avoir recours à des recettes littéraires, mais, comme ils disent, de « refaire l’homme ». Lorsqu’on a refait l’homme, lorsqu’on a rafraîchi l’esprit et fait naître une nouvelle vie affective, c’est d’elle que surgira, si elle surgit, une nouvelle poésie\footnote{CROCE \emph{: Cultura e vita morale}, pp. 241-242, chap. - « Troppa filosofia » (« Trop de philosophie ».) de 1922. (\emph{Note de Gramsci}.)} [4.]. »
\end{quoteblock}

\noindent Cette observation, le matérialisme historique peut la reprendre à son compte. La littérature n’engendre pas la littérature, etc., c’est-à-dire : les idéologies n’engendrent pas d’idéologies, les superstructures ne créent pas des superstructures sinon comme une descendance inerte et passive : elles sont engendrées, non par parthénogénèse, mais par l’intervention de l’élément « mâle », l’histoire, l’activité révolutionnaire qui crée l’ « homme nouveau », c’est-à-dire de nouveaux rapports sociaux.\par
On peut aussi en déduire ceci : que le vieil « homme », sous l’effet du changement, devient lui aussi « nouveau », parce qu’il entre dans une série de nouveaux rapports, une fois que les anciens rapports sont bouleversés. D'où le fait que, avant que le « nouvel homme » créé de façon positive ait donné sa poésie, on peut assister au « chant du cygne » du vieil homme rénové de façon négative : et souvent ce chant du cygne est d’une admirable splendeur ; le nouveau s’y mêle à l’ancien, les passion s’y brûlent d’un feu incomparable, etc. (\emph{La Divine Comédie} n’est-elle pas un peu le chant du cygne du moyen âge, tout en anticipant sur les temps nouveaux et sur la nouvelle histoire ?) (L.V.N., pp. 10-11.)\par
{\raggedleft \noindent [1930-1932]}
\section[{ Critères de critique littéraire}]{ Critères de critique littéraire}
\noindent L'idée que l’art est l’art, et non une propagande politique « voulue » et proposée, est-elle, en elle-même, un obstacle à la formation de courants culturels déterminés qui soient le reflet de leur époque et qui contribuent à renforcer des courants politiques déterminés ? Il ne semble pas, et il semble bien plutôt qu’une telle idée pose le problème en des termes plus radicaux et qui sont ceux d’une critique plus efficace et plus concluante. Une fois posé le principe qu’il ne faut rechercher que le caractère artistique d’une œuvre d’art, il n’est pas du tout exclu que l’on recherche quelle masse de sentiments, quelle attitude envers la vie, se dégagent de l’ œuvre elle-même. On voit même, chez De Sanctis et chez Croce, que cela est admis par les courants esthétiques modernes. Ce qui est exclu, c’est qu’une œuvre soit belle à cause de son contenu moral et politique, et non pas à cause de sa forme, dans laquelle le contenu abstrait s’est fondu, à laquelle il s’est identifié.\par
On peut aussi rechercher si une œuvre d’art n’est pas ratée parce que son auteur a été détourné par des préoccupations pratiques extérieures, c’est-à-dire postiches, sans sincérité. Il semble que ce soit là le point crucial de la polémique : X… « veut » exprimer artificiellement un certain contenu et ne crée pas une œuvre d’art. La faillite artistique de l’œuvre d’art donnée (car X… a montré qu’il était un artiste dans d’autres de ses œuvres réellement senties et vécues) démontre que ce contenu est pour X… une matière sourde et rebelle, que l’enthousiasme de X… est factice et voulu de l’extérieur, que X… n’est pas en réalité, dans ce cas précis, un artiste, mais un serviteur qui veut plaire à ses maîtres. Il y a donc deux séries de faits : les uns de caractère esthétique ou d’art pur, les autres de politique culturelle (C’est-à-dire de politique tout court). Le fait qu’on en arrive à nier le caractère artistique d’une œuvre, peut servir au critique politique pour démontrer que X…, en tant qu’artiste, n’appartient pas à un certain monde politique ; que du moment que sa personnalité est essentiellement artistique, ce monde n’exerce aucune action dans sa vie intime, la plus personnelle, que ce monde n’existe pas : X… est par conséquent un comédien de la politique, il veut faire croire qu’il est ce qu’il n’est pas, etc. Donc le critique politique dénoncera X…. non comme artiste, mais comme « opportuniste politique ».\par
Lorsque l’homme politique exerce une pression pour que l’art de son temps exprime un monde culturel donné, il s’agit d’une activité politique, non d’une critique artistique : si le monde culturel pour lequel on lutte est un fait vivant et nécessaire, son expansivité sera irrésistible et il trouvera ses artistes. Mais si, malgré la pression exercée, ce caractère irrésistible ne se voit pas, ne se manifeste pas, cela signifie qu’il s’agit d’un monde postiche et fictif, d’une élucubration livresque de gens médiocres qui se lamentent du fait que les hommes de plus grande envergure ne sont pas d’accord avec eux. La façon même de poser la question peut être un indice de la solidité d’un tel monde moral et culturel : en effet, ce qu’on appelle le « calligraphisme » n’est autre chose que la défense des petits artistes, qui affirment par opportunisme certains principes, mais se sentent incapables de les exprimer de façon artistique, c’est-à-dire par leur activité propre, et qui radotent sur une pure forme qui serait son propre contenu, etc., etc. Le principe formel de la distinction des catégories spirituelles et de leur unité de « circulation », même sous son aspect abstrait, permet de saisir la réalité effective et de critiquer le côté arbitraire et la pseudo-vie de ceux qui ne veulent pas abattre cartes sur table, ou qui sont simplement des médiocres placés par le hasard à un poste de commande. [1933]\par
Dans le numéro de mars 1933 de \emph{L'Educazione fascista}, voir l’article polémique de Argô avec Paul Nizan [« Idee d’oltre confine » (Idées d’au-delà les frontières) ! ] à propos de la conception d’une nouvelle littérature qui pourrait surgir d’un renouvellement intellectuel et moral intégral. Nizan semble bien poser le problème lorsqu’il commence par définir ce qu’est un renouvellement intégral des bases culturelles, et limite le champ de la recherche elle-même. La seule objection fondée de Argo est la suivante : l’impossibilité de sauter une étape nationale, autochtone, de la nouvelle littérature, et les dangers « cosmopolites » de la conception de Nizan. De ce point de vue, de nombreuses critiques adressées par Nizan à des groupes d’intellectuels français sont à réexaminer : \emph{N. R. F.}, le « populisme », etc., jusqu’au groupe de \emph{Monde}\footnote{Le groupe de \emph{Monde} : il s’agit de la revue fondée en 1928 par Henri Barbusse qui joua un grand rôle dans la création d’un front de lutte anti-fasciste des travailleurs et des intellectuels.],}, non parce que ses critiques ne frappent pas juste politiquement, mais précisément parce qu’il est impossible que la nouvelle littérature ne se manifeste pas « nationalement » par des combinaisons et des associations diverses, plus où moins hybrides. C'est le courant tout entier qu’il faut examiner et étudier, objectivement.\par
D'autre part, en ce qui concerne les rapports entre la littérature et la politique, il ne faut pas perdre de vue ce critère : que l’homme de lettres doit avoir des perspectives nécessairement moins précises et moins définies que l’homme politique, qu’il doit être moins « sectaire », si l’on peut dire, mais sous des formes en apparence contradictoires. Pour l’homme politique toute image « fixée » \emph{a priori} est réactionnaire -le politique considère l’ensemble du mouvement dans son devenir. L'artiste, au contraire, doit avoir des images « fixées » et coulées dans leur forme définitive. Le politique imagine l’homme tel qu’il est et, en même temps, tel qu’il devrait être pour atteindre un but déterminé ; son travail consiste précisément à amener les hommes à se mettre en mouvement, à sortir de leur être présent pour devenir capables collectivement d’atteindre le but que l’on se propose, c’est-à-dire à se « conformer » à ce but. L'artiste représente nécessairement « ce qu’il y a », à un certain moment, de personnel, de non-conformiste, etc., de façon réaliste. Aussi, du point de vue politique, l’homme politique ne sera jamais content de l’artiste, et ne pourra pas l’être : il le trouvera toujours en retard sur son époque, toujours anachronique, toujours dépassé par le mouvement réel. Si l’histoire est un processus continu de libération et d’auto-conscience, il est évident que chaque étape, du point de vue de l’histoire et dans ce cas du point de vue de la culture, sera aussitôt dépassée et n’intéressera plus. Il me semble qu’il faut tenir compte de tout cela pour apprécier les jugements de Nizan sur ces différents groupes.\par
Mais, d’un point de vue objectif, comme cela se produit encore aujourd’hui, Voltaire est actuel, pour certaines couches de la population, de même ces groupes littéraires et toutes les combinaisons qu’ils représentent peuvent être actuels, et sont même actuels : dans ce cas, « objectif » veut dire que le développement du renouveau moral et intellectuel n’est pas simultané chez toutes les couches sociales, bien loin de là : aujourd’hui encore, il n’est pas inutile de le répéter, beaucoup sont partisans de Ptolémée et non de Copernic\footnote{C'est-à-dire beaucoup sont encore partisans de la vieille représentation du monde selon le géographe grec Ptolémée, pour qui la terre était le centre de l’univers, et non de la conception scientifique moderne, née avec Copernic, qui affirma le premier que la terre n’était qu’une planète tournant autour du soleil.}. Il existe de nombreux « conformismes », de nombreuses luttes pour de nouveaux « conformismes » et des combinaisons variées entre ce qui est (et ce qui peut être envisagé sous des angles différents) et ce que l’on travaille à faire devenir (et beaucoup travaillent dans ce sens). Se placer du point de vue d’une « seule » ligne de mouvement progressif, selon laquelle toute acquisition nouvelle s’accumule et devient le point de départ de nouvelles acquisitions, est une grave erreur : non seulement les lignes sont multiples, mais on observe aussi des mouvements de recul dans la ligne la plus « progressive ». En outre, Nizan ne sait pas poser la question de ce qu’on appelle la « littérature populaire », c’est-à-dire du succès que connaît, parmi les masses populaires, la littérature des feuilletons (romans d’aventures, policiers, noirs, etc.), succès qui est aidé par le cinéma et par le journal. Et c’est pourtant cette question qui constitue la plus grande partie du problème d’une nouvelle littérature en tant qu’expression d’un renouveau intellectuel et moral : car c’est seulement à partir des lecteurs de romans-feuilletons que l’on peut sélectionner le public suffisant et nécessaire pour créer la base culturelle d’une nouvelle littérature. Il me semble que le problème doit être le suivant : comment créer un corps d’écrivains qui, du point de vue artistique, soient au roman-feuilleton ce que Dostoïevski était à Eugène Sue et Soulié ou, pour le roman policier, ce que Chesterton est à Conan Doyle ou à Wallace, etc. ? Dans cette perspective, il faut abandonner bien des idées préconçues, mais il faut tout particulièrement penser que, non seulement on ne peut monopoliser ce genre de littérature mais que l’on a contre soi la formidable organisation d’intérêts des éditeurs.\par
Le préjugé le plus répandu est que la nouvelle littérature doit s’identifier avec une école artistique d’origine intellectuelle, comme ce fut le cas pour le futurisme. Les prémisses de la nouvelle littérature doivent être nécessairement historiques, politiques, populaires ; elles doivent tendre à élaborer ce qui existe déjà, de façon polémique ou de toute autre façon, peu importe ; l’important est que cette nouvelle littérature plonge ses racines dans \emph{l’humus} de la culture populaire telle qu’elle est, avec ses goûts, ses tendances, etc., avec son monde moral et intellectuel, même s’il est arriéré et conventionnel. (L.V.N., pp.11-14.)\par
{\raggedleft \noindent [1933]}
\section[{ Critères de méthode}]{ Critères de méthode}
\noindent Il serait absurde de prétendre que chaque année ou même tous les dix ans la littérature d’un pays puisse produire un \emph{Promessi Sposi}\footnote{\emph{I Promessi Sposi} (\emph{LesFiancés}) : le célèbre roman historique d’AlexandreMANZONI.} ou un \emph{Sepolcri}\footnote{\emph{I Sepolcri} (\emph{Les Tombeaux}) : oeuvre du poète romantique italien Ugo FOSCOLO, écrite en 1807. Ce poème sur le thème de l’immortalité de la pensée humaine est l’une des plus grandes œuvres lyriques de la poésie italienne}, etc. Justement pour cette raison l’activité critique normale est obligée d’avoir principalement un caractère « culturel », et être une critique des « tendances », sous peine de devenir un massacre continuel (et, dans ce cas, comment choisir l’œuvre à massacrer, l’écrivain à rejeter hors de l’art ? C'est un problème qui paraît négligeable, mais qui, si l’on y réfléchit du point de vue de l’organisation moderne de la vie culturelle, est fondamental.) Une activité critique qui serait constamment négative, faite d’éreintements, de démonstrations qu’il s’agit de « non-poésie » et non de « poésie » \footnote{« Poésie » et « non poésie » : formule de Croce (c’est aussi le titre d’un de ses livres), qui distingue dans une œuvre littéraire ce qui appartient à l’art (poésie) et ce qui lui est étranger (non-poésie).} serait ennuyeuse et révoltante : le « choix » semblerait être une chasse à l’homme, ou bien on pourrait le considérer comme fortuit et par conséquent sans importance.\par
 Il semble certain que l’activité de la critique doive toujours avoir un aspect positif, en ce sens qu’elle doit mettre en relief, dans l’œuvre examinée, une valeur positive : si celle-ci ne peut être d’ordre artistique, elle peut être d’ordre culturel et alors chaque livre pris en particulier, sauf cas exceptionnel, ne présentera pas autant d’intérêt que les groupes de travaux groupés en séries selon leur tendance culturelle. A propos du choix : le critère le plus simple, en dehors de l’intuition du critique et de l’examen systématique de toute la littérature, travail colossal et presque impossible à faire individuellement, paraît être celui du « succès de librairie », dans les deux sens de « succès auprès des lecteurs » et de « succès auprès des éditeurs », ce qui, dans certains pays où la vie intellectuelle est contrôlée par des organes gouvernementaux, a aussi son sens, car il indique quelle orientation l’Etat voudrait donner à la culture nationale. (L.V.N., pp. 19-20.)\par
{\raggedleft \noindent [1934]}
\section[{ Divers types de romans populaires}]{ Divers types de romans populaires}
\noindent Il y a une grande variété de genres de roman populaire et il faut remarquer que si tous ces types connaissent en même temps une certaine diffusion et un certain succès, l’un d’eux cependant l’emporte sur les autres, et de loin. De cette prédominance on peut déduire un changement des goûts fondamentaux, de même que, du caractère simultané du succès des divers types de romans on peut tirer la preuve qu’il existe dans le peuple différentes couches culturelles, divers « ensembles de sentiments » qui dominent dans chaque couche, divers « modèles de héros » populaires. Dresser le catalogue de ces types et établir historiquement leur plus ou moins grande fortune relative, est donc important pour le but que se propose cet essai :\par

\begin{enumerate}[itemsep=0pt,]
\item type Victor Hugo, Eugène Sue \emph{(Les Misérables, les Mystères de Paris)} à caractère nettement idéologique-politique, de tendance démocratique liée à l’idéologie de 1848 ; 
\item type sentimental, qui n’est pas politique au sens étroit du mot, mais où s’exprime ce que l’on pourrait définir « une démocratie sentimentale » (Richebourg, Decourcelle, etc.) ; 
\item le type qui se présente comme étant de pure intrigue, mais qui a un contenu idéologique conservateur-réactionnaire (Montépin) ; 
\item le roman historique d’Alexandre Dumas et de Ponson du Terrail qui, outre son caractère historique, a un caractère idéologique-politique, mais moins marqué : Ponson du Terrail cependant est conservateur réactionnaire, et l’exaltation des aristocrates et de leurs fidèles serviteurs a un caractère bien différent des représentations historiques d’Alexandre Dumas, qui n’a pourtant pas une tendance politique démocratique nette, mais qui est plutôt imprégné de sentiments démocratiques génériques et « passif s », et se rapproche souvent du type « sentimental » ;
\item le roman policier sous son double aspect (Lecocq, Rocambole, Sherlock Holmes, Arsène Lupin) ;
\item le roman ténébreux (fantômes, châteaux mystérieux, etc. : Anna Radcliffe) ;
\item le roman scientifique d’aventures, géographique, qui peut avoir une tendance ou n’être qu’un roman d’intrigue (Jules Verne, Boussenard).
\end{enumerate}

\noindent De plus chacun de ces types présente divers aspects nationaux (en Amérique, le roman d’aventures est l’épopée des pionniers, etc.). On peut observer comment, dans la production d’ensemble de chaque pays, il y a, implicitement, un sentiment nationaliste, qui ne s’exprime pas de façon rhétorique, mais qui s’insinue habilement dans le récit. Chez Verne et chez les Français, le sentiment anti-anglais, lié à la perte des colonies et à l’irritation causée par les défaites maritimes, est très vif : dans le roman géographique d’aventures les Français ne se heurtent pas aux Allemands, mais aux Anglais. Mais le sentiment anti-anglais est vif également dans le roman historique et jusque dans le roman sentimental (par exemple George Sand ; réaction due à la guerre de Cent Ans et à l’assassinat de Jeanne d’Arc, et aussi à la fin de Napoléon).\par
En Italie, aucun de ces types de roman n’a eu d’écrivains (nombreux) de quelque relief (je ne parle pas de valeur littéraire, mais de valeur « commerciale », d’invention, de construction ingénieuse d’intrigues, à grand effet, certes, mais élaborées de façon assez rationnelle). Même le roman policier, qui a eu un tel succès international (et financier pour les auteurs et éditeurs) n’a pas eu d’écrivains en Italie ; et pourtant de nombreux romans, surtout historiques, ont pris pour sujet l’Italie et les événements historiques de ses villes, de ses régions, ses institutions et ses hommes. Ainsi l’histoire vénitienne, avec ses organisations politiques, judiciaires, policières, a fourni et continue à fournir des sujets aux romanciers populaires de tous les pays, sauf à l’Italie. La littérature populaire sur la vie des brigands a connu en Italie un certain succès, mais c’est une production de très basse valeur.\par
Le dernier, le plus récent type de livre populaire est la vie romancée qui représente de toute façon une tentative inconsciente pour satisfaire aux exigences culturelles de certaines couches populaires plus évoluées au point de vue culturel, qui ne se contentent pas de l’histoire du type Dumas. Même cette littérature n’a pas en Italie de nombreux représentants (Mazzucchelli, Cesare Giardini, etc.) : non seulement les écrivains italiens ne sont pas comparables parle nombre, la fécondité, le talent et le charme littéraire aux écrivains français, allemands, anglais, mais, chose plus significative, ils choisissent leurs sujets hors d’Italie (Mazzucchelli et Giardini en France, Eucardio Momigliano en Angleterre), pour s’adapter au goût populaire italien qui s’est formé d’après les romans historiques, plus spécialement français. L'homme de lettres italien n’écrirait pas une biographie romancée de Masaniello, de Michele de Lando, de Cola di Rienzo\footnote{\emph{Masaniello, Michele de Lando, Cola di Rienzo} : héros populaires de l’histoire italienne. Tommaso Aniello, dit Masaniello (1623-1647) fut un pêcheur napolitain qui dirigea l’insurrection du peuple de Naples contre le vice-roi espagnol en 1647. Il fut assassiné peu après. Michele de Lando fut le chef de l’insurrection des ouvriers cardeurs de Florence ; nommé gonfalonier en 1378, il gouverna la ville jusqu’à 1381, puis fut exilé. Cola di Rienzo (1313-1354), homme du peuple qui fut nommé tribun du peuple à Rome en 1347 ; il voulait fonder une république des Etats italiens qui rendrait à Rome sa grandeur passée ; mais, devenu impopulaire par son attitude dictatoriale, il fut assassiné au cours d’une émeute.}, sans se croire obligé de les bourrer, pour les « soutenir », d’ennuyeuses tirades de rhétorique, pour qu’on ne croie pas…, pour qu’on ne pense pas…. etc. Il est vrai que le succès remporté par les vies romancées a amené de nombreux éditeurs à commencer la publication de collections biographiques, mais il s’agit de livres qui sont par rapport à la vie romancée ce que \emph{la Religieuse de Monza}\footnote{« La Religieuse de Monza » : long épisode du célèbre roman d’Alexandre MANZONI\emph{ : Les Fiançés}, qui constitue comme un roman dans le roman lui-même. C'est l’histoire tragique de la supérieure du couvent de Monza, Gertrude, que ses parents avaient fait entrer contre son gré dans les ordres pour conserver à leur fils aîné le patrimoine de la famille. Gertrude, cloîtrée sans vocation et désespérée, se laissa aller à une liaison amoureuse qui l’entraîna jusqu’au crime.} est au \emph{Comte de Monte-Cristo ;} il s’agit du thème biographique habituel, souvent philologiquement correct, qui peut trouver au maximum quelques milliers de lecteurs, mais qui ne devient pas populaire.\par
Il faut remarquer que certains types de roman populaire indiqués plus haut ont leur correspondance dans le théâtre et aujourd’hui dans le cinéma. Au théâtre, la fortune considérable de Dario Niccodemi est certainement due au fait qu’il a su dramatiser certains traits et certains motifs éminemment liés à l’idéologie populaire ; ainsi dans \emph{Scampolo}, dans \emph{L'Aigrette, La Volata}, etc. Chez Gioacchino Forzano aussi on trouve quelque chose de ce genre, mais sur le modèle de Ponson du Terrail, avec des tendances conservatrices. L'œuvre de théâtre qui a eu en Italie le plus grand succès populaire est \emph{La Morte civile} de Giacometti, de caractère italien ; il n’a pas eu d’imitateur de valeur (toujours dans un sens non littéraire). Dans ce domaine du théâtre, on peut remarquer que toute une série d’auteurs dramatiques de grande valeur littéraire peuvent plaire beaucoup, même au public populaire : \emph{Maison de poupée} d’Ibsen est très bien accueillie par le peuple des villes, dans la mesure où les sentiments représentés et la tendance morale de l’auteur trouvent une résonance profonde dans la psychologie populaire. Et d’ailleurs que devrait être le fameux \emph{théâtre d’idées} sinon cela : la représentation de passions liées aux murs avec des solutions dramatiques représentant une \emph{catharsis} « progressive », représentant le drame de la partie la plus avancée intellectuellement et moralement d’une société, et exprimant le développement historique immanent dans les murs elles-mêmes, telles qu’elles sont ? Ces passions, ce drame doivent cependant être représentés et non développés comme une thèse, comme un discours de propagande, c’est-à-dire que l’auteur doit les vivre, dans le monde réel avec toutes leurs exigences contradictoires, et ne pas exprimer des sentiments uniquement tirés des livres. (L.V.N., pp. 110-113.)\par
{\raggedleft \noindent [1934-1935]}
\section[{ Jules Verne et le roman géographique-scientifique}]{ Jules Verne et le roman géographique-scientifique}
\noindent Dans les livres de Jules Verne, il n’y a jamais rien qui soit tout à fait impossible : les « possibilités » dont disposent les héros de Jules Verne sont supérieures à celles qui existent réellement à l’époque mais elles ne sont pas trop supérieures et surtout elles ne sont pas « en dehors » de la ligne de développement des conquêtes scientifiques obtenues ; son imagination n’est pas du tout « arbitraire » et a pour cela la faculté d’exciter l’imagination du lecteur déjà acquis à l’idée du développement fatal du progrès scientifique dans le domaine du contrôle des forces de la nature.\par
Le cas de Wells et de Poe est différent, car chez eux « l’arbitraire » domine en grande partie, même si le point de départ peut être logique et découler d’une réalité scientifique concrète : il y a chez Verne l’alliance ! de l’intelligence humaine et des forces matérielles ; chez Wells et chez Poe c’est l’intelligence humaine qui prédomine et c’est pourquoi Verne a été plus populaire, parce que plus compréhensible. Mais en même temps cet équilibre des constructions romanesques de Verne est devenu une limite, dans le temps, à sa popularité (mise à part sa maigre valeur artistique) : la science a dépassé Verne et ses livres ne sont plus des « excitants psychiques ».\par
On peut dire quelque chose de semblable des aventures policières, par exemple de Conan Doyle ; pour son époque elles étaient excitantes, aujourd’hui elles ne le sont presque plus et pour diverses raisons : parce que le monde des luttes policières est aujourd’hui mieux connu, alors que Conan Doyle, en grande partie, le révélait, du moins à un grand nombre de lecteurs pacifiques. Mais surtout parce que, en Sherlock Holmes il y a un équilibre rationnel (trop) entre l’intelligence et la science. Aujourd’hui on est plus intéressé par l’apport individuel du héros, par la technique « psychique » en soi, aussi Poe et Chesterton sont plus intéressants, etc.\par
Dans la revue \emph{Marzocco} du 19 février 1928, Adolfo Faggi (« Impressions sur Jules Verne » ) écrit que le caractère anti-anglais de nombreux romans de Jules Verne doit être situé dans cette période de rivalité entre la France et l’Angleterre qui atteignit son point culminant dans l’épisode de Fachoda. L'affirmation est erronée et entachée d’anachronisme : l’antibritannisme était (et il l’est peut-être encore) un élément fondamental de la psychologie populaire française ; le sentiment anti-allemand est relativement récent, moins enraciné que le sentiment anti-anglais, il n’existait pas avant la Révolution française et s’est développé après 1870, après la défaite et l’impression douloureuse que la France n’était plus la forte nation militaire et politique de l’Europe occidentale, car l’Allemagne, seule, sans coalition, avait vaincu la France. Le sentiment anti-anglais remonte à la formation de la France moderne, comme Etat unitaire et moderne c’est-à-dire à la guerre de Cent Ans, et aux échos dans l’imagination populaire de l’épopée de Jeanne d’Arc ; il a été renforcé dans les temps modernes par les guerres pour l’hégémonie sur le continent (et dans le monde) qui a atteint son maximum avec la Révolution française et avec Napoléon : l’épisode de Fachoda, malgré toute sa gravité, ne peut être comparé à cette imposante tradition dont témoigne toute la littérature française populaire. (L.V.N., pp. 114-115.)\par
{\raggedleft \noindent [1934-1935]\par}
\noindent  
\section[{Sur le roman policier}]{Sur le roman policier}
\noindent Le roman policier est né aux confins de la littérature sur les « causes célèbres ». C'est à celle-ci, d’ailleurs, que se rattache également le roman du type \emph{Comte de MonteCristo ;} ne s’agit-il pas, ici aussi, de « causes célèbres » romancées, colorées de toute l’idéologie populaire concernant l’administration de la justice, surtout si la passion politique s’y mêle ? Rodin, dans \emph{Le Juif errant}, n’est-il pas un type d’organisateur d’ « intrigues scélérates », que n’arrête aucun crime, aucun assassinat, et au contraire le prince Rodolphe\footnote{L'un des principaux personnages du roman-feuilleton d’Eugène Sue \emph{: Les Mystères de Paris.}} n’est-il pas « l’ami du peuple », qui déjoue les intrigues et les crimes ? Le passage de ce type de roman aux romans de pure aventure est marqué par un processus de schématisation de la pure intrigue, dépouillée de tout élément d’idéologie démocratique et petite-bourgeoise : ce n’est plus la lutte entre le peuple bon, simple et généreux, et les forces obscures de la tyrannie (Jésuites, police secrète liée à la raison d’Etat ou à l’ambition de certains princes, etc.), mais seulement la lutte entre la délinquance professionnelle ou spécialisée et les forces de l’ordre légal, privées ou publiques, sur la base de la loi écrite.\par
La collection des « causes célèbres », dans la célèbre collection française, a eu son équivalent dans les autres pays : elle a été traduite en italien, au moins en partie, pour les procès de renommée européenne, comme celui de Fualdès, pour l’assassinat du courrier de Lyon, etc.\par
L'activité « judiciaire » a toujours suscité de l’intérêt et continue à le faire ; l’attitude du sentiment public envers l’appareil de la justice (toujours discrédité, d’où le succès du policier privé ou amateur) et envers le délinquant a souvent changé ou du moins a pris diverses colorations. Le grand criminel a souvent été représenté comme supérieur à l’appareil de la justice, exactement comme le représentant de la « vraie » justice : influence du romantisme, \emph{Les Brigands}\footnote{\emph{Les Brigands} (1782) : drame du poète romantique allemand Schiller, œuvre pessimiste et satire de la société.} de Schiller ; les contes d’Hoffmann, Anna Radcliffe, le « Vautrin » de Balzac.\par
Le personnage de Javert des \emph{Misérables} est intéressant du point de vue de la psychologie populaire : Javert a tort du point de vue de la « vraie justice », mais Hugo l’a représenté de façon sympathique, comme un « homme de caractère », attaché à son devoir « abstrait », etc. ; c’est de Javert que naît peut-être une tradition selon laquelle même le policier peut être « respectable ».\par
\emph{Rocambole} de Ponson du Terrail. Gaboriau continue la réhabilitation du policier, avec \emph{Monsieur Lecocq} qui ouvre la voie à Sherlock Holmes. Il n’est pas vrai que les Anglais, dans le roman « judiciaire » représentent la « défense de la loi », alors que les Français représentent l’exaltation du délinquant. Il s’agit d’un passage « culturel » dû au fait que cette littérature se répand aussi dans certaines couches cultivées. Se rappeler qu’Eugène Sue, très lu par les démocrates des classes moyennes, a imaginé tout un système de répression de la délinquance professionnelle.\par
Dans cette littérature policière il y a toujours eu deux courants l’un mécanique, basé sur l’intrigue, l’autre artistique ; Chesterton est aujourd’hui le plus grand représentant de cet aspect « artistique » comme le fut en son temps Poe : Balzac, avec Vautrin, s’occupe du délinquant, mais il n’est pas, « techniquement » parlant, un écrivain de romans policiers.\par
{\raggedleft \noindent [1934-1935]\par}
\noindent Voir le livre de Henri Jagot : \emph{Vidocq}, éd. Berger-Levrault, Paris, 1932. Vidocq a servi de point de départ au Vautrin de Balzac et à Alexandre Dumas (on le retrouve aussi un peu dans le « Jean Valjean » de Victor Hugo et surtout dans \emph{Rocambole}). Vidocq fut condamné à huit ans de prison comme faux-monnayeur, à cause d’une imprudence ; vingt évasions, etc. En 1812, il entra dans la police de Napoléon et commanda pendant quinze ans une équipe d’agents créée exprès pour lui : il devint célèbre pour ses arrestations sensationnelles. Congédié par Louis-Philippe, il fonda une agence privée de \emph{détectives}, mais avec peu de succès : il ne pouvait opérer que dans les rangs de la police d’Etat. Mort en 1857. Il a laissé ses \emph{Mémoires}, qui n’ont pas été écrits par lui seul et qui contiennent beaucoup d’exagérations et de vantardises.\par
Voir l’article de Aldo Sorani : « Conan Doyle et le succès du roman policier » dans \emph{Pegaso} d’août 1930 : important pour l’analyse de ce genre de littérature et pour les diverses formes spécifiques qu’il a pris jusqu’ici. En parlant de Chesterton et de la série de nouvelles sur le père Brown\footnote{\emph{Le père Brown} : prêtre-détective, personnage de plusieurs romans policiers de l’écrivain anglais Chesterton, dont le modèle lui fut fourni par le Père O'Connor.}, Sorani ne tient pas compte de deux éléments culturels qui paraissent par contre essentiels : \emph{a)} il ne fait pas allusion à l’atmosphère caricaturale qui se manifeste spécialement dans le volume \emph{L'Innocence du père Brown} et qui constitue même l’élément artistique qui élève la nouvelle policière de Chesterton lorsque, ce qui n’est pas toujours le cas, l’expression en est parfaite ; \emph{b)} il ne signale pas le fait que les nouvelles du père Brown sont des « apologies » du catholicisme et du clergé romain, entraîné à connaître tous les replis de l’âme humaine par l’exercice de la confession et par sa fonction de guide spirituel et d’intermédiaire entre l’homme et la divinité contre le « scientisme » et la psychologie positive du protestant Conan Doyle. Sorani, dans son article, se réfère à diverses tentatives, particulièrement anglo-saxonnes, et de plus grande importance littéraire, pour perfectionner du point de vue technique le roman policier. L'archétype est Sherlock Holmes, dans ses deux caractères fondamentaux : de scientifique et de psychologue : il s’agit de perfectionner l’un oul’autre de ces traits caractéristiques, ou les deux ensemble. Chesterton a justement insisté sur l’élément psychologique, dans le jeu des inductions et des déductions du père Brown, mais semble avoir encore exagéré dans ce sens avec le personnage du poète-policier Gabriel Gale.\par
Sorani esquisse un tableau du succès inouï remporté par le roman policier dans tous les ordres de la société et cherche à en trouver l’origine psychologique : ce serait une manifestation de révolte contre le caractère mécanique et la standardisation de la vie moderne, une façon de s’évader de la grisaille de la vie quotidienne. Mais cette explication peut s’appliquer à toutes les formes de littérature, populaire ou artistique : depuis le poème chevaleresque (Don Quichotte ne cherche-t-il pas à s’évader lui aussi, et même pratiquement de la vie quotidienne grise et standardisée d’un village espagnol ?) jusqu’au roman-feuilleton de tous genres. L'ensemble de la littérature et de la poésie ne serait donc qu’unstupéfiant contre la banalité quotidienne ? De toute façon, l’article de Sorani est indispensable pour une future recherche plus organique sur ce genre de littérature populaire.\par
Le problème : Pourquoi la littérature policière est-elle répandue ? est un aspect particulier du problème plus général : Pourquoi la littérature non artistique est-elle répandue ? Sans aucun doute pour des raisons pratiques et culturelles (politiques et morales) : et cette réponse générale est la plus précise, dans ses limites approximatives. Mais la littérature artistique elle-même ne se répand-elle pas elle aussi pour des raisons pratiques ou politiques et morales, et seulement par l’intermédiaire de raisons de goût artistique, pour rechercher la beauté et pour en jouir ? En réalité on lit un livre poussé par des raisons pratiques (et il faut rechercher pourquoi certains élans se généralisent plus que d’autres) et on le relit pour des raisons artistiques. L'émotion esthétique ne naît presque jamais à la première lecture. Cela se manifeste encore davantage au théâtre, où l’émotion esthétique représente un « pourcentage » très réduit de l’intérêt du spectateur ; car à la scène d’autres éléments jouent, et nombre d’entre eux ne sont même pas d’ordre intellectuel, mais d’ordre purement physiologique, comme le \emph{sex appeal}, etc. Dans d’autres cas l’émotion esthétique au théâtre ne tire pas son origine de l’œuvre littéraire, mais de son interprétation par les acteurs et le metteur en scène : mais dans ce cas, il faut que le texte littéraire du drame qui fournit le prétexte à l’interprétation ne soit pas « difficile » et d’une psychologie recherchée, mais qu’il soit « élémentaire et populaire » dans ce sens que les passions représentées doivent être, le plus possible, profondément « humaines » et d’une expérience immédiate (vengeance, honneur, amour maternel, etc.) et par conséquent l’analyse se complique aussi dans ces cas-là.\par
Les grands acteurs traditionnels étaient acclamés dans \emph{La Mort civile}, dans \emph{Les Deux Orphelines}, dans \emph{Les Crochets du Père Martin}\footnote{ \noindent \emph{La Mort civile} : drame de Paolo Giacometti représenté pour la première fois en 1861, et considéré comme le chef-d’œuvre de cet auteur. Giacometti déclare avoir voulu poser le problème du caractère indissoluble du mariage dans le cas de la « mort civile » d’un des époux. Pièce pathétique, pleine de déclamation et de sentimentalité, mais qui portait sur la scène un des problèmes de la vie publique de la nouvelle société italienne née avec l’unité du pays.\par
 \emph{Les Deux Orphelines} : mélodrame en 5 actes et 8 tableaux d’Ennery et Cormon (1874), qui connut un gros succès populaire dans plusieurs pays. Située sous Louis XV, c’est l’histoire émouvante de deux orphelines qui, après des péripéties dramatiques, sont enfin réunies et trouvent le bonheur.\par
 \emph{Les Crochets du Père Martin} : comédie d’Eugène Cormon et Eugène Grange (1858), pièce aux caractères superficiels, mais qui connut un grand succès populaire à cause de la moralité de l’intrigue.
}, etc., plus que dans les intrigues à complications psychologiques : dans le premier cas les applaudissements étaient sans réserve ; dans le second cas, ils étaient plus froids, destinés à isoler l’acteur aimé du public de l’œuvre représentée, etc.\par
Une justification du succès des romans populaires semblable à celle que donne Sorani se trouve dans un article de Filippo Burzio sur \emph{Les Trois Mousquetaires} d’Alexandre Dumas (publié dans \emph{La Stampa} du 22 octobre 1930 et reproduit en extraits par \emph{L'Italia letteraria} du 9 novembre). Burzio considère \emph{Les Trois Mousquetaires} comme une excellente personnification, comme \emph{Don Quichotte} et le \emph{Roland furieux}\footnote{\emph{Orlando furioso (Roland furieux)} : poème chevaleresque d’aventures merveilleuses, écrit par l’Arioste. L'une des grandes œuvres de la littérature italienne.}, du mythe de l’aventure, c’est-à-dire de quelque chose d’essentiel à la nature humaine, qui semble gravement et progressivement s’éloigner de la vie moderne. Plus l’existence se fait rationnelle [ou rationalisée, plutôt, par contrainte, car si elle est rationnelle pour les groupes dominants, elle ne l’est pas pour les groupes dominés ; et elle est liée à l’activité économique-pratique, par laquelle la contrainte s’exerce, fut-ce même de façon indirecte, sur les couches « intellectuelles » elles-mêmes] et organisée, plus la discipline sociale devient rigoureuse et la tâche assignée à l’individu précise et prévisible [mais imprévisible pour les dirigeants, comme le manifestent les crises et les catastrophes historiques], et plus la marge d’aventure se trouve réduite, comme la libre forêt, qui appartient à tous, est réduite entre les murs étouffants de la propriété privée… Le taylorisme est une belle chose et l’homme est un animal qui peut s’adapter, mais il y a pourtant des limites à sa mécanisation. Si on me demandait quelles sont les raisons profondes de l’inquiétude occidentale, je répondrais sans hésiter : la décadence de la foi (!) et l’humiliation du désir d’aventure. Qui l’emportera, du taylorisme ou des \emph{Trois Mousquetaires} ? C'est une autre question, et quant à la réponse qui, il y a trente ans, semblait certaine, il vaut mieux la laisser en suspens. Si la civilisation actuelle ne sombre pas, nous assisterons peut-être à un intéressant mélange des deux. »\par
Le problème est celui-ci : Burzio ne tient pas compte du fait qu’il y a toujours eu une grande partie de l’humanité dont l’activité a toujours été taylorisée et soumise à une discipline de fer, et qu’elle a cherché à s’évader hors des limites étroites de l’organisation existante qui l’écrasait par l’imagination et par le rêve. La plus grande aventure, la plus grande « utopie » que l’humanité a créée collectivement, la religion, n’est-elle pas une façon de s’évader du « monde terrestre » ? Et n’est-ce pas dans ce sens que Balzac parle de la loterie comme d’un opium pour la misère, phrase qui a été reprise ensuite par d’autres\footnote{Dans une autre note \emph{(Note sul Machiavelli}, Einaudi, p. 288) Gramsci cite le passage de \emph{La Rabouilleuse} où Balzac parle de la loterie comme de « l’opium de la misère », et poursuit : « Ce passage de Balzac pourrait aussi être rattaché à l’expression « opium du peuple » employée dans l’introduction à la \emph{Contribution à la critique de la Philosophie du Droit de Hegel}, publiée en 1844, dont l’auteur (Karl Marx) fut un grand admirateur de Balzac. » Gramsci ajoute que le passage de l’expression « opium de la misère » employée par Balzac pour la loterie, à l’expression « opium du peuple » employée pour la religion, a été probablement influencé par la réflexion sur le « pari » de Pascal.} ? Mais le plus important est qu’à côté de Don Quichotte existe Sancho Pança, qui ne veut pas « d’aventures », mais une vie assurée, et que dans leur grande majorité les hommes sont tourmentés précisément par « l’impossibilité de prévoir le lendemain », par le caractère précaire de leur propre vie quotidienne, c’est-à-dire par un excès d’ « aventures » probables.\par
Dans le monde moderne le problème prend un aspect différent de celui qu’il avait dans le passé, parce que la rationalisation coercitive de l’existence frappe toujours plus les classes moyennes et les intellectuels, et d’une façon inouïe ; mais pour elles aussi il ne s’agit pas d’une décadence de l’aventure, mais bien du caractère trop aventureux de la vie quotidienne, c’est-à-dire du caractère trop précaire de l’existence, joint à la conviction qu’il n’y a aucun moyen individuel d’endiguer cette précarité de l’existence : aussi les gens aspirent-ils à l’aventure « belle » et intéressante, parce qu’elle est due à leur propre initiative contre la « laide », la révoltante aventure qui est due aux conditions que d’autres ne leur proposent pas, mais leur imposent.\par
La justification de Sorani et de Burzio sert aussi à expliquer la passion du sportif, c’est-à-dire que, expliquant trop de choses, elle n’explique rien. Le phénomène est vieux au moins comme la religion, et il est à plusieurs aspects et non unilatéral : il a même un aspect positif, c’est-à-dire le désir de « s’éduquer » par la connaissance d’un mode de vie que l’on considère comme supérieur au sien, le désir d’élever sa propre personnalité en se proposant un modèle idéal, le désir de connaître le monde et les hommes plus qu’il n’est possible dans certaines conditions de vie, le snobisme, etc., etc. (L.V.N., pp. 115-119.)\par
{\raggedleft \noindent [1934-1935]}
\section[{Dérivations culturelles du roman-feuilleton}]{Dérivations culturelles du roman-feuilleton}
\noindent Voir le fascicule de \emph{La Cultura} consacré à Dostoïevski en 1931. Vladimir Pozner soutient justement dans un article que les romans de Dostoïevski, du point de vue culturel, sont dérivés des romans-feuilletons genre Eugène Sue, etc. Il ! est utile de ne pas oublier cette dérivation pour le développement ultérieur de cette rubrique sur la littérature populaire, dans la mesure où elle montre comment certains courants culturels (motifs et intérêts moraux, sensibilité, idéologie, etc.) peuvent s’exprimer de deux façons : à la façon purement mécanique d’une intrigue sensationnelle (Sue, etc.) et à la façon « lyrique » (Balzac, Dostoïevski, et, dans une certaine mesure Victor Hugo). Les contemporains ne s’aperçoivent pas toujours de la dégradation d’une partie de ces manifestations littéraires, comme cela s’est produit partiellement pour Eugène Sue, qui a été lu par tous les groupes sociaux et qui « émouvait » même les gens « cultivés », tandis qu’il tomba ensuite au rang « d’écrivain lu seulement par le peuple. » (la « première lecture » donne simplement, ou presque, des sensations, qu’elles soient « culturelles » ou de contenu, et le « peuple » est un lecteur de première lecture, sans attitude critique, chez qui l’émotion naît de la sympathie qu’il éprouve pour l’idéologie générale dont le livre est l’expression souvent artificielle et voulue).\par
Sur ce même sujet, voir :\par

\begin{enumerate}[itemsep=0pt,]
\item Mario Praz : \emph{La Carne, lamorte e il diavolo nella letteratura romantica}, in-16, pp. X-505, Milan-Rome, éd. La Cultura (voir le compte rendu de Luigi Foscolo Benedetto dans \emph{Leonardo} de mars 1931 : il en ressort que Praz n’a pas distingué avec exactitude les différents degrés de « culture », d’où certaines objections de Benedetto, qui d’ailleurs ne paraît pas avoir saisi lui-même le lien historique du problème historico-littéraire) ;
\item Servais Etienne : \emph{Le Genre romanesque en France depuis l’apparition de la « Nouvelle Héloïse » jusqu’aux approches de la Révolution}, éd. Armand Colin ;
\item Alice Killen : \emph{Le roman terrifiant ou « roman noir » de Walpole à Anne Radcliffe, et son influence sur la littérature française jusqu’en 1840}, éd. Champion ; et Reginald W. Hartland (chez le même éditeur : \emph{Walter Scott et le roman « frénétique »}). 
\end{enumerate}

\noindent L'affirmation de Pozner, que le roman de Dostoïevski serait un « roman d’aventure » est probablement dérivée d’un essai de Jacques Rivière sur le « roman d’aventure » (peut-être publié à la N.R.F.) dont le sens serait : « une vaste représentation d’actions qui sont en même temps dramatiques et psychologiques », ainsi que l’ont entendu Balzac, Dostoïevski, Dickens et George Elio\footnote{L'étude de Vladimir Pozner sur « Dostoïevski et le roman d’aventures », écrit pour le numéro spécial de \emph{Cultura} a été repris en France par la revue \emph{Europe} (sept.-oct. 1931). Vladimir Pozner nous a précisé que, contrairement à ce que pense Gramsci, il n’avait jamais eu connaissance de cet essai de Jacques Rivière.}.\par
{\raggedleft \noindent [1934-1935]\par}
\noindent A approcher de ceci un article d’André Moufflet : « Le Style du roman-feuilleton » dans le \emph{Mercure de France} du 1er février 1931. Le roman-feuilleton - selon Moufflet - est né du besoin d’\emph{illusion}, qu’éprouvaient une infinité de petites existences, et qu’elles éprouvent sans doute encore, comme pour rompre la misérable monotonie à laquelle elles se voient condamnées. Observation générale : elle peut se faire pour tous les romans, et pas seulement pour les romans-feuilletons : il faut analyser quelle \emph{illusion particulière} le roman-feuilleton donne au peuple, et comment cette illusion change selon les périodes historiques-politiques : il y a le snobisme, mais il y a un fond d’aspirations démocratiques qui se reflètent dans le roman-feuilleton classique. Roman « ténébreux » à la Radcliffe, roman d’intrigue, d’aventures, policier à caractère scandaleux, de la pègre, etc. Le snob se reconnaît dans le roman-feuilleton qui décrit la vie des nobles ou, de façon générale, des hautes classes de la société, mais cela plaît aux femmes et particulièrement aux jeunes filles et chacune d’elles, d’ailleurs, pense que la beauté peut la faire entrer dans ces classes supérieures.\par
Pour Moufflet, il y a des « classiques » du roman feuilleton ; mais il entend cela dans un certain sens : il semble que le roman-feuilleton « classique » soit celui du genre « démocratique » avec diverses nuances de Victor Hugo à Eugène Sue, à Alexandre Dumas. L'article de Moufflet est à lire, mais sans perdre de vue qu’il examine le roman-feuilleton comme « genre littéraire », pour le style, etc., comme expression d’une « esthétique-populaire », ce qui est faux. Le peuple est « contenuiste », mais si le contenu est exprimé par de grands artistes, ce sont ceux-là qu’il préfère. Rappeler ce que j’ai écrit de l’amour du peuple pour Shakespeare, pour les classiques grecs et, dans la période moderne, pour les grands romanciers russes (Tolstoï, Dostoïevski). De même, en musique, Verdi.\par
{\raggedleft \noindent [1933-1934]\par}
\noindent Dans l’article « Le mercantilisme littéraire » de J.-H. Rosny aîné, dans \emph{Les Nouvelles littéraires} du 4 octobre 1930, il est dit que Victor Hugo a écrit \emph{Les Misérables} en s’inspirant des \emph{Mystères de Paris} d’Eugène Sue et du succès de ce dernier livre, succès si grand que, quarante ans après, son éditeur Lacroix en était encore stupéfait. Rosny écrit :\par

\begin{quoteblock}
 \noindent « Les feuilletons, soit dans l’intention du directeur de journal, soit dans celle de leur auteur, furent des productions inspirées par le goût du public et non par le goût des auteurs. »
\end{quoteblock}

\noindent Cette définition est, elle aussi, unilatérale. En effet Rosny n’écrit qu’une série d’observations sur la littérature « commerciale » en général (et aussi, par conséquent, sur la littérature pornographique) et sur l’aspect commercial de la littérature. Si le « commerce » et un certain « goût » du public se rencontrent, ce n’est pas le fait du hasard, tant il est vrai que les feuilletons écrits aux environs de 1848 avaient une orientation politico-sociale déterminée, qui aujourd’hui encore les fait rechercher et lire par un public qui est animé par ces mêmes sentiments de 1848 [A propos de Victor Hugo, se rappeler ses bons rapports avec Louis-Philippe et, par la suite, son attitude de monarchiste constitutionnel en 48. Il est intéressant de remarquer que, tandis qu’il écrivait \emph{Les Misérables}, il écrivait aussi les notes de \emph{Choses vues} (parues après sa mort), et que les deux façons d’écrire ne sont pas toujours en accord. Etudier ces questions, car d’habitude on considère Hugo comme un homme d’un seul bloc, etc. (dans \emph{la Revue des Deux-Mondes} de 1928 ou 1929, plus probablement de 1929, il doit y avoir un article là-dessus\footnote{Il s’agit en effet de l’article d’André LeBreton : « Victor Hugo académicien », paru dans \emph{la Revue des Deux-Mondes} du 15 février 1929. (\emph{Note de Gramsci}.) [1930]}) (L.V.N., pp. 119-121.)\par
{\raggedleft \noindent [1933-1934]}
\section[{ Origines populaires du « surhomme »}]{ Origines populaires du « surhomme » }
\noindent Chaque fois que l’on tombe sur quelque admirateur de Nietzsche il est opportun de se demander et de rechercher si ses conceptions « surhumaines », contre la morale conventionnelle, etc., sont purement d’origine nietzschéenne, c’est-à-dire sont le produit de l’élaboration d’une pensée qu’il faut situer dans la sphère de la « haute culture », ou bien si elles ont des origines beaucoup plus modestes, et si elles ne sont pas, par exemple, liées à la littérature des romans-feuilletons. (Et Nietzsche lui-même n’a-t-il en rien été influencé ; par les romans-feuilletons français ? Il faut se souvenir que ce genre de littérature, aujourd’hui dégradée aux loges de concierge, a été très répandue parmi les intellectuels, au moins jusqu’en 1870, comme le sont aujourd’hui ce qu’on appelle les romans « de la série noire »). Il semble de toute façon qu’on puisse affirmer qu’une grande partie de la soi-disant « surhumanité » nietzschéenne a simplement pour origine et pour modèle doctrinal non pas Zarathoustra, mais \emph{Le Comte de Monte-Cristo} d’A. Dumas. Le personnage le plus achevé qui est représenté par Dumas dans \emph{Le Comte de Monte-Cristo} a de nombreuses répliques dans d’autres romans du même auteur : on le retrouve, par exemple, dans Athos des \emph{Trois Mousquetaires}, dans \emph{Joseph Balsamo} et peut-être aussi dans d’autres personnages. De même, quand on lit que quelqu’un est un admirateur de Balzac, il faut être sur ses gardes : dans Balzac aussi il y a bien des choses quirelèvent du roman-feuilleton. Vautrin lui aussi est à sa façon un surhomme, et le discours qu’il tient à Rastignac dans \emph{Le Père Goriot} est fortement… nietzschéen au sens vulgaire du mot ; et l’on peut dire la même chose de Rastignac et de Rubempré\footnote{Deux types d’ambitieux des romans de Balzac. Leur histoire est révélatrice d’une société (celle de la Restauration) où, à la puissance du titre de noblesse revalorisé par la monarchie, tend à se substituer la puissance que confère à l’argent l’essor récent du capitalisme.}.\par
Le succès de Nietzsche a été très composite : ses oeuvres complètes sont publiées par l’éditeur Monanni et l’on connaît les origines culturelles idéologiques de Monanni et de sa plus fidèle clientèle.\par
Vautrin et « l’ami de Vautrin » ont laissé une trace profonde dans la littérature de Paolo Valera et dans sa \emph{Folla} (se rappeler le Turinois « ami de Vautrin » de la \emph{Folla).} L'idéologie du « mousquetaire » empruntée au roman de Dumas a trouvé par la suite un large écho dans le peuple.\par
Que l’on éprouve une certaine pudeur à justifier mentalement ses propres conceptions avec les romans de Dumas et de Balzac, cela se comprend aisément : c’est pourquoi on les justifie avec Nietzsche et l’on admire Balzac comme écrivain d’art et non comme créateur de personnages romanesques du genre feuilleton. Mais le lien réel paraît certain du point de vue de la culture. Le type du « surhomme » est Monte-Cristo, libéré de cette auréole particulière de « fatalisme » qui est propre au bas romantisme et qui est encore plus appuyé chez Athos et chez Joseph Balsamo. Monte-Cristo transporté dans la politique est certes tout à fait pittoresque (la lutte contre les « ennemis personnels » de Monte-Cristo, etc.). On peut observer à quel point certains pays ont pu rester, dans ce domaine également, provinciaux et arriérés par rapport à d’autres pays ; alors que Sherlock Holmes est déjà devenu anachronique pour une grande partie de l’Europe, dans certains pays on en est encore à \emph{Monte-Cristo} et à Fenimore Cooper (cf. « les sauvages », « barbiche de fer », etc\footnote{Types de personnages devenus populaires par les romans de Fenimore Cooper.}. ).\par
Voir le livre de Mario Praz \emph{: La carne, la morte e il diavolonella letteratura romantica} (édition \emph{La Cultura).} A côté de la recherche de Praz, il faudrait faire celle-ci : du « surhomme » dans la littérature populaire et de ses influences sur la vie réelle et sur les mœurs (la petite-bourgeoisie et les petits intellectuels sont particulièrement influencés par ce genre d’images romanesques, qui sont comme leur « opium », leur « paradis artificiel », en opposition avec leur vie mesquine et étroite dans la réalité immédiate) : d’où le succès de certains slogans comme : « il vaut mieux vivre un jour comme un lion que cent ans comme une brebis\footnote{Le coup de patte satirique de Gramsci utilise ici un des innombrables slogans fascistes dont Mussolini avait fait recouvrir les murs de toute l’Italie.}. », succès particulièrement grand chez ceux qui sont justement, et irrémédiablement, des brebis. Combien de ces « brebis » disent « Oh ! si j’avais le pouvoir, même un seul jour ! », etc. ; être des « justiciers » implacables, c’est à quoi aspirent ceux qui subissent l’influence de Monte-Cristo.\par
Adolfo Omodeo a observé qu’il existe une sorte de « mainmorte » culturelle, constituée par la littérature religieuse, dont personne ne semble vouloir s’occuper, comme si elle n’avait aucune importance et aucune fonction dans la vie nationale et populaire. A part l’épigramme de la « mainmorte » et la satisfaction du clergé devant le fait que sa littérature spéciale n’est pas soumise à un examen critique, il existe un autre secteur de la vie culturelle nationale et populaire dont personne ne s’occupe et ne se préoccupe de façon critique ; et c’est précisément la littérature de feuilleton proprement dite, au sens large du terme (au sens où Victor Hugo et même Balzac en font partie !)\par
Dans \emph{Monte-Cristo} il y a deux chapitres où l’on disserte explicitement sur le « surhomme » des feuilletons : le chapitre intitulé « Idéologie », lorsque MonteCristo rencontre le procureur Villefort ; et celui qui décrit le déjeuner chez le vicomte de Morcerf lors du premier voyage de Monte-Cristo à Paris. Voir si dans d’autres romans de Dumas il existe d’autres éléments « idéologiques » de ce genre. Dans \emph{Les Trois Mousquetaires}, Athos tient davantage du type général de l’homme fatal du bas romantisme : dans ce roman l’humeur individualiste des gens du peuple est plutôt excitée par l’activité aventureuse et extra-légale des mousquetaires en tant que tels. Dans \emph{Joseph Balsamo}, la puissance de l’individu est liée à des forces obscures de magie et à l’appui que lui donne la maçonnerie européenne, donc l’exemple est moins suggestif pour le lecteur populaire. Chez Balzac, les personnages ont un caractère artistique plus concret, mais ils font cependant partie de l’atmosphère du romantisme populaire. Rastignac et Vautrin ne doivent certes pas être confondus avec les personnages de Dumas et c’est justement pourquoi leur influence est plus facile à « avouer », non seulement pour des hommes comme Paolo Valera et ses collaborateurs de la \emph{Folla}, mais aussi pour de médiocres intellectuels comme Vincenzo Morello, que l’on considère pourtant (ou du moins qui sont considérés par beaucoup) comme appartenant au monde de la « haute culture ». De Balzac il faut rapprocher Stendhal avec Julien Sorel et d’autres personnages de ses romans.\par
Pour le « surhomme » de Nietzsche, outre l’influence romantique française (et de façon générale celle du culte de Napoléon) il faut voir les tendances racistes, qui ont atteint leur point culminant chez Gobineau et de là chez Chamberlain et dans le pangermanisme (Treitschke, la théorie de la puissance, etc.). Mais peut-être le « surhomme » populaire de Dumas doit-il être considéré comme une réaction « démocratique » à la conception d’origine féodale du racisme, qu’il faudrait lier à l’exaltation du « chauvinisme » français dans les romans d’Eugène Sue.\par
Comme réaction à cette tendance du roman populaire français il faut signaler Dostoïevski : Raskolnikov, c’est Monte-Cristo « critiqué » pas un panslaviste chrétien. Pour l’influence exercée sur Dostoïevski par le roman feuilleton français, voir le numéro spécial de \emph{La Cultura} consacré à Dostoïevski.\par
Le caractère populaire du « surhomme » contient de nombreux éléments théâtraux, extérieurs, qui conviennent à une « primadonna » plutôt qu’à un « surhomme » ; beaucoup de formalisme « subjectif et objectif », l’ambition enfantine d’être « le premier de la classe », mais surtout celle d’être considéré et proclamé tel. Sur les rapports entre le bas romantisme et certains aspects de la vie moderne (atmosphère digne du \emph{Comte de MonteCristo}), lire un article de Louis Gillet dans la \emph{Revue des Deux Mondes} du 15 décembre 1932. Ce type de « surhomme » trouve son expression au théâtre (surtout au théâtre français, qui continue sous bien des aspects la littérature de feuilleton style 1848) : voir le répertoire « classique » de Ruggero Ruggeri, comme \emph{Le Marquis de Priola, La Griffe}\footnote{\emph{Le Marquis de Priola}, drame de Henry Lavedan (1902), adaptation « fin de siècle » de la légende de Don Juan. – \emph{La Griffe}, drame d’Henri Bernstein (1906).}, etc., et de nombreuses pièces d’Henry Bernstein. (L.V.N., pp. 122-124.)\par
{\raggedleft \noindent [1933-1934]}
\section[{Balzac}]{Balzac}
\noindent Voir l’article de Paul Bourget : « Les idées politiques et sociales de Balzac » dans \emph{Les Nouvelles littéraires} du 8 août 1931. Bourget commence par remarquer qu’aujourd’hui on donne toujours plus d’importance aux idées de Balzac :\par

\begin{quoteblock}
 \noindent « L'école traditionaliste [c’est-à-dire réactionnaire extrémiste] que nous voyons grandir chaque jour, inscrit son nom à côté de celui de De Bonald, de Le Play, de Taine lui-même. »
\end{quoteblock}

\noindent Mais il n’en était pas ainsi autrefois. Sainte-Beuve, dans son article des « Lundis » consacré à Balzac après sa mort, ne fait même pas allusion à ses idées politiques et sociales. Taine, qui admirait l’auteur des romans, lui refusa toute importance doctrinale. Le critique catholique Caro lui-même, dans les premières années du Second Empire, jugeait futiles les idées de Balzac. Flaubert écrit que les idées politiques et sociales de Balzac ne valent pas la peine d’être discutées : « Il était catholique, légitimiste, propriétaire, écrit Flaubert, un immense bonhomme, mais de second ordre. » Zola écrit : « Rien de plus étrange que ce soutien du pouvoir absolu, dont le talent est essentiellement démocratique et qui a écrit l’œuvre la plus révolutionnaire. »\par
On comprend l’article de Paul Bourget. Il s’agit de trouver chez Balzac l’origine du roman positiviste, mais réactionnaire, la science au service de la réaction (genre Maurras), ce qui par ailleurs est le destin le plus exact du positivisme établi par Comte. (L.V.N., p. 125.)\par
{\raggedleft \noindent [1932-1933]}
\section[{Balzac et la science}]{Balzac et la science}

\begin{quoteblock}
 \noindent Voir la « Préface générale » de La Comédie Humaine, où Balzac écrit que le naturaliste aura l’éternel honneur d’avoir montré que \\
 « l’animal est un principe qui prend sa forme extérieure ou mieux, les différences de sa forme, dans les milieux où il est appelé à se développer. Les espèces zoologiques résultent de ces différences… Pénétré de ce système, je vis que la société ressemble à la nature. Ne fait-elle pas de l’homme, suivant les milieux où son action se déploie, autant d’hommes différents qu’il y a d’espèces zoologiques ?… Il adonc existé, il existera de tous temps des espèces sociales comme il y a des espèces zoologiques. La différence entre un soldat, un ouvrier, un administrateur, un oisif (!!), un savant, un homme d’Etat, un commerçant, un marin, un poète, un pauvre (!!), un prêtre, sont aussi considérables que celles qui distinguent le loup, le lion, l’âne, le corbeau, le requin, le veau marin, la brebis. »
\end{quoteblock}

\noindent Que Balzac ait écrit cela et même qu’il l’ait pris au sérieux et qu’il ait pu imaginer de construire tout un système social sur ces métaphores, cela n’a rien d’étonnant et ne diminue en rien la grandeur de Balzac artiste. Ce qui est remarquable, c’est qu’aujourd’hui Paul Bourget et, comme il le dit, « l’école traditionaliste », se soient appuyés sur ces pauvres fantaisies « scientifiques » pour construire des systèmes politico-sociaux qui ne peuvent même pas se justifier par leur valeur artistique. En partant de ces données, Balzac se pose le problème de « perfectionner au maximum ces espèces sociales » et de les harmoniser entre elles, mais comme les « espèces » sont créées par le milieu, il faudra « conserver » et organiser le milieu donné pour maintenir et perfectionner l’espèce donnée. Il semble que Flaubert n’avait pas tort d’écrire que cela ne vaut pas la peine de discuter les idées sociales de Balzac. Et l’article de Bourget montre seulement à quel point est fossilisée l’école traditionaliste française.\par
Mais si tout le système de Balzac est sans importance comme « programme pratique », c’est-à-dire du point de vue sous lequel l’examine Bourget, il y a en lui des éléments qui présentent de l’intérêt pour reconstruire le monde poétique de Balzac, sa conception du monde telle, qu’il l’a réalisée sur le plan artistique, son « réalisme » qui, tout en ayant des origines idéologiques réactionnaires, propres à la Restauration, monarchistes, etc., n’en est pas moins pour cela du réalisme effectif. Et l’on comprend l’admiration qu’eurent pour Balzac les fondateurs de la philosophie de la praxis : que l’homme soit tout le complexe résultant des conditions sociales au sein desquelles il s’est développé et il vit, que pour « changer » l’homme il faille changer cet ensemble complexe de conditions, c’est ce qu’a clairement compris Balzac. Que « politiquement et socialement » il ait été un réactionnaire, cela n’apparaît que dans la partie extra-artistique de ses écrits (divagations, préfaces, etc.). Et même il est vrai également que ce « complexe de conditions » ou ce « milieu » est compris de façon « naturaliste » ; en effet Balzac précède un certain courant littéraire français, etc. (L.V.N., pp. 125-126.)\par
{\raggedleft \noindent [1932-1933]}
\chapterclose


\chapteropen
\chapter[{Langue nationale et grammaire}]{Langue nationale et grammaire}\renewcommand{\leftmark}{Langue nationale et grammaire}


\chaptercont
\section[{Essai de Croce : cette table ronde est carrée}]{Essai de Croce : cette table ronde est carrée\protect\footnotemark }
\footnotetext{ \noindent Dans le court écrit : « Cette table ronde est carrée », \emph{Problèmes d’esthétique}, II° édition, pp. 139-173, Croce affirme que la proposition, bien que « grammaticalement rationnelle », est logiquement et esthétiquement « absurde » et donc « repoussée hors du domaine de l’esprit théorique » ; il y voit une preuve du fait que la grammaire n’est pas une science mais seulement « un ensemble d’abstractions et de choix entièrement pratiques ».\par
 Il reconnaît toutefois que « si je veux donner une représentation concrète à cette proposition, je dois considérer par exemple qu’elle a été construite intentionnellement pour représenter une incohérence mentale… ce que nous faisons justement en ce moment ». Gramsci part de cette dernière affirmation pour confirmer la « fonctionnalité » c’est-à-dire l’ « historicité » de toute vérité, y compris des vérités grammaticales.
}
\noindent [Cahier XXI (29) écrit en 1935.] (L.V.N., pp. 197-205.)\par
L'essai est erroné même du point de vue crocien (de la philosophie crocienne). L'emploi même que fait Croce de la proposition montre qu’elle est « expressive » donc justifiée : on peut dire la même chose de toutes les « propositions », y compris des propositions qui ne sont pas « techniquement » grammaticales, qui peuvent être expressives et justifiées dans la mesure où elles ont une fonction, même négative (pour mettre en évidence l’ « erreur » de grammaire, on peut employer une incorrection).\par
Le problème se pose donc différemment, en termes de « discipline à l’historicité du langage » dans le cas des « incorrections » (qui sont absence de « discipline » mentale, néologisme, particularisme provincial, jargon, etc.) ou dans d’autres termes (dans le cas proposé par l’essai de Croce, l’erreur est établie ainsi : une telle proposition peut apparaître dans la représentation d’un « fou », d’un anormal, etc., et acquérir une valeur expressive absolue ; comment représenter quelqu’un qui n’est pas « logique » sinon en lui faisant dire des « choses illogiques » ?). En réalité, tout ce qui est « grammaticalement exact » peut être justifié aussi du point de vue esthétique, logique, etc., si on l’envisage non pas dans la logique particulière de l’expression immédiatement mécanique, mais comme élément d’une représentation plus vaste et plus globale.\par
La question que veut poser Croce : « Qu'est-ce que la grammaire ? » ne peut pas avoir de solution dans son essai. La grammaire est « histoire » ou « document historique » : elle est la « photographie » (ou bien les traits fondamentaux de la « photographie » ) d’une phase déterminée d’un langage national (collectif) formé historiquement et en continuel développement. La question pratique peut être : dans quel but cette photographie ? Pour faire l’histoire d’un aspect de la civilisation, ou pour modifier un aspect de la civilisation ? La prétention de Croce conduirait à nier toute valeur à un tableau représentant entre autres, par exemple, une… sirène : autrement dit, on devrait conclure que toute proposition doit correspondre au \emph{vrai} ou au \emph{vraisemblable}, etc.\par
(La proposition peut être non logique en soi, contradictoire, mais en même temps « cohérente » dans un cadre plus vaste.)
\section[{ Combien peut-il y avoir de formes de grammaire ?}]{ Combien peut-il y avoir de formes de grammaire ?}
\noindent Pas mal, certainement. Il y a la forme « immanente » au langage lui-même, qui nous fait parler « selon la grammaire » sans le savoir, tout comme le personnage de Molière faisait de la prose sans le savoir. Et ce rappel ne doit pas sembler inutile, car Panzini \emph{(Guide de grammaire italienne}, 18° mille) ne paraît pas distinguer entre cette « grammaire » et la grammaire « normative » écrite dont il veut parler et qui lui semble être la seule grammaire pouvant exister. La préface à la première édition est pleine d’aménités, qui ont d’ailleurs leur signification chez un écrivain (tenu pour un spécialiste) de grammaire, telle l’affirmation que « nous pouvons écrire et parler même sans grammaire ».\par
En réalité, à côté de la « grammaire immanente » à tout langage, il existe aussi de fait, c’est-à-dire même si elle n’est pas écrite, une (ou plusieurs) grammaire « normative » et qui est constituée par le contrôle réciproque par l’enseignement réciproque, par la « censure » réciproque, qui se manifestent à travers les questions : « Qu'as-tu compris ? », « Que veux-tu dire ? », « Explique-toi mieux », etc., à travers la caricature et la moquerie, etc. Tout cet ensemble d’actions et de réactions contribue à déterminer un conformisme grammatical, autrement dit, à établir des « normes » et des jugements de correction ou d’incorrection, etc. Mais cette manifestation e spontanée » d’un conformisme grammatical est nécessairement décousue, discontinue, limitée à des couches sociales locales ou à des centres locaux. (Un paysan qui s’urbanise finit par se conformer au parler de la ville sous la pression du milieu citadin ; à la campagne on cherche à imiter le parler de la ville ; les classes subalternes cherchent à porter comme les classes dominantes et les intellectuels, etc.).\par
On pourrait tracer un tableau de la « grammaire normative » qui est spontanément à l’œuvre dans toute société, dans la mesure où celle-ci tend à s’unifier soit comme territoire, soit comme culture, c’est-à-dire dans la mesure où il y a, dans cette société, une couche dirigeante dont la fonction est reconnue et respectée. Le nombre des « grammaires spontanées » ou « immanentes » est incalculable et on peut dire, en théorie, que chacun a sa grammaire. Toutefois, il faut relever à côté de cette « désagrégation » de fait, les mouvements unificateurs de plus ou moins grande ampleur, soit comme aire territoriale, soit comme « volume linguistique ». Les « grammaires normatives » écrites tendent à embrasser tout un territoire national et tout le « volume linguistique » pour créer un conformisme linguistique national unitaire, qui situe d’ailleurs à un niveau supérieur l’ « individualisme » expressif parce qu’il donne un squelette plus robuste et plus homogène à l’organisme linguistique national dont chaque individu est le reflet et l’interprète (système Taylor et autodidacisme\footnote{Gramsci fait allusion ici au rapport entre ce que l’on pourrait appeler, en utilisant son langage, « une direction linguistique planifiée » ou une « taylorisation » dans le domaine de la langue, et l’initiative linguistique individuelle ou autoéducation critique dans le domaine du langage}).\par
Grammaires historiques et non seulement normatives. Il est évident que celui qui écrit une grammaire normative ne peut pas ignorer l’histoire de la langue dont il veut proposer une « phase exemplaire » comme la « seule » digne de devenir de façon « organique » et « totalitaire » la langue « commune » d’une nation, en lutte et en concurrence avec d’autres « phases », avec d’autres types ou d’autres schémas qui existent déjà (liés à des développements traditionnels ou à des tentatives inorganiques et incohérentes des forces qui, comme on l’a vu, agissent continuellement sur les « grammaires » spontanées et immanentes au langage). La grammaire historique ne peut pas ne pas être « comparative » : expression qui analysée à fond, indique la conscience intime que le fait linguistique, comme tout autre fait historique, ne peut pas avoir de frontières nationales strictement définies, mais que l’histoire est toujours « histoire mondiale » et que les histoires particulières ne vivent que dans le cadre de l’histoire mondiale. La grammaire normative a d’autres buts, même si elle ne peut pas imaginer la langue nationale hors du cadre des autres langues, qui influent par des voies innombrables et souvent difficiles à contrôler sur cette langue (qui peut contrôler l’apport des innovations linguistiques dû aux émigrants rapatriés, aux voyageurs, aux lecteurs de journaux en langues étrangères, aux traducteurs, etc.).\par
La grammaire normative écrite présuppose donc toujours un e choix », une orientation culturelle, c’est-à-dire qu’elle est toujours un acte de politique culturelle nationale. On pourra discuter sur la meilleure façon de présenter le « choix » et l’ « orientation » pour les faire accepter de bon gré, autrement dit on pourra discuter des moyens les plus opportuns pour obtenir le but ; il n’y a pas de doute qu’il y ait un but à atteindre, qui a besoin de moyens appropriés et conformes, c’est-à-dire qu’il s’agisse d’un acte politique.\par
Problèmes : quelle est la nature de cet acte politique et doit-il soulever des oppositions de « principe », une collaboration de fait, des oppositions de détail, etc. Si l’on part du présupposé qu’il faut centraliser ce qui existe déjà à l’état diffus, disséminé mais inorganique et incohérent, il semble évident qu’une opposition de principe n’est pas rationnelle ; il faut au contraire une collaboration de fait et une acceptation volontaire de tout ce qui peut servir à créer une langue nationale commune, dont la non-existence provoque des frictions surtout dans les masses populaires où les particularismes locaux et les phénomènes de psychologie étroite et provinciale sont plus tenaces qu’on ne le croit ; il s’agit en somme d’une intensification de la lutte contre l’analphabétisme, etc. L'opposition de fait existe déjà dans la résistance des masses à se dépouiller d’habitudes et de psychologies particularistes. Résistance stupide provoquée par les adeptes fanatiques des langues internationales\footnote{En 1918, Gramsci se battit dans\emph{ 'Avanti} contre les partisans de l’espéranto et surtout contre ceux qui voulaient engager officiellement le Parti socialiste dans le soutien de l’espérantisme, considéré comme l’expression linguistique de la bataille internationaliste. (Voir le résumé de cette polémique dans la note de G. Ferrata à l’article « La langue unique et l’espéranto » in \emph{2000 pagine di Gramsci}, tome 1, pp. 338-339).}. Il est clair que dans cet ordre de problèmes, on ne peut pas discuter la question de la lutte nationale d’une culture hégémonique contre d’autres nationalités ou restes de nationalités.\par
Panzini ne se pose pas même de loin ces problèmes, et ses publications grammaticales sont pour cette raison incertaines, contradictoires, oscillantes. Il ne se demande pas par exemple - et cette question ne manque pourtant pas d’importance pratique - quel est aujourd’hui le centre d’où, par le bas, irradient les innovations linguistiques : Florence, Rome, Milan. Mais il ne se demande d’ailleurs même pas s’il existe (et quel il est) un centre d’irradiation spontanée par le haut, c’est-à-dire sous une forme relativement organique, continue, efficiente, et si cette irradiation peut être réglée et intensifiée.
\section[{ Foyers d’irradiation des innovations linguistiques dans la tradition et d’un conformisme linguistique national dans les grandes masses nationales}]{ Foyers d’irradiation des innovations linguistiques dans la tradition et d’un conformisme linguistique national dans les grandes masses nationales}

\begin{enumerate}[itemsep=0pt,]
\item L'école ;
\item Les journaux ;
\item Les écrivains d’art et les écrivains populaires ;
\item Le théâtre et le cinéma sonore ;
\item La radio ;
\item Les réunions publiques de tout genre, y compris les réunions religieuses ;
\item Les relations de « conversation » entre les différentes couches de la population des plus cultivées aux moins cultivées\footnote{Un problème auquel on ne donne peut-être pas toute l’importance qu’il mérite est celuid e la part de « paroles » versifiées qui sont apprises par cœur sous la forme de chansons, d’extraits d’opéra, etc. Il faut remarquer combien le peuple se soucie peu de bien apprendre par cœur ces paroles, qui sont souvent biscornues, démodées, baroques, mais les réduit à des espèces de rengaines qui ne servent qu’à se rappeler le thème musical. \emph{(Note de Gramsci.)}} ;
\item Les dialectes locaux, entendus en divers sens (des dialectes les plus localisés à ceux qui embrassent des ensembles régionaux plus ou moins vastes ; comme le napolitain pour l’Italie méridionale, le palermitain et le catanais pour la Sicile, etc.).
\end{enumerate}

\noindent Puisque le processus de formation, de diffusion et de développement d’une langue nationale unitaire se fait à travers tout un ensemble de processus moléculaires, il est utile d’avoir conscience du processus dans son ensemble pour être en mesure d’y intervenir activement avec le maximum de résultats. Il ne faut pas considérer cette intervention comme « décisive » et imaginer que les buts proposés seront tous atteints dans leur détail, c’est-à-dire qu’on obtiendra une langue unitaire \emph{déterminée : on} obtiendra une \emph{langue unitaire} si elle est une nécessité, et l’intervention organisée accélèrera le rythme du processus déjà existant. On ne peut ni prévoir ni établir ce que sera cette langue ; en tout cas, si l’intervention est « rationnelle », elle sera organiquement liée à la tradition, ce qui n’est pas sans importance dans l’économie de la culture.\par
Manzoniens et « classicistes » avaient un type de langue à faire prévaloir. Il n’est pas juste de dire que ces discussions ont été inutiles et n’ont pas laissé de traces dans la culture moderne, même si elles ne sont pas très importantes. En réalité, en ce dernier siècle, la culture unitaire s’est étendue et donc aussi avec elle une langue unitaire commune. Mais la formation historique de la nation italienne, dans son ensemble, avait un rythme trop lent. Chaque fois qu’affleure d’une façon ou d’une autre la question de la langue\footnote{Voir dans « La question de la langue » \emph{Les Intellectuels et l’organisation de la culture}, pp. 23-24, l’esquisse rapide d’une histoire de la langue italienne comme expression des luttes d’hégémonie politique et culturelle dans le cadre de la société italienne : « Il n’existe pas encore d’histoire de la langue italienne, en ce sens : la grammaire historique ne constitue pas encore cette histoire, au contraire… Je crois que cette étude n’est ni inutile, ni purement érudite, si l’on entend la langue comme élément de la culture et donc de l’histoire générale, et comme manifestation principale de la « nationalité » et de la « popularité » d es intellectuels. » On trouve dans la lettre à Tatiana du 17 novembre 1930 (Lettre n° 170, Gallimard, pp. 265-266) une allusion sommaire mais limpide à cette histoire de la langue italienne qui coïncide, selon Gramsci, avec l’histoire de la « fonction cosmopolite » des intellectuels dans la société italienne. La conclusion en est que pendant toute l’histoire italienne « continua à exister une double langue, la langue populaire ou dialectale, et la langue savante, c’est-à-dire celle des intellectuels et des classes cultivées ». }, cela signifie qu’une série d’autres problèmes est en train de se poser ! : la formation et l’élargissement de la classe dirigeante, la nécessité d’établir des rapports plus intimes et plus sûrs entre les groupes dirigeants et la masse populaire nationale, c’est-à-dire de réorganiser l’hégémonie culturelle. Divers phénomènes se produisent aujourd’hui qui indiquent une renaissance de ces questions : publications de Panzini, Trabalza, Alladoli, Monelli, rubriques dans les journaux, interventions des directions syndicales, etc.
\section[{ Différents types de grammaire normative}]{ Différents types de grammaire normative}
\noindent Pour les écoles, pour les personnes dites cultivées. En réalité, la différence est due au degré différent de développement intellectuel du lecteur ou du spécialiste, et donc à la technique différente qu’il faut employer pour faire apprendre ou pour intensifier chez les jeunes élèves la connaissance organique de la langue nationale ; on ne peut pas négliger, en ce qui concerne ces jeunes élèves, le principe didactique d’une certaine rigidité autoritaire, péremptoire (« il faut dire comme cela » ) ; il faut au contraire « persuader » les « autres » pour leur faire accepter librement une solution déterminée comme la solution la meilleure (que l’on démontre être la meilleure en atteignant le but proposé et partagé, lorsqu’il est partagé).\par
Il ne faut pas oublier en outre que d’autres éléments du programme didactique d’enseignement général, comme certains éléments de logique formelle, ont été greffés sur l’étude traditionnelle de la grammaire normative : on pourra discuter pour savoir si cette greffe est opportune ou non, si l’étude de la logique formelle est justifiée ou non (elle semble justifiée et il semble juste aussi qu’elle accompagne l’étude de la grammaire, plus que de l’arithmétique, etc., en raison de leur similitude de nature, et parce que, liée à la grammaire, la logique formelle est relativement vivifiée et facilitée) mais il ne faut pas laisser la question de côté.
\section[{ Grammaire historique et grammaire normative}]{ Grammaire historique et grammaire normative}
\noindent Une fois établi que la grammaire normative est un acte politique et que ce n’est qu’en partant de ce point de vue que l’on peut justifier « scientifiquement » son existence et l’énorme travail de patience que réclame son apprentissage (autant de travail qu’il en faut pour obtenir que des centaines de milliers de recrues d’origine et de préparation intellectuelles les plus disparates naisse une armée homogène et capable de se mouvoir et d’agir simultanément et de façon disciplinée : autant de « leçons pratiques et théoriques » dérèglements, etc.), il faut poser son rapport avec la grammaire historique.\par
Le fait de ne pas avoir défini ce rapport explique un grand nombre des incongruités des grammaires normatives jusqu’à la grammaire de Trabalza-Allodoli\footnote{Ciro Trabalza et Ettore Allodoli : \emph{La Grammaire des Italiens}, Florence 1935.}. Il s’agit de deux choses distinctes et en partie différentes, comme l’histoire et la politique, mais qui ne peuvent pas être pensées indépendamment l’une de l’autre : comme la politique de l’histoire. D'ailleurs, puisque l’étude des langues comme phénomène culturel est née de besoins politiques (plus ou moins conscients et consciemment exprimés), les nécessités de la grammaire normative ont influé sur la grammaire historique et sur ses « conceptions législatives » (tout au moins cet élément traditionnel a renforcé au siècle dernier l’application de la méthode naturaliste-positiviste à l’étude de l’histoire des langues conçue comme « science du langage » ). Il ressort de la grammaire de Trabalza et aussi du compte rendu éreintant de Schiaffini \emph{(Nuova Antologia}, 16 septembre 1934) combien même les prétendus « idéalistes » n’ont pas compris le renouvellement apporté à la science du langage par les doctrines de Bartoli. La tendance de l’ « idéalisme » a trouvé son expression la plus complète chez Bertoni : il s’agit d’un retour à de vieilles conceptions rhétoriques sur les mots « beaux » et « laids » en soi et pour soi, conceptions revernies par un nouveau langage pseudo-scientifique. En réalité, on tente de trouver une justification extrinsèque à la grammaire normative, après 'en avoir tout aussi extrinsèquement « démontré » l’ « inutilité » théorique et même pratique.\par
L'essai de Trabalza sur l’\emph{Histoire de la grammaire} pourra fournir des indications utiles sur les interférences entre grammaire historique (ou mieux histoire du langage) et grammaire normative, sur l’histoire du problème, etc.
\section[{ Grammaire et technique}]{ Grammaire et technique}
\noindent Peut-on poser le problème pour la grammaire comme pour la « technique » en général ? La grammaire est-elle seulement la technique de la langue ? En tout cas, la thèse des idéalistes, surtout la thèse gentilienne, de l’inutilité de la grammaire et de son exclusion de l’enseignement scolaire est-elle justifiée ? Si l’on parle (on s’exprime avec des mots) d’une façon historiquement déterminée pour des nations ou pour des aires linguistiques, peut-on se passer d’enseigner cette « façon historiquement déterminée » ? Une fois admis que la grammaire normative traditionnelle est insuffisante, est-ce une bonne raison pour n’enseigner aucune grammaire, c’est-à-dire pour ne se préoccuper d’aucune façon d’accélérer l’apprentissage de la manière de parler déterminée d’une certaine aire linguistique, mais de laisser « apprendre la langue dans le langage vivant » ou autres expressions de ce genre employées par Gentile ou par les gentiliens ? Il s’agit au fond d’une forme de libéralisme des plus extravagantes et des plus biscornues.\par
Différences entre Croce et Gentile. D'habitude, Gentile s’appuie sur Croce, en en exagérant jusqu’à l’absurde quelques positions théoriques. Croce soutient que la grammaire ne fait partie d’aucune des activités spirituelles théoriques élaborées par lui, mais il finit par trouver dans la « pratique » la justification d’un grand nombre d’activités niées d’un point de vue théorique : Gentile exclut aussi de la pratique, dans un premier temps, ce qu’il nie théoriquement, quitte à trouver ensuite une justification théorique aux manifestations pratiques les plus dépassées et les plus injustifiées techniquement.\par
Doit-on apprendre « systématiquement » la technique ? Il est arrivé que la technique de l’artisan de village s’oppose à celle de Ford. On apprend la « technique industrielle » de bien des f acons : artisans, pendant le travail de l’usine lui-même, observant comment les autres travaillent (et donc avec une plus grande perte de temps et d’énergie et seulement partiellement) ; dans les écoles professionnelles (dans lesquelles on apprend systématiquement tout le métier, même si quelques-unes des notions apprises ne doivent servir qu’un petit nombre de fois dans la vie, et même jamais) ; par la combinaison des différentes manières, avec le système Taylor-Ford qui crée un nouveau type de qualification et de métier limité à des usines déterminées, et même à des machines ou à des moments du processus de production.\par
La grammaire normative, que l’on ne peut concevoir séparée du langage vivant que par abstraction, tend à faire apprendre l’ensemble de l’organisme de la langue déterminée et à créer une attitude spirituelle qui rend apte à s’orienter toujours dans le domaine linguistique (voir la note sur l’étude du latin dans les écoles classiques)\footnote{[Il est significatif que la défense de la grammaire à l’école se conclut par cette remarque polémique à l’égard de l’activisme rousseauiste et néo-idéaliste, dont Gramsci rappelle en outre le rapport de dépendance et la « fonction historique ». Mais cela ne signifie pas pour autant une indulgence à l’égard du formalisme grammatical comme on peut le voir dans ce passage d’une lettre à Carlo : « Dans l’apprentissage des langues que tu comptes entreprendre, je te conseille de ne pas trop te perdre dans les grammaires, mais de lire, lire, en feuilletant plus souvent le dictionnaire que la grammaire. La grammaire, selon moi, doit accompagner la traduction et non la précéder. Beaucoup commencent par les grammaires et ils n’en sortent plus, bien qu’ils se cassent la tête. » (Lettre n° 215, Gallimard, pp. 344-345).}.\par
Si la grammaire est exclue de l’école et n’est pas « écrite », on ne peut pas l’exclure, pour autant, de la « vie » réelle, comme on l’a déjà dit dans une autre note : on exclut seulement l’intervention organisée et unitaire dans l’apprentissage de la langue et, en réalité, on exclut de l’apprentissage de la langue cultivée la masse populaire nationale, puisque la couche dirigeante la plus élevée, qui parle traditionnellement le « beau langage », transmet cette langue de génération en génération, à travers un processus lent qui commence avec les premiers balbutiements de l’enfant sous la direction des parents, et qui se poursuit dans la conversation (avec ses « on dit ainsi », « on doit dire ainsi », etc.) toute la vie : en réalité, on étudie « toujours » la grammaire, etc. (par l’imitation des modèles admirés, etc.). Il y a, dans la position de Gentile, beaucoup plus de politique qu’on ne le croit et beaucoup d’attitude réactionnaire inconsciente comme du reste on l’a noté d’autres fois à d’autres occasions ; il y a toute l’attitude réactionnaire de la vieille conception libérale, il y a un « laisser faire, laisser passer » qui n’est pas justifié comme il l’était chez Rousseau (et Gentile est plus rousseauiste qu’il ne le croit) par l’opposition à la paralysie de l’école jésuite, mais il est devenu une idéologie abstraite et « anhistorique ».
\section[{ Ce qu’on appelle « question de la langue »}]{ Ce qu’on appelle « question de la langue »}
\noindent Il semble clair que le \emph{De vulgari eloquentia}\footnote{\emph{De vulgari eloquentia} (De l’éloquence vulgaire) : Ecrit de Dante destiné à défendre l’usage en littérature de la langue « vulgaire » (c’est-à-dire l’Italien), contre ceux qui préféreraient écrire en latin, langue « savante ». } de Dante est à considérer essentiellement comme un acte de politique culturelle-nationale (au sens qu’avait « national » à cette époque et chez Dante), de même que ce qu’on appelle « la question de la langue » - qui devient intéressante à étudier de ce point de vue - a toujours été un aspect de la lutte politique. La « question de la langue » a été une réaction des intellectuels face à l’écroulement de l’unité politique, survenu en Italie sous le nom d’ « équilibre des Etats italiens », face à l’écroulement et à la désintégration des classes économiques et politiques qui s’étaient formées après l’an Mille avec les Communes ; elle représente la tentative, dont on peut dire qu’elle a réussi pour une bonne part, de préserver et même de renforcer une couche intellectuelle unitaire dont l’existence allait revêtir une grande importance au XVIII° et au XIX° siècles (pendant le Risorgimento). Le petit livre de Dante a lui aussi une grande importance pour l’époque à laquelle il fut écrit : les intellectuels italiens de la période la plus luxuriante des Communes rompent, non seulement dans les faits mais aussi en élevant le fait à la théorie, avec le latin et justifient la langue vulgaire en l’exaltant contre le « mandarinat » latinisant, au moment même où la langue vulgaire s’illustre dans de grandes manifestations artistiques. Que la tentative de Dante ait eu une importance innovatrice énorme, on le voit plus tard avec le retour du latin comme langue des gens cultivés (et ici peut se greffer la question du double aspect de l’Humanisme et de la Renaissance, qui furent essentiellement réactionnaires du point de vue national populaire et progressistes comme expression du développement culturel des groupes intellectuels italiens et européens).
\chapterclose


\chapteropen
\chapter[{ Américanisme et Fordisme}]{ Américanisme et Fordisme}\renewcommand{\leftmark}{ Américanisme et Fordisme}


\chaptercont
\noindent  \par
[Notes extraites du Cahier V écrit en 1934.]\par
\section[{Quelques aspects de la question sexuelle}]{Quelques aspects de la question sexuelle}
\noindent Obsession de la question sexuelle et dangers de cette obsession. Tous les « auteurs de projets » placent en premier lieu la question sexuelle et la résolvent « ingénument ».\par
Il faut remarquer la très large part, et souvent la part prépondérante qu’a la question sexuelle dans les « utopies » (l’observation de Croce, disant que les solutions apportées par Campanella dans sa \emph{Cité du soleil}, ne peuvent s’expliquer par les besoins sexuels des paysans de Calabre, est stupide). Les instincts sexuels sont ceux qui ont été le plus fortement réprimés par la société en développement ; leur « régularisation », étant donné les contradictions qu’elle fait apparaître, et les perversions qu’on lui attribue, semble la chose la moins « naturelle », aussi les références à la « nature » se font-elles plus fréquentes dans ce domaine. La littérature « psychanalytique » est, elle aussi, une façon de critiquer la réglementation des instincts sexuels sous sa forme parfois « illuministe\footnote{\emph{Illuminisme, philosophie illuministe} : c’est l’appellation italienne du courant philosophique du siècle des « lumières », c’est-à-dire le XVIII° siècle français, avec ses grands écrivains qui ont apporté les « lumières » de la raison} », avec la création d’un nouveau mythe, celui du « sauvage », fondé sur la sexualité (y compris dans les rapports entre parents et enfants).\par
Grande différence, dans ce domaine, entre ville et campagne, mais pas dans le sens idyllique en ce qui concerne la campagne, où se produisent les crimes sexuels les plus monstrueux et les plus nombreux, où la bestialité et la pédérastie sont très répandues. Dans l’enquête parlementaire sur le Midi faite en 1911 on relève que dans les Abruzzes et la Basilicata\footnote{Les \emph{Abruzzes} : région centrale des Apennins qui a donné son nom à la province actuelle d’Abruzzes et Molise (villes principales : Pescara, l’Aquila). / La Basilicata : province du sud de l’Italie (villes principales Potenza, Matera). / Ces deux régions sont parmi les plus pauvres de l’Italie.} (où le fanatisme religieux et le système du patriarcat sont plus développés, et où l’influence des idées de la ville se fait moins sentir au point que, selon Serpieri, au cours des années 1919-1920 il n’y eut pas la moindre agitation paysanne) on trouve l’inceste dans 30\% des familles, et il ne semble pas que cette situation ait changé jusqu’à ces dernières années.\par
La sexualité comme fonction de reproduction et comme sport : l’idéal « esthétique » de la femme oscille entre la conception d’ « administratrice » et celle de « bibelot », de « jouet ». Mais il n’y a pas qu’en ville que la sexualité est devenue un « sport » ; les proverbes populaires : « L'homme est chasseur, la femme est tentatrice », « Celui qui n’a pas mieux, va coucher avec sa femme », etc. montrent la diffusion de cette conception sportive même à la campagne et dans les rapports sexuels entre éléments d’une même classe.\par
Fonction économique de la reproduction : elle n’est pas seulement un fait général, intéressant la société dans son ensemble, qui réclame une certaine proportion entre les divers âges en vue de la reproduction et de l’entretien de la partie passive de la population (passive de façon normale, à cause de l’âge, de l’individualité, etc.), mais c’est aussi un fait « moléculaire », qui se retrouve au sein des plus petits agrégats économiques, comme la famille. L'expression « le bâton de la vieillesse » montre la conscience instinctive de la nécessité économique d’un certain rapport entre jeunes et vieux dans toute l’étendue de la société. Le spectacle des mauvais traitements dont sont l’objet, dans les villages, les vieux et les vieilles sans enfants, incite les couples à désirer des enfants (le proverbe : « Une mère élève cent fils et cent fils ne soutiennent pas une mère », montre un autre aspect de la question) : les vieux sans enfants, dans les classes populaires, sont traités comme les « bâtards ». Les progrès de l’hygiène, qui ont élevé l’âge moyen de la vie humaine, posent sans cesse davantage la question sexuelle comme un aspect fondamental et autonome de la question économique, aspect si important qu’il peut poser à son tour des problèmes complexes du type des « superstructures ». L'augmentation de la moyenne de la vie en France, jointe à la faible natalité et à la nécessité de faire fonctionner un appareil de production très riche et très complexe, posent déjà de nos jours certains problèmes liés au problème national : les vieilles générations se trouvent dans des rapports toujours plus anormaux avec les jeunes générations de même culture nationale, et les masses travailleuses sont grossies par des éléments étrangers immigrés qui modifient leur base : on voit déjà apparaître, comme en Amérique, une certaine division du travail : métiers qualifiés pour les autochtones (en dehors des fonctions de direction et d’organisation) et métiers non qualifiés pour les immigrés.\par
Un rapport semblable, mais aux conséquences antiéconomiques importantes, s’établit dans toute une série de pays entre les villes industrielles à basse natalité et la campagne prolifique : la vie de l’industrie demande un apprentissage général, un processus d’adaptation psycho-physique à des conditions déterminées de travail, de nourriture, d’habitation, de murs, etc. qui n’est pas quelque chose d’inné, de « naturel », mais qui doit être acquis, alors que les caractères urbains acquis se transmettent de façon héréditaire ou sont assimilés au cours de l’enfance et de l’adolescence. Aussi la faible natalité urbaine entraîne-t-elle une dépense importante et régulière pour l’apprentissage continuel de nouveaux éléments non-urbains, et comporte un changement perpétuel de la composition économico-sociale de la ville, en replaçant perpétuellement sur de nouvelles bases le problème de l’hégémonie.\par
La question de morale et de civilisation la plus importante, liée à la question sexuelle, est celle de la formation d’une nouvelle personnalité féminine : aussi longtemps que la femme ne sera pas parvenue non seulement à une réelle indépendance par rapport à l’homme, mais aussi à une nouvelle façon de se concevoir elle-même et de concevoir son rôle dans les rapports sexuels, la question sexuelle demeurera encombrée de caractères morbides et il faudra être prudent dans toute innovation législative à ce sujet. Toute crise de coercition unilatérale dans le domaine sexuel conduit à un dérèglement « romantique » qui peut être aggravé par l’abolition de la prostitution légale et organisée. Tous ces éléments compliquent et rendent extrêmement difficile toute réglementation du fait sexuel et toute tentative de créer une nouvelle éthique sexuelle conforme aux nouvelles méthodes de production et de travail. D'autre part, il est nécessaire de procéder à une telle réglementation et à la création d’une nouvelle éthique. Il faut remarquer que les industriels (et particulièrement Ford) se sont intéressés aux rapports sexuels de ceux qui sont sous leur dépendance et, d’une façon générale, de l’installation de leurs familles ; les apparences de « puritanisme » qu’a pris cet intérêt (comme dans le cas de la « prohibition » ) ne doit pas faire illusion ; la vérité est que le nouveau type d’homme que réclame la rationalisation de la production et du travail ne peut se développer tant que l’instinct sexuel n’a pas été réglementé en accord avec cette rationalisation, tant qu’il n’a pas été lui aussi rationalisé. (\emph{Mach.}, pp. 323-326.)
\section[{« Animalité » et industrialisme}]{« Animalité » et industrialisme}
\noindent L'histoire de l’industrialisme a toujours été (et elle le devient aujourd’hui sous une forme plus accentuée et plus rigoureuse) une lutte continue contre l’élément « animalité » de l’homme, un processus ininterrompu, souvent douloureux et sanglant, de la soumission des instincts (instincts naturels, c’est-à-dire animaux et primitifs)à des règles toujours nouvelles, toujours plus complexes et plus rigides, et à des habitudes d’ordre, d’exactitude, de précision qui rendent possibles les formes toujours plus complexes de la vie collective, conséquences nécessaires du développement de l’industrialisme. Cette lutte est imposée de l’extérieur et les résultats obtenus jusqu’ici, malgré leur grande valeur pratique immédiate, sont en grande partie purement mécaniques et ne sont pas devenus une « seconde nature ». Mais chaque nouvelle façon de vivre, dans la période où s’impose la lutte contre l’ancien, n’a-t-elle pas toujours été pendant un certain temps le résultat d’une compression mécanique ? Même les instincts qui doivent être dominés aujourd’hui parce qu’ils sont encore trop « animaux », ont été en réalité un progrès important sur les instincts antérieurs, encore plus primitifs : qui pourrait décrire combien a coûté, en vies humaines et en douloureuse soumission des instincts, le passage du nomadisme à la vie sédentaire et agricole ? Cela comprend les premières formes de l’esclavage de la glèbe et du métier, etc. Jusqu’ici tous les changements dans la façon d’être et de vivre se sont produits par coercition brutale, par la domination d’un groupe social sur toutes les forces productives de la société ; la sélection ou l’ « éducation » de l’homme adaptée aux nouveaux types de civilisation c’est-à-dire aux nouvelles formes de production et de travail, s’est réalisée au moyen de brutalités inouïes, en jetant dans l’enfer dessous-classes les faibles et les réfractaires, ou en les éliminant complètement. A chaque apparition de nouveaux types de civilisation, ou au cours du processus de leur développement, des crises se sont produites. Mais qui a été entraîné dans ces crises ? Pas les masses travailleuses, mais les classes moyennes et une partie de la classe dominante elle-même, qui avaient éprouvé elles aussi la pression coercitive, qui s’était nécessairement exercée sur toute l’étendue de la société. Les crises de \emph{libertinisme} ont été nombreuses : chaque époque historique a eu la sienne.\par
Lorsque la pression coercitive s’exerce sur l’ensemble social (et cela se produit spécialement après la chute de l’esclavagisme et l’avènement du christianisme) on voit se développer des idéologies puritaines qui confèrent à l’emploi intrinsèque de la force les formes extérieures de la conviction et du consentement : mais une fois le résultat atteint, au moins dans une certaine mesure, la pression se disperse (cette cassure se manifeste historiquement de façons très diverses, comme il est naturel, car la pression a toujours pris des formes originales, et souvent personnelles : si elle s’est identifiée avec un mouvement religieux, elle a créé son propre appareil, qui s’est personnifié dans certaines couches ou castes, et a pris le nom de Cromwell ou de Louis XV, etc.), et la crise de libertinisme se produit (la crise française après la mort de Louis XV, par exemple, ne peut être comparée avec la crise américaine qui suivit l’avènement de Roosevelt, de même que la prohibition n’a pas d’équivalent dans les époques précédentes, avec les actes de banditisme qui l’ont suivie, etc.) ; pourtant cette crise ne touche que de façon superficielle les masses travailleuses, où elle ne les touche qu’indirectement car elle déprave leurs femmes : en effet ces masses, ou bien ont déjà acquis les habitudes et les façons de vivre rendues nécessaires par le nouveau système de vie et de travail, ou bien continuent à ressentir la pression coercitive pour les nécessités élémentaires de leur existence (l’anti-prohibitionnisme lui-même n’a pas été voulu par les ouvriers, et la corruption que la contrebande et le banditisme apportèrent avec eux était répandue dans les classes supérieures).\par
Dans l’après-guerre on a assisté à une crise des murs d’une étendue et d’une profondeur considérables, mais cette crise s’est manifestée contre une forme de coercition qui n’avait pas été imposée pour créer des habitudes conformes à une nouvelle forme de travail, mais en raison des nécessités, d’ailleurs considérées comme transitoires, de la vie de guerre dans les tranchées. Cette pression a réprimé en particulier les instincts sexuels, même normaux, chez une grande masse de jeunes gens, et la crise s’est déchaînée au moment du retour à la vie normale, et elle a été rendue encore plus violente par la disparition d’un si grand nombre d’hommes et par un déséquilibre permanent dans le rapport numérique entre les individus des deux sexes. Les institutions liées à la vie sexuelle ont subi une forte secousse et la question sexuelle a vu se développer de nouvelles formes d’utopie de tendance « illuministe ». La crise a été (et elle est encore) rendue plus violente du fait qu’elle a touché toutes les couches de la population et qu’elle est entrée en conflit avec les exigences de nouvelles méthodes de travail qui sont venues, entre temps, s’imposer (taylorisme et rationalisation en général). Ces nouvelles méthodes exigent une discipline rigide des instincts sexuels (du système nerveux), c’est-à-dire une consolidation de la « famille » au sens large (et non de telle ou telle forme de système familial), de la réglementation et de la stabilité des rapports sexuels.\par
Il faut insister sur le fait que, dans le domaine de la sexualité, le facteur idéologique le plus dépravant et le plus « régressif » est la conception « illuministe » et libertaire propre aux classes qui ne sont pas liées étroitement au travail producteur, et qui se propage de ces classes à celles des travailleurs. Cet élément devient d’autant plus important lorsque, dans un Etat, les classes travailleuses ne subissent plus la pression coercitive d’une classe supérieure, lorsque les nouvelles habitudes et aptitudes psycho-physiques liées aux nouvelles méthodes de production et de travail doivent être acquises par voie de persuasion réciproque ou de conviction proposée à l’individu et acceptée par lui. Il peut ainsi se créer peu à peu une situation à double fond, un conflit intime entre l’idéologie « verbale » qui reconnaît la nécessité nouvelle, et la pratique réelle, « animale », qui empêche les corps physiques d’acquérir effectivement de nouvelles aptitudes. Il se forme dans ce cas ce que l’on peut appeler une situation d’hypocrisie sociale totalitaire. Pourquoi totalitaire ? Dans les autres situations, les couches populaires sont contraintes à observer la « vertu » ; celui qui la prêche ne l’observe pas, tout en lui rendant un hommage en paroles, de sorte que l’hypocrisie est partielle, non totale ; cette situation, certes, ne peut durer et doit conduire à une crise de libertinisme, mais lorsque les masses auront déjà assimilé la « vertu » par des habitudes permanentes ou presque permanentes, c’est-à-dire avec des oscillations toujours plus faibles. Au contraire, dans le cas où il n’y a pas de pression coercitive d’une classe supérieure, la « vertu » est affirmée de façon générale et n’est observée ni par conviction ni par coercition, sans qu’il y ait cependant une acquisition des aptitudes psycho-physiques nécessaires pour les nouvelles méthodes de travail. La crise peut devenir « permanente », c’est-à-dire avoir des perspectives catastrophiques, car seule la contrainte pourra régler la question, une contrainte de type nouveau, dans la mesure où, exercée par l’ « élite » d’une classe sur sa propre classe, elle ne peut être qu’une auto-coercition, c’est-à-dire une auto-discipline (Alfieri se faisant attacher à sa chaise\footnote{Désireux de rompre avec son milieu et son éducation aristocratique, Alfieri résolut, tout jeune, de se former par lui-même une personnalité égale à celle des héros de Plutarque qu’il admirait : pour se forcer à étudier, il se faisait attacher à sa chaise, devant son bureau, quatre à cinq heures par jour.}). En tout cas, ce qui peut s’opposer à cette fonction des \emph{élites} c’est la mentalité « illuministe » et libertaire appliquée au monde des rapports sexuels ; de plus, lutter contre cette conception signifie justement créer les \emph{élites} nécessaires à cette tâche historique, ou du moins les développer pour que leur fonction s’étende à toutes les branches de l’activité humaine. (\emph{Mach.}, pp. 326-329.)
\section[{ Rationalisation de la production et du travail}]{ Rationalisation de la production et du travail}
\noindent La tendance de Leone Davidovi\footnote{Lev Davidovitch Bronstein (Trotsky)} était étroitement liée à cette série de problèmes, ce qui ne me paraît pas avoir été bien mis en lumière. Son contenu essentiel, à ce point de vue, consistait dans la volonté « trop » résolue (par conséquent non rationalisée) d’accorder la suprématie, dans la vie nationale, à l’industrie et aux méthodes industrielles, d’accélérer, par des moyens de contrainte extérieure, la discipline et l’ordre dans la production, et d’adapter les mœurs aux nécessités du travail. Etant donné la façon générale d’aborder tous les problèmes liés à cette tendance, celle-ci devait nécessairement aboutir à une forme de bonapartisme, d’où la nécessité impérieuse de la supprimer. Ses préoccupations étaient justes, mais ses solutions pratiques étaient profondément erronées. C'est dans ce décalage entre la théorie et la pratique que résidait le danger, qui du reste s’était déjà manifesté précédemment, en 1921. Le principe de la contrainte, directe ou indirecte, dans l’organisation de la production et du travail, est juste, mais la forme qu’elle avait prise était erronée ; le modèle militaire était devenu un préjugé funeste et les armées du travail échouèrent. Intérêt de Leone Davidovi pour l’américanisme ; ses articles, ses enquêtes sur le « byt\footnote{« \emph{Byt} » : signifie en russe le mode de vie.} « et sur la littérature ; ces activités étaient moins étrangères les unes aux autres qu’il ne pourrait sembler, car les nouvelles méthodes de travail sont indissolublement liées à un certain mode de vie, à une certaine façon de penser et de sentir la vie ; on ne peut obtenir des succès dans un domaine sans obtenir des résultats tangibles dans l’autre. En Amérique la rationalisation du travail et la prohibition sont sans aucun doute liées : les enquêtes des industriels sur la vie privée des ouvriers, les services d’inspection créés dans certaines entreprises pour contrôler la « moralité » des ouvriers, sont des nécessités de la nouvelle méthode de travail. Rire de ces initiatives (même si elles ont été un échec) et ne voir en elles qu’une manifestation hypocrite de « puritanisme », c’est se refuser la possibilité de comprendre l’importance, le sens et la \emph{portée objective} du phénomène américain, qui est \emph{aussi} le plus grand effort collectif qui se soit manifesté jusqu’ici pour créer, avec une rapidité prodigieuse et une conscience du but à atteindre sans précédent dans l’histoire, un type nouveau de travailleur et d’homme. L'expression « conscience du but à atteindre » peut paraître au moins humoristique si l’on se souvient de la phrase de Taylor sur le « gorille apprivoisé ». Taylor exprime en effet avec un cynisme brutal le but de la société américaine : développer au plus haut degré chez le travailleur les attitudes machinales et automatiques, briser l’ancien ensemble de liens psycho-physiques du travail professionnel qualifié qui demandait une certaine participation active de l’intelligence, de l’imagination, de l’initiative du travailleur, et réduire les opérations de la production à leur seul aspect physique et machinal. Mais, en réalité, il ne s’agit pas de nouveautés originales, il s’agit seulement de la phase la plus récente d’un long processus qui a commencé avec la naissance de l’industrialisme lui-même, phase qui est seulement plus intense que les précédentes et qui se manifeste sous des formes plus brutales, mais qui sera dépassée elle aussi par la création d’un nouvel ensemble de liens psycho-physiques d’un type différent des précédents et, à coup sûr, d’un type \emph{supérieur.} Il se produira inéluctablement une sélection forcée, une partie de l’ancienne classe ouvrière se trouvera impitoyablement éliminée du monde du travail et peut-être du monde \emph{tout court}\footnote{En français dans le texte}.\par
C'est à ce point de vue qu’il faut étudier les initiatives « puritaines » des industriels américains du type Ford. Il est certain qu’ils ne se souciaient pas de l’ « humanité » et de la « spiritualité » du travailleur, qui sont immédiatement brisées. Cette « humanité », cette « spiritualité » ne peuvent se réaliser que dans le monde de la production et du travail, dans la « création » productive ; elles existaient au plus haut point chez l’artisan, chez le « démiurge\footnote{\emph{Démiurge} : le nom du créateur du monde dans la philosophie grecque ancienne.} », lorsque la personnalité du travailleur se reflétait tout entière dans l’objet créé, lorsque le lien entre l’art et le travail était encore très fort. Mais c’est justement contre cet « humanisme » que le nouvel industrialisme entre en lutte. Les initiatives « puritaines » n’ont pour but que de conserver, en dehors du travail, chez le travailleur, exploité au maximum par la nouvelle méthode de production, un certain équilibre psychophysique qui l’empêche de s’effondrer physiologiquement. Cet équilibre ne peut être que purement extérieur et mécanique, mais il pourra devenir interne s’il est proposé par le travailleur lui-même et non imposé du dehors, s’il est proposé par une nouvelle forme de société, avec des moyens appropriés et originaux. L'industriel américain se préoccupe de maintenir la continuité de l’efficience physique du travailleur, de son efficience musculaire et nerveuse : il est de son intérêt d’avoir une main-d’œuvre stable, toujours enforme dans son ensemble, parce que l’ensemble du personnel (le travailleur collectif) d’une entreprise est une machine qui ne doit pas être trop souvent démontée et dont il ne faut pas trop souvent renouveler les pièces particulières, sans occasionner des pertes énormes.\par
Le fameux « haut salaire » est un élément qui se rattache à cette nécessité : il est l’instrument qui sert à sélectionner une main-d’œuvre adaptée au système de production et de travail, et à la maintenir stable. Mais le haut salaire est un instrument à double tranchant : il faut que le travailleur dépense « rationnellement » son salaire plus élevé, afin de maintenir, de rénover et, si possible, d’accroître son efficience musculaire et nerveuse, et non pour la détruire ou l’amoindrir. Et voilà que la lutte contre l’alcool, le facteur le plus dangereux de destruction des forces de travail, devient une affaire d’Etat. Il est possible que d’autres luttes « puritaines » deviennent elles aussi des fonctions d’Etat, si l’initiative privée des industriels se révèle insuffisante ou si se produit une crise de moralité trop profonde et trop étendue parmi les masses travailleuses, ce qui pourrait se produire à la suite d’une longue et importante crise de chômage.\par
A la question de l’alcool est liée la question sexuelle l’abus et l’irrégularité des fonctions sexuelles est, après l’alcoolisme, l’ennemi le plus dangereux de l’énergie nerveuse et l’on observe couramment que le travail « obsédant » provoque des dépravations alcooliques et sexuelles. Les tentatives faites par Ford d’intervenir, au moyen d’un corps d’inspecteurs, dans la vie privée de ses employés, et de contrôler la façon dont ils dépensent leur salaire et dont ils vivent, est un indice de ces tendances encore « privées » ou latentes, mais qui peuvent devenir, à un certain moment, une idéologie d’Etat, en se greffant sur le puritanisme traditionnel, c’est-à-dire en se présentant comme un renouveau de la morale des pionniers, du « véritable » américanisme, etc. Le fait le plus important du phénomène américain dans ce domaine est le fossé qui s’est creusé, et qui ira sans cesse en s’élargissant, entre la moralité et les habitudes de vie des travailleurs et celles des autres couches de la population.\par
La prohibition a déjà donné un exemple d’untel écart. Qui consommait l’alcool introduit en contrebande aux Etats-Unis ? C'était devenu une marchandise de grand luxe et même les plus hauts salaires ne pouvaient en permettre la consommation aux larges couches des masses travailleuses : celui qui travaille pour un salaire, avec un horaire fixe, n’a pas de temps à consacrer à la recherche de l’alcool, n’a pas le temps de s’adonner aux sports, ni de tourner les lois. On peut faire la même observation pour la sexualité. La « chasse à la femme » exige trop de loisirs. Chez l’ouvrier de type nouveau on verra se répéter, sous une autre forme, ce qui se produit chez les paysans dans les villages. La fixité relative des unions sexuelles paysannes est étroitement liée au système de travail à la campagne. Le paysan qui rentre chez lui le soir après une longue et fatigante journée de travail, veut la \emph{Venerem facilemparabilemque} dont parle Horace\footnote{« \emph{Venerem facilemparabilemque} » (l’amour facile et toujours à ma portée) : citation du poète latin Horace (\emph{Satires}, Livre II, vers 119).} ; il n’est pas disposé à aller tourner autour de femmes rencontrées au hasard ; il aime sa femme parce qu’il est sûr d’elle, parce qu’elle ne se dérobera pas, ne fera pas de manières et ne prétendra pas jouer la comédie de la séduction et du viol pour être possédée. Il semble qu’ainsi la fonction sexuelle soit mécanisée mais il s’agit en réalité de la naissance d’une nouvelle forme d’union sexuelle dépouillée des couleurs « éblouissantes » et du clinquant romantique propres au petit bourgeois et au « bohème » désœuvré. Il apparaît clairement que le nouvel industrialisme veut la monogamie, veut que le travailleur ne gaspille pas son énergie nerveuse dans la recherche désordonnée et excitante de la satisfaction sexuelle occasionnelle : l’ouvrier qui se rend au travail après une nuit de « débauche » n’est pas un bon travailleur ; l’exaltation passionnelle ne peut aller de pair avec les mouvements chronométrés des gestes de la production liés aux automatismes les plus parfaits. Cet ensemble complexe de pressions et de contraintes directes et indirectes exercées sur la masse donnera sans aucun doute des résultats et l’on verra naître une nouvelle forme d’union sexuelle dont la monogamie et la stabilité relative semblent devoir être les traits caractéristiques et fondamentaux.\par
Il serait intéressant de connaître les résultats statistiques concernant les phénomènes de déviation des habitudes sexuelles officiellement préconisées aux Etats-Unis, analysés par groupes sociaux : on constatera que, de façon générale, les divorces sont particulièrement nombreux dans les classes supérieures. Cet écart entre la moralité des masses travailleuses et celle d’éléments toujours plus nombreux des classes dirigeantes, aux Etats-Unis, semble être un des phénomènes les plus intéressants et les plus riches de conséquences. Jusqu’à ces derniers temps le peuple américain était un peuple de travailleurs : cette « vocation travailleuse » n’était pas seulement un caractère propre à la classe ouvrière, c’était aussi une qualité spécifique des classes dirigeantes. Le fait qu’un milliardaire continue pratiquement à travailler jusqu’à ce que la maladie ou la vieillesse l’oblige à se reposer, que son activité s’étende sur un très grand nombre d’heures de la journée, voilà des phénomènes typiquement américains, voilà le phénomène américain le plus stupéfiant pour l’Européen moyen. On a remarqué précédemment que cette différence entre les Américains et les Européens est due a l’absence de « traditions » aux Etats-Unis, dans la mesure où tradition signifie également résidu passif de toutes les formes sociales périmées de l’histoire. Par contre il existe aux Etats-Unis, toute récente, la « tradition » des pionniers, c’est-à-dire des fortes individualités chez qui la « vocation laborieuse » avait atteint la plus grande intensité et la plus grande vigueur, d’hommes qui, directement et non par l’intermédiaire d’une armée d’esclaves et de serviteurs, entraient en contact, de façon énergique, avec les forces naturelles pour les dominer et les exploiter victorieusement. Ce sont ces résidus passifs, qui, en Europe, résistent à l’américanisme (« ils représentent, la qualité », etc.) car ils sentent instinctivement que les nouvelles formes de production et de travail les balaieraient implacablement. Mais, s’il est vrai qu’en Europe, dans ce cas, les vieilleries qui ne sont pas encore enterrées seraient définitivement détruites, que voit-on se produire en Amérique même ? La différence de moralité dont nous avons parlé montre que sont en train de se créer des marges de passivité sociale sans cesse plus vastes. Il semble que les femmes jouent un rôle dominant dans ce phénomène. L'homme, l’industriel, continue à travailler même s’il est milliardaire, mais sa femme et ses filles tendent de plus en plus à être des « mammifères de luxe ». Les concours de beauté, les concours pour être actrice de cinéma (se rappeler qu’en 1926 30 000 jeunes Italiennes ont envoyé leur photographie en maillot de bain à la Fox\footnote{Une des grandes compagnies de production cinématographique de Hollywood. Sur l’expansion économique de Hollywood à cette époque, et sur le « star system », voir Georges Sadoul : \emph{Histoire d’un art, le cinéma.} Paris, Flammarion, chap. XX, et, du même auteur : \emph{Les Merveilles du cinéma}, Paris, Editeurs Français Réunis, 1957, chap. IV, pp. 72-74.}, de théâtre, etc., en sélectionnant la beauté féminine dans le monde et la mettant aux enchères, font naître une mentalité de prostitution ; c’est la « traite des blanches » devenue légale pour les classes supérieures. Les femmes oisives voyagent, traversent continuellement l’océan pour venir en Europe, échappent à la prohibition de leur patrie et contractent des « mariages » saisonniers (rappelons que les capitaines de marine américains se sont vu retirer le droit de célébrer des mariages à bord, car de nombreux couples se mariaient à leur départ d’Europe et divorçaient avant de débarquer en Amérique) : c’est la prostitution réelle qui se répand, à peine masquée sous de fragiles formalités juridiques.\par
Ces phénomènes propres aux classes supérieures rendront plus difficile l’exercice de la contrainte sur les masses travailleuses pour les rendre conformes aux besoins de la nouvelle industrie ; en tout cas ils déterminent une rupture psychologique et accélèrent la cristallisation et la saturation des groupes sociaux, en rendant évidente leur transformation en castes comme cela s’est produit en Europe. (\emph{Mach.}, pp. 329-334.)
\section[{Taylorisation et mécanisation du travailleur}]{Taylorisation et mécanisation du travailleur}
\noindent A propos de l’écart que le taylorisme déterminerait entre le travail manuel et le « contenu humain » du travail, on peut faire des observations utiles sur le passé, et particulièrement en ce qui concerne ces professions que l’on considère comme les « plus intellectuelles », c’est-à-dire celles qui sont liées à la reproduction des écrits en vue de la publication, ou de toute autre forme de diffusion et de transmission : les copistes d’avant l’invention de l’imprimerie, les typographes, les linotypistes, les sténographes, les dactylos. Si l’on y réfléchit, on s’aperçoit que, dans ces métiers, l’adaptation à la mécanisation est plus difficile que dans les autres. Pourquoi ? Parce qu’il est difficile d’atteindre au sommet de la qualification professionnelle, qui exige que l’ouvrier « oublie » le contenu intellectuel de l’écrit qu’il reproduit, ou qu’il n’y réfléchisse pas, pour ne fixer son attention que sur la calligraphie de chaque lettre s’il est copiste, ou pour décomposer les phrases en mots « abstraits » et ceux-ci en caractères d’imprimerie, choisir rapidement les morceaux de plomb dans les casses, pour décomposer non seulement chaque mot, mais des groupes de mots, dans le texte d’un discours, pour les grouper mécaniquement en abréviations sténographiques, ou pour obtenir la rapidité chez la dactylo, etc. L'intérêt que porte le travailleur au contenu intellectuel du texte se mesure a ses erreurs, autrement dit il constitue une déficience professionnelle : sa qualification se mesure précisément à son désintéressement intellectuel, c’est-à-dire à sa mécanisation. Le copiste du moyen âge qui s’intéressait au texte changeait l’orthographe, la morphologie, la syntaxe du texte qu’il recopiait, négligeait des passages entiers que sa faible culture ne lui permettait pas de comprendre ; le cours des idées que faisait naître en lui l’intérêt qu’il portait au texte, l’amenait à y intercaler des commentaires et des observations ; si son dialecte ou sa langue étaient différents de ceux du texte, il y introduisait des nuances étrangères ; c’était un mauvais copiste car en réalité il « refaisait » le texte. La lenteur de l’écriture médiévale (qui était aussi un art) explique bon nombre de ces déficiences : on avait trop de temps pour réfléchir et par conséquent la « mécanisation » était plus difficile. Le typographe, lui, doit être très rapide, ses mains et ses yeux doivent être sans cesse en mouvement, ce qui rend plus facile sa mécanisation. Mais, si l’on y réfléchit, l’effort que doivent faire ces travailleurs placés devant un texte dont le contenu est parfois passionnant (en effet dans ces cas-là on travaille moins vite et plus mal), pour n’en considérer que la graphie et ne s’attacher qu’à celle-ci, est peut-être le plus grand effort que l’on puisse exiger d’un métier. Cependant cet effort, l’homme l’accomplit sans tuer pour autant sa vie spirituelle. Une fois que l’adaptation s’est faite, on constate en réalité que le cerveau de l’ouvrier, loin de se momifier, atteint au contraire un état de complète liberté. Ce qui a été complètement mécanisé, c’est seulement le geste physique ; la mémoire du métier, réduit à des gestes simples répétés à une cadence très grande, a « fait son nid » dans les faisceaux musculaires et nerveux, laissant le cerveau libre et dégagé pour se livrer à d’autres occupations. Lorsqu’on marche, on n’a pas besoin de réfléchir à tous les mouvements nécessaires pour faire agir en synchronisme toutes les parties du corps d’une certaine façon : c’est de la même façon que l’on fait et que l’on continuera à faire, dans l’industrie, les gestes fondamentaux du métier. On marche automatiquement et l’on pense, en même temps, à tout ce que l’on veut. Les industriels américains ont fort bien compris cette dialectique propre aux nouvelles méthodes industrielles. Ils ont compris que le « gorille apprivoisé » n’est qu’une façon de parler, que l’ouvrier n’en reste pas moins, « malheureusement », un homme, et même que pendant son travail il réfléchit davantage, ou il a du moins une plus grande possibilité de penser, une fois qu’il a surmonté la crise de l’adaptation sans avoir été éliminé. Et non seulement il pense, mais le fait qu’il ne retire pas de son travail des satisfactions immédiates, et qu’il comprend qu’on veut le réduire à n’être qu’un « gorille apprivoisé », peut l’amener à avoir des idées peu conformistes. Qu'une telle préoccupation existe chez les industriels, c’est ce que nous montre toute la série de précautions et d’initiatives « éducatives » que l’on peut relever dans les livres de Ford et dans l’œuvre de Philip. (\emph{Mach.}, pp. 336-337.)
\chapterclose


\chapteropen
\chapter[{Passé et présent}]{Passé et présent}\renewcommand{\leftmark}{Passé et présent}


\chaptercont
\section[{ Le travailleur collectif}]{ Le travailleur collectif}
\noindent Dans l’exposé critique des événements qui ont suivi la guerre et des tentatives constitutionnelles (organiques)pour sortir de l’état de désordre et de dispersion des forces, montrer comment le mouvement pour revaloriser l’usine\footnote{Le mouvement des Conseils d’usine, lancé par le groupe \emph{L'Ordine nuovo.}}, en opposition avec l’organisation professionnelle (ou mieux de façon autonome) correspondait parfaitement à l’analyse du développement du système de l’usine qui est faite dans le I° volume de la \emph{Critique de l’économie politique}\footnote{\emph{Critique del’économie politique} est le sous-titre du \emph{Capital} de Karl Marx.}. Une division du travail toujours plus parfaite réduisant objectivement la fonction du travailleur dans l’usine à l’exécution de travaux de détails toujours plus « analytiques », de façon que la complexité de l’œuvre commune échappe à chacun d’eux pris en particulier et que, dans sa conscience elle-même, sa propre contribution ait si peu de prix qu’elle lui semble facilement remplaçable à chaque instant ; en même temps, le travail bien élaboré et bien organisé aboutissant à une plus grande productivité « sociale », enfin l’assimilation nécessaire de l’ensemble des travailleurs d’une usine à un « travailleur collectif » : telles sont les données implicites et nécessaires du mouvement des Conseils d’usine qui tend à rendre « subjectif » ce qui est donné « objectivement ». Mais dans ce cas, que veut dire objectif ? Pour le travailleur isolé « objectif » veut dire la rencontre des exigences du développement technique avec les intérêts de la classe dominante. Mais cette rencontre, cette unité entre le développement technique et les intérêts de la classe dominante n’est qu’une phase historique du développement de l’industrie et elle doit être conçue comme transitoire. Ce lien peut disparaître ; non seulement l’exigence technique peut être conçue de façon concrète, indépendamment des intérêts de la classe dominante, mais elle peut être unie aux intérêts de la classe qui est encore subalterne. Qu'une telle « scission », qu’une nouvelle synthèse soient historiquement mûres, cela est démontré de façon décisive par le fait même qu’un tel processus est compris par la classe subalterne et c’est précisément pour cela que celle-ci n’est plus subalterne, ou encore qu’elle manifeste la tendance à sortir de sa condition de classe subordonnée. Le « travailleur collectif » se conçoit comme tel, non seulement dans chaque usine particulière, mais dans des domaines plus vastes de la division du travail national et international et il est une manifestation extérieure, politique, de cette conscience qu’il a acquise précisément dans les organismes qui représentent l’usine comme productrice d’objets réels et non de profit. (P.P., pp. 78-79.)\par
{\raggedleft \noindent [1932]}
\section[{Bâtisseurs de greniers}]{Bâtisseurs de greniers}
\noindent On peut juger une génération d’après le jugement qu’elle-même donne de la génération précédente ; une période historique d’après la façon dont elle-même considère la période qui l’a précédée. Une génération qui abaisse la génération précédente, qui ne réussit pas à en voir les grandeurs et la signification nécessaire, ne peut qu’être mesquine et sans confiance en soi, même si elle adopte des poses de gladiateur et brûle de désirs de grandeur. C'est le rapport habituel du grand homme et du valet de chambre\footnote{(Pas de grand homme pour son valet de chambre.) Hegel est revenu plusieurs fois sur cette locution proverbiale, dont il conteste la portée : le point de vue du valet de chambre ne prouve pas la petitesse du héros, mais seulement celle du valet. Plus généralement l’historien empirique, anecdotique, déforme l’histoire en n’en retenant que les petits côtés.}. Faire le désert pour émerger et se distinguer : une génération vivante et forte qui se propose de travailler et de s’affirmer, tend au contraire à surestimer la génération précédente, car sa propre énergie lui donne l’assurance qu’elle ira encore plus loin ; végéter simplement, c’est déjà dépasser ce qui est décrit comme mort.\par
On reproche au passé de ne pas avoir accompli la tâche du présent : tout serait bien plus facile si les parents avaient déjà fait le travail des enfants. Il y a une justification implicite de la nullité du présent dans la dépréciation du passé : que n’aurions-nous pas fait, nous, si nos parents avaient fait ceci et cela… mais ils ne l’ont pas fait et, par conséquent, nous n’avons rien fait de plus. Un grenier sur un rez-de-chaussée est-il moins grenier qu’un grenier sur un dixième ou un trentième étage ? Une génération qui ne sait que bâtir des greniers se plaint de ce que les prédécesseurs n’aient pas déjà construit des palais de dix ou trente étages. Vous vous dites capables de bâtir des cathédrales, mais vous n’êtes capables que de construire des greniers.\par
Différence avec le \emph{Manifeste}\footnote{\emph{Le Manifeste du Parti communiste}, de Marx et Engels.}, qui exalte la grandeur de la classe qui va mourir. (P.P., pp. 102-103.)\par
{\raggedleft \noindent [1931-1932]}
\section[{Enquêtes sur les jeunes}]{Enquêtes sur les jeunes}
\noindent Enquête « sur la nouvelle génération » publiée dans la \emph{Fiera letteraria} du 2 décembre1928 au 17 février 1929.\par
Pas très intéressante. Les professeurs d’université connaissent mal les jeunes étudiants. Le refrain le plus fréquent est le suivant : les jeunes ne se consacrent plus aux recherches et aux études désintéressées, mais visent le gain immédiat. Agostino Lanzillo répond : « Aujourd’hui \emph{surtout} nous ne connaissons ni l’état d’esprit des jeunes ni leurs sentiments. Il est difficile de gagner leur esprit ; ils se taisent très \emph{volontiers} sur les problèmes culturels, sociaux et moraux. Est-ce méfiance ou manque d’intérêt ? » (\emph{Fiera letteraria}, 9 décembre 1928). Cette réponse de Lanzillo est l’unique remarque réaliste de l’enquête. Lanzillo remarque encore : « Une discipline de fer et une situation de paix intérieure et extérieure se développent dans le travail concret et réalisateur, mais elles ne permettent pas le déchaînement de conceptions politiques ou morales opposées. Il manque une arène aux jeunes pour s’agiter, pour manifester des formes exubérantes de passions et de tendances. De cette situation naît et dérive une attitude froide et silencieuse qui est une promesse mais qui contient aussi des \emph{inconnues. »}\par
La réponse de Giuseppe Lombardo-Radice dans le même numéro de la \emph{Fiera letteraria} est intéressante : « Les jeunes d’aujourd’hui ont peu de \emph{patience} pour les études scientifiques et historiques ; très peu affrontent un travail qui exige une longue préparation et présente des difficultés de recherche. Ils veulent en général \emph{se débarrasser} des études ; ils cherchent avant tout à s’établir rapidement et se détournent des recherches désintéressées en aspirant \emph{à gagner} et en répugnant aux carrières qui leur semblent trop lentes. Malgré toute la « philosophie » ambiante, leur intérêt spéculatif est pauvre ; leur culture est faite de fragments ; ils discutent peu, rarement ils se divisent en groupes et en cénacles dont l’enseigne soit une idée philosophique ou religieuse. Ils ont à l’égard des grands problèmes une attitude sceptique ou une attitude de respect tout à fait extérieur pour ceux qui les prennent au sérieux, ou une attitude \emph{d’adoption passive d’un « verbe doctrinal ». } « En général, les mieux disposés spirituellement sont les étudiants \emph{les plus pauvres »} et « les riches sont, le plus souvent, inquiets, impatientés par la discipline des études, pressés. Ce n’est pas d’eux que sortira une classe spirituellement capable de diriger notre pays ».\par
Ces notes de Lanzillo et de Lombardo-Radice sont les seules parties sérieuses de l’enquête, à laquelle n’ont d’ailleurs presque exclusivement participé que des professeurs de lettres. La plupart ont répondu par des « actes de foi » et non par des constatations objectives, ou bien ils ont avoué ne pas pouvoir répondre.\par
On trouve dans la \emph{Civiltà Cattolica} du 20 mai 1933 un bref résumé des « Conclusions de l’enquête sur la nouvelle génération\footnote{Extrait du fascicule 28 du « Saggiatore », Rome 1933, \emph{in 8°}, p. 32 \emph{(Note de Gramsci).}} « . On sait combien ces enquêtes sont nécessairement unilatérales et tendancieuses et combien elles donnent habituellement raison à la façon de penser de ceux qui les ont proposées. Il faut être d’autant plus prudent qu’il semble difficile aujourd’hui de connaître ce que pensent et veulent les nouvelles générations. Selon la \emph{Civiltà Cattolica}, la substance de l’enquête serait : « La nouvelle génération serait donc sans morale et sans principes immuables de moralité sans religiosité et même athée, avec peu d’idées et beaucoup d’instinct. » « La génération d’avant-guerre croyait en et était dominée par les idées de justice, de bien, de désintéressement et de religion ; la spiritualité moderne s’est débarrassée de ces idées, qui sont immorales en pratique. Les petits faits de la vie exigent souplesse et élasticité morales, que l’on commence à obtenir avec la liberté d’esprit de la nouvelle génération. Tous les principes moraux qui se sont imposés comme des axiomes aux consciences individuelles perdent leur valeur dans la nouvelle génération. La morale est devenue absolument pragmatique, elle jaillit de la vie pratique, des différentes situations dans lesquelles l’homme vient à se trouver. La nouvelle génération n’est ni spiritualiste, ni positiviste, ni matérialiste, elle tend à dépasser rationnellement aussi bien les attitudes spiritualistes que les positions positivistes et matérialistes surannées. Sa caractéristique principale est le manque de quelque forme de respect que ce soit pour tout ce qui incarne le vieux monde. Le sens religieux et l’ensemble des divers impératifs moraux abstraits, désormais inadaptés à la vie d’aujourd’hui, se sont affaiblis dans la masse des jeunes. Les très jeunes ont moins d’idées et plus de vie, ils ont en revanche acquis naturel et confiance dans l’acte sexuel, si bien que l’amour n’est plus conçu comme, un péché, une transgression, une chose prohibée. Les jeunes, activement orientés vers les directions qu’indique la vie ! moderne, apparaissent à l’abri de tout retour possible à une religiosité dogmatique et dissolvante. »\par
Cette série d’affirmations n’est, semble-t-il, ni plus ni moins que le programme du « Saggiatore » lui-même et il s’agit là, je crois, plus d’une curiosité que d’une chose sérieuse. C'est, au fond, une reprise populaire du thème du « surhomme », issue des expériences les plus récentes de la vie nationale, un « surhomme » « \emph{strapaesien}\footnote{\emph{Strapaese} (super-pays ou super-village) : était un mouvement littéraire de tendance traditionaliste s’opposant au mouvement \emph{Stracittà} (super-ville) qui se voulait plus moderne.} « bon pour les milieux distingués et la pharmacie philosophique. Si l’on y réfléchit, cela signifie que la nouvelle génération est devenue, sous l’aspect d’un volontarisme extrême, d’une très grande aboulie. Il n’est pas vrai qu’elle n’ait pas d’idéaux : seulement ceux-ci sont tous contenus dans le code pénal que l’on suppose fait une fois pour toutes dans son ensemble. Cela signifie aussi qu’il manque dans le pays une direction culturelle en dehors de la direction catholique, ce qui laisserait supposer que l’hypocrisie religieuse doit, à tout le moins, finir par s’accroître. Il serait toutefois intéressant de savoir de quelle nouvelle génération veut parler le « Saggiatore ».\par
Il semble que l’ « originalité » du « Saggiatore » consiste dans le fait d’avoir appliqué à la « vie » le concept d’ « expérience » propre, non pas à la science, mais à l’opérateur de laboratoire scientifique. Les conséquences de cette transposition mécanique ne sont guère brillantes ; elles correspondent à ce qui était bien connu sous le nom d’ « opportunisme » ou de manque de principes (rappeler certaines interprétations journalistiques de la relativité d’Einstein, lorsqu’en 1921 cette théorie devint la proie des journalistes). Le sophisme consiste en ceci : lorsque l’opérateur de laboratoire fait « l’épreuve et la contre-épreuve », son épreuve a des conséquences limitées à l’espace des éprouvettes et des alambics : il « éprouve » hors de soi, sans donner autre chose de lui-même dans l’expérience que l’attention physique et intellectuelle. Mais les choses se passent bien différemment dans les rapports entre les hommes, et les conséquences ont une tout autre portée. L'homme transforme le réel et ne se limite pas à l’examiner \emph{in vitro} pour en reconnaître les lois de régularité abstraite. On ne déclare pas une guerre pour « faire une expérience », ni on ne détruit l’économie d’un pays, etc., pour trouver les lois du meilleur ordre social possible. Qu'il faille pour construire ses propres plans de transformation de la vie se fonder sur l’expérience, c’est-à-dire sur l’importance exacte des rapports sociaux existants et non pas sur des idéologies vides ou sur des généralités rationnelles, n’implique pas que l’on ne doit pas avoir de principes, qui ne sont rien d’autre que les expériences mises sous forme de concepts ou de normes impératives. La philosophie du « Saggiatore », réaction plausible aux excès actualistes et religieux, est pourtant aussi essentiellement liée à des tendances conservatrices et passives et témoigne en réalité du plus haut « respect » pour ce qui existe, c’est-à-dire pour le passé cristallisé. On trouve dans un article de Giorgio Granata (« Saggiatore », rapporté dans \emph{Critica Fascista} du 1° mai 1933) de nombreux traits de cette philosophie : pour Granata, la conception du « parti politique » avec son « programme » utopique, « comme monde du devoir-être (!) par opposition au monde de l’être, de la réalité », a fait son temps, et pour cette raison la France serait « inactuelle » : comme si précisément la France n’avait pas toujours donné au XIX° siècle l’exemple de l’opportunisme politique le plus plat, c’est-à-dire de la servilité envers ce qui existe, envers la réalité, c’est-à-dire envers les « programmes » en acte de forces bien déterminées et identifiables. Et l’asservissement aux faits voulus et réalisés par d’autres est le vrai point de vue du « Saggiatore », c’est-à-dire l’indifférence et l’aboulie sous couleur d’une grande activité de fourmis : la philosophie de l’homme de Guichardin qui réapparaît toujours à certaines périodes de la vie italienne. Que pour ce résultat on ait dû se référer à Galilée et reprendre le titre de « Saggiatore », n’est qu’une belle impudence, et il est à parier que Messieurs Granata et Cien’ont à craindre ni de nouveaux bûchers ni de nouvelles inquisitions. La conception du « parti politique » de Granata coïncide d’ailleurs avec la conception qu’exprime Croce dans le chapitre « Le parti comme jugement et préjugé » de son livre \emph{Culture et vie morale}, ainsi qu’avec le « programme » de l’\emph{Unita} florentine, problémiste\footnote{L'\emph{Unita} de Salvemini.}, etc.\par
Et pourtant ce groupe du « Saggiatore » mérite d’être étudié et analysé :\par

\begin{enumerate}[itemsep=0pt,]
\item parce qu’il cherche à exprimer, quoique grossièrement, des tendances qui sont répandues et vaguement conçues par un grand nombre de gens ; 
\item parce qu’il est indépendant de tout « grand philosophe » traditionnel, et s’oppose même à toute tradition cristallisée ; 
\item parce qu’un grand nombre des affirmations de ce groupe sont indubitablement des répétitions approximatives de positions philosophiques de la philosophie de la praxis entrées dans la culture générale, etc.
\end{enumerate}

\noindent Rappeler le « en prouvant et en reprouvant » du député Giuseppe Canepa en tant que commissaire aux approvisionnements pendant la guerre : ce Galilée de la science administrative avait besoin d’une expérience avec des morts et des blessés pour savoir que là où manque le pain, le sang coule. (P.P., pp. 104-107.)\par
{\raggedleft \noindent [1933]}
\section[{ L'histoire maîtresse de la vie, les leçons del’expérience, etc.}]{ L'histoire maîtresse de la vie, les leçons del’expérience, etc.}
\noindent Benvenuto Cellini [\emph{Vie}, livre deuxième, dernières phrases du paragraphe XVII.] écrit aussi :\par

\begin{quoteblock}
 \noindent « Il est bien vrai que l’on dit : tu apprendras pour une autre occasion. Cela n’a pas de sens car la [fortune] survient toujours selon des modalités différentes et jamais imaginées. »
\end{quoteblock}

\noindent On peut dire, sans doute, que l’histoire est la maîtresse de la vie et que l’expérience comporte un enseignement ; mais onne peut pas le dire pour signifier qu’il est possible de tirer de la façon dont s’est développé un complexe d’événements, un principe sûr d’action et de conduite pour des événements semblables ; on ne peut le dire qu’au sens où la production des événements réels étant le résultat d’un concours contradictoire de forces, il faut chercher à être la force déterminante. Ce qui doit être compris en plusieurs sens, car on peut être la force déterminante non seulement parce qu’on est la force qui prévaut quantitativement (ce qui n’est ni toujours possible, ni toujours réalisable), mais parce qu’on est la force qui prévaut qualitativement. Et l’on peut être cette force si l’on a l’esprit d’initiative, si on saisit le « bon moment », si on maintient un continuel état de tension de la volonté de façon à être en mesure de démarrer au moment choisi (sans avoir besoin de longs préparatifs qui laissent passer l’instant le plus favorable, etc.). On trouve un aspect de cette façon de considérer les choses dans l’aphorisme selon lequel la meilleure tactique défensive est l’offensive. Nous sommes toujours sur la défensive contre le « hasard », c’est-à-dire contre le concours prévisible de forces opposées qui ne peuvent pas toujours être toutes identifiées (et négliger une seule de ces forces empêche de prévoir la combinaison effective des forces qui donne toujours leur originalité aux événements) et nous pouvons « offenser » le hasard au sens où nous intervenons activement dans sa production, où, de notre point de vue, nous le rendons moins « hasard », moins « nature » et davantage effet de notre activité et de notre volonté. (P.P., pp. 107-108.)\par
{\raggedleft \noindent [1932]}
\section[{« Rationalisme ». Concept romantique de l’inventeur}]{« Rationalisme ». Concept romantique de l’inventeur}
\noindent Selon ce concept est innovateur celui qui veut détruire tout l’existant, sans se soucier de ce qui arrivera ensuite puisque, c’est bien connu, métaphysiquement toute destruction estcréation, et même on ne détruit que ce qu’on remplace par une nouvelle création. A ce concept romantique se joint un concept rationnel ou « illuministe ». On pense que tout ce qui existe est un « piège » tendu par les forts aux faibles, par les malins aux pauvres d’esprit. Le danger vient du fait que, « du point de vue illuministe », ces mots sont pris à la lettre, matériellement. La philosophie de la praxis est contre cette façon de voir. La vérité est au contraire : toute chose qui existe est rationnelle, c’est-à-dire qu’elle a eu ou qu’elle a une fonction utile. Que ce qui existe ait existé, c’est-à-dire ait eu sa raison d’être en tant que « conforme » au mode de vie, de pensée, d’action de la classe dirigeante, ne signifie pas que ce soit devenu « irrationnel » parce que la classe dominante a été privée du pouvoir et de sa force de donner impulsion à toute la société. Une vérité que l’on oublie : ce qui existe a eu sa raison d’exister, a servi, a été rationnel, a « facilité » le « développement historique » et la vie. Qu'à partir d’un certain point cela n’ait plus eu lieu, que de modalités du progrès qu’elles étaient telles formes de vie soient devenues un empêchement et un obstacle, c’est vrai, mais n’est pas vrai « sur toute la ligne » : c’est vrai là où c’est vrai, c’est-à-dire dans les formes de vie les plus hautes, celles qui sont décisives, celles qui marquent la pointe du progrès, etc. Mais la vie ne se développe pas de façon homogène, elle se développe au contraire par des avancées partielles, de pointe, elle se développe pour ainsi dire par croissance « pyramidale ». De tout mode de vie il importe donc d’étudier l’histoire, c’est-à-dire l’originaire « rationalité » puis, celle-ci reconnue, se poser la question pour chaque cas pris à part : cette rationalité existe-t-elle encore, pour autant qu’existent encore les conditions sur lesquelles cette rationalité se fondait ? Le fait auquel, au contraire, on ne prête pas assez attention est le suivant : les modes de vie apparaissent à qui les vit comme absolus, « comme naturels », comme on dit, et c’est déjà énorme d’en montrer l’ « historicité », de démontrer qu’ils sont justifiés dans la mesure où existent telles conditions, mais qu’une fois changées ces conditions ils ne sont plus justifiés mais « irrationnels ». C'est pourquoi la discussion de telles façons de vivre et d’agir prend un caractère odieux, persécuteur, devient une affaire d’ « intelligence » ou de « stupidité », etc. Intellectualisme, illuminisme pur, qu’il importe de combattre sans répit.\par
On en déduit :\par

\begin{enumerate}[itemsep=0pt,]
\item que tout fait a été « rationnel »
\item qu’il est à combattre pour autant qu’il n’est plus rationnel, c’est-à-dire qu’il n’est plus conforme au but mais se traîne par la viscosité de l’habitude ; 
\item qu’il ne faut pas croire, parce qu’une façon de vivre, d’agir, de penser, est devenue irrationnelle dans un milieu donné, qu’elle soit devenue irrationnelle partout et pour tous, et que seule la méchanceté ou la bêtise la maintiennent en vie ; 
\item que pourtant, le fait qu’une façon de vivre, de penser, d’agir, soit devenue irrationnelle quelque part aune grande importance - c’est vrai, et il importe de le mettre en lumière par tous les moyens : c’est ainsi qu’on commence à modifier les coutumes, en introduisant une forme de pensée « historiciste » qui facilitera les changements effectifs dès que les convictions seront changées, qui autrement dit rendra moins « visqueuse » la coutume routinière.
\end{enumerate}

\noindent Un autre point à préciser : qu’une façon de vivre, d’agir, de penser se soit introduite dans toute la société parce qu’elle appartient proprement à la classe dirigeante, cela ne signifie pas qu’elle soit parelle-même irrationnelle et à rejeter. Si l’on y regarde de près, on voit que dans tout fait existent deux aspects : l’un « rationnel », c’est-à-dire conforme au but ou « économique », et l’autre relevant de la « mode », qui est une façon d’être déterminée du premier aspect rationnel. Porter des chaussures est rationnel, mais la forme déterminée de la chaussure est due à la mode. Porter le faux-col est rationnel, parce que cela permet de changer souvent cette partie du vêtement « chemise » qui se salit plus facilement, mais la forme du faux-col dépend de la mode, etc. On voit en somme que la classe dirigeante en « inventant » une utilité nouvelle, plus économique ou plus conforme aux conditions données ou au but donné a en même temps donné une « sienne » forme particulière à l’invention et à l’utilité nouvelle.\par
C'est penser avec des oeillères que de confondre l’utilité permanente (dans la mesure où permanence il y a) avec la mode. Au contraire la tâche du moraliste et du créateur de coutumes est d’analyser les façons d’être et de vivre, de les critiquer en séparant le permanent, l’utile, le rationnel, ce qui est conforme au but (dans la mesure où ce but subsiste) de ce qui est accidentel, de ce qui est snobisme, de ce qui est singerie, etc. Sur la base du « rationnel », il peut être utile, de créer une mode originale, c’est-à-dire une forme neuve qui intéresse.\par
Que la forme de pensée critiquée ne soit pas juste, cela se voit au fait qu’elle a des limites : par exemple personne (à moins d’être fou) n’ira prêcher qu’il ne faut plus apprendre à lire et à écrire, parce que la lecture et l’écriture ont assurément été introduites par la classe dirigeante, parce que l’écriture sert à diffuser certaine littérature ou à écrire les lettres de chantage ou les rapports des mouchards. (P.P., pp. 175-177.)\par
{\raggedleft \noindent [1932-1933]}
\section[{ Naturel, contre-nature, artificiel, etc.}]{ Naturel, contre-nature, artificiel, etc.}
\noindent Quand on dit que telle action, telle façon de vivre, telles mœurs sont « naturelles », ou qu’au contraire elles sont « contre-nature », qu’est-ce que cela signifie ? Chacun, dans son for intérieur, s’imagine le savoir exactement mais si l’on demande une réponse explicite et motivée on voit bien que la chose n’est pas si facile qu’on pouvait le croire. Pour commencer, qu’il soit bien entendu qu’on ne peut parler de « nature » comme de quelque chose de fixe, d’immuable, d’objectif. Il apparaît que presque toujours « naturel » signifie « juste » et « normal » selon notre conscience historique actuelle ; mais la plupart des gens n’ont pas conscience de cette actualité historiquement déterminée, ils tiennent leur façon de penser pour éternelle et immuable.\par
A remarquer, dans quelques groupes fanatiques de la « naturalité » l’opinion suivante : des actes qui paraissent « contre-nature » à notre conscience sont à leurs yeux « naturels » parce que les animaux agissent ainsi - et les animaux ne sont-ils pas « les êtres les plus naturels du monde » ? Opinion qu’on entend souvent exprimer, dans certains milieux à propos surtout des questions concernant les rapports sexuels. Par exemple : pourquoi l’inceste serait-il « contre-nature » s’il est répandu dans la « nature » ? D'abord, même quand il s’agit d’animaux, de telles affirmations ne sont pas toujours exactes, parce que les observations portent sur des animaux domestiqués par l’homme pour son propre avantage et contraints à une forme de vie qui pour les animaux eux-mêmes n’est pas « naturelle » mais conforme aux fins de l’homme. Mais quand il serait vrai que de tels actes se produisent parmi les animaux, en quoi cela aurait-il une signification pour l’homme ? Pourquoi devrait-il en résulter une règle de conduite ? La « nature » de l’homme est l’ensemble des rapports sociaux qui détermine une conscience historiquement définie ; seule cette conscience peut indiquer ce qui est « naturel » ou « contre-nature ». De plus, l’ensemble des rapports sociaux est contradictoire à chaque instant, et en continuel développement, si bien que la « nature » de l’homme n’est pas quelque chose d’homogène pour tous les hommes de tous les temps.\par
On a entendu souvent dire que telle habitude est devenue « une seconde nature » ; mais la « première nature » était-elle à proprement parler la « première » ? Cette façon de parler n’implique-t-elle pas que le sens commun entrevoit l’historicité de la « nature humaine » ? Dès qu’on a constaté que, l’ensemble des rapports sociaux étant contradictoire, la conscience des hommes ne peut pas ne pas être contradictoire, un problème se pose : comment se manifeste cette contradiction et comment l’unification peut-elle être progressivement obtenue ? La contradiction se manifeste dans l’ensemble du corps social par l’existence de consciences historiques de groupe (par l’existence de stratifications correspondant à diverses phases de développement historique de la civilisation et par des antithèses dans les groupes qui correspondent à un même niveau historique) ; elle se manifeste dans les individus pris à part comme reflet de cette désagrégation « verticale et horizontale ». Dans les groupes subalternes l’absence d’autonomie dans l’initiative historique aggrave la désagrégation et renforce la lutte pour se libérer des principes imposés et non proposés, pour atteindre une conscience historique autonome : dans une telle lutte, les points de repère sont ! disparates, et l’un d’eux, précisément la « naturalité », la « nature » posée en modèle rencontre un grand succès parce qu’il paraît évident et simple. Comment devrait se former, au contraire, cette conscience historique proposée de façon autonome ? Comment chacun devrait-il choisir et combiner les éléments nécessaires à la constitution d’une telle conscience autonome ?\par
Faudra-t-il rejeter \emph{a priori} tout élément « imposé » ? Il faudra le rejeter en tant qu’imposé, non en soi, c’est-à-dire qu’il faudra lui donner une forme nouvelle qui soit propre au groupe considéré. En effet : que l’instruction soit obligatoire, cela ne signifie pas qu’elle soit à rejeter, ni non plus que ne puisse être justifiée, avec de nouvelles raisons, une forme nouvelle d’obligation : il faut faire « liberté » de ce qui est « nécessité » mais il faut pour cela reconnaître une nécessité « objective », c’est-à-dire objective principalement pour le groupe en question. A cet effet, on doit se référer aux rapports techniques de production, à un type déterminé de civilisation économique dont le développement exige un mode de vie déterminée, des règles de conduite déterminées, certaines mœurs. Il faut se persuader que ce qui est « objectif » et nécessaire, ce n’est pas seulement tel outillage, mais aussi tel comportement, telle éducation, telle façon de vivre ensemble, etc. C'est sur cette objectivité et nécessité historique (qui par ailleurs n’est pas évidente : quelqu’un doit la reconnaître de façon critique, s’en faire le défenseur de façon complète et presque « capillaire » ) qu’on peut fonder l’ « universalité » du principe moral ; bien plus, il n’y a jamais eu d’autre universalité que cette nécessité objective de la technique sociale, même si elle est interprétée au moyen d’idéologies transcendantes ou transcendantales et présentée chaque fois sous la forme historiquement la plus efficace pour que soit atteint le résultat voulu.\par
Une conception comme celle qui vient d’être exposée paraît conduire à une forme de relativisme et par suite de scepticisme moral. Remarquons que l’on peut en dire autant de toutes les conceptions jusqu’ici élaborées par la philosophie : leur impérativité catégorique et objective a toujours été susceptible d’être réduite par la « mauvaise volonté » à des formes de relativisme et de scepticisme. Pour que la conception religieuse puisse apparaître absolue et objectivement universelle, elle devrait au moins se présenter comme monolithique, en tout cas intellectuellement uniforme chez tous les croyants, ce qui est très loin de la réalité (différences d’écoles, sectes, tendances, et différences de classes : simples et cultivés, etc.) : d’où la fonction du pape comme maître infaillible. On peut en dire autant de l’impératif catégorique de Kant : « Agis comme tu voudrais voir agir tous les hommes dans les mêmes circonstances\footnote{Kant écrit exactement : « Agis comme si la maxime de ton action devait être érigée par ta volonté en loi universelle de la nature. » Gramsci cite de mémoire.}. » Il est évident que chacun peut penser, \emph{bona fide}\footnote{De bonne foi.}, que tous devraient agir comme lui-même quand il commet des actes qui répugnent, au contraire, à des consciences plus évoluées ou de civilisation différente. Un mari jaloux qui tue sa femme infidèle pense que tous les maris devraient tuer les femmes infidèles, etc. On peut observer qu’il n’y a pas de délinquant qui ne justifie, en son for intérieur, le délit qu’il a commis pour scélérat qu’il puisse être ; de ce point de vue, les protestations d’innocence de tant de condamnés ne vont pas sans une certaine conviction de bonne foi ; en réalité, chacun de ces condamnés connaît exactement les circonstances objectives et subjectives dans lesquelles il a commis le délit et tire de cette connaissance, que souvent il ne peut transmettre rationnellement aux autres, la conviction d’être « justifié » ; c’est seulement un changement dans sa conception de la vie qui peut le conduire à un jugement différent : chose qui arrive souvent et explique nombre de suicides. La formule kantienne, analysée avec réalisme, ne dépasse pas les limites d’un milieu donné, quel qu’il soit, avec toutes ses superstitions morales et ses mœurs barbares ; elle est statique, c’est une forme vide qui peut être remplie par n’importe quel contenu historique, actuel ou anachronique (avec ses contradictions naturellement, qui font que ce qui est vérité au-delà des Pyrénées est mensonge en deçà). La formule kantienne paraît supérieure parce que les intellectuels la remplissent de leur façon particulière de vivre et d’agir et que parfois, on peut l’admettre, certains groupes d’intellectuels sont plus avancés et civilisés que leur entourage.\par
L'argument du danger de relativisme et de scepticisme n’est donc pas valable. Le problème à poser est autre : la conception morale envisagée comporte-t-elle des caractères qui lui promettent une certaine durée ? Ou bien est-elle variable d’un jour à l’autre ? Ou donne-t-elle, à l’intérieur d’un même groupe, matière à formuler la théorie de la double volonté ? De plus : peut-on se fonder sur elle pour constituer une \emph{élite} qui guide les multitudes, les éduque, ait la capacité d’être « exemplaire » ? Si les réponses à ces questions sont affirmatives, la conception est justifiée, valable. Mais il y aura une période de relâchement, de libertinage même et de dissolution morale. Ce n’est pas exclu, loin de là, mais ce n’est pas non plus un argument valable. Des périodes de dissolution morale, il y en a eu souvent dans l’histoire, même lorsqu’une conception morale inchangée maintenait sa prédominance : elles ont eu pour origine des causes réelles, concrètes, non des conceptions morales : très souvent elles sont l’indice qu’une conception a vieilli, s’est désagrégée, est devenue pure hypocrisie formaliste, mais tente de garder le haut du pavé par la coercition, en contraignant la société à une double vie : contre l’hypocrisie et la duplicité réagissent précisément, dans des formes excessives, les périodes de libertinage et de dissolution qui annoncent presque toujours qu’une nouvelle conception est en train de se former.\par
Le danger d’un manque de vitalité morale est représenté, au contraire, par le fatalisme de ces groupes qui partagent la conception de la « naturalité » selon la « nature » des brutes, et pour lesquels tout est justifié par le milieu social. Tout sens de la responsabilité individuelle finit ainsi par s’émousser et toute responsabilité particulière est noyée dans une abstraite et introuvable responsabilité sociale.\par
Si cette conception était vraie, le monde et l’histoire seraient toujours immobiles. En effet, si l’individu a besoin pour changer que toute la société ait changé avant lui, mécaniquement, par on ne sait quelle force extra-humaine, aucun changement n’aura jamais lieu. L'histoire est au contraire une lutte continuelle des individus et des groupes pour changer ce qui existe à un instant donné, mais pour que la lutte soit efficace, ces individus et ces groupes doivent se sentir supérieurs à l’existant, éducateurs de la société, etc. Donc le milieu ne justifie pas, il « explique » seulement le comportement des individus, surtout des plus passifs historiquement. « Explication » qui servira quelquefois à rendre indulgent envers les particuliers et fournira du matériel pour l’éducation, mais ne doit jamais devenir « justification », sous peine de conduire nécessairement à l’une des formes les plus hypocrites et les plus révoltantes de conservatisme et de « réaction ».\par
Au concept de « naturel » s’oppose celui d’ « artificiel », de « conventionnel ». Mais qui signifient « artificiel » et « conventionnel » quand on fait référence aux phénomènes de masse ? Cela signifie simplement « historique », acquis à travers le développement historique, et c’est en vain qu’on cherche à donner un sens péjoratif à la chose puisqu’elle a pénétré même dans la conscience commune avec l’expression « seconde nature ». On pourra donc parler d’artifice et de conventionnel par référence à des idiosyncrasies personnelles, non par référence à des phénomènes de masse déjà en acte. Voyager par chemin de fer est « artificiel », mais sûrement pas de la même façon que se farder le visage.\par
 D'après les indications du paragraphe précédent se pose le problème : qui devra décider qu’une conduite morale déterminée est la plus conforme au développement des forces productives ? Assurément, il ne peut être question de créer un « pape » spécial ou un office compétent. Les forces dirigeantes naîtront du fait même que la façon de penser sera orientée dans ce sens réaliste et naîtront du choc même des opinions contraires, sans « conventionnel », sans « artifice », mais « naturellement ». (P.P., pp. 200-204.)\par

\dateline{[1933-1934]}
\chapterclose

 


% at least one empty page at end (for booklet couv)
\ifbooklet
  \pagestyle{empty}
  \clearpage
  % 2 empty pages maybe needed for 4e cover
  \ifnum\modulo{\value{page}}{4}=0 \hbox{}\newpage\hbox{}\newpage\fi
  \ifnum\modulo{\value{page}}{4}=1 \hbox{}\newpage\hbox{}\newpage\fi


  \hbox{}\newpage
  \ifodd\value{page}\hbox{}\newpage\fi
  {\centering\color{rubric}\bfseries\noindent\large
    Hurlus ? Qu’est-ce.\par
    \bigskip
  }
  \noindent Des bouquinistes électroniques, pour du texte libre à participation libre,
  téléchargeable gratuitement sur \href{https://hurlus.fr}{\dotuline{hurlus.fr}}.\par
  \bigskip
  \noindent Cette brochure a été produite par des éditeurs bénévoles.
  Elle n’est pas faîte pour être possédée, mais pour être lue, et puis donnée.
  Que circule le texte !
  En page de garde, on peut ajouter une date, un lieu, un nom ; pour suivre le voyage des idées.
  \par

  Ce texte a été choisi parce qu’une personne l’a aimé,
  ou haï, elle a en tous cas pensé qu’il partipait à la formation de notre présent ;
  sans le souci de plaire, vendre, ou militer pour une cause.
  \par

  L’édition électronique est soigneuse, tant sur la technique
  que sur l’établissement du texte ; mais sans aucune prétention scolaire, au contraire.
  Le but est de s’adresser à tous, sans distinction de science ou de diplôme.
  Au plus direct ! (possible)
  \par

  Cet exemplaire en papier a été tiré sur une imprimante personnelle
   ou une photocopieuse. Tout le monde peut le faire.
  Il suffit de
  télécharger un fichier sur \href{https://hurlus.fr}{\dotuline{hurlus.fr}},
  d’imprimer, et agrafer ; puis de lire et donner.\par

  \bigskip

  \noindent PS : Les hurlus furent aussi des rebelles protestants qui cassaient les statues dans les églises catholiques. En 1566 démarra la révolte des gueux dans le pays de Lille. L’insurrection enflamma la région jusqu’à Anvers où les gueux de mer bloquèrent les bateaux espagnols.
  Ce fut une rare guerre de libération dont naquit un pays toujours libre : les Pays-Bas.
  En plat pays francophone, par contre, restèrent des bandes de huguenots, les hurlus, progressivement réprimés par la très catholique Espagne.
  Cette mémoire d’une défaite est éteinte, rallumons-la. Sortons les livres du culte universitaire, cherchons les idoles de l’époque, pour les briser.
\fi

\ifdev % autotext in dev mode
\fontname\font — \textsc{Les règles du jeu}\par
(\hyperref[utopie]{\underline{Lien}})\par
\noindent \initialiv{A}{lors là}\blindtext\par
\noindent \initialiv{À}{ la bonheur des dames}\blindtext\par
\noindent \initialiv{É}{tonnez-le}\blindtext\par
\noindent \initialiv{Q}{ualitativement}\blindtext\par
\noindent \initialiv{V}{aloriser}\blindtext\par
\Blindtext
\phantomsection
\label{utopie}
\Blinddocument
\fi
\end{document}
