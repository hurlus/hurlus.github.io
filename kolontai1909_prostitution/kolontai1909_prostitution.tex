%%%%%%%%%%%%%%%%%%%%%%%%%%%%%%%%%
% LaTeX model https://hurlus.fr %
%%%%%%%%%%%%%%%%%%%%%%%%%%%%%%%%%

% Needed before document class
\RequirePackage{pdftexcmds} % needed for tests expressions
\RequirePackage{fix-cm} % correct units

% Define mode
\def\mode{a4}

\newif\ifaiv % a4
\newif\ifav % a5
\newif\ifbooklet % booklet
\newif\ifcover % cover for booklet

\ifnum \strcmp{\mode}{cover}=0
  \covertrue
\else\ifnum \strcmp{\mode}{booklet}=0
  \booklettrue
\else\ifnum \strcmp{\mode}{a5}=0
  \avtrue
\else
  \aivtrue
\fi\fi\fi

\ifbooklet % do not enclose with {}
  \documentclass[french,twoside]{book} % ,notitlepage
  \usepackage[%
    papersize={105mm, 297mm},
    inner=12mm,
    outer=12mm,
    top=20mm,
    bottom=15mm,
    marginparsep=0pt,
  ]{geometry}
  \usepackage[fontsize=9.5pt]{scrextend} % for Roboto
\else\ifav
  \documentclass[french,twoside]{book} % ,notitlepage
  \usepackage[%
    a5paper,
    inner=25mm,
    outer=15mm,
    top=15mm,
    bottom=15mm,
    marginparsep=0pt,
  ]{geometry}
  \usepackage[fontsize=12pt]{scrextend}
\else% A4 2 cols
  \documentclass[twocolumn]{report}
  \usepackage[%
    a4paper,
    inner=15mm,
    outer=10mm,
    top=25mm,
    bottom=18mm,
    marginparsep=0pt,
  ]{geometry}
  \setlength{\columnsep}{20mm}
  \usepackage[fontsize=9.5pt]{scrextend}
\fi\fi

%%%%%%%%%%%%%%
% Alignments %
%%%%%%%%%%%%%%
% before teinte macros

\setlength{\arrayrulewidth}{0.2pt}
\setlength{\columnseprule}{\arrayrulewidth} % twocol
\setlength{\parskip}{0pt} % classical para with no margin
\setlength{\parindent}{1.5em}

%%%%%%%%%%
% Colors %
%%%%%%%%%%
% before Teinte macros

\usepackage[dvipsnames]{xcolor}
\definecolor{rubric}{HTML}{800000} % the tonic 0c71c3
\def\columnseprulecolor{\color{rubric}}
\colorlet{borderline}{rubric!30!} % definecolor need exact code
\definecolor{shadecolor}{gray}{0.95}
\definecolor{bghi}{gray}{0.5}

%%%%%%%%%%%%%%%%%
% Teinte macros %
%%%%%%%%%%%%%%%%%
%%%%%%%%%%%%%%%%%%%%%%%%%%%%%%%%%%%%%%%%%%%%%%%%%%%
% <TEI> generic (LaTeX names generated by Teinte) %
%%%%%%%%%%%%%%%%%%%%%%%%%%%%%%%%%%%%%%%%%%%%%%%%%%%
% This template is inserted in a specific design
% It is XeLaTeX and otf fonts

\makeatletter % <@@@


\usepackage{blindtext} % generate text for testing
\usepackage[strict]{changepage} % for modulo 4
\usepackage{contour} % rounding words
\usepackage[nodayofweek]{datetime}
% \usepackage{DejaVuSans} % seems buggy for sffont font for symbols
\usepackage{enumitem} % <list>
\usepackage{etoolbox} % patch commands
\usepackage{fancyvrb}
\usepackage{fancyhdr}
\usepackage{float}
\usepackage{fontspec} % XeLaTeX mandatory for fonts
\usepackage{footnote} % used to capture notes in minipage (ex: quote)
\usepackage{framed} % bordering correct with footnote hack
\usepackage{graphicx}
\usepackage{lettrine} % drop caps
\usepackage{lipsum} % generate text for testing
\usepackage[framemethod=tikz,]{mdframed} % maybe used for frame with footnotes inside
\usepackage{pdftexcmds} % needed for tests expressions
\usepackage{polyglossia} % non-break space french punct, bug Warning: "Failed to patch part"
\usepackage[%
  indentfirst=false,
  vskip=1em,
  noorphanfirst=true,
  noorphanafter=true,
  leftmargin=\parindent,
  rightmargin=0pt,
]{quoting}
\usepackage{ragged2e}
\usepackage{setspace} % \setstretch for <quote>
\usepackage{tabularx} % <table>
\usepackage[explicit]{titlesec} % wear titles, !NO implicit
\usepackage{tikz} % ornaments
\usepackage{tocloft} % styling tocs
\usepackage[fit]{truncate} % used im runing titles
\usepackage{unicode-math}
\usepackage[normalem]{ulem} % breakable \uline, normalem is absolutely necessary to keep \emph
\usepackage{verse} % <l>
\usepackage{xcolor} % named colors
\usepackage{xparse} % @ifundefined
\XeTeXdefaultencoding "iso-8859-1" % bad encoding of xstring
\usepackage{xstring} % string tests
\XeTeXdefaultencoding "utf-8"
\PassOptionsToPackage{hyphens}{url} % before hyperref, which load url package

% TOTEST
% \usepackage{hypcap} % links in caption ?
% \usepackage{marginnote}
% TESTED
% \usepackage{background} % doesn’t work with xetek
% \usepackage{bookmark} % prefers the hyperref hack \phantomsection
% \usepackage[color, leftbars]{changebar} % 2 cols doc, impossible to keep bar left
% \usepackage[utf8x]{inputenc} % inputenc package ignored with utf8 based engines
% \usepackage[sfdefault,medium]{inter} % no small caps
% \usepackage{firamath} % choose firasans instead, firamath unavailable in Ubuntu 21-04
% \usepackage{flushend} % bad for last notes, supposed flush end of columns
% \usepackage[stable]{footmisc} % BAD for complex notes https://texfaq.org/FAQ-ftnsect
% \usepackage{helvet} % not for XeLaTeX
% \usepackage{multicol} % not compatible with too much packages (longtable, framed, memoir…)
% \usepackage[default,oldstyle,scale=0.95]{opensans} % no small caps
% \usepackage{sectsty} % \chapterfont OBSOLETE
% \usepackage{soul} % \ul for underline, OBSOLETE with XeTeX
% \usepackage[breakable]{tcolorbox} % text styling gone, footnote hack not kept with breakable


% Metadata inserted by a program, from the TEI source, for title page and runing heads
\title{\textbf{ Les problèmes de la prostitution }}
\date{1909}
\author{Kollontaï, Alexandra}
\def\elbibl{Kollontaï, Alexandra. 1909. \emph{Les problèmes de la prostitution}}
\def\elsource{\href{https://www.marxists.org/francais/kollontai/works/1909/00/akoll_1909_prosti.htm}{\dotuline{https://www.marxists.org/francais/kollontai/works/1909/00/akoll\_1909\_prosti.htm}}\footnote{\href{https://www.marxists.org/francais/kollontai/works/1909/00/akoll_1909_prosti.htm}{\url{https://www.marxists.org/francais/kollontai/works/1909/00/akoll_1909_prosti.htm}}}}

% Default metas
\newcommand{\colorprovide}[2]{\@ifundefinedcolor{#1}{\colorlet{#1}{#2}}{}}
\colorprovide{rubric}{red}
\colorprovide{silver}{lightgray}
\@ifundefined{syms}{\newfontfamily\syms{DejaVu Sans}}{}
\newif\ifdev
\@ifundefined{elbibl}{% No meta defined, maybe dev mode
  \newcommand{\elbibl}{Titre court ?}
  \newcommand{\elbook}{Titre du livre source ?}
  \newcommand{\elabstract}{Résumé\par}
  \newcommand{\elurl}{http://oeuvres.github.io/elbook/2}
  \author{Éric Lœchien}
  \title{Un titre de test assez long pour vérifier le comportement d’une maquette}
  \date{1566}
  \devtrue
}{}
\let\eltitle\@title
\let\elauthor\@author
\let\eldate\@date


\defaultfontfeatures{
  % Mapping=tex-text, % no effect seen
  Scale=MatchLowercase,
  Ligatures={TeX,Common},
}


% generic typo commands
\newcommand{\astermono}{\medskip\centerline{\color{rubric}\large\selectfont{\syms ✻}}\medskip\par}%
\newcommand{\astertri}{\medskip\par\centerline{\color{rubric}\large\selectfont{\syms ✻\,✻\,✻}}\medskip\par}%
\newcommand{\asterism}{\bigskip\par\noindent\parbox{\linewidth}{\centering\color{rubric}\large{\syms ✻}\\{\syms ✻}\hskip 0.75em{\syms ✻}}\bigskip\par}%

% lists
\newlength{\listmod}
\setlength{\listmod}{\parindent}
\setlist{
  itemindent=!,
  listparindent=\listmod,
  labelsep=0.2\listmod,
  parsep=0pt,
  % topsep=0.2em, % default topsep is best
}
\setlist[itemize]{
  label=—,
  leftmargin=0pt,
  labelindent=1.2em,
  labelwidth=0pt,
}
\setlist[enumerate]{
  label={\bf\color{rubric}\arabic*.},
  labelindent=0.8\listmod,
  leftmargin=\listmod,
  labelwidth=0pt,
}
\newlist{listalpha}{enumerate}{1}
\setlist[listalpha]{
  label={\bf\color{rubric}\alph*.},
  leftmargin=0pt,
  labelindent=0.8\listmod,
  labelwidth=0pt,
}
\newcommand{\listhead}[1]{\hspace{-1\listmod}\emph{#1}}

\renewcommand{\hrulefill}{%
  \leavevmode\leaders\hrule height 0.2pt\hfill\kern\z@}

% General typo
\DeclareTextFontCommand{\textlarge}{\large}
\DeclareTextFontCommand{\textsmall}{\small}

% commands, inlines
\newcommand{\anchor}[1]{\Hy@raisedlink{\hypertarget{#1}{}}} % link to top of an anchor (not baseline)
\newcommand\abbr[1]{#1}
\newcommand{\autour}[1]{\tikz[baseline=(X.base)]\node [draw=rubric,thin,rectangle,inner sep=1.5pt, rounded corners=3pt] (X) {\color{rubric}#1};}
\newcommand\corr[1]{#1}
\newcommand{\ed}[1]{ {\color{silver}\sffamily\footnotesize (#1)} } % <milestone ed="1688"/>
\newcommand\expan[1]{#1}
\newcommand\foreign[1]{\emph{#1}}
\newcommand\gap[1]{#1}
\renewcommand{\LettrineFontHook}{\color{rubric}}
\newcommand{\initial}[2]{\lettrine[lines=2, loversize=0.3, lhang=0.3]{#1}{#2}}
\newcommand{\initialiv}[2]{%
  \let\oldLFH\LettrineFontHook
  % \renewcommand{\LettrineFontHook}{\color{rubric}\ttfamily}
  \IfSubStr{QJ’}{#1}{
    \lettrine[lines=4, lhang=0.2, loversize=-0.1, lraise=0.2]{\smash{#1}}{#2}
  }{\IfSubStr{É}{#1}{
    \lettrine[lines=4, lhang=0.2, loversize=-0, lraise=0]{\smash{#1}}{#2}
  }{\IfSubStr{ÀÂ}{#1}{
    \lettrine[lines=4, lhang=0.2, loversize=-0, lraise=0, slope=0.6em]{\smash{#1}}{#2}
  }{\IfSubStr{A}{#1}{
    \lettrine[lines=4, lhang=0.2, loversize=0.2, slope=0.6em]{\smash{#1}}{#2}
  }{\IfSubStr{V}{#1}{
    \lettrine[lines=4, lhang=0.2, loversize=0.2, slope=-0.5em]{\smash{#1}}{#2}
  }{
    \lettrine[lines=4, lhang=0.2, loversize=0.2]{\smash{#1}}{#2}
  }}}}}
  \let\LettrineFontHook\oldLFH
}
\newcommand{\labelchar}[1]{\textbf{\color{rubric} #1}}
\newcommand{\milestone}[1]{\autour{\footnotesize\color{rubric} #1}} % <milestone n="4"/>
\newcommand\name[1]{#1}
\newcommand\orig[1]{#1}
\newcommand\orgName[1]{#1}
\newcommand\persName[1]{#1}
\newcommand\placeName[1]{#1}
\newcommand{\pn}[1]{\IfSubStr{-—–¶}{#1}% <p n="3"/>
  {\noindent{\bfseries\color{rubric}   ¶  }}
  {{\footnotesize\autour{ #1}  }}}
\newcommand\reg{}
% \newcommand\ref{} % already defined
\newcommand\sic[1]{#1}
\newcommand\surname[1]{\textsc{#1}}
\newcommand\term[1]{\textbf{#1}}

\def\mednobreak{\ifdim\lastskip<\medskipamount
  \removelastskip\nopagebreak\medskip\fi}
\def\bignobreak{\ifdim\lastskip<\bigskipamount
  \removelastskip\nopagebreak\bigskip\fi}

% commands, blocks
\newcommand{\byline}[1]{\bigskip{\RaggedLeft{#1}\par}\bigskip}
\newcommand{\bibl}[1]{{\RaggedLeft{#1}\par\bigskip}}
\newcommand{\biblitem}[1]{{\noindent\hangindent=\parindent   #1\par}}
\newcommand{\dateline}[1]{\medskip{\RaggedLeft{#1}\par}\bigskip}
\newcommand{\labelblock}[1]{\medbreak{\noindent\color{rubric}\bfseries #1}\par\mednobreak}
\newcommand{\salute}[1]{\bigbreak{#1}\par\medbreak}
\newcommand{\signed}[1]{\bigbreak\filbreak{\raggedleft #1\par}\medskip}

% environments for blocks (some may become commands)
\newenvironment{borderbox}{}{} % framing content
\newenvironment{citbibl}{\ifvmode\hfill\fi}{\ifvmode\par\fi }
\newenvironment{docAuthor}{\ifvmode\vskip4pt\fontsize{16pt}{18pt}\selectfont\fi\itshape}{\ifvmode\par\fi }
\newenvironment{docDate}{}{\ifvmode\par\fi }
\newenvironment{docImprint}{\vskip6pt}{\ifvmode\par\fi }
\newenvironment{docTitle}{\vskip6pt\bfseries\fontsize{18pt}{22pt}\selectfont}{\par }
\newenvironment{msHead}{\vskip6pt}{\par}
\newenvironment{msItem}{\vskip6pt}{\par}
\newenvironment{titlePart}{}{\par }


% environments for block containers
\newenvironment{argument}{\itshape\parindent0pt}{\vskip1.5em}
\newenvironment{biblfree}{}{\ifvmode\par\fi }
\newenvironment{bibitemlist}[1]{%
  \list{\@biblabel{\@arabic\c@enumiv}}%
  {%
    \settowidth\labelwidth{\@biblabel{#1}}%
    \leftmargin\labelwidth
    \advance\leftmargin\labelsep
    \@openbib@code
    \usecounter{enumiv}%
    \let\p@enumiv\@empty
    \renewcommand\theenumiv{\@arabic\c@enumiv}%
  }
  \sloppy
  \clubpenalty4000
  \@clubpenalty \clubpenalty
  \widowpenalty4000%
  \sfcode`\.\@m
}%
{\def\@noitemerr
  {\@latex@warning{Empty `bibitemlist' environment}}%
\endlist}
\newenvironment{quoteblock}% may be used for ornaments
  {\begin{quoting}}
  {\end{quoting}}

% table () is preceded and finished by custom command
\newcommand{\tableopen}[1]{%
  \ifnum\strcmp{#1}{wide}=0{%
    \begin{center}
  }
  \else\ifnum\strcmp{#1}{long}=0{%
    \begin{center}
  }
  \else{%
    \begin{center}
  }
  \fi\fi
}
\newcommand{\tableclose}[1]{%
  \ifnum\strcmp{#1}{wide}=0{%
    \end{center}
  }
  \else\ifnum\strcmp{#1}{long}=0{%
    \end{center}
  }
  \else{%
    \end{center}
  }
  \fi\fi
}


% text structure
\newcommand\chapteropen{} % before chapter title
\newcommand\chaptercont{} % after title, argument, epigraph…
\newcommand\chapterclose{} % maybe useful for multicol settings
\setcounter{secnumdepth}{-2} % no counters for hierarchy titles
\setcounter{tocdepth}{5} % deep toc
\markright{\@title} % ???
\markboth{\@title}{\@author} % ???
\renewcommand\tableofcontents{\@starttoc{toc}}
% toclof format
% \renewcommand{\@tocrmarg}{0.1em} % Useless command?
% \renewcommand{\@pnumwidth}{0.5em} % {1.75em}
\renewcommand{\@cftmaketoctitle}{}
\setlength{\cftbeforesecskip}{\z@ \@plus.2\p@}
\renewcommand{\cftchapfont}{}
\renewcommand{\cftchapdotsep}{\cftdotsep}
\renewcommand{\cftchapleader}{\normalfont\cftdotfill{\cftchapdotsep}}
\renewcommand{\cftchappagefont}{\bfseries}
\setlength{\cftbeforechapskip}{0em \@plus\p@}
% \renewcommand{\cftsecfont}{\small\relax}
\renewcommand{\cftsecpagefont}{\normalfont}
% \renewcommand{\cftsubsecfont}{\small\relax}
\renewcommand{\cftsecdotsep}{\cftdotsep}
\renewcommand{\cftsecpagefont}{\normalfont}
\renewcommand{\cftsecleader}{\normalfont\cftdotfill{\cftsecdotsep}}
\setlength{\cftsecindent}{1em}
\setlength{\cftsubsecindent}{2em}
\setlength{\cftsubsubsecindent}{3em}
\setlength{\cftchapnumwidth}{1em}
\setlength{\cftsecnumwidth}{1em}
\setlength{\cftsubsecnumwidth}{1em}
\setlength{\cftsubsubsecnumwidth}{1em}

% footnotes
\newif\ifheading
\newcommand*{\fnmarkscale}{\ifheading 0.70 \else 1 \fi}
\renewcommand\footnoterule{\vspace*{0.3cm}\hrule height \arrayrulewidth width 3cm \vspace*{0.3cm}}
\setlength\footnotesep{1.5\footnotesep} % footnote separator
\renewcommand\@makefntext[1]{\parindent 1.5em \noindent \hb@xt@1.8em{\hss{\normalfont\@thefnmark . }}#1} % no superscipt in foot
\patchcmd{\@footnotetext}{\footnotesize}{\footnotesize\sffamily}{}{} % before scrextend, hyperref


%   see https://tex.stackexchange.com/a/34449/5049
\def\truncdiv#1#2{((#1-(#2-1)/2)/#2)}
\def\moduloop#1#2{(#1-\truncdiv{#1}{#2}*#2)}
\def\modulo#1#2{\number\numexpr\moduloop{#1}{#2}\relax}

% orphans and widows
\clubpenalty=9996
\widowpenalty=9999
\brokenpenalty=4991
\predisplaypenalty=10000
\postdisplaypenalty=1549
\displaywidowpenalty=1602
\hyphenpenalty=400
% Copied from Rahtz but not understood
\def\@pnumwidth{1.55em}
\def\@tocrmarg {2.55em}
\def\@dotsep{4.5}
\emergencystretch 3em
\hbadness=4000
\pretolerance=750
\tolerance=2000
\vbadness=4000
\def\Gin@extensions{.pdf,.png,.jpg,.mps,.tif}
% \renewcommand{\@cite}[1]{#1} % biblio

\usepackage{hyperref} % supposed to be the last one, :o) except for the ones to follow
\urlstyle{same} % after hyperref
\hypersetup{
  % pdftex, % no effect
  pdftitle={\elbibl},
  % pdfauthor={Your name here},
  % pdfsubject={Your subject here},
  % pdfkeywords={keyword1, keyword2},
  bookmarksnumbered=true,
  bookmarksopen=true,
  bookmarksopenlevel=1,
  pdfstartview=Fit,
  breaklinks=true, % avoid long links
  pdfpagemode=UseOutlines,    % pdf toc
  hyperfootnotes=true,
  colorlinks=false,
  pdfborder=0 0 0,
  % pdfpagelayout=TwoPageRight,
  % linktocpage=true, % NO, toc, link only on page no
}

\makeatother % /@@@>
%%%%%%%%%%%%%%
% </TEI> end %
%%%%%%%%%%%%%%


%%%%%%%%%%%%%
% footnotes %
%%%%%%%%%%%%%
\renewcommand{\thefootnote}{\bfseries\textcolor{rubric}{\arabic{footnote}}} % color for footnote marks

%%%%%%%%%
% Fonts %
%%%%%%%%%
\usepackage[]{roboto} % SmallCaps, Regular is a bit bold
% \linespread{0.90} % too compact, keep font natural
\newfontfamily\fontrun[]{Roboto Condensed Light} % condensed runing heads
\ifav
  \setmainfont[
    ItalicFont={Roboto Light Italic},
  ]{Roboto}
\else\ifbooklet
  \setmainfont[
    ItalicFont={Roboto Light Italic},
  ]{Roboto}
\else
\setmainfont[
  ItalicFont={Roboto Italic},
]{Roboto Light}
\fi\fi
\renewcommand{\LettrineFontHook}{\bfseries\color{rubric}}
% \renewenvironment{labelblock}{\begin{center}\bfseries\color{rubric}}{\end{center}}

%%%%%%%%
% MISC %
%%%%%%%%

\setdefaultlanguage[frenchpart=false]{french} % bug on part


\newenvironment{quotebar}{%
    \def\FrameCommand{{\color{rubric!10!}\vrule width 0.5em} \hspace{0.9em}}%
    \def\OuterFrameSep{\itemsep} % séparateur vertical
    \MakeFramed {\advance\hsize-\width \FrameRestore}
  }%
  {%
    \endMakeFramed
  }
\renewenvironment{quoteblock}% may be used for ornaments
  {%
    \savenotes
    \setstretch{0.9}
    \normalfont
    \begin{quotebar}
  }
  {%
    \end{quotebar}
    \spewnotes
  }


\renewcommand{\headrulewidth}{\arrayrulewidth}
\renewcommand{\headrule}{{\color{rubric}\hrule}}

% delicate tuning, image has produce line-height problems in title on 2 lines
\titleformat{name=\chapter} % command
  [display] % shape
  {\vspace{1.5em}\centering} % format
  {} % label
  {0pt} % separator between n
  {}
[{\color{rubric}\huge\textbf{#1}}\bigskip] % after code
% \titlespacing{command}{left spacing}{before spacing}{after spacing}[right]
\titlespacing*{\chapter}{0pt}{-2em}{0pt}[0pt]

\titleformat{name=\section}
  [block]{}{}{}{}
  [\vbox{\color{rubric}\large\raggedleft\textbf{#1}}]
\titlespacing{\section}{0pt}{0pt plus 4pt minus 2pt}{\baselineskip}

\titleformat{name=\subsection}
  [block]
  {}
  {} % \thesection
  {} % separator \arrayrulewidth
  {}
[\vbox{\large\textbf{#1}}]
% \titlespacing{\subsection}{0pt}{0pt plus 4pt minus 2pt}{\baselineskip}

\ifaiv
  \fancypagestyle{main}{%
    \fancyhf{}
    \setlength{\headheight}{1.5em}
    \fancyhead{} % reset head
    \fancyfoot{} % reset foot
    \fancyhead[L]{\truncate{0.45\headwidth}{\fontrun\elbibl}} % book ref
    \fancyhead[R]{\truncate{0.45\headwidth}{ \fontrun\nouppercase\leftmark}} % Chapter title
    \fancyhead[C]{\thepage}
  }
  \fancypagestyle{plain}{% apply to chapter
    \fancyhf{}% clear all header and footer fields
    \setlength{\headheight}{1.5em}
    \fancyhead[L]{\truncate{0.9\headwidth}{\fontrun\elbibl}}
    \fancyhead[R]{\thepage}
  }
\else
  \fancypagestyle{main}{%
    \fancyhf{}
    \setlength{\headheight}{1.5em}
    \fancyhead{} % reset head
    \fancyfoot{} % reset foot
    \fancyhead[RE]{\truncate{0.9\headwidth}{\fontrun\elbibl}} % book ref
    \fancyhead[LO]{\truncate{0.9\headwidth}{\fontrun\nouppercase\leftmark}} % Chapter title, \nouppercase needed
    \fancyhead[RO,LE]{\thepage}
  }
  \fancypagestyle{plain}{% apply to chapter
    \fancyhf{}% clear all header and footer fields
    \setlength{\headheight}{1.5em}
    \fancyhead[L]{\truncate{0.9\headwidth}{\fontrun\elbibl}}
    \fancyhead[R]{\thepage}
  }
\fi

\ifav % a5 only
  \titleclass{\section}{top}
\fi

\newcommand\chapo{{%
  \vspace*{-3em}
  \centering % no vskip ()
  {\Large\addfontfeature{LetterSpace=25}\bfseries{\elauthor}}\par
  \smallskip
  {\large\eldate}\par
  \bigskip
  {\Large\selectfont{\eltitle}}\par
  \bigskip
  {\color{rubric}\hline\par}
  \bigskip
  {\Large TEXTE LIBRE À PARTICPATION LIBRE\par}
  \centerline{\small\color{rubric} {hurlus.fr, tiré le \today}}\par
  \bigskip
}}

\newcommand\cover{{%
  \thispagestyle{empty}
  \centering
  {\LARGE\bfseries{\elauthor}}\par
  \bigskip
  {\Large\eldate}\par
  \bigskip
  \bigskip
  {\LARGE\selectfont{\eltitle}}\par
  \vfill\null
  {\color{rubric}\setlength{\arrayrulewidth}{2pt}\hline\par}
  \vfill\null
  {\Large TEXTE LIBRE À PARTICPATION LIBRE\par}
  \centerline{{\href{https://hurlus.fr}{\dotuline{hurlus.fr}}, tiré le \today}}\par
}}

\begin{document}
\pagestyle{empty}
\ifbooklet{
  \cover\newpage
  \thispagestyle{empty}\hbox{}\newpage
  \cover\newpage\noindent Les voyages de la brochure\par
  \bigskip
  \begin{tabularx}{\textwidth}{l|X|X}
    \textbf{Date} & \textbf{Lieu}& \textbf{Nom/pseudo} \\ \hline
    \rule{0pt}{25cm} &  &   \\
  \end{tabularx}
  \newpage
  \addtocounter{page}{-4}
}\fi

\thispagestyle{empty}
\ifaiv
  \twocolumn[\chapo]
\else
  \chapo
\fi
{\it\elabstract}
\bigskip
\makeatletter\@starttoc{toc}\makeatother % toc without new page
\bigskip

\pagestyle{main} % after style

  
\salute{Alexandra Kollontaï}

\begin{quoteblock}
 \noindent Si le mariage représente l’un des côtés de la vie sexuelle du monde bourgeois, la prostitution en représente l’autre. Le premier est la face de la médaille, la seconde en est le revers. Quand l’homme ne trouve pas sa satisfaction dans le mariage, il a le plus souvent recours à la prostitution (…) qu’il s’agisse de ceux qui, de gré ou de force, vivent dans le célibat, ou de ceux auxquels le mariage ne donne pas ce qu’ils en attendaient, les circonstances leur sont infiniment plus favorables pour les aider à satisfaire leur instinct sexuel que pour les femmes\footnote{Citation de Bebel, \emph{La femme et le socialisme} (NdE)}.
\end{quoteblock}

\noindent Méprisée par tous, pourchassée par tous, mais secrètement encouragée, la prostitution, sous ses fleurs somptueuses mais empoisonnées, étouffe tout ce qui reste des vertus familiales. Recouvrant la société d’une sorte de limon pourri, elle empoisonne de son haleine fétide les pures joies de l’union amoureuse des sexes.\par
De nos jours, la prostitution atteint des proportions colossales, telles que l’humanité n’en a jamais connues, même aux périodes de sa plus grande décadence spirituelle. Que pèsent les dichtérions grecs semi-religieux, ces lupanars romains, cette prostitution joyeuse des « filles à soldats » ou « sérieuse » des ateliers du Moyen-Âge, cette débauche cynique, ouvertement condamnée mais secrètement encouragée, de l’époque de la Réforme, que pèsent ces milliers de grisettes frivoles en face de la vente massive du corps féminin pratiquée aujourd’hui ? Telle une infection contagieuse, la prostitution se répand de place en place, de pays en pays, de ville en ville, empoisonnant l’atmosphère de la vie sociale contemporaine. Des professions entières, des couches entières de la société sont soumises à son influence délétère.\par
La duplicité hypocrite à l’égard de la prostitution est caractéristique de la bourgeoisie et met en relief le fait que là aussi, dans cette question qui semble concerner l’humanité tout entière, elle a une position de classe. En effet, la prostitution, cet appendice obligatoire de la société de classes contemporaine, ce correctif à la forme coercitive désuète de la famille actuelle, pèse de tout son poids sur les classes non possédantes. C’est ici, dans les bas-fonds obscurs et nauséabonds, que poussent ses germes funestes ; c’est dans le corps du prolétariat qu’elle plante le plus souvent ses griffes empoisonnées, et bien que son haleine fétide pourrisse toute l’atmosphère sociale, c’est d’abord pour la classe ouvrière qu’elle est un fléau. Voilà pourquoi la bourgeoisie n’est nullement pressée de sonner l’alarme : si le gros du contingent des femmes vénales était fourni par la classe possédante, il est à supposer que son attitude à l’égard de cette question serait fort différente.\par
Le pourquoi de l’attitude ambiguë des gouvernements de tous les pays à l’égard de la prostitution doit être recherché précisément dans ce point de vue de classe, dont cette question sociale est elle aussi totalement imprégnée. Condamnée par la religion, punie par la société et même par la loi, la prostitution n’en est pas moins non seulement tolérée, mais encore réglementée par l’État. Déclarée nécessaire pour la satisfaction des besoins sexuels naturels des hommes, la prostitution, depuis la formation de la société de classes, est devenue, sous une forme ou sous une autre, un « paratonnerre contre la débauche », la garantie des principes familiaux et la gardienne de la vertu des « honnêtes » bourgeoises.\par
Les rois jouissaient des services des prostituées, les admettaient à leur cour, nommaient des fonctionnaires spéciaux pour les administrer, mais cela ne les empêchait pas, en même temps, d’humilier, de persécuter et de martyriser de toutes les façons les prostituées, et parfois d’en faire périr des centaines, sous le coup d’une extase religieuse ou d’un moment de repentir hypocrite. La bourgeoisie et l’Église, qui jouissaient elles aussi largement des services de la prostitution, et qui la soutenaient en secret, la fustigeaient et la persécutaient ouvertement. Le peuple, qui y voyait une expression criante et terriblement dépouillée de sa propre servitude, la haïssait de toutes les forces de son âme impulsive et s’efforçait par tous les moyens de détruire de malheureuses victimes de cette « industrie honteuse », de leur « faire passer le goût du pain » en les couvrant d’injures, en les lapidant, en les torturant, en les tuant, en démolissant les maisons de tolérance. Mais le peuple avait beau lutter contre la vente du corps féminin, la société de classes, qui avait rendu inéluctable la vente de la force de travail, faisait sans cesse de nouvelles victimes de la « passion publique ».\par
La société contemporaine, en remplaçant la torture et le meurtre périodique des prostituées par l’assassinat moral de celles-ci à l’aide de lois et de règlements rigoureux, ne s’est guère éloigné de la cruauté médiévale. A l’époque du Consulat, le « tiers état », avec le « rationalisme » qui lui est propre et sa tendance à protéger ses intérêts à l’aide d’un arsenal juridique, a pour la première fois proclamé le principe d’une réglementation publique de la prostitution. La surveillance médico-policière a été instituée en France en 1800, et c’est en 1802 qu’a été délivré pour la première fois la « carte jaune ».\par
La prostitution, jusqu’alors seulement tolérée par l’État, est devenue phénomène reconnu par le pouvoir et légalisé. Cependant, l’hypocrisie habituelle ne permet pas d’avouer ouvertement la banqueroute des vieilles formes familiales et l’inévitable croissance de la prostitution sur le terrain des rapports capitalistes. Toute la législation russe sur l’« industrie honteuse » est pénétrée de cet esprit hypocrite. Dans l’intérêt de la sauvegarde de la famille bourgeoise, pépinière d’héritiers du capital, le commerce du corps féminin est encouragé, mais du point de vue de la « morale officielle », il est condamné sévèrement et sans indulgence ; et pour conserver à ses propres yeux le prestige de sa « haute pureté morale », la société bourgeoise s’empresse d’accuser les prostituées d’outrager son apparente vertu, et empoisonne par tous les moyens l’existence déjà pas si drôle de ces malheureuses « prêtresses du vice ».\par
Lorsqu’à Moscou il fut question d’instituer une commission médico-policière, on se proposa d’abord d’imposer aux maisons closes une contribution au profit de l’État. Mais cette idée fut abandonnée comme inconvenante, « en particulier parce que le premier prélèvement d’un impôt quelconque sur les femmes publiques ne s’accorderait pas à l’esprit de nos lois, et pourrait laisser croire que le gouvernement s’autorise à faire commerce de l’obscénité, alors que celle-ci est sévèrement réprimée par la loi ».\par
En Allemagne, on trouve la même duplicité – le propriétaire qui loge une prostituée est poursuivi en vertu du code pénal. Mais « d’autre part, la police est tenue de tolérer que des milliers de femmes se prostituent et doit protéger leurs activités dès l’instant où elles sont inscrites au registre des prostituées et se soumettent aux règlements établis pour elles, par exemple au contrôle médical périodique. Mais, si le gouvernement admet les prostituées et par là même encourage leur industrie, il doit aussi admettre qu’elles soient logées, et même – dans l’intérêt de l’ordre et de la santé publics – qu’il y ait des maisons spéciales où elles puissent exercer leur métier. Quelles contradictions ! D’un part, l’État reconnaît officiellement que la prostitution est nécessaire ; d’autre part il condamne les prostituées et le proxénétisme. Cette attitude de l’État montre que pour la société actuelle, la prostitution est un sphinx et qu’elle n’est pas en mesure de résoudre son énigme ». Oui, telle est la logique de la société bourgeoise actuelle ! La prostitution, en tant que phénomène social, est le fruit naturel de la société de classes contemporaine, mais ce n’est pas tout ; les textes eux-mêmes qui réglementent la prostitution sont entièrement imprégnés d’un point de vue de classe. » Une différenciation de classe de la prostitution – note le professeur Elistratov – soigneusement respectée dans la pratique, traverse comme un fil rouge toute une série de règlements locaux »\footnote{Prof. Elistratov, L’Enregistrement des femmes dans la catégorie des prostituées.}. Notre législation n’admet le contrôle forcé et la détention à l’hôpital que pour les filles qui « font le trottoir », les filles « louches », les putains « de bas étage » (c’est-à-dire de condition sociale inférieure). C’est ce que stipule l’article 158 des décrets de 1890 ; le vieil édit sénatorial de 1763 dit à peu près la même chose : « […] ordonnons cependant, pour les femmes convaincues d’obscénité, de n’examiner et déporter pour guérison que celles de bas étage ou vagabondes. » En ce sens, l’ordonnance du ministre de l’Intérieur adressée aux gouverneurs de provinces le 17 octobre 1844, et sur la base de laquelle s’effectue aujourd’hui encore la surveillance de la prostitution dans les provinces de Russie, prend une position encore plus nette. » Il va de soi que seules peuvent être soumises aux mesures que vous jugerez utile de prendre en l’occurrence les personnes qui en sont passibles de par leur mode de vie, leur \emph{qualité} et autres références sociales ». Le même principe entre dans les règlements spéciaux de certaines villes ; et s’il existe des dérogations, leur caractère accidentel et les indulgences consenties aux femmes des classes aisées soulignent avec une netteté particulière le caractère de classe de ces dispositions.\par
Le scandale de cette réglementation, c’est qu’elle retombe entièrement sur les femmes des classes pauvres ; devant les prostituées riches, la police comme les règlements ne font qu’ôter poliment leur chapeau. « On peut dire que partout, ce sont les prostituées les moins aisées qui sont placées sous surveillance. Les agents ne sont pas assez habiles – et parfois ils n’en ont même pas la possibilité – pour démasquer une prostituée de haut vol. Il y faut beaucoup de tact, sous peine d’avoir à le payer cher. En outre, les prostituées de cette espèce trouvent toujours des défenseurs prêts à les tirer d’embarras, ou tout au moins à se porter garants pour elles. Dans toutes les villes prédominent les prostituées de basse classe. Plus la surveillance est mal faite, moins il y a de prostituées dans les milieux aisés et cultivés. La police, afin d’éviter un travail supplémentaire et pour ne pas s’attirer de désagréments, se cantonne aux pauvres et à celles qui font le trottoir. » Du fait que la prostituée de « haut vol », dans la plupart des cas, appartient par ses origines à la classe bourgeoise, l’œil vigilant de la surveillance médico-policière glisse sur elle sans la voir, pour s’en prendre avec un zèle redoublé aux femmes dont la position sociale n’inspire pas confiance aux pouvoirs en place. « Dans les taudis où logent les femmes de la classe ouvrière, le malheur et le vice sont si étroitement mêlés qu’il n’est pas possible à première vue de les distinguer l’un de l’autre. Du reste, le sergent de ville n’a ni le temps ni l’envie de réfléchir – il tranche l’affaire rapidement et… sans appel : la femme qu’il a arrêtée dans la rue, dans le logement du coin ou dans l’asile de nuit est considérée comme prostituée ; on agit à son égard comme à l’égard d’une débauchée, même si, à part le fait qu’elle est sans abri ou sans travail, rien n’indique qu’elle se livre au commerce de la débauche \footnote{La prostitution surveillée, cité par ELISTRATOV, op. cit.}. » Les règles actuelles de la surveillance médico-policière constituent une menace dangereuse pour toutes les femmes du prolétariat, notamment celles qui vivent en banlieue. Sans même parler des périodes de chômage aigu, où la femme est naturellement, « sans raisons plausibles », dans la rue, la prolétaire risque, à n’importe quel jour férié, d’être soumise à un contrôle infamant. Le papier d’identité perdu ou tout autre coup du hasard redouble la gravité de sa situation et place souvent l’ouvrière devant cette alternative : ou bien accepter d’être expulsée et renvoyée sous escorte dans son pays natal, ou bien se soumettre à la surveillance médico-policière (et dans ce cas, mais seulement dans ce cas, la commission médicale se charge de lui faire obtenir un nouveau passeport). Bien entendu, cette situation n’existe pas seulement en Russie, mais dans tous les pays bourgeois. « N’est pas soumise au contrôle – dit le docteur Blachko – presque toute la prostitution élégante, ce qu’on appelle les dames du demi-monde, qui constitue pour la police une sorte de \emph{noli me tangere}. La masse soumise à la surveillance est presque partout formée de la lie la plus malheureuse et la plus déshéritée. Docilement et stupidement, chaque année et pendant des décennies, ces filles du destin accomplissent leur promenade habituelle aux centres d’examen. »\par
La société de classes actuelle a même trouvé le moyen de scinder la prostitution, méprisée par tous le monde, en deux classes. La « qualité supérieure », celle des prostituées aisées est accaparée par la classe bourgeoise, elle la sert, vit avec elle dans une certaine intimité et jusqu’à un certain point partage ses privilèges. La « qualité inférieure » – chair de la chair de la classe ouvrière ou de la paysannerie pauvre – boit jusqu’à la lie la coupe de la servitude, de l’humiliation et du chagrin…\par
Il est clair que le problème de l’abolition de la prostitution, le problème de l’assainissement des rapports entre les sexes, c’est le problème de la classe prolétarienne, problème lié de la façon la plus étroite et la plus indissoluble aux conditions du travail et de la production. Si, pour les autres classes et couches de la population, la solution des questions du mariage, et par suite de la prostitution, a surtout un intérêt psychologique et moral, pour le prolétariat, c’est l’une des questions fondamentales de la vie, l’un des éléments déterminants de l’avenir. La lutte contre la prostitution et les formes monstrueuses de la famille actuelle, en d’autres termes la lutte contre les institutions de classe du monde bourgeois contemporain, découle directement de la lutte générale du prolétariat et en constitue une partie intégrante. […] non, si effectivement le mouvement abolitionniste triomphait chez nous, si l’armée des prostituées se mettait à s’accroître plus lentement, les féministes seraient moins que quiconque responsables de ces heureux événements. Ce n’est pas aux résolutions maniérées des féministes que la femme en sera redevable, mais au parti ouvrier, qui lutte pour le changement des rapports, économiques et sociaux existants. On peut affirmer avec certitude que les cadres qui engendrent comme une nécessité la dépendance matérielle de la prostitution seront réduits à chaque nouvelle conquête de la classe ouvrière dans le domaine des rapports économiques et juridiques.\par
\bigbreak
\bigbreak
\bigbreak
 


% at least one empty page at end (for booklet couv)
\ifbooklet
  \pagestyle{empty}
  \clearpage
  % 2 empty pages maybe needed for 4e cover
  \ifnum\modulo{\value{page}}{4}=0 \hbox{}\newpage\hbox{}\newpage\fi
  \ifnum\modulo{\value{page}}{4}=1 \hbox{}\newpage\hbox{}\newpage\fi


  \hbox{}\newpage
  \ifodd\value{page}\hbox{}\newpage\fi
  {\centering\color{rubric}\bfseries\noindent\large
    Hurlus ? Qu’est-ce.\par
    \bigskip
  }
  \noindent Des bouquinistes électroniques, pour du texte libre à participation libre,
  téléchargeable gratuitement sur \href{https://hurlus.fr}{\dotuline{hurlus.fr}}.\par
  \bigskip
  \noindent Cette brochure a été produite par des éditeurs bénévoles.
  Elle n’est pas faîte pour être possédée, mais pour être lue, et puis donnée.
  Que circule le texte !
  En page de garde, on peut ajouter une date, un lieu, un nom ; pour suivre le voyage des idées.
  \par

  Ce texte a été choisi parce qu’une personne l’a aimé,
  ou haï, elle a en tous cas pensé qu’il partipait à la formation de notre présent ;
  sans le souci de plaire, vendre, ou militer pour une cause.
  \par

  L’édition électronique est soigneuse, tant sur la technique
  que sur l’établissement du texte ; mais sans aucune prétention scolaire, au contraire.
  Le but est de s’adresser à tous, sans distinction de science ou de diplôme.
  Au plus direct ! (possible)
  \par

  Cet exemplaire en papier a été tiré sur une imprimante personnelle
   ou une photocopieuse. Tout le monde peut le faire.
  Il suffit de
  télécharger un fichier sur \href{https://hurlus.fr}{\dotuline{hurlus.fr}},
  d’imprimer, et agrafer ; puis de lire et donner.\par

  \bigskip

  \noindent PS : Les hurlus furent aussi des rebelles protestants qui cassaient les statues dans les églises catholiques. En 1566 démarra la révolte des gueux dans le pays de Lille. L’insurrection enflamma la région jusqu’à Anvers où les gueux de mer bloquèrent les bateaux espagnols.
  Ce fut une rare guerre de libération dont naquit un pays toujours libre : les Pays-Bas.
  En plat pays francophone, par contre, restèrent des bandes de huguenots, les hurlus, progressivement réprimés par la très catholique Espagne.
  Cette mémoire d’une défaite est éteinte, rallumons-la. Sortons les livres du culte universitaire, cherchons les idoles de l’époque, pour les briser.
\fi

\ifdev % autotext in dev mode
\fontname\font — \textsc{Les règles du jeu}\par
(\hyperref[utopie]{\underline{Lien}})\par
\noindent \initialiv{A}{lors là}\blindtext\par
\noindent \initialiv{À}{ la bonheur des dames}\blindtext\par
\noindent \initialiv{É}{tonnez-le}\blindtext\par
\noindent \initialiv{Q}{ualitativement}\blindtext\par
\noindent \initialiv{V}{aloriser}\blindtext\par
\Blindtext
\phantomsection
\label{utopie}
\Blinddocument
\fi
\end{document}
