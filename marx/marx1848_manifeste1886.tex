%%%%%%%%%%%%%%%%%%%%%%%%%%%%%%%%%
% LaTeX model https://hurlus.fr %
%%%%%%%%%%%%%%%%%%%%%%%%%%%%%%%%%

% Needed before document class
\RequirePackage{pdftexcmds} % needed for tests expressions
\RequirePackage{fix-cm} % correct units

% Define mode
\def\mode{a4}

\newif\ifaiv % a4
\newif\ifav % a5
\newif\ifbooklet % booklet
\newif\ifcover % cover for booklet

\ifnum \strcmp{\mode}{cover}=0
  \covertrue
\else\ifnum \strcmp{\mode}{booklet}=0
  \booklettrue
\else\ifnum \strcmp{\mode}{a5}=0
  \avtrue
\else
  \aivtrue
\fi\fi\fi

\ifbooklet % do not enclose with {}
  \documentclass[french,twoside]{book} % ,notitlepage
  \usepackage[%
    papersize={105mm, 297mm},
    inner=12mm,
    outer=12mm,
    top=20mm,
    bottom=15mm,
    marginparsep=0pt,
  ]{geometry}
  \usepackage[fontsize=9.5pt]{scrextend} % for Roboto
\else\ifav
  \documentclass[french,twoside]{book} % ,notitlepage
  \usepackage[%
    a5paper,
    inner=25mm,
    outer=15mm,
    top=15mm,
    bottom=15mm,
    marginparsep=0pt,
  ]{geometry}
  \usepackage[fontsize=12pt]{scrextend}
\else% A4 2 cols
  \documentclass[twocolumn]{book}
  \usepackage[%
    a4paper,
    inner=15mm,
    outer=10mm,
    top=25mm,
    bottom=18mm,
    marginparsep=0pt,
  ]{geometry}
  \setlength{\columnsep}{20mm}
  \usepackage[fontsize=9.5pt]{scrextend}
\fi\fi

%%%%%%%%%%%%%%
% Alignments %
%%%%%%%%%%%%%%
% before teinte macros

\setlength{\arrayrulewidth}{0.2pt}
\setlength{\columnseprule}{\arrayrulewidth} % twocol
\setlength{\parskip}{0pt} % classical para with no margin
\setlength{\parindent}{1.5em}

%%%%%%%%%%
% Colors %
%%%%%%%%%%
% before Teinte macros

\usepackage[dvipsnames]{xcolor}
\definecolor{rubric}{HTML}{902c20} % the tonic
\def\columnseprulecolor{\color{rubric}}
\colorlet{borderline}{rubric!30!} % definecolor need exact code
\definecolor{shadecolor}{gray}{0.95}
\definecolor{bghi}{gray}{0.5}

%%%%%%%%%%%%%%%%%
% Teinte macros %
%%%%%%%%%%%%%%%%%
%%%%%%%%%%%%%%%%%%%%%%%%%%%%%%%%%%%%%%%%%%%%%%%%%%%
% <TEI> generic (LaTeX names generated by Teinte) %
%%%%%%%%%%%%%%%%%%%%%%%%%%%%%%%%%%%%%%%%%%%%%%%%%%%
% This template is inserted in a specific design
% It is XeLaTeX and otf fonts

\makeatletter % <@@@


\usepackage{blindtext} % generate text for testing
\usepackage{contour} % rounding words
\usepackage[nodayofweek]{datetime}
\usepackage{DejaVuSans} % font for symbols
\usepackage{enumitem} % <list>
\usepackage{etoolbox} % patch commands
\usepackage{fancyvrb}
\usepackage{fancyhdr}
\usepackage{fontspec} % XeLaTeX mandatory for fonts
\usepackage{footnote} % used to capture notes in minipage (ex: quote)
\usepackage{framed} % bordering correct with footnote hack
\usepackage{graphicx}
\usepackage{lettrine} % drop caps
\usepackage{lipsum} % generate text for testing
\usepackage[framemethod=tikz,]{mdframed} % maybe used for frame with footnotes inside
\usepackage{pdftexcmds} % needed for tests expressions
\usepackage{polyglossia} % non-break space french punct, bug Warning: "Failed to patch part"
\usepackage[%
  indentfirst=false,
  vskip=1em,
  noorphanfirst=true,
  noorphanafter=true,
  leftmargin=\parindent,
  rightmargin=0pt,
]{quoting}
\usepackage{ragged2e}
\usepackage{setspace}
\usepackage{tabularx} % <table>
\usepackage[explicit]{titlesec} % wear titles, !NO implicit
\usepackage{tikz} % ornaments
\usepackage{tocloft} % styling tocs
\usepackage[fit]{truncate} % used im runing titles
\usepackage{unicode-math}
\usepackage[normalem]{ulem} % breakable \uline, normalem is absolutely necessary to keep \emph
\usepackage{verse} % <l>
\usepackage{xcolor} % named colors
\usepackage{xparse} % @ifundefined
\XeTeXdefaultencoding "iso-8859-1" % bad encoding of xstring
\usepackage{xstring} % string tests
\XeTeXdefaultencoding "utf-8"
\PassOptionsToPackage{hyphens}{url} % before hyperref, which load url package
\usepackage{hyperref} % supposed to be the last one, :o) except for the ones to follow
\urlstyle{same} % after hyperref

% TOTEST
% \usepackage{hypcap} % links in caption ?
% \usepackage{marginnote}
% TESTED
% \usepackage{background} % doesn’t work with xetek
% \usepackage{bookmark} % prefers the hyperref hack \phantomsection
% \usepackage[color, leftbars]{changebar} % 2 cols doc, impossible to keep bar left
% \usepackage[utf8x]{inputenc} % inputenc package ignored with utf8 based engines
% \usepackage[sfdefault,medium]{inter} % no small caps
% \usepackage{firamath} % choose firasans instead, firamath unavailable in Ubuntu 21-04
% \usepackage{flushend} % bad for last notes, supposed flush end of columns
% \usepackage[stable]{footmisc} % BAD for complex notes https://texfaq.org/FAQ-ftnsect
% \usepackage{helvet} % not for XeLaTeX
% \usepackage{multicol} % not compatible with too much packages (longtable, framed, memoir…)
% \usepackage[default,oldstyle,scale=0.95]{opensans} % no small caps
% \usepackage{sectsty} % \chapterfont OBSOLETE
% \usepackage{soul} % \ul for underline, OBSOLETE with XeTeX
% \usepackage[breakable]{tcolorbox} % text styling gone, footnote hack not kept with breakable



% Metadata inserted by a program, from the TEI source, for title page and runing heads
\title{\textbf{ Le manifeste du parti communiste }\\ \medskip
\textit{ traduction Laura Marx-Lafargue, 1886 }\\ \medskip
Avec notes et préfaces de Friedrich Engels}
\date{1848}
\author{Marx \& Engels}
\def\elbibl{Marx \& Engels. 1848. \emph{Le manifeste du parti communiste}}
\def\elabstract{%
 
\labelblock{Les prolétaires de tous les pays ne vont pas s’unir}

 \noindent  Marx a reçu une éducation juive complète, se réfère souvent à \emph{la Bible}, c’est un prophète (athée). Il interroge l’histoire, dont il cherche le moteur et l’avenir. En 1848, il prophétisait la généralisation de l’exploitation rationnelle de la nature et de l’humanité. En conséquence, la planète tout entière serait soumise au calcul. Il en déduisait que partageant les mêmes conditions d’exploitation, l’humanité s’unirait pour renverser ses exploiteurs. \par
 
\begin{quoteblock}
 \noindent « La Bourgeoisie produit avant tout ses propres fossoyeurs. Sa chute et la victoire du Prolétariat sont également inévitables. » (\hyperref[fossoyeur]{\dotuline{32}})\par
 « À la place de l’ancienne société bourgeoise, avec ses classes et ses antagonismes de classes, surgit une association où le libre développement de chacun est la condition du libre développement pour tous. » (\hyperref[utopie]{\dotuline{54}})
 \end{quoteblock}

 \noindent  Après un siècle et demi d’expériences parfois meurtrières, il faut l’admettre, ce n’est pas ce qui s’est passé. La guerre de 1914 a par exemple montré que les ouvriers d’Europe avaient non seulement une patrie, mais qu’ils étaient même prêts à mourir par millions pour elle. \par
 \bigbreak
 \noindent  Ce siècle numérique généralise encore plus efficacement l’exploitation par le calcul, et nous n’assistons pas à une conscientisation universelle de l’humanité, au contraire.  \par
  Avec les écrans, des individus ont l’impression de s’ouvrir au monde entier, de sortir de leur famille et de leur quartier, qui ne les reconnaissent pas à la valeur qu’ils s’imaginent avoir. Les grands réseaux sociaux sont les plus grandes capitalisations boursières. Les masses leur abandonnent le meilleur de leur vie, elles se soumettent volontairement au Capital, elles se numérisent, pour mendier une flatterie. \par
  Mais plus les individus partagent les mêmes logiciels et les mêmes interfaces, plus ils prennent peur de perdre leur âme singulière, et plus ils jettent leurs identités imaginaires à la figure des étrangers, avec les mêmes mèmes. La personne violentée par la standardisation de ses conditions matérielles réagit d’abord par la fureur identitaire. \par
 
\labelblock{Il ne faut pas lire \emph{le Manifeste} en attendant que la prophétie se réalise, mais comme une invitation à repenser l’histoire, et à agir.}

 \noindent  Le parti communiste du temps de Marx ou de Lafargue était une assemblée houleuse mais encore pleine de débats honnêtes, d’amitiés et de combats ; loin des sinistres armées totalitaires à broyer les individus, ou des actuelles machines électorales à répéter les éléments de langage. \par
  Le \emph{Manifeste} témoigne d’une grande vitalité de la pensée socialiste d’alors, cataloguant des impasses encore tentantes aujourd’hui. Il faut transposer un peu, mais le « \emph{socialisme féodal} » (\hyperref[III1a]{\dotuline{III.1.a}}) semble déjà le royalisme corporatif de Maurras (au programme de Pétain ou de l’extrême-droite chrétienne actuelle) ; le « \emph{socialisme conservateur et bourgeois} » (\hyperref[III2]{\dotuline{III.2}}) caractérise bien le PS français depuis plusieurs décennies ; et sous les socialismes « \emph{critico-utopique} » (\hyperref[III3]{\dotuline{III.3}}), on trouvera les inventeurs de constitutions, de tirages au sort et autres revenus universels. \par
  Marx ne se mélange pas les repères et se moque déjà des unions électoralistes des gauches. Une alliance n’est pas une affaire de personnes, de bonnes ou de mauvaises intentions, mais un calcul du rapport des forces historiques. \par
 
\begin{quoteblock}
\noindent « En France, les communistes se rallient au parti démocrate-socialiste contre les bourgeoisies radicales [=PS] et conservatrices [=droite], tout en se réservant le droit de critiquer les phrases et les illusions léguées par la tradition révolutionnaire. » (\hyperref[par76]{\dotuline{76}})\end{quoteblock}

 \bigbreak
 \noindent  Le \emph{Manifeste} est ici donné dans la \href{https://fr.wikisource.org/wiki/Manifeste_du_parti_communiste/Lafargue}{\dotuline{première traduction en français de 1886, par Laura Marx-Lafargue}}\footnote{\href{https://fr.wikisource.org/wiki/Manifeste_du_parti_communiste/Lafargue}{\url{https://fr.wikisource.org/wiki/Manifeste_du_parti_communiste/Lafargue}}} (1845, 1911), fille de Karl Marx et mariée à Paul Lafargue (1880, \emph{Le Droit à la paresse}). Y ont été reportées les innovations de Andler (\href{https://fr.wikisource.org/wiki/Manifeste_du_parti_communiste/Andler}{\dotuline{traduction 1901}}\footnote{\href{https://fr.wikisource.org/wiki/Manifeste_du_parti_communiste/Andler}{\url{https://fr.wikisource.org/wiki/Manifeste_du_parti_communiste/Andler}}}) : une numéroration commode, les préfaces à d’autres éditions, ainsi que les notes de Engels. Quelques intertitres ont été insérés, pour aider à se reprérer dans un texte désormais historique, où chaque ligne compte, comme la loi, ou bien \emph{la Bible}. 
 
\clearpage
}
\def\elsource{ \href{https://fr.wikisource.org/wiki/Manifeste_du_parti_communiste/Lafargue}{\dotuline{Wikisource}}\footnote{\href{https://fr.wikisource.org/wiki/Manifeste_du_parti_communiste/Lafargue}{\url{https://fr.wikisource.org/wiki/Manifeste_du_parti_communiste/Lafargue}}} }

% Default metas
\newcommand{\colorprovide}[2]{\@ifundefinedcolor{#1}{\colorlet{#1}{#2}}{}}
\colorprovide{rubric}{red}
\colorprovide{silver}{Gray}
\@ifundefined{syms}{\newfontfamily\syms{DejaVu Sans}}{}
\newif\ifdev
\@ifundefined{elbibl}{% No meta defined, maybe dev mode
  \newcommand{\elbibl}{Titre court ?}
  \newcommand{\elbook}{Titre du livre source ?}
  \newcommand{\elabstract}{Résumé\par}
  \newcommand{\elurl}{http://oeuvres.github.io/elbook/2}
  \author{Éric Lœchien}
  \title{Un titre de test assez long pour vérifier le comportement d’une maquette}
  \date{1566}
  \devtrue
}{}
\let\eltitle\@title
\let\elauthor\@author
\let\eldate\@date


\defaultfontfeatures{
  % Mapping=tex-text, % no effect seen
  Scale=MatchLowercase,
  Ligatures={TeX,Common},
}

\@ifundefined{\columnseprulecolor}{%
    \patchcmd\@outputdblcol{% find
      \normalcolor\vrule
    }{% and replace by
      \columnseprulecolor\vrule
    }{% success
    }{% failure
      \@latex@warning{Patching \string\@outputdblcol\space failed}%
    }
}{}

\hypersetup{
  % pdftex, % no effect
  pdftitle={\elbibl},
  % pdfauthor={Your name here},
  % pdfsubject={Your subject here},
  % pdfkeywords={keyword1, keyword2},
  bookmarksnumbered=true,
  bookmarksopen=true,
  bookmarksopenlevel=1,
  pdfstartview=Fit,
  breaklinks=true, % avoid long links
  pdfpagemode=UseOutlines,    % pdf toc
  hyperfootnotes=true,
  colorlinks=false,
  pdfborder=0 0 0,
  % pdfpagelayout=TwoPageRight,
  % linktocpage=true, % NO, toc, link only on page no
}


% generic typo commands
\newcommand{\astermono}{\medskip\centerline{\color{rubric}\large\selectfont{\syms ✻}}\medskip\par}%
\newcommand{\astertri}{\medskip\par\centerline{\color{rubric}\large\selectfont{\syms ✻\,✻\,✻}}\medskip\par}%
\newcommand{\asterism}{\bigskip\par\noindent\parbox{\linewidth}{\centering\color{rubric}\large{\syms ✻}\\{\syms ✻}\hskip 0.75em{\syms ✻}}\bigskip\par}%

% lists
\newlength{\listmod}
\setlength{\listmod}{\parindent}
\setlist{
  itemindent=!,
  listparindent=\listmod,
  labelsep=0.2\listmod,
  parsep=0pt,
  % topsep=0.2em, % default topsep is best
}
\setlist[itemize]{
  label=—,
  leftmargin=0pt,
  labelindent=1.2em,
  labelwidth=0pt,
}
\setlist[enumerate]{
  label={\bf\color{rubric}\arabic*.},
  labelindent=0.8\listmod,
  leftmargin=\listmod,
  labelwidth=0pt,
}
\newlist{listalpha}{enumerate}{1}
\setlist[listalpha]{
  label={\bf\color{rubric}\alph*.},
  leftmargin=0pt,
  labelindent=0.8\listmod,
  labelwidth=0pt,
}
\newcommand{\listhead}[1]{\hspace{-1\listmod}\emph{#1}}

\renewcommand{\hrulefill}{%
  \leavevmode\leaders\hrule height 0.2pt\hfill\kern\z@}

% General typo
\DeclareTextFontCommand{\textlarge}{\large}
\DeclareTextFontCommand{\textsmall}{\small}


% commands, inlines
\newcommand{\anchor}[1]{\Hy@raisedlink{\hypertarget{#1}{}}} % link to top of an anchor (not baseline)
\newcommand\abbr{}
\newcommand{\autour}[1]{\tikz[baseline=(X.base)]\node [draw=rubric,thin,rectangle,inner sep=1.5pt, rounded corners=3pt] (X) {\color{rubric}#1};}
\newcommand\corr{}
\newcommand{\ed}[1]{ {\color{silver}\sffamily\footnotesize (#1)} } % <milestone ed="1688"/>
\newcommand\expan{}
\newcommand\gap{}
\renewcommand{\LettrineFontHook}{\color{rubric}}
\newcommand{\initial}[2]{\lettrine[lines=2, loversize=0.3, lhang=0.3]{#1}{#2}}
\newcommand{\initialiv}[2]{%
  \let\oldLFH\LettrineFontHook
  % \renewcommand{\LettrineFontHook}{\color{rubric}\ttfamily}
  \IfSubStr{QJ’}{#1}{
    \lettrine[lines=4, lhang=0.2, loversize=-0.1, lraise=0.2]{\smash{#1}}{#2}
  }{\IfSubStr{É}{#1}{
    \lettrine[lines=4, lhang=0.2, loversize=-0, lraise=0]{\smash{#1}}{#2}
  }{\IfSubStr{ÀÂ}{#1}{
    \lettrine[lines=4, lhang=0.2, loversize=-0, lraise=0, slope=0.6em]{\smash{#1}}{#2}
  }{\IfSubStr{A}{#1}{
    \lettrine[lines=4, lhang=0.2, loversize=0.2, slope=0.6em]{\smash{#1}}{#2}
  }{\IfSubStr{V}{#1}{
    \lettrine[lines=4, lhang=0.2, loversize=0.2, slope=-0.5em]{\smash{#1}}{#2}
  }{
    \lettrine[lines=4, lhang=0.2, loversize=0.2]{\smash{#1}}{#2}
  }}}}}
  \let\LettrineFontHook\oldLFH
}
\newcommand{\labelchar}[1]{\textbf{\color{rubric} #1}}
\newcommand{\milestone}[1]{\autour{\footnotesize\color{rubric} #1}} % <milestone n="4"/>
\newcommand\name{}
\newcommand\orig{}
\newcommand\orgName{}
\newcommand\persName{}
\newcommand\placeName{}
\newcommand{\pn}[1]{{\sffamily\textbf{#1.}} } % <p n="3"/>
\newcommand\reg{}
% \newcommand\ref{} % already defined
\newcommand\sic{}
\def\mednobreak{\ifdim\lastskip<\medskipamount
  \removelastskip\nopagebreak\medskip\fi}
\def\bignobreak{\ifdim\lastskip<\bigskipamount
  \removelastskip\nopagebreak\bigskip\fi}

% commands, blocks
\newcommand{\byline}[1]{\bigskip{\RaggedLeft{#1}\par}\bigskip}
\newcommand{\bibl}[1]{{\RaggedLeft{#1}\par\bigskip}}
\newcommand{\biblitem}[1]{{\noindent\hangindent=\parindent   #1\par}}
\newcommand{\dateline}[1]{\medskip{\RaggedLeft{#1}\par}\bigskip}
\newcommand{\labelblock}[1]{\medbreak{\noindent\color{rubric}\bfseries #1}\par\mednobreak}
\newcommand{\salute}[1]{\bigbreak{#1}\par\medbreak}
\newcommand{\signed}[1]{\bigbreak\filbreak{\raggedleft #1\par}\medskip}

% environments for blocks (some may become commands)
\newenvironment{borderbox}{}{} % framing content
\newenvironment{citbibl}{\ifvmode\hfill\fi}{\ifvmode\par\fi }
\newenvironment{docAuthor}{\ifvmode\vskip4pt\fontsize{16pt}{18pt}\selectfont\fi\itshape}{\ifvmode\par\fi }
\newenvironment{docDate}{}{\ifvmode\par\fi }
\newenvironment{docImprint}{\vskip6pt}{\ifvmode\par\fi }
\newenvironment{docTitle}{\vskip6pt\bfseries\fontsize{18pt}{22pt}\selectfont}{\par }
\newenvironment{msHead}{\vskip6pt}{\par}
\newenvironment{msItem}{\vskip6pt}{\par}
\newenvironment{titlePart}{}{\par }


% environments for block containers
\newenvironment{argument}{\fontlight\parindent0pt}{\vskip1.5em}
\newenvironment{biblfree}{}{\ifvmode\par\fi }
\newenvironment{bibitemlist}[1]{%
  \list{\@biblabel{\@arabic\c@enumiv}}%
  {%
    \settowidth\labelwidth{\@biblabel{#1}}%
    \leftmargin\labelwidth
    \advance\leftmargin\labelsep
    \@openbib@code
    \usecounter{enumiv}%
    \let\p@enumiv\@empty
    \renewcommand\theenumiv{\@arabic\c@enumiv}%
  }
  \sloppy
  \clubpenalty4000
  \@clubpenalty \clubpenalty
  \widowpenalty4000%
  \sfcode`\.\@m
}%
{\def\@noitemerr
  {\@latex@warning{Empty `bibitemlist' environment}}%
\endlist}
\newenvironment{quoteblock}% may be used for ornaments
  {\begin{quoting}}
  {\end{quoting}}

% table () is preceded and finished by custom command
\newcommand{\tableopen}[1]{%
  \ifnum\strcmp{#1}{wide}=0{%
    \begin{center}
  }
  \else\ifnum\strcmp{#1}{long}=0{%
    \begin{center}
  }
  \else{%
    \begin{center}
  }
  \fi\fi
}
\newcommand{\tableclose}[1]{%
  \ifnum\strcmp{#1}{wide}=0{%
    \end{center}
  }
  \else\ifnum\strcmp{#1}{long}=0{%
    \end{center}
  }
  \else{%
    \end{center}
  }
  \fi\fi
}


% text structure
\newcommand\chapteropen{} % before chapter title
\newcommand\chaptercont{} % after title, argument, epigraph…
\newcommand\chapterclose{} % maybe useful for multicol settings
\setcounter{secnumdepth}{-2} % no counters for hierarchy titles
\setcounter{tocdepth}{5} % deep toc
\markright{\@title} % ???
\markboth{\@title}{\@author} % ???
\renewcommand\tableofcontents{\@starttoc{toc}}
% toclof format
% \renewcommand{\@tocrmarg}{0.1em} % Useless command?
% \renewcommand{\@pnumwidth}{0.5em} % {1.75em}
\renewcommand{\@cftmaketoctitle}{}
\setlength{\cftbeforesecskip}{\z@ \@plus.2\p@}
\renewcommand{\cftchapfont}{}
\renewcommand{\cftchapdotsep}{\cftdotsep}
\renewcommand{\cftchapleader}{\normalfont\cftdotfill{\cftchapdotsep}}
\renewcommand{\cftchappagefont}{\bfseries}
\setlength{\cftbeforechapskip}{0em \@plus\p@}
% \renewcommand{\cftsecfont}{\small\relax}
\renewcommand{\cftsecpagefont}{\normalfont}
% \renewcommand{\cftsubsecfont}{\small\relax}
\renewcommand{\cftsecdotsep}{\cftdotsep}
\renewcommand{\cftsecpagefont}{\normalfont}
\renewcommand{\cftsecleader}{\normalfont\cftdotfill{\cftsecdotsep}}
\setlength{\cftsecindent}{1em}
\setlength{\cftsubsecindent}{2em}
\setlength{\cftsubsubsecindent}{3em}
\setlength{\cftchapnumwidth}{1em}
\setlength{\cftsecnumwidth}{1em}
\setlength{\cftsubsecnumwidth}{1em}
\setlength{\cftsubsubsecnumwidth}{1em}

% footnotes
\newif\ifheading
\newcommand*{\fnmarkscale}{\ifheading 0.70 \else 1 \fi}
\renewcommand\footnoterule{\vspace*{0.3cm}\hrule height \arrayrulewidth width 3cm \vspace*{0.3cm}}
\setlength\footnotesep{1.5\footnotesep} % footnote separator
\renewcommand\@makefntext[1]{\parindent 1.5em \noindent \hb@xt@1.8em{\hss{\normalfont\@thefnmark . }}#1} % no superscipt in foot


% orphans and widows
\clubpenalty=9996
\widowpenalty=9999
\brokenpenalty=4991
\predisplaypenalty=10000
\postdisplaypenalty=1549
\displaywidowpenalty=1602
\hyphenpenalty=400
% Copied from Rahtz but not understood
\def\@pnumwidth{1.55em}
\def\@tocrmarg {2.55em}
\def\@dotsep{4.5}
\emergencystretch 3em
\hbadness=4000
\pretolerance=750
\tolerance=2000
\vbadness=4000
\def\Gin@extensions{.pdf,.png,.jpg,.mps,.tif}
% \renewcommand{\@cite}[1]{#1} % biblio

\makeatother % /@@@>
%%%%%%%%%%%%%%
% </TEI> end %
%%%%%%%%%%%%%%


%%%%%%%%%%%%%
% footnotes %
%%%%%%%%%%%%%
\renewcommand{\thefootnote}{\bfseries\textcolor{rubric}{\arabic{footnote}}} % color for footnote marks

%%%%%%%%%
% Fonts %
%%%%%%%%%
\usepackage[]{roboto} % SmallCaps, Regular is a bit bold
% \linespread{0.90} % too compact, keep font natural
\newfontfamily\fontrun[]{Roboto Condensed Light} % condensed runing heads
\ifav
  \setmainfont[
    ItalicFont={Roboto Light Italic},
  ]{Roboto}
\else\ifbooklet
  \setmainfont[
    ItalicFont={Roboto Light Italic},
  ]{Roboto}
\else
\setmainfont[
  ItalicFont={Roboto Italic},
]{Roboto Light}
\fi\fi
\renewcommand{\LettrineFontHook}{\bfseries\color{rubric}}
% \renewenvironment{labelblock}{\begin{center}\bfseries\color{rubric}}{\end{center}}

%%%%%%%%
% MISC %
%%%%%%%%

\setdefaultlanguage[frenchpart=false]{french} % bug on part


\newenvironment{quotebar}{%
    \def\FrameCommand{{\color{rubric!10!}\vrule width 0.5em} \hspace{0.9em}}%
    \def\OuterFrameSep{\itemsep} % séparateur vertical
    \MakeFramed {\advance\hsize-\width \FrameRestore}
  }%
  {%
    \endMakeFramed
  }
\renewenvironment{quoteblock}% may be used for ornaments
  {%
    \savenotes
    \setstretch{0.9}
    \normalfont
    \begin{quotebar}
  }
  {%
    \end{quotebar}
    \spewnotes
  }


\renewcommand{\pn}[1]{{\footnotesize\autour{ #1}}} % <p n="3"/>
\renewcommand{\headrulewidth}{\arrayrulewidth}
\renewcommand{\headrule}{{\color{rubric}\hrule}}

% delicate tuning, image has produce line-height problems in title on 2 lines
\titleformat{name=\chapter} % command
  [display] % shape
  {\vspace{1.5em}\centering} % format
  {} % label
  {0pt} % separator between n
  {}
[{\color{rubric}\huge\textbf{#1}}\bigskip] % after code
% \titlespacing{command}{left spacing}{before spacing}{after spacing}[right]
\titlespacing*{\chapter}{0pt}{-2em}{0pt}[0pt]

\titleformat{name=\section}
  [block]{}{}{}{}
  [\vbox{\color{rubric}\large\raggedleft\textbf{#1}}]
\titlespacing{\section}{0pt}{0pt plus 4pt minus 2pt}{\baselineskip}

\titleformat{name=\subsection}
  [block]
  {}
  {} % \thesection
  {} % separator \arrayrulewidth
  {}
[\vbox{\large\textbf{#1}}]
% \titlespacing{\subsection}{0pt}{0pt plus 4pt minus 2pt}{\baselineskip}

\ifaiv
  \fancypagestyle{main}{%
    \fancyhf{}
    \setlength{\headheight}{1.5em}
    \fancyhead{} % reset head
    \fancyfoot{} % reset foot
    \fancyhead[L]{\truncate{0.45\headwidth}{\fontrun\elbibl}} % book ref
    \fancyhead[R]{\truncate{0.45\headwidth}{ \fontrun\nouppercase\leftmark}} % Chapter title
    \fancyhead[C]{\thepage}
  }
  \fancypagestyle{plain}{% apply to chapter
    \fancyhf{}% clear all header and footer fields
    \setlength{\headheight}{1.5em}
    \fancyhead[L]{\truncate{0.9\headwidth}{\fontrun\elbibl}}
    \fancyhead[R]{\thepage}
  }
\else
  \fancypagestyle{main}{%
    \fancyhf{}
    \setlength{\headheight}{1.5em}
    \fancyhead{} % reset head
    \fancyfoot{} % reset foot
    \fancyhead[RE]{\truncate{0.9\headwidth}{\fontrun\elbibl}} % book ref
    \fancyhead[LO]{\truncate{0.9\headwidth}{\fontrun\nouppercase\leftmark}} % Chapter title, \nouppercase needed
    \fancyhead[RO,LE]{\thepage}
  }
  \fancypagestyle{plain}{% apply to chapter
    \fancyhf{}% clear all header and footer fields
    \setlength{\headheight}{1.5em}
    \fancyhead[L]{\truncate{0.9\headwidth}{\fontrun\elbibl}}
    \fancyhead[R]{\thepage}
  }
\fi

\ifav % a5 only
  \titleclass{\section}{top}
\fi

\newcommand\chapo{{%
  \vspace*{-3em}
  \centering % no vskip ()
  {\Large\addfontfeature{LetterSpace=25}\bfseries{\elauthor}}\par
  \smallskip
  {\large\eldate}\par
  \bigskip
  {\Large\selectfont{\eltitle}}\par
  \bigskip
  {\color{rubric}\hline\par}
  \bigskip
  {\Large LIVRE LIBRE À PRIX LIBRE, DEMANDEZ AU COMPTOIR\par}
  \centerline{\small\color{rubric} {hurlus.fr, tiré le \today}}\par
  \bigskip
}}


\begin{document}
\pagestyle{empty}
\ifbooklet{
  \thispagestyle{empty}
  \centering
  {\LARGE\bfseries{\elauthor}}\par
  \bigskip
  {\Large\eldate}\par
  \bigskip
  \bigskip
  {\LARGE\selectfont{\eltitle}}\par
  \vfill\null
  {\color{rubric}\setlength{\arrayrulewidth}{2pt}\hline\par}
  \vfill\null
  {\Large LIVRE LIBRE À PRIX LIBRE, DEMANDEZ AU COMPTOIR\par}
  \centerline{\small{hurlus.fr, tiré le \today}}\par
  \newpage\null\thispagestyle{empty}\newpage
  \addtocounter{page}{-2}
}\fi

\thispagestyle{empty}
\ifaiv
  \twocolumn[\chapo]
\else
  \chapo
\fi
{\it\elabstract}
\bigskip
\makeatletter\@starttoc{toc}\makeatother % toc without new page
\bigskip

\pagestyle{main} % after style

  \renewcommand{\leftmark}{Prologue}
\section[Prologue]{Prologue}
\noindent \initialiv{U}{n spectre hante l’Europe}, le spectre du communisme. Toutes les puissances de la vieille Europe se sont unies en une Sainte-Alliance pour traquer ce spectre : le Pape et le Czar, Metternich et Guizot, les radicaux de France et les policiers d’Allemagne.\par
Quelle est l’opposition que n’ont pas accusée de communisme ses adversaires au pouvoir ? Quelle est l’opposition qui, à son tour, n’a pas relancé à ses adversaires de droite ou de gauche l’épithète flétrissante de communiste ?\par
Deux choses ressortent de ces faits :\par

\begin{enumerate}[itemsep=0pt,]
\item Déjà le communisme est reconnu par toutes les puissances d’Europe comme une puissance ;
\item Il est grand temps que les communistes exposent, à la face du monde entier, leur manière de voir, leurs buts et leurs tendances ; qu’ils opposent au conte du spectre communiste un manifeste du parti.

\end{enumerate}\noindent Dans ce but, des communistes de diverses nationalités se sont réunis à Londres et ont rédigé le manifeste suivant, qui sera publié en anglais, français, allemand, italien, flamand et danois.

\chapteropen
\renewcommand{\leftmark}{I. Bourgeois et prolétaires }
\chapter[I. Bourgeois et prolétaires ]{I. Bourgeois et prolétaires \protect\footnotemark }\phantomsection
\label{I}
\footnotetext{(Engels, 1888, ed. anglaise) On entend par bourgeoisie la classe des capitalistes modernes, propriétaires des moyens de production sociale et qui emploient le travail salarié. On entend par prolétariat la classe des ouvriers salariés modernes qui, privés de leurs propres moyens de production, sont obligés pour subsister, de vendre leur force de travail.}

\chaptercont
\noindent \initialiv{L\kern-0.08em{’}}{histoire de toute société} jusqu’à nos jours  n’a été que l’histoire de luttes de classes. Hommes libres et esclaves, patriciens et plébéiens, barons et serfs, maîtres de jurandes \footnote{(Engels, 1888, ed. anglaise) Maître de jurande, c’est-à-dire membre de plein droit d’une corporation, maître du corps de métier et non juré.} et compagnons, en un mot oppresseurs et opprimés, en opposition constante, ont mené une guerre ininterrompue, tantôt ouverte, tantôt dissimulée ; une guerre qui finissait toujours, ou par une transformation révolutionnaire de la société tout entière, ou par la destruction des deux classes en lutte.\par
Dans les premières époques historiques, nous constatons presque partout une division hiérarchique de la société, une échelle graduée de positions sociales. Dans la Rome antique, nous trouvons des patriciens, des chevaliers, des plébéiens, des esclaves ; au Moyen Âge, des seigneurs, des vassaux, des maîtres, des compagnons, des serfs ; et dans chacune de ces classes, des gradations spéciales.\par
\bigbreak
\noindent \phantomsection
\label{par3}\pn{3} La société bourgeoise moderne, élevée sur les ruines de la société féodale, n’a pas aboli les antagonismes de classes. Elle n’a fait que substituer aux anciennes, de nouvelles classes, de nouvelles conditions d’oppression, de nouvelles formes de lutte.\par
Cependant, le caractère distinctif de notre époque, de l’ère de la Bourgeoisie, est d’avoir simplifié les antagonismes de classes. La société se divise de plus en plus en deux vastes camps opposés, en deux classes ennemies : la Bourgeoisie et le Prolétariat.\par
Des serfs du Moyen Âge naquirent les éléments des premières communes ; de cette population municipale sortirent les éléments constitutifs de la Bourgeoisie.\par
La découverte de l’Amérique, la circumnavigation de l’Afrique, offrirent à la Bourgeoisie naissante un nouveau champ d’action. Les marchés de l’Inde et de la Chine, la colonisation de l’Amérique, le commerce colonial, l’accroissement des moyens d’échange et des marchandises, imprimèrent une impulsion, inconnue jusqu’alors, au commerce, à la navigation, à l’industrie, et assurèrent, par conséquent, un rapide développement à l’élément révolutionnaire de la société féodale en dissolution.\par

\labelblock{La bourgeoisie dans l’histoire}

\noindent \phantomsection
\label{par4}\pn{4} L’ancien mode de production ne pouvait plus satisfaire aux besoins qui croissaient avec l’ouverture de nouveaux marchés. Le métier, entouré de privilèges féodaux, fut remplacé par la manufacture. La petite bourgeoisie industrielle supplanta les maîtres de jurande ; la division du travail entre les différentes corporations disparut devant la division du travail dans l’atelier même.\par
Mais les marchés s’agrandissaient sans cesse : la demande croissait toujours. La manufacture, elle aussi, devint insuffisante ; alors la machine et la vapeur révolutionnèrent la production industrielle. La grande industrie moderne supplanta la manufacture ; la petite bourgeoisie manufacturière céda la place aux industriels millionnaires, – chefs d’armées de travailleurs, – aux bourgeois modernes.\par
La grande industrie a créé le marché mondial, préparé par la découverte de l’Amérique. Le marché mondial accéléra prodigieusement le développement du commerce, de la navigation, de tous les moyens de communication. Ce développement réagit à son tour sur la marche de l’industrie ; et au fur et à mesure que l’industrie, le commerce, la navigation, les chemins de fer se développaient, la Bourgeoisie grandissait, décuplant ses capitaux et refoulant à l’arrière-plan les classes léguées par le Moyen Âge.\par
\bigbreak
\noindent \phantomsection
\label{par5}\pn{5} La Bourgeoisie, nous le voyons, est elle-même le produit d’un long développement, d’une série de révolutions dans les modes de production et de communication.\par
Chaque étape de l’évolution parcourue par la Bourgeoisie était accompagnée d’un progrès correspondant.\par
État opprimé par le despotisme féodal, association se gouvernant elle-même dans la commune \footnote{ \noindent (Engels, 1888, ed. anglaise) On désignait sous le nom de communes les villes qui surgissaient en France avant même qu’elles eussent conquis sur leurs seigneurs et maîtres féodaux l’autonomie locale et les droits politiques du \emph{tiers état}. D’une façon générale, l’Angleterre apparaît ici en tant que pays type du développement économique de la bourgeoisie ; la France en tant que pays type de son développement politique.\par
 (Engels, 1890, ed. allemande) C’est ainsi que les habitants des villes, en Italie et en France appelaient leur communauté urbaine, une fois achetés ou arrachés à leurs seigneurs féodaux leurs premiers droits à une administration autonome.
} ; ici république municipale, là tiers-état taxable de la monarchie ; puis, durant la période manufacturière, contrepoids de la noblesse dans les monarchies limitées ou absolues ; pierre angulaire des grandes monarchies, la Bourgeoisie, depuis l’établissement de la grande industrie et du marché mondial, s’est enfin emparée du pouvoir politique – à l’exclusion des autres classes, – dans l’État représentatif moderne. Le gouvernement moderne n’est qu’un comité administratif des affaires de la classe bourgeoise.\par
\bigbreak
\noindent \phantomsection
\label{par6}\pn{6} La bourgeoisie a joué dans l’histoire un rôle essentiellement révolutionnaire.\par
Partout où elle a conquis le pouvoir, elle a foulé aux pieds les relations féodales, patriarcales et idylliques. Tous les liens multicolores qui unissaient l’homme féodal à ses supérieurs naturels, elle les a brisés sans pitié, pour ne laisser subsister d’autre lien entre l’homme et l’homme que le froid intérêt, que le dur \emph{argent comptant}. Elle a noyé l’extase religieuse, l’enthousiasme chevaleresque, la sentimentalité du petit-bourgeois, dans les eaux glacées du calcul égoïste. Elle a fait de la dignité personnelle une simple valeur d’échange ; elle a substitué aux nombreuses libertés, si chèrement conquises, l’unique et impitoyable liberté du commerce. En un mot, à la place de l’exploitation, voilée par des illusions religieuses et politiques, elle a mis une exploitation ouverte, directe, brutale et éhontée.\par
La Bourgeoisie a dépouillé de leur auréole toutes les professions jusqu’alors réputées vénérables, et vénérées. Du médecin, du juriste, du prêtre, du poète, du savant, elle a fait des travailleurs salariés.\par
La Bourgeoisie a déchiré le voile de sentimentalité qui recouvrait les relations de famille et les a réduites à n’être que de simples rapports d’argent.\par
La Bourgeoisie a démontré comment la brutale manifestation de la force au Moyen-Âge, si admirée de la réaction, trouve son complément naturel dans la plus crasse paresse. C’est elle qui, la première, a prouvé ce que peut accomplir l’activité humaine : elle a créé bien d’autres merveilles que les pyramides d’Égypte, les aqueducs romains, les cathédrales gothiques ; elle a conduit bien d’autres expéditions que les antiques migrations de peuples et les croisades.\par

\labelblock{Le progrès}

\noindent \phantomsection
\label{par7}\pn{7} La Bourgeoisie n’existe qu’à la condition de révolutionner sans cesse les instruments de travail, ce qui veut dire le mode de production, ce qui veut dire tous les rapports sociaux. La conservation de l’ancien mode de production était, au contraire, la première condition d’existence de toutes les classes industrielles antérieures. Ce bouleversement continuel des modes de production, ce constant ébranlement de tout le système social, cette agitation et cette insécurité perpétuelles, distinguent l’époque bourgeoise de toutes les précédentes. Tous les rapports sociaux traditionnels et figés, avec leur cortège de croyances et d’idées admises et vénérées, se dissolvent ; celles qui les remplacent deviennent surannées avant de se cristalliser. Tout ce qui était solide et stable est ébranlé, tout ce qui était sacré est profané, et les hommes sont forcés, enfin, d’envisager leurs conditions d’existence et leurs relations réciproques avec des yeux dégrisés.\par
\bigbreak
\noindent \phantomsection
\label{par8}\pn{8} Poussée par le besoin de débouchés toujours nouveaux, la bourgeoisie envahit le globe entier. Il lui faut pénétrer partout, s’établir partout, créer partout des moyens de communication.\par
Par l’exploitation du marché mondial, la bourgeoisie donne un caractère cosmopolite à la production de tous les pays. Au désespoir des réactionnaires, elle a enlevé à l’industrie sa base nationale. Les vieilles industries nationales sont détruites, ou sur le point de l’être. Elles sont supplantées par de nouvelles industries dont l’introduction devient une question vitale pour toutes les nations civilisées, industries qui n’emploient plus des matières premières indigènes, mais des matières premières venues des régions les plus éloignées, et dont les produits se consomment non seulement dans le pays même, mais dans tous les coins du globe.\par
\phantomsection
\label{utopie}À la place des anciens besoins, satisfaits par les produits nationaux, naissent de nouveaux besoins, réclamant pour leur satisfaction les produits des contrées les plus lointaines et des climats les plus divers. À la place de l’ancien isolement des nations se suffisant à elles-mêmes, se développe un trafic universel, une interdépendance des nations. Et ce qui est vrai pour la production matérielle s’applique à la production intellectuelle. Les productions intellectuelles d’une nation deviennent la propriété commune de toutes. L’étroitesse et l’exclusivisme nationaux deviennent de jour en jour plus impossibles ; des nombreuses littératures nationales et locales se forme une littérature universelle.\par
\bigbreak
\noindent \phantomsection
\label{par9}\pn{9} Par le rapide développement des instruments de production et des moyens de communication, la bourgeoisie entraîne dans le courant de la civilisation jusqu’aux nations les plus barbares. Le bon marché de ses produits est la grosse artillerie qui bat en brèche toutes les murailles de Chine et fait capituler les barbares les plus opiniâtrement hostiles aux étrangers. Sous peine de mort elle force toutes les nations à adopter le mode de production bourgeois. En un mot, elle modèle le monde à son image.\par
\bigbreak
\noindent \phantomsection
\label{par10}\pn{10} La Bourgeoisie a soumis la campagne à la ville. Elle a créé d’énormes cités ; elle a prodigieusement augmenté la population des villes aux dépens de celle des campagnes, et par là, elle a préservé une grande partie de la population de l’idiotisme de la vie des champs. De même qu’elle a subordonné la campagne à la ville, les nations barbares ou demi-civilisées aux nations civilisées, elle a subordonné les pays agricoles aux pays industriels, l’Orient à l’Occident.\par
\bigbreak
\noindent \phantomsection
\label{par11}\pn{11} La Bourgeoisie supprime de plus en plus l’éparpillement des moyens de production, de la propriété et de la population. Elle a aggloméré les populations, centralisé les moyens de production et concentré la propriété dans les mains de quelques individus. La conséquence fatale de ces changements a été la centralisation politique. Des provinces indépendantes, reliées entre elles par des liens fédéraux, mais ayant des intérêts, des lois, des gouvernements, des tarifs douaniers différents, ont été réunies en une seule nation, sous un seul gouvernement, une seule loi, un seul tarif douanier et un seul intérêt national de classe.\par
\bigbreak
\noindent \phantomsection
\label{par12}\pn{12} La Bourgeoisie, depuis son avènement, à peine séculaire, a créé des forces productives plus variées et plus colossales que toutes les générations passées prises ensemble. La subjugation des forces de la nature, les machines, l’application de la chimie à l’industrie et à l’agriculture, la navigation à vapeur, les chemins de fer, les télégraphes électriques, le défrichement de continents entiers, la canalisation des rivières, des populations entières sortant de terre comme par enchantement, quel siècle antérieur a soupçonné que de pareilles forces productives dormaient dans le travail social ?\par
\bigbreak
\noindent \phantomsection
\label{par13}\pn{13} Voici donc ce que nous avons vu : les moyens de production et d’échange servant de base à l’évolution bourgeoise furent créés dans le sein de la société féodale. À un certain degré du développement de ces moyens de production et d’échange, les conditions dans lesquelles la société féodale produisait et échangeait ses produits, l’organisation féodale de l’industrie et de la manufacture, en un mot, les rapports de la propriété féodale, cessèrent de correspondre aux nouvelles forces productives. Ils entravaient la production au lieu de la développer. Ils se transformèrent en autant de chaînes. Il fallait briser ces chaînes. On les brisa. À la place s’éleva la libre concurrence, avec une constitution sociale et politique correspondante, avec la domination économique et politique de la classe bourgeoise.\par

\labelblock{Crises économiques}

\noindent \phantomsection
\label{par14}\pn{14} Sous nos yeux il se produit un phénomène analogue. La société bourgeoise moderne, qui a mis en mouvement de si puissants moyens de production et d’échange ressemble au magicien qui ne sait plus dominer les puissances infernales qu’il a évoquées. Depuis trente ans au moins, l’histoire de l’industrie et du commerce n’est que l’histoire de la révolte des forces productives contre les rapports de propriété qui sont les conditions d’existence de la Bourgeoisie et de son règne. Il suffit de mentionner les crises commerciales qui, par leur retour périodique, mettent de plus en plus en question l’existence de la société bourgeoise. Chaque crise détruit régulièrement non seulement une masse de produits déjà créés, mais encore une grande partie des forces productives elles-mêmes. Une épidémie, qui, à toute autre époque, eût semblé un paradoxe, s’abat sur la société, – l’épidémie de la surproduction. La société se trouve subitement rejetée dans un état de barbarie momentanée ; on dirait qu’une famine, qu’une guerre d’extermination lui coupent tous les moyens de subsistance ; l’industrie et le commerce semblent annihilés. Et pourquoi ? Parce que la société a trop de civilisation, trop de moyens de subsistance, trop d’industrie, trop de commerce. Les forces productives dont elle dispose ne favorisent plus le développement des conditions de la propriété bourgeoise ; au contraire, elles sont devenues trop puissantes pour ces conditions qui se tournent en entraves ; et toutes les fois que les forces productives sociales s’affranchissent de ces entraves, elles précipitent dans le désordre la société tout entière et menacent l’existence de la propriété bourgeoise. Le système bourgeois est devenu trop étroit pour contenir les richesses créées dans son sein.\par
Comment la Bourgeoisie surmonte-t-elle ces crises ? D’une part, par la destruction forcée d’une masse de forces productives ; d’autre part, par la conquête de nouveaux marchés, et l’exploitation plus parfaite des anciens. C’est-à-dire qu’elle prépare des crises plus générales et plus formidables et diminue les moyens de les prévenir.\par
Les armes dont la Bourgeoisie s’est servie pour abattre la féodalité se retournent aujourd’hui contre la bourgeoisie elle-même.\par

\labelblock{Travail et consommation aliénées}

\noindent \phantomsection
\label{par15}\pn{15} Mais la Bourgeoisie n’a pas seulement forgé les armes qui doivent lui donner la mort ; elle a produit aussi les hommes qui manieront ces armes, – les ouvriers modernes, \emph{les Prolétaires}.\par
Avec le développement de la Bourgeoisie, c’est-à-dire du capital, se développe aussi le Prolétariat, la classe des ouvriers modernes, qui ne vivent qu’à la condition de trouver du travail, et qui n’en trouvent plus dès que leur travail cesse d’agrandir le capital. Les ouvriers, contraints de se vendre au jour le jour, sont une marchandise comme tout autre article du commerce ; ils subissent, par conséquent, toutes les vicissitudes de la concurrence, toutes les fluctuations du marché.\par
L’introduction des machines et la division du travail, dépouillant le travail de l’ouvrier de son caractère individuel, lui ont enlevé tout attrait. Le producteur devient un simple appendice de la machine ; on n’exige de lui que l’opération la plus simple, la plus monotone, la plus vite apprise. Par conséquent, le coût de production de l’ouvrier se réduit à peu près aux moyens d’entretien dont il a besoin pour vivre et pour propager sa race. Or, le prix du travail, comme celui de toute marchandise, est égal au coût de sa production. Donc, plus le travail devient répugnant, plus les salaires baissent. Bien plus, la somme de travail s’accroît avec le développement de la machine et de la division du travail, soit par la prolongation de la journée du travail, soit par l’accélération du mouvement des machines.\par
\bigbreak
\noindent \phantomsection
\label{par16}\pn{16} L’industrie moderne a transformé le petit atelier de l’ancien patron patriarcal en la grande fabrique du bourgeois capitaliste. Des masses d’ouvriers, entassés dans la fabrique, sont organisés militairement. Traités comme des soldats industriels, ils sont placés sous la surveillance d’une hiérarchie complète d’officiers et de sous-officiers. Ils ne sont pas seulement les esclaves de la classe bourgeoise, du gouvernement bourgeois, mais encore, journellement et à toute heure, les esclaves de la machine, du contremaître et surtout du maître de la fabrique. Plus ce despotisme proclame hautement le profit comme son but unique, plus il est mesquin, odieux et exaspérant.\par
\bigbreak
\noindent \phantomsection
\label{par17}\pn{17} Moins le travail exige d’habileté et de force, c’est-à-dire plus l’industrie moderne progresse, plus le travail des hommes est supplanté par celui des femmes. Les distinctions d’âge et de sexe n’ont plus d’importance sociale pour la classe ouvrière. Il n’y a plus que des instruments de travail dont le prix varie suivant l’âge et le sexe.\par
\bigbreak
\noindent \phantomsection
\label{par18}\pn{18} Une fois que l’ouvrier a subi l’exploitation du fabricant et qu’il a reçu son salaire en argent comptant, il devient la proie d’autres membres de la bourgeoisie, du petit propriétaire, du prêteur sur gages.\par
La petite Bourgeoisie, les petits industriels, les marchands, les petits rentiers, les artisans et les paysans propriétaires, tombent dans le Prolétariat ; d’une part, parce que leurs petits capitaux ne leur permettant pas d’employer les procédés de la grande industrie, ils succombent dans leur concurrence avec les grands capitalistes, d’autre part, parce que leur habileté spéciale est dépréciée par les nouveaux modes de production. De sorte que le Prolétariat se recrute dans toutes les classes de la population.\par

\labelblock{Organisation du Prolétariat en classe}

\noindent \phantomsection
\label{par19}\pn{19} Le Prolétariat passe par différentes phases d’évolution. Sa lutte contre la Bourgeoisie commence dès sa naissance.\par
D’abord la lutte est engagée par des ouvriers isolés, ensuite par les ouvriers d’une même fabrique, enfin par les ouvriers du même métier dans une localité, contre le bourgeois qui les exploite directement. Ils ne se contentent pas de diriger leurs attaques contre le mode bourgeois de production, ils les dirigent contre les instruments de production : ils détruisent les marchandises étrangères qui leur font concurrence, brisent les machines, brûlent les fabriques et s’efforcent de reconquérir la position perdue de l’artisan du Moyen-Âge.\par
\bigbreak
\noindent \phantomsection
\label{par20}\pn{20} À ce moment du développement, le Prolétariat forme une masse incohérente, disséminée sur tout le pays, et désunie par la concurrence. Si parfois les ouvriers s’unissent pour agir en masse compacte, cette action n’est pas encore le résultat de leur propre union, mais de celle de la Bourgeoisie qui, pour atteindre ses fins politiques, doit mettre en branle le Prolétariat tout entier, et qui, pour le moment, possède encore le pouvoir de le faire. Durant cette phase, les prolétaires ne combattent pas encore leurs propres ennemis, mais les ennemis de leurs ennemis, c’est-à-dire les restes de la monarchie absolue, les propriétaires fonciers, les bourgeois non industriels, les petits bourgeois. Tout le mouvement historique est de la sorte concentré entre les mains de la Bourgeoisie ; toute victoire remportée dans ces conditions est une victoire bourgeoise.\par
\bigbreak
\noindent \phantomsection
\label{par21}\pn{21} Or l’industrie, en se développant, non seulement grossit le nombre des prolétaires mais les concentre en masses plus considérables ; les prolétaires augmentent en force et prennent conscience de leur force. Les intérêts, les conditions d’existence des prolétaires s’égalisent de plus en plus, à mesure que la machine efface toute différence dans le travail et presque partout réduit le salaire à un niveau également bas. La croissante concurrence des bourgeois entre eux et les crises commerciales qui en résultent, rendent les salaires de plus en plus incertains ; le constant perfectionnement de la machine rend la position de l’ouvrier de plus en plus précaire ; les collisions individuelles entre l’ouvrier et le bourgeois prennent de plus en plus le caractère de collisions entre deux classes. Les ouvriers commencent par se coaliser contre les bourgeois pour le maintien de leurs salaires. Ils vont jusqu’à former des associations permanentes en prévision de ces luttes occasionnelles. Çà et là la résistance éclate en émeute.\par
\bigbreak
\noindent \phantomsection
\label{par22}\pn{22} Parfois les ouvriers triomphent ; mais c’est un triomphe éphémère. Le véritable résultat de leurs luttes est moins le succès immédiat que la solidarité croissante des travailleurs. Cette solidarisation est facilitée par l’accroissement des moyens de communication qui permettent aux ouvriers de localités différentes d’entrer en relation. Or, il suffit de cette mise en contact pour transformer les nombreuses luttes locales qui partout revêtent le même caractère en une lutte nationale, en une lutte de classe. Mais toute lutte de classe est une lutte politique. Et l’union que les bourgeois du Moyen-Âge mettaient des siècles à établir par leurs chemins vicinaux, les prolétaires modernes l’établissent en quelques années par les chemins de fer.\par
\bigbreak
\noindent \phantomsection
\label{par23}\pn{23} L’organisation du Prolétariat en classe, et par suite en parti politique, est sans cesse détruite par la concurrence que se font les ouvriers entre eux. Mais elle renaît toujours ; et toujours plus forte, plus ferme, plus formidable. Elle profite des divisions intestines des bourgeois pour les obliger à donner une garantie légale à certains intérêts de la classe ouvrière, par exemple, la loi de dix heures de travail en Angleterre.\par
\bigbreak
\noindent \phantomsection
\label{par24}\pn{24} En général, les collisions dans la vieille société favorisent de diverses manières le développement du Prolétariat. La Bourgeoisie vit dans un état de guerre perpétuelle ; d’abord contre l’aristocratie, puis contre cette catégorie de la Bourgeoisie dont les intérêts viennent en conflit avec les progrès de l’industrie, toujours, enfin, contre la Bourgeoisie des pays étrangers. Dans toutes ces luttes, elle se voit forcée de faire appel au Prolétariat, d’user de son concours et de l’entraîner dans le mouvement politique, en sorte que la Bourgeoisie fournit aux Prolétaires les éléments de sa propre éducation politique et sociale, c’est-à-dire des armes contre elle-même.\par
\bigbreak
\noindent \phantomsection
\label{par25}\pn{25} De plus, ainsi que nous venons de le voir, des fractions entières de la classe dominante sont précipitées dans le Prolétariat, ou sont menacées, tout au moins, dans leurs conditions d’existence. Elles aussi apportent au Prolétariat de nombreux éléments de progrès.\par
Enfin au moment où la lutte des classes approche de l’heure décisive, le processus de dissolution de la classe régnante, de la société tout entière, prend un caractère si violent et si âpre qu’une fraction de la classe régnante s’en détache et se rallie à la classe révolutionnaire, à la classe qui représente l’avenir. De même que jadis, une partie de la noblesse se rangea du côté de la Bourgeoisie, de nos jours une partie de la Bourgeoisie fait cause commune avec le prolétariat, notamment cette partie des idéologues bourgeois parvenue à l’intelligence théorique du mouvement historique dans son ensemble.\par

\labelblock{Le Prolétariat est tout parce qu’il n’a rien}

\noindent \phantomsection
\label{par26}\pn{26} De toutes les classes qui à l’heure présente se trouvent face à face avec la Bourgeoisie, le Prolétariat seul est la classe vraiment révolutionnaire. Les autres classes périclitent et périssent avec la grande industrie ; le Prolétariat, au contraire, est son produit tout spécial.\par
La classe moyenne, les petits fabricants, les détaillants, les paysans combattent la Bourgeoisie, parce qu’elle compromet leur existence en tant que classe moyenne. Ils ne sont donc pas révolutionnaires, mais conservateurs ; qui plus est, ils sont réactionnaires ; ils demandent que l’histoire fasse machine en arrière. S’ils agissent révolutionnairement, c’est par crainte de tomber dans le Prolétariat : ils défendent alors leurs intérêts futurs et non leurs intérêts actuels ; ils abandonnent leur propre point de vue pour se placer à celui du Prolétariat.\par
La voyoucratie des grandes villes, cette putréfaction passive, cette lie des plus basses couches de la société, est çà et là entraîné dans le mouvement par une révolution prolétarienne ; cependant, ses conditions de vie la prédisposeront plutôt à se vendre à la réaction.\par
\bigbreak
\noindent \phantomsection
\label{par27}\pn{27} Les conditions d’existence de la vieille société sont déjà détruites dans les conditions d’existence du Prolétariat. Le prolétaire est sans propriété ; ses relations de famille n’ont rien de commun avec celles de la famille bourgeoise. Le travail industriel moderne, qui implique l’asservissement de l’ouvrier par le capital, aussi bien en France qu’en Angleterre, qu’en Amérique, qu’en Allemagne, a dépouillé le Prolétaire de tout caractère national. Les lois, la morale, la religion sont pour lui autant de préjugés bourgeois, derrière lesquels se cachent autant d’intérêts bourgeois.\par
\bigbreak
\noindent \phantomsection
\label{par28}\pn{28} Toutes les classes précédentes qui avaient conquis le pouvoir ont essayé de consolider leur situation acquise en soumettant la société à leur propre mode d’appropriation. Les Prolétaires ne peuvent s’emparer des forces productives sociales qu’en abolissant leur propre mode d’appropriation et par suite le mode d’appropriation en vigueur jusqu’à nos jours. Les Prolétaires n’ont rien à eux à assurer ; ils ont, au contraire, à détruire toute garantie privée, toute sécurité privée existantes.\par
\bigbreak
\noindent \phantomsection
\label{par29}\pn{29} Tous les mouvements historiques ont été, jusqu’ici, des mouvements de minorités au profit de minorités. Le mouvement prolétarien est le mouvement spontané de l’immense majorité au profit de l’immense majorité. Le Prolétariat, la dernière couche de la société actuelle, ne peut se redresser sans faire sauter toutes les couches superposées qui constituent la société officielle.\par
\bigbreak
\noindent \phantomsection
\label{par30}\pn{30} La lutte du Prolétariat contre la Bourgeoisie, bien qu’elle ne soit pas au fond une lutte nationale, en revêt cependant, tout d’abord, la forme. Il va sans dire que le Prolétariat de chaque pays doit en finir, avant tout, avec sa propre Bourgeoisie.\par
En esquissant à grands traits les phases du développement prolétarien, nous avons décrit l’histoire de la guerre civile, plus ou moins occulte, qui travaille la société jusqu’à l’heure où cette guerre éclate en une révolution ouverte, et où le Prolétariat établit les bases de sa domination par le renversement violent de la bourgeoisie.\par

\labelblock{En résumé}

\noindent \phantomsection
\label{par31}\pn{31} Toutes les sociétés antérieures, nous l’avons vu, ont reposé sur l’antagonisme de la classe oppressive et de la classe opprimée. Mais pour opprimer une classe il faut, au moins, pouvoir lui garantir les conditions d’existence qui lui permettent de vivre en esclave. Le serf, en pleine féodalité, parvenait à se faire membre de la Commune ; le bourgeois embryonnaire du Moyen-Âge atteignait la position de bourgeois, sous le joug de l’absolutisme féodal. L’ouvrier moderne, au contraire, loin de s’élever avec le progrès de l’industrie, descend toujours plus bas, au-dessous même du niveau des conditions de sa propre classe. Le travailleur tombe dans le paupérisme, et le paupérisme s’accroît plus rapidement encore que la population et la richesse. Il est donc manifeste que la Bourgeoisie est incapable de remplir le rôle de classe régnante et d’imposer à la société comme loi suprême les conditions d’existence de sa classe. Elle ne peut plus régner, parce qu’elle ne peut plus assurer l’existence à son esclave, même dans les conditions de son esclavage ; parce qu’elle est obligée de le laisser tomber dans une situation telle, qu’elle doit le nourrir au lieu de s’en faire nourrir. La société ne peut plus exister sous sa domination, ce qui revient à dire que son existence est désormais incompatible avec celle de la société.\par
\bigbreak
\noindent \phantomsection
\label{par32}\pn{32} La condition essentielle d’existence et de suprématie pour la classe bourgeoise est l’accumulation de la richesse dans des mains privées, la formation et l’accroissement du capital ; la condition du capital est le salariat. Le salariat repose exclusivement sur la concurrence des ouvriers entre eux. Le progrès de l’industrie, dont la Bourgeoisie est l’agent passif et inconscient, remplace l’isolement des ouvriers par leur union révolutionnaire au moyen de l’association. Le développement de la grande industrie sape sous les pieds de la bourgeoisie le terrain même sur lequel elle a établi son système de production et d’appropriation.\par
\phantomsection
\label{fossoyeur}La Bourgeoisie produit avant tout ses propres fossoyeurs. Sa chute et la victoire du Prolétariat sont également inévitables.
\chapterclose


\chapteropen
\renewcommand{\leftmark}{II. Prolétaires et communistes}
\chapter[II. Prolétaires et communistes]{II. Prolétaires et communistes}\phantomsection
\label{II}

\chaptercont
\noindent \initialiv{Q}{uelle est la position des communistes} vis-à-vis des prolétaires pris en masse ? – Les communistes ne forment pas un parti distinct opposé aux autres partis ouvriers. – Ils n’ont point d’intérêts qui les séparent du prolétariat en général. – Ils ne proclament pas de principes sectaires sur lesquels ils voudraient modeler le mouvement ouvrier.\par
Les communistes ne se distinguent des autres partis ouvriers que sur deux points :\par

\begin{enumerate}[itemsep=0pt,]
\item Dans les différentes luttes nationales des prolétaires, ils mettent en avant et font valoir les intérêts du Prolétariat ;
\item Dans les différentes phases évolutives de la lutte entre prolétaires et bourgeois, ils représentent toujours et partout les intérêts du mouvement général.

\end{enumerate}
\labelblock{Avant-garde}

\noindent \phantomsection
\label{par34}\pn{34} Pratiquement, les communistes sont donc la section la plus résolue, la plus avancée de chaque pays, la section qui anime toutes les autres ; théoriquement ils ont sur le reste du prolétariat l’avantage d’une intelligence nette des conditions, de la marche et des fins générales du mouvement prolétarien.\par
Le but immédiat des communistes est le même que celui de toutes les fractions du Prolétariat : organisation des prolétaires en parti de classe, destruction de la suprématie bourgeoise, conquête du pouvoir politique par le Prolétariat.\par
\bigbreak
\noindent \phantomsection
\label{par35}\pn{35} Les propositions théoriques des communistes ne reposent nullement sur des idées et des principes inventés ou découverts par tel ou tel réformateur du monde.\par
Elles ne sont que l’expression, en termes généraux, des conditions réelles d’une lutte de classe existante, d’un mouvement historique évoluant sous nos yeux. L’abolition des rapports de propriété qui ont existé jusqu’ici n’est pas le caractère distinctif du communisme.\par

\labelblock{La propriété}

\noindent \phantomsection
\label{par36}\pn{36} La propriété a subi de constants changements, de continuelles transformations historiques.\par
La Révolution française, par exemple, abolit la propriété féodale en faveur de la propriété bourgeoise.\par
Le caractère distinctif du communisme n’est pas l’abolition de la propriété en général, mais l’abolition de la propriété bourgeoise.\par
Or, la propriété privée, la propriété bourgeoise moderne, est la dernière et la plus parfaite expression du mode de production et d’appropriation basé sur les antagonismes de classes, sur l’exploitation des uns par les autres.\par
En ce sens, les communistes peuvent résumer leur théorie dans cette proposition unique : abolition de la \emph{propriété privée}.\par
\bigbreak
\noindent \phantomsection
\label{par37}\pn{37} On nous a reproché, à nous autres communistes, de vouloir abolir la propriété personnelle, péniblement acquise par le travail, propriété que l’on déclare être la base de toute liberté, de toute activité, de toute indépendance individuelle.\par
La propriété personnelle, fruit du travail d’un homme ! Veut-on parler de la propriété du petit-bourgeois, du petit paysan, forme de propriété antérieure à la propriété bourgeoise ? Nous n’avons que faire de l’abolir, le progrès de l’industrie l’a abolie, ou est en train de l’abolir. Ou bien veut-on parler de la propriété privée, de la propriété bourgeoise moderne ?\par
Est-ce que le travail salarié crée de la propriété pour le prolétaire ? Nullement. Il crée le capital, c’est-à-dire la propriété qui exploite le travail salarié, et qui ne peut s’accroître qu’à la condition de produire du nouveau travail salarié afin de l’exploiter de nouveau. Dans sa forme présente la propriété se meut entre ces deux termes antinomiques : capital et travail. Examinons les deux côtés de cet antagonisme.\par

\labelblock{Le Capital et le Travail}

\noindent \phantomsection
\label{par38}\pn{38} Être capitaliste signifie occuper non seulement une position personnelle, mais encore une position sociale dans le système de la production. Le capital est un produit collectif ; il ne peut être mis en mouvement que par les efforts combinés de beaucoup de membres de la société, et même, en dernière instance, que par les efforts combinés de tous les membres de la société.\par
Le capital n’est donc pas une force personnelle ; il est une force sociale.\par
Dès lors, quand le capital est transformé en propriété commune, appartenant à tous les membres de la société, ce n’est pas là une propriété personnelle transformée en propriété sociale. Il n’y a que le caractère social de la propriété qui soit transformé. Elle perd son caractère de propriété de classe.\par
\bigbreak
\noindent \phantomsection
\label{par39}\pn{39} Arrivons au travail salarié.\par
Le prix moyen du travail salarié est le minimum du salaire, c’est-à-dire la somme des moyens d’existence dont l’ouvrier a besoin pour vivre en ouvrier. Par conséquent, ce que l’ouvrier s’approprie par son activité est tout juste ce qui lui est nécessaire pour entretenir une maigre existence, et pour se reproduire.\par
Nous ne voulons en aucune façon abolir cette appropriation personnelle des produits du travail, indispensable à l’entretien et à la reproduction de la vie humaine, cette appropriation ne laissant aucun profit net qui donne du pouvoir sur le travail d’autrui. Ce que nous voulons, c’est supprimer ce triste mode d’appropriation qui fait que l’ouvrier ne vit que pour accroître le capital et ne vit que juste autant que l’exigent les intérêts de la classe régnante.\par
\bigbreak
\noindent \phantomsection
\label{par40}\pn{40} Dans la société bourgeoise, le travail vivant n’est qu’un moyen d’accroître le travail accumulé. Dans la société communiste, le travail accumulé n’est qu’un moyen d’élargir, d’enrichir, et d’embellir l’existence.\par
Dans la société bourgeoise, le passé domine le présent ; dans la société communiste c’est le présent qui domine le passé. Dans la société bourgeoise, le capital est indépendant et personnel, tandis que l’individu agissant est dépendant et privé de personnalité.\par
C’est l’abolition d’un pareil état de choses que la bourgeoisie flétrit comme l’abolition de l’individualité et de la liberté. Et avec juste raison. Car il s’agit effectivement de l’abolition de l’individualité, de l’indépendance et de la liberté bourgeoises.\par

\labelblock{La liberté}

\noindent \phantomsection
\label{par41}\pn{41} Par liberté, dans les conditions actuelles de la production bourgeoise, on entend la liberté du commerce, du libre-échange.\par
Mais avec le trafic, le trafic libre disparaît aussi. Au reste, tous les grands mots sur le libre-échange, de même que toutes les forfanteries libérales de nos bourgeois n’ont un sens que par contraste au commerce entravé, au bourgeois asservi du Moyen-Âge ; ils n’en ont aucun lorsqu’il s’agit de l’abolition, par les communistes, du trafic, des rapports de la production bourgeoise et de la bourgeoisie elle-même.\par
Vous êtes saisi d’horreur parce que nous voulons abolir la propriété privée. Mais dans votre société la propriété privée est abolie pour les neuf dixièmes de ses membres. C’est précisément parce qu’elle n’existe pas pour ces neuf dixièmes qu’elle existe pour vous. Vous nous reprochez donc de vouloir abolir une forme de la propriété qui ne peut se constituer qu’à la condition de priver l’immense majorité de la société de toute propriété.\par
\bigbreak
\noindent \phantomsection
\label{par42}\pn{42} En un mot, vous nous accusez de vouloir abolir votre propriété à vous. À la vérité, c’est bien là notre intention.\par
Dès que le travail ne peut plus être converti en capital, en argent, en propriété foncière, bref, en pouvoir social, capable d’être monopolisé, c’est-à-dire dès que la propriété individuelle ne peut plus se transformer en société bourgeoise, vous déclarez que l’individualité est supprimée.\par
Vous avouez donc que lorsque vous parlez de l’individu, vous n’entendez parler que du bourgeois. Et cet individu-là, sans contredit, doit être supprimé.\par
Le communisme n’enlève à personne le pouvoir de s’approprier sa part des produits sociaux, il n’ôte que le pouvoir d’assujettir, à l’aide de cette appropriation, le travail d’autrui.\par
\bigbreak
\noindent \phantomsection
\label{par43}\pn{43} On a objecté encore qu’avec l’abolition de la propriété privée toute activité cesserait, qu’une paresse générale s’emparerait du monde.\par
Si cela était, il y a beau jour que la société bourgeoise aurait succombé à la fainéantise, puisque ceux qui y travaillent ne gagnent pas et que ceux qui y gagnent ne travaillent pas.\par
Toute l’objection se réduit à cette tautologie qu’il n’y a plus de travail salarié là où il n’y a plus de capital.\par

\labelblock{La culture}

\noindent \phantomsection
\label{par44}\pn{44} Les accusations portées contre le mode communiste de production et d’appropriation des produits matériels ont été également portées contre la production et l’appropriation intellectuelles. De même que pour le bourgeois la disparition de la propriété de classe équivaut à la disparition de toute production, de même la disparition de la culture intellectuelle de classe signifie, pour lui, la disparition de toute culture intellectuelle.\par
La culture, dont il déplore la perte, n’est pour l’immense majorité, que le façonnement à devenir machine.\par
\bigbreak
\noindent \phantomsection
\label{par45}\pn{45} Mais ne nous querellez pas tant que vous appliquerez à l’abolition de la propriété bourgeoise l’étalon de vos notions bourgeoises de liberté, de culture, de droit, etc. Vos idées sont elles-mêmes les produits des rapports de la production et de la propriété bourgeoises, comme votre droit n’est que la volonté de votre classe érigée en loi, volonté dont le contenu est déterminé par les conditions matérielles d’existence de votre classe.\par
La conception intéressée qui vous fait ériger en lois éternelles de la nature et de la raison les rapports sociaux qui naissent de votre mode de production – rapports sociaux transitoires, qui surgissent et disparaissent au cours de la production – cette conception vous la partagez avec toutes les classes jadis régnantes et disparues aujourd’hui. Ce que vous concevez pour la propriété antique, ce que vous comprenez pour la propriété féodale, il vous est défendu de l’admettre pour la propriété bourgeoise.\par

\labelblock{La famille}

\noindent \phantomsection
\label{par46}\pn{46} Vouloir abolir la famille ! Jusqu’aux plus radicaux qui s’indignent de cet infâme dessein des communistes.\par
Sur quelle base repose la famille bourgeoise de notre époque ? Sur le capital, le gain individuel. La famille, à l’état complet, n’existe que pour la bourgeoisie ; mais elle trouve son complément dans la suppression forcée de toute famille pour le prolétaire, et dans la prostitution publique.\par
La famille bourgeoise s’évanouit naturellement avec l’évanouissement de son complément nécessaire, et l’un et l’autre disparaissent avec la disparition du capital.\par
Nous reprochez-vous de vouloir abolir l’exploitation des enfants par leurs parents ? Nous avouons le crime.\par
\bigbreak
\noindent \phantomsection
\label{par47}\pn{47} Mais nous brisons, dites-vous, les liens les plus sacrés, en substituant à l’éducation de famille, l’éducation sociale.\par
Et votre éducation à vous, n’est-elle pas, elle aussi, déterminée par la société ? Par les conditions sociales dans lesquelles vous élevez vos enfants, par l’intervention directe ou indirecte de la société à l’aide des écoles, etc. ? Les communistes n’inventent pas cette ingérence de la société de l’éducation, ils ne cherchent qu’à en changer le caractère et à arracher l’éducation à l’influence de la classe régnante.\par
Les déclamations bourgeoises sur la famille et l’éducation, sur les doux liens qui unissent l’enfant à ses parents, deviennent de plus en plus écœurantes à mesure que la grande industrie détruit tout lien de famille pour le prolétaire et transforme les enfants en simples articles de commerce, en simples instruments de travail.\par

\labelblock{Les femmes}

\noindent \phantomsection
\label{par48}\pn{48} Mais de la bourgeoisie tout entière s’élève une clameur : vous autres communistes, vous voulez introduire la communauté des femmes !\par
Pour le bourgeois sa femme n’est rien qu’un instrument de production. Il entend dire que les instruments de production doivent être mis en commun et il conclut naturellement qu’il y aura communauté des femmes.\par
Il ne soupçonne pas qu’il s’agit précisément d’assigner à la femme un autre rôle que celui de simple instrument de production.\par
Rien de plus grotesque, d’ailleurs, que l’horreur ultra-morale qu’inspire à nos bourgeois la prétendue communauté officielle des femmes chez les communistes. Les communistes n’ont pas besoin d’introduire la communauté des femmes. Elle a presque toujours existé.\par
Nos bourgeois, non contents d’avoir à leur disposition les femmes et les filles de leurs prolétaires, sans parler de la prostitution officielle, trouvent un plaisir singulier à se cocufier mutuellement.\par
Le mariage bourgeois est, en réalité, la communauté des femmes mariées. Tout au plus pourrait-on accuser les communistes de vouloir mettre à la place d’une communauté de femmes hypocrite et dissimulée, une autre qui serait franche et officielle. Il est évident, du reste, qu’avec l’abolition des rapports de production actuels, la communauté des femmes qui en dérive, c’est-à-dire la prostitution officielle et non officielle, disparaîtra.\par

\labelblock{La patrie}

\noindent \phantomsection
\label{par49}\pn{49} En outre, on accuse les communistes de vouloir abolir la patrie, la nationalité.\par
Les ouvriers n’ont pas de patrie. On ne peut leur ravir ce qu’ils n’ont pas. Comme le prolétariat de chaque pays doit, en premier lieu, conquérir le pouvoir politique, s’ériger en classe maîtresse de la nation, il est par là encore national lui-même, quoique nullement dans le sens bourgeois.\par
Déjà les démarcations et les antagonismes nationaux des peuples disparaissent de plus en plus avec le développement de la bourgeoisie, la liberté du commerce et le marché mondial, avec l’uniformité de la production industrielle et les conditions d’existence qui y correspondent.\par
L’avènement du prolétariat les fera disparaître plus vite encore. L’action commune des différents prolétariats, dans les pays civilisés, tout au moins, est une des premières conditions de leur émancipation.\par
Abolissez l’exploitation de l’homme par l’homme, et vous abolissez l’exploitation d’une nation par une autre nation.\par
Lorsque l’antagonisme des classes, à l’intérieur des nations, aura disparu, l’hostilité de nation à nation disparaîtra.\par

\labelblock{La religion}

\noindent \phantomsection
\label{par50}\pn{50} Quant aux accusations portées contre les communistes, au nom de la religion, de la philosophie et de l’idéologie en général, elles ne méritent pas un examen approfondi.\par
Est-il besoin d’un esprit bien profond pour comprendre que les vues, les notions et les conceptions, en un mot, que la conscience de l’homme change avec tout changement survenu dans ses relations sociales, dans son existence sociale ?\par
Que démontre l’histoire de la pensée si ce n’est que la production intellectuelle se transforme avec la production matérielle ? Les idées dominantes d’une époque n’ont jamais été que les idées de la classe dominante.\par
Lorsqu’on parle d’idées qui révolutionnent une société tout entière, on annonce seulement le fait que dans le sein de la vieille société les éléments d’une nouvelle société se sont formés et que la dissolution des vieilles idées marche de pair avec la dissolution des anciennes relations sociales.\par
Quand l’ancien monde était à son déclin, les vieilles religions furent vaincues par la religion chrétienne ; quand au \textsc{xviii}\emph{e} siècle, les idées chrétiennes cédèrent la place aux idées philosophiques, la société féodale livrait sa dernière bataille à la bourgeoisie, alors révolutionnaire. Les idées de liberté religieuse et de liberté de conscience ne firent que proclamer le règne de la libre concurrence dans le domaine de la connaissance.\par
\bigbreak
\noindent \phantomsection
\label{par51}\pn{51} « Sans doute, dira-t-on, les idées religieuses, morales, philosophiques, politiques et juridiques se sont modifiées dans le cours du développement historique. Mais la religion, la morale, la philosophie se maintenaient toujours à travers ces transformations.\par
« Il y a de plus des vérités éternelles, telles que la liberté, la justice, etc., qui sont communes à toutes les conditions sociales. Or, le communisme abolit les vérités éternelles, il abolit la religion et la morale au lieu de les constituer sur une nouvelle base, ce qui est contradictoire à tout le développement historique antérieur. »\par
À quoi se réduit cette objection ? L’histoire de toute société se résume dans le développement des antagonismes des classes, antagonismes qui ont revêtu des formes différentes aux différentes époques.\par
Mais quelle qu’ait été la forme revêtue par ces antagonismes, l’exploitation d’une partie de la société par l’autre est un fait commun à tous les siècles antérieurs. Donc, rien d’étonnant à ce que la conscience sociale de tous les âges, en dépit de toute divergence et de toute diversité, se soit toujours mue dans de certaines formes communes, dans des formes de conscience qui ne se dissoudront complètement qu’avec l’entière disparition de l’antagonisme des classes.\par
La révolution communiste est la rupture la plus radicale avec les rapports de propriété traditionnels ; rien d’étonnant à ce que, dans le cours de son développement, elle rompe de la façon la plus radicale avec les vieilles idées traditionnelles.\par

\labelblock{Propriété publique des moyens de production}

\noindent \phantomsection
\label{par52}\pn{52} Cependant laissons là les objections faites par la bourgeoisie au communisme.\par
Ainsi que nous l’avons vu plus haut, la première étape dans la révolution ouvrière est la constitution du prolétariat en classe régnante, la conquête du pouvoir public par la démocratie.\par
Le prolétariat se servira de sa suprématie politique pour arracher petit à petit tout capital à la bourgeoisie, pour centraliser tous les instruments de production dans les mains de l’État, c’est-à-dire du prolétariat organisé en classe régnante, et pour augmenter au plus vite les masses des forces productives disponibles.\par
Ceci, naturellement, ne pourra s’accomplir, au début, que par une violation despotique des droits de propriété et des rapports de production bourgeoise, c’est-à-dire par la prise de mesures qui, au point de vue économique, paraîtront insuffisantes et insoutenables, mais qui au cours du mouvement se dépassent elles-mêmes et sont indispensables comme moyen de révolutionner le mode de production tout entier.\par
Ces mesures, bien entendu, seront différentes dans les différents pays.\par
Cependant, pour les pays les plus avancés, les mesures suivantes pourront assez généralement être applicables.\par

\begin{enumerate}[itemsep=0pt,]
\item Expropriation de la propriété foncière et confiscation de la rente foncière au profit de l’État.
\item Impôt fortement progressif.
\item Abolition de l’héritage.
\item Confiscation de la propriété de tous les émigrants et de tous les rebelles.
\item Centralisation du crédit dans les mains de l’État au moyen d’une banque nationale avec capital de l’État et avec le monopole exclusif.
\item Centralisation, dans les mains de l’État, de tous les moyens de transport.
\item Augmentation des manufactures nationales et des instruments de production, défrichement des terrains incultes et amélioration des terres cultivées d’après un système général.
\item Travail obligatoire pour tous, organisation d’armées industrielles, particulièrement pour l’agriculture.
\item Combinaison du travail agricole et industriel, mesures tendant à faire disparaître la distinction entre ville et campagne.
\item Éducation publique et gratuite de tous les enfants, abolition du travail des enfants dans les fabriques, tel qu’il est pratiqué aujourd’hui. Combinaison de l’éducation avec la production matérielle, etc.

\end{enumerate}\bigbreak
\noindent \phantomsection
\label{par54}\pn{54} Les antagonismes de classes une fois disparus dans le cours du développement, et toute la production concentrée dans les mains des individus associés, le pouvoir public perd son caractère politique. Le pouvoir politique, à proprement parler, est le pouvoir organisé d’une classe pour l’oppression d’une autre. Si le prolétariat, dans sa lutte contre la bourgeoisie, se constitue forcément en classe, s’il s’érige par une révolution en classe régnante, et, comme classe régnante détruit violemment les anciens rapports de production, il détruit, en même temps que ces rapports de production, les conditions d’existence de l’antagonisme des classes ; il détruit les classes en général et, par là, sa propre domination comme classe.\par
À la place de l’ancienne société bourgeoise, avec ses classes et ses antagonismes de classes, surgit une association où le libre développement de chacun est la condition du libre développement pour tous.
\chapterclose


\chapteropen
\renewcommand{\leftmark}{III. Littérature socialiste et communiste}
\chapter[III. Littérature socialiste et communiste]{III. Littérature socialiste et communiste}\phantomsection
\label{III}

\chaptercont
\section[1. Le Socialisme Réactionnaire]{1. Le Socialisme Réactionnaire}\phantomsection
\label{III1}
\subsection[a) Le Socialisme Féodal]{a) Le Socialisme Féodal}\phantomsection
\label{III1a}
\noindent \initial{P}{ar leur position historique}, les aristocraties françaises et anglaises se trouvèrent appelées à lancer des libelles contre la société bourgeoise. Dans la révolution française de 1830, dans le mouvement réformiste anglais, elles avaient succombé une fois de plus sous les coups du parvenu abhorré. Pour elles, il ne pouvait plus désormais être question d’une lutte politique sérieuse, il ne leur restait plus que la lutte littéraire. Or, dans le domaine littéraire aussi, la vieille phraséologie de la Restauration était devenue impossible.\par
Pour se créer des sympathies, il fallait que l’aristocratie fit semblant de perdre de vue ses intérêts propres, et qu’elle dressât son acte d’accusation contre la bourgeoisie, dans le seul intérêt de la classe ouvrière exploitée. Elle se ménagea de la sorte la satisfaction de faire des chansons satiriques sur son nouveau maître et de fredonner à ses oreilles des prophéties grosses de malheurs.\par
C’est ainsi que naquit le socialisme féodal, mélange de jérémiades et de pasquinades, d’échos du passé et de vagissements de l’avenir. Si parfois sa critique mordante et spirituelle frappa au cœur la bourgeoisie, son impuissance absolue à comprendre la marche de l’histoire moderne, finit toujours par le rendre ridicule.\par
\bigbreak
\noindent \phantomsection
\label{par56}\pn{56} En guise de drapeau, ces messieurs arboraient la besace du mendiant, afin d’attirer à eux le peuple ; mais dès que le peuple accourut, il aperçut leurs derrières ornés du vieux blason féodal et se dispersa avec de grands et d’irrévérencieux éclats de rires.\par
Une partie des légitimistes français et la jeune Angleterre ont donné au monde ce réjouissant spectacle.\par
Quand les champions de la féodalité démontrent que le mode d’exploitation de la féodalité était autre que celui de la bourgeoisie, ils n’oublient qu’une chose, c’est qu’elle exploitait dans des conditions tout à fait différentes et aujourd’hui surannées. Quand ils font remarquer que sous leur régime le prolétariat moderne n’existait pas, ils oublient que la bourgeoisie est précisément un rejeton fatal de la société féodale.\par
\bigbreak
\noindent \phantomsection
\label{par57}\pn{57} Ils cachent si peu, d’ailleurs, le caractère réactionnaire de leur critique, que leur premier chef d’accusation contre la bourgeoisie est justement d’avoir créé sous son régime une classe qui fera sauter tout l’ancien ordre social.\par
Aussi, n’est-ce pas tant d’avoir produit un prolétariat qu’ils imputent à crime à la bourgeoisie que d’avoir produit un prolétariat révolutionnaire.\par
Dans la lutte politique ils prennent donc une part active à toutes les mesures violentes contre la classe ouvrière. Et dans leur vie de tous les jours ils savent, en dépit de leur phraséologie boursoufflée, s’abaisser pour ramasser les fruits d’or qui tombent de l’arbre de l’industrie, et troquer toutes les vertus chevaleresques, l’honneur, l’amour et la fidélité, contre la laine, le sucre de betterave et l’eau-de-vie.\par
\bigbreak
\noindent \phantomsection
\label{par58}\pn{58} De même que le prêtre et le seigneur féodal marchèrent jadis la main dans la main, voyons-nous aujourd’hui le socialisme clérical marcher côte à côte avec le socialisme féodal.\par
Rien n’est plus facile que de recouvrir d’un vernis de socialisme l’ascétisme chrétien. Le christianisme, lui aussi, ne s’est-il pas élevé contre la propriété privée, le mariage, l’État ? Et à leur place n’a-t-il pas prêché la charité et les guenilles, le célibat et la mortification de la chair, la vie monastique et l’Église ? Le socialisme chrétien n’est que l’eau bénite avec laquelle le prêtre consacre le mécontentement de l’aristocratie.
\subsection[b) Le Socialisme Petit-bourgeois]{b) Le Socialisme Petit-bourgeois}\phantomsection
\label{III1b}
\noindent \initial{L\kern-0.08em{’}}{aristocratie féodale} n’est pas la seule classe ruinée par la bourgeoisie, elle n’est pas la seule classe dont les conditions d’existence s’étiolent et dépérissent dans la société bourgeoise moderne. Les petits bourgeois et les petits paysans du Moyen-Âge étaient les précurseurs de la bourgeoisie moderne. Dans les pays où le commerce et l’industrie sont peu développés, cette classe continue à végéter à côté de la bourgeoisie qui s’épanouit.\par
Dans les pays où la civilisation moderne est florissante, il s’est formé une nouvelle classe de petits bourgeois qui oscillent entre le Prolétariat et la Bourgeoisie ; partie complémentaire de la société bourgeoise, elle se constitue toujours de nouveau. Mais les individus qui la composent se voient sans cesse précipités dans le prolétariat, par suite de la concurrence et, qui plus est, avec la marche progressive de la grande production, ils voient approcher le moment où ils disparaîtront complètement comme fraction indépendante de la société moderne et où ils seront remplacés dans le commerce, la manufacture et l’agriculture par des contre-maîtres, des garçons de boutiques et des laboureurs.\par
\bigbreak
\noindent \phantomsection
\label{par60}\pn{60} Dans les pays comme la France, où les paysans forment bien au delà de la moitié de la population, il était naturel que des écrivains, prenant fait et cause pour le prolétariat contre la bourgeoisie, devaient critiquer le régime bourgeois et défendre le parti ouvrier au point de vue du petit-bourgeois et du paysan. C’est ainsi que se forma le socialisme du petit-bourgeois. Sismondi est le chef de cette littérature, aussi bien pour l’Angleterre que pour la France.\par
Ce socialisme analysa avec beaucoup de pénétration les contradictions inhérentes aux rapports de production modernes. Il mit à nu les hypocrites apologies des économistes. \phantomsection
\label{extermination} Il démontra d’une façon irréfutable les effets meurtriers de la machine et de la division du travail, la concentration des capitaux et de la propriété foncière, la surproduction, les crises, la misère du prolétariat, l’anarchie dans la production, la criante disproportion dans la distribution des richesses, la guerre industrielle d’extermination des nations entre elles, la dissolution des vieilles mœurs, des vieilles relations familiales, des vieilles nationalités.\par
\bigbreak
\noindent \phantomsection
\label{par61}\pn{61} Le but positif, toutefois, de ce socialisme des petits bourgeois est, soit de rétablir les anciens moyens de production et d’échange, et, avec eux, les anciens rapports de propriété et l’ancienne société, soit de faire rentrer de force les moyens modernes de production et d’échange dans le cadre étroit des anciens rapports de production qui ont été brisés et fatalement brisés par eux. Dans l’un et l’autre cas, ce socialisme est à la fois réactionnaire et utopique.\par
Pour la manufacture, le système des corporations, pour l’agriculture, des relations patriarcales ; voilà son dernier mot.\par
Finalement, quand les faits historiques l’eurent tout à fait désenivrée, cette forme de socialisme s’est abandonnée à une lâche mélancolie.
\subsection[c) Le Socialisme allemand ou le “vrai” Socialisme]{c) Le Socialisme allemand ou le “vrai” Socialisme}\phantomsection
\label{III1c}
\noindent \initial{L}{a littérature socialiste} et communiste de la France, née sous la pression d’une bourgeoisie régnante, est l’expression littéraire de la révolte contre ce règne. Elle fut introduite en Allemagne au moment où la bourgeoisie commençait sa lutte contre l’absolutisme féodal.\par
Des philosophes, des demi-philosophes, et des beaux esprits allemands se jetèrent avidement sur cette littérature, mais ils oublièrent qu’avec l’importation de la littérature française en Allemagne, il n’y avait pas eu en même temps importation des conditions sociales de la France. Par rapport aux conditions allemandes, la littérature française perdit toute signification pratique immédiate et prit un caractère purement littéraire. Elle ne devait plus paraître qu’une spéculation oiseuse sur la \emph{réalisation de la nature humaine}. C’est ainsi que pour les philosophes allemands du \textsc{xviii}\emph{e} siècle, les revendications de la première révolution française n’étaient que les revendications de la « raison pratique » en général, et la manifestation de la volonté des bourgeois révolutionnaires de la France ne signifiait, à leurs yeux, que la manifestation des lois de la volonté pure, de la volonté telle qu’elle doit être, de la véritable volonté humaine.\par
\bigbreak
\noindent \phantomsection
\label{par63}\pn{63} Le travail des gens de lettres allemands se bornait à mettre d’accord les idées françaises avec leur vieille conscience philosophique, ou plutôt à s’approprier les idées françaises en les accommodant à leur point de vue philosophique.\par
Ils se les approprièrent comme on s’assimile une langue étrangère, par la traduction.\par
On sait comment les moines superposèrent sur les manuscrits des auteurs classiques du paganisme, les absurdes légendes des saints catholiques. Les gens de lettres allemands agirent en sens inverse à l’égard de la littérature française. Ils glissèrent leurs non-sens sous l’original français. Par exemple, sous la critique française des fonctions économiques de l’argent, ils écrivirent : \emph{« Aliénation de l’être humain »}, sous la critique française de l’État bourgeois, ils écrivirent : \emph{« Élimination de la catégorie de l’universalité abstraite »}, et ainsi de suite.\par
L’introduction de cette phraséologie philosophique au milieu des développements français, ils la baptisèrent : « Philosophie de l’action », « Vrai Socialisme », « Science allemande du socialisme », « Base philosophique du socialisme », etc.\par
De cette façon, on émascula complètement la littérature socialiste et communiste française. Et parce qu’elle cessa, entre les mains des Allemands, d’être l’expression de la lutte d’une classe contre une autre, ceux-ci se félicitèrent de s’être élevés au-dessus de \emph{l’étroitesse française}, et d’avoir défendu non pas de vrais besoins, mais « le besoin du vrai » ; d’avoir défendu, non pas les intérêts du prolétaire, mais les intérêts de l’être humain, de l’homme en général ; de l’homme qui n’appartient à aucune classe ni à aucune réalité et qui n’existe que dans le ciel embrumé de la fantaisie philosophique.\par
\bigbreak
\noindent \phantomsection
\label{par64}\pn{64} Ce socialisme allemand qui prenait si solennellement au sérieux ses maladroits exercices d’écolier et qui les tambourinait à la façon des saltimbanques, perdit cependant petit à petit son innocence de pédant.\par
La lutte de la bourgeoisie allemande et principalement de la bourgeoisie prussienne contre la monarchie absolue et féodale, en un mot, le mouvement libéral, devint plus sérieux.\par
De la sorte, le \emph{vrai} socialisme eut l’occasion tant souhaitée de confronter les réclamations socialistes avec le mouvement politique. Il put lancer les anathèmes traditionnels contre le libéralisme, contre l’état représentatif, contre la concurrence bourgeoise, contre la liberté bourgeoise de la presse, contre le droit bourgeois, contre la liberté et l’égalité bourgeoises ; il put prêcher aux masses qu’elles n’avaient rien à gagner, mais, au contraire, tout à perdre à ce mouvement bourgeois. Le socialisme allemand oublia, bien à propos, que la critique française, dont il était le niais écho, présupposait la société bourgeoise moderne, avec les conditions matérielles d’existence qui y correspondent et une constitution politique conforme, choses précisément que, pour l’Allemagne, il s’agissait encore de conquérir.\par
\bigbreak
\noindent \phantomsection
\label{par65}\pn{65} Pour les gouvernements absolus, avec leur cortège de prêtres, de pédagogues, de hobereaux et de bureaucrates, ce socialisme servit d’épouvantail pour faire peur à la bourgeoisie qui se dressait menaçante.\par
Il compléta, par son hypocrisie doucereuse, les amers coups de fouet et les balles que ces mêmes gouvernements administrèrent aux ouvriers allemands qui se soulevaient.\par
Si le \emph{vrai} socialisme devint ainsi une arme entre les mains des gouvernements, il représentait directement, en outre, l’intérêt réactionnaire, l’intérêt du petit-bourgeois. La classe des petits bourgeois, léguée par le \textsc{xvi}\emph{e} siècle, et depuis lors sans cesse renaissante sous des formes diverses, constitue pour l’Allemagne la vraie base sociale de l’état de choses existant.\par
\bigbreak
\noindent \phantomsection
\label{par66}\pn{66} Maintenir la petite bourgeoisie, c’est maintenir les conditions allemandes actuelles. La suprématie industrielle et politique de la bourgeoisie menace cette classe de destruction certaine, d’une part par la concentration des capitaux, d’autre part par le développement d’un prolétariat révolutionnaire. Le vrai socialisme devait tuer d’une pierre ces deux oiseaux. Il se propagea comme une épidémie.\par
Le vêtement tissé avec les fils immatériels de la spéculation, brodé de fleurs de rhétorique et tout saturé d’une rosée sentimentale, ce vêtement transcendant, dans lequel les socialistes allemands enveloppèrent leurs quelques maigres « vérités éternelles », ne fit qu’activer la vente de leur marchandise auprès d’un pareil public.\par
De son côté le socialisme allemand comprit de mieux en mieux que c’était sa vocation d’être le représentant pompeux de cette petite bourgeoisie.\par
Il proclama la nation allemande la nation normale et le philistin allemand l’homme normal. À toutes les infamies de cet homme normal il donna un sens occulte, un sens supérieur et socialiste qui les faisait tout le contraire de ce qu’elles étaient. Il alla jusqu’au bout, en s’élevant contre la tendance « brutalement destructive » du communisme et en déclarant que, impartial, il planait au-dessus de toutes les luttes de classes.\par
À quelques exceptions près, les publications soi-disant socialistes ou communistes, qui circulent en Allemagne (en 1847) appartiennent à cette sale et énervante littérature.
\section[2. Le Socialisme Conservateur et bourgeois]{2. Le Socialisme Conservateur et bourgeois}\phantomsection
\label{III2}
\noindent \initial{U}{ne partie de la bourgeoisie} cherche à porter remède aux maux sociaux dans le but d’assurer l’existence de la société bourgeoise.\par
Dans cette catégorie se rangent les économistes, les philanthropes, les humanitaires, les améliorateurs du sort de la classe ouvrière, les organisateurs de bienfaisance, les protecteurs des animaux, les fondateurs des sociétés de tempérance, les réformateurs en chambre de tout acabit. Et l’on est allé jusqu’à élaborer ce socialisme bourgeois en systèmes complets.\par
Citons, comme exemple, la \emph{Philosophie de la Misère} de Proudhon.\par
Les socialistes bourgeois veulent les conditions de vie de la société moderne sans les dangers et les luttes qui en dérivent fatalement. Ils veulent la société actuelle, mais avec élimination des éléments qui la révolutionnent et la dissolvent. Ils veulent la bourgeoisie sans le prolétariat. La bourgeoisie, comme de juste, se représente le monde où elle domine comme le meilleur des mondes possibles. Le socialisme bourgeois élabore cette représentation consolante en système ou en demi-système. Lorsqu’il somme le prolétariat de réaliser ces systèmes et de faire son entrée dans la nouvelle Jérusalem, il ne fait pas autre chose au fond que de l’engager à s’en tenir à la société actuelle, mais à se débarrasser de sa conception haineuse de cette société.\par
\bigbreak
\noindent \phantomsection
\label{par68}\pn{68} Une seconde forme de ce socialisme, moins systématique, mais plus pratique, essaya de dégoûter les ouvriers de tout mouvement révolutionnaire, en leur démontrant que ce n’était pas tel ou tel changement politique, mais seulement une transformation des rapports de la vie matérielle et des conditions économiques qui pouvait leur profiter. Notez que par transformation des rapports matériels de la société, ce socialisme n’entend pas parler de l’abolition des rapports de production bourgeois, mais uniquement de réformes administratives s’accomplissant sur la base même de la production bourgeoise, qui, par conséquent, n’affectent pas les relations du capital et du salariat, et qui, dans les meilleurs cas, ne font que diminuer les frais et simplifier le travail administratif du gouvernement bourgeois.\par
Le socialisme bourgeois n’atteint son expression adéquate qu’alors qu’il devient une simple figure de rhétorique.\par
Libre échange ! dans l’intérêt de la classe ouvrière ; droit protecteur ! dans l’intérêt de la classe ouvrière ; prisons cellulaires ! dans l’intérêt de la classe ouvrière : voilà son dernier mot, le seul mot dit sérieusement par le socialisme bourgeois.\par
Car le socialisme bourgeois tient tout entier dans cette phrase : les bourgeois sont des bourgeois dans l’intérêt de la classe ouvrière.
\section[3. Socialisme et Communisme critico-utopique]{3. Socialisme et Communisme critico-utopique}\phantomsection
\label{III3}
\noindent \initial{I}{l ne s’agit pas ici de la littérature} qui, dans toutes les grandes révolutions modernes, a formulé les revendications du prolétariat (les écrits de Babœuf, etc.).\par
\bigbreak
\noindent \phantomsection
\label{par70}\pn{70} Les premières tentatives directes du prolétariat pour faire prévaloir ses propres intérêts de classe, faites en un temps d’effervescence générale, dans la période du renversement de la société féodale, échouèrent nécessairement, aussi bien à cause de l’état embryonnaire du prolétariat lui-même qu’à cause de l’absence des conditions matérielles de son émancipation, conditions qui ne pouvaient être produites que sous l’ère bourgeoise. La littérature révolutionnaire qui accompagnait ces premiers mouvements du prolétariat eut forcément un caractère réactionnaire. Elle préconise un ascétisme général et un grossier égalitarisme.\par
\bigbreak
\noindent \phantomsection
\label{par71}\pn{71} Les systèmes socialistes et communistes proprement dits, les systèmes de Saint-Simon, de Fourier, d’Owen, etc., font leur apparition dans la première période de la lutte entre le prolétariat et la bourgeoisie, période décrite ci-dessus (voir \emph{Bourgeoisie et Prolétariat.})\par
Les inventeurs de ces systèmes se rendent bien compte de l’antagonisme des classes, ainsi que de l’action d’éléments dissolvants dans la société dominante elle-même. Mais ils n’aperçoivent du côté du prolétariat aucune action historique, aucun mouvement politique qui lui soient propres.\par
Comme le développement de l’antagonisme des classes marche de pair avec le développement de l’industrie, ils ne trouvent pas davantage les conditions matérielles de l’émancipation du prolétariat et se mettent en quête d’une science sociale, de lois sociales, dans le but de créer ces conditions.\par
L’activité sociale doit céder la place à leur activité cérébrale personnelle, les conditions historiques de l’émancipation à des conditions fantastiques, l’organisation graduelle et spontanée du prolétariat en classe à une organisation fabriquée de toute pièce par eux-mêmes. L’histoire future du monde se résout pour eux dans la propagande et la mise en pratique de leurs plans de société.\par
\bigbreak
\noindent \phantomsection
\label{par72}\pn{72} Dans la formation de leurs plans, toutefois, ils ont la conscience de défendre avant tout les intérêts de la classe ouvrière, parce qu’elle est la classe la plus souffrante. La classe ouvrière n’existe pour eux que sous cet aspect de la classe la plus souffrante.\par
Mais, ainsi que le comportent la forme peu développée de la lutte des classes et leur propre position sociale, ils se considèrent bien au-dessus de tout antagonisme des classes. Ils désirent améliorer les conditions matérielles de la vie pour tous les membres de la société, même des plus privilégiés. Par conséquent, ils ne cessent de faire appel à la société tout entière sans distinction, ou plutôt ils s’adressent de préférence à la classe régnante. Puisque, aussi bien, il suffit de comprendre leur système pour reconnaître que c’est le meilleur de tous les plans possibles de la meilleure des sociétés possibles.\par
Ils repoussent donc toute action politique et surtout toute action révolutionnaire, ils cherchent à atteindre leur but par des moyens paisibles et essayent de frayer un chemin au nouvel évangile social par la force de l’exemple, par des expériences en petit, condamnées d’avance à l’insuccès.\par
La peinture fantastique de la société future, faite à une époque où le prolétariat, peu développé encore, envisage sa propre position d’une manière fantastique, correspond aux premières aspirations instinctives des ouvriers vers une complète transformation de la société.\par
\bigbreak
\noindent \phantomsection
\label{par73}\pn{73} Mais les écrits socialistes et communistes renferment aussi des éléments critiques. Ils attaquent la société existante à ses bases. Ils ont fourni, par conséquent, dans leur temps, des matériaux d’une grande valeur pour éclairer les ouvriers. Leurs propositions positives relatives à la société future, telle que la suppression de la distinction entre ville et campagne, l’abolition de la famille, du gain privé et du travail salarié, la proclamation de l’harmonie sociale et la transformation de l’État en une simple administration de la production, toutes ces propositions ne font qu’indiquer la disparition de l’antagonisme des classes, antagonisme qui commence seulement à se dessiner et dont les faiseurs de systèmes ne connaissent encore que les premières formes indistinctes et indéterminées. Aussi ces propositions n’ont-elles encore qu’un sens purement utopique.\par
\bigbreak
\noindent \phantomsection
\label{par74}\pn{74} L’importance du socialisme et du communisme critico-utopique est en raison inverse du développement historique. À mesure que la lutte des classes s’accentue et prend une forme, ce fantastique dédain pour la lutte, cette fantastique opposition à la lutte, perdent toute valeur pratique, toute justification théorique. C’est pourquoi si, à beaucoup d’égards, les fondateurs de ces systèmes étaient des révolutionnaires, les sectes formées par leurs disciples sont toujours réactionnaires, car ces disciples s’obstinent à opposer les vieilles conceptions des maîtres à l’évolution historique du prolétariat. Ils cherchent donc, et en cela ils sont conséquents, à émousser la lutte des classes et à concilier les antagonismes. Ils rêvent toujours la réalisation expérimentale de leurs utopies sociales, l’établissement de phalanstères isolés \footnote{(Engels, 1888, ed. anglaise) Le phalanstère était le nom des colonies socialistes imaginées par Fourier. Cabet a donné le nom d’Icarie à son pays utopique, et plus tard à sa colonie communiste en Amérique.}, la création de colonies à l’intérieur et la fondation d’une petite Icarie \footnote{(Engels, 1890, ed. allemande) Home-colonies (colonies à l’intérieur du pays). Owen appelait de ce nom ses sociétés communistes modèles. Les phalanstères étaient des palais sociaux imaginés par Fourier. On donnait le nom d’Icarie au pays utopique dont Cabet a décrit les institutions communistes.} – édition in-douze de la nouvelle Jérusalem ; et pour donner une réalité à tous ces châteaux en Espagne, ils se voient forcés de faire appel aux cœurs et aux caisses des bourgeois. Petit à petit, ils tombent dans la catégorie des socialistes réactionnaires ou conservateurs, dépeints plus haut, et ne s’en distinguent plus que par un pédantisme plus systématique et une foi superstitieuse et fanatique dans l’efficacité miraculeuse de leur science sociale.\par
Ils s’opposent donc avec acharnement à toute action politique de la classe ouvrière, une pareille action ne pouvant provenir, à leur avis, que d’un aveugle manque de foi dans le nouvel évangile.\par
Les Owenistes en Angleterre, les Fouriéristes en France réagissent, là contre les Chartistes, ici contre les Réformistes.
\chapterclose


\chapteropen
\renewcommand{\leftmark}{IV. Position des communistes vis-à-vis des différents partis de l’opposition}
\chapter[IV. Position des communistes vis-à-vis des différents partis de l’opposition]{IV. Position des communistes vis-à-vis des différents partis de l’opposition}\phantomsection
\label{IV}

\chaptercont
\noindent \initialiv{D\kern-0.08em{’}}{après} ce que nous avons dit plus haut (voir Section II), la position des communistes vis-à-vis des partis ouvriers déjà constitués, s’explique d’elle-même, et, partant, leur position vis-à-vis des Chartistes en Angleterre et des réformateurs agraires dans l’Amérique du Nord.\par
Ils combattent pour les intérêts et les buts immédiats de la classe ouvrière, mais dans le mouvement du présent, ils défendent et représentent en même temps l’avenir du mouvement.\par
\bigbreak
\noindent \phantomsection
\label{par76}\pn{76} En France, les communistes se rallient au parti démocrate-socialiste contre les bourgeoisies radicales et conservatrices, tout en se réservant le droit de critiquer les phrases et les illusions léguées par la tradition révolutionnaire.\par
En Suisse ils appuient les radicaux, sans méconnaître que ce parti se compose d’éléments contradictoires, moitié de démocrates socialistes \footnote{ \noindent (Engels, 1888, ed. anglaise) Ce parti était alors représenté au Parlement par Ledru-Rollin, dans la littérature par Louis Blanc et dans la presse quotidienne par La Réforme. Ils désignaient par démocratique-socialiste, nom qu’ils inventèrent, la fraction du parti démocratique ou républicain, qui était plus ou moins nuancée de socialisme.\par
 (Engels, 1890, ed. allemande) Ce qu’on appelait alors en France le Parti démocrate-socialiste était représenté en politique par Ledru-Rollin et dans la littérature par Louis Blanc ; il était donc à cent lieues de la social-démocratie allemande d’aujourd’hui.
}, dans l’acception française du mot, moitié de bourgeois radicaux.\par
En Pologne les communistes soutiennent le parti qui voit dans une révolution agraire la condition de l’affranchissement national, c’est-à-dire le parti qui fit la révolution de Cracovie en 1846.\par
\bigbreak
\noindent \phantomsection
\label{par77}\pn{77} En Allemagne le parti communiste lutte d’accord avec la bourgeoisie, toutes les fois que la bourgeoisie agit révolutionnairement, contre la monarchie absolue, la propriété foncière féodale et la petite bourgeoisie.\par
Mais jamais, à aucun moment, ce parti ne néglige d’éveiller chez les ouvriers une conscience claire et nette de l’antagonisme profond qui existe entre la bourgeoisie et le prolétariat, afin que, l’heure venue, les ouvriers allemands sachent convertir les conditions sociales et politiques, créées par le régime bourgeois, en autant d’armes contre la bourgeoisie ; afin que, sitôt les classes réactionnaires de l’Allemagne détruites, la lutte puisse s’engager contre la bourgeoisie elle-même.\par
C’est vers l’Allemagne surtout que se tourne l’attention des communistes, parce que l’Allemagne se trouve à la veille d’une révolution bourgeoise, et parce qu’elle accomplira cette révolution dans des conditions plus avancées de la civilisation européenne et avec un prolétariat infiniment plus développé que l’Angleterre et la France n’en possédaient au \textsc{xvii}\emph{e} et au \textsc{xviii}\emph{e} siècles, et que, par conséquent, la révolution bourgeoise allemande ne saurait être que le court prélude d’une révolution prolétarienne.\par
\bigbreak
\noindent \phantomsection
\label{par78}\pn{78} En somme, les communistes appuient partout tout mouvement révolutionnaire contre l’état de choses social et politique existant.\par
Dans tous ces mouvements, ils mettent en avant la question de la propriété, quelle que soit la forme plus ou moins développée qu’elle ait revêtue, comme la question fondamentale du mouvement.\par
Enfin les communistes travaillent à l’union et à l’entente des partis démocratiques de tous les pays.\par
Les communistes ne s’abaissent pas à dissimuler leurs opinions et leurs buts. Ils proclament hautement que ces buts ne pourront être atteints sans le renversement violent de tout ordre social actuel. Que les classes régnantes tremblent à l’idée d’une révolution communiste. Les prolétaires n’ont rien à y perdre, hors leurs chaînes. Ils ont un monde à gagner.\par

\labelblock{Prolétaires de tous les pays, unissez-vous !}

\chapterclose


\chapteropen
\renewcommand{\leftmark}{Préfaces}
\chapter[Préfaces]{Préfaces}

\chaptercont
\section[Préface à l’édition allemande de 1872]{Préface à l’édition allemande de 1872}
\noindent La Ligue des communistes, association ouvrière internationale qui, dans les circonstances d’alors, ne pouvait être évidemment que secrète, chargea les soussignés, délégués au congrès tenu à Londres en novembre 1847, de rédiger un programme détaillé, à la fois théorique et pratique, du Parti et destiné à la publicité. Telle est l’origine de ce Manifeste dont le manuscrit, quelques semaines avant la révolution de Février, fut envoyé à Londres pour y être imprimé. Publié d’abord en allemand, il a eu dans cette langue au moins douze éditions différentes en Allemagne, en Angleterre et en Amérique. Traduit en anglais par Miss Hélène Macfarlane, il parut en 1850, à Londres, dans le \emph{Red Republican}, et, en 1871, il eut, en Amérique, au moins trois traductions anglaises. Il parut une première fois en français à Paris, peu de temps avant l’insurrection de juin 1848, et, récemment, dans \emph{Le Socialiste} de New York. Une traduction nouvelle est en préparation. On en fit une édition en polonais à Londres, peu de temps après la première édition allemande. Il a paru en russe à Genève, après 1860. Il a été également traduit en danois peu après sa publication.\par
Bien que les circonstances aient beaucoup changé au cours des vingt-cinq dernières années, les principes généraux exposés dans ce Manifeste conservent dans leurs grandes lignes, aujourd’hui encore, toute leur exactitude. Il faudrait revoir, çà et là, quelques détails. Le Manifeste explique lui-même que l’application des principes dépendra partout et toujours des circonstances historiques données, et que, par suite, il ne faut pas attribuer trop d’importance aux mesures révolutionnaires énumérées à la fin du chapitre II. Ce passage serait, à bien des égards, rédigé tout autrement aujourd’hui. Etant donné les progrès immenses de la grande industrie dans les vingt-cinq dernières années et les progrès parallèles qu’a accomplis, dans son organisation en parti, la classe ouvrière, étant donné les expériences, d’abord de la révolution de février, ensuite et surtout de la Commune de Paris qui, pendant deux mois, mit pour la première fois aux mains du prolétariat le pouvoir politique, ce programme est aujourd’hui vieilli sur certains points. La Commune, notamment, a démontré que \emph{« la classe ouvrière ne peut pas se contenter de prendre telle quelle la machine de l’Etat et de la faire fonctionner pour son propre compte »} (voir \emph{Der Bürgerkrieg in Frankreich}. Adresse des Generalrats der Internationalen Arbeiterassoziation, édition allemande, S. 19, où cette idée est plus longuement développée). En outre, il est évident que la critique de la littérature socialiste présente une lacune pour la période actuelle, puisqu’elle s’arrête à 1847. Et, de même, si les remarques sur la position des communistes à l’égard des différents partis d’opposition (chapitre IV) sont exactes aujourd’hui encore dans leurs principes, elles sont vieillies dans leur application parce que la situation politique s’est modifiée du tout au tout et que l’évolution historique a fait disparaître la plupart des partis qui y sont énumérés.\par
Cependant, le Manifeste est un document historique que nous ne nous attribuons plus le droit de modifier. Une édition ultérieure sera peut-être précédée d’une introduction qui comblera la lacune entre 1847 et nos jours ; la réimpression actuelle nous a pris trop à l’improviste pour nous donner le temps de l’écrire.\par

\byline{Karl Marx, Friedrich Engels ; \\
Londres, 24 juin 1872}
\section[Préface à l’édition russe de 1882]{Préface à l’édition russe de 1882}
\noindent La première édition russe du Manifeste du Parti communiste, traduit par Bakounine, parut peu après 1860 à l’imprimerie du \emph{Kolokol}. A cette époque une édition russe de cet ouvrage avait tout au plus pour l’Occident l’importance d’une curiosité littéraire. Aujourd’hui, il n’en va plus de même. Combien était étroit le terrain où se propageait le mouvement prolétarien à cette époque (décembre 1847), c’est ce qui ressort parfaitement du dernier chapitre : \emph{« Position des communistes envers les différents partis d’opposition dans les divers pays. »} La Russie et les Etats-Unis notamment n’y sont pas mentionnés. C'était le temps où la Russie formait la dernière grande réserve de la réaction européenne, et où l’émigration aux Etats-Unis absorbait l’excédent des forces du prolétariat européen. Ces deux pays fournissaient à l’Europe des matières premières et lui offraient en même temps des débouchés pour l’écoulement de ses produits industriels. Tous deux servaient donc, de l’une ou l’autre manière, de contrefort à l’organisation sociale de l’Europe.\par
Que tout cela est changé aujourd’hui ! C'est précisément l’émigration européenne qui a rendu possible le développement colossal de l’agriculture en Amérique du Nord, développement dont la concurrence ébranle dans ses fondements la grande et la petite propriété foncière en Europe. C'est elle qui a, du même coup, donné aux Etats-Unis la possibilité de mettre en exploitation ses énormes ressources industrielles, et cela avec une énergie et à une échelle telles que le monopole industriel de l’Europe occidentale, et notamment celui de l’Angleterre, disparaîtra à bref délai. Ces deux circonstances réagissent à leur tour de façon révolutionnaire sur l’Amérique elle-même. La petite et la moyenne propriété des farmers, cette assise de tout l’ordre politique américain, succombe peu à peu sous la concurrence de fermes gigantesques, tandis que, dans les districts industriels, il se constitue pour la première fois un nombreux prolétariat à côté d’une fabuleuse concentration du Capital.\par
Passons à la Russie. Au moment de la révolution de 1848-1849, les monarques d’Europe, tout comme la bourgeoisie d’Europe, voyaient dans l’intervention russe le seul moyen de les sauver du prolétariat qui commençait tout juste à prendre conscience de sa force. Le tsar fut proclamé chef de la réaction européenne. Aujourd’hui, il est, à Gatchina, le prisonnier de guerre de la révolution, et la Russie est à l’avant-garde du mouvement révolutionnaire de l’Europe.\par
Le Manifeste communiste avait pour tâche de proclamer la disparition inévitable et prochaine de la propriété bourgeoise. Mais en Russie, à côté de la spéculation capitaliste qui se développe fiévreusement et de la propriété foncière bourgeoise en voie de formation, plus de la moitié du sol est la propriété commune des paysans. Il s’agit, dès lors, de savoir si la communauté paysanne russe, cette forme déjà décomposée de l’antique propriété commune du sol, passera directement à la forme communiste supérieure de la propriété foncière, ou bien si elle doit suivre d’abord le même processus de dissolution qu’elle a subi au cours du développement historique de l’Occident.\par
La seule réponse qu’on puisse faire aujourd’hui à cette question est la suivante : si la révolution russe donne le signal d’une révolution prolétarienne en Occident, et que toutes deux se complètent, la propriété commune actuelle de la Russie pourra servir de point de départ à une évolution communiste.\par

\byline{Karl Marx, Friedrich Engels ; \\
Londres, 21 janvier 1882}
\section[Préface à l’édition allemande de 1883]{Préface à l’édition allemande de 1883}
\noindent Il me faut malheureusement signer seul la préface de cette édition. Marx, l’homme auquel toute la classe ouvrière d’Europe et d’Amérique doit plus qu’à tout autre, Marx repose au cimetière de Highgate, et sur sa tombe verdit déjà le premier gazon. Après sa mort, il ne saurait être question moins que jamais de remanier ou de compléter le Manifeste. Je crois d’autant plus nécessaire d’établir expressément, une fois de plus, ce qui suit.\par
L'idée fondamentale et directrice du Manifeste, à savoir que la production économique et la structure sociale qui en résulte nécessairement forment, à chaque époque historique, la base de l’histoire politique et intellectuelle de cette époque ; que par suite (depuis la dissolution de la propriété commune du sol des temps primitifs), toute l’histoire a été une histoire de luttes de classes, de luttes entre classes exploitées et classes exploitantes, entre classes dominées et classes dominantes, aux différentes étapes de leur développement social ; mais que cette lutte a actuellement atteint une étape où la classe exploitée et opprimée (le prolétariat) ne peut plus se libérer de la classe qui l’exploite et l’opprime (la bourgeoisie), sans libérer en même temps et à tout jamais la société entière de l’exploitation, de l’oppression et des luttes de classes ; cette idée maîtresse appartient uniquement et exclusivement à Marx \footnote{ \noindent (Engels, 1890, ed. allemande) Cette idée, ai-je écrit dans la préface à l’édition anglaise, cette idée qui selon moi, est appelée à marquer pour la science historique le même progrès que la théorie de Darwin pour la biologie, nous nous en étions tous deux approchés peu à peu, plusieurs années déjà avant 1845. Jusqu’où j’étais allé moi-même dans cette direction, de mon propre gré, on peut en juger par mon livre \emph{La situation de la classe laborieuse en Angleterre}. Quand au printemps 1845 je revis Marx à Bruxelles, il l’avait déjà élaborée et il me l’a exposée à peu près aussi clairement que je l’ai fait ici, moi-même. »
 }.\par
Je l’ai souvent déclaré, mais il faut maintenant que cette déclaration figure aussi en tête du \emph{Manifeste}.\par

\byline{Friedrich Engels ; \\
Londres, 28 juin 1883}
\section[Préface à l’édition anglaise de 1888]{Préface à l’édition anglaise de 1888}
\noindent Le Manifeste est le programme de la Ligue des communistes, association ouvrière, d’abord exclusivement allemande, ensuite internationale et qui, dans les conditions politiques qui existaient sur le Continent avant 1848, ne pouvait qu’être une société secrète. Au congrès de la Ligue qui s’est tenu à Londres, en novembre 1847, Marx et Engels se voient confier la tâche de rédiger, aux fins de publication, un ample programme théorique et pratique du Parti. Travail achevé en janvier 1848, et dont le manuscrit allemand fut envoyé à Londres pour y être imprimé, à quelques semaines de la révolution française du 24 février. La traduction française vit le jour à Paris, dès avant l’insurrection de juin 1848. La première traduction anglaise, due à Miss Hélène Macfarlane, parut dans le \emph{Red Republican} de George Julian Harney, Londres 1850. Ont paru également les éditions danoise et polonaise.\par
La défaite de l’insurrection parisienne de juin 1848 – la première grande bataille entre prolétariat et bourgeoisie – devait de nouveau, pour une certaine période, refouler à l’arrière-plan les revendications sociales et politiques de la classe ouvrière européenne. Depuis lors, seuls les divers groupes de la classe possédante s’affrontaient de nouveau dans la lutte pour la domination, tout comme avant la révolution de février ; la classe ouvrière a dû combattre pour la liberté d’action politique et s’aligner sur les positions extrêmes de la partie radicale des classes moyennes. Tout mouvement prolétarien autonome, pour peu qu’il continuât à donner signe de vie, était écrasé sans merci. Ainsi, la police prussienne réussit à dépister le Comité central de la Ligue des communistes, qui résidait alors à Cologne. Ses membres furent arrêtés et, après dix-huit mois de détention, déférés en jugement, en octobre 1852. Ce fameux \emph{procès des communistes à Cologne} dura du 4 octobre au 12 novembre ; sept personnes parmi les prévenus furent condamnées à des peines allant de trois à six ans de forteresse. Immédiatement après le verdict, la Ligue fut officiellement dissoute par les membres demeurés en liberté. Pour ce qui est du Manifeste, on l’eût cru depuis lors voué à l’oubli.\par
Lorsque la classe ouvrière d’Europe eut repris suffisamment de forces pour un nouvel assaut contre les classes dominantes, naquit l’Association internationale des travailleurs. Cependant, cette Association qui s’était constituée dans un but précis – fondre en un tout les forces combatives du prolétariat d’Europe et d’Amérique ne pouvait proclamer d’emblée les principes posés dans le Manifeste. Le programme de l’Internationale devait être assez vaste pour qu’il fût accepté et par les trade-unions anglaises, et par les adeptes de Proudhon en France, Belgique, Italie et Espagne, et par les lassaliens \footnote{(Engels) Lassale nous a toujours affirmé, personnellement, qu’il était le disciple de Marx et, comme tel, il se plaçait sur le terrain du Manifeste. Mais dans sa propagande publique (1862-1864), il n’allait pas au-delà des associations productives créditées par l’Etat.} en Allemagne. Marx qui rédigea ce programme de façon à donner satisfaction à tous ces partis, s’en remettait totalement au développement intellectuel de la classe ouvrière, qui devait être à coup sur le fruit de l’action et de la discussion commune. Par eux-mêmes les événements et les péripéties de la lutte contre le Capital- les défaites plus encore que le succès – ne pouvaient manquer de faire sentir aux ouvriers l’insuffisance de toutes leurs panacées et les amener à comprendre à fond les conditions véritables de leur émancipation. Et Marx avait raison. Quand, en 1874, l’Internationale cessa d’exister, les ouvriers n’étaient plus du tout les mêmes que lors de sa fondation en 1864. Le proudhonisme en France, le lassallisme en Allemagne étaient à l’agonie et même les trade-unions anglaises, alors ultra-conservatrices, et ayant depuis longtemps, dans leur majorité, rompu avec l’Internationale, approchaient peu à peu du moment où le président de leur congrès qui s’est tenu l’an dernier à Swansea, pouvait dire en leur nom : \emph{« Le socialisme continental a cessé d’être pour nous un épouvantail. »} A la vérité, les principes du Manifeste avaient pris un large développement parmi les ouvriers de tous les pays.\par
Ainsi, le Manifeste s’est mis une nouvelle fois au premier plan. Après 1850, le texte allemand fut réédité plusieurs fois en Suisse, Angleterre et Amérique. En 1872, il est traduit en anglais à New York et publié \emph{dans Woodhull and Claflin’s Weekly}. D'après ce texte anglais, \emph{Le Socialiste} new-yorkais a publié la traduction française. Par la suite, parurent en Amérique au moins encore deux traductions anglaises plus ou moins déformées, dont l’une fut rééditée en Angleterre. La première traduction en russe, faite par Bakounine, fut éditée aux environs de 1863 par l’imprimerie du \emph{Kolokol} d’Herzen, à Genève ; la deuxième traduction, due à l’héroïque Véra Zassoulitch, sortit de même à Genève en 1882. Une nouvelle édition danoise est lancée par la \emph{Socialdemokratisk Bibliothek} à Copenhague en 1885 ; une nouvelle traduction française a été publiée par\emph{ Le Socialiste} de Paris, en 1886. D'après cette traduction, a paru une version espagnole, publiée à Madrid en 1886. Point n’est besoin de parler des éditions allemandes réimprimées, on en compte au moins douze. La traduction arménienne, qui devait paraître il y a quelques mois à Constantinople, n’a pas vu le jour, comme on me l’a dit, uniquement parce que l’éditeur avait craint de sortir le livre avec le nom de Marx, tandis que le traducteur refusait de se dire l’auteur du Manifeste. Pour ce qui est des nouvelles traductions en d’autres langues, j’en ai entendu parler, mais n’en ai jamais vu. Ainsi donc, l’histoire du Manifeste reflète notablement celle du mouvement ouvrier contemporain ; à l’heure actuelle, il est incontestablement l’œuvre la plus répandue, la plus internationale de toute la littérature socialiste. Le programme commun de millions d’ouvriers, de la Sibérie à la Californie.\par
Et, cependant, au moment où nous écrivions, nous ne pouvions toutefois l’intituler le Manifeste socialiste. En 1847, on donnait le nom de socialistes, d’une part, aux adeptes des divers systèmes utopiques : les owenistes en Angleterre et les fouriéristes en France, et qui n’étaient déjà plus les uns et les autres, que de simples sectes agonisantes ; d’autre part, aux médicastres sociaux de tout acabit qui promettaient, sans aucun préjudice pour le Capital et le profit, de guérir toutes les infirmités sociales au moyen de toutes sortes de replâtrage. Dans les deux cas, c’étaient des gens qui vivaient en dehors du mouvement ouvrier et qui cherchaient plutôt un appui auprès des classes \emph{cultivées}. Au contraire, cette partie des ouvriers qui, convaincue de l’insuffisance de simples bouleversements politiques, réclamait une transformation fondamentale de la société, s’appelait alors communiste. C'était un communisme à peine dégrossi, purement instinctif, parfois un peu grossier, mais cependant il pressentait l’essentiel et se révéla assez fort dans la classe ouvrière pour donner naissance au communisme utopique : en France, celui de Cabet et en Allemagne, celui de Weitling. En 1847, le socialisme signifiait un mouvement bourgeois, le communisme, un mouvement ouvrier. Le socialisme avait, sur le continent tout au moins, ses entrées dans le monde, pour le communisme, c’était exactement le contraire. Et comme, dès ce moment, nous étions d’avis que \emph{« l’émancipation des travailleurs doit être l’œuvre des travailleurs eux-mêmes »}, nous ne pouvions hésiter un instant sur la dénomination à choisir. Depuis, il ne nous est jamais venu à l’esprit de la rejeter.\par
Bien que le Manifeste soit notre œuvre commune, j’estime néanmoins de mon devoir de constater que la thèse principale, qui en constitue le noyau, appartient à Marx. Cette thèse est qu’à chaque époque historique, le mode prédominant de la production économique et de l’échange et la structure sociale qu’il conditionne, forment la base sur laquelle repose l’histoire politique de ladite époque et l’histoire de son développement intellectuel, base à partir de laquelle seulement elle peut être expliquée ; que de ce fait toute l’histoire de l’humanité (depuis la décomposition de la communauté primitive avec sa possession commune du sol) a été une histoire de luttes de classes, de luttes entre classes exploiteuses et exploitées et opprimées ; que l’histoire de cette lutte de classes atteint à l’heure actuelle, dans son développement, une étape où la classe exploitée et opprimée – le prolétariat – ne peut plus s’affranchir du joug de la classe qui l’exploite et l’opprime – la bourgeoisie – sans affranchir du même coup, une fois pour toutes, la société entière de toute exploitation, oppression, division en classes et lutte de classes.\par
Cette idée qui selon moi est appelée à marquer pour la science historique le même progrès que la théorie de Darwin pour la biologie, nous nous en étions tous deux approchés peu à peu, plusieurs années déjà avant 1845. Jusqu’où j’étais allé moi-même dans cette direction, de mon propre gré, on peut en juger par mon livre \emph{La situation de la classe laborieuse en Angleterre} \footnote{(Engels) \emph{The Condition of the Working Class in England in 1844}. By Frederick Engels. Translated by Florence K. Wischnewetzky, New York, Lovell-London, W. Reeves, 1888.}. Quand au printemps 1845 je revis Marx à Bruxelles, il l’avait déjà élaborée et il me l’a exposée à peu près aussi clairement que je l’ai fait ici, moi-même.\par
Je reproduis les lignes suivantes empruntées à notre préface commune à l’édition allemande de 1872 :\par

\begin{quoteblock}
 \noindent « Bien que les circonstances aient beaucoup changé au cours des vingt dernières années, les principes généraux exposés dans ce Manifeste conservent dans leurs grandes lignes, aujourd’hui encore, toute leur exactitude. Il faudrait revoir, çà et là, quelques détails. Le Manifeste explique lui-même que l’application des principes dépendra partout et toujours des circonstances historiques données, et que, par suite, il ne faut pas attribuer trop d’importance aux mesures révolutionnaires énumérées à la fin du chapitre II. Ce passage serait, à bien des égards, rédigé tout autrement aujourd’hui. Etant donné les progrès immenses de la grande industrie dans les vingt-cinq dernières années et les progrès parallèles qu’a accomplis, dans son organisation en parti, la classe ouvrière, étant donné les expériences, d’abord de la révolution de Février, ensuite et surtout de la Commune de Paris qui, pendant deux mois, mit pour la première fois aux mains du prolétariat le pouvoir politique, ce programme est aujourd’hui vieilli sur certains points. La Commune, notamment, a démontré que \emph{« la classe ouvrière ne peut pas se contenter de prendre telle quelle la machine d’Etat et de la faire fonctionner pour son propre compte »} (voir The Civil War in France ; Address of the General Council of the International Working-men’s Association. London Truelove, 1871, p. 15, où cette idée est plus longuement développée). En outre, il est évident que la critique de la littérature socialiste présente une lacune pour la période actuelle, puisqu’elle s’arrête à 1847. Et, de même, si les remarques sur la position des communistes à l’égard des différents partis d’opposition (chapitre IV) sont exactes aujourd’hui encore dans leurs principes, elles sont vieillies dans leur application parce que la situation politique s’est modifiée du tout au tout et que l’évolution historique a fait disparaître la plupart des partis qui y sont énumérés.\par
 Cependant, le Manifeste est un document historique que nous n’avons plus le droit de modifier. »
 \end{quoteblock}

\noindent La traduction que nous présentons est de M. Samuel Moore, traducteur de la plus grande partie du \emph{Capital} de Marx. Nous l’avons revue ensemble et j’ai ajouté quelques remarques explicatives d’ordre historique.\par

\byline{Friedrich Engels ; \\
Londres, 30 janvier 1888}
\section[Préface à l’édition allemande de 1890]{Préface à l’édition allemande de 1890}
\noindent Depuis que j’ai écrit les lignes qui précèdent, une nouvelle édition allemande du Manifeste est devenue nécessaire. Il convient en outre de mentionner ici qu’il s’est produit bien des choses autour du Manifeste.\par
Une deuxième traduction russe, par Véra Zassoulitch, parut à Genève en 1882 ; nous en rédigeâmes, Marx et moi, la préface. Malheureusement, j’ai égaré le manuscrit allemand original et je suis obligé de retraduire du russe, ce qui n’est d’aucun profit pour le texte même. Voici cette préface :\par

\begin{quoteblock}
 \noindent « La première édition russe du Manifeste du Parti communiste, traduit par Bakounine, parut peu après 1860 à l’imprimerie du Kolokol. A cette époque, une édition russe de cet ouvrage avait tout au plus pour l’Occident l’importance d’une curiosité littéraire. Aujourd’hui, il n’en va plus de même.\par
 Combien était étroit le terrain où se propageait le mouvement prolétarien à cette époque (décembre 1847), c’est ce qui ressort parfaitement du dernier chapitre : « Position des communistes envers les différents partis d’opposition dans les divers pays. » La Russie et les Etats-Unis notamment n’y sont pas mentionnés C’était le temps où la Russie formait la dernière grande réserve de la réaction européenne, et où l’émigration aux Etats-Unis absorbait l’excédent des forces du prolétariat européen. Ces deux pays fournissaient à l’Europe des matières premières et lui offraient en même temps des débouchés pour l’écoulement de ses produits industriels. Tous deux servaient donc, de l’une ou l’autre manière, de contrefort à l’organisation sociale de l’Europe.\par
 Que tout cela est changé aujourd’hui ! C'est précisément l’émigration européenne qui a rendu possible le développement colossal de l’agriculture en Amérique du Nord, développement dont la concurrence ébranle dans ses fondements la grande et la petite propriété foncière en Europe. C'est elle qui a, du même coup, donné aux Etats-Unis la possibilité de mettre en exploitation ses énormes ressources industrielles, et cela avec une énergie et à une échelle telles que le monopole industriel de l’Europe occidentale, et notamment celui de l’Angleterre, disparaîtra à bref délai. Ces deux circonstances réagissent à leur tour de façon révolutionnaire sur l’Amérique elle-même. La petite et la moyenne propriété des farmers, cette assise de tout l’ordre politique américain, succombe peu a peu sous la concurrence de fermes gigantesques, tandis que, dans les districts industriels, il se constitue pour la première fois un nombreux prolétariat à côté d’une fabuleuse concentration du Capital.\par
 Passons à la Russie. Au moment de la révolution de 1848-1849, les monarques d’Europe, tout comme la bourgeoisie d’Europe, voyaient dans l’intervention russe le seul moyen de les sauver du prolétariat qui commençait tout juste à prendre conscience de sa force. Le tsar fut proclamé chef de la réaction européenne. Aujourd’hui, il est, à Gatchina, le prisonnier de guerre de la révolution, et la Russie est à l’avant-garde du mouvement révolutionnaire de l’Europe.\par
 Le Manifeste communiste avait pour tâche de proclamer la disparition inévitable et prochaine de la propriété bourgeoise. Mais en Russie, à côté de la spéculation capitaliste qui se développe fiévreusement et de la propriété foncière bourgeoise en voie de formation, plus de la moitié du sol est la propriété commune des paysans. Il s’agit, dès lors, de savoir si la communauté paysanne russe, cette forme déjà décomposée de l’antique propriété commune du sol, passera directement à la forme communiste supérieure de la propriété foncière, ou bien si elle doit suivre d’abord le même processus de dissolution qu’elle a subi au cours du développement historique de l’Occident.\par
 La seule réponse qu’on puisse faire aujourd’hui à cette question est la suivante : si la révolution russe donne le signal d’une révolution ouvrière en Occident, et que toutes deux se complètent, la propriété commune actuelle de la Russie pourra servir de point de départ à une évolution communiste. »\par
 
\byline{Karl Marx, Friedrich Engels \\
Londres, 21 janvier 1882}
 \end{quoteblock}

\noindent  \par
Une nouvelle traduction polonaise parut, à la même époque, à Genève : \emph{Manifest Kommunistyczny}.\par
Depuis, une nouvelle traduction danoise a paru dans la \emph{Socialdemokratisk Bibliothek}, Copenhague, 1885. Elle n’est malheureusement pas tout à fait complète ; quelques passages essentiels, qui semblent avoir arrêté le traducteur, ont été omis, et çà et là, on peut relever des traces de négligences, dont l’effet est d’autant plus regrettable qu’on voit, d’après le reste, que la traduction aurait pu, avec un peu plus de soin, être excellente.\par
En 1886 parut une nouvelle traduction française dans \emph{Le Socialiste} de Paris ; c’est jusqu’ici la meilleure.\par
D'après cette traduction a paru la même année une version espagnole, d’abord dans \emph{El Socialista} de Madrid, et ensuite en brochure : \emph{Manifesto del Partido Communista}, por Carlos Marx y F. Engels, Madrid, Administracion de \emph{El Socialista}, Herman Cortès, 8.\par
A titre de curiosité, je dirai qu’en 1887 le manuscrit d’une traduction arménienne a été offert à un éditeur de Constantinople ; l’excellent homme n’eut cependant pas le courage d’imprimer une brochure qui portait le nom de Marx et estima que le traducteur devrait bien plutôt s’en déclarer l’auteur, ce que celui-ci refusa de faire.\par
A plusieurs reprises ont été réimprimées en Angleterre certaines traductions américaines plus ou moins inexactes ; enfin, une traduction authentique a paru en 1888. Elle est due à mon ami Samuel Moore, et nous l’avons revue ensemble avant l’impression. Elle a pour titre : \emph{Manifesto of the Communist Party}, by Karl Marx and Frederick Engels, Authorized English translation, edited and annotated by Frederick Engels, 1888. London, William Reeves, 185 Fleet st., E.C. J'ai repris dans la présente édition quelques-unes des notes de cette traduction anglaise.\par
Le Manifeste a eu sa destinée propre. Salué avec enthousiasme, au moment de son apparition, par l’avant-garde peu nombreuse encore du socialisme scientifique (comme le prouvent les traductions signalées dans la première préface), il fut bientôt refoulé à l’arrière-plan par la réaction qui suivit la défaite des ouvriers parisiens en juin 1848, et enfin il fut proscrit « de par la loi » avec la condamnation des communistes de Cologne en novembre 1852. Avec le mouvement ouvrier datant de la révolution de Février, le Manifeste aussi disparaissait de la scène publique.\par
Lorsque la classe ouvrière européenne eut repris suffisamment de forces pour un nouvel assaut contre la puissance des classes dominantes, naquit l’Association internationale des travailleurs. Elle avait pour but de fondre en une immense armée unique toute la classe ouvrière d’Europe et d’Amérique capable d’entrer dans la lutte. Elle ne pouvait donc partir directement des principes posés dans le Manifeste. Il lui fallait un programme qui ne fermât pas la porte aux trade-unions anglaises, aux proudhoniens français, belges, italiens et espagnols, ni aux lassalliens allemands \footnote{(Engels) Lassalle se déclarait toujours personnellement avec nous, le disciple de Marx, et, comme tel, il se tenait évidemment sur le terrain du Manifeste. Il en est autrement de ceux de ses partisans qui n’allèrent pas au-delà de son programme d’associations de production bénéficiant de crédits de l’Etat et qui divisèrent toute la classe ouvrière en ouvriers comptant sur l’Etat et en ouvriers ne comptant que sur eux-mêmes.} – Ce programme – le préambule des Statuts de l’Internationale – fut rédigé par Marx avec une maîtrise à laquelle Bakounine et les anarchistes eux-mêmes ont rendu hommage. Pour la victoire définitive des propositions énoncées dans le Manifeste, Marx s’en remettait uniquement au développement intellectuel de la classe ouvrière, qui devait résulter de l’action et de la discussion communes. Les événements et les vicissitudes de la lutte contre le Capital, les défaites plus encore que les succès, ne pouvaient manquer de faire sentir aux combattants l’insuffisance de toutes leurs panacées et les amener à comprendre à fond les conditions véritables de l’émancipation ouvrière. Et Marx avait raison. La classe ouvrière de 1874, après la dissolution de l’Internationale, était tout autre que celle de 1864, au moment de sa fondation. Le proudhonisme des pays latins et le lassallisme proprement dit en Allemagne étaient à l’agonie, et même les trade-unions anglaises, alors ultra-conservatrices, approchaient peu à peu du moment où, en 1887, le président de leur congrès à Swansea pouvait dire en leur nom : « Le socialisme continental a cessé d’être pour nous un épouvantail. » Mais dès 1887, le socialisme continental s’identifiait presque entièrement avec la théorie formulée dans le Manifeste. Et ainsi l’histoire du Manifeste reflète jusqu’à un certain point l’histoire du mouvement ouvrier moderne depuis 1848. A l’heure actuelle, il est incontestablement l’œuvre la plus répandue, la plus internationale de toute la littérature socialiste, le programme commun de millions d’ouvriers de tous les pays, de la Sibérie à la Californie.\par
Et, cependant, lorsqu’il parut, nous n’aurions pu l’intituler Manifeste socialiste. En 1847, on comprenait sous ce nom de socialiste deux sortes de gens. D'abord, les adhérents des divers systèmes utopiques, notamment les owenistes en Angleterre et les fouriéristes en France, qui n’étaient déjà plus, les uns et les autres, que de simples sectes agonisantes. D'un autre côté, les charlatans sociaux de tout acabit qui voulaient, à l’aide d’un tas de panacées et avec toutes sortes de rapiéçages, supprimer les misères sociales, sans faire le moindre tort au Capital et au profit. Dans les deux cas, c’étaient des gens qui vivaient en dehors du mouvement ouvrier et qui cherchaient plutôt un appui auprès des classes \emph{cultivées}. Au contraire, cette partie des ouvriers qui, convaincue de l’insuffisance des simples bouleversements politiques, réclamait une transformation fondamentale de la société, s’appelait alors communiste. C'était un communisme à peine dégrossi purement instinctif, parfois un peu grossier ; mais il était assez puissant pour donner naissance à deux systèmes de communisme utopique : en France l’lcarie de Cabet et en Allemagne le système de Weitling. En 1847, le socialisme signifiait un mouvement bourgeois, le communisme, un mouvement ouvrier. Le socialisme avait, sur le continent tout au moins, ses entrées dans le monde ; pour le communisme, c’était exactement le contraire. Et comme, dès ce moment, nous étions très nettement d’avis que « l’émancipation des travailleurs doit être l’œuvre des travailleurs eux-mêmes », nous ne pouvions hésiter un instant sur la dénomination à choisir. Depuis, il ne nous est jamais venu à l’esprit de la rejeter.\par
« Prolétaires de tous les pays, unissez-vous ! » Quelques voix seulement nous répondirent, lorsque nous lançâmes cet appel par le monde, il y a maintenant quarante-deux ans, à la veille de la première révolution parisienne dans laquelle le prolétariat se présenta avec ses revendications à lui. Mais le 28 septembre 1864, des prolétaires de la plupart des pays de l’Europe occidentale s’unissaient pour former l’Association internationale des travailleurs, de glorieuse mémoire. L'Internationale elle-même ne vécut d’ailleurs que neuf années. Mais que l’alliance éternelle établie par elle entre les prolétaires de tous les pays existe encore et qu’elle soit plus puissante que jamais, il n’en est pas de meilleure preuve que la journée d’aujourd’hui. Au moment où j’écris ces lignes, le prolétariat d’Europe et d’Amérique passe la revue de ses forces, pour la première fois mobilisées en une seule armée, sous un même drapeau et pour un même but immédiat : la fixation légale de la journée normale de huit heures, proclamée dès 1866 par le Congrès de l’Internationale à Genève, et de nouveau par le Congrès ouvrier de Paris en 1889. Le spectacle de cette journée montrera aux capitalistes et aux propriétaires fonciers de tous les pays que les prolétaires de tous les pays sont effectivement unis.\par
Que Marx n’est-il à côté de moi, pour voir cela de ses propres yeux !\par

\byline{Friedrich Engels ; \\
Londres, 1er mai 1890}
\section[Préface à l’édition polonaise de 1892]{Préface à l’édition polonaise de 1892}
\noindent Qu'il ait été nécessaire de faire paraître une nouvelle édition polonaise du Manifeste du Parti communiste, permet de faire maintes conclusions.\par
D'abord, il faut constater que le Manifeste est devenu, ces derniers temps, une sorte d’illustration du progrès de la grande industrie sur le continent européen. A mesure que celle-ci évolue dans un pays donné, les ouvriers de ce pays ont de plus en plus tendance à voir clair dans leur situation, en tant que classe ouvrière, par rapport aux classes possédantes ; le mouvement socialiste prend de l’extension parmi eux et le Manifeste devient l’objet d’une demande accrue. Ainsi, d’après le nombre d’exemplaires diffusés dans la langue du pays, il est possible de déterminer avec assez de précision non seulement l’état du mouvement ouvrier, mais aussi le degré d’évolution de la grande industrie dans ce pays.\par
La nouvelle édition polonaise du Manifeste est donc une preuve du progrès décisif de l’industrie de la Pologne. Que ce progrès ait effectivement eu lieu durant les dix années qui se sont écoulées depuis que la dernière édition a vu le jour, nul doute ne saurait subsister. Le Royaume de Pologne, la Pologne du Congrès, s’est transformé en une vaste région industrielle de l’empire de Russie. Tandis que la grande industrie russe est dispersée dans maints endroits, une partie tout près du golfe de Finlande, une autre dans la région centrale (Moscou, Vladimir), la troisième sur les côtes de la mer Noire et de la mer d’Azov, etc., l’industrie polonaise se trouve concentrée sur une étendue relativement faible et éprouve aussi bien les avantages que les inconvénients de cette concentration. Ces avantages furent reconnus par les fabricants concurrents de Russie lorsque, malgré leur désir ardent de russifier tous les Polonais, ils réclamèrent l’institution de droits protecteurs contre la Pologne. Quant aux inconvénients – pour les fabricants polonais comme pour le gouvernement russe – , ils se traduisent par une rapide diffusion des idées socialistes parmi les ouvriers polonais et par une demande accrue pour le Manifeste.\par
Cependant, cette évolution rapide de l’industrie polonaise qui a pris le pas sur l’industrie russe, offre à son tour une nouvelle preuve de la vitalité tenace du peuple polonais et constitue une caution nouvelle de son futur rétablissement national. Or, le rétablissement d’une Pologne autonome puissante, nous concerne nous tous et pas seulement les Polonais. Une coopération internationale de bonne foi entre les peuples d’Europe n’est possible que si chacun de ces peuples reste le maître absolu dans sa propre maison. La révolution de 1848, au cours de laquelle les combattants prolétariens ont dû, sous le drapeau du prolétariat, exécuter en fin de compte la besogne de la bourgeoisie, a réalisé du même coup, par le truchement de ses commis – Louis Bonaparte et Bismarck \footnote{« La révolution de 1848, comme nombre de celles qui la précédèrent, a connu d’étranges destins. Les mêmes gens qui l’écrasèrent, sont devenus, selon le mot de Marx, ses exécuteurs testamentaires. Louis-Napoléon fut contraint de créer une Italie unie et indépendante, Bismarck fut contraint de faire en Allemagne une révolution à sa manière et de rendre à la Hongrie une certaine indépendance... » (Engels, \emph{La situation de la classe laborieuse en Angleterre}. Préface à l’édition allemande de 1892.)} – l’indépendance de l’Italie, de l’Allemagne, de la Hongrie. Pour ce qui est de la Pologne qui depuis 1792 avait fait pour la révolution plus que ces trois pays pris ensemble, à l’heure où, en 1863, elle succombait sous la poussée des forces russes, dix fois supérieures aux siennes propres, elle fut abandonnée à elle-même. La noblesse a été impuissante à défendre et à reconquérir l’indépendance de la Pologne ; la bourgeoisie se désintéresse actuellement, pour ne pas dire plus, de cette indépendance. Néanmoins, pour la coopération harmonieuse des nations européennes, elle s’impose impérieusement. Seul peut conquérir cette indépendance le jeune prolétariat polonais, qui en est même le garant le plus sûr. Car pour les ouvriers du reste de l’Europe cette indépendance est aussi nécessaire que pour les ouvriers polonais eux-mêmes.\par

\byline{Friedrich Engels ; \\
Londres, 10 février 1892}
\section[Préface à l’édition italienne de 1893]{Préface à l’édition italienne de 1893}

\salute{Au lecteur italien,}
\noindent La publication du Manifeste du Parti communiste a presque exactement coïncidé avec la date du 18 mars 1848, avec les révolutions de Milan et de Berlin, soulèvements armés de deux nations, dont l’une est située au centre du continent européen, l’autre, au centre des pays méditerranéens, deux nations affaiblies Jusque-là par leur morcellement et les dissensions internes, ce qui les fit tomber sous la domination étrangère. Tandis que l’Italie était soumise à l’empereur d’Autriche, l’Allemagne n’en subissait pas moins le joug, tout aussi sensible encore que moins direct, du tsar de toutes les Russies. Les conséquences des événements du 18 mars 1848 délivrèrent l’Italie et l’Allemagne de cette infamie ; si, de 1848 à 1871, ces deux grandes nations furent rétablies et purent recouvrer, de l’une ou de l’autre façon, leur indépendance, cela tient, selon Marx, au fait que ceux-là mêmes qui avaient écrasé la révolution de 1848, étaient devenus, bien malgré eux, ses commis.\par
Partout cette révolution fut l’œuvre de la classe ouvrière : c’est elle qui dressa les barricades et offrit sa vie en sacrifice. Cependant, seuls les ouvriers parisiens en renversant le gouvernement, étaient tout à fait décidés à renverser aussi le régime bourgeois. Mais, bien qu’ils fussent conscients de l’antagonisme inéluctable entre leur propre classe et la bourgeoisie, ni le progrès économique du pays, ni la formation intellectuelle de la masse des ouvriers français n’avaient pas encore atteint le niveau qui eut pu favoriser la transformation sociale. C'est bien pourquoi les fruits de la révolution devaient revenir en fin de compte à la classe capitaliste. Dans les autres pays – Italie, Allemagne, Autriche – les ouvriers, dès le début, ne firent qu’aider la bourgeoisie à accéder au pouvoir Mais il n’est pas un seul où la domination de la bourgeoisie soit possible sans l’indépendance nationale. Aussi la révolution de 1848 devait-elle déboucher sur l’unité et l’indépendance des nations qui en étaient privées jusque-là : l’Italie, l’Allemagne, la Hongrie. Maintenant, c’est le tour de la Pologne.\par
Ainsi, si la révolution de 1848 n’était pas une révolution socialiste, elle a du moins déblayé la route, préparé le terrain pour cette dernière. Le régime bourgeois, qui a suscité dans tous les pays l’essor de la grande industrie, a du même coup créé partout, durant ces derniers quarante-cinq ans, un prolétariat nombreux, bien cimenté et fort ; il a engendré ainsi, comme le dit le Manifeste, ses propres fossoyeurs. Sans le rétablissement de l’indépendance et de l’unité de chaque nation prise à part, il est impossible de réaliser, sur le plan international, ni l’union du prolétariat ni la coopération pacifique et consciente de ces nations en vue d’atteindre les buts communs. Essayez de vous représenter une action commune internationale des ouvriers italiens, hongrois, allemands, polonais et russes dans le cadre des conditions d’avant 1848 !\par
Donc, les combats de 1848 n’ont pas été vains. De même les quarante-cinq années qui nous séparent de cette période révolutionnaire. Ses fruits commencent à mûrir, et je voudrais seulement que la parution de cette traduction italienne fût bon signe, signe avant-coureur de la victoire du prolétariat italien, de même que la parution de l’original a été le précurseur de la révolution internationale.\par
Le Manifeste rend pleine justice au rôle révolutionnaire que le capitalisme a joué dans le passé. L'Italie fut la première nation capitaliste. La fin du Moyen-Âge féodal, le début de l’ère capitaliste moderne trouvent leur expression dans une figure colossale. C'est l’Italien Dante, le dernier poète du Moyen-Âge et en même temps le premier poète des temps nouveaux. Maintenant, comme en 1300, s’ouvre une ère historique nouvelle. L'Italie nous donnera-t-elle un nouveau Dante qui perpétuera l’éclosion de cette ère nouvelle, prolétarienne ?\par

\byline{Friedrich Engels ; \\
Londres, 1er février 1893}
\chapterclose

 


% at least one empty page at end (for booklet couv)
\ifbooklet
  \newpage\null\thispagestyle{empty}\newpage
\fi

\ifdev % autotext in dev mode
\fontname\font — \textsc{Les règles du jeu}\par
(\hyperref[utopie]{\underline{Lien}})\par
\noindent \initialiv{A}{lors là}\blindtext\par
\noindent \initialiv{À}{ la bonheur des dames}\blindtext\par
\noindent \initialiv{É}{tonnez-le}\blindtext\par
\noindent \initialiv{Q}{ualitativement}\blindtext\par
\noindent \initialiv{V}{aloriser}\blindtext\par
\Blindtext
\phantomsection
\label{utopie}
\Blinddocument
\fi
\end{document}
