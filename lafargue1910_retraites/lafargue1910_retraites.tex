%%%%%%%%%%%%%%%%%%%%%%%%%%%%%%%%%
% LaTeX model https://hurlus.fr %
%%%%%%%%%%%%%%%%%%%%%%%%%%%%%%%%%

% Needed before document class
\RequirePackage{pdftexcmds} % needed for tests expressions
\RequirePackage{fix-cm} % correct units

% Define mode
\def\mode{a4}

\newif\ifaiv % a4
\newif\ifav % a5
\newif\ifbooklet % booklet
\newif\ifcover % cover for booklet

\ifnum \strcmp{\mode}{cover}=0
  \covertrue
\else\ifnum \strcmp{\mode}{booklet}=0
  \booklettrue
\else\ifnum \strcmp{\mode}{a5}=0
  \avtrue
\else
  \aivtrue
\fi\fi\fi

\ifbooklet % do not enclose with {}
  \documentclass[french,twoside]{book} % ,notitlepage
  \usepackage[%
    papersize={105mm, 297mm},
    inner=12mm,
    outer=12mm,
    top=20mm,
    bottom=15mm,
    marginparsep=0pt,
  ]{geometry}
  \usepackage[fontsize=9.5pt]{scrextend} % for Roboto
\else\ifav
  \documentclass[french,twoside]{book} % ,notitlepage
  \usepackage[%
    a5paper,
    inner=25mm,
    outer=15mm,
    top=15mm,
    bottom=15mm,
    marginparsep=0pt,
  ]{geometry}
  \usepackage[fontsize=12pt]{scrextend}
\else% A4 2 cols
  \documentclass[twocolumn]{report}
  \usepackage[%
    a4paper,
    inner=15mm,
    outer=10mm,
    top=25mm,
    bottom=18mm,
    marginparsep=0pt,
  ]{geometry}
  \setlength{\columnsep}{20mm}
  \usepackage[fontsize=9.5pt]{scrextend}
\fi\fi

%%%%%%%%%%%%%%
% Alignments %
%%%%%%%%%%%%%%
% before teinte macros

\setlength{\arrayrulewidth}{0.2pt}
\setlength{\columnseprule}{\arrayrulewidth} % twocol
\setlength{\parskip}{0pt} % classical para with no margin
\setlength{\parindent}{1.5em}

%%%%%%%%%%
% Colors %
%%%%%%%%%%
% before Teinte macros

\usepackage[dvipsnames]{xcolor}
\definecolor{rubric}{HTML}{800000} % the tonic 0c71c3
\def\columnseprulecolor{\color{rubric}}
\colorlet{borderline}{rubric!30!} % definecolor need exact code
\definecolor{shadecolor}{gray}{0.95}
\definecolor{bghi}{gray}{0.5}

%%%%%%%%%%%%%%%%%
% Teinte macros %
%%%%%%%%%%%%%%%%%
%%%%%%%%%%%%%%%%%%%%%%%%%%%%%%%%%%%%%%%%%%%%%%%%%%%
% <TEI> generic (LaTeX names generated by Teinte) %
%%%%%%%%%%%%%%%%%%%%%%%%%%%%%%%%%%%%%%%%%%%%%%%%%%%
% This template is inserted in a specific design
% It is XeLaTeX and otf fonts

\makeatletter % <@@@


\usepackage{blindtext} % generate text for testing
\usepackage[strict]{changepage} % for modulo 4
\usepackage{contour} % rounding words
\usepackage[nodayofweek]{datetime}
% \usepackage{DejaVuSans} % seems buggy for sffont font for symbols
\usepackage{enumitem} % <list>
\usepackage{etoolbox} % patch commands
\usepackage{fancyvrb}
\usepackage{fancyhdr}
\usepackage{float}
\usepackage{fontspec} % XeLaTeX mandatory for fonts
\usepackage{footnote} % used to capture notes in minipage (ex: quote)
\usepackage{framed} % bordering correct with footnote hack
\usepackage{graphicx}
\usepackage{lettrine} % drop caps
\usepackage{lipsum} % generate text for testing
\usepackage[framemethod=tikz,]{mdframed} % maybe used for frame with footnotes inside
\usepackage{pdftexcmds} % needed for tests expressions
\usepackage{polyglossia} % non-break space french punct, bug Warning: "Failed to patch part"
\usepackage[%
  indentfirst=false,
  vskip=1em,
  noorphanfirst=true,
  noorphanafter=true,
  leftmargin=\parindent,
  rightmargin=0pt,
]{quoting}
\usepackage{ragged2e}
\usepackage{setspace} % \setstretch for <quote>
\usepackage{tabularx} % <table>
\usepackage[explicit]{titlesec} % wear titles, !NO implicit
\usepackage{tikz} % ornaments
\usepackage{tocloft} % styling tocs
\usepackage[fit]{truncate} % used im runing titles
\usepackage{unicode-math}
\usepackage[normalem]{ulem} % breakable \uline, normalem is absolutely necessary to keep \emph
\usepackage{verse} % <l>
\usepackage{xcolor} % named colors
\usepackage{xparse} % @ifundefined
\XeTeXdefaultencoding "iso-8859-1" % bad encoding of xstring
\usepackage{xstring} % string tests
\XeTeXdefaultencoding "utf-8"
\PassOptionsToPackage{hyphens}{url} % before hyperref, which load url package

% TOTEST
% \usepackage{hypcap} % links in caption ?
% \usepackage{marginnote}
% TESTED
% \usepackage{background} % doesn’t work with xetek
% \usepackage{bookmark} % prefers the hyperref hack \phantomsection
% \usepackage[color, leftbars]{changebar} % 2 cols doc, impossible to keep bar left
% \usepackage[utf8x]{inputenc} % inputenc package ignored with utf8 based engines
% \usepackage[sfdefault,medium]{inter} % no small caps
% \usepackage{firamath} % choose firasans instead, firamath unavailable in Ubuntu 21-04
% \usepackage{flushend} % bad for last notes, supposed flush end of columns
% \usepackage[stable]{footmisc} % BAD for complex notes https://texfaq.org/FAQ-ftnsect
% \usepackage{helvet} % not for XeLaTeX
% \usepackage{multicol} % not compatible with too much packages (longtable, framed, memoir…)
% \usepackage[default,oldstyle,scale=0.95]{opensans} % no small caps
% \usepackage{sectsty} % \chapterfont OBSOLETE
% \usepackage{soul} % \ul for underline, OBSOLETE with XeTeX
% \usepackage[breakable]{tcolorbox} % text styling gone, footnote hack not kept with breakable


% Metadata inserted by a program, from the TEI source, for title page and runing heads
\title{\textbf{ Intervention contre la loi des retraites au congrès de la S.F.I.O. }}
\date{1910}
\author{Lafargue, Paul}
\def\elbibl{Lafargue, Paul. 1910. \emph{Intervention contre la loi des retraites au congrès de la S.F.I.O.}}
\def\elsource{https://www.marxists.org/francais/lafargue/works/1910/00/lafargue\_19100000.htm}

% Default metas
\newcommand{\colorprovide}[2]{\@ifundefinedcolor{#1}{\colorlet{#1}{#2}}{}}
\colorprovide{rubric}{red}
\colorprovide{silver}{lightgray}
\@ifundefined{syms}{\newfontfamily\syms{DejaVu Sans}}{}
\newif\ifdev
\@ifundefined{elbibl}{% No meta defined, maybe dev mode
  \newcommand{\elbibl}{Titre court ?}
  \newcommand{\elbook}{Titre du livre source ?}
  \newcommand{\elabstract}{Résumé\par}
  \newcommand{\elurl}{http://oeuvres.github.io/elbook/2}
  \author{Éric Lœchien}
  \title{Un titre de test assez long pour vérifier le comportement d’une maquette}
  \date{1566}
  \devtrue
}{}
\let\eltitle\@title
\let\elauthor\@author
\let\eldate\@date


\defaultfontfeatures{
  % Mapping=tex-text, % no effect seen
  Scale=MatchLowercase,
  Ligatures={TeX,Common},
}


% generic typo commands
\newcommand{\astermono}{\medskip\centerline{\color{rubric}\large\selectfont{\syms ✻}}\medskip\par}%
\newcommand{\astertri}{\medskip\par\centerline{\color{rubric}\large\selectfont{\syms ✻\,✻\,✻}}\medskip\par}%
\newcommand{\asterism}{\bigskip\par\noindent\parbox{\linewidth}{\centering\color{rubric}\large{\syms ✻}\\{\syms ✻}\hskip 0.75em{\syms ✻}}\bigskip\par}%

% lists
\newlength{\listmod}
\setlength{\listmod}{\parindent}
\setlist{
  itemindent=!,
  listparindent=\listmod,
  labelsep=0.2\listmod,
  parsep=0pt,
  % topsep=0.2em, % default topsep is best
}
\setlist[itemize]{
  label=—,
  leftmargin=0pt,
  labelindent=1.2em,
  labelwidth=0pt,
}
\setlist[enumerate]{
  label={\bf\color{rubric}\arabic*.},
  labelindent=0.8\listmod,
  leftmargin=\listmod,
  labelwidth=0pt,
}
\newlist{listalpha}{enumerate}{1}
\setlist[listalpha]{
  label={\bf\color{rubric}\alph*.},
  leftmargin=0pt,
  labelindent=0.8\listmod,
  labelwidth=0pt,
}
\newcommand{\listhead}[1]{\hspace{-1\listmod}\emph{#1}}

\renewcommand{\hrulefill}{%
  \leavevmode\leaders\hrule height 0.2pt\hfill\kern\z@}

% General typo
\DeclareTextFontCommand{\textlarge}{\large}
\DeclareTextFontCommand{\textsmall}{\small}

% commands, inlines
\newcommand{\anchor}[1]{\Hy@raisedlink{\hypertarget{#1}{}}} % link to top of an anchor (not baseline)
\newcommand\abbr[1]{#1}
\newcommand{\autour}[1]{\tikz[baseline=(X.base)]\node [draw=rubric,thin,rectangle,inner sep=1.5pt, rounded corners=3pt] (X) {\color{rubric}#1};}
\newcommand\corr[1]{#1}
\newcommand{\ed}[1]{ {\color{silver}\sffamily\footnotesize (#1)} } % <milestone ed="1688"/>
\newcommand\expan[1]{#1}
\newcommand\foreign[1]{\emph{#1}}
\newcommand\gap[1]{#1}
\renewcommand{\LettrineFontHook}{\color{rubric}}
\newcommand{\initial}[2]{\lettrine[lines=2, loversize=0.3, lhang=0.3]{#1}{#2}}
\newcommand{\initialiv}[2]{%
  \let\oldLFH\LettrineFontHook
  % \renewcommand{\LettrineFontHook}{\color{rubric}\ttfamily}
  \IfSubStr{QJ’}{#1}{
    \lettrine[lines=4, lhang=0.2, loversize=-0.1, lraise=0.2]{\smash{#1}}{#2}
  }{\IfSubStr{É}{#1}{
    \lettrine[lines=4, lhang=0.2, loversize=-0, lraise=0]{\smash{#1}}{#2}
  }{\IfSubStr{ÀÂ}{#1}{
    \lettrine[lines=4, lhang=0.2, loversize=-0, lraise=0, slope=0.6em]{\smash{#1}}{#2}
  }{\IfSubStr{A}{#1}{
    \lettrine[lines=4, lhang=0.2, loversize=0.2, slope=0.6em]{\smash{#1}}{#2}
  }{\IfSubStr{V}{#1}{
    \lettrine[lines=4, lhang=0.2, loversize=0.2, slope=-0.5em]{\smash{#1}}{#2}
  }{
    \lettrine[lines=4, lhang=0.2, loversize=0.2]{\smash{#1}}{#2}
  }}}}}
  \let\LettrineFontHook\oldLFH
}
\newcommand{\labelchar}[1]{\textbf{\color{rubric} #1}}
\newcommand{\milestone}[1]{\autour{\footnotesize\color{rubric} #1}} % <milestone n="4"/>
\newcommand\name[1]{#1}
\newcommand\orig[1]{#1}
\newcommand\orgName[1]{#1}
\newcommand\persName[1]{#1}
\newcommand\placeName[1]{#1}
\newcommand{\pn}[1]{\IfSubStr{-—–¶}{#1}% <p n="3"/>
  {\noindent{\bfseries\color{rubric}   ¶  }}
  {{\footnotesize\autour{ #1}  }}}
\newcommand\reg{}
% \newcommand\ref{} % already defined
\newcommand\sic[1]{#1}
\newcommand\surname[1]{\textsc{#1}}
\newcommand\term[1]{\textbf{#1}}

\def\mednobreak{\ifdim\lastskip<\medskipamount
  \removelastskip\nopagebreak\medskip\fi}
\def\bignobreak{\ifdim\lastskip<\bigskipamount
  \removelastskip\nopagebreak\bigskip\fi}

% commands, blocks
\newcommand{\byline}[1]{\bigskip{\RaggedLeft{#1}\par}\bigskip}
\newcommand{\bibl}[1]{{\RaggedLeft{#1}\par\bigskip}}
\newcommand{\biblitem}[1]{{\noindent\hangindent=\parindent   #1\par}}
\newcommand{\dateline}[1]{\medskip{\RaggedLeft{#1}\par}\bigskip}
\newcommand{\labelblock}[1]{\medbreak{\noindent\color{rubric}\bfseries #1}\par\mednobreak}
\newcommand{\salute}[1]{\bigbreak{#1}\par\medbreak}
\newcommand{\signed}[1]{\bigbreak\filbreak{\raggedleft #1\par}\medskip}

% environments for blocks (some may become commands)
\newenvironment{borderbox}{}{} % framing content
\newenvironment{citbibl}{\ifvmode\hfill\fi}{\ifvmode\par\fi }
\newenvironment{docAuthor}{\ifvmode\vskip4pt\fontsize{16pt}{18pt}\selectfont\fi\itshape}{\ifvmode\par\fi }
\newenvironment{docDate}{}{\ifvmode\par\fi }
\newenvironment{docImprint}{\vskip6pt}{\ifvmode\par\fi }
\newenvironment{docTitle}{\vskip6pt\bfseries\fontsize{18pt}{22pt}\selectfont}{\par }
\newenvironment{msHead}{\vskip6pt}{\par}
\newenvironment{msItem}{\vskip6pt}{\par}
\newenvironment{titlePart}{}{\par }


% environments for block containers
\newenvironment{argument}{\itshape\parindent0pt}{\vskip1.5em}
\newenvironment{biblfree}{}{\ifvmode\par\fi }
\newenvironment{bibitemlist}[1]{%
  \list{\@biblabel{\@arabic\c@enumiv}}%
  {%
    \settowidth\labelwidth{\@biblabel{#1}}%
    \leftmargin\labelwidth
    \advance\leftmargin\labelsep
    \@openbib@code
    \usecounter{enumiv}%
    \let\p@enumiv\@empty
    \renewcommand\theenumiv{\@arabic\c@enumiv}%
  }
  \sloppy
  \clubpenalty4000
  \@clubpenalty \clubpenalty
  \widowpenalty4000%
  \sfcode`\.\@m
}%
{\def\@noitemerr
  {\@latex@warning{Empty `bibitemlist' environment}}%
\endlist}
\newenvironment{quoteblock}% may be used for ornaments
  {\begin{quoting}}
  {\end{quoting}}

% table () is preceded and finished by custom command
\newcommand{\tableopen}[1]{%
  \ifnum\strcmp{#1}{wide}=0{%
    \begin{center}
  }
  \else\ifnum\strcmp{#1}{long}=0{%
    \begin{center}
  }
  \else{%
    \begin{center}
  }
  \fi\fi
}
\newcommand{\tableclose}[1]{%
  \ifnum\strcmp{#1}{wide}=0{%
    \end{center}
  }
  \else\ifnum\strcmp{#1}{long}=0{%
    \end{center}
  }
  \else{%
    \end{center}
  }
  \fi\fi
}


% text structure
\newcommand\chapteropen{} % before chapter title
\newcommand\chaptercont{} % after title, argument, epigraph…
\newcommand\chapterclose{} % maybe useful for multicol settings
\setcounter{secnumdepth}{-2} % no counters for hierarchy titles
\setcounter{tocdepth}{5} % deep toc
\markright{\@title} % ???
\markboth{\@title}{\@author} % ???
\renewcommand\tableofcontents{\@starttoc{toc}}
% toclof format
% \renewcommand{\@tocrmarg}{0.1em} % Useless command?
% \renewcommand{\@pnumwidth}{0.5em} % {1.75em}
\renewcommand{\@cftmaketoctitle}{}
\setlength{\cftbeforesecskip}{\z@ \@plus.2\p@}
\renewcommand{\cftchapfont}{}
\renewcommand{\cftchapdotsep}{\cftdotsep}
\renewcommand{\cftchapleader}{\normalfont\cftdotfill{\cftchapdotsep}}
\renewcommand{\cftchappagefont}{\bfseries}
\setlength{\cftbeforechapskip}{0em \@plus\p@}
% \renewcommand{\cftsecfont}{\small\relax}
\renewcommand{\cftsecpagefont}{\normalfont}
% \renewcommand{\cftsubsecfont}{\small\relax}
\renewcommand{\cftsecdotsep}{\cftdotsep}
\renewcommand{\cftsecpagefont}{\normalfont}
\renewcommand{\cftsecleader}{\normalfont\cftdotfill{\cftsecdotsep}}
\setlength{\cftsecindent}{1em}
\setlength{\cftsubsecindent}{2em}
\setlength{\cftsubsubsecindent}{3em}
\setlength{\cftchapnumwidth}{1em}
\setlength{\cftsecnumwidth}{1em}
\setlength{\cftsubsecnumwidth}{1em}
\setlength{\cftsubsubsecnumwidth}{1em}

% footnotes
\newif\ifheading
\newcommand*{\fnmarkscale}{\ifheading 0.70 \else 1 \fi}
\renewcommand\footnoterule{\vspace*{0.3cm}\hrule height \arrayrulewidth width 3cm \vspace*{0.3cm}}
\setlength\footnotesep{1.5\footnotesep} % footnote separator
\renewcommand\@makefntext[1]{\parindent 1.5em \noindent \hb@xt@1.8em{\hss{\normalfont\@thefnmark . }}#1} % no superscipt in foot
\patchcmd{\@footnotetext}{\footnotesize}{\footnotesize\sffamily}{}{} % before scrextend, hyperref


%   see https://tex.stackexchange.com/a/34449/5049
\def\truncdiv#1#2{((#1-(#2-1)/2)/#2)}
\def\moduloop#1#2{(#1-\truncdiv{#1}{#2}*#2)}
\def\modulo#1#2{\number\numexpr\moduloop{#1}{#2}\relax}

% orphans and widows
\clubpenalty=9996
\widowpenalty=9999
\brokenpenalty=4991
\predisplaypenalty=10000
\postdisplaypenalty=1549
\displaywidowpenalty=1602
\hyphenpenalty=400
% Copied from Rahtz but not understood
\def\@pnumwidth{1.55em}
\def\@tocrmarg {2.55em}
\def\@dotsep{4.5}
\emergencystretch 3em
\hbadness=4000
\pretolerance=750
\tolerance=2000
\vbadness=4000
\def\Gin@extensions{.pdf,.png,.jpg,.mps,.tif}
% \renewcommand{\@cite}[1]{#1} % biblio

\usepackage{hyperref} % supposed to be the last one, :o) except for the ones to follow
\urlstyle{same} % after hyperref
\hypersetup{
  % pdftex, % no effect
  pdftitle={\elbibl},
  % pdfauthor={Your name here},
  % pdfsubject={Your subject here},
  % pdfkeywords={keyword1, keyword2},
  bookmarksnumbered=true,
  bookmarksopen=true,
  bookmarksopenlevel=1,
  pdfstartview=Fit,
  breaklinks=true, % avoid long links
  pdfpagemode=UseOutlines,    % pdf toc
  hyperfootnotes=true,
  colorlinks=false,
  pdfborder=0 0 0,
  % pdfpagelayout=TwoPageRight,
  % linktocpage=true, % NO, toc, link only on page no
}

\makeatother % /@@@>
%%%%%%%%%%%%%%
% </TEI> end %
%%%%%%%%%%%%%%


%%%%%%%%%%%%%
% footnotes %
%%%%%%%%%%%%%
\renewcommand{\thefootnote}{\bfseries\textcolor{rubric}{\arabic{footnote}}} % color for footnote marks

%%%%%%%%%
% Fonts %
%%%%%%%%%
\usepackage[]{roboto} % SmallCaps, Regular is a bit bold
% \linespread{0.90} % too compact, keep font natural
\newfontfamily\fontrun[]{Roboto Condensed Light} % condensed runing heads
\ifav
  \setmainfont[
    ItalicFont={Roboto Light Italic},
  ]{Roboto}
\else\ifbooklet
  \setmainfont[
    ItalicFont={Roboto Light Italic},
  ]{Roboto}
\else
\setmainfont[
  ItalicFont={Roboto Italic},
]{Roboto Light}
\fi\fi
\renewcommand{\LettrineFontHook}{\bfseries\color{rubric}}
% \renewenvironment{labelblock}{\begin{center}\bfseries\color{rubric}}{\end{center}}

%%%%%%%%
% MISC %
%%%%%%%%

\setdefaultlanguage[frenchpart=false]{french} % bug on part


\newenvironment{quotebar}{%
    \def\FrameCommand{{\color{rubric!10!}\vrule width 0.5em} \hspace{0.9em}}%
    \def\OuterFrameSep{\itemsep} % séparateur vertical
    \MakeFramed {\advance\hsize-\width \FrameRestore}
  }%
  {%
    \endMakeFramed
  }
\renewenvironment{quoteblock}% may be used for ornaments
  {%
    \savenotes
    \setstretch{0.9}
    \normalfont
    \begin{quotebar}
  }
  {%
    \end{quotebar}
    \spewnotes
  }


\renewcommand{\headrulewidth}{\arrayrulewidth}
\renewcommand{\headrule}{{\color{rubric}\hrule}}

% delicate tuning, image has produce line-height problems in title on 2 lines
\titleformat{name=\chapter} % command
  [display] % shape
  {\vspace{1.5em}\centering} % format
  {} % label
  {0pt} % separator between n
  {}
[{\color{rubric}\huge\textbf{#1}}\bigskip] % after code
% \titlespacing{command}{left spacing}{before spacing}{after spacing}[right]
\titlespacing*{\chapter}{0pt}{-2em}{0pt}[0pt]

\titleformat{name=\section}
  [block]{}{}{}{}
  [\vbox{\color{rubric}\large\raggedleft\textbf{#1}}]
\titlespacing{\section}{0pt}{0pt plus 4pt minus 2pt}{\baselineskip}

\titleformat{name=\subsection}
  [block]
  {}
  {} % \thesection
  {} % separator \arrayrulewidth
  {}
[\vbox{\large\textbf{#1}}]
% \titlespacing{\subsection}{0pt}{0pt plus 4pt minus 2pt}{\baselineskip}

\ifaiv
  \fancypagestyle{main}{%
    \fancyhf{}
    \setlength{\headheight}{1.5em}
    \fancyhead{} % reset head
    \fancyfoot{} % reset foot
    \fancyhead[L]{\truncate{0.45\headwidth}{\fontrun\elbibl}} % book ref
    \fancyhead[R]{\truncate{0.45\headwidth}{ \fontrun\nouppercase\leftmark}} % Chapter title
    \fancyhead[C]{\thepage}
  }
  \fancypagestyle{plain}{% apply to chapter
    \fancyhf{}% clear all header and footer fields
    \setlength{\headheight}{1.5em}
    \fancyhead[L]{\truncate{0.9\headwidth}{\fontrun\elbibl}}
    \fancyhead[R]{\thepage}
  }
\else
  \fancypagestyle{main}{%
    \fancyhf{}
    \setlength{\headheight}{1.5em}
    \fancyhead{} % reset head
    \fancyfoot{} % reset foot
    \fancyhead[RE]{\truncate{0.9\headwidth}{\fontrun\elbibl}} % book ref
    \fancyhead[LO]{\truncate{0.9\headwidth}{\fontrun\nouppercase\leftmark}} % Chapter title, \nouppercase needed
    \fancyhead[RO,LE]{\thepage}
  }
  \fancypagestyle{plain}{% apply to chapter
    \fancyhf{}% clear all header and footer fields
    \setlength{\headheight}{1.5em}
    \fancyhead[L]{\truncate{0.9\headwidth}{\fontrun\elbibl}}
    \fancyhead[R]{\thepage}
  }
\fi

\ifav % a5 only
  \titleclass{\section}{top}
\fi

\newcommand\chapo{{%
  \vspace*{-3em}
  \centering % no vskip ()
  {\Large\addfontfeature{LetterSpace=25}\bfseries{\elauthor}}\par
  \smallskip
  {\large\eldate}\par
  \bigskip
  {\Large\selectfont{\eltitle}}\par
  \bigskip
  {\color{rubric}\hline\par}
  \bigskip
  {\Large TEXTE LIBRE À PARTICPATION LIBRE\par}
  \centerline{\small\color{rubric} {hurlus.fr, tiré le \today}}\par
  \bigskip
}}

\newcommand\cover{{%
  \thispagestyle{empty}
  \centering
  {\LARGE\bfseries{\elauthor}}\par
  \bigskip
  {\Large\eldate}\par
  \bigskip
  \bigskip
  {\LARGE\selectfont{\eltitle}}\par
  \vfill\null
  {\color{rubric}\setlength{\arrayrulewidth}{2pt}\hline\par}
  \vfill\null
  {\Large TEXTE LIBRE À PARTICPATION LIBRE\par}
  \centerline{{\href{https://hurlus.fr}{\dotuline{hurlus.fr}}, tiré le \today}}\par
}}

\begin{document}
\pagestyle{empty}
\ifbooklet{
  \cover\newpage
  \thispagestyle{empty}\hbox{}\newpage
  \cover\newpage\noindent Les voyages de la brochure\par
  \bigskip
  \begin{tabularx}{\textwidth}{l|X|X}
    \textbf{Date} & \textbf{Lieu}& \textbf{Nom/pseudo} \\ \hline
    \rule{0pt}{25cm} &  &   \\
  \end{tabularx}
  \newpage
  \addtocounter{page}{-4}
}\fi

\thispagestyle{empty}
\ifaiv
  \twocolumn[\chapo]
\else
  \chapo
\fi
{\it\elabstract}
\bigskip
\makeatletter\@starttoc{toc}\makeatother % toc without new page
\bigskip

\pagestyle{main} % after style

  \chapter[{Intervention contre la loi des retraites au congrès de la S.F.I.O. (1910)}]{Intervention contre la loi des retraites au congrès de la S.F.I.O. (1910)}
\noindent Citoyens, je n’ai pas l’intention d’examiner la loi des retraites ouvrières. Elle a été étudiée dans la discussion que j’ai eue avec Jaurès dans l’\emph{Humanité} et beaucoup d’entre vous l’ont lue; par conséquent, je n’ai pas à revenir sur cette question, et puis le temps du Congrès étant limité, ce n’est pas le moment de l’aborder. Le seul point que je veuille examiner aujourd’hui devant vous, c’est de savoir s’il est de l’intérêt du Parti Socialiste de voter cette loi, de s’abstenir au Parlement quand le vote viendra, ou bien de voter contre. Je dois déclarer immédiatement, pour ne pas avoir une demande du camarade Renaudel  : je suis pour qu’on vote contre la loi. (\emph{Approbations sur certains bancs}) Et je vais vous en dire les raisons.\par
La loi des retraites ouvrières est une réclame électorale : et, pour s’en convaincre, on n’a qu’à rappeler son historique. En 1901, à la veille des élections, on la sert au public électoral; en 1906, on la sert à nouveau, et, en 1910, on la ressert. Camarade Vaillant, bien souvent, ainsi que vous l’avez dit, on a réclamé la discussion de la loi au Sénat, mais on s’en gardait bien, on a voulu attendre au dernier moment, au quart d’heure de Rabelais. Elle ne reviendra en discussion à la Chambre que lorsqu’il sera trop tard pour y changer un point sur un i, car il faudrait la renvoyer au sénat, qui la remanierait et la retournerait à la Chambre, et pendant ces voyages, les élections arriveraient, et voyez-vous la tête des radicaux si la loi n’était pas votée ?\par
Vous souvenez-vous que lorsqu’en 1906, les radicaux arrivèrent en majorité à la Chambre, et qu’ils avaient à leur tête un des plus distingués chefs du radicalisme, on se dit : Voilà le moment pour le parti radical d’exécuter son programme, il est au pied du mur, on va le voir à l’œuvre ? Nous répondîmes : le parti radical fera faillite. On nous accusa d’être des oiseaux de mauvaise augure; nous étions, au contraire, des oiseaux de trop bon augure; nous n’avions pas prévu que ce gouvernement, que présidait Clemenceau et qui était orné de deux renégats socialistes, Briand et Viviani, serait le gouvernement qui, depuis la Semaine sanglante de 1871, verserait le plus de sang ouvrier. \emph{(Applaudissements.)}\par
Il est compréhensible que le parti radical ait besoin, pour se présenter devant les électeurs, de cette loi sur les retraites… Le mot retraites exerce une action magique sur l’imagination des ouvriers, c’est le seul mérite de la loi; ils ont peur d’avoir à mendier leur pain dans la vieillesse, eux qui ont nourri et enrichi la société : les radicaux, spéculant sur cette peur, les leurrent avec des promesses de retraites.\par
Mais le parti radical pourrait se tromper et faire un aussi mauvais calcul que le parti libéral anglais, qui, avant d’aller aux élections, vota une loi de retraites, qui ne demande pas un sou aux ouvriers. \emph{(Mouvements divers.)} Il espérait battre le parti conservateur avec sa loi philanthropique et c’est lui qui a été battu dans les campagnes qu’il croyait avoir gagnées.\par
Vous ne voulez pas d’une loi d’assistance me dites-vous. Que vous avez raison ! Mais votre loi de retraites est pire que la loi d’assistance existante. Jaurès a contesté mes chiffres, il a dit, qu’ils étaient monstrueusement faux… J'ai dit et je répète que l’on dépense 199 fr. 91 par assisté. C'est là un de mes chiffres monstrueusement faux.\par
\textbf{JAURÉS} - Ce n’est pas celui-là que j’ai contesté.\par
\textbf{LAFARGUE} - Vous n’avez pu contester aucun de mes chiffres ayant rapport avec la loi des retraites. La fameuse loi que vous voulez voter ne promet que 100 francs au retraité, tandis que l’assisté reçoit 199 francs.\par
Je conçois que le parti radical, le parti le plus meurtrier de France \emph{(Très bien !)} veuille se faire pardonner le sang ouvrier qu’il a versé si abondamment. \emph{(Approbation.)} Le Parti socialiste n’a pas besoin de cette réclame électorale. L'assemblée réunie ici est la preuve de sa force. J'ai assisté à bien des Congrès, mais je n’en ai jamais vu où bouillonnaient tant de passions et de sentiments contraires et où régnait tant de calme. \emph{(Vive approbation.)}\par
Le Parti socialiste n’a pas à craindre les élections avec l’élite qu’il a ici et avec les terribles arguments qu’il a contre le parti radical et tous les autres partis. La loi est faite non pour nous, mais contre nous. Voyons si le Parti socialiste a intérêt à la voter.\par
Trois points de la loi ont été examinés.\par
La capitalisation, qui a été si longuement discutée, est, selon moi, de moindre importance pour les socialistes. Elle intéresse surtout les Millerand et les autres brasseurs d’affaires du Parlement, rêvant de mettre la main sur les 200 millions qui tomberont tous les ans dans la caisse pour entreprendre des travaux publics et lancer des panamas. Jaurès, vous avez soutenu cette opinion en 1890. Lorsque Millerand, votre ami d’alors, prétendait qu’on ne capitalisait que pour des Panamas, vous étiez de cet avis alors, et aujourd’hui vous voulez qu’on capitalise et qu’on emploie les capitaux accumulés à des travaux publics. N'êtes-vous pas effrayé ? Quels travaux proposera-t-on ? On remettra au jour tous les vieux projets, qui n’ont jamais pu trouver un capitaliste assez naïf pour avancer cent sous afin qu’ils soient exécutés; par exemple, le canal de Paris à la mer. Ces mirifiques projets, on ne le proposera que pour enrichir la France et faire le bonheur des ouvriers… Le patriotisme et la philanthropie ne coûtent pas cher aux fricoteurs. Ces travaux qui les emmillionneront engloutiront les capitaux et rapporteront peu ou pas d’intérêts.\par
Vous dites qu’on emploiera les capitaux à l’achat des fonds publics. Cette idée épouvante les financiers. L'achat des titres de rente par milliards, les fera monter au delà du pair; de sorte qu’au lieu de rapporter 3\% ils ne donneront que 2 \% ou moins, comme les consolidés anglais avant la guerre du Transvaal.\par
Par malheur, les capitaux capitalisés courent d’autres dangers. Est-ce qu’il n’y a pas eu déjà des caisses de retraites en  France ? Est-ce qu’on n’a pas capitalisé les cotisations des marins ? Où est cette caisse ? Groussier, vous êtes député, demandez donc au ministère des Finances s’il s’y trouve un lapin.\emph{ (Rires.)} Vous me répondrez que mon histoire est vieille et que comme elle a des cheveux blancs, ainsi que moi, n’en parlons plus. (\emph{Nouveaux rires}.)\par
\textbf{JAURÉS}. - Et moi qui les ai gris, qu’est-ce que je vais faire ?\par
\textbf{LAFARGUE}. - Mais il y a une autre histoire, moins vieille, à cheveux gris, comme Jaurès. L'empire créa, en 1853, une caisse de retraites pour les fonctionnaires; bien entendu, ils devaient l’alimenter. La Caisse ne devait donner des retraites qu’après 25 ans de services actifs et 30 de services sédentaires; donc, pendant 25 et 30 ans, l’Etat a reçu, sans rien dépenser, il a accumulé, il a capitalisé les cotisations des fonctionnaires. On calcule qu’en 1884, c’est-à-dire au bout de 30 ans, la Caisse devait posséder un milliards 96 millions. Je prierai Jaurès et Groussier de demander au ministre des Finances où se trouve ce milliard; se serait-il évaporé comme les milliards des inscrits maritimes ?\par
Jaurès, vous parlerez après moi, je le regrette, et bien que vous soyez un de nos grands orateurs, j’aurais aimé parler après vous, afin de connaître vos arguments pour essayer de les rétorquer. \emph{(Rires.)}\par
\textbf{JAURÉS}. - Nous sommes tous comme cela.\par
\textbf{LAFARGUE}. - Je sais que vous avez réponse à tout.\par
\textbf{JAURÉS}. - Alors, je n’avais pas besoin de parler avant. \emph{(Rires.)}\par
\textbf{LAFARGUE}. - Vous avez écrit : Oui, les fonds des inscrits maritimes se sont fondus; mais les marins n’ont pas à se plaindre, puisqu’ils touchent maintenant une retraite supérieure à celle qu’ils recevaient auparavant. Ce n’est donc pas l’intérêt de leurs cotisations capitalisées qui fournit les retraites, mais l’impôt que paient les contribuables, c’est-à-dire les travailleurs. Les ouvriers paient donc des impôts pour que les marins reçoivent une retraite, parce que les gouvernements bourgeois ont escamoté les capitaux capitalisés de leur Caisse de retraite.\par
Renaudel va me lancer un anathème : Comment peut-on oser dire que le gouvernement bourgeois est un gouvernement voleur! - Mais voler est une fonction vitale de tout gouvernement bourgeois, puisque voler est la condition même de l’existence de la classe bourgeoise. La classe capitaliste, est-ce qu’elle travaille, est-ce qu’elle produit ? Est-ce que toute la richesse des nations bourgeoises n’est pas créée par le travail des salariés manuels et intellectuels ? Est-ce que ces richesses leur appartiennent ? Elles sont accaparées par la classe fainéante, par la classe pour qui son Jésus-Christ a formulé la prière Notre Père, où il est dit : Notre Père, qui êtes aux cieux, donnez-nous notre pain quotidien, parce que nous ne voulons pas l’obtenir par le travail.\par
Le vol des fruits du travail des salariés étant l’occupation essentielle de la classe bourgeoise, il s’ensuit que le gouvernement qui la représente doit être et est un gouvernement essentiellement voleur.\par
L'Etat vole en France avec une certaine modération, il y met des gants; traversez l’Atlantique, Jaurès, et vous verrez comment il vole dans la Grande République américaine : nous comptons par les francs, les Yankees par dollars; quand l’Etat vole 1 franc en France, il vole 5 francs aux Etats-Unis. \emph{(Rires.)}\par
C'est parce que l’Etat ne peut faire autrement que voler que j’ai demandé des comptes détaillés sur ce que versaient les fonctionnaires et sur ce qu’ils touchaient comme retraites. Jaurès en a été indigné et il m’a reproché de semer la panique dans la classe ouvrière, en lui faisant croire que l’Etat vole les fonctionnaires. Sembat n’est pas tout à fait de cet avis, il trouve, au contraire, ma demande intéressante et il m’a promis de la poser au ministre des Finances. Il faut, en effet, avoir comment les retraites sont distribuées; il faut savoir si l’on donne une croûte de pain aux petits qui peinent et paient, et si on empiffre de chapons et de bouteilles de médoc les fainéants des hauts grades.\par
Si ces comptes étaient publiés et portés à la connaissance de tous, ils feraient un joli scandale, pareil à celui que créa Louis XVI quand, en 1790, il dut envoyer à l’Assemblée constituante le \emph{Livre rouge} des pensions de la monarchie; sa publication précipita les événements révolutionnaires.\par
Vous voulez de l’action, camarade Vaillant, voilà le moyen d’en avoir. La révélation des malpropretés parlementaires, qui se cachent dans la caisse des retraites des fonctionnaires, sèmera le mécontentement et la colère contre les gouvernements bourgeois, radicaux et conservateurs, qui oppriment les travailleurs salariés.\par
Cette publication est nécessaire : souvenez-vous avec quel méprisant dédain les radicaux et les monarchistes et leurs valets de plume et de gouvernement traitaient les postiers, qui avaient l’impudence de se mettre en grève, de se révolter contre le gouvernement bourgeois qui les paie et leur donne une retraite. C'était le grand, l’unique argument des folliculaires de la presse et des orateurs du Parlement.\par
Cette publication étranglera cet argument : on verra alors combien chèrement les petits fonctionnaires paient cette maigre retraite dont on veut faire une chaîne d’esclavage.\par
Le mot "retraite" qu’on accole à la loi n’est pas le mot propre. \emph{Retraite} veut dire pension donnée à quelqu’un pour reconnaître des services rendus, sans lui demander de contribuer pour un sou à cette pension.\par
La loi devrait être nommée loi des rentes viagères des salariés; parce qu’une rente viagère est constituée par des prélèvements faits pendant des années sur le revenu ou le salaire de celui qui le reçoit et c’est ce que se propose de faire la loi dites des retraites ouvrières. Il n’y a que mensonge dans cette loi, jusque dans son nom. Elle est monstrueusement mensongère, Jaurès !\par
Ainsi que je le disais en débutant, la capitalisation, qui intéresse les brasseurs d’affaires et les fricoteurs du Parlement,  n’est pas, pour le socialiste le plus grand vice de la loi des retraites : il y a deux iniquités - la cotisation et l’âge de la retraite, 65 ans - que le Parti socialiste ne doit pas accepter et qui, bien plus que la capitalisation, enflammeront l’indignation et la colère ouvrières.\par
La Chambre avait fixé, en 1906, l’âge de la retraite à 60 ans; le Sénat le porte à 65 ans, et comme les radicaux veulent, coûte que coûte, avoir la loi pour aller aux élections, ils voteront les 65 ans, la mort dans l’âme, diront-ils.\par
J'ai été étonné d’entendre Groussier nous assurer que, dans la population ouvrière, il y avait 25 \% de vieillards de 65 ans; mais j’ai été encore plus étonné d’entendre l’assertion de Groussier répétée par Vaillant, dont je connais et apprécie l’esprit scientifique. Je vais démontrer l’erreur de Groussier et de Vaillant en me servant de chiffres officiels. Le rapport du sénateur Cuvinot, établi sur les chiffres fournis par le ministère du Travail, compte 11.408.000 salariés dans l’agriculture, l’industrie et le commerce, dans lesquels il y en a 867.000 âgés de plus de 60 ans, soit 8 \%.\par
\textbf{GROUSSIER}. - Ce rapport ne signifie rien.\par
\textbf{LAFARGUE}. - Parce qu’il contredit votre chiffre. Donc, un document officiel reconnaît seulement 8 \% de salariés âgés de plus de 60 ans.\par
Il y a d’autres documents officiels auxquels je renvoie Groussier, qui dit avoir étudié les chiffres de la loi avec beaucoup de soin. J'ai cité dans mes articles de l’\emph{Humanité} le rapport de 1901 de M. Guieysse, qui passe pour un actuaire des plus capables; il calcule que, d’après le recensement de 1896, il ne restait que 7\% de salariés âgés de 60 ans. J'ai cité un autre document de Bertillon, non pas le Bertillon policier anthropométrique, mais son père, le savant démographe. Il donne dans l’annuaire statistique de Paris de 1882 un classement de la population par âge, d’après le recensement de 1876, et il arrive à ce résultat qu’à 65 ans, il ne reste en France que 7\% de la population, et à Paris que 4 \% et une fraction.\par
\textbf{JAURES}. - Pour la France tout entière, combien dites-vous ?\par
\textbf{LAFARGUE}. - 7 \% et 4 \% à Paris, ce qui démontre que le milieu parisien est plus meurtrier que le milieu général de la France, et l’on ne peut prétendre que le milieu dans lequel vit et meurt la classe ouvrière soit plus hygiénique que le milieu parisien; donc, si l’on veut avoir le chiffre de la survie ouvrière, ce n’est pas 7 \% des salariés de 65 ans qui survivent, mais 4 \%. Ce chiffre est encore trop élevé, parce que si vous prenez, comme je l’ai fait…\par
\textbf{JAURES}. - Est-ce que M. Bertillon calcule la survie des hommes à partir de 18 ans, ou bien compte-t-il…\par
\textbf{LAFARGUE}. - A partir de zéro âge; quant à moi, j’ai calculé à partir de 18 ans jusqu’à 80 ans et plus… Je soutiens que le chiffre de survie de Paris est encore trop fort, parce qu’à Paris, il y a des arrondissement pauvres, dans lesquels la mortalité est différente. J'ai comparé les statistiques officielles publiées dans les annuaires de Paris et j’ai trouvé qu’il existe une différence de 33 \% entre la mortalité des quartiers riches et des quartiers pauvres, et que dans un arrondissement, le XIII\textsuperscript{e}, essentiellement ouvrier, la mortalité pour dix mille habitants est, par année, de 250; par conséquent, au bout de quarante ans, la population tout entière disparaît. Cette population, qui pendant quarante années sera condamnée à payer l’impôt pour les retraites, les touchera dans la tombe.\par
Ils ne seront pas nombreux les salariés des villes qui vivront jusqu’à 65 ans pour toucher la retraite : dans les campagnes, il est encore possible de rencontrer des paysans de 65 ans, et même plus âgés, assez gaillards pour jouir encore de la vie. Mais dans les villes, cela n’existe pas, et cependant Groussier vient de nous dire qu’il y avait dans la population de la France 25 \% de vieillards de 65 ans…\par
\textbf{GROUSSIER}. - Voulez-vous me permettre un mot ? Nous ne nous entendons pas sur le mode de calculer : en comparant le nombre de personnes qui ont plus de 65 ans avec celles qui ont de 18 à 65 ans. C'est le calcul que vous faites. Je dis que ce n’est pas celui qu’il faut. Ce que nous devons rechercher, c’est le nombre de personnes qui, nées ensemble ou qui ayant ensemble 18 ans, arrivent à 65 ans. C'est là ce que nous voulons savoir. Nous sommes aujourd’hui 100 personnes qui adhérons à la caisse âgés de 18 ans. Combien en restera-t-il qui auront 65 ans ? Et c’est alors que j’ai dit qu’il en reste 25 \%. Et si vous voulez me permettre un mot… \emph{(Protestations.)}\par
\textbf{LAFARGUE}. - Au contraire !\par
\textbf{GROUSSIER}. - Je comprends bien qu’il est délicat, surtout étant président…\par
\textbf{COMPERE-MOREL}. - Ce n’est pas pour cela : nous ne voulons pas contester votre droit à la parole, mais nous protestons contre les chiffres.\par
\textbf{GROUSSIER}. - Ce que j’indique, lorsqu’on veut avoir des résultats et des chiffres précis, c’est qu’il faut tenir compte des documents qui sont fournis, des statistiques et des tables de mortalité. Eh bien, du dernier recensement, on a tiré un tableau de mortalité extrêmement précis. Même si je prends - parce que je vais bien plus loin - d’après les recensements, le nombre des personnes qui arrivent à 65 ans depuis le moment de leur naissance, sur 100 000 personnes qui naissent, il y a en France, 35 000 qui arrivent à 65 ans, c’est-à-dire 35 \%. Mais j’ai indiqué l’autre jour que cela, c’est la mortalité générale et que lorsqu’il s’agit de prendre de 18 à 61 ans, il y en a plus de 40 \%, et ce que j’ai cherché à montrer, c’est qu’en effet, comme la mortalité des patrons est moins forte que celle des ouvriers, il faut en tenir compte, et j’arrive à montrer qu’il y a 25 \% d’ouvriers qui de 18 ans arrivent à 65 ans. Je ne dis pas que ces ouvriers vont toucher pendant de longues années : il y en a 25 \% qui arrivent à 65 ans. Dans ces 25 \%, il y en a qui vont mourir peu de temps après. Il est extrêmement difficile de savoir - si je l’avais pu ici, je l’aurais cherché - combien il y en a de 70 ans. Je voyais qu’il en restait 10 \%. Il y a de 65 à 70 ans une mortalité trop forte, il y a un très grand nombre de ces 25 \% dont je parle qui toucheront à peine de la retraite, je vous le concède. Mais lorsqu’on dit qu’il y en a que 4 ou 5 \% qui arrivent à 65 ans, on  se trompe. J'ai cherché à établir des chiffres précis. On pense que j’ai voulu défendre la loi : pas du tout ! J'ai voulu montrer ce qu’est la loi. Nous nous appuyons sur ce qui est pour tâcher d’en obtenir davantage. Je serai tout à fait d’accord avec Lafargue pour demander qu’on réduise l’âge et il y a un point sur lequel nous serons d’accord encore, c’est qu’en effet, la mortalité est différente suivant les professions et qu’il y aura lieu de demander à ce que si même on entend laisser le minimum à 65 ans, il y ait des professions où la mortalité est trop forte - les verriers par exemple - et où on doit réduire ce chiffre de 65 ans.\par
\textbf{LAFARGUE}. - Je crois que si Groussier avait étudié la mortalité par profession, il n’aurait pas trouvé beaucoup de salariés âgés de 65 ans. Mais, en tout cas, les 25 \% de vieilards dont il parle ne jouiront pas pendant longtemps de la retraite.\par
Il vient de nous rappeler qu’en Angleterre il ne s’était trouvé, pour réclamer la retraite, que 692 000 vieillards de 70 ans, sur 41 millions d’habitants, soit 1½ \%, et il ajoute que les conditions de vie d’Angleterre et de France étant à peu près semblables, il devait se trouver chez nous la même proportion de vieux de 70 ans, c’est-à-dire 1½ \%, et comme il assure que le nombre des vieux de 65 ans était de 25 \%, donc, de 65 à 70 ans 23 ½ \% de ces vieillards ne toucheront que dans la tombe leur retraite, après l’avoir reçue pendant quelques mois sur terre. On peut donc faire les généreux avec cette retraite, que si peu toucheront si peu de temps.\par
Je conteste les chiffres de Groussier, mais il est important de les discuter ici.\par
Cette retraite que si peu toucheront, tous les salariés devront la payer en impôt pour ne pas la recevoir; la retraite sera illusoire, mais l’impôt etn ainsi que le disait Renard, il y a dans le Nord beaucoup de familles où deux enfants, la mère et le père vont à la fabrique; il paieront donc 24 francs par an, soit 2 francs par mois; quand on connaît les misérables salaires du Nord, on ne peut qu’être effrayés de cet impôt de 24 francs par famille ouvrière. Les pauvres salariés qui ont cherché à réduire leurs dépenses par la Coopération, et malgré les Coopératives, si nombreuses dans le Nord, beaucoup d’ouvriers, qui travaillent, sont obligés de recourir à l’Assistance publique pour nourrir leur famille, et c’est sur ces pauvres que vous allez mettre un nouvel impôt ! "\emph{Craignez, vous} disait Renard, \emph{que lorsque la loi sera votée et appliquée, ils ne se retournent le poing menaçant contre le Parti socialiste}."\par
Aussi ai-je été un de ceux qui ont été étonnés de sa conclusion; parce que ou son argumentation est fausse d’un bout à l’autre, ou sa conclusion est fausse.\par
\textbf{RAPPOPORT}. - L'argument reste.\par
\textbf{LAFARGUE}. - Cet impôt si lourd pour les ouvriers des villes sera écrasant pour les travailleurs des champs. Vous, Jaurès, qui êtes élu par les paysans, vous Mauger, qui habitez le village, vous connaissez leur misère; vous savez que le salaire moyen des campagnes est de 400 à 300 francs par an, et pour que la famille puisse manger, il faut que le père, la mère et les enfants travaillent, et c’est à ces misérables qu’on demandera 24 francs ! A eux qui ont tant de peine à se procurer de l’argent; car bien souvent on les paie en produits, en marchandises; l’argent est une rareté à la campagne; il a même pour le paysan propriétaire une valeur inappréciable. C'est lorsque l’argent a une valeur si grande, plus grande que dans la ville, que vous allez obliger la famille paysanne à payer 24 francs en argent comptant. Si, comme le disait Renard, les villes se retourneront furieuses contre le Parti socialiste, plus furieuses encore se retourneront contre lui les campagnes, s’il vote l’impôt des retraites. La propagande dans les campagnes deviendra bien plus difficile, parce que l’on met l’impôt non seulement sur les journaliers, les valets et les filles de ferme, mais encore sur les fermiers et les propriétaires qui les emploient.\par
\textbf{COMPERE-MOREL}. - Ils ne pourront pas payer, ils ne gagnent pas assez.\par
\textbf{LAFARGUE}. - Les propriétaires et les fermiers diront aux journaliers et aux valets de ferme : "\emph{C'est le Parti socialiste qui vous a mis cet impôt sur les épaules}", et ne croyez-vous pas que lorsque dans les villes le patron prélèvera l’impôt sur les salariés et lorsque ceux-ci grommelleront, il leur dira : "\emph{Ce n’est pas contre moi qu’il faut vous fâcher, mais contre le Parti socialiste, qui a voté l’impôt des retraites pour les morts}."\par
Lorsque les ouvriers seront organisés, syndiqués, le patron, de peur de la grève et du tapage, n’osera pas leur faire payer sa quote-part d’impôt; mais, dans les petits ateliers, où travaillent de pauvres diables, sans organisation, d’autant moins payés qu’ils sont plus misérables, le patron prélèvera sur leur salaire sa part d’impôt en même temps que celle du salarié.\par
Les radicaux, qui n’ont pas voulu établir l’impôt sur le revenu, l’ont mis sur les salaires.\emph{ (Applaudissements.)} Et cet impôt est progressif, mais à rebours; ceux qui gagnent le moins, ceux qui sont les plus misérables, les plus écrasés, paieront le plus.\par
Voilà la belle loi que vous allez voter. Je comprends la colère, la fureur de la C.G.T., quand elle voit à quoi aboutissent les efforts parlementaires du parti radical et du Parti socialiste. Il fut un temps où la C.G.T. était un Dieu, aussi terrible que le dieu des juifs et des chrétiens, qu’il ne fallait pas irriter; on ne devait rien dire qui pût lui déplaire, sous peine de dangereuse colère. La C.G.T. était tabou, il fallait s’aplatir devant elle, ne jamais la contredire, lui passer toutes ses fantaisies. J'ai été un de ceux qui ne se sont pas inclinés devant elle et qui ont osé l’attaquer, quand c’était utile, et ce sont précisément ceux qui s’agenouillaient humblement devant elle qui, aujourd’hui, lui dénient le droit de s’occuper des retraites ouvrières et proclament que nous devons être un Parti autonome.\par
\textbf{VAILLANT}. - Nous avons toujours dit cela : autonomie des deux côtés.\par
\textbf{LAFARGUE}. - Vous n’étiez pas autonome vis-à-vis de la C.G.T.; souvenez-vous de votre déclaration à Nancy : alors vous n’hésitiez pas à sacrifier l’opinion et l’intérêt du Parti socialiste au caprice de la C.G.T., et aujourd’hui vous ne voulez  pas accepter son indication sur une question ouvrière, essentiellement de son ressort. Méric vous le rappelait, elle a deux fois, en 1901 et en 1907, nettement indiqué sa volonté, et pour moi, il est du devoir du Parti socialiste d’en tenir compte, afin d’établir et d’entretenir de bonnes relations avec la C.G.T.\par
Est-ce que vous croyez que la Confédération, à qui vous avez attribué une puissance extraordinaire, puissance à laquelle je n’ai jamais cru, va être satisfaite ? Votez la loi des retraites, c’est déclarer la guerre…\par
\textbf{VAILLANT}. - Jamais !\par
\textbf{LAFARGUE}. - … entre la C.G.T. et le Parti socialiste; partout où ses délégués iront, ils reprocheront au parti socialiste l’impôt sur les salaires et l’âge de la retraite. Ils commencent déjà. J'ai reçu par l’intermédiaire de l’\emph{Humanité} une lettre de l’Emancipatrice de Hermes, que Héliès doit connaître…\par
\textbf{HELIES}. - L'Emancipation.\par
\textbf{LAFARGUE}. - … qui me dit : "\emph{Le 12, nous organisons une conférence, avec un délégué de la C.G.T., sur les retraites ouvrières; nous savons que vous partagez les opinions de la C.G.T.; voulez-vous venir traiter avec lui le même sujet} ?"\par
Si vous avez l’intelligence et le courage de repousser la loi, il sera facile de s’entendre pour une action commune avec la Confédération du Travail. Moi, qui n’ai jamais cru à l’autonomie de la Coopération, de la Confédération et du Parti socialiste, mais qui pense que ces trois organisations ouvrières doivent unir leur forces pour une action commune…\par
\textbf{VAILLANT}. - Il faut pouvoir.\par
\textbf{LAFARGUE}. - … Ce n’est pas du jour au lendemain que l’union se fera; en ce moment, une occasion exceptionnelle nous est offerte de la faire. Le Parti socialiste peut dire aux confédérés : "\emph{Menons ensemble la campagne contre cette loi de mensonge et d’escroquerie; allons ensemble dans toute la France, soulever la classe ouvrière contre les députés et les sénateurs qui la voteront}. " \emph{(Applaudissements.)}\par
Vaillant, vous parliez d’action et de mouvement à créer, combien grande serait l’agitation, si les délégués de la C.G.T. et du Parti socialiste allaient dans toutes les Bourses du Travail, dans les villes et dans les campagnes, dénoncer cette loi de duperie, exciter les salariés et les patrons à la révolte pour qu’ils ne paient pas l’impôt des retraites, et réclamer une loi d’assurance contre le chômage, la maladie, l’invalidité et la vieillesse, et demander que la vieillesse ouvrière ne commence ni à 60 ans, ni à 50 ans, mais quand le salarié surmené et épuisé ne peut plus travailler et doit se reposer.\par
Vous savez aussi bien que moi que les forces du salarié baissent à 40 ans, et que, dans la grande industrie, on n’embauche plus à cet âge; on commence alors à désembaucher et à diminuer les salaires; alors s’abattent sur le travailleur la misère, la maladie, la souffrance physique et morale : c’est pour cela que je suis avec Vaillant pour réclamer une loi d’assurance contre les misères et les dangers qui assaillent le salarié.\par
Vous, Vaillant, qui dans votre longue vie de luttes avez vu tant d’injustices, vous avez un optimisme parlementaire qui me surprend : comment pouvez-vous croire que la loi, une fois votée, vous pourrez la pétrir comme de l’argile, lui faire un nez, des oreilles, des mains… \emph{(Rires.)} Ah! Non, la loi restera telle que vous la voterez, pendant des dix ou douze ans; tout au plus modifiera-t-on des articles insignifiants, à moins que la classe ouvrière et paysanne, rendue furieuse par l’impôt qu’on prélèvera sur elle, ne se mette de la partie. Rien de plus difficile que de changer une loi par la méthode parlementaire : depuis 1830, c’est-à-dire depuis quatre-vingt ans, on veut diminuer les frais judiciaires, si énormes que parfois ils sont supérieurs à la valeur des objets saisis ou vendus; à différentes époques, des projets de loi ont été déposés, discutés et enterrés, et la spoliation judiciaire continue; il n’y a rien de plus permanent que les iniquités qui profitent aux gens de justice. Jaurès, la justice légalise le vol.\par
\textbf{HERVÉ}. - N'insultez pas la justice.\emph{ (Rires.)}\par
\textbf{LAFARGUE}. - Mon pauvre Hervé, vous allez renouer connaissance avec les iniquités de la justice. Je vous plains, et si elle vous emprisonne, nous regretterons la force de propagande qu’elle paralysera. \emph{(Applaudissements.)} Mais si vous avez le bonheur d’échapper aux sales pattes de la justice, j’espère que vous viendrez avec nous faire la campagne contre la loi des retraites et pour la loi d’assurance.\par
Ma conclusion, la voici : le Parti socialiste doit voter contre la loi et déposer immédiatement un projet de loi contre le chômage, la maladie, l’invalidité et la vieillesse et organiser avec la C.G.T., avec Hervé, avec n’importe qui, une vaste agitation dans tout le pays.\par
\textbf{SEMBAT}. - Ils ne vous suivront plus, ils ne sont pas pour la loi, contre oui, pour non.\par
\textbf{LAFARGUE}. - S'ils ne marchent pas, tant pis. Nous sommes assez nombreux pour la besogne. Je suis vieux, mais je m’y mettrai avec courage. Je termine donc en vous disant : Votez contre la loi ! \emph{(Applaudissements.)}
 


% at least one empty page at end (for booklet couv)
\ifbooklet
  \pagestyle{empty}
  \clearpage
  % 2 empty pages maybe needed for 4e cover
  \ifnum\modulo{\value{page}}{4}=0 \hbox{}\newpage\hbox{}\newpage\fi
  \ifnum\modulo{\value{page}}{4}=1 \hbox{}\newpage\hbox{}\newpage\fi


  \hbox{}\newpage
  \ifodd\value{page}\hbox{}\newpage\fi
  {\centering\color{rubric}\bfseries\noindent\large
    Hurlus ? Qu’est-ce.\par
    \bigskip
  }
  \noindent Des bouquinistes électroniques, pour du texte libre à participation libre,
  téléchargeable gratuitement sur \href{https://hurlus.fr}{\dotuline{hurlus.fr}}.\par
  \bigskip
  \noindent Cette brochure a été produite par des éditeurs bénévoles.
  Elle n’est pas faîte pour être possédée, mais pour être lue, et puis donnée.
  Que circule le texte !
  En page de garde, on peut ajouter une date, un lieu, un nom ; pour suivre le voyage des idées.
  \par

  Ce texte a été choisi parce qu’une personne l’a aimé,
  ou haï, elle a en tous cas pensé qu’il partipait à la formation de notre présent ;
  sans le souci de plaire, vendre, ou militer pour une cause.
  \par

  L’édition électronique est soigneuse, tant sur la technique
  que sur l’établissement du texte ; mais sans aucune prétention scolaire, au contraire.
  Le but est de s’adresser à tous, sans distinction de science ou de diplôme.
  Au plus direct ! (possible)
  \par

  Cet exemplaire en papier a été tiré sur une imprimante personnelle
   ou une photocopieuse. Tout le monde peut le faire.
  Il suffit de
  télécharger un fichier sur \href{https://hurlus.fr}{\dotuline{hurlus.fr}},
  d’imprimer, et agrafer ; puis de lire et donner.\par

  \bigskip

  \noindent PS : Les hurlus furent aussi des rebelles protestants qui cassaient les statues dans les églises catholiques. En 1566 démarra la révolte des gueux dans le pays de Lille. L’insurrection enflamma la région jusqu’à Anvers où les gueux de mer bloquèrent les bateaux espagnols.
  Ce fut une rare guerre de libération dont naquit un pays toujours libre : les Pays-Bas.
  En plat pays francophone, par contre, restèrent des bandes de huguenots, les hurlus, progressivement réprimés par la très catholique Espagne.
  Cette mémoire d’une défaite est éteinte, rallumons-la. Sortons les livres du culte universitaire, cherchons les idoles de l’époque, pour les briser.
\fi

\ifdev % autotext in dev mode
\fontname\font — \textsc{Les règles du jeu}\par
(\hyperref[utopie]{\underline{Lien}})\par
\noindent \initialiv{A}{lors là}\blindtext\par
\noindent \initialiv{À}{ la bonheur des dames}\blindtext\par
\noindent \initialiv{É}{tonnez-le}\blindtext\par
\noindent \initialiv{Q}{ualitativement}\blindtext\par
\noindent \initialiv{V}{aloriser}\blindtext\par
\Blindtext
\phantomsection
\label{utopie}
\Blinddocument
\fi
\end{document}
