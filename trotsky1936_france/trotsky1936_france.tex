%%%%%%%%%%%%%%%%%%%%%%%%%%%%%%%%%
% LaTeX model https://hurlus.fr %
%%%%%%%%%%%%%%%%%%%%%%%%%%%%%%%%%

% Needed before document class
\RequirePackage{pdftexcmds} % needed for tests expressions
\RequirePackage{fix-cm} % correct units

% Define mode
\def\mode{a4}

\newif\ifaiv % a4
\newif\ifav % a5
\newif\ifbooklet % booklet
\newif\ifcover % cover for booklet

\ifnum \strcmp{\mode}{cover}=0
  \covertrue
\else\ifnum \strcmp{\mode}{booklet}=0
  \booklettrue
\else\ifnum \strcmp{\mode}{a5}=0
  \avtrue
\else
  \aivtrue
\fi\fi\fi

\ifbooklet % do not enclose with {}
  \documentclass[french,twoside]{book} % ,notitlepage
  \usepackage[%
    papersize={105mm, 297mm},
    inner=12mm,
    outer=12mm,
    top=20mm,
    bottom=15mm,
    marginparsep=0pt,
  ]{geometry}
  \usepackage[fontsize=9.5pt]{scrextend} % for Roboto
\else\ifav
  \documentclass[french,twoside]{book} % ,notitlepage
  \usepackage[%
    a5paper,
    inner=25mm,
    outer=15mm,
    top=15mm,
    bottom=15mm,
    marginparsep=0pt,
  ]{geometry}
  \usepackage[fontsize=12pt]{scrextend}
\else% A4 2 cols
  \documentclass[twocolumn]{report}
  \usepackage[%
    a4paper,
    inner=15mm,
    outer=10mm,
    top=25mm,
    bottom=18mm,
    marginparsep=0pt,
  ]{geometry}
  \setlength{\columnsep}{20mm}
  \usepackage[fontsize=9.5pt]{scrextend}
\fi\fi

%%%%%%%%%%%%%%
% Alignments %
%%%%%%%%%%%%%%
% before teinte macros

\setlength{\arrayrulewidth}{0.2pt}
\setlength{\columnseprule}{\arrayrulewidth} % twocol
\setlength{\parskip}{0pt} % classical para with no margin
\setlength{\parindent}{1.5em}

%%%%%%%%%%
% Colors %
%%%%%%%%%%
% before Teinte macros

\usepackage[dvipsnames]{xcolor}
\definecolor{rubric}{HTML}{800000} % the tonic 0c71c3
\def\columnseprulecolor{\color{rubric}}
\colorlet{borderline}{rubric!30!} % definecolor need exact code
\definecolor{shadecolor}{gray}{0.95}
\definecolor{bghi}{gray}{0.5}

%%%%%%%%%%%%%%%%%
% Teinte macros %
%%%%%%%%%%%%%%%%%
%%%%%%%%%%%%%%%%%%%%%%%%%%%%%%%%%%%%%%%%%%%%%%%%%%%
% <TEI> generic (LaTeX names generated by Teinte) %
%%%%%%%%%%%%%%%%%%%%%%%%%%%%%%%%%%%%%%%%%%%%%%%%%%%
% This template is inserted in a specific design
% It is XeLaTeX and otf fonts

\makeatletter % <@@@


\usepackage{blindtext} % generate text for testing
\usepackage[strict]{changepage} % for modulo 4
\usepackage{contour} % rounding words
\usepackage[nodayofweek]{datetime}
% \usepackage{DejaVuSans} % seems buggy for sffont font for symbols
\usepackage{enumitem} % <list>
\usepackage{etoolbox} % patch commands
\usepackage{fancyvrb}
\usepackage{fancyhdr}
\usepackage{float}
\usepackage{fontspec} % XeLaTeX mandatory for fonts
\usepackage{footnote} % used to capture notes in minipage (ex: quote)
\usepackage{framed} % bordering correct with footnote hack
\usepackage{graphicx}
\usepackage{lettrine} % drop caps
\usepackage{lipsum} % generate text for testing
\usepackage[framemethod=tikz,]{mdframed} % maybe used for frame with footnotes inside
\usepackage{pdftexcmds} % needed for tests expressions
\usepackage{polyglossia} % non-break space french punct, bug Warning: "Failed to patch part"
\usepackage[%
  indentfirst=false,
  vskip=1em,
  noorphanfirst=true,
  noorphanafter=true,
  leftmargin=\parindent,
  rightmargin=0pt,
]{quoting}
\usepackage{ragged2e}
\usepackage{setspace} % \setstretch for <quote>
\usepackage{tabularx} % <table>
\usepackage[explicit]{titlesec} % wear titles, !NO implicit
\usepackage{tikz} % ornaments
\usepackage{tocloft} % styling tocs
\usepackage[fit]{truncate} % used im runing titles
\usepackage{unicode-math}
\usepackage[normalem]{ulem} % breakable \uline, normalem is absolutely necessary to keep \emph
\usepackage{verse} % <l>
\usepackage{xcolor} % named colors
\usepackage{xparse} % @ifundefined
\XeTeXdefaultencoding "iso-8859-1" % bad encoding of xstring
\usepackage{xstring} % string tests
\XeTeXdefaultencoding "utf-8"
\PassOptionsToPackage{hyphens}{url} % before hyperref, which load url package

% TOTEST
% \usepackage{hypcap} % links in caption ?
% \usepackage{marginnote}
% TESTED
% \usepackage{background} % doesn’t work with xetek
% \usepackage{bookmark} % prefers the hyperref hack \phantomsection
% \usepackage[color, leftbars]{changebar} % 2 cols doc, impossible to keep bar left
% \usepackage[utf8x]{inputenc} % inputenc package ignored with utf8 based engines
% \usepackage[sfdefault,medium]{inter} % no small caps
% \usepackage{firamath} % choose firasans instead, firamath unavailable in Ubuntu 21-04
% \usepackage{flushend} % bad for last notes, supposed flush end of columns
% \usepackage[stable]{footmisc} % BAD for complex notes https://texfaq.org/FAQ-ftnsect
% \usepackage{helvet} % not for XeLaTeX
% \usepackage{multicol} % not compatible with too much packages (longtable, framed, memoir…)
% \usepackage[default,oldstyle,scale=0.95]{opensans} % no small caps
% \usepackage{sectsty} % \chapterfont OBSOLETE
% \usepackage{soul} % \ul for underline, OBSOLETE with XeTeX
% \usepackage[breakable]{tcolorbox} % text styling gone, footnote hack not kept with breakable


% Metadata inserted by a program, from the TEI source, for title page and runing heads
\title{\textbf{ Où va la France ? }}
\date{1936}
\author{Léon Trotksy}
\def\elbibl{Léon Trotksy. 1936. \emph{Où va la France ?}}
\def\elsource{ \href{http://gallica.bnf.fr/ark:/12148/bpt6k5813514w}{\dotuline{http://gallica.bnf.fr/ark:/12148/bpt6k5813514w}}\footnote{\href{http://gallica.bnf.fr/ark:/12148/bpt6k5813514w}{\url{http://gallica.bnf.fr/ark:/12148/bpt6k5813514w}}}  \href{http://efele.net/ebooks/livres/000293}{\dotuline{http://efele.net/ebooks/livres/000293}}\footnote{\href{http://efele.net/ebooks/livres/000293}{\url{http://efele.net/ebooks/livres/000293}}} }

% Default metas
\newcommand{\colorprovide}[2]{\@ifundefinedcolor{#1}{\colorlet{#1}{#2}}{}}
\colorprovide{rubric}{red}
\colorprovide{silver}{lightgray}
\@ifundefined{syms}{\newfontfamily\syms{DejaVu Sans}}{}
\newif\ifdev
\@ifundefined{elbibl}{% No meta defined, maybe dev mode
  \newcommand{\elbibl}{Titre court ?}
  \newcommand{\elbook}{Titre du livre source ?}
  \newcommand{\elabstract}{Résumé\par}
  \newcommand{\elurl}{http://oeuvres.github.io/elbook/2}
  \author{Éric Lœchien}
  \title{Un titre de test assez long pour vérifier le comportement d’une maquette}
  \date{1566}
  \devtrue
}{}
\let\eltitle\@title
\let\elauthor\@author
\let\eldate\@date


\defaultfontfeatures{
  % Mapping=tex-text, % no effect seen
  Scale=MatchLowercase,
  Ligatures={TeX,Common},
}


% generic typo commands
\newcommand{\astermono}{\medskip\centerline{\color{rubric}\large\selectfont{\syms ✻}}\medskip\par}%
\newcommand{\astertri}{\medskip\par\centerline{\color{rubric}\large\selectfont{\syms ✻\,✻\,✻}}\medskip\par}%
\newcommand{\asterism}{\bigskip\par\noindent\parbox{\linewidth}{\centering\color{rubric}\large{\syms ✻}\\{\syms ✻}\hskip 0.75em{\syms ✻}}\bigskip\par}%

% lists
\newlength{\listmod}
\setlength{\listmod}{\parindent}
\setlist{
  itemindent=!,
  listparindent=\listmod,
  labelsep=0.2\listmod,
  parsep=0pt,
  % topsep=0.2em, % default topsep is best
}
\setlist[itemize]{
  label=—,
  leftmargin=0pt,
  labelindent=1.2em,
  labelwidth=0pt,
}
\setlist[enumerate]{
  label={\bf\color{rubric}\arabic*.},
  labelindent=0.8\listmod,
  leftmargin=\listmod,
  labelwidth=0pt,
}
\newlist{listalpha}{enumerate}{1}
\setlist[listalpha]{
  label={\bf\color{rubric}\alph*.},
  leftmargin=0pt,
  labelindent=0.8\listmod,
  labelwidth=0pt,
}
\newcommand{\listhead}[1]{\hspace{-1\listmod}\emph{#1}}

\renewcommand{\hrulefill}{%
  \leavevmode\leaders\hrule height 0.2pt\hfill\kern\z@}

% General typo
\DeclareTextFontCommand{\textlarge}{\large}
\DeclareTextFontCommand{\textsmall}{\small}

% commands, inlines
\newcommand{\anchor}[1]{\Hy@raisedlink{\hypertarget{#1}{}}} % link to top of an anchor (not baseline)
\newcommand\abbr[1]{#1}
\newcommand{\autour}[1]{\tikz[baseline=(X.base)]\node [draw=rubric,thin,rectangle,inner sep=1.5pt, rounded corners=3pt] (X) {\color{rubric}#1};}
\newcommand\corr[1]{#1}
\newcommand{\ed}[1]{ {\color{silver}\sffamily\footnotesize (#1)} } % <milestone ed="1688"/>
\newcommand\expan[1]{#1}
\newcommand\foreign[1]{\emph{#1}}
\newcommand\gap[1]{#1}
\renewcommand{\LettrineFontHook}{\color{rubric}}
\newcommand{\initial}[2]{\lettrine[lines=2, loversize=0.3, lhang=0.3]{#1}{#2}}
\newcommand{\initialiv}[2]{%
  \let\oldLFH\LettrineFontHook
  % \renewcommand{\LettrineFontHook}{\color{rubric}\ttfamily}
  \IfSubStr{QJ’}{#1}{
    \lettrine[lines=4, lhang=0.2, loversize=-0.1, lraise=0.2]{\smash{#1}}{#2}
  }{\IfSubStr{É}{#1}{
    \lettrine[lines=4, lhang=0.2, loversize=-0, lraise=0]{\smash{#1}}{#2}
  }{\IfSubStr{ÀÂ}{#1}{
    \lettrine[lines=4, lhang=0.2, loversize=-0, lraise=0, slope=0.6em]{\smash{#1}}{#2}
  }{\IfSubStr{A}{#1}{
    \lettrine[lines=4, lhang=0.2, loversize=0.2, slope=0.6em]{\smash{#1}}{#2}
  }{\IfSubStr{V}{#1}{
    \lettrine[lines=4, lhang=0.2, loversize=0.2, slope=-0.5em]{\smash{#1}}{#2}
  }{
    \lettrine[lines=4, lhang=0.2, loversize=0.2]{\smash{#1}}{#2}
  }}}}}
  \let\LettrineFontHook\oldLFH
}
\newcommand{\labelchar}[1]{\textbf{\color{rubric} #1}}
\newcommand{\milestone}[1]{\autour{\footnotesize\color{rubric} #1}} % <milestone n="4"/>
\newcommand\name[1]{#1}
\newcommand\orig[1]{#1}
\newcommand\orgName[1]{#1}
\newcommand\persName[1]{#1}
\newcommand\placeName[1]{#1}
\newcommand{\pn}[1]{\IfSubStr{-—–¶}{#1}% <p n="3"/>
  {\noindent{\bfseries\color{rubric}   ¶  }}
  {{\footnotesize\autour{ #1}  }}}
\newcommand\reg{}
% \newcommand\ref{} % already defined
\newcommand\sic[1]{#1}
\newcommand\surname[1]{\textsc{#1}}
\newcommand\term[1]{\textbf{#1}}

\def\mednobreak{\ifdim\lastskip<\medskipamount
  \removelastskip\nopagebreak\medskip\fi}
\def\bignobreak{\ifdim\lastskip<\bigskipamount
  \removelastskip\nopagebreak\bigskip\fi}

% commands, blocks
\newcommand{\byline}[1]{\bigskip{\RaggedLeft{#1}\par}\bigskip}
\newcommand{\bibl}[1]{{\RaggedLeft{#1}\par\bigskip}}
\newcommand{\biblitem}[1]{{\noindent\hangindent=\parindent   #1\par}}
\newcommand{\dateline}[1]{\medskip{\RaggedLeft{#1}\par}\bigskip}
\newcommand{\labelblock}[1]{\medbreak{\noindent\color{rubric}\bfseries #1}\par\mednobreak}
\newcommand{\salute}[1]{\bigbreak{#1}\par\medbreak}
\newcommand{\signed}[1]{\bigbreak\filbreak{\raggedleft #1\par}\medskip}

% environments for blocks (some may become commands)
\newenvironment{borderbox}{}{} % framing content
\newenvironment{citbibl}{\ifvmode\hfill\fi}{\ifvmode\par\fi }
\newenvironment{docAuthor}{\ifvmode\vskip4pt\fontsize{16pt}{18pt}\selectfont\fi\itshape}{\ifvmode\par\fi }
\newenvironment{docDate}{}{\ifvmode\par\fi }
\newenvironment{docImprint}{\vskip6pt}{\ifvmode\par\fi }
\newenvironment{docTitle}{\vskip6pt\bfseries\fontsize{18pt}{22pt}\selectfont}{\par }
\newenvironment{msHead}{\vskip6pt}{\par}
\newenvironment{msItem}{\vskip6pt}{\par}
\newenvironment{titlePart}{}{\par }


% environments for block containers
\newenvironment{argument}{\itshape\parindent0pt}{\vskip1.5em}
\newenvironment{biblfree}{}{\ifvmode\par\fi }
\newenvironment{bibitemlist}[1]{%
  \list{\@biblabel{\@arabic\c@enumiv}}%
  {%
    \settowidth\labelwidth{\@biblabel{#1}}%
    \leftmargin\labelwidth
    \advance\leftmargin\labelsep
    \@openbib@code
    \usecounter{enumiv}%
    \let\p@enumiv\@empty
    \renewcommand\theenumiv{\@arabic\c@enumiv}%
  }
  \sloppy
  \clubpenalty4000
  \@clubpenalty \clubpenalty
  \widowpenalty4000%
  \sfcode`\.\@m
}%
{\def\@noitemerr
  {\@latex@warning{Empty `bibitemlist' environment}}%
\endlist}
\newenvironment{quoteblock}% may be used for ornaments
  {\begin{quoting}}
  {\end{quoting}}

% table () is preceded and finished by custom command
\newcommand{\tableopen}[1]{%
  \ifnum\strcmp{#1}{wide}=0{%
    \begin{center}
  }
  \else\ifnum\strcmp{#1}{long}=0{%
    \begin{center}
  }
  \else{%
    \begin{center}
  }
  \fi\fi
}
\newcommand{\tableclose}[1]{%
  \ifnum\strcmp{#1}{wide}=0{%
    \end{center}
  }
  \else\ifnum\strcmp{#1}{long}=0{%
    \end{center}
  }
  \else{%
    \end{center}
  }
  \fi\fi
}


% text structure
\newcommand\chapteropen{} % before chapter title
\newcommand\chaptercont{} % after title, argument, epigraph…
\newcommand\chapterclose{} % maybe useful for multicol settings
\setcounter{secnumdepth}{-2} % no counters for hierarchy titles
\setcounter{tocdepth}{5} % deep toc
\markright{\@title} % ???
\markboth{\@title}{\@author} % ???
\renewcommand\tableofcontents{\@starttoc{toc}}
% toclof format
% \renewcommand{\@tocrmarg}{0.1em} % Useless command?
% \renewcommand{\@pnumwidth}{0.5em} % {1.75em}
\renewcommand{\@cftmaketoctitle}{}
\setlength{\cftbeforesecskip}{\z@ \@plus.2\p@}
\renewcommand{\cftchapfont}{}
\renewcommand{\cftchapdotsep}{\cftdotsep}
\renewcommand{\cftchapleader}{\normalfont\cftdotfill{\cftchapdotsep}}
\renewcommand{\cftchappagefont}{\bfseries}
\setlength{\cftbeforechapskip}{0em \@plus\p@}
% \renewcommand{\cftsecfont}{\small\relax}
\renewcommand{\cftsecpagefont}{\normalfont}
% \renewcommand{\cftsubsecfont}{\small\relax}
\renewcommand{\cftsecdotsep}{\cftdotsep}
\renewcommand{\cftsecpagefont}{\normalfont}
\renewcommand{\cftsecleader}{\normalfont\cftdotfill{\cftsecdotsep}}
\setlength{\cftsecindent}{1em}
\setlength{\cftsubsecindent}{2em}
\setlength{\cftsubsubsecindent}{3em}
\setlength{\cftchapnumwidth}{1em}
\setlength{\cftsecnumwidth}{1em}
\setlength{\cftsubsecnumwidth}{1em}
\setlength{\cftsubsubsecnumwidth}{1em}

% footnotes
\newif\ifheading
\newcommand*{\fnmarkscale}{\ifheading 0.70 \else 1 \fi}
\renewcommand\footnoterule{\vspace*{0.3cm}\hrule height \arrayrulewidth width 3cm \vspace*{0.3cm}}
\setlength\footnotesep{1.5\footnotesep} % footnote separator
\renewcommand\@makefntext[1]{\parindent 1.5em \noindent \hb@xt@1.8em{\hss{\normalfont\@thefnmark . }}#1} % no superscipt in foot
\patchcmd{\@footnotetext}{\footnotesize}{\footnotesize\sffamily}{}{} % before scrextend, hyperref


%   see https://tex.stackexchange.com/a/34449/5049
\def\truncdiv#1#2{((#1-(#2-1)/2)/#2)}
\def\moduloop#1#2{(#1-\truncdiv{#1}{#2}*#2)}
\def\modulo#1#2{\number\numexpr\moduloop{#1}{#2}\relax}

% orphans and widows
\clubpenalty=9996
\widowpenalty=9999
\brokenpenalty=4991
\predisplaypenalty=10000
\postdisplaypenalty=1549
\displaywidowpenalty=1602
\hyphenpenalty=400
% Copied from Rahtz but not understood
\def\@pnumwidth{1.55em}
\def\@tocrmarg {2.55em}
\def\@dotsep{4.5}
\emergencystretch 3em
\hbadness=4000
\pretolerance=750
\tolerance=2000
\vbadness=4000
\def\Gin@extensions{.pdf,.png,.jpg,.mps,.tif}
% \renewcommand{\@cite}[1]{#1} % biblio

\usepackage{hyperref} % supposed to be the last one, :o) except for the ones to follow
\urlstyle{same} % after hyperref
\hypersetup{
  % pdftex, % no effect
  pdftitle={\elbibl},
  % pdfauthor={Your name here},
  % pdfsubject={Your subject here},
  % pdfkeywords={keyword1, keyword2},
  bookmarksnumbered=true,
  bookmarksopen=true,
  bookmarksopenlevel=1,
  pdfstartview=Fit,
  breaklinks=true, % avoid long links
  pdfpagemode=UseOutlines,    % pdf toc
  hyperfootnotes=true,
  colorlinks=false,
  pdfborder=0 0 0,
  % pdfpagelayout=TwoPageRight,
  % linktocpage=true, % NO, toc, link only on page no
}

\makeatother % /@@@>
%%%%%%%%%%%%%%
% </TEI> end %
%%%%%%%%%%%%%%


%%%%%%%%%%%%%
% footnotes %
%%%%%%%%%%%%%
\renewcommand{\thefootnote}{\bfseries\textcolor{rubric}{\arabic{footnote}}} % color for footnote marks

%%%%%%%%%
% Fonts %
%%%%%%%%%
\usepackage[]{roboto} % SmallCaps, Regular is a bit bold
% \linespread{0.90} % too compact, keep font natural
\newfontfamily\fontrun[]{Roboto Condensed Light} % condensed runing heads
\ifav
  \setmainfont[
    ItalicFont={Roboto Light Italic},
  ]{Roboto}
\else\ifbooklet
  \setmainfont[
    ItalicFont={Roboto Light Italic},
  ]{Roboto}
\else
\setmainfont[
  ItalicFont={Roboto Italic},
]{Roboto Light}
\fi\fi
\renewcommand{\LettrineFontHook}{\bfseries\color{rubric}}
% \renewenvironment{labelblock}{\begin{center}\bfseries\color{rubric}}{\end{center}}

%%%%%%%%
% MISC %
%%%%%%%%

\setdefaultlanguage[frenchpart=false]{french} % bug on part


\newenvironment{quotebar}{%
    \def\FrameCommand{{\color{rubric!10!}\vrule width 0.5em} \hspace{0.9em}}%
    \def\OuterFrameSep{\itemsep} % séparateur vertical
    \MakeFramed {\advance\hsize-\width \FrameRestore}
  }%
  {%
    \endMakeFramed
  }
\renewenvironment{quoteblock}% may be used for ornaments
  {%
    \savenotes
    \setstretch{0.9}
    \normalfont
    \begin{quotebar}
  }
  {%
    \end{quotebar}
    \spewnotes
  }


\renewcommand{\headrulewidth}{\arrayrulewidth}
\renewcommand{\headrule}{{\color{rubric}\hrule}}

% delicate tuning, image has produce line-height problems in title on 2 lines
\titleformat{name=\chapter} % command
  [display] % shape
  {\vspace{1.5em}\centering} % format
  {} % label
  {0pt} % separator between n
  {}
[{\color{rubric}\huge\textbf{#1}}\bigskip] % after code
% \titlespacing{command}{left spacing}{before spacing}{after spacing}[right]
\titlespacing*{\chapter}{0pt}{-2em}{0pt}[0pt]

\titleformat{name=\section}
  [block]{}{}{}{}
  [\vbox{\color{rubric}\large\raggedleft\textbf{#1}}]
\titlespacing{\section}{0pt}{0pt plus 4pt minus 2pt}{\baselineskip}

\titleformat{name=\subsection}
  [block]
  {}
  {} % \thesection
  {} % separator \arrayrulewidth
  {}
[\vbox{\large\textbf{#1}}]
% \titlespacing{\subsection}{0pt}{0pt plus 4pt minus 2pt}{\baselineskip}

\ifaiv
  \fancypagestyle{main}{%
    \fancyhf{}
    \setlength{\headheight}{1.5em}
    \fancyhead{} % reset head
    \fancyfoot{} % reset foot
    \fancyhead[L]{\truncate{0.45\headwidth}{\fontrun\elbibl}} % book ref
    \fancyhead[R]{\truncate{0.45\headwidth}{ \fontrun\nouppercase\leftmark}} % Chapter title
    \fancyhead[C]{\thepage}
  }
  \fancypagestyle{plain}{% apply to chapter
    \fancyhf{}% clear all header and footer fields
    \setlength{\headheight}{1.5em}
    \fancyhead[L]{\truncate{0.9\headwidth}{\fontrun\elbibl}}
    \fancyhead[R]{\thepage}
  }
\else
  \fancypagestyle{main}{%
    \fancyhf{}
    \setlength{\headheight}{1.5em}
    \fancyhead{} % reset head
    \fancyfoot{} % reset foot
    \fancyhead[RE]{\truncate{0.9\headwidth}{\fontrun\elbibl}} % book ref
    \fancyhead[LO]{\truncate{0.9\headwidth}{\fontrun\nouppercase\leftmark}} % Chapter title, \nouppercase needed
    \fancyhead[RO,LE]{\thepage}
  }
  \fancypagestyle{plain}{% apply to chapter
    \fancyhf{}% clear all header and footer fields
    \setlength{\headheight}{1.5em}
    \fancyhead[L]{\truncate{0.9\headwidth}{\fontrun\elbibl}}
    \fancyhead[R]{\thepage}
  }
\fi

\ifav % a5 only
  \titleclass{\section}{top}
\fi

\newcommand\chapo{{%
  \vspace*{-3em}
  \centering % no vskip ()
  {\Large\addfontfeature{LetterSpace=25}\bfseries{\elauthor}}\par
  \smallskip
  {\large\eldate}\par
  \bigskip
  {\Large\selectfont{\eltitle}}\par
  \bigskip
  {\color{rubric}\hline\par}
  \bigskip
  {\Large TEXTE LIBRE À PARTICPATION LIBRE\par}
  \centerline{\small\color{rubric} {hurlus.fr, tiré le \today}}\par
  \bigskip
}}

\newcommand\cover{{%
  \thispagestyle{empty}
  \centering
  {\LARGE\bfseries{\elauthor}}\par
  \bigskip
  {\Large\eldate}\par
  \bigskip
  \bigskip
  {\LARGE\selectfont{\eltitle}}\par
  \vfill\null
  {\color{rubric}\setlength{\arrayrulewidth}{2pt}\hline\par}
  \vfill\null
  {\Large TEXTE LIBRE À PARTICPATION LIBRE\par}
  \centerline{{\href{https://hurlus.fr}{\dotuline{hurlus.fr}}, tiré le \today}}\par
}}

\begin{document}
\pagestyle{empty}
\ifbooklet{
  \cover\newpage
  \thispagestyle{empty}\hbox{}\newpage
  \cover\newpage\noindent Les voyages de la brochure\par
  \bigskip
  \begin{tabularx}{\textwidth}{l|X|X}
    \textbf{Date} & \textbf{Lieu}& \textbf{Nom/pseudo} \\ \hline
    \rule{0pt}{25cm} &  &   \\
  \end{tabularx}
  \newpage
  \addtocounter{page}{-4}
}\fi

\thispagestyle{empty}
\ifaiv
  \twocolumn[\chapo]
\else
  \chapo
\fi
{\it\elabstract}
\bigskip
\makeatletter\@starttoc{toc}\makeatother % toc without new page
\bigskip

\pagestyle{main} % after style

   \section[{Préface}]{Préface}\phantomsection
\label{p1}\renewcommand{\leftmark}{Préface}

\noindent  \emph{Cette brochure est formée d’articles écrits à des moments divers des deux dernières années et demie. Pour parler plus précisément : de l’offensive de la coalition fasciste-bonapartiste-royaliste du 6 février 1934 à la grandiose grève de masse de fin mai début juin 1936. Quelle grandiose amplitude politique ! Les chefs du Front populaire sont, assurément, enclins à attribuer le mérite du déplacement qui s’est produit vers la gauche à la clairvoyance et à la sagesse de leur politique. Mais il n’en est rien. Le cartel tripartite s’est révélé être un facteur de troisième ordre dans la marche de la crise politique. Communistes, socialistes et radicaux n’ont rien prévu ni rien dirigé : ils ont subi les événements. Le coup, inattendu pour eux, du 6 février 1934 les força, à l’encontre de leurs mots d’ordre et de leurs doctrines de la veille, à chercher leur salut dans l’alliance de l’un avec l’autre. La grève de mai-juin 1936, aussi inattendue, porte à ce rassemblement parlementaire un coup mortel. Ce qui à un coup d’œil superficiel peut sembler être l’apogée du Front populaire est en réalité son agonie.} \par
 \emph{Comme les différentes parties de cette brochure datent de divers moments, reflétant les diverses étapes de la crise que traverse la France,  le lecteur trouvera dans ces pages d’inévitables répétitions. Les supprimer aurait signifié détruire complètement la construction de chacune des parties et, ce qui est beaucoup plus important, enlever à tout le travail sa dynamique, qui reflète la dynamique des événements eux-mêmes. L’auteur a préféré conserver les répétitions. Elles peuvent même ne pas être sans utilité pour le lecteur. Nous vivons à une époque de liquidation générale du marxisme dans les sommets officiels du mouvement ouvrier. Les préjugés les plus vulgaires servent actuellement de doctrine officielle aux chefs politiques et syndicaux de la classe ouvrière française. Au contraire, la voix du réalisme révolutionnaire résonne dans cette acoustique artificielle comme la voix du « sectarisme ». Avec d’autant plus d’insistance faut-il \emph{répéter} et \emph{répéter} les vérités fondamentales de la politique marxiste devant l’auditoire des ouvriers avancés.} \par
 \emph{Dans telles ou telles affirmations particulières de l’auteur le lecteur trouvera, peut-être, certaines contradictions. Nous ne les avons pas écartées. En fait ces prétendues « contradictions » viennent seulement du fait qu’ont été soulignés divers côtés d’un seul et même phénomène à diverses étapes du processus. Dans l’ensemble la brochure, nous semble-t-il, a supporté l’épreuve des événements et, peut-être, se trouvera capable de faciliter leur compréhension.} \par
 \emph{Les journées de grande grève auront, sans aucun doute, le mérite de renouveler l’atmosphère stagnante, putride des organisations ouvrières, en la purifiant des miasmes du réformisme et du  patriotisme, des genres « socialiste », « communiste » et « syndicaliste ». Bien entendu, cela ne se produira pas d’un seul coup ni de soi-même. Nous allons au-devant d’une opiniâtre lutte idéologique sur la base d’une âpre lutte des classes. Mais la marche à venir de la crise montrera que seul le marxisme permet de se retrouver à temps dans la texture des événements et de prévoir leur développement futur.} \par
 \emph{Les journées de février 1934 ont marqué la première offensive sérieuse de la contre-révolution unie. Les journées de mai-juin 1936 sont le signe de la première vague puissante de la révolution prolétarienne. Ces deux jalons marquent deux voies possibles : l’italienne et la russe. La démocratie parlementaire, au nom de laquelle agit le gouvernement Blum, sera réduite en poudre entre deux puissantes meules. Quels que soient les prochaines étapes, les combinaisons et les regroupements transitoires, les flux et les reflux momentanés, les épisodes tactiques, dès maintenant il n’y a plus à choisir qu’entre le fascisme et la révolution prolétarienne. Tel est le sens du présent travail.} \par
Le 10 juin 1936.\par

\byline{L. T{\scshape rotsky}.}
  \section[{Où va la france ? (Fin octobre 1934)}]{Où va la france ? \\
(Fin octobre 1934)}\phantomsection
\label{p2}\renewcommand{\leftmark}{Où va la france ? \\
(Fin octobre 1934)}

\noindent Dans ces pages, nous voulons expliquer aux ouvriers avancés quel sort attend la France dans les années qui viennent. Pour nous, la France, ce n’est ni la Bourse, ni les banques, ni les trusts, ni le gouvernement, ni l’état-major, ni l’Eglise, — tous ceux-là, ce sont les oppresseurs de la France, — mais c’est la classe ouvrière et les paysans exploités.\par
\subsection[{L’effondrement de la démocratie bourgeoise}]{L’effondrement de la démocratie bourgeoise}
\noindent Après la guerre se produisit une série de révolutions, qui remportèrent de brillantes victoires : en Russie, en Allemagne, en Autriche-Hongrie, plus tard en Espagne. Mais c’est seulement en Russie que le prolétariat a pris pleinement le pouvoir en main, a exproprié ses exploiteurs et a su, grâce à cela, créer et maintenir un Etat ouvrier. Dans tous les autres cas le prolétariat, malgré la victoire, s’est arrêté, par la faute de sa direction, à mi-chemin. Le résultat en fut que le pouvoir s’est échappé de ses mains et, se déplaçant de gauche à droite, est devenu la proie du fascisme. Dans une série d’autres pays le pouvoir est tombé dans les mains d’une dictature militaire. Dans aucun de ces pays le parlement ne s’est trouvé avoir la force de concilier les contradictions des classes et d’assurer  une marche pacifique de l’évolution. Le conflit s’est résolu les armes à la main.\par
Certes, en France on a longtemps pensé que le fascisme n’avait rien à voir avec ce pays. Car la France est une république, toutes les questions y sont tranchées par le peuple souverain au moyen du suffrage universel. Mais le 6 février quelques milliers de fascistes et de royalistes, armés de revolvers, de matraques et de rasoirs, ont imposé au pays le gouvernement réactionnaire Doumergue, sous la protection duquel les bandes fascistes continuent à croître et à s’armer. Que nous prépare demain ?\par
Certes, en France, comme dans certains autres pays d’Europe (Angleterre, Belgique, Hollande, Suisse, Pays scandinaves), il existe encore un Parlement, des élections, des libertés démocratiques ou leurs débris. Mais dans tous ces pays la lutte des classes s’exacerbe dans le même sens qu’elle s’est développée auparavant en Italie et en Allemagne. Qui se console avec la phrase : « La France n’est pas l’Allemagne », est un imbécile sans espoir. Dans tous les pays agissent maintenant les mêmes lois ; ce sont celles de la décadence capitaliste. Si les moyens de production continuent à rester dans les mains d’un petit nombre de capitalistes, il n’y a pas de salut pour la société. Elle est condamnée à aller de crise en crise, de misère en misère, de mal en pis. Dans les divers pays les conséquences de la décrépitude et de la décadence du capitalisme s’expriment sous des formes diverses et se développent à des rythmes inégaux. Mais le fond du processus est le même partout. La bourgeoisie a mené sa société à une banqueroute complète. Elle n’est capable d’assurer au peuple ni le pain ni la paix. C’est précisément pourquoi elle ne peut supporter plus longtemps l’ordre démocratique. Elle est contrainte d’écraser les ouvriers à l’aide de la violence physique. Mais on ne peut pas venir  à bout du mécontentement des ouvriers et des paysans par la police seule. Faire marcher l’armée contre le peuple, c’est trop souvent impossible : elle commence par se décomposer et cela s’achève par le passage d’une grande partie des soldats du côté du peuple. C’est pourquoi le grand capital est contraint de créer des bandes armées particulières, spécialement dressées contre les ouvriers, comme certaines races de chien sont dressées contre le gibier. La signification historique du \emph{fascisme} est d’écraser la classe ouvrière, de détruire ses organisations, d’étouffer la liberté politique à l’heure où les capitalistes s’avèrent déjà incapables de diriger et de dominer à l’aide de la mécanique démocratique.\par
Son matériel humain, le fascisme le trouve surtout au sein de la petite-bourgeoisie. Celle-ci est finalement ruinée par le grand capital. Avec la structure sociale actuelle, il n’y a pas de salut pour elle. Mais elle ne connaît pas d’autre issue. Son mécontentement, sa révolte, son désespoir, les fascistes les détournent du grand capital et les dirigent contre les ouvriers. On peut dire du fascisme que c’est une opération de luxation des cerveaux de la petite-bourgeoisie dans les intérêts de ses pires ennemis. Ainsi, le grand capital ruine d’abord les classes moyennes, ensuite, à l’aide de ses agents mercenaires, les démagogues fascistes, dirigent contre le prolétariat la petite-bourgeoisie tombée dans le désespoir. Ce n’est que par de tels procédés de brigand que le régime bourgeois est encore capable de se maintenir. Jusqu’à quand ? Jusqu’à ce qu’il soit renversé par la révolution prolétarienne.
\subsection[{Le commencement du bonapartisme en France}]{Le commencement du bonapartisme en France}
\noindent En France le mouvement de la démocratie vers le fascisme n’en est encore qu’à sa première étape. Le Parlement existe, mais il n’a plus les pouvoirs d’autrefois et il  ne les reprendra jamais plus. Morte de peur, la majorité du Parlement a, après le 6 février, appelé au pouvoir Doumergue, le sauveur, l’arbitre. Son gouvernement se tient au-dessus du Parlement. Il s’appuie non pas sur la majorité « démocratiquement » élue, mais directement et immédiatement sur l’appareil bureaucratique, sur la police et l’armée. C’est précisément pourquoi Doumergue ne peut souffrir aucune liberté pour les fonctionnaires et, en général, pour les serviteurs de l’Etat. Il lui faut un appareil bureaucratique docile et discipliné, au sommet duquel il puisse se tenir sans danger de tomber. La majorité parlementaire est contrainte de s’incliner devant Doumergue dans sa frayeur devant les fascistes et devant le « front commun ». Actuellement on écrit beaucoup sur la « réforme » prochaine de la Constitution, sur le droit de dissolution de la Chambre des députés, etc. Toutes ces questions n’ont qu’un intérêt juridique. Dans le sens politique la question est déjà résolue. La réforme s’est accomplie sans voyage à Versailles. L’apparition sur l’arène de bandes fascistes armées a donné la possibilité aux agents du grand capital de s’élever au-dessus du Parlement. C’est en cela que consiste maintenant l’essence de la Constitution française. Tout le reste n’est qu’illusions, phrases ou tromperie consciente.\par
Le rôle actuel de Doumergue (comme de ses successeurs possibles, dans le genre du maréchal Pétain ou de Tardieu) n’est pas une chose nouvelle. C’est un rôle analogue à celui que, dans d’autres conditions, jouèrent Napoléon I\textsuperscript{er} et Napoléon III. L’essence du bonapartisme consiste en ceci ; s’appuyant sur la lutte de deux camps, il « sauve », à l’aide d’une dictature bureaucratico-militaire, la « nation ». Napoléon I\textsuperscript{èr} représente le bonapartisme de la jeunesse impétueuse de la société bourgeoise. Le bonapartisme de Napoléon III, c’est celui du moment où, sur le crâne de la bourgeoisie, apparaît déjà  la calvitie. En la personne de Doumergue, nous rencontrons le bonapartisme sénile du déclin capitaliste. Le gouvernement Doumergue, c’est le premier degré du passage du parlementarisme au bonapartisme. Pour maintenir l’équilibre de Doumergue, il faut à celui-ci à sa droite les bandes fascistes et autres qui l’ont porté au pouvoir. Réclamer de lui qu’il dissolve — non pas sur le papier, mais dans la réalité — les Jeunesses patriotes, les Croix de feu, les Camelots du roi, etc., c’est réclamer qu’il coupe la branche sur laquelle il se tient. Des oscillations temporaires de tel ou tel côté sont, bien entendu, possibles. Ainsi, une offensive prématurée du fascisme pourrait provoquer dans les sommets gouvernementaux quelque écart « à gauche ». Doumergue ferait place pour un moment non pas à Tardieu, mais à Herriot. Mais, premièrement, on n’a jamais dit que les fascistes feraient une tentative prématurée de coup d’Etat. Deuxièmement, un écart temporaire à gauche dans les sommets ne changerait pas la direction générale du développement, il ne ferait plutôt qu’ajourner un peu le dénouement. Pour revenir en arrière, à la démocratie pacifique, il n’y a pas de voie. Le développement conduit inévitablement, infailliblement, à un conflit entre le prolétariat et le fascisme.
\subsection[{Le bonapartisme sera-t-il de longue durée ?}]{Le bonapartisme sera-t-il de longue durée ?}
\noindent Combien de temps peut se maintenir l’actuel régime bonapartiste de transition ? Ou, autrement dit : combien de temps reste-t-il au prolétariat pour se préparer au combat décisif ? A cette question il est impossible, naturellement, de répondre exactement. Mais on peut cependant établir quelques données pour apprécier la vitesse du développement de tout le processus. L’élément le plus important pour pouvoir juger, c’est la question du sort à venir du parti radical.\par
 Par son apparition, le bonapartisme actuel est lié, comme on l’a dit, à un commencement de guerre civile entre les camps politiques extrêmes. Son principal appui matériel, il le trouve dans la police et dans l’armée. Mais il a aussi un appui à gauche : c’est le parti radical-socialiste. La base de ce parti de masse est constituée par la petite-bourgeoisie des villes et des campagnes. Les sommets du parti sont formés par les agents « démocratiques » de la grande bourgeoisie, qui ont donné de loin en loin au peuple des petites réformes et le plus souvent des phrases démocratiques, l’ont sauvé chaque jour (en paroles) de la réaction et du cléricalisme, mais dans toutes les questions importantes ont fait la politique du grand capital. Sous la menace du fascisme, et encore plus du prolétariat, les radicaux-socialistes se sont trouvés contraints de passer du camp de la « démocratie » parlementaire dans le camp du bonapartisme. Comme le chameau sous le fouet du chamelier, le radicalisme s’est mis sur ses quatre genoux, pour permettre à la réaction capitaliste de s’asseoir entre ses bosses. Sans le soutien politique des radicaux, le gouvernement Doumergue serait, au moment présent, encore impossible.\par
Si l’on compare l’évolution politique de la France à celle de l’Allemagne, le gouvernement Doumergue et ses successeurs possibles correspondent aux gouvernements Brüning, Papen, Schleicher, qui comblèrent l’intervalle entre la démocratie de Weimar et Hitler. Il y a, pourtant, une différence qui, politiquement, peut prendre une importance énorme. Le bonapartisme allemand est entré en scène quand les partis démocratiques avaient fondu, alors que les nazis croissaient avec une force prodigieuse. Les trois gouvernements « bonapartistes » d’Allemagne, ayant un très faible appui politique propre, se trouvaient en équilibre sur une corde tendue au-dessus de l’abîme entre les deux camps hostiles : le prolétariat et le  fascisme. Ces trois gouvernements tombèrent rapidement. Le camp du prolétariat était alors scindé, non préparé pour la lutte, désorienté et trahi par les chefs. Les nazis purent prendre le pouvoir presque sans combat.\par
Le fascisme français ne représente pas encore maintenant une force de masse. Par contre, le bonapartisme a un appui, certes pas très sûr et pas très stable, mais de masse, dans la personne des radicaux. Entre ces deux faits existe un lien interne. Par le caractère social de son appui, le radicalisme est un parti de la petite-bourgeoisie. Et le fascisme ne peut devenir une force de masse qu’en conquérant la petite-bourgeoisie. En d’autres termes : \emph{en France, le fascisme peut se développer avant tout sur le compte des radicaux}. Ce processus se produit déjà actuellement, mais il se trouve encore à son premier stade.
\subsection[{Le rôle du parti radical}]{Le rôle du parti radical}
\noindent Les dernières élections cantonales ont donné les résultats qu’on pouvait et devait en attendre : les flancs, c’est-à-dire les réactionnaires et le bloc ouvrier, ont gagné, et le centre, c’est-à-dire les radicaux, perdu. Mais gains et pertes sont encore infimes. S’il s’était agi d’élections parlementaires, ces phénomènes auraient pris, sans aucun doute, des dimensions plus considérables. Les déplacements qui se sont marqués ont, pour nous, de l’importance non pas en eux-mêmes, mais seulement en tant que symptômes de changements dans la conscience des masses. Ils montrent que le centre petit-bourgeois a déjà commencé à fondre en faveur des deux camps extrêmes. Cela veut dire que les restes du régime parlementaire vont être rongés de plus en plus ; les camps extrêmes vont croître ; les heurts entre eux approchent. Il n’est pas difficile de comprendre que ce processus est absolument inévitable.\par
 Le parti radical est le parti à l’aide duquel la grande bourgeoisie maintenait les espoirs de la petite-bourgeoisie en une amélioration progressive et pacifique de sa situation. Ce rôle des radicaux ne fut possible qu’aussi longtemps que la situation économique de la petite-bourgeoisie restait supportable, tolérable, aussi longtemps qu’elle ne subissait pas une ruine massive, qu’elle gardait espoir en l’avenir. Certes, le programme des radicaux est toujours resté un simple morceau de papier. Les radicaux n’ont accompli aucune réforme sociale sérieuse en faveur des travailleurs et ne pouvaient en accomplir, — cela ne leur eût pas été permis par la grande bourgeoisie, dans les mains de laquelle sont tous les véritables leviers du pouvoir : les banques et la Bourse, la grande presse, les hauts fonctionnaires, la diplomatie, l’état-major. Mais ce sont quelques petites aumônes, surtout dans le cadre de la province, qu’obtenaient les radicaux de temps à autre, en faveur de leur clientèle, maintenant par là les illusions des masses populaires. Ainsi en allait-il jusqu’à la dernière crise. Actuellement pour le paysan le plus arriéré il devient clair qu’il s’agit non pas d’une crise passagère ordinaire, comme il y en eut pas mal avant la guerre, mais d’une crise de tout le système social. Il faut des mesures hardies et décisives. Lesquelles ? Le paysan ne le sait pas. Personne ne le lui a dit, comme il l’eût fallu.\par
Le capitalisme a porté les moyens de production à un tel niveau qu’ils se sont trouvés paralysés par la misère des masses populaires, ruinées par le même capitalisme. Par cela même tout le système est entré dans une période de décadence, de décomposition, de pourriture. Le capitalisme non seulement ne peut pas donner aux travailleurs de nouvelles réformes sociales, ni même seulement de petites aumônes, il est contraint de reprendre même les anciennes. Toute l’Europe est entrée dans une époque de contre-réformes économiques et politiques. La politique de spoliation  et d’étouffement des masses est provoquée non pas par les caprices de la réaction, mais par la décomposition du système capitaliste. C’est là le fait fondamental, qui doit être assimilé par chaque ouvrier, s’il ne veut pas qu’on le dupe avec des phrases creuses. C’est précisément pourquoi les partis réformistes démocratiques se décomposent et perdent leurs forces l’un après l’autre dans toute l’Europe. C’est le même sort qui attend également les radicaux français. Seuls des gens sans cervelle peuvent penser que la capitulation de Daladier ou la servilité d’Herriot devant la pire réaction sont le résultat de causes fortuites, temporaires ou du manque de caractère de ces deux lamentables chefs. Non ! Les grands phénomènes politiques doivent toujours avoir de profondes causes sociales. La décadence des partis démocratiques est un phénomène universel qui a ses raisons dans la décadence du capitalisme même. La grande bourgeoisie dit aux radicaux : « Maintenant, ce n’est plus le moment de plaisanter ! Si vous ne cessez pas de faire des coquetteries aux socialistes et de flirter avec le peuple en lui promettant monts et merveilles, alors j’appelle les fascistes. Comprenez bien que le 6 février ne fut qu’un premier avertissement ! » Après quoi le chameau radical se met sur ses quatre genoux. Il ne lui reste rien d’autre à faire.\par
Mais le radicalisme ne trouvera pas son salut dans cette voie. Liant, aux yeux de tout le peuple, son sort au sort de la réaction, il abrège inévitablement sa fin ! La perte de voix et de mandats aux élections cantonales n’est qu’un commencement. Puis le processus d’effondrement du parti radical ira de plus en plus. vite. Toute la question est de savoir en faveur de qui, de la révolution prolétarienne ou du fascisme, se fera cet effondrement inévitable, irrésistible.\par
Qui présentera le premier, le plus largement, le plus hardiment aux classes moyennes, le programme le plus  convaincant, et — c’est là le plus important — conquerra leur confiance, en leur montrant en paroles et en fait qu’il est capable de briser tous les obstacles sur la voie d’un avenir meilleur : le socialisme révolutionnaire ou la réaction fasciste ? De cette question dépend le sort de la France, pour de nombreuses années. Non seulement de la France, mais de toute l’Europe. Non seulement de l’Europe, mais du monde entier.
\subsection[{Les « classes moyennes », le parti radical et le fascisme}]{Les « classes moyennes », le parti radical et le fascisme}
\noindent Depuis le moment de la victoire des nazis en Allemagne, dans les partis et les groupes de « gauche » on fait beaucoup de discours sur la nécessité de se tenir près des « classes moyennes » pour barrer la route au fascisme. La fraction Renaudel et C\textsuperscript{ie} s’est séparée du Parti socialiste avec le but spécial de se rapprocher des radicaux. Mais à l’heure même où Renaudel, qui vit sur les idées de 1848, tendait les deux mains à Herriot, ce dernier avait les deux mains prises : l’une par Tardieu, l’autre par Louis Marin.\par
De là, pourtant, il ne s’ensuit pas du tout que la classe ouvrière puisse tourner le dos à la petite-bourgeoisie, en la laissant à son malheur. Oh, non ! Se rapprocher des paysans et des petites gens des villes, les attirer de notre côté, c’est la condition nécessaire du succès de la lutte contre le fascisme, sans même parler de la conquête du pouvoir. Il faut seulement poser d’une façon juste le problème. Mais pour cela, il faut comprendre clairement quelle est la nature des « classes moyennes ». Rien n’est plus dangereux en politique, surtout dans une période critique, que de répéter des formules générales, sans examiner quel contenu social elles recouvrent.\par
La société contemporaine se compose de trois classes : la  grande bourgeoisie, le prolétariat et les « classes moyennes », ou petite bourgeoisie. Les relations entre ces trois classes déterminent en fin de compte la situation politique dans le pays. Les classes fondamentales de la société sont la grande bourgeoisie et le prolétariat. Seules ces deux classes peuvent avoir une politique indépendante, claire et conséquente. La petite bourgeoisie se distingue par sa dépendance économique et son hétérogénéité sociale. Sa couche supérieure touche immédiatement la grande bourgeoisie. La couche inférieure se fond avec le prolétariat et tombe même à l’état de lumpen-prolétariat. Conformément à sa situation économique, la petite-bourgeoisie ne peut avoir de politique indépendante. Elle oscille toujours entre les capitalistes et les ouvriers. Sa propre couche supérieure la pousse à droite ; ses couches inférieures, opprimées et exploitées, sont capables, dans certaines conditions, de tourner brusquement à gauche. C’est par ces relations contradictoires des différentes couches des « classes moyennes » qu’a toujours été déterminée la politique confuse et absolument inconsistante des radicaux, leurs hésitations entre le cartel avec les socialistes, pour calmer la base, et le bloc national avec la réaction capitaliste, pour sauver la bourgeoisie. La décomposition définitive du radicalisme commence au moment où la grande bourgeoisie, elle-même dans l’impasse, ne lui permet plus d’osciller. La petite-bourgeoisie, en la personne des masses ruinées des villes et des campagnes, commence à perdre patience. Elle prend une attitude de plus en plus hostile envers sa propre couche supérieure ; elle se convainc en fait de l’inconsistance et de la perfidie de sa direction politique. Le paysan pauvre, l’artisan, le petit commerçant se convainquent en fait qu’un abîme les sépare de tous ces maires, de tous ces avocats, de tous ces arrivistes politiques, dans le genre d’Herriot, Daladier, Chautemps et C\textsuperscript{ie}, qui,  par leur mode de vie et par leurs conceptions, sont de grands bourgeois. C’est précisément cette désillusion de la petite-bourgeoisie, son impatience, son désespoir que le fascisme exploite. Ses agitateurs stigmatisent et maudissent la démocratie parlementaire qui épaule les carriéristes et les staviskrates, mais ne donne rien aux petits travailleurs. Eux, ces démagogues, brandissent le poing à l’adresse des banquiers, des gros commerçants, des capitalistes. Ces paroles et ces gestes répondent pleinement aux sentiments des petits propriétaires, tombés dans une situation sans issue. Les fascistes montrent de l’audace, descendent dans la rue, s’attaquent à la police, tentent par la force de chasser le Parlement. Cela en impose au petit bourgeois tombé dans le désespoir. Il se dit : « Les radicaux, parmi lesquels il y a trop de coquins, se sont vendus définitivement aux banquiers ; les socialistes promettent depuis longtemps d’anéantir l’exploitation, mais ils ne passent jamais des paroles aux actes ; les communistes, on ne peut rien y comprendre ; aujourd’hui c’est une chose, demain c’en est une autre ; il faut voir si les fascistes ne peuvent pas apporter le salut ».
\subsection[{Le passage des classes moyennes dans le camp du fascisme est-il inévitable ?}]{Le passage des classes moyennes dans le camp du fascisme est-il inévitable ?}
\noindent Renaudel, Frossard et leurs semblables s’imaginent que la petite bourgeoisie est attachée avant tout à la démocratie et que c’est précisément pourquoi il faut se joindre aux radicaux. Quelle monstrueuse aberration ! La démocratie n’est qu’une forme politique. La petite bourgeoisie ne se soucie pas de la coquille de la noix, mais de son amande. Elle cherche à se sauver de la misère et de la ruine. Que la démocratie s’avère impuissante — et au diable la démocratie ! Ainsi raisonne ou  sent chaque petit bourgeois. Dans la révolte grandissante des couches inférieures de la petite bourgeoisie contre ses propres couches supérieures, « instruites », municipales, cantonales et parlementaires, se trouve la principale source sociale et politique du fascisme. A cela il faut ajouter la haine de la jeunesse intellectuelle, écrasée par la crise, pour les avocats, les professeurs, les députés et les ministres parvenus, Ici aussi, par conséquent, les intellectuels petits bourgeois inférieurs se rebellent contre leurs sommets.\par
Cela signifie-t-il que le passage de la petite bourgeoisie sur la voie du fascisme soit inévitable, inéluctable ? Non, une telle conclusion serait du fatalisme honteux. Ce qui est réellement inévitable, inéluctable, c’est la fin du radicalisme et de tous les groupements politiques qui lient leur sort au sien. Dans les conditions de la décadence capitaliste, il ne reste plus de place pour un parti de réformes démocratiques et de progrès « pacifique ». Quelle que soit la voie par laquelle passe le développement à venir de la France, le radicalisme disparaîtra de toute façon de la scène, rejeté et honni par la petite bourgeoisie, qu’il a définitivement trahie. Que notre prédiction réponde à la réalité, tout ouvrier conscient s’en convaincra dès maintenant sur la base des faits, et de l’expérience de chaque jour. De nouvelles élections apporteront aux radicaux des défaites. Des couches vont se séparer d’eux les unes après les autres, les masses populaires en bas, les groupes de carriéristes effrayés en haut. Des départs, des scissions, des trahisons vont suivre d’une façon ininterrompue. Aucune manœuvre et aucun bloc ne sauveront le parti radical. Il entraînera avec lui dans l’abîme le « parti » de Renaudel-Déat et C\textsuperscript{ie}. La fin du parti radical est le résultat inévitable du fait que la société bourgeoise ne peut plus venir à bout de ses difficultés à l’aide de méthodes soi-disant démocratiques.  La scission entre la base de la petite bourgeoisie et ses sommets est inévitable.\par
Mais cela ne signifie pas du tout que les masses qui suivent le radicalisme doivent infailliblement reporter leurs espoirs sur le fascisme. Certes, la partie la plus démoralisée, la plus déclassée et la plus avide de la jeunesse des classes moyennes a déjà porté son choix dans cette direction. C’est à ce réservoir que se forment surtout les bandes fascistes. Mais les lourdes masses petites-bourgeoises des villes et des campagnes n’ont pas encore fait leur choix. Elles hésitent devant une grande décision. C’est précisément parce qu’elles hésitent qu’elles continuent encore, mais déjà sans confiance, à voter pour les radicaux. Cette situation d’hésitation et d’irrésolution ne durera pourtant pas des années, mais des mois. Le développement politique va prendre dans la période qui vient un rythme fébrile. La petite bourgeoisie ne repoussera la démagogie du fascisme que si elle met sa foi dans la réalité d’une autre voie. Mais l’autre voie, c’est la voie de la révolution prolétarienne.
\subsection[{Est-il vrai que la petite bourgeoisie craint la révolution ?}]{Est-il vrai que la petite bourgeoisie craint la révolution ?}
\noindent Les routiniers parlementaires, qui croient être des connaisseurs du peuple, aiment à répéter : « Il ne faut pas effrayer les classes moyennes avec la révolution, elles n’aiment pas les extrêmes ». Sous cette forme générale, cette affirmation est absolument fausse. Naturellement, le petit propriétaire tient à l’ordre, tant que ses affaires vont bien et aussi longtemps qu’il espère que demain elles iront encore mieux. Mais quand cet espoir est perdu, il entre facilement en rage et est prêt à se livrer aux mesures les plus extrêmes. Sinon, comment aurait-il pu renverser l’Etat démocratique et amener le fascisme au  pouvoir en Italie et en Allemagne ? Les petites gens désespérées voient avant tout dans le fascisme une force combative contre le grand capital et croient qu’à la différence des partis ouvriers qui travaillent seulement de la langue, le fascisme se servira des poings pour établir plus de « justice ». Le paysan et l’artisan sont, à leur manière, des réalistes : ils comprennent qu’on ne pourra pas se passer des poings. Il est faux, trois fois faux d’affirmer que la petite bourgeoisie actuelle ne va pas aux partis ouvriers parce qu’elle craint les « mesures extrêmes ». Bien au contraire. La couche inférieure de la petite bourgeoisie, ses grandes masses ne voient dans les partis ouvriers que des machines parlementaires, ne croient pas à la force des partis ouvriers, ne croient pas qu’ils soient capables de lutter, qu’ils soient prêts à mener cette fois la lutte jusqu’au bout. Et s’il en est ainsi, est-ce la peine de remplacer le radicalisme par ses confrères parlementaires de gauche ? Voilà comment raisonne ou sent le propriétaire à demi exproprié, ruiné et révolté. Sans la compréhension de cette psychologie des paysans, des artisans, des employés, des petits fonctionnaires, etc. — psychologie qui découle de la crise sociale — il est impossible d’élaborer une politique juste.\par
La petite bourgeoisie est économiquement dépendante et politiquement morcelée. C’est pourquoi elle ne peut avoir une politique propre. Elle a besoin d’un « chef », qui lui inspire confiance. Ce chef individuel ou collectif, c’est-à-dire un personnage ou un parti, peut lui être donné par l’une ou l’autre des classes fondamentales, soit par la grande bourgeoisie, soit par le prolétariat. Le fascisme unit et arme les masses disséminées ; d’une « poussière humaine » — selon notre expression — il fait des détachements de combat. Il donne ainsi à la petite bourgeoisie l’illusion d’être une force indépendante. Elle commence à s’imaginer qu’elle commandera réellement à  l’Etat. Rien d’étonnant à ce que ces espoirs et ces illusions lui montent à la tête.\par
Mais la petite bourgeoisie peut trouver aussi un chef dans la personne du prolétariat. Elle l’a montré en Russie, partiellement en Espagne. Elle y tendit en Italie, en Allemagne et en Autriche. Mais les partis du prolétariat ne s’y montrèrent pas à la hauteur de leur tâche historique. Pour amener à lui la petite bourgeoisie, le prolétariat doit conquérir sa confiance. Et, pour cela, il doit avoir lui-même confiance en sa force. Il lui faut avoir un clair programme d’action et être prêt à lutter pour le pouvoir par tous les moyens possibles. Soudé par son parti révolutionnaire pour une lutte décisive et impitoyable, le prolétariat dit aux paysans et aux petites gens des villes : « Je lutte pour le pouvoir ; voici mon programme ; je suis prêt à m’entendre avec vous pour des changements dans ce programme ; je n’emploierai la force que contre le grand capital et ses laquais ; mais avec vous, travailleurs, je veux conclure une alliance sur la base d’un programme donné ». Un tel langage, le paysan le comprendra. Il faut seulement qu’il ait confiance dans la capacité du prolétariat de s’emparer du pouvoir. Or, pour cela, il faut épurer le front unique de toute équivoque, de toute indécision, des phrases creuses ; il faut comprendre la situation et se mettre sérieusement sur la voie de la lutte révolutionnaire.
\subsection[{Une alliance avec les radicaux serait une alliance contre les classes moyennes}]{Une alliance avec les radicaux serait une alliance contre les classes moyennes}
\noindent Renaudel, Frossard et leurs semblables s’imaginent sérieusement qu’une alliance avec les radicaux, c’est une alliance avec les « classes moyennes » et, par conséquent, une barrière contre le fascisme. Ces gens ne voient rien d’autre que les ombres parlementaires. Ils ignorent l’évolution  réelle des masses et se tournent vers le parti radical, qui se survit, alors que celui-ci leur a entre temps tourné le derrière. Ils pensent qu’à une époque de grande crise sociale une alliance des classes mises en mouvement peut être remplacée par un bloc avec une clique parlementaire compromise et vouée à la perte. Une alliance véritable du prolétariat et des classes moyennes, c’est une question non pas de statique parlementaire, mais de dynamique révolutionnaire. Cette alliance, il faut la créer, la forger dans la lutte.\par
Tout le fond de la situation politique actuelle est dans le fait que la petite bourgeoisie désespérée commence à se débarrasser du joug de la discipline parlementaire et de la tutelle de la clique « radicale » conservatrice, qui a toujours trompé le peuple et l’a maintenant définitivement trahi. Se lier dans cette situation aux radicaux signifie se condamner soi-même au mépris des masses et pousser la petite bourgeoisie dans les bras du fascisme, comme le seul sauveur.\par
Le parti ouvrier doit s’occuper non pas d’une tentative sans espoir de sauver le parti des banqueroutiers ; il doit, au contraire, de toutes ses forces accélérer le processus d’affranchissement des masses de l’emprise radicale. Plus il mettra de zèle et de hardiesse à accomplir ce travail, et plus il préparera véritablement et rapidement l’alliance de la classe ouvrière avec la petite bourgeoisie. Il faut prendre les classes dans leur mouvement. Il faut se régler sur leur tête, et non sur leur queue. L’histoire travaille maintenant rapidement. Malheur à qui reste sur place !\par
Quand Frossard dénie au Parti socialiste le droit de démasquer, d’affaiblir, de décomposer le parti radical, il agit en radical conservateur, mais non en socialiste. Seul a droit d’existence historique le parti qui croit en son programme et s’efforce de rassembler tout le peuple sous son drapeau. Sinon, ce n’est pas un parti historique, mais une  coterie parlementaire, une clique de carriéristes. C’est non seulement le droit, mais le devoir élémentaire du parti du prolétariat d’affranchir les masses travailleuses de l’influence funeste de la bourgeoisie. Cette tâche historique prend actuellement une acuité particulière, car les radicaux s’efforcent plus que jamais de couvrir le travail de la réaction, endorment et trompent le peuple et préparent ainsi la victoire du fascisme. Les radicaux de gauche ? Mais ils capitulent aussi fatalement devant Herriot qu’Herriot devant Tardieu.\par
Frossard veut espérer que l’alliance des socialistes avec les radicaux aboutira à un gouvernement de « gauche », qui dissoudra les organisations fascistes et sauvera la République. Il est difficile d’imaginer un amalgame plus monstrueux d’illusions démocratiques et de cynisme policier. Quand nous disons — nous en parlerons en détail plus loin — qu’il faut une milice du peuple, Frossard et ses semblables objectent : « Contre le fascisme, il faut lutter non pas avec des moyens physiques, mais idéologiques ». Quand nous disons : seule une mobilisation révolutionnaire hardie des masses, qui n’est possible que dans une lutte contre le radicalisme, est capable de miner le terrain sous les pieds du fascisme, les mêmes gens nous répliquent : « Non, seule peut nous sauver la police du gouvernement Daladier-Frossard ».\par
Pitoyable bredouillement ! Les radicaux ont eu le pouvoir et, s’ils ont consenti à le céder à Doumergue, ce n’est pas parce qu’il leur manquait l’aide de Frossard, mais parce qu’ils craignaient le fascisme, qu’ils craignaient la grande bourgeoisie qui les menaçait des rasoirs royalistes, qu’ils craignaient encore plus le prolétariat, qui commençait à se dresser contre le fascisme. Pour comble de scandale, c’est Frossard lui-même qui, effrayé de l’effroi des radicaux, conseilla à Daladier de capituler ! Si on admet pour un instant — hypothèse manifestement invraisemblable !  — que les radicaux aient consenti à rompre l’alliance avec Doumergue pour l’alliance avec Frossard, les bandes fascistes, cette fois avec la [{\corr collaboration}] directe de la police, seraient descendues trois fois plus nombreuses dans la rue, et les radicaux ensemble avec Frossard se seraient fourrés sous les tables ou se seraient cachés dans les réduits les plus secrets de leurs ministères.\par
Mais faisons encore une hypothèse fantastique : la police de Daladier-Frossard « désarme » les fascistes. Est-ce que cela résout la question ? Et qui désarmera la même police, qui, de la main droite, rendra aux fascistes ce qu’elle leur aurait pris de la main gauche ? La comédie du désarmement par la police n’aurait fait qu’accroître l’autorité des fascistes, en tant que combattants contre l’Etat capitaliste. Des coups contre les bandes fascistes ne peuvent être réels que dans la mesure où ces bandes sont en même temps isolées politiquement. Cependant l’hypothétique gouvernement Daladier-Frossard ne donnerait rien ni aux ouvriers ni aux masses petites bourgeoises, car il ne pourrait attenter aux fondements de la propriété privée. Et sans expropriation des banques, des grandes entreprises commerciales, des industries-clés, des transports, sans monopole du commerce extérieur et sans une série d’autres mesures profondes, il n’est nullement possible de venir en aide au paysan, à l’artisan, au petit commerçant. Par sa passivité, son impuissance, son mensonge, le gouvernement Daladier-Frossard provoquerait une tempête de révolte dans la petite bourgeoisie et la pousserait définitivement sur la voie du fascisme, si... si ce gouvernement était possible.\par
Il faut pourtant reconnaître que Frossard n’est pas seul. Le même jour (24 octobre) où le modéré Zyromski intervenait dans le \emph{Populaire} contre la tentative de Frossard de faire renaître le cartel, Cachin intervenait dans l’\emph{Humanité} pour défendre l’idée d’un bloc avec les radicaux-socialistes.  Lui, Cachin, saluait avec enthousiasme le fait que les radicaux s’étaient prononcés pour le « désarmement » des fascistes. Certes, les radicaux se sont prononcés pour le désarmement de tous, y compris des organisations ouvrières. Certes, dans les mains de l’Etat bonapartiste, une telle mesure serait dirigée surtout contre les ouvriers. Certes, les fascistes « désarmés » recevraient le lendemain le double d’armes, non sans l’aide de la police. Mais à quoi bon se faire de la peine avec de sombres réflexions ? Tout homme a besoin d’espoir. Et voilà Cachin qui va sur les traces de Wels et d’Otto Bauer, qui attendirent aussi en leur temps le salut d’un désarmement par la police de Brüning et de Dollfuss. Faisant un tournant de 180°, Cachin identifie les radicaux aux classes moyennes. Les paysans opprimés, il ne les voit qu’à travers le prisme du radicalisme. L’alliance avec les petits propriétaires travailleurs, il ne se la représente pas autrement que sous la forme d’un bloc avec les affairistes parlementaires qui commencent, enfin, à perdre la confiance des petits propriétaires. Au lieu de nourrir et d’attiser la révolte commençante du paysan et de l’artisan contre les exploiteurs « démocratiques » et de diriger cette révolte sur la voie d’une alliance avec le prolétariat, Cachin s’apprête à soutenir les banqueroutiers radicaux de l’autorité du « front commun » et de pousser ainsi la révolte des couches inférieures de la petite bourgeoisie dans la voie du fascisme.\par
La nonchalance théorique se venge toujours cruellement dans la politique révolutionnaire. L’ « antifascisme » comme le « fascisme », ce ne sont pas pour les stalinistes des conceptions concrètes, mais deux grands sacs vides où ils fourrent tout ce qui leur tombe sous la main. Doumergue est pour eux un fasciste, comme auparavant Daladier lui aussi fut pour eux un fasciste. En fait, Doumergue est un exploiteur capitaliste de l’aile fasciste de la petite bourgeoisie radicale. Actuellement ces deux systèmes se  combinent dans le régime bonapartiste. Doumergue aussi est à sa manière un « antifasciste », car il préfère une dictature « paisible », militaire et policière, du grand capital à une guerre civile avec une issue toujours incertaine. Par frayeur devant le fascisme et encore plus devant le prolétariat, l’ « antifasciste » Daladier s’est joint à Doumergue. Mais le régime de Doumergue est inconcevable sans l’existence des bandes fascistes. L’analyse marxiste élémentaire démontre ainsi la complète inconsistance de l’idée de l’alliance avec les radicaux contre le fascisme ! Les radicaux eux-mêmes prennent soin de montrer en fait combien fantastiques et réactionnaires sont les chimères politiques de Frossard et de Cachin.
\subsection[{La milice ouvrière et ses adversaires}]{La milice ouvrière et ses adversaires}
\noindent Pour lutter, il faut conserver et renforcer les instruments et les moyens de lutte ; les organisations, la presse, les réunions, etc. Tout cela, le fascisme le menace directement et immédiatement. Il est encore trop faible pour se mettre à la lutte directe pour le pouvoir ; mais il est assez fort pour tenter d’abattre les organisations ouvrières morceau par morceau, pour tremper dans ces attaques ses bandes, semer dans les rangs ouvriers l’accablement et le manque de confiance dans leur force. En outre, le fascisme trouve des auxiliaires inconscients dans la personne de tous ceux qui disent que la « lutte physique » est inadmissible et sans espoir et réclament de Doumergue le désarmement de ses gardes fascistes. Rien n’est si dangereux pour le prolétariat, surtout dans les conditions actuelles, que le poison sucré des faux espoirs. Rien n’accroît autant l’insolence des fascistes que le « pacifisme » mollasse des organisations ouvrières. Rien ne détruit autant la confiance des classes moyennes dans le  prolétariat que la passivité expectante, que l’absence de volonté de lutte.\par
Le \emph{Populaire} et surtout l’\emph{Humanité} écrivent chaque jour : « Le front unique, c’est une barrière contre le fascisme », « Le front unique ne permettra pas », « Les fascistes n’oseront pas », et ainsi de suite. Ce sont des phrases. Il faut dire carrément aux ouvriers, socialistes et communistes : « Ne permettez pas aux journalistes et aux orateurs superficiels et irresponsables de vous bercer avec des phrases. Il s’agit de vos têtes et de l’avenir du socialisme ». Ce n’est pas nous qui nions l’importance du front unique : Nous l’exigions alors que les chefs des deux partis étaient contre lui. Le front unique ouvre d’énormes possibilités. Mais rien de plus. En lui-même, le front unique ne décide rien. Seule la lutte des masses décide. Le front unique s’avérera une grande chose lorsque les détachements communistes viendront en aide aux détachements socialistes — et inversement — au cas d’une attaque des bandes fascistes contre le \emph{Populaire} ou l’\emph{Humanité.} Mais pour cela, les détachements de combat prolétariens doivent exister, s’éduquer, s’exercer, s’armer. Et s’il n’y a pas d’organisation de défense, c’est-à-dire de milice ouvrière, le \emph{Populaire} et l’\emph{Humanité} pourront écrire autant d’articles qu’ils voudront sur la toute-puissance du front unique, les deux journaux se trouveront sans défense devant la première attaque bien préparée des fascistes. Essayons de faire l’examen critique des « arguments » et des « théories » des adversaires de la milice ouvrière, qui sont très nombreux et très influents dans les deux partis ouvriers.\par
 — Il nous faut l’auto-défense de masse, et non la milice, nous dit-on souvent. Mais qu’est-ce que cette « auto-défense de masse » ? Sans organisation de combat ? Sans cadres spécialisés ? Sans armement ? Remettre aux masses non-organisées, non-préparées, laissées à elles- mêmes, la défense contre le fascisme, ce serait jouer un rôle incomparablement plus bas que celui de Ponce-Pilate. Nier le rôle de la milice, c’est nier le rôle de l’avant-garde. Alors pourquoi un parti ? Sans le soutien des masses la milice n’est rien. Mais sans détachements de combat organisés la masse la plus héroïque sera écrasée, morceau par morceau, par les bandes fascistes. Opposer la milice à l’autodéfense est absurde. \emph{La milice est l’organe de l’autodéfense}.\par
— Appeler à l’organisation de la milice, disent certains adversaires, certes peu sérieux et peu honnêtes, c’est de la « provocation ». Ce n’est pas un argument, mais une insulte. Si la nécessité de défendre les organisations ouvrières découle de toute la situation, comment peut-on donc ne pas appeler à la création de milices ? Peut-être veut-on nous dire que la création de milices « provoque » les attaques des fascistes et la répression du gouvernement ? Alors, c’est un argument absolument réactionnaire. Le libéralisme a toujours dit aux ouvriers que par leur lutte de classes ils « provoquent » la réaction. Les réformistes répétèrent cette accusation contre les marxistes ; les menchéviks contre les bolchéviks. Ces accusations se réduisent en fin de compte à cette pensée profonde que, si les opprimés ne remuaient pas, les oppresseurs ne seraient pas contraints de les battre. C’est la philosophie de Tolstoï et de Gandhi, mais aucunement celle de Marx et de Lénine. Si désormais l’\emph{Humanité} veut aussi développer la doctrine de la « non-résistance au mal par la violence », il lui faut prendre pour symbole non pas la faucille et le marteau, emblème de la révolution d’Octobre, mais la pieuse chèvre qui nourrit Gandhi de son lait.\par
— Mais l’armement des ouvriers n’est opportun que dans une situation révolutionnaire qui n’existe pas encore. Cet argument profond signifie que les ouvriers doivent se laisser battre jusqu’à ce que la situation devienne révolutionnaire.  Ceux qui prêchaient hier la « troisième période » ne veulent pas voir ce qui s’est produit devant leurs yeux. La question elle-même de l’armement n’a surgi pratiquement que parce la situation « pacifique », « normale », « démocratique » a fait place à une situation agitée, critique et instable, qui peut aussi bien se changer en situation révolutionnaire que contre-révolutionnaire. Cette alternative dépend avant tout de ceci : les ouvriers avancés se laisseront-ils battre impunément morceau par morceau ou bien à chaque coup répondront-ils par deux coups, élevant le courage des opprimés et les unissant autour d’eux ? Une situation révolutionnaire ne tombe pas du ciel. Elle se forme avec la participation active de la classe révolutionnaire et de son parti.\par
Les stalinistes français invoquent maintenant le fait que la milice n’a pas sauvé de la défaite le prolétariat allemand. Hier encore ils niaient toute défaite en Allemagne et affirmaient que la politique des stalinistes allemands avait été juste d’un bout à l’autre. Aujourd’hui ils voient tout le mal dans la milice ouvrière allemande \emph{(Rote Front)}. Ainsi, d’une faute ils tombent dans la faute opposée, non moins monstrueuse. La milice ne résout pas par elle-même la question. \emph{Il faut une politique juste.} Et la politique des stalinistes en Allemagne (« le social-fascisme, c’est l’ennemi principal », scission syndicale, flirt avec le nationalisme, putschisme) conduisit fatalement à l’isolement de l’avant-garde prolétarienne et à son effondrement. Avec une stratégie bonne à rien, aucune milice ne pouvait sauver la situation.\par
C’est une sottise de dire que par elle-même l’organisation de la milice mène sur la voie des aventures, provoque l’ennemi, remplace la lutte politique par la lutte physique, etc. Dans toutes ces phrases il n’y a rien d’autre que de la couardise politique. La milice, en tant que forte organisation de l’avant-garde, est en fait le moyen le plus  sûr contre les aventures, contre le terrorisme individuel, contre les sanglantes explosions spontanées. La milice est en même temps le seul moyen sérieux de réduire au minimum la guerre civile que le fascisme impose au prolétariat. Que seulement les ouvriers, malgré l’absence de « situation révolutionnaire », corrigent quelquefois à leur gré les « fils à papa » patriotes, et le recrutement de nouvelles bandes fascistes deviendra du coup incomparablement plus difficile.\par
Mais ici les stratèges, embrouillés dans leur propre raisonnement, sortent contre nous des arguments encore plus stupéfiants. Nous lisons textuellement : « Si nous répondons aux coups de revolver des bandes fascistes par d’autres coups de revolver, écrit l’\emph{Humanité} du 23 octobre, nous perdons de vue que le fascisme est le produit du régime capitaliste et qu’en luttant contre le fascisme, c’est tout le système que nous visons ». Il est difficile d’accumuler en quelques lignes plus de confusion et plus d’erreurs. Impossible de se défendre contre les fascistes, parce qu’ils représentent... « un produit du régime capitaliste ». Cela veut dire qu’il faut renoncer à toute lutte, car tous les maux sociaux contemporains représentent des « produits du système capitaliste ».\par
Quand les fascistes tuent un révolutionnaire ou incendient le siège d’un journal prolétarien, les ouvriers doivent constater philosophiquement : « Ah ! les meurtres et les incendies sont les produits du système capitaliste », et rentrer chez eux la conscience tranquille. A la théorie militante de Marx est substituée une prostration fataliste au seul avantage de l’ennemi de classe. La ruine de la petite bourgeoisie est, bien entendu, le produit du capitalisme. La croissance des bandes fascistes est, à son tour, le produit de la ruine de la petite bourgeoisie. Mais, d’un autre coté, l’accroissement de la misère et de la révolte du prolétariat est aussi le produit du capitalisme, et la milice,  à son tour, est le produit de l’exacerbation de la lutte des classes. Pourquoi donc pour les « marxistes » de l’\emph{Humanité} les bandes fascistes sont-elles le produit légitime du capitalisme, et la milice ouvrière, le produit illégitime des... trotskistes. Décidément, il est impossible d’y rien comprendre.\par
Il faut, nous dit-on, viser tout le « système ». Comment ? Par dessus la tête des êtres humains ? Pourtant, les fascistes dans les différents pays ont commencé par des coups de revolver et ont fini par la destruction de tout le « système » des organisations ouvrières. Comment donc arrêter l’offensive armée de l’ennemi, sinon par une défense armée, pour ensuite, à notre tour, passer à l’offensive ?\par
Certes, l’\emph{Humanité} admet maintenant en paroles la défense, mais seulement en tant qu’ « auto-défense de masse » : la milice est nuisible, parce que, voyez-vous, elle coupe des masses les détachements de combat. Mais pourquoi donc chez les fascistes existe-t-il des détachements armés indépendants, qui ne se coupent pas des masses réactionnaires, mais, au contraire, par leurs coups bien organisés élèvent le courage de la masse et renforcent son audace ? Ou peut-être la masse prolétarienne par ses qualités combatives serait-elle inférieure à la petite bourgeoisie déclassée ?\par
Embrouillée jusqu’au bout, l’\emph{Humanité} commence à hésiter : voilà que l’auto-défense de masse a besoin de créer ses « groupes d’auto-défense ». Au lieu de la milice répudiée on met des groupes spéciaux, des détachements. Il semble à première vue que la différence ne soit que dans le nom. A vrai dire même le nom proposé par l’\emph{Humanité} ne vaut rien. On peut parler d’ « autodéfense de masse », mais il est impossible de parler de « groupes d’auto-défense », car les groupes ont pour but de défendre non pas eux-mêmes, mais les organisations  ouvrières. Cependant, il ne s’agit pas, bien entendu, du nom. Les « groupes d’auto-défense » doivent, de l’avis de l’\emph{Humanité,} renoncer à l’emploi d’armes, pour ne pas tomber dans le « putschisme ». Ces sages traitent la classe ouvrière comme un enfant à qui il ne faut pas laisser un rasoir entre les mains. D’ailleurs, les rasoirs, comme on sait, sont le monopole des camelots du roi, qui, étant un légitime « produit du capitalisme », ont renversé à l’aide de rasoirs le « système » de la démocratie. Pourtant, comment les « groupes d’auto-défense » vont-ils se défendre contre les revolvers fascistes ? « Idéologiquement », bien entendu. Autrement dit : il ne leur reste qu’à se cacher. N’ayant pas ce qu’il faut entre les mains, ils doivent chercher l’ « auto-défense » dans leurs jambes. Et les fascistes pendant ce temps saccageront impunément les organisations ouvrières. Mais si le prolétariat subit une terrible défaite, en revanche il ne se rendra pas coupable de « putschisme ». Du dégoût et du mépris, voilà ce que provoque ce bavardage de poltron sous le drapeau du « bolchévisme » !\par
Déjà lors de la « troisième période » d’heureuse mémoire, quand les stratèges de l’\emph{Humanité} avaient le délire des barricades, « conquéraient » chaque jour la rue et traitaient de « social-fascistes » tous ceux qui ne partageaient pas leurs extravagances, nous prédisions : « Dès l’instant où ces gens se seront brûlé le bout des doigts, ils deviendront les pires opportunistes ». La prédiction s’est maintenant complètement confirmée. Au moment où dans le Parti socialiste se renforce et croit le mouvement en faveur de la milice, les chefs du parti qu’on appelle communiste courent prendre la lance d’incendie pour refroidir les aspirations des ouvriers avancés à se former en colonnes de combat. Peut-on se figurer un travail plus néfaste et plus démoralisant ?
 \subsection[{Il faut bâtir la milice ouvrière}]{Il faut bâtir la milice ouvrière}
\noindent Dans les rangs du Parti socialiste, il arrive d’entendre parfois cette objection : « Il faut faire la milice, mais il n’est pas besoin d’en parler tout haut ». On ne peut que féliciter les camarades qui ont le souci de soustraire le coté pratique de l’affaire aux yeux et aux oreilles importuns. Mais il est trop naïf de penser qu’on puisse créer la milice imperceptiblement, secrètement, entre quatre murs. Il nous faut des dizaines et ensuite des centaines de milliers de combattants. Ils ne viendront que si des millions d’ouvriers et d’ouvrières, et derrière eux les paysans aussi, comprennent la nécessité de la milice et créent, autour des volontaires, une atmosphère de sympathie ardente et de soutien actif. La conspiration peut et doit envelopper uniquement le côté technique de l’affaire. Mais quant à la campagne politique, elle doit se développer ouvertement, dans les réunions, dans les usines, dans les rues et sur les places publiques.\par
Les cadres fondamentaux de la milice doivent être des ouvriers d’usine, groupés suivant le lieu de travail, se connaissant l’un l’autre et pouvant protéger leurs détachements de combat contre l’infiltration des agents de l’ennemi beaucoup plus facilement et beaucoup plus sûrement que les bureaucrates les plus élevés. Des états-majors conspiratifs sans mobilisation ouverte des masses resteront au moment du danger suspendus en l’air. Il faut que toutes les organisations ouvrières se mettent à l’œuvre. Dans cette question, il ne peut y avoir de ligne de démarcation entre les partis ouvriers et les syndicats. La main dans la main ils doivent mobiliser les masses. Le succès de la milice ouvrière sera alors pleinement assuré.\par
Mais où donc les ouvriers vont-ils prendre des armes ? objectent les sérieux « réalistes », c’est-à-dire les philistins effrayés. C’est que l’ennemi de classe a des fusils,  des canons, des tanks, des gaz, des avions. Et les ouvriers, des centaines de revolvers et des couteaux de poche ?\par
Dans cette objection, tout arrive en tas pour effrayer les ouvriers. D’un côté nos sages identifient l’armement des fascistes à l’armement de l’Etat ; de l’autre, ils se tournent vers l’Etat en le priant de désarmer les fascistes. Remarquable logique ! En fait, leur position est fausse dans les deux cas. En France, les fascistes sont encore loin de s’emparer de l’Etat. Le 6 février ils sont entrés en conflit armé avec la police de l’Etat. C’est pourquoi il serait faux de parler de canons et de tanks, quand il s’agit \emph{immédiatement} de lutte armée contre les fascistes. Les fascistes, bien entendu, sont plus riches que nous, il leur est plus facile d’acheter des armes. Mais les ouvriers sont plus nombreux, plus décidés, plus dévoués, du moins quand ils sentent une ferme direction révolutionnaire. Entre autres sources, les ouvriers peuvent s’armer aux dépens des fascistes, en les désarmant systématiquement. C’est maintenant une des plus sérieuses formes de la lutte contre le fascisme. Quand les arsenaux ouvriers commenceront à se remplir au compte des dépôts fascistes, les banques et les trusts deviendront plus prudents pour financer l’armement de leurs gardes assassins. On peut même admettre que dans ce cas, — \emph{mais dans ce cas seulement} — les autorités alarmées commenceront réellement à empêcher l’armement des fascistes pour ne pas procurer une source supplémentaire d’armement aux ouvriers. On sait depuis longtemps que seule une tactique révolutionnaire engendre, comme produit accessoire, des « réformes », ou des concessions du gouvernement.\par
Mais comment donc désarmer les fascistes ? Naturellement, il est impossible de le faire uniquement au moyen de seuls articles de journaux. Il faut créer des escouades de combat. Il faut créer les états-majors de la milice. Il  faut instituer un bon service de renseignements. Des milliers d’informateurs et d’auxiliaires bénévoles arriveront de tous côtés, quand ils apprendront que l’affaire est sérieusement arrangée par nous. Il faut une volonté d’action prolétarienne\footnote{ \noindent Dans l’\emph{Humanité} du 30 octobre, Vaillant-Couturier montre très bien qu’exiger du gouvernement le désarmement des fascistes est absurde, qu’un mouvement des masses seul peut les désarmer. Puisqu’il s’agit, évidemment, d’un désarmement non pas « idéologique », mais physique, nous voulons espérer que l’\emph{Humanité }reconnaîtra maintenant la nécessité de la milice ouvrière. Nous sommes prêts à saluer sincèrement tout pas des stalinistes dans la voie juste.\par
 ...Mais, hélas ! dès le 1\textsuperscript{er} novembre, Vaillant-Couturier fait un pas décisif en arrière : le désarmement des fascistes ne se fera pas par le Front unique, mais par la police de Doumergue, « sous la pression et le contrôle » du Front unique. Fameuse idée : sans révolution, par la seule pression « idéologique », changer la police en organe exécutif du prolétariat ! A quoi bon conquérir le pouvoir quand les mêmes résultats peuvent être obtenus par la voie pacifique ? « Sous la pression et le contrôle » du Front unique, Germain-Marlin va nationaliser les banques et Marchandeau mettre à la Santé les conspirateurs réactionnaires, en commençant par son collègue Tardieu. L’idée de la « pression et du contrôle », au lieu de la lutte révolutionnaire, n’a pas été inventée par Vaillant-Couturier, il l’a empruntée à Otto Bauer, à Hilferding et au menchévik russe Dan. Le but de cette idée est celui-ci : détourner les ouvriers de la lutte révolutionnaire. En fait, il est cent fois plus facile d’écraser les fascistes de ses propres mains que par les mains d’une police hostile. Et quand le Front unique deviendra suffisamment puissant pour « contrôler » l’appareil de l’Etat, — par conséquent après la prise du pouvoir, et nullement avant, — il chassera simplement la police bourgeoise et mettra à sa place la milice ouvrière.
 }.\par
Mais les armements fascistes ne sont pas, bien entendu, l’unique source. En France, il y a plus d’un million d’ouvriers organisés. A généralement parler, c’est très peu. Mais c’est pleinement suffisant pour établir un commencement de milice ouvrière. Si les partis et les syndicats  armaient seulement le dixième de leurs membres, cela ferait déjà une milice de 100.000 hommes. Il n’y a aucun doute que le nombre des volontaires, le lendemain de l’appel du « front unique » pour la milice, dépasserait de loin ce nombre. Les contributions des partis et des syndicats, les collectes et les souscriptions volontaires donneraient la possibilité, au cours d’un ou deux mois, d’assurer des armes à 100.000 ou 200.000 combattants ouvriers. La racaille fasciste mettrait rapidement la queue entre les jambes. Toute la perspective du développement deviendrait incomparablement plus favorable.\par
Invoquer l’absence d’armement ou d’autres causes objectives pour expliquer pourquoi jusqu’à maintenant on ne s’est pas mis à la création de la milice, c’est tromper soi-même et les autres. Le principal obstacle, on peut dire le seul obstacle, a sa racine dans le caractère conservateur et passif des organisations ouvrières dirigeantes. Les sceptiques que sont les chefs ne croient pas à la force du prolétariat. Ils mettent leur espoir en toutes sortes de miracles d’en haut, au lieu de donner une issue révolutionnaire à l’énergie d’en-bas. Les ouvriers conscients doivent forcer leurs chefs soit à passer immédiatement à la création de la milice du peuple, soit à céder la place à des forces plus jeunes et plus fraîches.
\subsection[{L’armement du prolétariat}]{L’armement du prolétariat}
\noindent Une grève est inconcevable sans propagande et sans agitation, mais aussi sans piquets qui, où ils le peuvent, agissent par la persuasion, mais, là où ils y sont contraints, ont recours à la force physique. La grève est la forme la plus élémentaire de la lutte de classes, laquelle combine toujours, en des proportions diverses, les procédés « idéologiques » et les procédés physiques. La lutte contre le fascisme est, dans son fond, une lutte politique.  « qui a besoin, pourtant, d’une milice, comme la grève a besoin de piquets. Au fond le piquet, c’est l’embryon de la milice ouvrière. Celui qui pense qu’il faut renoncer à la lutte physique doit renoncer à toute lutte, car l’esprit ne vit pas sans la chair.\par
Suivant la magnifique expression du théoricien militaire Clausewitz, la guerre est la continuation de la politique par d’autres moyens. Cette définition convient pleinement aussi à la guerre civile. La lutte physique n’est qu’un « autre moyen » de la lutte politique. Il est impossible de les opposer l’une à l’autre, car il est impossible d’arrêter à volonté la lutte politique quand elle se transforme, par la force des nécessités internes, en lutte physique. Le devoir d’un parti révolutionnaire est de prévoir l’inéluctabilité de la transformation de la politique en conflit armé déclaré et, de toutes ses forces, de se préparer pour ce moment comme s’y préparent les classes dominantes.\par
Les détachements de milice pour la défense contre le fascisme sont les premiers pas sur la voie de l’armement du prolétariat, mais non pas le dernier. Notre mot d’ordre est : \emph{Armement du prolétariat et des paysans révolutionnaires.} La milice du peuple doit, en fin de compte, embrasser tous les travailleurs. Remplir ce programme \emph{complètement}, on ne le pourra que dans l’Etat ouvrier, dans les mains duquel passeront tous les moyens de production, et par conséquent aussi les moyens de destruction, c’est-à-dire tous les armements et toutes les usines qui les produisent.\par
Pourtant, il est impossible d’arriver à l’Etat ouvrier les mains vides. D’une voie pacifique, constitutionnelle vers le socialisme ne peuvent parler maintenant que des invalides politiques, dans le genre de Renaudel. La voie constitutionnelle est coupée par des tranchées occupées par les bandes fascistes. Il y a pas mal de ces tranchées devant  nous. La bourgeoisie ne reculera pas même devant une douzaine de coups d’Etat, à l’aide de la police et de l’armée, pourvu que le prolétariat n’arrive pas au pouvoir. Un Etat ouvrier socialiste ne peut être créé autrement que par la voie d’une révolution victorieuse. Toute révolution est préparée par la marche du développement économique et politique, mais elle se décide toujours par des conflits armés déclarés entre les classes hostiles. Une victoire révolutionnaire ne devient possible que grâce à une longue agitation politique, un long travail d’éducation, une longue organisation des masses. Mais le conflit armé lui-même doit être également préparé longtemps à l’avance. Los ouvriers doivent savoir qu’il leur faudra se battre dans une lutte à mort. Ils doivent tendre à avoir des armes, comme un gage de leur affranchissement. Dans une époque aussi critique que l’époque actuelle, le parti de la révolution doit prêcher inlassablement aux ouvriers la nécessité de s’armer et doit tout faire pour assurer l’armement, au moins, de l’avant-garde prolétarienne. Sans cela la victoire est impossible.\par
Les dernières grandes victoires électorales du Labour Party britannique n’affaiblissent nullement ce qui vient d’être dit. Si l’on suppose même que les prochaines élections parlementaires donnent au parti ouvrier la majorité absolue, — ce qui n’est nullement assuré ; si l’on admet même que le parti se mette réellement sur la voie de réformes socialistes, — ce qui est peu vraisemblable, — il rencontrera immédiatement une opposition si enragée de la Chambre des lords, de la couronne, des banques, de la bourse, de la bureaucratie, de la grande presse, que la scission dans la fraction parlementaire deviendra inévitable, et l’aile gauche, la plus radicale, se trouvera être la minorité parlementaire. En même temps que cela le mouvement fasciste prendra des dimensions sans précédent. Effrayée par les élections municipales, la bourgeoisie  anglaise, sans aucun doute, se prépare réellement dès maintenant à une lutte extra-parlementaire, au moment même où les sommets du parti ouvrier bercent le prolétariat avec des succès électoraux et des illusions parlementaires. Les ouvriers socialistes sont contraints, malheureusement, de voir les événements britanniques à travers les lunettes roses de Jean Longuet. En fait la bourgeoisie britannique imposera au prolétariat une guerre civile d’autant plus cruelle que les chefs du Labour Party s’y préparent moins.\par
— Mais où donc prendrez-vous des armes pour tout le prolétariat ? objectent de nouveau des sceptiques, qui prennent leur inconsistance intérieure pour une impossibilité objective. Ils oublient que la même question s’est posée devant toute révolution tout au long de l’histoire. Et malgré tout les révolutions victorieuses marquent des étapes importantes dans le développement de l’humanité.\par
Le prolétariat produit les armes, les transporte, construit les bâtiments où elles sont déposées, défend ces bâtiments contre lui-même, sert dans l’armée et crée tout l’équipement de celle-ci. Ce ne sont ni des serrures ni des murs qui séparent les armes du prolétariat, mais l’habitude de la soumission, l’hypnose de la domination de classe, le poison nationaliste. Il suffit de détruire ces murs psychologiques, — et aucun mur de pierre ne résistera. Il suffît que le prolétariat veuille des armes, — et il les trouvera. La tâche du parti révolutionnaire est d’éveiller en lui cette volonté et de faciliter sa réalisation.\par
Mais voilà que Frossard et des centaines de parlementaires, de journalistes et de fonctionnaires syndicaux apeurés lancent leur dernier argument, le plus pesant : « Est-ce que des gens sérieux peuvent en général mettre leurs espoirs dans les succès de la lutte physique après les dernières expériences tragiques d’Autriche et d’Espagne ?  Songez à la technique actuelle : des tanks ! des gaz !! des aéroplanes !!! » Cet argument démontre seulement que quelques « gens sérieux » non seulement ne veulent rien apprendre, mais qu’avec la peur ils oublient même le peu qu’ils ont appris autrefois. L’histoire de ces vingt dernières années démontre, d’une façon particulièrement claire, que les problèmes fondamentaux dans les relations entre les classes, comme aussi entre les nations, se résolvent par la force physique. Les pacifistes ont \emph{espéré} longtemps que l’accroissement de la technique militaire rendrait la guerre impossible. Les philistins depuis de nombreuses dizaines d’années ont répété que l’accroissement de la technique militaire rendrait la révolution impossible. Pourtant, les guerres et les révolutions vont leur train. Jamais il n’y a eu autant de révolutions, y compris de révolutions victorieuses, que depuis la dernière guerre, qui a dévoilé toute la force de la technique militaire.\par
Sous la forme des découvertes les plus nouvelles, Frossard et Cie présentent de vieux clichés, se bornant à invoquer, au lieu des fusils automatiques et des mitrailleuses, les tanks et les avions de bombardement. Nous répondons : derrière chaque machine il y a des hommes qui sont liés par des liens non seulement techniques, mais aussi sociaux et politiques. Quand le développement historique pose devant la société une tâche révolutionnaire inéluctable, comme une question de vie ou de mort, quand il existe une classe progressive à la victoire de laquelle est lié le salut de la société, la marche même de la lutte politique ouvre devant la classe révolutionnaire les pénibilités les plus diverses : tantôt paralyser la force militaire de l’ennemi, tantôt s’en emparer, au moins partiellement. A la conscience d’un philistin, ces possibilités se présentent toujours comme des « succès occasionnels », qui ne se répéteront plus jamais. En fait,  dans les combinaisons les plus inattendues, mais pleinement naturelles au fond, des possibilités de toute sorte s’ouvrent dans toute grande révolution véritablement populaire. Mais la victoire ne vient pas, malgré tout, d’elle-même. Pour utiliser des possibilités favorables, il faut une volonté révolutionnaire, une ferme résolution de vaincre, une direction hardie et perspicace.\par
L’\emph{Humanité} admet en paroles le mot d’ordre de l’ « armement des ouvriers », mais seulement pour y renoncer en fait. Actuellement, dans la période présente, selon l’affirmation de ce journal, il est inadmissible de lancer un mot d’ordre, qui n’est opportun qu’ « en pleine crise révolutionnaire ». Il est dangereux de charger votre fusil, dit le chasseur trop « prudent », tant que ne s’est pas montré le gibier. Mais quand le gibier se montre, il est un peu tard pour charger le fusil. Est-ce que les stratèges de l’\emph{Humanité} pensent qu’ « en pleine crise révolutionnaire » ils pourront, sans préparation, mobiliser et armer le prolétariat ? Pour se procurer beaucoup d’armes, il faut au moins une certaine quantité d’armes. Il faut des cadres militaires. Il faut le désir invincible des masses de s’emparer d’armes. Il faut un travail préparatoire ininterrompu, non pas seulement dans les salles de gymnastique, mais en liaison indissoluble avec la lutte quotidienne des masses. Cela veut dire : Il faut immédiatement construire la milice et en même temps mener une propagande en faveur de l’armement général des ouvriers et des paysans révolutionnaires.
\subsection[{Mais les défaites d’Autriche et d’Espagne...}]{Mais les défaites d’Autriche et d’Espagne...}
\noindent L’impuissance du parlementarisme dans les conditions de la crise de tout le système social du capitalisme est si évidente que les démocrates vulgaires dans le camp ouvrier (Renaudel, Frossard et leurs imitateurs) ne trouvent  pas un argument pour défendre leurs préjugés pétrifiés. D’autant plus volontiers saisissent-ils tous les échecs et toutes les défaites subis sur la voie révolutionnaire. La marche de leur pensée est celle-ci : si le parlementarisme pur n’ouvre pas d’issue, avec la lutte armée non plus ça ne va pas mieux. Les défaites des insurrections prolétariennes d’Autriche et d’Espagne sont maintenant pour eux, bien entendu, un argument de choix. En fait, dans la critique de la méthode révolutionnaire, l’inconsistance théorique et politique des démocrates vulgaires apparaît encore plus clairement que dans leur défense des méthodes de la démocratie bourgeoise pourrissante.\par
Personne n’a dit que la méthode révolutionnaire assurait automatiquement la victoire. Ce qui décide, ce n’est pas la méthode en soi, mais sa juste application, l’orientation marxiste dans les événements, une organisation puissante, la confiance des masses conquise par une longue expérience, une direction perspicace et hardie. L’issue de tout combat dépend du moment et des conditions du conflit, du rapport des forces. Le marxisme est bien loin de la pensée que le conflit armé est la seule méthode révolutionnaire, une panacée bonne dans toutes les conditions. Le marxisme en général ne connaît pas de fétiches, ni parlementaires, ni insurrectionnels. Tout est bien à sa place et en son temps. Il y a une chose qu’on peut dire dès le début : sur la voie parlementaire le prolétariat socialiste nulle part et jamais n’a encore conquis le pouvoir, et ne s’en est même pas rapproché. Les gouvernements de Scheidemann, Hermann Müller, Mac Donald n’avaient rien de commun avec le socialisme. La bourgeoisie n’a laissé venir les social-démocrates et les travaillistes au pouvoir qu’à la condition qu’ils défendent le capitalisme contre ses ennemis. Et ils ont rempli scrupuleusement cette condition. Le socialisme purement parlementaire, antirévolutionnaire, n’a abouti nulle part et  jamais, à un ministère socialiste ; par contre, il a réussi à former de méprisables renégats, qui exploitèrent le parti ouvrier pour faire une carrière ministérielle : Millerand, Briand, Viviani, Laval, Paul-Boncour, Marquet.\par
D’autre part, il est montré par l’expérience historique que la méthode révolutionnaire peut mener à la conquête du pouvoir par le prolétariat : en Russie, en 1917, en Allemagne et en Autriche en 1918, en Espagne en 1930. En Russie, il y avait un puissant parti bolchevik qui, pendant de longues années, prépara la révolution et sut solidement s’emparer du pouvoir. Les partis réformistes d’Allemagne, d’Autriche et d’Espagne ne préparèrent ni ne dirigèrent la révolution, mais la subirent. Effrayés par le pouvoir qui, contre leur gré, leur était tombé entre les mains, ils le passèrent bénévolement à la bourgeoisie. Par cette voie, ils minèrent la confiance du prolétariat en lui-même et, encore plus, la confiance de la petite bourgeoisie dans le prolétariat. Ils préparèrent les conditions de la croissance de la réaction fasciste, ils en tombèrent victimes.\par
La guerre civile, avons-nous dit après Clausewitz, est la continuation de la politique, mais par d’autres moyens. Cela veut dire : le résultat de la guerre civile dépend seulement pour 1/4, pour ne pas dire 1/10, de la marche de la guerre civile elle-même, de ses moyens techniques, de la direction purement militaire, et pour les 3/4, sinon pour les 9/10, de la préparation politique. En quoi consiste cette préparation politique ? Dans la cohésion révolutionnaire des masses, dans leur affranchissement des espoirs serviles en la clémence, la générosité, la loyauté des esclavagistes « démocratiques », dans l’éducation de cadres révolutionnaires, sachant braver l’opinion publique officielle et capables de montrer à l’égard de la bourgeoisie ne fût-ce que le 1/10\textsuperscript{e} de l’implacabilité que la bourgeoisie montre à l’égard des travailleurs.  Sans cette trempe la guerre civile, quand les conditions l’imposeront, — \emph{et elles finissent toujours par l’imposer} — se déroulera dans les conditions les plus défavorables pour le prolétariat, dépendra de beaucoup de hasards, puis, même en cas de victoire militaire, le pouvoir pourra échapper des mains du prolétariat. Qui ne voit pas que la lutte des classes mène inévitablement à un conflit armé est aveugle. Mais n’est pas moins aveugle celui qui derrière le conflit armé et son issue ne voit pas toute la politique préalable des classes en lutte.\par
En Autriche a subi la défaite non pas la méthode de l’insurrection, mais l’austro-marxisme ; en Espagne — le réformisme parlementaire sans principes. En 1918, la social-démocratie autrichienne derrière le dos du prolétariat transmit le pouvoir, qu’il avait conquis, à la bourgeoisie. En 1927, non seulement elle se détourna lâchement de l’insurrection prolétarienne qui avait toutes les chances de vaincre, mais dirigea le Schutzbund ouvrier contre les masses insurgées. Par là elle prépara la victoire de Dollfuss. Bauer et Cie disaient : « Nous voulons une évolution paisible ; mais si l’ennemi perd la tête et nous attaque, alors... » Cette formule apparaît très « sage » et très « réaliste ». Malheureusement c’est sur ce modèle austro-marxiste que construit ses raisonnements Marceau Pivert aussi : « Si... alors ». En fait cette formule représente un piège pour les ouvriers ; elle les tranquillise, les endort, les trompe. « Si » signifie : les formes de la lutte dépendent de la bonne volonté de la bourgeoisie, et non pas de l’absolue inconciliabilité des intérêts des classes. « Si » signifie : \emph{si} nous sommes paisibles, prudents, conciliants, la bourgeoisie sera loyale, et tout se passera paisiblement. Courant après le fantôme « si », Otto Bauer et les autres chefs de la social-démocratie autrichienne reculèrent passivement devant la réaction, lui cédèrent une position après l’autre, démoralisèrent les masses, reculèrent  de nouveau, jusqu’au moment où ils se trouvèrent définitivement acculés dans l’impasse ; là, dans la dernière redoute, ils acceptèrent la bataille et... la perdirent.\par
En Espagne, les événements passèrent par une autre voie, mais les causes de la défaite, au fond, sont les mêmes. Le Parti socialiste, comme les « socialistes-révolutionnaires » et les menchéviks russes, partagea le pouvoir avec la bourgeoisie républicaine, pour empêcher les ouvriers de mener la révolution jusqu’au bout. Durant deux années, les socialistes au pouvoir aidèrent la bourgeoisie à se débarrasser des masses par des miettes de réformes agraires, sociales et nationales. Contre les couches les plus révolutionnaires du peuple, les socialistes employèrent la répression. Le résultat fut double. L’anarcho-syndicalisme qui, avec une juste politique du parti ouvrier, aurait fondu dans le feu de la révolution comme de la cire, en fait se renforça et souda autour de lui les couches combatives du prolétariat. A l’autre pôle, la démagogie sociale-catholique exploitait habilement le mécontentement des masses envers le gouvernement bourgeois-socialiste. Quand le Parti socialiste se trouva suffisamment compromis, la bourgeoisie le chassa du pouvoir et passa à l’offensive sur tout le front. Au Parti socialiste il fallut se défendre dans les conditions extrêmement défavorables que lui avait préparées sa propre politique antérieure. La bourgeoisie avait déjà un appui de masse à droite. Les chefs anarcho-syndicalistes, qui commirent au cours de la révolution toutes les fautes tombant à la portée de la main de ces confusionnistes professionnels, refusèrent de soutenir l’insurrection dirigée par les « politiciens »-traîtres. Le mouvement ne prit pas un caractère général, mais sporadique. Le gouvernement porta ses coups sur les diverses cases de l’échiquier. La guerre civile ainsi imposée par la réaction se termina par la défaite du prolétariat.\par
De l’expérience espagnole il n’est pas difficile de tirer  une conclusion contre la participation socialiste à un gouvernement bourgeois. La conclusion est en soi indiscutable, mais absolument insuffisante. Le prétendu « radicalisme » austro-marxiste n’est nullement meilleur que le ministérialisme espagnol. La différence entre eux est technique, et non politique. Tous deux attendaient que la bourgeoisie leur rende « loyauté » pour « loyauté ». Et tous deux ont mené le prolétariat à des catastrophes. En Espagne comme en Autriche subirent la défaite non pas les méthode de la révolution, mais les méthodes opportunistes dans une situation révolutionnaire. Ce n’est pas la même chose !\par

\asterism

\noindent Nous ne nous arrêterons pas ici sur la politique de l’Internationale communiste en Autriche et en Espagne et nous renvoyons le lecteur aux collections de la \emph{Vérité }des dernières années et à une série de brochures\footnote{ \noindent Voir, en particulier : Léon Trotsky : \emph{L’internationale Communiste après Lénine,} Rieder éditeur, 1930 ;\par
 \emph{La crise autrichienne} et \emph{le communisme}, dans \emph{Les problèmes de la Révolution allemande}, édité par la \emph{Vérité,} 1931.\par
 \emph{La Révolution Espagnole et les tâches des Communistes,} édité par la \emph{Vérité,} 1931 ;\par
 \emph{La Révolution Espagnole et les dangers qui la menacent,} édité par la \emph{Vérité}, 1931 ;\par
 (Ces deux derniers documents sont reproduits dans \emph{La Révolution Permanente,} Rieder éditeur, qui contient aussi des \emph{Lettres }sur la crise espagnole.)\par
 On lira aussi avec fruit, quoique concernant plus particulièrement l’Allemagne, \emph{Et Maintenant ?} et \emph{La seule voie,} édités par la \emph{Vérité,} 1932.
 }.\par
Dans une situation politique exceptionnellement favorable, les Partis communistes autrichien et espagnol, accablés par la théorie de la « troisième période », du « social-fascisme », etc., se trouvèrent voués à un isolement complet. Compromettant les méthodes de la révolution par  l’autorité de « Moscou », ils barrèrent par cela même la voie à une politique véritablement marxiste, véritablement bolchevique. La propriété fondamentale de la révolution est de soumettre à un examen rapide et impitoyable toutes les doctrines et toutes les méthodes. Le châtiment suit presque immédiatement le crime. La responsabilité de l’Internationale communiste pour les défaites du prolétariat en Allemagne, en Autriche, en Espagne, est incommensurable. Il n’est pas suffisant de mener une politique « révolutionnaire » (en paroles). Il faut une juste politique. Personne n’a encore trouvé d’autre secret de victoire.
\subsection[{Le front unique et la lutte pour le pouvoir}]{Le front unique et la lutte pour le pouvoir}
\noindent Nous avons déjà dit : le front unique des Partis socialiste et communiste renferme en soi de grandioses possibilités. Si seulement il le veut sérieusement, il deviendra demain le maître de la France. Mais il doit le vouloir.\par
Le fait que Jouhaux et, en général, la bureaucratie de la C.G.T. se tiennent en \emph{dehors} du front unique, en conservant leur « indépendance », semble contredire ce que nous avançons. Mais c’est seulement à première vue. A une époque de grandes tâches et de grands dangers, qui dressent sur pied les masses, les cloisons entre les organisations politiques et syndicales du prolétariat disparaissent. Les ouvriers veulent savoir comment se sauver du chômage et du fascisme, comment conquérir leur indépendance envers le capital, et ils ne se soucient guère de l’ « indépendance » de Jouhaux envers la politique prolétarienne (de la politique bourgeoise Jouhaux est — hélas ! — fort dépendant). Si l’avant-garde prolétarienne, en la personne du front unique, trace justement la voie de la lutte, toutes les bornes établies par la bureaucratie syndicale seront renversées par le torrent vivant du prolétariat. La clé de la situation est maintenant dans le front unique. S’il  ne se sert pas de cette clé, il jouera le rôle lamentable qu’aurait inévitablement joué le front unique des menchéviks et des « socialistes-révolutionnaires » en 1917 en Russie, si... si les bolchéviks ne les en avaient empêchés.\par
Nous ne parlons pas des Partis socialiste et communiste en particulier parce que, politiquement, tous deux ont renoncé à leur indépendance en faveur du front unique. Dès que les deux partis ouvriers, qui se concurrençaient vivement dans le passé, ont renoncé à se critiquer l’un l’autre et à se conquérir l’un à l’autre des adhérents, par cela même ils ont cessé d’exister en tant que partis distincts. Invoquer les « divergences principielles » qui demeurent, ne change rien à l’affaire. Dès que les divergences principielles ne se manifestent pas ouvertement et activement, en un moment si plein de responsabilités qu’actuellement, elles cessent par cela même d’exister politiquement ; elles sont semblables aux trésors qui dorment au fond de l’océan. Le travail commun finira-t-il ou non par la fusion ? Nous ne voulons pas le prédire. Mais pour la présente période, qui a une importance décisive dans les destinées de la France, le front unique agit comme un parti inachevé, construit sur le principe fédéraliste.\par
Que veut le front unique ? Jusqu’à maintenant il ne l’a pas dit aux masses. La lutte contre le fascisme ? Mais jusqu’à maintenant le front unique n’a même pas expliqué \emph{comment} il pense lutter contre le fascisme. D’ailleurs le bloc défensif contre le fascisme ne pourrait être suffisant que si, pour tout le reste, les deux partis conservaient une complète indépendance. Mais non, nous avons un front unique qui embrasse presque toute l’activité publique des deux partis et exclut leur lutte réciproque pour conquérir la majorité du prolétariat. De cette situation, il faut tirer toutes les conséquences. La première et la plus importante est celle-ci : \emph{la lutte pour le pouvoir.} Le but du front unique ne peut être qu’un gouvernement du front unique, c’est-à- dire un gouvernement socialiste-communiste, un ministère Blum-Cachin. Il faut le dire ouvertement. Si le front unique se prend au sérieux — et c’est à cette seule condition que le prendront au sérieux les masses populaires — il ne peut se dérober au mot d’ordre de conquête du pouvoir. Par quels moyens ? Par tous les moyens qui mènent au but. Le front unique ne renonce pas à la lutte parlementaire. Mais il utilise le Parlement avant tout pour démasquer l’impuissance du Parlement et expliquer au peuple que le gouvernement actuel a une base extra-parlementaire et qu’on ne peut le renverser que par un puissant mouvement des masses. La lutte pour le pouvoir signifie l’utilisation de toutes les possibilités, qu’ouvre le régime bonapartiste semi-parlementaire, pour renverser ce régime par une poussée révolutionnaire, pour remplacer l’Etat bourgeois par un Etat ouvrier.\par
Les dernières élections cantonales ont donné un accroissement de voix socialistes et surtout communistes. En lui-même ce fait ne règle rien. Le Parti communiste allemand, à la veille de son effondrement, a eu un afflux incomparablement plus impétueux de voix. De nouvelles larges couches d’opprimés sont poussées à gauche par toute la situation, indépendamment même de la politique des partis extrêmes. Le Parti communiste français a gagné plus de voix, car par tradition il reste, malgré toute sa politique conservatrice actuelle, « l’extrême-gauche ». Les masses ont manifesté par là leur tendance à donner une impulsion \emph{à gauche} aux partis ouvriers, car les masses sont énormément plus à gauche que leurs partis. De cela témoigne aussi l’état d’esprit révolutionnaire de la jeunesse socialiste. Il ne faut pas oublier que la jeunesse représente le baromètre sensible de toute la classe et de son avant-garde ! Si le front unique ne sort pas de la passivité ou, pis encore, entreprend un roman indigne avec les radicaux, « à gauche » du front unique commenceront à se renforcer  les anarchistes, les anarcho-syndicalistes et autres groupements semblables de désagrégation politique. En même temps se renforcera l’indifférence, précurseur de la catastrophe. Par contre, d’autre part, si le front unique, en assurant ses derrières et ses flancs contre les bandes fascistes, ouvre une large offensive politique sous le mot d’ordre de conquête du pouvoir, il rencontrera un écho si puissant qu’il dépassera les attentes les plus optimistes. Ne pas comprendre cela ne le peuvent que des bavards creux, pour qui les grands mouvements des masses resteront toujours le livre aux sept sceaux.
\subsection[{Pas un programme de passivité, mais un programme de révolution}]{Pas un programme de passivité, mais un programme de révolution}
\noindent La lutte pour le pouvoir doit partir de l’idée fondamentale que, si une opposition à une aggravation future de la situation des masses sur le terrain du capitalisme est encore possible, aucune amélioration réelle de leur situation n’est concevable sans incursion révolutionnaire dans le droit de propriété capitaliste. La campagne du front unique doit s’appuyer sur un programme de transition bien élaboré, c’est-à-dire sur une système de mesures, qui — avec un gouvernement ouvrier et paysan — doivent assurer la transition du capitalisme au socialisme\footnote{ \noindent Sur le contenu du programme lui-même nous ne nous arrêtons pas ici et renvoyons le lecteur au \emph{Programme d’action} édité par la Ligue communiste en 1934, qui représente le projet d’un tel programme de transition.
 }.\par
Or, il faut un programme non pas pour tranquilliser sa conscience, mais pour mener une action révolutionnaire. Que vaut le programme, s’il reste lettre morte ? Le Parti ouvrier belge a adopté, par exemple, le pompeux plan De Man, avec toutes les « nationalisations » ; mais quel  sens cela a-t-il, s’il ne veut pas lever le petit doigt pour sa réalisation ? Les programmes du fascisme sont fantastiques, mensongers, démagogiques. Mais le fascisme mène une lutte enragée pour le pouvoir. Le socialisme peut lancer le programme le plus savant ; mais sa valeur sera à zéro si l’avant-garde du prolétariat ne déploie pas une lutte hardie pour s’emparer de l’Etat. La crise sociale, dans son expression politique, est la crise du pouvoir. Le vieux maître de la société est banqueroutier. Il faut un nouveau maître. Si le prolétariat révolutionnaire ne s’empare pas du pouvoir, c’est inévitablement le fascisme qui s’en emparera !\par
Un programme de revendications transitoires pour les « classes moyennes » peut, naturellement, prendre une grande importance, si ce programme répond, d’une part, aux besoins réels des classes moyennes, de l’autre, aux exigences du développement vers le socialisme\footnote{ \noindent Dans l’\emph{Ecole émancipée} le camarade G. Serret publie un intéressant questionnaire au sujet de la situation économique des différentes couches de la paysannerie et de leurs tendances politiques. Les instituteurs pourraient devenir des agents irremplaçables du Front Unique au village et jouer, dans la période qui vient, un rôle historique. Mais pour cela ils doivent sortir de leur coquille. Ce n’est vraiment pas le moment de se livrer à de petites expériences dans de petits laboratoires. \emph{Les instituteurs révolutionnaires doivent entrer dans le Parti socialiste pour renforcer son aile révolutionnaire et le lier aux masses paysannes.} Il serait criminel de perdre du temps !\par
 (Ceci fut écrit fin 1934. Depuis, les bolchéviks-léninistes ont été chassés du Parti socialiste, mais les instituteurs ne sont pas sortis de leur coquille... N. du T.) .
 }. Mais encore une fois, le centre de gravité ne se trouve pas actuellement dans un programme spécial. Les « classes moyennes » ont vu beaucoup de programmes. Ce qu’il leur faut, c’est avoir confiance que le programme sera réalisé. Au moment où le paysan se dira : « Cette fois il semble bien que le parti ouvrier ne reculera pas », la  cause du socialisme sera gagnée. Mais, pour cela, il faut montrer en fait que nous sommes fermement prêts à briser tous les obstacles sur notre route.\par
Il n’est pas besoin d’inventer des moyens de lutte ; ils sont donnés par toute l’histoire du mouvement ouvrier mondial : Une campagne concentrée de la presse ouvrière frappant sur le même clou ; des discours véritablement socialistes à la tribune parlementaire, non pas en députés apprivoisés, mais en chefs du peuple ; l’utilisation de toutes les campagnes électorales pour des buts révolutionnaires ; des meetings répétés, où les masses viennent non pas simplement pour entendre les orateurs, mais recevoir les mots d’ordre et les directives de l’heure ; la création et le renforcement de la milice ouvrière ; des manifestations bien organisées, balayant de la rue les bandes réactionnaires ; des grèves de protestation ; une campagne ouverte pour l’unification et l’élargissement des rangs des syndicats sous le signe d’une lutte de classes résolue ; des actions opiniâtres et bien calculées pour la conquête de l’armée à la cause du peuple ; des grèves plus larges ; des manifestations plus puissantes ; la grève générale des travailleurs des villes et des champs ; une offensive générale contre le gouvernement bonapartiste pour le pouvoir des ouvriers et des paysans.\par
Pour préparer la victoire, il est encore temps. Le fascisme n’est pas encore devenu un mouvement de masse. La décomposition inévitable du radicalisme signifiera, pourtant, le rétrécissement de la base du bonapartisme, la croissance des camps extrêmes et le rapprochement du dénouement. Il ne s’agit pas d’années, mais de mois. Ce délai, assurément, n’est écrit nulle part. Il dépend de la lutte des forces vives, en premier chef de la politique du prolétariat et de son Front unique. Les forces potentielles de la révolution dépassent de beaucoup les forces du fascisme et en général de toute la réaction réunie. Les sceptiques qui pensent que tout est perdu doivent être  impitoyablement chassés des rangs ouvriers. Les couches profondes font un écho vibrant à chaque parole hardie, à chaque mot d’ordre véritablement révolutionnaire. Les masses profondes veulent la lutte.\par
Ce n’est pas l’esprit de combinaison des parlementaires et des journalistes, mais la haine légitime et créatrice des opprimés contre les oppresseurs qui est maintenant le seul facteur progressif de l’histoire. Il faut se tourner vers les masses, vers leurs couches les plus profondes. Il faut faire appel à leur passion et à leur raison. Il faut rejeter cette fausse « prudence », qui est le pseudonyme de la couardise et qui, dans les grands tournants historiques, équivaut à la trahison. Le front unique doit prendre pour devise la formule de Danton : « De l’audace, toujours de l’audace, et encore de l’audace ».\par
Bien comprendre la situation et en tirer toutes les conclusions pratiques — hardiment, sans peur, jusqu’au bout — c’est assurer la victoire du socialisme.
 \section[{Encore une fois, où va la France. (Fin mars 1935)}]{Encore une fois, où va la France \\
(Fin mars 1935)}\phantomsection
\label{p3}\renewcommand{\leftmark}{Encore une fois, où va la France \\
(Fin mars 1935)}

\noindent Au moment où Flandin succéda à Doumergue, nous avons posé devant l’avant-garde prolétarienne la question : « Où va la France ? » Les quatre mois et demi écoulés n’ont rien changé d’essentiel et n’ont affaibli ni notre analyse, ni notre pronostic. Le peuple français est arrivé à un carrefour : une voie mène à la révolution socialiste, l’autre — à la catastrophe fasciste. Le choix de la voie dépend du prolétariat. A sa tête se trouve son avant-garde organisée. Nous posons de nouveau la question : où l’avant-garde prolétarienne va-t-elle mener la France ?\par
\subsubsection[{Le diagnostic de l’Internationale communiste est faux et funeste}]{Le diagnostic de l’Internationale communiste est faux et funeste}
\noindent La C.A.P. du Parti socialiste a lancé en janvier un programme de \emph{lutte pour le pouvoir, de destruction de l’armature de l’Etat bourgeois, d’instauration de la démocratie ouvrière et paysanne, d’expropriation des banques et des branches concentrées de l’industrie}. Pourtant, le Parti jusqu’à maintenant n’a pas remué le petit doigt pour porter ce programme devant les masses. A son tour, le Parti communiste se refuse bel et bien à se mettre sur la voie de la lutte pour le pouvoir. La cause ? « La situation n’est pas révolutionnaire. »\par
 La milice ? L’armement des ouvriers ? Le contrôle ouvrier ? Un plan de nationalisation ? Impossible ! « La situation n’est pas révolutionnaire ». Que peut-on faire ? Lancer de grandes pétitions avec les cléricaux, s’exercer à l’éloquence creuse avec les radicaux et attendre. Jusqu’à quand ? Tant que la situation ne deviendra pas d’elle-même révolutionnaire. Les savants médecins de l’Internationale communiste ont un thermomètre, qu’ils mettent sous l’aisselle de la vieille femme qu’est l’Histoire et par ce moyen déterminent infailliblement la température révolutionnaire. Mais ils ne montrent leur thermomètre à personne.\par
Nous affirmons : le diagnostic de l’Internationale communiste est radicalement faux. La situation est révolutionnaire autant qu’elle peut être révolutionnaire \emph{avec la politique non-révolutionnaire} des partis ouvriers. Le plus exact est de dire que la situation est \emph{pré-révolutionnaire.} Pour que cette situation mûrisse, il faut une mobilisation immédiate, hardie et inlassable des masses sous les mots d’ordre de conquête du pouvoir au nom du socialisme. C’est à cette seule condition que la situation \emph{pré-révolutionnaire} se changera en situation \emph{révolutionnaire}. Dans le cas contraire, c’est-à-dire si on continue à piétiner sur place, la situation pré-révolutionnaire se changera infailliblement en situation contre-révolutionnaire et amènera la victoire du fascisme.\par
La phrase sacramentelle sur la « situation non-révolutionnaire » sert actuellement uniquement à bourrer le crâne aux ouvriers, à paralyser leur volonté et à délier les mains à l’ennemi de classe. Sous le couvert de pareilles phrases s’assemblent dans les sommets du prolétariat le conservatisme, la mollesse, l’étourderie, la lâcheté, et se prépare la catastrophe, comme en Allemagne.
 \subsubsection[{La tâche et le but de ce travail}]{La tâche et le but de ce travail}
\noindent Dans les pages qui suivent, nous, bolchéviks-léninistes, nous soumettons le diagnostic et le pronostic de l’Internationale communiste à une critique marxiste détaillée. A l’occasion, nous nous arrêterons sur les points de vue des divers chefs socialistes dans la mesure où cela sera nécessaire pour notre but fondamental : montrer la \emph{fausseté radicale de la politique du Comité central du Parti communiste français}. Aux cris et aux injures des stalinistes nous opposerons des faits et des arguments.\par
Nous ne nous bornerons pas, bien entendu, à une simple critique. Aux points de vue et aux mots d’ordre faux nous opposerons les idées et les méthodes créatrices de Marx et de Lénine.\par
Nous demandons au lecteur une attention concentrée. Il s’agit, dans le sens le plus direct et le plus immédiat, de la tête du prolétariat français. Pas un seul ouvrier conscient n’a le droit de rester impassible devant ces questions, de la solution desquelles dépend le sort de sa classe !
\subsection[{I. — Comment se forme une situation révolutionnaire ?}]{I. — Comment se forme une situation révolutionnaire ?}
\subsubsection[{La prémisse économique de la révolution socialiste}]{La prémisse économique de la révolution socialiste}
\noindent La première et la plus importante prémisse d’une situation révolutionnaire, c’est l’exacerbation intolérable des contradictions entre les forces productives et les formes de la propriété. \emph{La nation cesse d’aller de l’avant}. L’arrêt dans le développement de la puissance économique et, encore plus, sa régression signifient que le système capitaliste de production s’est définitivement épuisé et doit céder la place au système socialiste.\par
 La crise actuelle, qui embrasse tous les pays et rejette l’économie à des dizaines d’années en arrière, a définitivement poussé le système bourgeois jusqu’à l’absurde. Si à l’aurore du capitalisme des ouvriers affamés et ignorants ont brisé les machines, maintenant ceux qui détruisent les machines et les usines ce sont les capitalistes eux-mêmes. Avec le maintien ultérieur de la propriété privée des moyens de production, l’humanité est menacée de barbarie et de dégénérescence.\par
La base de la société, c’est son économie. Cette base est mûre pour le socialisme dans un double sens : la \emph{technique} moderne a atteint un tel degré qu’elle pourrait assurer un bien-être élevé au peuple et à toute l’humanité ; mais la \emph{propriété capitaliste}, qui se survit, voue les peuples à une pauvreté et à des souffrances toujours plus grandes.\par
La prémisse fondamentale, économique, du socialisme existe depuis déjà longtemps. Mais le capitalisme ne disparaîtra pas de \emph{lui-même} de la scène. Seule la classe ouvrière peut arracher les forces productives des mains des exploiteurs et des étrangleurs. L’histoire pose avec acuité cette tâche devant nous. Si le prolétariat se trouve pour telle ou telle raison incapable de renverser la bourgeoisie et de prendre le pouvoir, s’il est, par exemple, paralysé par ses propres partis et ses propres syndicats, le déclin de l’économie et de la civilisation se poursuivra, les calamités s’accroîtront, le désespoir et la prostration s’empareront des masses, le capitalisme — décrépit, pourrissant, vermoulu — étranglera toujours plus fort les peuples, en les entraînant dans l’abîme de nouvelles guerres. \emph{Hors de la révolution socialiste, point de salut.}
\subsubsection[{Est-ce la dernière crise du capitalisme ou non ?}]{Est-ce la dernière crise du capitalisme ou non ?}
\noindent Le présidium de l’Internationale communiste essaya d’abord d’expliquer que la crise, commencée en 1929, était la dernière crise du capitalisme. Deux ans après, Staline  déclara que la crise actuelle \emph{n’est} « vraisemblablement » \emph{pas} encore la \emph{dernière.} Nous rencontrons aussi dans le camp socialiste la même tentative de prophétie : la dernière crise ou non ?\par
« Il est imprudent d’affirmer, écrit Blum dans le \emph{Populaire} du 23 février, que la crise actuelle est comme un spasme suprême du capitalisme, le dernier sursaut avant l’agonie et la décomposition. » C’est le même point de vue qu’a Grumbach, qui dit, le 26 février, à Mulhouse : « D’aucuns affirment que cette crise est passagère ; les autres y voient la crise finale du système capitaliste. Nous n’osons pas encore nous prononcer définitivement. »\par
Dans cette façon de poser la question, il y a deux erreurs cardinales : premièrement, on mêle ensemble la \emph{crise conjoncturelle} et la crise \emph{historique de tout le système capitaliste ;} secondement, on admet, qu’\emph{indépendamment de l’activité consciente des classes}, une crise puisse \emph{d’elle-même} être la « dernière » crise.\par
Sous la domination du capital industriel, à l’époque de la libre concurrence, les montées conjoncturelles dépassaient de loin les crises ; les premières étaient la « règle », les secondes l’ « exception » ; le capitalisme dans son ensemble était en montée. Depuis la guerre, avec la domination du capital financier monopolisateur, les crises conjoncturelles surpassent de loin les ranimations ; on peut dire que les crises sont devenues la règle, les montées l’exception ; le développement économique dans son ensemble va vers le bas, et non vers le haut.\par
Néanmoins, des oscillations conjoncturelles sont inévitables et avec le capitalisme malade elles se perpétueront tant qu’existera le capitalisme. Et le capitalisme se perpétuera tant que la révolution prolétarienne ne l’aura pas achevé. Telle est la seule réponse correcte.
 \subsubsection[{Fatalisme et marxisme}]{Fatalisme et marxisme}
\noindent Le révolutionnaire prolétarien doit avant tout comprendre que le \emph{marxisme,} seule théorie scientifique de la révolution prolétarienne, n’a rien de commun avec l’attente fataliste de la « dernière » crise. Le marxisme est par son essence même une \emph{direction pour l’action révolutionnaire. }Le marxisme n’ignore pas la volonté et le courage, mais les aide à trouver la voie juste.\par
Il n’y a aucune crise qui \emph{d’elle-même} puisse être « mortelle » pour le capitalisme. Les oscillations de la conjoncture créent seulement une situation dans laquelle il sera plus facile ou plus difficile au prolétariat de renverser le capitalisme. Le passage de la société bourgeoise à la société socialiste présuppose l’activité de gens vivants, qui font leur propre histoire. Ils ne la font pas nu hasard ni selon leur bon plaisir, mais sous l’influence de causes objectives déterminées. Cependant, leurs propres actions — leur initiative, leur audace, leur dévouement ou, au contraire, leur sottise et leur lâcheté — entrent comme des anneaux nécessaires dans la chaîne du développement historique.\par
Personne n’a numéroté les crises du capitalisme et n’a indiqué par avance laquelle d’entre elles serait la « dernière ». Mais toute notre époque et surtout la crise actuelle dictent impérieusement au prolétariat : \emph{Prends le pouvoir !} Si, pourtant, le parti ouvrier, malgré des conditions favorables. se révèle incapable de mener le prolétariat à la conquête du pouvoir, la vie de la société continuera nécessairement sur les bases capitalistes — jusqu’à une nouvelle crise ou une nouvelle guerre, peut-être jusqu’au complet effondrement de la civilisation européenne.
\subsubsection[{La « dernière » crise et la « dernière » guerre}]{La « dernière » crise et la « dernière » guerre}
\noindent La guerre impérialiste de 1914-1918 représenta aussi une « crise » dans la marche du capitalisme, et bien la  plus terrible de toutes les crises possibles. Dans aucun livre il ne fut prédit si cette guerre serait la \emph{dernière} folie sanglante du capitalisme ou non. L’expérience de la Russie a montré que la guerre \emph{pouvait} être la fin du capitalisme. En Allemagne et en Autriche le sort de la société bourgeoise dépendit entièrement en 1918 de la social-démocratie, mais ce parti se révéla être le domestique du capital. En Italie et en France, le prolétariat aurait pu à la fin de la guerre conquérir le pouvoir, mais il n’avait pas à sa tête un parti révolutionnaire. En un mot, si la II\textsuperscript{e} Internationale n’avait pas trahi au moment de la guerre la cause du socialisme pour le patriotisme bourgeois, toute l’histoire de l’Europe et de l’humanité se présenterait maintenant tout autrement. Le passé, assurément, n’est pas réparable. Mais on peut et on doit apprendre les leçons du passé.\par
Le développement du fascisme est en soi le témoignage irréfutable du fait que la classe ouvrière a terriblement tardé à remplir la tâche posée depuis longtemps devant elle par le déclin du capitalisme.\par
La phrase : cette crise n’est pas encore la « dernière », ne peut avoir qu’un seul sens : malgré les leçons de la guerre et des convulsions de l’après-guerre, les partis ouvriers n’ont pas encore su préparer ni eux-mêmes, ni le prolétariat, à la prise du pouvoir ; pis encore, les chefs de ces partis ne voient pas encore jusqu’à maintenant la tâche elle-même, en la faisant retomber d’eux-mêmes, du parti et de la classe sur le « développement historique ». Le fatalisme est une trahison théorique envers le marxisme et la justification de la trahison politique envers le prolétariat, c’est-à-dire la préparation d’une nouvelle capitulation devant une nouvelle « dernière » guerre.
 \subsubsection[{L’Internationale communiste est passée sur les positions du fatalisme social-démocrate}]{L’Internationale communiste est passée sur les positions du fatalisme social-démocrate}
\noindent Le fatalisme de la social-démocratie est un héritage de l’avant-guerre, quand le capitalisme grandissait presque sans cesse, que s’accroissait le nombre des ouvriers, qu’augmentait le nombre des membres du parti, des voix aux élections et des mandats. De cette montée automatique naquit peu à peu l’illusion réformiste qu’il suffit de continuer dans l’ancienne voie (propagande, élections, organisation) et la victoire viendra d’elle-même.\par
Certes, la guerre a détraqué l’automatisme du développement. Mais la guerre est un phénomène « exceptionnel ». Genève aidant, il n’y aura plus de nouvelle guerre, tout rentrera dans la norme, et l’automatisme du développement sera rétabli.\par
A la lumière de cette perspective, les paroles : « Ce n’est pas encore la dernière crise », doivent signifier : « Dans cinq ans, dans dix ans, dans vingt ans, nous aurons plus de voix et de mandats, alors, il faut l’espérer, nous prendrons le pouvoir ». (Voir les articles et discours de Paul Faure.) Ce fatalisme optimiste, qui semblait convaincant il y a un quart de siècle, résonne aujourd’hui comme une voix d’outre-tombe. Radicalement fausse est l’idée qu’en allant vers la crise future le prolétariat deviendra infailliblement plus puissant que maintenant. Avec la putréfaction ultérieure inévitable du capitalisme le prolétariat ne croîtra pas et ne se renforcera pas, mais se décomposera, rendant toujours plus grande l’armée des chômeurs et des lumpen-prolétaires ; la petite bourgeoisie entre temps se déclassera et tombera dans le désespoir. La perte de temps ouvre une perspective au fascisme, et non à la révolution prolétarienne.\par
Il est remarquable que l’Internationale communiste aussi, bureaucratisée jusqu’à la moelle, a remplacé la  théorie de l’action révolutionnaire par la religion du fatalisme. Il est impossible de lutter, car « il n’y a pas de situation révolutionnaire ». Mais une situation révolutionnaire ne tombe pas du ciel, elle se forme dans la lutte des classes. Le parti du prolétariat est le plus important facteur \emph{politique} quant à la formation d’une situation révolutionnaire. Si ce parti tourne le dos aux tâches révolutionnaires, en endormant et en trompant les ouvriers pour jouer aux pétitions et pour fraterniser avec les radicaux, il doit alors se former non pas une situation révolutionnaire, mais une situation contre-révolutionnaire.
\subsubsection[{Comment la bourgeoisie apprécie-t-elle la situation ?}]{Comment la bourgeoisie apprécie-t-elle la situation ?}
\noindent Le \emph{déclin du capitalisme}, avec le degré extraordinairement élevé des forces productives, est la prémisse économique de la révolution socialiste. Sur cette base se déroule la \emph{lutte des classes}. Dans la lutte vive des classes se forme et mûrit une \emph{situation révolutionnaire,}\par
Comment la \emph{grande bourgeoisie,} maîtresse de la société contemporaine, apprécie-t-elle la situation actuelle, et comment agit-elle ? Le 6 février 1934 ne fut inattendu que pour les organisations ouvrières et la petite bourgeoisie. Les centres du grand capital participaient depuis longtemps au complot, avec le but de substituer par la violence au parlementarisme le bonapartisme (régime « personnel »). Cela veut dire : les banques, les trusts, l’état-major, la grande presse jugeaient le danger de la révolution si proche et si immédiat qu’ils se dépêchèrent de s’y préparer par un « petit » coup d’Etat.\par
Deux conclusions importantes découlent de ce fait : 1) les capitalistes, dès avant 1934, jugeaient la situation comme révolutionnaire ; 2) ils ne restèrent pas à attendre passivement le développement des événements, pour recourir à la dernière minute à une défense « légale », mais  ils prirent eux-mêmes l’initiative, en faisant descendre leurs bandes dans la rue. La grande bourgeoisie a donné aux ouvriers une leçon inappréciable de stratégie de classe !\par
L’\emph{Humanité} répète que le « front unique » a chassé Doumergue. Mais c’est, pour parler modérément, une fanfaronnade creuse. Au contraire, si le grand capital a jugé possible et raisonnable de remplacer Doumergue par Flandin, c’est uniquement parce que le Front unique, comme la bourgeoisie s’en est convaincue par l’expérience, ne représente pas encore un danger révolutionnaire immédiat. « Puisque les terribles chefs de l’Internationale communiste, malgré la situation dans le pays, ne se préparent pas à la lutte, mais tremblent de peur, cela veut dire qu’on peut attendre pour passer au fascisme. Inutile de forcer les événements et de compromettre prématurément les radicaux, dont on peut encore avoir besoin. » C’est ce que disent les véritables maîtres de la situation. Ils maintiennent l’union nationale et ses décrets bonapartistes, ils mettent le Parlement sous la terreur, mais ils laissent se reposer Doumergue. Les chefs du capital ont apporté ainsi une certaine correction à leur appréciation primitive, en reconnaissant que la situation n’est pas immédiatement révolutionnaire, mais pré-révolutionnaire.\par
Seconde leçon remarquable de stratégie de classe ! Elle montre que même le grand capital, qui a à sa disposition tous les leviers de commande, ne peut apprécier d’un seul coup \emph{a priori} et infailliblement la situation politique dans toute sa réalité : il entre en lutte et dans le processus de la lutte, sur la base de l’expérience de la lutte, il corrige et précise son appréciation. Tel est en général le seul moyen possible de s’orienter en politique exactement et en même temps activement.\par
Et les chefs de l’Internationale communiste ? A Moscou, à l’écart du mouvement ouvrier français, quelques médiocres bureaucrates, mal renseignés, en majorité ne  lisant même pas le français, donnent à l’aide de leur thermomètre le diagnostic infaillible : « La situation n’est pas révolutionnaire ». Le Comité central du Parti communiste français est tenu, en fermant yeux et oreilles, de répéter cette phrase creuse. La voie de l’Internationale communiste est la voie la plus courte vers l’abîme !
\subsubsection[{Le sens de la capitulation des radicaux}]{Le sens de la capitulation des radicaux}
\noindent Le \emph{parti radical} représente l’\emph{instrument politique de la grande bourgeoisie}, qui est le mieux adapté \emph{aux traditions et aux préjugés de la petite bourgeoisie}. Malgré cela, les chefs les plus responsables du radicalisme, sous le fouet du capital financier, se sont humblement inclinés devant le coup d’Etat du 6 février, dirigé immédiatement contre eux. Ils ont reconnu ainsi que la marche de la lutte des classes menace les intérêts fondamentaux de la « nation », c’est-à-dire de la bourgeoisie, et se sont vus contraints de sacrifier les intérêts électoraux de leur parti. La capitulation du plus puissant parti parlementaire devant les revolvers et les rasoirs des fascistes est l’expression extérieure de l’\emph{effondrement complet de l’équilibre politique du pays. }Mais celui qui prononce ces mots, dit par cela même : la situation est révolutionnaire ou, pour parler plus exactement, prérévolutionnaire\footnote{ \noindent Il est extrêmement caractéristique pour \emph{la bureaucratie ouvrière} petite-bourgeoise effrayée, surtout par les stalinistes, qu’elle se soit alliée aux radicaux « pour lutter contre le fascisme », après que les radicaux eurent révélé leur complète incapacité de lutter contre le fascisme. Le cartel électoral avec les radicaux, qui était un crime du point de vue des intérêts historiques du prolétariat, avait au moins, dans les cadres restreints du parlementarisme son sens pratique. \emph{L’alliance extra-parlementaire avec les radicaux contre le fascisme est non seulement un crime, mais encore une idiotie.}
 }.
 \subsubsection[{La petite bourgeoisie et la situation pré-révolutionnaire}]{La petite bourgeoisie et la situation pré-révolutionnaire}
\noindent Les processus qui se déroulent dans les masses de la petite bourgeoisie ont une importance exceptionnelle pour apprécier la situation politique. La crise politique du pays est avant tout la crise de la confiance des masses petites-bourgeoises dans leurs partis et leurs chefs traditionnels. \emph{Le mécontentement, la nervosité, l’instabilité, l’emportement facile de la petite bourgeoisie} sont des traits extrêmement importants d’une situation pré-révolutionnaire. De même que le malade brûlant de fièvre se met sur le côté droit ou le côté gauche, la petite bourgeoisie fébrile peut se tourner à droite ou à gauche. Selon le côté vers lequel se tourneront dans la prochaine période les millions de paysans, d’artisans, de petits commerçants, de petits fonctionnaires français, la situation pré-révolutionnaire actuelle peut se changer aussi bien en situation révolutionnaire que contre-révolutionnaire.\par
L’amélioration de la conjoncture économique pourrait — pas pour longtemps — retarder, mais non pas arrêter la différenciation à droite ou à gauche de la petite bourgeoisie. Au contraire, si la crise allait s’approfondissant, la faillite du radicalisme et de tous les groupements parlementaires qui gravitent autour de lui irait à une vitesse redoublée.
\subsubsection[{Comment peut se produire un coup d’Etat fasciste en France ?}]{Comment peut se produire un coup d’Etat fasciste en France ?}
\noindent Il ne faut pas toutefois penser que le fascisme doive nécessairement devenir un puissant parti parlementaire, avant qu’il se soit emparé du pouvoir. C’est ainsi que cela se passa en Allemagne, mais, en Italie ce fut autrement. Pour le succès du fascisme il n’est pas du tout obligatoire que la petite bourgeoisie ait rompu \emph{préalablement} avec les  anciens partis « démocratiques » : il suffit qu’elle ait perdu la confiance qu’elle avait en eux et qu’elle regarde avec inquiétude autour d’elle, en cherchant de nouvelles voies.\par
Aux prochaines élections municipales, la petite bourgeoisie peut encore donner un nombre très important de ses voix aux radicaux et aux groupes voisins, par l’absence d’un nouveau parti politique, qui réussirait à conquérir la confiance des paysans et des petites gens des villes. Et en même temps un coup de force militaire du fascisme peut se produire, avec l’aide de la grande bourgeoisie, dès quelques mois après les élections et par sa pression attirer à lui les sympathies des couches les plus désespérées de la petite bourgeoisie.\par
C’est pourquoi ce serait une grossière illusion de se consoler en pensant que le drapeau du fascisme n’est pas encore devenu populaire dans la province et dans les villages. Les tendances antiparlementaires de la petite bourgeoisie peuvent, en s’échappant du lit de la politique parlementaire officielle des partis, soutenir directement et immédiatement un coup d’Etat militaire, lorsque celui-ci deviendra nécessaire pour le salut du grand capital. Un tel mode d’action correspond beaucoup plus à la fois aux traditions et au tempérament de la France\footnote{ \noindent Le marxisme n’ignore nullement — notons-le en passant — ces éléments comme la tradition et le tempérament national. La direction fondamentale du développement est déterminée, évidemment, par la marche de la lutte des classes. Mais les \emph{formes} du mouvement, son \emph{rythme}, etc., peuvent varier beaucoup sous l’influence du tempérament et des traditions nationales, qui, à leur tour, se sont formées dans le passé sous l’influence de la marche de la lutte des classes.
 }.\par
Les chiffres des élections ont, bien entendu, une importance symptomatique. Mais s’appuyer sur ce \emph{seul} indice serait faire preuve de crétinisme parlementaire. Il s’agit de processus plus profonds, qui, un mauvais matin, peuvent prendre à l’improviste messieurs les parlementaires. Là,  comme dans les autres domaines, la question est tranchée non pas par l’arithmétique, mais par la dynamique de la lutte. La grande bourgeoisie n’enregistre pas passivement l’évolution des classes moyennes, mais prépare les tenailles d’acier, à l’aide desquelles elle pourra saisir au moment opportun les masses torturées par elle et désespérées.
\subsubsection[{Dialectique et métaphysique}]{Dialectique et métaphysique}
\noindent La pensée marxiste est \emph{dialectique ;} elle considère tous les phénomènes dans leur développement, dans leur passage d’un état à un autre. La pensée du petit bourgeois conservateur est métaphysique : ses conceptions sont immobiles et immuables, entre les phénomènes il y a des cloisonnements imperméables. L’opposition absolue entre une situation révolutionnaire et une situation non-révolutionnaire représente un exemple classique de pensée métaphysique, selon la formule : ce qui est, est — ce qui n’est pas, n’est pas, et tout le reste vient du Malin.\par
Dans le processus de l’histoire, on rencontre des situations stables tout à fait non-révolutionnaires. On rencontre aussi des situations notoirement révolutionnaires. Il existe aussi des situations contre-révolutionnaires (il ne faut pas l’oublier !). Mais ce qui existe surtout à notre époque de capitalisme pourrissant ce sont des situations \emph{intermédiaires, transitoires :} entre une situation non-révolutionnaire et une situation pré-révolutionnaire, entre une situation pré-révolutionnaire et une situation révolutionnaire ou... contre-révolutionnaire. C’est précisément ces états transitoires qui ont une importance décisive du point de vue de la stratégie politique.\par
Que dirions-nous d’un artiste qui ne distinguerait que les deux couleurs extrêmes dans le spectre ? Qu’il est daltonien ou à moitié aveugle et qu’il lui faut renoncer au pinceau. Que dire d’un homme politique, qui ne serait  capable de distinguer que deux états : « révolutionnaire » et « non-révolutionnaire » ? Que ce n’est pas un marxiste, mais un staliniste, qui peut faire un bon fonctionnaire, mais en aucun cas un chef prolétarien.\par
Une situation révolutionnaire se forme par l’action réciproque de facteurs objectifs et subjectifs. Si le parti du prolétariat se montre incapable d’analyser à temps les tendances de la situation prérévolutionnaire et d’intervenir activement dans son développement, au lieu d’une situation révolutionnaire surgira inévitablement une situation contrerévolutionnaire. C’est précisément \emph{devant ce danger que se trouve actuellement le prolétariat français.} La politique à courte vue, passive, opportuniste du front unique, et surtout des stalinistes, qui sont devenus son aile droite, voilà ce qui constitue le \emph{principal obstacle sur la voie de la révolution prolétarienne en France.}
\subsection[{II. — Les revendications immédiates et la lutte pour le pouvoir}]{II. — Les revendications immédiates et la lutte pour le pouvoir}
\subsubsection[{La stagnation du front unique}]{La stagnation du front unique}
\noindent Le Comité central du Parti communiste repousse la lutte pour la nationalisation des moyens de production, comme une revendication incompatible avec l’Etat bourgeois. Mais le Comité central repousse aussi la lutte pour le pouvoir pour la création de l’Etat ouvrier. A ces tâches il \emph{oppose} un programme de « revendications immédiates ».\par
\emph{Le front unique est actuellement privé de quelque programme que ce soit}. En même temps l’expérience propre du Parti communiste dans le domaine de la lutte pour les « revendications immédiates » a un caractère extrêmement lamentable. Tous les discours, articles et résolutions sur la nécessité de riposter au capital par des grèves n’ont  jusqu’à présent abouti à rien, ou presque. Malgré une situation de plus en plus tendue dans le pays, il règne dans la classe ouvrière une \emph{stagnation dangereuse}.\par
Le Comité central du Parti communiste accuse de cette stagnation tout le monde, sauf lui. Nous ne nous disposons à blanchir personne. Nos points de vue sont connus. Mais nous pensons que le \emph{principal obstacle} sur la voie du développement de la lutte révolutionnaire est actuellement le programme unilatéral, contredisant toute la situation, presque maniaque, des « revendications immédiates ». Nous voulons ici faire la lumière sur les considérations et les arguments du Comité central du Parti communiste avec toute l’ampleur nécessaire. Non pas que ces arguments soient sérieux et profonds : au contraire, ils sont misérables. Mais il s’agit d’une question, dont dépend le sort du prolétariat français.
\subsubsection[{La résolution du Comité central du Parti communiste sur les « revendications immédiates »}]{La résolution du Comité central du Parti communiste sur les « revendications immédiates »}
\noindent Le document le plus autorisé sur la question des « revendications immédiates » est la résolution programmatique du Comité central du Parti communiste. (Voir l’\emph{Humanité} du 24 février.) Nous nous arrêterons à ce document.\par
L’énoncé des revendications immédiates est fait très généralement : défense des salaires, amélioration des assurances sociales, conventions collectives, « contre la vie chère », etc. On ne dit pas un mot sur le caractère que peut et doit prendre dans les conditions de la crise sociale actuelle la lutte pour ces revendications. Pourtant, tout ouvrier comprend qu’avec deux millions de chômeurs complets et partiels, la lutte syndicale ordinaire pour des conventions collectives est une utopie. Pour contraindre, dans les conditions actuelles les capitalistes à faire des concessions  sérieuses il faut \emph{briser leur volonté ;} on ne peut y parvenir que par une offensive révolutionnaire. Mais une offensive révolutionnaire, qui oppose une classe à une classe, ne peut se développer uniquement sous des mots d’ordre économiques partiels. On tombe dans un \emph{cercle vicieux.} C’est là qu’est la principale cause de la stagnation du front unique.\par
La thèse marxiste générale : \emph{les réformes sociales ne sont que les sous-produits de la lutte révolutionnaire,} prend à l’époque du déclin capitaliste l’importance la plus immédiate et la plus brûlante. Les capitalistes ne peuvent céder aux ouvriers \emph{quelque chose} que s’ils sont menacés du danger de perdre \emph{tout.}\par
Mais même les plus grandes « concessions », dont est capable le capitalisme contemporain, lui-même acculé dans l’impasse, resteront absolument insignifiantes en comparaison avec la misère des masses et la profondeur de la crise sociale. Voilà pourquoi la plus immédiate de toutes les revendications doit être de revendiquer l’\emph{expropriation des capitalistes et la nationalisation (socialisation) des moyens de production.} Cette revendication est irréalisable sous la domination de la bourgeoisie ? Evidemment. C’est pourquoi il faut conquérir le pouvoir.
\subsubsection[{Pourquoi les masses ne font-elles pas écho aux appels du Parti communiste ?}]{Pourquoi les masses ne font-elles pas écho aux appels du Parti communiste ?}
\noindent La résolution du Comité central reconnaît en passant que « le Parti n’a pas encore réussi à organiser et à développer la résistance à l’offensive du capital ». Mais la résolution ne s’arrête pas du tout sur la question de savoir pourquoi donc, malgré les efforts du P.C. et de la C.G.T.U., les succès dans le domaine de la lutte économique défensive sont absolument insignifiants. A la grève générale du 12 février, qui ne poursuivait aucune « revendication  immédiate », participèrent des millions d’ouvriers et d’employés. Cependant, à la défense contre l’offensive du capital n’a participé jusqu’à maintenant qu’une fraction infime de ce même chiffre. Est-ce que ce fait étonnamment clair ne conduit les « chefs » du Parti communiste à aucune conclusion ? Pourquoi des \emph{millions d’ouvriers se risquent-ils} à \emph{participer à la grève générale,} à des manifestations de rues agitées, à des conflits avec les bandes fascistes, \emph{mais se refusent-ils à participer à des grèves économiques dispersées ?}\par
« Il faut comprendre — dit la résolution — les sentiments qui agitent les ouvriers désireux de passer à l’action. » Il faut comprendre... Mais le malheur est que les auteurs eux-mêmes de la résolution ne comprennent rien. Quiconque fréquente les réunions ouvrières sait comme nous que les discours généraux sur les « revendications immédiates » laissent le plus souvent les auditeurs dans un état d’indifférence renfrognée ; par contre, les mots d’ordre révolutionnaires clairs et précis provoquent en réponse une vague de sympathie. Cette différence de réaction de la masse caractérise de la façon la plus claire la situation politique de notre pays.\par
« Dans la période présente, — remarque inopinément la résolution, — la lutte économique nécessite de la part des ouvriers de \emph{lourds sacrifices}. » Il faudrait encore ajouter : et ce n’est que par exception qu’elle promet des résultats positifs. Et pourtant, la lutte pour les revendications immédiates a pour tâche \emph{d’améliorer} la situation des ouvriers. En mettant cette lutte au premier plan, en renonçant pour elle aux mots d’ordre révolutionnaires, les stalinistes considèrent, sans doute, que c’est précisément la lutte économique partielle qui est le plus capable de soulever de larges masses. Il s’avère justement le contraire : les masses ne font presque aucun écho aux appels  pour des grèves économiques. Comment peut-on donc en politique ne pas tenir compte des faits ?\par
Les masses comprennent ou sentent que dans les conditions de la crise et du chômage des conflits économiques partiels exigent des sacrifices inouïs, que ne justifieront en aucun cas les résultats obtenus. Les masses attendent et réclament d’autres méthodes, plus efficaces. Messieurs les stratèges, apprenez chez les masses : elles sont guidées par un sûr instinct révolutionnaire.
\subsubsection[{La conjoncture économique et la lutte gréviste}]{La conjoncture économique et la lutte gréviste}
\noindent S’appuyant sur des citations mal assimilées de Lénine, les stalinistes répètent : « La lutte gréviste est possible même en temps de crise ». Ils ne comprennent pas qu’il y a crise et crise. A l’époque du capitalisme ascendant, à la fois industriels et ouvriers, même pendant une crise aiguë, regardent en avant, vers la nouvelle ranimation prochaine. \emph{La crise actuelle est la règle, et non l’exception}. Dans le domaine purement économique le prolétariat par la terrible pression de la catastrophe économique est rejeté dans une retraite désordonnée. D’autre part, le déclin du capitalisme pousse de tout son poids le prolétariat sur la voie de la lutte politique révolutionnaire de masse. Pourtant, la direction du Parti communiste tend de toutes ses forces à barrer cette voie. Ainsi, dans les mains des stalinistes, le programme des « revendications immédiates » devient un instrument de désorientation et de désorganisation du prolétariat. Cependant, une offensive politique (lutte pour le pouvoir) avec une défense armée active (milice) renverserait d’un seul coup le rapport des forces des classes et, chemin faisant, ouvrirait, \emph{pour les couches ouvrières les plus retardataires}, la possibilité d’une lutte économique victorieuse.
 \subsubsection[{La possibilité d’une ranimation de la conjoncture}]{La possibilité d’une ranimation de la conjoncture}
\noindent Le capitalisme agonisant, comme nous le savons, a aussi ses cycles, mais des cycles déclinants, malades. Seule la révolution prolétarienne peut mettre fin à la crise du \emph{système capitaliste}. La crise \emph{conjoncturelle} fera inévitablement place à une nouvelle et brève ranimation, si ne survient pas entre temps la guerre ou la révolution.\par
En cas de ranimation de la conjoncture économique la lutte gréviste pourra, sans aucun doute, prendre une étendue beaucoup plus grande. C’est pourquoi il faut suivre attentivement le mouvement du commerce et de l’industrie, particulièrement les changements dans le marché du travail, sans se fier aux météorologues de l’école de Jouhaux et en aidant pratiquement les ouvriers à faire pression au moment nécessaire sur les capitalistes. Mais même dans le cas d’une lutte gréviste étendue il serait criminel de se borner à des revendications économiques partielles. La ranimation de la conjoncture ne peut être ni profonde, ni longue, car nous avons affaire avec les cycles d’un capitalisme irrémédiablement malade. La nouvelle crise — après une brève ranimation — peut se trouver être plus terrible que la présente. Tous les problèmes fondamentaux surgiront de nouveau, et avec une force et une acuité redoublées. Si l’on perd du temps, la croissance du fascisme peut se révéler irrésistible.\par
Mais aujourd’hui, la ranimation économique n’est qu’une hypothèse. La réalité, c’est l’approfondissement de la crise, le service militaire de deux ans, le réarmement de l’Allemagne, le danger de guerre.\par
C’est de cette réalité qu’il faut partir.
\subsubsection[{Les dépouilles du réformisme en guise de programme révolutionnaire}]{Les dépouilles du réformisme en guise de programme révolutionnaire}
\noindent L’idée finale de la résolution programmatique du Comité  Central couronne dignement tout l’édifice. Citons-la littéralement :\par

\begin{quoteblock}
 \noindent « En combattant chaque jour pour soulager les masses laborieuses des misères que leur impose le régime capitaliste, les communistes \emph{soulignent} que la libération définitive ne peut être obtenue que par l’abolition du régime capitaliste et l’instauration de la dictature du prolétariat. »
 \end{quoteblock}

\noindent Cette formule ne sonnait pas mal à l’aube de la social-démocratie, il y a un demi-siècle et plus. La social-démocratie dirigeait alors non sans succès la lutte des ouvriers pour des revendications et des réformes isolées, pour ce qu’on appelait le « programme-minimum », en « soulignant » bien que l’affranchissement \emph{définitif} du prolétariat ne serait réalisé que par la révolution. Le « but final » du socialisme était alors tracé dans le lointain nébuleux des années. C’est cette conception, qui déjà à la veille de la guerre s’était complètement survécue, que le Comité central du Parti communiste a transportée inopinément dans notre époque, en la répétant mot pour mot, jusqu’à la dernière virgule. Et ces gens invoquent Marx et Lénine !\par
Quand ils « soulignent » que l’ « \emph{affranchissement définitif} » ne peut être obtenu que par l’abolition du régime capitaliste, ils s’ingénient à l’aide de cette vérité élémentaire à tromper les ouvriers. Car ils leur suggèrent l’idée qu’une certaine amélioration, même importante, peut être obtenue dans les cadres du régime actuel. Ils représentent le capitalisme pourrissant et déclinant comme leurs pères et leurs grands-pères représentaient le capitalisme robuste et ascendant. Le fait est indiscutable : les stalinistes se parent des dépouilles du réformisme.\par
La formule politique marxiste, en fait, doit être celle-ci :\par
 
\begin{quoteblock}
 \noindent En expliquant chaque jour aux masses que le capitalisme pourrissant ne laisse pas de place non seulement pour l’amélioration de leur situation, mais même pour le maintien du niveau de misère habituel, en posant ouvertement devant les masses la tâche de la révolution socialiste, comme la tâche immédiate de nos jours, en mobilisant les ouvriers pour la prise du pouvoir, en défendant les organisations ouvrières au moyen de la milice, — les communistes (ou les socialistes) ne perdent pas, en même temps, une seule occasion pour arracher, chemin faisant, à l’ennemi telle ou telle concession partielle, ou, au moins, pour l’empêcher d’abaisser encore plus le niveau de vie des ouvriers.
 \end{quoteblock}

\noindent Comparez attentivement cette formule aux lignes de la résolution du Comité central citées plus haut. La différence, espérons-nous, est claire. D’un coté, le \emph{stalinisme} ; de l’autre, le \emph{léninisme.} Entre eux, un abîme.
\subsubsection[{Un moyen sûr contre le chômage}]{Un moyen sûr contre le chômage}
\noindent L’augmentation des salaires, les conventions collectives, l’abaissement du prix de la vie... Mais que faire avec le chômage ? La résolution du Comité central nous vient aussi en aide là-dessus. Citons-la :\par

\begin{quoteblock}
 \noindent « Ils (les communistes) réclament l’ouverture de travaux publics. A cet effet, ils élaborent des propositions concrètes adaptées à chaque situation locale ou régionale, ils préconisent les moyens de financer ces travaux (projet de prélèvement sur te capital, emprunts avec la garantie de l’Etat, etc.). »
 \end{quoteblock}

\noindent N’est-ce pas étonnant ? Cette recette de charlatan est copiée presque mot pour mot chez Jouhaux : les stalinistes  repoussent les revendications progressives du « Plan » et adoptent sa partie la plus fantaisiste et la plus utopique.\par
Les principales forces productives de la société sont paralysées ou à demi-paralysées par la crise. Les ouvriers sont dans la torpeur devant les machines qu’ils ont créées. Le Comité central sauveur propose : en dehors de l’économie capitaliste réelle, à coté d’elle, créer une autre économie capitaliste, sur la base de « travaux publics ».\par
Que l’on ne nous dise pas qu’il s’agit d’entreprises épisodiques : le chômage actuel n’a pas un caractère épisodique ; ce n’est pas simplement un chômage conjoncturel, mais un chômage de structure, l’expression la plus pernicieuse du déclin capitaliste. Pour le faire disparaître, le Comité central propose de créer un système de grands travaux, adapté à chaque région du pays, à l’aide d’un système particulier de financement, à côté des finances en désarroi du capitalisme. En un mot, le Comité central du Parti communiste propose tout simplement au capitalisme de changer de domicile. Et c’est ce « plan » qu’on oppose à la lutte pour le pouvoir et au programme de nationalisation ! \emph{Il n’y a pas de pires opportunistes que les aventuristes effrayés}.\par
\emph{Comment parvenir} à la réalisation des travaux publics, au prélèvement sur le capital, aux emprunts garantis, etc., là-dessus on ne nous dit pas un mot. Sans doute, à l’aide de... \emph{pétitions}. C’est le moyen d’action le plus opportun et le plus efficace. Aux pétitions ne résistent ni la crise, ni le fascisme, ni le militarisme. En outre, les pétitions font revivre l’industrie du papier et adoucissent le chômage. Notons donc : l’organisation de pétitions, partie fondamentale du système de travaux publics selon le plan de Thorez et compagnie.\par
De qui ces gens se moquent-ils ? D’eux-mêmes ou du prolétariat ?
 \subsubsection[{Le Parti communiste est un frein}]{Le Parti communiste est un frein}
\noindent « Il est étonnant que le prolétariat supporte passivement de telles privations et de telles violences après une lutte de classes plus que centenaire. » On peut entendre à chaque pas cette phrase si hautaine de la bouche d’un socialiste ou d’un communiste en chambre. La résistance est insuffisante ? On met cette faute sur le dos des masses ouvrières. Comme si les partis et les syndicats se trouvaient à l’écart du prolétariat et n’étaient pas ses organes de lutte ! C’est précisément parce que le prolétariat, en résultat de l’histoire plus que centenaire de ses luttes, a créé ses organisations politiques et syndicales, qu’il lui est difficile, presque impossible, de mener sans \emph{elles} et \emph{contre elles} la lutte contre le capital. Et pourtant, ce qui a été édifié comme le ressort de l’action est devenu un poids mort ou un frein.\par
Toute la situation inspire aux travailleurs l’idée que les actions révolutionnaires sont nécessaires pour changer toutes les conditions de l’existence. Mais précisément parce qu’il s’agit d’une lutte décisive, qui doit embrasser des millions d’hommes, son initiative repose naturellement sur les \emph{organisations dirigeantes}, sur les partis ouvriers, sur le Front unique. C’est d’eux que doivent partir un programme clair, des mots d’ordre, des mobilisations de combat. \emph{Pour soulever les masses, les partis doivent s’engager eux-mêmes,} en ouvrant une campagne révolutionnaire hardie dans le pays. Mais les organisations dirigeantes, le Parti communiste y compris, n’en ont pas le courage. Le P.C. rejette ses tâches et ses responsabilités sur les masses. Il exige que des millions d’hommes laissés par lui sans direction révolutionnaire entreprennent des combats dispersés pour des revendications partielles et montrent ainsi aux bureaucrates sceptiques qu’ils sont prêts à mener la lutte. Peut-être alors les grands chefs  consentiront-ils à commander l’offensive. Au lieu de \emph{diriger} les masses, le Comité central bureaucratique examine les masses, leur donne une mauvaise note et justifie ainsi son opportunisme et sa lâcheté.
\subsubsection[{Des recettes toutes faîtes « selon Lénine »}]{Des recettes toutes faîtes « selon Lénine »}
\noindent Au moment de l’équilibre économique et politique relatif de la France (1929-1933) le Comité central du Parti communiste proclama la « troisième période » et ne voulait se satisfaire que de la conquête de la rue par les barricades. Maintenant, au moment de la crise économique, sociale et politique, le même Comité central se contente d’un modeste programme de « revendications immédiates ». Cette contradiction absurde est le produit complexe de plusieurs facteurs ; l’effroi devant ses dernières fautes, l’incapacité de prêter l’oreille à la masse, l’habitude bureaucratique de prescrire au prolétariat une feuille de route toute faite, enfin, l’anarchie intellectuelle, résultat de zigzags, de falsifications, de mensonges et de répressions sans nombre.\par
L’auteur immédiat du nouveau programme est, sans doute, le « chef » actuel de l’Internationale communiste Bela Kun, qui va toujours tour à tour de l’aventurisme à l’opportunisme. Ayant lu dans Lénine que les bolchéviks furent \emph{dans certaines conditions} pour les grèves, et les menchéviks contre, Bela Kun fonda en un clin d’œil sur cette découverte sa politique « réaliste ». Mais pour son malheur, Bela Kun n’avait pas ouvert Lénine... à la bonne page.\par
A certaines périodes, les grèves économiques jouèrent réellement un rôle énorme dans le mouvement révolutionnaire du prolétariat russe. Or, le capitalisme russe n’était pas pourri à ce moment-là, mais grandissait et s’élevait rapidement. Le prolétariat russe était une classe vierge et  les grèves étaient pour lui la première forme d’éveil et d’activité. Enfin, le large débordement des grèves coïncida chaque fois avec l’essor conjoncturel de l’industrie.\par
Aucune de ces conditions n’existe en France. Le prolétariat français a derrière lui une grandiose école de révolution, de lutte syndicale et parlementaire, avec tout l’héritage positif et négatif de ce riche passé. Il serait difficile d’attendre un débordement spontané du mouvement gréviste en France, même en période d’essor économique, d’autant plus lorsque la crise conjoncturelle approfondit les plaies du déclin capitaliste.\par
Non moins important est l’autre côté de la question. Au moment du premier mouvement gréviste impétueux en Russie, il y eut une seule fraction de la social démocratie russe qui tenta de se borner à des revendications économiques partielles : ce fut ceux qu’on appela les « économistes ». Selon leur opinion, il fallait repousser le mot d’ordre : « A bas l’autocratie ! » jusqu’à l’apparition d’une « situation révolutionnaire ». Lénine jugea les « économistes » comme de misérables opportunistes. Il montra \emph{qu’il fallait préparer activement une situation révolutionnaire} même en période de mouvement gréviste.\par
Il est en général absurde de tenter de transporter mécaniquement en France les diverses étapes et les divers épisodes du mouvement révolutionnaire russe. Mais il est encore moins possible de le faire à la manière de Bela Kun, qui ne connaît ni la Russie, ni la France, ni le marxisme. A l’école de Lénine, il faut apprendre la \emph{méthode d’action} et non pas changer le léninisme en citations et en recettes, bonnes pour tous les cas de la vie.
\subsubsection[{« La Paix, le Pain et la Liberté ! »}]{« La Paix, le Pain et la Liberté ! »}
\noindent Ainsi, la situation en France, selon l’opinion des stalinistes, n’est pas révolutionnaire ; les mots d’ordre révolutionnaires,  par ce fait, sont inopportuns ; il faut concentrer toute l’attention sur les grèves économiques et les revendications partielles. Tel est le programme. C’est un programme opportuniste et sans vie, mais c’est un programme.\par
A côté de lui, il y en a, pourtant, un autre. L’\emph{Humanité} répète chaque jour le triple mot d’ordre : « La paix, le pain, la liberté ». C’est sous ce drapeau, explique l’\emph{Humanité}, que les bochéviks ont vaincu en 1917. A la suite des stalinistes, Just répète la même idée. Très bien. Mais en 1917, en Russie, il y avait une situation notoirement révolutionnaire. Comment donc des mots d’ordre, qui ont assuré le succès de la révolution prolétarienne, se trouvent-ils bons comme « revendications immédiates » dans une situation non-révolutionnaire ? Que les augures de l’\emph{Humanité} nous expliquent à nous, simples mortels, ce mystère.\par
Nous, pour notre part, nous rappellerons quelles « revendications immédiates » renfermait le triple mot d’ordre des bolchéviks.\par
« \emph{Pour la paix !} » Cela signifiait en 1917, dans les conditions de la guerre, la lutte contre tous les partis patriotiques, des monarchistes aux menchéviks, la revendication de la publication de tous les traités secrets, la mobilisation révolutionnaire des soldats contre le commandement et l’\emph{organisation de la fraternisation} sur les \emph{fronts}. « Pour la paix ! », cela signifiait un défi au militarisme de l’Autriche et de l’Allemagne, d’une part, de l’Entente, de l’autre. Le mot d’ordre des bolcheviks signifiait ainsi la politique la plus hardie et la plus révolutionnaire qu’ait jamais connue l’histoire de l’humanité.\par
« Lutter » pour la paix en 1935, en alliance avec Herriot et les « pacifistes » bourgeois, c’est-à-dire les impérialistes hypocrites, signifie simplement soutenir le statu quo, bon \emph{au moment présent} pour la bourgeoisie  française. Cela signifie endormir et démoraliser les ouvriers par les illusions du « désarmement », des « pactes de non-agression », par le mensonge de la Société des Nations, en préparant une nouvelle capitulation des partis ouvriers au moment ou la bourgeoisie française ou ses rivaux trouveront bon de renverser le statu quo.\par
« \emph{Pour le pain !} » Cela signifiait pour les bolchéviks en 1917 l’\emph{expropriation de la terre et des réserves de blé chez les propriétaires fonciers et les spéculateurs} et le \emph{monopole du commerce du blé dans les mains du gouvernement des ouvriers et des paysans}. Que signifie « Pour le pain ! » chez nos stalinistes en 1935 ? Une simple répétition verbale !\par
« \emph{Pour la liberté !} » Les bolchéviks montraient aux masses que la liberté reste une fiction, tant que les écoles, la presse, les lieux de réunion restent dans les mains de la bourgeoisie. « Pour la liberté ! » signifiait : la prise du pouvoir par les soviets, l’expropriation des propriétaires fonciers, le contrôle ouvrier sur la production.\par
« Pour la liberté ! », en alliance avec Herriot et les vénérables dames des deux sexes de la Ligue des droits de l’homme, signifie soutenir les gouvernements semi-bonapartistes, semi-parlementaires, et rien d’autre. La bourgeoisie a besoin actuellement non seulement des bandes de de La Rocque, mais aussi de la réputation « gauche » de Herriot. Le capital financier s’occupe d’armer les Fascistes. Les stalinistes restaurent la réputation gauche de Herriot à l’aide des mascarades du « Front populaire ». Voilà à quoi servent en 1935 les mots d’ordre de la Révolution d’Octobre !
\subsubsection[{Dragons et puces}]{Dragons et puces}
\noindent A titre de seul exemple de la nouvelle politique « réaliste », la résolution du Comité central raconte que les  chômeurs de Villejuif mangent la soupe des Croix de feu et crient : « La Rocque au poteau ! ». Combien d’hommes mangent la soupe, combien crient, on ne nous le dit pas : les stalinistes ne peuvent souffrir les chiffres. Mais là n’est pas la question... Jusqu’où doit tomber le parti « révolutionnaire », pour, dans une résolution programmatique, ne pas trouver d’autre exemple de politique prolétarienne que les cris impuissants d’ouvriers accablés et affamés, contraints de se nourrir des miettes de la philanthropie fasciste. Et ces chefs ne se sentent ni humiliés, ni honteux !\par
Marx citait une fois, en parlant de certains de ses disciples, les paroles de Heine : « J’ai semé des dragons, et j’ai récolté des puces. » Nous craignons bien que les fondateurs de la III\textsuperscript{e} Internationale ne doivent répéter ces mêmes paroles... Et pourtant, notre époque a besoin non pas de puces, mais de dragons !
\subsection[{III — La lutte contre le fascisme et la grève générale}]{III — La lutte contre le fascisme et la grève générale}
\subsubsection[{Le programme de l’Internationale communiste et le fascisme}]{Le programme de l’Internationale communiste et le fascisme}
\noindent Le programme de l’Internationale communiste, écrit en 1928, dans la période de déclin théorique de l’I.C., dit : « L’époque de l’impérialisme est l’époque du capitalisme agonisant », En soi, cette affirmation, formulée bien auparavant par Lénine, est absolument indiscutable et a une importance décisive pour la politique du prolétariat à notre époque. Mats les auteurs du programme de l’Internationale communiste n’ont absolument pas compris la thèse mécaniquement adoptée par eux sur le capitalisme \emph{agonisant} ou \emph{pourrissant}. Cette incompréhension apparut  d’une façon particulièrement claire dans la question la plus brûlante pour nous : le fascisme.\par
Le programme de l’Internationale communiste dit à ce sujet : « \emph{A côté} de la social-démocratie, qui aide la bourgeoisie à étouffer le prolétariat et à en endormir sa vigilance, apparaît le fascisme ». L’Internationale communiste n’a pas compris que la mission du fascisme n’est pas d’agir \emph{à coté} de la social-démocratie, mais d’écraser toutes les anciennes organisations ouvrières, y compris les organisations réformistes. La tâche du fascisme, c’est, selon les termes du programme, d’ « anéantir les couches \emph{communistes} du prolétariat et leurs cadres dirigeants ». Le fascisme ne menacerait absolument pas la social-démocratie et les syndicats réformistes ; au contraire, la social-démocratie elle-même jouerait de plus en plus un « rôle fasciste ». Le fascisme ne ferait que compléter l’œuvre du réformisme, en agissant « \emph{à coté} de la social-démocratie ».\par
Nous citons non pas l’article de quelconques Thorez ou Duclos qui se contredisent à chaque pas, mais le document fondamental de l’Internationale communiste, son programme. (Voir chapitre II, paragraphe 3 : « La crise du capitalisme et le fascisme ».) Nous avons là devant nous tous les éléments fondamentaux de la théorie du \emph{social-fascisme.} Les chefs de l’Internationale communiste n’ont pas compris que le capitalisme pourrissant ne peut plus s’accommoder de la social-démocratie la plus modérée et la plus servile, ni en tant que parti au pouvoir ni en tant que parti dans l’opposition. Le fascisme est appelé à prendre place non pas « à côté de la social-démocratie », mais sur ses os. C’est précisément de là que vinrent la possibilité, la nécessité et l’urgence du front unique. Mais la malheureuse direction de l’Internationale communiste n’a tenté d’appliquer la politique du front unique que dans la période où celle-ci n’était pas imposée à la social-démocratie. Dès que la situation du  réformisme fut ébranlée et que la social-démocratie tomba sous les coups, l’Internationale communiste se refusa au front unique. Ces gens ont le fâcheux penchant de mettre un manteau en été et d’aller en hiver sans même une feuille de vigne !\par
Malgré l’expérience instructive de l’Italie, l’Internationale communiste a inscrit sur son drapeau l’aphorisme génial de Staline : « La social-démocratie et le fascisme ne sont pas des antipodes, mais des jumeaux ». C’est la principale cause de la défaite du prolétariat allemand. Certes, dans la question du front unique, l’I.C. a accompli un brusque tournant : les faits se sont trouvés plus puissants que le programme. Mais le programme de l’Internationale communiste n’a été ni supprimé, ni modifié. Ses erreurs fondamentales n’ont pas été expliquées aux ouvriers. Les chefs de l’Internationale communiste, qui ont perdu confiance en eux-mêmes, conservent \emph{pour tous les cas} un pont de retraite vers les positions du « social-fascisme ». Cela donne à la politique du Front unique un caractère sans principe, diplomatique et instable.
\subsubsection[{Les illusions réformistes et stalinistes}]{Les illusions réformistes et stalinistes}
\noindent L’incompréhension du sens de la thèse de Lénine sur le « capitalisme agonisant » donne à toute la politique actuelle du Parti communiste français un caractère d’impuissance criarde, complétée par des illusions réformistes. Alors que le fascisme représente le produit organique du déclin capitaliste, les stalinistes sont subitement persuadés de la possibilité de mettre fin au fascisme, sans toucher aux bases de la société bourgeoise.\par
Le 6 mars, Thorez écrivait pour la 101\textsuperscript{e} fois dans l’\emph{Humanité :}\par

\begin{quoteblock}
 \noindent « Afin d’assurer l’échec \emph{définitif} du fascisme, nous proposons de nouveau au parti socialiste  l’action commune pour la défense des revendications immédiates... »
 \end{quoteblock}

\noindent Tout ouvrier conscient doit bien réfléchir à cette phrase « programmatique ». Le fascisme, comme nous le savons, naît de l’union du désespoir des classes moyennes et de la politique terroriste du grand capital. Les « revendications immédiates », ce sont les revendications qui ne sortent pas du cadre du capitalisme. Comment donc, en restant sur le terrain du capitalisme pourrissant, peut-on « assurer l’échec définitif (!) » du fascisme ?\par
Quand Jouhaux dit : en mettant fin à la crise (ce n’est pas si simple !), nous aurons vaincu par cela même le fascisme, Jouhaux, au moins, est fidèle à lui-même : il garde encore et toujours espoir dans la régénération et le rajeunissement du capitalisme. Or, les stalinistes reconnaissent en paroles l’inéluctabilité de la décomposition prochaine du capitalisme. Comment peuvent-ils donc promettre d’assainir la superstructure politique, en assurant l’échec définitif du fascisme, et en même temps laisser intacte la base économique pourrissante de la société ?\par
Pensent-ils que le grand capital peut à sa guise faire tourner la roue de l’histoire en arrière et se mettre de nouveau sur la voie des concessions et des « réformes » ? Croient-ils que la petite bourgeoisie peut être sauvée, à l’aide de « revendications immédiates », de la ruine croissante, du déclassement et du désespoir) Comment accorder alors ces illusions trade-unionistes et réformistes avec la thèse sur le capitalisme agonisant ?\par
Prise dans son plan théorique, la position du Parti communiste représente, comme nous le voyons, l’absurdité la plus complète. Regardons comment apparaît cette position à la lumière de la lutte pratique.
 \subsubsection[{La lutte pour les revendications immédiates et le fascisme}]{La lutte pour les revendications immédiates et le fascisme}
\noindent Le 28 février, Thorez exprimait dans les termes suivants la même idée centrale et radicalement fausse de la politique actuelle du Parti communiste :\par

\begin{quoteblock}
 \noindent « Pour battre définitivement le fascisme, il faut de toute évidence enrayer l’offensive économique du capital contre le niveau de vie des masses travailleuses. »
 \end{quoteblock}

\noindent Pourquoi la milice ouvrière ? Pourquoi une lutte directe contre le fascisme ? Il faut tendre à élever le niveau de vie des masses et le fascisme disparaîtra comme par enchantement.\par
Hélas ! dans ces lignes, toute la perspective de la lutte prochaine est complètement défigurée, les relations réelles sont mises la tête en bas. Les capitalistes viennent au fascisme non pas selon leur bon plaisir, mais par nécessité : ils ne peuvent plus conserver la propriété privée des moyens de production qu’en menant l’offensive contre les ouvriers, en renforçant l’oppression, en semant autour d’eux la misère et le désespoir. Craignant en même temps la [{\corr riposte}] inévitable des ouvriers, les capitalistes, par l’entremise de leurs agents, excitent la petite bourgeoisie contre le prolétariat, en accusant celui-ci de rendre la crise plus longue et plus profonde, et financent les bandes fascistes pour écraser les ouvriers.\par
Si la riposte des ouvriers à l’offensive du capital se renforce demain, si les grèves deviennent plus fréquentes et plus importantes, le fascisme, à l’encontre des paroles de Thorez, ne disparaîtra pas, mais au contraire grandira deux fois plus. La croissance du mouvement gréviste provoquera une mobilisation des briseurs de grève. Tous les bandits « patriotes » entreront dans le mouvement. Des attaques quotidiennes contre les ouvriers viendront à l’ordre  du jour. Fermer les yeux là-dessus, c’est aller à une perte assurée.\par
Est-ce à dire, répliqueront Thorez et consorts, qu’il ne faut pas riposter ? (Et suivront à notre adresse les injures habituelles, par-dessus lesquelles nous passerons, comme par-dessus une flaque d’eau sale). Non, il est nécessaire de riposter. Nous n’appartenons nullement à l’école qui pense que le meilleur moyen de sauvegarde est le silence, la retraite, la capitulation. « Ne provoquez pas l’ennemi ! », « Ne vous défendez pas ! », « Ne vous armez pas ! », « Couchez-vous sur le dos, les quatre pattes en l’air ! ». Il faut chercher les théoriciens de cette école stratégique non pas chez nous, mais à la rédaction de l’\emph{Humanité !} Il est nécessaire au prolétariat de riposter s’il ne veut pas être écrasé. Mais alors aucune illusion réformiste et pacifiste n’est admissible. La lutte sera féroce. Il faut prévoir d’avance les conséquences inévitables de la riposte et s’y préparer.\par
Par son offensive actuelle, la bourgeoisie donne un caractère \emph{nouveau,} incomparablement plus aigu, à la relation entre la situation économique et la situation sociale du capitalisme pourrissant. Exactement de même les ouvriers doivent aussi donner à leur défense un caractère nouveau, qui réponde aux méthodes de l’ennemi de classe. En se défendant contre les coups économiques du capital, il faut savoir défendre en même temps ses organisations contre les bandes mercenaires du capital. Il est impossible de le faire autrement qu’à l’aide de la \emph{milice ouvrière.} Aucune affirmation verbale, aucun cri, aucune injure de l’\emph{Humanité} ne pourront infirmer cette conclusion. En particulier à l’adresse des syndicats il est nécessaire de dire : camarades, vos locaux et vos journaux seront saccagés, vos organisations réduites en poussière, si vous ne passez pas immédiatement à la création de \emph{détachements de défense syndicale} (« milice syndicale »), si  vous ne démontrez pas en fait que vous ne céderez pas au fascisme un seul pouce sans combat.
\subsubsection[{La grève générale n’est pas un jeu de cache-cache}]{La grève générale n’est pas un jeu de cache-cache}
\noindent Dans le même article (du 28 février) Thorez se plaint :\par

\begin{quoteblock}
 \noindent « Le Parti socialiste n’a pas accepté nos propositions d’une large action, \emph{grève} y compris, contre les décrets-lois toujours en vigueur. »
 \end{quoteblock}

\noindent Grève y compris ? Quelle grève ? Puisqu’il s’agit de l’abolition des décrets-lois, ce que Thorez a en vue, apparemment, ce sont non pas des grèves économiques partielles, mais la grève générale, c’est-à-dire politique. Il ne prononce pas les mots de « grève générale » pour ne pas faire remarquer qu’il ne fait que répéter notre ancienne proposition. A quelles ruses humiliantes doivent avoir recours ces pauvres gens pour masquer leurs oscillations et leurs contradictions !\par
Ce procédé est devenu, semble-t-il, une méthode. Dans la lettre ouverte du 12 mars, le Comité central du Parti communiste propose au Parti socialiste d’ouvrir contre le service militaire de deux ans une campagne décisive « par tous les moyens, y compris la \emph{grève} ». De nouveau la même formule mystérieuse ! Le Comité central a, évidemment, en vue la grève comme moyen de lutte politique, c’est-à-dire révolutionnaire. Mais pourquoi donc craint-il alors de prononcer tout haut le mot de grève générale et parle-t-il de grève tout court ? Avec qui le Comité central joue-t-il à cache-cache ? N’est-ce pas avec le prolétariat ?
\subsubsection[{La préparation de la grève générale}]{La préparation de la grève générale}
\noindent Mais si on laisse de côté ces procédés déplacés pour sauver le « prestige », il reste le fait que le Comité central  du Parti communiste propose pour la lutte contre la législation bonapartiste de Doumergue-Flandin la grève générale. Nous sommes pleinement d’accord avec cela. Mais nous exigeons que les chefs des organisations ouvrières comprennent eux-mêmes et expliquent aux masses ce que signifie dans les conditions actuelles la grève générale et comment il faut s’y préparer.\par
Déjà une simple grève économique exige d’ordinaire une organisation de combat, en particulier des \emph{piquets}. Dans les conditions de l’exacerbation actuelle de la lutte des classes, de provocation et de terreur fascistes, une sérieuse organisation de piquets est la condition vitale de tout conflit économique important. Imaginons, pourtant, que quelque chef de syndicat déclare : « Il ne faut pas de \emph{piquets,} c’est une provocation, — il suffit de l’\emph{autodéfense} des grévistes ! ». N’est-il pas évident que les ouvriers devraient conseiller amicalement à un tel « chef » d’aller à l’hôpital, sinon directement dans un asile d’aliénés. C’est que les piquets représentent précisément l’organe le plus important de l’auto-défense des grévistes !\par
Etendons ce raisonnement à la grève générale. Nous avons en vue non pas une simple manifestation, ni une grève symbolique d’une heure ou même de 24 heures, mais une opération de combat, avec le but de contraindre l’adversaire à céder. Il n’est pas difficile de comprendre quelle exacerbation terrible de la lutte des classes signifierait la grève générale dans les conditions actuelles ! Les bandes fascistes surgiraient de toutes parts comme des champignons après la pluie et tenteraient de toutes leurs forces d’apporter le trouble, la provocation et la désagrégation dans les rangs grévistes. Comment pourrait-on préserver la grève générale de victimes superflues et même d’un complet écrasement sinon à l’aide de détachements de combat ouvriers sévèrement disciplinés ? \emph{La grève} générale \emph{est une grève partielle généralisée. La milice ouvrière  est le piquet de grève généralisé.} Seuls des bavards et des fanfarons misérables peuvent [{\corr dans}] les conditions actuelles jouer avec l’idée de la grève générale, en se refusant en même temps à un travail opiniâtre pour la création de la milice ouvrière !
\subsubsection[{La grève générale dans une « situation non révolutionnaire » ?}]{La grève générale dans une « situation non révolutionnaire » ?}
\noindent Mais les malheurs du Comité central du Parti communiste ne sont pas finis.\par
La grève générale, comme le sait tout marxiste, est un des moyens de lutte les plus révolutionnaires. La grève générale ne se trouve possible que lorsque la lutte des classes s’élève au-dessus de toutes les exigences particulières et corporatives, s’étend à travers tous les compartiments des professions et des quartiers, efface les frontières entre les syndicats et les partis, entre la légalité et l’illégalité et mobilise la majorité du prolétariat, en l’opposant activement à la bourgeoisie et à l’Etat. Au-dessus de la grève générale, il ne peut y avoir que l’insurrection armée. Toute l’histoire du mouvement ouvrier témoigne que toute grève générale, quels que soient les mots d’ordre sous lesquels elle soit apparue, a une tendance interne à se transformer en conflit révolutionnaire déclaré, en lutte directe pour le pouvoir. En d’autres termes ; la grève générale n’est possible que dans les conditions d’une extrême tension politique et c’est pourquoi elle est toujours l’expression indiscutable du caractère révolutionnaire de la situation. Comment, en ce cas, le Comité central peut-il proposer la grève générale ? « La situation n’est pas révolutionnaire » !\par
Peut-être Thorez nous objectera-t-il qu’il a en vue, non pas la véritable grève générale, mais une petite grève bien docile, tout juste bonne pour l’usage propre de la rédaction  de l’\emph{Humanité ?} Ou, peut-être, ajoutera-t-il discrètement que, prévoyant le refus des chefs de la S.F.I.O., il ne risque rien en leur proposant la grève générale ? Mais le plus vraisemblable est que Thorez, en guise d’objection, nous accusera simplement de monter un complot avec Chiappe, l’ex-Alphonse XIII et le pape ; ce sont les réponses qui réussissent le mieux à Thorez !\par
Mais tout ouvrier communiste, qui a une tête sur les épaules, doit réfléchir aux contradictions criantes de ses malheureux chefs : il est impossible, voyez-vous, de bâtir la milice ouvrière parce que la situation n’est pas révolutionnaire : il est même impossible de faire de la propagande en faveur de l’armement du prolétariat, c’est-à-dire de préparer les ouvriers à la situation révolutionnaire futures mais il est possible, semble-t-il, d’appeler maintenant même les ouvriers à la grève générale, malgré l’absence de situation révolutionnaire. En vérité on dépasse ici toutes les bornes de l’étourderie et de l’absurdité !
\subsubsection[{« Les Soviets partout ! »}]{« Les Soviets partout ! »}
\noindent Dans toutes les réunions, on peut entendre les communistes répéter le mot d’ordre, qui leur est resté en héritage de la « troisième période » : « Les Soviets partout ! » Il est absolument évident que ce mot d’ordre, si on le prend au sérieux, a un caractère profondément révolutionnaire : il est impossible d’établir le régime des soviets autrement qu’au moyen de l’insurrection armée contre la bourgeoisie. Mais l’insurrection \emph{armée} suppose une \emph{arme} dans les mains du prolétariat. Ainsi le mot d’ordre des « soviets partout » et le mot d’ordre de « l’armement des ouvriers » sont étroitement et indissolublement liés l’un à l’autre. Pourquoi donc le premier mot d’ordre est-il sans cesse répété par les stalinistes et le second déclaré « provocation des trotskistes » ?\par
 Notre perplexité est d’autant plus légitime que le mot d’ordre de l’armement des ouvriers correspond beaucoup plus à la situation politique actuelle et à l’état psychologique du prolétariat. Le mot d’ordre des « soviets » a, par son essence même, un caractère offensif et suppose une révolution victorieuse. Pourtant le prolétariat se trouve actuellement dans une situation défensive. Le fascisme le menace directement d’écrasement physique. La nécessité de la défense, même les armes à la main, est actuellement plus compréhensible et plus à la portée de masses beaucoup plus larges que l’idée de l’offensive révolutionnaire. Ainsi le mot d’ordre de l’armement peut à l’étape présente compter sur un écho beaucoup plus large et beaucoup plus actif que le mot d’ordre des soviets. Comment donc un parti ouvrier peut-il, s’il n’a pas véritablement trahi les intérêts de la révolution, laisser échapper une situation si exceptionnelle et compromettre malhonnêtement l’idée de l’armement au lieu de la populariser ardemment ?\par
Nous sommes prêts à reconnaître que notre question est dictée par notre nature « contre-révolutionnaire », en particulier par nos aspirations à provoquer l’intervention militaire : on sait que dès que le mikado et Hitler se seront persuadés par notre question qu’un courant d’air souffle dans les têtes de Bela Kun et de Thorez, ils déclareront la guerre à l’U.R.S.S.\par
Tout cela a été irréfutablement établi par Duclos et n’a pas besoin de preuves. Mais, tout de même, daignez répondre : comment peut-on arriver aux soviets sans insurrection armée ? Comment arriver à l’insurrection sans armement des ouvriers ? Comment se défendre contre le fascisme sans armes ? Comment arriver à l’armement, même partiel, sans propagande pour ce mot d’ordre ?
 \subsubsection[{Mais la grève générale est-elle possible dans un proche avenir ?}]{Mais la grève générale est-elle possible dans un proche avenir ?}
\noindent A une question de ce genre, il n’y a pas de réponse à priori, c’est-à-dire toute faite d’avance. Pour avoir une réponse, il faut savoir interroger. Qui ? La masse. Comment l’interroger ? Au moyen de l’agitation.\par
L’agitation n’est pas seulement le moyen de communiquer à la masse tels ou tels mots d’ordre, d’appeler les masses à l’action, etc. L’agitation est aussi pour le parti un moyen de prêter l’oreille à la masse, de sonder son état d’esprit et ses pensées et, selon les résultats, de prendre telle ou telle décision pratique. C’est seulement les stalinistes qui ont transformé l’agitation en un monologue criard : pour les marxistes, pour les léninistes, \emph{l’agitation est toujours un dialogue avec la masse}.\par
Mais pour que ce dialogue donne les résultats nécessaires, le parti doit apprécier correctement la situation générale dans le pays et tracer la voie générale de la lutte prochaine. A l’aide de l’agitation et du sondage de la masse, le parti doit apporter dans sa conception les corrections et les précisions nécessaires, en particulier dans tout ce qui concerne le \emph{rythme du mouvement et les dates des grandes actions}.\par
La situation dans le pays a été définie plus haut : elle a un caractère pré-révolutionnaire avec le caractère non-révolutionnaire de la direction du prolétariat. Et puisque la politique du prolétariat est le principal facteur dans le développement d’une situation révolutionnaire, le caractère non-révolutionnaire de la direction prolétarienne entrave la transformation de la situation pré-révolutionnaire en situation révolutionnaire déclarée, et par cela même contribue à la transformer en situation contre-révolutionnaire.\par
Dans la réalité objective il n’y a pas, bien entendu, de strictes délimitations entre les différents stades du processus  politique. Une étape s’insère dans l’autre, en résultat de quoi la situation révèle diverses contradictions. Ces contradictions, assurément, rendent plus difficiles le diagnostic et le pronostic, mais ne les rendent nullement impossibles.\par
\emph{Les forces du prolétariat français non seulement ne sont pas épuisées, mais sont même intactes. Le fascisme comme fadeur politique dans les masses petites-bourgeoises est encore relativement faible} (quoique beaucoup plus puissant, malgré tout, qu’il semble aux parlementaires). Ces deux faits politiques très importants permettent de dire avec une ferme conviction : rien n’est encore perdu, la possibilité de transformer la situation pré-révolutionnaire en situation révolutionnaire est encore complètement ouverte.\par
Or, dans un pays capitaliste comme la France, il ne peut y avoir de luttes révolutionnaires sans grève générale : si les ouvriers et les ouvrières, pendant les journées décisives, restent dans les usines, qui donc se battra ? La grève générale s’inscrit ainsi à l’ordre du jour.\par
Mais la question du \emph{moment} de la grève générale est la question de savoir si les masses sont prêtes à lutter et si les organisations ouvrières sont prêtes à les mener au combat.
\subsubsection[{Les masses veulent-elles lutter ?}]{Les masses veulent-elles lutter ?}
\noindent Est-il vrai pourtant qu’il ne manque que la direction révolutionnaire ? N’y a-t-il pas une grande force de conservatisme dans les masses elles-mêmes, dans le prolétariat ? Des voix s’élèvent de différents côtés. Et ce n’est pas étonnant ! Quand approche une crise révolutionnaire, de nombreux chefs, qui craignent les responsabilités, se cachent derrière le pseudo-conservatisme des masses. L’histoire nous enseigne que quelques semaines et même  quelques jours avant l’insurrection d’Octobre, des bolcheviks marquants comme Zinoviev, Kaménev, Rykov (de certains comme Losovsky, Manouilsky, etc., inutile de parler) affirmaient que les masses étaient fatiguées et ne voulaient pas se battre. Et pourtant Zinoviev, Kaménev et Rykov, comme révolutionnaires, étaient cent coudées au-dessus des Cachin, Thorez et Monmousseau.\par
Celui qui dit que notre prolétariat ne veut ou n’est pas capable de mener la lutte révolutionnaire, celui-là lance une calomnie, en reportant sa propre mollesse et sa propre lâcheté sur les masses laborieuses. \emph{Jusqu’à maintenant il n’y a eu aucun cas ni à Paris ni en province où les masses soient restées sourdes à l’appel d’en-haut.}\par
Le plus grand exemple est la grève générale du 12 février 1934. Malgré la complète division de la direction, l’absence de toute préparation sérieuse, les efforts tenaces des chefs de la C.G.T. de réduire le mouvement au minimum, puisque ne pouvant pas l’éviter complètement, la grève générale eut le succès le plus grand qu’elle put avoir dans les conditions données. C’est clair : les masses voulaient combattre. Tout ouvrier conscient doit se dire : la pression de la base doit être bien puissante, si Jouhaux lui-même est sorti pour un moment de l’immobilité. Certes, il ne s’agissait pas d’une grève générale au sens propre, mais seulement d’une manifestation de 24 heures. Mais cette limitation \emph{ne} fut \emph{pas} apportée par les \emph{masses :} elle fut dictée \emph{d’en-haut.}\par
La manifestation de la place de la République, le 10 février de cette année, confirme la même conclusion. Le seul instrument qu’aient utilisé les centres dirigeants pour la préparation fut la lance de pompiers. Le seul mot d’ordre que les masses entendirent fut : Chut ! Chut ! Et, néanmoins, le nombre des manifestants dépassa toutes les attentes. En province, la chose s’est présentée et se présente dans la dernière année absolument de la même  façon. Il est impossible d’apporter un seul fait sérieux qui témoignerait que les chefs veulent lutter et que les masses se refuseraient à les suivre. Toujours et partout, on observa un rapport absolument inverse. Il garde toute sa force encore aujourd’hui. La base veut lutter, les sommets freinent. C’est là le principal danger et il \emph{peut} aboutir à une véritable catastrophe.
\subsubsection[{Les bases et les sommets à l’intérieur des partis}]{Les bases et les sommets à l’intérieur des partis}
\noindent Le même rapport se retrouve, non seulement entre les partis (ou les syndicats) et le prolétariat, mais aussi à l’intérieur de chacun des partis. Ainsi. \emph{Frossard} n’a pas, dans la base de la S.F.I.O., le moindre appui : seuls, le soutiennent les députés et les maires, qui veulent que tout reste comme par le passé. Au contraire, \emph{Marceau Pivert}, grâce à ses interventions de plus en plus claires et résolues, devient dans la base une des figures les plus populaires. Nous le reconnaissons d’autant plus volontiers que nous n’avons jamais renoncé dans le passé, comme nous n’y renoncerons pas dans l’avenir, à dire ouvertement quand nous ne sommes pas d’accord avec Pivert.\par
Pris comme symptôme politique, ce fait dépasse pourtant de loin, par son importance, la question personnelle de Frossard et de Pivert : il montre la tendance générale du développement. \emph{La base du Parti socialiste, comme du Parti communiste, est plus à gauche, plus révolutionnaire, plus hardie que les sommets :} c’est précisément pourquoi elle est prête à donner sa confiance seulement aux chefs de gauche. Plus encore : elle pousse les socialistes sincères toujours plus à gauche. Pourquoi donc la base se radicalise-t-elle elle-même ? Parce qu’elle se trouve en contact direct avec les masses populaires, avec leur misère, avec leur révolte, avec leur haine. Ce symptôme est infaillible. On peut se fier à lui.
 \subsubsection[{Les « revendications immédiates » et la radicalisation des masses}]{Les « revendications immédiates » et la radicalisation des masses}
\noindent Les chefs du Parti communiste peuvent, certes, invoquer le fait que les masses ne font pas écho à leurs appels. Or, ce fait n’infirme pas, mais confirme notre analyse. Les masses ouvrières comprennent ce que ne comprennent pas les « chefs », c’est-à-dire : dans les conditions d’une très grande crise sociale une seule lutte économique partielle, qui exige d’énormes efforts et d’énormes sacrifices, ne peut pas donner de résultats sérieux. Pis encore : elle peut affaiblir et épuiser le prolétariat. Les ouvriers sont prêts à participer à des manifestations de lutte et même à la grève générale, mais non pas à de petites grèves épuisantes sans perspective. Malgré les appels, les manifestes et les articles de l’\emph{Humanité}, les agitateurs communistes n’apparaissent presque nullement devant les masses en prêchant des grèves au nom des « revendications partielles immédiates ». Ils sentent que les plans bureaucratiques des chefs ne correspondent absolument pas à la situation objective, ni à l’état d’esprit des masses. Sans grande perspective, les masses ne pourront et ne commenceront à lutter. La politique de l’\emph{Humanité} est \emph{la politique} d’un \emph{pseudo-« réalisme » artificiel et faux}. L’insuccès de la C.G.T.U. dans la déclaration de grèves partielles est la confirmation indirecte, mais très réelle de la profondeur de la crise et de la tension morale dans les quartiers ouvriers.\par
Il ne faut pourtant pas croire que la radicalisation des masses continuera d’elle-même, automatiquement. La classe ouvrière attend une initiative de ses organisations. Quand elle en sera venue à la conclusion que ses attentes sont trompées — et cette heure n’est, peut-être, pas si loin — le processus de radicalisation se brisera, se transformera en manifestations de découragement, de prostration,  en des explosions isolées de désespoir. A la périphérie du prolétariat, des tendances anarchistes côtoieront des tendances fascistes. Le vin se sera changé en vinaigre.\par
Les changements de l’état d’esprit politique des masses exigent la plus grande attention. Sonder cette dialectique vivante à chaque étape — c’est la tâche de l’agitation. Actuellement, le Front unique reste criminellement à la fois en retard sur le développement de la crise sociale et sur l’état d’esprit des masses. Il est encore possible de rattraper le temps perdu. Mais il ne faut plus perdre de temps. L’histoire compte maintenant non pas par années, mais par mois et par semaines.
\subsubsection[{Le programme de la grève générale}]{Le programme de la grève générale}
\noindent Pour déterminer à quel degré les masses sont prêtes à la grève générale et en même temps renforcer l’état d’esprit combattif des masses, il faut mettre devant elles un programme d’action révolutionnaire. Des mots d’ordre partiels tels que l’abolition des décrets-lois bonapartistes et du service militaire de deux ans, trouveront, assurément, dans ce programme une place marquante. Mais ces deux mots d’ordre épisodiques sont absolument insuffisants.\par
\emph{Au-dessus de toutes les tâches et revendications partielles de notre époque} se trouve \emph{la QUESTION DU POUVOIR}. Depuis le 6 février 1934, la question du pouvoir est posée ouvertement comme une question de force. Les élections municipales et parlementaires peuvent avoir leur importance, en tant qu’évaluation des forces — pas plus. La question sera tranchée par le conflit déclaré des deux camps. Les gouvernements du type Doumergue. Flandin, etc., n’occupent l’avant-scène que jusqu’au jour du dénouement définitif. Demain, ce sera ou bien le fascisme ou bien le prolétariat qui gouvernera la France.\par
 C’est précisément parce que le régime étatique intermédiaire actuel est extrêmement instable, que la grève générale \emph{peut} donner de très grands succès partiels, en contraignant le gouvernement à en venir à des concessions dans la question des décrets-lois bonapartistes, du service militaire de deux ans, etc. Mais un tel succès, extrêmement précieux et important en lui-même, ne rétablira pas l’équilibre de la « démocratie » : le capital financier redoublera les subsides au fascisme et la question du pouvoir, peut-être après une courte pause, se posera avec une force redoublée.\par
L’importance fondamentale de la grève générale, indépendamment des succès partiels qu’elle peut donner, mais aussi ne pas donner, est dans le fait qu’elle pose d’une façon révolutionnaire la question du pouvoir. Arrêtant les usines, les transports, en général tous les moyens de liaison, les stations électriques, etc., le prolétariat paralyse par cela même non seulement la production, mais aussi le gouvernement. Le pouvoir étatique reste suspendu en l’air. Il doit, soit dompter le prolétariat par la faim et par la force, en le contraignant à remettre de nouveau en mouvement la machine de l’Etat bourgeois, soit céder la place devant le prolétariat.\par
Quels que soient les mots d’ordre et le motif pour lesquels la grève générale ait surgi, si elle embrasse les véritables masses et si ces masses sont bien décidées à lutter, la grève générale pose inévitablement devant toutes les classes de la nation la question : \emph{qui va être le maître de la maison ?}\par
Les chefs du prolétariat doivent comprendre cette logique interne de la grève générale, sinon ce ne sont pas des chefs, mais des dilettantes et des aventuriers. Politiquement, cela signifie : les chefs sont tenus dès maintenant de poser devant le prolétariat le problème de la conquête révolutionnaire du pouvoir. Sinon, ils ne doivent  pas se hasarder à parler de grève générale. Mais en renonçant à la grève générale, ils renoncent par cela même à toute lutte révolutionnaire, c’est-à-dire ils livrent le prolétariat au fascisme.\par
\emph{Ou la capitulation complète ou la lutte révolutionnaire pour le pouvoir} — telle est l’alternative, qui découle de toutes les conditions de la crise actuelle. Celui qui n’a pas compris cette alternative n’a rien à faire dans le camp du prolétariat.
\subsubsection[{La grève générale et la C.G.T.}]{La grève générale et la C.G.T.}
\noindent La question de la grève générale se complique par le fait que la C.G.T. proclame son monopole à déclarer et à conduire la grève générale. Il en résulte que cette question ne regarde pas du tout les partis ouvriers. Et ce qui est, à première vue, le plus étonnant, c’est qu’il se trouve des parlementaires socialistes, qui pensent que cette prétention est dans l’ordre des choses ; en vérité, ils veulent simplement se débarrasser de cette responsabilité.\par
La grève générale, comme l’indique déjà son nom, a pour but d’embrasser, autant que possible, tout le prolétariat. La C.G.T. ne réunit dans ses rangs probablement pas plus de 5 à 8 \% du prolétariat. L’influence propre de la C.G.T. en dehors des limites des syndicats est absolument insignifiante, dans la mesure où, dans telle ou telle question, elle ne coïncide pas avec l’influence des partis ouvriers. Peut-on, par exemple, comparer l’influence du \emph{Peuple} à l’influence du \emph{Populaire} ou de l’\emph{Humanité)}\par
La direction de la C.G.T., par ses conceptions et ses méthodes, est encore incomparablement plus loin des tâches de l’époque actuelle que la direction des partis ouvriers. Plus on va des sommets de l’appareil vers la base des syndiqués, moins il y a de confiance en Jouhaux et en son groupe. Le manque de confiance se change de plus  en plus en méfiance active. L’appareil conservateur actuel de la C.G.T. sera inévitablement balayé par le développement ultérieur de la crise révolutionnaire.\par
La grève générale est, par son essence même, une opération politique. Elle oppose la classe ouvrière dans son ensemble à l’Etat bourgeois. Elle rassemble les ouvriers syndiqués et non-syndiqués, socialistes, communistes et sans-parti. Elle a besoin d’un appareil de presse et d’agitateurs tel que la C.G.T. seule n’en dispose absolument pas.\par
La grève générale pose carrément la question de la conquête du pouvoir par le prolétariat. La C.G.T. tourna et tourne le dos à cette tâche (les chefs de la C.G.T. tournent la face vers le pouvoir bourgeois). Les chefs de la C.G.T. eux-mêmes sentent, assurément, que la direction de la grève générale est au-dessus de leurs forces. S’ils proclament, néanmoins, leur monopole à la diriger, c’est uniquement parce qu’ils espèrent par cette voie étouffer la grève générale avant même sa naissance.\par
Et la grève générale du 12 février 1934 ? Ce ne fut qu’une brève et pacifique démonstration, imposée à la C.G.T. par les ouvriers socialistes et communistes. Jouhaux et consorts ont pris sur eux la direction formelle de la riposte précisément pour l’empêcher de se transformer en grève générale révolutionnaire.\par
Dans les instructions à ses propagandistes, la C.G.T. communiquait : « Au lendemain du 6 février, la population laborieuse et \emph{tous les démocrates,} à l’appel de la C.G.T., ont manifesté leur ferme volonté de barrer la route aux \emph{factieux} ». A part elle-même, la C.G.T. n’a noté ni les socialistes ni les communistes — seulement les « démocrates ». Dans cette seule phrase, c’est tout Jouhaux. C’est précisément pourquoi il serait criminel de faire confiance à Jouhaux pour trancher la question de savoir s’il faut ou non la lutte révolutionnaire.\par
 Bien entendu, dans la préparation et la conduite de la grève générale, les syndicats auront un rôle très influent ; mais non pas en vertu d’un monopole, mais côte à côte avec les partis ouvriers. Du point de vue révolutionnaire, il est particulièrement important de collaborer étroitement avec les organisations syndicales \emph{locales,} sans la moindre atteinte, bien entendu, à leur autonomie.\par
Quant à la C.G.T., il lui faudra, soit prendre place dans le front commun prolétarien, en se détachant des « démocrates », soit rester à l’écart. Collaborer loyalement à droits égaux ? Oui. Examiner en commun les délais et les moyens de la conduite de la grève générale ? Oui ! Reconnaître le monopole de Jouhaux d’étouffer le mouvement révolutionnaire ? Jamais !
\subsection[{IV. — Socialisme et lutte armée}]{IV. — Socialisme et lutte armée}
\subsubsection[{La grande leçon du 6 février 1935}]{La grande leçon du 6 février 1935}
\noindent Ce jour-là — le 6 février 1935 — les ligues fascistes avaient projeté de manifester place de la Concorde. Que fait donc le Front unique et, en particulier, le Comité central du Parti communiste ? Il appelle les ouvriers de Paris à manifester à la Concorde en même temps que les fascistes. Peut-être les fascistes seraient-ils sans armes ? Non, depuis un an, ils s’arment de façon redoublée. Peut-être le Comité central du Parti communiste allait-il armer suffisamment de détachements de défense ? Oh ! non, le Comité central est contre le « putschisme » et la « lutte physique ». Comment est-il donc possible de lancer des dizaines de milliers d’ouvriers sans armes, sans préparation, sans défense contre des bandes fascistes admirablement organisées et armées, qui nourrissent une haine sanglante pour le prolétariat révolutionnaire ?\par
 Que les malins ne nous disent pas : le Comité central du Parti communiste ne se disposait nullement à jeter les ouvriers sous les revolvers des fascistes ; il voulait seulement donner à Flandin un prétexte convenable pour interdire la manifestation fasciste. Mais c’est encore pis. Le Comité central du Parti communiste, apparaît-il donc, jouait avec les têtes des ouvriers, et l’issue de ce jeu dépendait entièrement de Flandin, plus exactement, des chefs de la police de l’école de Chiappe. Et que serait-il arrivé si à la préfecture de police on avait décidé de profiter de la bonne occasion et de donner une leçon aux ouvriers révolutionnaires par l’entremise des fascistes, en faisait de plus retomber la responsabilité de la boucherie sur les chefs du Front unique ? Il n’est pas difficile de se représenter les conséquences ! Si le massacre sanglant ne s’est pas produit cette fois-ci, en cas de continuation de la même politique, il se produira inévitablement, infailliblement à la première occasion semblable.
\subsubsection[{« Putschisme » et aventurisme}]{« Putschisme » et aventurisme}
\noindent La conduite du Comité central fut la plus pure forme d’aventurisme bureaucratique. Les marxistes ont toujours enseigné que \emph{l’opportunisme et l’aventurisme représentent les deux faces de la même médaille}. Le 6 février 1935, avec une clarté remarquable, nous montre avec quelle facilité la médaille se retourne.\par
« Nous sommes contre le putschisme, contre l’insurrectionnalisme ! » répéta pendant des années Otto Bauer, qui ne sut pas se débarrasser du Schutzbund (milice ouvrière), laissé en héritage par la révolution de 1918. La puissante social-démocratie autrichienne recula lâchement, s’adapta à la bourgeoisie, recula de nouveau, lança des « pétitions » ineptes, créa une fausse apparence de lutte, mit ses espoirs en son Flandin (il avait nom Dollfuss),  céda position sur position, et quand elle se vit au fond de l’impasse ! elle se mit à crier hystériquement ; « Ouvriers, au secours ! ». Les meilleurs combattants, sans liaison avec les masses désorientées, accablées, trompées, se jetèrent dans le combat et subirent une défaite inévitable. Après quoi, Otto Bauer et Julius Deutsch déclarèrent : « \emph{Nous} avons agi en révolutionnaires, mais le \emph{prolétariat} ne nous a pas soutenus ! ».\par
Les événements d’Espagne se sont déroulés selon le même schéma. Les chefs social-démocrates ont appelé les ouvriers à l’insurrection après qu’ils eurent cédé à la bourgeoisie toutes les positions révolutionnaires conquises et lassé les masses populaires par leur politique de reculade. Les « antiputschistes » professionnels se sont trouvés contraints d’appeler à 1a défense armée dans des conditions telles qu’ils lui avaient donné pour une large part le caractère d’un « putsch ».\par
\emph{Le 6 février 1935 fut en France une petite répétition des événements d’Autriche et d’Espagne.} Pendant de nombreux mois, les stalinistes ont endormi et démoralisé les ouvriers, ridiculisé le mot d’ordre de la milice et « repoussé » la lutte physique, puis tout d’un coup, sans la moindre préparation, ils ont commandé au prolétariat : « A la Concorde, en avant, marche ! ». Pour cette fois, le bon Langeron les sauva. Mais si demain, quand l’atmosphère sera encore plus ardente, quand les voyous fascistes assassineront des dizaines de chefs ouvriers ou incendieront l’\emph{Humanité} — qui dit que cela est invraisemblable ? — le sage Comité central criera infailliblement : « Ouvriers, aux Armes ! ». Et puis, mis dans un camp de concentration ou se promenant dans les rues de Londres s’ils arrivent jusque-là, les mêmes chefs déclareront fièrement : « Nous avons appelé à l’insurrection, mais les ouvriers ne nous ont pas soutenus ! ».
 \subsubsection[{Il faut prévoir et se préparer}]{Il faut prévoir et se préparer}
\noindent Le secret du succès, évidemment, n’est pas dans la « lutte physique » elle-même, mais dans une juste politique. Or, nous appelons juste la politique qui répond aux conditions du temps et du lieu. \emph{En soi}, la milice ouvrière ne résout pas le problème. Mais la milice ouvrière est une \emph{partie intégrante nécessaire} de la politique, qui répond aux conditions du temps et du lieu. Il serait absurde de tirer à coups de revolver sur l’urne électorale. Mais il serait encore plus absurde de se défendre contre les bandes fascistes avec le bulletin de vote.\par
Les premiers noyaux de la milice ouvrière se trouvent inévitablement faibles, isolés, inexpérimentés. Les routiniers et les sceptiques secouent la tête avec mépris. Il se trouve des cyniques qui n’ont pas honte de railler l’idée de la milice ouvrière dans un entretien avec les journalistes du Comité des Forges. S’ils pensent s’assurer ainsi contre les camps de concentration, ils se trompent. L’impérialisme se fiche de l’avilissement de tel ou tel chef ; il lui faut écraser la classe.\par
Quand Guesde et Lafargue, tout jeunes, entreprirent la propagande du marxisme, ils passèrent aux yeux des sages philistins pour des solitaires impuissants et des utopistes naïfs. Néanmoins, ce sont eux qui ont creusé le lit de ce mouvement, qui porte sur lui tant de routiniers parlementaires. Dans le domaine littéraire, syndical, coopératif les premiers pas du mouvement ouvrier furent faibles, chancelants, peu assurés. Mais malgré sa pauvreté, le prolétariat, grâce à son nombre et à son esprit de sacrifice, a créé de puissantes organisations.\par
L’\emph{organisation armée du prolétariat,} qui au moment présent coïncide presque complètement avec la \emph{défense contre le fascisme}, est une nouvelle branche de la lutte de classes. Les premiers pas sont ici aussi inexpérimentés,  maladroits. Il faut s’attendre à des fautes. Il est même impossible d’éviter complètement la provocation. La sélection des cadres s’obtiendra peu à peu et cela d’autant plus sûrement, d’autant plus solidement que la milice sera plus près des usines, là où les ouvriers se connaissent bien l’un l’autre.\par
Mais l’initiative doit partir nécessairement d’en-haut. \emph{Le parti peut et doit donner les premiers cadres}. Sur la même voie doivent aussi se mettre — et ils s’y mettront inévitablement — les syndicats. Ces cadres se souderont et se renforceront d’autant plus vite qu’ils rencontreront une plus grande sympathie et un plus grand soutien dans les organisations ouvrières, puis dans la masse des travailleurs.\par
Que dire de ces messieurs, qui, en guise de sympathie et de soutien, apportent blâme et raillerie ou, pis encore, représentent devant l’ennemi de classe des détachements d’auto-défense ouvrière comme des détachements d’ « insurrection » et de « putsch » ? Regardez, en particulier le « Combat (?) Marxiste (!) ». Des pédants savants et à demi savants, des adjudants théoriciens de Jouhaux, dirigés par les menchéviks russes, raillent méchamment les premiers pas de la milice ouvrière. Il est impossible de donner à de tels messieurs d’autre nom que celui d’ennemis directs de la révolution prolétarienne.
\subsubsection[{La milice ouvrière et l’armée}]{La milice ouvrière et l’armée}
\noindent Mais ici les routiniers conservateurs lancent leur dernier argument : « Est-ce que vous pensez qu’à l’aide de détachements de milice mal armés le prolétariat pourra conquérir le pouvoir, c’est-à-dire remporter la victoire sur l’armée actuelle, avec sa technique moderne (les tanks ! l’aviation !! les gaz !!!) ?... ». Il est difficile d’imaginer un argument plus plat et plus trivial, d’ailleurs cent fois  contredit par la théorie et par l’histoire. Néanmoins on le présente chaque fois comme le dernier mot d’une pensée « réaliste ».\par
Si on admet même pour un instant que les détachements de la milice se révéleront demain inaptes dans la lutte pour le pouvoir, ils n’en sont pas moins nécessaires \emph{aujourd’hui} pour la défense des organisations ouvrières. Les chefs de la C.G.T. se refusent, comme on sait, à toute lutte pour le pouvoir. Cela n’arrêtera nullement les fascistes devant l’écrasement de la C.G.T. Les syndicalistes, qui ne prennent pas à temps des mesures de défense, commettent un crime contre les syndicats, indépendamment de leur orientation politique.\par
Considérons de plus près, pourtant, l’argument capital des pacifistes : « Les détachements armés d’ouvriers sont impuissants contre l’armée contemporaine ». Cet « argument » est dirigé, au fond, non pas contre la milice, mais \emph{contre l’idée même de révolution prolétarienne}. Si on admet pour un instant que l’armée outillée jusqu’aux dents se trouvera \emph{dans toutes les conditions} du côté du grand capital, alors il faut renoncer non seulement à la milice ouvrière, mais au socialisme en général. Alors le capitalisme est éternel.\par
Heureusement, il n’en est pas ainsi. La révolution prolétarienne suppose une exacerbation extrême de la lutte des classes, à la ville et au village, et par conséquent aussi dans l’armée. La révolution ne remportera la victoire que lorsqu’elle aura conquis à elle ou, au moins, neutralisé le noyau fondamental de l’armée. Cette conquête. pourtant ne peut s’improviser : il faut la préparer systématiquement.\par
Ici, le doctrinaire pacifiste nous interrompt pour tomber en paroles — d’accord avec nous. « Evidemment — dira-t-il — il faut conquérir l’armée au moyen d’une propagande continuelle. Or, c’est ce que nous faisons.  La lutte contre la grande mortalité dans les casernes, contre les deux ans, contre la guerre — le succès de cette lutte rend inutile l’armement des ouvriers. »\par
Cela est-il vrai ? Non, c’est radicalement faux. Une conquête pacifique, sereine de l’armée est encore moins possible que la conquête pacifique d’une majorité parlementaire. Déjà, les campagnes très modérées contre la mortalité dans les casernes et contre les deux ans vont sans aucun doute conduire à un rapprochement entre les ligues patriotiques et les officiers réactionnaires, à un complot direct de leur part et aussi au versement redoublé des subsides que le capital financier donne aux fascistes. \emph{Plus l’agitation antimilitariste aura de succès, plus le danger fasciste croîtra rapidement}. Telle est la dialectique réelle et non inventée de la lutte. La conclusion est que dans le processus même de la propagande et de la préparation il faut savoir se défendre les armes à la main, et de mieux en mieux.
\subsubsection[{Pendant la révolution}]{Pendant la révolution}
\noindent Pendant la révolution se produiront dans l’armée des oscillations inévitables, une lutte intérieure s’y mènera. Même les fractions les plus avancées ne passeront ouvertement et activement du côté du prolétariat que si elles voient de leurs yeux que \emph{les ouvriers veulent se battre et sont capables de vaincre.} La tâche des détachements fascistes sera de ne pas permettre le rapprochement entre le prolétariat révolutionnaire et l’armée. Les fascistes s’effaceront d’écraser l’insurrection ouvrière dès son début pour enlever aux meilleures fractions de l’armée la possibilité de soutenir les insurgés. En même temps les fascistes viendront en aide aux détachements réactionnaires de l’armée pour désarmer les régiments les plus révolutionnaires et les moins sûrs.\par
 Quelle sera en ce cas notre tâche ?\par
Il est impossible de dire par avance la marche concrète de la révolution dans un pays donné. Mais on peut, sur la base de toute l’expérience de l’histoire, affirmer avec certitude que l’insurrection en aucun cas et dans aucun pays ne prendra le caractère d’un simple duel entre la milice ouvrière et l’armée. Le rapport des forces sera bien plus complexe et incomparablement plus favorable au prolétariat. La \emph{milice ouvrière} — non par son armement, mais par sa conscience et son héroïsme — sera l’\emph{avant-garde de la révolution.} Le \emph{fascisme} sera l’\emph{avant-garde de la contre-révolution.} La milice ouvrière, avec le soutien de toute la classe, avec la sympathie de tous les travailleurs, devra battre, désarmer, terroriser les bandes de brigands de la réaction et ouvrir ainsi aux ouvriers la voie vers la \emph{fraternisation révolutionnaire avec l’armée}. L’alliance des ouvriers et des soldats viendra à bout des fractions contre-révolutionnaires. Ainsi sera assurée la victoire.\par
Les sceptiques hausseront les épaules avec mépris. Mais les sceptiques font le même geste à la veille de toute révolution victorieuse. Le prolétariat fera bien de prier à temps les sceptiques de s’en aller bien loin. Le temps est trop précieux pour expliquer la musique aux sourds, les couleurs aux aveugles et aux sceptiques la révolution socialiste.
\subsection[{V. — Le prolétariat, les paysans, l’armée, les femmes, les jeunes}]{V. — Le prolétariat, les paysans, l’armée, les femmes, les jeunes}
\subsubsection[{Le Plan de la C.G.T. et le Front unique}]{Le Plan de la C.G.T. et le Front unique}
\noindent Jouhaux a emprunté l’idée du Plan à de Man. Chez tous deux le but est le même : masquer le dernier \emph{krach  du réformisme} et inspirer au prolétariat de nouveaux espoirs, pour le détourner de la révolution.\par
Ni de Man, ni Jouhaux n’ont inventé leurs « plans ». Ils ont pris tout simplement les revendications fondamentales du \emph{programme de transition marxiste}, la nationalisation des banques et des industries-clés, ont jeté par-dessus bord la lutte de classes et à la place de l’expropriation révolutionnaire des expropriateurs ils ont mis une opération financière de \emph{rachat}.\par
Le pouvoir doit comme auparavant rester dans les mains du « peuple », c’est-à-dire de la bourgeoisie. Mais l’Etat rachète les plus importantes branches de l’industrie (on ne nous dit pas lesquelles exactement) à leurs propriétaires actuels, qui deviennent pour deux ou trois générations des rentiers parasitaires : la pure et simple exploitation privée capitaliste fait place à une exploitation indirecte, par l’intermédiaire d’un capitalisme d’Etat.\par
Comme Jouhaux comprend que même ce programme émasculé de nationalisation est absolument irréalisable sans lutte révolutionnaire, il déclare par avance qu’il est prêt à \emph{changer son} « \emph{plan} » en \emph{petite monnaie} de réformes parlementaires dans le style, à la mode, de l’économie dirigée. L’idéal pour Jouhaux serait qu’au moyen d’arrangements dans les coulisses il réduise toute opération à ce que dans différents conseils économiques et industriels siègent des bureaucrates syndicaux, sans pouvoir et sans autorité, mais avec des jetons de présence.\par
Ce n’est pas pour rien que le plan de Jouhaux — son plan réel qu’il cache derrière le « Plan » de papier — a reçu le soutien des néos et même l’approbation de Herriot !\par
Le sage idéal du syndicalisme « indépendant » ne sera, pourtant, réalisé que si le capitalisme se régénère de nouveau et si les masses ouvrières retombent sous le joug. Mais si le déclin capitaliste se poursuit ? Alors le  plan, lancé pour détourner les ouvriers de « mauvaises pensées », peut devenir le drapeau du mouvement révolutionnaire.\par
En Belgique, ce danger existe déjà. Le P.O.B. s’est trouvé contraint de mener une agitation pour le plan de Man. Les ouvriers ont pris le plan tout à fait au sérieux. Sous le drapeau du plan une aile gauche s’est mise à se former, en particulier au sein de la jeunesse. Le falsificateur théorique de Man ressemble de plus en plus au sorcier qui a invoqué les esprits, mais qui ne sait pas comment les faire rentrer dans l’au-delà. Les bolchéviks-léninistes belges ont pleinement raison de se placer sur le terrain du mouvement de masse pour le plan pour, au moyen de la critique marxiste, le faire aller de l’avant.\par
Evidemment, effrayé par l’exemple belge, Jouhaux s’est empressé de reculer. Le point le plus important de l’ordre du jour du Comité national de la C.G.T., au milieu de mars, — la propagande pour le plan — s’est trouvé inopinément escamoté. Si cette manœuvre a plus ou moins réussi, la faute en retombe entière sur la direction du Front unique.\par
Les chefs de la C.G.T. ont lancé leur « Plan » pour avoir la possibilité de concurrencer les partis de la révolution. Par là, Jouhaux a montré qu’à la suite de ses inspirateurs bourgeois il apprécie la situation comme révolutionnaire (dans le sens large du mot). Mais l’\emph{adversaire révolutionnaire n’est pas apparu sur l’arène.} Jouhaux décida de ne pas s’engager plus loin dans une voie pleine de risques. Il recula et maintenant il attend.\par
En janvier, la C.A.P. du Parti socialiste proposa au Parti communiste une lutte commune pour le pouvoir au nom de la socialisation des banques et des branches concentrées de l’industrie. Si dans le Comité central du Parti communiste avaient siégé des révolutionnaires, ils auraient dû saisir cette proposition des deux mains. En ouvrant une  large campagne pour le pouvoir, ils auraient accéléré la mobilisation révolutionnaire à l’intérieur de la S.F.I.O. et en même temps auraient contraint Jouhaux à mener de l’agitation pour son « Plan ». En suivant cette voie, on aurait pu forcer la C.G.T. à prendre sa place dans le Front unique. Le poids spécifique du prolétariat français se serait accru de plusieurs fois.\par
Mais dans le Comité central du Parti communiste siègent non pas des révolutionnaires, mais des mandarins. « Il n’y a pas de situation révolutionnaire », répondirent-ils, en contemplant leur nombril. Les réformistes de la S.F.I.O. respirèrent de soulagement : le danger était passé. Jouhaux se hâta de retirer de l’ordre du jour la question de la propagande pour le Plan. Le prolétariat est resté dans la grande crise sociale \emph{sans aucun programme}. L’Internationale communiste a joué encore une fois un rôle réactionnaire.
\subsubsection[{Alliance révolutionnaire avec la paysannerie}]{Alliance révolutionnaire avec la paysannerie}
\noindent La crise de l’agriculture constitue maintenant le principal réservoir des tendances bonapartistes et fascistes. Quand la misère prend le paysan à la gorge, il est capable de faire les sauts les plus inattendus. Il regarde la démocratie avec une méfiance croissante.\par
« Le mot d’ordre de la défense des libertés démocratiques — écrit Monmousseau (\emph{Cahiers du Bolchévisme}, 1\textsuperscript{er} septembre 1934, page 1017) — correspond parfaitement à l’esprit de la paysannerie. » Cette phrase remarquable montre que Monmousseau comprend aussi peu la question paysanne que la question syndicale. Les paysans commencent à tourner le dos aux partis de « gauche », précisément parce que ceux-ci sont incapables de leur proposer rien d’autres que des paroles en l’air sur la « défense de la démocratie ».\par
 Aucun programme de « revendications immédiates » ne peut donner quelque chose de sérieux au village. Le prolétariat doit parler avec les paysans le \emph{langage de la révolution :} il ne trouvera pas d’autre langue commune. Les ouvriers doivent élaborer un programme de \emph{mesures révolutionnaires pour le salut de l’agriculture} en commun avec les paysans.\par
Les paysans craignent surtout la \emph{guerre}. Peut-être, avec Laval et Litvinov, allons-nous les leurrer d’espoirs en la Société des Nations et dans le « désarmement » ? Le seul moyen d’éviter la guerre, c’est de renverser sa propre bourgeoisie et de donner le signal de la transformation de l’Europe en \emph{Etats-Unis des Républiques Ouvrières et Paysannes.} Sans révolution, point de salut contre la guerre.\par
Les paysans travailleurs souffrent des conditions usuraires du \emph{crédit.} Pour changer ces conditions, il n’y a qu’une voie : exproprier les banques, les concentrer dans les mains de l’Etat ouvrier et, sur le compte des requins financiers, créer un \emph{crédit de faveur} pour les petits paysans, pour les coopératives paysannes en particulier. Sur les banques de crédit agricole doit être instauré le \emph{contrôle paysan}.\par
Les paysans souffrent de l’exploitation des trusts d’engrais et de la meunerie. Il n’y a pas d’autre voie que de \emph{nationaliser les trusts d’engrais et la grande meunerie} et de les subordonner complètement aux intérêts des paysans et des consommateurs.\par
Différentes catégories de paysans (fermiers, métayers) souffrent de l’exploitation des grands propriétaires fonciers. Il n’y a pas d’autre moyen de lutte contre l’usure \emph{foncière} que l’expropriation des usuriers fonciers par les \emph{comités de paysans} sous le contrôle de l’Etat ouvrier et paysan.\par
Aucune de ces mesures n’est concevable sous la domination  de la bourgeoisie. De petites aumônes ne sauveront pas le paysan, des palliatifs ne lui serviront de rien. Il faut des mesures révolutionnaires hardies. Le paysan les comprendra, les approuvera et les soutiendra, si l’ouvrier lui propose sérieusement de \emph{lutter en commun pour le pouvoir}.\par
Ne pas attendre que la petite bourgeoisie se détermine elle-même, mais \emph{former sa pensée, forger sa volonté}, — voilà la tâche du parti ouvrier. C’est seulement ainsi que pourra se réaliser l’union des ouvriers et des paysans.
\subsubsection[{L’armée}]{L’armée}
\noindent L’état d’esprit de la majorité des officiers de l’armée reflète l’état d’esprit réactionnaire des classes dominantes du pays, mais sous une forme encore plus concentrée. L’état d’esprit de la masse des soldats reflète l’état d’esprit des ouvriers et des paysans, mais sous une forme affaiblie : la bourgeoisie sait beaucoup mieux maintenir la liaison avec les officiers que le prolétariat avec les soldats.\par
Le fascisme en impose extrêmement aux officiers, car ses mots d’ordre sont décisifs et car il est prêt à trancher les questions difficiles par le revolver et la mitrailleuse. On dispose de pas mal de renseignements dispersés sur la liaison entre les ligues fascistes et l’armée, aussi bien par l’intermédiaire d’officiers de réserve que d’active. Cependant il ne nous parvient qu’une partie infime de ce qui se passe en fait. Maintenant doit grandir dans l’armée le rôle des rengagés. En eux la réaction trouvera pas mal d’agents supplémentaires. \emph{Le noyautage fasciste de l’armée, sous la protection du grand Etat-major, est en pleine marche.}\par
Les jeunes ouvriers conscients à la caserne pourraient offrir avec succès une résistance à la démoralisation fasciste.  Mais le grand malheur est qu’eux-mêmes sont politiquement désarmés : ils n’ont pas de programme. Le jeune chômeur, le fils du petit paysan, du petit commerçant ou du petit fonctionnaire apportent dans l’armée le mécontentement des milieux sociaux, d’où ils sortent. Que leur dira le communiste à la caserne, sinon que « la situation n’est pas révolutionnaire » ? Les fascistes pillent le programme marxiste, en changeant avec succès certaines de ses parties en instrument de démagogie sociale. Les « communistes » (?) renient en fait leur programme, en lui substituant les débris pourris du réformisme. Peut-on concevoir une banqueroute plus frauduleuse ?\par
L’\emph{Humanité} se concentre sur les « revendications immédiates » des soldats : c’est nécessaire, mais ce n’est qu’une centième partie du programme. L’armée vit maintenant plus que jamais d’une vie politique. Toute crise sociale est, nécessairement, une crise de l’armée. \emph{Le soldat français attend et cherche des réponses claires}. Il n’y a pas et il ne peut y avoir de meilleure réponse aux questions de la crise sociale et de meilleure réplique à la démagogie des fascistes que le \emph{programme du socialisme}. Il faut le déployer hardiment dans le pays et par mille canaux il pénétrera dans l’armée !
\subsubsection[{Les femmes}]{Les femmes}
\noindent La crise sociale, avec son cortège de calamités, pèse le plus lourdement sur les femmes travailleuses. Elles sont opprimées doublement par la classe possédante et par leur propre famille.\par
Il se trouve des « socialistes », qui craignent que les femmes aient le droit de vote, vu l’influence qu’a sur elles l’Eglise. Comme si le sort du peuple dépendait du plus ou moins grand nombre des municipalités de « gauche » en 1935, et non de la situation morale, sociale  et politique des millions d’ouvrières et de paysannes à l’époque prochaine !\par
Toute crise révolutionnaire se caractérise par l’éveil des meilleures qualités de la femme des classes travailleuses ; la passion, l’héroïsme, le dévouement. L’influence de l’Eglise sera balayée non pas par le rationalisme impuissant des « libres-penseurs », ni par la fade cagoterie des francs-maçons, mais par la lutte révolutionnaire pour l’émancipation de l’humanité, par conséquent, en premier lieu, de l’ouvrière.\par
Le programme de la révolution socialiste doit retentir de nos jours comme le tocsin pour les femmes de la classe ouvrière !
\subsubsection[{Les Jeunes}]{Les Jeunes}
\noindent La condamnation la plus cruelle de la direction des organisations ouvrières, politiques et syndicales, est la faiblesse des organisations de jeunes. Sur le terrain de la \emph{philanthropie}, du \emph{divertissement} et du \emph{sport}, la bourgeoisie et l’Eglise sont incomparablement plus fortes que nous. On ne peut leur arracher la jeunesse ouvrière que par le programme socialiste et l’action révolutionnaire.\par
A la jeune génération du prolétariat, il faut une direction politique, mais non une tutelle importune. Le bureaucratisme conservateur étouffe et repousse la jeunesse. Si le régime des Jeunesses communistes avait existé en 1848, il n’y aurait pas eu de Gavroche. La politique de passivité et d’adaptation se reflète d’une façon particulièrement funeste sur les \emph{cadres} de la jeunesse. Les jeunes bureaucrates deviennent vieux avant l’âge : ils connaissent tous les genres de manœuvres dans les coulisses, mais ils ne connaissent pas l’ABC du marxisme. Ils se forment des « convictions » à telle ou telle occasion, selon les exigences de la manœuvre. Au dernier congrès de  l’Entente de la Seine, on a pu observer d’assez près ce \emph{type.}\par
Il faut poser devant la jeunesse ouvrière le problème de la révolution dans toute son ampleur. En se tournant vers la nouvelle génération, il faut savoir faire appel à son audace et à son courage, sans lesquels rien de grand ne se fait dans l’histoire. La révolution ouvrira largement les portes à la jeunesse. La jeunesse ne peut pas ne pas être pour la révolution !
\subsection[{VI. — Pourquoi la IVe Internationale}]{VI. — Pourquoi la IV\textsuperscript{e} Internationale}
\subsubsection[{La faillite de l’Internationale communiste}]{La faillite de l’Internationale communiste}
\noindent Dans sa lettre au Conseil national du Parti socialiste, le Comité central du Parti communiste a proposé, comme base pour l’unification, « le programme de l’Internationale communiste, qui a conduit à la victoire du socialisme en U.R.S.S. alors que le programme de la II\textsuperscript{e} Internationale n’a pas résisté à l’épreuve tragique de la guerre et a abouti au douloureux bilan de l’Allemagne et de l’Autriche ». Que la II\textsuperscript{e} Internationale ait fait faillite, les marxistes révolutionnaires l’ont proclamé dès août 1914. Tous les événements ultérieurs n’ont fait que confirmer cette appréciation. Mais en montrant la banqueroute incontestable de la social-démocratie en Allemagne et en Autriche, les stalinistes oublient de répondre à une question : \emph{que sont donc devenues les sections allemande et autrichienne de l’Internationale communiste ? }Le Parti communiste allemand s’est écroulé devant l’épreuve historique aussi ignominieusement que la social-démocratie allemande. Pourquoi ? Les ouvriers allemands voulaient lutter et croyaient que « Moscou » les mènerait au combat ; ils tendaient sans cesse à gauche. Le Parti  communiste allemand grossissait rapidement ; à Berlin, il dépassait numériquement la social-démocratie. Mais il se trouva intérieurement ravagé avant que vînt l’heure de l’épreuve. L’étouffement de la vie intérieure, la volonté de commander au lieu de convaincre, la politique de zigzags, la nomination des chefs par en-haut, le système de mensonge et de tromperie des masses — tout cela démoralisa le parti jusqu’à la moelle. Quand on approcha le danger, le parti se trouva être un cadavre. Il est impossible d’effacer ce fait de l’histoire.\par
Après la capitulation honteuse de l’Internationale communiste en Allemagne, les bolchéviks-léninistes, sans hésiter une minute, proclamèrent : la Troisième Internationale est morte ! Il n’est pas besoin de rappeler les injures que lancèrent contre nous les stalinistes de tous les pays. L’\emph{Humanité,} déjà après l’avènement définitif de Hitler, affirmait de numéro en numéro : « Il n’y a pas de défaite en Allemagne », « Seuls des renégats peuvent parler de défaite », « Le Parti communiste allemand croît d’heure en heure », « Le Parti de Thælmann se prépare à la prise du pouvoir ». Rien d’étonnant si ces fanfaronnades criminelles après la plus grande catastrophe historique ont démoralisé encore plus les autres sections de l’Internationale communiste : une organisation, qui a perdu la capacité d’apprendre de sa propre défaite, est irrémédiablement condamnée.
\subsubsection[{La leçon de la Sarre}]{La leçon de la Sarre}
\noindent La preuve ne tarda pas. Le plébiscite de la Sarre fut pour ainsi dire monté pour montrer quels restes de confiance le prolétariat allemand gardait dans la Deuxième et la Troisième Internationale. Les résultats sont là : mises devant la nécessité de choisir entre la violence triomphante de Hitler et l’impuissance pourrie des partis ouvriers  banqueroutiers, les masses donnèrent à Hitler 90 \% des voix et au front commun de la Deuxième et de la Troisième Internationale (si on excepte la bourgeoisie juive, les affairistes intéressés, les pacifistes, etc.), probablement pas plus de 7 \%. Tel est le \emph{bilan} commun \emph{du réformisme et du stalinisme.} Malheur à celui qui n’a pas compris cette leçon !\par
Les masses travailleuses ont voté pour Hitler parce qu’elles ne voyaient pas d’autre voie. Les partis, qui pendant des dizaines d’années les avaient éveillées et rassemblées au nom du socialisme, les ont trompées et trahies. Voilà la conclusion commune qu’ont faite les travailleurs ! Si en France le drapeau de la révolution socialiste s’était élevé bien haut, le prolétariat de la Sarre aurait tourné ses regards vers l’Ouest et aurait placé la solidarité de classe au-dessus de la solidarité nationale. Mais, hélas, le coq gaulois n’annonçait pas au peuple sarrois une aurore révolutionnaire. Bien que sous le couvert du front unique, en France se mène la même politique de faiblesse, d’indécision, de piétinement sur place, de manque de confiance dans ses forces, qui a perdu la cause du prolétariat allemand. C’est pourquoi le plébiscite sarrois est non seulement la preuve des résultats de la catastrophe allemande, mais aussi un \emph{avertissement redoutable pour le prolétariat français}. Malheur aux partis qui glissent à la surface des événements, se bercent de paroles, espèrent en des miracles et permettent à l’ennemi mortel et de s’organiser impunément, de s’armer, d’occuper des positions avantageuses et de choisir le moment le plus favorable pour porter le coup décisif !\par
Voilà ce que nous dit la leçon sarroise.
\subsubsection[{Le programme de l’Internationale communiste}]{Le programme de l’Internationale communiste}
\noindent De nombreux réformistes et centristes (c’est-à-dire ceux qui hésitent entre le réformisme et la révolution), se tournant  à gauche, essaient maintenant de graviter vers l’Internationale Communiste ; certains d’entre eux, surtout ouvriers, espèrent sincèrement trouver dans le programme de Moscou le reflet de la Révolution d’Octobre ; d’autres, surtout fonctionnaires, s’efforcent simplement de lier amitié avec la puissante bureaucratie soviétique. Laissons les carriéristes à eux-mêmes. Mais aux socialistes, qui espèrent sincèrement trouver dans l’Internationale communiste une direction révolutionnaire, nous disons : vous vous trompez cruellement ! Vous ne connaissez pas bien l’histoire de l’Internationale communiste, qui dans les dix dernières années fut l’histoire des erreurs, des catastrophes, des capitulations et de la dégénérescence bureaucratique.\par
Le programme actuel de l’Internationale communiste fut adopté au VI\textsuperscript{e} congrès, en 1928, après l’écrasement de la direction léniniste\footnote{ \noindent Le programme de l’Internationale communiste fut écrit par Boukharine, qui bientôt après fut officiellement déclaré « libéral bourgeois ». Dans son « Testament », Lénine jugea nécessaire de prévenir que Boukharine ne possédait pas le marxisme, car sa pensée était pénétrée de \emph{scolastique}. J’ai donné une critique du programme éclectique de l’Internationale communiste dans mon livre : \emph{L’Internationale communiste après Lénine} (Rieder éd., 1931). Cette critique est restée jusqu’à maintenant sans réponse.
 }. Entre le programme actuel et celui avec lequel le bolchévisme remporta la victoire de 1917, il y a un abîme. Le programme du bolchévisme partait du point de vue que le sort de la Révolution d’Octobre est inséparable du sort de la révolution internationale. Le programme de 1928, malgré toutes ses phrases « internationalistes », part de la perspective de la \emph{construction indépendante du socialisme en U.R.S.S.} Le programme de Lénine disait : « Sans révolution en Occident et en Orient nous sommes vaincus ». Ce programme, \emph{par son essence même,} excluait la possibilité de  sacrifier les intérêts du mouvement ouvrier mondial aux intérêts de l’U.R.S.S. Le programme de l’Internationale communiste signifie en pratique : aux intérêts de l’U.R.S.S. (en fait, aux intérêts des combinaisons diplomatiques de la bureaucratie soviétique), on peut et on doit sacrifier les intérêts de la révolution prolétarienne en France. Le programme de Lénine enseignait : le bureaucratisme soviétique est le pire ennemi du socialisme ; reflétant la pression des forces et des tendances bourgeoises sur le prolétariat, le bureaucratisme peut conduire à la renaissance de la bourgeoisie ; le succès de la lutte contre le fléau du bureaucratisme ne peut être assuré que par la victoire du prolétariat européen et mondial. Contrairement à cela le programme actuel de l’Internationale communiste dit : le socialisme peut être construit indépendamment des succès et des défaites du prolétariat mondial, sous la direction de la bureaucratie soviétique infaillible et toute-puissante ; tout ce qui est dirigé contre son infaillibilité est contre-révolutionnaire et mérite d’être exterminé.\par
Dans le programme actuel de l’Internationale communiste, il y a, bien entendu, beaucoup d’expressions, de formules, de phrases empruntées au programme de Lénine (la bureaucratie conservatrice de Thermidor et du Consulat utilisa aussi en France la terminologie des Jacobins) ; mais au fond ces deux programmes s’excluent l’un l’autre. Pratiquement, en effet, la bureaucratie staliniste a remplacé depuis longtemps le programme de la révolution prolétarienne internationale par un programme de réformes soviétiques nationales. Désagrégeant et affaiblissant le prolétariat mondial par sa politique, qui représente une mixture d’opportunisme et d’aventurisme, l’Internationale communiste mine par cela même les intérêts fondamentaux de l’U.R.S.S. Nous sommes pour l’U.R.S.S. mais contre la bureaucratie usurpatrice et son instrument aveugle, l’Internationale communiste.
 \subsubsection[{Bela Kun, chef de l’Internationale communiste}]{Bela Kun, chef de l’Internationale communiste}
\noindent Manouilsky, hier chef de l’Internationale communiste, s’est noyé sans laisser de traces dans la « troisième période » (où il n’y avait, hélas ! que de la mousse). Manouilsky fut remplacé, sans que les sujets y prennent la moindre part, par Bela Kun. Sur ce nouveau souverain de l’Internationale communiste il est nécessaire de dire quelques mots. En tant que prisonnier de guerre hongrois en Russie, Bela Kun, comme beaucoup d’autres prisonniers, devînt communiste, et à son retour en Hongrie chef du petit parti. La prostration du gouvernement du comte Karoly devant l’Entente se termina par la transmission consentie et pacifique du pouvoir aux partis ouvriers, sans aucune révolution. Les communistes du parti de Bela Kun s’empressèrent de s’unir aux social-démocrates. Inspirateur de la Hongrie soviétique, Bela Kun fit preuve d’une complète carence, surtout dans la question paysanne, ce qui aboutit rapidement à l’effondrement des soviets. Rentré comme émigrant en U.R.S.S., Bela Kun eut toujours des rôles de troisième plan, car il ne jouissait pas du tout de la confiance politique de Lénine. On connaît le discours extrêmement violent de Lénine au Plénum du Comité exécutif de l’Internationale communiste, à la veille du III\textsuperscript{e} Congrès : presque chaque phrase rappelait les « bêtises de Bela Kun ». Dans ma brochure sur la direction de l’Internationale communiste\footnote{ \noindent Cette brochure : \emph{Qui dirige aujourd’hui l’Internationale communiste ?} est reproduite dans \emph{L’internationale communiste après Lénine}.
 }, j’ai raconté comment Lénine m’expliqua son attaque violente contre Bela Kun : « Il faut apprendre aux gens à ne pas avoir confiance en lui ». Depuis ce temps, non seulement Bela Kun n’a rien appris, mais il a même oublié le peu qu’il s’était  assimilé à l’école de Lénine, On peut voir combien cet homme est fait pour le rôle de chef de l’Internationale communiste et, en particulier, du prolétariat français.
\subsubsection[{L’unité organique}]{L’unité organique}
\noindent Admettons que le Parti communiste maintenant même croisse. Non pas grâce à sa politique, mais malgré sa politique. Les événements poussent les ouvriers à gauche, et le Parti communiste, malgré son tournant opportuniste, reste pour les masses l’ « extrême-gauche ». La croissance numérique du Parti communiste ne renferme pourtant pas en soi la moindre garantie pour l’avenir : le Parti communiste allemand, comme nous l’avons dit, grossit jusqu’au moment même de la capitulation, et beaucoup plus rapidement.\par
En tout cas, le fait de l’existence de deux partis ouvriers, qui rend absolument nécessaire, en face du danger commun, une politique de front unique, suffit à expliquer en même temps les aspirations des ouvriers à l’unité organique. S’il y avait en France un parti révolutionnaire conséquent, nous serions les adversaires résolus de la fusion avec le parti opportuniste. Dans les conditions de l’exacerbation de la crise sociale le parti révolutionnaire, en lutte contre le réformisme, rassemblerait infailliblement sous son drapeau la majorité écrasante des ouvriers. Le problème historique n’est pas d’unir mécaniquement toutes les organisations, qui subsistent des différentes étapes de la lutte des classes, mais de rassembler le prolétariat dans la lutte et pour la lutte. Ce sont deux problèmes absolument différents, parfois même contradictoires.\par
Mais c’est un fait qu’en France il n’y a pas de parti révolutionnaire. La légèreté avec laquelle le Parti communiste — sans la moindre discussion — est passé de la  théorie et de la pratique du « social-fascisme » au bloc avec les radicaux et au refus des tâches révolutionnaires au nom des « revendications immédiates », témoigne que l’appareil du parti est complètement rongé par le cynisme et la hase désorientée et déshabituée de penser. C’est un \emph{parti malade.}\par
Nous avons critiqué assez ouvertement la position de la S.F.I.O. pour ne pas répéter ce que nous avons déjà dit plus d’une fois. Mais il est incontestable malgré tout que l’aile gauche, révolutionnaire, de la S.F.I.O. devient peu à peu le laboratoire où se forment les mots d’ordre et les méthodes de la lutte prolétarienne. Si cette aile se fortifie et se trempe, elle pourra devenir le facteur décisif pour agir sur les ouvriers communistes. C’est dans cette seule voie que le salut est possible. Au contraire, la situation se trouverait définitivement perdue, si l’aile révolutionnaire du Parti socialiste tombait dans le système d’engrenages, qui a pour nom appareil de l’Internationale communiste et qui sert à hacher les colonnes vertébrales et les caractères, à faire perdre l’habitude de penser et à apprendre à obéir aveuglément : ce système est franchement funeste pour former des révolutionnaires.\par
— Seriez-vous contre l’unité organique ? nous demanderont, non sans indignation, quelques camarades.\par
Non, nous ne sommes pas contre l’unité. Mais nous sommes contre le fétichisme, la superstition et l’aveuglement. L’unité en soi ne résout encore rien. La social-démocratie autrichienne rassembla presque tout le prolétariat, mais seulement pour le mener à la perte. Le Parti ouvrier belge a le droit de se dire le seul parti du prolétariat, mais cela ne l’empêche pas d’aller de capitulation en capitulation. Seuls des gens d’une naïveté sans espoir peuvent espérer que le Labour Party, qui domina complètement dans le prolétariat britannique, est capable d’assurer la victoire de celui-ci. Ce qui décide, ce  n’est pas l’unité en soi, mais son contenu politique réel.\par
Si la S.F.I.O. s’unissait aujourd’hui même au Parti Communiste, cela n’assurerait pas encore plus la victoire que le Front unique ne l’assure : seule une juste politique révolutionnaire peut donner la victoire. Mais nous sommes prêts à reconnaître que l’unification faciliterait, dans les conditions présentes, le regroupement et le rassemblement des éléments véritablement révolutionnaires, dispersés dans les deux partis. C’est dans ce sens — et dans ce sens seulement — que l’unification pourrait être un pas en avant.\par
Mais l’unification — disons-le ici même — serait un pas en arrière, pis encore, un pas vers l’abîme, si la lutte contre l’opportunisme dans le parti unifié se dirigeait suivant le lit de l’Internationale communiste. L’appareil staliniste est capable d’exploiter une révolution victorieuse, mais il est organiquement incapable d’assurer la victoire d’une nouvelle révolution. Il est conservateur jusqu’à la moelle. Répétons-le encore une fois : \emph{la bureaucratie soviétique a autant à voir avec l’ancien parti bolchévik que la bureaucratie du Directoire et du Consulat avec le jacobinisme.}\par
L’unification des deux partis ne nous conduirait en avant que si elle était affranchie d’illusions, d’aveuglement et de pure tromperie. Pour ne pas tomber victime de la maladie de l’Internationale communiste, il faut aux socialistes de gauche une sérieuse inoculation de léninisme. C’est précisément pourquoi, entre autres, nous suivons avec un esprit si attentif et si critique l’évolution des groupements de gauche. D’aucuns se sentent offensés par nous. Mais nous pensons que dans le domaine révolutionnaire les règles de responsabilité sont incomparablement plus importantes que les règles de courtoisie. La critique dirigée contre nous, nous l’apprécions aussi d’un point de vue révolutionnaire et non sentimental.
 \subsubsection[{Dictature du prolétariat}]{Dictature du prolétariat}
\noindent Zyromski a essayé, dam une série d’articles, d’indiquer les principes fondamentaux du futur parti unifié. C’est une chose beaucoup plus sérieuse que de répéter des phrases générales sur l’unité, à la manière de Lebas. Par malheur, Zyromski fait dans ses articles un pas du centrisme réformiste non vers le léninisme, mais vers le centrisme bureaucratique (stalinisme). Cela apparaît de la façon la plus claire, comme nous allons le montrer, dans la question de la dictature du prolétariat.\par
Zyromski, pour quelque raison, répète avec une insistance particulière dans une série d’articles l’idée — en invoquant d’ailleurs Staline comme source première ! — que « la dictature du prolétariat ne peut jamais être considérée comme un but ». Comme s’il existait quelque part au monde des théoriciens insensés qui pensent que la dictature du prolétariat est un « but en soi » ! Mais dans ces étranges répétitions, il y a une idée : Zyromski s’excuse pour ainsi dire par avance devant les droitiers de vouloir la dictature. Par malheur, il est difficile d’établir la dictature, si on commence par s’excuser.\par
Bien pire, pourtant, est l’idée suivante : « Cette dictature du prolétariat... doit se desserrer et se transformer progressivement en démocratie prolétarienne au fur et à mesure que se développe l’édification socialiste ». Dans ces quelques lignes il y a deux profondes erreurs principielles. La dictature du prolétariat \emph{est opposée} à la démocratie prolétarienne. Pourtant, la dictature du prolétariat par son essence même peut et doit être l’épanouissement suprême de la démocratie prolétarienne. Pour accomplir une grandiose révolution sociale, il faut au prolétariat la manifestation suprême de toutes ses forces et de toutes ses capacités : Il s’organise démocratiquement précisément pour venir à bout de ses ennemis. La dictature  doit, selon Lénine, « apprendre à chaque cuisinière à diriger l’Etat ». Le glaive de la dictature est dirigé contre les ennemis de classe ; la \emph{base de la dictature est constituée par la démocratie prolétarienne}.\par
Chez Zyromski la démocratie prolétarienne vient remplacer la dictature « au fur et à mesure que se développe l’édification socialiste ». C’est une perspective absolument fausse. Au fur et à mesure que la société bourgeoise se transforme en société socialiste, la démocratie prolétarienne dépérit avec la dictature, car l’Etat lui-même dépérit. Dans la société socialiste il n’y aura pas de place pour la « démocratie prolétarienne », premièrement, par l’absence de prolétariat, deuxièmement, par l’absence de la nécessité de violence étatique. C’est pourquoi le développement de la société socialiste doit signifier non pas la transformation de la dictature en démocratie, mais leur dissolution commune dans l’organisation économique et culturelle de la société socialiste.
\subsubsection[{Adaptation à la bureaucratie staliniste}]{Adaptation à la bureaucratie staliniste}
\noindent Nous ne nous serions pas arrêtés à cette erreur, si elle avait eu un caractère purement théorique. En fait se cache derrière elle tout un dessein politique. La théorie de la dictature du prolétariat, que, selon son propre aveu, il a empruntée à Dan, Zyromski tente de l’adapter au régime actuel de la bureaucratie soviétique. Il ferme d’ailleurs consciemment les yeux sur cette question-ci : pourquoi, malgré les énormes succès économiques de l’U.R.S.S., la dictature prolétarienne évolue-t-elle non pas vers la démocratie, mais vers un bureaucratisme monstrueux, qui prend définitivement le caractère d’un régime personnel ? Pourquoi, « au fur et à mesure que se développe l’édification socialiste », étouffe-t-on le parti, étouffe-t-on les soviets, étouffe-t-on les syndicats ? Il est impossible de  répondre à cette question sans une critique décisive du stalinisme. Mais c’est précisément ce que Zyromski veut éviter à tout prix.\par
Cependant, le fait qu’une bureaucratie indépendante et incontrôlée ait usurpé la défense des conquêtes sociales de la révolution prolétarienne, témoigne que nous avons devant nous une dictature malade, en dégénérescence, qui, si on la laisse à elle-même, aboutira non pas à la « démocratie prolétarienne », mais à l’effondrement complet du régime soviétique.\par
Seule la révolution en Occident peut sauver de la perte la Révolution d’Octobre. La théorie du « socialisme en un seul pays » est fausse dans toutes ses bases. Le programme de l’Internationale communiste ne vaut pas plus qu’elle. Adopter ce programme, ce serait lancer le train de la révolution internationale dans le ravin. La première condition de succès pour le prolétariat français est l’indépendance complète de son avant-garde à l’égard de la bureaucratie soviétique, nationale et conservatrice. Bien entendu, le Parti communiste a le droit de proposer de prendre comme base de l’unification le programme de l’Internationale communiste : il ne peut rien offrir d’autre. Mais les marxistes révolutionnaires qui sont conscients de leurs responsabilités pour le sort du prolétariat, doivent soumettre le programme de Boukharine-Staline à une critique impitoyable. L’unité est une chose magnifique, mais pas sur une base pourrie. La tâche progressive consiste à rassembler les ouvriers socialistes et communistes sur la base du programme international de Marx et de Lénine. Les intérêts dû prolétariat mondial comme les intérêts de l’U.R.S.S. (ils ne sont pas différents) exigent la même lutte contre le réformisme que contre le stalinisme.
 \subsubsection[{La IVe Internationale}]{La IV\textsuperscript{e} Internationale}
\noindent Les deux Internationales, non seulement la Deuxième, mais aussi la Troisième, sont atteintes jusqu’à la moelle. Il y a des preuves historiques, qui ne trompent pas. Les grands événements (Chine, Angleterre, Allemagne, Autriche, Espagne) ont rendu leur verdict. A ce verdict, confirmé en Sarre, aucun appel n’est plus possible. La préparation d’une nouvelle Internationale, s’appuyant sur l’expérience tragique des dix dernières années, est mise à l’ordre du jour. Cette tâche grandiose est étroitement liée, bien entendu, à toute la marche de la lutte de classe du prolétariat, avant tout, à la lutte contre le fascisme en France. Pour vaincre l’ennemi, l’avant-garde du prolétariat doit s’assimiler les méthodes marxistes révolutionnaires, incompatibles et avec l’opportunisme et avec le stalinisme. Réussira-t-elle à remplir cette tâche ? Engels écrivit jadis : « Les Français s’améliorent toujours à l’approche des combats ». Espérons qu’ils justifieront pleinement cette fois-ci l’appréciation de notre grand maître. Mais la victoire du prolétariat français n’est concevable que si du feu de la lutte il fait sortir un parti véritablement révolutionnaire, qui deviendra la pierre angulaire de la nouvelle Internationale. Telle serait la voie la plus courte, la plus avantageuse, la plus favorable pour la révolution internationale.\par
Ce serait une stupidité d’affirmer qu’elle est assurée. Si la victoire est possible, la défaite, non plus, malheureusement, n’est pas exclue. \emph{La politique actuelle du front unique, comme des deux organisations syndicales, ne facilite pas, mais entrave la victoire}. Il est absolument évident qu’en cas d’écrasement du prolétariat français, ses deux partis disparaîtraient définitivement de la scène. La nécessité d’une nouvelle Internationale, sur de nouvelles bases, deviendrait alors évidente pour tout ouvrier. Mais il est  absolument évident par avance que l’édification de la IV\textsuperscript{e} Internationale, en cas de triomphe du fascisme en France, rencontrerait mille obstacles, irait avec une extrême lenteur et le centre de tout le travail révolutionnaire se trouverait, selon toute vraisemblance, transporté en Amérique.\par
Ainsi, les deux variantes historiques — la victoire et la défaite du prolétariat français — conduisent également, quoique à des rythmes différents, sur la voie de la IV\textsuperscript{e} Internationale. C’est précisément cette tendance historique qu’expriment les bolchéviks-léninistes. L’aventurisme sous toutes ses formes nous est étranger. Il s’agit non pas de « proclamer » d’une façon artificielle la IV\textsuperscript{e} Internationale, mais de la préparer systématiquement. Il faut, par l’expérience des événements, montrer et démontrer aux ouvriers avancés que les programmes et les méthodes des deux Internationales se trouvent en contradiction insurmontable avec les exigences de la révolution prolétarienne et que ces contradictions ne s’amoindrissent pas, mais au contraire croissent sans cesse. De cette analyse découle la seule ligne générale possible : il faut préparer théoriquement et pratiquement la IV\textsuperscript{e} Internationale.
\subsubsection[{Jacques Doriot ou le couteau sans lame}]{Jacques Doriot ou le couteau sans lame}
\noindent En février s’est tenue une conférence internationale de plusieurs organisations n’appartenant ni à la II\textsuperscript{e} ni à la III\textsuperscript{e} Internationale (deux partis hollandais, le S.A.P. allemand, l’I.L.P. britannique, etc.). Sauf les Hollandais, qui sont sur la position du marxisme révolutionnaire, tous les autres participants représentent différentes variétés, en majorité très conservatrices, de centrisme. J. Doriot, qui participa à cette conférence, écrit dans son compte rendu : « Au moment où la crise du capitalisme apporte la vérification  éclatante des thèses du marxisme... les partis créés à cet effet, soit par la II\textsuperscript{e}, soit par la III\textsuperscript{e} Internationale, \emph{ont tous failli à leur mission} ». Nous ne nous arrêterons pas sur le fait que Doriot lui-même, au cours d’une lutte de dix ans contre l’Opposition de gauche, aida à décomposer l’Internationale communiste. Nous ne rappellerons pas, en particulier, le triste rôle de Doriot à l’égard de la Révolution chinoise. Prenons simplement acte qu’en février 1935, Doriot a compris et reconnu la faillite de la II\textsuperscript{e} et de la III\textsuperscript{e} Internationale. Conclut-il de là à la nécessité de préparer la nouvelle Internationale ? Faire une telle supposition, ce serait bien mal connaître ce qu’est le centrisme. Sur l’idée de la nouvelle Internationale, Doriot écrit : « Cette idée du trotskisme a été formellement condamnée par la Conférence ». Doriot se laisse entraîner, quand il parle de « condamnation formelle », mais il est vrai que la Conférence, contre les deux délégués hollandais, \emph{a repoussé} l’idée de la IV\textsuperscript{e} Internationale ? Quel est en ce cas le programme réel de la Conférence ? Il est de n’avoir aucun programme. Dans leur travail quotidien, les participants de la Conférence se tiennent à l’écart des tâches internationales de la révolution prolétarienne, et y pensent bien peu. Mais tous les ans et demi, ils tiennent congrès pour soulager leur cœur et déclarer : « La II\textsuperscript{e} et la III\textsuperscript{e} Internationale ont fait faillite ». Après avoir hoché tristement la tête, ils se séparent. Il faudrait plutôt appeler cette « organisation » : Bureau pour la célébration d’un service funèbre annuel pour la II\textsuperscript{e} et la III\textsuperscript{e} Internationale.\par
Ces vénérables personnes se figurent être des « réalistes », des « tacticiens », voire des « marxistes ». Elles ne font que répandre des aphorismes : « Il ne faut pas anticiper ». « Les masses n’ont pas encore compris », etc. Mais pourquoi anticipez-vous donc vous-mêmes, en constatant la banqueroute des deux Internationales : les « masses » ne l’ont pas encore compris ? Et les masses qui  l’ont compris — sans votre aide — elles... votent pour Hitler (Sarre). Vous subordonnez la préparation de la IV\textsuperscript{e} Internationale à un « processus historique ». Mais n’êtes-vous pas vous-mêmes une partie de ce processus ? Les marxistes se sont toujours efforcés d’être à la tête du processus historique. Quelle partie représentez-vous exactement ?\par
« Les masses n’ont pas encore compris. » Mais les masses ne sont pas homogènes. Les idées nouvelles sont assimilées d’abord par les éléments avancés et, par leur intermédiaire, pénètrent dans les masses. Si vous-mêmes, sages ailiers, avez compris la nécessité et l’inéluctabilité de la IV\textsuperscript{e} Internationale, comment pouvez-vous donc cacher cette conclusion aux masses ? Pis encore : après avoir reconnu la faillite des Internationales existantes, Doriot « condamne » (!!!) l’idée de la nouvelle Internationale. Quelle perspective pratique donne-t-il donc à l’avant-garde révolutionnaire ? Aucune ! Mais cela signifie semer la confusion, le trouble et la démoralisation.\par
Telle est la nature du centrisme. Il faut comprendre cette nature jusqu’au bout. Sous la pression des circonstances, tel centriste peut aller très loin dans l’analyse, les appréciations, la critique : dans ce domaine, les chefs du S.A.P., qui dirigèrent la conférence dont nous venons de parler, répètent fort scrupuleusement beaucoup de ce que les bolchéviks-léninistes ont dit il y a deux, trois ou dix ans. Mais le centriste s’arrête toujours craintivement devant les conclusions révolutionnaires. Célébrer en famille un service funèbre pour l’Internationale communiste ? Pourquoi pas ! Mais se mettre à la préparation de la nouvelle Internationale ? Non, plutôt... « condamner » le trotskisme.\par
Doriot n’a aucune position. Et il n’en veut pas. Après la rupture avec la bureaucratie de l’Internationale communiste, il aurait pu jouer un rôle progressif et sérieux.  Mais jusqu’à maintenant il ne s’en est même pas approché. Il se dérobe aux tâches révolutionnaires. Il s’est choisi pour maître les chefs du S.A.P. Veut-il s’inscrire définitivement dans la corporation des centristes ? Qu’il sache : un centriste, c’est un couteau sans lame !
\subsection[{VII. — Conclusion}]{VII. — Conclusion}
\subsubsection[{Le rapport des forces}]{Le rapport des forces}
\noindent « Attendre », « faire durer », « gagner du temps », tels sont les mots d’ordre des réformistes, des pacifistes, des syndicalistes, des stalinistes. Cette politique se nourrit de l’idée que \emph{le temps travaille pour nous}. Est-ce vrai ? C’est radicalement faux ! Si, dans une situation pré-révolutionnaire, nous ne menons pas une politique révolutionnaire, alors le temps travaille \emph{contre nous}.\par
Malgré les hymnes creux en l’honneur du \emph{front unique, }le rapport des forces s’est modifié dans la dernière année au détriment du prolétariat. Pourquoi ? Marceau Pivert a donné une juste réponse à cette question dans son article : « \emph{Tout se tient} » \emph{(Populaire} du 18 mars 1935). Dirigés derrière les coulisses par le capital financier, toutes les forces et tous les détachements de la réaction mènent une politique incessante d’offensive, envahissent de nouvelles positions, les renforcent et vont de l’avant (industrie, agriculture, enseignement, presse, justice, armée). Du côté du prolétariat il n’y a que des phrases sur l’offensive ; en fait il n’y a même pas de défensive. Les positions ne sont pas renforcées, mais se rendent sans combat ou se préparent à se rendre.\par
Le rapport politique des forces est déterminé non pas seulement par des données objectives (rôle dans la production, nombre, etc.), mais subjectives ; \emph{la conscience de sa  force} est le plus important élément de \emph{force réelle}. Tandis que le fascisme élève de jour en jour la confiance des petits-bourgeois déclassés en eux-mêmes, les groupes dirigeants du front unique affaiblissent la volonté du prolétariat. Les pacifistes, disciples de Bouddha et de Gandhi, et non de Marx et de Lénine, s’exercent à prêcher contre la violence, contre l’armement, contre la lutte physique. Les stalinistes prêchent au fond la même chose, en invoquant seulement la « situation non-révolutionnaire ». Entre les fascistes et les pacifistes de toute nuance s’établit une division du travail : les uns renforcent le camp de la réaction, les autres affaiblissent le camp de la révolution. Telle est la vérité non camouflée !
\subsubsection[{Est-ce que cela signifie que la situation est désespérée?.. Oh ! non...}]{Est-ce que cela signifie que la situation est désespérée?.. Oh ! non...}
\noindent Deux facteurs importants agissent contre les réformistes et contre les stalinistes. Premièrement : les exemples frais d’Allemagne, d’Autriche, d’Espagne sont devant les yeux de tous ; les masses ouvrières sont alarmées, les réformistes et les stalinistes sont décontenancés. Secondement : les marxistes ont réussi à poser à temps devant l’avant-garde prolétarienne les problèmes de la révolution.\par
Les bolchéviks-léninistes sont loin de vouloir exagérer leur nombre. Mais la force de leurs mots d’ordre vient de ce qu’ils reflètent la logique du développement de la situation pré-révolutionnaire actuelle. Les événements, à chaque étape, confirment notre analyse et notre critique. L’aile gauche du Parti socialiste croit. Dans le Parti communiste la critique est comme auparavant étouffée. Mais la croissance de l’aile révolutionnaire dans la S.F.I.O. ouvrira inévitablement une brèche dans la discipline bureaucratique meurtrière des stalinistes : les  révolutionnaires des deux partis se tendront l’un à l’autre la main pour travailler en commun.\par
Notre règle reste comme toujours : \emph{exprimer ce qui est. }C’est le plus grand service qu’on peut rendre actuellement à la cause de la révolution. Les forces du prolétariat ne sont pas dépensées. La petite bourgeoisie n’a pas encore fait son choix. On a perdu beaucoup de temps, mais les derniers délais ne sont pas encore épuisés.\par
\emph{La victoire est possible !} Plus encore : \emph{la victoire est assurée} — autant que la victoire puisse être assurée par avance — à une seule et unique condition : \emph{il faut vouloir la victoire, il faut aspirer à la victoire, il faut renverser les obstacles, il faut culbuter l’ennemi et lui mettre le genou sur la poitrine} .\par
 \emph{Camarades, amis, frères et sœurs ! Les bolchéviks-léninistes vous appellent à la lutte et à la victoire !} 
 \section[{Front populaire et comité d’action, (26 novembre 1935)}]{Front populaire et comité d’action \\
(26 novembre 1935)}\phantomsection
\label{p4}\renewcommand{\leftmark}{Front populaire et comité d’action \\
(26 novembre 1935)}

\noindent Le « Front populaire » est une coalition du prolétariat avec la bourgeoisie impérialiste, en la personne du Parti radical et d’une série de pourritures de la même espèce et de plus petite taille. La coalition s’étend au domaine parlementaire. Dans les deux domaines, le Parti radical, qui conserve son entière liberté d’action, limite brutalement la liberté d’action du prolétariat.\par
Le Parti radical lui-même se trouve dans un processus de décomposition : chaque nouvelle élection montre que les électeurs le quittent pour la droite et pour la gauche. Au contraire, les Partis socialiste et communiste — en l’absence d’un parti véritablement révolutionnaire — se renforcent. La tendance générale des masses travailleuses, y compris les masses petites-bourgeoises, est tout à fait évidente : \emph{vers la gauche}. L’orientation des chefs des partis ouvriers n’est pas moins évidente : \emph{vers la droite. }Pendant que les masses, par leurs votes et par leur lutte, veulent renverser le Parti radical, les chefs du front unique, au contraire, aspirent à le sauver. Après avoir gagné la confiance des masses ouvrières sur la base d’un programme « socialiste », les chefs des partis ouvriers cèdent volontairement la part du lion de cette confiance aux radicaux en qui les masses ouvrières n’ont aucune confiance.\par
Le « Front populaire », dans son aspect actuel, foule  aux pieds non seulement la démocratie ouvrière, mais aussi la démocratie formelle, c’est-à-dire bourgeoise. La plupart des électeurs radicaux ne participent pas à la lutte des travailleurs, et, par conséquent, au « Front populaire ». Cependant, le Parti radical occupe dans ce « Front » une position non pas seulement égale aux autres, mais privilégiée ; les partis ouvriers sont contraints de limiter leur activité par le programme du Parti radical. Cette idée est mise en pratique avec le moins de gêne par les cyniques de l’\emph{Humanité}. Les dernières élections sénatoriales manifestent avec une très grande clarté la situation \emph{privilégiée} des radicaux dans le Front populaire. Les chefs du Parti communiste se vantent ouvertement d’avoir renoncé en faveur des partis non-prolétariens à quelques places qui appartiennent, de plein droit, aux ouvriers. Cela signifie tout simplement que le front unique a rétabli partiellement le cens électoral sur la base de la fortune en faveur de la bourgeoisie.\par
Le « Front » est par définition l’organisation directe et immédiate de la lutte. Là où il s’agit de la lutte, chaque ouvrier vaut une dizaine de bourgeois, même adhérant au « Front populaire ». Au point de vue de la combativité révolutionnaire du « Front », les privilèges électoraux devraient être donnés non aux bourgeois radicaux, mais aux ouvriers. Mais au fond les privilèges ne sont pas nécessaires. Le « Front populaire » défend « la démocratie » ? Qu’il commence alors par l’appliquer dans ses propres rangs. Cela signifie : \emph{la direction du « Front populaire » doit directement et immédiatement réfléter la volonté des masses en lutte}.\par
Comment ? Très simplement : par des élections. Le prolétariat n’interdit à personne de lutter à côté de lui contre le fascisme, le gouvernement bonapartiste de Laval, le complot militaire des impérialistes et toutes les autres formes d’oppression et d’ignominie. La seule chose qu’exigent  les ouvriers conscients de leurs alliés véritables, ou possibles, c’est qu’ils luttent \emph{effectivement.} Chaque groupe de population, qui participe réellement à la lutte à une étape donnée et qui est prêt à se soumettre à la discipline commune doit, à droit égal, influencer la direction du « Front populaire ».\par
Chaque deux cents, cinq cents ou mille citoyens qui adhèrent au « Front populaire » dans la ville, le quartier, l’usine, la caserne, la campagne, doit, pendant les actions de combat, élire son représentant dans les \emph{comités d’action} locaux.\par
Tous les participants de la lutte s’engagent à reconnaître leur discipline.\par
Le dernier congrès de l’Internationale communiste, dans sa résolution sur le rapport de Dimitrov, s’est prononcé dans le sens de la création de comités d’action élus, comme appui de masse du « Front populaire ». C’est certes la seule idée progressive de toute la résolution. Mais c’est précisément pourquoi les stalinistes ne font rien pour sa réalisation. Ils ne peuvent pas s’y décider sans rompre la collaboration de classe avec la bourgeoisie.\par
Il est vrai que participer aux élections des comités d’action le peuvent, non seulement les ouvriers, mais aussi les employés, les fonctionnaires, les anciens combattants, les artisans, les petits commerçants et les petits paysans. C’est ainsi que les comités d’action peuvent répondre le mieux aux tâches de la lutte pour conquérir l’influence sur la petite bourgeoisie. Mais, par contre, ils rendent extrêmement difficile la collaboration de la bureaucratie ouvrière avec celle de la bourgeoisie. Cependant, le « Front populaire », dans sa forme actuelle, n’est pas autre chose que l’organisation de la collaboration de classe entre les exploiteurs politiques du prolétariat (réformistes et stalinistes) et les exploiteurs de la petite bourgeoisie (radicaux). De véritables élections de masse de comités d’action  doivent automatiquement expulser les affairistes bourgeois (radicaux) du « Front populaire », et ainsi taire sauter en l’air la politique criminelle, dictée par Moscou.\par
Il serait, néanmoins, erroné de croire que l’on peut simplement, à un jour et à une heure donnés, faire appel aux masses prolétariennes et petites-bourgeoises pour les élections des comités d’action sur la base de statuts déterminés. Une telle manière d’aborder la question serait purement bureaucratique, et par conséquent stérile. Elire les comités d’action, les ouvriers ne le peuvent que dans le cas où ils participent eux-mêmes à une action et éprouvent la nécessité d’une direction révolutionnaire.\par
Il ne s’agit pas d’une représentation \emph{démocratique} de \emph{toutes} et \emph{n’importe quelles masses,} mais d’une représentation \emph{révolutionnaire} des masses en \emph{lutte.} Le comité d’action est l’appareil de la lutte. Inutile de présumer d’avance quelles couches de travailleurs seront alliées à la création des comités d’action : les frontières de masses qui luttent se détermineront dans la lutte même.\par
Le danger énorme en France consiste en ce que l’énergie révolutionnaire des masses dépensée morceau par morceau dans des explosions isolées, comme à Toulon, à Brest, à Limoges, fasse place à l’apathie. Seuls des traîtres conscients ou des cerveaux désespérément obtus peuvent croire que l’on peut, dans la situation actuelle, tenir les masses dans l’immobilité jusqu’à ce qu’on leur fasse cadeau d’en haut du gouvernement de « Front populaire ».\par
Les grèves, les protestations, les escarmouches de rue, les révoltes directes sont tout à fait inévitables dans la situation actuelle. La tâche du parti prolétarien consiste non pas à freiner et à paralyser ces mouvements, mais à les unifier et à leur donner la plus grande force.\par
Les réformistes, et surtout les stalinistes, ont peur d’effrayer les radicaux. L’appareil du « front unique » joue tout à fait consciemment le rôle de désorganisateur à  l’égard des mouvements spontanés des masses. Et les gauches, du type Marceau Pivert, ne font que protéger cet appareil de la colère des masses. On ne peut sauver la situation que si l’on aide les masses en lutte, dans le processus de la lutte même, à créer un nouvel appareil, qui réponde aux nécessités du moment. C’est en cela que réside précisément la fonction des comités d’action.\par
Pendant la lutte à Toulon et à Brest, les ouvriers auraient sans hésitation créé une organisation locale de combat, si on les avait appelés à le faire. Au lendemain de la répression sanglante de Limoges, les ouvriers et une partie considérable de la petite bourgeoisie auraient sans aucun doute manifesté leur disposition à créer des comités élus pour enquêter sur les événements sanglants et les prévenir dans l’avenir. Pendant le mouvement dans les casernes qui eut lieu au cours de cet été contre le « rabiot », les soldats, sans hésiter, auraient élu des comités d’action de compagnie, de régiment et de garnison si on leur avait indiqué cette voie. De tels cas se présentent et se présenteront à chaque pas. Plus souvent à l’échelle locale, moins souvent à l’échelle nationale. La tâche consiste en ce qu’il ne faut pas manquer une seule de ces occasions. La première condition pour cela : comprendre clairement soi-même la signification des comités d’action, comme \emph{le seul moyen de briser la résistance antirévolutionnaire des appareils des partis et des syndicats}.\par
Cela signifie-t-il que les comités d’action remplacent les organisations des partis et des syndicats ? Il serait absurde de poser ainsi la question. Les masses entrent en lutte avec toutes leurs idées, groupements, traditions et organisations. Les partis continuent de vivre et de lutter. Pendant les élections aux comités d’action, chaque parti tentera naturellement de faire passer ses partisans. Les comités d’action prendront des décisions à la majorité des voix avec l’entière liberté des partis et des fractions de se grouper. Par  rapport aux partis, les comités d’action peuvent être appelés des \emph{parlements révolutionnaires :} les partis ne sont pas exclus, au contraire, ils sont supposés nécessaires ; en même temps ils sont contrôlés dans l’action, et les masses apprennent à se libérer de l’influence des partis pourris.\par
Cela signifie-t-il que les comités d’action sont des \emph{soviets} ? Dans certaines conditions, les comités d’action peuvent devenir des soviets. Il serait, néanmoins, erroné d’appeler les comités d’action de ce nom. Aujourd’hui, en 1935, les masses populaires sont habituées à lier au mot de soviet l’idée du pouvoir déjà conquis. Mais le moment n’en est pas encore proche en France. Les soviets en Russie, dans leurs premiers pas, n’ont pas du tout été ce qu’ils sont devenus par la suite, ils ont même souvent porté à l’époque le nom modeste de comités ouvriers ou de comités de grève.\par
Les comités d’action, dans leur stade actuel, ont pour tâche d’unifier la lutte défensive des masses travailleuses en France et aussi de donner à ces masses la conscience de leur propre force pour l’offensive à venir. La chose aboutira-t-elle aux véritables soviets ? Cela dépend du fait de savoir si la situation critique actuelle en France se développera ou non jusqu’à sa conclusion révolutionnaire. Cela ne dépend pas seulement, bien entendu, de la volonté de l’avant-garde révolutionnaire, mais aussi d’une série de conditions objectives. En tout cas, le mouvement de masses qui se heurte actuellement à la barrière du « Front populaire » n’avancera pas sans les comités d’action.\par
Des tâches, telles que la création de la \emph{milice ouvrière}, \emph{l’armement des ouvriers,} la préparation de la \emph{grève générale}, resteront sur le papier, si la masse elle-même ne s’attelle pas à la lutte, en la personne de ses organes responsables. Seuls ces comités d’action sortis de la lutte peuvent assurer la véritable milice, comptant non pas des milliers, mais des dizaines de milliers de combattants. Ce  n’est que les comités d’action, englobant les centres principaux du pays, qui pourront choisir le moment pour passer à des méthodes plus décidées de la lutte, dont la direction leur appartiendra de droit.\par
Des considérations tracées plus haut, il découle une série de conclusions pour l’activité politique des révolutionnaires prolétariens en France. La première de ces conclusions concerne la soi-disant « Gauche révolutionnaire ». Ce groupement est caractérisé par une incompréhension totale des lois du mouvement des masses. Les centristes ont beau bavarder sur les « masses », ils s’orientent toujours sur l’appareil réformiste. En répétant tel ou tel mot d’ordre révolutionnaire, Marceau Pivert les subordonne au principe abstrait de l’ « unité organique », qui, en fait, s’avère l’unité avec les patriotes contre les révolutionnaires.\par
Pendant que pour les masses révolutionnaires la question de vie ou de mort est de briser la résistance des appareils social-patriotes unis, les centristes de gauche considèrent l’ « unité » de ces appareils comme un bien absolu, au-dessus des intérêts de la lutte révolutionnaire. Ne peut bâtir des comités d’action que celui qui a compris jusqu’au bout la nécessité de \emph{libérer les masses de la direction traître des social-patriotes}. Cependant Pivert s’accroche à Zyromski, qui s’accroche à Blum, qui ensemble avec Thorez s’accroche à Herriot, qui s’accroche à Laval. Pivert entre dans le système du « Front populaire » (ce n’est pas pour rien que la Gauche révolutionnaire a voté au dernier Conseil national la résolution honteuse de Blum) et le « Front populaire » entre comme aile dans le régime bonapartiste de Laval. L’effondrement du régime bonapartiste est inévitable. Si la direction du « Front populaire » (Herriot-Blum-Cachin-Thorez-Zyromski-Pivert) parvient à se maintenir pendant la période proche décisive, alors le régime bonapartiste, inévitablement, cédera sa place au fascisme.\par
 La condition de la victoire du prolétariat est la \emph{liquidation de la direction actuelle.} Le mot d’ordre de l’ « unité » devient dans ces conditions non seulement une bêtise, mais un crime. Aucune \emph{unité avec les agents de l’impérialisme français et de la Société des Nations}. A leur direction perfide il faut opposer les unités d’action révolutionnaires. On ne peut construire des comités qu’en démasquant impitoyablement la politique antirévolutionnaire de la soi-disant « Gauche révolutionnaire » avec Marceau Pivert en tête. Des illusions et des doutes à cet égard ne peuvent, bien entendu, avoir place dans nos rangs.
 \section[{La France a un tournant (28 mars 1936)}]{La France a un tournant\protect\footnotemark  \\
(28 mars 1936)}\phantomsection
\label{p5}\renewcommand{\leftmark}{La France a un tournant \\
(28 mars 1936)}

\footnotetext{ \noindent Cette étude a été écrite en guise de préface à la nouvelle édition de \emph{Terrorisme et communisme,} parue aux Editions de la Nouvelle Revue critique, sous le titre de \emph{Défense du Terrorisme}.
 }
\noindent Ce livre est consacré à l’éclaircissement des méthodes de la politique révolutionnaire du prolétariat à notre époque. L’exposé a un caractère polémique, comme la politique révolutionnaire elle-même. En gagnant les masses opprimées, la polémique dirigée contre la classe dominante se transforme, à un moment donné, en révolution.\par
Comprendre clairement la nature sociale de la société moderne, de son Etat, de son Droit, de son idéologie constitue le fondement théorique de la politique révolutionnaire. La bourgeoisie opère par abstraction (« nation », « patrie », « démocratie ») pour camoufler l’exploitation qui est à la base de sa domination. Le \emph{Temps,} l’un des plus infâmes journaux de l’univers, enseigne chaque jour aux masses populaires françaises le patriotisme et le désintéressement. Cependant ce n’est un secret pour personne que le désintéressement du \emph{Temps} s’estime d’après un tarif international bien établi.\par
Le premier acte de la politique révolutionnaire est de démasquer les fictions bourgeoises qui intoxiquent le [{\corr sentiment}] des masses populaires. Ces fictions deviennent  particulièrement malfaisantes quand elles s’amalgament avec les idées de « socialisme » et de « révolution ». Aujourd’hui plus qu’à n’importe quel autre moment, ce sont les fabricants de ce genre d’amalgames qui donnent le ton dans les organisations ouvrières françaises.\par
La première édition de cet ouvrage a exercé une certaine influence sur la formation du Parti communiste français : l’auteur en a reçu maints témoignages, dont au demeurant il ne serait pas difficile de trouver la trace dans l’\emph{Humanité} jusqu’en 1924. Au cours des douze années qui ont suivi, il a été procédé dans l’internationale communiste — après plusieurs zigzags fébriles — à une révision fondamentale des valeurs : il suffit de dire qu’aujourd’hui cet ouvrage figure sur l’index des livres interdits. Par leurs idées et leurs méthodes, les chefs actuels du Parti communiste français (nous sommes obligé de conserver cette appellation qui est en contradiction complète avec la réalité) ne se différencient par aucun principe de Kautsky, contre lequel notre ouvrage est dirigé : ils sont toutefois immensément plus ignorants et cyniques. Le nouvel accès de réformisme et de patriotisme que subissent Cachin et C\textsuperscript{ie} aurait pu à lui seul justifier une nouvelle édition de ce livre. Il y a cependant d’autres raisons plus sérieuses : elles ont leurs racines dans la profonde crise prérévolutionnaire qui secoue le régime de la III\textsuperscript{e} République.\par
Après dix-huit ans d’absence, l’auteur de cet ouvrage a eu la possibilité de passer deux ans en France (1933-1935), il est vrai en simple qualité d’observateur de province qui, par surcroît, était l’objet d’une surveillance serrée. Pendant cette période, il y eut dans le département de l’Isère, où l’auteur eut l’occasion de séjourner, un petit incident pareil à beaucoup d’autres qui cependant donne la clé de toute la politique française. Dans un sanatorium appartenant au Comité des Forges, un jeune  ouvrier, qui était sous le coup d’une grave opération, s’était permis de lire un journal révolutionnaire (plus exactement : le journal que naïvement il considérait comme révolutionnaire était \emph{l’Humanité}). L’administration posa à l’imprudent malade et ensuite à quatre autres malades qui partageaient ses sympathies cet ultimatum : renoncer à recevoir des publications indésirables, ou être jetés à la rue. Les malades eurent beau indiquer que l’on se livrait ouvertement dans le sanatorium à une propagande cléricale et réactionnaire, cela n’eut évidemment aucun effet. Comme il s’agissait de simples ouvriers qui ne risquaient ni mandats parlementaires ni portefeuilles ministériels, mais tout simplement leur santé et leur vie, l’ultimatum n’eut pas de succès : cinq malades, dont un à la veille d’être opéré, furent mis à la porte du sanatorium. Grenoble avait alors une municipalité socialiste que présidait le docteur Martin, un de ces bourgeois conservateurs qui généralement donnent le ton dans le parti socialiste et dont Léon Blum est le représentant achevé. Les ouvriers expulsés essayèrent de trouver une défense auprès du maire. Ce fut en vain : malgré leur insistance, leurs lettres, leurs démarches, ils ne furent pas même reçus. Ils s’adressèrent au journal local de gauche, la \emph{Dépêche}. où radicaux et socialistes forment un cartel indissoluble. En apprenant qu’il s’agissait du sanatorium du Comité des Forges, le directeur du journal refusa catégoriquement d’intervenir : tout ce que vous voudrez mais pas ça. Pour une imprudence à l’égard de cette puissante organisation, la \emph{Dépêche} fut privée une fois déjà de publicité et subit de ce fait une perte de 20.000 francs. A la différence des prolétaires, le directeur du journal de gauche, ainsi que le maire, avaient quelque chose à perdre : aussi renoncèrent-ils à une lutte illégale en abandonnant les ouvriers avec leurs intestins et leurs reins malades à leur propre sort.\par
Une ou deux fois par semaine, le maire socialiste, remuant  de vagues souvenirs de jeunesse, fait un discours pour vanter les avantages du socialisme sur le capitalisme. Pendant les élections, la \emph{Dépêche} soutient le maire et son parti. Tout est pour le mieux. Le Comité des Forges regarde avec une tolérance toute libérale ce genre de socialisme qui ne cause pas le plus petit préjudice aux intérêts matériels du capital. Avec 20.000 francs de publicité par an (si bon marché coûtent ces messieurs !), les féodaux de l’industrie lourde et de la banque tiennent pratiquement à leur dévotion un grand journal du cartel ! Et pas seulement ce journal : le Comité des Forges a bien sûr assez de moyens, directs ou indirects, pour agir sur messieurs les maires, sénateurs, députés, y compris les maires, les sénateurs, les députés socialistes. Toute la France officielle est placée sous la dictature du capital financier. Dans le dictionnaire Larousse, ce système est désigné sous le nom de « République démocratique ».\par
Messieurs les députés de gauche et les journalistes, non seulement de l’Isère mais de tous les départements de France, croyaient que leur cohabitation pacifique avec la réaction capitaliste n’aurait pas de fin. Ils se trompaient. Depuis longtemps vermoulue, la démocratie sentit soudain sur sa tempe le canon d’un revolver. De même que les armements de Hitler — acte matériel brutal — causèrent une véritable révolution dans les rapports entre les Etats en démontrant la vanité et le caractère illusoire de ce qu’il est convenu d’appeler le « droit international », de même les bandes armées du colonel de La Rocque ont jeté la perturbation dans les rapports intérieurs de la France en obligeant tous les partis sans exception à se réorganiser, à se restreindre et à se regrouper.\par
Frédéric Engels a écrit un jour que l’Etat, y compris la République démocratique, ce sont des bandes armées pour la défense de la propriété ; tout le reste n’est là que pour enjoliver ou masquer ce fait. Les éloquents  défenseurs du « Droit », dans le genre d’Herriot et de Blum, ont toujours été révoltés par ce cynisme. Mais Hitler, de même que de La Rocque, chacun dans sa sphère, ont de nouveau montré qu’Engels avait raison.\par
Au début de 1934, Daladier était président du Conseil par la volonté du suffrage universel, direct et secret : il portait la souveraineté nationale dans sa poche avec son mouchoir. Mais dès que les bandes de de La Rocque, Maurras et C\textsuperscript{ie} montrèrent qu’elles avaient l’audace de tirer des coups de revolver et de couper les jarrets des chevaux de la police, Daladier et sa souveraineté cédèrent la place à l’invalide politique que désignèrent les chefs de ces bandes. Ce fait a infiniment plus d’importance que toutes les statistiques électorales et on ne saurait l’effacer de l’histoire récente de la France, car il est une indication pour l’avenir.\par
Il est certain qu’il n’est pas donné à \emph{n importe quel groupe} armé de revolvers de modifier à tout moment l’orientation politique d’un pays. Seules les bandes armées qui sont les organes d’une classe déterminée peuvent, dans \emph{certaines} circonstances, jouer un rôle décisif. Le colonel de La Rocque et ses partisans veulent assurer l’ « ordre » contre les secousses. Et comme en France « ordre » signifie domination du capital financier sur la petite et moyenne bourgeoisie et domination de l’ensemble de la bourgeoisie sur le prolétariat et les couches sociales qui lui sont proches, les troupes de de La Rocque sont tout simplement des bandes armées du capital financier.\par
Cette idée n’est pas neuve. On peut même la trouver fréquemment dans le \emph{Populaire} et l’\emph{Humanité,} encore qu’ils n’aient pas été les premiers à la formuler. Cependant ces publications ne disent que la moitié de la vérité. L’autre moitié, non moins importante, est que Herriot et Daladier, avec leurs partisans, sont aussi une agence du capital financier : autrement les radicaux n’auraient pas  pu être le parti gouvernemental de la France pendant des dizaines d’années. Si l’on ne veut pas jouer à cache-cache. il est nécessaire de dire que de La Rocque et Daladier travaillent pour le même patron. Cela ne signifie pas, évidemment, qu’il y ait entre eux ou leurs méthodes une complète identité. Bien au contraire. Ils se font une guerre acharnée, comme deux agences spécialisées, dont chacune possède le secret du salut. Daladier promet de maintenir l’ordre au moyen de la même démocratie tricolore. De La Rocque estime que le parlementarisme périmé doit être balayé en faveur d’une dictature militaire et policière déclarée. Les méthodes politiques sont antagonistes, mais les intérêts sociaux sont les mêmes.\par
La décadence du système capitaliste, sa crise incurable, sa décomposition forment la base historique de l’antagonisme qui existe entre de La Rocque et Daladier (nous prenons ces deux noms uniquement pour faciliter l’exposé). Malgré les progrès incessants de la technique et les résultats remarquables de certaines branches industrielles, le capitalisme dans l’ensemble freine le développement des forces productives, ce qui détermine une extrême instabilité des rapports sociaux et internationaux. La démocratie parlementaire est intimement liée à l’époque de la libre concurrence et de la liberté du commerce international. La bourgeoisie put tolérer le droit de grève, de réunion, la liberté de la presse aussi longtemps que les forces productives furent en pleine ascension, que les débouchés s’élargirent, que le bien-être des masses populaires, quoique restreint, s’accrut et que les nations capitalistes purent vivre et laisser vivre les autres. Mais plus aujourd’hui. L’époque impérialiste est caractérisée, abstraction faite de l’Union soviétique, par une stagnation et une diminution du revenu national, par une crise agraire chronique et un chômage organique. Ces phénomènes  internes sont inhérents à la phase actuelle du capitalisme comme la goutte et la sclérose à un âge déterminé de l’individu. Vouloir expliquer le chaos économique mondial par les conséquences de la dernière guerre, c’est faire preuve d’un esprit désespérément superficiel, à l’instar de M. Caillaux, du comte Sforza et autres. La guerre ne fut pas autre chose qu’une tentative des pays capitalistes de faire retomber sur le dos de l’adversaire le krach qui, dès ce moment, menaçait. La tentative échoua. La guerre ne fit qu’aggraver les signes de décomposition dont l’accentuation ultérieure prépare une nouvelle guerre.\par
Aussi mauvaises que soient les statistiques économiques de la France, qui passent intentionnellement sous silence les antagonismes de classe, elles ne peuvent pas dissimuler les signes manifestes de la décomposition sociale. Parallèlement à la diminution du revenu national, à la chute en vérité catastrophique du revenu des campagnes, à la ruine des petites gens des villes, à l’accroissement du chômage, les entreprises géantes, ayant un chiffre d’affaires annuel de 100 à 200 millions et même davantage, font de brillants bénéfices. Le capital financier, dans toute l’acception du terme, suce le sang du peuple français. Telle est la base sociale de l’idéologie et de la politique de l’ « union nationale ».\par
Des adoucissements et des éclaircies dans le processus de décomposition sont possibles, voire inévitables ; mais ils garderaient un caractère strictement conditionné par la conjoncture. Quant à la tendance générale de notre époque, elle place la France, après bien d’autres pays, devant cette alternative : ou le prolétariat doit renverser l’ordre bourgeois foncièrement gangrené, ou le capital, en vue de sa propre conservation, doit remplacer la démocratie par le fascisme. Pour combien de temps ? Le sort de Mussolini et de Hitler répondra à cette question.\par
Les fascistes ont tiré, le 6 février 1934, sur l’ordre  direct de la Bourse, des banques et des trusts. De ces mêmes positions de commande, Daladier a été sommé de remettre le pouvoir à Doumergue, Et si le ministre radical, président du Conseil, a capitulé — avec la pusillanimité qui caractérise les radicaux — c’est parce qu’il a reconnu dans les bandes de de La Rocque les troupes de son propre patron. Autrement dit : Daladier, ministre souverain, céda le pouvoir à Doumergue pour la même raison que le directeur de la \emph{Dépêche} et le maire de Grenoble refusèrent de dénoncer l’odieuse cruauté des agents du Comité des Forges.\par
Cependant, le passage de la démocratie au fascisme comporte des risques de secousses sociales. D’où les hésitations et les désaccords tactiques que l’on constate dans les hautes sphères de la bourgeoisie. Tous les magnats du capital sont pour qu’on continue à renforcer les bandes armées qui pourront constituer une réserve salutaire à l’heure du danger. Mais quelle place accorder à ces bandes dès aujourd’hui ? Doit-on leur permettre de passer tout de suite à l’attaque ou les garder en attendant comme un moyen d’intimidation ? Autant de questions qui ne sont pas encore résolues. Le capital financier ne croit plus qu’il soit possible aux radicaux d’entraîner derrière eux les masses de la petite bourgeoisie et de maintenir, par la pression de ces masses, le prolétariat dans les limites de la discipline « démocratique ». Mais il ne croit pas davantage que les organisations fascistes, qui manquent encore d’une véritable base de masse, soient capables de s’emparer du pouvoir et d’établir un régime fort.\par
Ce qui a fait comprendre, aux dirigeants de la coulisse la nécessité d’être prudents, ce n’est pas la rhétorique parlementaire, mais la révolte des ouvriers, la tentative de grève générale étouffée, certes, dès le début par la bureaucratie de Jouhaux, et ultérieurement les émeutes locales (Toulon, Brest). Les fascistes ayant été remis quelque  peu en place, les radicaux respirèrent plus librement. Le \emph{Temps}, qui dans une série d’articles avait déjà trouvé le moyen d’offrir sa main et son cœur à la « jeune génération », découvrit de nouveau les avantages du régime libéral, conforme d’après lui au génie français. Ainsi s’est établi un régime instable, transitoire, bâtard, conforme non pas au génie de la France, mais au déclin de la Troisième République. Dans ce régime, ce sont les traits \emph{bonapartistes} qui apparaissent avec le plus de netteté : indépendance du gouvernement à l’égard des partis et des programmes, liquidation du pouvoir législatif au moyen des pleins pouvoirs, le gouvernement se situant au-dessus des fractions en lutte, c’est-à-dire en fait au-dessus de la nation, pour jouer le rôle d’ « arbitre ». Les ministères Doumergue, Flandin, Laval, tous les trois, avec l’immanquable participation des radicaux humiliés et compromis, ont représenté de petites variantes sur un seul et même thème.\par
Lorsque le ministère Sarraut fut constitué, Léon Blum, dont la perspicacité a deux dimensions au lieu de trois, annonça : « Les derniers effets du 6 février sont détruits sur le plan parlementaire » (\emph{Populaire} du 2 février 1936). Voilà ce qui s’appelle brosser l’ombre du carrosse avec l’ombre d’une brosse ! Comme si l’on pouvait supprimer « sur le plan parlementaire » la pression des bandes armées du capital financier ! Comme si Sarraut pouvait ne pas sentir cette pression et ne pas trembler devant elle ! En réalité, le gouvernement Sarraut-Flandin est une variété de ce même « bonapartisme » semi-parlementaire, toutefois légèrement incliné à « gauche ». Sarraut lui-même, réfutant l’accusation d’avoir pris des mesures arbitraires, répondit on ne peut mieux au Parlement : « Si mes mesures sont arbitraires, c’est parce que je veux être un arbitre ». Cet aphorisme n’aurait pas été déplacé dans la bouche de Napoléon III. Sarraut se sent non pas le  mandataire d’un parti déterminé ou d’un bloc de partis au pouvoir, comme le veulent les règles du parlementarisme, mais un arbitre au-dessus des classes et des partis comme le veulent les lois du bonapartisme.\par
L’aggravation de la lutte de classes et surtout l’entrée en scène des bandes armées de la réaction n’ont pas moins révolutionné les organisations ouvrières. Le Parti socialiste, qui jouait paisiblement le rôle de la cinquième roue dans la charrette de la III\textsuperscript{e} République, se vit contraint de répudier à demi ses traditions cartellistes et même de rompre avec son aile droite (néos). Dans le même temps, les communistes accomplirent l’évolution contraire, mais sur une échelle infiniment plus vaste. Pendant des années ces messieurs avaient rêvé de barricades, de conquête de la rue, etc. (ce rêve, il est vrai, avait surtout un caractère littéraire). Après le 6 février, comprenant que l’affaire était sérieuse, les artisans des barricades se jetèrent à droite. Le réflexe spontané de ces phraseurs apeurés coïncida d’une façon frappante avec la nouvelle orientation internationale de la diplomatie soviétique.\par
Devant le danger que représente l’Allemagne hitlérienne, la politique du Kremlin se tourna vers la France. \emph{Statu quo} dans les rapports internationaux ! \emph{Statu quo }dans le régime intérieur de la France ! Espoirs de révolution socialiste ? Chimères ! Les milieux dirigeants du Kremlin ne parlent qu’avec mépris du communisme français. Il faut donc garder ce qui existe pour ne pas avoir pire. La démocratie parlementaire en France ne se concevant pas sans les radicaux, faisons en sorte que les socialistes les soutiennent ; ordonnons aux communistes de ne pas gêner le bloc Blum-Herriot ; s’il est possible, faisons-les entrer eux-mêmes dans ce bloc. Ni secousses, ni menaces ! Telle est l’orientation du Kremlin.\par
Quand Staline répudie la révolution mondiale, les partis bourgeois français ne veulent pas le croire. C’est bien  à tort ! En politique, une confiance aveugle n’est évidemment pas une vertu supérieure. Mais une méfiance aveugle ne vaut pas mieux. Il faut savoir confronter les paroles avec les actes et discerner la tendance générale de l’évolution pour plusieurs années. La politique de Staline, qui est déterminée par les intérêts de la bureaucratie soviétique privilégiée, est devenue foncièrement conservatrice. La bourgeoisie française a tout lieu de faire confiance à Staline. Le prolétariat français a tout autant de raisons d’être méfiant.\par
Au congrès d’unité de Toulouse, le « communiste » Racamond a donné de la politique du Front populaire une formule digne de passer à la postérité : « Comment vaincre la timidité du Parti radical ? » Comment vaincre la peur qu’a la bourgeoisie du prolétariat ? « Très simplement : les terribles révolutionnaires doivent jeter le couteau qu’ils serraient entre les dents, se pommader les cheveux et prendre le sourire de la plus charmante des odalisques : Vaillant-Couturier dernière manière en sera le prototype. Sous la pression des « communistes » pommadés, qui de toutes leurs forces poussaient à droite les socialistes qui évoluaient vers la gauche, Blum dut changer une fois de plus d’orientation. Il le fit, heureusement, dans le sens habituel. Ainsi se forma le Front populaire : compagnie d’assurance de banqueroutiers radicaux aux frais du capital des organisations ouvrières.\par
Le radicalisme est inséparable de la franc-maçonnerie. C’est tout dire. Lors des débats qui eurent lieu à la Chambre des Députés sur les ligues, M. Xavier-Vallat rappela que Trotsky avait, à une époque, « interdit » aux communistes d’adhérer aux loges maçonniques. M. Jammy Schmidt, qui est, paraît-il, une autorité en la matière, s’empressa d’expliquer cette interdiction par l’incompatibilité du bolchévisme despotique avec l’ « esprit de liberté ». Nous ne voyons pas la nécessité de polémiquer  sur ce thème avec le député radical. Mais aujourd’hui encore nous estimons que le représentant ouvrier qui va chercher son inspiration ou sa consolation dans la fade religion maçonnique de la collaboration des classes ne mérite pas la moindre confiance. Ce n’est pas par hasard que le Cartel a été complété par une large participation des socialistes aux loges maçonniques. Mais le temps est venu pour les communistes repentis de ceindre eux-mêmes le tablier. Au demeurant, en tablier, il sera plus commode aux compagnons nouvellement initiés de servir les vieux patrons du Cartel.\par
Le Front populaire, nous dit-on non sans indignation, n’est nullement un cartel, mais un mouvement de masse. Les définitions pompeuses ne manquent pas, certes, mais elles ne changent rien aux choses. Le but du Cartel a toujours été de \emph{freiner} le mouvement de masse en l’orientant vers la collaboration de classe. Le Front populaire a exactement le même but. La différence entre eux — et elle est de taille — est que le Cartel traditionnel a été appliqué dans les époques de stabilité et de câline du régime parlementaire. Mais aujourd’hui que les masses sont impatientes et prêtes à exploser, un frein plus solide, avec la participation des « communistes » est devenu indispensable. Les meetings communs, les cortèges à grand spectacle, les serments, l’union du drapeau de la Commune avec le drapeau de Versailles, le tintamarre, la démagogie, tout cela n’a qu’un but : contenir et démoraliser le mouvement de masse.\par
Pour se justifier devant les droites, Sarraut déclara à la Chambre que ses concessions inoffensives au Front populaire ne constituent rien de plus que la \emph{soupape de sûreté }du régime. Cette franchise aurait pu paraître imprudente. Mais l’extrême-gauche la couvrit d’applaudissements. Sarraut n’avait donc aucune raison de se gêner. De toute façon, il a réussi à donner, peut-être sans le vouloir, une  définition du Front populaire : une soupape de sûreté contre le mouvement de masse. En général, M. Sarraut a la main heureuse pour les aphorismes !\par
La politique extérieure est la continuation de la politique intérieure. Ayant complètement abandonné le point de vue du prolétariat, Blum, Cachin et C\textsuperscript{ie} adoptent — sous le masque de la « sécurité collective » et du « droit international » — le point de vue de l’impérialisme national. Ils préparent la même politique d’abdication et de platitude que celle qu’ils ont suivie de 1914 à 1918 en y ajoutant seulement : « pour la défense de l’U.R.S.S. ». Cependant de 1918 à 1923, quand la diplomatie soviétique se vit fréquemment obligée de louvoyer et de passer des accords, il ne vint jamais à l’esprit d’une seule section de l’Internationale communiste qu’elle pourrait faire bloc avec sa bourgeoisie ! A elle seule, cette chose n’est-elle pas une preuve suffisante de la sincérité de Staline quand il répudie la révolution mondiale ?\par
Pour les mêmes motifs que les chefs actuels de l’Internationale communiste se collent aux mamelles de la « démocratie » dans la période de son agonie, ceux-ci découvrent le radieux visage de la Société des Nations alors qu’elle a déjà le hoquet de la mort. Ainsi s’est créée une plate-forme commune de politique extérieure entre les radicaux et l’Union soviétique. Le programme intérieur du Front populaire est un assemblage de lieux communs qui permettent une interprétation aussi libre que le Covenant de Genève. Le sens général du programme est celui-ci : pas de changement. Or les masses veulent du changement et c’est en cela que réside le fond de la crise politique.\par
En désarmant politiquement le prolétariat, les Blum, Paul Faure, Cachin, Thorez, s’intéressent surtout à ce qu’il ne s’arme pas physiquement. La propagande de ces messieurs ne se différencie pas des sermons religieux sur  la supériorité des principes moraux. Engels, qui enseignait que la possession du pouvoir d’Etat est une question de bandes armées, Marx qui regardait l’insurrection comme un art, apparaissent aux députés, sénateurs et maires actuels du Front populaire comme des sauvages du moyen âge. Le \emph{Populaire} a passé pour la centième fois un dessin figurant un ouvrier désarmé avec cette légende : « Vous comprendrez que nos poings nus sont plus solides que toutes vos matraques ». Quel splendide mépris pour la technique militaire ! A cet égard, le Négus lui-même a des vues plus avancées. Pour ces gens, les coups d’Etat en Italie, en Allemagne, en Autriche n’existent pas. Cesseront-ils de vanter les « poings nus », quand de La Rocque leur passera les menottes ? Par moment, on en arrive presque à regretter qu’on ne puisse pas faire subir séparément cette expérience à messieurs les chefs, sans que les masses aient à en souffrir !\par
Vu sous l’angle du régime bourgeois, le Front populaire est un épisode de la rivalité entre le radicalisme et le fascisme pour gagner l’attention et les faveurs du grand capital. En fraternisant d’une façon théâtrale avec les socialistes et les communistes, les radicaux veulent montrer au patron que le régime n’est pas aussi malade que les droites le prétendent ; que le danger de révolution est exagéré ; que Vaillant-Couturier lui-même a troqué son couteau contre un collier ; que par les « révolutionnaires » apprivoisés, on peut discipliner les masses ouvrières et, par conséquent, sauver le système parlementaire de la faillite.\par
Cependant, tous les radicaux ne croient pas à cette manœuvre ; les plus sérieux et les plus influents, Herriot en tête, préfèrent adopter une attitude d’attente. Mais en fin de compte eux-mêmes ne peuvent pas proposer autre chose. La crise du parlementarisme est avant tout une crise de confiance de l’électeur à l’égard du radicalisme.\par
Tant qu’on n’aura pas découvert le moyen de rajeunir  le capitalisme, il n’y aura pas de recette pour sauver le parti radical. Celui-ci n’a le choix qu’entre différents genres de mort politique. Un succès relatif aux prochaines élections n’empêcherait pas et même ne retarderait pas bien longtemps son effondrement.\par
Les chefs du Parti socialiste, les politiciens les plus insouciants de France, ne s’embarrassent pas de la sociologie du Front populaire : personne ne peut rien tirer d’intéressant des interminables monologues de Léon Blum. Quant aux communistes, qui sont extrêmement fiers d’avoir pris l’initiative de la collaboration avec la bourgeoisie, ils présentent le Front populaire comme l’\emph{alliance du prolétariat avec les classes moyennes.} Quelle parodie du marxisme ! Non, le parti radical n’est pas le parti de la petite bourgeoisie. Il n’est pas davantage un « bloc de la moyenne et de la petite bourgeoisie », selon la définition absurde de la \emph{Pravda}. Non seulement la moyenne bourgeoisie exploite la petite bourgeoisie sur le plan économique comme sur le plan politique, mais elle est elle-même une agence du capital financier. Désigner sous le terme neutre de « bloc » des rapports politiques hiérarchiques fondés sur l’exploitation, c’est se moquer de la réalité. Un cavalier n’est pas un bloc entre l’homme et le cheval. Si le parti de Herriot-Daladier a des racines dans les masses petites bourgeoises et, dans une certaine mesure, jusque dans les milieux ouvriers, c’est uniquement dans le but de les duper dans l’intérêt du régime capitaliste. \emph{Les radicaux sont le parti démocratique de l’impérialisme français} — toute autre définition est un leurre.\par
La crise du système capitaliste désarme les radicaux en leur enlevant les moyens traditionnels qui leur permettaient d’endormir la petite bourgeoisie. Les « classes moyennes » commencent à sentir, sinon à comprendre, qu’on ne sauvera pas la situation par de misérables  réformes et qu’une refonte hardie du régime actuel est devenue nécessaire. Mais radicalisme et hardiesse vont ensemble comme l’eau et le feu. Le fascisme s’alimente avant tout de la méfiance croissante de la petite bourgeoisie à l’égard du radicalisme. On peut dire sans exagérer que le sort politique de la France ne tardera pas à se décider dans une large mesure selon la manière dont le radicalisme sera liquidé et selon que le fascisme ou le parti du prolétariat prendra sa succession, c’est-à-dire héritera de son influence sur les masses petites bourgeoises.\par
Un principe élémentaire de la stratégie marxiste est que l’alliance du prolétariat avec les petites gens des villes et des campagnes doit se réaliser uniquement dans la lutte irréductible contre la représentation parlementaire traditionnelle de la petite bourgeoisie. Pour gagner le paysan à l’ouvrier, il faut le détacher du politicien radical qui l’asservit au capital financier. Contrairement à cela, le Front populaire, complot de la bureaucratie ouvrière avec les pires exploiteurs politiques des classes moyennes, est tout simplement susceptible de tuer la foi des masses dans les méthodes révolutionnaires et de les jeter dans les bras de la contre-révolution fasciste.\par
Quelle que soit la difficulté qu’on ait à le croire, il n’en est pas moins vrai que quelques cyniques essayent de justifier la politique du Front populaire en se référant à Lénine qui, paraît-il, a démontré qu’on ne peut pas se passer de « compromis » et notamment d’accords avec d’autres partis. Pour les chefs de l’Internationale communiste d’aujourd’hui, outrager Lénine est devenu une règle ; ils piétinent la doctrine du fondateur du parti bolchevik et vont ensuite, à Moscou, s’incliner devant son mausolée.\par
Lénine a commencé sa tâche dans la Russie tsariste où non seulement les ouvriers, les paysans, les intellectuels,  mais de larges milieux bourgeois combattaient l’ancien régime. Si, d’une façon générale, la politique du Front populaire avait pu avoir sa justification, il semblerait que ce fût avant tout dans un pays qui n’avait pas encore fait sa révolution bourgeoise. Messieurs les falsificateurs feraient bien d’indiquer dans quelle phase, à quel moment et dans quelles circonstances le parti bolchévik a réalisé en Russie un simulacre de Front populaire ? Qu’ils fassent travailler leurs méninges et fouillent dans les documents historiques !\par
Les bolchéviks ont passé des accords d’ordre pratique avec les organisations révolutionnaires petites-bourgeoises pour le transport clandestin en commun des écrits révolutionnaires, parfois pour l’organisation en commun d’une manifestation dans la rue ou pour riposter aux bandes de pogromistes. Lors des élections à la Douma, ils ont eu recours, dans certaines circonstances et au deuxième degré\footnote{ \noindent L’élection des députés à la Douma se faisait par des collèges électoraux désignés au deuxième et au troisième degré. (N. d. T.)
 }, à des blocs électoraux avec les menchéviks ou avec les socialistes révolutionnaires. C’est tout. Ni « programmes » communs ni organismes permanents, ni renoncement à critiquer les alliés du moment. Ce genre d’accords et de compromis épisodiques, strictement limités à des buts précis — Lénine n’avait en vue que ceux-là — n’avait rien de commun avec le Front populaire qui représente un conglomérat d’organisations hétérogènes, une alliance durable de classes différentes liées pour toute une période — et quelle période ! — par une politique et un programme communs — par une politique de parade, de déclamation et de poudre aux yeux. A la première épreuve sérieuse, le Front populaire se brisera et toutes ses parties constitutives en sortiront avec de  profondes lézardes. La politique du Front populaire est une politique de trahison.\par
La règle du bolchévisme en ce qui concerne les blocs était la suivante : \emph{Marcher séparément, battre ensemble !} La règle des chefs de l’Internationale communiste d’aujourd’hui est celle-ci : \emph{Marcher ensemble pour être battu séparément}. Que ces messieurs se cramponnent à Staline et à Dimitrov, mais qu’ils s’arrangent pour laisser Lénine en paix.\par
Il est impossible de ne pas s’indigner quand on lit les déclarations de chefs vantards prétendant que le Front populaire a « sauvé » la France du fascisme ; en réalité, cela veut dire tout simplement que nos héros apeurés se sont sauvés par leurs encouragements mutuels d’une frayeur plus grande. Pour combien de temps ? Entre le premier soulèvement de Hitler et son arrivée au pouvoir, il s’est écoulé dix années marquées par des alternatives de flux et de reflux. A l’époque, les Blum et les Cachin allemands ont maintes fois proclamé leur « victoire » sur le national-socialisme. Nous ne les avons pas crus et nous n’avons pas eu tort. Néanmoins cette expérience n’a rien appris aux cousins français de Wels et de Thælmann. Certes, en Allemagne, les communistes n’ont pas participé au Front populaire qui groupait la social-démocratie, la bourgeoisie de gauche et le Centre catholique (« alliance du prolétariat avec les classes moyennes » !). En ce temps-là, l’Internationale communiste repoussait même les accords de combat entre organisations ouvrières contre le fascisme. Les résultats sont connus. Notre sympathie la plus chaleureuse pour Thælmann, en tant que prisonnier des bourreaux, ne peut pas nous empêcher de dire que sa politique, c’est-à-dire la politique de Staline, a plus fait pour la victoire de Hitler que la politique de Hitler lui-même. Ayant tourné casaque, l’Internationale communiste applique aujourd’hui en France la politique  suffisamment connue de la social-démocratie allemande. Est-il vraiment si difficile d’en prévoir les résultats ?\par
Les prochaines élections parlementaires, quelle que soit leur issue, n’apporteront pas, \emph{par elles-mêmes}, de changements sérieux dans la situation : en définitive, les électeurs sont priés de choisir entre un arbitre genre Laval et un arbitre genre Herriot-Daladier. Mais comme Herriot a tranquillement collaboré avec Laval et que Daladier les a soutenus tous les deux, la différence qui les sépare, si on la mesure à l’échelle des problèmes historiques qui sont posés, est insignifiante.\par
Croire que Herriot-Daladier sont capables de déclarer la guerre aux « deux cents familles » qui gouvernent la France, c’est duper impudemment le peuple. Les deux cents familles ne sont pas suspendues entre ciel et terre, elles constituent le couronnement organique du système du capital financier. Pour avoir raison des deux cents familles, il faut renverser le régime économique et politique au maintien duquel Herriot et Daladier ne sont pas moins intéressés que Tardieu et de La Rocque. Il ne s’agit pas de la lutte de la « nation » contre quelques féodaux, comme le représente l’\emph{Humanité,} mais de la lutte du prolétariat contre la bourgeoisie, de la lutte de classes qui ne peut être tranchée que par la révolution. Le complot antiouvrier des chefs du Front populaire est devenu le principal obstacle dans cette voie.\par
On ne peut pas dire d’avance combien de temps encore des ministères semi-parlementaires, semi-bonapartistes continueront en France à se succéder et par quelles phases précises le pays passera au cours de la prochaine période. Cela dépendra de la conjoncture économique nationale et mondiale, de l’atmosphère internationale, de la situation en U.R.S.S., du degré de stabilité du fascisme italien et allemand, de la marche des événements en Espagne, enfin — et ce n’est pas le facteur le  moins important — de la clairvoyance et de l’activité des éléments avancés du prolétariat français. Les convulsions du franc peuvent hâter le dénouement. Une coopération plus étroite de la France avec l’Angleterre est de nature à le retarder. De toute façon, l’agonie de la « démocratie » peut durer beaucoup plus de temps en France que la période préfasciste Brüning-Papen-Schleicher n’a duré en Allemagne ; mais elle ne cessera pas pour cela d’être une agonie. La démocratie sera balayée. La question est uniquement de savoir qui la balayera.\par
La lutte contre les « deux cents familles », contre le fascisme et la guerre — pour la paix, le pain, la liberté et autres belles choses — est ou bien un leurre ou bien une lutte pour renverser le capitalisme. Le problème de la conquête révolutionnaire du pouvoir se pose devant les travailleurs français non pas comme un objectif lointain, mais comme une tâche de la période qui s’ouvre. Or, les chefs socialistes et communistes non seulement se refusent à procéder à la mobilisation révolutionnaire du prolétariat, mais ils s’y opposent de toutes leurs forces. En même temps qu’ils fraternisent avec la bourgeoisie, ils traquent et expulsent les bolchéviks. Telle est la violence de leur haine de la révolution et de la peur qu’elle leur inspire ! Dans cette situation, le plus mauvais rôle est joué par les pseudo-révolutionnaires du type Marceau Pivert qui promettent de renverser la bourgeoisie, mais pas autrement qu’avec la permission de Léon Blum !\par
Toute la marche du mouvement ouvrier français au cours de ces douze dernières années a mis à l’ordre du jour la nécessité de créer un \emph{nouveau parti révolutionnaire.}\par
Vouloir deviner si les événements laisseront « suffisamment » de temps pour former le nouveau parti, c’est ce livrer à la plus stérile des occupations. Les ressources  de l’Histoire en ce qui concerne les possibilités diverses, les formes de transition, les étapes, les accélérations et les retards, sont inépuisables. Sous l’empire des difficultés économiques, le fascisme peut prendre l’offensive prématurément et subir une défaite. Un répit durable en résulterait. Au contraire, il peut par prudence adopter trop longtemps une attitude d’attente et de ce fait offrir de nouvelles chances aux organisations révolutionnaires. Le Front populaire peut se briser sur ses contradictions avant que le fascisme soit capable de livrer une bataille générale : il en résulterait une période de regroupements et de scissions dans les partis ouvriers et une cristallisation rapide d’une avant-garde révolutionnaire. Les mouvements spontanés des masses, selon l’exemple de Toulon et de Brest, peuvent prendre une grande ampleur et créer un point d’appui sûr pour le levier révolutionnaire. Enfin, même une victoire du fascisme en France, ce qui théoriquement n’est pas impossible, ne veut pas dire que celui-ci restera au pouvoir un millier d’années, comme Hitler l’annonce, ni que cette victoire lui accordera une période comme celle dévolue à Mussolini. Si le crépuscule du fascisme commençait en Italie ou en Allemagne, il ne tarderait pas à s’étendre à la France. Dans l’hypothèse la moins favorable, construire un parti révolutionnaire c’est hâter l’heure de la revanche. Les sages qui se débarrassent de cette tâche urgente, en prétendant que les « conditions ne sont pas mûres », ne font que démontrer qu’eux-mêmes ne sont pas mûrs pour ces conditions.\par
Les marxistes français, comme ceux de tous les pays, doivent, dans un certain sens, recommencer à nouveau, mais à un degré historiquement plus élevé que leurs prédécesseurs. La chute de l’Internationale communiste, plus honteuse que la chute de la social-démocratie en 1914, gêne considérablement au début la marche en avant. Le recrutement des nouveaux cadres se fait avec lenteur au  cours d’une lutte cruelle dans la classe ouvrière contre le front uni de la bureaucratie réactionnaire et patriote. D’un autre côté, ces difficultés, qui ne se sont pas abattues par hasard sur le prolétariat, constituent un facteur important pour une bonne sélection et une solide trempe des premières phalanges du nouveau parti et de la nouvelle Internationale.\par
Seule une infime partie des cadres de l’Internationale communiste avaient commencé leur éducation révolutionnaire au début de la guerre, avant la Révolution d’Octobre. Tous ceux-là, presque sans exception, se trouvent actuellement en dehors de la III\textsuperscript{e} Internationale. La lignée suivante a adhéré à la Révolution d’Octobre quand celle-ci était déjà triomphante : c’était plus facile. Mais de cette [{\corr deuxième}] lignée elle-même il ne reste que peu de chose. La majeure partie des cadres actuels de l’Internationale communiste a adhéré non pas au programme bolchevik, non pas au drapeau révolutionnaire, mais à la bureaucratie soviétique. Ce ne sont pas des lutteurs, mais des fonctionnaires dociles, des aides de camp, des grooms. De là vient que la III\textsuperscript{e} Internationale se décompose d’une manière si peu glorieuse dans une situation historique riche de grandioses possibilités révolutionnaires.\par
La IV\textsuperscript{e} Internationale se hisse sur les épaules de ses trois devancières. Elle reçoit des coups, de front, de flanc et par derrière. Les carriéristes, les poltrons et les philistins n’ont rien à faire dans ses rangs. Une portion, inévitable au début, de sectaires et d’aventuriers s’en ira au fur et à mesure que le mouvement grandira. Laissons les pédants et les sceptiques hausser les épaules au sujet des « petites » organisations qui publient de « petits » journaux et lancent des défis au monde entier. Les révolutionnaires sérieux passeront à côté d’eux avec mépris.  La Révolution d’Octobre avait, elle aussi, commencé à marcher dans des souliers d’enfant...\par
Les puissants partis russes socialiste-révolutionnaire et menchévik qui, pendant des mois, formèrent un « Front populaire » avec les cadets, tombèrent en poussière sous les coups d’une « poignée de fanatiques » du bolchévisme. La social-démocratie allemande, le Parti communiste allemand et la social-démocratie autrichienne ont trouvé une mort sans gloire sous les coups du fascisme. L’époque qui va commencer pour l’humanité européenne ne laissera pas trace dans le mouvement ouvrier de tout ce qui est équivoque et gangrené. Tous ces Jouhaux, Citrine, Blum, Cachin, Vandervelde, Caballero ne sont que des fantômes. Les sections de la II\textsuperscript{e} et III\textsuperscript{e} Internationale quitteront la scène sans éclat les unes après les autres. Un nouveau et grandiose regroupement des rangs ouvriers est inévitable. Les jeunes cadres révolutionnaires acquerront de la chair et du sang. La victoire n’est concevable que sur la base des méthodes bolchéviks...
 \section[{L’étape décisive, (5 juin 1936)}]{L’étape décisive \\
(5 juin 1936)}\phantomsection
\label{p6}\renewcommand{\leftmark}{L’étape décisive \\
(5 juin 1936)}

\noindent Le rythme des événements en France s’est brusquement accéléré. Auparavant il fallait apprécier le caractère \emph{pré-révolutionnaire} de la situation sur la base de l’analyse théorique et de divers symptômes politiques. Maintenant, les faits parlent d’eux-mêmes. On peut dire sans exagération que, dans toute la France, il n’y a que deux partis dont les chefs ne voient, ne comprennent ou ne veulent pas voir toute la profondeur de la crise révolutionnaire : les Partis « socialiste » et « communiste ». On peut, assurément, leur ajouter les chefs syndicaux « indépendants ». Les masses ouvrières créent maintenant une situation révolutionnaire à l’aide de l’action directe. La bourgeoisie craint mortellement le développement des événements et prend dans les coulisses, sous le nez du nouveau gouvernement, toutes les mesures nécessaires de résistance, de salut, de tromperie, d’écrasement et de revanche sanglante. Seuls, les chefs « socialistes » et « communistes » continuent à bavarder sur le Front populaire, comme si la lutte des classes n’avait pas déjà renversé leur méprisable château de cartes.\par
Blum déclare : « Le pays a donné mandat au Front populaire, et nous ne pouvons sortir des cadres de ce mandat ». Blum trompe son propre parti et tente de tromper le prolétariat. Les stalinistes. (ils se nomment toujours « communistes ») l’aident à le faire. En fait, socialistes et communistes utilisent les trucs, les ficelles et les nœuds  coulants de la mécanique électorale, pour venir à bout des masses laborieuses dans l’intérêt de l’alliance avec le radicalisme bourgeois. L’essence politique de la crise s’exprime dans le fait que \emph{le peuple a la nausée des radicaux et de leur III\textsuperscript{e} République}. C’est ce que tentent d’utiliser les fascistes. Qu’ont donc fait socialistes et communistes ?\par
Ils se sont portés garants des radicaux devant le peuple, ont représenté les radicaux comme injustement calomniés, ont fait croire aux ouvriers et aux paysans que tout leur salut était dans le ministère Daladier. C’est sur ce diapason que fut orchestrée toute la campagne électorale. Comment ont répondu les masses ? Elles ont donné une énorme augmentation de voix et de mandats aux communistes, en tant qu’extrême gauche. Les tournants et les zigzags des mercenaires de la diplomatie soviétique ne sont pas compris des masses, car ils ne sont pas vérifiés par leur propre expérience. \emph{Les masses n’apprennent que dans l’action. Elles n’ont pas le temps de faire de connaissances théoriques.} Quand un million et demi d’électeurs donnent leur voix aux communistes, leur majorité dit à ceux-ci : « Nous voulons que vous fassiez en France ce que les bolcheviks russes ont fait chez eux en octobre 1917 ». Telle est la volonté réelle de la partie la plus active de la population, de celle qui est capable de lutter et d’assurer l’avenir de la France. Telle est la première leçon des élections.\par
Les socialistes ont à peu près maintenu leur ancien nombre de voix, malgré la scission de l’important groupé néo. Dans cette question aussi, les masses ont donné à leurs « chefs » une grande leçon. Les néos voulaient le cartel à tout prix, c’est-à-dire la collaboration avec la bourgeoisie républicaine au nom du salut et de l’épanouissement de la « République ». C’est précisément sur cette ligne qu’ils se sont scindés des socialistes et se sont présentés en concurrents aux élections. Les électeurs se sont  éloignés d’eux. Les néos se sont effondrés. Il y a deux ans, nous avons prédit que le développement politique à venir tuerait avant tout les petits groupes qui gravitaient autour des radicaux. Ainsi, dans le conflit entre socialistes et néos, les masses ont jugé et ont rejeté le groupe qui proposait le plus systématiquement et le plus résolument, le plus bruyamment et le plus ouvertement, l’alliance avec la bourgeoisie. Telle est la seconde leçon des élections.\par
Le Parti socialiste n’est un parti ouvrier ni par sa politique, ni par sa composition sociale. C’est le parti des nouvelles couches moyennes (fonctionnaires, employés, etc.), partiellement de la petite bourgeoisie et de l’aristocratie ouvrière. Une analyse sérieuse de la statistique électorale démontrerait indubitablement que les socialistes ont cédé aux communistes une importante fraction d’ouvriers et de paysans pauvres et en échange ont reçu des radicaux d’importants groupes des classes moyennes. Cela signifie que le mouvement de la petite bourgeoisie va des radicaux vers la gauche — vers les socialistes et les communistes — tandis que des groupes de la grande et moyenne bourgeoisie se séparent des radicaux vers la droite. Le regroupement s’opère selon les axes des classes, et non pas suivant la ligne artificielle du « Front populaire ». La polarisation rapide des rapports politiques souligne le caractère révolutionnaire de la crise. Telle est la troisième leçon, la leçon fondamentale.\par
L’électeur a manifesté, par conséquent, sa volonté — autant qu’il a eu en général la possibilité de la manifester dans la camisole de force du parlementarisme — non pas en faveur de la politique du Front populaire, mais contre elle. Certes, au second tour, socialistes et communistes, en retirant leurs candidatures en faveur des bourgeois radicaux, ont encore plus altéré la volonté politique des travailleurs de France. Malgré cela, les radicaux sont sortis des épreuves les côtes rompues, perdant un bon tiers de  leurs mandats. Le \emph{Temps} dit : « C’est parce qu’ils sont entrés dans un bloc avec les révolutionnaires ». Daladier réplique : « Sans le Front populaire nous aurions perdu plus ». Daladier a incontestablement raison. Si socialistes et communistes avaient mené une politique de classe, c’est-à-dire avaient lutté pour l’alliance des ouvriers et des éléments semi-prolétariens de la ville et du village contre toute la bourgeoisie, y compris aussi son aile radicale pourrie, ils auraient eu considérablement plus de voix, et les radicaux seraient revenus à la Chambre en un groupe insignifiant.\par
Tous les faits politiques témoignent que, ni dans les rapports sociaux de la France, ni dans l’état d’esprit politique des masses, il n’y a d’appui pour le Front populaire. Cette politique est imposée par en-haut : par la bourgeoisie radicale, par les maquignons et les affairistes socialistes, par les diplomates soviétiques et leurs laquais « communistes ». De leurs forces réunies ils font tout ce que l’on peut faire à l’aide du plus malhonnête de tous les systèmes électoraux, pour tromper et leurrer politiquement les masses populaires et altérer leur volonté réelle. Néanmoins les masses ont su, même dans ces conditions, montrer qu’elles veulent non pas une coalition avec les radicaux, mais le rassemblement des travailleurs contre \emph{toute} la bourgeoisie.\par
Si, dans toutes les circonscriptions électorales, où socialistes et communistes se sont écartés en faveur des radicaux, avaient été posées au second tour des candidatures ouvrières révolutionnaires, elles auraient recueilli un nombre très important de voix. Par malheur, il ne s’est pas trouvé d’organisation capable d’une telle initiative. Ceci démontre que les groupes révolutionnaires, centraux et locaux, restent en dehors de la dynamique des événements et préfèrent s’abstenir et s’esquiver là où il faut agir. C’est  triste ! Mais l’orientation générale des masses est malgré tout absolument claire.\par
Socialistes et communistes avaient préparé de toutes leurs forces un ministère Herriot ; à la rigueur, un ministère Daladier. Qu’ont donc fait les masses ? Elles ont \emph{imposé }aux socialistes et aux communistes un ministère Blum. Est-ce que ce n’est pas un vote direct contre la politique du Front populaire ?\par
Ou peut-être faut-il de nouvelles preuves ? La manifestation à la mémoire des communards, semble-t-il, a dépassé cette année toutes les manifestations populaires, qu’avait vues Paris. Cependant, les radicaux n’avaient et ne pouvaient avoir avec cette manifestation le moindre rapport. Les masses laborieuses de Paris, avec un instinct politique sans égal, ont montré qu’elles sont prêtes à être le double là où elles ne sont pas contraintes de souffrir la fraternisation répugnante de leurs chefs avec les exploiteurs bourgeois. La puissance de la manifestation du 24 mai est le désaveu le plus convaincant, le plus infaillible du Paris ouvrier à la politique du Front populaire.\par
Mais sans le Front populaire, le Parlement, où socialistes et communistes n’ont malgré tout pas la majorité, ne serait pas viable, et les radicaux — oh ! malheur — se trouveraient rejetés « dans les bras de la réaction ». Ce raisonnement est pleinement digne des philistins poltrons, qui se trouvent à la tête des Partis socialiste et communiste. \emph{La non-viabilité du Parlement est la conséquence inévitable du caractère révolutionnaire de la crise}. A l’aide d’une série de fourberies politiques, on a réussi à masquer tant bien que mal cette non-viabilité ; mais elle se révélera malgré tout demain. Pour ne pas pousser les radicaux réactionnaires jusqu’à la moelle des os « dans les bras de la réaction », il faut s’unir avec les radicaux pour la défense du capital. C’est en cela, et seulement en  cela, que réside la mission du Front populaire. Mais les ouvriers l’empêchent.\par
Le Parlement n’est pas viable, parce que la crise actuelle n’ouvre aucune issue dans la voie parlementaire. Et, de nouveau, les masses travailleuses françaises, avec le sûr instinct révolutionnaire qui les distingue, ont saisi infailliblement ce trait important de la situation. A Toulon et à Brest, elles ont tiré les premiers signaux d’alarme. Les protestations des soldats contre le « rabiot » (prolongation du service militaire) signifiaient la forme d’action directe des masses la plus dangereuse pour l’ordre bourgeois. Enfin, dans les journées où le congrès socialiste acceptait unanimement (en commun avec le phraseur creux Marceau Pivert) le mandat du « Front populaire » et remettait ce mandat à Léon Blum ; dans les journées où Blum se regardait dans la glace de tous côtés, faisait des gestes prégouvernementaux, poussaient des exclamations prégouvernementales et les commentait dans des articles, où il s’agissait toujours de Blum et jamais du prolétariat — précisément dans ces journées, une vague magnifique, véritablement printanière, de grèves a déferlé sur la France. Ne trouvant pas de direction et marchant sans elle, les ouvriers ont accompli avec hardiesse et assurance l’occupation des usines après l’arrêt du travail.\par
Le nouveau gendarme du capital, Salengro, même avant de prendre le pouvoir, a déclaré (absolument tout comme l’aurait fait Herriot, Laval, Tardieu ou de La Rocque) qu’il défendrait « l’ordre contre l’anarchie ». Cet individu appelle ordre l’anarchie capitaliste. Il nomme anarchie la lutte pour l’ordre socialiste. L’occupation, quoique encore pacifique, des fabriques et des usines par les ouvriers a une énorme importance symptomatique. Les travailleurs disent : « Nous voulons être les maîtres dans les établissements où nous n’avons été jusqu’à maintenant que des esclaves ».\par
 Mortellement effrayé, Léon Blum, désirant faire peur aux ouvriers, dit: « Je ne suis pas Kérensky; et à la place de Kérensky en France viendrait non pas Lénine, mais quelqu’un d’autre ». On peut imaginer que le Kérensky russe avait compris la politique de Lénine ou avait prévu son arrivée. En fait, exactement comme Blum, Kérensky faisait croire aux ouvriers qu’au cas de sa chute viendrait au pouvoir non le bolchévisme, mais « quelqu’un d’autre ». Précisément là où Blum veut se distinguer de Kérensky, il l’imite servilement. Il est impossible, pourtant, de ne pas reconnaître que, dans la mesure où l’affaire dépend de Blum, il fraye en réalité la voie au fascisme et non au prolétariat.\par
Plus criminelle et plus infâme que tout est dans cette situation la conduite des communistes: ils ont promis de soutenir à fond le gouvernement Blum, sans y entrer. « Nous sommes des révolutionnaires trop terribles — disent Cachin et Thorez — nos collègues radicaux peuvent mourir d’effroi, il vaut mieux que nous nous tenions à l’écart ». Le ministérialisme dans les coulisses est dix fois pire que le ministérialisme ouvert et déclaré. En fait, les communistes veulent conserver leur indépendance extérieure, pour assujettir d’autant mieux les masses ouvrières au Front populaire, c’est-à-dire à la discipline du capital. Mais là aussi un obstacle apparaît avec la lutte des classes. La simple et honnête grève de masse détruit impitoyablement la mystique et la mystification du Front populaire. Il a déjà reçu un coup mortel, et dès maintenant il ne peut plus que périr.\par
Dans la voie parlementaire, il n’y a pas d’issue. Blum n’inventera pas la poudre, car il craint la poudre. Les machinations ultérieures du Front populaire ne peuvent que prolonger l’agonie du parlementarisme et donner à de La Rocque un délai pour se préparer à un nouveau coup,  plus sérieux, si... si les révolutionnaires ne le devancent pas.\par
Après le 6 février 1934, quelques camarades impatients pensaient que le dénouement viendrait « demain », et que c’était pourquoi il fallait immédiatement faire quelque miracle. Une telle « politique » ne pouvait rien donner, sinon des aventures et des zigzags, qui ont extraordinairement entravé le développement du parti révolutionnaire. On ne peut pas rattraper le temps perdu. Mais il ne faut plus perdre de temps désormais, car il en reste peu. Même aujourd’hui, nous ne fixerons pas de délai. Mais après la grande vague de grève, les événements ne peuvent se développer que du côté de la révolution ou du côté du fascisme. L’organisation qui ne trouvera pas appui dans le mouvement gréviste actuel, qui ne saura pas se lier étroitement aux ouvriers en lutte, est indigne du nom d’organisation révolutionnaire. Ses membres feraient mieux de se chercher une place dans les hospices ou dans les loges maçonniques (avec la protection de M. Pivert) !\par
En France, il y a d’assez nombreux messieurs des deux sexes, ex-communistes, ex-socialistes, ex-syndicalistes, qui vivent en groupes et en cliques, échangent entre quatre murs leurs impressions sur les événements et pensent que le moment n’est pas venu de leur participation éclairée. « Il est encore trop tôt. » Et quand viendra de La Rocque, ils diront : « Il est maintenant trop tard ». Des raisonneurs stériles de ce genre sont nombreux, en particulier parmi l’aile gauche du syndicat des instituteurs. Ce serait le plus grand crime de perdre pour ce public, ne fût-ce qu’une seule minute. Que les morts enterrent les morts !\par
Le sort de la France ne se décide maintenant ni au Parlement, ni dans les salles de rédaction des journaux conciliateurs, réformistes et stalinistes, ni dans les cercles de sceptiques, de geignards et de phraseurs. Le sort de la  France se décide dans les usines, qui ont su, par l’action, montrer la voie de l’issue de l’anarchie capitaliste vers l’ordre socialiste. La place des révolutionnaires est dans les usines !\par
Le dernier congrès de l’Internationale communiste, dans sa cuisine éclectique, a posé l’une à côté de l’autre la coalition avec les radicaux et la création de comités d’action de masse, c’est-à-dire de soviets embryonnaires. Dimitrov, comme aussi ses inspirateurs, s’imaginent qu’on peut combiner la collaboration des classes avec la lutte des classes, le bloc avec la bourgeoisie et la lutte pour le pouvoir du prolétariat, l’amitié avec Daladier et l’édification des soviets. Les stalinistes français ont donné aux comités d’action le nom de comités de Front populaire, s’imaginant qu’ainsi ils conciliaient la lutte révolutionnaire avec la défense de la démocratie bourgeoise. Les grèves actuelles mettent radicalement en pièces cette pitoyable illusion. Les radicaux craignent les comités. Les socialistes craignent l’effroi des radicaux. Les communistes craignent la peur des uns et des autres. Le mot d’ordre des comités ne peut être abordé que par une organisation véritablement révolutionnaire, absolument dévouée aux masses, à leur cause, à leur lutte. Les ouvriers français ont de nouveau montré qu’ils sont dignes de leur réputation historique. Il faut leur faire confiance. Les soviets sont toujours nés des grèves. La grève de masse est l’élément naturel de la révolution prolétarienne. Les comités d’action ne peuvent actuellement être rien d’autre que les comités des grévistes, qui occupent les entreprises. D’atelier en atelier, d’usine en usine, de quartier en quartier, de ville en ville, les comités d’action doivent établir entre eux une liaison étroite, se réunir en conférences par villes, par groupes de production, par arrondissements, pour terminer par un congrès de tous les comités d’action de France. C’est cela qui sera le nouvel ordre, qui doit remplacer l’anarchie actuelle.
 \section[{La révolution française a commencé, (9 juin 1936)}]{La révolution française a commencé \\
(9 juin 1936)}\phantomsection
\label{p7}\renewcommand{\leftmark}{La révolution française a commencé \\
(9 juin 1936)}

\noindent Jamais la radio ne s’est trouvée être aussi précieuse que dans ces derniers jours. Elle donne la possibilité de suivre d’un lointain village de Norvège les battements du pouls de la révolution française. Il serait plus exact de dire : le reflet de ces battements dans la conscience et dans la voix de messieurs les ministres, les secrétaires syndicaux et autres chefs mortellement effrayés.\par
Les mots de « révolution française » peuvent sembler exagérés. Mais non ! Ce n’est pas une exagération. C’est précisément ainsi que naît la révolution. En général elle ne peut pas naître autrement. La révolution française a commencé.\par
Certes, Léon Jouhaux, à la suite de Léon Blum, assure à la bourgeoisie qu’il s’agit d’un mouvement purement économique, dans les cadres stricts de la loi. Sans doute, les ouvriers sont les maîtres des usines pendant la grève, établissant leur contrôle sur la propriété et son administration. Mais on peut fermer les yeux sur ce regrettable « détail ». Dans l’ensemble ce sont « des grèves économiques, mais non pas politiques », affirment messieurs les chefs. Cependant, sous l’effet de grèves « non-politiques », toute la situation politique du pays change radicalement. Le gouvernement décide d’agir avec une promptitude à laquelle il n’avait pas songé la veille : car,  selon les paroles de Blum, la véritable force est patiente ! Les capitalistes font preuve d’un esprit accommodant tout à fait inattendu. Toute la contre-révolution en attente se cache derrière le dos de Blum et de Jouhaux. Et tout ce miracle est produit par... de simples grèves « corporatives ». Qu’est-ce que ce serait si les grèves avaient eu un caractère politique ?\par
Mais non, les chefs disent une contre-vérité. La corporation embrasse les ouvriers d’une profession donnée, les séparant des autres professions. Le trade-unionisme et le syndicalisme réactionnaire tendent tous leurs efforts à maintenir le mouvement ouvrier dans les cadres corporatifs. C’est là-dessus que s’assoit la dictature de fait de la bureaucratie syndicale sur la classe ouvrière (la pire de toutes les dictatures !) avec la dépendance servile de la clique Jouhaux-Racamond envers l’Etat capitaliste. L’essence du mouvement actuel réside précisément dans le fait qu’il brise les cadres professionnels, corporatifs et locaux, en élevant au-dessus d’eux les revendications, les espoirs, la volonté de \emph{tout} le prolétariat. Le mouvement prend le caractère d’une épidémie. La contagion s’étend d’usine en usine, de corporation en corporation, de quartier en quartier. Toutes les couches de la classe ouvrière se répondent, pour ainsi dire, l’une à l’autre. Les métallurgistes ont commencé : c’est l’avant-garde. Mais la force du mouvement réside dans le fait qu’à une petite distance de l’avant-garde suivent les lourdes réserves de la classe, y compris les professions les plus diverses, puis son arrière-garde, que d’ordinaire messieurs les chefs parlementaires et syndicaux oublient complètement. Ce n’est pas pour rien que le \emph{Peuple} reconnaissait ouvertement que pour lui plusieurs catégories particulièrement mal payées de la population parisienne étaient apparues comme un fait complètement « inattendu ». Cependant c’est précisément dans les profondeurs de ces couches les plus exploitées que  se cachent des sources intarissables d’enthousiasme, de dévouement, de courage. Le fait même de leur éveil est le signe infaillible d’un grand combat. Il faut trouver accès à ces couches à tout prix !\par
S’arrachant des cadres corporatifs et locaux, le mouvement gréviste est devenu redoutable non seulement pour la société bourgeoise, mais aussi pour sa propre représentation parlementaire et syndicale, qui, actuellement, est avant tout occupée à ne pas voir la réalité. Selon une légende historique, à la question de Louis XVI : « Mais c’est une révolte ? », un des courtisans répondit : « Non, sire, c’est une révolution ». Actuellement, à la question de la bourgeoisie : « C’est une révolte ? », ses courtisans répondent : « Non, ce ne sont que des grèves corporatives ». Rassurant les capitalistes, Blum et Jouhaux se rassurent eux-mêmes. Mais les paroles ne peuvent rien. Certes, au moment où ces lignes paraîtront dans la presse, la première vague peut s’être calmée. Apparemment la vie rentrera, semblera-t-il, dans son ancien lit. Mais cela ne change rien à l’affaire. Ce qui s’est passé, ce n’est pas des grèves corporatives. Ce n’est même pas des grèves. C’est la \emph{grève.} C’est le rassemblement au grand jour des opprimés contre les oppresseurs. C’est le début classique de la révolution.\par
Toute l’expérience passée de la classe ouvrière, son histoire d’exploitation, de malheurs, de luttes, de défaites, revit sous le choc des événements et s’élève dans la conscience de chaque prolétaire, même du plus arriéré, le poussant dans les rangs communs. Toute la classe est entrée en mouvement. Il est impossible d’arrêter avec des paroles cette masse gigantesque. La lutte doit aboutir soit à la plus grande des victoires, soit au plus terrible des écrasements.\par
 
\asterism

\noindent Le \emph{Temps} a appelé la grève les « \emph{grandes manœuvres de la révolution} ». C’est incomparablement plus sérieux que ce que disent Blum et Jouhaux. Mais même la définition du \emph{Temps} est malgré tout inexacte, car elle est dans un certain sens exagérée. Des manœuvres présupposent l’existence d’un commandement, d’un état-major, d’un plan. Il n’y a rien eu de tel dans la grève. Les centres des organisations ouvrières, y compris le Parti communiste, ont été pris à l’improviste. Ils craignent avant tout que la grève ne dérange tous leurs plans. La radio transmet une phrase remarquable de Marcel Cachin : « Nous sommes, les uns et les autres, devant le fait de la grève ». En d’autres termes, la grève est notre malheur commun. Avec ces paroles le sénateur menaçant convainc les capitalistes de faire des concessions pour ne pas exacerber la situation. Les secrétaires parlementaires et syndicaux, qui s’adaptent à la grève avec l’intention de l’étouffer le plus tôt possible, se trouvent en réalité en dehors de la grève, s’agitent en l’air et ne savent pas eux-mêmes s’ils retomberont à terre sur les pieds ou sur la tête. La masse éveillée n’a pas encore d’état-major révolutionnaire.\par
Le véritable état-major est chez l’ennemi de classe. Cet état-major ne coïncide nullement avec le gouvernement Blum, quoiqu’il s’en serve très habilement La réaction capitaliste joue actuellement un grand jeu risqué, mais elle le joue savamment. Au moment présent elle joue à qui perd gagne : « Cédons aujourd’hui à toutes ces revendications désagréables, qui ont rencontré l’approbation commune de Blum, de Jouhaux et de Daladier. De la reconnaissance de principe à la réalisation de fait il y a encore un grand chemin. Il y a le Parlement, il y a le Sénat, il y a l’administration — tout cela ce sont des machines d’obstruction. Les masses manifesteront de l’impatience et  tenteront de serrer plus fort. Daladier se séparera de Blum. Thorez tentera de se détacher à gauche. Blum et Jouhaux se sépareront des masses. Alors nous nous rattraperons de toutes les concessions actuelles et même avec usure ». Ainsi raisonne le véritable état-major de la contre-révolution : les fameuses « 200 familles » et leurs stratèges mercenaires. Elles agissent selon un plan. Et ce serait une légèreté de dire que leur plan est sans base solide. Non, avec l’aide de Blum, de Jouhaux et de Cachin, la contre-révolution \emph{peut} arriver au but.\par
Le fait que le mouvement des masses atteint, sous une forme improvisée, des dimensions si grandioses et un effet politique si grand, souligne au mieux le caractère profond, organique, véritablement révolutionnaire de la vague de grèves. C’est en cela qu’est le gage de la durée du mouvement, de sa ténacité, de l’inéluctabilité d’une série de vagues croissantes. Sans cela la victoire serait impossible. Mais tout cela est insuffisant pour vaincre. Contre l’état-major et le plan des « 200 familles », il faut l’état-major et le plan de la révolution prolétarienne. Ni l’un, ni l’autre n’existent encore. Mais ils peuvent être créés. Il existe toutes les prémisses et tous les éléments d’une nouvelle cristallisation des masses.\par

\asterism

\noindent Le déclenchement de la grève est provoqué, dit-on, par les « espoirs » dans le gouvernement de Front populaire. Ce n’est qu’un quart de la vérité, et même moins. S’il ne s’était agi que de pieux \emph{espoirs}, les ouvriers n’auraient pas couru le risque de la lutte. Dans la grève s’exprime, avant tout, la \emph{méfiance} ou le \emph{manque de confiance} des ouvriers, sinon dans la bonne volonté du gouvernement, du moins dans sa capacité de briser les obstacles et de venir à bout de ses tâches. Les prolétaires veulent « aider » le gouvernement,  mais à leur façon, à la façon prolétarienne. Assurément, ils n’ont pas encore pleine conscience de leur force. Mais ce serait une grossière caricature de dessiner la chose, comme si la masse n’était guidée que par des « espoirs » en Blum. Il ne lui est pas facile de rassembler ses pensées sous l’oppression des vieux chefs, qui s’efforcent de la faire rentrer le plus tôt possible dans la vieille ornière de l’esclavage et de la routine. Malgré tout, le prolétariat français ne reprend pas l’histoire au commencement. Toujours et partout la grève a fait apparaître à la surface les ouvriers les plus conscients et les plus hardis. L’initiative leur appartient. Ils agissent encore prudemment, tâtant le terrain. Les détachements avancés s’efforcent de ne pas se couper en avant, pour ne pas s’isoler. L’écho amical qui leur vient de l’arrière leur donne courage. L’écho que se font les différentes parties de la classe est devenu un essai d’auto-mobilisation. Le prolétariat lui-même a le plus grand besoin de cette manifestation de sa propre force. Les succès pratiques obtenus, quelque incertains qu’ils soient en eux-mêmes, doivent extraordinairement élever la confiance des masses en elles-mêmes, surtout des couches les plus arriérées et les plus opprimées.\par
La principale conquête de la première vague est dans le fait que sont apparus des chefs dans les ateliers et les usines. Ont été créés les éléments d’états-majors locaux et de quartiers. La masse les connaît. Ils se connaissent l’un l’autre. Les véritables révolutionnaires chercheront liaison avec eux. Ainsi la première auto-mobilisation de la masse a marqué et en partie désigné les premiers éléments d’une direction révolutionnaire. La grève a secoué, ranimé, renouvelé tout le gigantesque organisme de la classe. La vieille écaille organisationnelle est encore loin d’être disparue, au contraire, elle se maintient avec assez d’obstination. Mais sous elle apparaît déjà une nouvelle peau.\par
Sur le rythme des événements, qui, sans aucun doute,  s’accéléreront, nous ne dirons rien maintenant. Ici ne sont encore possibles que des suppositions et des conjectures. La seconde vague, son déclenchement et sa tension permettront, sans aucun doute, de faire un pronostic beaucoup plus concret que ce qui est actuellement possible. Mais une chose est claire par avance : la seconde vague sera loin d’avoir le même caractère pacifique, presque débonnaire, printanier, que la première. Elle sera plus mûre, plus tenace et plus âpre, car elle sera provoquée par la déception des masses dans les résultats pratiques de la politique du Front populaire et de la première offensive. Dans le gouvernement se produiront des fissures, de même que dans la majorité parlementaire. La contre-révolution deviendra du coup plus assurée et plus insolente. Il ne faut pas attendre de nouveaux succès faciles des masses. En face du danger de perdre ce qui a semblé être conquis ; devant la résistance croissante de l’ennemi ; devant la confusion et la débandade de la direction officielle, les masses sentiront l’ardente nécessité d’avoir un programme, une organisation, un plan, un état-major. C’est à cela qu’il faut se préparer et préparer les ouvriers avancés. Dans l’atmosphère de la révolution, la rééducation de la masse, la sélection des cadres et leur trempe s’accompliront rapidement.\par
Un état-major révolutionnaire ne peut naître au moyen de combinaisons de sommets. L’organisation de combat ne coïnciderait pas avec le parti, même s’il existait en France un parti révolutionnaire de masse, car le mouvement est incomparablement plus large que le parti. L’organisation ne peut pas non plus coïncider avec les syndicats, car les syndicats n’embrassent qu’une partie insignifiante de la classe et sont soumis à une bureaucratie archi-réactionnaire. La nouvelle organisation doit répondre à la nature du mouvement lui-même, refléter la masse en lutte, exprimer sa volonté la plus ferme. Il s’agît d’un gouvernement direct de la classe révolutionnaire. Il n’est pas besoin d’inventer  ici de nouvelles formes : il y a des précédents historiques. Les ateliers et les usines élisent leurs députés, qui se réunissent pour élaborer en commun les plans de la lutte et pour la diriger. Il n’y a même pas à inventer le nom d’une telle organisation : \emph{c’est les soviets de députés ouvriers}.\par
La principale masse des ouvriers révolutionnaires marche actuellement derrière le Parti communiste. Ils ont crié plus d’une fois dans le passé : « Les Soviets partout ! ». Leur majorité a pris, sans aucun doute, ce mot d’ordre au sérieux. Il fut un temps où nous pensions que ce mot d’ordre n’était pas opportun. Mais, actuellement, la situation est radicalement changée. Le puissant conflit des classes va vers un dénouement redoutable. Celui qui hésite, qui perd du temps, est un traître. Il faut choisir entre la plus grande des victoires historiques et la plus terrible des défaites. Il faut préparer la victoire. « Les Soviets partout » ? D’accord. Mais il est temps de passer des paroles aux actes !
 \section[{Devant la seconde étape, (9 juillet 1936)}]{Devant la seconde étape \\
(9 juillet 1936)}\phantomsection
\label{p8}\renewcommand{\leftmark}{Devant la seconde étape \\
(9 juillet 1936)}

\noindent Il faut le répéter encore une fois : la presse sérieuse du capital, comme le \emph{Temps} de Paris ou le \emph{Times} de Londres, a apprécié de façon beaucoup plus juste et perspicace l’importance des événements de juin en France et en Belgique, que ne l’a fait la presse du Front populaire. Tandis que les journaux officiels socialistes et communistes, à la suite de Léon Blum, parlent de la « réforme pacifique du régime social de la France », qui a commencé, la presse conservatrice affirme que la révolution s’est ouverte en France et qu’à l’une des prochaines étapes elle prendra inévitablement des formes violentes. Il serait inexact de voir dans ce pronostic seulement ou surtout une tentative d’effrayer les propriétaires. Les représentants du grand capital savent regarder la lutte sociale de façon très réaliste. Les politiciens petits-bourgeois, au contraire, prennent volontiers leurs désirs pour la réalité : se tenant entre les classes fondamentales, le capital financier et le prolétariat, messieurs les « réformateurs » proposent aux deux adversaires de s’entendre sur la ligne moyenne, qu’ils ont à grand’peine élaborée à l’état-major du Front populaire et qu’eux-mêmes interprètent de façon différente. Il leur faudra, pourtant, se convaincre très rapidement qu’il est beaucoup plus facile de concilier les contradictions des classes dans des articles leaders que dans le travail gouvernemental, surtout au plus fort de la crise sociale.\par
 Au Parlement on a lancé à Blum l’accusation ironique qu’il avait mené les pourparlers au sujet des revendications des grévistes avec les représentants des « deux cents familles ». « Et avec qui m’aurait-il fallu parler ? » répondit ingénieusement le président du Conseil. C’est vrai, s’il faut mener des pourparlers avec la bourgeoisie, alors il faut choisir les véritables maîtres, qui sont capables de trancher pour eux et d’ordonner aux autres. Mais il était alors inutile de leur déclarer bruyamment la guerre ! Dans les cadres du régime bourgeois, de ses lois, de sa mécanique, chacune des « deux cents familles » est incomparablement plus puissante que le gouvernement Blum. Les magnats de la finance représentent le couronnement du système bourgeois de la France, et le gouvernement Blum, malgré tous ses succès électoraux, ne « couronne » qu’un intervalle temporaire entre les deux camps en lutte.\par
Actuellement, dans la première moitié de juillet, à un regard superficiel il peut sembler que tout est plus ou moins rentré dans la norme. En fait, dans les profondeurs du prolétariat, comme dans les sommets des classes dominantes, la préparation presque automatique d’un nouveau conflit est en marche. Tout le fond de la chose est en ceci : les réformes, très piètres en réalité, sur lesquelles se sont mis d’accord les capitalistes et les chefs des organisations ouvrières, ne sont pas viables, car elles sont au-dessus des forces du capitalisme déjà décadent, pris dans son ensemble. L’oligarchie financière, qui fait des affaires magnifiques au plus fort de la crise, peut, assurément, s’accommoder de la semaine de 40 heures, des congés payés, etc. Mais des centaines de milliers de moyens et petits industriels, sur qui s’appuie le capital financier et sur qui il fait retomber maintenant les frais de son accord avec Blum, doivent soit se ruiner docilement, soit tenter, à leur tour, de faire retomber les frais  des réformes sociales sur les ouvriers et les paysans, comme sur les consommateurs.\par
Certes, Blum a plus d’une fois développé à la Chambre et dans la presse la séduisante perspective d’une ranimation économique générale et d’une circulation qui détendrait rapidement, donnant la possibilité d’abaisser considérablement les frais généraux de production et permettant par cela d’augmenter les dépenses en force de travail sans élever les prix des marchandises. C’est vrai, de tels processus économiques combinés se sont rencontrés plus d’une fois dans le passé ; toute l’histoire du capitalisme ascendant en est marquée. Le malheur est seulement que Blum tente d’appeler dans l’avenir un passé parti sans retour. Des politiciens soumis à de telles aberrations peuvent s’appeler socialistes et même communistes, en fait ils regardent non pas en avant, mais en arrière, et c’est pourquoi ils sont des freins du progrès.\par
Le capitalisme français, avec son célèbre « équilibre » entre l’agriculture et l’industrie, est entré, après l’Italie et l’Allemagne, dans le stade du déclin, mais de façon non moins irrésistible. Ce n’est pas une phrase de proclamation révolutionnaire, mais une réalité incontestable. Les forces productives de la France ont dépassé les cadres de la propriété privée et les frontières de l’Etat. L’ingérence gouvernementale sur les bases du régime capitaliste ne peut qu’aider à faire passer les faux frais de la décadence de certaines classes sur d’autres. Sur lesquelles précisément ? Quand il faut au président du Conseil socialiste mener des pourparlers sur une répartition plus « équitable » du revenu national, il ne trouve pas, comme nous l’avons déjà entendu, de partenaires plus dignes que les représentants des deux cents familles. Ayant dans leurs mains tous les leviers fondamentaux de l’industrie, du crédit et du commerce, les magnats de la finance font retomber les frais de l’accord sur les  « classes moyennes », les contraignant par cela même à entrer en lutte avec les ouvriers. C’est en cela qu’est actuellement le nœud de la situation.\par
Les industriels et les commerçants montrent aux ministres leurs livres de comptes et disent : « Nous ne pouvons pas », Le gouvernement, se souvenant des vieux manuels d’économie politique, répond ; « Il faut diminuer les frais de production ». Mais c’est plus facile à dire qu’à faire. En outre, accroître la technique, c’est dans les conditions actuelles augmenter le chômage et en fin de compte approfondir la crise. Les ouvriers, de leur côté, protestent contre le fait que la montée des prix, qui commence, menace de dévorer leurs conquêtes. Le gouvernement ordonne aux préfets d’ouvrir la lutte contre la vie chère. Mais les préfets savent par une longue expérience qu’il est beaucoup plus facile de faire baisser le ton des journaux d’opposition que de faire baisser le prix de la viande. La vague de vie chère est encore entièrement devant nous.\par
Les petits industriels, les petits commerçants et derrière eux les paysans aussi seront de plus en plus déçus par le Front populaire, dont, avec une spontanéité et une naïveté plus grandes que les ouvriers, ils attendaient le salut immédiat. La contradiction politique fondamentale du Front populaire réside dans le fait que ceux qui sont à la tête de sa politique du juste milieu, craignant d’ « effrayer » les classes moyennes, ne sortent pas des cadres de l’ancien régime social, c’est-à-dire de l’impasse historique. Cependant les soi-disant classes moyennes, non pas leurs sommets, bien entendu, mais leurs couches inférieures, qui sentent l’impasse à chaque pas, ne craignent nullement les décisions hardies, au contraire, les réclament, comme une délivrance du nœud coulant qui les étreint. « N’attendez pas des miracles de nous ! » répètent les pédants qui se trouvent au pouvoir. Mais  précisément sans « miracle », c’est-à-dire sans décisions héroïques, sans une complète révolution dans les rapports de propriété — sans concentration du système (bancaire, des branches fondamentales de l’industrie et du commerce extérieur dans les mains de l’Etat — il n’y a pas de salut pour la petite bourgeoisie de la ville et de la campagne. Si les « classes moyennes », au nom desquelles s’est précisément édifié le Front populaire, ne trouvent pas de hardiesse à gauche, elles iront en chercher à droite. La petite bourgeoisie tremble de fièvre et elle se jettera inévitablement d’un côté sur l’autre. Entre temps le grand capital stimulera à coup sûr ce tournant, qui doit marquer le début du fascisme en France, non seulement comme organisation semi-militaire des fils de famille, avec automobiles et avions, mais aussi comme véritable mouvement de masses.\par
Les ouvriers ont exercé en juin une grandiose pression sur les classes dirigeantes, mais ils ne l’ont pas mené jusqu’au bout. Ils ont dévoilé leur puissance révolutionnaire, mais aussi leur faiblesse : l’absence de programme et de direction. Tous les fondements de la société capitaliste, mais aussi tous ses ulcères incurables, sont restés en place. Maintenant s’est ouverte la période de la contre-pression : répression contre les agitateurs de gauche, agitation toujours plus maligne des agitateurs de droite, tentatives d’augmenter les prix, mobilisation d’industriels pour des lock-outs massifs. Les syndicats de France, qui à la veille de la grève, ne comptaient même pas un million de membres, approchent maintenant de quatre millions. Cet afflux massif inouï montre quels sentiments animent les masses ouvrières. Il ne peut même pas être question de permettre que sans combat on fasse retomber sur elles les frais de leurs propres conquêtes. Ministres et chefs officiels exhortent inlassablement les ouvriers à se tenir tranquilles et à ne pas empêcher le  gouvernement de travailler à résoudre les problèmes. Mais puisque le gouvernement, par le fond même des choses, ne peut résoudre aucun problème, puisque les concessions de juin furent obtenues grâce à la grève, et non par une attente patiente, puisque chaque jour nouveau dévoilera l’inconsistance du gouvernement devant la contre-offensive grandissante du capital, les exhortations monotones perdront très rapidement leur force de persuasion. La logique de la situation, qui découle de la victoire de juin, plus exactement, du caractère semi-fictif de cette victoire, forcera les ouvriers à répondre à l’appel, c’est-à-dire à entrer de nouveau dans la lutte. Dans l’effroi devant cette perspective, le gouvernement se déplace à droite. Sous la pression immédiate des alliés radicaux, mais, en fin de compte, sur l’exigence des « deux cents familles », le ministre socialiste de l’Intérieur a déclaré au Sénat que les occupations par les grévistes d’usines, de magasins et de fermes ne seraient plus tolérées. Un avertissement de ce genre, assurément, n’arrêtera pas la lutte ; mais il est capable de lui donner un caractère incomparablement plus décisif et plus aigu.\par
Une analyse absolument objective, partant des faits, et non des désirs, conduit, ainsi, à la conclusion que des deux côtés se prépare un nouveau conflit social, qui doit éclater avec une inéluctabilité presque mécanique. Il n’est pas difficile de déterminer en général dès maintenant la nature de ce conflit. Dans toutes les périodes révolutionnaires de l’histoire on peut trouver deux étapes successives, étroitement liées l’une à l’autre : il y a d’abord un mouvement « spontané » des masses, qui prend l’adversaire à l’improviste et lui arrache de sérieuses concessions, au moins, des promesses ; après quoi la classe dominante, sentant les bases de sa domination menacée, prépare la revanche. Les masses semi-victorieuses manifestent de l’impatience. Les chefs traditionnels  de « gauche », pris par le mouvement à l’improviste, tout comme les adversaires, espèrent sauver la situation à l’aide de l’éloquence conciliatrice et, en fin de compte, perdent leur influence. Les masses entrent dans la nouvelle lutte presque sans direction, sans programme clair et sans compréhension des difficultés prochaines. Ainsi le conflit, s’élevant inévitablement de la première demi-victoire des masses, a conduit souvent à leur défaite ou à leur demi-défaite. Il n’est guère probable que dans l’histoire des révolutions on puisse trouver une exception à cette règle. La différence, pourtant — et elle n’est pas mince — est dans le fait que dans certains cas la défaite a pris le caractère d’un \emph{écrasement :} telles furent, par exemple, les journées de juin 1846, en France, qui marquèrent la fin de la révolution ; dans d’autres cas la demi-défaite fut seulement une \emph{étape vers la victoire :} c’est le rôle que joua, par exemple, la défaite des ouvriers et des soldats pétersbourgeois en juillet 1917. Précisément la défaite de juillet accéléra la montée des bolchéviks, qui non seulement avaient su apprécier justement la situation, sans illusions et sans fard, mais ne s’étaient pas non plus détachés des masses dans les journées les plus difficiles d’insuccès, de victimes et de persécutions.\par
Oui, la presse conservatrice analyse mûrement la situation. Le capital financier et ses organes politiques et militaires auxiliaires préparent la revanche avec un froid calcul. Dans les sommets du Front populaire il n’y a rien d’autre que l’effarement et la lutte interne. Les journaux de gauche font des sermons. Les chefs se gargarisent de phrases. Les ministres s’efforcent de montrer à la Bourse qu’ils sont murs pour diriger l’Etat. Tout cela signifie que le prolétariat entrera dans le prochain conflit non seulement \emph{sans} la direction de ses organisations traditionnelles, comme en juin, mais aussi \emph{contre} elles. Cependant  il n’y a pas encore de nouvelle direction reconnue de tous. Dans de telles conditions il est difficile de compter sur une victoire immédiate. La tentative d’aller de l’avant conduira bientôt à l’alternative : journées de juin 1848 ou journées de juillet 1917 ? Autrement dit : écrasement pour de longues années, avec l’inévitable triomphe de la réaction fasciste, ou bien seulement une sévère leçon de stratégie, en résultat de quoi la classe ouvrière sera incomparablement plus mûrie, renouvellera sa direction et préparera les conditions de la victoire future.\par
Le prolétariat français n’est pas un novice. Il a derrière lui le plus grand nombre dans l’histoire de batailles historiques. Certes, il faut à la nouvelle génération apprendre à chaque pas de sa propre expérience — mais non pas depuis le début ni dans l’ensemble, mais pour ainsi dire suivant un cours abrégé. Une grande tradition vit dans les os et aide à choisir le chemin. Déjà en juin les chefs anonymes de la classe en éveil, avec un magnifique tact révolutionnaire, ont trouvé les méthodes et les formes de la lutte. Le travail moléculaire de la conscience de la masse, actuellement, ne s’arrête pas une heure. Tout cela permet de compter que non seulement la nouvelle couche de chefs restera fidèle à la masse aux jours de l’inévitable et, vraisemblablement, assez proche nouveau conflit, mais aussi saura retirer du combat, sans écrasement, l’armée insuffisamment préparée.\par
Il n’est pas vrai que les révolutionnaires de France soient intéressés à ce que le conflit soit accéléré ou à ce qu’il soit provoqué « artificiellement » : ne peuvent penser ainsi que les cerveaux obtus de policiers. Les révolutionnaires marxistes voient leur devoir en ceci : regarder clairement en face la réalité et nommer chaque chose par son nom. Tirer à temps de la situation objective la  perspective de la seconde étape, c’est aider les ouvriers avancés à ne pas être pris à l’improviste et à apporter la plus grande clarté possible dans la conscience des masses en lutte. C’est en cela précisément que consiste actuellement la véritable tâche d’une sérieuse direction politique.
 


% at least one empty page at end (for booklet couv)
\ifbooklet
  \pagestyle{empty}
  \clearpage
  % 2 empty pages maybe needed for 4e cover
  \ifnum\modulo{\value{page}}{4}=0 \hbox{}\newpage\hbox{}\newpage\fi
  \ifnum\modulo{\value{page}}{4}=1 \hbox{}\newpage\hbox{}\newpage\fi


  \hbox{}\newpage
  \ifodd\value{page}\hbox{}\newpage\fi
  {\centering\color{rubric}\bfseries\noindent\large
    Hurlus ? Qu’est-ce.\par
    \bigskip
  }
  \noindent Des bouquinistes électroniques, pour du texte libre à participation libre,
  téléchargeable gratuitement sur \href{https://hurlus.fr}{\dotuline{hurlus.fr}}.\par
  \bigskip
  \noindent Cette brochure a été produite par des éditeurs bénévoles.
  Elle n’est pas faîte pour être possédée, mais pour être lue, et puis donnée.
  Que circule le texte !
  En page de garde, on peut ajouter une date, un lieu, un nom ; pour suivre le voyage des idées.
  \par

  Ce texte a été choisi parce qu’une personne l’a aimé,
  ou haï, elle a en tous cas pensé qu’il partipait à la formation de notre présent ;
  sans le souci de plaire, vendre, ou militer pour une cause.
  \par

  L’édition électronique est soigneuse, tant sur la technique
  que sur l’établissement du texte ; mais sans aucune prétention scolaire, au contraire.
  Le but est de s’adresser à tous, sans distinction de science ou de diplôme.
  Au plus direct ! (possible)
  \par

  Cet exemplaire en papier a été tiré sur une imprimante personnelle
   ou une photocopieuse. Tout le monde peut le faire.
  Il suffit de
  télécharger un fichier sur \href{https://hurlus.fr}{\dotuline{hurlus.fr}},
  d’imprimer, et agrafer ; puis de lire et donner.\par

  \bigskip

  \noindent PS : Les hurlus furent aussi des rebelles protestants qui cassaient les statues dans les églises catholiques. En 1566 démarra la révolte des gueux dans le pays de Lille. L’insurrection enflamma la région jusqu’à Anvers où les gueux de mer bloquèrent les bateaux espagnols.
  Ce fut une rare guerre de libération dont naquit un pays toujours libre : les Pays-Bas.
  En plat pays francophone, par contre, restèrent des bandes de huguenots, les hurlus, progressivement réprimés par la très catholique Espagne.
  Cette mémoire d’une défaite est éteinte, rallumons-la. Sortons les livres du culte universitaire, cherchons les idoles de l’époque, pour les briser.
\fi

\ifdev % autotext in dev mode
\fontname\font — \textsc{Les règles du jeu}\par
(\hyperref[utopie]{\underline{Lien}})\par
\noindent \initialiv{A}{lors là}\blindtext\par
\noindent \initialiv{À}{ la bonheur des dames}\blindtext\par
\noindent \initialiv{É}{tonnez-le}\blindtext\par
\noindent \initialiv{Q}{ualitativement}\blindtext\par
\noindent \initialiv{V}{aloriser}\blindtext\par
\Blindtext
\phantomsection
\label{utopie}
\Blinddocument
\fi
\end{document}
