%%%%%%%%%%%%%%%%%%%%%%%%%%%%%%%%%
% LaTeX model https://hurlus.fr %
%%%%%%%%%%%%%%%%%%%%%%%%%%%%%%%%%

% Needed before document class
\RequirePackage{pdftexcmds} % needed for tests expressions
\RequirePackage{fix-cm} % correct units

% Define mode
\def\mode{a4}

\newif\ifaiv % a4
\newif\ifav % a5
\newif\ifbooklet % booklet
\newif\ifcover % cover for booklet

\ifnum \strcmp{\mode}{cover}=0
  \covertrue
\else\ifnum \strcmp{\mode}{booklet}=0
  \booklettrue
\else\ifnum \strcmp{\mode}{a5}=0
  \avtrue
\else
  \aivtrue
\fi\fi\fi

\ifbooklet % do not enclose with {}
  \documentclass[french,twoside]{book} % ,notitlepage
  \usepackage[%
    papersize={105mm, 297mm},
    inner=12mm,
    outer=12mm,
    top=20mm,
    bottom=15mm,
    marginparsep=0pt,
  ]{geometry}
  \usepackage[fontsize=9.5pt]{scrextend} % for Roboto
\else\ifav
  \documentclass[french,twoside]{book} % ,notitlepage
  \usepackage[%
    a5paper,
    inner=25mm,
    outer=15mm,
    top=15mm,
    bottom=15mm,
    marginparsep=0pt,
  ]{geometry}
  \usepackage[fontsize=12pt]{scrextend}
\else% A4 2 cols
  \documentclass[twocolumn]{report}
  \usepackage[%
    a4paper,
    inner=15mm,
    outer=10mm,
    top=25mm,
    bottom=18mm,
    marginparsep=0pt,
  ]{geometry}
  \setlength{\columnsep}{20mm}
  \usepackage[fontsize=9.5pt]{scrextend}
\fi\fi

%%%%%%%%%%%%%%
% Alignments %
%%%%%%%%%%%%%%
% before teinte macros

\setlength{\arrayrulewidth}{0.2pt}
\setlength{\columnseprule}{\arrayrulewidth} % twocol
\setlength{\parskip}{0pt} % classical para with no margin
\setlength{\parindent}{1.5em}

%%%%%%%%%%
% Colors %
%%%%%%%%%%
% before Teinte macros

\usepackage[dvipsnames]{xcolor}
\definecolor{rubric}{HTML}{800000} % the tonic 0c71c3
\def\columnseprulecolor{\color{rubric}}
\colorlet{borderline}{rubric!30!} % definecolor need exact code
\definecolor{shadecolor}{gray}{0.95}
\definecolor{bghi}{gray}{0.5}

%%%%%%%%%%%%%%%%%
% Teinte macros %
%%%%%%%%%%%%%%%%%
%%%%%%%%%%%%%%%%%%%%%%%%%%%%%%%%%%%%%%%%%%%%%%%%%%%
% <TEI> generic (LaTeX names generated by Teinte) %
%%%%%%%%%%%%%%%%%%%%%%%%%%%%%%%%%%%%%%%%%%%%%%%%%%%
% This template is inserted in a specific design
% It is XeLaTeX and otf fonts

\makeatletter % <@@@


\usepackage{blindtext} % generate text for testing
\usepackage[strict]{changepage} % for modulo 4
\usepackage{contour} % rounding words
\usepackage[nodayofweek]{datetime}
% \usepackage{DejaVuSans} % seems buggy for sffont font for symbols
\usepackage{enumitem} % <list>
\usepackage{etoolbox} % patch commands
\usepackage{fancyvrb}
\usepackage{fancyhdr}
\usepackage{float}
\usepackage{fontspec} % XeLaTeX mandatory for fonts
\usepackage{footnote} % used to capture notes in minipage (ex: quote)
\usepackage{framed} % bordering correct with footnote hack
\usepackage{graphicx}
\usepackage{lettrine} % drop caps
\usepackage{lipsum} % generate text for testing
\usepackage[framemethod=tikz,]{mdframed} % maybe used for frame with footnotes inside
\usepackage{pdftexcmds} % needed for tests expressions
\usepackage{polyglossia} % non-break space french punct, bug Warning: "Failed to patch part"
\usepackage[%
  indentfirst=false,
  vskip=1em,
  noorphanfirst=true,
  noorphanafter=true,
  leftmargin=\parindent,
  rightmargin=0pt,
]{quoting}
\usepackage{ragged2e}
\usepackage{setspace} % \setstretch for <quote>
\usepackage{tabularx} % <table>
\usepackage[explicit]{titlesec} % wear titles, !NO implicit
\usepackage{tikz} % ornaments
\usepackage{tocloft} % styling tocs
\usepackage[fit]{truncate} % used im runing titles
\usepackage{unicode-math}
\usepackage[normalem]{ulem} % breakable \uline, normalem is absolutely necessary to keep \emph
\usepackage{verse} % <l>
\usepackage{xcolor} % named colors
\usepackage{xparse} % @ifundefined
\XeTeXdefaultencoding "iso-8859-1" % bad encoding of xstring
\usepackage{xstring} % string tests
\XeTeXdefaultencoding "utf-8"
\PassOptionsToPackage{hyphens}{url} % before hyperref, which load url package

% TOTEST
% \usepackage{hypcap} % links in caption ?
% \usepackage{marginnote}
% TESTED
% \usepackage{background} % doesn’t work with xetek
% \usepackage{bookmark} % prefers the hyperref hack \phantomsection
% \usepackage[color, leftbars]{changebar} % 2 cols doc, impossible to keep bar left
% \usepackage[utf8x]{inputenc} % inputenc package ignored with utf8 based engines
% \usepackage[sfdefault,medium]{inter} % no small caps
% \usepackage{firamath} % choose firasans instead, firamath unavailable in Ubuntu 21-04
% \usepackage{flushend} % bad for last notes, supposed flush end of columns
% \usepackage[stable]{footmisc} % BAD for complex notes https://texfaq.org/FAQ-ftnsect
% \usepackage{helvet} % not for XeLaTeX
% \usepackage{multicol} % not compatible with too much packages (longtable, framed, memoir…)
% \usepackage[default,oldstyle,scale=0.95]{opensans} % no small caps
% \usepackage{sectsty} % \chapterfont OBSOLETE
% \usepackage{soul} % \ul for underline, OBSOLETE with XeTeX
% \usepackage[breakable]{tcolorbox} % text styling gone, footnote hack not kept with breakable


% Metadata inserted by a program, from the TEI source, for title page and runing heads
\title{\textbf{ L’Homme et la Société }}
\date{1798}
\author{Salaville, Jean-Baptiste (1755-1832)}
\def\elbibl{Salaville, Jean-Baptiste (1755-1832). 1798. \emph{L’Homme et la Société}}
\def\elsource{Jean-Baptiste Salaville, {\itshape L’Homme et la Société, ou Nouvelle théorie de la nature humaine et de l’état social}, Paris, Carteret, an VII [1798], X-420 p. PDF : \href{http://gallica.bnf.fr/ark:/12148/bpt6k1135138}{\dotuline{Gallica}}\footnote{\href{http://gallica.bnf.fr/ark:/12148/bpt6k1135138}{\url{http://gallica.bnf.fr/ark:/12148/bpt6k1135138}}}.}

% Default metas
\newcommand{\colorprovide}[2]{\@ifundefinedcolor{#1}{\colorlet{#1}{#2}}{}}
\colorprovide{rubric}{red}
\colorprovide{silver}{lightgray}
\@ifundefined{syms}{\newfontfamily\syms{DejaVu Sans}}{}
\newif\ifdev
\@ifundefined{elbibl}{% No meta defined, maybe dev mode
  \newcommand{\elbibl}{Titre court ?}
  \newcommand{\elbook}{Titre du livre source ?}
  \newcommand{\elabstract}{Résumé\par}
  \newcommand{\elurl}{http://oeuvres.github.io/elbook/2}
  \author{Éric Lœchien}
  \title{Un titre de test assez long pour vérifier le comportement d’une maquette}
  \date{1566}
  \devtrue
}{}
\let\eltitle\@title
\let\elauthor\@author
\let\eldate\@date


\defaultfontfeatures{
  % Mapping=tex-text, % no effect seen
  Scale=MatchLowercase,
  Ligatures={TeX,Common},
}


% generic typo commands
\newcommand{\astermono}{\medskip\centerline{\color{rubric}\large\selectfont{\syms ✻}}\medskip\par}%
\newcommand{\astertri}{\medskip\par\centerline{\color{rubric}\large\selectfont{\syms ✻\,✻\,✻}}\medskip\par}%
\newcommand{\asterism}{\bigskip\par\noindent\parbox{\linewidth}{\centering\color{rubric}\large{\syms ✻}\\{\syms ✻}\hskip 0.75em{\syms ✻}}\bigskip\par}%

% lists
\newlength{\listmod}
\setlength{\listmod}{\parindent}
\setlist{
  itemindent=!,
  listparindent=\listmod,
  labelsep=0.2\listmod,
  parsep=0pt,
  % topsep=0.2em, % default topsep is best
}
\setlist[itemize]{
  label=—,
  leftmargin=0pt,
  labelindent=1.2em,
  labelwidth=0pt,
}
\setlist[enumerate]{
  label={\bf\color{rubric}\arabic*.},
  labelindent=0.8\listmod,
  leftmargin=\listmod,
  labelwidth=0pt,
}
\newlist{listalpha}{enumerate}{1}
\setlist[listalpha]{
  label={\bf\color{rubric}\alph*.},
  leftmargin=0pt,
  labelindent=0.8\listmod,
  labelwidth=0pt,
}
\newcommand{\listhead}[1]{\hspace{-1\listmod}\emph{#1}}

\renewcommand{\hrulefill}{%
  \leavevmode\leaders\hrule height 0.2pt\hfill\kern\z@}

% General typo
\DeclareTextFontCommand{\textlarge}{\large}
\DeclareTextFontCommand{\textsmall}{\small}

% commands, inlines
\newcommand{\anchor}[1]{\Hy@raisedlink{\hypertarget{#1}{}}} % link to top of an anchor (not baseline)
\newcommand\abbr[1]{#1}
\newcommand{\autour}[1]{\tikz[baseline=(X.base)]\node [draw=rubric,thin,rectangle,inner sep=1.5pt, rounded corners=3pt] (X) {\color{rubric}#1};}
\newcommand\corr[1]{#1}
\newcommand{\ed}[1]{ {\color{silver}\sffamily\footnotesize (#1)} } % <milestone ed="1688"/>
\newcommand\expan[1]{#1}
\newcommand\foreign[1]{\emph{#1}}
\newcommand\gap[1]{#1}
\renewcommand{\LettrineFontHook}{\color{rubric}}
\newcommand{\initial}[2]{\lettrine[lines=2, loversize=0.3, lhang=0.3]{#1}{#2}}
\newcommand{\initialiv}[2]{%
  \let\oldLFH\LettrineFontHook
  % \renewcommand{\LettrineFontHook}{\color{rubric}\ttfamily}
  \IfSubStr{QJ’}{#1}{
    \lettrine[lines=4, lhang=0.2, loversize=-0.1, lraise=0.2]{\smash{#1}}{#2}
  }{\IfSubStr{É}{#1}{
    \lettrine[lines=4, lhang=0.2, loversize=-0, lraise=0]{\smash{#1}}{#2}
  }{\IfSubStr{ÀÂ}{#1}{
    \lettrine[lines=4, lhang=0.2, loversize=-0, lraise=0, slope=0.6em]{\smash{#1}}{#2}
  }{\IfSubStr{A}{#1}{
    \lettrine[lines=4, lhang=0.2, loversize=0.2, slope=0.6em]{\smash{#1}}{#2}
  }{\IfSubStr{V}{#1}{
    \lettrine[lines=4, lhang=0.2, loversize=0.2, slope=-0.5em]{\smash{#1}}{#2}
  }{
    \lettrine[lines=4, lhang=0.2, loversize=0.2]{\smash{#1}}{#2}
  }}}}}
  \let\LettrineFontHook\oldLFH
}
\newcommand{\labelchar}[1]{\textbf{\color{rubric} #1}}
\newcommand{\milestone}[1]{\autour{\footnotesize\color{rubric} #1}} % <milestone n="4"/>
\newcommand\name[1]{#1}
\newcommand\orig[1]{#1}
\newcommand\orgName[1]{#1}
\newcommand\persName[1]{#1}
\newcommand\placeName[1]{#1}
\newcommand{\pn}[1]{\IfSubStr{-—–¶}{#1}% <p n="3"/>
  {\noindent{\bfseries\color{rubric}   ¶  }}
  {{\footnotesize\autour{ #1}  }}}
\newcommand\reg{}
% \newcommand\ref{} % already defined
\newcommand\sic[1]{#1}
\newcommand\surname[1]{\textsc{#1}}
\newcommand\term[1]{\textbf{#1}}

\def\mednobreak{\ifdim\lastskip<\medskipamount
  \removelastskip\nopagebreak\medskip\fi}
\def\bignobreak{\ifdim\lastskip<\bigskipamount
  \removelastskip\nopagebreak\bigskip\fi}

% commands, blocks
\newcommand{\byline}[1]{\bigskip{\RaggedLeft{#1}\par}\bigskip}
\newcommand{\bibl}[1]{{\RaggedLeft{#1}\par\bigskip}}
\newcommand{\biblitem}[1]{{\noindent\hangindent=\parindent   #1\par}}
\newcommand{\dateline}[1]{\medskip{\RaggedLeft{#1}\par}\bigskip}
\newcommand{\labelblock}[1]{\medbreak{\noindent\color{rubric}\bfseries #1}\par\mednobreak}
\newcommand{\salute}[1]{\bigbreak{#1}\par\medbreak}
\newcommand{\signed}[1]{\bigbreak\filbreak{\raggedleft #1\par}\medskip}

% environments for blocks (some may become commands)
\newenvironment{borderbox}{}{} % framing content
\newenvironment{citbibl}{\ifvmode\hfill\fi}{\ifvmode\par\fi }
\newenvironment{docAuthor}{\ifvmode\vskip4pt\fontsize{16pt}{18pt}\selectfont\fi\itshape}{\ifvmode\par\fi }
\newenvironment{docDate}{}{\ifvmode\par\fi }
\newenvironment{docImprint}{\vskip6pt}{\ifvmode\par\fi }
\newenvironment{docTitle}{\vskip6pt\bfseries\fontsize{18pt}{22pt}\selectfont}{\par }
\newenvironment{msHead}{\vskip6pt}{\par}
\newenvironment{msItem}{\vskip6pt}{\par}
\newenvironment{titlePart}{}{\par }


% environments for block containers
\newenvironment{argument}{\itshape\parindent0pt}{\vskip1.5em}
\newenvironment{biblfree}{}{\ifvmode\par\fi }
\newenvironment{bibitemlist}[1]{%
  \list{\@biblabel{\@arabic\c@enumiv}}%
  {%
    \settowidth\labelwidth{\@biblabel{#1}}%
    \leftmargin\labelwidth
    \advance\leftmargin\labelsep
    \@openbib@code
    \usecounter{enumiv}%
    \let\p@enumiv\@empty
    \renewcommand\theenumiv{\@arabic\c@enumiv}%
  }
  \sloppy
  \clubpenalty4000
  \@clubpenalty \clubpenalty
  \widowpenalty4000%
  \sfcode`\.\@m
}%
{\def\@noitemerr
  {\@latex@warning{Empty `bibitemlist' environment}}%
\endlist}
\newenvironment{quoteblock}% may be used for ornaments
  {\begin{quoting}}
  {\end{quoting}}

% table () is preceded and finished by custom command
\newcommand{\tableopen}[1]{%
  \ifnum\strcmp{#1}{wide}=0{%
    \begin{center}
  }
  \else\ifnum\strcmp{#1}{long}=0{%
    \begin{center}
  }
  \else{%
    \begin{center}
  }
  \fi\fi
}
\newcommand{\tableclose}[1]{%
  \ifnum\strcmp{#1}{wide}=0{%
    \end{center}
  }
  \else\ifnum\strcmp{#1}{long}=0{%
    \end{center}
  }
  \else{%
    \end{center}
  }
  \fi\fi
}


% text structure
\newcommand\chapteropen{} % before chapter title
\newcommand\chaptercont{} % after title, argument, epigraph…
\newcommand\chapterclose{} % maybe useful for multicol settings
\setcounter{secnumdepth}{-2} % no counters for hierarchy titles
\setcounter{tocdepth}{5} % deep toc
\markright{\@title} % ???
\markboth{\@title}{\@author} % ???
\renewcommand\tableofcontents{\@starttoc{toc}}
% toclof format
% \renewcommand{\@tocrmarg}{0.1em} % Useless command?
% \renewcommand{\@pnumwidth}{0.5em} % {1.75em}
\renewcommand{\@cftmaketoctitle}{}
\setlength{\cftbeforesecskip}{\z@ \@plus.2\p@}
\renewcommand{\cftchapfont}{}
\renewcommand{\cftchapdotsep}{\cftdotsep}
\renewcommand{\cftchapleader}{\normalfont\cftdotfill{\cftchapdotsep}}
\renewcommand{\cftchappagefont}{\bfseries}
\setlength{\cftbeforechapskip}{0em \@plus\p@}
% \renewcommand{\cftsecfont}{\small\relax}
\renewcommand{\cftsecpagefont}{\normalfont}
% \renewcommand{\cftsubsecfont}{\small\relax}
\renewcommand{\cftsecdotsep}{\cftdotsep}
\renewcommand{\cftsecpagefont}{\normalfont}
\renewcommand{\cftsecleader}{\normalfont\cftdotfill{\cftsecdotsep}}
\setlength{\cftsecindent}{1em}
\setlength{\cftsubsecindent}{2em}
\setlength{\cftsubsubsecindent}{3em}
\setlength{\cftchapnumwidth}{1em}
\setlength{\cftsecnumwidth}{1em}
\setlength{\cftsubsecnumwidth}{1em}
\setlength{\cftsubsubsecnumwidth}{1em}

% footnotes
\newif\ifheading
\newcommand*{\fnmarkscale}{\ifheading 0.70 \else 1 \fi}
\renewcommand\footnoterule{\vspace*{0.3cm}\hrule height \arrayrulewidth width 3cm \vspace*{0.3cm}}
\setlength\footnotesep{1.5\footnotesep} % footnote separator
\renewcommand\@makefntext[1]{\parindent 1.5em \noindent \hb@xt@1.8em{\hss{\normalfont\@thefnmark . }}#1} % no superscipt in foot
\patchcmd{\@footnotetext}{\footnotesize}{\footnotesize\sffamily}{}{} % before scrextend, hyperref


%   see https://tex.stackexchange.com/a/34449/5049
\def\truncdiv#1#2{((#1-(#2-1)/2)/#2)}
\def\moduloop#1#2{(#1-\truncdiv{#1}{#2}*#2)}
\def\modulo#1#2{\number\numexpr\moduloop{#1}{#2}\relax}

% orphans and widows
\clubpenalty=9996
\widowpenalty=9999
\brokenpenalty=4991
\predisplaypenalty=10000
\postdisplaypenalty=1549
\displaywidowpenalty=1602
\hyphenpenalty=400
% Copied from Rahtz but not understood
\def\@pnumwidth{1.55em}
\def\@tocrmarg {2.55em}
\def\@dotsep{4.5}
\emergencystretch 3em
\hbadness=4000
\pretolerance=750
\tolerance=2000
\vbadness=4000
\def\Gin@extensions{.pdf,.png,.jpg,.mps,.tif}
% \renewcommand{\@cite}[1]{#1} % biblio

\usepackage{hyperref} % supposed to be the last one, :o) except for the ones to follow
\urlstyle{same} % after hyperref
\hypersetup{
  % pdftex, % no effect
  pdftitle={\elbibl},
  % pdfauthor={Your name here},
  % pdfsubject={Your subject here},
  % pdfkeywords={keyword1, keyword2},
  bookmarksnumbered=true,
  bookmarksopen=true,
  bookmarksopenlevel=1,
  pdfstartview=Fit,
  breaklinks=true, % avoid long links
  pdfpagemode=UseOutlines,    % pdf toc
  hyperfootnotes=true,
  colorlinks=false,
  pdfborder=0 0 0,
  % pdfpagelayout=TwoPageRight,
  % linktocpage=true, % NO, toc, link only on page no
}

\makeatother % /@@@>
%%%%%%%%%%%%%%
% </TEI> end %
%%%%%%%%%%%%%%


%%%%%%%%%%%%%
% footnotes %
%%%%%%%%%%%%%
\renewcommand{\thefootnote}{\bfseries\textcolor{rubric}{\arabic{footnote}}} % color for footnote marks

%%%%%%%%%
% Fonts %
%%%%%%%%%
\usepackage[]{roboto} % SmallCaps, Regular is a bit bold
% \linespread{0.90} % too compact, keep font natural
\newfontfamily\fontrun[]{Roboto Condensed Light} % condensed runing heads
\ifav
  \setmainfont[
    ItalicFont={Roboto Light Italic},
  ]{Roboto}
\else\ifbooklet
  \setmainfont[
    ItalicFont={Roboto Light Italic},
  ]{Roboto}
\else
\setmainfont[
  ItalicFont={Roboto Italic},
]{Roboto Light}
\fi\fi
\renewcommand{\LettrineFontHook}{\bfseries\color{rubric}}
% \renewenvironment{labelblock}{\begin{center}\bfseries\color{rubric}}{\end{center}}

%%%%%%%%
% MISC %
%%%%%%%%

\setdefaultlanguage[frenchpart=false]{french} % bug on part


\newenvironment{quotebar}{%
    \def\FrameCommand{{\color{rubric!10!}\vrule width 0.5em} \hspace{0.9em}}%
    \def\OuterFrameSep{\itemsep} % séparateur vertical
    \MakeFramed {\advance\hsize-\width \FrameRestore}
  }%
  {%
    \endMakeFramed
  }
\renewenvironment{quoteblock}% may be used for ornaments
  {%
    \savenotes
    \setstretch{0.9}
    \normalfont
    \begin{quotebar}
  }
  {%
    \end{quotebar}
    \spewnotes
  }


\renewcommand{\headrulewidth}{\arrayrulewidth}
\renewcommand{\headrule}{{\color{rubric}\hrule}}

% delicate tuning, image has produce line-height problems in title on 2 lines
\titleformat{name=\chapter} % command
  [display] % shape
  {\vspace{1.5em}\centering} % format
  {} % label
  {0pt} % separator between n
  {}
[{\color{rubric}\huge\textbf{#1}}\bigskip] % after code
% \titlespacing{command}{left spacing}{before spacing}{after spacing}[right]
\titlespacing*{\chapter}{0pt}{-2em}{0pt}[0pt]

\titleformat{name=\section}
  [block]{}{}{}{}
  [\vbox{\color{rubric}\large\raggedleft\textbf{#1}}]
\titlespacing{\section}{0pt}{0pt plus 4pt minus 2pt}{\baselineskip}

\titleformat{name=\subsection}
  [block]
  {}
  {} % \thesection
  {} % separator \arrayrulewidth
  {}
[\vbox{\large\textbf{#1}}]
% \titlespacing{\subsection}{0pt}{0pt plus 4pt minus 2pt}{\baselineskip}

\ifaiv
  \fancypagestyle{main}{%
    \fancyhf{}
    \setlength{\headheight}{1.5em}
    \fancyhead{} % reset head
    \fancyfoot{} % reset foot
    \fancyhead[L]{\truncate{0.45\headwidth}{\fontrun\elbibl}} % book ref
    \fancyhead[R]{\truncate{0.45\headwidth}{ \fontrun\nouppercase\leftmark}} % Chapter title
    \fancyhead[C]{\thepage}
  }
  \fancypagestyle{plain}{% apply to chapter
    \fancyhf{}% clear all header and footer fields
    \setlength{\headheight}{1.5em}
    \fancyhead[L]{\truncate{0.9\headwidth}{\fontrun\elbibl}}
    \fancyhead[R]{\thepage}
  }
\else
  \fancypagestyle{main}{%
    \fancyhf{}
    \setlength{\headheight}{1.5em}
    \fancyhead{} % reset head
    \fancyfoot{} % reset foot
    \fancyhead[RE]{\truncate{0.9\headwidth}{\fontrun\elbibl}} % book ref
    \fancyhead[LO]{\truncate{0.9\headwidth}{\fontrun\nouppercase\leftmark}} % Chapter title, \nouppercase needed
    \fancyhead[RO,LE]{\thepage}
  }
  \fancypagestyle{plain}{% apply to chapter
    \fancyhf{}% clear all header and footer fields
    \setlength{\headheight}{1.5em}
    \fancyhead[L]{\truncate{0.9\headwidth}{\fontrun\elbibl}}
    \fancyhead[R]{\thepage}
  }
\fi

\ifav % a5 only
  \titleclass{\section}{top}
\fi

\newcommand\chapo{{%
  \vspace*{-3em}
  \centering % no vskip ()
  {\Large\addfontfeature{LetterSpace=25}\bfseries{\elauthor}}\par
  \smallskip
  {\large\eldate}\par
  \bigskip
  {\Large\selectfont{\eltitle}}\par
  \bigskip
  {\color{rubric}\hline\par}
  \bigskip
  {\Large TEXTE LIBRE À PARTICPATION LIBRE\par}
  \centerline{\small\color{rubric} {hurlus.fr, tiré le \today}}\par
  \bigskip
}}

\newcommand\cover{{%
  \thispagestyle{empty}
  \centering
  {\LARGE\bfseries{\elauthor}}\par
  \bigskip
  {\Large\eldate}\par
  \bigskip
  \bigskip
  {\LARGE\selectfont{\eltitle}}\par
  \vfill\null
  {\color{rubric}\setlength{\arrayrulewidth}{2pt}\hline\par}
  \vfill\null
  {\Large TEXTE LIBRE À PARTICPATION LIBRE\par}
  \centerline{{\href{https://hurlus.fr}{\dotuline{hurlus.fr}}, tiré le \today}}\par
}}

\begin{document}
\pagestyle{empty}
\ifbooklet{
  \cover\newpage
  \thispagestyle{empty}\hbox{}\newpage
  \cover\newpage\noindent Les voyages de la brochure\par
  \bigskip
  \begin{tabularx}{\textwidth}{l|X|X}
    \textbf{Date} & \textbf{Lieu}& \textbf{Nom/pseudo} \\ \hline
    \rule{0pt}{25cm} &  &   \\
  \end{tabularx}
  \newpage
  \addtocounter{page}{-4}
}\fi

\thispagestyle{empty}
\ifaiv
  \twocolumn[\chapo]
\else
  \chapo
\fi
{\it\elabstract}
\bigskip
\makeatletter\@starttoc{toc}\makeatother % toc without new page
\bigskip

\pagestyle{main} % after style

  \section[{Avant-propos.}]{Avant-propos.}\renewcommand{\leftmark}{Avant-propos.}

\noindent {\itshape Si vous commencez par la certitude}, dit Bacon, {\itshape vous finirez par le doute} ; c’est ce qui m’est arrivé relativement à la plupart des opinions que je combats aujourd’hui : elles ont eu d’abord pour moi le caractère de l’évidence ; je les ai adoptées aussitôt que je les ai connues ; mais, conduit ensuite à examiner si elles s’accordaient avec la nature des choses ; la nature des choses et ces opinions m’ont tellement paru s’exclure, qu’il a bien fallu passer de la certitude au doute.\par
J’ai fait alors ce que recommande le célèbre philosophe que je viens de citer : j’ai, pour ainsi dire, nettoyé l’aire sur laquelle j’avais bâti mon premier édifice ; j’en ai revu tous les matériaux, et je me suis supposé cherchant pour la première fois les vérités dont je m’étais cru possesseur.\par
L’ouvrage qu’on va lire est le fruit de cette investigation : je sens trop mon insuffisance pour ne pas le croire imparfait sous tous les rapports ; mais peut-être fournira-t-il à d’autres l’occasion d’en faire un meilleur, et cette considération seule me déterminerait à le publier.\par
Je ne demande pas qu’on adopte de confiance les principes qu’il contient ; mais je désirerais, s’il était possible, qu’on voulût bien les examiner sans prévention, et qu’on ne les rejetât pas précisément parce qu’ils se trouveraient en opposition avec les systèmes reçus.\par
Le plus grand obstacle que nous ayons à vaincre pour arriver à la découverte de la vérité n’est peut-être pas la difficulté de la découvrir, mais l’erreur qui nous persuade que nous l’avons trouvée ; car, cette, illusion retient notre esprit dans une sorte d’inertie, l’empêche de se livrer à de nouvelles recherches, et nous prévient contre celles qu’osent tenter les autres.\par
C’est ainsi que d’époque en époque on essaie, de circonscrire la liberté de penser, de l’enfermer dans tel ou tel système théologique, philosophique, politique ou moral, comme dans une espèce de labyrinthe d’où on lui fait une loi de ne pas sortir ; mais, quand elle en a parcouru tous les détours, sa nature incoercible la porte à de nouvelles excursions, et elle s’échappe, au grand étonnement de ceux qui croyaient l’avoir cantonnée à perpétuité.\par
Laissons-lui donc son allure naturelle. Est-ce qu’il est possible de cerner la pensée ; de tracer autour d’elle le cercle de Popilius ? Que l’inutilité de cette prétention nous en fasse à la fin apercevoir l’inconvenance : soyons assez sages pour ne donner qu’un assentiment provisoire aux opinions les plus accréditées ; songeons que les systèmes les plus absurdes, les principes les plus faux, ont eu, si je puis m’exprimer ainsi, leur temps de vérité, c’est-à-dire d’assentiment général, et qu’il a fallu pourtant ensuite les abandonner. Pourquoi n’en serait-il pas de même, de ceux qui sont actuellement en circulation ?\par
Une expérience tant de fois répétée ne devrait-elle pas nous rendre un peu plus circonspects, un peu moins prompts à repousser tout ce qui s’éloigne de notre manière de voir, et substituer en nous, à la disposition de nous scandaliser des nouvelles idées, celle de les examiner sans partialité ? Sans doute, cela devrait être ainsi ; mais, quoiqu’on dise que les leçons de l’expérience sont les seules qui profitent aux hommes, je ne crois pas qu’ils en mettent le plus grand nombre à profit ; et il me semble qu’à cet égard comme à bien d’autres, la somme des non-valeurs l’emporte infiniment sur celle des produits effectifs.

\chapteropen
\chapter[{Chapitre I. De la nature composée de l’homme, ou des deux principes qui le constituent intérieurement mixte ou double.}]{Chapitre I. De la nature composée de l’homme, ou des deux principes qui le constituent intérieurement mixte ou double.}\renewcommand{\leftmark}{Chapitre I. De la nature composée de l’homme, ou des deux principes qui le constituent intérieurement mixte ou double.}


\chaptercont
\noindent Pour peu que nous nous observions nous-mêmes, nous reconnaîtrons en nous deux manières de sentir, ou, si l’on veut, deux sensibilités différentes ; l’une que nous appellerons {\itshape physique}, parce qu’elle est immédiatement affectée par les objets physiques ; l’autre {\itshape morale}, parce qu’elle s’exerce sur les objets moraux.\par
Le plaisir qu’on éprouve en mangeant une pêche on en respirant l’odeur d’une rose, ne ressemble point à cet autre plaisir que des témoignages d’estime ou de reconnaissance déterminent en nous : la sensation pénible, occasionnée par un mets repoussant ou par une odeur fétide, ne ressemble pas non plus au sentiment douloureux que produit l’injustice ou le mépris.\par
On est pourtant dans l’usage de rapporter ces deux sensibilités à un même principe ; on veut qu’elles ne soient que des modifications différentes d’un être simple qu’on suppose dans l’homme, et qu’on appelle {\itshape âme}. Cette doctrine est tellement accréditée, que la proposition d’admettre deux principes de sentiment paraîtra d’abord fort extraordinaire ; mais nous prions nos lecteurs de suspendre leur jugement, jusqu’à ce que nous ayons développé les motifs qui nous portent à l’adopter.\par
En admettant dans l’homme intérieur deux principes de sensibilité, ou, pour mieux dire, deux êtres différents, nous n’examinerons point s’ils sont tous les deux spirituels, tous les deux matériels, ou si l’un est esprit, et l’autre matière. Pour résoudre ces différentes questions, il faudrait connaître les substances dans leur essence, et c’est à quoi il est démontré que nous ne pouvons pas parvenir. On suivra donc à cet égard telle opinion qu’on voudra. Nous serons d’accord avec tous les systèmes : ce que nous avons à dire n’en exclut aucun\footnote{La question de la spiritualité n’a occasionné tant de débats que parce qu’on a cru qu’elle tenait à la destination future de l’homme ; mais rien ne nous démontre qu’une substance spirituelle ne puisse être anéantie, ni qu’une substance matérielle ne puisse exister toujours : ne connaissant pas plus l’une que l’autre dans leur essence, pouvons-nous affirmer ou nier quelque chose de cette essence que nous ne connaissons pas ? Il me semble, dès lors, que la question de la spiritualité perd toute son importance, et que nous n’avons pas plus d’intérêt à l’admettre qu’à la rejeter, puisque, dans les deux cas, les résultats peuvent être les mêmes.}.\par
Nous caractériserons les deux êtres que nous supposons dans l’homme, et qui le constituent intérieurement mixte ou double, par leur différente manière de sentir ; l’un s’appellera {\itshape l’être physique}, et l’autre {\itshape l’être moral} : par l’un, nous n’entendons pas le corps, et par l’autre l’âme, comme on le fait communément. Ces deux êtres remplacent dans notre hypothèse l’être simple et absolu, qu’on désigne sous le nom d’{\itshape âme} dans les autres systèmes. Nous pourrions dire, pour nous conformer aux idées reçues, que ce sont deux âmes dans un même individu.
\chapterclose


\chapteropen
\chapter[{Chapitre II. Que la doctrine de l’homme intérieur mixte ou double est celle de la plupart des philosophes de l’antiquité et de plusieurs écrivains célèbres parmi les modernes.}]{Chapitre II. Que la doctrine de l’homme intérieur mixte ou double est celle de la plupart des philosophes de l’antiquité et de plusieurs écrivains célèbres parmi les modernes.}\renewcommand{\leftmark}{Chapitre II. Que la doctrine de l’homme intérieur mixte ou double est celle de la plupart des philosophes de l’antiquité et de plusieurs écrivains célèbres parmi les modernes.}


\chaptercont
\noindent En disant que l’homme intérieur {\itshape est} mixte ou double, ce n’est pas une nouvelle doctrine que nous annonçons, C’est celle de presque tous les anciens philosophes. On sait que les Pythagoriciens admettaient l’âme raisonnable et l’âme sensitive. Les Éclectiques, qui choisirent dans les opinions des autres écoles celles dont ils composèrent leur propre système, admirent aussi cette {\itshape biduité} intérieure de l’homme, qu’une fausse idée de perfection attachée à la simplicité nous a fait rejeter.\par
« L’homme, disaient-ils, a deux âmes ; l’une qu’il tient du premier intelligible, et l’autre qu’il a reçue dans le monde sensible : chacune a conservé des caractères distinctifs de son origine ; l’âme du monde intelligible retourne sans cesse à sa source, et les lois de la fatalité ne peuvent rien sur elle ; l’autre est asservie au mouvement des mondes\footnote{Voyez les œuvres de Diderot : {\itshape Opinions des anciens philosophes}, tome I, page 288.}. »\par
Quoique les modernes aient abandonné cette théorie, on n’en trouve pas moins dans leurs écrits des passages qui la supposent évidemment.\par
« En méditant sur la nature de l’homme, dit Rousseau, j’y découvre deux principes distincts, dont l’un l’élève à l’étude des vérités éternelles, à l’amour de la justice et du beau moral, aux régions du monde intellectuel, dont la contemplation fait les délices du sage, et dont l’autre le ramène bassement en lui-même, l’asservit à l’empire des sens, aux passions qui en sont les ministres, et contrarie par elles tout ce que lui inspire le sentiment du premier. En me sentant entraîné, combattu par ces deux mouvements contraires, je me dis : {\itshape Non, l’homme n’est point un.} Je veux et je ne veux pas. Je me sens à la fois esclave et libre. Je vois le bien, je l’aime, et je fais le mal. Je suis actif, quand j’écoute la raison ; passif, quand mes passions m’entraînent ; et mon pire tourment, quand je succombe, est de sentir que j’ai pu résister. »\par
Rien de tout cela ne pouvant s’appliquer au corps proprement dit, il faut bien que Rousseau établisse sa distinction dans ce qu’on est convenu d’appeler {\itshape âme}, et qu’il trouve que {\itshape l’homme n’est point un} sous ce rapport.\par
Cette doctrine est également celle de Smith ; car voici de quelle manière il s’exprime dans sa {\itshape Théorie des sentiments moraux} :\par
« Lorsque j’examine ma propre conduite pour la juger, et que je l’approuve ou que je la condamne, il est évident que je me divise en quelque sorte en deux personnes, et que le moi qui examine et qui juge remplit un rôle différent de l’autre moi, dont la conduite est examinée et jugée… Le premier est le juge, et le second le moi jugé. Il est aussi impossible que l’un soit l’autre à tous égards, qu’il est impossible que la cause et l’effet ne soient qu’une même chose\footnote{When I endeavaur to examine my own conduct, when I endeavour to pass sentence upon it, and either to approve or condemn it, it is evident that, in all such cases, I divide myself, as it were, into two persons ; and that I the examiner and judge represent a different character from that other I the person whose conduct is examined into and judged of… the first is the judge, the second the person judged of. But that the judge should in every respect, be the same with the person judged of, is as impossible, as that the cause should, in every respect, be the same with the effect. {\itshape Theory of moral sentiment}, vol. I, page 282.}. »\par
Il y a donc deux {\itshape moi} dans l’homme ; car il est impossible de concevoir la division du même moi ; et, en supposant même cette division, elle constitue deux êtres.\par
Mais si ce phénomène particulier à l’homme, d’être à la fois juge et personne jugée, ne peut s’expliquer qu’en admettant en lui deux êtres, dont l’un juge, et dont l’autre est jugé ; l’amour de soi, qui est également particulier à l’homme, ainsi que nous le verrons par la suite, suppose encore en lui un double moi ; car le même moi ne peut pas être actif et passif dans le même sens. Il faut donc un moi qui aime, et un moi qui soit aimé. L’amour ne peut se passer de ce double rapport, et ce double rapport exige rigoureusement deux êtres ; il s’ensuit que si, dans cette passion que nous appelons {\itshape amour de nous-mêmes}, nous nous sentons à la fois l’objet aimant et l’objet aimé, c’est que notre individu renferme deux êtres différents. Il serait impossible, sans cela, qu’un semblable effet pût avoir lieu.\par
Nous pouvons ajouter l’autorité de Buffon à celle des auteurs que nous venons de citer ; car il dit positivement que « l’homme intérieur est double ; qu’il est composé de deux principes différents par leur nature, et qu’il est aisé, en rentrant en soi-même, de reconnaître l’existence de l’un et de l’autre. » Si nous rapportons ici ces divers témoignages, c’est pour prouver à nos lecteurs que, dans le cas où notre proposition ne serait qu’une erreur, elle nous est commune avec des hommes dont le génie et les lumières ne sont pas contestés, et que dès lors elle mérite d’être examinée. À la vérité, ces célèbres écrivains n’ont pas fait de cette opinion la base de leurs théories, et il semble que c’est un tort qu’on peut leur reprocher ; car si l’homme intérieur se compose effectivement de deux principes ou de deux êtres différents, on ne l’expliquera point ou on l’expliquera mal, tant qu’on le considérera comme un être simple. En partant de cette fausse donnée, il y aura nécessairement une multitude de faits dont on ne trouvera point la cause ; d’autres dont on ne donnera que des explications gratuites qui ne satisferont ni la raison ni le sentiment ; et, malgré tout ce qu’on a écrit sur l’homme, n’est-ce pas là l’état actuel de la science ? ne sent-on pas même l’impossibilité d’aller plus loin par la route qu’on a suivie ?
\chapterclose


\chapteropen
\chapter[{Chapitre III. Réponse à une objection de Condillac, contre le système de l’homme intérieur double.}]{Chapitre III. Réponse à une objection de Condillac, contre le système de l’homme intérieur double.}\renewcommand{\leftmark}{Chapitre III. Réponse à une objection de Condillac, contre le système de l’homme intérieur double.}


\chaptercont
\noindent Il y aurait donc dans chaque homme, dit ce célèbre métaphysicien, deux moi, deux personnes, qui, n’ayant rien de commun dans la manière de sentir, ne sauraient avoir aucune sorte de commerce ensemble, et dont chacune ignorerait absolument ce qui se passerait dans l’autre\footnote{{\itshape Traité des animaux}, chap. 2.}. »\par
Ces conséquences ne nous semblent pas nécessairement déduites de la biduité intérieure de l’homme : car les deux êtres qui le composent intérieurement, quoiqu’ayant chacun leur moi, leur action propre et particulière, ne peuvent-ils pas être tellement sympathiques, que l’un ne puisse dérober à l’autre aucun de ses mouvements, aucune de ses impressions, et que, par l’effet de cette sympathie, ils jouissent en quelque sorte, d’une mutuelle intuition et d’une conscience commune ?\par
Si le principe de sympathie, dont Smith a développé les effets extérieurs, nous identifie, pour ainsi dire, avec nos semblables, et nous donne la conscience des affections qu’ils éprouvent, quoique leur individualité soit séparée de la nôtre, quelle influence ce même principe ne doit-il pas avoir sur les deux êtres collatéraux qui se trouvent si intimement unis dans notre constitution intérieure ? Loin de n’avoir aucune sorte de commerce ensemble ; loin d’ignorer absolument chacun ce qui se passe dans l’autre, pourquoi cette sympathie mutuelle ne leur en donnerait-elle pas au même degré la connaissance et le sentiment respectif ?\par
L’objection de Condillac ne prouve donc rien contre la doctrine de l’homme intérieur double : elle tient uniquement à la prévention de cet écrivain pour le système de l’unité et de la simplicité de l’âme.\par
Au reste, pourquoi n’en serait-il pas de ces deux êtres intérieurs relativement à la personne, comme il en est de la vue, dont l’organe est double, et dont l’effet est cependant un ? Rien n’autorise donc à dire, comme le fait Condillac, que « l’unité de personne suppose nécessairement l’unité de l’être sentant ». Nous ne voyons pas en quoi il impliquerait qu’il y en eût deux. Le moi personnel se composerait des deux moi intérieurs ; et, malgré l’assertion de ce profond métaphysicien, nous osons croire que c’est là la vraie doctrine.
\chapterclose


\chapteropen
\chapter[{Chapitre IV. De la double volonté dans l’homme.}]{Chapitre IV. De la double volonté dans l’homme.}\renewcommand{\leftmark}{Chapitre IV. De la double volonté dans l’homme.}


\chaptercont
\noindent Une nouvelle preuve que le système de l’unité n’est pas, quelque accrédité qu’il soit, le vrai système de l’homme, c’est qu’il existe en nous une double volonté comme une double sensibilité, et qu’il est contre toute vraisemblance d’en conclure l’existence d’un être simple et unique.\par
Comment rapporter en effet à une seule volonté ce phénomène si commun de vouloir et de ne vouloir pas en même temps ? N’impliquerait-il pas qu’une même volonté fût dans le même instant positive et négative ? S’il n’y avait que de la fluctuation, si ce n’était que par intermittence qu’on voulût et qu’on ne voulût pas, quelque rapide que fût le passage de l’un de ces états à l’autre, on pourrait croire qu’ils résultent d’une seule faculté ; mais il n’est personne qui n’ait souvent éprouvé l’effet de leur existence simultanée, et je voudrais bien qu’on m’expliquât comment on peut la concilier avec l’hypothèse d’une seule volonté.\par
Rien n’est, au contraire plus simple, ni plus facile à concevoir si l’on admet deux volontés, car elles peuvent être à la fois positives, à la fois négatives, tour à tour l’un ou l’autre ; ou tandis que l’une est positive, l’autre peut persévérer dans la négation : voilà toutes les difficultés, toutes les bizarreries du cœur humain, dont une seule faculté ne vous donnera jamais la solution clairement et naturellement expliquée dans le système des deux volontés.\par
On demandera peut-être pourquoi l’opinion d’une seule volonté est l’opinion commune, tandis qu’on admet généralement deux sensibilités ; c’est que les deux volontés, quoique souvent opposées, s’accordent ordinairement ou ne diffèrent que faiblement dans les déterminations usuelles de la vie, il en résulte qu’elles sont plus habituellement et sur un plus grand nombre de points en paix qu’en guerre ; et dans leur accord parfait, elles semblent ne constituer qu’une seule faculté, cet état étant le plus ordinaire et celui d’une forte opposition le plus rare, on ne doit pas être surpris que l’unité de volonté soit l’opinion commune, et que ce principe une fois admis, on ne songe pas à rapporter nos contradictions intérieures à l’opposition de deux volontés, dont on ne soupçonne pas l’existence ; mais il suffit qu’on ait quelquefois à se déterminer entre deux volontés opposées, pour qu’on soit fondé à admettre que cette opposition résulte de la lutte qui s’établit entre deux facultés ; car, s’il n’y en avait qu’une, il pourrait y avoir de la fluctuation par la différence des motifs ; mais jamais une véritable opposition.\par
Au reste, je ne vois pas que la supposition de deux volontés ait quelque chose de plus étrange, que celle de deux sensibilités : cependant cette dernière distinction ne choque personne ; elle est, pour ainsi dire, vulgaire. On admet sans la moindre difficulté une sensibilité physique et une sensibilité morale ; pourquoi n’admettrait-on pas également une volonté morale et une volonté physique ? La volonté n’est-elle pas la conséquence de la sensibilité, et s’il existe en nous deux sensibilités d’une nature différente, ne faut-il pas qu’il y ait aussi deux volontés de différente nature ? Quoique cette opinion ne nous paraisse pas susceptible d’être contestée, nous lui donnerons tout à l’heure de nouveaux développements.
\chapterclose


\chapteropen
\chapter[{Chapitre V. Caractère distinctif des deux êtres qui constituent l’homme intérieur.}]{Chapitre V. Caractère distinctif des deux êtres qui constituent l’homme intérieur.}\renewcommand{\leftmark}{Chapitre V. Caractère distinctif des deux êtres qui constituent l’homme intérieur.}


\chaptercont
\noindent Si ce que nous avons dit jusqu’à présent n’explique point la mécanique de l’association des deux êtres ou principes intérieurs, que nous supposons dans l’homme, nous croyons du moins qu’il suffit pour en démontrer l’existence ; car si nous trouvons en nous une double sensibilité, une double volonté, un double moi, comment ne pas nous croire composés intérieurement de deux êtres absolus auxquels il ne manquerait rien pour exister séparément ? À quoi bon cette duplicité de facultés identiques pour un être simple et unique ?\par
D’ailleurs les deux êtres que nous supposons ont des caractères distinctifs qui ne permettent guères de les confondre : la présence des objets physiques affecte l’être physique ; la présence des notions déduites de l’entendement affecte l’être moral : les sensations que le premier reçoit des objets extérieurs, produisent en lui le plaisir et la douleur ; la sensibilité de l’autre affectée par les notions intellectuelles, détermine en lui un autre genre de plaisir et de douleur. Leur volonté respective se détermine par ces motifs respectifs : ce sont donc deux êtres, pour ainsi dire, accolés l’un à l’autre. C’est l’homme et l’animal réunis.\par
L’être physique se montre seul dans les premiers temps de l’enfance ; l’être moral ne se développe qu’à une époque postérieure, et la raison en est bien simple : c’est que, l’être physique est d’abord mis en activité par les sensations qu’il reçoit des objets extérieurs ; tandis que l’être moral ne pouvant déployer la sienne qu’en conséquence des notions, il faut qu’il existe une sorte de mobilier d’idées simples dans l’homme, pour que leur combinaison donne lieu aux notions ; ces notions étant pour l’être moral ce que les objets extérieurs sont pour l’être physique, on peut dire que le monde que le premier doit habiter n’existe pas encore, quand l’autre est en possession du sien.\par
L’être physique peut donc exister sans l’être moral. Les animaux qui n’ont point d’être moral ; les enfants, chez lesquels il n’est pas encore développé, existent par l’être physique seul. Nous n’avons pas la même preuve de fait à donner de l’existence indépendante de l’être moral ; cependant on en conçoit la possibilité ; on sent qu’il pourrait se passer de l’être physique ; il ne lui faudrait pour cela qu’un monde intellectuel : or ce monde existe dans notre entendement ; et, quoiqu’il s’y forme, en quelque sorte, sur le patron du monde physique, on sent parfaitement qu’il pourrait subsister sans son modèle, car les idées survivent aux objets qui les ont produites ; et la destruction des objets n’entraînant pas celle des idées, on peut concevoir le monde matériel anéanti, et le monde intellectuel survivant à sa destruction. Celui-ci suffirait à l’être moral, puisqu’il y trouverait le principe de ses perceptions, de ses sensations et de son activité morale : il pourrait donc exister dans ce monde intellectuel sans l’être physique, comme l’être physique existe dans le monde matériel sans l’être moral.
\chapterclose


\chapteropen
\chapter[{Chapitre VI. Suite du chapitre précédent.}]{Chapitre VI. Suite du chapitre précédent.}\renewcommand{\leftmark}{Chapitre VI. Suite du chapitre précédent.}


\chaptercont
\noindent La différence la plus caractéristique entre les deux êtres que nous analysons, c’est que l’un est libre, et l’autre dépendant ou nécessité.\par
En effet, celui que nous appelons l’être physique est d’abord mis en activité par les impressions qu’il reçoit du dehors ; il faut qu’il sente avant d’être actif : il dépend donc des objets extérieurs pour son activité propre ; il en reçoit des sensations qui déterminent sa volonté ; sa volonté détermine ses actions : le voilà par conséquent entièrement subordonné à l’impression des objets extérieurs qui, en agissant sur sa sensibilité, commandent son attention, sa volonté, son activité : il n’est donc pas libre.\par
L’être moral, au contraire, est d’abord actif, parce que l’attention précède en lui la sensation ; celle-ci est une impression reçue ; c’est une action de l’objet sur l’être qui l’éprouve : dans l’attention, au contraire, l’être agit sur l’objet qui détermine son attention, c’est une activité spontanée de l’être attentif : l’objet qui l’appelle, la détermine ; mais il ne la commande pas : par conséquent, elle est libre.\par
C’est sur les idées simples que l’être moral développe d’abord son activité par l’attention qu’il leur accorde ; il les combine par la réflexion qui n’est que l’attention continuée, et, en les combinant, il en forme des notions qui agissent sur sa sensibilité, comme les objets extérieurs agissent sur celle de l’être physique : il est donc lui-même l’auteur de ses propres sensations, puisqu’il ne les éprouve que par les notions qu’il a formées ; s’il est l’auteur de ses propres sensations, il l’est aussi des déterminations de sa volonté, qui succèdent à ces mêmes sensations ; et il s’ensuit que l’acte émané de cette volonté est un acte {\itshape libre} ; car toutes les opérations qui le précèdent, et qui le déterminent, étant libres, l’acte qui en est le résultat et la conséquence, est de la même nature que ses prémisses.\par
L’être moral est donc libre et indépendant dans toute la teneur de son système, comme l’être physique est dépendant et nécessité dans la plénitude du sien : voilà pourquoi les mêmes actions ne sont ni imputées ni imputables aux animaux ni aux enfants, tandis qu’elles le sont à l’homme. Cette différence ne provient que de l’absence ou du non-développement de l’être moral dans les uns, et de sa présence dans l’autre. Les premiers ne sont pas, à proprement parler, les auteurs de leurs actions, dont ils n’ont pas l’initiative ; c’est le privilège de l’être moral : aussi, dès que cet être est développé dans l’homme, l’homme devient responsable de ses actions, parce qu’il acquiert la faculté d’en être le promoteur, et qu’il les exécute librement.
\chapterclose


\chapteropen
\chapter[{Chapitre VII. Impossibilité d’expliquer la liberté de l’homme, dans l’hypothèse d’une seule volonté.}]{Chapitre VII. Impossibilité d’expliquer la liberté de l’homme, dans l’hypothèse d’une seule volonté.}\renewcommand{\leftmark}{Chapitre VII. Impossibilité d’expliquer la liberté de l’homme, dans l’hypothèse d’une seule volonté.}


\chaptercont
\noindent La liberté consiste à faire ce qu’on veut : telle est la définition adoptée par les philosophes et par le vulgaire ; mais s’il n’existe qu’une volonté dans l’homme, cette faculté identique dans l’homme et dans l’animal ne produira-t-elle pas le même effet dans l’un et dans l’autre ? Je voudrais bien que, dans le système d’une seule volonté, on m’apprît pourquoi l’homme est libre en faisant ce qu’il veut, tandis que les animaux ne le sont pas en faisant ce qu’ils veulent ? Si vous privez ceux-ci de la liberté, ne faudra-t-il pas également en priver l’homme ; et si vous l’accordez à l’homme, ne faudra-t-il pas la leur accorder ?\par
Charles Bonnet, dans son {\itshape Essai analytique sur les facultés de l’âme}, soutient que les animaux et les enfants sont libres : il se sauve par là de l’inconséquence qu’on peut reprocher aux autres métaphysiciens ; mais si les animaux et les enfants sont libres, pourquoi n’y a-t-il ni mérite ni démérite dans leurs actions, et pourquoi en est-il autrement de celles de l’homme ? Cette différence ne tient-elle pas à la liberté des unes et à la non-liberté des autres ? N’est-on pas d’autant plus fondé à le croire, que, lorsqu’il arrive à l’homme d’être contraint dans ses actions, elles ne lui sont pas plus imputées que ne le sont les leurs aux animaux et aux enfants ?\par
Il faut donc bien que les actions des animaux et des enfants, quoique volontaires, ne soient pas libres ; car si elles l’étaient, elles leur seraient imputées : puisqu’elles ne leur sont pas imputables, il est clair que la définition de la liberté ne leur convient point, et qu’ils ne sont pas libres en faisant ce qu’ils veulent : il en serait de même de l’homme, s’il n’y avait en lui que la volonté physique dont les animaux et les enfants sont pourvus ; il faut nécessairement lui supposer, outre la volonté animale, une volonté d’un autre ordre, une volonté privilégiée, à laquelle se rapporte la définition de la liberté : c’est même la seule qui, à proprement parler, mérite le nom de volonté, et, dans ce sens, on peut dire qu’il n’y a que l’homme qui veuille ; car les autres animaux voulant, en quelque sorte, par impulsion, et n’ayant pas sur leur volonté cette initiative que l’être moral exerce sur la sienne dans l’homme, ne peuvent pas être considérés à la rigueur comme des êtres volontaires.\par
Au reste, si vous n’admettez qu’une volonté dans l’homme, la définition de la liberté entraînera les plus étrangères contradictions, et mettra dans les raisonnements qu’on pourra faire à ce sujet, une foule d’inconséquences qui rendront à jamais insoluble la question de la liberté, si longtemps et si infructueusement agitée.\par
En effet, s’il n’y a qu’une volonté dans l’homme, il n’y a qu’une manière de vouloir ; il s’ensuit que lorsque je sacrifie la satisfaction de mes désirs sensuels à l’accomplissement de mes devoirs moraux, je ne fais pas évidemment ce que je veux ; car en résistant à mes désirs, c’est à ma volonté que je résiste. Donc aux termes de votre définition, je ne sais pas libre. Pour l’être véritablement, il faut que je me livre à la recherche des plaisirs sensuels, que je satisfasse mes appétits : les actes qui exigent quelque empire sur soi, seront autant d’actes non libres.\par
Voudrez-vous réformer votre définition, en maintenant l’hypothèse d’une seule volonté ? Direz-vous que la liberté ne consiste pas à faire ce qu’on veut ? Mais alors, en quoi consistera-t-elle ? Sera-ce à faire ce qu’on ne veut pas ? mais cette résistance même à une volonté positive, il faut la vouloir ; il y aura donc une autre volonté ; car celle qui détermine la résistance n’est certainement pas celle à laquelle on résiste. Deux effets aussi contradictoires ne peuvent pas partir d’une même cause : vous serez donc forcé d’admettre deux volontés dans l’homme, ou de renoncer à expliquer sa liberté. Ce système sera le seul dans lequel votre définition de la liberté puisse recevoir l’interprétation qu’il faut nécessairement lui donner pour en reconnaître l’exactitude.
\chapterclose


\chapteropen
\chapter[{Chapitre VIII. Objection à laquelle le système de l’homme intérieur double, peut donner lieu relativement à la liberté.}]{Chapitre VIII. Objection à laquelle le système de l’homme intérieur double, peut donner lieu relativement à la liberté.}\renewcommand{\leftmark}{Chapitre VIII. Objection à laquelle le système de l’homme intérieur double, peut donner lieu relativement à la liberté.}


\chaptercont
\noindent Si l’homme intérieur est effectivement composé de deux êtres différents, dont l’un soit libre et l’autre dépendant ou nécessité, on ne pourra plus regarder l’homme comme un être parfaitement libre. Cette définition ne lui conviendra point dans un sens absolu ; libre par l’être moral, esclave par l’être physique, il n’aura pas ce qu’on peut appeler une liberté positive. Continuellement ballotté entre les deux êtres qui le constituent, tantôt libre, tantôt esclave, il n’aura qu’une liberté équivoque, ou pour mieux dire, il n’en aura point ; il ne sera ni libre ni nécessité ; quelque opinion qu’on adopte à cet égard, on sera également dans l’erreur.\par
Pour répondre à cette objection, nous observerons que, si l’homme est double pour la volonté, il est un pour l’action ; il peut donc agir toujours en conséquence des déterminations de l’être libre, et cette puissance le constitue dans un état de parfaite liberté, malgré la composition mixte de son individu.\par
Supposons un soldat en présence de l’ennemi : l’imminence du danger affecte sa sensibilité physique, et la volonté qui en dépend est par cette impression déterminée à la désertion. D’un autre côté, la notion du mépris qui accompagne la lâcheté, notion que l’être moral a formée en lui, agit dans un sens inverse sur la sensibilité morale, et porte la volonté de cet ordre à une détermination contraire. Voilà bien deux volontés opposées, mais l’action ne peut pas être double ; c’est-à-dire, qu’il n’est pas possible à l’individu, de fuir et de rester en même temps à son poste : il peut donc s’en tenir à la détermination de l’être libre, lui donner son plein et entier effet, et neutraliser la volonté dépendante ; il est donc libre par la puissance de l’action et par l’unité de cette puissance.\par
Si l’action suit la détermination non libre de l’être physique, malgré la détermination contraire de l’être moral ; ce n’est point la liberté dans ce cas-là qui manque à l’individu ; c’est lui, au contraire, qui manque à sa liberté, car il fait ce qu’intérieurement il veut ne pas faire, et quoique aucune puissance extérieure ne contraigne son action, elle n’est pas libre puisqu’elle s’effectue malgré l’opposition de sa volonté morale ; c’est lui-même qui la contraint.\par
Et en effet, la liberté consistant à faire ce que veut la volonté morale, quand celle-ci se trouve en opposition avec la volonté physique, céder à cette dernière, ce n’est pas choisir, c’est manquer à sa propre liberté ; la surmonter c’est être libre. Voilà pourquoi les sacrifices de la vertu, du devoir, de l’héroïsme, sont des actes éminemment libres, parce qu’ils supposent que les déterminations de la volonté morale, ont prévalu sur les déterminations les plus énergiques de la volonté physique.\par
Il ne faut donc pas croire qu’il suffise pour être libre d’écarter tous les objets extérieurs de contrainte, car nous portons en nous-mêmes un principe de compression et d’assujettissement dont nous seuls pouvons nous libérer, et contre lequel nous avons continuellement à défendre notre liberté : c’est l’habitude de le vaincre qui constitue l’exercice libre de la vie ; quelque indépendant qu’on soit des autres, on est esclave si l’on se laisse soi-même dans la dépendance de ce principe intérieur ; l’être le plus libre est celui qui sait le mieux s’en affranchir.\par
Mais on a de si fausses notions de la liberté, qu’on se croit libre dans l’obéissance absolue et non contrariée à ce principe nécessité, tandis que la générosité qui lui résiste, passe pour une violence qu’on se fait à soi-même.
\chapterclose


\chapteropen
\chapter[{Chapitre IX. De l’amour de soi, et des préjugés qu’on s’est fait sur cette passion.}]{Chapitre IX. De l’amour de soi, et des préjugés qu’on s’est fait sur cette passion.}\renewcommand{\leftmark}{Chapitre IX. De l’amour de soi, et des préjugés qu’on s’est fait sur cette passion.}


\chaptercont
\noindent L’amour de soi, comme nous l’avons déjà dit, suppose nécessairement deux êtres en nous. Le moi qui aime et le moi qui est aimé, ne sont pas sans doute le même moi ; nous pouvons dire à cet égard, ce que Smith a dit du moi qui juge et du moi qui est jugé : il est aussi impossible que l’un soit l’autre, qu’il est impossible que la cause et l’effet ne soient qu’une même chose.\par
Si la distinction du physique et du moral dans l’homme constitue cet hermaphrodisme intérieur qui le rend capable de s’aimer ; cette distinction n’existant point dans les autres animaux, le physique étant leur seul apanage, il est naturel d’en conclure que l’amour de soi n’entre point dans le système de leurs passions, et qu’ils ne s’aiment point eux-mêmes.\par
On est pourtant dans l’usage de leur attribuer cette passion, parce qu’on lui rapporte dans l’homme des effets qui ne lui appartiennent pas : on la regarde comme le principe du soin que nous prenons de notre conservation. La recherche du plaisir, la fuite de la douleur, sont encore des inclinations qu’on motive par l’amour de soi. Tout cela se trouvant dans les animaux comme dans l’homme, on en conclut qu’ils s’aiment comme nous nous aimons.\par
Cependant pour peu qu’on veuille réfléchir, on verra que ces effets qu’on rapporte à l’amour de soi sont des penchants naturels de l’être physique, absolument étrangers à la passion d’où l’on veut qu’ils dérivent.\par
Il ne manque pas, en effet, de gens qui, loin de s’aimer, sont très mal avec eux-mêmes, ou, pour mieux dire, qui se haïssent, et qui cependant n’en recherchent pas le plaisir avec moins d’avidité, n’en fuient pas moins la douleur, n’en sont pas moins soigneux de se conserver.\par
Ce n’est donc pas à l’amour de soi qu’appartiennent toutes ces inclinations ; car s’il suffisait de ne pas s’aimer pour ne plus les avoir, ceux qui tiennent beaucoup à la vie, sans s’en aimer davantage, la quitteraient sans doute pour se séparer d’eux-mêmes ; et ces divorces seraient plus communs qu’on ne l’imagine parmi ceux qu’on croit les plus égoïstes.\par
Si ces penchants ne dérivent pas de l’amour de soi dans l’homme, ils ne supposent pas cet amour dans les animaux ; c’est donc bien gratuitement qu’on leur en fait honneur : il n’y a que l’homme qui en soit susceptible.\par
Une autre erreur dans laquelle on est tombé en rapportant à cette passion des effets qui ne lui appartiennent pas, c’est de la croire générale, universelle, indélébile : les penchants qu’on lui attribue se trouvant en nous à toutes les époques de la vie, on en a conclu l’existence permanente de la passion à laquelle on les rapportait ; mais il est de fait que nous sommes susceptibles envers nous-mêmes des mêmes sentiments que nous éprouvons pour les autres ; qu’ainsi nous pouvons nous aimer ou nous haïr nous-mêmes, selon que nous nous trouvons dignes d’amour ou de haine : on s’aimait dans le temps de l’innocence, on se déteste après le crime : la haine de soi peut donc succéder à l’amour de soi.\par
Cependant les penchants qu’on rapporte à cette passion n’en existent pas moins, lorsqu’elle a cessé d’exister : ils sont toujours les mêmes, soit qu’on s’aime, soit qu’on se haïsse. Ce sont donc dans l’homme comme dans l’animal des déterminations naturelles de l’être physique, qui ne supposent ni n’excluent l’amour de soi.
\chapterclose


\chapteropen
\chapter[{Chapitre X. Suite du chapitre précédent.}]{Chapitre X. Suite du chapitre précédent.}\renewcommand{\leftmark}{Chapitre X. Suite du chapitre précédent.}


\chaptercont
\noindent Il paraît bien extraordinaire qu’on ait pu prendre le change sur ce qu’on devait entendre par {\itshape amour de soi} ; car cette expression même indique que nous sommes l’objet propre et direct de cet amour. Or, en lui donnant pour objet le plaisir, la satisfaction des sens, etc., on change, pour ainsi dire, son acception : c’est l’amour du plaisir, l’amour des voluptés ; ce n’est plus l’amour de nous-mêmes.\par
Par amour de soi dans l’homme, nous devons nécessairement entendre {\itshape l’amour de notre être}, et, par amour de notre être, {\itshape l’amour de l’être moral} ; car ce n’est pas ce que nous avons de commun avec les autres créatures qui fait, en quelque, sorte, que nous sommes {\itshape nous}. Le moi moral est notre moi particulier : c’est véritablement le moi humain. Le moi physique étant le moi commun à tous les animaux, n’est pas plus particulier à l’homme qu’à la bête : c’est le moi de l’animalité. Le soi de l’homme réside donc dans l’être moral, et par conséquent l’amour de soi est l’amour de cet être.\par
Cette passion ne diffère pas de celle que nous éprouvons pour les objets extérieurs : ce sont, en effet, les excellentes qualités, les perfections que nous découvrons dans ceux-ci, qui nous déterminent à les aimer : l’amour de nous-mêmes naît également en nous de l’excellence de notre être moral ; il s’exalte par la considération de la dignité, de la perfection, et en un mot de la beauté morale de cet être : alors il nous inspire pour lui le dévouement qu’il est dans la nature de l’amour de produire pour tout ce qui en est l’objet.\par
Quelque énergiques que soient nos penchants, nous les sacrifions à ses déterminations libres et généreuses ; et ces sacrifices dans l’amour de soi n’ont rien de plus extraordinaire que ceux qu’on fait pour les objets extérieurs, quand la même passion les motive. Ne voit-on pas tous les jours ceux qui aiment sacrifier leurs propres inclinations à celles de l’objet aimé ? Ne s’estiment-ils pas heureux de souffrir pour lui plaire ? trouvent-ils rien de coûteux ni de pénible pour arriver à ce but ? Quelque force qu’on veuille donner à nos penchants, ils ne sont pas plus énergiques en nous que la passion de l’or chez l’avare ; celui-ci pourtant sacrifie cet or à la beauté dont il est épris : voilà l’image visible de ce qu’opère l’amour de soi dans l’homme.\par
Ceci nous donne l’explication naturelle des jouissances de la vertu, et nous indique pourquoi elle n’a pas les mêmes attraits pour tous ; car il est facile de voir que sans l’amour de soi, tel que nous l’avons défini, nous ne serons pas disposés à sacrifier à notre être moral, la moindre de nos inclinations physiques : ce serait vouloir qu’on fît pour un objet indifférent, les mêmes sacrifices que pour un objet aimé, et cela n’est pas dans la nature des choses.\par
Cet avare dont nous venons de parler, qui prodigue son or à l’objet dont il est épris ; demandez-lui les mêmes libéralités pour un être qui lui soit indifférent ; vous ne les obtiendrez point. Il en sera de même de l’homme indifférent pour son être moral ; vous aurez beau lui vanter les charmes de la vertu, du désintéressement, de l’héroïsme ; entièrement livré à la servitude de ses penchants, incapable de résister à l’attrait du plaisir, esclave de la vie sans l’aimer, et ne pouvant l’aimer, parce qu’il en est l’esclave, ce ne sera que par la violence, par la contrainte, par la terreur ; des châtiments, que vous le déterminerez à ne pas obéir à ses appétits, comme la brute obéit aux siens.\par
Cependant ce même homme qui trouvera si pénible de lutter contre ses penchants, quand il y sera déterminé par des motifs de crainte ou de coercition et qui leur cèdera toutes les fois qu’il croira pouvoir le faire avec impunité ce même homme, disons-nous, eût fait spontanément avec l’amour de soi, ce que vous ne pouvez obtenir qu’il fasse en employant les moyens les plus rigoureux : car, l’effet de cette passion, eût été de subordonner en lui la volonté physique à la volonté morale, et par conséquent les penchants de l’être physique, aux libres déterminations de l’être moral : celui-ci disposant alors de toutes ces inclinations et les modifiant à volonté, l’homme n’eût point trouvé d’obstacle en lui-même qui l’empêchât de régler sa conduite sur la convenance des choses. Voilà comment l’amour de soi rentre dans l’amour de la vertu, et l’amour de la vertu dans l’amour de la liberté.\par
On se plaint ordinairement de ce que les hommes s’aiment trop eux-mêmes ; on voudrait qu’ils fussent moins égoïstes mais s’il est un reproche à leur faire, c’est de ne pas s’aimer assez : le secret de leur perfectionnement moral, n’est pas de chercher à les détacher d’eux-mêmes comme on se plaît à le répéter ; mais de leur apprendre à aimer leur être pour lequel, quoiqu’on les accuse d’égoïsme, ils sont presque tous indifférents : il faut leur révéler le mystère et les voluptés intérieures de cet amour, pour qu’ils en soient touchés et séduits.\par
Le moyen de produire en eux cette disposition, d’où dépend leur amélioration morale, n’est pas de les dégrader, de les terrifier, de les asservir, de leur inspirer des idées fausses ou abjectes de leur nature, comme on l’a fait jusqu’à présent ; mais de leur donner la conscience de leur dignité, d’en relever, s’il est possible, l’excellence ; car, si l’amour de soi tient aux mêmes motifs que celui que nous éprouvons pour les objets extérieurs, s’il est également fondé sur les perfections senties de l’être qui en est l’objet, plus nous serons intimement convaincus de l’excellence et de la dignité de notre être moral, plus nous serons disposés à l’aimer ; et plus nous l’aimerons, plus les heureuses conséquences que nous avons supposées à cet amour, se développeront, se multiplieront dans la société et concourront par leur développement au bonheur individuel et au bonheur collectif.
\chapterclose


\chapteropen
\chapter[{Chapitre XI. Développement de la théorie de l’amour de soi.}]{Chapitre XI. Développement de la théorie de l’amour de soi.}\renewcommand{\leftmark}{Chapitre XI. Développement de la théorie de l’amour de soi.}


\chaptercont
\noindent Considérez l’homme intérieur sous un seul aspect, il ne présente qu’un être sentant, avide de plaisir, ennemi de la douleur et toujours animé du désir de sa conservation ; examinez-le dans un autre sens, vous voyez un être libre, qui ne veut être commandé ni par l’attrait du plaisir, ni par la crainte de la douleur, ni même par le motif de sa conservation et qui s’indigne de l’empire que ces différentes passions exercent sur nous.\par
Au premier coup d’œil rien ne paraît plus bizarre, plus monstrueux que l’association de ces deux êtres dans un même individu ; car, comment cet individu pourra-t-il goûter un seul instant de repos ni de bonheur ? L’être purement sentant voudra qu’il recherche le plaisir, quand l’être libre demandera qu’il le dédaigne ; l’un lui fera toujours une loi de se conserver, l’autre voudra qu’il préfère sa liberté à sa conservation, et quelque parti qu’il adopte, il sera toujours en proie à des volontés contraires.\par
Voilà bien le portrait de l’homme si vous en écartez l’amour de soi ; mais introduisez-y cet agent, il va tout concilier, tout arranger et fondre ensemble ces deux moi qui paraissent si opposés, pour n’en former qu’un moi unique.\par
D’abord cette biduité intérieure qui nous paraît étrange et bizarre était nécessaire pour que nous puissions nous aimer nous-mêmes ; car elle rompt, pour ainsi dire, notre identité, nous divise nous partage, et nous fournit, en cela, le moyen de nous voir, de nous considérer, comme si nous étions en quelque sorte, hors de nous, ce qui nous met à même de nous apprécier et d’éprouver pour notre être, tous les sentiments qu’un objet extérieur pourrait nous inspirer ; nous pouvons donc nous plaire à nous-mêmes, et cette faculté de nous plaire, entraîne celle de nous aimer.\par
Mais il ne suffisait pas que nous eussions cette faculté de nous aimer, il fallait encore que nous pussions nous donner des preuves de cet amour, et comme les preuves d’amour consistent principalement dans les sacrifices qu’on fait pour l’objet aimé ; il fallait que nous eussions en nous les moyens de nous faire des sacrifices à nous-mêmes ; voilà pourquoi la nature a donné un caractère différent et des inclinations différentes aux deux moi qui sont, si je puis m’exprimer ainsi, que {\itshape nous} aime nous, car s’ils étaient identiques, l’un n’aurait rien à sacrifier à l’autre.\par
Ne soyons donc pas surpris de cette contradiction de nos vœux, ne la regardons pas comme une erreur de la nature ; c’est notre indifférence pour nous-mêmes qui nous la rend onéreuse ; elle disparaît dans l’amour de soi, ou plutôt, elle fournit à cet amour le moyen de s’entretenir et de développer son énergie : ceux qui s’aiment véritablement, seraient bien fâchés de n’avoir point de sacrifice à se faire ; en les privant de cette faculté, on les priverait de la plus grande partie de leur bonheur.\par
Faute d’avoir pénétré assez avant dans l’homme, on l’a renfermé tout entier dans le moi purement sentant ; on n’a pas vu que son moi véritable, était le moi libre, et que par conséquent, l’amour de soi devait se rapporter à ce moi ; qu’il fallait donc dans cette passion, que nous fussions en quelque sorte épris de notre moi libre, et que nous trouvassions en nous-même, le moyen de lui prouver cet amour : c’est à quoi servent les penchants déterminés de l’être physique.\par
Les hommes paraissent s’être singulièrement abusés sur la vraie destination de ces penchants, dont ils se sont partout rendus plus ou moins esclaves : l’intention de la nature en nous les donnant, n’a certainement pas été qu’ils fussent en nous des instruments de servitude ; il faut donc leur supposer un autre emploi dans l’économie humaine, un emploi qui leur ôte leur dépendance naturelle et qui les libéralise ; c’est ce qu’opère l’amour de soi ; car, dans cet amour, ils deviennent la matière de nos sacrifices : ils nous fournissent l’occasion et le moyen de bien mériter de nous-mêmes, et c’est parce qu’ils ont une autre propriété dans leur essence, celle de nous attacher à la vie, de nous lier à notre propre conservation, que le sacrifice que nous nous en faisons pour être parfaitement libres, à un prix qui nous charme intérieurement, et nous porte à nous en tenir compte à nous-mêmes.\par
Et en effet, s’il n’existait en nous aucune disposition à nous faire ces sortes de sacrifices, nous suivrions invariablement la direction de nos penchants, nous leur obéirions machinalement, et comme dans une foule de circonstances, ils nous tireraient en sens contraire, des déterminations du moi libre, si nous n’avions pas le pouvoir de les lui sacrifier, ou, quoique ayant ce pouvoir, si nous n’étions pas dans la disposition d’en faire usage, le moi libre serait le plus souvent en nous dans un état de contrainte et d’oppression ; il lui importe donc que ces penchants lui soient subordonnés, et ce que nous pouvons faire de plus agréable pour lui, c’est de les lui soumettre : ils perdent alors, comme nous venons de le dire, leur dépendance naturelle ; ils ne sont plus des inclinations déterminées de notre nature ; car ils deviennent la propriété du moi libre, qui les modifie dans le sens de ses déterminations spontanées ; c’est librement que nous recherchons le plaisir ; c’est librement que nous fuyons la douleur et que nous nous intéressons à notre conservation. Si la convenance des choses exige une autre détermination, nous sommes prêts à l’adopter. Cet affranchissement de la servitude naturelle des penchants est le produit de l’amour de soi, d’où l’on voit qu’en dernière analyse, cet amour se résout dans l’amour de la liberté.\par
Voilà ce qui fait dire à Montesquieu, en parlant de cette coutume si générale chez les Romains, de se donner la mort : « Tel est le cas que nous faisons de nous-mêmes, que nous consentons à cesser de vivre par un instinct naturel et obscur, qui fait que nous nous aimons plus que notre vie même. »\par
C’est que l’amour de la vie, quelque fort qu’il soit en nous, n’est qu’un penchant de l’être physique ; mais la liberté est l’essence de notre être moral qui est notre véritable être, et il s’ensuit qu’en aimant notre liberté, c’est nous-mêmes que nous aimons ; cet amour exige donc à la rigueur, que le désir de vivre soit subordonné à celui d’être libre. Aimer sa liberté, c’est s’aimer soi-même : aimer la vie, c’est simplement aimer la vie.\par
La pensée de Montesquieu perd toute son obscurité par cette explication : on voit qu’elle est d’une vérité profonde. Plus on fait de cas de soi-même, moins on doit en faire de la vie : plus on en fait de la vie, moins on en fait de soi.\par
L’amour de la vie étant en nous le plus fort de nos penchants, est par cette raison le plus grand sacrifice que nous puissions nous faire à nous-mêmes ; ce sacrifice est donc la plus grande preuve d’amour que nous puissions nous donner, et si nous ne l’envisageons pas ainsi ; ce n’est pas parce que nous nous aimons trop ; mais parce que nous ne nous aimons pas assez.\par
C’est une erreur de croire que la lâcheté provienne de l’amour immodéré de soi : le lâche ne tient tant à la vie, que parce qu’il ne tient pas assez à lui-même. Voilà sans doute la raison du mépris qu’il inspire : le peu de cas qu’il fait lui-même de son être, nous porte à le mépriser comme il le méprise, et c’est sans doute aussi par la raison contraire, que nous estimons et que nous admirons le courage.\par
Au reste, le sacrifice de nos penchants à notre amour pour nous-mêmes ou pour notre liberté, n’exige point l’abnégation de ces mêmes penchants ; c’est-à-dire, qu’il n’est pas nécessaire de renoncer au plaisir, de rechercher la douleur, et enfin de se suicider, pour se donner la dernière preuve d’amour ; il suffit de la disposition constante et positive ; à se faire ces sortes de sacrifices, si l’intérêt de la liberté le demande. Cette disposition suffit à l’amour de soi, comme la disposition à tout donner, suffit à notre amour pour les autres. On sait que l’amour tient compte de l’intention comme du fait. Si j’ai la certitude que tout ce qui est à mon ami est à moi, qu’il est prêt à m’en faire le sacrifice ; cette disposition équivaut dans nos rapports mutuels, au sacrifice reçu : il n’est pas besoin qu’il soit consommé. Je passerais toute ma vie sans emprunter un écu à cet ami, qu’il n’en aurait pas moins le mérite de s’être dépouillé pour moi.\par
Notre amour pour nous-mêmes, tire aussi toute son énergie, de la disposition où nous sommes à nous sacrifier nos penchants ; car, du reste, il faut que les occasions déterminent le sacrifice, et que l’être moral le demande comme une chose nécessaire à sa liberté. Et remarquez qu’à cet égard, nous ne pouvons pas nous en imposer à nous-mêmes, ni nous persuader que nous sommes dans la disposition à nous sacrifier nos penchants, si nous ne sommes pas réellement dans cette disposition. Nous ne pouvons pas nous tromper nous-mêmes par de fausses démonstrations de dévouement, comme il nous arrive quelquefois de tromper les autres.\par
Concluons de tout ce qui précède, que l’amour de nous-mêmes est proprement l’amour de notre liberté ; que par conséquent cet amour ne comprend point dans son acception l’amour de la vie, l’amour du plaisir et autres inclinations déterminées de notre nature, qui constituent les penchants de l’être purement sentant ; qu’il faut, au contraire, pour nous aimer véritablement, que nous soyons toujours dans la disposition de sacrifier ces inclinations déterminées aux déterminations spontanées du moi libre. Si nous ne sommes pas dans cette disposition positive, nous ne tenons point à notre liberté ; ne tenant point à notre liberté, nous ne tenons point à notre véritable moi : dès lors quelque soin que nous prenions de conserver notre individu, nous sommes réellement indifférents pour nous-mêmes.
\chapterclose


\chapteropen
\chapter[{Chapitre XII. Suite du chapitre précédent.}]{Chapitre XII. Suite du chapitre précédent.}\renewcommand{\leftmark}{Chapitre XII. Suite du chapitre précédent.}


\chaptercont
\noindent Si c’est de l’amour de soi, comme on le prétend, que dérivent l’amour du plaisir, le désir de notre conservation ; pourquoi ne sommes-nous pas touchés des preuves d’amour que nous nous donnons dans ce sens à nous-mêmes ? pourquoi ne me sais-je pas plus de gré d’avoir fait un bon repas, qu’une bonne action ? Puis-je, dans les systèmes reçus, me donner une plus grande preuve d’aversion pour moi-même, que d’exposer ma vie pour sauver mon semblable ? D’où vient pourtant que je ne m’en veux pas de l’avoir exposée ? que c’est au contraire une chose dont je me sais toujours gré, qui me satisfait intérieurement, et me met, pour ainsi dire, en bonne intelligence avec moi-même ?\par
Voilà certes une étrange contradiction dans les effets que vous rapportez à l’amour de soi ; car les actes les plus contraires à cet amour, dans votre système, sont précisément ceux dont nous nous savons le plus de gré, tandis que nous ne nous tenons aucun compte de ceux qui en émanent directement : nous sommes d’une ingratitude sans égale envers nous-mêmes pour ces sortes de soins.\par
J’en appelle au témoignage de ces hommes qu’on nomme si improprement égoïstes, qui passent toute leur vie à se choyer, à se délecter ; qui poussent jusqu’au scrupule les attentions délicates pour leur personne ; demandez-leur s’ils en sont mieux avec eux-mêmes. S’ils veulent être sincères, ils vous répondront que leur indifférence pour eux s’accroît en raison des soins qu’ils se donnent pour s’aimer ; que quoiqu’on les accuse de s’aimer excessivement, ils ont le malheur de ne pas s’aimer du tout ; que leur être n’est nullement touché de tous les soins qu’ils prennent de lui plaire ; qu’au lieu de répondre à leurs avances, il reste intérieurement froid et glacé pour eux comme pour les autres.\par
D’après la manière dont vous expliquez l’amour de soi, il faudra croire que Socrate, qui ne voulut pas sortir de sa prison quand on lui en offrit les moyens, était fort mal avec lui-même : que Scipion, qui s’honora par sa continence autant que par ses victoires, ne s’aimait pas du tout ; que la même indifférence régnait dans ce Valerius, qui, trois fois consul, mourut sans laisser de quoi se faire enterrer. Il en sera de même d’Aristide, d’Epaminondas, des deux Brutus, de Caton, et, en un mot, de tous ceux qui ont le plus honoré la nature humaine. Il n’y aura que les lâches, les fripons, les dilapidateurs, les hommes sensuels qui seront en possession de s’aimer véritablement ; ce seront les amants d’eux-mêmes par excellence ; car, pour être conséquent à votre définition de l’amour de soi, il faut, pour m’aimer véritablement, que je cherche le plaisir à tout prix ; qu’à tout prix je fuie la douleur, que je sacrifie tout à ma conservation : m’écarter de cette direction, c’est me faire une infidélité.\par
Telle est en dernier résultat la conclusion forcée de toutes ces doctrines qu’on appelle philosophiques, parce qu’on les croit appuyées sur la connaissance approfondie de la nature humaine.\par
Il est vrai que leurs auteurs modifient les principes par des considérations d’intérêt personnel qui ne permettent pas de les suivre à la rigueur : le plaisir goûté à l’excès amène la douleur, il faut donc se modérer dans le plaisir : la douleur est quelquefois salutaire, il ne faut donc pas toujours la fuir, et quelquefois il faut exposer sa vie pour la conserver. On ne peut donc pas s’aimer autant qu’on le voudrait, ni comme on le voudrait dans ce système ; il faut maintenir une sorte d’équilibre entre les divers penchants, et soumettre l’amour de soi au calcul de l’intérêt : on ne peut pas toujours s’en donner des preuves. Il y a même des infidélités forcées ; car de quelque manière qu’on s’arrange, on n’a pas toujours le plaisir à sa disposition ; il faut bien se résoudre quelquefois à sentir la douleur et quelque soin qu’on prenne de se conserver, il vient un moment où, malgré soi, l’on manque au devoir de sa conservation.\par
Il y aurait donc une bien singulière contradiction dans les procédés de la nature à notre égard ; elle nous aurait imposé d’un côté le devoir de nous aimer nous-mêmes, et de l’autre elle nous aurait forcés à la déloyauté en soumettant cet amour à des conditions que tôt ou tard il nous serait impossible de remplir ! Telle est l’incohérence et l’absurdité du système qui base l’amour de soi sur la satisfaction ou plutôt sur la servitude de nos penchants. Il était difficile de s’égarer plus complètement dans la recherche de la vérité.\par
Mais ce qui démontre, surtout, la fausseté de cette théorie, c’est que l’amour, quel qu’il soit, se fonde sur des sacrifices ; et quels sacrifices se fait-on à soi-même dans la satisfaction de ses penchants, ou dans les soins qu’on se donne pour les satisfaire ? Quelque pénibles qu’on suppose les soins de cette nature, ce ne sont jamais que des sollicitudes mercenaires qui ont leur salaire proportionné dans la satisfaction du désir qui les motive ; ils n’ont, par conséquent, aucun des caractères de désintéressement et de générosité qui constituent les sacrifices. Si c’est le plaisir que vous cherchez, le plaisir obtenu vous paye de vos peines. Ne l’obtinssiez-vous pas, elles n’ont rien de méritoire à raison de leur but intéressé ; c’est, pour ainsi dire, une spéculation manquée et voilà tout. Ce commerce, absolument étranger à l’amour de soi, ne peut jamais établir aucun rapport d’intimité entre nous et notre être, quelque actif que vous le supposiez il ne vous en laissera pas moins dans la plus parfaite indifférence pour vous-même ; car, au fonds, vous n’aurez rien fait pour vous.\par
Mais si vous avez sacrifié la satisfaction de vos penchants aux déterminations de votre moi libre ; si vous avez dédaigné le plaisir, affronté la douleur, bravé la mort par un motif de rectitude morale, ces penchants non satisfaits sont un bien dont vous vous êtes privé pour maintenir la liberté de votre être ; vous avez été généreux envers lui, c’est évidemment une preuve d’amour que vous lui avez donnée ; il doit vous en tenir compte, et il le fait toujours. Il s’ensuit, malgré l’opinion contraire, que plus on tient à ses penchants, moins on tient à soi ; et que plus on tient à soi, moins on tient à ses penchants.
\chapterclose


\chapteropen
\chapter[{Chapitre XIII. Impossibilité de concilier l’existence simultanée de la liberté et des penchants dans le système de l’unité de l’homme intérieur. Confirmation de la théorie de l’amour de soi, exposée dans les chapitres précédents.}]{Chapitre XIII. Impossibilité de concilier l’existence simultanée de la liberté et des penchants dans le système de l’unité de l’homme intérieur. Confirmation de la théorie de l’amour de soi, exposée dans les chapitres précédents.}\renewcommand{\leftmark}{Chapitre XIII. Impossibilité de concilier l’existence simultanée de la liberté et des penchants dans le système de l’unité de l’homme intérieur. Confirmation de la théorie de l’amour de soi, exposée dans les chapitres précédents.}


\chaptercont
\noindent Par un être libre, on doit nécessairement entendre un être qui, sans inclination déterminée dans sa nature, a la puissance virtuelle de prendre et de quitter toutes celles qu’il lui plaît d’adopter ; car, si vous lui supposez une tendance naturelle vers un objet quelconque, ou un éloignement naturel pour quoi que ce soit, il n’est plus libre. Cette contradiction existe pourtant dans l’homme ; il est libre et néanmoins il tend naturellement vers le plaisir : il fuit naturellement la douleur ; il désire naturellement sa conservation. Comment concilier sa liberté avec ces déterminations positives et innées, en quelque sorte, dans sa constitution. C’est à quoi l’on ne parviendra jamais dans le système de l’unité de l’être sentant et voulant dans l’homme ; car il impliquera toujours qu’un même être soit à la fois libre et nécessité.\par
Mais cette contradiction disparaît dans le système de l’homme intérieur mixte ; car il n’implique pas que des deux êtres qui le composent, l’un soit libre et l’autre absolument nécessité ; que les penchants se trouvent dans celui-ci, et la liberté dans celui-là. Il en résulte que ces penchants, quoique en nous, sont, en quelque sorte, hors de nous. Ce n’est pas à notre être proprement dit qu’ils appartiennent, mais à l’être collatéral qui lui est associé dans notre constitution intérieure. Quoiqu’il y ait entre eux communauté d’existence, il n’y a pas communauté de propriétés. La liberté est l’essence de l’un, les\par
Cette explication une fois admise, il est aisé de voir qu’en travaillant à satisfaire les penchants, on ne fait réellement rien pour soi : quoique cette proposition paraisse le plus étrange de tous les paradoxes, elle n’en est pas moins de la plus exacte vérité. En effet, le moi libre étant le véritable moi de l’homme, et les penchants n’appartenant ni ne pouvant appartenir à ce moi, il est clair que nos soins pour les satisfaire se rapportent exclusivement à l’être nécessité, qui est en nous, sans être nous-mêmes. Notre moi, proprement dit, ne peut donc pas nous savoir gré de ce que nous ne faisons pas pour lui : nous ne devons pas être surpris qu’il ne nous tienne aucun compte des soins que nous nous donnons pour satisfaire des penchants qui lui sont étrangers ; il approuve ces sollicitudes, il s’y prête volontairement quand elles ne choquent point sa liberté ; mais elles n’ont ni ne peuvent avoir rien de méritoire envers nous-mêmes. C’est faute d’être convaincus de cette vérité, que nous sommes perpétuellement le jouet des désirs qui naissent de nos penchants, que nous croyons trouver le bonheur dans leur satisfaction, et que nous ne l’y trouvons jamais.\par
Au reste, pour que notre liberté ne souffre point de ces inclinations déterminées que la nature a mis en nous sans notre participation, il faut qu’elles nous deviennent propres ; c’est-à-dire qu’il soit en nous de les modifier à notre gré, et il faut pour cela que nous les mettions à la disposition de notre moi libre : c’est moins un sacrifice qu’une appropriation de ces sortes de penchants ; car si nous les laissons dans leur indépendance naturelle de nous-mêmes ; si, par la disposition à nous les sacrifier, nous n’exerçons pas sur eux une juridiction effective, ils sont en nous sans être à nous ; ils nous appartiennent moins que nous ne leur appartenons ; ils nous commandent, et nous obéissons.\par
Il faut donc, pour qu’ils soient à nous, que nous en acquérions la propriété ; et acquérir cette propriété, c’est acquérir celle du moi nécessité, car ce moi se composant de ces penchants, le moi libre en les absorbant, l’absorbe, pour ainsi dire, lui-même : c’est une espèce de transsubstantiation d’un moi dans un autre. Par cette opération, le moi moral, le moi libre, le véritable moi de l’homme devient le moi absolu. De double qu’il était, l’homme intérieur devient un ; c’est une sorte de génération intérieure dans laquelle l’homme, pour ainsi dire, se refait lui-même.\par
Par l’unité de moi qui en résulte, la volonté devient une ; elle dispose de tous les moyens qui sont dans notre nature, les soumet tous à ses déterminations libres et spontanées : de là, la force, la vertu, l’énergie, la puissance, tous termes synonymes qui désignent l’empire de l’être moral dans notre constitution, la souveraineté du moi libre, et l’anéantissement ou la subordination du moi nécessité.\par
La nature, en constituant l’homme intérieur double, semble s’être reposée sur l’amour de soi, du soin d’achever son ouvrage : elle a voulu qu’il devînt un par cette passion ; mais notre indifférence pour notre moi positif, trompe ses intentions, et, dans l’imperfection où nous laisse cette indifférence pour nous-mêmes, notre biduité intérieure, loin de s’effacer, se prononce tous les jours davantage : nous ressemblons à des époux divorcés, qui se trouveraient forcés d’habiter continuellement ensemble.\par
Cette mésintelligence, que nous mettons sur le compte de la nature, est absolument de notre fait. Est-ce sa faute si nous ne nous aimons pas si nous prenons le change sur cet amour, et enfin si nous sommes avares des penchants qu’elle avait mis en nous pour que nous fussions généreux envers nous-mêmes ? Ce que vous appelez intérêt personnel, n’est pas seulement un mal dans vos rapports avec les autres, il l’est bien plus envers vous ; car c’est contre vous-même que vous êtes intéressé.\par
Quel est, en effet, le premier intérêt d’un être libre ? N’est-ce pas de rapporter tout à sa liberté, de s’affranchir de tous les genres de servitude ? Est-ce là ce que vous entendez par intérêt personnel ? Non : c’est d’être esclave de vos penchants, de goûter le plaisir, d’être exempts de douleur, de veiller à votre conservation, d’obéir, sans la moindre contradiction, à ces inclinations déterminées. Vous faites consister l’amour de soi dans cette servitude, et vous ne voyez pas que c’est vous isoler de vous-même ; que c’est vous réduire, autant qu’il est en vous, à un état d’invirtualité, d’abjection et d’impuissance : car c’est par notre liberté que nous sommes forts ; et notre liberté demande que nous soyons supérieurs à nos penchants. L’intérêt personnel n’est donc pas de leur obéir ; mais de se les soumettre ; l’amour de soi ne consiste pas à les satisfaire ; mais à se les subordonner.
\chapterclose


\chapteropen
\chapter[{Chapitre XIV. De la nature du bonheur.}]{Chapitre XIV. De la nature du bonheur.}\renewcommand{\leftmark}{Chapitre XIV. De la nature du bonheur.}


\chaptercont
\noindent L’idée que nous nous formons du bonheur, nous présente un état permanent de satisfaction intérieure : nous ne pouvons donc pas le rapporter à la sensibilité, car la sensibilité doit nécessairement éprouver différentes modifications ; il s’ensuit que l’être physique étant un être purement sentant, est par cela même, incapable de bonheur, et qu’il faut en chercher le principe dans l’être moral ou dans l’être libre, ce qui est la même chose : le bonheur est à celui-ci, ce que le plaisir est à l’autre, c’est-à-dire, qu’il constitue la satisfaction qu’éprouve l’être moral du développement et de la jouissance de sa liberté, comme le plaisir constitue la satisfaction qu’éprouve l’être physique, lorsque sa sensibilité est agréablement affectée.\par
On voit donc que le bonheur et le plaisir qu’on est dans l’habitude de confondre, se rapportent à des propriétés différentes, et sont réellement indépendants l’un de l’autre. Il est possible que l’être moral soit heureux par la conscience de sa liberté, au moment où l’être physique souffre, comme il est possible que celui-ci jouisse et que l’autre soit mécontent de son état ; ce phénomène absolument inexplicable dans les systèmes basés sur l’unité de l’être, s’explique naturellement dans celui de l’homme intérieur double.\par
Il est toujours en notre pouvoir d’être heureux, parce qu’il est toujours en notre pouvoir d’être libres ; mais il ne dépend pas de nous d’avoir toujours du plaisir : je ne puis pas répondre que le mets le plus délicat ne me deviendra point insipide, ni que telle femme dont la beauté me ravit actuellement me charmera toujours. Qu’on me répète continuellement le morceau de musique le plus délicieux, je finirai malgré moi par le trouver insupportable. Nous ne sommes donc pas les maîtres d’éprouver le plaisir à volonté ; mais le bonheur est toujours à notre disposition : voilà pourquoi son idée ne comprend pas celle de l’inconstance qui entre comme partie intégrante dans celle du plaisir. Cet état n’est point un rêve de l’imagination, ni une abstraction métaphysique, comme on le croit ordinairement ; il a la raison de son existence positive dans la nature même de l’homme, il ne tient qu’à lui d’être heureux, puisqu’il ne tient qu’à lui d’être libre.\par
C’est notre indifférence pour notre liberté qui est la source de toutes nos fausses opinions sur le bonheur ; qui nous porte tantôt à le regarder comme un être chimérique, et tantôt à le confondre avec le plaisir. On se tourmente, on se fatigue, pour entasser des moyens de plaisir avec lesquels on se figure qu’on composera le bonheur. C’est la même erreur que celle des alchimistes, qui veulent faire de l’or avec les autres métaux.\par
Quoique cette méprise soit à peu près générale, il est pourtant vrai que nous sentons confusément la différence spécifique du bonheur et du plaisir, et c’est parce qu’elle existe dans la nature des choses, que deux systèmes opposés de civilisation et de mœurs, ont à diverses époques partagé l’opinion des philosophes et des politiques.\par
Les uns ont voulu qu’on bannît de la société le luxe, les richesses, les plaisirs sensuels ; les autres au contraire n’ont cessé de provoquer la culture des arts, du commerce, de l’industrie, qui introduit dans une nation les richesses, le luxe et le goût des plaisirs : il est bien évident que les premiers se proposent une fin qui n’est pas celle que les autres ont en vue, et que par conséquent ces deux systèmes ne peuvent pas avoir une base commune ; cependant, tous les deux s’étayent également de la nature humaine qui fournit à chacun ses motifs, et leur prête en quelque sorte le même appui ; tout cela ne peut s’expliquer que par la duplicité de cette nature, qui fait que ces deux systèmes quoique opposés n’impliquent point, et offrent les mêmes rapports de convenance avec elle.\par
En effet, l’un trouve à se justifier par l’exigence de la liberté et du bonheur qui sont les besoins de l’être moral, et l’autre se fonde sur la sensibilité et le plaisir qui constituent la propriété de l’être physique. Pour concilier ces deux systèmes qui semblent s’exclure mutuellement, il faut appliquer aux nations la théorie de l’amour de soi, que nous avons développée relativement à l’homme : il faut que l’amour de soi, dans ces êtres politiques, soit ce qu’il est dans l’homme, c’est-à-dire, l’amour de leur liberté, et qu’ils lui subordonnent leurs arts, leurs richesses, leur luxe, leurs plaisirs, de la manière dont nous ayons dit que l’homme devait subordonner ses penchants à la sienne. Alors, il n’y aura plus rien de contradictoire dans les deux systèmes : l’opulence et la liberté, le bonheur et le plaisir se combineront dans l’économie sociale par le même artifice qui combine les penchants avec la liberté dans l’économie humaine.\par
Cette combinaison dont nous n’avons pas encore eu d’exemple, serait sans doute le chef-d’œuvre de la politique ; mais les intérêts de la liberté doivent l’emporter sur tout, et sous ce rapport le système qui en fait exclusivement son objet, mérite bien certainement la préférence sur celui qui ne paraît occupé que de ce que j’appellerais volontiers l’homme inférieur, ou la nature inférieure de l’homme. L’énergie que le premier donne à la liberté, celui-ci la communique aux penchants, sans fortifier la liberté dans la même proportion ; il en résulte que cette dernière s’invirtualise, qu’elle est opprimée dans l’homme par les penchants dominateurs, et que dès lors, il n’y a point de bonheur à espérer pour lui, parce qu’il n’est pas ce qu’il doit être ; le caractère propre et distinctif de sa nature est d’être libre ; le reste lui est commun avec les autres animaux, d’où il suit que s’il n’est pas libre, il ne vit pas conformément à sa nature, et, s’il ne vit pas conformément à sa nature, il est clair qu’il ne peut pas être heureux.\par
D’ailleurs, le système qui sembla faire abstraction du plaisir pour tout donner à la liberté et au bonheur qui en est la conséquence, sert peut-être mieux les intérêts de ce qu’il paraît négliger, que celui qui en fait l’objet essentiel de sa sollicitude ; car le plaisir est une espèce de bonne fortune qu’on trouve sans la chercher, et qui fuit presque toujours celui qui la cherche. C’est un bien que nous n’avons pas besoin de poursuivre, que nous pouvons négliger sans inconvénient, parce que la nature en fait les frais quand nous ne nous en mêlons pas. Comparez une danse de village où elle est la seule maîtresse des cérémonies, avec un bal de l’opéra, et dites-moi de quel côté se trouve la plus grande somme de plaisir.\par
Rien n’est donc plus insensé que de faire provision de jouissances ; car nous ne savons pas si nous serons flattés demain, de celles que nous ramassons aujourd’hui ; cependant, la cupidité de ces sortes de biens nous fait négliger peux qui sont toujours à notre disposition, et dont le principe est dans notre nature ; en effet, à mesure que les arts et la civilisation nous apprennent tout le parti que nous pouvons tirer des objets extérieurs, nous oublions de tirer parti de nous-mêmes : nous ne savons plus nous aider de nos propres forces ; nous perdons, pour ainsi dire, la conscience de nos moyens intrinsèques. En nous enrichissant à l’extérieur, nous nous appauvrissons intérieurement.\par
Pour nous en convaincre, nous n’avons qu’à nous comparer à ces hommes non civilisés que nous appelons sauvages ; je ne parle point de cet état de pénurie dans lequel ils vivent, qui, chez nous, se nommerait indigence, sur lequel nous nous lamenterions du matin au soir, et qui ne leur inspire pas même l’idée de se trouver misérables ; on sait encore qu’ils bravent les tourments les plus atroces, qu’ils supportent des douleurs inouïes sans proférer la moindre plainte. Leur constance inaltérable défie la cruauté la plus ingénieuse : on dirait qu’ils peuvent, quand il leur plaît, abandonner leur corps à la douleur, sans en être affectés.\par
Ce sont pourtant des hommes comme nous, et c’est en vain qu’on voudrait attribuer ce phénomène à une sorte d’insensibilité physique, car il faudrait les supposer d’une insensibilité absolue.\par
Les Hurons, les Algonquins, les Chérokées, ont sans doute l’épiderme moins délicat que nos petites maîtresses et nos jeunes adonis ; mais il serait absurde d’en conclure qu’ils ne doivent pas souffrir, quand le fer et le feu les déchirent ; pourquoi donc n’en paraissent-ils pas affectés, tandis que la moindre égratignure nous met au désespoir ? pourquoi ne se trouvent-ils pas malheureux dans la condition la plus pénurieuse, tandis qu’avec du superflu nous nous trouvons extrêmement à plaindre ? C’est qu’ils ont l’énergie du courage et la force de la volonté morale, qui est presque nulle chez nous ; c’est que la cupidité du plaisir ne les a pas invirtualisés, qu’ils sont nos maîtres dans une science de l’homme, que des habitudes molles et pusillanimes nous ont fait oublier, celle de souffrir ; c’est qu’ils trouvent, dans le sentiment énergique de la liberté, la force de vaincre la douleur.\par
La sensibilité n’est, en quelque sorte, qu’un élément qui nous entoure et dans lequel nous existons ; elle est, pour ainsi dire, l’atmosphère de notre être moral ; c’est par elle qu’il communique avec les objets extérieurs : elle est, si je puis m’exprimer ainsi, le conducteur ou le milieu à travers lequel il exerce son activité ; mais elle n’est pas une de ses qualités propres comme la liberté. Voilà pourquoi il est dans l’ordre qu’elle soit subordonnée à celle-ci ; c’est-à-dire que l’homme, pour être libre, n’en tienne pas plus de compte que si elle n’existait pas.\par
S’il n’en était pas ainsi dans la nature des choses, la plupart des actes répréhensibles trouveraient leur excuse dans le ménagement de la sensibilité. On pardonnerait à la lâcheté que motiverait ce ménagement ; elle n’aurait pas même besoin d’indulgence ; mais si, malgré soi, l’on est toujours sévère à son égard, et si l’on ne se la pardonne pas plus à soi qu’on ne la pardonne aux autres, c’est qu’il faut qu’il y ait en nous un principe supérieur à toutes les modifications de la sensibilité, et qui exige que nous le maintenions dans cette indépendance.\par
La fin de la sensibilité n’est pas proprement dans les impressions que nous en recevons, c’est-à-dire, qu’elle ne nous délecte pas pour nous délecter ; elle ne nous affecte pas non plus de douleur pour que nous souffrions ; nous ne sommes pas l’objet définitif de ces sensations ; elles n’ont lieu en nous que pour nous attirer vers les objets extérieurs, nous mettre en rapport avec les autres, et nous donner la conscience de leur état, par celui où nous nous trouvons nous-mêmes. C’est, comme nous l’avons dit plus haut, une espèce de conducteur ou de milieu, à travers lequel nous allons chez eux, et qu’ils passent pour se rendre chez nous. Il s’ensuit que la destination de la sensibilité en nous, se rapporte moins à nous qu’aux autres ; en la bornant à nous-mêmes, c’est-à-dire, en terminant à nous l’effet de ses impressions, en nous regardant comme leur objet définitif, et ne prenant en considération que le plaisir ou la douleur qu’elles nous procurent, nous en dénaturons l’emploi, nous en intervertissons l’usage : car s’il était dans les vues de la nature, que la sensibilité n’eût trait qu’à nous, nous manquerions à ses lois, en la détournant de cette direction pour la porter sur les autres : on n’aurait de sensibilité que pour soi ; plus de bienveillance, d’humanité, de commisération ni de pitié pour personne : il est pourtant de fait que ces sentiments, rapportés à nous-mêmes, nous avilissent et nous dégradent, tandis qu’appliqués à nos semblables, ils nous honorent et nous dignifient.\par
Il y a donc cette différence entre la liberté et la sensibilité, que l’une nous appartient exclusivement, et que tout le monde a des droits sur l’autre. C’est, pour ainsi dire, un serviteur que nos compagnons d’existence ont chez nous, et que nous ne pouvons employer à notre usage, que lorsqu’il n’a rien à faire pour eux. Autant nous devons être égoïstes et jaloux de notre liberté, autant nous devons être généreux et prodigues de sensibilité ; car il est une foule de circonstances où la plus grande preuve d’amour que nous puissions nous donner à nous-mêmes, est d’être sans pitié pour les maux que nous souffrons ; sévérité qui constitue la vertu, parce que nous en sommes l’objet, tandis qu’elle ne serait qu’humanité, barbarie et férocité, si nous nous permettions de l’appliquer aux autres.\par
Mais nous avons tellement pris le change sur les véritables intentions de la nature, que la plupart des hommes jettent leur liberté à la tête du premier venu, la sacrifient pour le moindre intérêt, et se font le centre exclusif de leur sensibilité, s’imaginant trouver le bonheur dans ce lâche égoïsme. Qu’en résulte-t-il ? qu’ils n’ont ni le bonheur qui leur eût procuré la jouissance de leur liberté, ni les plaisirs d’une sensibilité généreuse ; en se l’appropriant exclusivement, ils la stérilisent.\par
Il faut donc, comme nous l’avons dit dans les chapitres précédents, que le moi moral domine dans l’économie humaine, et que la sensibilité lui soit subordonnée, pour qu’il puisse en disposer au gré de ses intentions libres et généreuses. En renonçant à votre liberté, pour vous concentrer dans la sensibilité, vous avez pris à contresens le vrai système de l’homme ; vous avez lâché ce qu’il fallait retenir, et retenu ce qu’il fallait lâcher. La liberté que vous avez aliénée valait mieux que tous les biens que vous avez acquis. La nature vous l’avait donnée pour vous, et vous l’avez sacrifiée à la sensibilité qui vous appartient moins qu’elle n’appartient aux autres, et qu’il faut prodiguer pour en jouir. Cette méprise vous a rendu l’esclave des objets par lesquels la sensibilité peut être délectée ; et c’est ainsi que, pour acquérir la propriété des choses, vous avez perdu la propriété de vous-même.\par
On croit communément que le riche a des moyens d’être heureux, qui ne sont pas à la portée du pauvre ; mais encore une fois, un mobilier de plaisir n’est pas un mobilier de bonheur. Celui-ci tient à l’énergie de la liberté : plus on est libre, plus on est heureux. Les éléments du bonheur sont donc en nous-mêmes, et ceux du plaisir hors de nous. Le riche et le pauvre, pour être heureux, ont également à se défendre de la dépendance des objets extérieurs ; car il faut, pour se maintenir libres, que l’un lutte contre ses besoins, et l’autre contre ses jouissances.\par
Ce n’est point la misère qui fait le malheur du pauvre : il n’est malheureux que lorsqu’il est sans courage, sans amour pour sa liberté ; voilà la véritable indigence. L’opulence ne met point à l’abri de cette misère effective, et les chaumières la supposent, moins que les palais.\par
La vie est un théâtre, où malgré la diversité des costumes, les acteurs égaux et camarades n’ont qu’un même rôle à remplir, celui d’être libres. La différence des positions ne met pas plus d’inégalité entre eux, que la diversité des emplois n’en met dans une troupe de comédiens. Juger du bonheur ou du malheur des individus par leur condition respective dans la société, c’est prendre le personnage pour l’acteur. Si vous êtes véritablement homme, vous vous trouverez bien dans toutes les conditions ; si vous ne l’êtes pas, vous ne serez satisfait d’aucune, parce que vous ne serez jamais satisfait de vous-même.\par
Notre attachement pour des objets quelconques ne peut remplir nos vœux qu’autant qu’il est libre ; car s’il est servile il n’est plus conforme à notre nature, et dès lors il nous dégrade et nous rend malheureux au sein même des plaisirs qu’il nous procure. L’art d’être libre est donc encore l’art de jouir. Les vrais épicuriens sont les hommes libres : cela est si vrai qu’on peut mettre en fait que l’homme qui goûterait le mieux le bienfait de la vie, serait celui qui se sentirait le maître de la quitter à volonté.\par
Ce principe est d’une application générale à tous les objets de notre cupidité ; c’est pour en mieux jouir qu’il faut s’en rendre indépendant : attirons-les à nous ; mais ne souffrons pas qu’ils nous envahissent. La plupart de ceux que dévore la soif de s’enrichir ne savent pas qu’ils sont les gens les moins propres à la condition après laquelle ils soupirent, et que ce métier qui leur paraît si facile, ne convient réellement qu’à celui auquel il serait indifférent d’être pauvre.\par
S’ils étaient bien convaincus de cette vérité, s’ils pouvaient se persuader que pour être riche et heureux en même temps, il faut l’être indépendamment de sa fortune, je ne sais s’ils voudraient être riches, s’ils ne trouveraient pas que cette condition leur surfait le bonheur au lieu de le leur offrir à grand marché, comme ils l’imaginent.\par
Voilà pourquoi l’on ne les voit jamais satisfaits des biens acquis ; ils entassent richesses sur richesses, cherchant toujours dans l’accroissement de leurs possessions, ce qu’ils ne peuvent trouver que dans l’indépendance de ce qu’ils possèdent.\par
Je ne serais pas même surpris qu’ils ne tendissent à la liberté par une marche opposée ; qu’ils ne se figurassent qu’à force de multiplier leurs acquisitions, ils se libéreront de la soif d’acquérir ; qu’ils seront moins dominés par une fortune immense que par une propriété exiguë, et que leur servitude perdra, pour ainsi dire, de son intensité, en obtenant plus de surface ; mais en croyant la neutraliser, ils lui donnent une plus grande consistance. Ce n’est pas se libérer comme ils se l’imaginent, c’est simplement se renfermer dans le cercle vicieux de la cupidité.\par
La nature n’a pas mis votre liberté dans les choses, pourquoi donc la cherchez-vous où elle n’est pas ? Ce n’est point à votre qualité de propriétaire foncier qu’elle est attachée ; mais à la propriété de vous-même. En abandonnant celle-ci vous abandonnez celle qui rapporte le bonheur ; vous ne pouvez donc pas recueillir le fruit d’une propriété que vous avez aliénée.\par
Que cette économie est fausse et misérable ! Quel contresens dans la juste appréciation des choses ! Nous avons si bien pris l’inverse de notre intérêt, bien entendu, qu’au lieu de mettre tout dans notre dépendance, nous nous sommes mis dans la dépendance de tout. Nous nous sommes dépréciés nous-mêmes pour donner du prix aux objets qui n’en avaient pas. Cette folie a été poussée au point qu’il est peu de choses dont nous ne fassions plus de cas que de nous-mêmes, et auxquelles, pour peu qu’on s’avise de nous marchander, nous ne soyons toujours prêts à nous vendre.
\chapterclose


\chapteropen
\chapter[{Chapitre XV. De l’homme double de Buffon, et de la critique qu’en a fait Condillac.}]{Chapitre XV. De l’homme double de Buffon, et de la critique qu’en a fait Condillac.}\renewcommand{\leftmark}{Chapitre XV. De l’homme double de Buffon, et de la critique qu’en a fait Condillac.}


\chaptercont
\noindent C’est dans le {\itshape Discours sur la Nature des Animaux}, qu’on trouve la théorie de Buffon sur la biduité intérieure de l’homme. « {\itshape Homo duplex}, dit-il, l’homme intérieur est double, il est composé de deux principes différents par leur nature, et contraires par leur action. L’âme, ce principe spirituel, ce principe de toute connaissance, est toujours en opposition avec cet autre principe animal et purement matériel : le premier est une lumière pure qu’accompagnent le calme et la sérénité, une source salutaire d’où émanent la science, la raison, la sagesse ; l’autre est une fausse lueur, qui ne brille que par la tempête et dans l’obscurité, un torrent impétueux qui entraîne à sa suite les passions et les erreurs.\par
« Le principe animal se développe le premier ; comme il est purement matériel, et qu’il consiste dans la durée des ébranlements et le renouvellement des impressions formées dans notre sens intérieur matériel, par les objets analogues ou contraires à nos appétits, il commence à agir dès que le corps peut sentir de la douleur ou du plaisir, il nous détermine le premier, et aussitôt que nous pouvons faire usage de nos sens. Le principe spirituel se manifeste plus tard, il se développe, il se perfectionne au moyen de l’éducation. C’est par la communication des pensées d’autrui que l’enfant en acquiert et devient lui-même pensant et raisonnable ; et sans cette communication, il ne serait que stupide ou fantasque, selon le degré d’inaction ou d’activité de son sens intérieur matériel.\par
« Il est aisé, en rentrant en soi-même, de reconnaître l’existence de ces deux principes : il y a des instants dans la vie, il y a même des heures, des jours, des saisons, où nous pouvons juger, non seulement de la certitude de leur existence, mais aussi de leur contrariété d’action. Je veux parler de ces temps d’ennui, d’indolence, de dégoût, où nous ne pouvons nous déterminer à rien, où nous voulons ce que nous ne faisons pas, et faisons ce que nous ne voulons pas… Si nous nous observons dans cet état, notre {\itshape moi} nous paraîtra divisé en deux personnes, dont la première, qui représente la faculté raisonnable, blâme ce que fait la seconde, mais n’est pas assez forte pour s’y opposer efficacement et la vaincre ; au contraire, cette dernière étant formée de toutes les illusions de nos sens, et de notre imagination, elle contraint, elle enchaîne, et souvent accable la première, et nous fait agir contre ce que nous pensons, ou nous force à l’inaction, quoique nous ayons la volonté d’agir.\par
« … Lorsque l’un des deux principes agit sans opposition de la part de l’autre, nous ne sentons aucune contrariété intérieure ; notre {\itshape moi} nous paraît simple, parce que nous n’éprouvons qu’une impulsion simple, et c’est dans cette unité d’action que consiste notre bonheur ; car pour peu que, par des réflexions, nous, venions à blâmer nos plaisirs, ou que, par la violence de nos passions, nous cherchions à haïr la raison, nous cessons dès lors d’être heureux, nous perdons l’unité de notre existence, en quoi consiste notre tranquillité ; la contrariété intérieure se renouvelle, les deux personnes se représentent en opposition, et les deux principes se font sentir, et se manifestent par les doutes, les inquiétudes et les remords.\par
« De là, on peut conclure que le plus malheureux de tous les états, est celui où ces deux puissances souveraines de la nature de l’homme, sont toutes deux en grand mouvement, mais en mouvement égal et qui fait équilibre ; c’est là le point de l’ennui le plus profond et de cet horrible dégoût de soi-même, qui ne nous laisse d’autre désir que celui de cesser d’être...\par
« Toutes les situations voisines de cette situation, tous les états qui approchent de cet état d’équilibre, et dans lesquels les deux principes opposés ont peine à se surmonter, et agissent en même temps, et avec des forces presque égales, sont des temps de trouble, d’irrésolution et de malheur ; le corps même vient à souffrir de ce désordre et de ces combats intérieurs, il languit dans l’accablement, ou se consume par l’agitation que cet état produit...\par
« C’est donc parce que la nature de l’homme est composée de deux principes opposés, qu’il a tant de peine à se concilier avec lui-même ; c’est de là que viennent son inconstance, son irrésolution, ses ennuis.\par
« Les animaux au contraire dont la nature est simple et purement matérielle, ne ressentent ni combats intérieurs, ni opposition, ni trouble ; ils n’ont ni nos regrets, ni nos remords, ni nos espérances, ni nos craintes.\par
« Séparons de nous tout ce qui appartient à l’âme, ôtons-nous l’entendement, l’esprit et la mémoire ; ce qui nous restera sera la partie matérielle par laquelle nous sommes animaux : nous aurons encore des besoins, des sensations, des appétits ; nous aurons de la douleur et du plaisir, nous aurons même des passions ; car une passion est-elle autre chose qu’une sensation plus forte que les autres, et qui se renouvelle à tout instant ? Or nos sensations pourront se renouveler dans notre sens intérieur matériel : nous aurons donc toutes les passions, du moins toutes les passions aveugles que l’âme, ce principe de la connaissance, ne peut ni produire ni fomenter. »\par
Telle est la doctrine de Buffon sur la nature mixte de l’homme. Ce célèbre écrivain, en reconnaissant deux êtres dans cette nature, détermine leur essence sur laquelle il n’avait ni ne pouvait avoir aucune notion positive : il nous dit que l’un est spirituel et l’autre matériel. Il conserve le nom d’âme à l’un de ces principes, et par là il rentre dans l’opinion commune sur l’union de l’âme et du corps, attribuant à celui-ci, la sensibilité physique, les appétits et les passions qui en résultent, donnant à l’autre l’intelligence, la raison et les autres facultés ; de manière que c’est à la fois, admettre et ne pas admettre deux êtres dans l’homme ; car le corps eût-il les propriétés qu’on lui attribue, comme il est l’homme extérieur dans lequel l’âme est, pour ainsi dire, renfermée, celle-ci constituant l’homme intérieur, l’homme intérieur serait toujours un être simple : pour qu’il soit double, il faut qu’il y ait deux âmes ou deux principes auxquels le corps serve d’asile.\par
C’est bien là ce que Buffon a prétendu ; mais la manière dont il s’est exprimé, par ménagement sans doute, pour le temps où il a écrit, a mis du louche dans sa pensée, et l’a rendue vague et indéterminée. En donnant à l’un de ces principes la dénomination de sens intérieur matériel, ce n’est plus un être différent du corps, c’est le corps même, puisque les sens matériels appartiennent à celui-ci. Cependant il a voulu désigner autre chose ; il a voulu, qu’outre l’âme humaine, on admît dans l’homme, l’âme animale qui, de quelque nature qu’on la suppose, esprit ou matière, est autre chose que le corps proprement dit, dans les animaux, et par conséquent, doit être autre chose que le corps dans l’homme.\par
En rapportant à ces deux êtres intérieurs nos contradictions, nos irrésolutions, nos inconséquences, ce savant naturaliste n’a pas vu qu’elles provenaient du caractère différent de ces deux êtres, de la liberté de l’un et de la non-liberté de l’autre. Il a dit avec raison, que cette biduité intérieure n’existant point dans les animaux, ils n’étaient susceptibles ni de ces contradictions, ni de ces combats intérieurs qui ont lieu dans l’homme ; mais il devait ajouter, que la simplicité de leur nature les constitue dans une solitude perpétuelle, tandis que la nature composée de l’homme, fait qu’il n’est pas seul quand il est avec lui-même.\par
Ceci est un mal sans doute, lorsque les deux êtres intérieurs sont en opposition, l’homme souffre de leur mésintelligence ; mais cette mésintelligence n’est qu’accidentelle, c’est-à-dire qu’elle n’est pas dans leur nature respective ; car, c’est par eux que l’homme est susceptible de s’aimer lui-même : il doit à sa biduité intérieure, d’avoir en soi le principe actif et passif de cette passion, de se sentir à la fois, comme nous l’avons déjà dit, l’objet aimant et l’objet aimé. La nature simple des animaux, n’admettant pas ce double rapport, il n’y a point pour eux d’amour de soi, ni probablement de conscience d’eux-mêmes.\par
C’est pour n’avoir pas rapporté à l’amour de soi, le parfait accord et l’union des deux principes, que Buffon a voulu que l’unité intérieure de l’homme, de laquelle dépend son bonheur, résultât de la lutte de ces deux principes et de la victoire remportée par l’un d’eux, n’importe lequel, car il semble ne redouter que leur équilibre ou leur rivalité de puissance. Notre théorie est à cet égard absolument différente de la sienne ; c’est dans l’amour de soi que nous trouvons le moyen naturel de subordination de l’un des principes à l’autre, et le moi libre étant proprement le moi humain ou le moi de l’homme, doit être l’objet aimé. La subordination de l’autre principe doit donc s’effectuer à son profit ; elle n’est pas arbitraire, c’est-à-dire qu’il n’est pas indifférent lequel des deux principes soit le dominateur. C’est un point auquel nous croyons avoir donné assez de développement, pour qu’il ne soit pas nécessaire d’y revenir ici.\par
Au reste, malgré ces inexactitudes et ce défaut de précision dans les termes, le {\itshape Discours sur la nature des animaux} n’en est pas moins, à notre avis, l’ouvrage dans lequel la différence entre l’homme et la bête a été le mieux saisie et le mieux caractérisée. Nous osons dire qu’il ne méritait point la critique parfois dure, pour ne pas dire indécente, que s’en est permis Condillac dans son {\itshape Traité des animaux}.\par
Le sujet de ce dernier écrit est le même que celui du discours de Buffon. Condillac, après avoir relevé les erreurs qu’il croit apercevoir dans son rival, présente un autre système, d’où il résulte que les animaux sentent comme nous, ont des idées comme nous, pensent comme nous, raisonnent comme nous, apprennent comme nous, mais dans de moindres degrés, parce que leurs besoins sont plus bornés que les nôtres.\par
Dans Buffon, la différence de l’animal à l’homme est celle du simple ou composé ; dans Condillac, ils ne diffèrent que du plus au moins.\par
Cependant, sans le vouloir, Condillac rentre dans l’opinion de Buffon sur la nature composée de l’homme ; car il introduit aussi deux {\itshape moi} dans son système. Cette idée est si naturelle, qu’elle fait violence à ceux même qui la rejettent. Après avoir combattu l’homme intérieur double de Buffon, Condillac n’en est pas moins forcé de supposer en nous un double moi : il appelle l’un le moi d’habitude, et l’autre le moi de réflexion.\par
« C’est le premier, dit-il, qui touche, qui voit ; c’est lui qui dirige toutes les facultés animales. Son objet est de conduire le corps, de le garantir de tout accident, et de veiller continuellement à sa conservation.\par
« Le second lui abandonnant tous ces détails, se porte à d’autres objets.\par
Il s’occupe du soin d’ajouter à notre bonheur. Ses succès multiplient ses désirs ; ses méprises les renouvellent avec plus de force : les obstacles sont autant d’aiguillons ; la curiosité le meut sans cesse ; l’industrie fait son caractère : celui-là est tenu en action par les objets, dont les impressions reproduisent dans l’âme les idées, les besoins et les désirs qui déterminent dans le corps les mouvements correspondants, nécessaires à la conservation de l’animal : celui-ci est excité par toutes les choses qui, en nous donnant de la curiosité, nous portent à multiplier nos besoins.\par
« Mais quoiqu’ils tendent à un but particulier, ils agissent souvent ensemble. Lorsqu’un géomètre, par exemple, est fort occupé de la solution d’un problème, les objets continuent encore d’agir sur ses sens. Le moi d’habitude obéit donc à leurs impressions. C’est lui qui traverse Paris, qui évite les embarras, tandis que le moi de réflexion est tout entier à la solution qu’il cherche. »\par
Ce moi d’habitude, constitue selon Condillac, l’instinct des animaux ; mais il prétend que ce moi dérive de la réflexion, c’est-à-dire, qu’on ne fait par habitude que la chose à laquelle on a d’abord réfléchi, de manière que par cette interprétation, son moi d’habitude rentre dans le moi de réflexion, et le moi de réflexion dans le moi d’habitude. La distinction qu’il a d’abord établie entre eux, disparaît, et fait place à l’identité : aussi, dans sa lettre à un critique, qui n’avait vu dans ces deux moi qu’une faible imitation de l’homme double de Buffon, Condillac répond : « Cela est vrai, si vous vous arrêtez aux mots ; mais si vous allez jusqu’aux idées, vous trouverez deux pensées bien différentes. »\par
En effet, si l’on admet que ce que Condillac appelle moi d’habitude, dérive du moi de réflexion, ce ne sont plus au fond, deux moi différents : ce n’est qu’un seul et même moi ; mais cet écrivain ne s’est-il pas trompé dans cette assertion ? N’a-t-il pas confondu ce qu’il ne fallait pas confondre ? Il est très vrai qu’il nous arrive de faire, par habitude, ce que nous avons d’abord fait avec réflexion ; mais ce principe est-il applicable aux opérations naturelles auxquelles Condillac en fait l’application ? Il faudrait donc qu’avant de marcher par habitude, l’enfant eût réfléchi qu’il allait marcher ; qu’avant de voir par habitude, il eût vu par réflexion ; qu’il eût touché par réflexion avant de toucher par habitude.\par
Sans doute l’enfant apprend toutes ces choses, il ne les sait point en arrivant au monde ; mais il n’en est pas de cet apprentissage comme de celui d’un art ou d’un métier quelconque, c’est-à-dire qu’il les apprend, parce qu’il les fait et non parce qu’il veut les faire : elles ne sont donc, à proprement parler, ni le produit de la réflexion, ni le produit de l’habitude ; mais d’un instinct qui n’est ni l’une ni l’autre : la réflexion et l’habitude les modifient ensuite dans l’homme, sans en altérer ni changer le principe qui reste toujours ce qu’il est dans la nature animale.\par
Le moi de réflexion est donc indépendant de ce que Condillac appelle le moi d’habitude ; et ce qu’il appelle moi d’habitude, est également indépendant du moi de réflexion. Les animaux n’ont pas besoin, comme il le croit, de réfléchir pour acquérir celui-là, puisqu’il n’est pas le produit de la réflexion chez nous. Cet instinct les conserve et détermine leurs actions sans qu’ils y pensent, comme il nous conserve et détermine les nôtres, sans que nous y pensions ; mais outre ce moi animal, nous avons le moi de réflexion qu’ils n’ont pas et que Condillac leur accorde, parce qu’en rapportant le premier à l’habitude, il le fait dériver de l’autre.\par
« C’est en réfléchissant, dit-il, que les bêtes l’acquièrent : mais comme elles ont peu de besoins, le temps arrive bientôt où elles ont fait tout ce que la réflexion peut leur apprendre. Il ne leur reste plus qu’à répéter tous les jours les mêmes choses : elles doivent donc n’avoir enfin que des habitudes ; elles doivent être bornées à l’instinct. »\par
Voilà certes, une singulière progression. La faculté de réfléchir ou le moi de réflexion, ne se développe dans l’homme, qu’après cet autre moi, improprement appelé par Condillac, moi d’habitude : plus nous avançons dans la carrière de la vie, plus nous réfléchissons ; mais l’animal, s’il fallait en croire ce système, réfléchirait d’abord et ne réfléchirait plus ensuite ; cette faculté périrait en lui ou ne lui serait plus d’aucun usage. Est-ce là juger par analogie ?\par
Si l’animal avait en effet cette faculté de réfléchir, n’est-il pas probable qu’elle acquerrait en lui comme en nous plus d’énergie et plus d’intensité avec l’âge ; qu’elle le rendrait capable de se réformer, de se perfectionner, de faire ce qu’il n’aurait pas déjà fait, ou de mieux faire ce qu’il faisait auparavant ? Si l’on n’observe rien de tout cela dans la conduite des animaux, n’est-il pas naturel d’en conclure que cette faculté leur manque absolument ; qu’en aucun temps ils ne font rien par réflexion, ni même par habitude, dans le sens que nous attachons à ce mot ; car celle-ci suppose une pensée ou une réflexion primitive qui l’a déterminée ; que par conséquent ils vont, viennent, agissent et exécutent tout ce qu’ils font par un instinct qui les conduit à leur insu, comme celui qui nous fait marcher, éviter un embarras, traverser Paris, choisir les rues par où nous devons passer, quoique nous ayons notre attention ailleurs, et que nous ne pensions nullement à toutes ces choses.\par
Il semble que Condillac devait d’autant moins accorder la réflexion aux animaux, qu’il leur refuse la liberté ; et l’on ne voit pas que des êtres non libres, aient aucun besoin de réfléchir. L’homme compare ses idées ; il agit, pour ainsi dire, en lui-même, avant d’agir au-dehors : il fait une espèce de répétition intérieure de ce qu’il veut faire extérieurement, parce que s’il n’est pas content de cet essai préliminaire, il peut en demeurer là, ou faire à ce même essai, tous les changements qu’il juge convenables ; mais l’animal en qui cette liberté n’existe point, et dont la conduite est, de l’aveu même de Condillac, {\itshape commandée par les circonstances}, n’a nul besoin de cet essai intérieur d’action, de ce préliminaire que la liberté motive dans l’homme et qui, dans l’animal, serait absolument sans motif.\par
Les bêtes ne se représentent donc pas d’avance ce qu’elles vont faire, pas plus que nous ne nous représentons ce que nous faisons sans réflexion, et qui se trouve fait et bien fait, sans que nous y ayons nullement songé. Cette représentation intérieure n’a lieu en nous, qu’à raison de la liberté inhérente à notre nature, de modifier l’action représentée, de lui donner ou de ne pas lui donner l’existence positive ; c’est ce qui rend nécessaire le dessein qui en est tracé d’avance dans notre esprit et l’examen de sa convenance. Cette image et cet examen antérieurs seraient sans but et sans motif, si nous n’étions pas libres, si le système de nos actions était déterminé et commandé par les circonstances, comme celui des autres animaux.\par
On peut donc être assuré qu’ils n’ont ni l’image anticipée des actions, ni le sentiment de leur convenance, parce que tout cela serait en eux un effet sans cause, et que tout, dans la nature, a sa raison suffisante.\par
Le langage nous serait tout aussi inutile que la représentation anticipée de nos actions, si nous étions nécessités comme les autres créatures ; car, cette communication d’idées qui a lieu entre nous, par la parole et par l’écriture, est-elle autre chose que la communication de ces images intérieures des actions, de ces desseins figurés d’après lesquels nous les effectuons. Or, cette communication serait évidemment superflue et sans effet, si liés et astreints à un système déterminé de conduite, nous n’avions pas la faculté de nous en écarter ou de le modifier, car ce serait bien en vain que ces divers tableaux se succéderaient dans notre esprit, nous n’en retirerions aucune espèce d’avantage ; il n’en résulterait pour nous, que la triste conviction de notre impuissante nécessité, tandis que, par la liberté de notre nature, elle y produit l’émulation et le désir d’un perfectionnement toujours possible.\par
Ce n’est donc ni de la différence de leur conformation, ni de la moindre somme de leurs besoins, que provient le défaut de réflexion et le non-usage de la parole chez les autres animaux, mais de leur non-liberté. La nature a été sage et généreuse en leur refusant ces dons, comme elle a été sage et généreuse en les accordant à l’homme ; car, on peut mettre en fait, que le don de la parole et de la communication des idées qui contribue tant à nous faire aimer l’existence, leur rendrait la leur insupportable, s’ils ne recevaient pas en même temps la liberté par laquelle il nous est avantageux, et dont la privation le leur rendrait inutile et onéreux jusqu’au désespoir.\par
Pour nous convaincre de cette vérité, nous n’avons qu’à nous rappeler l’effet que produit sur nous une passion sans espérance, une situation que nous voudrions changer et dans laquelle nous sommes condamnés à rester. Telle serait la position des animaux, en qui le don de la parole et de la communication des idées exciterait l’émulation, produirait le désir de faire ce qu’ils ne font pas, de mettre du choix dans leurs actions, de varier leurs habitudes, de changer leur condition ou d’améliorer celle dans laquelle ils vivent, et qui cependant seraient condamnés à marcher toujours sur la même ligne, à ne faire jamais que les mêmes choses ; assaillis d’une foule de vœux toujours impuissants, obsédés par des images successives d’actions que la nécessité de leur nature ne leur permettrait jamais de réaliser, peut-on se faire l’idée d’une situation plus déplorable, plus désespérante, et ne faudrait-il pas être le plus cruel ennemi des bêtes, pour leur souhaiter un bienfait qui les soumettrait à un état dont les théologiens ont fait le supplice des réprouvés.\par
La nature, plus sage et plus indulgente qu’on ne l’imagine, n’a pas voulu que l’existence fit le malheur des êtres qu’elle en a gratifiés. Elle a mis de l’économie et de la proportion entre les moyens qu’elle leur a départis et le but qu’il leur est donné d’atteindre. Quand nous croyons la surprendre en défaut, nous devons être persuadés que c’est notre ignorance qui la calomnie. Si elle avait accordé aux animaux la faculté de penser, de réfléchir, de raisonner, elle leur aurait accordé celle de se communiquer leurs pensées, leurs réflexions, leurs raisonnements, et si elle leur avait accordé tout cela, elle leur aurait accordé la liberté du choix des actions, d’où résulterait la possibilité spontanée de modifier leur état, de changer leurs habitudes, d’améliorer leur condition. De ce que nous voyons que cette liberté leur manque, nous devons en conclure qu’elles ne pensent point, qu’elles ne réfléchissent pas, qu’elles ne raisonnent pas, que par conséquent, elles n’ont rien à se communiquer ; que ce qu’elles font, elles le font parce que la sensibilité seule les détermine à le faire.\par
Buffon a donc eu raison de dire que {\itshape le principe du sentiment n’est pas celui de la connaissance}, car, les bêtes sentent et ne connaissent pas : il n’y a pas seulement entre elles et nous, la différence du plus au moins ; mais la différence spécifique du simple au composé : nous avons comme elles, le moi sentant, ou le moi physique, et de plus, le moi pensant, réfléchissant, voulant indépendamment des sensations, en un mot, le moi moral ou le moi libre ; car, toutes ces facultés sont liées, naissent les unes des autres, et sont, pour ainsi dire, indivisibles\footnote{Nous ne croyons pas avoir besoin d’avertir que ceci n’a rien de contraire à ce que nous avons dit dans le premier chapitre, où nous avons admis deux principes de sensibilité.}. L’être qui pense, réfléchit, l’être qui réfléchit, examine la convenance des actions ; l’examen de cette convenance suppose la puissance de les faire ou de s’en abstenir, par conséquent, la liberté morale. Admettre l’absence de l’une de ces facultés, c’est admettre l’absence des autres ; ainsi Condillac, en refusant aux animaux la liberté qu’il est impossible de leur accorder dans aucun système, est forcé, par cela même, de leur refuser la pensée, la réflexion et le raisonnement ; car, le choix ou la liberté des actions, est en nous le but et la fin de toutes ces opérations intérieures ; et il est clair que cette fin n’existant pas pour les autres animaux, ils ne doivent pas avoir non plus les moyens de l’obtenir. S’ils avaient ces moyens, ils auraient la fin pour laquelle ils sont institués, et s’ils avaient la fin, ils auraient aussi les moyens.
\chapterclose


\chapteropen
\chapter[{Chapitre XVI. Suite du chapitre précédent ; réponse à une objection ; explication de ce qui se passe en nous, par ce qui a lieu hors de nous.}]{Chapitre XVI. Suite du chapitre précédent ; réponse à une objection ; explication de ce qui se passe en nous, par ce qui a lieu hors de nous.}\renewcommand{\leftmark}{Chapitre XVI. Suite du chapitre précédent ; réponse à une objection ; explication de ce qui se passe en nous, par ce qui a lieu hors de nous.}


\chaptercont
\noindent S’il est vrai, dira-t-on, que le principe du sentiment ne soit pas celui de la connaissance, on ne peut plus admettre que toutes les idées viennent des sens ou des sensations ; car la connaissance tient aux idées, et le principe qui connaît n’étant pas celui qui sent, il semble que, dans ce système, les idées sont indépendantes des sensations.\par
Avant de répondre à cette objection, il est bon de rappeler les différentes opinions sur l’origine des idées : les uns les font dériver de je ne sais quelles traces imprimées sur le cerveau : ils couvrent, pour ainsi dire, cet organe de ces sortes de stigmates. Leur empreinte conservée ou renouvelée, constitue le souvenir, et l’oubli a lieu quand elle s’efface.\par
D’autres ont substantialisé les idées ; ils en ont fait des êtres positifs, et ont établi dans le cerveau des cases pour les recevoir. L’âme n’a été occupée qu’à les arranger dans ces cases, et c’est de l’ordre qu’elle a mis dans cet arrangement qu’est résulté pour elle la facilité ou la difficulté de les retrouver au besoin, et de s’en servir de la manière la plus convenable.\par
Une troisième opinion, qui est encore celle de la plupart des philosophes, c’est celle des ébranlements. Le cerveau, dans ce système, est une espèce de clavecin dont les fibres sont les touches. Ces fibres, remuées par les objets extérieurs, produisent une idée, comme les touches d’un clavecin sur lesquelles on applique les doigts, produisent un son.\par
Condillac a adopté cette comparaison sans admettre la réalité de ces prétendus ébranlements des fibres, à l’aide desquels Bonnet a cru de très bonne foi expliquer toute la mécanique des sensations et des idées. Le premier de ces écrivains, pour mettre plus de simplicité dans les opérations de l’entendement, l’a réduit à la faculté de sentir : l’attention, la réflexion, le raisonnement, le désir, ne sont, selon lui, que la sensation transformée en autant de modes différents.\par
J’ignore si ce célèbre métaphysicien s’est bien entendu lui-même ; mais j’avoue qu’il m’est impossible de comprendre sa transformation de la sensation, ou sa sensation transformée. Je ne conçois rien à cette prétendue métamorphose de la sensation ; il me semble même qu’elle implique avec les effets qu’on en déduit ; car l’idée est en nous la perception des sensations ; or, la perception d’une chose n’est certainement pas la chose perçue.\par
Le système de nos rapports intérieurs paraît, en quelque sorte, moulé sur celui que nous observons à l’extérieur. L’être libre, intelligent et moral, ou le principe qui connaît, voit en nous les sensations ou les modifications de l’être sentant, comme notre œil voit au-dehors les actions et les objets sur lesquels il porte ses regards : or, nous savons que, pour voir les objets, il est de rigueur qu’on ne soit pas dans ces objets et surtout qu’on ne soit pas ces objets même ; donc l’être qui perçoit en nous les sensations, n’est pas l’être qui les éprouve, et par conséquent le principe du sentiment n’est pas celui de la connaissance.\par
Il n’en est pourtant pas moins vrai que toute connaissance dérive du sentiment ; car les perceptions de l’être intelligent ne pouvant s’étendre que sur les sensations ou sur les modifications de l’être sentant ; si celui-ci n’éprouvait ni sensation ni modification, l’autre n’aurait rien à percevoir : ce serait en quelque sorte pour lui un état de ténèbres, dans lequel il ne verrait plus rien, ne distinguerait plus rien ; d’où il suit que, n’ayant plus de perceptions, il n’aurait plus d’idées, et s’il n’y avait plus d’idées, il n’y aurait plus de connaissance.\par
Pour nous faire une image sensible de ce qui se passe en nous par la réunion du principe intelligent au principe sentant, nous n’avons qu’à nous supposer rencontrant une personne que nous n’avons jamais vue : les circonstances nous forcent à rester avec elle, à en faire notre société habituelle. D’abord, nous ne connaissons pas cette personne ; mais elle est toujours sous nos yeux : nous sommes témoins des impressions qu’elle reçoit, de l’effet qu’elles produisent sur elle : ces perceptions répétées nous donnent la connaissance de ce qui la flatte, de ce qui lui déplaît : nous jugeons d’après cela si ce qui la flatte est bien, si ce qui lui déplaît est mal : nous voilà initiés dans son caractère : le passé nous donne à son égard la prévoyance de l’avenir ; parce que nous lui avons vu faire ou éprouver dans certaines circonstances, nous avons la connaissance anticipée de ce qu’elle éprouvera, de ce qu’elle fera dans des circonstances semblables ou analogues. En conséquence nous prévenons ou provoquons ces circonstances, et nous sommes d’autant moins sujets à nous tromper à cet égard, que nous l’avons mieux observée, mieux étudiée, et que nous la connaissons mieux. Eh bien ! la connaissance de nous-mêmes tient aux mêmes rapports, aux mêmes procédés. C’est également une connaissance d’observation et d’expérience. Si nous faisons si peu de progrès dans ce genre d’instruction ; c’est qu’en général nous nous plaçons trop près de nous, nous ne savons pas nous mettre à la distance où il faut être de soi pour se bien regarder, et pour se voir comme si on était un autre.\par
La personne dont nous avons supposé la rencontre, c’est notre être sentant, ou physique avec lequel la nature unit notre être intelligent et moral, qui d’abord n’a nulle connaissance de cet être auquel il est réuni dans notre constitution intérieure ; mais il perçoit ses sensations et ses modifications successives, de la manière à peu près dont nous voyons les actions et les impressions de ceux avec lesquels nous vivons ; et c’est ainsi qu’il parvient à connaître l’être sentant, c’est-à-dire, à savoir que, quand il reçoit telle impression, il est modifié de telle manière, et que quand il est modifié de telle manière ; il veut ou ne veut pas telle ou telle chose.\par
C’est d’après cette expérience acquise, que l’être intelligent modifie à son tour l’être sentant par des impressions intérieures, qui secondent ou combattent celles qui viennent du dehors : ceci peut encore s’expliquer par la manière dont nous influons sur les personnes avec lesquelles nous vivons : « Prenez garde, leur disons-nous, n’allez pas dans tel endroit, vous savez ce qui vous arriva la dernière fois que vous y fûtes ; ne cédez point à l’attrait de cette femme, elle vous perdra ; si vous continuez à voir telle société ou tel individu, je me brouille avec vous. »\par
Il est clair que la personne à qui nous donnons ces avis est, pour ainsi dire, entre deux mobiles qui la poussent ou la tirent en sens contraire ; elle reste en équilibre ou cède à l’un des deux. Selon le degré d’affection qu’elle a pour nous, nous l’emportons ou nous ne l’emportons pas sur la force attractive des autres impressions ; et, si celles-ci l’entraînent, nous lui reprochons plus ou moins amèrement sa conduite, selon le plus ou le moins d’importance que nous attachions à ce qu’elle en tînt une autre, et selon les conséquences qui résultent de celle qu’elle a tenue.\par
Je demande si ce n’est point là ce qui se passe en nous-mêmes, et si l’analogie ne doit pas nous porter à en conclure qu’il existe en nous deux êtres, dont l’un observe, connaît et avise l’autre. C’est ce moi observateur et moniteur de l’autre moi, qui constitue en nous ce spectateur impartial dont parle Smith, lequel intérieurement approuve ou désapprouve notre conduite, et déclare qu’elle est bien ou mal, selon qu’elle s’éloigne ou se rapproche des avis qu’il nous a donnés.\par
Au reste, cet être n’est impartial que parce qu’il est libre, et il n’est libre que parce qu’il n’est point le théâtre des sensations : elles ne se passent point en lui, il n’en est que le spectateur : il les voit dans l’être sentant, comme nous voyons les marques extérieures de plaisir ou de douleur sur la physionomie des autres. Il en a la perception et non l’impression, l’idée et non le sentiment : la connaissance qu’il en acquiert est toute intellectuelle : il a donc toute la liberté et toute l’impartialité nécessaires pour examiner la convenance des actions que ces sensations provoquent, et par conséquent pour donner ou pour refuser son consentement à ces mêmes actions. C’est, selon les circonstances, l’objet d’une délibération tranquille, ou d’un très violent débat entre lui et l’être sentant. Cela dépend du plus ou du moins de passion de celui-ci, c’est-à-dire, du plus ou du moins d’énergie des impressions reçues ; il faut que celles qu’il reçoit intérieurement de l’être observateur, balancent ou neutralisent les autres ; car, par lui-même, il ne peut que céder : voilà pourquoi nous ne faisons rien sans motif intérieur ou extérieur, et pourquoi cela n’empêche pas que nous ne soyons libres ; car notre liberté ne tient point à l’être qui se détermine nécessairement d’après les motifs reçus, et qui n’est en quelque sorte, que la partie exécutive de nous-mêmes ; mais à celui qui conçoit ces motifs qui, sur la convenance des choses, les enfante, les fournit à l’être nécessité, et qui, par conséquent, est supérieur aux motifs dont il est le principe.\par
Ceci répond aux différentes questions qu’on a fait sur la liberté d’indifférence, la liberté de spontanéité, et je ne sais quelles autres semblables questions dont on a surchargé la philosophie pour prouver la liberté ou la non-liberté de l’homme. Il est facile de voir combien toutes ces questions sont oiseuses et, pour ainsi dire, étrangères à la question : c’est de notre nature mixte ou composée que dépend la solution de cette difficulté, et nous osons dire qu’on ne la résoudra jamais, tant qu’on n’admettra pas la biduité intérieure de l’homme.\par
C’est parce que les animaux ne sont qu’un dans leur intérieur, qu’ils ne sont pas libres ; c’est parce qu’ils n’ont point en eux d’être observateur d’eux-mêmes, qu’ils ne peuvent pas se connaître. Leur être perçoit les objets extérieurs ; mais rien en eux ne le perçoit lui-même.\par
Voilà pourquoi ils s’ignorent, et ne peuvent jamais parvenir à se faire aucune notion de ce qu’ils sont ; leur caractère leur est absolument inconnu. Ils ne vivent, pour ainsi dire, qu’en surface, et n’ont point, à proprement parler, de vie intérieure. Ils ne trouvent point en eux d’être avec lequel ils puissent s’entretenir, et par conséquent délibérer sur ce qu’ils doivent faire ou ne pas faire ; car, pour délibérer, il faut être deux. Il s’ensuit qu’ils n’ont aucune idée de la convenance des choses ; qu’ils font tout par impulsion, et rien par délibération.\par
L’homme reçoit les mêmes impressions qu’eux, et céderait avec tout aussi peu de retenue à ce qu’elles exigent de lui, s’il n’avait pas en lui-même un être indépendant de leur influence, qui les considère et les voit dans l’être sur lequel elles développent leur énergie : comme elles n’arrivent pas jusqu’à cet être indépendant, le calme de sa position lui permet de les examiner de sang-froid, d’en prévoir les effets, d’en calculer les conséquences éloignées, et par conséquent de dire à l’être passionné : « Faites ou ne faites pas ; cédez à ces impressions ou résistez, car vous vous trouveriez mal de leur avoir obéi. »\par
Quelquefois ces avis prévalent sur la force des sensations, et il est de notre intérêt qu’ils l’emportent toujours ; car, pourquoi la nature aurait-elle mis en nous cet être observateur et indépendant des impressions sensuelles, si elle n’avait pas voulu par là nous rendre supérieurs à ces impressions, c’est-à-dire, nous mettre en puissance d’en modifier les effets, et de n’en tenir compte qu’autant qu’il nous paraîtrait convenable de le faire. L’intention de la nature a été évidemment que nous eussions cette puissance ; et puisqu’elle a eu cette intention, elle nous a donné des moyens suffisants pour la remplir ; c’est donc notre faute si nous ne la remplissons pas, si nous nous laissons, dominer par nos sensations, et en un mot si nous ne sommes pas libres.\par
Les autres animaux ne tiennent pas seulement au monde sensible ; ils y sont renfermés ; ils en font partie comme les végétaux et les minéraux : ce monde n’est pas une chose différente d’eux-mêmes ; mais l’homme, à raison de sa nature mixte, y tient, et cependant il en est dehors : il y tient par l’être sentant, il en est dehors par l’être libre et intelligent. Voilà pourquoi il acquiert la connaissance intuitive et générale de ce monde, que les autres créatures ne peuvent point acquérir. Il conçoit les autres êtres, et les autres êtres ne le conçoivent pas. Pour contempler ainsi la nature universelle, pour la voir en quelque sorte devant nous, et en saisir, comme nous le faisons, l’ensemble et les détails, il faut que nous soyons hors de cette nature, sans quoi elle ne pourrait pas être l’objet de notre investigation.\par
Il y a donc en nous un être, pour ainsi dire, surnaturel, ou qui est hors des lois naturelles du monde sensible, et qui habite ce monde sans être soumis à l’ordre nécessité qui le règle et le dirige ; d’où il suit que cet être est libre par sa nature. Le but de son association à l’être sentant n’est pas de l’asservir, mais de l’introduire dans le monde sensible, comme dans le lieu où doit s’exercer sa liberté. {\itshape Être libre dans un ordre de choses nécessité}, voilà quelle est la destination de l’homme, c’est le problème qu’il est appelé à résoudre, et à la solution duquel doivent tendre tous ses moyens. C’est là ce qui constitue la perfectibilité humaine ; elle consiste dans la recherche de la solution de ce problème, car quand on l’aura trouvée, on aura trouvé la perfection.\par
Pourquoi étudions-nous l’homme ? pourquoi étudions-nous la nature ? quel est le but commun de toutes ces recherches ? N’est-ce pas, d’un côté, pour apprendre à modifier les lois de notre organisation, et, de l’autre, pour apprendre également à modifier les lois des êtres qui nous environnent ? Nous travaillons donc à nous rendre indépendants de ces lois ; nous les étudions pour les soumettre à nos déterminations, et pour nous affranchir des leurs. C’est notre liberté qui est le motif réel et ignoré de toutes ces recherches ; quoique nous ne rapportions pas toutes nos études à ce but, le but de tout ce que nous apprenons est d’apprendre à être libres, à nous affranchir de l’empire de la nécessité à laquelle tous les autres êtres sont asservis, et dont l’homme seul a le droit de secouer les entraves.\par
S’il est donc vrai que la liberté est le but commun de tous les hommes ; si c’est là le chemin de leur perfectionnement et de leur bonheur, il est de rigueur que toutes les institutions qui les éloignent de ce but généreux, les dégradent, les avilissent et les rendent malheureux ; il est de rigueur aussi que tous les vices, tous les préjugés, toutes les erreurs qui nous retiennent dans l’esclavage, soient tôt ou tard dissipés par l’action lente, mais non interrompue de la nature libérale et généreuse de l’homme.\par
Il ne faut donc pas désespérer de l’amélioration future de notre condition, ni croire que nous sommes parvenus au {\itshape nec plus ultra} de l’art de vivre ; pour être assuré qu’on fait l’emploi le plus convenable d’une chose, il faut qu’on la connaisse parfaitement ; rien n’est plus ordinaire que d’employer à contresens celles qu’on ne connaît pas.\par
Donnez un fusil à charger à celui qui n’a nulle idée de cette arme ni du moyen de s’en servir, il sera possible qu’il mette la balle ou le papier avant la poudre. Telle plante dont vous ne connaîtrez point les propriétés vous empoisonnera si vous vous avisez d’en faire usage, tandis qu’un autre la modifiera de manière à lui ôter sa qualité nuisible, et trouvera le secret d’en extraire un aliment ou un médicament salutaire.\par
Pourquoi l’ignorance de nous-mêmes ne nous exposerait-elle pas aux mêmes méprises, aux mêmes contresens dans notre conduite, qui n’est autre chose que l’emploi que nous faisons de notre individu ? Savons-nous assez ce que nous sommes, pour être persuadés que nous tirons de nous le meilleur parti possible ? Connaissons-nous la nature de l’homme, connaissons-nous celle de la société, au point d’affirmer qu’il n’y a rien de mieux à faire que ce qu’on a fait jusqu’ici ?\par
Malgré la certitude de nos principes actuels, je ne voudrais pas répondre que nous ne fussions encore dans l’ignorance à cet égard : faisons donc abstraction de toutes ; nos prétendues connaissances ; étudions l’homme, étudions la société : nous avons essayé de jeter quelque jour sur la nature mixte du premier : voyons si nous ne pourrons pas également pénétrer dans celle de l’autre ; c’est toujours nous occuper de l’homme ; car il nous semble qu’il ne servirait à rien de connaître la nature de l’homme, si l’on ne connaissait pas celle de la société ; comme il ne servirait à rien de connaître celle de la société, si l’on ne connaissait pas celle de l’homme.
\chapterclose


\chapteropen
\chapter[{Chapitre XVII. Nécessité de la société pour le développement de l’être moral.}]{Chapitre XVII. Nécessité de la société pour le développement de l’être moral.}\renewcommand{\leftmark}{Chapitre XVII. Nécessité de la société pour le développement de l’être moral.}


\chaptercont
\noindent « S’il était possible, dit Smith, qu’une créature humaine parvînt à la maturité de l’âge dans quelque lieu inhabité et sans aucune communication avec son espèce, elle n’aurait pas plus d’idée de la convenance ou de l’inconvenance de ses sentiments et de sa conduite, que de la beauté ou de la difformité de son visage. »\par
C’est que, dans une pareille position, l’homme ne recevrait pas les sensations appropriées au développement de son être moral, et que dès lors il n’y aurait que l’être physique de développé en lui. Les hommes étant les seuls êtres moraux qui puissent affecter nos sens, sont les seuls objets dont la présence puisse nous faire éprouver les sensations nécessaires au développement de l’être qui nous constitue ce qu’ils sont eux-mêmes ; si nous n’avions aucune communication avec eux, il ne se développerait point en nous, et ce non-développement équivaudrait à la privation ou à la négation absolue de cet être.\par
Il est clair que, dans cet état, nous n’aurions pas la perception ou l’idée de nos propres sensations, parce que l’être observateur de ces mêmes sensations n’étant pas développé en nous, n’en prendrait aucune connaissance ; dès lors, point de souvenir idéal ou intellectuel du passé, point de prévoyance idéale ou intellectuelle de l’avenir, point de réflexion ni de délibération intérieure ; en un mot, point de connaissance, car nous avons vu que tout cela résulte de la biduité intérieure de l’homme ; d’où il est facile de conclure que l’un des êtres qui constituent cette biduité n’étant pas développé, les effets qui en sont la conséquence ne pourraient pas avoir lieu.\par
Telle était sans doute la position de ce jeune homme dont parle Condillac, qui, à l’âge d’environ dix ans, fut trouvé dans des forêts de la Lithuanie, vivant parmi les ours : « Il ne donnait aucune marque de raison, marchait sur ses pieds et sur ses mains, n’avait aucun langage, et formait des sons qui ne ressemblaient en rien à ceux d’un homme. Il fut longtemps sans pouvoir proférer quelques paroles, encore le fit-il d’une manière bien barbare. Aussitôt qu’il put parler, on l’interrogea sur son premier état ; mais il ne s’en souvint non plus que nous nous souvenons de ce qui nous est arrivé au berceau. »\par
Condillac essaie d’expliquer cette particularité, en disant que cet enfant, dans son premier état, n’étant occupé que du soin de sa conservation, bornait toutes ses réflexions à ce seul objet ; qu’il était donc naturel que les idées qu’il acquit ensuite dans la société, lui fissent totalement oublier celles-là.\par
Cependant, il n’est pas vraisemblable que si le principe de la connaissance avait été développé dans cet enfant, il ne se fût pas trouvé dans sa première position, des circonstances assez frappantes pour en déterminer le souvenir. La seule différence de son existence passée à son existence présente, devait, en supposant l’intelligence développée dans celle-là, former un contraste assez remarquable pour que celle-ci ne pût pas l’effacer. Comment ne se serait-il pas souvenu de ses premiers compagnons, les ours, dont la forme différait absolument de celle des êtres à deux pieds et sans fourrure avec lesquels il vivait actuellement ? N’est-il pas d’expérience que nous gardons toute la vie le souvenir des lieux dans lesquels s’est écoulée notre enfance, et que les idées des objets que nous avons vus à cette époque sont moins susceptibles de s’effacer, que celles qui nous viennent dans le moyen âge.\par
Il faut donc bien admettre, pour expliquer l’ignorance absolue des premières années de ce jeune homme, qu’il n’y avait que l’être sentant de développé en lui, et que, par conséquent, il ne pouvait pas avoir la connaissance d’un état dans lequel il était sans connaissance, et où il ne vivait que de sensibilité.\par
Condillac, après avoir rapporté ce fait dans son {\itshape Traité des sensations}, le cite encore dans {\itshape l’Essai sur l’origine des connaissances humaines}, et, dans ce dernier ouvrage, il attribue au défaut de signes pour fixer les idées, l’oubli de celles que le jeune homme avait eues dans les bois ; mais en admettant que ce défaut de signes ne lui eût effectivement pas permis de lier ses idées et de les étiqueter en quelque sorte, pour avoir plus de facilité à les retrouver, tout ce qu’on pourrait en conclure, c’est qu’il n’aurait pas pu faire le journal de sa vie forestière ; mais il ne s’ensuivrait pas qu’il dût en perdre totalement le souvenir, il aurait toujours eu en bloc l’image de cette condition qu’il ne lui aurait pas été possible de détailler. S’il ne s’en souvint pas plus que nous ne nous souvenons de ce qui nous est arrivé au berceau, c’est qu’il fallait qu’il y eût en lui le même défaut de connaissance ; il n’y avait point d’oubli de sa part, car on n’oublie point ce qu’on n’a pas connu.\par
L’explication la plus naturelle et la plus probable de ce phénomène, est donc que l’être qui perçoit en nous les sensations, n’étant pas développé dans l’individu dont il s’agit, il n’avait pas la perception de ses propres sensations, et que, n’ayant pas cette perception, il n’avait point d’idées. Ce ne fut que lorsque la présence de ses semblables eût développé en lui le principe observateur de l’être sentant, qu’il commença de se voir intérieurement, et qu’il fit, pour ainsi dire, sa première connaissance avec lui-même. Là commencèrent aussi ses idées. Ce fut, en quelque sorte, l’époque de sa génération intellectuelle et morale, d’où il suit que les temps antérieurs à cette époque devaient lui être absolument inconnus. S’il avait continué de vivre dans les bois, loin de toute société humaine, sa vie entière se serait écoulée dans cette ignorance absolue de soi, il aurait pu mourir dans la plus extrême vieillesse, qu’il n’aurait pas plus su que le premier jour, qu’il vivait ni qu’il avait vécu.\par
Ceci démontre la vérité de ce que nous avons dit relativement aux animaux ; car, il est certain qu’ils sont toute leur vie, ce qu’était cet enfant dans les bois ; c’est, pour ainsi dire, un animal devenu homme, et puisqu’il est constaté que, dans son état de pure animalité, il n’avait ni mémoire, ni réflexion, ni connaissance de lui-même, et que cependant il vivait, se conservait et pourvoyait à sa subsistance. Nous devons en conclure que les animaux, par la sensibilité seule vivent, se conservent, pourvoient à leurs besoins, et font, sans intelligence, toutes les opérations qui nous portent à les supposer intelligents.\par
Une autre conséquence à tirer de ce fait, c’est qu’il y a dans notre nature un besoin de rapport avec nos semblables, qui n’est pas dans la leur. Notre faiblesse physique, dans l’enfance, exige que nous soyons, comme eux, alimentés et secourus ; mais tout se borne là pour eux, parce que le développement de l’être physique complète leur existence, tandis que la nôtre ne serait pas même commencée par ce seul développement. Au moment où ils n’ont plus besoin de leurs semblables, et où ils peuvent s’en séparer, comme ils s’en séparent effectivement, le développement de notre être moral nous retient auprès des nôtres : c’est lui qui perpétue les rapports d’affinité, et qui, par conséquent, est le principe de la sociabilité parmi les hommes, ainsi que nous le verrons dans le chapitre suivant.
\chapterclose


\chapteropen
\chapter[{Chapitre XVIII. De la permanence des rapports d’affinité.}]{Chapitre XVIII. De la permanence des rapports d’affinité.}\renewcommand{\leftmark}{Chapitre XVIII. De la permanence des rapports d’affinité.}


\chaptercont
\noindent Aussitôt que l’oiseau peut passer de son nid sur la branche de l’arbre voisin, tout est fini entre lui et ses parents : la sollicitude paternelle et maternelle qui, un moment auparavant, éclatait encore dans toute son énergie, cesse sur-le-champ et disparaît sans retour : il en est de même dans les espèces qui vivent rapprochées et qui vont en troupes plus ou moins nombreuses. L’agneau distingue sa mère au milieu d’une foule de brebis, tant qu’il a besoin de son lait. Il reconnaît son bêlement, et ne prend point les mamelles d’une autre pour les siennes ; mais du moment où cet aliment ne lui est plus nécessaire, sa mère ne le connaît plus, il ne reconnaît plus sa mère ; plus d’affinité entre elle et lui : ils passent l’un à côté de l’autre, sans se donner aucun signe d’intelligence.\par
Pourquoi n’en est-il pas de même parmi les hommes ? pourquoi le terme de l’enfance n’est-il pas également pour eux le terme de l’affinité ? On a cherché à expliquer cette différence, en disant que l’enfance de l’homme est plus longue que celle des autres animaux, de manière que, dans celui-là, l’affinité dégénère en habitude et se perpétue par ce moyen ; mais si elle prend ce caractère dans l’homme, pourquoi ne le prend-t-elle pas dans les autres espèces ? À en juger par les signes extérieurs, les affections sont aussi vives et peut-être plus vives dans celles-ci que dans l’autre ; cependant elles cessent tout à coup pour faire place à une indifférence absolue, tandis que l’homme en conserve toujours le sentiment. Quoique l’enfance des autres animaux soit plus courte que la sienne, on ne voit pas pourquoi elle ne produirait pas la même habitude, si la permanence de l’affinité était dans l’homme l’effet de l’habitude. En avançant cette proposition, a-t-on calculé le temps qu’il fallait pour donner lieu à ce phénomène.\par
Millar, dans ses {\itshape Observations sur la distinction des rangs}, dit que les jeunes animaux sont bientôt en état de pourvoir à leur propre subsistance ; mais la longue enfance de l’espèce humaine, ajoute-t-il, exigeant plusieurs années de soins, le père et la mère d’un enfant, en voient ordinairement naître un second, avant que le premier ait cessé d’avoir besoin de leurs secours : ainsi, tant que la mère est en état d’avoir des enfants, leur affection et leurs soins passent successivement d’un objet à un autre, et leur union se continue par la même cause qui y a donné naissance.\par
Quand il serait possible d’expliquer ainsi la permanence de l’union de l’homme et de la femme, ou ce que nous appelons le mariage, on n’en pourrait rien conclure pour les enfants ; c’est-à-dire, qu’il n’en résulterait pas que les enfants, n’ayant plus besoin du secours de leurs parents, dussent conserver avec eux des rapports que ce besoin ne motiverait plus ; mais je doute qu’on puisse admettre la raison que cet auteur donne de la permanence de l’union des deux sexes, car on sait assez que dans l’état d’incivilisation, les grossesses ne se succèdent pas avec tant de rapidité, et que d’ailleurs, les soins de l’enfance y sont extrêmement abrégés par la nature même de cette condition. Ce n’est donc pas dans toutes ces considérations qu’il faut chercher la cause de la continuité des affinités domestiques.\par
Il est de fait que toute affinité cesse après l’enfance dans les autres animaux, et qu’il n’en est pas de même dans l’homme. La longueur ou la brièveté de ce premier âge, ne peut pas nous donner la raison de cette différence ; car, il a un terme dans l’homme comme dans les autres animaux, et, après cette époque, il ne peut pas motiver la continuité de l’affinité. Elle cesserait donc en nous comme elle cesse dans les autres espèces, s’il n’y avait pas en nous un principe de sa durée, qui n’existe point ailleurs.\par
Ce principe tient à la nature humaine : il ne dépend point des institutions ; car, partout où l’on a trouvé des hommes, on leur a reconnu ce caractère particulier ; partout on les a vus persévérer dans les rapports d’affinité, ce qui prouve que c’est une propriété de l’espèce humaine.\par
À la vérité, quelques écrivains ont supposé un temps où il n’en était pas ainsi ; où le terme de l’enfance était pour l’homme, comme pour les autres animaux, le terme de l’affinité ; mais cette supposition n’a jamais été appuyée d’aucun fait positif, et il est facile de voir que si tel avait été l’état de l’humanité, à une époque quelconque, il subsisterait encore, car son changement n’aurait pas dépendu d’un changement dans les choses, il aurait fallu que la nature même de l’homme fût changée, ce qui est évidemment impossible.\par
En effet, si la nature humaine a pu changer à cet égard, pourquoi celle des autres animaux ne pourrait-elle pas éprouver quelque jour le même changement ? Pourquoi le loup ou le lion, loin de se borner à nourrir leurs petits comme ils le font maintenant, ne pourraient-ils pas quelque jour leur conserver la même affection, les reconnaître pour leurs enfants, quand ils n’auraient plus besoin de leurs secours, et ceux-ci à leur tour reconnaître leur père ? Pourquoi ne s’aviseraient-ils pas de vivre en famille, de se donner mutuellement des témoignages de bienveillance, de former des tribus, d’abord, et ensuite des sociétés policées ?\par
Cette supposition vous paraît absurde ; mais si l’homme a pu passer de l’état du loup ou du lion à celui où nous le voyons maintenant, pourquoi le loup et le lion ne pourraient-ils pas passer un jour à l’état actuel de l’homme ? Transportez-vous à cette prétendue époque, où selon vous, l’homme vivait comme les autres animaux, où l’affinité cessait entre les pères et les enfants, comme elle cesse parmi eux ; dites-moi si toutes les objections que vous pouvez me faire sur le changement éventuel de condition des animaux dont je viens de parler, ne seront point applicables à l’homme ? Il est pourtant sorti de cette condition, s’il faut vous en croire, et s’il en est sorti, pourquoi les autres n’en sortiraient-ils pas ? Vous êtes forcé d’admettre cette possibilité ou de rejeter votre hypothèse sur l’homme.\par
Voulez-vous savoir d’où résulte la permanence de l’affinité dans l’homme, et sa non-permanence dans les autres animaux ? C’est que ceux-ci, bornés par leur nature ou principe du sentiment, ne sont susceptibles que de l’affinité de sensibilité qui n’existe qu’autant que les sensations qui la déterminent continuent d’avoir lieu : quand elles cessent, elle cesse. Lorsque leur sensibilité n’est plus modifiée par l’état de faiblesse de leurs petits, il ne reste rien des impressions que cet état a fait sur eux ; elles sont comme non avenues pour les pères et pour les enfants. Il n’est donc pas étonnant qu’ils deviennent étrangers les uns aux autres.\par
L’homme, au contraire, à raison de sa nature mixte, persévère dans les rapports d’affinité, parce que le principe qui connaît, ou le principe observateur des impressions que reçoit l’être sentant, conserve en lui le souvenir ou la connaissance de ces impressions : il les lui rappelle quand elles ont cessé d’avoir lieu, et en les lui rappelant il le modifie dans le sens où elles l’ont modifié et détermine en lui les mêmes affections. C’est donc l’affinité de connaissance qui succède en lui à l’affinité de sensibilité. Cet effet naturel dans l’homme ne peut jamais avoir lieu dans les autres animaux, puisque la simplicité de leur nature exclut le principe de la connaissance.\par
Voilà la seule explication qu’il nous semble qu’on puisse donner de ce phénomène. Au reste, nous voyons clairement ici que l’être moral est en nous le principe de la sociabilité, puisque c’est lui qui perpétue en nous les rapports d’affinité qui sont les éléments de la société humaine. La durée de ces rapports étant limitée dans la nature de l’être sentant ou physique, les propriétés de celui-ci ne peuvent pas donner lieu à la société. Il faut absolument qu’elle résulte des facultés de l’autre. C’est une vérité dont il importe essentiellement qu’on soit convaincu, et à laquelle nous tâcherons de donner tout le développement dont nous sommes capables.
\chapterclose


\chapteropen
\chapter[{Chapitre XIX. De la nature des rapports sociaux.}]{Chapitre XIX. De la nature des rapports sociaux.}\renewcommand{\leftmark}{Chapitre XIX. De la nature des rapports sociaux.}


\chaptercont
\noindent Pour savoir de quel principe ou de quelle propriété de la nature humaine dérivent les rapports sociaux, nous n’avons qu’à examiner quelle est celle de nos qualités qui leur est essentielle. Ce n’est point, la sensibilité ; car les animaux sentent, et ils ne sont pas sociables. S’ils sont incapables de conserver les rapports d’affinité, à plus forte raison ne sont-ils pas susceptibles de rapports sociaux. Ce n’est pourtant pas le défaut de sensibilité qui les met dans cette impuissance, mais le défaut de liberté. Pour nous en convaincre, nous n’avons qu’à faire abstraction de la liberté en nous-mêmes, et nous verrons si, en nous dépouillant de cette faculté, nous ne nous dépouillerons pas également de la sociabilité.\par
N’est-il pas, en effet, bien aisé de concevoir que, si nous n’étions pas libres, c’est-à-dire si nous n’avions pas le pouvoir de délibérer intérieurement nos actions, d’en examiner la convenance, et en un mot de les faire ou de ne pas les faire, il nous serait impossible d’ordonner notre conduite ni de nous engager à rien ; car n’étant pas les maîtres d’agir ou de ne point agir, de faire ou de ne pas faire, comment pourrions-nous nous engager à faire ou à éviter telle ou telle chose dans des circonstances déterminées ? Il est évident que nous ne le pourrions pas, parce qu’il ne dépendrait pas de nous d’être fidèles à cet engagement ; nous ne pourrions pas nous servir de caution à nous-mêmes de son exécution, et dès lors à quoi nous servirait de le prendre ? Il est donc certain que, sans la liberté, le pouvoir de contracter ces engagements n’existerait point en nous, ou y serait entièrement illusoire. Or détruire en nous cette faculté, ne serait-ce pas y détruire la faculté sociale ?\par
Une fois privés de cette puissance, quelles que fussent nos actions et nos affections, eussent-elles la propriété de nous rapprocher les uns des autres, de nous faire exécuter des opérations communes, de mêler, si l’on veut, et, de confondre nos intérêts, en résulterait-il jamais de société proprement dite ? Ne serions-nous pas, dans tout cela, des instruments passifs, comme les rouages d’une montre qui remplissent le même but, et concourent à la même fin, sans qu’on puisse leur appliquer l’idée de société ?\par
Il n’y a donc que les rapports émanés de la liberté qui soient véritablement sociaux ; ceux que la sensibilité seule détermine et soutient n’ont rien de social ; d’où il suit qu’il ne suffit pas des phénomènes observés dans la manière d’être des abeilles, des fourmis, des castors et autres animaux de ce genre, pour affirmer, comme on le fait communément, que ces animaux vivent en société ; car l’existence des rapports qu’ils ont entre eux n’autorise point une pareille conséquence : il faudrait prouver que les êtres qui les soutiennent sont libres, c’est-à-dire qu’ils peuvent persévérer dans ces rapports, ou les changer et les modifier. Or leur constante uniformité ne permet pas d’admettre une pareille supposition. Elle indique assez qu’ils ne sont pas les maîtres de rien changer dans leur système ; que, par conséquent, il est déterminé à leur insu, sans qu’ils y prennent une part intelligente et libre ; et dès lors on a beau admirer en eux les effets, pour ainsi dire, mécaniques de leur sensibilité, il ne s’ensuit pas qu’il y ait rien de social dans leur conduite.\par
Quant à nous, il est de fait que nous établissons de nouveaux rapports, et que nous sommes toujours les maîtres de changer et de modifier ceux qui existent ; et c’est à cette propriété qu’ils doivent le caractère social qu’ils n’auraient point s’ils émanaient d’une nature nécessitée ; car la nécessité, soit naturelle, soit artificielle, est toujours la nécessité ; et, quel que soit le mode de ses opérations dans l’ordre de la nature comme dans celui de l’art, elle ne peut jamais constituer de société proprement dite : une société forcée n’est pas une société.\par
Il serait donc possible qu’il n’y eût aucune différence ostensible entre la manière d’être des autres animaux et la nôtre ; que tout ce que nous faisons, ils le fissent également, et cependant que la dénomination de société convînt a la nôtre, et ne convînt pas à la leur : il suffirait pour cela que nous fissions librement ce qu’ils feraient nécessairement ; que les mêmes actes eussent en nous un principe libre, et en eux un principe nécessité.\par
D’après ce que nous venons de dire, il est facile de voir que c’est l’être moral intelligent et libre, qui est en nous l’être social, car les rapports que l’être sentant ou physique détermine dans les autres espèces, seraient sociaux comme les nôtres, si la sociabilité dérivait de ce principe ; mais cet être n’étant pas libre, ne peut pas constituer des rapports libres, et ne pouvant pas constituer des rapports libres, il ne peut pas constituer des rapports sociaux.\par
Ceux que nous avons avec nos semblables, reçoivent ce caractère de notre nature même : il n’est pas besoin que nous le leur donnions ou que nous ayons l’intention de le leur donner, pour qu’ils l’aient effectivement. Notre liberté les socialise, en quelque sorte, à notre insu : ils sont naturellement sociaux, parce qu’ils sont naturellement libres : c’est, pour ainsi dire, une couleur qui leur est propre, et qui les distingue de ceux qu’ont entre eux les êtres nécessités.\par
Il ne faut donc pas croire, comme on le fait vulgairement, que les rapports antérieurs aux conventions positives ne fussent pas des rapports sociaux ; car, ce caractère leur étant donné par la nature humaine, ils l’ont indépendamment des conventions ; du moment où ils existent, la société existe, et comme on ne conçoit pas que les hommes aient pu vivre en aucun temps sans quelque relation entre eux, il est clair que remonter à l’origine de la société, c’est remonter à l’origine de l’homme.
\chapterclose


\chapteropen
\chapter[{Chapitre XX. De l’intérêt social, ou du motif qui nous porte à vivre en société.}]{Chapitre XX. De l’intérêt social, ou du motif qui nous porte à vivre en société.}\renewcommand{\leftmark}{Chapitre XX. De l’intérêt social, ou du motif qui nous porte à vivre en société.}


\chaptercont
\noindent Si c’est dans l’être moral ou libre qu’est le principe de notre existence sociale, c’est dans l’intérêt même de cet être qu’est la raison fondamentale de la société. Or, cet être n’a d’autre intérêt que celui de sa liberté : les considérations tirées de nos penchants et de nos besoins physiques, lui sont parfaitement étrangères ; il ne veut qu’être libre : les penchants, par cela même qu’ils sont en nous des inclinations déterminées, ne peuvent pas faire partie de son système : ce n’est pas de lui, comme nous l’avons observé, que nous vient le désir de notre conservation, l’amour du plaisir, la crainte de la douleur. L’existence de ces inclinations déterminées et de la liberté impliquerait dans un même être. Ce serait vouloir, comme nous l’avons déjà dit, qu’il fût à la fois libre et nécessité. Il est évident que cela ne se peut pas : les penchants excluent la liberté, la liberté exclut les penchants, et le seul moyen de concilier cette contradiction dans l’homme, est de le supposer comme nous l’avons fait, composé de deux êtres à l’un desquels appartiennent les penchants, et à l’autre la liberté. Il en résultera que ces deux êtres auront chacun un intérêt différent. L’un aura l’intérêt de ses penchants, l’autre, celui de sa liberté. Il ne faudra donc pas motiver par l’intérêt de l’un ce qui le sera par celui de l’autre, et c’est ce qu’on a fait relativement à la société.\par
En effet, en voulant qu’elle soit motivée par l’intérêt de notre conservation, et par des considérations tirées de nos penchants et de nos besoins physiques, on la rapporte à l’être qui n’en est pas le principe, et dont l’intérêt ne peut pas par conséquent la motiver ; on lui ôte son vrai motif, son motif naturel, pour lui en substituer un qui n’est pas le sien.\par
Il semble pourtant qu’il était très facile d’éviter un pareil contresens, car l’intérêt de notre conservation, pour lequel on veut que nous nous soyons réunis en société, n’existe-t-il pas dans les autres animaux comme dans l’homme ? ne s’intéressent-ils pas comme nous à leur conservation ? comme nous ne recherchent-ils pas le plaisir, ne fuient-ils pas la douleur ; et si l’intérêt de ces penchants n’exige pas qu’ils vivent en société, pourquoi l’exigerait-il dans l’homme ? pourquoi serions-nous sociables par des motifs et par des intérêts qui nous sont communs avec des êtres qui ne le sont pas ?\par
Les mêmes causes ne doivent-elles pas produire les mêmes effets ? Si celles que vous assignez à la société, dans l’homme, se trouvent également dans les autres animaux, pourquoi ne produisent-elles pas dans ceux-ci l’effet qu’elles produisent dans celui-là. Jusqu’à ce que vous m’ayez donné la raison de cette différence, je serai fondé dans votre système, à vous dire avec Rousseau, que je ne vois pas pourquoi un homme aurait plutôt besoin d’un autre homme, qu’un singe ou un loup de son semblable. L’enfant qui vivait parmi les ours s’était conservé dans les bois sans le secours de personne, il n’avait donc pas besoin des autres pour remplir un but qu’il remplissait sans eux.\par
D’ailleurs en motivant la société par l’intérêt de notre conservation, on est si peu d’accord avec la nature des choses, que la société, de quelque manière qu’on la suppose organisée, en exige impérieusement la subordination et le désintéressement.\par
En effet, une condition essentielle de la société, c’est que tous les individus qui la composent, la défendent et la soutiennent au péril de leur vie ; sans cette obligation ou sans cet engagement, la société serait illusoire, car on ne pourrait pas compter sur sa stabilité ; or, pour s’imposer cette obligation ou pour prendre cet engagement dont l’exécution peut être à chaque instant réclamée, ne faut-il pas s’être fait à soi-même le sacrifice de cette conservation, dont on suppose que la garantie est le motif déterminant de l’institution sociale ? ne faut-il pas s’être mis dans une telle disposition à l’égard de la vie qu’on soit prêt à la quitter, le moment d’après, si le bien de la société l’exige, et conçoit-on qu’on puisse baser une institution sur l’intérêt d’une chose dont cette institution même suppose le plus parfait désintéressement ?\par
Vainement dirait-on que l’engagement de sacrifier sa vie au maintien de la société, étant pris par tous les membres qui la composent, il en résulte, pour la conservation de chacun, la garantie de tous, et qu’ainsi cet engagement, loin de supposer le désintéressement de la vie, est, au contraire, motivé par le désir de sa conservation.\par
Cette interprétation, applicable à l’effet, ne le serait point à la cause ; c’est-à-dire que, quoiqu’il entre effectivement dans le système de la société de conserver ses membres, son adoption ou son institution n’en exige pas moins, de la part de chacun en particulier et de tous en général, le désintéressement de la vie ; car, pour se soumettre à la lui sacrifier, il faut que ce sacrifice soit en leur puissance, et qu’ils aient l’intention de l’effectuer, par conséquent qu’ils fassent trêve au désir de se conserver ; sans quoi, ils s’en imposeraient les uns aux autres sur leurs dispositions mutuelles ; ils manifesteraient l’intention de sacrifier généreusement leur vie pour l’intérêt commun, tandis qu’intérieurement ils ne songeraient qu’à s’en assurer la conservation ; dès lors, la société se trouverait basée sur la mauvaise foi, il y aurait dans chacun de ses membres une arrière-pensée, une fraude cachée dont on apercevrait l’effet, à l’approche du moindre péril, par la désertion de tous ceux qui s’étaient engagés à le repousser.\par
Il faut donc que les termes de l’obligation soient l’expression formelle et positive de la disposition intérieure de l’homme, et cette disposition loin d’accuser en lui, comme on le prétend, plus de sollicitude pour l’intérêt de sa conservation, suppose au contraire la subordination de cet intérêt : elle annonce qu’il n’est pas commandé par le désir de vivre, puisqu’il se soumet à perdre la vie pour ses semblables. Je ne vois pas trop comment on peut rapporter une obligation de cette nature à l’intérêt de la conservation, et il me semble, quoi qu’on en puisse dire, que dans cette hypothèse, il y a contradiction entre le motif qu’on suppose à l’homme, et la teneur positive de son engagement.\par
Mais, admettons l’interprétation d’où l’on tire l’étrange conséquence, que c’est par l’intérêt de sa propre conservation que l’homme limite le droit de se conserver, il faudrait pour autoriser un semblable système qu’il ne pût pas se conserver sans s’imposer cette obligation ; qu’il lui fût impossible d’exister dans l’isolement, et qu’alors sa réunion à ses semblables lui offrant des chances conservatrices, il consentît pour les acquérir à créer lui-même celle de sa destruction ; mais l’exemple des autres animaux et celui de l’homme même, ne permettent pas d’accorder à l’état social ce privilège exclusif. L’expérience d’ailleurs l’en aurait bientôt dépouillé, nous n’en sommes pas à savoir combien les causes de destruction y sont multipliées. Or, si l’homme pouvait se conserver hors de cet état, et si cet état n’est pas décidément conservateur, pourquoi les hommes l’auraient-ils adopté, et pourquoi le soutiendraient-ils par le motif qu’on leur suppose ?\par
Ici, nous sommes loin de trouver ce rapport de l’effet à la cause qui fait dire que l’un est la suite de l’autre : jamais rien ne parut se repousser ni s’exclure davantage. L’engagement de cesser de vivre ne peut pas résulter du désir de vivre ; il faut absolument qu’il ait un autre motif dans la nature humaine ; qu’il se rapporte en nous à un intérêt supérieur à celui de la vie. Or nous avons fait voir que celui de la liberté devait l’emporter sur celui de notre conservation ; et, puisque ce dernier ne motive point l’état social, il faut qu’il soit motivé par l’autre.\par
On demandera comment l’intérêt de notre liberté peut exiger que nous vivions avec nos semblables, et que nous nous soumettions à des devoirs, à des obligations et à des engagements envers eux ; c’est ce que nous examinerons bientôt : mais il faut auparavant que nous fassions voir pourquoi cette vérité a été méconnue, et pourquoi l’on a cru que la société, loin de favoriser le développement de la liberté dans l’homme, en exigeait, au contraire l’abnégation et le sacrifice.
\chapterclose


\chapteropen
\chapter[{Chapitre XXI. Erreur des philosophes et des publicistes dans la recherche du principe et du motif de la société.}]{Chapitre XXI. Erreur des philosophes et des publicistes dans la recherche du principe et du motif de la société.}\renewcommand{\leftmark}{Chapitre XXI. Erreur des philosophes et des publicistes dans la recherche du principe et du motif de la société.}


\chaptercont
\noindent En n’admettant dans l’homme qu’un seul être ou qu’un seul principe, on l’a dépouillé du caractère mixte qui le distingue des autres animaux ; et dès lors il n’est pas étonnant qu’on n’ait pu se faire aucune idée de son mécanisme intérieur. Il a fallu rapporter à cet être unique les facultés et les propriétés de deux êtres différents, lui donner à la fois les penchants et la liberté, et, pour qu’il n’y eût pas de contradictions dans cette double attribution, faire consister la liberté dans l’obéissance non contrariée aux penchants, c’est-à-dire, qu’on a cru que l’homme était parfaitement libre lorsqu’il pouvait en suivre les déterminations, et qu’il cessait de l’être lorsqu’un motif ou un obstacle quelconque l’empêchait de s’y conformer.\par
Or cette liberté, qu’on pourrait appeler physique, n’est pas proprement la liberté ; car, quoique les animaux éprouvent une violence réelle quand on les empêche d’obéir à leurs penchants, et qu’on soit dans l’usage de dire alors qu’ils ne sont pas libres, quoique, par la même raison, on les regarde comme libres lorsque rien ne les gêne dans cette obéissance, il est pourtant vrai qu’ils ne le sont dans l’un ni dans l’autre cas. La seule différence, c’est que, dans le premier, une nécessité artificielle empêche, ou contrarie, en eux, la nécessité naturelle, et que, dans l’autre, celle-ci n’est ni empêchée ni contrariée ; mais ils n’en sont pas moins nécessités, ou plutôt ils n’en, sont pas plus libres.\par
On dit aussi d’un ruisseau que son cours est libre lorsque rien n’en change la direction ; cependant il n’est pas plus libre, en suivant sa pente nécessaire, que lorsqu’il est contenu par une digue.\par
Nous appliquons donc le mot à une chose naturellement nécessitée, mais dont rien ne contrarie ou n’empêche l’effet, et ce n’est pas dans ce sens que nous devons entendre la liberté de l’homme : celle-ci est elle-même naturelle, c’est-à-dire étrangère à toute nécessité ; elle ne tient pas seulement aux effets, c’est la cause même qui est libre. Elle ne peut donc pas consister dans l’obéissance non contrariée aux penchants qui, à raison de leur détermination naturelle et indépendante de nous-mêmes, sont en nous une cause nécessitée ; de manière que, si nous n’avions pas le pouvoir de nous subordonner ces inclinations, elles nous commanderaient ; et si elles nous commandaient, nous ne serions pas libres.\par
C’est précisément la position dans laquelle se trouvent les autres animaux. Ils suivent invariablement les déterminations des penchants, parce qu’il n’est pas en leur pouvoir de les modifier : ils sont donc toujours commandés par ces inclinations. L’attrait du plaisir, la crainte de la douleur, le désir de leur conservation, les déterminent nécessairement : on ne trouve point parmi eux de Scévola ni de Scipion ; aucun ne donne l’exemple d’une mort volontaire ni d’une continence spontanée, d’où nous devons conclure qu’ils sont naturellement nécessités, et que l’homme est naturellement libre, puisqu’il est indépendant des inclinations auxquelles ils sont assujettis, et qu’il ne leur obéit qu’après avoir délibéré s’il doit ou ne doit pas leur obéir.\par
Mais ce pouvoir, il ne l’aurait point, si son être moral n’était point développé ; car dans cette supposition, il serait borné à l’être physique comme les autres animaux, et dès lors, il serait commandé comme eux par ses penchants, il en suivrait nécessairement les déterminations, et en un mot, il ne serait pas plus libre qu’ils ne le sont eux-mêmes. Sa liberté tient donc au développement de son être moral, et cet être, comme nous l’avons déjà vu, ne pouvant se développer que par les rapports sociaux, si vous faites abstraction de ces rapports, vous faites abstraction du développement moral de l’homme ; et en faisant abstraction de ce développement, vous faites abstraction de sa liberté.\par
Ceci prouve combien est absurde la supposition d’un état primitif d’insociabilité, dans lequel on accorde à l’homme une liberté illimitée ; car si, dans un temps quelconque, les hommes s’étaient trouvés dans une pareille condition, le principe de la liberté n’étant pas développé en eux, ils n’auraient point eu de liberté. C’eût été pour eux un état de nécessité absolue et non de liberté absolue, comme l’ont prétendu ceux qui ont adopté cette hypothèse.\par
Ce qui les a induits en erreur, c’est l’acception commune du mot liberté ; ils l’ont appliqué à l’homme dans le sens vulgaire qui n’exclut point la nécessité naturelle : ils l’ont cru libre dans l’obéissance non contrariée aux penchants, et comme on ne voit pas ce qui pourrait l’empêcher d’en suivre les déterminations hors de la société, on en a conclu que dans cet état il serait parfaitement et absolument libre ; voilà la source de l’égarement dans lequel les fausses notions qu’on s’est fait de la liberté ont jeté une foule d’écrivains, dont les systèmes règlent encore l’opinion, et forment la doctrine reçue.\par
Après avoir ainsi supposé l’homme parfaitement et absolument libre, hors de l’état social, il n’était plus possible de donner l’intérêt de la liberté pour motif à cette institution, puisque cet intérêt se trouvait pleinement satisfait avant qu’elle existât : il a donc bien fallu lui chercher un autre motif, et supposer, comme on l’a fait, que c’est pour l’intérêt de leur conservation que les hommes se réunissent et forment des corporations politiques.\par
Ces réunions donnant lieu à des obligations et à des devoirs qui n’existeraient pas sans elles, et sans lesquels elles ne pourraient pas subsister ; leur condition de rigueur étant que chacun des membres qui les composent, s’impose l’obligation de ne pas se livrer à des plaisirs qui pourraient nuire aux autres ; de souffrir pour eux, et de mourir, s’il le faut, pour leur intérêt commun ; on a dû croire, en partant de l’idée que la liberté consiste dans l’obéissance non contrariée aux penchants, que ces devoirs, ces obligations ou ces engagements, étaient autant de restrictions de la liberté antérieure, autant de sacrifices de cette liberté que chacun consentait à faire, pour obtenir de ses coassociés la garantie de sa propre conservation.\par
Telle est la manière dont on a raisonné jusqu’ici. On n’a pas vu ou l’on n’a pas voulu voir l’espèce de contradiction qu’il y a de supposer que, pour l’intérêt de sa conservation, l’homme s’engage à périr quand les autres l’exigeront, et à créer ainsi lui-même la chance de sa destruction ; on n’a pas vu non plus que, dans cette hypothèse, l’homme s’étant conservé sans le secours de ses semblables, jusqu’au moment où on lui fait prendre la résolution de vivre avec eux, c’est donner à cette résolution un motif illusoire ; car, si l’homme se conservait effectivement et pouvait se conserver dans ce qu’on a appelé l’état de nature, comment veut-on qu’il ait été déterminé à le quitter par le motif de sa conservation ? d’où lui est venu ce besoin de ses semblables, dont on admet qu’il s’était passé jusqu’alors ?\par
Je me lasse de relever toutes les inconséquences et toutes les contradictions de cette théorie. Le fait est que, dans ces sortes de systèmes, la société qui est le phénomène le plus général et le plus constant parmi les hommes, se trouve un effet sans cause déterminante, une condition purement facultative ; et c’est bien ainsi que l’envisageait Rousseau, lorsqu’il a prétendu que, sans des événements qui pouvaient ne pas arriver, elle n’aurait jamais eu lieu.\par
Cependant il est assez difficile de croire qu’une condition à laquelle tient le caractère distinctif de l’homme, et sans laquelle il ne serait pas ce qu’il est, puisse être pour lui un état arbitraire et non un mode d’existence essentiel à sa nature. Séparez-vous un moment de l’idée de la société, faites-en abstraction ; supposez qu’elle n’a jamais existé, comment pourrez-vous me dire : Voilà l’homme ? Où en est la preuve, vous répondrai-je ? rien ne le distinguant des autres animaux, comment lui délivrer un certificat d’humanité ?\par
Vous m’objecterez sa forme extérieure ; mais si l’on découvrait, dans quelque partie du globe, une peuplade de singes ou de loups, délibérant sur leurs intérêts communs, et remplissant, les uns envers les autres, des devoirs et des obligations dont ils auraient la connaissance, quelque différence matérielle qu’il y eût entre eux et nous, ne serions-nous pas forcés de les classer dans notre espèce, comme nous y classons les hommes de différente forme et de différente couleur ?\par
Si l’on découvrait en même temps des êtres d’un extérieur parfaitement semblable au nôtre, mais incapables de se prescrire aucun devoir, de prendre et de remplir aucun engagement ; la ressemblance, ou, si l’on veut, l’identité de leur forme avec la nôtre, serait-elle une raison de les déclarer hommes ?*\par
L’humanité ne se caractérise donc pas par la forme extérieure, mais par la liberté intérieure et morale des actions, et cette liberté ne pouvant se développer que dans les rapports sociaux et par les rapports sociaux, on est forcé d’en conclure qu’ils sont absolument nécessaires à l’humanité.\par
Écartons donc cette hypothèse d’un état primitif d’insociabilité, qu’on nous présente comme l’état naturel de l’homme, et dans lequel nous trouvons l’anéantissement de sa nature caractéristique : écartons également l’intérêt de notre conservation, comme motif de l’état social, puisque nous avons vu que cet intérêt ne peut pas le motiver, et que si les autres êtres satisfont à ce qu’il exige d’eux, sans former de société proprement dite, il n’y a point de raison qui puisse nous porter à croire que nous ayons besoin de cette institution pour nous conserver. Cherchons le motif de la société dans un intérêt qui nous soit particulier, puisque la société nous est particulière ; et comme celui de la liberté n’existe que pour nous, voyons si nous ne trouverons pas dans cette propriété privilégiée de notre nature, l’explication d’un phénomène qu’on a vainement essayé d’expliquer jusqu’ici, en le rapportant à des causes qui nous sont communes avec les êtres nécessités.
\chapterclose


\chapteropen
\chapter[{Chapitre XXII. Que c’est l’intérêt de notre liberté, et non celui de notre conservation, qui constitue en nous l’intérêt social.}]{Chapitre XXII. Que c’est l’intérêt de notre liberté, et non celui de notre conservation, qui constitue en nous l’intérêt social.}\renewcommand{\leftmark}{Chapitre XXII. Que c’est l’intérêt de notre liberté, et non celui de notre conservation, qui constitue en nous l’intérêt social.}


\chaptercont
\noindent On est si accoutumé à regarder les devoirs, les obligations et les engagements comme des gênes ou des restrictions de la liberté ; qu’on aura sans doute bien de la peine à se persuader qu’ils en soient un besoin et une propriété essentielle : cependant, si la liberté, comme nous l’avons déjà fait sentir, ne consiste point dans l’obéissance illimitée aux penchants ; s’il faut, au contraire, que l’homme, pour être libre, les domine et se les subordonne, nous verrons naître, du besoin de cette subordination, le besoin des devoirs ; et dès lors il sera évident que ceux-ci, loin de restreindre ou d’anéantir la liberté, comme l’ont prétendu la plupart des écrivains, sont véritablement institués à son profit.\par
En effet, les penchants, par leur nature, sont absolus en nous comme dans les autres animaux : ils nous provoquent également à les satisfaire ; mais ceux-là n’étant pas libres, ne peuvent se refuser à leurs provocations qui les déterminent toujours, et auxquelles ils ne peuvent jamais résister spontanément. Ce pouvoir qu’ils n’ont pas, l’homme en est investi par sa nature libre ; mais, outre qu’il faut que cette nature soit développée, comme nous l’avons dit dans le chapitre précédent, il faut encore que des motifs de convenance déterminant l’exercice du pouvoir dont elle est en nous le principe. Or ces motifs n’existent et ne peuvent exister que par la société et dans la société ; car, dans l’isolement, je voudrais bien qu’on me dît où seraient pour l’homme les motifs de convenance d’exercer quelque empire sur ses penchants ; il est certain que cette condition ne pourrait pas nous les procurer, et que dès lors nous serions condamnés à nous laisser dominer par ces sortes d’inclinations, comme les créatures à qui la faculté de les modifier n’a pas été accordée.\par
L’intérêt de notre liberté nous porte donc à chercher ces motifs, à nous placer dans les circonstances qui peuvent leur donner lieu ; et comme la seule position qui puisse les amener est celle de vivre avec nos semblables, l’intérêt de notre liberté nous porte à vivre avec eux. Ce mode d’existence crée pour nous la convenance de nous subordonner nos penchants, et par conséquent d’exercer notre liberté. En nous obligeant à ne rechercher le plaisir qu’autant qu’il ne nuira point aux autres, à ne fuir la douleur qu’autant qu’il ne leur sera point avantageux que nous la supportions, à ne nous conserver qu’autant que la société n’aura pas besoin du sacrifice de notre vie, nous déterminons les cas où nous entendons obéir à nos penchants et ceux où nous voulons leur résister, et par là nous exerçons sur eux l’empire qu’ils exerceraient sur nous, si nous ne prenions pas cette détermination.\par
Voilà le véritable motif des engagements et des devoirs sociaux : nous avons besoin de ces devoirs pour ôter à nos penchants leur caractère absolu, et les rendre, pour ainsi dire, facultatifs ; modification qui les approprie à la liberté à laquelle ils ne pourraient pas s’adapter sans cette modification : Il faut donc qu’ils la subissent, pour que nous soyons libres ; et s’ils ne peuvent la subir qu’autant que nous vivons avec nos semblables, et que nous nous imposons des devoirs envers eux, il est clair que c’est dans notre liberté que la société a la raison fondamentale de son existence, et non dans les autres considérations sur lesquelles on l’a basée jusqu’à ce moment.\par
L’homme, pour vivre conformément à sa nature libre, ne peut pas, comme les autres animaux, s’en tenir à l’usage déterminé de la vie, c’est-à-dire qu’elle ne peut pas être en lui, comme elle l’est en eux, une tâche à laquelle il ne puisse rien changer, et qu’il soit tenu de remplir conformément à un type prescrit et imposé ; cependant son existence étant soumise aux mêmes lois physiques et aux mêmes besoins que celle des autres animaux, comment, avec une telle conformité, mettra-t-il entre eux et lui cette différence que la sienne sera libre, et que la leur sera nécessitée ?\par
C’est que ceux-ci la laisseront telle qu’ils l’auront reçue : ils ne disposeront point de leur vie ; ce sera leur vie qui disposera d’eux : l’homme, au contraire, exercera son empire sur la sienne ; il lui donnera un but, une intention ; il prendra l’engagement de la sacrifier pour ses semblables, et, par cet engagement, il se l’appropriera ; car, dès lors, elle ne sera plus en lui de nécessité naturelle comme elle l’était auparavant, mais de sa propre obligation, puisque, l’ayant soumise de son chef à des engagements qui lui sont propres et personnels, c’est pour ces engagements qu’il sera censé vivre, et non parce que la nature lui a imposé la vie.\par
Voilà par quel artifice il sortira de la classe des êtres nécessités, avec lesquels la nature semble avoir voulu le confondre, en l’enfermant, pour ainsi dire, dans un système de lois nécessaires, et qu’il résoudra, comme nous l’avons annoncé, le problème de vivre libre dans un ordre de choses nécessité.\par
L’intérêt de notre liberté est donc le motif réel des obligations que nous nous imposons envers nos semblables ; nous disposons de notre vie à leur profit, parce que nous avons besoin d’en faire cette disposition pour qu’elle soit à nous : il nous faut l’engager, et l’aliéner en quelque sorte, pour en devenir propriétaires ; car, en lui imprimant cette modification, nous en exproprions en nous la nature, pour nous substituer à sa place, et exercer ses droits sur le bien qu’elle nous a donné : nous en faisons le même usage qu’elle en a fait ; nous en disposons comme elle en a disposé : c’est ainsi que nous nous assurons qu’il est à nous, et qu’au lieu de tenir servilement à la vie comme les êtres que la nécessité enchaîne à sa conservation, nous la mettons elle-même dans notre dépendance, et lui imposons nos propres lois.\par
Telle est la nature du but social ; l’intérêt qui nous y porte n’est pas, comme on l’a cru, celui de nous conserver notre vie, mais bien de nous l’approprier ; et cette appropriation, étant un besoin de notre liberté, les devoirs, les obligations, et les engagements par lesquels elle s’effectue, sont également motivés par le même principe ; car ils ne sont autre chose que des moyens dont l’appropriation de la vie est la fin ; et, comme le besoin de la fin constitue le besoin des moyens, si celle-là est en nous un besoin de la liberté, il est de rigueur que les autres se rapportent à la même cause.\par
Ici, nous apercevons pourquoi l’homme est sociable et pourquoi les autres animaux ne le sont pas ; pourquoi l’un s’impose des devoirs, des obligations, et prend des engagements, et pourquoi les autres ne font ni ne peuvent rien faire de semblable. Cette différence, que l’intérêt de notre conservation n’expliquerait pas, puisqu’ils ont le même intérêt que nous à se conserver, s’explique par ce besoin de l’appropriation de la vie, que l’intérêt de la liberté motive en nous, et qu’il ne peut pas motiver en eux, puisqu’ils ne sont pas libres : ce genre d’intérêt n’existant point dans leur nature nécessitée, ne peut pas y produire l’effet qu’il produit dans la nôtre : il ne peut pas leur donner comme à nous le besoin de s’approprier leur vie, ni par conséquent d’en faire la disposition sociale que nous en faisons, pour arriver à cette fin. L’usage de la vie est déterminé pour eux, tandis qu’il faut, pour être libres, que nous le déterminions nous-mêmes, et que son intention naturelle ne la rapportant qu’à nous, nous lui en donnions une de notre chef qui la rapporte aux autres.\par
C’est ainsi qu’elle devient en nous une propriété libre, et qu’elle n’est dans les autres animaux qu’une tâche imposée, ou plutôt un effet indépendant de leur participation ; car ils n’ont pas le pouvoir de la modifier, de l’engager, de l’aliéner, de la garder ou de s’en démettre, et, en un mot, d’en disposer comme d’un bien à eux appartenant. Cette faculté qu’exclut la nécessité de leur nature, est à la fois une propriété et un besoin de la nôtre : s’ils étaient libres, ils auraient, comme nous, ce besoin d’exercer une suprême juridiction sur leur propre existence ; et de ce besoin naîtrait, pour eux comme pour nous, celui des obligations et des devoirs sociaux par lesquels l’autre se satisfait.\par
Dans tout état de société, l’appropriation, et par conséquent la libération de la vie, a lieu par les devoirs et les engagements convenus ou non convenus de l’homme envers ses semblables. Le sauvage vit et meurt pour sa tribu : il dispose donc socialement de sa vie, et satisfait, par cette disposition, au besoin de se l’approprier, et de l’affranchir du mode nécessité dans lequel s’écoule celle des autres créatures.\par
Cette appropriation libérale et généreuse de la vie dans l’état le plus sauvage, met une différence totale entre cet état et celui des autres animaux, auquel on a voulu le comparer ; l’un est social et libre, et l’autre insocial et nécessité. Voilà ce qui constitue la vraie ligne de démarcation entre l’homme et la bête. Les écrivains qui, en remontant l’échelle graduée de la civilisation, ont cru pouvoir arriver à ce terme où toute distinction s’effacerait entre l’homme et les autres animaux, ne nous ont donné que des romans et des abstractions métaphysiques. Ils n’ont pas vu que cette distinction tient à la liberté de l’un et à la non-liberté des autres, à la sociabilité qui est la conséquence de la liberté dans celui-là, et à l’insociabilité qui résulte de la nécessité dans ceux-ci ; que cette distinction, fondée sur leur nature respective, est indépendante de la civilisation, et se montre, chez les peuples les moins civilisés, avec autant ou même avec plus d’énergie que chez les autres ; que par conséquent elle est indélébile, et a dû subsister à toutes les époques ; et qu’une des erreurs dont il importe le plus de se guérir, c’est de croire qu’en remontant au principe de la civilisation, on remonte au principe de la société.\par
Notre théorie diffère de celles qu’on a données jusqu’ici, en ce que les autres supposent que l’homme ne s’impose les devoirs sociaux que pour le ménagement de ce qu’on appelle ses intérêts ; et nous, nous prétendons, au contraire, que ces devoirs lui étant nécessaires pour être libre, ont en eux-mêmes l’objet propre et déterminant de leur imposition ; c’est pour eux-mêmes que l’homme se les impose, et qu’il les remplit, parce qu’il y trouve le moyen d’exercer sur ses penchants un empire qui les concilie avec sa liberté. Cette vérité que nous avons essayé d’établir dans ce chapitre, va devenir plus sensible encore par les développements où nous allons entrer dans ceux qui suivent, et qui tendent tous à l’éclaircir et à la confirmer.
\chapterclose


\chapteropen
\chapter[{Chapitre XXIII. Que la considération de la réciprocité n’entre pour rien dans le motif déterminant des devoirs et des obligations que l’homme s’impose envers ses semblables.}]{Chapitre XXIII. Que la considération de la réciprocité n’entre pour rien dans le motif déterminant des devoirs et des obligations que l’homme s’impose envers ses semblables.}\renewcommand{\leftmark}{Chapitre XXIII. Que la considération de la réciprocité n’entre pour rien dans le motif déterminant des devoirs et des obligations que l’homme s’impose envers ses semblables.}


\chaptercont
\noindent Une suite de l’erreur dans laquelle on est tombé, en motivant la société par l’intérêt de notre conservation, a été de regarder la réciprocité comme une condition essentielle de cet ordre des choses, c’est-à-dire, de supposer que l’homme ne s’imposait des obligations envers ses semblables que parce qu’ils s’imposaient les mêmes obligations envers lui : de là, l’idée d’un pacte ou d’un contrat social, pour stipuler et consacrer cette prétendue clause fondamentale de la société.\par
Tout cela tient au préjugé qui basé la société sur l’intérêt de notre conservation. La réciprocité est une condition de rigueur dans ce système ; car, lorsqu’un danger imminent porte les hommes à se réunir pour leur mutuelle conservation, c’est qu’ils ne croient pas que la force individuelle puisse le surmonter, et qu’ils espèrent en venir à bout, en réunissant leurs forces isolées. Il faut donc que l’application réciproque de ces forces ait lieu, et que chacun fasse pour le salut des autres ce que les autres font pour son propre salut. La nature de cette association, déterminée par la présence du péril, et motivée par le besoin de forces et de moyens réunis, suppose nécessairement que chacun attend les secours des autres en prodiguant les siens, et qu’il compte sur cette réciprocité.\par
Mais la société naturelle des hommes : entre eux n’est pas sans doute fondée sur une semblable hypothèse ; un péril imminent n’en a point déterminé l’existence ; car, dans cette supposition, elle aurait été accidentelle et momentanée comme sa cause. Le danger dissipé, son but aurait été rempli, et le rétablissement du calme aurait produit le licenciement des individus, jusqu’à ce qu’une nouvelle crise eût encore provoqué leur réunion.\par
L’intérêt de notre conservation ne peut pas motiver un phénomène aussi général et aussi constant que celui de la société parmi les hommes ; car il est d’expérience que cet intérêt ne les porte à se réunir que lorsqu’il est compromis. En perdant ses craintes, ses alarmes, il perd, pour ainsi dire, toute sa puissance attractive ; d’où il suit qu’il faudrait, pour que la société se soutînt par cet intérêt, que les hommes fussent dans un état perpétuel d’inquiétude sur leur existence : si vous leur accordez un moment de sécurité, il n’y a plus de raison dans votre système pour qu’ils restent unis.\par
L’erreur de ce système est donc de rapporter à un phénomène général et constant la cause d’un phénomène accidentel et particulier ; de vouloir que l’intérêt de leur conservation, qui détermine les hommes à se réunir quand il est compromis, soit la cause de leur-union naturelle, générale et permanente : aussi la nécessité d’approprier l’effet à la cause force-t-elle les auteurs de ce système de donner à la société le caractère d’un phénomène accidentel,, de supposer que des événements alarmants ont déterminé son existence, et par conséquent d’admettre un temps où elle n’avait pas lieu ; elle n’est plus, alors dans la nature de l’homme, mais dans les circonstances de sa position ; et l’on ne voit pas comment le changement de ces circonstances n’aurait pas changé la disposition de l’homme à vivre en société : {\itshape Sublatâ causâ tollitur effectus.}\par
Plus on examine cette doctrine, plus on est surpris qu’elle ait pu faire illusion à des hommes d’un profond génie, qui l’ont soutenue de la meilleure foi, sans se douter qu’elle était fausse dans son principe et dans ses conséquences.\par
En rejetant l’intérêt de notre conservation comme motif déterminant de la société, nous n’avons plus les mêmes raisons de regarder la réciprocité comme une condition de rigueur dans cette institution, ou du moins comme une considération déterminante.\par
Mais comment concevoir, dira-t-on, que l’homme s’engage à se dévouer pour les autres, à souffrir pour eux, à mourir pour eux, sans y être déterminé par l’espoir qu’ils en feront autant pour lui ? Je sens que, d’après nos préjugés et d’après les fausses notions qu’on nous a données de l’homme et de la société, on doit trouver bien extraordinaire l’imposition gratuite de ces obligations ; mais ne voyez-vous pas que c’est dire, en d’autres termes, qu’on ne se laissera point commander par l’attrait du plaisir, par la crainte de la douleur, ni par l’amour de la vie, et que s’il est nécessaire à l’homme de se mettre dans cette disposition envers lui-même pour être libre, c’est dans sa propre liberté qu’est le motif déterminant de cet engagement ; que par conséquent il ne le prend pas envers les autres, pour les autres, ni parce qu’ils le prennent envers lui, mais parce qu’il a lui-même besoin de le prendre, indépendamment de toute considération de réciprocité.\par
L’obligation sociale est donc à la fois intéressée et généreuse ; elle est intéressée, parce que c’est l’intérêt de notre liberté qui nous détermine à nous l’imposer ; et elle est généreuse, en ce que le motif de cet engagement étant en nous-mêmes, nous n’avons pas besoin d’un retour d’obligation de la part des autres, pour obtenir le but de cette disposition ; il est dans cette disposition même : elle est, en quelque sorte, sa propre fin ; il ne s’agit donc pas ici du {\itshape do ut des} qui détermine les transactions d’un intérêt synallagmatique, ni, par la même raison, d’un contrat, pour stipuler une clause que la nature de l’engagement ne comporte pas.\par
L’idée de ce contrat n’est venue que parce qu’on a voulu que la société fût basée sur l’intérêt de notre conservation, et que, dans cette hypothèse, on ne pouvait pas se passer de réciprocité ; mais, dans le fait, cette institution est purement et absolument généreuse dans son essence, c’est-à-dire que les membres qui la composent, ayant, chacun dans leur nature, le même intérêt à disposer de leur vie pour les autres, à la leur engager pour en libérer l’exercice, s’imposent cette obligation, par ce seul motif qui, pouvant se satisfaire, soit que les autres remplissent la même obligation, soit qu’ils ne la remplissent pas, se suffit, pour ainsi dire, à lui-même, et n’a nul besoin de retour.\par
Tel est le véritable esprit de la société : la supposition d’un pacte ou d’un contrat ne tend qu’à le dénaturer, qu’à nous donner une fausse idée de l’état social, qu’à nous en dérober le vrai motif, le véritable intérêt, pour leur substituer des motifs et des intérêts imaginaires.\par
Ce n’est pas que la réciprocité ne résulte effectivement de l’identité des obligations ; car si plusieurs individus s’imposent les mêmes devoirs les uns envers les autres, et disposent également de leur vie, il y aura bien réciprocité entre eux, mais elle sera l’effet, et non la cause des devoirs qu’ils se seront imposés. Pour en être la cause, il faudrait qu’elle fût entrée dans le motif déterminant de l’imposition de ces mêmes devoirs : or, si ce motif, comme nous l’avons fait voir, est indépendant de cette considération ; celle-ci, pour nous servir du langage de l’école, ne s’y trouvera point {\itshape à priori} ; elle sera, si l’on veut, la conséquence des devoirs imposés, mais elle n’en sera point le motif.\par
Et une preuve bien sensible que cela doit être ainsi, c’est-à-dire que le motif des obligations et des devoirs sociaux est indépendant de toute considération de réciprocité, c’est que, quand cette réciprocité ne résulterait pas de l’accomplissement unanime des devoirs entre les divers membres de la société, celui qui les observerait n’en obtiendrait pas moins la fin pour laquelle ils sont institués ; il n’en trouverait pas moins dans leur observation l’avantage de vivre conformément à sa nature libre, ou, si l’on aime mieux, de répondre à l’intérêt de sa liberté ; tandis que ceux qui violeraient ces mêmes devoirs manqueraient ce but, et se trouveraient frustrés de ses avantages, quelque fidèles que fussent les autres à les remplir à leur égard.\par
Voilà pourquoi l’inobservation ou la violation des devoirs par les uns, n’en dégage pas les autres ; ce qui arriverait infailliblement s’ils étaient fondés sur l’avantage de la réciprocité ; car, dans cette supposition, la conséquence nécessaire de l’oubli que se permettraient les uns de remplir leurs obligations, serait d’autoriser le même oubli dans les autres ; mais s’il n’en est pas ainsi ; si l’obligation subsiste toujours, malgré le défaut de réciprocité ; et, en un mot, si je sens très bien en moi-même que, quoiqu’un autre viole les devoirs sociaux à mon égard, ce n’est pas une raison pour moi de les violer envers lui ; s’il est encore évident que, quand toute la société me proscrirait et m’accablerait d’outrages et d’injustices, je n’en dois pas moins persister dans mes obligations envers elle, c’est que mon intérêt à les remplir est indépendant de la réciprocité d’où l’on voudrait le faire dépendre, c’est que ces devoirs sont fondés sur l’intérêt propre et personnel de celui qui les remplit, et qu’en les violant, il se lèse lui-même.\par
Si l’on n’admettait pas ce principe, comment concevrait-on qu’il eût pu exister des hommes tels que Socrate, Thémistocle, Aristide ? Pourquoi admirerions-nous la conduite de ces hommes qui n’auraient été que des dupes, et qui ne mériteraient tout au plus que l’espèce de pitié qu’on accorde aux imbéciles ? Quoique leurs imitateurs soient rares, si l’on sent en soi-même que, placé dans les mêmes circonstances, on voudrait pouvoir les imiter, c’est qu’il faut qu’ils aient agi conséquemment à l’intérêt bien entendu de la nature humaine, c’est qu’ils ont satisfait éminemment à leur propre intérêt, en supportant la spoliation, l’exil, la mort, sans forfaire à leur liberté par l’abnégation des devoirs. Cependant cette persévérance, j’ai presque dit cette opiniâtreté, n’eût été, dans ceux qui en ont donné de si rares exemples, qu’une très gratuite déloyauté envers eux-mêmes, si les devoirs sociaux étaient fondés sur l’intérêt de la réciprocité, si le motif de leur institution n’était pas indépendant de cette considération ultérieure, s’ils n’avaient pas en eux-mêmes la cause déterminante et le but final de leur observation.\par
La société n’est proprement que le moyen et l’occasion de l’imposition des devoirs, elle n’est, pour ainsi dire, que leur point d’appui ; car ces devoirs étant un besoin de l’homme même, un besoin de sa nature libre, s’il pouvait le satisfaire hors de l’état social, il n’y aurait plus de raison pour lui de vivre dans cet état ; mais comme on ne voit pas que l’imposition d’aucun devoir pût avoir lieu dans l’isolement, il faut nécessairement qu’il vive en société pour satisfaire ce besoin propre et personnel des devoirs : c’est donc pour eux qu’il la recherche et qu’il la soutient, et non pour les intérêts auxquels on est dans l’usage de la rapporter : il s’ensuit que les devoirs qu’on regarde comme la partie onéreuse de l’état social, sont précisément l’avantage réel et positif de cette condition ; et l’on peut dire qu’elle perdrait tout son prix s’il était possible de les en exclure.\par
Voilà ce qui fait que les engagements sociaux n’ont pas besoin d’être convenus ni stipulés comme on se le figure, et qu’ils datent d’une époque bien antérieure à l’institution des lois positives ; car l’intérêt de ces engagements étant indépendant de la réciprocité, celui qui les prend, les prenant pour son propre compte, et parce que c’est un besoin pour lui de les prendre, n’a point d’intérêt à s’assurer si les autres les prennent ; quoiqu’il les contracte en leur faveur, ce n’est pas avec eux qu’il contracte, mais avec lui-même ; et, quoique les autres paraissent être le but final de ces engagements, ils se rapportent, en dernière analyse, à celui qui les souscrit : c’est à lui-même qu’il manque, lorsqu’il ne les remplit pas. L’idée d’un contrat pour l’établissement de la société est donc l’idée la plus chimérique qu’on ait jamais pu imaginer : c’est le système le plus faux dont on ait pu s’aviser pour trouver la raison de l’état de société parmi les hommes.
\chapterclose


\chapteropen
\chapter[{Chapitre XXIV. Suite du chapitre précédent.}]{Chapitre XXIV. Suite du chapitre précédent.}\renewcommand{\leftmark}{Chapitre XXIV. Suite du chapitre précédent.}


\chaptercont
\noindent Si nous ne sentons pas ce besoin naturel d’engager notre vie aux autres pour nous l’approprier et pour en libérer l’exercice, c’est que ce besoin a été, en quelque sorte, prévu en nous, et qu’on y a satisfait d’avance. En effet, nous naissons dans cet engagement en naissant dans la société ; et cette préexistence d’une obligation que nous n’avons pas contractée, n’a rien de choquant dans notre système ; car les pères, en engageant les enfants avant qu’ils soient nés, ne font que prévoir en eux le besoin de cet engagement ; et cette prescience ne pouvant pas les tromper, ils ont, pour croire que la société sera le vœu de leurs enfants, la même certitude que s’ils étaient présents, et qu’ils pussent l’exprimer eux-mêmes.\par
Ils ne sont donc que les interprètes de ce vœu ; caractère qu’ils n’auraient point, si la société n’était pas un besoin de la nature humaine, et s’il fallait un contrat et des conventions pour lui donner l’existence ; car, dans cette hypothèse, on ne conçoit pas comment les pères pourraient soumettre leurs arrière-neveux à vivre dans un état qui pourrait très bien ne pas leur convenir : où serait pour ceux-ci l’obligation de reconnaître un pacte qu’ils n’auraient pas souscrit, et que signifie dès lors cette prétendue charte primitive, imaginée, par quelques écrivains, comme le titre fondamental de la société civile ?\par
Sans doute, les devoirs et les rapports sociaux sont susceptibles d’une infinie variété de combinaisons et de modifications, à raison de la liberté de leur principe : ils peuvent donc donner lieu à des conventions particulières, temporaires et locales : ce n’est pas sur ce point que les pères peuvent prévoir le vœu de leurs enfants, encore moins le limiter ; mais ils en sont les interprètes fidèles et assurés, en ce qui concerne le besoin positif de la société, et par conséquent des devoirs et des engagements qui sont essentiels à son existence ; en les prenant pour leurs enfants, ils ne font que ce qu’ils feraient inévitablement eux-mêmes ; ils satisfont d’avance, comme nous l’avons déjà dit, un besoin effectif de leur nature ; de manière qu’en naissant, ils se trouvent dans l’état qui leur convient, et dont ils ne sauraient se passer, quoiqu’ils puissent d’ailleurs y faire toutes les modifications dont il est susceptible.\par
Mais parce que nous trouvons, pour ainsi dire, notre état tout fait, il s’ensuit que nous ne sentons pas le besoin de nous le donner, et que son motif nous échappe, lorsque nous cherchons à nous en rendre raison. Si, pour le découvrir, nous nous supposons, par abstraction, hors de l’état social, nous nous y transportons tels que nous sommes, c’est-à-dire avec le développement que nous avons reçu de la société : nous ne pouvons pas nous faire l’idée d’une condition dans laquelle notre être moral ne serait point développé, et où il nous serait impossible de donner à notre vie une intention autre que celle de la nature, qui est de nous faire vivre : nous nous y portons, malgré nous, avec notre développement moral, avec la liberté de suivre ou de ne pas suivre les déterminations de nos penchants ; liberté que nous n’aurions point : nous supposons que nous n’en ferions usage que pour les suivre, parce que nous n’aurions point de motif d’en agir autrement, et que dès lors il y aurait un accord parfait entre notre liberté et nos penchants.\par
Telle est l’illusion à laquelle nous nous livrons : nous ne faisons pas attention que notre être moral n’étant pas développé, nous serions dans la servitude physique de nos penchants ; que, n’ayant pas le pouvoir de les modifier, ils seraient absolus en nous, et nous commanderaient ; d’où il suivrait que nous serions nécessités dans toute la teneur de notre vie.\par
Si nous pouvions nous faire une idée juste de cette position, nous concevrions parfaitement que l’intérêt de notre liberté motive l’état social, et que les devoirs, les obligations, les engagements qui le constituent, sont un besoin de notre nature libre : cet intérêt nous paraîtrait assez puissant pour déterminer, un semblable effet ; mais, par cela même que cet effet existe, l’intérêt que nous avons à son existence disparaît, en quelque sorte, pour nous, à peu près comme celui de nous bien porter, auquel nous ne donnons guère d’attention, tant que nous ne sommes pas malades.\par
Il n’est donc pas étonnant que nous prenions le change sur la cause naturelle de la société, qui n’a rien de frappant ni de remarquable, et que nous croyons la trouver dans l’intérêt de notre conservation qui, dans les moments de péril, déterminant des réunions subites, et en quelque sorte d’instinct, doit nous paraître bien plus énergique que cet autre intérêt, par lequel nous vivons en état constant de société.\par
Celui-ci doit être méconnu ou à peine soupçonné, en sorte que la méprise, est presque inévitable ; et, quand une fois nous avons donné pour motif à la société l’intérêt de notre conservation, il est de rigueur que nous y fassions entrer la réciprocité comme une condition essentielle : or voici les conséquences qui résultent naturellement d’un semblable système ; c’est que nos devoirs et nos obligations dans cette théorie ne profitant qu’aux autres, et ceux des autres étant à notre profit, il ne faut pas un grand effort de calcul pour nous persuader qu’il est de notre intérêt que les autres remplissent parfaitement leurs obligations, et que nous puissions nous dispenser de remplir les nôtres : de là l’indifférence pour des devoirs dont nous ne sentons pas l’avantage pour nous-mêmes : ce ne sont plus que des charges pénibles et onéreuses, dont la sagesse et l’habileté consistent à s’affranchir ; il s’ensuit qu’à l’émulation de remplir les devoirs doit bientôt succéder celle de s’y soustraire, et que si les hommes étaient conséquents à ce système, il amènerait infailliblement la dissolution de la société.\par
Voilà l’effet certain de cette doctrine de l’état social, basé sur l’intérêt de notre conservation et sur la clause essentielle de la réciprocité : nous en voyons découler l’anéantissement de toute vertu, l’abnégation ou l’oubli des devoirs, leur négligence calculée, et une prédilection antisociale pour ses intérêts des penchants.\par
Une doctrine aussi contraire à la société, dans ses conséquences, ne peut certainement pas lui être adaptée dans ses principes, et quoiqu’elle semble justifiée par l’état actuel de la société, où l’on voit que la plupart négligent les devoirs, pour soigner uniquement ce qu’ils appellent leurs intérêts, s’imaginant qu’il n’existe pour eux aucun profit dans l’observation de ceux-là, et que tout est bénéfice dans la satisfaction de ceux-ci ; il est pourtant vrai que la société, même la plus corrompue, dépose contre le système que nous combattons. On pourrait même dire que plus elle est corrompue, plus il est facile d’apercevoir la fausseté des motifs par lesquels on veut qu’elle ait été instituée et qu’elle se soutienne.\par
En effet, si les motifs qu’on donne à la société, étaient les vrais motifs de son institution ; si elle était basée sur l’intérêt des penchants et sur la réciprocité de ces intérêts, comment voudrait-on qu’elle pût subsister dans la lésion de ces mêmes intérêts ? comment serait-elle à l’épreuve des oppressions, des injustices et des iniquités les plus révoltantes ? Si pour tenir à la société, l’homme n’avait pas, dans sa nature, un motif indépendant des intérêts d’où l’on veut qu’elle dérive, ne l’abandonnerait-il pas aussitôt qu’il verrait que ceux des autres prévalent sur les siens ? Pourquoi cette classe nombreuse qui, dans toutes les associations politiques, souffre, travaille et ne vit que de privations, s’obstinerait-elle à rester unie à ceux qui dans la mollesse et dans l’oisiveté, s’engraissent de ses sueurs et lui rendent constamment en mépris, ce qu’ils en reçoivent en bienfaits ? Je voudrais bien qu’on m’expliquât une contradiction aussi étrange entre la cause et l’effet, et qu’on m’apprît comment une société fondée sur la réciprocité des intérêts, pourrait subsister, quand la partialité la plus scandaleuse et la plus révoltante, s’introduirait dans son sein, et en altérerait les dispositions fondamentales. J’avoue que la permanence de la société, dans une semblable hypothèse, me paraîtrait le phénomène le plus incompréhensible et le plus contraire aux premières notions des engagements fondés sur la réciprocité.\par
Mais il est bien évident, par la nature même des choses, que ce n’est pas ce genre d’engagement qui constitue la société, et que ceux qui ont prétendu l’expliquer dans ce sens, se sont abusés eux-mêmes, et ont ensuite abusé les autres ; car, en admettant leur contrat social, où serait la garantie de cette réciprocité qui, selon eux, est la clause fondamentale de ce pacte primitif ?\par
Dans les transactions particulières, la sanction de la loi donne aux engagements que les individus prennent entre eux la garantie de l’autorité publique. Si l’individu envers lequel je m’engage ne remplit pas son engagement envers moi, je connais d’avance le moyen de l’y contraindre, en le traduisant devant les tribunaux : il existe des précautions et des sûretés qui me déterminent ; mais, dans votre pacte social, où sont-elles ? Qui me répond que ceux envers lesquels je m’engage ne manqueront point à leur engagement envers moi ? d’où puis-je tirer la certitude de leur bonne foi, de leur adhésion constante aux conditions que nous venons de souscrire ; et, s’ils manquent, à mon égard, à la transaction sociale, lorsque j’en aurai fidèlement rempli les obligations, devant quel tribunal pourrai-je les citer pour en obtenir justice ? à quelle autorité pourrai-je appeler de la violation du pacte social ?\par
Ce serait donc le plus fou, le plus indiscret, le plus téméraire de tous les engagements, si la réciprocité en était le motif déterminant ; car il s’ensuivrait que l’homme ferait aux autres le sacrifice de son existence, ou s’obligerait à la sacrifier pour eux, sans avoir la moindre certitude d’en retirer les avantages qu’il s’en serait promis : or ce n’est pas ainsi que les hommes se conduisent dans les spéculations de cette nature. Ils ne s’engagent pas avec tant de légèreté dans les moindres affaires ; ils pèsent auparavant les certitudes, les probabilités, et prennent des précautions infinies pour s’assurer de n’être pas trompés : voudrait-on qu’ils n’eussent pas apporté les mêmes dispositions à l’établissement de la société, si la société avait été pour eux une affaire de spéculation ? auraient-ils donné, tête baissée, dans un établissement de cette importance, eux qu’on voit, dans les plus petites affaires, ne s’engager qu’à bon escient, et calculer toutes les chances avant de s’immiscer dans la plus légère entreprise ?\par
Ceci, sans doute, peut être considéré comme une démonstration {\itshape à priori} que l’idée de la réciprocité n’entre pour rien dans le motif déterminant de la société.\par
Mais les preuves {\itshape à posteriori} sont peut-être encore plus frappantes et plus énergiques ; car supposons des hommes assez confiants pour avoir pu croire que l’engagement qu’ils prenaient envers les autres, et que les autres prenaient envers eux, serait fidèlement rempli, comment imaginer que, se voyant ensuite horriblement et cruellement trompés dans leur attente, ils eussent pu persister dans cette association devenue léonine ? Quoi ! voilà des individus qui, avec des droits égaux, forment un établissement, dont les avantages doivent leur être communs ! Cependant il se trouve bientôt, par le fait, que quelques-uns ont tout, et que la grande majorité n’a rien ; que tous les avantages sont pour ceux-là, toutes les charges pour ceux-ci ; et, malgré ce renversement des conditions essentielles et fondamentales de l’association, l’établissement subsiste ! il ne vient pas même dans l’idée de ceux qui sont le plus lésés de l’abandonner pour rentrer dans leur premier état !\par
Vous ne me ferez jamais comprendre que cet effet pût avoir lieu dans votre système : s’il était le vrai système de la société, elle se serait maintenue dans les conditions de son pacte primitif, ou elle aurait cessé d’avoir lieu, comme tous les établissements qui reposent sur des stipulations de réciprocité, et qui ne se soutiennent qu’autant que ces stipulations sont observées et maintenues.\par
La société n’a donc rien de commun avec les établissements de ce genre ; la leur assimiler, c’est en donner une fausse notion, c’est en dénaturer le caractère, c’est substituer au véritable intérêt social un intérêt antisocial, c’est induire les hommes à poursuivre un but qui les égare, et qui trompe toujours leurs vœux, soit qu’ils l’atteignent, soit qu’ils le manquent, parce qu’il n’est pas celui qu’ils doivent se proposer.\par
Au reste, si l’homme ne sent pas le vrai motif de la société ; s’il ne l’aperçoit pas dans l’intérêt de sa liberté, qui veut qu’il s’impose des devoirs envers les autres, et qu’il les remplisse ; s’il n’a pas la conscience de l’avantage qu’il retire ou qu’il peut retirer de ces devoirs et de leur religieuse observation, quelle que soit la conduite des autres à son égard ; en un mot, s’il ne fait pas consister le bonheur de vivre en société dans l’unique bonheur d’en remplir les obligations, ce n’est pas seulement parce que, ne s’étant jamais trouvé hors de la société, il ne peut se faire aucune idée d’un état, où l’impossibilité de s’imposer ces sortes d’obligations lui en ferait sentir le besoin propre et personnel ; c’est encore parce que des institutions tyranniques lui ayant fait contracter des habitudes et des inclinations serviles, l’ont rendu indifférent pour sa liberté, et que, dès lors, ne cherchant plus à se subordonner ses penchants pour être libre, il n’a plus senti l’avantage des devoirs qui sont le moyen de cette subordination : il leur a ôté cette destination pour leur donner celle d’obtenir par eux une réciprocité avantageuse aux intérêts de ses penchants qui, dans son indifférence pour sa liberté, sont devenus le motif de toutes ses déterminations.\par
C’est de cette erreur sur la nature et sur la fin des devoirs sociaux que sont nées toutes ces doctrines qui les ont présentés comme des sacrifices auxquels l’homme s’était soumis envers les autres, dans l’espoir d’obtenir de leur part des avantages qui en compenseraient la rigueur ; et c’est ainsi que le véritable esprit de la société s’est altéré et corrompu dans la théorie et dans la pratique : on a perdu de vue la générosité de son essence, pour la transformer en un marché frauduleux, où chacun s’engage avec le désir secret de ne pas remplir son engagement, et ne cherche, dans la vaine promesse qu’il fait de lui être fidèle, qu’à se donner un titre qui l’autorise à presser son exécution dans les autres.\par
Je ne vois pas trop, d’après les systèmes reçus, ce qu’on peut trouver d’extraordinaire dans une pareille conduite, qui n’est au fond que la très juste conséquence des principes qu’on établit ; car si les devoirs ne sont par eux-mêmes d’aucun intérêt pour celui qui les adopte ; s’il ne les observe que pour les autres, et afin d’obtenir de leur part une réciprocité avantageuse, l’intérêt bien calculé de la société sera d’obtenir le plus possible avec la moindre mise de devoirs : les mauvais raisonneurs, les calculateurs maladroits, seront ceux qui mettront toujours les devoirs en avant, sans s’inquiéter du produit ; ceux-là prendront à contresens la spéculation sociale ; ce seront de pauvres gens qui n’entendront rien à leurs affaires, et dont l’imbécillité mériterait l’interdiction.\par
S’il est vrai pourtant qu’on n’en juge pas ainsi ; si on les regarde, au contraire, comme les hommes les plus sages ; si le désespoir de les imiter humilie notre faiblesse, c’est sans doute parce qu’ils entendent mieux leurs vrais intérêts que les autres ; qu’ils les raisonnent mieux, et les calculent mieux ; et comment reconnaître en eux cette supériorité, si l’on n’admet pas qu’il est avantageux d’observer les devoirs pour eux-mêmes ? Ce principe une fois admis, nous trouverons que ceux qui négligent les devoirs pour les intérêts sont les mauvais raisonneurs, les mauvais calculateurs, les insensés qui cherchent le bénéfice de la société où il n’est pas, qui font abstraction de leur propre intérêt en ne s’occupant que de leurs intérêts, et enfin qui se dupent eux-mêmes en croyant duper les autres.
\chapterclose


\chapteropen
\chapter[{Chapitre XXV. Objection à laquelle le désintéressement des devoirs relativement aux autres peut donner lieu. Réponse à cette objection.}]{Chapitre XXV. Objection à laquelle le désintéressement des devoirs relativement aux autres peut donner lieu. Réponse à cette objection.}\renewcommand{\leftmark}{Chapitre XXV. Objection à laquelle le désintéressement des devoirs relativement aux autres peut donner lieu. Réponse à cette objection.}


\chaptercont
\noindent Si nous avons en nous-mêmes le motif propre et déterminant de l’imposition et de l’observation des devoirs, et si ce motif est indépendant de toute considération de réciprocité, que nous importe, dira-t-on, que les autres remplissent ou ne remplissent pas ces mêmes devoirs ? c’est une chose qui doit nous être parfaitement indifférente. Pourquoi donc voulons-nous qu’ils les remplissent, et quel droit avons-nous d’employer des moyens de rigueur pour les y contraindre ?\par
Telle est l’objection qu’on peut nous opposer, et à laquelle nous sentons la nécessité de répondre : il est certain que l’homme ayant besoin de s’imposer des devoirs envers ses semblables pour être libre, trouve dans l’intérêt de sa liberté le motif propre et déterminant de l’imposition et de l’observation de ces devoirs : quand il les néglige, c’est un mal qu’il se fait à lui-même ; il est donc intéressé à les remplir, quelle que soit la conduite des autres.\par
Dans ce cas-là, dit-on, pourquoi ne pas les laisser les maîtres d’agir comme il leur plaît ? Mais je demanderai à mon tour pourquoi ne pas laisser se jeter dans la rivière celui qui veut s’y jeter ? Votre vie n’est-elle pas indépendante de la sienne ? votre intérêt à vivre a-t-il quelque chose de commun avec celui qu’il a de vivre lui-même ? Quel droit avez-vous donc d’user de violence pour le retenir à la vie malgré lui ? Vous me direz que c’est un devoir ; eh bien, c’en est un également de forcer les autres à remplir leurs devoirs ; il doit même l’emporter sur celui de leur conserver la vie ; car supposez-vous dans l’alternative d’empêcher un homme de commettre un crime atroce, ou de l’empêcher de périr, n’agirez-vous pas conséquemment à son propre intérêt, en préférant son inculpabilité à sa vie ?\par
Or, si cette sollicitude pour l’observation des devoirs de la part des autres est elle-même un devoir en nous, elle est motivée par l’intérêt que nous avons nous-mêmes à remplir nos devoirs, et non par aucune considération intéressée dans son rapport à ceux qui en sont l’objet ; car, si nous ne l’exercions que parce qu’il est avantageux aux autres que nous observions nos devoirs, et qu’il nous est avantageux à nous-mêmes qu’ils observent les leurs, serions-nous affectés des violations qui ne se rapporteraient point directement à nous-mêmes ? Pourquoi exciteraient-elles notre indignation et notre zèle à les réprimer ? pourquoi manifesterions-nous notre joie et notre admiration pour l’observation héroïque des devoirs, qui ne tournerait point à notre propre avantage ?\par
D’ailleurs, si cette sollicitude avait le motif qu’on lui suppose, il s’ensuivrait que les hommes qui tiennent le plus à leur fortune, à leur vie et à ce qu’on appelle communément leurs intérêts, devraient en être bien plus susceptibles que les autres ; cependant l’expérience prouve le contraire : on voit qu’en général ces hommes, si intéressés à l’observation des devoirs sociaux, sont précisément ceux qui s’y intéressent le moins ; car ils s’affectent peu des violations qui n’ont pas un rapport direct à eux-mêmes : ils murmurent, se plaignent, et réclament vivement quand les devoirs ne sont pas observés à leur égard ; mais ce n’est pas l’inobservation des devoirs qui détermine leurs plaintes et leurs murmures, c’est la souffrance de leurs intérêts : ils crieraient également, si elle avait lieu par toute autre cause ; leur désespoir serait le même. Qu’on leur assure que leur maison ne brûlera pas, et les voilà prêts à laisser brûler celle de leur voisin.\par
On ne peut donc pas dire que ces gens-là qui, dans les systèmes reçus, paraissent éminemment intéressés à l’observation des devoirs sociaux, y prennent effectivement un très grand intérêt. Leur indifférence sur ce point inexplicable dans ces sortes de systèmes, s’explique naturellement par les principes que nous avons établis. En effet, la sensibilité pour ce qu’on appelle les intérêts, se développe presque toujours aux dépens des devoirs, c’est-à-dire que, plus on tient à ceux-ci, moins on doit tenir à ceux-là, et plus on tient à ceux-là, moins on est attaché aux autres : or, si c’est en raison de notre attachement à nos propres devoirs que s’exerce notre sollicitude pour que les autres observent les leurs, il est évident qu’elle doit avoir peu d’activité dans les hommes qui, très chauds pour leurs intérêts, sont par cela même très froids pour leurs devoirs : on ne doit donc pas être surpris que les violations les plus scandaleuses se commettent sous leurs yeux, sans qu’ils montrent le moindre zèle à les prévenir ou à les réprimer.\par
Ce sont, au contraire, ceux qui tiennent le moins aux intérêts, et à qui par conséquent la violation des devoirs de la part des autres semblerait le moins dommageable ; ce sont ceux-là, dis-je, qui manifestent le zèle le plus actif pour leur observation, et qui se sacrifient pour empêcher qu’ils ne soient violés : cependant, d’après les idées communes et d’après nos systèmes sociaux, c’est l’inverse qui devrait avoir lieu.\par
Au reste, il est facile de comprendre que notre sollicitude, pour que les autres remplissent leurs devoirs, n’est pas en nous l’effet de la considération intéressée des avantages qui en résultent pour nous, car elle entre implicitement dans cette disposition de notre vie, que nous faisons à leur profit pour être libres nous-mêmes. En effet, si ce motif nous détermine à nous imposer l’obligation de donner notre propre vie pour la conservation de celle des autres, il entre nécessairement dans cette obligation de la leur conserver dans le mode conforme à leur nature, par conséquent de prendre à ce mode le même intérêt qu’à leur vie même. Or, s’il faut, pour qu’ils vivent conformément à leur nature, c’est-à-dire, pour qu’ils soient libres, qu’ils remplissent leurs devoirs, il est de notre devoir à nous envers eux de vouloir qu’ils les remplissent, et de nous dévouer à l’observation de leurs devoirs comme à la conservation de leur vie.\par
L’induction qu’on a tirée de cette sollicitude pour fonder l’observation des devoirs sur la réciprocité des intérêts, est donc fausse en principe ; et, ce qui le prouve, c’est que le zèle pour l’observation commune des devoirs, loin de favoriser nos intérêts, en exige souvent le sacrifice : ne faut-il pas s’exposer au courroux et à la puissance des violateurs ? Est-ce un homme très attaché à sa fortune, à sa vie, à ses plaisirs, qui aura le courage de dévoiler et de poursuivre les oppresseurs de ses concitoyens, les dilapidateurs de la fortune publique ? Ne gardera-t-il pas un silence motivé sur la crainte de compromettre ses intérêts ? ne préférera-t-il pas au rôle périlleux de se montrer leur ennemi celui de devenir leur complice ?\par
Certes, loin de songer à ses intérêts, il faut être bien dégagé de leur influence pour prendre un vif intérêt à l’observation des devoirs de la part des autres. Il n’est peut-être pas de sollicitude plus généreuse, de devoir plus difficile à remplir, surtout dans une société corrompue. Ce n’est donc pas la considération de ce qu’on appelle communément nos intérêts qui en détermine en nous l’exercice, c’est celle de nos devoirs, puisqu’elle tient à leur système, et participe à la générosité de leur essence.
\chapterclose


\chapteropen
\chapter[{Chapitre XXVI. Appel à l’expérience contre ceux qui basent la société sur les intérêts.}]{Chapitre XXVI. Appel à l’expérience contre ceux qui basent la société sur les intérêts.}\renewcommand{\leftmark}{Chapitre XXVI. Appel à l’expérience contre ceux qui basent la société sur les intérêts.}


\chaptercont
\noindent Les intérêts, dans le sens ordinaire, ne comprennent que la satisfaction des penchants ; car on les oppose aux devoirs qui en exigent la subordination et le sacrifice, de manière que, dans l’opinion commune, les devoirs ne sont d’aucun intérêt par eux-mêmes ; et voilà pourquoi l’on suppose que l’homme ne les adopte que pour le ménagement des intérêts des autres, et afin que les autres ménagent les siens ; c’est par le prétendu besoin de cette réciprocité qu’on motive l’état social qui, par là, se trouve, en dernière analyse, institué au profit des intérêts, c’est-à-dire des penchants.\par
La conséquence naturelle de cette doctrine, c’est qu’il faut, pour remplir parfaitement nos vues, que cet état pourvoie efficacement à notre conservation ; qu’il multiplie la somme de nos plaisirs, et diminue les chances de la douleur ; car c’est là ce que demandent nos penchants, et par conséquent nos intérêts : plus il remplit ces conditions, plus, dans le système des intérêts, il approche de sa perfection, et plus il semble que nous devons tenir à cet état ; car on voit que nous nous attachons aux choses à raison de leur aptitude à remplir le but de leur existence, et c’est surtout relativement à la société que l’application de ce principe doit avoir lieu.\par
D’où vient cependant qu’on observe précisément le contraire ? pourquoi trouve-t-on les citoyens moins zélés pour la défense de la société à laquelle ils appartiennent, lorsqu’ils y ont goûté, pendant une longue paix, le bonheur de vivre sans inquiétude pour leur conservation ? pourquoi la même indifférence se manifeste-t-elle, lorsque les arts et les richesses y ont introduit tous les raffinements du plaisir, et multiplié les moyens de prévenir ou d’écarter la douleur ? C’est l’époque à laquelle on devrait être le plus passionné pour un état qui tient tout ce qu’on s’en était promis ; cependant ouvrez l’histoire de tous les peuples, et voyez quel a été le signe précurseur des révolutions et de la chute des empires !\par
C’est, à ce qu’il me paraît, un terrible argument contre les systèmes modernes ; car si les intérêts sont le but et la fin de la société ; si c’est pour eux qu’elle est établie, pourquoi le lien social se relâche-t-il à mesure qu’ils sont mieux traités ? pourquoi se détache-t-on de la société en raison des progrès qu’elle fait pour arriver à ce que vous regardez comme sa perfection, c’est-à-dire à l’accomplissement des vues pour lesquelles vous la supposez instituée ?\par
Je crains bien que, faute de connaître le vrai système de l’homme, vous ne vous soyez égarés dans la recherche du vrai système de la société ; que vous ne l’ayez ôtée à l’intérêt qui la motive pour la rapporter à des intérêts qui ne la motivent pas, et qui par conséquent ne peuvent la soutenir. L’extrême attention donnée à ceux-ci, serait dès lors un soin à contresens de la vraie sollicitude sociale, une fausse économie de la société, qui, loin de la perfectionner, doit l’altérer, la corrompre et la détruire.\par
En effet, si l’homme a besoin d’ôter à ses penchants leur caractère absolu, pour être libre ; s’il ne peut le leur ôter que par les devoirs, et si les devoirs ne peuvent exister que par la société, c’est proprement pour les devoirs qu’elle est établie : en la rapportant aux intérêts, vous dénaturez l’esprit de son institution, vous substituez au véritable intérêt social des intérêts étrangers à son système. L’exagération de sensibilité que vous provoquez pour ceux-ci produit l’indifférence pour l’autre : il n’y a donc rien d’extraordinaire qu’il s’affaiblisse et s’éteigne à mesure que vous mettez plus de recherche à ménager ceux dans lesquels vous croyez trouver le bénéfice de la société, et que, dans le fait, vous ne cultivez qu’à son désavantage.\par
C’est de cette politique insensée qu’est né le préjugé qui fonde la consistance de la société sur la consistance des intérêts. L’esprit, une fois prévenu de cette fausse opinion, ne voit la société que dans les riches, ne regarde comme citoyens que les propriétaires : ceux-là, dit-on, sont intéressés à maintenir la société ; plus ils tiennent à leurs propriétés, plus ils doivent tenir au corps social : alors on mesure le civisme sur l’échelle des propriétés ; les grosses fortunes sont des câbles qui attachent ceux qui les possèdent à la patrie, et en font les plus fermes appuis : de là, l’estime et la vénération pour les grands propriétaires : ce sont les honnêtes gens, les hommes probes, les citoyens par excellence : on les gratifie de toutes les qualités qui dignifient la nature humaine, on les appelle aux premières places de la république.\par
Les autres sont les ennemis naturels de la société ; ils n’ont aucun intérêt à la défendre, et par conséquent aucun droit à vivre dans son sein. N’a-t-on pas poussé l’extravagance de cette doctrine jusqu’à confondre le sol avec la cité, jusqu’à soutenir que les propriétaires avaient le droit de chasser ceux qui ne le sont pas !\par
Mais si ce système, comme nous croyons l’avoir suffisamment démontré, est faux dans son principe et dans ses conséquences ; si la société ne tient pas aux intérêts ; si c’est le besoin des devoirs qui détermine son existence, c’est de la consistance des devoirs, et non de celle des intérêts, que résulte la consistance de la société. Or, les devoirs étant un besoin personnel de l’homme, les pauvres et les non-propriétaires, par cela même qu’ils sont hommes, ont le même intérêt et le même droit à la société que les riches ; ils doivent même lui être plus attachés, car la sensibilité des intérêts dans ceux-là nuit au sentiment des devoirs qui, dans les autres, a toute son énergie ; j’en excepte les circonstances où le pauvre, corrompu par le riche, n’aspire, comme lui, qu’à soigner ses intérêts, ne regarde les devoirs que comme la partie onéreuse de l’état social, et devient, par cette dégradation des sentiments naturels de l’humanité, le vil esclave des riches s’il n’en est pas le bourreau, ou leur bourreau s’il n’en est pas l’esclave.\par
Mais à qui les riches peuvent-ils s’en prendre de ces révolutions qu’ils ont eux-mêmes provoquées, en dénaturant l’esprit de la société, en substituant les intérêts aux devoirs, en corrompant les opinions et les sentiments de la multitude par la perversité de leurs inclinations ? Tant qu’on basera la société sur les intérêts, il faudra s’attendre à y voir éclater tôt ou tard ces horribles secousses : elles auront lieu, quelque soin qu’on apporte à les prévenir, car ce sont des crises morales, des efforts que la nature fait pour ramener la société à ses vrais principes ; et il est dans l’ordre que ceux qui l’en ont écartée, ou qui la retiennent dans cette déviation, en soient les victimes.\par
Telle est la conséquence nécessaire de nos systèmes politiques, économiques et civilisateurs, dans lesquels, donnant aux intérêts l’importance qu’il faudrait donner aux devoirs, on ne s’occupe que des propriétés et des propriétaires ; de manière que, si le travail de la multitude n’était pas nécessaire à l’oisiveté des riches, on ne voit pas trop pourquoi elle serait admise dans la société, ni quel droit elle aurait d’y prétendre. Que peut-il résulter de l’adoption de ces principes ? Ce qu’il en est déjà résulté ; l’insolence des uns, l’humiliation des autres, l’immoralité de tous.\par
S’il était vrai, comme vous le prétendez, que la multitude non propriétaire fût l’ennemie naturelle de la société, croyez-vous que l’état social pût se maintenir, malgré tous les moyens de répression et de compression que pourraient employer vos riches et vos propriétaires ? Il faut donc bien admettre qu’elle a un motif naturel d’adhérer à la société, et par conséquent de la soutenir : ce motif est dans la nature humaine, et non dans la possession du soi ni dans l’accaparement de ses productions. Renforcez-le, au lieu de l’affaiblir, et vous y trouverez le vrai palladium de la société que vous chercheriez inutilement dans l’intérêt de la propriété qui ne peut ici servir de caution ni de garantie.\par
En faisant consister la cité dans les devoirs, comme le veut la nature des choses, vous aurez des citoyens non propriétaires, en la faisant consister dans les intérêts ; vous aurez des propriétaires non citoyens, qui tiendront beaucoup à leurs propriétés sans tenir à la cité, et dont l’attachement aux devoirs sociaux sera toujours en raison inverse de leur attachement pour leurs intérêts. S’agira-t-il de verser son sang pour la patrie, ils y enverront le non-propriétaire ; faudra-t-il des impôts pour subvenir aux dépenses publiques, ils feront vendre le grabat du pauvre, et garderont leur lit de duvet.\par
Voilà comment ces hommes qui tiennent au corps social par des intérêts si puissants, s’il faut en croire, la politique du jour, et qui, par conséquent, en sont les plus fermes appuis ; voilà, dis-je, comment ils prouveront leur zèle à défendre et à soutenir la société : ce seront pourtant d’honnêtes gens, d’excellents citoyens. Comme on ne cessera de leur prodiguer ce titre qu’ils mériteront si bien, la multitude, corrompue par leur exemple, voudra devenir honnête à son tour, et regarder la friponnerie comme le noviciat de la probité, jusqu’à ce que, trouvant ce moyen trop lent, trop incertain, trop périlleux, sous la domination de votre aristocratie propriétaire, elle saisira la première occasion de chasser et d’exproprier vos honnêtes gens pour devenir elle-même honnête tout à coup.\par
Vous ferez alors de belles complaintes, des descriptions bien pathétiques des scènes atroces qui auront signalé cette épouvantable catastrophe, et vous ne verrez pas qu’elle aura été l’effet de vos doctrines insensées, de votre impolitique partialité pour une classe privilégiée, de cette faveur et de cette considération exclusivement accordées aux riches, qui auront jeté la multitude dans cet état d’abjection, de corruption et de mépris, d’où naîtront ces effroyables désordres.\par
C’est ainsi qu’en exagérant les avantages de la propriété, vous provoquerez la violation des propriétés : en cherchant dans les intérêts une garantie contre les foreurs révolutionnaires ; vous enfermerez les hommes dans le cercle vicieux des révolutions ; le tout pour n’avoir pas connu le vrai système de la société, pour l’avoir basée sur des intérêts qui ne peuvent lui servir d’appui, pour avoir voulu que cette institution, qui n’est relative qu’à la liberté de l’homme, tournât uniquement au profit de ses penchants, et qu’étant généreuse dans son essence, elle devînt, par l’altération de son principe, une froide spéculation d’égoïsme et de cupidité.
\chapterclose


\chapteropen
\chapter[{Chapitre XXVII. Suite du chapitre précédent.}]{Chapitre XXVII. Suite du chapitre précédent.}\renewcommand{\leftmark}{Chapitre XXVII. Suite du chapitre précédent.}


\chaptercont
\noindent S’il est vrai qu’on ne tienne à sa patrie qu’à raison des intérêts, dites-moi quels sont les puissants intérêts qui attachent le Tartare à sa horde, le sauvage américain à sa tribu ? Est-ce parce qu’ils y possèdent une tante ou une méchante cabane qu’ils trouveraient partout ailleurs ? Motiverez-vous par votre grand principe de la propriété leur attachement inaltérable pour la société dans laquelle ils vivent ? Ne devraient-ils pas, dans votre système, être absolument cosmopolites ? ne sont-ce pas de ces gens que vous appelez {\itshape gens qui n’ont rien à perdre}, et qui, par conséquent, n’ont aucune raison de tenir au corps social ?\par
Il semblerait donc que rien ne devrait être plus précaire, plus incertain, que ces sortes de sociétés ; le moindre choc devrait les dissoudre ; car le lien social ne s’y trouvant resserré par aucun des intérêts, d’où, selon vous, il tire toute sa force, serait, dans cette supposition, le plus relâché, le plus faible de tous les liens, et ne pourrait constituer que des associations éphémères : cependant l’expérience prouve le contraire. On sait que ces sociétés subsistent sans se confondre ; qu’elles tiennent fortement à leur conservation et à leur indépendance : il n’existe parmi elles ni défection ni division de la part d’aucun des membres qui les composent. Ce n’est point là qu’est né le proverbe : {\itshape Ubi benè ibi patria}. C’est, au contraire, chez les peuples qui ont pris l’intérêt de la propriété pour base de leur association, et qui l’ont regardé comme le plus ferme appui de l’union sociale.\par
Je voudrais bien que les hommes, imbus de cette doctrine, m’expliquassent pourquoi des peuples, chez lesquels il n’existe ni propriétaires ni capitalistes, sont toujours prêts à verser leur sang pour une société dans laquelle ils ne possèdent rien ? Que leur importe qu’elle subsiste, ou qu’elle soit détruite ; n’ayant point de propriétés à défendre, pourquoi s’exposent-ils gratuitement à perdre la vie, au danger plus terrible encore d’être faits prisonniers ; sachant très bien que, s’ils ont ce malheur, ils n’en seront pas quittes pour périr ; qu’on leur fera subir auparavant des tourments horribles, prolongés avec tout le raffinement de la cruauté {\itshape }? Eh bien, malgré toutes ces considérations, et quoique la plus éclatante victoire ne leur promette d’autre butin que les chevelures de leurs ennemis, ils partent au moindre signal de la volonté publique, et pas un ne s’avise de fuir, de donner sa démission, ni de demander un congé.\par
Mais, dans nos grandes sociétés civilisées, où se trouvent tant d’excellents citoyens, tant de riches propriétaires, qui tiennent fortement au corps social, et qui ont le plus grand intérêt d’y tenir, si la patrie est menacée, il faut s’industrier pour trouver des défenseurs, recourir au tirage de la milice, recruter des hommes achetés, employer la presse et les réquisitions forcées. C’est à qui se dispensera de marcher ; et les excellents citoyens, c’est-à-dire les riches que leurs intérêts attachent fortement à la société, ne sont pas les derniers à chercher des exemptions : si l’on veut qu’ils se rangent sous les drapeaux, il faut leur faire un pont d’or, leur donner des grades et des distinctions, leur promettre des récompenses honorifiques et pécuniaires.\par
Voilà comment les intérêts attachent à la cité. Ce sont précisément les non-propriétaires, c’est-à-dire les gens qu’on regarde comme étrangers au corps social, et auxquels on refuse les droits de citoyen, qui, sans marchander, volent spontanément à l’ennemi, et versent inciviquement leur sang pour une patrie dont ils sont censés les ennemis naturels. Les autres restent paisiblement dans leurs foyers, lisent tranquillement les gazettes, intriguent pour arriver aux places les plus lucratives, et spéculent sur les malheurs publics.\par
Il me semble qu’on ne peut pas trop conclure de tout cela que les intérêts sont la base de l’état social ; car plus ils acquièrent d’intensité, plus on voit que le lien social se relâche, et que personne ne prend moins d’intérêt à la société que ceux qui en prennent beaucoup à leurs intérêts.\par
Pendant fort longtemps on a cru qu’on ne pouvait être honnête homme sans religion : quelques personnes commencent à soupçonner que cette opinion pourrait bien n’être qu’un préjugé ; peut-être en sera-t-il de même de celle qui veut qu’on ne puisse être bon citoyen qu’autant qu’on est propriétaire : il est très possible qu’on découvre, quelque jour, que la propriété est dans la cité, et non la cité dans la propriété ; que par conséquent on peut être citoyen sans être propriétaire, comme on peut être propriétaire sans être citoyen. Malgré le symbole de nos philosophes économistes et publicistes, je suis tenté d’affirmer qu’on séparera le civisme de la propriété, comme on sépare déjà la probité de la religion, et qu’on ne croira pas plus au civisme exclusif des propriétaires qu’à la probité exclusive des dévots. Peut-être alors les premiers perdront-ils la haute considération dont ils jouissent dans la société, comme les autres ont déjà perdu la profonde vénération qu’on leur a longtemps accordée.\par
C’est, en effet, un autre genre de superstition, qu’on pourrait appeler la superstition de la propriété : elle n’a rien de commun avec le respect de la propriété, qui tient au devoir social. On peut même dire que la propriété ne sera jamais mieux respectée que quand on en aura banni la superstition.\par
La société est un besoin naturel de l’homme : tout homme est donc citoyen par sa nature. Cette qualité tient à la personne ; car on sent très bien que la fin de la société serait illusoire, s’il était au pouvoir de quelques-uns d’en recueillir tout l’avantage, et d’en priver les autres : or c’est ce qui arriverait, si elle était fondée sur les intérêts. Il faut donc qu’elle ait un but que tous puissent atteindre, et dont aucun ne puisse être écarté : c’est ce que nous offre la liberté obtenue par les devoirs sociaux ; car ces devoirs sont à la disposition de tous : chacun est toujours le maître de se les imposer, de les remplir, et d’être libre par eux. Voilà le principe et la fin de la société. En manquant à nos devoirs, ou en les négligeant pour les intérêts, nous pouvons vivre dans son sein sans en recueillir le véritable fruit ; mais c’est un bien dont personne ne peut nous priver ni nous déshériter. Chacun se fait son lot à cet égard, et se rend heureux ou misérable, indépendamment des autres.\par
Tout homme a donc en soi les mêmes moyens d’être citoyen, et le même intérêt à l’être. La fin de la société est la même pour tous : elle exige de tous la subordination des intérêts aux devoirs, et leur en assure à tous le même avantage. Voilà la parfaite égalité naturelle et sociale. Toute la différence entre le bon et le mauvais citoyen, c’est que l’un préfère constamment les devoirs aux intérêts, tandis que l’autre donne la préférence aux intérêts sur les devoirs ; et il est clair que plus les intérêts ont de l’intensité, plus il doit être difficile de les subordonner aux devoirs ; ce qui n’est pas du tout à l’avantage du riche.\par
Celui qui, n’étant pas propriétaire, ou qui l’étant déjà, usurpe la propriété d’un autre, fait un acte de mauvais citoyen, parce qu’il donne à ses intérêts la préférence sur ses devoirs ; les biens qu’il acquiert ne le dédommagent pas de celui dont il se prive, en manquant à l’obligation sociale de respecter les propriétés : les moyens de jouissance que lui procure son usurpation ne compensent en rien le tort qu’il se fait à lui-même par sa dégradation morale. Quand il n’aurait rien à craindre des lois, il est malheureux par la conscience de son invirtualité. Il a péché sans dédommagement ; car le tort qu’il s’est fait, est absolu.\par
Le riche, le propriétaire, qui n’est pas dans la disposition constante de sacrifier ses propriétés à l’utilité publique, est encore un mauvais citoyen, car il donne aussi la préférence aux intérêts sur les devoirs ; par cette prédilection funeste à lui-même, il s’isole de ses semblables pour se vouer et s’enchaîner à des êtres inanimés dont il devient l’esclave ; il se dégrade, s’avilit, et perd cette belle conscience de lui-même qu’il eût acquise s’il eût été fidèle à l’intérêt de sa liberté, en dévouant généreusement sa personne et ses biens à sa patrie. Ce trésor vaut bien sans doute ceux auxquels il attache un si grand prix ; mais il faut ne pas tenir à ceux-ci pour avoir celui-là. Voulez-vous être riche de votre propre estime, soyez généreux de votre personne et de vos biens ; regardez-les comme un dépôt que la patrie vous a confié, et que vous devez lui rendre à sa première réquisition.\par
D’après cette condition exigée des propriétaires pour être bons citoyens, je demande s’il en est beaucoup qui le soient effectivement ? si tous ou presque tous, au lieu d’être disposés à faire à la patrie le sacrifice généreux de leurs intérêts, ne les défendent pas de tout leur pouvoir contre ses besoins les plus urgents, et si la garantie de leurs propriétés n’est pas encore le motif déterminant des sacrifices qu’ils font, après avoir épuisé tous les moyens de ne pas les faire ?\par
Où est, dans tout cela, la preuve de ce grand attachement qu’on suppose que la propriété donne à l’homme pour la société dans laquelle il vit ? Dire que les propriétaires tiennent beaucoup au corps social, parce qu’ils tiennent beaucoup à leurs propriétés, me paraît le plus vicieux de tous les raisonnements ; car, pour tenir beaucoup au corps social, il faudrait qu’ils ne tinssent pas du tout à leurs propriétés, qu’ils fussent toujours prêts à en faire un hommage généreux à la patrie ; et certes ce n’est là ni l’esprit du système que nous combattons, ni la profession de foi de ces riches propriétaires qu’on présente à notre vénération comme les membres les plus recommandables de la cité : on peut en juger par les cris qu’ils poussent, pour peu qu’on touche à leurs intérêts : montrent-ils la même sensibilité pour les maux qui blessent la patrie ? ne se consolent-ils pas volontiers de son humiliation et de son asservissement ? ne sont-ils pas les premiers à les provoquer, s’ils croient les faire tourner au profit de leurs intérêts ? Que leur importent la misère et l’oppression du peuple ? ils se sont fait un intérêt à part de la cause commune : quel intérêt voulez-vous donc qu’ils prennent à celle-ci ? Des batailles perdues, des flots de sang versés ne les affligeront que parce qu’ils y verront le présage d’un accroissement de contributions ; ils pleureront même sur les succès qui les menaceront d’une pareille chance. Est-ce qu’ils peuvent prendre quelque part à la gloire de leur pays, à la prospérité de leurs concitoyens ? Ne sont-ils pas renfermés dans leurs domaines, dans leurs palais, ou dans leurs coffres forts, comme le rat de la fable dans son fromage de Hollande ?\par
Si c’est là ce qu’on appelle tenir à la société, j’avoue que je ne conçois rien à cette doctrine, et que personne ne me paraît plus dégagé du lien social que ceux qu’on en suppose exclusivement étreints : ce n’est pas que je regarde la propriété comme une chose antisociale ; mais il me semble qu’on a des idées bien fausses à cet égard : quoiqu’on ait beaucoup écrit sur cette matière, on ne l’a guère traitée d’une manière philosophique ; et l’on pourrait presque dire qu’elle est neuve sous ce rapport. Nous allons donc essayer de remonter à ses principes, et d’établir, s’il est possible, le vrai système de la propriété.
\chapterclose


\chapteropen
\chapter[{Chapitre XXVIII. De la propriété.}]{Chapitre XXVIII. De la propriété.}\renewcommand{\leftmark}{Chapitre XXVIII. De la propriété.}


\chaptercont
\noindent Les autres animaux usent des choses comme nous ; elles sont aussi nécessaires à leur existence qu’à la nôtre, et il semblerait que cette nécessité devrait leur donner un droit naturel à ces choses ; cependant nous ne leur reconnaissons pas ce droit, et nous le reconnaissons dans l’homme : il faut donc que l’homme imprime aux choses dont il acquiert la propriété un caractère qu’il n’est pas au pouvoir des autres animaux d’imprimer à rien de ce qu’ils touchent ; car d’où naîtrait, sans cela, la propriété, la légitimité de la propriété, et le respect pour la propriété ?\par
Ce n’est pas le besoin des choses qui constitue en nous la légitimité de leur appropriation ; car ce besoin étant le même dans les autres animaux, y produirait le même effet s’il était de sa nature de le produire ; ce n’est pas non plus la force ou la violence qui peuvent nous rendre propriétaires des choses que nous acquérons par ce moyen ; car les autres animaux l’emploient quand ils sont pressés par des besoins qu’ils ne peuvent pas satisfaire autrement, et pour eux comme pour nous, il n’en résulte que l’usurpation. Serait-ce la première occupation des choses qui en déterminerait la propriété ? Si l’on était propriétaire, parce qu’on serait premier occupant, les animaux, susceptibles de cette priorité, pourraient être propriétaires ; il y aurait violation de leurs droits, quand on les chasserait d’un lieu où ils étaient établis avant nous.\par
Fera-t-on dériver la propriété des conventions sociales ; mais si la propriété n’existait pas indépendamment de ces conventions, comment ces conventions lui donneraient-elles l’existence ? S’il ne suffit pas à un individu de dire : Ceci m’appartient, pour qu’en effet cette chose lui appartienne, comment se pourrait-il que plusieurs individus s’appropriassent légitimement un sol, en disant : Ce sol nous appartient ? Où serait pour les autres l’obligation morale de respecter cette prise de possession qui pourrait s’étendre à tout le globe ? Si l’on pressait les occupateurs d’établir leur droit sur ce terrain, ils n’auraient aucune bonne raison à donner ; ils seraient forcés de dire qu’ils s’en emparent, parce qu’ils s’en empirent, et que si on veut les en exclure, ils en appellent à la force.\par
Le droit de premier occupant est donc un droit chimérique : la propriété des choses suppose qu’on se les est rendu propres, c’est-à-dire qu’on y a mis du sien, qu’on a contribué à les faire ce qu’elles sont, en s’associant à la puissance créatrice, en y ajoutant sa propre coopération ; sans cela, l’usage que nous pourrions en faire, la possession que nous pourrions en prendre, n’en altéreraient point l’indépendance naturelle, et ne nous en conféreraient jamais la propriété : voilà pourquoi les animaux usent de tout, et n’acquièrent la propriété de rien.\par
Il en serait de même de l’homme, s’il se bornait à l’usage déterminé des productions spontanées ; mais il entre, pour ainsi dire, dans le laboratoire de la nature, s’unit à ses travaux, les augmente, les perfectionne, et devient par-là copropriétaire de leur produit ; car il est de moitié avec elle dans l’existence des choses. Ainsi l’homme, à son choix, stérilise ou féconde un terrain ; il y fait croître des épis ou des roses : l’animal, au contraire, broute ce qu’il y trouve, et, s’il n’y trouve rien, il ne peut rien y mettre.\par
Pourquoi cette limitation de l’un à l’usage des choses produites, et cette faculté dans l’autre de contribuer à leur production, et d’en obtenir la propriété ? N’est-ce pas dans la nature nécessitée de celui-là et dans la nature libre de l’autre, qu’il faut chercher la raison de cette différence ? Le premier, soumis en tout à la nécessité, ne peut rien changer à l’ordre dans lequel il existe ; il ne lui est pas permis d’en modifier les lois, tout est déterminé pour lui : son existence elle-même est déterminée, il n’en a point la propriété ; car il vit nécessairement : il n’y a donc pas d’injustice à l’en priver, parce qu’en la lui ôtant, on ne lui ôte pas une chose qui lui appartienne ; et si l’on n’est pas injuste en lui ôtant la vie, à plus forte raison ne l’est-on pas en le privant des autres biens, car rien de tout cela n’est à lui.\par
L’homme, au contraire, à raison de sa liberté, peut changer les déterminations naturelles, et appliquer à tout ses propres déterminations ; il s’ensuit que les choses qu’il a déterminées sont à lui ; car elles lui doivent leur existence, ou du moins la modification de leur existence. Cela s’étend jusqu’à sa vie même, dont l’appropriation, ainsi que nous l’avons développé dans les chapitres précédents, résulte de l’intention sociale qu’il lui prescrit, en la dévouant à ses semblables, en la consacrant à la société ; de là le respect pour la vie de l’homme, et l’injustice à l’en dépouiller ; car, en la lui ôtant, c’est véritablement sa chose qu’on lui ôte.\par
À ce respect pour la personne, s’identifie celui des biens acquis, car c’est la personne qu’on respecte dans les biens. Qu’est-ce, en effet, que le travail auquel la propriété doit son existence ? n’est-ce pas une avance généreuse que l’homme fait à la terre, une mise, pour ainsi dire, de son être, qui s’incorpore, en quelque sorte, avec le sol auquel il se prodigue. En respectant ce sol, c’est donc la personne qu’on respecte. C’est la libre émanation de la personne dans la chose par le travail, qui donne à celle-ci le même caractère qu’à l’autre, et la rend également respectable.\par
Aussi les peuples non cultivateurs n’ont point de sol, à proprement parler ; le terrain qu’ils s’arrogent ne leur appartient pas plus qu’aux autres, pas plus qu’aux animaux qui l’occupent avec eux : leur séjour sur ce terrain, ni les conventions par lesquelles ils s’en déclareraient possesseurs, ne leur en conféreraient jamais la propriété ; il faut qu’ils le cultivent pour en devenir propriétaires. Nul n’est obligé de respecter une terre à laquelle ils ne se sont point communiqués par le travail, et qui n’offre point l’empreinte généreuse des avances qu’ils lui ont faites : elle est toujours dans son indépendance naturelle, parce qu’elle n’a rien à eux ; il faut qu’ils lui donnent pour l’acquérir.\par
Mais, dira-t-on, à quoi bon le travail et l’appropriation des choses ? L’homme ne pourrait-il pas, comme les autres animaux, vivre des productions spontanées ? Sans doute, s’il ne s’agissait pour lui que de vivre ; mais il est dans sa nature de vivre librement, et ce mode d’existence demande l’appropriation de la vie et des choses : voilà pourquoi l’homme ne peut pas, comme les autres animaux, s’en tenir à leur usage déterminé ; c’est de sa liberté que dérive le besoin de les modifier, et non, comme on le croit communément, du désir d’en extraire de nouvelles jouissances, ou de multiplier celles qu’elles peuvent lui procurer naturellement.\par
Où serait, en effet, notre liberté, si nous ne pouvions rien changer à l’existence indépendante des choses ; si nous étions condamnés à les laisser telles qu’elles sont, si l’usage en était absolument déterminé pour nous comme pour les autres créatures ? ne faut-il pas, pour en user librement, c’est-à-dire d’une manière conforme à notre nature, que nous ayons sur elles un empire absolu, que nous puissions contribuer à leur production, et leur faire subir ensuite des modifications indéterminées ? l’appropriation est-elle autre chose que la libre disposition des objets appropriés ? La propriété n’est donc pas naturellement en nous un effet de l’égoïsme ; elle ne le devient que par la corruption de son principe. En lui donnant l’intérêt des jouissances sensuelles pour objet, nous la dénaturons en quelque sorte, nous lui ôtons son principe généreux, pour la rapporter à des motifs étrangers à son essence.\par
Remarquez qu’il n’est presque pas de chose que l’homme emploie à son usage telle que la nature la produit : toutes subissent dans ses mains des préparations et des modifications qui leur donnent une nouvelle existence : pourquoi cette manipulation, ce manufacturage universel ? pourquoi changer la face de la nature ? serait-ce pour vivre ? Mais les animaux vivent très bien sans cela. Le besoin de vivre n’est donc pas en nous le motif de toutes ces sollicitudes : voulez-vous que ce soit pour acquérir de nouvelles jouissances ? certes, si elles n’avaient pas d’autre but, ce serait une bien grande sottise ; car, outre qu’il est fort incertain si elles ajoutent quelque chose à cet égard aux bienfaits spontanés de la nature, il est d’expérience que les trois quarts des hommes ont la peine de ces occupations sans en retirer les avantages ; et, parmi ceux qui jouissent du produit de ces travaux sans les partager, le supplice de l’oisiveté, le dégoût et la satiété de tous ces raffinements, prouvent assez que ce n’est point là la vraie fin de nos labeurs.\par
La nature nous aurait singulièrement surfait la jouissance des choses, s’il fallait que nous l’achetassions par des soins dont les autres animaux seraient dispensés ; il s’ensuivrait qu’ils auraient été beaucoup mieux traités que nous ; qu’ils seraient les pensionnaires de la nature, et que nous n’en serions que les manœuvres : s’il en était ainsi, je ne vois pas qu’il fût si extravagant aux théologiens de supposer l’homme coupable de quelque grand méfait, en punition duquel il aurait été condamné à travailler : il faudrait bien imaginer quelque chose de semblable pour nous rendre raison du lot qui nous serait échu de ne manger notre pain qu’à la sueur de notre front, tandis que les autres n’auraient qu’à se baisser et prendre.\par
Le fait est que ceux-ci ne sont heureux ni malheureux de vivre de cette manière adaptée à leur nature, et que l’homme, en vivant comme eux, sortirait de la sienne. Laissons donc le péché originel pour ce qu’il est ; laissons également les systèmes des philosophes et des économistes qui expliquent la nécessité du travail et de la propriété par le besoin de vivre, et de se procurer des jouissances ; disons, avec plus de raison, que le travail et la propriété sont un besoin de la liberté qui ne peut pas s’en tenir à l’usage déterminé des choses, et qui, par conséquent, est obligée de se les approprier pour les soumettre à ses propres déterminations. Voilà le motif naturel de l’agriculture, des arts, du commerce, de l’industrie, et, tel un mot, de toutes les occupations auxquelles nous nous livrons ; c’est ce motif qui les détermine à notre insu : nous croyons n’obéir qu’à la soif du gain, à l’amour des commodités, à l’espoir des jouissances ; mais ce qui prouve notre erreur à cet égard, c’est que, ce but obtenu, nous n’en sommes pas plus avancés. La même sollicitude nous presse toujours d’agir : il semble que nous n’ayons rien fait ; car nous éprouvons toujours le besoin de faire, et malheur à nous si nous essayons d’en rester là ; nous ne tardons pas à nous apercevoir que l’oisiveté est pour nous un état contre-nature, une position plus fatigante que le travail, parce qu’elle arrête ou suspend l’exercice de notre liberté qui s’alimente du choix et de la variété de nos occupations.\par
Si nous avions des idées justes de ce que nous sommes, nous organiserions l’économie de la vie humaine sur un plan conforme à notre nature : nous regarderions le travail comme un élément nécessaire à notre existence, et non comme une tâche pénible dont on est heureux d’être débarrassé ; c’est l’ambition insensée de l’oisiveté qui corrompt les bienfaits du travail, le rend fatigant, soucieux, inquiet, vil et misérable : il perdrait ce caractère, si nous le rapportions au besoin généreux de notre liberté qui en est le vrai motif, et non à celui de vivre, d’acquérir une plus grande somme de superfluités, ou d’arriver à ce repos absolu qui nous paraît la plus désirable de toutes les conditions, et qui trompe toujours l’espoir de ceux qui l’obtiennent.\par
La liberté est le grand but de la vie humaine : elle est, à proprement parler, la science de l’homme : apprendre à être homme, c’est apprendre à être libre ; mais cette science est à peine soupçonnée, parce qu’on a voulu tout expliquer par la sensibilité : on n’a pas fait attention que cette propriété nous étant commune avec les autres animaux, on ne pouvait lui rapporter les phénomènes particuliers à l’homme ; tels que celui de vivre en société, et d’acquérir la propriété des choses. Dans la plupart des systèmes, quoiqu’on dise d’abord que l’homme est un être libre, on ne tient ensuite aucun compte de sa liberté ; c’est une cause sans effet, car elle n’explique rien, ne motive rien, ne sert à rien, n’est nécessaire à rien ; on en fait, pour ainsi dire, abstraction, et l’on ne s’aperçoit pas combien il est absurde d’expliquer mécaniquement un être qu’on a d’abord supposé libre.
\chapterclose


\chapteropen
\chapter[{Chapitre XXIX. Que le besoin de la société, relativement à la propriété, n’est pas de s’en assurer la garantie, mais d’en acquérir la libre disposition.}]{Chapitre XXIX. Que le besoin de la société, relativement à la propriété, n’est pas de s’en assurer la garantie, mais d’en acquérir la libre disposition.}\renewcommand{\leftmark}{Chapitre XXIX. Que le besoin de la société, relativement à la propriété, n’est pas de s’en assurer la garantie, mais d’en acquérir la libre disposition.}


\chaptercont
\noindent En prouvant que la propriété est un besoin de la liberté, nous avons prouvé qu’elle est naturelle à l’homme : pour user de la vie et des choses d’une manière conforme à sa nature, il faut qu’il se les approprie : il est donc inutile de déclamer contre la propriété ni de chercher à la proscrire, car ce serait proscrire la liberté ; mais, pour remédier aux mauvais effets dont nous voyons qu’elle est la source, il faut qu’elle soit dégagée des motifs d’égoïsme qui la corrompent et l’insocialisent : or il est un préjugé très invétéré, très accrédité, relativement à la propriété, c’est celui de croire que sa conservation ou sa garantie est un des principaux motifs qui déterminent les hommes à vivre en société : il n’y a pas plus de vérité dans cette opinion que dans celle qui fonde la société sur l’intérêt de notre conservation. Nous avons réfuté cette dernière erreur dans les chapitres précédents : il s’agit maintenant de combattre l’autre qui n’en diffère que par son objet, car elle tient au même système, et dérive du même principe.\par
La production des choses par le travail, ou les modifications qu’il leur imprime, ne suffisent pas à leur entière appropriation : pour qu’elles nous soient véritablement propres, il faut que nous puissions en disposer librement : or nous n’en disposerions pas librement, si nous ne pouvions en faire usage que pour nous-mêmes ; car cet usage limité et déterminé n’en constituerait pas la libre disposition, ni par conséquent l’appropriation effective : il faut, pour que celle-ci ait lieu, qu’on puisse faire un emploi généreux des choses acquises ; c’est une condition tellement essentielle à la propriété, que si vous l’en excluez, elle n’existe pas.\par
Cependant où serait hors de la société, je ne dis pas le motif, mais le moyen, la possibilité de disposer généreusement des choses acquises ? Il est évident qu’on ne le pourrait pas, et que par conséquent cet état exclurait l’appropriation des choses, non parce qu’il priverait de leur usage, mais parce qu’il n’en comporterait pas la libre disposition : c’est donc pour celle-ci que la société est absolument nécessaire à la propriété ; le travail la crée en quelque sorte, mais c’est la société qui la virtualise, et qui lui donne le complément de son existence.\par
En effet, vous avez cultivé ce terrain ; après l’avoir défriché, vous y avez planté des choux ou des raves : c’est fort bien, vous avez certainement un droit sur ce terrain ; mais si vous ne pouvez en disposer que pour vous-même ; s’il vous est impossible de le transmettre à d’autres, en êtes-vous propriétaire ? Suffirait-il, pour le devenir, qu’on vous assurât et qu’on vous garantît que vous ne serez jamais troublé dans sa possession ? Non, si la possibilité d’en disposer n’existait pas pour vous, il serait toujours dans la même indépendance de vous-même, et par conséquent vous n’en auriez pas la propriété.\par
Nous n’avons qu’à supposer un homme dans une île où il n’y ait personne, et où personne ne puisse aborder, il aura pour la possession de cette île, toute la sûreté, toute la garantie qu’on peut désirer ; en sera-t-il pour cela propriétaire ? J’aimerais autant qu’on dît que l’huître a la propriété du rocher sur lequel elle est fixée. Pour qu’une chose soit véritablement à nous, le pouvoir de la lâcher est aussi nécessaire que celui de la retenir.\par
Cependant, sans s’en apercevoir, on abstrait communément cette libre disposition des notions qu’on se fait de la propriété ; on n’y fait entrer que la conservation de la chose appropriée : on ne voit pas que, s’il était impossible de s’en dessaisir, on n’en aurait plus la propriété : ce ne seraient pas les choses qui nous appartiendraient, ce serait nous qui appartiendrions aux choses.\par
L’avance de travail que l’homme fait à la terre pour en acquérir la propriété, prouve que celle-ci est généreuse dans son principe ; elle doit l’être également dans sa fin, c’est-à-dire que le but de l’homme, en se prodiguant à la terre n’est pas de s’incorporer dans cette terre, de s’en rendre l’esclave ; mais, en quelque sorte, de l’attirer à lui, et c’est ce que signifie le mot : appropriation. Il faut donc qu’il puisse disposer de cette terre avec autant de liberté que de lui-même ; qu’elle obéisse à toutes ses déterminations, et non qu’elle les commande ; et elle les commanderait, s’il tenait exclusivement à sa possession, s’il ne se sentait pas toujours libre de s’en départir.\par
Voilà précisément à quoi la société est nécessaire, relativement à la propriété : c’est à lui donner ce caractère libéral et généreux qu’il lui serait impossible d’obtenir hors de cet état, et qui lui est tellement essentiel, qu’elle ne peut s’en passer.\par
Il s’en fait bien que les motifs de conservation et de garantie provoquent aussi impérieusement cette institution ; car si l’homme n’avait cherché qu’à se renfermer dans l’usage intéressé des choses, sa meilleure garantie n’était-elle pas dans l’isolement de ses semblables et dans l’état naturel de la terre ? Avait-il à craindre qu’on l’exclût de ce domaine, tant qu’il l’aurait possédé par indivis avec les autres ? n’a-t-il pas, au contraire, créé la chance de l’expropriation, en devenant membre de la société ? ne s’est-il pas soumis lui-même à cette chance, en s’engageant au sacrifice de sa vie, et, à plus forte raison, de ses biens ?\par
C’est que cette aliénation est nécessaire à l’appropriation de l’une et des autres ; c’est qu’elle constitue, en quelque sorte, la mise en possession de la vie et des biens ; c’est que, tant que l’homme n’en a pas fait cette disposition sociale, il n’a point fait acte de propriétaire, il n’est simplement qu’usufruitier : ce n’est qu’en vertu de cette généreuse disposition qu’il peut dire : La vie est à moi, les biens sont à moi. J’en ai disposé comme de ma chose ; je leur ai assigné de mon chef un but, une destination, un emploi que la nature ne leur avait pas donné ; j’en ai fait ce qu’il m’a plu ; j’en suis donc le maître.\par
Nous pouvons tirer de ceci la conséquence qu’il faut être citoyen avant d’être propriétaire, on plutôt qu’on n’est pas citoyen, parce qu’on est propriétaire ; mais qu’on est propriétaire, parce qu’on est citoyen : la première de ces deux qualités est subordonnée à l’autre. En effet, si l’appropriation de la vie et des choses n’a lieu que par l’hommage généreux qu’on en fait à la société, ceux qui n’en ont pas fait cette disposition, ou qui la rétractent intérieurement par un attachement servile à la vie et aux biens, n’en sont réellement pas propriétaires ; ils en prennent le titre, mais ils n’en ont pas la qualité. Leurs biens ne leur appartiennent pas ; ce sont eux qui appartiennent à leurs biens, puisqu’ils s’y sont, pour ainsi dire, inféodés ; ils ont abjuré les devoirs de citoyen pour être propriétaires, et ils ne sont ni propriétaires ni citoyens.\par
Voilà quelle est la position de ces riches qui, consubstantialisés avec leur fortune, et incapables de la sacrifier à la patrie, ne regardent la société que comme une banque d’assurance de la vie et des biens, et s’imaginent qu’elle n’est instituée que pour leur en garantir la conservation. On ne peut pas dire qu’ils sont propriétaires, puisqu’ils tiennent servilement à leurs possessions, et que la faculté de s’en dessaisir est un effort qui n’est pas en leur puissance : on ne peut pas dire non plus qu’ils sont citoyens ; car ils ne mettent ni leur personne ni leurs biens dans la société ; quoique vivant dans son sein, ils en sont isolés par le fait ; ils séparent, autant qu’ils le peuvent, leurs intérêts des siens, se retranchent absolument dans les leurs, et ne font pour les autres que ce que la nécessité les contraint de faire : ils ne tiennent donc pas au corps social ; ils sont hors de la cité, dont on les suppose les membres les plus intéressants. Je ne vois pas pourquoi la considération s’attacherait exclusivement à leur personne, pourquoi l’on s’occuperait tant du ménagement de leurs intérêts qui les absorbe entièrement eux-mêmes, ni enfin quel serait le motif de l’espèce de respect et de vénération qu’on veut que nous ayons pour leur servitude dorée.\par
Mais il est possible que je me trompe, car, d’après nos grands principes de sociabilité, de civilisation et d’économie politique, le soin de notre conservation et la garantie des biens étant le motif déterminant du pacte social et le but essentiel de la société, les bons citoyens sont sans doute ceux qui atteignent le mieux ce but, qui soignent le mieux leur conservation et leur fortune : sous ce rapport, on n’a rien à reprocher aux riches ; ils remplissent le devoir social avec la plus édifiante ponctualité ; et nul, dans ce système, n’a des droits mieux acquis à la considération publique.\par
Ne soyons pas surpris que toutes ces belles théories ne nous donnent qu’une société nominale, une cité sans citoyens, une chose publique qui n’est celle de personne ; car si la conservation de ma vie et de mes biens est le motif qui me détermine à vivre en société, mon intérêt et mon habilité dans cette condition est de faire en sorte que les autres exposent et sacrifient tout cela pour moi, tandis que, de mon côté, je dois chercher à le compromettre et à l’exposer le moins possible pour eux. Chacun faisant à part soi le même raisonnement, et réglant sa conduite en conséquence, vous n’aurez bientôt de la société que le nom, car les intérêts ne vous la donneront pas. Elle ne peut résulter que des devoirs ; et ceux-ci étant négligés ou dédaignés, il n’y a : plus de société.\par
Quand on en est à ce point, il n’est pas étonnant qu’on regarde comme indispensables les moyens coercitifs, car c’est la seule ressource qui reste pour maintenir un ordre factice dans une société factice ; mais ce remède, loin de guérir le mal, ne sert qu’à l’aggraver ; plus on l’emploie, et plus on l’exagère, plus on se met en opposition la nature libérale et généreuse de l’homme : il en est de cela comme de certains médicaments dont on fait usage pour provoquer ou pour accélérer des fonctions naturelles au corps humain : on les donne d’abord à petites doses, pour faciliter l’effet qu’on désire ; bientôt cet effet n’a lieu qu’autant que le remède est administré ; il faut ensuite l’augmenter si l’on veut qu’il réussisse, enfin, la résistance de la nature l’emporte sur l’énergie des moyens dont on se sert pour la provoquer. Vous avez commencé par un grain d’opium pour déterminer le sommeil de votre malade, et maintenant il faut le tuer pour le faire dormir.\par
Voilà ce qui résulte de l’emploi des moyens coercitifs. C’est en les prodiguant qu’on débilite et qu’on tue le corps social : l’expérience a beau prouver qu’ils ne rendent pas les hommes meilleurs, on n’en continue pas moins à leur supposer un pouvoir qu’ils n’ont pas ; et quand une fois on a fondé sur eux l’espoir de la sécurité sociale, les vrais moyens de l’ordre et du bonheur négligés et inactifs dans la nature humaine, sont méconnus et calomniés : on les croit contraires à leur propre but : on s’imagine qu’il n’en peut résulter que le désordre et la confusion. Et, en effet, si l’on regarde la coercition comme le principe de l’ordre, comment ne pas croire que le désordre émane de la liberté ? Si l’on fait de l’égoïsme le système absolu de l’homme, comment lui supposer un principe de générosité ?\par
Tout cela tient aux fausses notions qu’on acquiert des hommes et des choses, dans un état, où tout est également faux, où l’homme n’est pas l’homme, où la société n’est pas la société. C’est parce que les hommes ont été corrompus par ces fausses opinions et par les habitudes serviles qui en sont résultées, qu’il est devenu indispensable d’obtenir, par la crainte ou par la cupidité, ce que dans la rectitude de leur esprit et de leur cœur, ils feraient par amour des devoirs, par amour de leur liberté, par amour d’eux-mêmes.\par
Il n’en est pas de la société comme d’un établissement de finance, dont le bénéfice est dans les produits : c’est dans la mise qu’est celui de la société ; on s’enrichit de ce qu’on y met et non de ce qu’on en retire : ils entendent donc bien mal ce genre de spéculation, ceux qui veulent que la société fasse tout pour eux, sans rien faire pour elle, ou en faisant le moins possible. La politique moderne, en plaçant le bénéfice social dans les intérêts, favorise cette fausse économie de la société qui l’enrichit en apparence, et l’appauvrit dans le fait ; car ce n’est pas l’or, ce n’est pas l’opulence des villes ni la richesse des particuliers qui constitue la richesse sociale ; c’est le dévouement généreux, le sacrifice spontané des intérêts aux devoirs, de la part de tous ceux qui composent la société, qui rend celle-ci puissante, fortunée au-dedans, redoutable au-dehors : c’est un luxe de sentiment généreux et de vertus patriotiques qu’il lui faut, et non un étalage puéril de superfluités, de colifichets et de brillantes niaiseries.\par
Le mal ne vient donc point de ce qu’il y a des riches dans la société ; mais de ce que ces riches ne sont pas citoyens ; car, s’ils l’étaient, la division des propriétés n’aurait aucun des inconvénients qu’on lui reproche.\par
La communauté des biens n’est pas une chose sans exemple : on sait qu’elle a existé dans différents pays : « Les anciennes annales de la Chine, attestent que pendant très longtemps, les habitants y jouissaient de la terre, ainsi que des autres éléments, presque en commun. Le pays était divisé en petits districts égaux, et chaque district était cultivé en commun par huit familles qui composaient chaque village et jouissaient de tout le fruit de leurs labeurs, à l’exception d’une partie qu’on réservait pour les dépenses publiques. Ce ne fut qu’à la suite d’une révolution, dont parlent avec douleur toutes les histoires chinoises antérieures à l’ère chrétienne, que l’usurpateur distribua toutes les terres aux compagnons de ses victoires, allouant seulement aux cultivateurs, une petite portion du revenu\footnote{{\itshape Voyage dans l’intérieur de la Chine}, par lord Macarteney, tome 2, page 372.}. »\par
Il en était à peu près de même chez les Suèves, du temps de César ; les Égyptiens, les Esséniens, chez les Juifs ; les Troglodites, \phantomsection
\label{\_GoBack}les Spartiates ; les Mexicains, avant la découverte de l’Amérique, et plus récemment, les habitants du Paraguai, offrent la preuve historique, que la communauté des biens n’est pas une abstraction métaphysique, et qu’elle peut avoir lieu dans la société ; mais quoique ce mode présente une idée d’égalité et de fraternité, qui séduit d’abord, surtout quand on lui oppose les abus monstrueux dont la cupidité a souillé la propriété individuelle, il est pourtant vrai qu’il est contraire à la nature de l’homme, en ce qu’il gène le développement de sa liberté, dont nous avons établi que l’appropriation des choses est un besoin naturel. Voilà sans doute pourquoi cette communauté si désirable et si belle en apparence, n’a eu lieu qu’à certaines époques, et n’a pu se soutenir chez les peuples même qui l’avaient adoptée.\par
Maintenons donc le principe de la division des propriétés ; mais bannissons en l’égoïsme et la cupidité qui le corrompent et l’insocialisent ; qu’il devienne libéral et généreux, c’est-à-dire, que chacun subordonnant les intérêts aux devoirs, soit toujours prêt à se dévouer sans réserve à la patrie, qu’il ait toujours présente cette obligation, et qu’il fasse consister son bonheur à la remplir. Si cet esprit domine dans la société, vous n’aurez plus besoin de parler de loi agraire, de partage égal, ni de tous ces autres moyens de limiter la propriété, qui, fussent-ils mis à exécution, ne rempliraient point le but qu’on en attendrait et enfanteraient des maux pires que ceux auxquels on voudrait remédier ; car ce serait encore ne s’occuper que des intérêts, dans une institution essentiellement généreuse et où il faut qu’ils soient entièrement oubliés ; c’est précisément alors qu’ils sont le mieux servis.\par
Si l’on était bien pénétré de cette vérité, la propriété respectée d’individu à individu n’en serait pas moins commune dans le tout : il importerait fort peu que les uns eussent plus et les autres moins : les moyens chercheraient les besoins, comme l’eau cherche le niveau, tandis que dans nos systèmes enrichisseurs et insociaux, les besoins courent presque toujours inutilement après les moyens.\par
Aristote combat la communauté des biens que Platon a voulu introduire dans sa république, il donne d’excellentes raisons contre ce système ; mais, en admettant la propriété individuelle, il veut aussi qu’elle remplisse le même but que la communauté des biens : « Organisez, dit-il, vos lois de manière que les biens appartenant aux individus, les produits soient, en quelque sorte, communs. En partageant les soins, vous tarissez la source des querelles. Bien plus, vous donnez plus de ressort à l’industrie, qui veut toujours améliorer. Faites ensuite, que la vertu soit la dispensatrice de ces biens, suivant la belle maxime, que {\itshape tout est commun entre amis}. Ces institutions sont comprises implicitement dans les lois de plusieurs cités, où il ne serait pas impossible de les mettre à exécution. Il y a plus, les gouvernements les plus sages les ont adoptées ou peuvent aisément les mettre en vigueur. Là, chaque citoyen a sa propriété qu’il partage avec ses amis ; mais il use du bien des autres, comme s’il était commun. Ainsi, à Sparte, on se sert d’un esclave étranger comme du sien. Un citoyen est-il surpris par le besoin au milieu des campagnes ? il entre chez le premier venu et lui prend son cheval, ses provisions, et même son chien. Il est donc évident que la plus sage des lois serait celle qui, en consacrant le principe de la propriété individuelle, porterait les citoyens à regarder leurs biens comme communs. Mais comment inspirer aux hommes cette philanthropie ? C’est au législateur à opérer ce prodige\footnote{Voyez {\itshape la Politique d’Aristote}, traduite par le citoyen Champagne, tome I, page 97.}. »
\chapterclose


\chapteropen
\chapter[{Chapitre XXX. Que c’est de l’amour de la liberté qu’on doit attendre la parfaite observation des devoirs, et non des autres moyens par lesquels on a voulu l’obtenir : impuissance de ces moyens démontrée par l’expérience : réponse aux objections qu’on fait contre la liberté.}]{Chapitre XXX. Que c’est de l’amour de la liberté qu’on doit attendre la parfaite observation des devoirs, et non des autres moyens par lesquels on a voulu l’obtenir : impuissance de ces moyens démontrée par l’expérience : réponse aux objections qu’on fait contre la liberté.}\renewcommand{\leftmark}{Chapitre XXX. Que c’est de l’amour de la liberté qu’on doit attendre la parfaite observation des devoirs, et non des autres moyens par lesquels on a voulu l’obtenir : impuissance de ces moyens démontrée par l’expérience : réponse aux objections qu’on fait contre la liberté.}


\chaptercont
\noindent De tout ce que nous avons dit jusqu’ici, nous devons conclure que si la liberté est dans l’homme, le motif naturel de l’imposition et de l’observation des devoirs sociaux, plus il aimera sa liberté, plus il sera porté à remplir ces devoirs : s’il est indifférent pour celle-là, il le sera nécessairement pour les autres, et dès lors il n’est pas étonnant qu’il les viole ou qu’il les néglige, car le motif propre et naturel de leur observation étant chez lui sans énergie, il n’en doit résulter que peu ou point d’effet. L’observation des devoirs, qui est un besoin pour les hommes libres, devient une tâche et une obligation pénible pour ceux qui ne le font pas. Autant les premiers trouvent en eux de puissance et de facilité pour l’exécution des actes les plus héroïques, autant les simples obligations sociales coûtent aux autres.\par
Cependant la société ne peut se maintenir que par les devoirs, et lorsque leur motif naturel est éteint dans ceux qui la composant, elle ne peut subsister qu’en y suppléant par d’autres motifs ; mais ceux-ci n’étant pas le vrai principe des devoirs, les forcent sans les spontanéiser, de manière que l’observation en est coûteuse, difficile, presque toujours imparfaite, le plus souvent en défaut, sans satisfaction pour celui qui en est l’agent, sans séduction et sans entraînement pour les autres.\par
Aussi voyez si tous ces moyens en apparence, si énergiques de menaces et de châtiments civils ou religieux, quelque développement et quelque exagération qu’on ait pu leurs donner, voyez, dis-je, s’ils ont jamais pu déterminer l’observation générale des devoirs sociaux ; si la corruption, la démoralisation et l’incivisme ne se sont pas accrus, en proportion des efforts qu’on a faits pour les réprimer, par ces sortes de moyens.\par
Leur peu de succès est sans doute la meilleure preuve qu’ils ne sont pas adaptés à la fin pour laquelle on les prodigue ; mais il résulte encore de leur emploi un mal qu’on ne soupçonne pas, c’est que loin de ramener l’homme au principe de rectitude qui est dans sa nature, ils l’en écartent ; en l’habituant à motiver par eux l’observation des devoirs, ils lui en interceptent, pour ainsi dire, le vrai motif.\par
En effet, si vous dites à l’homme et si vous lui persuadez, qu’il doit être probe, bon citoyen, fidèle observateur des lois de son pays, parce qu’il s’acquerra par là l’estime et la considération de ses semblables, dont une conduite contraire lui attirerait l’aversion et le mépris ; ou bien parce qu’on le punirait sévèrement, et que s’il échappait au châtiment dans ce monde, des supplices affreux l’attendraient dans une autre vie ; en motivant les devoirs par ces sortes de considérations, vous l’induisez à croire que si elles n’existaient pas, ils seraient sans motif ; que par conséquent ils ne sont pas un besoin de sa nature. Dès lors il ne les rapporte point à leur véritable cause : il ne voit pas qu’étant un être libre avec des penchants déterminés, il est absolument nécessaire à sa liberté qu’il modifie ces penchants, qu’il se les subordonne et que c’est là le but des devoirs qu’il s’impose envers ses semblables : la fin pour laquelle il doit les remplir est en lui-même, et vous la lui faites chercher ailleurs.\par
En faisant ainsi prendre le change à l’homme sur le motif pour lequel il doit observer les devoirs sociaux, vous dépouillez pour lui l’observation de ces devoirs, de son avantage propre et déterminant, c’est-à-dire qu’il n’en retire pas le bien qu’il doit en retirer ; sa liberté ne s’enrichit point de ce qu’il ne fait pas pour elle, et dès lors ces devoirs sont, en quelque sorte, gratuits ; car ils ne remplissent pas le but qu’ils devraient remplir ils ne produisent pas dans celui qui les observe l’effet qu’ils devraient y produire, et comme cet effet est le plus puissant encouragement à les pratiquer, il ne faut pas être surpris que quand il n’a pas lieu, tous les autres moyens soient inefficaces, ou n’obtiennent qu’une observation précaire et incertaine des devoirs.\par
Voulez-vous l’établir sur sa véritable base, rappelez l’amour de la liberté dans l’homme, passionnez-le pour cette propriété privilégiée de sa nature : quand il aimera véritablement sa liberté, il aimera ses devoirs, et c’est la meilleure garantie que vous puissiez avoir de leur parfaite observation.\par
Quoique les principes d’égoïsme, de cupidité, d’abjection et de servitude aient, pour ainsi dire, absorbé la raison dans la théorie et les sentiments dans la pratique, il est pourtant une vérité que tous les systèmes admettent sans se donner la peine d’en déduire les conséquences, parce qu’il leur en faut d’absolument opposées : c’est la nature libre de l’homme ; voilà le point convenu d’où il faut partir pour montrer les fausses routes qu’on a suivies et qu’on s’obstine à suivre encore.\par
Si l’homme est un être libre, sa conservation n’est pas le but essentiel de son existence, ni son premier intérêt ; il s’agit moins pour lui de vivre que de vivre librement, car la vie n’est en lui qu’un moyen dont, la liberté est la fin, en excluant l’intérêt de la liberté vous excluez l’intérêt de la vie.\par
Si l’homme est un être libre, il ne doit pas non plus se coller, pour ainsi dire, aux choses, s’en rendre l’esclave ; mais les soumettre à ses déterminations libres et indépendantes.\par
Vous voyez donc que la liberté est dans l’homme le principe du désintéressement et de la générosité, par conséquent de toute sociabilité, de toute moralité humaine ; or, si elle est la source productive de ces résultats, comment pourrait-on se flatter de les obtenir en la tarissant ; en lui substituant la sensibilité exagérée des intérêts qui met les hommes en opposition, le despotisme qui les comprime, la superstition qui les terrifie ?\par
Quel est le but de tous ces écrivains qui se plaignent amèrement de l’immoralité, de l’égoïsme, du relâchement des devoirs sociaux, et qui tâchent en même temps d’étouffer l’amour de la liberté dans tous les cœurs ; lui imputent le désordre, l’anarchie, en font une espèce de fléau et l’avilissent autant qu’il est en eux, sans s’apercevoir qu’exclure la liberté, c’est exclure la générosité, qu’exclure la générosité, c’est provoquer l’égoïsme dont ils se plaignent, l’immoralité contre laquelle ils déclament, le relâchement des devoirs sociaux qu’ils déplorent ?\par
Peut-être nous apprendront-ils qu’il faut invalider le moyen pour obtenir la fin, étouffer le principe de toute vertu dans les hommes pour les rendre vertueux, renforcer l’attachement aux plus vils intérêts, pour provoquer l’amour des devoirs, tourner le patriotisme en dérision pour faire aimer la patrie, abjurer le titre de citoyen pour mieux tenir à la cité : voilà certes des découvertes auxquelles on ne s’attendait pas, et qui sans doute étaient réservées à notre âge.\par
Mais de quels motifs s’appuie-t-on ordinairement pour proscrire la liberté ; on l’accuse de barbarie, d’inhumanité, d’insensibilité, et parce qu’elle donne à l’homme la force de surmonter les affections privées quand le devoir l’exige, on suppose qu’elle étouffe ces affections ; on ne croit pas qu’elle puisse coexister avec elles.\par
Cependant que trouverez-vous de grand dans l’action de Brutus s’il n’a point des entrailles de père ? Plus vous admettrez qu’il avait de tendresse pour ses enfants, plus il vous paraîtra magnanime : ôtez-lui l’amour paternel, c’est un juge ordinaire qui prononce sur des coupables.\par
Des hommes libres sentiront parfaitement que Brutus a pu aimer ses enfants avec toute l’énergie naturelle à cette affection, et cependant leur donner la mort qu’ils avaient méritée par leur crime, c’est-à-dire, qu’il a pu être à la fois père et citoyen ; mais c’est une chose que vous ne ferez jamais entendre aux autres ; ils vous diront toujours que Brutus n’aimait point ses enfants, que s’il les avait aimés il n’aurait pas pu les condamner : voilà toute leur logique, et il leur est impossible d’en avoir une autre, car leur indifférence pour leur liberté fait qu’ils sont dominés par leurs affections, et ils confondent cette domination avec l’énergie de ces mêmes affections : ils ne savent pas que les passions les plus fortes se développent dans l’homme vraiment libre, sans le dominer, tandis que la moindre fantaisie enchaîne les autres ; que par conséquent l’amour paternel pouvait être bien plus énergique dans Brutus que dans tous ceux qui s’avisent de le condamner ; qu’il est même très présumable qu’il aimait ses enfants mieux que n’aiment les leurs tous ces bons pères qui ne demanderaient qu’à les absoudre quelque coupables qu’ils fussent.\par
Cette impossibilité pour les hommes corrompus par l’esclavage de concevoir l’accord de la liberté et de la sensibilité, est la source de la fausse opinion que des mœurs libres sont des mœurs féroces, et que par conséquent il n’y a d’aménité que dans la servitude.\par
La liberté sans doute a quelque chose de terrible quand elle exige des sacrifices dont l’humanité frémit : un père plongeant le couteau dans le sein de sa fille, pour l’empêcher de devenir la proie d’un décemvir, ne vous paraît qu’un être barbare ; remarquez cependant que l’amour paternel, en versant des larmes sur le tombeau de Virginie, avoue l’inflexible rigueur qui la priva du jour, tandis qu’il repousse la honteuse bassesse de l’esclave qui, pour assurer le sort de sa fille, la prostitue à ses tyrans.\par
Voulez-vous en savoir la raison ? c’est que la vie n’a de prix que par la liberté, et que nous ôter notre liberté, c’est nous ôter le motif de notre existence : il n’est donc pas étonnant qu’une fois dépouillée de ce bien, l’homme libre ne voie en elle qu’un inutile fardeau, qu’une dégradation à laquelle sa nature lui fait un devoir de ne pas consentir, d’où il suit que par amour même pour ses enfants il préférera leur mort à leur esclavage ; mais des hommes indifférents pour leur liberté ne vivent que pour vivre, se soumettent à tout pour exister, et impriment à toutes leurs affections, le même caractère d’impuissance et de servilité : comment s’indigneraient-ils de la dégradation de leurs enfants, lorsqu’ils n’ont pas le courage de s’indigner de leur propre dégradation ?\par
Si quelqu’un doit aimer la vie, c’est sans doute l’homme libre, parce qu’en lui la vie est conforme aux intentions de la nature, elle remplit le but qu’elle doit remplir, elle est ce qu’elle doit être, il est donc impossible qu’elle n’ait pas pour lui plus d’attrait que pour les autres ; cependant l’homme libre quitte la vie à son choix, et l’esclave qui voudrait la quitter ne le peut pas. On s’imagine que c’est par un excès d’amour pour elle, on se trompe ; certainement Socrate qui pouvant éviter la ciguë, l’attendit dans sa prison ; Caton qui pour hâter son dernier moment déchira lui-même sa blessure, aimaient plus la vie qu’un Néron, ni qu’un Vitellius qui la prolongèrent avec tant de lâcheté.\par
Il ne faut donc pas croire que la puissance de tout sacrifier dans les hommes libres, résulte de leur indifférence pour les objets qu’ils sacrifient, ni se figurer qu’en travaillant à se détacher de tout on travaille à acquérir cette puissance ; car on n’est pas libre si l’on n’a pas la force de quitter les choses auxquelles on tient le plus : le propre de la liberté est de tout embrasser dans son amour et de quitter tout ; l’esclavage ne nous attache point aux choses, il nous y enchaîne : on n’aime véritablement que ce qu’on aime librement : quelque forte que soit la passion, il est dans la nature humaine que la liberté soit plus forte encore.\par
Vous voyez qu’il n’implique point que Brutus aimât passionnément ses enfants, et que néanmoins il prononçât leur arrêt de mort : tel homme l’accuse de barbarie et se croit bien meilleur père qui n’a que la servitude de la paternité. Celui qui eut le courage d’immoler ses enfants coupables se fût immolé lui-même pour les sauver innocents ; et je voudrais bien savoir si ceux qui le condamnent et qui se disent si bons pères seraient capables de cet effort. Je crains bien qu’ils ne soient que de mauvais citoyens.\par
En effet, un homme n’a pas le courage de venger durement sur les siens la patrie outragée, la cité trahie ; il sait qu’ils sont coupables, et il cherche à les sauver ; pour l’excuser vous me dites qu’il cède à la tendresse paternelle, à la pitié filiale ; eh bien, placez-le dans des circonstances où ces affections privées exigent de grands sacrifices, et vous verrez s’il ne leur manquera pas, comme il manque aux vertus publiques.\par
N’imaginez donc pas qu’on puisse être bon père, bon fils, bon époux, bon ami quand on est mauvais citoyen ? dans quels pays a-t-on vu les plus rares exemples de tendresse paternelle et de piété filiale ? C’est précisément dans ceux où ces affections ont été le plus souvent sacrifiées au devoir de citoyen : c’est à Rome que Brutus offre ce grand exemple sur ses propres enfants ; que Virginius poignarde sa fille ; qu’un autre dit que si son père revenait sur la terre avec le projet d’attenter à la liberté publique, il le tuerait comme il avait tué César ; eh bien, c’est à Rome que Coriolan prêt à se venger du sénat et du peuple, immole sa vengeance et sa vie aux larmes de sa mère : c’est à Rome que l’énergie du sentiment inspire à une fille éplorée de soutenir la vieillesse défaillante de son malheureux père, en partageant entre son fils et lui cet aliment précieux que la nature dépose dans le sein maternel.\par
Il n’est donc pas vrai que la liberté rende les mœurs féroces : elle subordonne les affections privées à l’amour de la patrie ; mais loin de s’affaiblir de cette subordination, elles n’en deviennent que plus énergiques. L’esclave qui n’a point de patrie à chérir, en sera-t-il plus attaché à sa femme, à ses enfants, à ses amis ? Non, il n’est, à proprement parler, ni mari, ni père, ni citoyen ; il est esclave.\par
J’aime bien l’humanité, la sensibilité de ces peuples qui, pour ne pas être libres, s’en croient plus sensibles, plus humains, plus doux, plus bienveillants : ils ont en effet des mères bien tendres, pourvu que d’autres allaitent leurs enfants ; de bons pères qui enterrent leurs filles dans des cloîtres, des enfants très affectionnés qui rendent à leurs pères des visites d’étiquette ; des époux bien unis dans leurs débauches mutuelles ; des amis chauds et ardents, quand il s’agit d’une partie de plaisir : voilà ce qu’on appelle des mœurs d’une aménité charmante, d’une douceur et d’une sensibilité exquises !\par
La servitude ne donne que l’hypocrisie de la sensibilité ; car, au fonds, elle dessèche et pétrifie le cœur qu’elle stérilise de sentiments généreux. On connaît le trait de ce lâche courtisan dont le fils, atteint d’une flèche mortelle, expire à ses yeux ; il le voit et il applaudit à l’adresse du tyran qui a décoché la flèche homicide ; il lui en fait compliment en disant, qu’Apollon n’aurait pas mieux visé. Eh bien, cet homme, si toutefois on peut lui donner ce nom, avait à coup sûr, toute l’aménité dont on fait si grand cas dans certaines sociétés. Les gens du bon ton de ce temps-là devaient le trouver d’un commerce délicieux.\par
C’est une erreur bien grande et bien funeste, de croire, comme on le fait communément, qu’en se pénétrant d’une belle indifférence pour la patrie, en s’isolant de l’intérêt public pour se renfermer dans le cercle étroit de sa famille, on se renforcera dans l’amour des siens : c’est le plus faux de tous les calculs ; car en vous conduisant de cette manière, vous ne concentrez pas votre sensibilité, vous la neutralisez : tout ce que la patrie gagne d’affection tourne au profit des affections domestiques ; tout ce qu’elle perd, elles le perdent.\par
C’est, au reste, une chose bien facile à concevoir dans nos principes, car, l’indifférence pour la patrie, suppose l’indifférence pour les devoirs sociaux ; l’indifférence pour les devoirs sociaux, suppose qu’on est indifférent pour sa propre liberté ; or, de l’indifférence pour celle-ci résulte, comme nous l’avons déjà vu, la servitude des penchants, et de la servitude des penchants, l’égoïsme qui, sans nous attacher à nous-mêmes, nous détache des autres, et détache les autres de nous.\par
Ne soyons donc pas surpris que l’incivisme produise le relâchement des liens domestiques, car ce sont des nœuds secondaires qui tiennent au nœud principal de la société : en resserrant celui-ci, vous resserrez les autres ; en le relâchant, les autres doivent nécessairement se relâcher.\par
Certes, si l’indifférence pour la chose publique, tournait effectivement à l’avantage des affections privées, rien ne devrait égaler l’union des familles dans les pays où la patrie n’est qu’un mot et la chose publique un objet de gaspillage ou de dérision. Pourquoi donc y trouve-t-on tant de pères égoïstes, tant d’époux libertins, tant d’enfants qui soupirent après la mort de leurs pères, parce qu’il leur tarde d’hériter ? Pourquoi toutes les amitiés s’y réduisent-elles à des protestations de dévouement qui ne trompent plus personne, parce que personne n’est tenté d’y croire ?\par
Voilà ce me semble une preuve de fait que le plus ou le moins d’énergie des affections privées, dépend du plus ou du moins d’énergie et d’intensité des affections publiques. La cité est, en quelque sorte, le foyer générateur des affinités domestiques. Voulez-vous aimer véritablement vos proches et vos amis ? aimez votre patrie : Voulez-vous que votre femme, vos enfants, vos amis, vous aiment d’un amour généreux et désintéressé, apprenez-leur à aimer leur patrie. Voulez-vous apprendre à aimer votre patrie ? aimez votre liberté ; car si vous aimez votre liberté, vous aimerez vos devoirs de citoyen, et si vous aimez vos devoirs de citoyen, vous aimerez votre patrie. L’amour de la patrie et l’amour de la liberté ne sont donc pas deux passions différentes ; ce n’est qu’une même passion ; elle laisse à la sensibilité tout son développement et à la vertu toute sa puissance : avec elle on est passionné ; et cependant on est libre ; aussi voyez si l’on entend les hommes libres plaindre continuellement de la violence ou de la tyrannie des passions ; c’est que cette violence et cette tyrannie n’existent pas pour eux ; c’est qu’ils sont libres dans les passions les plus énergiques, tandis que la moindre fantaisie, comme nous l’avons déjà dit, tyrannise et maîtrise les autres.\par
Ce n’est donc pas à la violence des passions qu’il faut s’en prendre, des désordres qui troublent la société ; mais à cette indifférence pour la liberté, que le vice des institutions, les habitudes serviles qui en sont la conséquence, et les préjugés de toute espèce, concourent à maintenir : au lieu de déclamer contre les passions, de chercher à les contenir par la sévérité des peines ou par l’espoir des récompenses, opposez-leur l’amour de la liberté, stimulez et fortifiez cet amour dans l’homme, et quand vous l’aurez une fois armé de ce sentiment généreux, laissez les passions développer toute leur énergie. Ne voyez-vous pas que ceux qu’elles entraînent, ne succombent que parce qu’ils n’ont pas su mettre leur liberté à la hauteur où il fallait la placer pour empêcher qu’elle ne fût submergée ?
\chapterclose


\chapteropen
\chapter[{Chapitre XXXI. Nouvelle preuve de la fausseté des systèmes, qui font de l’insociabilité, l’état naturel de l’homme.}]{Chapitre XXXI. Nouvelle preuve de la fausseté des systèmes, qui font de l’insociabilité, l’état naturel de l’homme.}\renewcommand{\leftmark}{Chapitre XXXI. Nouvelle preuve de la fausseté des systèmes, qui font de l’insociabilité, l’état naturel de l’homme.}


\chaptercont
\noindent S’il y avait quelque vérité dans toutes les théories qu’on nous a données jusqu’ici, de l’homme et de la société, il s’ensuivrait que l’homme se trouverait dans l’alternative de faire une violence continuelle à sa nature pour rester dans l’état social, ou d’abandonner l’état social pour obéir à sa nature.\par
Or, on conviendra sans doute, qu’il est impossible qu’un être quelconque s’écarte de sa propre nature, à moins qu’il n’y soit forcé ; car, si cette aberration pouvait avoir lieu spontanément, il n’y aurait rien de positif dans la nature des êtres ; ils pourraient n’avoir pas demain celle que vous leur auriez reconnue aujourd’hui ; on ne pourrait plus dire : {\itshape Natura est semper sconsona.}\par
S’il était donc vrai qu’il eût fallu que l’homme fît violence à sa nature, pour entrer dans l’état social, cette violence ne pouvant pas provenir de lui-même, l’état social n’aurait pas été de son choix ; il faudrait admettre qu’il aurait été jeté dans cette condition, par des circonstances forcées, et ces circonstances ne pouvant pas la lui approprier, il n’y aurait resté qu’autant qu’il lui aurait été impossible de revenir à son état naturel ; car c’est là ce que nous observons dans tous les êtres qui, distraits de leur condition naturelle, y rentrent aussitôt qu’il leur est possible d’y rentrer.\par
La permanence de la société serait donc inexplicable dans ce système ; son universalité ne le serait pas moins, car comment concevoir que les hommes eussent été partout distraits de leur condition naturelle, et que des circonstances, pour ainsi dire, exceptives, eussent eu lieu partout, et à toutes les époques. Voilà ce qu’on ne peut pas raisonnablement supposer ; l’insociabilité, d’après ces principes, devrait être la règle générale, et la société, l’exception. Cependant, nous voyons tellement le contraire, qu’on n’a pas un seul exemple d’insociabilité à citer à l’appui du système que nous combattons.\par
Il doit, d’après cela, paraître fort étrange qu’on ait pu imaginer une semblable hypothèse. Par quelle suite de faits ou de raisonnements a-t-on pu être conduit à établir des principes absolument opposés à ceux qu’exige l’ordre des choses existant ? comment n’a-t-on pas senti que la permanence et l’universalité de la société donnaient un démenti continuel à ces principes et aux conséquences qu’il fallait nécessairement en déduire ?\par
Sans doute, ces considérations n’ont point échappé aux auteurs de ces sortes de théories ; mais voici de quelle manière ils ont raisonné : L’homme, ont-ils dit, est déterminé par sa nature à désirer sa conservation, à rechercher le plaisir, à fuir la douleur, et il faut, pour vivre en société, qu’il se soumette à sacrifier sa vie pour les autres, à se priver des plaisirs qui pourraient leur nuire, à souffrir pour leur avantage commun : il n’a donc pas une disposition naturelle à vivre en société ; sa nature l’écarte de ce mode d’existence. Cependant la société est un fait positif ; et comment expliquer ce fait que nous trouvons en opposition avec les dispositions essentielles de la nature humaine ? Il faut nécessairement que des circonstances impérieuses aient déterminé l’homme à faire violence à sa nature pour se soumettre aux lois de la société.\par
Telle a été la conclusion à laquelle ils se sont arrêtés, sans examiner s’il était possible qu’un être quelconque fît violence à sa nature : ils n’ont pas vu que celle de l’homme étant composée, ils n’en saisissaient qu’une moitié, et ne la considéraient, pour ainsi dire, que de profil ; que, dès lors, il leur était impossible de ne pas s’égarer dans les jugements qu’ils en portaient.\par
En effet, leurs observations ne sont tombées que sur ce que nous avons appelé l’être physique, qui, par sa nature, est impérieusement déterminé à désirer sa conservation, à rechercher le plaisir, à fuir la douleur ; mais cet être, quoique dans l’homme, ainsi que nous croyons l’avoir démontré, n’est pas l’homme ; sa nature n’est point, à proprement parler, la nature humaine : c’est celle de l’être moral et libre, qui constitue éminemment celle-ci ; d’où il suit qu’en faisant violence aux inclinations de l’autre, ce n’est pas à sa propre nature que l’homme fait violence ; il la sert au contraire : car conçoit-on que l’être moral pût remplir le vœu naturel de sa liberté, si les penchants de l’être physique ne lui étaient pas subordonnés, s’ils demeuraient absolus, comme ils le sont par leur nature ?\par
La modification de ces penchants n’est donc pas une violence que l’homme se fait à lui-même ; c’est, comme nous l’avons développé dans le cours de cet ouvrage, un besoin de sa liberté. La société, en exigeant cette subordination, n’exige que ce que notre nature elle-même veut et demande impérieusement, et c’est là ce qui établit une convenance parfaite entre la nature de l’homme et les devoirs qui constituent la société. On n’est tombé dans l’erreur à cet égard que parce qu’on a confondu les deux natures qu’il est essentiel de séparer dans l’homme, et qu’on a rapporté à la nature humaine l’éloignement pour la société qui ne provient que de la nature animale. Savez-vous dans quels cas la subordination de l’une à l’autre est effectivement une violence exercée sur nous-mêmes ? C’est lorsqu’elle s’effectue en sens inverse, c’est-à-dire quand nous nous laissons commander par l’attrait du plaisir, par la crainte de la douleur ou par le désir de notre conservation ; quand nous ne nous sentons pas les maîtres d’agir en sens contraire de ces impulsions, et que nous souffrons qu’elles nous dominent absolument ; car alors l’être libre, dont la nature constitue éminemment notre nature, se trouve dans la dépendance de l’être nécessité, d’où résulte un état violent pour nous-mêmes ; et ce qui le prouve, c’est que nous sentons très bien qu’il nous serait avantageux de dominer les inclinations qui nous dominent, et de nous placer dans une telle position, que nous fussions les maîtres de ne leur accorder que ce qu’il nous plairait de leur accorder.\par
Ce besoin naturel de la subordination des penchants une fois admis, la société entre sans effort dans l’économie de la nature humaine, elle en devient même la conséquence nécessaire ; car elle est l’occasion et le moyen de cette subordination. Si notre nature libre demande que nous ne soyons pas dominés par le désir de vivre, la société, en exigeant que nous nous engagions à sacrifier notre vie pour nos semblables, veut ce que veut notre nature, c’est-à-dire que nous dominions le penchant qui nous porte à nous conserver ; elle ne fait que rendre explicite une disposition implicitement renfermée dans la nature humaine.\par
S’il est également dans notre nature libre que nous ne soyons pas commandés par l’amour du plaisir ni par la crainte de la douleur, la société, en exigeant de nous l’engagement de nous priver des plaisirs nuisibles aux autres, et de souffrir, s’il le faut, pour leur avantage, ne veut non plus que ce que veut notre nature, car c’est exiger que nous maîtrisions nos penchants : il y a donc identité de but entre les dispositions essentielles de la nature humaine et les devoirs constitutifs de la société.\par
C’est l’ignorance de celle-là qui a fait croire qu’elle était en opposition avec ceux-ci, et qu’il fallait lui faire violence pour la contraindre à les adopter.\par
Les animaux, à raison de la simplicité de leur nature, seraient dans le cas où l’on suppose l’homme, s’il était possible qu’ils s’imposassent les devoirs que nous nous imposons ; il faudrait, en effet, qu’ils fissent violence à leur propre nature, et aucun être ne pouvant avoir cette faculté, il est clair que celle de s’imposer des devoirs ou de contracter des obligations, est par cela même absolument interdite aux animaux.\par
Il en serait de même de l’homme s’il n’y avait pas en lui une double nature ; la violence qu’il ferait à ses penchants pour s’imposer les devoirs sociaux, serait une violation de sa nature, et c’est là ce qui a trompé ceux qui n’ont point admis sa biduité intérieure ; ils ont été forcés de supposer cette violence comme servant, en quelque sorte, de préliminaire à l’introduction de l’état social parmi les hommes. L’esprit, une fois prévenu de cette fausse opinion, a dû croire que l’état d’insociabilité n’exigeant aucune coercibilité des penchants, était l’état naturel de l’homme, et que la société ne pouvant se passer de leur subordination, était opposée à sa nature ; de là la recherche des motifs qui avaient pu déterminer l’homme à quitter ce prétendu état naturel, pour en adopter un auquel sa nature se refusait. Les hypothèses et les conjectures se sont multipliées sur cet objet, sans qu’on se soit encore avisé d’en mettre les principes en question, c’est-à-dire, d’examiner si la subordination des penchants n’est pas effectivement selon la nature de l’homme, et si le besoin de cette subordination ne constitue pas en lui le besoin de la société.\par
D’où vient cependant que les actes de courage, de générosité, d’héroïsme produisent en nous ce charme, ce ravissement, cette généreuse émulation de les imiter ? N’est-ce pas parce qu’ils accusent dans leur agent une subordination absolue des penchants, un empire positif sur ces inclinations déterminées ; et, s’il n’était pas dans la pâture humaine de se les subordonner, si c’était, comme on le prétend, une violence exercée sur cette même nature, conçoit-on qu’elle pût se complaire dans cette violence, et y trouver sa propre satisfaction ?\par
Pourquoi d’un autre côté les actes de lâcheté, d’égoïsme, de bassesse et de servilité nous indignent-ils, nous révoltent-ils ? N’est-ce pas parce qu’ils annoncent que les penchants dominent celui qui en est l’agent, et que cette domination répugne à notre nature morale, à peu près comme une odeur fétide répugne à notre nature physique ?\par
Il est donc inutile de chercher le motif des obligations sociales ailleurs, que dans ce besoin propre et naturel à l’homme de se subordonner ses penchants pour être libre ; car le but de toutes ces obligations n’étant en dernière analyse que cette subordination, elles contribuent à nous faire obtenir la fin vers laquelle il est de notre nature morale de tendre constamment, d’où il suit qu’elles ont pour l’homme un intérêt qui les lui fait rechercher pour elles-mêmes, et non à cause des prétendus avantages sociaux par lesquels on veut qu’il soit déterminé à se les imposer.\par
La nature mixte de l’homme nous donne la solution de toutes ces difficultés, qui dans le système de l’unité sont absolument inexplicables, ou dont l’explication ne présente qu’un amas d’inconséquences, de contradictions et d’absurdités plus révoltantes les unes que les autres ; telles que d’admettre que quoique l’homme vive partout en société, cependant l’insociabilité est son état naturel, et que quoiqu’il ne soit au pouvoir d’aucun être de faire violence à sa nature, il a fallu que l’homme fît violence à la sienne pour entrer dans l’état social.
\chapterclose


\chapteropen
\chapter[{Chapitre XXXII. Impossibilité de concilier les devoirs sociaux avec l’amour de soi dans les autres systèmes : convenance de ces devoirs avec la théorie que nous avons donnée de l’amour de nous-mêmes.}]{Chapitre XXXII. Impossibilité de concilier les devoirs sociaux avec l’amour de soi dans les autres systèmes : convenance de ces devoirs avec la théorie que nous avons donnée de l’amour de nous-mêmes.}\renewcommand{\leftmark}{Chapitre XXXII. Impossibilité de concilier les devoirs sociaux avec l’amour de soi dans les autres systèmes : convenance de ces devoirs avec la théorie que nous avons donnée de l’amour de nous-mêmes.}


\chaptercont
\noindent L’amour de soi, tel qu’on le définit dans les systèmes de nos philosophes et de nos publicistes, est évidemment en contradiction avec les dispositions qu’exige la société, car c’est l’amour de notre conservation, l’amour du plaisir, l’antipathie pour la douleur, et la société ne pouvant exister qu’à condition que chacun de ses membres sera toujours prêt à sacrifier sa vie pour les autres, et à se priver des plaisirs qui pourraient leur nuire, demande nécessairement une suite de déterminations et d’actions absolument opposées à celles qu’exigerait ce prétendu amour de soi : comment se peut-il donc qu’une institution dans laquelle il faut renoncer à s’aimer soi-même, ait pu se généraliser et s’universaliser parmi les hommes ? comment se peut-il qu’ils aient tous adopté une pareille condition ?\par
Voilà certes le phénomène le plus étrange et le plus incompréhensible ; cependant il faut admettre cette contradiction entre les devoirs sociaux et l’amour de soi, ou convenir que ce qu’on a pris pour l’amour de nous-mêmes n’est pas effectivement l’amour de nous-mêmes ; c’est ce que nous avons essayé d’établir, en traitant de cette passion que nous avons rapportée à l’amour de notre liberté : dès lors, il n’y a plus ni contradiction ni opposition entre l’amour de soi et les dispositions dans lesquelles il faut que nous soyons pour vivre en société ; et c’est, à ce qu’il nous semble, une preuve démonstrative de la fausseté des autres systèmes et de la vérité de celui que nous tâchons de leur substituer.\par
Pour rendre ceci plus sensible, nous n’avons qu’à confronter, pour ainsi dire, notre théorie de l’amour de soi avec les conditions qu’exige impérieusement l’existence de la société, et nous verrons que ces deux choses, loin de se repousser et de s’exclure, comme elles le font dans les autres systèmes, s’appellent en quelque sorte, s’adaptent l’une à l’autre sans effort, s’éclairent réciproquement, et se servent mutuellement de preuve et de confirmation.\par
En effet, l’amour de nous-mêmes étant l’amour de notre liberté, exige la subordination de nos penchants, c’est-à-dire la disposition constante à nous les sacrifier ; car ce n’est pas l’existence de ces penchants qui est contraire à notre liberté, c’est leur domination ; et il est clair que, quand nous sommes disposés à nous les sacrifier, ils ne nous dominent pas ; c’est nous, au contraire, qui les dominons.\par
« Cette disposition, avons-nous dit\footnote{Chapitre XI, page 55.}, suffit à l’amour de soi, comme la disposition à tout donner suffit à notre amour pour les autres. On sait que l’amour tient compte de l’intention comme du fait. Si j’ai la certitude que tout ce qui est à mon ami est à moi, et qu’il est prêt à m’en faire le sacrifice, cette disposition équivaut, dans nos rapports mutuels, au sacrifice reçu : il n’est pas besoin qu’il soit consommé. Je passerais toute ma vie sans emprunter un écu à cet ami, qu’il n’en aurait pas moins le mérite de s’être dépouillé pour moi. Notre amour pour nous-mêmes tire aussi toute son énergie de la disposition où nous sommes à nous sacrifier nos penchants ; car, du reste, il faut que les occasions déterminent ce sacrifice. »\par
Voyons maintenant si la société s’approprie ou ne s’approprie pas à cette doctrine. La disposition à nous sacrifier nos penchants étant nécessaire à l’amour de nous-mêmes, rien n’est plus propre à nous mettre dans cette disposition, à nous faire un besoin positif de l’acquérir, que la contingence des circonstances qui peuvent exiger le sacrifice effectif de nos penchants. Or la société crée cette contingence, et par là elle concourt à nous mettre dans la disposition où il faut que nous soyons pour nous aimer nous-mêmes. Cette disposition est la seule condition positive des obligations sociales ; car leur acquit est éventuel, surtout celui de sacrifier sa vie ; ce sacrifice est subordonné à des circonstances qui peuvent ne se présenter jamais, et qui rarement ont lieu pour le plus grand nombre.\par
Si la société n’est jamais troublée ; si elle ne se trouve jamais en péril, l’obligation de mourir pour sa défense reste sans effet pour ceux qui la composent : il en est de même de l’obligation de supporter pour elle des privations et des souffrances ; elle reste également sans effet, tant que les circonstances de l’état social n’en exigent point l’acquit. On voit donc que la société n’exclut pas le ménagement des intérêts ; elle ne demande pas l’abnégation des penchants ; mais elle veut qu’ils nous soient subordonnés, qu’ils soit toujours en notre pouvoir de nous les sacrifier, et en cela elle veut ce que veut l’amour de nous-mêmes ; il y a par conséquent une convenance parfaite entre son système et celui de notre amour pour nous : on croit communément que l’homme ne la soutient que pour se lier avec les autres ; mais c’est qu’il en a besoin pour se lier avec soi ; en s’isolant de ses semblables, il s’isolerait de lui-même ; la dissolution de l’état social entraînerait la dissolution de l’homme. Il s’ensuit, qu’aimer les devoirs sociaux, c’est s’aimer soi-même, parce que c’est aimer sa propre liberté. Aussi rien, ne peut détacher de ces sortes de devoirs, celui qui a la conscience de l’effet qui en résulte dans son économie intérieure. On peut le bannir du sol ; mais on ne le bannit pas de la cité. Il l’emporte avec lui dans l’exil, dans les fers et jusques dans les supplices. Il sent qu’il a besoin de ce dévouement pour être libre, et qu’il le sera tant qu’il lui restera fidèle : s’il cessait un moment de l’éprouver, c’est alors qu’il se croirait vaincu par ses persécuteurs, et il n’est pas assez ennemi de lui-même pour leur accorder un semblable triomphe.\par
« Les dieux m’enlèveront tout, dit Brutus à Cicéron, plutôt que de me faire consentir à l’oppression de ma patrie : je ne souffrirais pas même de mon père, s’il revivait, qu’il eût dans Rome plus de pouvoir que le sénat et les lois : croiriez-vous nous avoir sauvés en obtenant d’Octave, notre retour dans cette ville ? Qu’importe qu’il nous laissât la vie, quand nous lui aurions sacrifié notre liberté. Je me croirai dans Rome, partout où il me sera permis d’être libre, partout où la servitude me paraîtra le pire de tous les maux. N’attendez pas de moi que je devienne le complice de ceux qui se soumettent à leur propre dégradation. Je souffrirai tout, je tenterai tout pour arracher notre patrie à l’esclavage. Si la fortune seconde mes efforts, nous en jouirons tous ; si j’échoue, j’aurai fait mon devoir ; j’aurai préféré ma liberté à tout, et la conscience qui m’en restera, est pour moi, d’un prix supérieur à tous les biens de la vie. »\par
Voilà quel est l’attachement que l’amour de la liberté donne pour les devoirs. Quand les hommes sauront que leur destination est d’être libres, qu’ils ne peuvent être heureux qu’en la remplissant, et qu’ils la remplissent en remplissant leurs devoirs ; quand ils seront bien convaincus que, tenir à ses devoirs, c’est tenir à sa liberté, et que tenir à sa liberté, c’est tenir à soi-même ; l’ordre et la paix régneront dans la société, sans effort, sans contrainte ; chacun trouvant la fin de la société dans les devoirs, n’imaginera pas, comme dans le système des intérêts, qu’elle n’est le partage que de quelques-uns : il n’en voudra point à ceux-là, ceux-là n’en voudront point aux autres. Un but que tout le monde peut atteindre, n’excite point l’envie ; et tel doit être celui de la société : croire se l’approprier exclusivement, c’est le manquer ; se sentir toujours le maître de l’obtenir, c’est l’atteindre. C’est par le système généreux de la liberté, que nous sortirons de l’état violent dans lequel nous ont jetés les fausses doctrines sur l’intérêt personnel ; c’est par lui que nous rentrerons dans l’ordre naturel de la société, dont tout le monde est appelé à recueillir les avantages, et qui, dans l’état actuel de l’opinion et des sentiments, sont perdus pour tout le monde.
\chapterclose


\chapteropen
\chapter[{Chapitre XXXIII. De la guerre et de la philanthropie.}]{Chapitre XXXIII. De la guerre et de la philanthropie.}\renewcommand{\leftmark}{Chapitre XXXIII. De la guerre et de la philanthropie.}


\chaptercont
\noindent Nous croyons avoir démontré jusqu’à l’évidence, que l’état social est l’état naturel de l’homme ; mais il est impossible que le genre humain ne forme qu’une seule société ; il faut donc qu’il se divise en sociétés particulières, et cela, pour l’intérêt même des obligations sociales.\par
En effet, on ne s’obligerait à rien, en s’obligeant à se dévouer pour le genre humain, à souffrir pour le genre humain, à mourir pour le genre humain ; car n’étant pas dans l’ordre des choses possibles qu’aucune chance pût exiger qu’on remplît cette obligation, il est clair qu’elle serait purement nominale. Or, nous avons vu qu’elle est un besoin de l’homme, parce qu’elle est absolument nécessaire à sa liberté, d’où il suit qu’il faut qu’elle soit effective, et pour qu’elle soit effective, il faut la contingence des circonstances qui puissent en déterminer l’acquit.\par
La paix perpétuelle entre les différentes sociétés, produirait le même effet que la réunion de l’espèce humaine, en une seule famille ; c’est-à-dire, qu’elle anéantirait aussi l’éventualité de l’acquit des obligations sociales, et par conséquent elle les rendrait illusoires.\par
Si une paix parfaite et inaltérable pouvait s’établir entre les nations ; si elles n’avaient à craindre ni guerres étrangères ni guerres intestines, leur sécurité serait absolue, et dès lors, rien ne motiverait de la part des citoyens, l’obligation de s’exposer et de mourir pour elles : en supprimant toutes les chances de la guerre, on supprimerait aussi cette obligation, et l’homme ne pouvant être libre que par elle, en l’en dépouillant, on le dépouillerait de sa liberté : sa nature exige donc qu’il soit maintenu dans le devoir de s’exposer et de mourir s’il le faut, pour sa patrie. Cette chance entre essentiellement dans la raison formelle et positive de son existence, et l’on sent qu’il faut, pour qu’elle ait lieu, que la société à laquelle on appartient, puisse être compromise. Or, elle ne peut l’être que par une violence exercée sur elle de la part des autres sociétés. De là, la nécessité naturelle de la guerre. Quelque soin qu’on prenne de stabiliser la paix, elle ne peut être qu’une trêve plus ou moins longue ; ainsi le veut la nature des choses.\par
Au reste, la guerre ne nous paraît un si grand mal, que parce que nous regardons la vie comme le premier des biens ; mais si nous faisons attention que notre premier intérêt est d’être libres, et que la contingence de la guerre est nécessaire à notre liberté, ou il faudra que la liberté soit un mal, ou l’on sera forcé de convenir que la guerre est un bien.\par
C’est peut-être dans cette nécessité naturelle de la guerre entre les nations, pour le maintien et l’efficacité des devoirs sociaux, qu’il faut chercher la cause des guerres continuelles que les peuples non policés ont entre eux ; car la démarcation entre ces peuples n’étant fixée par aucune convention positive, ni par aucune forme déterminée de gouvernement, peut-être ont-ils un besoin constant d’être en guerre pour conserver la conscience de leur moi politique, et s’empêcher de se fondre les uns dans les autres.\par
C’est la seule explication qu’il me semble qu’on puisse donner de l’état toujours hostile de ces peuples ; car n’ayant ni propriété, ni commerce, ni sol à proprement parler ; n’étant guidés ni par l’espoir du butin, ni par celui des conquêtes, ils ne font la guerre que pour la guerre même ; elle n’est pourtant pas un effet sans cause, et puisque nous ne trouvons pas cette cause dans les circonstances physiques de ces peuples, il faut bien la chercher dans leurs circonstances morales.\par
À ne considérer les choses que d’après les notions communes, la guerre devrait être plus fréquente chez les peuples civilisés que chez les peuples sauvages ; cependant ceux-ci sont toujours en guerre, et les autres jouissent de longs intervalles de paix ; on en fait honneur à la prétendue férocité de ceux-là et à la civilité polie de ceux-ci ; mais je crains bien que la guerre ne soit un besoin de la liberté chez les premiers, et la paix un besoin de l’esclavage chez les autres.\par
Si la liberté se perd dans les grandes nations, c’est peut-être parce que la guerre même n’en altère pas la sécurité. Des hommes gagés se battent sur les frontières, et la masse des citoyens, tranquille spectatrice dans l’intérieur, ne prend qu’un intérêt de curiosité aux événements de la guerre : il en résulte que la chance d’exposer sa vie pour le salut commun, est un de ces événements qui dans le cours naturel des choses n’existe pas pour le plus grand nombre. Dès lors on ne se prépare point à ce qui ne doit jamais arriver : on ne se met pas dans la disposition que motiverait la probabilité de ce dévouement effectif : on laisse oblitérer en soi le sentiment de cette obligation sociale, et en perdant le sentiment de cette obligation, on perd celui de sa liberté.\par
Voilà l’effet inévitable de la sécurité dont jouissent les grandes nations : c’est là qu’il importerait surtout que tous les jeunes gens sans distinction reçussent une éducation militaire, que les fêtes, les jeux, les institutions tendissent à rappeler sans cesse aux citoyens l’obligation de se dévouer à la patrie, pour laquelle tout conspire à les rendre indifférents. La politique ne devrait être occupée qu’à surmonter cet inconvénient naturel aux grandes sociétés ; mais elle trouve plus commode de se prêter à son influence ou même de la renforcer : elle donne des armes à quelques-uns et favorise l’inclination des autres à rester non armés ; ensuite avec les premiers, elle enchaîne ceux-ci qui ne demandent qu’à recevoir des fers, pourvu qu’on les laisse vivre ; ils paient de leur liberté cette misérable existence, et croient avoir fait encore un excellent marché.\par
Il est dans la nature de l’homme de s’attacher aux objets de sa munificence : il tient à la société, parce qu’il fait pour elle et non parce qu’elle fait pour lui. Il faut que la patrie ait besoin de nous, pour que nous ayons besoin d’elle : c’est parce qu’on s’est arrangé de manière à se passer du concours de tous les citoyens pour sa défense, que ceux-ci ont fini par n’être plus citoyens ; aussi ont-ils une merveilleuse facilité à quitter la société dans laquelle ils vivent, lorsqu’ils n’y font pas ce qu’ils appellent leurs affaires, preuve certaine que les affaires de la société ne sont pas les leurs, car ils ne pourraient les faire que dans son sein ; mais ce qu’ils sont, ils peuvent l’être partout ; partout ils peuvent être commerçants, propriétaires, artistes, manufacturiers ; ce cosmopolitisme inconnu aux peuples, où quand la patrie est menacée, tout marche à l’ennemi, démontre évidemment qu’en neutralisant les devoirs au profit des intérêts, on neutralise l’affection pour la cité, on détruit le lien social qui se relâche en raison du relâchement des devoirs.\par
Nous pouvons maintenant apprécier à leur juste valeur les déclamations de la philosophie moderne contre la guerre : elle est un mal sans doute, mais c’est quand les intérêts en sont le but, quand il ne s’agit, comme dans la plupart des guerres qui ont lieu entre les peuples civilisés, que d’acquérir quelque portion de terrain, de donner plus d’étendue au commerce ou de satisfaire l’ambition d’un despote ; mais elle est un bien et un très grand bien lorsqu’elle est véritablement nationale ; c’est-à-dire, lorsqu’elle tourne au profit des devoirs, lorsqu’elle rappelle à tous les citoyens l’obligation de les remplir, et qu’elle renforce en eux le sentiment de cette obligation.\par
Par un autre préjugé qui n’est que la conséquence de celui qui regarde la guerre comme un mal absolu, les philosophes se sont attachés à prêcher la philanthropie ou l’amour de tous les hommes en général : ils n’ont pas vu que c’était prêcher l’insociabilité ; car par cela même que la nature a voulu que l’espèce humaine fût divisée en sociétés particulières, elle a voulu aussi que les hommes circonscrivissent leurs affections, dans leurs sociétés respectives.\par
Si cet amour exalté du genre humain pouvait avoir lieu ; si d’un bout de la terre à l’autre tous les hommes pouvaient se regarder comme frères, et ne former, suivant l’expression vulgaire, qu’une seule famille, la paix universelle serait la conséquence de cette fraternité. Rien ne troublerait la profonde tranquillité des différentes sociétés, et nous avons vu que cette sécurité absolue anéantirait les devoirs sociaux dont l’existence est essentielle à la nature humaine.\par
Il est donc dans la nature des choses qu’on tienne exclusivement à ses concitoyens ; nous osons même dire que c’est dans cette préférence exclusive que gît le vrai principe de l’humanité, et non dans cet amour abstrait du genre humain, qui n’est, dans le fait, qu’une indifférence systématique. Votre amour exclusif pour vos concitoyens vous empêchera-t-il de bien accueillir l’étranger, de lui accorder l’hospitalité, le soulagement dans le malheur ? au contraire, ce sentiment vous fournira le motif propre et déterminant de cette généreuse conduite ; car vous penserez que vos concitoyens peuvent se trouver individuellement étrangers dans les autres pays, et plus ils vous seront chers, plus cette considération vous rendra généreux, humain, bienfaisant et hospitalier envers les autres.\par
L’expérience au reste confirme ce que nous avançons ici ; les peuples chez lesquels les étrangers sont le mieux accueillis, sont précisément ceux qu’on peut le moins soupçonner de philanthropie ; on sait avec quelle bienveillance et quelle humanité les Européens furent reçus par les Américains qu’ils traitèrent ensuite avec tant de barbarie ; cependant ces Américains, n’étaient pas à coup sûr des philanthropes : l’amour du genre humain n’était pas en eux le principe déterminant du bon accueil qu’ils firent à leurs frères d’Europe, dont ils n’avaient jamais entendu parler : on peut en dire autant des Otahiticiens, des Kamchadales, des Ostiaques, et en un mot de tous les peuples qui, n’ayant aucune idée des autres nations, ne peuvent tenir qu’à la leur, et doivent nécessairement concentrer toutes leurs affections. Lisez les relations des voyageurs et vous verrez combien ils sont humains et hospitaliers.\par
Il s’en faut bien qu’on trouve les mêmes qualités chez nous, qui ne parlons sans cesse que de l’intérêt du genre humain et de l’amour de tous les hommes en général : je ne vois pas trop que cette passion que nous affectons pour la masse de l’espèce humaine profite beaucoup aux individus, ni que notre non-prédilection pour nos concitoyens tourne infiniment à l’avantage des étrangers. Je croirais plutôt qu’elle nous rend étrangers les premiers, sans nous donner plus d’inclination pour les autres.\par
Cette prétendue philanthropie est donc bien moins amie des hommes qu’on ne le pense. L’humanité, ainsi que nous l’avons déjà dit, a son principe dans le patriotisme ; prêcher le patriotisme c’est prêcher l’humanité. L’espèce humaine ne s’enrichit-elle pas de tout ce que les hommes font pour leurs sociétés respectives ? Sparte a-t-elle seule profité de la journée des Thermopiles ? L’exemple mémorable de ce dévouement héroïque, n’est-il pas devenu le patrimoine de toutes les nations ? Ne voyez-vous pas que l’amour de la patrie est le stimulant le plus actif qu’on puisse donner à la pratique des vertus et à la recherche des vérités utiles aux hommes ? Au lieu des trois cents républicains qui se dévouèrent aux Thermopiles, supposez trois cents philanthropes, et cette belle leçon de courage et de patriotisme sera perdue pour l’humanité.\par
Qu’Archimède, soit un vrai philanthrope, un citoyen du monde, il fera dans son cabinet de belles exhortations au genre humain, il invitera tous les hommes à vivre en frères ; mais songera-t-il à brûler les vaisseaux ennemis dans le port de Syracuse ? N’est-ce pas à son patriotisme que l’espèce humaine fut redevable de la découverte du miroir ardent ?\par
Certes, il n’est pas seulement de l’intérêt de chaque société que les membres qui lui appartiennent lui soient exclusivement dévoués, c’est l’intérêt général de l’espèce humaine. Le monde se compose de nations, les nations se composent d’individus, les individus sont dans les nations, et les nations dans le monde : tel est l’ordre universel ; il s’ensuit que les individus sont des parties dont la nation est le tout, et l’excellence du tout consiste dans l’adhérence de ses parties. Tenir à sa patrie, c’est tenir au genre humain. Les devoirs du citoyen, sont les devoirs de l’homme, parce que l’homme naît citoyen.\par
La vraie philanthropie est donc l’opposé de la fraternité universelle : celle-ci n’est qu’une débauche de l’imagination, un libertinage insocial. L’avantage des nations nous le répétons encore, n’est pas que les membres des unes, fraternisent avec ceux des autres ; mais qu’au contraire ils soient tous exclusivement attachés à la nation dont ils sont membres. Et n’est-ce pas là ce que comprend implicitement le vœu de la liberté des peuples ? Il y aurait une très grande inconséquence à voter cette liberté, si l’on voulait que la fraternité entre ces peuples en fût le résultat ; car plus ils seront libres, plus ils tiendront à leur moi politique respectif ; et plus ils tiendront à leur moi politique respectif, plus ils concentreront leurs affections dans leurs sociétés particulières : rien ne serait donc moins philanthropique que de voter la liberté des peuples, si le patriotisme exclusif n’était pas philanthropique.\par
Les peuples, sans liberté, sans patriotisme, soumis à des gouvernements absolus, sont, dans leurs rapports aux autres nations, ce que de mauvais pères, de mauvais époux, des enfants vicieux, sont dans la société ; c’est-à-dire, des individus immoraux, dont l’espèce est intéressée à détruire le scandale ; et, lorsqu’en généralisant les idées, on aura fait quelques progrès dans la morale du genre humain, il est probable qu’on ne souffrira plus de peuples esclaves.
\chapterclose


\chapteropen
\chapter[{Chapitre XXXIV. De l’établissement des lois positives.}]{Chapitre XXXIV. De l’établissement des lois positives.}\renewcommand{\leftmark}{Chapitre XXXIV. De l’établissement des lois positives.}


\chaptercont
\noindent « Les hommes, dit Condillac, ignorent ce qu’ils peuvent, tant que l’expérience ne leur a pas fait remarquer ce qu’ils font d’après la nature seule ; c’est pourquoi ils n’ont jamais fait avec dessein que des choses qu’ils avaient déjà faites sans avoir eu le projet de les faire. Je crois que cette observation se confirmera toujours ; et je crois encore que, si elle n’avait pas échappé, on raisonnerait mieux qu’on ne fait. »\par
Nous pensons, comme ce célèbre métaphysicien, que ce principe est d’une application générale\footnote{Condillac réfute ici lui-même la proposition par laquelle, dans son {\itshape Traité des Animaux}, il veut qu’on ne fasse par habitude que ce qu’on a d’abord fait avec réflexion ; car, s’il faut que l’expérience fasse remarquer aux hommes ce qu’ils font d’après la nature seule, il est certain que ces opérations non remarquées sont des opérations non réfléchies, et que, puisqu’elles précèdent la réflexion dans l’homme, elles ne la supposent pas dans les animaux. C’est la propriété que nous avons de nous identifier à tout, de nous substituer à tout, et d’exister, pour ainsi dire, en tout, qui nous porte à joindre la réflexion à l’instinct dans les animaux, parce que nous l’y trouvons jointe en nous-mêmes : sans nous en apercevoir, c’est nous-mêmes que nous analysons en croyant analyser les animaux.} : en effet, nos sens s’exercent avant que nous sachions disposer de leur exercice. Nous apprenons à sentir, quoique nous soyons naturellement sentants : il faut également que nous apprenions à être libres et sociaux, quoique nous soyons naturellement l’un et l’autre. Il n’a donc point existé de temps où l’homme ne fut pas un être libre ; il n’a point existé de temps où l’homme ne fut pas un être social ; il n’a point existé de temps où l’homme n’exerçât pas ces deux qualités, mais il les exerçait sans savoir qu’il les exerçait ; à peu près comme l’enfant exerce les sens de la vue et du toucher ayant de savoir qu’il en est pourvu, avant de se former l’idée des sensations qui en résultent : c’est en les répétant, qu’il en acquiert l’art et la science ; il en est de même de la liberté et de la sociabilité.\par
L’application de la liberté à un acte quelconque n’a pas suffi pour donner à l’homme la conscience de sa liberté ; il a fallu que cette application se répétât sur un grand nombre d’actes différents, pour qu’il pût en conclure qu’il était libre : il exerçait donc sa liberté avant de savoir qu’il l’exerçait. La sociabilité, par la même raison, a dû s’exercer antérieurement à tout système d’association. Le projet de vivre en société n’a pu naître que de la société déjà formée ; car d’où aurait-on tiré la connaissance anticipée de l’état social ? Serait-ce une idée innée ? Depuis longtemps, on ne croit plus à ce système : il a donc bien fallu qu’on apprît de l’expérience qu’on était susceptible de l’état social ; et, pour qu’on l’ait appris de l’expérience, il a fallu que cette condition existât avant qu’on imaginât de la faire exister : elle était en fait depuis longtemps, quand on s’est avisé de la mettre en principe.\par
Le premier qui circonscrivit un terrain dans une espèce de palissade, ne dit pas : ceci est à moi, comme le prétend Rousseau ; il ne prévit pas que la propriété serait la conséquence d’une action à laquelle il attachait ni ne pouvait attacher cette idée : il ne songeait sans doute qu’à se faire une sorte d’abri, contre les injures de l’air, ou contre les incursions de quelques animaux incommodes ou dangereux. Ce ne fut que lorsque, à son imitation, les autres eurent fait de semblables clôtures, et qu’on vit qu’elles étaient mutuellement respectées, que ce qui était en fait depuis longtemps fut érigé en droit. Voilà la propriété reconnue ; mais on en ignore encore les différentes modifications : il faut que l’expérience apprenne qu’elle est susceptible de mutation, d’aliénation, d’échange, de transmission héréditaire. Ayant qu’on en vienne à faire avec dessein, ces sortes de transactions, il faut, comme le dit Condillac, qu’on les ait déjà faites sans avoir eu le projet de les faire.\par
Une cabane abandonnée reçoit un nouveau propriétaire, qui, soit inconstance, soit tout autre motif, vient l’habiter ; celle qu’il a quittée, un autre la prend : tous deux satisfaits de leur nouveau domicile, ne songent nullement à s’inquiéter. C’est de l’exemple réitéré de ces sortes de mutations que naît l’idée de la mutabilité de la propriété.\par
Des enfants, après la mort de leurs pères, continuent d’habiter la même cabane, parce qu’ils s’y trouvent, et que l’habitude les y retient : personne d’ailleurs, n’entreprend de les y troubler, de là, l’idée de la transmission héréditaire : il faut que toutes ces choses existent avant qu’on veuille les faire exister. Ce ne sont pas les lois qui leur donnent l’existence, ce sont elles au contraire qui donnent l’existence aux lois.\par
Nous voyons donc que la société se forme sans dessein prémédité. Les hommes, par l’exercice de leurs facultés, arrivent à des résultats communs sans avoir prévu l’identité de ces résultats ; il se trouve qu’ils ont voulu les mêmes choses, sans qu’aucun ait eu l’intention de conformer sa conduite à celle des autres : c’est ainsi que la société se forme, pour ainsi dire, d’elle-même, sans qu’aucun de ceux qui la composent ait eu pour but de la former.\par
Ce n’est qu’après l’avoir observée dans sa marche naturelle, qu’on convient de vouloir encore ce qu’on a déjà voulu, de faire ce qu’on a déjà fait, et voilà ce qu’on appelle les lois positives : ces lois ne créent rien, ne déterminent rien, n’instituent rien. Tout ce qu’elles statuent existait avant leur institution. Elles ne sont, à proprement parler, que la déclaration d’un fait : elles ne constituent pas la société, puisque la société est déjà constituée. Leur office se borne à déclarer son existence, et à donner à ceux qui la composent, l’idée de faire avec intention ce qu’ils faisaient auparavant sans dessein réfléchi.\par
Tel est l’ordre naturel des sociétés. Il ne faut pas croire qu’elles aient été l’effet d’un plan quelconque, c’est-à-dire, qu’on ait fait les lois d’abord, et qu’ensuite chacun s’y soit soumis ; car il a fallu que ce que les lois ont sanctionné, existât auparavant. Comment, en effet, aurait-on pu mettre dans ces lois quelque chose qui n’eût pas eu déjà lieu ? Si la société ne leur avait pas été antérieure, comment aurait-elle pu entrer dans leur dispositif ? Peut-on faire une loi sur un objet dont on n’a pas même l’idée ? et d’où serait venue l’idée de la société, si l’on n’en avait pas eu l’expérience ? Ne fallait-il pas, comme nous venons de le dire, qu’on pût la mettre en fait avant de la mettre en principe ?\par
J’aime bien les conceptions de ces écrivains, qui nous donnant leurs abstractions pour des réalités, font, pour ainsi dire, l’appel du genre humain, et voient aussitôt accourir les individus dispersés, se réunissant pour commencer la société dont ils ne se doutaient point auparavant. Ces êtres, qu’ils nous peignent d’une ignorance et d’une férocité absolues, n’obéissant qu’à leurs appétits, sans conscience du passé, sans prévoyance de l’avenir ; se trouvent tout à coup, s’il faut les en croire, métamorphosés en législateurs. Ils créent des lois, et des formes de gouvernement ; ils discutent leurs intérêts respectifs avec toute la subtilité des jurisconsultes et des politiques les plus raffinés ; et voilà ce qu’on nous présente comme l’histoire naturelle des associations humaines !\par
Voulez-vous savoir pourquoi nous ne datons la société que de l’établissement des lois positives, et pourquoi nous ne concevons pas son existence antérieure ? C’est que l’habitude où nous sommes de nous régler d’après ces lois, nous accoutume à leur rapporter le principe de nos actions ; à croire que nous ne les faisons que parce qu’elles les ont ordonnées, et non parce qu’il est dans notre nature de les faire. Ce préjugé., qui nous fait regarder les lois comme la cause effective de tout ce qui est réglé par elles, ne nous permet pas d’en faire remonter l’existence aux temps antérieurs ; il nous semble que rien de ce qu’elles ont statué ne pouvait exister avant leur institution, et cela doit nous paraître ainsi ; car nous ne pouvons faire remonter les effets qu’à leur cause ; et en plaçant dans les lois celle des effets postérieurs à leur institution, nous devons nécessairement en dépouiller le temps qui les a précédées.\par
On sent déjà toutes les conséquences de cette erreur sur la nature des lois : ces lois ordonnent le respect des personnes ; donc, avant leur établissement, le respect des personnes n’existait pas. De là ce prétendu état de guerre, qui, selon Hobbes et quelques autres écrivains, était l’état du genre humain avant l’institution des lois positives. Ces mêmes lois ordonnent que la propriété soit respectée, donc, avant leur établissement, ce que l’un avait, un autre cherchait à s’en emparer. Voilà les raisonnements dans lesquels on se trouve nécessairement entraîné par le préjugé de la causalité des lois.\par
On ne fait pas attention que si tel avait été l’état des hommes avant l’établissement des lois positives, l’idée de les établir n’aurait pu tomber dans l’esprit de personne ; car tous les individus de l’espèce humaine n’ayant offert d’autre spectacle que celui du meurtre et du brigandage, d’où aurait-on pu conclure la possibilité de l’ordre et de la paix parmi eux ? comment aurait-on eu même les idées d’ordre et de paix ? Voilà ce qu’on devrait bien nous expliquer dans le système de Hobbes et de tous les autres calomniateurs de la nature humaine.\par
Certes, loin de tirer de la loi positive du respect des personnes et des propriétés, la fausse conséquence que ceux qui l’ont instituée étaient en état de guerre avant cette loi, il est naturel au contraire d’en induire qu’ils vivaient dans la meilleure intelligence. Et en effet, sur quoi auraient-ils fondé la probabilité de l’observation future de la loi, si le passé ne la leur avait pas fournie, et comment le passé leur aurait-il fourni cette probabilité, si le respect des personnes et des propriétés avait été une chose inconnue parmi eux {\itshape }?\par
Nous nous sommes singulièrement mépris sur la nature des lois ; nous leur avons attribué un pouvoir qu’elles n’ont pas, celui de déterminer nos actions : nous leur avons supposé une vertu coactive, une puissance à elles inhérente, de produire des effets dont elles ne peuvent que reconnaître et consacrer l’existence. Les dispositions de la nature humaine, une fois consignées dans les lois, on a imaginé qu’elles créaient en nous ces dispositions. On a cru recevoir d’elles ce qu’on leur a donné.\par
Ceci confirme parfaitement la théorie de la société que nous avons exposée dans cet ouvrage ; car, si les lois ne sont la cause de rien, s’il faut que ce qu’elles statuent ait eu lieu avant qu’elles existassent, il est clair que la société a dû préexister à leur institution, et que par conséquent, elles n’ont rien innové, rien changé à l’état antérieur des hommes ; ils n’ont fait après que ce qu’ils faisaient auparavant ; or, ce qu’ils faisaient auparavant, ils le faisaient de leur propre mouvement, sans être convenus de le faire, sans qu’aucun engagement respectif les y déterminât ; d’où il résulte qu’on a méconnu le vrai caractère de la société, quand on l’a rapportée à des conventions, à des engagements, à des stipulations de réciprocité, puisqu’elle existait avant toutes ces choses dans lesquelles on croit en trouver le principe.\par
Les lois positives n’ayant rien innové dans l’ordre préexistant, n’ont été qu’une simple déclaration de l’état actuel de la société et de l’intention manifestée de la part de ses membres de persister dans cet état, c’est-à-dire, de persévérer dans le même système de volontés et d’actions. Voilà tout ce que renferment les lois : ce qu’on leur attribue de plus est illusoire. Elles n’ont pas été le motif des actions contenues dans leur dispositif, puisque ces actions leur étant antérieures, avaient leur motif qu’elles ont conservé après. Ces lois ne sont pas la cause déterminante de la conduite ultérieure de ceux qui s’y soumettent, puisque leur conduite était auparavant la même. Ce n’est pas par la considération de la réciprocité convenue ou constatée dans ces lois, qu’on fait ce à quoi elles obligent ; car on le faisait avant cette considération : ce n’est pas non plus parce qu’on s’y est engagé, puisqu’on agissait de même, quand il n’existait point d’engagement.\par
À la vérité on parle d’un {\itshape engagement tacite} antérieur aux lois positives ; mais je voudrais bien qu’on nous dît ce que c’est que cet engagement, quelles en sont les clauses, et enfin s’il peut exister un engagement de cette nature ; car cet engagement est déterminé, ou il ne l’est pas ; s’il est déterminé, il n’est plus tacite ; s’il n’est pas déterminé, comment en déterminerez-vous l’infraction ? Je crains bien que ce ne soit qu’un mot vide de sens, une de ces expressions qu’on met à la place des choses, et qui en tiennent lieu, jusqu’à ce qu’on reconnaisse qu’ils ne signifient rien.\par
Toutes ces opinions d’état de nature, d’état de guerre entre les hommes, de société formée par engagement respectif, par pacte ou contrat synallagmatique, ne sont réellement que des chimères, des abstractions, des fables inventées par l’ignorance de la nature humaine et par le préjugé de la causalité des lois.\par
Il importe plus qu’on ne l’imagine de combattre ce préjugé qui dénature l’essence même des lois, et nous en dérobe la vraie notion ; car il tend à faire regarder les lois comme arbitraires, tandis que leur nature les circonscrit dans ce qui a déjà eu lieu : on en prend les éléments dans l’avenir, et c’est du passé qu’on doit les prendre : leur objet propre est ce qui s’est fait, et non ce qui se fera ; c’est au mieux existant, et non au mieux possible, à leur servir de base. Le rôle du législateur n’est pas de devancer la société, mais de l’attendre, et de marcher avec elle. Ses fonctions sont plutôt d’un observateur que d’un penseur ; car, pour remplir véritablement sa mission, il faut qu’il travaille sur des faits, et non sur des principes : c’est au moraliste à rechercher ceux-ci, à les mettre en évidence, pour qu’on les réduise en fait ; et c’est seulement lorsqu’ils ont subi cette modification, que le législateur peut en faire la matière des lois.\par
Mais, parce qu’on attribue à celles-ci le pouvoir de déterminer elles-mêmes leur observation, on fait, si je puis m’exprimer ainsi, des lois purement idéales ; on en cherche le principe dans ce qui n’existe pas, souvent même dans ce qui ne peut exister, et l’on attend de la loi la cause de son existence : on croit créer cette cause, en créant la loi qu’on regarde comme la raison suffisante de l’effet dépendant de son observation ; mais les lois, comme nous l’avons déjà vu, n’étant que de simples déclarations, ne peuvent être la cause de rien. C’est donc vainement qu’on leur assigne un emploi qu’elles ne peuvent remplir : alors on leur adjoint la force ; et, quand la force est jointe à la loi, ce n’est pas la loi qu’on observe, c’est à la force qu’on obéit. Il y a véritablement oppression et tyrannie légale.\par
On peut mettre en fait que, si la loi n’avait jamais excédé les bornes de sa nature, s’il n’y avait point eu ce que j’appellerais des lois ambitieuses, dont le propre fût absolument d’innover, et par conséquent d’exiger des hommes ce qu’elles n’avaient pas le droit d’exiger ; on peut mettre en fait, dis-je, qu’on n’aurait jamais eu besoin de coercition : elle n’est devenue nécessaire que par les mauvaises lois, et ce sont encore les mauvaises lois qui en déterminent l’usage.\par
Parce qu’on se figure que les lois dans le principe ont changé l’état des hommes, et donné l’existence à un nouvel ordre des choses, on croit qu’elles peuvent faire encore ce qu’on suppose qu’elles ont déjà fait ; on s’imagine qu’un ordre nouveau peut être l’effet de lois nouvelles. Tant que cette erreur subsistera, l’oppression résultera des lois ; il faudra, pour qu’elle cesse, qu’on sente que les lois doivent être l’effet des innovations, et non les innovations l’effet des lois.
\chapterclose


\chapteropen
\chapter[{Chapitre XXXV. De la volonté générale.}]{Chapitre XXXV. De la volonté générale.}\renewcommand{\leftmark}{Chapitre XXXV. De la volonté générale.}


\chaptercont
\noindent On a dit une chose vraie, quand on a dit que la loi est l’expression de la volonté générale ; mais cette volonté est un être abstrait qui, à proprement parler, n’a point de réalité : il n’y a pas plus de volonté générale, qu’il n’y a d’homme en général, d’arbre en général ; c’est par abstraction des volontés particulières que nous créons la notion de la volonté générale, comme c’est par abstraction des arbres en particulier, des hommes en particulier, que nous créons la notion abstraite d’arbre et d’homme en général ; et, comme il a fallu qu’on eût vu plusieurs hommes, plusieurs arbres, et qu’on eût observé leurs qualités communes pour en déduire la notion de l’homme et de l’arbre, il a fallu, par la même raison, qu’on eût vu, si je puis m’exprimer ainsi, plusieurs volontés particulières, et qu’on eût observé leurs qualités communes pour en déduire la notion de la volonté générale.\par
Or, la volonté ne peut se voir d’une manière positive que dans ses effets : quand on a remarqué que plusieurs individus vivaient ensemble, sans se détruire, on a reconnu qu’il était dans la volonté de chacun de respecter son semblable ; quand on a vu que ce que l’un avait, un autre ne cherchait point à s’en emparer, on a été assuré que la volonté positive de chacun était de respecter la propriété dans les autres : c’est de ce caractère commun et reconnu des volontés particulières, qu’on a formé la notion abstraite de la volonté générale ; elle est née de l’expérience et de l’observation, comme toutes les autres notions de ce genre, et non, comme on se l’imagine, de la simple déclaration des vœux individuels.\par
En effet, des volontés non exécutées, ou avant leur exécution, ne peuvent pas être considérées comme des volontés positives ; elles ne sont, dans l’ordre moral, si je puis me servir de cette comparaison, que ce que sont les germes dans l’ordre physique : il faut qu’elles se développent dans leur effet pour obtenir leur caractère positif, Comme il faut qu’un arbre renfermé dans la graine qui le contient, se développe pour être un arbre. Les vœux ne sont donc, en quelque sorte, que le germe de la volonté, et comme la notion abstraite de la volonté générale, ne peut se former que d’après les volontés positives des individus, il est clair qu’il faut que les vœux soient devenus les volontés, c’est-à-dire, qu’ils aient obtenu leur effet, pour que la notion de la volonté générale ait lieu, car des vœux énoncés, déclarés ou présumés, ne sauraient en fournir les éléments.\par
Le vœu général qui tend à une fin n’est donc pas ce qu’il faut à la loi ; car nous admettons que celle-ci doit exprimer la volonté générale, et pour qu’elle en contienne véritablement l’expression, il faut que le vœu général ait obtenu sa fin, c’est-à-dire, qu’il soit devenu la volonté générale jusqu’alors la loi est au moins prématurée ; car son objet, ainsi que nous l’avons dit dans l’autre chapitre, n’est pas ce qui sera, mais ce qui est ; elle n’a point à statuer sur le mieux possible, mais sur le mieux existant.\par
C’est pour n’avoir pas fait cette distinction qu’on est tombé dans les erreurs les plus graves, et qu’une subtile métaphysique s’est glissée dans la théorie de la législation, pour en dénaturer et corrompre les vrais principes. Au lieu de baser les lois sur des faits, comme le veut la nature des choses, on les a basées sur des opinions ; de là ce système de délibérations et de votes individuels ou collectifs pour la confection de la loi. On s’est figuré qu’en exprimant des vœux on exprimait la volonté générale. Si toute la nation, a-t-on dit, pouvait se réunir en assemblée délibérante, et donner son vœu sur les questions d’intérêt public, le recensement et la confrontation des votes, en donnant un vœu commun, donnerait la volonté générale.\par
Jusqu’à présent on n’a objecté contre cette doctrine que l’impossibilité de son exécution, surtout dans les grandes sociétés ; mais quand il serait possible de réunir en effet tout le peuple dans une seule assemblée et d’en recueillir le vœu positif ; supposez même que tous les membres de cette assemblée puissent voter avec connaissance de cause sur les questions soumises à leur délibération, ce qui n’est et ne sera jamais vrai pour le plus grand nombre, il est clair, d’après ce que nous avons dit plus haut, qu’on se tromperait encore si l’on croyait, en explorant le vœu commun, explorer la volonté commune ; car la notion abstraite de celle-ci, ne peut se former que par l’observation des rapports constants qu’ont entre elles les volontés particulières : or, ces volontés se composent d’un système suivi de vœux et d’actions ; d’où il suit que c’est de l’uniformité aperçue entre ces différents systèmes librement développés, que doit se conclure la volonté générale, et non de l’émission des vœux à telles ou telles époques.\par
Il n’est donc pas besoin de consulter le peuple pour savoir si une loi est l’expression, de sa volonté ; on peut le voir, en quelque sorte, par ses propres yeux : on n’a qu’à examiner jusqu’à quel point elle se rapproche ou s’éloigne des habitudes communes, ou de ce qui se fait généralement ; si elle s’en écarte absolument dans son dispositif, quelque satisfaisante qu’elle soit pour la raison, elle n’exprime point la volonté générale ; quelque bonne qu’elle soit en principe, elle est mauvaise comme loi.\par
Pour connaître la volonté générale, nous le répétons encore, il s’agit moins d’interroger que d’observer : le moyen de parvenir à cette connaissance, n’est pas de demander au peuple ce qu’il veut, mais d’examiner ce qu’il fait.\par
Au reste, ceci ne peut et ne doit s’entendre que d’un peuple libre ; car, si vous le supposez dominé par ses chefs, c’est vainement que vous observeriez ce qu’il fait, pour en déduire la notion de la volonté générale, puisque ce qu’il fait n’est pas sa volonté. Pour qu’on eût toujours pu compter sur cette indication, il aurait fallu que les législateurs et les gouvernants, au lieu de se croire institués pour diriger la marche de la société, se fussent pénétrés de l’idée qu’il ne leur appartenait que d’en suivre les directions, et que c’était toujours d’elle-même qu’ils devaient recevoir l’avertissement de créer une institution ou de la détruire. Mais quand ils se sont une fois intitulés les maîtres du peuple, quand ils l’ont forcé de se soumettre à leur volonté particulière, et que, dans la rédaction de ce qu’ils ont appelé des lois, ils n’ont consulté que leur caprice, il est bien certain que tous les éléments d’où l’on pouvait extraire la notion de la volonté générale, ont été détruits, et que l’uniformité des actions qui l’aurait fournie dans un autre temps n’a pu donner désormais que celle de la servitude générale.\par
En effet, pour qu’un homme ou quelques hommes soient les maîtres des autres, il faut nécessairement que ceux-ci règlent leurs volontés particulières sur la leur. L’obéissance fût-elle volontaire, comme elle le devient à la longue par l’effet de la résignation et de l’habitude, il n’en résulte pas moins qu’en faisait la volonté d’un autre, les hommes ne font pas leur volonté. On ne peut donc pas, dans cet ordre des choses, abstraire la volonté générale des volontés individuelles, puisqu’à proprement parler, il n’y a point de volontés individuelles. Tous ne font que ce que leurs maîtres veulent qu’ils fassent ; il s’ensuit que ce n’est que par un véritable abus des mots, qu’on donne le nom de lois aux règlements qui ont lieu chez les peuples qui vivent dans cette malheureuse condition ; car il n’y a point de loi possible pour eux : il faudrait, pour lui donner l’existence qu’on pût abstraire de leurs rapports mutuels, une volonté générale ; et c’est ce que leurs rapports mutuels ne peuvent pas donner. Toute législation leur est donc interdite : ils sont inévitablement condamnés à l’arbitraire.\par
La première chose à faire, pour qu’ils rentrent dans la voie naturelle des sociétés humaines, c’est de les rendre à la liberté ; mais il ne s’ensuivra pas qu’ils soient tout à coup susceptibles de législation ; car il faut pour cela que les volontés particulières prennent l’initiative de leurs déterminations, c’est-à-dire, qu’elles soient elles-mêmes ; or, il est assez difficile qu’on soit tout à coup ce qu’on n’était pas ; qu’on marche quand on n’a pas l’habitude de marcher, et enfin, qu’on veuille être soi, lorsque toute sa vie on a été un autre.\par
Mais admettons cette métamorphose, il n’y aura point encore lieu à faire des lois, parce qu’il n’y a pas encore de volonté générale ; il faut pour cela que les volontés particulières licenciées et rendues à elles-mêmes, s’exercent sans dominateur, et que leur libre exercice produise dans les actions une uniformité qu’on ne puisse pas imputer à la force ni à la contrainte. Ce n’est que lorsque vous apercevrez cette uniformité produite par le libre développement des volontés particulières, dans le système général des actions, que vous pourrez dire ; voilà la volonté générale.\par
Ceci nous conduit à regarder comme une idée profondément politique, celle d’abandonner, en quelque sorte, le peuple Français à lui-même, après la mémorable révolution qui l’a rendu à la liberté ; rien n’était plus sage que la conception de ce qu’on a appelé gouvernement révolutionnaire, s’il avait été ce qu’il devait être, c’est-à-dire, un obstacle à toute usurpation d’autorité, à toute domination extérieure et intérieure, et en même-temps la suspension de toute loi, jusqu’à ce que les résultats de la liberté en eussent amené le motif.\par
Mais on sent combien, chez un peuple longtemps asservi, il est difficile, pour ne pas dire impossible, d’attendre cette époque ; car l’intervalle est nécessairement occupé par l’anarchie : point d’ordre, point de régularité, tout est bouleversé ; c’est un chaos dont les éléments se trouvent froissés dans tous les sens possibles, de manière que les hommes comparant leur situation présente à celle qui la précédée, et n’ayant pas l’expérience d’une organisation sociale émanée du concours des volontés librement développées, regardent ce dernier résultat comme absolument impossible, se méfient de la force des choses, et ne voient de salut que dans la domination.\par
Il n’est pas extraordinaire que, dans cette disposition des esprits, les idées de liberté, de souveraineté du peuple, de volonté générale, ne soient plus envisagées, par ceux mêmes qui les avaient d’abord accueillies, que comme des rêveries métaphysiques : cependant s’il est vrai que la société se soit d’abord formée sans qu’on songeât à la former, elle a été bien certainement l’effet du libre développement des volontés particulières ; il a bien fallu qu’il résultât du libre exercice de ces volontés, un système uniforme d’actions ; que cette uniformité servît de base à l’établissement des lois positives, et tant que les lois ont eu cette base, elles ont véritablement été l’expression de la volonté générale : on ne peut pas disconvenir que le peuple, dans cet état de la société, n’ait été libre et souverain de droit et de fait.\par
Ces idées, qu’on traite aujourd’hui avec tant de mépris, n’ont donc pas toujours été de pures abstractions, puisque la société leur a dû son existence positive, et il est assez difficile de croire qu’elle puisse remplir son but, en s’écartant absolument de son principe : il n’est peut-être pas aussi aisé qu’on l’a cru de l’y ramener ; mais, on peut du moins pressentir que ce retour est inévitable. Il faudra bien que les idées de liberté qui, toujours proscrites se reproduisent toujours, prévalent enfin dans l’opinion générale : la sottise et l’erreur pourront différer leur triomphe ; mais il n’en est pas moins assuré dans les développements ultérieurs de la perfectibilité humaine.\par
Que les avocats du despotisme s’égaient tant qu’il leur plaira, sur les principes généreux que notre corruption et nos habitudes serviles nous mettent dans l’impuissance d’adopter, ils ne prouveront pas la fausseté de ces principes. Ils pourront faire croire à des esclaves que la servitude est pour eux la meilleure de toutes les conditions ; mais ils ne l’ennobliront pas ; ils ne dignifieront point ce qui est vil de sa nature : ils pourront de même outrager la liberté de leurs sarcasmes, mais ils ne la flétriront pas ; malgré toutes leurs saillies, elle sera toujours ce qu’elle est, toujours supérieure à leur déplorable démence.\par
Dans aucun temps, la servitude n’a manqué d’apologistes, ni la liberté de détracteurs : c’est surtout aux époques semblables à la nôtre, que les défenseurs officieux de l’obéissance passive ont eu des prétentions à l’estime ; ils se sont figurés que la postérité leur saurait un gré infini du soin qu’ils auraient pris de lui donner des maîtres ; mais elle a toujours été assez ingrate pour ne pas sentir l’importance de ce service : ce sont les noms des défenseurs et des martyrs de la liberté, traités de scélérats par les amis des rois leurs contemporains, qui sont restés constamment dans la mémoire des hommes. Le champion de la royauté, Saumaise, n’épargna contre Milton, aucune des épithètes que les Saumaises de nos jours prodiguent avec tant de libéralités : eh bien, malgré le rétablissement de la royauté en Angleterre, on n’a pu ni mépriser Milton ni estimer Saumaise.
\chapterclose


\chapteropen
\chapter[{Chapitre XXXVI. De l’altération et de la corruption de la Société.}]{Chapitre XXXVI. De l’altération et de la corruption de la Société.}\renewcommand{\leftmark}{Chapitre XXXVI. De l’altération et de la corruption de la Société.}


\chaptercont
\noindent Nous n’avons pas parlé des écrivains qui, comme Grotius, ont fait dériver la société de ce qu’ils ont appelé le droit du plus fort ; car, outre qu’ils ont été réfutés par ceux qui ont voulu la baser sur des conventions et des engagements respectifs, il est bien démontré par tous les renseignements historiques sur l’état des hommes, antérieurement à la civilisation, que la force n’a aucune part à leurs réunions en sociétés séparées.\par
C’est avec aussi peu de fondement que d’autres ont voulu rapporter à l’autorité paternelle, le principe de l’état social ; car on sait qu’avant la division des propriétés, l’autorité paternelle n’existe pas : elle n’est ni une prétention dans les pères ni un droit reconnu de la part des enfants, et cependant la société est antérieure à la division des propriétés.\par
« Les parents, dit Robertson dans son {\itshape Histoire de l’Amérique}, aussitôt qu’ils ont conduit leurs enfants jusqu’au-delà de cet âge de faiblesse où ils ne peuvent point subvenir à leurs propres besoins, leur laissent une entière liberté : ils ne leur donnent presque jamais des conseils ; ils ne les grondent et ne les châtient point, ils les laissent enfin maîtres absolus de leurs propres actions. »\par
Cette indépendance absolue des enfants, attestée par tous les voyageurs et par tous ceux qui nous ont donné l’historique des premiers temps de la société, dément formellement la prétendue royauté paternelle dont il a plu à quelques écrivains de gratifier cette époque. Il est donc évident que toutes les opinions émises jusqu’à ce jour sur le principe fondamental de la société sont également fausses. La société, comme nous l’avons dit, se forme sans dessein prémédité, par la seule impulsion de la nature humaine ; par le besoin naturel à l’homme de se subordonner ses penchants pour être libre, et par la nécessité des rapports sociaux pour effectuer cette subordination ; il en résulte naturellement des effets communs, des actions communes, des dévouements identiques, et c’est de l’expérience et de l’observation de ces phénomènes constants, produits par le libre exercice des volontés particulières, que résulte la notion de la société, c’est-à-dire, la connaissance de l’état social qu’on ignorait auparavant quoiqu’on vécût dans cet état.\par
Le système des lois positives se forme à peu près de la même manière : le fait le plus général et le plus constant est le premier qu’on observe et qu’on déclare ; on dira, par exemple, que, quand la société est insultée ou menacée par les autres sociétés, tout ce qui est en état de la défendre accourt spontanément à sa défense ; rien d’impératif, ni de coercitif dans cette déclaration : ce n’est à proprement parler, qu’une maxime traditionnelle fondée sur l’observation d’un fait constant. Ces observations et ces maximes se multiplient, et forment par la suite des temps une espèce de code. Les vieillards, à raison de leur expérience, en sont naturellement les dépositaires ; ils ont vu ce qui s’est fait dans les mêmes circonstances, et en disant ce qui s’est fait, ils disent ce qu’on fera. De là les assemblées des vieillards ou anciens qui remplissent d’abord, sans s’en douter, l’office de législateur, et déclarent la volonté générale, en déclarant les faits généraux.\par
Mais à mesure que les rapports se multiplient, que les idées se développent, que l’industrie fait des progrès, il survient nécessairement des modifications dans l’état des hommes : à des faits généraux, dans un temps succèdent d’autres faits généraux, qui quelquefois leur sont diamétralement opposés : cependant les vieilles maximes subsistent toujours. Ce qu’on doit faire par ces maximes n’est plus ce qu’on fait : si l’on était assez éclairé pour démêler les véritables causes de ce changement, on verrait que la société ne peut pas rester dans un état stationnaire ; que, composée d’éléments toujours actifs, elle porte en elle-même le principe de ses changements et de ses modifications ; on reconnaîtrait que, par le libre exercice des volontés particulières, ce qui était la volonté générale dans un temps n’est plus la volonté générale dans un autre ; que par conséquent il faut en changer la déclaration, et réformer une maxime qui ne la contient plus, pour lui en substituer une autre. Par ce simple artifice, la société, toujours dirigée par le même principe, se conserverait sans altération, et se civiliserait sans être asservie ; mais on ne fait pas, ou l’on ne peut pas faire ces sortes de réflexions : on veut que les vieilles maximes expriment la volonté générale quand elles ne l’expriment plus, et c’est à cette prétention que commence le dérangement de l’économie naturelle de la société.\par
La volonté générale, dans ces premiers temps, n’est susceptible de varier que sur les choses qui ne tiennent point à l’essence de l’état social ; car la défense commune, le respect des personnes et des propriétés, sont des objets sur lesquels elle ne peut pas varier.\par
Comment, en effet, la société se maintiendrait-elle si, quand elle est attaquée ou menacée, ceux de ses membres que leurs forces individuelles rendent capables de la faire respecter n’en prenaient point généreusement la défense ? comment se maintiendrait-elle encore, si les membres de cette société ne se respectaient point réciproquement, s’ils ne travaillaient, comme on l’a prétendu, qu’à se détruire ou à s’enlever mutuellement leur proie ? Ou il faut admettre qu’il n’existe point de société avant l’établissement d’un gouvernement positif (ce qui est démenti par l’expérience), ou il faut reconnaître que, dans ces sociétés préexistantes, la défense commune, le respect des personnes et des propriétés, sont toujours un fait général et constant, d’où l’on doit nécessairement conclure qu’ils sont toujours la volonté générale.\par
Mais il est d’autres choses qui, ne tenant point à l’essence même de la société, peuvent et doivent varier, soit par l’effet des circonstances, soit par les développements de l’intelligence et de l’industrie. La société, par exemple, subsistait, dans un temps, des produits de la chasse : la difficulté de se procurer la subsistance par ce moyen, ou la découverte de quelque procédé d’agriculture, porteront naturellement les individus à l’adoption de ce nouveau mode ; il s’ensuivra qu’à la volonté générale de chasser succédera la volonté générale de cultiver. Cette innovation en entraînera nécessairement d’autres dans l’économie générale de la société : les anciens, tenant toujours aux vieilles coutumes, en rappelleront les maximes surannées, et traiteront tout ce qui se fait de dangereuses innovations : ils croiront qu’il importe au bien de la société de mettre un frein à ce prétendu désordre ; ils feront quelque règlement pour l’empêcher, pour maintenir l’ancienne volonté générale qui n’est plus la volonté générale, ou qui, par la force des choses, va cesser de l’être.\par
Bientôt s’apercevant qu’ils sont obéis, c’est-à-dire qu’ils font faire aux autres ce qu’ils veulent, ils ne se bornent plus à déclarer ce qui s’est fait ; c’est, au contraire, ce qu’ils veulent qu’on fasse qui devient le motif de leurs déclarations ; de manière que la loi, qui était auparavant l’expression de la volonté générale, n’exprime plus maintenant que leur volonté particulière : on s’y conforme d’abord par déférence ; mais il vient un moment où il faut que la force en détermine l’observation, et que ceux dont les fonctions étaient simplement de déclarer, s’arrogent le droit d’ordonner, et soient investis du pouvoir de contraindre. Ils finissent par dire : Vous ferez cela, parce que nous le voulons ; et si vous ne le faites pas, vous serez punis.\par
Quand les choses en sont à ce point, la société n’est plus dans sa marche naturelle ; elle s’est, pour ainsi dire, fourvoyée, et a quitté le droit chemin pour en prendre un qui ne peut jamais la conduire à son but. La loi n’était auparavant que l’expression de ce qui se faisait librement avant sa confection ; maintenant au contraire, on fait la loi pour que telle chose qu’on n’a nulle disposition à faire se fasse ; elle résultait du libre développement des volontés particulières, maintenant elle commande à ces volontés ; c’était le peuple qui en dernière analyse était son propre législateur, et maintenant il y a des législateurs du peuple.\par
Par ce renversement des principes naturels de la société, il se fait dans l’homme une révolution morale qui change entièrement ses dispositions du moment où sa conduite est soumise au contrôle et à la coercition d’une volonté extérieure, le principe intérieur et moral d’action s’oblitère ; en quelque sorte, il devient passif. Rien n’était prescrit auparavant, et maintenant il faut que tout le soit, et qu’on attache une peine à l’infraction de tout ce qu’on prescrit, pour en déterminer l’observation ; ainsi la défense commune, le respect des personnes et des propriétés, ces devoirs inviolés auparavant et remplis sans la moindre contrainte, l’homme ne se croit plus capable de les observer, si la loi ne les lui prescrivait point, et si elle n’attachait pas les peines les plus sévères à leur infraction. Le misérable dépouille sa nature de tout, et attend de la loi tout ce qu’elle devait recevoir de lui.\par
De cet avilissement de la nature humaine résultent les idées les plus fausses sur l’homme et sur la société. On suppose d’abord que l’homme a été obligé de faire violence à sa nature, pour se plier à l’état social ; et en effet, la tournure qu’on a fait prendre à la société, motive cette opinion ; car au système libre des sociétés humaines, on a substitué un système, où ce qui était l’effet de la liberté est devenu l’effet de l’obéissance ; et comme on ne connaît que cet ordre actuel des choses, et qu’on ne peut pas avoir la conscience de celui qui l’a précédé, on ne peut juger l’homme que sur ce qu’il est dans ce système, et ce jugement qui n’est que relatif, doit avoir pour ceux qui le portent le caractère d’un jugement absolu.\par
Comment pourrait-on, par exemple, juger l’homme capable de respecter librement les personnes et les propriétés, quand ce respect lui est commandé par les lois, et qu’on voit un appareil formidable de châtiments, et une force publique instituée pour en déterminer l’observation ? Il est impossible qu’on n’en conclue pas qu’abandonné à lui-même, l’homme serait une espèce de tigre toujours enclin au meurtre et au pillage ; que par conséquent il a fallu lui faire violence pour qu’il adoptât d’autres errements, et enfin que c’est pour n’être pas détruit qu’il s’est engagé à ne pas détruire.\par
Toutes ces conséquences qu’on croit déduites de la nature humaine, ne le sont que du système qui s’est introduit dans la société. L’homme n’agit plus par le même principe : des lois arbitraires, appuyées de la force, ont commandé toutes ses actions, de manière qu’il lui est devenu impossible de distinguer ce qu’il aurait fait de ce qu’il n’aurait pas fait s’il avait été libre ; il a tout rapporté à l’obéissance, et motivant tout par ce principe, il s’est imaginé qu’il ferait absolument l’inverse dans le mode spontané.\par
Cette erreur est d’autant plus naturelle, que l’homme, à raison de sa liberté, n’étant pas fait pour obéir, sent toujours en lui-même quelque chose qui le tire en sens contraire de l’obéissance ; c’est un état contre lequel sa nature ne cesse, en quelque sorte, de protester ; et il prend cette disposition à ne pas obéir pour une disposition naturelle à ne pas faire la chose qui lui est ordonnée. Ce qui n’est relatif qu’au mode d’activité, il le rapporte à l’action même, et il se croit de l’éloignement pour l’action quand il n’en a que pour le principe d’après lequel il l’a fait. Cette répugnance qu’il ne trouve point à faire ce qui ne lui est pas ordonné, lui persuade qu’en suivant sa propre inclination, il ne ferait rien de conforme à ses devoirs : il ne voit pas que cette répugnance pour les actions ordonnées, et cette tendance vers celles qui ne le sont pas, sont en lui des qualités étrangères à ces actions, et proviennent uniquement du soin qu’on a pris de les lui ordonner ou de les lui défendre ; que, si elles étaient à son choix, il en résulterait de sa part un autre système de vœux et de déterminations, et que par conséquent il ne peut rien affirmer de ses propres dispositions tant qu’il n’est pas libre. Voilà la grande source de nos erreurs sur l’homme, et c’est encore de là que dérivent nos erreurs sur la société ; car cette disposition à ne pas obéir, ne pouvant être surmontée que par la crainte, et tous les devoirs sociaux se trouvant compris dans le système de l’obéissance, la crainte doit paraître absolument nécessaire à l’accomplissement de ces devoirs, et elle l’est, en effet, dans ce système ; car la société, d’un état libre qu’elle était, devenue un état violent par sa fausse manutention, ne peut se soutenir que par la violence. L’erreur ne consiste qu’à prendre cet état de déviation pour son état naturel ; qu’à croire qu’il lui soit impossible d’exister autrement : ce préjugé fait qu’au lieu de travailler à la ramener dans sa voie originelle, on la confirme, pour ainsi dire, dans son égarement ; on cherche son perfectionnement dans le principe même de sa corruption ; on renforce ce qu’on appelle la terreur salutaire des lois de tout ce que la superstition peut inventer de plus horrible ; on frappe les sens d’effroi par les exemples multipliés d’une atroce sévérité légale, et l’imagination par la crainte des supplices à venir. Si cette double crainte cessait de peser sur les consciences, on se figure qu’il n’y aurait plus ni rectitude ni moralité parmi les hommes.\par
Tout cela n’est vrai qu’à raison du mal invétéré qu’a produit le système de l’obéissance : ce qu’il était dans la nature de l’homme de faire pour être libre, vous avez voulu qu’il le fît pour n’être pas pendu ou pour n’être pas damné. Il n’est donc pas extraordinaire qu’il se persuade que, sans ces motifs que vous avez substitués aux motifs naturels de rectitude et de moralité, il n’y aurait plus pour lui ni rectitude ni moralité possibles.\par
On raconte qu’un soldat anglais, fait prisonnier dans la guerre d’Amérique, demandait aux insurgents pourquoi ils se battaient : Moi, disait-il, je me bats pour mon roi ; mais vous qui n’avez pas de roi, pour qui et pourquoi vous battez-vous ? Ce malheureux, accoutumé à ne régler sa conduite que sur des motifs d’obéissance, ne concevoir pas qu’on pût faire pour la liberté ce qu’il faisait pour son roi. Voilà l’histoire de ceux qui demandent gravement ce qu’on peut mettre à la place de la superstition et de l’obéissance passive : les raisonnements qu’ils font à ce sujet peuvent se réduire à une question tout aussi niaise et tout aussi ridicule que celle du soldat anglais : habitués à ne faire dépendre la probité, la loyauté, la justice, que de la crainte de l’enfer ou de l’échafaud, il ne leur est pas plus possible de soupçonner le motif naturel de ces vertus dans l’amour de la liberté, qu’il n’était possible au soldat anglais de voir dans cet amour généreux le motif de l’insurrection américaine.\par
En ôtant son roi à ce soldat qui trouvait si extraordinaire qu’on se battît quand on n’avait pas de roi, on lui eût sans doute ôté toute sa vertu guerrière : il se serait du moins cru sans motif pour rester fidèle à ses drapeaux ; pourrait-on cependant en conclure qu’il faut que les hommes soient asservis pour être braves ? Des esclaves doivent raisonner de cette manière ; c’est une conséquence nécessaire de leur position. Les motifs qui dirigent leur conduite étant les seuls dont ils aient la conscience, il ne leur est pas possible d’en supposer d’autres.\par
Il n’en est pourtant pas moins vrai que la liberté en fournit d’un autre ordre qui déterminent, sans effort et sans contrainte, les actions dont la crainte est le motif dans les systèmes serviles ; celle-ci comprime les penchants sans en opérer la subordination ; l’autre se les subordonne ; et, quand une fois elle se les est appropriés, l’homme pratique volontairement toutes les vertus sociales qui coûtent tant à remplir, et auxquelles on manque si souvent quand elles sont observées par d’autres considérations.\par
Quelle est, en effet, l’espèce de garantie que vous donnent vos systèmes d’obéissance et de contrainte ? Malgré toutes vos précautions, en dormez-vous plus tranquilles ; n’êtes-vous pas dans des alarmes continuelles ? La sécurité règne-t-elle parmi vous ? Non, elle n’y règne point, et ne peut y régner ; car, avec toutes vos terreurs civiles et religieuses, vous asservissez l’homme, mais vous ne le moralisez pas. Ses penchants n’en sont pas moins absolus ; ils n’en dominent pas moins dans son économie intérieure, parce qu’ils sont retenus sans être subordonnés ; et chacun ayant en soi la conscience de leur insubordination, prend une fausse idée de la nature humaine : de là cette méfiance que les hommes, dans cet état, ont les uns pour les autres ; ils se regardent comme naturellement enclins à se nuire. Cette opinion, fondée sur ce qu’ils sont, les tient dans la réserve et dans la circonspection : quelque moyen qu’on emploie pour les rassurer, les douceurs de la confiance leur sont interdites. La sévérité des lois, la force réprimante, le frein de l’opinion, les menaces théologiques ; en un mot, quelque chose qu’on fasse pour exagérer le puissant ressort de la crainte, tous ces moyens, en apparence si énergiques, n’empêchent point l’homme de se troubler à l’approche de l’homme, ou du moins de se mettre sur ses gardes ; d’appréhender sa rencontre dans les lieux déserts, de se barricader pour qu’on ne viole pas son asile à la faveur ombres de la nuit. Ces moyens, dont on vante tant l’efficacité, n’empêchent pas que, dans le commerce habituel de la vie, il ne faille se méfier des démonstrations les plus bienveillantes, supposer partout la fraude, et avoir rarement le malheur de se tromper dans ses suppositions.\par
C’est que vous avez substitué au principe naturel de moralité un principe contraire à la nature de l’homme ; vous aurez beau le renforcer, il ne produira jamais l’effet que vous en attendez, parce qu’il est faux dans son application. Avant qu’on s’avisât de l’employer, la société n’était jamais troublée par ses propres membres : l’ordre et la sécurité régnaient dans son sein ; le respect des personnes et des propriétés avait lieu ; l’on ne soupçonnait pas même qu’on pût y déroger : il a fallu pour cela que le principe de l’obéissance succédât à celui de la liberté ; qu’on commandât ce qui se pratiquait auparavant sans qu’on l’eût ordonné ; dès lors on s’est vu réduit à ne plus compter sur l’observation de la chose ordonnée ; il a fallu prendre des précautions pour se mettre à l’abri des violences qu’on ne redoutait point auparavant. En invirtualisant l’homme, ce système l’a démoralisé, l’a mis hors de sa loi, et l’a fait tout autre qu’il était par sa nature.\par
Voyez, en effet, les peuples qui vivent encore dans l’état naturel de la société : connaissent-ils nos craintes, nos précautions ? sont-ils, comme nous, dans une insécurité perpétuelle, au milieu de leurs concitoyens ? Hélas ! ils dorment les portes ouvertes ; ils n’ont pas besoin que des patrouilles nocturnes rôdent autour de leurs habitations pour en écarter le voleur ou l’assassin ; ils se rencontrent dans les bois sans être les uns pour les autres un objet d’épouvante. Si quelqu’un d’entre eux a tué quelque animal qu’il ne puisse emporter seul, il met son carquois dessus, et va chercher ses commensaux pour qu’ils l’aident à le transporter chez lui ; les autres passent dans cet intervalle, voient la proie, et n’y touchent point : c’est la propriété de leur frère\footnote{\noindent Cook, dans ses Voyages, parle continuellement des vols que lui faisaient les habitants des îles qu’il a parcourues ; mais, comme il le dit lui-même, c’était moins pour voler que pour satisfaire une curiosité enfantine sur des objets qu’ils n’avaient jamais vus ; ce qui le prouve, c’est que ces objets étaient le plus souvent rendus sans nulle contestation, et que l’action de les prendre n’entraînait aucune idée de délit parmi ces peuples : il est probable, quoi qu’en dise le voyageur, qu’ils n’ont aucune notion de propriété individuelle ; ce qui semble le confirmer, c’est que, dans un endroit de ses Voyages, il rapporte que tous les objets échangés individuellement furent retrouvés dans une même cabane ; d’où il est naturel de conclure que c’était le dépôt commun de la société.\par
Il est possible que ce célèbre navigateur ait mis beaucoup d’exactitude dans ses descriptions topographiques ; mais, à l’égard des mœurs et des usages, ses récits doivent nécessairement contenir bien des choses fausses ou hasardées : il parle de rois, de grands, de bas peuple dans des pays où certainement toutes ces distinctions ne sont pas connues ; il nous dit l’âge des chefs qui, sans doute, ne le savaient pas eux-mêmes, puisqu’il nous apprend, d’un autre côté, qu’ils n’ont aucun moyen artificiel de calculer la durée, et qu’ils ne peuvent se rappeler au juste ce qui l’est passé au-delà d’un an.\par
Comment décrire avec exactitude les mœurs et les habitudes d’un peuple dont on ignore absolument l’idiome, et chez lequel on ne reste que quelques jours. Quelque talent qu’on ait pour l’observation, on rapporte naturellement tout ce qu’on voit à ses propres idées ; on investit de ses préjugés des coutumes et des habitudes qui leur sont absolument étrangères : on donne des motifs à des choses qui n’en ont pas, ou qui en ont d’entièrement opposés à ceux qu’on leur prête : on crée des explications que la nature des choses désavoue ; et, en racontant les faits avec exactitude, on n’en fait pas moins un roman par la manière dont on les interprète ; c’est peut-être un écueil que ne peuvent éviter les voyageurs même les plus véridiques.
}. On sait cependant qu’ils supportent des abstinences qui, chez nous, légitimeraient ou excuseraient du moins le vol des aliments : on n’est donc pas fondé à dire, comme on l’a fait, que, s’ils ne volent point, c’est qu’il n’existe rien chez eux qui puisse tenter la cupidité ; car un mets n’est pas la chose la moins séduisante pour celui qui a faim. D’où vient donc cette retenue chez des gens qui ne craignent ni les galères dans ce monde ni l’enfer dans l’autre ? Donnez-leur ces deux grands motifs de rectitude et de moralité, et bientôt je ne réponds pas qu’on soit sûr de retrouver la bête qu’on aura laissée sous la sauvegarde du carquois : elle aura disparu, comme disparaissent chez nous les choses que l’imprévoyance laisse traîner : il faudra des serrures et des verrous. Adieu la confiance et la sécurité.\par
Peut-être regardera-t-on ceci comme le plus étrange de tous les paradoxes ; car, dira-t-on, multiplier les motifs, n’est-ce pas donner plus de certitude à l’effet ? Sans doute, mais ce n’est point là votre hypothèse ; vous ne multipliez pas les motifs, vous les changez ; ce que l’homme faisait par un principe et par des motifs puisés dans sa nature, vous le lui faites faire par un autre principe et par d’autres motifs : il faut nécessairement qu’il abandonne ceux-là pour adopter ceux-ci : il y a donc changement de système ; ce n’est pas, comme vous le croyez, multiplier les motifs, c’est leur en substituer d’autres. Or ceux que vous établissez ne dérivant point de la nature humaine, ne tenant point à ses pouvoirs virtuels, mais lui étant, pour ainsi dire, administrés du dehors, ne sont pas naturellement appropriés à la fin pour laquelle vous les instituez. Ils doivent donc le plus souvent manquer leur but ; quelque énergiques qu’ils vous paraissent, il ne faut pas être surpris que l’effet en soit toujours incertain, et qu’on ne puisse avoir en eux qu’une confiance douteuse.\par
Nous pouvons conclure de ce qui précède que ce n’est pas, comme on le croit, de l’affaiblissement, ni de la dégénération de ses principes naturels, que sont provenues l’altération et la corruption de la société, ou, pour mieux dire, qu’elle ne s’est pas dépravée dans son propre système : elle n’est devenue ce qu’elle est que parce qu’elle a passé dans un autre qui, ne lui étant pas approprié, en a changé l’essence et la destination. Le principe de l’obéissance, substitué à celui de la liberté, en a fait une autre institution basée sur d’autres motifs, et se conservait par d’autres moyens.\par
Comment revenir de cet état à celui dans lequel notre nature voudrait que nous fussions ? Comment rentrer dans la route que nous avons abandonnée ? quels sont les moyens à prendre, les obstacles à vaincre ? Telles sont les questions sur lesquelles nous essaierons de jeter quelque jour dans un autre temps ; mais, quelque difficulté que présentent de semblables recherches, il est peut-être plus difficile encore d’en faire sentir l’importance à des hommes imbus de tous les préjugés de la servitude, qui se sont familiarisés avec la dépendance, et que cette habitude a modifiés au point de leur faire redouter la liberté, comme il serait dans l’ordre naturel des choses qu’ils redoutassent l’esclavage.\par

\begin{center}
\noindent \centerline{FIN.}
\end{center}

\chapterclose

 


% at least one empty page at end (for booklet couv)
\ifbooklet
  \pagestyle{empty}
  \clearpage
  % 2 empty pages maybe needed for 4e cover
  \ifnum\modulo{\value{page}}{4}=0 \hbox{}\newpage\hbox{}\newpage\fi
  \ifnum\modulo{\value{page}}{4}=1 \hbox{}\newpage\hbox{}\newpage\fi


  \hbox{}\newpage
  \ifodd\value{page}\hbox{}\newpage\fi
  {\centering\color{rubric}\bfseries\noindent\large
    Hurlus ? Qu’est-ce.\par
    \bigskip
  }
  \noindent Des bouquinistes électroniques, pour du texte libre à participation libre,
  téléchargeable gratuitement sur \href{https://hurlus.fr}{\dotuline{hurlus.fr}}.\par
  \bigskip
  \noindent Cette brochure a été produite par des éditeurs bénévoles.
  Elle n’est pas faîte pour être possédée, mais pour être lue, et puis donnée.
  Que circule le texte !
  En page de garde, on peut ajouter une date, un lieu, un nom ; pour suivre le voyage des idées.
  \par

  Ce texte a été choisi parce qu’une personne l’a aimé,
  ou haï, elle a en tous cas pensé qu’il partipait à la formation de notre présent ;
  sans le souci de plaire, vendre, ou militer pour une cause.
  \par

  L’édition électronique est soigneuse, tant sur la technique
  que sur l’établissement du texte ; mais sans aucune prétention scolaire, au contraire.
  Le but est de s’adresser à tous, sans distinction de science ou de diplôme.
  Au plus direct ! (possible)
  \par

  Cet exemplaire en papier a été tiré sur une imprimante personnelle
   ou une photocopieuse. Tout le monde peut le faire.
  Il suffit de
  télécharger un fichier sur \href{https://hurlus.fr}{\dotuline{hurlus.fr}},
  d’imprimer, et agrafer ; puis de lire et donner.\par

  \bigskip

  \noindent PS : Les hurlus furent aussi des rebelles protestants qui cassaient les statues dans les églises catholiques. En 1566 démarra la révolte des gueux dans le pays de Lille. L’insurrection enflamma la région jusqu’à Anvers où les gueux de mer bloquèrent les bateaux espagnols.
  Ce fut une rare guerre de libération dont naquit un pays toujours libre : les Pays-Bas.
  En plat pays francophone, par contre, restèrent des bandes de huguenots, les hurlus, progressivement réprimés par la très catholique Espagne.
  Cette mémoire d’une défaite est éteinte, rallumons-la. Sortons les livres du culte universitaire, cherchons les idoles de l’époque, pour les briser.
\fi

\ifdev % autotext in dev mode
\fontname\font — \textsc{Les règles du jeu}\par
(\hyperref[utopie]{\underline{Lien}})\par
\noindent \initialiv{A}{lors là}\blindtext\par
\noindent \initialiv{À}{ la bonheur des dames}\blindtext\par
\noindent \initialiv{É}{tonnez-le}\blindtext\par
\noindent \initialiv{Q}{ualitativement}\blindtext\par
\noindent \initialiv{V}{aloriser}\blindtext\par
\Blindtext
\phantomsection
\label{utopie}
\Blinddocument
\fi
\end{document}
