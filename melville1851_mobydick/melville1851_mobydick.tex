%%%%%%%%%%%%%%%%%%%%%%%%%%%%%%%%%
% LaTeX model https://hurlus.fr %
%%%%%%%%%%%%%%%%%%%%%%%%%%%%%%%%%

% Needed before document class
\RequirePackage{pdftexcmds} % needed for tests expressions
\RequirePackage{fix-cm} % correct units

% Define mode
\def\mode{a4}

\newif\ifaiv % a4
\newif\ifav % a5
\newif\ifbooklet % booklet
\newif\ifcover % cover for booklet

\ifnum \strcmp{\mode}{cover}=0
  \covertrue
\else\ifnum \strcmp{\mode}{booklet}=0
  \booklettrue
\else\ifnum \strcmp{\mode}{a5}=0
  \avtrue
\else
  \aivtrue
\fi\fi\fi

\ifbooklet % do not enclose with {}
  \documentclass[french,twoside]{book} % ,notitlepage
  \usepackage[%
    papersize={105mm, 297mm},
    inner=12mm,
    outer=12mm,
    top=20mm,
    bottom=15mm,
    marginparsep=0pt,
  ]{geometry}
  \usepackage[fontsize=9.5pt]{scrextend} % for Roboto
\else\ifav
  \documentclass[french,twoside]{book} % ,notitlepage
  \usepackage[%
    a5paper,
    inner=25mm,
    outer=15mm,
    top=15mm,
    bottom=15mm,
    marginparsep=0pt,
  ]{geometry}
  \usepackage[fontsize=12pt]{scrextend}
\else% A4 2 cols
  \documentclass[twocolumn]{report}
  \usepackage[%
    a4paper,
    inner=15mm,
    outer=10mm,
    top=25mm,
    bottom=18mm,
    marginparsep=0pt,
  ]{geometry}
  \setlength{\columnsep}{20mm}
  \usepackage[fontsize=9.5pt]{scrextend}
\fi\fi

%%%%%%%%%%%%%%
% Alignments %
%%%%%%%%%%%%%%
% before teinte macros

\setlength{\arrayrulewidth}{0.2pt}
\setlength{\columnseprule}{\arrayrulewidth} % twocol
\setlength{\parskip}{0pt} % classical para with no margin
\setlength{\parindent}{1.5em}

%%%%%%%%%%
% Colors %
%%%%%%%%%%
% before Teinte macros

\usepackage[dvipsnames]{xcolor}
\definecolor{rubric}{HTML}{800000} % the tonic 0c71c3
\def\columnseprulecolor{\color{rubric}}
\colorlet{borderline}{rubric!30!} % definecolor need exact code
\definecolor{shadecolor}{gray}{0.95}
\definecolor{bghi}{gray}{0.5}

%%%%%%%%%%%%%%%%%
% Teinte macros %
%%%%%%%%%%%%%%%%%
%%%%%%%%%%%%%%%%%%%%%%%%%%%%%%%%%%%%%%%%%%%%%%%%%%%
% <TEI> generic (LaTeX names generated by Teinte) %
%%%%%%%%%%%%%%%%%%%%%%%%%%%%%%%%%%%%%%%%%%%%%%%%%%%
% This template is inserted in a specific design
% It is XeLaTeX and otf fonts

\makeatletter % <@@@


\usepackage{blindtext} % generate text for testing
\usepackage[strict]{changepage} % for modulo 4
\usepackage{contour} % rounding words
\usepackage[nodayofweek]{datetime}
% \usepackage{DejaVuSans} % seems buggy for sffont font for symbols
\usepackage{enumitem} % <list>
\usepackage{etoolbox} % patch commands
\usepackage{fancyvrb}
\usepackage{fancyhdr}
\usepackage{float}
\usepackage{fontspec} % XeLaTeX mandatory for fonts
\usepackage{footnote} % used to capture notes in minipage (ex: quote)
\usepackage{framed} % bordering correct with footnote hack
\usepackage{graphicx}
\usepackage{lettrine} % drop caps
\usepackage{lipsum} % generate text for testing
\usepackage[framemethod=tikz,]{mdframed} % maybe used for frame with footnotes inside
\usepackage{pdftexcmds} % needed for tests expressions
\usepackage{polyglossia} % non-break space french punct, bug Warning: "Failed to patch part"
\usepackage[%
  indentfirst=false,
  vskip=1em,
  noorphanfirst=true,
  noorphanafter=true,
  leftmargin=\parindent,
  rightmargin=0pt,
]{quoting}
\usepackage{ragged2e}
\usepackage{setspace} % \setstretch for <quote>
\usepackage{tabularx} % <table>
\usepackage[explicit]{titlesec} % wear titles, !NO implicit
\usepackage{tikz} % ornaments
\usepackage{tocloft} % styling tocs
\usepackage[fit]{truncate} % used im runing titles
\usepackage{unicode-math}
\usepackage[normalem]{ulem} % breakable \uline, normalem is absolutely necessary to keep \emph
\usepackage{verse} % <l>
\usepackage{xcolor} % named colors
\usepackage{xparse} % @ifundefined
\XeTeXdefaultencoding "iso-8859-1" % bad encoding of xstring
\usepackage{xstring} % string tests
\XeTeXdefaultencoding "utf-8"
\PassOptionsToPackage{hyphens}{url} % before hyperref, which load url package

% TOTEST
% \usepackage{hypcap} % links in caption ?
% \usepackage{marginnote}
% TESTED
% \usepackage{background} % doesn’t work with xetek
% \usepackage{bookmark} % prefers the hyperref hack \phantomsection
% \usepackage[color, leftbars]{changebar} % 2 cols doc, impossible to keep bar left
% \usepackage[utf8x]{inputenc} % inputenc package ignored with utf8 based engines
% \usepackage[sfdefault,medium]{inter} % no small caps
% \usepackage{firamath} % choose firasans instead, firamath unavailable in Ubuntu 21-04
% \usepackage{flushend} % bad for last notes, supposed flush end of columns
% \usepackage[stable]{footmisc} % BAD for complex notes https://texfaq.org/FAQ-ftnsect
% \usepackage{helvet} % not for XeLaTeX
% \usepackage{multicol} % not compatible with too much packages (longtable, framed, memoir…)
% \usepackage[default,oldstyle,scale=0.95]{opensans} % no small caps
% \usepackage{sectsty} % \chapterfont OBSOLETE
% \usepackage{soul} % \ul for underline, OBSOLETE with XeTeX
% \usepackage[breakable]{tcolorbox} % text styling gone, footnote hack not kept with breakable


% Metadata inserted by a program, from the TEI source, for title page and runing heads
\title{\textbf{ Moby Dick }}
\date{}
\author{Melville, Hermann}
\def\elbibl{Melville, Hermann. \emph{Moby Dick}}

% Default metas
\newcommand{\colorprovide}[2]{\@ifundefinedcolor{#1}{\colorlet{#1}{#2}}{}}
\colorprovide{rubric}{red}
\colorprovide{silver}{lightgray}
\@ifundefined{syms}{\newfontfamily\syms{DejaVu Sans}}{}
\newif\ifdev
\@ifundefined{elbibl}{% No meta defined, maybe dev mode
  \newcommand{\elbibl}{Titre court ?}
  \newcommand{\elbook}{Titre du livre source ?}
  \newcommand{\elabstract}{Résumé\par}
  \newcommand{\elurl}{http://oeuvres.github.io/elbook/2}
  \author{Éric Lœchien}
  \title{Un titre de test assez long pour vérifier le comportement d’une maquette}
  \date{1566}
  \devtrue
}{}
\let\eltitle\@title
\let\elauthor\@author
\let\eldate\@date


\defaultfontfeatures{
  % Mapping=tex-text, % no effect seen
  Scale=MatchLowercase,
  Ligatures={TeX,Common},
}


% generic typo commands
\newcommand{\astermono}{\medskip\centerline{\color{rubric}\large\selectfont{\syms ✻}}\medskip\par}%
\newcommand{\astertri}{\medskip\par\centerline{\color{rubric}\large\selectfont{\syms ✻\,✻\,✻}}\medskip\par}%
\newcommand{\asterism}{\bigskip\par\noindent\parbox{\linewidth}{\centering\color{rubric}\large{\syms ✻}\\{\syms ✻}\hskip 0.75em{\syms ✻}}\bigskip\par}%

% lists
\newlength{\listmod}
\setlength{\listmod}{\parindent}
\setlist{
  itemindent=!,
  listparindent=\listmod,
  labelsep=0.2\listmod,
  parsep=0pt,
  % topsep=0.2em, % default topsep is best
}
\setlist[itemize]{
  label=—,
  leftmargin=0pt,
  labelindent=1.2em,
  labelwidth=0pt,
}
\setlist[enumerate]{
  label={\bf\color{rubric}\arabic*.},
  labelindent=0.8\listmod,
  leftmargin=\listmod,
  labelwidth=0pt,
}
\newlist{listalpha}{enumerate}{1}
\setlist[listalpha]{
  label={\bf\color{rubric}\alph*.},
  leftmargin=0pt,
  labelindent=0.8\listmod,
  labelwidth=0pt,
}
\newcommand{\listhead}[1]{\hspace{-1\listmod}\emph{#1}}

\renewcommand{\hrulefill}{%
  \leavevmode\leaders\hrule height 0.2pt\hfill\kern\z@}

% General typo
\DeclareTextFontCommand{\textlarge}{\large}
\DeclareTextFontCommand{\textsmall}{\small}

% commands, inlines
\newcommand{\anchor}[1]{\Hy@raisedlink{\hypertarget{#1}{}}} % link to top of an anchor (not baseline)
\newcommand\abbr[1]{#1}
\newcommand{\autour}[1]{\tikz[baseline=(X.base)]\node [draw=rubric,thin,rectangle,inner sep=1.5pt, rounded corners=3pt] (X) {\color{rubric}#1};}
\newcommand\corr[1]{#1}
\newcommand{\ed}[1]{ {\color{silver}\sffamily\footnotesize (#1)} } % <milestone ed="1688"/>
\newcommand\expan[1]{#1}
\newcommand\foreign[1]{\emph{#1}}
\newcommand\gap[1]{#1}
\renewcommand{\LettrineFontHook}{\color{rubric}}
\newcommand{\initial}[2]{\lettrine[lines=2, loversize=0.3, lhang=0.3]{#1}{#2}}
\newcommand{\initialiv}[2]{%
  \let\oldLFH\LettrineFontHook
  % \renewcommand{\LettrineFontHook}{\color{rubric}\ttfamily}
  \IfSubStr{QJ’}{#1}{
    \lettrine[lines=4, lhang=0.2, loversize=-0.1, lraise=0.2]{\smash{#1}}{#2}
  }{\IfSubStr{É}{#1}{
    \lettrine[lines=4, lhang=0.2, loversize=-0, lraise=0]{\smash{#1}}{#2}
  }{\IfSubStr{ÀÂ}{#1}{
    \lettrine[lines=4, lhang=0.2, loversize=-0, lraise=0, slope=0.6em]{\smash{#1}}{#2}
  }{\IfSubStr{A}{#1}{
    \lettrine[lines=4, lhang=0.2, loversize=0.2, slope=0.6em]{\smash{#1}}{#2}
  }{\IfSubStr{V}{#1}{
    \lettrine[lines=4, lhang=0.2, loversize=0.2, slope=-0.5em]{\smash{#1}}{#2}
  }{
    \lettrine[lines=4, lhang=0.2, loversize=0.2]{\smash{#1}}{#2}
  }}}}}
  \let\LettrineFontHook\oldLFH
}
\newcommand{\labelchar}[1]{\textbf{\color{rubric} #1}}
\newcommand{\milestone}[1]{\autour{\footnotesize\color{rubric} #1}} % <milestone n="4"/>
\newcommand\name[1]{#1}
\newcommand\orig[1]{#1}
\newcommand\orgName[1]{#1}
\newcommand\persName[1]{#1}
\newcommand\placeName[1]{#1}
\newcommand{\pn}[1]{\IfSubStr{-—–¶}{#1}% <p n="3"/>
  {\noindent{\bfseries\color{rubric}   ¶  }}
  {{\footnotesize\autour{ #1}  }}}
\newcommand\reg{}
% \newcommand\ref{} % already defined
\newcommand\sic[1]{#1}
\newcommand\surname[1]{\textsc{#1}}
\newcommand\term[1]{\textbf{#1}}

\def\mednobreak{\ifdim\lastskip<\medskipamount
  \removelastskip\nopagebreak\medskip\fi}
\def\bignobreak{\ifdim\lastskip<\bigskipamount
  \removelastskip\nopagebreak\bigskip\fi}

% commands, blocks
\newcommand{\byline}[1]{\bigskip{\RaggedLeft{#1}\par}\bigskip}
\newcommand{\bibl}[1]{{\RaggedLeft{#1}\par\bigskip}}
\newcommand{\biblitem}[1]{{\noindent\hangindent=\parindent   #1\par}}
\newcommand{\dateline}[1]{\medskip{\RaggedLeft{#1}\par}\bigskip}
\newcommand{\labelblock}[1]{\medbreak{\noindent\color{rubric}\bfseries #1}\par\mednobreak}
\newcommand{\salute}[1]{\bigbreak{#1}\par\medbreak}
\newcommand{\signed}[1]{\bigbreak\filbreak{\raggedleft #1\par}\medskip}

% environments for blocks (some may become commands)
\newenvironment{borderbox}{}{} % framing content
\newenvironment{citbibl}{\ifvmode\hfill\fi}{\ifvmode\par\fi }
\newenvironment{docAuthor}{\ifvmode\vskip4pt\fontsize{16pt}{18pt}\selectfont\fi\itshape}{\ifvmode\par\fi }
\newenvironment{docDate}{}{\ifvmode\par\fi }
\newenvironment{docImprint}{\vskip6pt}{\ifvmode\par\fi }
\newenvironment{docTitle}{\vskip6pt\bfseries\fontsize{18pt}{22pt}\selectfont}{\par }
\newenvironment{msHead}{\vskip6pt}{\par}
\newenvironment{msItem}{\vskip6pt}{\par}
\newenvironment{titlePart}{}{\par }


% environments for block containers
\newenvironment{argument}{\itshape\parindent0pt}{\vskip1.5em}
\newenvironment{biblfree}{}{\ifvmode\par\fi }
\newenvironment{bibitemlist}[1]{%
  \list{\@biblabel{\@arabic\c@enumiv}}%
  {%
    \settowidth\labelwidth{\@biblabel{#1}}%
    \leftmargin\labelwidth
    \advance\leftmargin\labelsep
    \@openbib@code
    \usecounter{enumiv}%
    \let\p@enumiv\@empty
    \renewcommand\theenumiv{\@arabic\c@enumiv}%
  }
  \sloppy
  \clubpenalty4000
  \@clubpenalty \clubpenalty
  \widowpenalty4000%
  \sfcode`\.\@m
}%
{\def\@noitemerr
  {\@latex@warning{Empty `bibitemlist' environment}}%
\endlist}
\newenvironment{quoteblock}% may be used for ornaments
  {\begin{quoting}}
  {\end{quoting}}

% table () is preceded and finished by custom command
\newcommand{\tableopen}[1]{%
  \ifnum\strcmp{#1}{wide}=0{%
    \begin{center}
  }
  \else\ifnum\strcmp{#1}{long}=0{%
    \begin{center}
  }
  \else{%
    \begin{center}
  }
  \fi\fi
}
\newcommand{\tableclose}[1]{%
  \ifnum\strcmp{#1}{wide}=0{%
    \end{center}
  }
  \else\ifnum\strcmp{#1}{long}=0{%
    \end{center}
  }
  \else{%
    \end{center}
  }
  \fi\fi
}


% text structure
\newcommand\chapteropen{} % before chapter title
\newcommand\chaptercont{} % after title, argument, epigraph…
\newcommand\chapterclose{} % maybe useful for multicol settings
\setcounter{secnumdepth}{-2} % no counters for hierarchy titles
\setcounter{tocdepth}{5} % deep toc
\markright{\@title} % ???
\markboth{\@title}{\@author} % ???
\renewcommand\tableofcontents{\@starttoc{toc}}
% toclof format
% \renewcommand{\@tocrmarg}{0.1em} % Useless command?
% \renewcommand{\@pnumwidth}{0.5em} % {1.75em}
\renewcommand{\@cftmaketoctitle}{}
\setlength{\cftbeforesecskip}{\z@ \@plus.2\p@}
\renewcommand{\cftchapfont}{}
\renewcommand{\cftchapdotsep}{\cftdotsep}
\renewcommand{\cftchapleader}{\normalfont\cftdotfill{\cftchapdotsep}}
\renewcommand{\cftchappagefont}{\bfseries}
\setlength{\cftbeforechapskip}{0em \@plus\p@}
% \renewcommand{\cftsecfont}{\small\relax}
\renewcommand{\cftsecpagefont}{\normalfont}
% \renewcommand{\cftsubsecfont}{\small\relax}
\renewcommand{\cftsecdotsep}{\cftdotsep}
\renewcommand{\cftsecpagefont}{\normalfont}
\renewcommand{\cftsecleader}{\normalfont\cftdotfill{\cftsecdotsep}}
\setlength{\cftsecindent}{1em}
\setlength{\cftsubsecindent}{2em}
\setlength{\cftsubsubsecindent}{3em}
\setlength{\cftchapnumwidth}{1em}
\setlength{\cftsecnumwidth}{1em}
\setlength{\cftsubsecnumwidth}{1em}
\setlength{\cftsubsubsecnumwidth}{1em}

% footnotes
\newif\ifheading
\newcommand*{\fnmarkscale}{\ifheading 0.70 \else 1 \fi}
\renewcommand\footnoterule{\vspace*{0.3cm}\hrule height \arrayrulewidth width 3cm \vspace*{0.3cm}}
\setlength\footnotesep{1.5\footnotesep} % footnote separator
\renewcommand\@makefntext[1]{\parindent 1.5em \noindent \hb@xt@1.8em{\hss{\normalfont\@thefnmark . }}#1} % no superscipt in foot
\patchcmd{\@footnotetext}{\footnotesize}{\footnotesize\sffamily}{}{} % before scrextend, hyperref


%   see https://tex.stackexchange.com/a/34449/5049
\def\truncdiv#1#2{((#1-(#2-1)/2)/#2)}
\def\moduloop#1#2{(#1-\truncdiv{#1}{#2}*#2)}
\def\modulo#1#2{\number\numexpr\moduloop{#1}{#2}\relax}

% orphans and widows
\clubpenalty=9996
\widowpenalty=9999
\brokenpenalty=4991
\predisplaypenalty=10000
\postdisplaypenalty=1549
\displaywidowpenalty=1602
\hyphenpenalty=400
% Copied from Rahtz but not understood
\def\@pnumwidth{1.55em}
\def\@tocrmarg {2.55em}
\def\@dotsep{4.5}
\emergencystretch 3em
\hbadness=4000
\pretolerance=750
\tolerance=2000
\vbadness=4000
\def\Gin@extensions{.pdf,.png,.jpg,.mps,.tif}
% \renewcommand{\@cite}[1]{#1} % biblio

\usepackage{hyperref} % supposed to be the last one, :o) except for the ones to follow
\urlstyle{same} % after hyperref
\hypersetup{
  % pdftex, % no effect
  pdftitle={\elbibl},
  % pdfauthor={Your name here},
  % pdfsubject={Your subject here},
  % pdfkeywords={keyword1, keyword2},
  bookmarksnumbered=true,
  bookmarksopen=true,
  bookmarksopenlevel=1,
  pdfstartview=Fit,
  breaklinks=true, % avoid long links
  pdfpagemode=UseOutlines,    % pdf toc
  hyperfootnotes=true,
  colorlinks=false,
  pdfborder=0 0 0,
  % pdfpagelayout=TwoPageRight,
  % linktocpage=true, % NO, toc, link only on page no
}

\makeatother % /@@@>
%%%%%%%%%%%%%%
% </TEI> end %
%%%%%%%%%%%%%%


%%%%%%%%%%%%%
% footnotes %
%%%%%%%%%%%%%
\renewcommand{\thefootnote}{\bfseries\textcolor{rubric}{\arabic{footnote}}} % color for footnote marks

%%%%%%%%%
% Fonts %
%%%%%%%%%
\usepackage[]{roboto} % SmallCaps, Regular is a bit bold
% \linespread{0.90} % too compact, keep font natural
\newfontfamily\fontrun[]{Roboto Condensed Light} % condensed runing heads
\ifav
  \setmainfont[
    ItalicFont={Roboto Light Italic},
  ]{Roboto}
\else\ifbooklet
  \setmainfont[
    ItalicFont={Roboto Light Italic},
  ]{Roboto}
\else
\setmainfont[
  ItalicFont={Roboto Italic},
]{Roboto Light}
\fi\fi
\renewcommand{\LettrineFontHook}{\bfseries\color{rubric}}
% \renewenvironment{labelblock}{\begin{center}\bfseries\color{rubric}}{\end{center}}

%%%%%%%%
% MISC %
%%%%%%%%

\setdefaultlanguage[frenchpart=false]{french} % bug on part


\newenvironment{quotebar}{%
    \def\FrameCommand{{\color{rubric!10!}\vrule width 0.5em} \hspace{0.9em}}%
    \def\OuterFrameSep{\itemsep} % séparateur vertical
    \MakeFramed {\advance\hsize-\width \FrameRestore}
  }%
  {%
    \endMakeFramed
  }
\renewenvironment{quoteblock}% may be used for ornaments
  {%
    \savenotes
    \setstretch{0.9}
    \normalfont
    \begin{quotebar}
  }
  {%
    \end{quotebar}
    \spewnotes
  }


\renewcommand{\headrulewidth}{\arrayrulewidth}
\renewcommand{\headrule}{{\color{rubric}\hrule}}

% delicate tuning, image has produce line-height problems in title on 2 lines
\titleformat{name=\chapter} % command
  [display] % shape
  {\vspace{1.5em}\centering} % format
  {} % label
  {0pt} % separator between n
  {}
[{\color{rubric}\huge\textbf{#1}}\bigskip] % after code
% \titlespacing{command}{left spacing}{before spacing}{after spacing}[right]
\titlespacing*{\chapter}{0pt}{-2em}{0pt}[0pt]

\titleformat{name=\section}
  [block]{}{}{}{}
  [\vbox{\color{rubric}\large\raggedleft\textbf{#1}}]
\titlespacing{\section}{0pt}{0pt plus 4pt minus 2pt}{\baselineskip}

\titleformat{name=\subsection}
  [block]
  {}
  {} % \thesection
  {} % separator \arrayrulewidth
  {}
[\vbox{\large\textbf{#1}}]
% \titlespacing{\subsection}{0pt}{0pt plus 4pt minus 2pt}{\baselineskip}

\ifaiv
  \fancypagestyle{main}{%
    \fancyhf{}
    \setlength{\headheight}{1.5em}
    \fancyhead{} % reset head
    \fancyfoot{} % reset foot
    \fancyhead[L]{\truncate{0.45\headwidth}{\fontrun\elbibl}} % book ref
    \fancyhead[R]{\truncate{0.45\headwidth}{ \fontrun\nouppercase\leftmark}} % Chapter title
    \fancyhead[C]{\thepage}
  }
  \fancypagestyle{plain}{% apply to chapter
    \fancyhf{}% clear all header and footer fields
    \setlength{\headheight}{1.5em}
    \fancyhead[L]{\truncate{0.9\headwidth}{\fontrun\elbibl}}
    \fancyhead[R]{\thepage}
  }
\else
  \fancypagestyle{main}{%
    \fancyhf{}
    \setlength{\headheight}{1.5em}
    \fancyhead{} % reset head
    \fancyfoot{} % reset foot
    \fancyhead[RE]{\truncate{0.9\headwidth}{\fontrun\elbibl}} % book ref
    \fancyhead[LO]{\truncate{0.9\headwidth}{\fontrun\nouppercase\leftmark}} % Chapter title, \nouppercase needed
    \fancyhead[RO,LE]{\thepage}
  }
  \fancypagestyle{plain}{% apply to chapter
    \fancyhf{}% clear all header and footer fields
    \setlength{\headheight}{1.5em}
    \fancyhead[L]{\truncate{0.9\headwidth}{\fontrun\elbibl}}
    \fancyhead[R]{\thepage}
  }
\fi

\ifav % a5 only
  \titleclass{\section}{top}
\fi

\newcommand\chapo{{%
  \vspace*{-3em}
  \centering % no vskip ()
  {\Large\addfontfeature{LetterSpace=25}\bfseries{\elauthor}}\par
  \smallskip
  {\large\eldate}\par
  \bigskip
  {\Large\selectfont{\eltitle}}\par
  \bigskip
  {\color{rubric}\hline\par}
  \bigskip
  {\Large TEXTE LIBRE À PARTICPATION LIBRE\par}
  \centerline{\small\color{rubric} {hurlus.fr, tiré le \today}}\par
  \bigskip
}}

\newcommand\cover{{%
  \thispagestyle{empty}
  \centering
  {\LARGE\bfseries{\elauthor}}\par
  \bigskip
  {\Large\eldate}\par
  \bigskip
  \bigskip
  {\LARGE\selectfont{\eltitle}}\par
  \vfill\null
  {\color{rubric}\setlength{\arrayrulewidth}{2pt}\hline\par}
  \vfill\null
  {\Large TEXTE LIBRE À PARTICPATION LIBRE\par}
  \centerline{{\href{https://hurlus.fr}{\dotuline{hurlus.fr}}, tiré le \today}}\par
}}

\begin{document}
\pagestyle{empty}
\ifbooklet{
  \cover\newpage
  \thispagestyle{empty}\hbox{}\newpage
  \cover\newpage\noindent Les voyages de la brochure\par
  \bigskip
  \begin{tabularx}{\textwidth}{l|X|X}
    \textbf{Date} & \textbf{Lieu}& \textbf{Nom/pseudo} \\ \hline
    \rule{0pt}{25cm} &  &   \\
  \end{tabularx}
  \newpage
  \addtocounter{page}{-4}
}\fi

\thispagestyle{empty}
\ifaiv
  \twocolumn[\chapo]
\else
  \chapo
\fi
{\it\elabstract}
\bigskip
\makeatletter\@starttoc{toc}\makeatother % toc without new page
\bigskip

\pagestyle{main} % after style

  \section[{ÉTYMOLOGIE}]{ÉTYMOLOGIE}\renewcommand{\leftmark}{ÉTYMOLOGIE}

\noindent (fournie par un pion de collège qui mourut tuberculeux) \par
Je le revois, ce blême surveillant usé jusqu’à la transparence depuis ses vêtements jusqu’au cœur, au cerveau. Il époussetait éternellement ses vieux lexiques et ses grammaires avec un étrange mouchoir ironiquement égayé de tous les joyeux drapeaux de toutes les nations connues du monde. Il aimait à épousseter ses vieilles grammaires ; d’une certaine manière, cela lui rappelait avec douceur qu’il était mortel.\par
Quand vous assumez d’enseigner les autres et de leur apprendre comment appeler la baleine (whale-fish) en notre langue, omettant par ignorance la lettre H qui à elle seule contient presque tout le sens du mot, vous ne respectez pas la vérité.\par

\bibl{Hackluyt}
\noindent {\itshape Wahle…} suéd. et dan. {\itshape Hval.} Sa rondeur et sa nage en roulis valent son nom à cet animal ; car en danois hvalt signifie : en arche, en voûte.\par

\bibl{Dictionnaire de Webster}
\noindent Wahle… plus directement du hollandais et de \hspace{1em}l’allemand. Wallen ; A. S. Walw-ian : rouler, se vautrer. Hébreu {\itshape Khtoζ\hspace{1em}}Grec {\itshape Cetus\hspace{1em}}Latin {\itshape Whoel\hspace{1em}}Anglo-Saxon {\itshape Hvalt\hspace{1em}}Danois {\itshape Wal\hspace{1em}}Hollandais {\itshape Hwal\hspace{1em}}Suédois {\itshape Hvalur\hspace{1em}}Islandais {\itshape Whale\hspace{1em}}Anglais {\itshape Baleine\hspace{1em}}Français {\itshape Ballena\hspace{1em}}Espagnol {\itshape Pekee-Nuee-Nuee\hspace{1em}}Fidjien {\itshape Pehee-Nuee-Nuee\hspace{1em}}Dialecte d’Erromango
\section[{EXTRAITS}]{EXTRAITS}\renewcommand{\leftmark}{EXTRAITS}

\noindent (fournis par un très obscur bibliothécaire) \par
Tout semble prouver que ce personnage falot, fouineur acharné, ce pauvre diable de très obscur bibliothécaire parcourut d’interminables galeries de la Bibliothèque vaticane et tous les étalages de livres de la terre, glanant au hasard les moindres allusions qu’il pouvait tant bien que mal trouver dans n’importe quel livre tant sacré que profane. Aussi ne devez-vous pas tenir indifféremment dans cette cueillette toute assertion pour parole d’un évangile de cétologie. Loin de là. En ce qui concerne les auteurs anciens et les poètes en général, les extraits, cités ici, n’ont d’autre valeur et d’autre attrait que ceux que leur confère une vue d’ensemble sur les pièces et morceaux de ce qu’ont dit, pensé, imaginé et chanté sur le Léviathan de nombreuses nations et de nombreuses générations dont la nôtre.\par
Alors, adieu, porte-toi bien, pauvre diable de bibliothécaire très obscur dont je suis le commentateur. Tu appartiens à la race blafarde et incurable qu’aucun vin de ce monde ne saurait jamais réchauffer ; et pour qui le pâle Xérès aurait une trop capiteuse rutilance ; mais, auprès de toi, l’on aime à s’asseoir parfois et à se sentir un identique pauvre diable, à communier dans les larmes et, les yeux noyés et secs les verres, étreint d’une tristesse non point absolument déplaisante, affirmer carrément : Renonce, sous-fifre ! Car plus tu te donnes de peine pour faire plaisir au monde, plus tu grossiras le nombre de ceux qui ne t’en auront jamais de reconnaissance. Je voudrais pouvoir pour toi vider Hampton Court et les Tuileries ! Mais ravalez vos larmes et hissez vos cœurs jusqu’au sommet du mât de cacatois ; car les amis vous y ont précédés font reluire à votre intention les sept demeures célestes et, en vue de votre arrivée, mettent au ban Gabriel, Michel et Raphaël si longuement choyés. Ici-bas l’on ne trinque qu’avec des cœurs déjà brisés, mais là-haut vous lèverez des verres d’un cristal infrangible !\par
Et Dieu créa les grandes baleines.\par

\bibl{Genèse.}
\noindent Léviathan laisse derrière lui un sillage lumineux l’abîme semble couvert d’une toison blanche.\par

\bibl{Job.}
\noindent L’Éternel fit venir un grand poisson qui engloutit Jonas\par

\bibl{Jonas.}
\noindent Là se promènent les navires\par
Et ce Léviathan que tu as formé pour se jouer dans les flots.\par

\bibl{Psaumes.}
\noindent Ce jour-là, Yahvé châtiera de son épée dure, grande et forte Léviathan, serpent fuyard,\par
Léviathan, serpent tortueux ; Et il tuera le dragon de la mer.\par

\bibl{Isaïe.}
\noindent Et toute chose quelle qu’elle soit, venant à s’approcher du gouffre de la gueule de ce monstre, bête, navire ou roc, sombre tout aussitôt dans son gosier horrible et périt dans l’antre sans fond de sa panse.\par

\bibl{PLUTARQUE, \emph{Trad. Holland.}}
\noindent L’océan Indien enfante les poissons les plus divers et les plus grands qui soient, dont les cétacés et les tourbillons d’eau dits baleines dont la longueur atteint quatre acres ou arpents de terre.\par

\bibl{PLINE L’ANCIEN, \emph{Trad. Holland.}}
\noindent Nous n’étions pas en mer depuis deux jours qu’au lever du soleil apparurent des baleines et d’autres monstres marins. Parmi les premières, il y en avait une d’une taille monstrueuse… Elle avançait sur nous, la gueule ouverte, soulevant des vagues de tous côtés et creusant devant elle, un sillon écumant.\par

\bibl{LUCIEN : \emph{De la manière d’écrire l’histoire. Trad. Tooke.}}
\noindent Il visita ce pays avec l’arrière-pensée de prendre des chevaux marins qui ont au lieu de dents des os de grand prix dont il offrit quelques-uns au roi… Les meilleures baleines furent prises dans son propre pays, certaines mesuraient environ 48 à 50 mètres de long. Il disait avoir été l’un des six hommes qui en avaient tué soixante en deux jours.\par
{\itshape Le Périple d’Other ou Othere, récit recueilli par le Roi Alfred le Grand. A. D. 890.}- \par
… au lieu que tout autre chose, soit beste ou vaisseau, qui entre dans l’horrible chaos de la bouche de ce monstre (la baleine), est incontinent perdu et englouti, ce petit poisson (le gayon de mer) s’y retire en toute seureté et y dort…\par

\bibl{MONTAIGNE, \emph{Apologie de Raimond Sebond.}}
\noindent {\itshape « …} c’est, par la mort bœuf, léviathan descript par le noble prophète Moïse en la vie du saint homme Job ! Il nous avalera tous, et gens et naufz comme pillules. »\par

\bibl{RABELAIS.}
\noindent Le foie de cette baleine remplissait deux tombereaux.\par

\bibl{Annales de Stowe.}
\noindent Le grand Léviathan faisait bouillonner les mers comme une chaudière.\par

\bibl{Traduction des Psaumes par Bacon.}
\noindent Nous n’avons rien appris de certain au sujet du volume énorme de la baleine ou orque. Elles deviennent si grasses qu’on extrait une quantité incroyable d’huile d’une seule baleine.\par

\bibl{Ibid. Histoire de la Vie et de la Mort.}
\noindent Le remède souverain par excellence pour une contusion interne était le blanc de baleine.\par

\bibl{Le Roi Henri.}
\noindent Tout à fait comme une baleine.\par

\bibl{Hamlet.}
\noindent Aucun art médical n’aurait su guérir cette blessure Il lui fallait retourner à celle qui, d’un dard exquis,\par
Lui avait transpercé la poitrine, engendrant cette douleur lancinante\par
Pareille à celle qui pousse, à travers l’Océan, la baleine au rivage.\par

\bibl{ SPENSER, \emph{La Reine des Fées.}}
\noindent Immenses comme les baleines qui, par le mouvement de leurs corps énormes, troublent jusqu’au bouillonnement la plus paisible mer d’huile.\par

\bibl{SIR WILLIAM DAVENANT, \emph{Préface à Gondibert.}}
\noindent C’est à juste titre que les hommes peuvent demeurer dans le doute et ignorer ce qu’est le spermaceti, puisque le savant Hosmannus, dans son œuvre de trente années, dit clairement : Nescio quid sit.\par

\bibl{SIR T. BROWNE : \emph{Du Sperma Ceti et de la baleine à sperm ceti.}}
\noindent Pareil au Talus de Spenser avec son fléau moderne Sa puissante queue est menace de désastre Il porte leurs javelots fichés dans ses flancs Et un bouquet de piques fleurit sur son dos.\par

\bibl{WALLER, \emph{La Bataille des îles de Summer.}}
\noindent Ce grand Léviathan fut créé artificiellement qui a nom État (civitas en latin) et n’est rien de plus qu’un homme artificiel.\par

\bibl{Introduction de HOBBES à son \emph{Léviathan.}}
\noindent Ce sot de Mansoul l’avala sans mâcher comme si c’eût été un sprat dans la gueule d’une baleine.\par

\bibl{\emph{Le Voyage du Pèlerin}.}
\noindent Léviathan, que Dieu, de toutes ses créatures, fit la plus grande entre celles qui nagent dans le cours de l’Océan.\par

\bibl{\emph{Paradis Perdu}.}
\noindent Le Léviathan. La plus grande de toutes les créatures vivantes Étiré dans les profondeurs pareil à un promontoire,\par
Dort ou nage.\par
On dirait une terre mouvante ; il aspire toute une mer Par ses ouïes et la rejette en souffle.\par

\bibl{Ibid.}
\noindent Les puissantes baleines qui nagent dans une mer d’eau et portent en elle un océan d’huile.\par

\bibl{FULLER, \emph{L’État profane et l’État sacré.}}
\noindent Ainsi couchés derrière quelque promontoire Les Léviathans énormes attendent leurs proies Et sans lui laisser de chance avalent le fretin Qui s’égare entre leurs mâchoires béantes.\par

\bibl{DRYDEN, \emph{Annus Mirabilis.}}
\noindent Tandis que la baleine flotte à l’arrière du navire, ils lui coupent la tête et la remorquent en bateau aussi près que possible du rivage, mais douze ou treize pieds de profondeur suffisent à l’échouer.\par

\bibl{THOMAS EDGE, \emph{Dix Voyages au Spitzberg, dans Purchas.}}
\noindent Sur leur route, ils virent de nombreuses baleines s’amusant dans l’Océan, et faisant jaillir l’eau par les tuyaux et trous dont la nature a pourvu leurs épaules.\par

\bibl{\emph{Voyages d’Herbert en Asie et en Afrique, Harris : recueil de voyages}.}
\noindent Ils virent une troupe si nombreuse de baleines qu’ils furent contraints d’avancer avec les plus grandes précautions de crainte de précipiter leur navire sur elles…\par

\bibl{SCHOUTEN, \emph{Sixième Circumnavigation.}}
\noindent Nous fîmes voile de l’embouchure de l’Elbe, vent N. E., à bord du navire le {\itshape Jonas-dans-la-Baleine…} Certains disent que la baleine ne peut ouvrir la gueule, mais c’est une fable… Ils grimpent souvent aux mâts pour tenter d’apercevoir une baleine, car celui qui a la primeur de la découverte reçoit la récompense d’un ducat… On m’a raconté qu’une baleine prise près des Shetland avait dans le ventre plus d’une caque de harengs… L’un de nos harponneurs m’a dit avoir pris une fois, au Spitzberg, une baleine entièrement blanche.\par

\bibl{Un Voyage au Groenland, A. D. 1671, Harris, Recueil de Voyages.}
\noindent Plusieurs baleines se sont échouées sur cette côte (de Fife) en l’an 1652… L’une d’elles de l’espèce à fanons mesurait quatre-vingts pieds de long et à ce que l’on m’a dit) donna, outre une grande quantité d’huile, 500 livres de baleines. Ses mâchoires servent de portail au jardin de Pitfferren.\par

\bibl{SIBBALD, \emph{Fife et Kinross.}}
\noindent Je me suis engagé à essayer de maîtriser et de tuer ce cachalot car je n’ai jamais entendu dire qu’un cétacé de cette espèce ait jamais été tué par un homme tant sont grandes sa férocité et sa rapidité.\par

\bibl{RICHARD STRAFFORD, \emph{Lettre des Bermudes, Phil. Trans. A. D. 1668.}}
\noindent Les baleines dans la mer Obéissent à la voix de Dieu.\par

\bibl{Premier livre de lecture de la Nouvelle-Angleterre.}
\noindent Nous vîmes également en abondance de grandes baleines, car dans ces mers du sud, je dirais qu’on en rencontre cent contre une dans le nord.\par

\bibl{CAPITAINE COWLEY, \emph{Voyage autour du Monde, A. D. 1729.}}
\noindent … Et le souffle de la baleine exhale souvent une puanteur si intolérable qu’elle peut provoquer des troubles cérébraux.\par

\bibl{ULLOA, \emph{Amérique du Sud.}}
\noindent À cinquante sylphides élues avec une attention particulière Nous confions la mission importante du jupon Nous avons souvent vu tomber cette septuple barrière Renforcé pourtant de cerceaux et armées de baleines\par

\bibl{\emph{La Boucle de cheveux enlevée} (Pope).}
\noindent Si nous comparons la taille des animaux terrestres avec celle de ceux qui habitent les profondeurs des mers, la comparaison nous les fera paraître méprisables. La baleine est certainement le plus grand animal de la création.\par

\bibl{GOLDSMITH, \emph{Histoire Naturelle.}}
\noindent Si vous deviez écrire une fable pour les petits poissons, vous leur prêteriez le langage des grandes baleines.\par

\bibl{Goldsmith à Johnson.}
\noindent Dans l’après-midi, nous vîmes ce que nous avions présumé être un rocher et qui se révéla être une baleine que des Asiatiques avaient tuée et qu’ils remorquaient à la côte. Ils semblaient vouloir se dissimuler à nos regards en se cachant derrière le corps de la baleine.\par

\bibl{Les Voyages de Cook.}
\noindent Ils ont tant de peur de quelques-unes de ces baleines, que par une espèce de superstition, ils n’osent même pas les nommer quand ils sont en mer. Ils apportent avec eux dans les bateaux du fumier, du souffre et du genièvre, et toutes choses qui font fuir ces mammifères, et ils en répandent autour du bateau pour empêcher qu’ils ne s’approchent d’eux.\par

\bibl{\emph{Lettre d’UNO} DE TROIL \emph{sur le voyage de Banks et Solender en Islande en 1772.}}
\noindent Le cachalot découvert par les Nantuckais est un animal vif et féroce qui réclame de la part des pêcheurs une adresse extrême et de la témérité.\par

\bibl{\emph{Rapport de} THOMAS JEFFERSON \emph{sur la pêche à la baleine, présenté au Ministère de France en 1778.}}
\noindent Et, sir, je vous le demande, y a-t-il au monde quelque chose qui l’égale ?\par

\bibl{\emph{Discours d}’EDMUND BURKE \emph{au Parlement, rapport sur la pêche à la baleine à Nantucket.}}
\noindent Espagne – une grande baleine échouée sur les côtes d’Europe.\par

\bibl{EDMUND BURKE \emph{(quelque part…).}}
\noindent Une dixième source des revenus ordinaires du roi provient de son droit aux poissons royaux : la baleine et l’esturgeon, fondée sur la considération qu’il assure la protection des mers contre les pirates et les bandits. Ces poissons, lorsqu’ils sont rejetés à la côte ou péchés non loin de terre sont propriété du roi.\par

\bibl{BLACKSTONE.}
\noindent Déjà l’équipage s’apprête à reprendre le tournoi de la mort L’infaillible Rodmond, au-dessus de sa tête, élève Le harpon d’acier qui jamais ne manque sa proie\par

\bibl{FALCONNIER, \emph{Le Naufrage.}}
\noindent Toits, dômes et flèches étincelaient, Traversant seules le ciel, les fusées Suspendaient leur feu éphémère À la voûte nocturne Et pour rendre rivaux le feu et l’eau, Une baleine fait fuser dans les airs Pour exprimer sa maladroite joie Un bouquet d’Océan.\par

\bibl{COWPER, \emph{Visite de la Reine à Londres.}}
\noindent À chaque battement, le cœur envoie, à une vitesse extrême, dix ou quinze gallons de sang.\par

\bibl{JOHN HUNTER, \emph{Rapport sur la dissection d’une baleine de petite taille.}}
\noindent Le diamètre de l’aorte d’une baleine est plus grand que celui de la conduite d’eau principale de London-Bridge et le rugissement dans la canalisation est moindre en violence et en rapidité que celui du sang, jaillissant du cœur d’une baleine.\par

\bibl{PALEY, \emph{Théologie Naturelle.}}
\noindent Les cétacés sont les mammifères sans pieds de derrière ; leur tronc se continue avec une queue épaisse.\par

\bibl{BARON CUVIER,}
\noindent Par 40 degrés sud, nous vîmes des cachalots mais, la mer en étant alors envahie, nous n’en prîmes aucun jusqu’au 1\textsuperscript{er} mai.\par

\bibl{COLNETT, \emph{Voyage dans le but d’étendre la pêcherie au cachalot.}}
\noindent Dans le libre élément qui s’étendait sous moi Se débattaient et plongeaient, par jeu, chasse et combat Des poissons de toutes couleurs, toutes formes, toutes espèces Que les mots ne sauraient dépeindre et que jamais marin ne vit, Depuis le redoutable Léviathan Jusqu’aux êtres infimes qui peuplent chaque vague Serrés en bancs immenses, pareils à des îles flottantes. Et conduits par un mystérieux instinct à travers ce désert Désolé et bien que de toutes parts assaillis Par de voraces ennemis, Baleines, requins et monstres armés au front et aux mâchoires D’épées, de scies, de cornes spiralées ou de crocs recourbés.\par

\bibl{MONTGOMERY, \emph{Univers avant le Déluge.}}
\noindent Io ! péan ! Io ! chante À la gloire du roi des poissons. Le vaste Atlantique n’a point De plus puissante baleine ; Il n’est pas de plus gras poisson À s’ébattre dans la mer polaire.\par

\bibl{CHARLES LAMB, \emph{Triomphe de la Baleine.}}
\noindent En l’an 1690, quelques personnes observaient, du sommet d’une colline élevée, les ébats et les souffles des baleines, lorsque quelqu’un, désignant la mer, dit : voici le vert pâturage qui donnera leur pain aux petits-enfants de nos enfants.\par

\bibl{OBED MACY, \emph{Histoire de Nantucket.}}
\noindent J’ai construit pour Susan et moi un cotage et j’ai fait un portail en arc gothique avec les os d’une mâchoire de baleine.\par

\bibl{HAWTHORNE, \emph{Contes racontés deux fois.}}
\noindent Elle vint commander un monument qu’elle voulait dresser à la mémoire de son premier amour tué par une baleine dans l’océan Pacifique quarante ans auparavant.\par

\bibl{Ibid.}
\noindent {\itshape –} Un cachalot ! non, Monsieur, répondit Tom, c’est une baleine franche, j’ai vu son souffle faire, au-dessus de l’eau, une paire d’arc-en-ciel aussi jolis qu’un Chrétien peut en souhaiter voir. C’est une vraie barrique d’huile, celle-là !\par

\bibl{FENIMORE COOPER, \emph{Le Pilote.}}
\noindent On apporta des journaux, ils annonçaient que sur les théâtres de Berlin, on montrait des monstres marins et des baleines.\par

\bibl{Conversations de Goethe avec Eckermann. }
\noindent Seigneur ! M. Chace que se passe-t-il ? je répondis : nous avons été défoncés par un cachalot.\par

\bibl{\emph{Récit du naufrage du navire baleinier l’}« \emph{Essex » de Nantucket, qui fut attaqué et coulé par un grand cachalot dans l’océan Pacifique.}}

\bibl{\emph{Par} OWEN CHACE \emph{de Nantucket, premier second du navire susdit}. New York, 1821}
\noindent Une nuit qu’un marin était dans les haubans Le vent jouait un air de flûte Tandis qu’une pâle lune se voilait et se dévoilait Allumant le sillage argenté d’une baleine Qui dessinait sur la mer la trace de son jeu.\par

\bibl{ELISABETH OAKES SMITH.}
\noindent Pour la capture d’une seule baleine, la longueur des lignes filées par les diverses pirogues s’élevait à un total de 10.440 yards soit près de six milles anglais…\par
Parfois la baleine agite sa queue effrayante qui cingle l’air comme un fouet et dont le bruit résonne jusqu’à trois ou quatre milles.\par

\bibl{SCORESBY.}
\noindent Rendu fou par les douleurs infligées par ces nouvelles blessures, le cachalot en fureur roule sur lui-même et, dressant sa tête énorme, les mâchoires largement ouvertes, il les referme brutalement sur tout ce qui se trouve à sa portée ; il donne du front dans les pirogues, les projette devant lui à la vitesse de l’éclair, les défonçant parfois tout à fait.\par
… Il y a certes matière à s’étonner grandement que l’étude des mœurs d’un animal aussi digne d’intérêt et d’un rapport commercial aussi important que le cachalot ait été totalement dédaignée et ait suscité si peu de curiosité parmi les nombreux observateurs, certains fort compétents, qui auraient eu, au cours de ces dernières années, les occasions les plus abondantes de noter son comportement.\par

\bibl{THOMAS BEALE. \emph{Histoire Naturelle du cachalot, 1849.}}
\noindent Le cachalot est non seulement mieux armé que la baleine franche (ou du Groenland) et redoutablement armé aux deux extrémités de son corps mais encore il a une propension à en user bien souvent pour attaquer d’une manière à la fois si rusée, si téméraire et si méchante qu’on en est venu à le considérer comme le plus dangereux à chasser de tous les cétacés connus.\par

\bibl{FRÉDÉRIC DEBELL BENNETT. \emph{Expédition baleinière au}-}
\noindent {\itshape tour du Monde}, 1840.\par
13 octobre : Une voix claironna de la tête de mât : La voilà qui souffle ! À quelle distance ? demanda le capitaine. Trois points sous le vent par bâbord, sir. Redressez la barre ! Gouvernez droit ! Droit ! Ohé d’en haut ! La voyez-vous encore cette baleine.\par
Oui, oui, toute une troupe de cachalots ! Et la voilà qui souffle ! La voilà qui saute ! Donnez de la voix ! donnez de la voix chaque fois !\par
Oui, oui, sir ! La voilà qui souffle ! La voilà qui sououffle qui sou-ou-ou-ffle ! À quelle distance ? Deux milles et demi. Par tous les tonnerres ! si près ! Tout le monde sur le pont !\par

\bibl{J. ROSS BROWN. \emph{Gravures d’une croisière baleinière},}
\noindent 1846. \par
Le navire baleinier {\itshape le Globe}, à bord duquel eurent lieu les affreux événements dont nous allons rendre compte, avait son port d’attache dans l’île de Nantucket.\par

\bibl{\emph{Récit de} LAY ET HUSSEY, \emph{Survivants du Globe, A. D. 1828.}}
\noindent Poursuivi, une fois, par une baleine qu’il avait blessée, il la tint momentanément en respect avec une lance ; mais finalement le monstre furieux se rua sur le bateau ; lui-même et ses compagnons n’eurent la vie sauve que parce qu’en voyant que l’attaque ne pourrait être évitée, ils se jetèrent à l’eau.\par

\bibl{Journal des Missions de Tyerman et Benett.}
\noindent Monsieur Webster dit que « Nantucket est en soi d’un intérêt national aussi singulier que frappant. Sa population de huit à neuf mille personnes, vivant sur mer, ajoute largement chaque année à la prospérité nationale grâce à leur industrie qui réclame autant de courage que de persévérance. »\par

\bibl{\emph{Tiré du discours au Sénat des États-Unis dans lequel Daniel Webster lança une motion pour la construction d’une digue à Nantucket}, 1828.}
\noindent La baleine tomba droit sur lui et le tua probablement sur le coup.\par

\bibl{\emph{La baleine et ses chasseurs ou les Aventures du Baleinier et la biographie de la baleine, récits recueillis lors du voyage de retour du Commodore Preble par le} RÉV. HENRY T. CHEEVER.}
\noindent Si vous faites le moindre sacré bruit, répondit Samuel je vous expédie en enfer !\par

\bibl{\emph{Vie de Samuel Comstock (Le mutin) par son frère} WILLIAM COMSTOCK.}
\noindent Autre version de l’histoire du navire baleinier {\itshape Le Globe.}\par
Les expéditions hollandaises et britanniques dans l’Arctique, si elles échouèrent dans leur objectif majeur qui était de découvrir un passage vers l’Inde, révélèrent du moins les habitats de la baleine.\par

\bibl{\emph{Dictionnaire commercial} de Mc Culloch.}
\noindent En toutes choses existe la réciprocité, la balle ne bondit que pour rebondir, et en cherchant les lieux fréquentés par la baleine, les pêcheurs semblent avoir trouvé de nouveaux chenaux vers le mystérieux passage du Nord-Ouest.\par

\bibl{… \emph{inédit…}}
\noindent On ne peut rencontrer, en mer, un navire baleinier sans être frappé à son approche. La voilure réduite, ses hommes en vigie, perchés au sommet des mâts, scrutant ardemment l’étendue immense, le vaisseau offre une image toute différente de celle des long-courriers.\par

\bibl{« \emph{Des courants et de la pêche à la baleine. » Expéditions d’explorations faites par les États-Unis.}}
\noindent Les promeneurs des banlieues de Londres, ou d’ailleurs, se souviennent peut-être d’avoir vu fichés en terres d’immenses os recourbés, soit pour former une arche au-dessus d’un portail, ou l’entrée d’une tonnelle, et peut-être leur aura-t-on dit que c’étaient là des côtes de baleines.\par

\bibl{Récit d’une expédition baleinière dans l’Arctique.}
\noindent Lorsque les pirogues regagnèrent le navire, leur chasse terminée, les Blancs virent qu’il était tombé aux mains meurtrières des sauvages qui faisaient partie de l’équipage.\par

\bibl{Relation d’un journal au sujet de la prise et de la reprise du navire baleinier Hobomack.}
\noindent Bien des hommes d’équipage appartenant à un navire baleinier (américain) – c’est un fait de notoriété publique – désertent les navires sur lesquels ils étaient partis.\par

\bibl{Croisière sur une baleinière.}
\noindent Tout à coup, une masse puissante émergea des eaux et bondit à la verticale. C’était la baleine.\par

\bibl{Minant Coffin ou le pêcheur de baleines.}
\noindent La baleine, certes, est harponnée mais, songez-y, comment maîtriseriez-vous une fougueuse pouliche non rompue si elle n’avait qu’une corde attachée à la naissance de la queue.\par

\bibl{Tiré d’un chapitre sur la pêche à la baleine, dans « De la quille à la pomme de mât ».}
\noindent J’ai vu, une fois, deux de ces monstres (des baleines), sans doute le mâle et la femelle, nager lentement l’un derrière l’autre, à moins de distance d’un ricochet du rivage (en Terre de Feu) où le hêtre étendait ses branches.\par

\bibl{DARWIN. \emph{Voyage d’un naturaliste.}}
\noindent Sciez ! s’écria le second lorsque tournant la tête il vit les mâchoires béantes d’un immense cachalot à la proue de la pirogue la menaçant d’un destruction foudroyante. Sciez ! Il y va de vos vies !\par

\bibl{Wharton le tueur de baleines.}
\noindent Gardez votre bonne humeur, mes gars, ayez du cœur au ventre, Tandis que l’audacieux harponneur pique la baleine !\par

\bibl{Chanson nantuckaise.}
\noindent Ô le vieux et fier cachalot, dans tempête et la rafale Sera dans son Océan natal un géant de puissance Là où la puissance fait loi Et sera roi de la mer infinie.\par

\bibl{Chanson de baleinier.}

\chapteropen
\chapter[{CHAPITRE I. Mirages}]{CHAPITRE I \\
Mirages}\renewcommand{\leftmark}{CHAPITRE I \\
Mirages}


\chaptercont
\noindent Appelez-moi Ismaël. Voici quelques années – peu importe combien – le porte-monnaie vide ou presque, rien ne me retenant à terre, je songeai à naviguer un peu et à voir l’étendue liquide du globe. C’est une méthode à moi pour secouer la mélancolie et rajeunir le sang. Quand je sens s’abaisser le coin de mes lèvres, quand s’installe en mon âme le crachin d’un humide novembre, quand je me surprends à faire halte devant l’échoppe du fabricant de cercueils et à emboîter le pas à tout enterrement que je croise, et, plus particulièrement, lorsque mon hypocondrie me tient si fortement que je dois faire appel à tout mon sens moral pour me retenir de me ruer délibérément dans la rue, afin d’arracher systématiquement à tout un chacun son chapeau… alors, j’estime qu’il est grand temps pour moi de prendre la mer. Cela me tient lieu de balle et de pistolet. Caton se lance contre son épée avec un panache philosophique, moi, je m’embarque tranquillement. Il n’y a là rien de surprenant. S’ils en étaient conscients, presque tous les hommes ont, une fois ou l’autre, nourri, à leur manière, envers l’Océan, des sentiments pareils aux miens.\par
Voyez votre cité sur l’île de Manhattan, ceinturée de quais comme les récifs de corail entourent les îles des mers du sud, et que le commerce bat de toutes parts de son ressac. À droite et à gauche ses rues mènent à la mer. La Batterie forme l’extrême pointe de la ville basse, dont le noble môle est balayé par les vagues et les vents frais encore éloignés de la terre quelques heures auparavant. Voyez, se réunir là, la foule des badauds de la mer !\par
Flânez dans la ville par une rêveuse après-midi de Sabbat. Allez de Corlears Hook à Coenties Slip, de là poussez au nord par Whitehall. Que voyez-vous ? Sentinelles silencieuses, plantées partout dans la ville, des milliers et des milliers d’hommes sont figés dans des songes océaniques. Les uns sont adossés aux pilotis, les autres assis au bout des digues, certains se penchent vers les pavois des navires de la Chine, d’autres, comme pour mieux contempler la mer, se sont hissés dans les gréements. Mais tous sont des terriens, cloîtrés toute la semaine entre des cloisons de bois ou de plâtre… rivés à des comptoirs, cloués à des bancs, courbés sur des bureaux. Comment cela se fait-il ? Les vertes prairies ont-elles disparu ? Que font-ils là ?\par
Mais voyez ! voici que des foules nouvelles arrivent, fonçant droit vers l’eau, destinées, semble-t-il, à un plongeon. Étrange ! Rien ne paraît devoir les satisfaire hormis l’ultime limite de la terre, une halte dans l’ombre abritée des entrepôts ne leur suffit pas. Non. Il leur faut s’approcher de l’eau d’aussi près qu’ils le peuvent sans y tomber. Et ils sont là, échelonnés sur des milles, sur des lieues. Tous venus, de l’intérieur des terres, par les sentiers et les allées, les rues et les avenues, du nord, de l’est, du sud et de l’ouest. Ils se sont tous agglutinés là, pourtant. Dites-moi, le pouvoir magnétique des aiguilles de tous ces compas marins les a-t-il attirés d’aussi loin ?\par
Et encore. Imaginez que vous êtes à la campagne, dans quelque haute région de lacs. Prenez le chemin qu’il vous plaira, n’importe lequel, neuf fois sur dix, il vous amènera au fond d’un vallon près d’une flaque abandonnée par un ruisseau. C’est de la magie ! Prenez le plus distrait des hommes, absorbé dans la plus profonde des rêveries, dressez-le sur ses jambes, incitez-le à poser un pied devant l’autre, et il vous conduira infailliblement vers l’eau, pour autant qu’il y en ait dans la région. Viendriezvous à mourir de soif dans le grand désert américain, tentez l’expérience si un professeur de métaphysique fait partie de votre caravane. Certes, chacun le sait, l’eau et la méditation vont de pair à jamais.\par
Mais voici un artiste ! Son désir est d’exprimer pour vous, sur sa toile, le coin de paysage le plus enchanteur et le plus romantique de toute la vallée du Saco, le plus pénétré de rêve, d’ombre et de paix. Quel est son procédé ? Là se dressent ses arbres dont chacun a un tronc creux propre à abriter un ermite et son crucifix ; là, sa prairie sommeille et son troupeau s’assoupit ; de sa chaumière, au loin, s’élève une indolente fumée. Plus loin encore, dans la distance, à travers les bois, le dédale d’un chemin grimpe et s’enroule jusqu’aux éperons des montagnes baignées d’azur. Mais bien que l’image traduise l’extase, et bien que le pin secoue ses soupirs comme des feuilles sur la tête de ce berger, tout serait vain, si le regard du pâtre n’était pas subjugué par l’eau qui coule devant lui. Parcourez la prairie de juin, vous frayant, des lieues et des lieues durant, une voie à travers les lys tigrés qui croissent à la hauteur de vos genoux – quelle est votre nostalgie ? – L’eau… il n’y a pas là une goutte d’eau ! Si le Niagara déversait une chute de sable, feriezvous des milliers de milles pour l’aller voir ? Pourquoi le malheureux poète du Tennessee, lorsqu’il reçut soudain deux poignées d’écus, en vint-il à peser s’il s’achèterait le manteau dont il avait tristement besoin, ou s’il investirait sa fortune à accomplir un voyage à pied jusqu’à Rockaway Beach ? Pourquoi presque tous les vigoureux garçons possédant une âme saine dans un corps sain sont-ils, une fois ou l’autre, pris de la folie d’aller voir la mer ? Pourquoi vous-même, lors de votre premier voyage comme passager, avez-vous ressenti ce frémissement mystique, lorsqu’on vous a annoncé que votre navire et vous-même aviez atteint la haute mer ? Pourquoi les anciens Perses ont-ils tenu la mer pour sacrée ? Pourquoi les Grecs lui ont-ils donné un dieu distinct, le propre frère de Jupiter ? Tout cela ne saurait être vide de sens. Plus lourde encore de signification l’histoire du Narcisse qui, ne pouvant faire sienne l’image tourmentante et douce que lui renvoyait la fontaine, s’y précipita dans la mort. Cette même image nous la percevons nous-mêmes sur tous les fleuves et tous les océans. C’est le spectre insaisissable de la vie, la clef de tout.\par
Toutefois, quand je dis avoir l’habitude de prendre la mer chaque fois que mon regard commence à s’embrumer, quand je me préoccupe par trop de mes poumons, je n’aimerais pas qu’on en conclue que j’y vais en tant que passager. Il y faudrait une bourse, or une bourse, s’il n’y a rien dedans, n’est qu’une loque. D’autre part les passagers ont le mal de mer, deviennent hargneux, insomniaques et n’ont, en général, pas grand plaisir. Non, je ne m’embarque jamais comme passager, et bien que j’ai quelque chose du loup de mer, je ne pars jamais non plus comme Commodore, Capitaine ou Maître-coq. Je laisse à ceux qui les apprécient ces distinctions et ces titres de gloire. Pour ma part, j’abhorre tout labeur honorable, respectable, les épreuves et tribulations de quelque nature qu’elles soient. J’ai bien assez à faire à m’occuper de moi-même sans assumer la responsabilité de navires, de trois-mâts barques, de bricks, de goélettes et que sais-je encore. Quant à m’engager comme coq – bien qu’il me faille reconnaître le prestige de cet emploi, le cuisinier valant à bord un officier, d’une certaine manière –, je n’ai jamais éprouvé de penchant au rôtissage des volailles… quoiqu’une fois rôties, raisonnablement beurrées, judicieusement salées et poivrées, personne ne parlera desdites volailles avec plus de respect, pour ne pas dire de révérence, que moi… C’est grâce à la passion idolâtre des Égyptiens que vous pouvez voir encore dans les fournils géants que sont les Pyramides les momies des ibis qu’ils ont rissolés et des hippopotames qu’ils ont fait rôtir.\par
Non. Quand je prends la mer c’est comme simple matelot de devant le mât, d’aplomb au gaillard d’avant et au sommet du mât de cacatois. À vrai dire, je reçois pas mal d’ordres, on me contraint de sauter d’espar en espar, comme une sauterelle dans la prairie de mai. Au début, cette soumission est assez déplaisante. Vous vous sentez blessé dans votre dignité surtout si vous êtes issu d’une vieille aristocratie terrienne, comme celle des Van Rensselear, des Randolph ou des Hardicanute. Combien davantage encore si, juste avant de mettre la main dans le pot à brai, vous posiez en grand seigneur parce que, instituteur de campagne, vous teniez en respect vos plus forts gaillards. Cuisant changement, je vous le garantis, que de passer de la fonction de maître d’école à celle de marin ; pour trouver le courage de sourire et de le supporter, il convient d’absorber une forte décoction de Sénèque et des Stoïques. Mais ceci même s’use avec le temps.\par
Si quelque vieux rat de capitaine m’ordonne de prendre un balai et de nettoyer les ponts, alors quoi ? Quel est le poids de cette humiliation, pesée, il s’entend, sur la balance du Nouveau Testament ? Pensez-vous que l’archange Gabriel aura de moi une opinion meilleure si, dans cette circonstance donnée, j’obéis à ce vieux rat avec promptitude et déférence ? Qui n’est pas esclave ? je vous le demande. De sorte que les vieux capitaines peuvent bien me donner des ordres, m’accabler de coups et de horions, j’ai la satisfaction de savoir que c’est dans l’ordre des choses, que tout un chacun est à peu près logé à la même enseigne – que ce soit sur le plan physique ou métaphysique – et que, l’universel coup de matraque ayant achevé sa tournée, les hommes n’ont plus qu’à se frictionner mutuellement les omoplates et s’estimer contents.\par
Je m’embarque aussi toujours comme matelot parce que ces messieurs se font un point d’honneur de me payer pour ma peine, alors que je n’ai jamais ouï-dire qu’ils aient donné un liard à un passager. Au contraire, les passagers doivent payer. Et il n’y a pas de différence au monde plus grande qu’entre payer et être payé. Le fait de payer est peut-être le pire fléau que nous aient attiré les maraudeurs du Paradis terrestre. Mais être payé… qu’y a-t-il de comparable à cela ? C’est merveille de \hspace{1em} voir l’empressement courtois avec lequel un homme reçoit de l’argent alors que nous sommes tous fermement convaincus que l’argent est la source de tous les maux affligeant le genre humain, et qu’en aucun cas le riche ne peut entrer au ciel. Ah ! comme nous nous livrons de gaieté de cœur à la perdition !\par
En dernier lieu, je m’engage toujours comme matelot parce que c’est un sain exercice et pour l’air pur qui fouette le gaillard d’avant. Car, en ce monde, les vents debout prédominent toujours sur les vents arrière (à condition toutefois de ne pas violer les règles de Pythagore) de sorte que, la plupart du temps, le Commodore ne reçoit au gaillard d’arrière qu’un air déjà frelaté par les marins du gaillard d’avant. Il croit avoir la primeur du vent, mais il n’en est rien. D’une manière à peu près identique, la roture mène ses chefs en bien des domaines sans que ceux-ci s’en doutent. Mais j’ignore la raison pour laquelle, après avoir goûté à plusieurs reprises des embruns en tant que marin marchand, je me mis dans la tête d’embarquer sur un navire baleinier. L’agent secret du Destin qui, invisible, exerce sur moi une surveillance constante, me suit discrètement et m’influence de manière inexplicable, répondra mieux que quiconque à cette question. Sans doute aucun, mon départ pour la pêche à la baleine figurait depuis bien longtemps au programme grandiose de la Providence, tel un court intermède musical, un solo dans une exécution orchestrale. L’affiche, j’imagine, devait l’annoncer à peu près en ces termes :\par
Élection brillamment disputée à la Présidence des États-unis Expédition baleinière par un certain Ismaël \par
Batailles sanglantes en Afghanistan. \par
Je ne vois pas cependant la raison précise qui poussa les Parques imprésarios de ce Grand Théâtre à m’assigner ce rôle minable à bord d’un navire baleinier, alors que d’autres \hspace{1em}se voient accordé de jouer les vedettes dans de somptueuses tragédies, ou des rôles courts et faciles dans des comédies de bon ton, ou d’être un gai luron dans une farce ; bien que je ne puisse discerner leur raison précise, en me remémorant maintenant toutes les circonstances, je crois deviner la vérité sur les ressorts et les motifs qui, présentés sous des déguisements astucieux et divers, m’induisirent à entreprendre de jouer mon rôle, tout en me berçant de l’illusion que ce choix émanait de ma propre volonté et de mon libre-arbitre.\par
Le motif majeur fut engendré par l’image écrasante de la grande baleine elle-même. Un monstre aussi mystérieux que de mauvais augure échauffait toute ma curiosité de même que les mers sauvages et lointaines où il roule l’île de son corps massif ; et les périls sans nom auxquels il vous expose et contre lesquels on ne peut se prémunir. Tout cela et les merveilles par milliers promises à mes yeux et à mes oreilles par la Patagonie firent pencher ma décision du côté de mon vœu. Pour d’autres, ces motifs n’auraient peut-être pas constitué une tentation, mais moi, une éternelle démangeaison des choses lointaines me tourmente. J’adore naviguer sur des mers interdites et poser le pied sur des rivages inhumains. Sans méconnaître le Bien, je suis prompt à percevoir l’horreur et capable de fraterniser avec elle – si elle me le permet – puisque mieux vaut entretenir des relations amicales avec nos commensaux.\par
Pour toutes ces raisons, cette expédition baleinière fut la bienvenue ; les grandes écluses du pays des merveilles s’ouvraient brutalement et, à travers les mirages qui me poussèrent à céder à mon désir, pénétrèrent jusqu’au tréfonds de mon âme d’interminables processions de baleines, flottant, deux par deux, escortant un fantôme magnifique encapuchonné de blanc, telle une colline neigeuse sur le ciel.
\chapterclose


\chapteropen
\chapter[{CHAPITRE II. Le sac de voyage}]{CHAPITRE II \\
Le sac de voyage}\renewcommand{\leftmark}{CHAPITRE II \\
Le sac de voyage}


\chaptercont
\noindent Je fourrai une ou deux chemises dans mon vieux sac de voyage, le serrai sous mon bras puis me mis en route pour le cap Horn et le Pacifique. Ayant quitté la bonne vieille ville de Manhattan, j’arrivai en temps voulu à New Bedford. C’était un samedi soir de décembre. Je fus bien déçu d’y apprendre que le petit paquebot desservant Nantucket était déjà parti et qu’il ne se présenterait aucune occasion de départ avant le lundi.\par
Si la plupart des jeunes candidats aux souffrances et aux supplices de la pêche à la baleine s’embarquent de ce même New Bedford, il faut bien le dire, cette idée ne m’effleurait pas. J’étais résolu à ne prendre la mer que sur un navire de Nantucket, parce que tout ce qui se rapportait à cette fameuse vieille île était, pour moi, empreint d’un charme subtil et violent qui m’envoûtait. D’autre part, si, récemment, New Bedford s’était assuré le monopole de l’industrie baleinière, laissant bien à la traîne cette pauvre vieille Nantucket, celle-ci – Tyr de cette Carthage – la vit naître ; c’est là que fut hâlée à la côte la première baleine américaine. D’où partirent, sinon de Nantucket, les pêcheurs aborigènes, les Peaux-Rouges, lançant leurs pirogues à la poursuite du Léviathan ? Et d’où fit-il voile, sinon de Nantucket encore, cet audacieux petit sloop, partiellement chargé de pierres emportées – ainsi le veut l’histoire – pour être jetées aux baleines dans le but de juger de l’instant où elles seraient assez proches pour être harponnées depuis le beaupré ?\par
Avant de pouvoir gagner mon port d’embarquement j’avais à présent une nuit, un jour et une nuit encore, à passer à New Bedford. L’inquiétude me vint de savoir où j’allais manger et dormir pendant ce temps. C’était une nuit louche, pis que cela, une nuit très sombre et très lugubre, d’un froid cinglant. Je ne connaissais personne ici. Je draguai mes poches avec angoisse pour ne ramener que menue monnaie, aussi me fis-je la recommandation suivante, tandis que, debout au milieu d’une morne rue, mon sac sur l’épaule, je comparais les ténèbres du nord à l’obscurité du sud : « Ainsi où que tu ailles, Ismaël, où qu’en ta sagesse tu décides de loger pour la nuit, mon cher Ismaël, ne manque pas de te renseigner sur le prix et ne fais pas le difficile. »\par
J’arpentai lentement les rues et je passai devant les « Harpons croisés » mais cela sembla trop coûteux et trop gai. Plus loin, les fenêtres embrasées de l’« Auberge de l’Espadon » déversaient de si ardents rayons qu’ils paraissaient avoir fondu, sur son seuil net, la neige et la glace entassées partout ailleurs, dont le dur revêtement de plus de dix centimètres d’épaisseur offrait des aspérités tranchantes comme du silex, lesquelles m’étaient fort pénibles chaque fois que j’y butais, les semelles de mes bottes, longuement soumises à de cruelles épreuves, se trouvant en piteux état. Et je me répétai, m’arrêtant un instant pour contempler la lumière généreuse répandue dans la rue et pour écouter les tintements des verres à l’intérieur : « Trop coûteux et trop gai. » Et je m’encourageai enfin : « Va de l’avant, Ismaël, tu m’entends ! Ôte-toi de devant la porte, tes bottes rapiécées bouchent le chemin. » Je repartis. Mes pas se dirigèrent, instinctivement, vers les rues menant à l’eau, c’est là que, sans aucun doute, devaient se trouver les auberges les moins coûteuses, sinon les plus gaies.\par
De si lugubres rues ! de part et d’autre c’étaient des blocs d’encre, non des maisons, avec ici et là une chandelle, pareille à un feu follet sur une tombe. À cette heure de la nuit, en ce dernier jour de semaine, ce quartier de la ville était désert. Mais j’arrivai enfin devant la lueur fumeuse d’une bâtisse, large et basse, dont la porte grande ouverte était une invitation. Elle avait un air négligé, comme si elle était d’utilité publique et la première chose que je fis en entrant fut de trébucher sur un seau de cendres. Ah ! ah ! me dis-je, tandis que la poussière me prenait à la gorge, sont-ce là les cendres de Gomorrhe, la cité détruite ? S’il y a des « Harpons croisés » et un « Espadon », ceci ne peut être qu’à l’enseigne du « Traquenard ». Me ressaisissant toutefois, j’entendis à l’intérieur une voix forte, j’avançai et ouvris une seconde porte.\par
On se serait cru à Topheth, lors d’une séance du Grand Parlement noir. Des centaines de visages sombres alignés se tournèrent pour me regarder. Au fond, en chaire, l’Ange du Jugement, noir lui aussi, frappait sur un livre. C’était une église nègre et le prédicateur parlait de la noirceur des Ténèbres, de ses gémissements, de ses larmes et de ses grincements de dents. Ah ! Ismaël, me dis-je en faisant demi-tour : représentation de misère à l’enseigne du « Traquenard » !\par
Poursuivant ma route, j’arrivai enfin, près des docks, vers une lumière confuse et j’entendis, venue d’en haut, une plainte désespérée ; levant les yeux, je vis un panneau qui se balançait au-dessus de la porte. Un trait de peinture blanche représentait faiblement un jet vertical empanaché d’écume sous lequel on lisait : « Auberge du Souffleur Peter Coffin\footnote{Coffin signifie cercueil en anglais.} » !\par
Cercueil ? Souffleur ? Rapprochement de mauvais augure, pensais-je. Mais on dit que c’est un nom répandu à Nantucket et je présumai que ce Peter en était originaire. La lumière était si falote, l’endroit paraissait en ce moment si tranquille, la petite maison était si délabrée – comme si elle eût été bâtie là avec les épaves d’un quartier incendié – la pauvreté semblait \hspace{1em}tellement arracher à l’enseigne son cri déchirant, que je me dis que c’était là par excellence le logement bon marché où l’on servirait le meilleur des cafés de pois grillés.\par
C’était une étrange maison : un toit à pignons, un côté tristement penché par la paralysie. Elle se tenait à un angle de rues exposé et glacial, où Eurus, le vent des tempêtes, hurlait comme jamais il ne le fit autour du navire secoué de ce pauvre Paul. Pourtant le vent le plus sauvage est un zéphyr lorsqu’on est au coin du feu, les pieds contre la plaque de cheminée, se rôtissant paisiblement avant d’aller au lit. Au sujet d’Eurus, ce vent de tempêtes – dit un auteur ancien dont je possède le seul manuscrit existant – il y a une différence admirable suivant que vous l’observez d’une fenêtre dont seule la vitre extérieure est givrée ou que vous l’éprouviez de cette fenêtre, sans carreaux, où le gel sévit des deux côtés et dont le pauvre hère a la Mort pour seul vitrier. C’est bien vrai, pensais-je, tandis que me revenait en mémoire ce passage du vieux manuscrit. Le raisonnement est sage. Oui, ces yeux sont des fenêtres et ce corps qui est mien en est la maison. Quel dommage pourtant qu’on n’ait pas bouché les fentes et les lézardes avec un peu de charpie ici ou là. Mais il n’est plus temps d’y porter remède. L’univers est achevé, on y a mis la dernière main et les débris ont été charroyés, voici des millions d’années. Pauvre Lazare, gisant là, un pavé en guise d’oreiller, claquant des dents contre la pierre, grelottant dans ses haillons, il pourrait bien se boucher les oreilles avec des chiffons, mordre un épi de maïs, que cela ne le garantirait pas du tempétueux Eurus. Eurus ! s’écrie le vieux riche drapé dans son vêtement de soie pourpre (celui qu’il eut par la suite fut bien plus rouge encore !) beuh… beuh !… quelle belle nuit de gel, quel n’est pas l’éclat d’Orion, quelle lumière septentrionale ! Qu’ils parlent des serres chaudes de leurs éternels étés et qu’on me laisse le privilège de me créer mon été à moi avec mon propre charbon.\par
Mais qu’en pense Lazare ? Peut-il réchauffer ses mains bleuies en les tendant vers les aurores boréales ? Lazare ne préférerait-il pas être à Sumatra plutôt qu’ici ? N’aimerait-il pas mieux, de beaucoup, s’étendre de tout son long contre la ligne de l’équateur ? Oui, juste Ciel, oui ! Il se jetterait bien dans la fournaise de l’enfer pour échapper à ce gel.\par
Mais il y a davantage de matière à étonnement à voir Lazare gisant là, sur le seuil de l’homme riche, qu’à découvrir un iceberg échoué au rivage d’une des Moluques. Pourtant le riche lui-même vit, tel un Tsar, dans le palais de glace d’un monde de soupirs transis, et comme il préside une société de tempérance, il ne boit que les larmes tièdes des orphelins.\par
Mais assez pleurniché ! Nous partons pour la pêche à la baleine et cela nous en fournira des occasions nouvelles. Raclons la glace de nos semelles gelées et voyons quelle sorte d’endroit peut bien être cette « Auberge du Souffleur ».
\chapterclose


\chapteropen
\chapter[{CHAPITRE III. L’Auberge du Souffleur}]{CHAPITRE III \\
L’Auberge du Souffleur}\renewcommand{\leftmark}{CHAPITRE III \\
L’Auberge du Souffleur}


\chaptercont
\noindent En entrant dans cette auberge à pignons, on se trouvait dans le dédale d’une entrée, large et basse, dont les lambris anciens rappelaient la carène d’un navire condamné. D’un côté, se trouvait un immense tableau, à l’huile, si parfaitement enfumé, si effacé que, sous ce faux-jour, il fallait se livrer à une étude approfondie et à un déchiffrage systématique, prendre des renseignements auprès des voisins, pour arriver à deviner un tant soit peu de ce qu’il représentait. D’innombrables ombres et pénombres vous invitaient, de prime abord, à penser que quelque jeune artiste ambitieux avait, à l’époque des sorcières de la Nouvelle-Angleterre, entrepris d’y dépeindre un chaos satanique. Mais en se concentrant dans une contemplation forcenée, en se livrant à des méditations répétées, et surtout en ouvrant la petite fenêtre du fond de l’entrée, on en venait à la conclusion qu’une telle idée, démentielle qu’elle fût, n’était peut-être pas infondée.\par
Mais le plus intrigant et le plus déconcertant, c’était au centre du tableau une masse souple, longue, noire, planant audessus de trois lignes perpendiculaires, bleues et pâles, flottant dans une écume innommable. Un tableau marécageux, bourbeux, pâteux, en vérité ! propre à déranger l’esprit d’un nerveux. Pourtant on se sentait figé par ce qu’il en émanait de sublimité indéfinissable, suggérée, défiant l’imagination, et l’on en venait à se jurer de découvrir le sens de cette peinture étonnante. De temps en temps, une idée lumineuse, mais hélas décevante, vous traversait : c’est une tempête sur la mer Noire à minuit ; c’est le combat cosmique des quatre éléments ; c’est une lande désolée ; c’est une scène hivernale hyperboréenne ; c’est la débâcle des glaces sur le fleuve Temps. Toutes ces divagations cédaient enfin devant ce quelque chose d’inquiétant au centre du tableau. Cela étant identifié, le reste perdait son mystère. Mais voyons ! cela ne ressemble-t-il pas vaguement à un poisson gigantesque ? au grand Léviathan lui-même ?\par
En fait, l’intention de l’artiste semblait avoir été la suivante : d’après ma propre théorie, appuyée sur les opinions recueillies auprès des personnes d’âge que je consultai sur la question : le tableau représente un cap-hornier pris dans un ouragan, sur le point de sombrer dans le bouillonnement des flots, seuls ses trois mâts brisés émergent encore, et une baleine en fureur, décidée à bondir par-dessus le navire, se trouve dans la situation insensée de s’empaler sur les trois têtes de mâts.\par
Sur le mur opposé de l’entrée s’étalait un éventail barbare de massues et de lances monstrueuses. Les unes portaient des dents serrées et luisantes pareilles à des scies d’ivoire, d’autres des touffes de cheveux humains ; l’une d’elles, en forme de faucille, était pourvue d’un large manche en demi-lune, dessinant la courbe dont un faucheur, au bras démesuré, abat l’andain. On était envahi d’horreur en la regardant et l’on se demandait quel sauvage dénaturé et cannibale avait bien pu faire une moisson de mort avec cet outil affreux et tranchant. Parmi ces armes se trouvaient des lances à baleines rouillées et des harpons cassés ou tordus. Certaines avaient une histoire. Avec cette lance, jadis longue, et à présent tristement coudée, Nathan Swaine tuait, il y a de cela cinquante ans, quinze baleines entre le lever et le coucher du soleil. Et ce harpon – transformé en tire-bouchon – fut emporté, dans les mers de Java, par une baleine qui, des années plus tard, fut tuée au cap Blanc. Le fer l’avait pénétrée près de la queue et, comme une aiguille va et vient dans le corps d’un homme, il fit un trajet de quarante pieds et fut retrouvé implanté dans la bosse.\par
Traversant cette entrée sombre, on passait ensuite sous une arche surbaissée, pratiquée dans ce qui avait dû être, autrefois, une grande cheminée centrale et ronde, puis on entrait dans la salle commune. Un endroit plus sombre encore, avec de grosses poutres si basses, et un vieux plancher si racorni qu’on s’imaginait être dans la coque d’un vieux navire, surtout par une pareille nuit où le vent hurlait, secouant furieusement cette arche archaïque ancrée à son carrefour. D’un côté, une table basse et longue, semblable à une étagère, portait des boîtes dont les vitres craquelées abritaient des objets rares et poussiéreux recueillis dans tous les recoins de ce vaste monde. Un antre obscur formait l’angle le plus reculé de la pièce – le bar – ébauchant grossièrement une tête de baleine. Il s’abritait d’ailleurs sous la voûte d’un maxillaire de baleine si grand qu’il eût aisément laissé passer une diligence. À l’intérieur, des rayons crasseux supportaient des vieilles carafes, des bouteilles, des flacons ; et entre ses mâchoires voraces, tel un autre Jonas poursuivi par une malédiction (et on le surnommait Jonas), s’affairait un petit vieillard desséché qui, contre espèces sonnantes, servait amoureusement, aux marins, le delirium et la mort.\par
Les verres dans lesquels il versait son poison étaient abominables. Honnêtes cylindres à l’extérieur, ces verres infâmes, boursouflés et verts, s’amincissaient à l’intérieur en un cône de tricherie. Des parallèles étaient gravés sur le pourtour de ces verres de voleur. Empli jusqu’à l’un deux, cela vous coûtait un penny, jusqu’à cet autre, un penny de plus, et ainsi de suite jusqu’à ce que le verre fût plein : la mesure « cap Horn » que vous avaliez pour un shilling.\par
Lorsque j’entrai dans la pièce, un groupe de jeunes marins, réunis autour d’une table, examinaient à la faible lumière un lot de ces petits objets de fantaisie qu’ils fabriquent au cours des voyages. Je demandai une chambre au patron, il me répondit que la maison était comble… plus un lit disponible.\par
– Mais baste ! ajouta-t-il en se tapant le front. Auriez pas d’objections à partager le plumard d’un harponneur, pas vrai ? Je suppose que vous partez pour la pêche à la baleine, alors mieux vaut vous habituer tout de suite à ce genre de choses.\par
Je lui dis que je n’aimais pas dormir à deux ; mais que si j’y étais vraiment obligé, j’aimerais au moins savoir qui était ce harponneur, et ajoutai que s’il n’y avait vraiment pas d’autre place, et si le harponneur n’était pas inqualifiable, plutôt que de rôder dans une ville étrangère par une si cruelle nuit, je partagerais la couche de n’importe quel homme correct.\par
– C’est bien ce que je me disais. Très bien. Asseyez-vous. Souper ? Voulez-vous souper ? Ce sera tout de suite prêt.\par
Je m’assis sur un vieux banc à dossier, gravé comme un banc de la digue. À l’autre extrémité, un matelot songeur contribuait à l’orner davantage à l’aide de son couteau de po- che ; penché en avant, il travaillait assidûment le bois entre ses jambes. Il s’exerçait à sculpter un navire sous toute sa toile mais, à mon avis, ça n’avançait guère.\par
Enfin quatre ou cinq d’entre nous furent invités à passer dans une pièce voisine pour le repas. On s’y serait cru en Islande tant il y faisait froid – point de feu du tout –, le patron prétendait qu’il n’avait pas les moyens d’un tel luxe. Seulement deux funèbres chandelles de suif, dans leur suaire… Il ne nous restait qu’à boutonner nos vestes de singes et à porter, entre nos mains à demi gelées, un thé brûlant jusqu’à nos lèvres. La nourriture, en revanche, était des plus substantielles, non seulement de la viande et des pommes de terre mais encore des boulettes bouillies. Juste ciel ! ces boulettes indigestes pour le souper ! Un jeune gars en houppelande verte leur faisait un sort.\par
– À ce train-là, mon garçon, lui dit le patron, sûr et certain que vous aurez des cauchemars.\par
– Patron, chuchotai-je, est-ce que c’est le harponneur ?\par
– Oh ! non, dit-il avec un sourire diabolique, le harponneur est un gars au teint sombre. Il ne mange pas de boulettes, lui, il ne mange que des steaks… et saignants !\par
– Du diable ! Où est ce harponneur ? Est-ce qu’il est ici ? demandai-je.\par
– Il ne va pas tarder… répondit-il.\par
Ce fut plus fort que moi, je commençai à trouver suspect ce harponneur au « teint sombre ». Et je résolus que, si nous devions décidément dormir ensemble, il faudrait qu’il se déshabille et se couche avant moi.\par
Le repas terminé, notre compagnie retourna dans la salle du bar, où ne sachant que faire de moi-même, je décidai de passer le reste de la soirée en spectateur.\par
Un chahut formidable se faisait justement entendre audehors. Se levant, le patron s’écria : « Voici l’équipage du {\itshape Grampus}. On en parlait ce matin, un voyage de trois ans et un navire plein. Hourrah, les gars ! nous allons avoir les dernières nouvelles des Fidji. »\par
Des bottes résonnèrent lourdement dans l’entrée, la porte s’ouvrit brutalement et une fougueuse équipe de marins entra. Enveloppés dans leurs capotes poilues, emmitouflés de cachenez de laine, reprisés et loqueteux, les barbes raides de glaçons, on eût dit une invasion d’ours du Labrador. Ils venaient de débarquer et c’était la première maison dans laquelle ils posaient le pied. Il n’y a, dès lors, rien d’étonnant à ce qu’ils se dirigeassent, comme un seul homme, vers la gueule de la baleine – le bar – où l’antique petit Jonas ridé, qui y officiait, versa à tous des rasades. L’un deux se plaignant d’un mauvais rhume de cerveau, Jonas lui fit une mixture de gin et de mélasse, assez pareille à de la poix, lui jurant que c’était la panacée idéale pour tous les rhumes et corizas de la création, même les plus rebelles, qu’on les eût attrapés sur la côte du Labrador ou exposé au vent sur une banquise.\par
L’alcool eût tôt fait de leur monter à la tête, comme il arrive généralement même aux plus fieffés ivrognes lorsqu’ils débarquent et ils commencèrent à mener grand tapage.\par
Pourtant je remarquai que l’un d’eux se tenait à l’écart, et bien qu’il parût soucieux de n’être pas un rabat-joie pour ses camarades à cause de sa gravité, il évitait de se mêler au chahut. Il m’intéressa aussitôt ; et étant donné que des dieux marins voulurent qu’il devînt très bientôt mon compagnon (de lit seulement dans l’histoire) je vais tenter de le décrire un peu. Il faisait six bons pieds de haut, d’imposantes épaules et une poitrine comme un batardeau. J’ai rarement vu un homme ayant une pareille carrure. Son visage tanné, d’un brun sombre, faisait paraître encore plus étincelantes ses dents blanches ; au fond de ses yeux passaient les ombres profondes de souvenirs qui ne paraissaient guère lui apporter de joie. Son accent trahissait un homme du Sud et sa belle prestance me fit penser qu’il était un de ces puissants montagnards venus des Alleghanies en Virginie. Lorsque l’orgie de ses camarades atteignit son paroxysme, cet homme disparut discrètement, et je ne le revis pas jusqu’au moment où il devint mon compagnon en mer. Pourtant, quelques instants à peine s’étaient écoulés que les matelots remarquèrent son absence et comme il semblait avoir du prestige auprès d’eux, ils hurlèrent d’une même voix : Bulkington ! Bulkington ! Où est Bulkington ? et s’élancèrent à sa poursuite hors de la maison.\par
Il était maintenant près de neuf heures et un calme, qui paraissait surnaturel après ces bacchanales, régna dans la pièce, je commençai à me féliciter du petit plan que j’avais échafaudé juste avant l’entrée des marins.\par
Aucun homme n’aime coucher à deux ; en fait, votre propre frère lui-même n’est pas le bienvenu dans votre lit. J’en ignore la raison mais tout le monde préfère la solitude du sommeil. Et quand il faut dormir avec un étranger, dans une auberge étrangère, en une ville étrangère et que cet étranger est un harponneur, alors vos objections se multiplient. Je ne voyais pas non plus de raison valable à être contraint, moi plutôt qu’un autre, de partager mon lit ; car les marins ne dorment pas davantage à deux en mer que les rois célibataires de la terre ferme. Naturellement ils dorment tous dans le même carré, mais chacun a son hamac, se couvre avec sa propre couverture et dort dans sa propre peau.\par
Plus je songeais à ce harponneur, plus me devenait intolérable l’idée de dormir avec lui. On pouvait raisonnablement se dire qu’un harponneur n’avait pas des dessous – de coton ou de laine – des plus propres, et moins encore des plus raffinés. J’en avais la chair de poule. D’autre part, il se faisait tard, et mon respectable harponneur aurait dû être rentré et prêt à aller au lit. En supposant encore qu’il me tombe dessus à minuit, comment saurais-je de quel bouge infâme il sortait ?\par
– Patron ! J’ai changé d’avis au sujet du harponneur. Je ne coucherai pas avec lui. J’essayerai ce banc.\par
– Comme vous voudrez ; je regrette de ne pas pouvoir vous prêter une nappe en guise de matelas, cette planche est diablement raboteuse, dit-il en palpant les nœuds et les entailles. Mais attendez un peu, poltron, j’ai un rabot de charpentier dans le bar, attendez, vous dis-je, je vais vous installer douillettement. Aussitôt dit aussitôt fait, il prit le rabot et, essuyant d’abord le banc avec son vieux mouchoir de soie, il s’attaqua énergiquement au rabotage de mon lit tout en ricanant comme un singe. Les copeaux volaient de droite et de gauche, jusqu’à ce qu’enfin la lame vînt buter contre un nœud rébarbatif. Le patron était près de se fouler le poignet et je l’adjurai par tous les grands dieux de renoncer ; ce lit était assez doux pour me convenir, et je ne voyais pas très bien comment on eût pu transformer en duvet une planche de pin. De sorte qu’il ramassa les copeaux avec un dernier ricanement, les jeta dans le grand fourneau au milieu de la pièce, puis partit vaquer à ses affaires, m’abandonnant à une sombre méditation.\par
Je pris la mesure du banc et constatai qu’il était d’un pied trop court ; on pouvait y remédier avec une chaise. Mais il était d’un pied trop étroit et l’autre banc qui se trouvait dans la pièce avait deux pouces de plus que celui qui venait d’être raboté, de sorte qu’on ne pouvait les accoupler. Je disposai alors le premier banc en long contre la seule partie libre du mur, en ménageant un petit intervalle pour mon dos. Mais je découvris bientôt qu’un tel courant glacé passait sous la fenêtre que cette formule n’était pas du tout satisfaisante, d’autant plus que la porte branlante envoyait un courant d’air rejoindre celui de la fenêtre, produisant de petits tourbillons exactement à l’endroit où j’envisageais de passer la nuit.\par
Que le diable emporte ce harponneur, pensai-je, mais au fond, ne pourrais-je pas prendre les devants sur lui, tirer le verrou depuis l’intérieur, sauter dans son lit, sans être réveillé par des coups violents dans la porte ? Cela paraissait n’être pas une mauvaise idée ; réflexion faite je l’abandonnai. Car comment savoir si le lendemain matin, dès que je sortirai de la chambre, le harponneur ne se tiendra pas sur le seuil, prêt à me rosser.\par
Jetant encore un regard autour de moi, ne voyant aucune possibilité de passer une nuit supportable ailleurs que dans le lit de quelqu’un d’autre, j’en vins à me dire qu’après tout je nourrissais peut-être des préjugés malséants sur ce harponneur inconnu. Je vais attendre un moment, me dis-je ; il ne saurait tarder. Alors je le dévisagerai attentivement, et peut-être que nous pourrons être de bons compères après tout… on ne peut pas savoir. Cependant, les autres pensionnaires arrivaient, seuls ou par deux, ou par trois et s’allaient coucher. De mon harponneur, pas l’ombre.\par
– Patron, dis-je, quelle sorte de gars est-ce ? Rentre-t-il toujours à des heures pareilles ? Il était à présent près de minuit.\par
Le patron gloussa et parut fort émoustillé par une idée qui ne pouvait que m’échapper.\par
– Non, répondit-il, d’habitude il se couche comme les poules – tôt couché, tôt levé – eh oui, c’est l’oiseau qui trouve le ver. Mais ce soir, il fait du colportage, voyez, et je ne sais pas ce qui le retient si longtemps, à moins, peut-être, qu’il n’ait pas réussi à vendre sa tête.\par
– Vendre sa tête ? Que me racontez-vous comme sornet- tes ? demandai-je avec colère. Voulez-vous me faire accroire, patron, que ce harponneur passe cette nuit bénie de samedi, enfin ce dimanche matin, à essayer de monnayer sa tête de porte à porte ?\par
– C’est exactement ça, dit le patron, je lui ai bien dit qu’ici il n’y aurait rien à faire, le marché est saturé.\par
– Saturé de quoi ? hurlai-je.\par
– De têtes, pour sûr. Est-ce qu’y a pas déjà trop de têtes au monde ?\par
– Je vais vous parler net, patron, dis-je tranquillement, vous feriez mieux de renoncer à allonger ces bobards… je ne suis pas un novice.\par
– Peut-être pas, dit-il en se taillant un cure-dent, mais j’ai comme une idée que vous seriez passé à tabac comme un novice si ce harponneur vous entendait dire du mal de sa tête.\par
– Je la lui casserai, répondis-je, au comble de la fureur que faisait naître en moi l’invraisemblable farrago du patron.\par
– Elle est déjà cassée, dit-il.\par
– Cassée, vous voulez bien dire cassée ?\par
– Pour sûr, c’est bien pour ça qu’il n’arrive pas à la vendre, j’imagine.\par
– Patron, lui dis-je en marchant sur lui, glacial comme le mont Hécla sous une tempête de neige, patron, cessez de tailler ce cure-dent. Vous et moi devons arriver à nous comprendre, et cela sans tarder. Je suis venu chez vous pour avoir un lit, vous me répondez que vous ne pouvez m’en offrir que la moitié d’un, l’autre moitié étant propriété d’un certain harponneur. Et à propos de ce harponneur, que je n’ai pas encore aperçu, vous insistez pour me raconter les histoires les plus insensées et les plus exaspérantes, dans le but de semer en moi des sentiments de méfiance envers un gars que vous destinez à être mon compagnon de lit – ce qui représente, patron, le comble de l’intimité et de la confiance. Je vous demande à présent de me répondre franchement et de me dire qui est ce harponneur, quelle sorte d’individu c’est, et si je peux, en toute sécurité et à tous égards, passer la nuit avec lui. Et tout d’abord je vous demanderai de vous dédire au sujet de cette histoire de vente de tête qui, si elle est vraie, prouve assez clairement que ce harponneur est fou à lier, et il n’entre pas dans mes vues de coucher avec un fou, et vous, monsieur, vous-même veux-je dire, patron, j’entends bien, vous, monsieur, en m’incitant sciemment à le faire, vous vous exposez à des poursuites pénales.\par
– Ben, dit le patron, en respirant profondément, voilà un sermon joliment long pour un gars qui aime plaisanter de temps à autre comme moi. Mais calmez-vous, calmez-vous, ce fameux harponneur dont je vous ai parlé vient d’arriver des mers du Sud, où il a acheté un paquet de têtes momifiées en Nouvelle- Zélande (une curiosité, vous savez), il les a toutes vendues sauf une, et celle-là il essaie de la vendre ce soir, parce que demain c’est dimanche, et que ce ne serait pas convenable d’aller vendre des têtes humaines dans les rues quand les gens vont à l’église. C’est ce qu’il voulait faire dimanche passé, mais je l’ai arrêté sur le pas de la porte comme il partait avec quatre têtes suspendues à une ficelle, comme un chapelet d’oignons.\par
Ce récit dissipa ce mystère autrement inexplicable, et s’il prouvait, somme toute, que le patron n’avait pas l’intention de se moquer de moi, il me poussait également à me demander quelle opinion je devais me faire d’un harponneur qui passait toute la nuit du samedi, et jusqu’à l’aube du jour du sabbat, engagé dans cette affaire de cannibalisme consistant à vendre les têtes de défunts idolâtres.\par
– Soyez sûr, patron, que ce harponneur est un homme dangereux.\par
– Il paye régulièrement. Mais allons, il se fait tard, il vaut mieux que vous sondiez, voilà un bon lit : Sal et moi on a dormi dans ce lit la nuit qu’on s’est collés. Il y a bien assez de place pour se retourner à deux dans ce lit ; c’est un puissant grand lit. Juste avant qu’on le mette au rancart, Sal y faisait dormir notre Sam et le petit Johnny aux pieds. Mais une nuit, en rêvant, je l’ai envoyé bouler par terre et Sam a failli se casser le bras. Après ça, Sal a dit que ça n’allait plus. Suivez-moi, je vais vous faire de la lumière en un clin d’œil, et sur ces mots il alluma une chandelle, me la tendit en s’offrant à me diriger. Mais je ne bougeais pas, hésitant ; regardant une pendule dans un coin, il s’écria : Ma foi, c’est dimanche, vous ne verrez pas ce harponneur cette nuit, l’a dû trouver un port quelque part… Alors, venez ; venez donc, est-ce que vous n’allez pas vous décidez à venir ?\par
Je considérai la question un instant, nous montâmes et il m’introduisit dans une petite chambre froide comme un coquillage, indubitablement meublée d’un lit prodigieux, presque assez vaste, en fait, pour que quatre harponneurs pussent y dormir côte à côte.\par
– Voilà, dit le patron en posant la chandelle sur un vieux coffre de marin branlant qui tenait lieu à la fois de meuble de toilette et de table. Là, mettez-vous à l’aise, à présent, je vous souhaite une bonne nuit. Je quittai le lit des yeux pour me tourner vers lui mais déjà il avait disparu.\par
Je rabattis le couvre-lit et me penchai ; bien que ce lit ne fût pas élégant, il supportait vaillamment un examen approfondi. Je fis alors le tour de la chambre ; le lit et la table mis à part, il n’y avait point d’autres meubles, les quatre murs, une grossière étagère et un devant de cheminée représentant un homme frappant une baleine. Parmi les objets qui ne faisaient pas à proprement parler partie de la chambre, se trouvaient un hamac roulé dans un coin et un grand sac de marin contenant la garderobe du harponneur et tenant incontestablement lieu de malle. Sur l’étagère au-dessus de la cheminée il y avait encore un paquet de curieux hameçons en os de poissons, et un long harpon appuyé à la tête du lit.\par
Mais qu’y a-t-il donc là sur le coffre ? Je pris la chose, l’approchai de la chandelle, la tâtai, la humai, fis toutes les tentatives imaginables pour arriver à une conclusion satisfaisante à son sujet. Je ne pouvais la comparer qu’à un grand paillasson bordé de petites ferrures cliquetantes, assez semblables aux piquants de porc-épic peints des mocassins indiens. Comme dans un poncho sud-américain, un trou ou une fente s’ouvrait au milieu. Mais serait-il possible qu’un harponneur sain d’esprit puisse endosser un paillasson pour parader dans les rues d’une ville chrétienne ainsi accoutré ? Je le passai pour l’essayer et il m’écrasa sous son poids, étant anormalement épais, velu et, j’en avais le sentiment, un peu humide, comme si ce mystérieux harponneur l’avait porté un jour de pluie. Ainsi vêtu, je me dirigeai vers un bout de miroir collé contre le mur et de ma vie je ne vis spectacle si affreux. Je sortis si précipitamment de ce poncho que j’en attrapai le torticolis.\par
Je m’assis au bord du lit et me mis à penser à ce harponneur colporteur de têtes et à son paillasson. Ainsi perché je me livrai à la méditation, puis je me levai, ôtai ma veste et poursuivis mes réflexions debout au milieu de la chambre. Enfin j’enlevai mon gilet et derechef je méditai quelque peu en bras de chemise. Commençant à avoir froid, à demi dévêtu de la sorte, et me souvenant que le patron avait dit que le harponneur ne rentrerait plus de la nuit, étant donné l’heure tardive, sans autre forme de procès, je giclai hors de mes bottes et, soufflant la chandelle, je me jetai dans le lit en me confiant à la Providence.\par
Que le matelas fût bourré d’épis de maïs ou de bris de vaisselle, impossible de le savoir, mais je me tournai et me retournai sans pouvoir dormir de longtemps. Tandis que je glissais dans une légère somnolence, presque sur le point de tomber dans les bras de Morphée, j’entendis un pas pesant sonner dans le corridor et je vis un rai de lumière filtrer sous la porte.\par
Dieu me préserve, pensai-je, ce doit être le harponneur, ce diable vendeur de têtes. Mais je demeurai étendu sans bouger, résolu à ne pas piper mot qu’on ne m’eût adressé la parole. La chandelle dans une main, ladite tête de Nouvelle-Zélande dans l’autre, l’étranger entra dans la chambre et sans un regard vers le lit, il posa la lumière à bonne distance de moi dans un coin à terre et commença à se débattre avec les cordons du grand sac dont j’ai déjà parlé. J’avais hâte de voir son visage, mais il était détourné et penché, occupé momentanément à dénouer les lacets du sac. Cela fait, il se retourna pourtant et alors ! Dieu du ciel ! quel spectacle ! Quel visage ! d’une couleur tout à la fois tirant sur le noir, le pourpre, le jaune et marqué, ici et là, de damiers d’aspect noirâtre. Oui, c’est bien ce que je pensais, un terrible compagnon de lit ; il s’est bagarré, a été affreusement balafré et il sort tout droit d’entre les mains du chirurgien. Mais comme il se tournait vers la lumière, je vis clairement que les carreaux noirs sur ses joues ne pouvaient être des emplâtres. C’étaient des taches d’une espèce ou d’une autre. D’abord je ne sus qu’en penser, puis j’eus le pressentiment de la vérité. Il me revint en mémoire l’histoire d’un Blanc – un baleinier lui aussi – qui, tombé entre les mains de cannibales, avait été tatoué par eux. J’en conclus que ce harponneur, au cours de ses lointains voyages, avait connu semblable aventure. Et qu’est-ce que cela, après tout ? pensai-je. Il ne s’agit que de son apparence, on peut être un honnête homme dans n’importe quelle peau. Mais alors que penser de ce teint d’un autre monde, je veux dire, de cette partie de sa peau qui n’était pas tatouée. Naturellement ce pouvait n’être qu’un cuir épais tanné par les ciels des tropiques. Mais je n’avais jamais entendu dire qu’un soleil brûlant fît virer un homme blanc à une pourpre nuancée de jaune. Toutefois je n’étais jamais allé dans les mers du Sud, et peut-être que le soleil de ces régions produisait sur la peau cet effet extraordinaire. Pendant tout le temps que ces pensées fulgurantes traversaient mon esprit, le harponneur ne remarqua nullement ma présence. Mais après avoir eu quelque difficulté à ouvrir son sac, il commença à fouiller à l’intérieur et en tira bientôt une sorte de tomahawk et une sacoche en peau de phoque poilue. Il les plaça sur le coffre au milieu de la chambre, puis saisissant la tête de Nouvelle-Zélande, une chose assez effroyable, il la fourra dans la sacoche. Il enleva son chapeau – un chapeau de castor tout neuf – et je fus sur le point de hurler de saisissement. Il n’avait pas un cheveu sur la tête – rien du moins dont il valût la peine de parler – rien… sauf un petit scalp noué en boucle sur le front. Cette tête chauve et pourpre apparaissait à présent comme un crâne léprosé. Si l’étranger ne s’était trouvé entre la porte et moi, j’aurais bondi en dehors plus rapidement que je n’ai jamais englouti un bon repas.\par
Et bien qu’on fût au second étage, ces circonstances me firent songer à sauter par la fenêtre. Je ne suis pas lâche, mais que penser de ce bandit pourpre, colporteur en têtes ? Cela défiait absolument ma compréhension. L’ignorance est la mère de l’épouvante, j’étais tellement désemparé et confondu par cet étranger que, je dois l’avouer, je n’aurais pas eu davantage peur du diable lui-même faisant irruption dans ma chambre au milieu de la nuit. En fait, j’étais à ce point pénétré d’effroi que je n’avais pas le cran nécessaire, à ce moment-là, pour lui adresser la parole et obtenir de lui la réponse qui eût expliqué tout ce qui paraissait si inexplicable en lui.\par
Cependant il se déshabillait et je vis sa poitrine et ses bras. Aussi vrai que je vis, ces parties cachées de son corps étaient un échiquier identique à son visage, tout son dos était également carrelé de noir, on aurait dit qu’il revenait d’une guerre de Trente Ans et qu’il aurait fui portant seulement une chemise d’emplâtres. Qui plus est, l’impression n’épargnait pas ses jambes, on eût cru voir une légion de grenouilles vert foncé assaillir des troncs de jeunes palmiers… Il était bien évident à présent que ce devait être quelque abominable sauvage, embarqué à bord d’un baleinier dans les mers du Sud, posant ainsi le pied en terre chrétienne. Je tremblais rien que d’y penser. De plus un vendeur ambulant de têtes, peut-être de celles de ses propres frères. Il pourrait lui venir le goût d’avoir la mienne… Seigneur ! Voyez, ce tomahawk !\par
Mais je n’eus pas le loisir de trembler car le sauvage se livra alors à une occupation qui me subjugua et retint toute mon attention, me convainquant que j’avais bel et bien affaire à un païen. Allant jusqu’à son lourd pardessus, ou paletot-pilote, ou noroît qu’il avait auparavant déposé sur une chaise, il en fouilla les poches et finit par en extraire une étrange figurine, informe, bossue, exactement de la couleur d’un bébé congolais de trois jours. Pensant à la tête réduite, j’en vins presque à croire un instant que cet homoncule noir était véritablement un nouveau-né conservé par un procédé identique. Puis, remarquant que cela n’avait aucune souplesse, que c’était brillant, sensiblement comme de l’ébène polie, j’en conclus que ce n’était rien de plus qu’une idole de bois, ce que cela s’avéra être en effet. Le sauvage se dirigea alors vers la cheminée vide, ôta l’écran de papier et installa sa figurine bossue, telle une quille, entre les landiers. Le chambranle de la cheminée, les briques à l’intérieur étaient couverts d’une suie épaisse, de sorte que je me disais que ce foyer était un autel ou une chapelle tout indiquée pour une idole congolaise.\par
Les yeux rivés sur la figurine à demi visible, me sentant toujours bien mal à l’aise, j’attendais ce qui allait suivre. Il puisa d’abord dans la poche de son pardessus une poignée de copeaux, et les disposa avec soin devant l’idole ; dessus, il ajouta un biscuit de mer et, approchant la lampe, il alluma les copeaux pour le feu du sacrifice. Après plusieurs tentatives, avançant brusquement ses doigts dans la flamme et les en retirant non moins brusquement (ce qui, semble-t-il, dut les rôtir cruellement), il parvint à reprendre le biscuit, puis soufflant dessus, tant pour le refroidir que pour le débarrasser des cendres, il l’offrit poliment au petit nègre. Mais ce diable en réduction ne parut pas tenté par une nourriture aussi desséchée, il ne remua pas les lèvres. Ces étranges singeries s’accompagnaient de bruits gutturaux encore plus étranges émis par l’adepte qui paraissait traduire ses prières en mélopées ou chanter quelque psalmodie païenne et dont le visage, en même temps, se convulsait d’une manière tout à fait contre nature. Enfin, après avoir éteint le feu, il se saisit de l’idole sans cérémonie aucune, et l’emballa dans la poche de sa capote d’un geste aussi peu religieux que celui d’un chasseur fourrant une bécasse dans sa gibecière.\par
Toutes ces manigances bizarres augmentaient mon malaise et, reconnaissant tous les symptômes annonçant qu’il en aurait bientôt terminé avec ses opérations, et qu’il allait sauter dans le lit, je songeai qu’il était urgent, avant que la lumière fût éteinte, de rompre maintenant ou jamais le charme qui m’avait si longtemps envoûté.\par
Mais le temps que je mis à chercher ce que j’allais dire fut fatal. Prenant sur la table son tomahawk-pipe, il en examina un instant le foyer, l’approcha de la lumière et les lèvres serrées sur le tuyau, il souffla de grands nuages de fumée. L’instant suivant la lumière était éteinte, et ce sauvage cannibale, son calumettomahawk entre les dents, sauta à mes côtés dans le lit. Je hurlai, cette fois-ci je ne pus me retenir et, avec un soudain grognement d’étonnement, il se mit à me palper.\par
Bafouillant quelque chose, je ne sais même pas quoi, je me réfugiai hors de son atteinte contre le mur, et je le suppliai, qui qu’il fût ou quoi qu’il fût, de se tenir tranquille, de me laisser me lever et de rallumer la lampe. Ses réponses gutturales m’indiquèrent aussitôt qu’il ne comprenait guère ce que je lui demandais.\par
– Que vous êtes… quoi diable ? dit-il enfin… Vous pas parler moi… crédieu… moi touer vous. Il brandissait son tomahawk, en parlant, et un feu d’artifice s’épanouissait dans l’ombre autour de moi.\par
– Patron, pour l’amour du ciel ! Peter Coffin ! hurlai-je. Patron ! À moi Coffin ! Tous les anges, sauvez-moi !\par
– Vous parler ! Vous dit qui vous êtes, ou crédieu, moi touer vous ! gronda encore le cannibale, tandis que les tournoiements affreux de son tomahawk éparpillaient des cendres de tabac brûlant autour de moi, à tel point que je craignais qu’il ne mît le feu à ma chemise. Mais, Dieu merci, à ce moment le patron entra dans la chambre, une lampe à la main et, sautant du lit, je courus à lui.\par
– Allons, allons, n’ayez pas peur, dit-il en ricanant une fois de plus, Queequeg ne toucherait pas un seul de vos cheveux.\par
– Finissez-en avec vos sourires ! criai-je, pourquoi ne m’avez-vous pas dit que ce maudit harponneur était un cannibale ?\par
– J’ai cru que vous le saviez… ne vous ai-je pas averti qu’il colportait des têtes en ville ? mais mouillez l’ancre et dormez. Écoutez-moi bien, Queequeg, vous comprend moi, moi comprend vous – cet homme dormir avec vous – vous compris ?\par
– Moi compris beaucoup, grogna Queequeg, tirant sur sa pipe et s’asseyant sur le lit.\par
– Vous entrez là, ajouta-t-il, pointant vers moi son tomahawk et rejetant les couvertures de côté. Non seulement il mit à ce geste de la courtoisie mais encore une charitable bonté. Je m’attardai à le regarder un instant. Malgré ses tatouages, c’était en somme un cannibale propre et avenant. Pourquoi ai-je fait tous ces embarras, me dis-je, cet homme est un être humain tout comme moi, il a autant de raisons de me craindre que moi de le redouter. Et mieux vaut dormir avec un cannibale à jeun qu’avec un chrétien ivre.\par
– Patron, dis-je, dites-lui de déposer là son tomahawk, ou sa pipe, ou son je ne sais quoi, dites-lui d’arrêter de fumer, et je veux bien me coucher auprès de lui. Mais je n’apprécie guère un homme en train de fumer dans le même lit que moi. C’est dangereux. D’autre part je n’ai pas d’assurance.\par
Cela ayant été traduit à Queequeg, il se soumit aussitôt, et à nouveau m’invita aimablement à me coucher, se serrant tout d’un côté comme pour dire : je n’effleurerai même pas votre jambe.\par
– Bonne nuit, patron, dis-je, vous pouvez disposer.\par
Je me couchai et je ne dormis jamais mieux de ma vie.
\chapterclose


\chapteropen
\chapter[{CHAPITRE IV. Le couvre-lit}]{CHAPITRE IV \\
Le couvre-lit}\renewcommand{\leftmark}{CHAPITRE IV \\
Le couvre-lit}


\chaptercont
\noindent En m’éveillant, au point du jour, le lendemain matin, je constatai que le bras de Queequeg m’entourait de la manière la plus tendre et la plus affectueuse. On aurait presque pu croire que j’étais sa femme. Le couvre-lit était curieusement façonné de carrés et de triangles de tissus de différentes couleurs et le bras, tatoué d’un interminable labyrinthe crétois, imitait à s’y méprendre une bande de ce même couvre-lit fait de pièces et de morceaux. En effet ce fameux bras ne présentait pas en deux endroits une nuance précise, sans doute, pensai-je, parce qu’il fut, en mer, exposé au soleil et à l’ombre sans aucune méthode, les manches de chemise roulées à des hauteurs irrégulières. Et à dire vrai, à demi endormi encore, je le distinguais à peine du couvre-lit tant leurs couleurs se fondaient intimement ; seule la sensation de poids et de pression me laissa deviner que Queequeg me tenait enlacé.\par
Je vais essayer d’exprimer les impressions étranges dont j’étais envahi. Lorsque j’étais enfant, je m’en souviens nettement, il m’arriva une aventure assez semblable. Rêve ou réali- \hspace{1em}té ? c’est une question que je ne pus jamais éclaircir tout à fait. Mais l’aventure fut la suivante : j’avais dû faire des miennes, je crois que j’avais essayé de grimper dans la cheminée, ce que j’avais vu faire quelques jours auparavant par le petit ramoneur ; et ma marâtre, qui ne manquait pas une occasion de me donner le fouet ou de m’envoyer coucher sans souper, me tira par les pieds hors du manteau et m’expédia au lit, bien qu’il ne fût que deux heures de l’après-midi, un 21 juin, le jour le plus long de l’année dans notre hémisphère. J’étais désespéré mais c’était sans rémission et je m’acheminai vers le troisième étage où se trouvait ma petite chambre, me déshabillai aussi lentement que possible pour tuer le temps et m’enfilai entre les draps en soupirant amèrement.\par
Je restai étendu, calculant sombrement qu’il fallait que seize heures entières s’écoulassent avant que je puisse espérer une résurrection. Seize heures au lit ! le creux des reins me faisait mal rien que d’y penser. De plus, il faisait si beau, le soleil entrait par la fenêtre et des rues montait le fracas des charrettes, des bruits joyeux de voix retentissaient par toute la maison. L’angoisse me gagnait toujours davantage, finalement je me relevai, me rhabillai, et descendant sans bruit sur mes chaussettes, je me mis en quête de ma belle-mère et me jetai soudain à ses pieds, l’implorant de m’octroyer la faveur particulière d’une bonne raclée pour mon inconduite, n’importe quel châtiment plutôt que de me condamner à passer au lit un temps si intolérablement long. Mais c’était la meilleure et la plus consciencieuse des marâtres, et je dus regagner ma chambre. Je restai éveillé de longues heures, souffrant comme j’ai rarement souffert depuis, malgré les malheurs plus grands qui m’arrivèrent par la suite. Enfin je dus m’assoupir et sombrer dans l’angoisse d’un cauchemar ; j’en émergeai lentement – encore engourdi de rêve – j’ouvris les yeux, la pièce auparavant ensoleillée était plongée dans l’obscurité, et aussitôt un frisson me secoua de la tête aux pieds. Il n’y avait rien qu’on pût voir ou entendre, mais ma main semblait tenir une main surnaturelle. Mon bras pendait sur le couvre-lit, et le fantôme, ou l’apparition silencieuse, innommable, inimaginable, à qui cette main appartenait, paraissait assise à mon chevet. Pendant un laps de temps, au cours duquel les siècles semblaient s’ajouter aux siècles, je demeurai étendu, glacé d’une terreur affreuse, sans oser retirer ma main, me disant toutefois que si j’osais tenter le plus faible mouvement, ce charme horrible serait rompu. Je ne sus pas comment je fus enfin libéré de ce maléfice, mais en m’éveillant, le \hspace{1em}lendemain matin, je tremblai à ce souvenir, et durant des jours, des semaines, des mois après je m’évertuai à trouver quelque explication plausible à ce mystère. Que dis-je maintenant, encore, je me surprends à m’en intriguer.\par
Si l’on fait abstraction de mon épouvante, j’éprouvai le même sentiment d’étrangeté et de surnaturel qu’alors en m’éveillant enfermé dans le bras païen de Queequeg. Mais lorsque les événements de la nuit précédente me revinrent, un à un, en mémoire, en leur pleine réalité, seul le comique de la situation m’apparut. Car malgré mes efforts pour dégager son bras et dénouer sa nuptiale étreinte, et bien qu’il dormît profondément, son embrassement si étroit me laissa à croire que seule la mort parviendrait à nous désunir. Je m’efforçai alors de le réveiller : « Queequeg ! » Seul un ronflement me répondit. Puis j’essayai de me tourner sur le côté, mais j’étais comme pris dans un harnais et soudain quelque chose me griffa légèrement. Je rejetai le couvre-lit : le tomahawk dormait au flanc du sauvage pareil à un bébé à tête de hache. Quel pétrin ! pensai-je, se trouver au lit, en plein jour, dans une maison étrangère, avec un cannibale et un tomahawk ! « Queequeg ! pour l’amour de Dieu ! debout, Queequeg ! » Finalement, à force de contorsions, de remontrances, à haute voix réitérées, sur l’inconvenance d’une étreinte d’un style pareillement matrimonial de la part d’un autre mâle, je parvins à lui soutirer un grognement et il retira bientôt son bras ; il se secoua comme un terre-neuve au sortir de l’eau, s’assit sur le lit, raide comme un piquet, me regarda et se frotta les yeux comme s’il ne se souvenait pas tout à fait de quelle manière j’étais arrivé là, puis une vague conscience de me connaître quelque peu parut naître en lui. Cependant que, désormais sans méfiance, je l’observai paisiblement, appliqué à l’étude d’une aussi curieuse créature. Lorsque enfin il parut réaliser le caractère de son compagnon nocturne et s’en accommoder, il sauta à bas du lit et, avec divers gestes et des bruits variés, me donna à comprendre que, si cela me convenait, il s’habillerait le premier, puis m’abandonnerait toute la chambre pour me vêtir à mon tour. Que voilà de courtoises propositions en la circonstance, Queequeg, pensai-je ! En vérité, on peut dire ce qu’on veut, ces sauvages ont une délicatesse innée, leur raffinement de politesse est admirable. Je rends un hommage tout spécial à Queequeg vu qu’il me traitait avec civilité et considération, tandis que moi je me rendais coupable de grossièreté, en le dévisageant depuis le lit et en épiant tous les gestes qu’il consacrait à sa toilette. La curiosité avait momentanément raison de mon éducation. Il faut bien dire qu’on ne rencontre pas tous les jours un homme comme Queequeg et que, tant lui que ses façons valaient la peine d’être contemplés avec insistance.\par
Il commença à s’habiller par en haut, coiffa d’abord son chapeau de castor, un très haut chapeau entre parenthèses, puis, toujours sans pantalons, il se mit en quête de ses bottes. Je ne sais quelles pouvaient être ses obscures raisons, mais toujours est-il qu’il s’aplatit sous le lit, ses bottes à la main et son chapeau sur la tête. Je conclus de ses violents soupirs et de ses gémissements d’efforts qu’il avait un rude travail pour les enfiler, je n’ai jamais entendu dire que les convenances exigent d’un homme une pareille pudeur pour mettre ses bottes. Mais, voyezvous, Queequeg était une créature en voie de transformation, ni chenille, ni papillon. Il était tout juste assez civilisé pour que ses incongruités soient étrangement mises en relief. Son éducation était incomplète. Il n’en avait pas terminé avec l’école. S’il n’avait aucunement été touché par la civilisation, il ne se serait vraisemblablement pas soucié de porter des bottes, mais s’il n’était pas resté à l’état sauvage, l’idée ne l’aurait pas effleuré de se glisser sous le lit pour se chausser. Il en émergea enfin, le chapeau cabossé et enfoncé jusqu’aux yeux puis, d’un pas grinçant et boitillant, il alla et vint dans la chambre, comme si, n’étant guère habitué à porter des bottes, les siennes humides, et de cuir racorni, sans doute pas faites sur mesure, le serraient et le tourmentaient lorsqu’il venait de les enfiler par un matin glacial.\par
M’apercevant qu’il n’y avait pas de rideaux aux fenêtres, que la rue était très étroite, que la maison d’en face avait une vue plongeante dans la chambre et prenant une conscience toujours plus aiguë de l’aspect peu décoratif de Queequeg, vêtu de pas grand-chose d’autre que son chapeau et ses bottes, je le suppliai tant bien que mal d’accélérer un peu sa toilette et plus spécialement de mettre ses pantalons au plus tôt. Il obéit, puis entreprit de se laver. À cette heure matinale, n’importe quel chrétien se serait lavé la figure, mais à mon étonnement, Queequeg se contenta de limiter ses ablutions à sa poitrine, à ses bras et à ses mains. Il endossa ensuite sa veste et prenant un dur morceau de savon sur la table de toilette, il le plongea dans l’eau et commença à le faire mousser sur son visage. J’attendais de voir d’où il allait extraire son rasoir et voilà qu’il saisit le harpon dressé contre le lit, en ôta le long manche de bois et retira du fourreau la lame, la repassa un peu sur sa botte, et s’approchant du morceau de miroir accroché au mur, entreprit un raclage ou plutôt un harponnage énergique de ses joues. Queequeg, pensai-je, c’est là mésuser de la coutellerie d’un pirate. Je m’étonnai moins ultérieurement de cette opération quand j’appris de quel admirable acier est faite une tête de harpon, et à quel point on maintient effilé son long tranchant.\par
Le reste de sa toilette fut vite terminé et il sortit fièrement de la chambre, drapé dans sa veste de singe, arborant son harpon comme un bâton de maréchal.
\chapterclose


\chapteropen
\chapter[{CHAPITRE V. Le petit déjeuner}]{CHAPITRE V \\
Le petit déjeuner}\renewcommand{\leftmark}{CHAPITRE V \\
Le petit déjeuner}


\chaptercont
\noindent Je le suivis de près et descendis au bar où j’abordai le ricanant patron très affablement. Je ne nourrissais pas de rancune à son égard quand bien même il avait poussé la plaisanterie un peu loin au sujet de mon compagnon de lit.\par
Toutefois un bon rire est chose excellentissime, une bonne chose par trop rare, ce qui est d’autant plus regrettable. De sorte que si un homme paie de sa personne pour fournir la matière à une bonne plaisanterie, qu’il n’en soit pas rebuté, mais en fasse les frais en riant le premier, avec générosité, vous pouvez être sûrs que cet homme a plus d’étoffe que vous n’auriez pu le croire.\par
Tous les pensionnaires arrivés la nuit précédente, que je n’avais pas encore eu le temps de bien voir, étaient réunis dans le bar. Presque tous étaient des baleiniers, les premiers, deuxièmes et troisièmes seconds, charpentiers, tonneliers, forgerons de bord, harponneurs, et gardiens de navire, une équipe tannée, musclée, broussailleuse, poilue, hirsute, portant leurs vestes de singe en guise de robe de chambre.\par
Sans grand risque de se tromper, on pouvait deviner depuis combien de temps un homme était à terre. Les joues saines de ce jeune gars, pareilles à une poire dorée par le soleil, à croire qu’il devait en émaner le même parfum musqué, disaient qu’il ne pouvait pas avoir débarqué depuis plus de trois jours, revenant des Indes. Cet homme, près de lui, de quelques nuances plus claires, a un ton acajou. Le teint d’un troisième garde la brûlure attardée du soleil tropical mais se décolore déjà pardessous, celui-là, sans l’ombre d’un doute, est depuis des semaines à terre. Mais qui pouvait faire montre d’une joue comme celle de Queequeg, striée de diverses couleurs, et ressemblant \hspace{1em}au versant ouest des Andes dont l’éventail offre par tranches le contraste des climats.\par
– Hoé, à la bouffe ! cria le patron, ouvrant la porte et, nous entrâmes pour le petit déjeuner.\par
On dit que les hommes qui ont bourlingué, gagnent de l’aisance et de la maîtrise de soi en société. Pas toujours cependant. Ledyard, le grand voyageur de la Nouvelle-Angleterre, et Mungo Park l’Écossais étaient de tous les hommes ceux qui avaient le moins d’assurance dans les salons. Mais peut-être n’est-ce pas tout à fait le bon moyen d’acquérir du vernis social que de traverser la Sibérie dans un traîneau tiré par des chiens, comme Ledyard, ou de faire, jusqu’au cœur noir de l’Afrique, une longue randonnée solitaire avec un estomac vide, comme le pauvre Mungo. Pourtant ce vernis est nécessaire presque partout où l’on va.\par
Les circonstances m’invitèrent à ces réflexions, car lorsque nous fûmes tous réunis autour de la table, je m’apprêtais à entendre de bonnes histoires de pêche à la baleine, or, à mon grand étonnement, presque tous les hommes gardaient un profond silence. Qui plus est, ils avaient l’air embarrassé. Oui, ils étaient là, tous les loups de mer dont un bon nombre avaient pour la première fois abordé sans l’ombre d’une hésitation les grandes baleines sur des mers lointaines et les avaient, sans broncher, provoquées en combat singulier ; et pourtant ils étaient assis là, à la table d’un fraternel petit déjeuner – tous du même métier, tous ayant des goûts semblables – et ils se jetaient des regards timides comme s’ils n’avaient jamais quitté quelque bergerie perdue dans les Montagnes Vertes. Curieux spectacle que ces ours pudiques, ces timides guerriers de la pêche à la baleine !\par
Quant à Queequeg, car Queequeg était assis avec eux et le hasard voulait qu’il présidât la table, il était froid comme un glaçon. Bien sûr, je ne puis faire grand éloge de son éducation. Son plus grand admirateur n’aurait guère pu lui trouver d’aimables excuses à avoir pris son fer avec lui et à s’en servir sans plus de façons, le brandissant au-dessus de la table, mettant en danger de nombreuses têtes, pour harponner les biftecks. Mais il avait pour ce faire des gestes calmes et tout un chacun sait que, selon une opinion répandue, faire quoi que ce soit froidement, c’est le faire avec distinction.\par
Nous n’aborderons pas ici le thème de toutes les singularités de Queequeg, nous passerons sous silence sa manière de s’abstenir de café et de petits pains chauds pour concentrer une attention sans faille sur les seuls biftecks saignants. Qu’il suffise de dire que le petit déjeuner terminé, il se retira comme tout le monde dans la salle commune, alluma son tomahawk-calumet, s’assit pour digérer paisiblement, et fuma coiffé de son inséparable chapeau, tandis que je sortais pour aller faire un tour.
\chapterclose


\chapteropen
\chapter[{CHAPITRE VI. La rue}]{CHAPITRE VI \\
La rue}\renewcommand{\leftmark}{CHAPITRE VI \\
La rue}


\chaptercont
\noindent Si je m’étais étonné tout d’abord, après le premier regard jeté sur un être aussi incongru que Queequeg, en l’imaginant côtoyant le beau monde d’une ville civilisée, mon étonnement disparut incontinent dès que je parcourus de jour les rues de New Bedford.\par
Tout grand port maritime offre, aux alentours de ses quais, le spectacle d’étrangers des plus bizarres et des plus hétéroclites. Même dans Broadway et Chesnut Streets, il arrivera que des marins venus de la Méditerranée bousculent des dames effarouchées. Regent Street n’est pas inconnue des lascars et des Malais et, sur l’Apollo Green de Bombay, d’entreprenants Yankees ont souvent effrayé les natifs. Mais New Bedford bat le record de toutes les Water Street et Wapping, où vous ne rencontrerez guère que des marins, tandis qu’à New Bedford d’authentiques cannibales taillent une bavette au coin des rues, parfaits sauvages, dont beaucoup ont sur les os une chair non sanctifiée. Insolite spectacle !\par
Mais en plus des insulaires des Fidji, de Tongatabou, de Panjang, d’Erromango Bright et en plus des échantillons farouches produits par les équipages des baleiniers qui déambulent inaperçus, la rue vous proposera des spectacles encore plus singuliers et certainement plus comiques. Chaque semaine, on voit arriver des citadins de Vermont et du New Hampshire dévorés par l’appât du gain et de la gloire des grandes pêches. Ce sont pour la plupart des gars jeunes et robustes qui, après avoir abattu des forêts entières, cherchent à troquer la hache contre le harpon. Ils sont aussi verts que les montagnes dont ils descendent. À bien des égards, on les croirait tout frais sortis de la coquille. Voyez-moi ce gars qui fait la roue, là-bas au coin ! Il porte un chapeau de castor, un manteau en queue d’hirondelle, une ceinture de marin serrée à la taille et un couteau dans sa gaine. Et cet autre qui arrive en suroît et cape d’alépine.\par
Il n’est pas de dandy citadin capable de rivaliser avec un gandin sorti de la campagne – j’entends un vrai rustre du dandysme – un gars qui, en pleine canicule, fauchera ses deux acres les mains gantées de daim de crainte qu’elles ne se tannent. Eh bien, quand un dandy de cette trempe s’est mis dans l’idée de se tailler une réputation à sa hauteur, et cherche à s’embarquer pour la pêche à la baleine, vous devriez voir à quelles cocasseries il s’adonne en arrivant au port. En commandant l’habit qu’il portera en mer, il réclame des boutons d’uniforme d’apparat pour sa vareuse, des bretelles pour ses pantalons de toile. Ah ! pauvre rustaud ! combien cruellement, au premier hurlement de la tempête, vont s’envoler ces ornements tandis que vous serez précipités, bretelles, boutons et tout, droit au fond du gosier de la tempête. Mais n’allez pas croire que cette ville fameuse ne peut montrer à ses hôtes que des harponneurs, des cannibales et des blancs-becs. Pas du tout. Pourtant, New Bedford est un lieu insolite. Mais sans nous autres baleiniers, cette région serait peut-être encore maintenant, telles les côtes du Labrador, le seul domaine du vent. Son arrière-pays est assez rocailleux pour effrayer encore son monde. Mais la ville est peut-être l’endroit où la vie est la plus chère de toute la Nouvelle-Angleterre. C’est bien une terre grasse, certes, mais non à la façon de Canaan, une terre de blé et de vin. Le lait ne ruisselle pas dans les rues, et le printemps ne les pave pas d’œufs frais. Pourtant, malgré cela, nulle part ailleurs en Amérique, vous ne trouverez autant de maisons patriciennes, des parcs et des jardins plus somptueux qu’à New Bedford. D’où viennent-ils et comment poussèrent-ils sur la scorie de cette terre maigre ?\par
Allez faire un tour et regardez bien, autour de ces hautaines demeures, se dresser l’emblème des harpons de fer, c’est la réponse à votre question. Ces jardins en fleurs et ces pimpantes maisons sortent tout droit des océans Atlantique, Pacifique et Indien. Toutes ces demeures ont été prises au harpon et arrachées au fond des mers. Herr Alexandre lui-même aurait-il pu accomplir pareil tour de prestidigitation ?\par
À New Bedford, dit-on, les pères dotent leurs filles avec des baleines et pourvoient leurs nièces grâce à quelques marsouins. Si vous voulez voir un beau mariage, c’est à New Bedford qu’il faut aller, car on raconte que les maisons ont des réserves d’huile inépuisables et que chaque nuit on y brûle avec insouciance des bougies de blanc de baleine tout entières.\par
L’été, la ville est exquise à voir, des érables sans nombre bordent de vert et d’or les longues avenues. Et en août, les beaux marronniers altiers et munificents, tels des candélabres, offrent au passant les cierges dressés de leurs bouquets en touffes. La toute-puissance de l’art est telle qu’en plus d’un quartier de New Bedford il a transformé en éclatantes terrasses fleuries les rochers nus, déchets inutiles rejetés au dernier jour de la création.\par
Les femmes de New Bedford, elles, fleurissent à l’égal de leurs propres roses rouges. Mais tandis que les roses ne s’épanouissent qu’en été, le délicat incarnat de leurs joues est éternel comme la lumière du septième ciel. Vous ne trouveriez nulle part ailleurs pareille splendeur, sauf peut-être à Salem, où les jeunes filles, dit-on, répandent un parfum si musqué que leurs amoureux marins le reconnaissent à des milles de la côte, comme s’ils approchaient des odorantes Moluques bien plutôt que des sables puritains.
\chapterclose


\chapteropen
\chapter[{CHAPITRE VII. La chapelle}]{CHAPITRE VII \\
La chapelle}\renewcommand{\leftmark}{CHAPITRE VII \\
La chapelle}


\chaptercont
\noindent C’est à New Bedford aussi que se trouve une chapelle des baleiniers. De tous les pêcheurs ombrageux, sur le point de partir vers l’océan Indien ou vers le Pacifique, rares sont ceux qui n’y font pas leur visite dominicale. Je n’ai pas fait exception.\par
À peine rentré de ma promenade matinale, je ressortis dans cette intention. Le temps, froid et ensoleillé, s’était mis à la neige et au brouillard. Serrant ma veste poilue, de ce tissu dit peau d’ours, je fonçai tête baissée dans la tempête inexorable. Quelques groupes épars de marins, de femmes et de veuves de marins y étaient déjà réunis quand j’entrai. Le silence ouaté n’y était brisé parfois que par les sifflements du vent. Chaque fidèle paraissait s’être volontairement assis loin l’un de l’autre, comme si chaque douleur silencieuse était une île inapprochable. Le pasteur n’était pas encore arrivé, et ces îlots muets d’hommes et de femmes attendaient en fixant des yeux les plaques de marbre, bordées de noir, encastrées dans le mur de part et d’autre de la chaire. Trois d’entre elles disaient à peu près ceci, mais je ne prétends pas citer :\par

\begin{center}
\noindent \centerline{À la mémoire de JOHN TALBOT Perdu en mer à l’âge de dix-huit ans près de l’île de la Désolation, au large de la Patagonie 1\textsuperscript{er} novembre 1836}\par
\end{center}


\begin{center}
\noindent \centerline{Cette plaque est érigée à sa mémoire par sa sœur.}\par
\end{center}


\begin{center}
\noindent \centerline{* * *}\par
\end{center}


\begin{center}
\noindent \centerline{À la mémoire de}\par
\end{center}


\begin{center}
\noindent \centerline{ROBERT LONG, WILLIS ELLERY, NATHAN COLEMAN, WALTER CANNY, SETH MACY, ET SAMUEL GLEIG}\par
\end{center}


\begin{center}
\noindent \centerline{de l’équipage de L’{\itshape Eliza}}\par
\end{center}


\begin{center}
\noindent \centerline{Qui fut entraîné par une baleine et perdu au large dans le Pacifique}\par
\end{center}


\begin{center}
\noindent \centerline{31 décembre 1839}\par
\end{center}


\begin{center}
\noindent \centerline{Ce marbre a été posé ici par leurs compagnons survivants.}\par
\end{center}


\begin{center}
\noindent \centerline{* * *}\par
\end{center}


\begin{center}
\noindent \centerline{À la mémoire du regretté CAPITAINE EZECHIEL HARDY Qui fut tué par un cachalot à l’étrave de sa pirogue sur la côte du Japon}\par
\end{center}


\begin{center}
\noindent \centerline{3 août 1833}\par
\end{center}


\begin{center}
\noindent \centerline{Cette plaque est érigée à son souvenir par sa veuve.}\par
\end{center}

\noindent Secouant la neige fondue de mon chapeau et de ma veste lustrés de blanc, je m’assis près de la porte et fus surpris en me tournant de découvrir Queequeg près de moi. Ému par la solennité du lieu, son attitude exprimait un étonnement mêlé d’incrédule curiosité. Ce sauvage fut le seul dans l’assemblée qui parut remarquer mon entrée, car il était seul à ne pas savoir lire et n’était pas, dès lors, absorbé à déchiffrer les froides inscriptions des murs. Y avait-il là des membres des familles dont les noms y étaient gravés, je n’en savais rien ; car d’innombrables accidents de pêche ne sont jamais relatés, et bien des femmes présentes avaient l’expression sinon les atours d’un deuil inconsolable, et j’eus la certitude qu’ici se rouvraient et saignaient à nouveau d’incurables blessures à la vue de ces plaques funèbres.\par
Oh ! vous dont les morts dorment dans le linceul d’une herbe verte et qui, debout parmi les fleurs, pouvez dire : là… c’est là que repose mon bien-aimé, vous ne pouvez savoir quelle désolation ronge des poitrines comme celles-là. Vous ne pouvez savoir le vide amer suscité par ces plaques bordées de noir qui ne recouvrent point de cendres ! Quel désespoir dans ces inscriptions immuables ! Quelle désertion que l’on n’attendait pas, quelle absence dévastatrice disent ces lignes qui semblent corroder toute foi et paraissent refuser la résurrection à ceux qui sont morts sans sépulture en des lieux inconnus ! Ces plaques pourraient être érigées dans les grottes d’Elephanta tout aussi bien qu’ici.\par
Où les morts de l’humanité figurent-ils au recensement des vivants ? Pourquoi un proverbe universel veut-il que les morts ne parlent pas alors qu’ils détiennent plus de secrets que les sables de Goodwin ? Comment osons-nous, parlant de celui qui, hier, s’en est allé pour l’autre monde, employer une expression aussi lourde que celle de départ définitif, alors que nous ne la risquerions pas s’il s’était simplement embarqué pour l’endroit le plus reculé des Indes sur la terre des vivants ? Pourquoi les assurances sur la vie paient-elles des primes face à l’immortalité ? Notre ancêtre Adam, qui mourut il y a quelque soixante siècles, de quel sommeil ne dort-il pas encore dans une immobilité absolue, éternelle, dans une catalepsie sans espoir ! Comment se fait-il que nous soyons inconsolables alors que nous croyons à la béatitude indicible des morts ? Pourquoi tous les vivants s’acharnent-ils à les réduire au silence ? Pourquoi le simple racontar relatif à un coup frappé dans une tombe peut-il répandre la terreur dans une cité entière ? Tout cela doit avoir un sens sans doute.\par
Mais la Foi, tel un chacal, se nourrit parmi les tombes et sa plus vivante espérance naît des doutes qui planent sur la mort.\par
Il est à peine besoin de dire quels étaient mes sentiments, à la veille de partir pour Nantucket alors que je contemplais ces marbres où s’inscrivait le sort des pêcheurs embarqués avant moi, tandis que la journée lugubre ne dispensait que ténèbres. Oui, Ismaël, pareille destinée pourrait bien être tienne. Pourtant, je me sentais redevenir joyeux. L’attrait enchanteur du départ… l’occasion, semble-t-il, unique d’être promu à un grade honoraire de l’immortalité en sombrant en mer. Oui, la mort est présente à la pêche à la baleine, elle y expédie un homme dans l’éternité sauvagement et en moins de temps qu’il n’en faut pour le dire. Mais alors ? Nous avons fait, je crois, une erreur formidable sur cette question de la vie et de la mort. Je crois que ce qu’on appelle mon ombre sur la terre est ma substance vraie… Je crois qu’en matière de spiritualité nous ressemblons par trop à des huîtres qui, contemplant le soleil à travers la mer, prennent l’eau la plus épaisse pour l’air le plus léger. Je crois que mon corps n’est pas que la lie de mon être supérieur. En fait, qu’emporte mon corps qui veut, prenez-le, dis-je, il n’est pas moi. Et dès lors applaudissons trois fois Nantucket et que sombre le navire et que sombre le corps car mon âme, Jupiter luimême ne pourrait l’envoyer par le fond.
\chapterclose


\chapteropen
\chapter[{CHAPITRE VIII. La chaire}]{CHAPITRE VIII \\
La chaire}\renewcommand{\leftmark}{CHAPITRE VIII \\
La chaire}


\chaptercont
\noindent Je n’étais pas assis depuis longtemps lorsque entra un homme d’un certain âge, encore vigoureux ; l’orage s’engouffra à sa suite et la porte se referma. Le regard déférent que lui adressa rapidement l’assemblée me prouva que ce beau vieillard était le pasteur. C’était, en effet, le père Mapple, comme l’appelaient les baleiniers qui l’estimaient grandement. Il avait été marin et harponneur au temps de sa jeunesse mais depuis de nombreuses années il s’était consacré au ministère. À l’heure dont je parle, une saine robustesse accordait à son âge un hiver victorieux ; sa vieillesse était de celles qui semblent s’épanouir en une nouvelle jeunesse, car tous les sillons de ses rides rayonnaient d’un doux renouveau, verdure printanière transperçant les neiges de février. Même en ignorant tout de son histoire, on ne pouvait manquer de considérer le père Mapple avec le plus grand intérêt car la vie aventureuse qu’il avait menée en mer marquait de singularité son attitude cléricale. Je remarquai, dès son entrée, qu’il n’avait pas de parapluie et il n’avait certes pas fait le trajet en voiture car son chapeau de toile goudronnée dégoulinait de neige et son grand manteau de drap était tellement alourdi d’eau que son poids semblait le tirer vers le sol. Chapeau, manteau et guêtres furent toutefois enlevés tour à tour et suspendus dans un coin puis, vêtu pour la circonstance, il se dirigea silencieusement vers la chaire.\par
Comme la plupart des chaires d’autrefois, celle-ci était très haute ; un escalier normal grimpant jusqu’à pareille altitude aurait fait un angle très aigu avec le plancher et de ce fait aurait très sérieusement empiété sur l’espace déjà restreint de la chapelle, aussi l’architecte s’était-il apparemment soumis aux suggestions du père Mapple et avait-il terminé sans marches la chaire à laquelle on accédait sur le côté par une échelle verticale pareille à celle de la coupée qui, sur le flanc d’un navire, permet d’y monter depuis les embarcations. La femme d’un capitaine baleinier l’avait pourvue d’une main courante de laine rouge ; elle se terminait par de jolis pommeaux et était teintée de couleur acajou ; ce dispositif, étant donné la nature de cette chapelle, n’était aucunement de mauvais goût. S’arrêtant un instant au pied de l’échelle, le père Mapple saisit à deux mains les nœuds ornant la main courante, leva les yeux, puis avec l’adresse d’un vrai marin, sans rien perdre d’une attitude respectueuse, les mains s’accrochant l’une après l’autre, il escalada les barreaux comme s’il se hissait au sommet du grand mât de son navire.\par
Les montants verticaux de cette échelle, comme le sont habituellement ceux de toute échelle volante, étaient de corde recouverte de tissu, seuls les échelons étaient en bois, de sorte que chacun comportait une articulation. Au premier coup d’œil je les avais remarquées, songeant que si elles étaient de toute utilité sur un navire, elles paraissaient ne pas se justifier ici. Ce à quoi je ne m’attendais pas, c’était à voir le père Mapple, une fois installé sur ses hauteurs, se retourner lentement, se pencher par-dessus la chaire et délibérément remonter, marche par marche, l’échelle, jusqu’à ce qu’elle fût toute rentrée, le laissant lui, inexpugnable, dans son petit Québec.\par
Je méditai un moment, sans comprendre pleinement, quelles pouvaient être ses raisons. Sa réputation de sincérité et de sainteté était si largement établie que je ne pouvais le soupçonner de rechercher une gloriole grâce à de méchants trucs de mise en scène. Non, me disais-je, il doit avoir un motif raisonnable, par ailleurs ce peut être le symbole d’une chose invisible. Peut-être s’isolant ainsi physiquement, tend-il à exprimer la suppression momentanée de tous les liens et de tous ses rapports avec le monde extérieur. Oui, cette chaire, pour ce fidèle homme de Dieu, est emplie de la chair du sang de la Parole, je comprends quelle est une forteresse autonome, une altière Ehrenbreistein, possédant entre ses murs une source éternelle.\par
Le pasteur n’avait pas emprunté à son passé de lointains voyages cette seule caractéristique étrange qu’était l’échelle. Entre les cénotaphes de marbre dressés de part et d’autre de la chaire, le mur qui en formait le dos était orné d’une fresque représentant un vaillant navire luttant contre une effroyable tempête au large d’une côte sous le vent dont les rochers noirs étaient enneigés par les brisants. Mais loin au-dessus des sombres nuées en fuite, une petite île de lumière flottait, un ange rayonnant de là se penchait et son visage éblouissant posait, lointainement sur le pont du navire secoué, un reflet assez semblable à la plaque d’argent qui rappelle, à bord du Victory, l’endroit où y tomba Nelson. Et l’ange semblait dire : « Ah ! toi, noble vaisseau, tiens bon et gouverne hardiment, car voici que le soleil revient, les nuages s’éloignent et l’azur le plus serein est proche. »\par
L’échelle et le tableau n’étaient pas seuls à donner à cette chaire un goût d’embrun. Elle se renflait comme une proue et la Sainte Bible reposait sur un enroulement imitant la volute des proues des navires.\par
Rien ne pouvait être plus lourd de sens, car une chaire est une étrave. Elle entraîne tout le monde dans son sillage et ouvre la voie à l’humanité. De là on voit approcher la brusque tempête de la colère divine et la proue est la première à soutenir l’attaque. De là montent les implorations pour des vents favorables vers les dieux qui régissent leurs forces bonnes ou mauvaises. Oui, le monde est un navire éphémère qui ne parfait pas son voyage et la chaire est son étrave.
\chapterclose


\chapteropen
\chapter[{CHAPITRE IX. Le sermon}]{CHAPITRE IX \\
Le sermon}\renewcommand{\leftmark}{CHAPITRE IX \\
Le sermon}


\chaptercont
\noindent Le père Mapple se leva et d’une voix douce, autoritaire avec modestie, il invita ses ouailles éparses à se grouper : « Que les tribordais se rapprochent des bâbordais ! Que les bâbordais se rapprochent des tribordais ! Tous au centre du navire ! »\par
Un remue-ménage de lourdes bottes de mer s’accompagna entre les bancs du pas traînant mais plus léger des femmes, puis le silence se fit et tous les regards se tournèrent vers le prédicateur.\par
Il se tut un instant puis, agenouillé à l’étrave de la chaire, il croisa sur sa poitrine ses larges mains brunes, leva son visage aux yeux clos, et pria avec tant de ferveur qu’on eût dit qu’il adressait sa supplication du fond de l’Océan.\par
Sa voix avait ce timbre solennel de la cloche qui sonne sans arrêt à bord d’un navire pris dans le brouillard et, sa prière terminée, il poursuivit, sur le même timbre, la lecture de l’hymne suivant dont il fit retentir les dernières strophes d’une joie exultante :\par


\begin{verse}
La voûte terrifiante de la baleine\\
Arquait au-dessus de moi ses lugubres ténèbres.\\
Tandis que les vagues roulaient dans la lumière bénie\\
Me soulevant et m’envoyant plus profond à ma perte.\\!

Je vis s’ouvrir la gueule de l’enfer\\
Avec ses tourments, ses douleurs éternelles\\
Connues des seuls damnés\\
Ah ! je sombrai dans le désespoir !\\!

Du fond de ma détresse je criai vers Dieu\\
N’osant le croire disposé à m’entendre\\
Pourtant il écouta ma plainte\\
Et la baleine me rejeta.\\!

Il vola en hâte à mon secours\\
Comme porté par un dauphin radieux\\
Insoutenable et pourtant brillante comme la foudre\\
Était la face de mon Dieu Sauveur.\\!

Mon chant pour jamais redira\\
Cette heure de joie terrible\\
Je rendrai gloire à mon Dieu\\
Pour sa miséricorde et sa puissance.\\
\end{verse}

\noindent Presque toutes les voix s’unirent en ce chant qui domina les hurlements de la tempête. Un silence suivit. Le prédicateur tourna lentement les pages de la Bible et enfin, posant la main sur la page choisie, il dit : « Mes bien-aimés camarades de bord, étalinguez au dernier verset de Jonas : Et l’Éternel envoya un grand poisson qui engloutit Jonas… »\par
« Camarades, ce livre, avec ses seuls quatre chapitres, quatre bitords, est l’un des plus petits torons du puissant câble des Écritures. Et pourtant à quelle profondeur de l’âme Jonas n’envoie-t-il pas la sonde ? Quelle fécondité dans la leçon du prophète ! Quelle ne fut pas sa noblesse à entonner ce cantique dans le ventre même du poisson ! Quelle majesté de grandes vagues tumultueuses ! Nous sentons les flots passer par-dessus nos têtes, avec lui nous tâtons du varech des grands fonds, tous les goémons et les limons de la mer nous enveloppent ! Mais quelle est cette leçon que nous enseigne le livre de Jonas ? Camarades, c’est une leçon à deux bitords, une leçon qui s’adresse à nous tous pécheurs, et à moi en particulier en tant que pilote du Dieu vivant. C’est une leçon qui s’adresse à nous tous pécheurs parce qu’elle relate l’histoire du péché, de la dureté du cœur, des craintes soudain éveillées, d’un prompt châtiment, du repentir, des prières et enfin de la délivrance et de la joie de Jonas. Le péché de tous les hommes, comme celui de ce fils d’Amittaï, est celui d’une désobéissance délibérée au commandement de Dieu ; nous ne parlerons pas maintenant de ce qu’était cet ordre ni de la manière dont il fut transmis et qu’il trouva si difficile à respecter. Mais tout ce que Dieu nous demande, souvenez-vous en, est ardu à accomplir, c’est pourquoi il ordonne plus souvent qu’il n’entreprend de persuader. Et si nous obéissons à Dieu, nous devons nous désobéir à nousmêmes, et c’est dans cette désobéissance à nous-mêmes que réside la difficulté d’obéir à Dieu.\par
« Portant en lui ce péché de désobéissance, Jonas l’aggrave encore, narguant Dieu en cherchant à le fuir. Il croit qu’un navire construit par des hommes l’emportera vers des pays où Dieu ne règne pas mais dont seuls sont maîtres les capitaines de ce monde. Il rôde furtivement sur les appontements de Joppé, en quête d’un bateau en partance pour Tarsis. Il y a peut-être à cela un sens jusqu’ici dédaigné. Toutes les études veulent que Tarsis ne soit rien d’autre que la moderne Cadix. Telle est l’opinion des savants. Et où se trouve Cadix, camarades ? Cadix est en Espagne. C’est le point le plus éloigné, par mer, de Joppé que peut-être Jonas pouvait atteindre en ces temps anciens où l’Atlantique était encore une mer presque inconnue. Car Joppé, la moderne Jaffa, camarades, se trouve sur la côte extrême est de la Méditerranée, en Syrie ; et Tarsis ou Cadix est à plus de deux mille milles à l’ouest de là, au-delà du détroit de Gibraltar. Ne voyez-vous pas alors, camarades, que Jonas cherchait à mettre entre Dieu et lui l’immensité du monde ? Misérable individu ! Ô misérable, le plus digne de tous les mépris, s’éloignant de son Dieu avec son regard coupable et son chapeau rabattu sur les yeux ; cherchant sournoisement à s’embarquer comme un voleur infâme, anxieux de traverser l’Océan. Sa mine trahit un tel désarroi qu’elle est sa propre condamnation, s’il y avait eu des policiers en ces temps anciens, sur le simple soupçon offert par son air inquiet, Jonas eût été arrêté avant de monter sur un pont de navire. Il se trahit ouvertement comme fuyard, aucun bagage, ni une valise, ni une boîte à chapeau, ni un sac de voyage, point d’amis pour l’accompagner de leurs adieux jusqu’à l’estacade. Enfin, après bien des louvoyantes recherches, il trouve, en partance pour Tarsis, le navire finissant de compléter sa cargaison et, tandis qu’il pose le pied à bord pour aller vers le capitaine, tous les marins cessent d’embarquer les marchandises, devant le regard mauvais de l’étranger. Jonas le remarque, vainement il essaie de paraître à l’aise et sûr de lui, vainement il ébauche un sourire malheureux. Les forts pressentiments qu’ils ont de l’homme assurent aux marins qu’il ne saurait être innocent. À leur manière d’exprimer des choses sérieuses de façon badine, l’un chuchote à l’autre : « Jack, il vient de dévaliser une veuve » ou « Joe, regardez-le bien, c’est un bigame » ou encore, « Harry, mon vieux, je pense que c’est un adultère échappé des prisons de Gomorrhe, ou peut-être l’un des meurtriers portés manquant de Sodome ». Un autre se précipite vers la pile de l’appontement où le navire est amarré pour lire l’annonce offrant une prime de cinq cents pièces d’or pour l’arrestation d’un parricide dont elle donne le signalement. Tout en lisant, il regarde tantôt l’affiche, tantôt Jonas, tandis que, faisant chorus avec lui, ses camarades de bord se rassemblent autour de Jonas, prêts à se saisir de lui. Jonas tremble d’épouvante, son effort pour prendre une contenance audacieuse le fait paraître plus lâche encore. Il ne peut se reconnaître suspect, et cela même est en soi une présomption défavorable. Aussi fait-il bonne figure à mauvais jeu et, quand les marins se rendent compte qu’il n’est pas l’homme recherché, ils le laissent passer et il descend dans la cabine du capitaine.\par
– Qui est là ? crie ce dernier, affairé à sa table, préparant hâtivement des papiers pour la douane. – Qui est là ? Ah ! comme cette innocente question déchire le cœur de Jonas ! Il est sur le point de faire demi-tour pour fuir à nouveau. Mais il se ressaisit. – « Je cherche à embarquer à votre bord à destination de Tarsis, quand partez-vous, monsieur ? » Le capitaine, débordé de travail, n’avait pas jusque-là levé les yeux vers Jonas bien que celui-ci se trouvât droit devant lui à présent, mais à peine a-t-il entendu cette voix sourde, qu’il lui lance un regard inquisiteur. « Nous appareillons à la prochaine marée », répondit-il enfin avec lenteur sans le quitter du regard. « Pas avant, monsieur ? » « C’est bien assez tôt pour n’importe quel honnête passager. » Ah ! Jonas, voilà un nouveau coup au cœur ! Mais il se hâte de détourner le capitaine de cette piste. « Je m’embarquerai avec vous, dit-il, le prix du voyage, à combien se monte-t-il ? Je paie tout de suite. » Car cela est écrit, camarades, afin de ne pas passer inaperçu dans cette histoire, et il est dit qu’il paya le prix du passage avant le départ du navire. Cette phrase, prise avec le contexte, pèse lourd.\par
« Maintenant le capitaine de Jonas, camarades, était de ceux dont la perspicacité décelait le crime là où il se trouvait mais il était cupide au point de ne le livrer que s’il n’y avait pas d’argent à l’appui. Car, en ce monde, camarades, le Péché qui paie son passage peut voyager librement et sans papiers, alors que la Vertu, en pauvresse se fait, elle, arrêter à toutes les frontières. De sorte que le capitaine s’apprête à sonder le portemonnaie de Jonas avant de le juger ouvertement. Il lui demande trois fois plus que le prix habituel et ce prix est accepté. Alors le capitaine comprend que Jonas est un fuyard mais il est également disposé à favoriser une fuite qui sème l’or sur ses talons. Pourtant, quand Jonas sort loyalement sa bourse, une prudence soupçonneuse tenaille encore le capitaine. Il fait sonner chaque pièce de crainte qu’il n’y en ait une fausse. Il marmonne : « En tout cas, pas un faux-monnayeur » et il inscrit Jonas pour son passage. « Voulez-vous me montrer ma cabine, je vous prie, monsieur, dit à présent Jonas, je suis fatigué d’un long voyage et j’ai besoin de dormir. » « Cela se voit, en effet, répond le capitaine, voici votre cabine. » Jonas entre, et fermerait la porte à clef si la serrure comportait une clef. En l’entendant tracasser sottement ce loquet, le capitaine rit dans sa barbe et marmotte quelque chose au sujet des portes des geôles des bagnards qui ne sont jamais autorisés à s’enfermer. Jonas se jette sur sa couchette, tout habillé et couvert de poussière qu’il est, pour s’apercevoir qu’il touche presque du front le plafond de la petite cabine. L’air est confiné et Jonas étouffe. Ainsi, dans ce trou resserré, situé de plus au-dessous de la ligne de flottaison, Jonas vit déjà le pressentiment de cette heure suffocante où la baleine le serrera au plus étroit de ses entrailles.\par
« Suspendue à la Cardan contre la paroi, une lampe oscille et se balance légèrement dans la cabine de Jonas et, le navire donnant de la bande du côté du quai à cause du poids des derniers ballots embarqués, la lampe et sa flamme, malgré ce mouvement, conservent une obliquité parfaite par rapport à la cabine, et, bien qu’en vérité inflexiblement droite, elle révèle les niveaux mensongers parmi lesquels elle est suspendue. Cette lampe angoisse et terrifie Jonas, tandis qu’il est là, étendu, son regard inquiet fait le tour de la cabine, mais ce fuyard, jusqu’à maintenant en sécurité, ne trouve pas où reposer ses yeux fureteurs. Et les contradictions que souligne la lampe lui inspirent une horreur grandissante. Tout est de guingois, plancher, plafond, parois. « Oh ! ma conscience en moi est pareillement suspendue et sa flamme brûle droit, mais toutes les parois de mon âme sont distordues ! » gémit-il.\par
« Comme celui qui, après une nuit de beuverie, titube encore en courant jusqu’à son lit, avec une conscience en éveil qui l’aiguillonne encore, tel le cheval de course romain dont le mors, pourvu de crocs, pénètre plus profond chaque fois qu’il s’élance. L’homme pris dans cette misérable situation se tourne et se retourne dans le vertige de l’angoisse, supplie Dieu de l’anéantir jusqu’à ce que la crise soit passée. Enfin une stupeur profonde l’arrache au tourbillon de sa douleur, pareille à celle qui envahit l’homme qui saigne à mort, car la conscience est une plaie dont rien ne saurait étancher le flot hémorragique. Ainsi Jonas, après s’être douloureusement débattu sur sa couche, sombra dans le sommeil, entraîné par le poids d’une prodigieuse misère.\par
« Et maintenant l’heure de la marée est venue, le navire largue ses amarres et quittant le quai désert où personne ne salue son départ, il glisse sur la mer, donnant de la bande, vers Tarsis. Mes amis, ce navire fut le premier connu à faire de la contrebande, et la marchandise non déclarée c’était Jonas. Mais la mer se révolte, elle ne veut pas porter ce mauvais fardeau. Un orage terrible se déclare, le navire est sur le point de se briser. Lorsque le maître d’équipage appelle tous les hommes pour alestir le vaisseau, lorsque les coffres, les ballots et les jarres clapotent par-dessus bord, lorsque le vent grince et hurlent les hommes, et que chaque planche tonne sous les piétinements audessus de sa tête, à travers ce tumulte enragé, Jonas poursuit son hideux sommeil. Il ne voit ni le ciel obscur, ni la mer en furie, il n’entend pas craquer les membrures, à peine perçoit-il, ou remarque-t-il dans le lointain, la ruée de la puissante baleine qui, d’ores et déjà, la gueule béante, fend les mers à sa poursuite. Oui, camarades, Jonas, dans les flancs d’un navire, étendu sur sa couchette, dormait profondément. Mais le maître d’équipage, dans sa terreur, vint à lui et cria dans son oreille morte : « Pourquoi dors-tu ? Lève-toi ! » Tiré en sursaut de sa léthargie par ce lugubre cri, Jonas chancela sur ses pieds et, trébuchant jusqu’au pont, saisit un hauban, et contempla la mer. Mais au même instant, bondissant par-dessus la lisse, une vague se jeta sur lui comme une panthère. Ainsi une vague après l’autre bondit sur le navire et, les dalots n’étant pas assez prompts à les boire, elles vont rugissant de l’avant à l’arrière, noyant presque les marins avant le naufrage. Et tandis que la lune blanche montre un visage apeuré dans les ravins d’un \hspace{1em}ciel de ténèbres, Jonas, figé, voit le beaupré se dresser, pointer vers le ciel, et s’abattre aussitôt vers les profondeurs suppliciées.\par
« La terreur poursuit la terreur en hurlant à travers son âme. Son échine courbée ne révèle que trop sa fuite devant Dieu. Les marins s’en aperçoivent, les soupçons qu’ils nourrissent envers lui grandissent et enfin, pour faire éclater la vérité, et s’en remettant complètement au jugement du ciel, ils tirent au sort pour savoir lequel d’entre eux leur attire cette grande tempête. Le sort tombe sur Jonas, quelle fureur ne mettent-ils pas alors à l’assaillir de questions : « Quelles sont tes affaires, et d’où viens-tu ? » « Ton pays ? » « Ton peuple ? » Mais, camarades, remarquez à présent le comportement du malheureux Jonas. Le pressant, les marins lui demandaient seulement qui il était et d’où il venait, or non seulement ils reçoivent une réponse à leurs questions mais encore à une autre question qu’ils n’ont pas posée, et cette réponse non sollicitée est arrachée à Jonas par la dure main de Dieu qui pèse sur lui.\par
« Je suis Hébreu, s’écrie-t-il, et je crains l’Éternel, le Dieu des cieux qui a fait la mer et la terre ! » Tu le crains, ô Jonas ? Oui, tu avais de bonnes raisons de craindre le Seigneur ton Dieu, en ce moment ! Aussitôt il fait un aveu complet qui amène les marins au comble de l’épouvante et toutefois les emplit de pitié. Car lorsque Jonas, qui n’implorait pas encore la miséricorde de Dieu, sachant trop bien quelles ténèbres il méritait, lorsque le misérable Jonas leur crie de le prendre et de le jeter dans la mer, reconnaissant qu’il leur avait attiré cette grande tempête, ils se détournent compatissants et se concertent pour trouver un autre moyen de sauver le navire. En vain ! L’ouragan indigné hausse la voix, alors une main levée en supplication vers Dieu, ils ferment à contre-cœur l’autre sur Jonas.\par
« Et voyez à présent Jonas saisi comme une ancre et jeté à la mer. Sur-le-champ, à l’est s’étale une mer d’huile et les flots sont apaisés car Jonas emporte avec lui la tempête et l’eau \hspace{1em}derrière lui est sans rides. Il est happé dans le maelström d’un remous si irrésistible qu’il s’aperçoit à peine de l’instant où le bouillonnement le jette entre les mâchoires béantes qui l’attendent, et la baleine claque ses dents d’ivoire et ferme sur sa prison autant de barreaux blancs. Alors Jonas pria Dieu dans le ventre de la baleine. Mais méditez sa prière et tirez-en une leçon majeure. Car tout pécheur qu’il soit, Jonas ne pleure ni ne gémit pour son immédiate délivrance. Il trouve juste ce châtiment affreux. Il laisse à Dieu le soin entier de sa délivrance, car malgré ses affres et ses douleurs, il met son bonheur à voir encore son saint temple. Et cela, camarades, c’est le vrai repentir, sans cris pour demander un pardon et reconnaissant de la punition. Combien cette attitude de Jonas fut agréable à Dieu, sa délivrance hors de la mer et de la baleine le prouve bien. Camarades, je ne vous propose pas Jonas en exemple pour son péché, mais comme modèle du repentir. Ne péchez pas ; mais si vous le faites, tâchez de le regretter à la manière de Jonas. »\par
Tandis qu’il disait ces mots, la tempête au-dehors hululante, hurlante et cinglante paraissait ajouter un poids nouveau à ces mots et le prédicateur, en décrivant l’ouragan dans lequel Jonas s’était trouvé pris, semblait lui-même secoué par l’orage. Une lame de fond soulevait sa large poitrine, les mouvements de ses bras imitaient la guerre que se livraient les quatre éléments, le tonnerre naissait sous son sourcil brun et l’éclair en son œil, ses auditeurs, dans la simplicité de leur âme, le regardaient avec une crainte soudaine et inaccoutumée.\par
Puis l’accalmie s’étendit jusqu’à lui et il tourna une fois de plus, en silence, les pages du Livre et, enfin, se tenant immobile, les yeux clos, il rentra en lui-même et en Dieu.\par
Mais il se pencha à nouveau vers les fidèles et, baissant très bas la tête avec la plus profonde et la plus virile humilité, il ajouta :\par
« Camarades, Dieu n’a posé qu’une seule main sur vous, il pèse sur moi de ses deux mains. Je vous ai lu, avec la pauvre lumière qui est mienne, la leçon que Jonas enseigne à tous les pécheurs, à vous, à moi plus encore, car je suis un plus grand pécheur que vous. Et maintenant je descendrai avec joie du grand mât pour venir m’asseoir à votre place sur les écoutilles, tandis que quelqu’un me lirait cette autre et plus terrible leçon que Jonas m’apprend à moi, en tant que pilote du Dieu vivant. Combien l’Oint du Seigneur étant son pilote-prophète, le porteparole de la vérité, lorsqu’il reçut de Dieu l’ordre d’aller faire entendre ces vérités importunes à la perverse Ninive, combien Jonas, épouvanté par l’hostilité qu’il soulèverait, mit d’ardeur à refuser cette mission en tentant d’échapper à son devoir et à son Dieu en embarquant sur un navire à Joppé. Mais Dieu est partout ; et Jonas n’arriva jamais à Tarsis. Comme nous l’avons vu, Dieu le rattrapa par l’entremise de la baleine, et l’engloutit dans l’abîme du châtiment, l’entraîna rapidement « dans le cœur de la mer » où les remous des courants l’aspirèrent à une profondeur de dix mille brasses et « les roseaux entourèrent sa tête » et toutes les vagues et les flots du malheur passèrent sur lui. Mais alors même qu’il se trouvait où ne saurait atteindre aucune sonde, dans « le sein du séjour des morts », alors même que la baleine était descendue jusqu’aux racines des montagnes, Dieu entendit les cris de son prophète englouti et repentant. Alors l’Éternel parla au poisson et, du fond des ténèbres glacées, la baleine remonta vers le chaud et bon soleil, vers toutes les délices de l’air et de la terre ; et elle « vomit Jonas sur la terre », alors la parole de l’Éternel fut adressée à Jonas une seconde fois, et Jonas, vaincu et meurtri, ses oreilles, comme deux coquillages, répétant à l’infini l’écho de l’Océan, Jonas se soumit à la volonté du Tout-Puissant. Et qu’était-elle, camarades ? Prêcher la Vérité à la face du Mensonge ! Oui, c’était bien cela !\par
« Et ceci, camarades, ceci est cette autre leçon : malheur au pilote du Dieu vivant qui se dérobe. Malheur à celui qui, séduit par ce monde, se soustrait au devoir de répandre l’Évangile ! Malheur à celui qui cherche à verser de l’huile sur les eaux que Dieu a soulevées en tempête ! Malheur à celui qui cherche à plaire plutôt qu’à semer la crainte ! Malheur à celui qui préfère le renom à la bonté ! Malheur à celui qui, en ce monde, ne va pas au-devant des affronts ! Malheur à celui qui ne reste pas fidèle à la vérité lorsqu’un mensonge peut le sauver ! Oui, malheur à celui qui, avec le grand Pilote Paul, lorsqu’il prêche aux autres, ne se reconnaît pas pour le plus grand des pécheurs ! »\par
Il parut un instant abattu et absent, puis, relevant la tête, il regarda à nouveau l’assemblée, une joie profonde illuminait son regard tandis qu’il s’écriait avec une ferveur céleste : « Mais, ô camarades ! à tribord de toute douleur, la joie vous attend, elle s’élèvera d’autant plus haut que l’abîme de la douleur aura été plus profond. La pomme du grand mât n’est-elle pas d’autant plus haute que la contre-quille est plus profonde ? La joie est le partage – une joie culminante et une joie intérieure – de celui qui, contre les dieux orgueilleux et les commodores de cette terre, demeure inexorablement fidèle à lui-même. La joie est à celui dont les bras restent fermes à le soutenir quand le navire de ce monde trompeur a sombré sous lui. La joie est à celui qui, sans merci devant la vérité, tue, brûle et détruit tout péché même s’il se cache dans les toges des juges et des sénateurs. La joie de la flèche de mât de perroquet est à celui qui ne reconnaît d’autre loi que celle du Seigneur, d’autre maître que son Dieu et n’a d’autre patrie que le ciel. La joie est à celui que toutes les vagues et les lames de cette mer tumultueuse de la foule ne peuvent arracher à la quille infaillible des siècles. Éternelles seront la joie et les délices de celui qui, proche de son ultime repos, peut dire avec son dernier souffle : « Ô Père – Toi dont je connais avant tout la colère – mortel ou immortel, me voici sur le point de mourir. J’ai lutté pour être tien, plus que pour appartenir à ce monde ou m’appartenir à moi-même. Et pourtant ce n’est rien… Je t’abandonne l’Éternité, car l’homme, qu’est-il pour prétendre à la durée de son Dieu ? »\par
Il se tut, fit le geste lent d’une bénédiction, enfouit son visage entre ses mains, et resta là, agenouillé, jusqu’à ce que, tout le monde étant parti, il demeura seul.
\chapterclose


\chapteropen
\chapter[{CHAPITRE X. Un ami de cœur}]{CHAPITRE X \\
Un ami de cœur}\renewcommand{\leftmark}{CHAPITRE X \\
Un ami de cœur}


\chaptercont
\noindent De retour à l’Auberge du Souffleur, revenant de la chapelle, j’y trouvai Queequeg tout seul ; il était parti avant la bénédiction. Assis sur un banc, devant le feu, les pieds dans le foyer, il tenait tout contre son visage sa petite idole noire ; la contemplant ardemment, il lui amenuisait doucement le nez avec son couteau tout en fredonnant pour lui-même quelque païenne mélopée.\par
Se trouvant interrompu, il abandonna sa figurine et bientôt se dirigea vers la table, y prit un grand livre qui s’y trouvait, le posa sur ses genoux et se mit à en compter les pages avec une méthodique régularité ; à chaque cinquantième page – ce fut du moins ce que j’en conclus – il s’arrêtait un instant, regardait dans le vague, et émettait un long sifflement perlé qui témoignait de son étonnement ; il semblait recommencer à un comme s’il ne savait pas compter au-delà de cinquante et ce n’était que cette accumulation de cinquantaines qui excitait son admiration quant au nombre de pages.\par
Je le regardais avec beaucoup d’intérêt. Tout sauvage qu’il était, et si hideusement défiguré – à mon goût du moins – l’expression de son visage était loin d’être déplaisante. Une âme ne peut se dissimuler. Sous ses tatouages de l’autre monde, je croyais découvrir un cœur simple et honnête, et dans ses larges yeux profonds, ardents, sombres et hardis, un esprit propre à défier mille démons. D’autre part, le païen avait une attitude altière que sa singularité n’entamait guère. Il avait l’air d’un homme qui ne s’était jamais montré obséquieux et qui n’avait rien dû à personne. Que son front parût plus franc, plus lumineux et plus grand du fait qu’il avait la tête rasée, je ne me risquerai pas à l’affirmer, mais il est certain que le moule de son crâne était phrénologiquement des meilleurs. Si ridicule que cela paraisse, il me rappelait le général Washington, tel que le représentent les bustes populaires. Au-dessus d’arcades sourcilières saillantes, pareilles à deux promontoires boisés dru, il avait cette même dépression longue et régulière… Queequeg était un sosie de George Washington en plus cannibale.\par
Tandis que feignant à demi de regarder la tempête par la croisée, je me livrais à cet examen minutieux de sa personne, il ne prit nullement garde à ma présence, ne broncha pas et ne leva même pas les yeux, paraissant tout entier absorbé à compter les pages du livre merveilleux. En pensant à la manière fraternelle dont nous avions dormi ensemble la nuit précédente, en me remémorant plus particulièrement ce bras qui me tenait tendrement au matin, je trouvai son indifférence très étrange. Mais les sauvages sont des êtres singuliers, parfois on ne sait pas comment les prendre. De prime abord ils sont impressionnants, leur sereine maîtrise d’eux-mêmes et leur simplicité paraissent une sagesse socratique. J’avais également remarqué que Queequeg frayait à peine, pour ne pas dire pas du tout, avec les marins qui fréquentaient l’auberge. Il ne faisait point d’avance et ne paraissait pas désireux d’étendre le cercle de ses relations. J’en étais surpris et frappé, et en y réfléchissant j’y trouvais un côté sublime. Voilà un homme qui, étant à quelque vingt mille milles de chez lui, en passant par le cap Horn – la seule route possible – était, dès lors, jeté parmi des êtres qui lui étaient aussi étrangers que s’il eût été transplanté sur la planète Jupiter. Pourtant il paraissait parfaitement à l’aise, conservant un calme absolu, se suffisant à lui-même, d’humeur égale. Tout cela exprimait les nuances d’une admirable philosophie, encore qu’il ignorât sans aucun doute jusqu’à l’existence d’une telle chose. Mais, pour être de vrais philosophes, nous ne devrions peut-être pas avoir une conscience si aiguë de vivre et de combattre. Quand j’entends que tel ou tel se donne pour philosophe, j’en conclus que, pareil à la vieille femme dyspeptique, il doit « avoir fendu son digesteur ».\par
Et j’étais assis là, dans cette pièce maintenant déserte, le feu, son ardeur épuisée, en était à cet instant de douceur où, après avoir répandu avec intensité sa chaleur, il ne brillait plus que pour le regard ; les ombres et les spectres de la nuit, rassemblés aux battants des fenêtres, épiaient cette paire silencieuse et solitaire que nous formions ; au-dehors la tempête enflait avec solennité ses mugissements, je commençai à me sensibiliser à d’étranges impressions. Un attendrissement m’envahissait. Mon cœur blessé, ma main crispée ne se retournaient plus contre un monde cruel que ce sauvage apaisant rachetait pour moi. Il était assis là, son indifférence même prouvait une nature qui ne cachait pas les hypocrisies civilisées ni les mielleuses fourberies. Sauvage il était. Un spectacle entre tous les spectacles. Pourtant je commençai à me sentir mystérieusement attiré vers lui, aimanté par ces mêmes choses qui eussent rebuté la plupart. Je vais goûter d’un ami païen, pensais-je, puisque la bienveillance chrétienne n’est que vide civilité. Je rapprochai de lui mon banc et fis des signes et des gestes d’amitié tout en essayant d’engager la conversation. Il ne remarqua d’abord que peu ces avances, mais comme je lui rappelais son hospitalité de la nuit, il se décida à me demander si nous allions partager la même chambre cette nuit encore. Je lui dis que oui, ce qui parut, me sembla-t-il, lui être agréable, voire même peut-être l’honorer.\par
Alors nous nous mîmes à feuilleter le livre ensemble et j’entrepris de lui expliquer le but des lettres et le sens des quelques gravures qu’il contenait. Son intérêt s’éveilla aussitôt et dès lors nous passâmes à un baragouinage, fait d’efforts réciproques, ayant trait aux diverses curiosités qu’il y avait à voir dans cette ville fameuse. Je proposai bientôt de fumer la pipe de l’amitié. Sortant sa blague et son tomahawk, il m’offrit tranquillement d’en tirer quelques bouffées, de sorte que nous restâmes assis là, fumant chacun à notre tour sa pipe barbare, nous la passant l’un à l’autre à intervalles réguliers.\par
Si le cœur du païen abritait encore à mon égard quelque glaçon d’indifférence, il eut tôt fait de fondre à la réconfortante chaleur de cette pipe commune et nous devînmes compères. Sa sympathie semblait venir à moi tout aussi naturellement et spontanément que la mienne allait à lui. Lorsque nous eûmes fini de fumer, il appuya son front contre le mien, m’enserra la taille et me dit que dès lors nous étions mariés, ce qui signifiait, dans le langage de son pays, que nous étions des amis de cœur et que, si besoin en était, il donnerait joyeusement sa vie pour moi. Chez un compatriote, cette flamme soudaine d’affection aurait paru par trop prématurée et tout à fait suspecte mais ces règles générales ne pouvaient en aucun cas s’appliquer à ce sauvage simple.\par
Après le souper, une nouvelle conversation et une nouvelle pipe, nous montâmes tous deux dans notre chambre. Il me fit don de sa tête réduite, sortit sa gigantesque blague et, fouillant sous le tabac, il en tira quelque trente dollars d’argent. Les étalant alors sur la table, il en fit deux parts égales et poussant l’une vers moi il me dit qu’ils m’appartenaient. J’allais protester mais il me réduisit au silence en les versant dans les poches de mon pantalon. Je les y laissai. Il commença alors ses prières du soir, sortit son idole et libéra la cheminée de son écran de papier. Certains signes et symptômes me donnèrent à croire qu’il désirait que je me joigne à lui. Sachant pertinemment ce qui allait suivre, je réfléchis un instant pour savoir si j’accepterais ou non si je m’y trouvais invité.\par
Je suis un bon chrétien, né et élevé dans le sein de l’infaillible église presbytérienne. Comment, dès lors, pouvais-je partager les dévotions que ce sauvage idolâtre rendait à son morceau de bois ? Mais qu’est-ce que rendre un culte ? me demandai-je. Vas-tu te figurer, Ismaël, que le Dieu magnanime du ciel et de la terre – et de tous les hommes, païens y compris – puisse éprouver l’ombre d’une jalousie envers un insignifiant morceau de bois noir ? Impensable. Mais qu’est-ce qu’adorer Dieu sinon faire sa volonté ? C’est là l’hommage à lui rendre. Et quelle est la volonté de Dieu ? sinon faire à mon prochain ce que je voudrais qu’il me fît. Telle est Sa volonté. Queequeg est mon prochain. Et que souhaiterais-je voir Queequeg faire pour moi ? Eh bien ! s’unir à moi dans ma manière presbytérienne et particulière de rendre grâces. Donc je dois me joindre à lui dans son culte personnel et par conséquent me muer en idolâtre. De sorte que j’allumai les copeaux et l’aidai à redresser l’innocente petite figurine. Avec Queequeg, je lui offris du biscuit brûlé, fis des salamalecs par deux ou trois fois et lui embrassai le nez. Ces rites terminés, nous nous déshabillâmes et nous nous couchâmes, en paix avec notre conscience et avec le monde entier. Mais nous ne nous endormîmes pas sans avoir bavardé un moment.\par
Il n’y a pas de lieu plus favorable qu’un lit aux révélations confidentielles entre amis, je ne sais pourquoi. On dit que mari et femme s’y dévoilent l’un à l’autre le tréfonds de leur âme et il est des vieux couples qui, étendus, y parlent presque jusqu’au matin du bon vieux temps. Ainsi dans la lune de miel de nos cœurs, étais-je allongé auprès de Queequeg. Couple envahi de bien-être et de tendresse.
\chapterclose


\chapteropen
\chapter[{CHAPITRE XI. En robes de chambre}]{CHAPITRE XI \\
En robes de chambre}\renewcommand{\leftmark}{CHAPITRE XI \\
En robes de chambre}


\chaptercont
\noindent Nous étions ainsi restés étendus, tantôt bavardant, tantôt nous endormant brièvement ; de temps à autre Queequeg jetait affectueusement ses brunes jambes tatouées par-dessus les miennes puis les retirait, tant nous nous sentions libres, fraternels et sans contrainte ; quand enfin nos causeries eurent chassé la plus légère somnolence, nous nous sentîmes d’humeur à nous relever bien que l’aube fût encore loin de poindre.\par
Oui, nous étions tout à fait réveillés, à tel point que notre position allongée commença à nous peser et que, progressivement, nous nous retrouvâmes assis, les couvertures bien bordées, appuyés à la tête du lit, nos quatre genoux serrés et levés contre nos poitrines, nos deux nez penchés sur nos rotules comme si elles eussent été des bassinoires. Notre confort nous paraissait d’autant plus agréable qu’il faisait froid dehors et même hors de nos couvertures dans cette chambre sans feu. Je dis d’autant plus encore parce que le fait d’avoir une petite partie du corps exposée au froid peut seul vous faire savourer pleinement votre propre chaleur animale, car tout plaisir, en ce monde, ne vaut que par contraste. Rien n’existe en soi. Si vous vous flattez d’être envahi de bien-être de la tête aux pieds et qu’il en ait été ainsi pendant fort longtemps, alors on ne peut pas dire que vous sachiez encore ce qu’est le bien-être. Mais si, à l’instar de Queequeg et moi au lit, vous avez eu le bout du nez, le front et les oreilles légèrement gelés, alors en vérité vous serez infiniment persuadés d’avoir délicieusement chaud. C’est pour cette raison qu’on ne devrait jamais faire du feu dans \hspace{1em} une chambre à coucher, un de ces luxueux inconforts des riches. Car la vraie volupté est de n’avoir entre la chaleur et le bien-être de votre corps et le froid extérieur qu’une simple couverture. Vous êtes alors l’unique étincelle vivante au cœur d’un cristal arctique.\par
Nous étions assis, ainsi recroquevillés, depuis un certain temps quand soudain je décidai d’ouvrir les yeux, car j’ai l’habitude de toujours les garder fermés quand je suis au lit entre les draps, qu’il fasse jour ou qu’il fasse nuit, que je dorme ou que je sois éveillé, cela afin de savourer pleinement le confort d’un lit. Car un homme ne peut prendre une conscience juste de lui-même que les yeux fermés, comme si les ténèbres étaient vraiment l’élément naturel de notre essence, cependant que la lumière est plus favorable à l’argile dont nous sommes pétris. Tandis que j’ouvrais alors les yeux, sortant de cette obscurité agréable et voulue pour plonger dans l’ombre extérieure, imposée et épaisse, des minuits non illuminés, un revirement désagréable s’opéra en moi. Aussi ne fis-je aucune objection à Queequeg qui suggérait de rallumer la lampe, puisque nous étions parfaitement réveillés, et qu’il souhaitait d’autre part ardemment tirer quelques paisibles bouffées de son tomahawk. Il faut le dire, si la nuit précédente j’avais éprouvé une répugnance profonde à ce qu’il fumât au lit – et c’est à cela qu’on s’aperçoit à quel point nos plus rigides préjugés s’assouplissent lorsque l’amour vient à les courber – cette nuit, je n’aimais rien tant que d’avoir Queequeg fumant à mes côtés et même au lit tant il semblait alors pénétré d’une joie sereine et familière. Je ne me sentais plus soucieux à l’excès de la police d’assurance du patron ! Je n’étais sensible qu’à l’intense et réconfortant partage d’une pipe et d’une couverture avec un véritable ami. Nos vareuses poilues sur les épaules, nous nous passions le tomahawk, jusqu’à ce que la fumée suspendît au-dessus de nous un baldaquin bleu, illuminé par la lampe que nous venions de rallumer.\par
Que ce rideau mouvant emportât le sauvage vers de très lointains décors, je ne sais, toujours est-il qu’il se mit à parler de son île natale et, ardent à connaître son histoire, je l’encourageais à me la conter. Il fut heureux d’accéder à mon désir. À ce moment-là j’eus grand-peine à comprendre simplement quelques mots, mais des conversations ultérieures me rendirent plus familière sa phraséologie décousue et me permettent de retracer à présent le récit tel qu’on peut le reconstituer d’après la simple ébauche que j’en donne.
\chapterclose


\chapteropen
\chapter[{CHAPITRE XII. Récit d’une vie}]{CHAPITRE XII \\
Récit d’une vie}\renewcommand{\leftmark}{CHAPITRE XII \\
Récit d’une vie}


\chaptercont
\noindent Queequeg était natif de Rokovoko, une île très lointaine dans l’ouest et dans le sud. Elle ne figure sur aucune carte, c’est le propre des endroits vrais.\par
Lorsqu’un sauvage frais éclos court librement dans ses forêts natales, vêtu d’un pagne de feuilles, suivi par de grignotantes chèvres, il n’est rien de plus qu’un adolescent ; toutefois, dans l’âme ambitieuse de Queequeg couvait alors un fervent désir de voir de la chrétienté autre chose qu’un occasionnel échantillon fourni par les baleiniers. Son père était un grand chef, un roi ; son oncle un grand prêtre, et il se vantait d’avoir, du côté maternel, des tantes qui étaient les épouses d’invincibles guerriers. Il avait du bon sang dans les veines, du sang bleu, je le crains, tristement corrompu par le penchant cannibalesque dans lequel son ignorante jeunesse avait été entretenue.\par
Un navire, en provenance de Sag Harbor, relâcha dans la baie paternelle, et Queequeg demanda à s’y embarquer pour les terres chrétiennes, mais, l’équipage du navire étant au complet, sa requête fut repoussée et toute l’influence de son royal père resta inefficace. Mais Queequeg s’était fait un serment. Seul dans son canoë, il pagaya jusqu’à un détroit qu’il semblait impossible au navire d’éviter après avoir quitté l’île. Il était bordé d’un côté par un récif de corail, de l’autre par une bande de terre basse couverte de palétuviers émergeant de l’eau. Dissimulant son canoë parmi ces buissons, la proue pointant vers la mer, il s’assit à l’arrière, tenant bas sa pagaie, et lorsque le navire passa à portée il jaillit comme une flèche, gagna ses flancs, d’un coup de pied fit chavirer et sombrer son embarcation, grimpa aux chaînes et se jetant de tout son long sur le pont, il s’agrippa à une boucle d’amarrage et jura de ne pas lâcher prise dût-on le hacher menu.\par
En vain le capitaine le menaça-t-il de le jeter par-dessus bord, en vain suspendit-on un sabre d’abordage au-dessus de ses poignets nus ; Queequeg était fils de roi et Queequeg était inébranlable. Frappé par cette intrépidité désespérée et par ce désir farouche de voir la chrétienté, le capitaine s’adoucit enfin et lui dit de se considérer comme chez lui à bord. Toutefois ce superbe jeune sauvage, ce prince de Galles des mers, ne vit jamais la cabine du capitaine. Il fut inscrit comme matelot et on fit de lui un baleinier. Mais tel le tsar Pierre qui était satisfait de travailler sur n’importe quel chantier naval des villes étrangères, Queequeg ne reculait devant aucun travail prétendu ignoble, s’il pouvait acquérir ainsi l’heureux pouvoir de rapporter quelque lumière à ses ignorants compatriotes. Car, me dit-il, il était poussé par un désir profond d’apprendre des chrétiens les arts qui auraient ajouté au bonheur des siens et, qui plus est, les auraient rendus meilleurs. Mais hélas ! les habitudes des baleiniers eurent tôt fait de le convaincre que les chrétiens pouvaient être à la fois malheureux et méchants ; infiniment plus que tous les païens de son père. Arrivé enfin dans le vieux Sag Harbor, il vit ce que les marins y faisaient puis, allant à Nantucket, il vit à quoi ils dépensaient leurs paies là aussi, et le pauvre Queequeg considéra la cause comme perdue. Il pensa : c’est un monde pourri sous tous les méridiens et je mourrai païen.\par
Ainsi, bien qu’il restât en son cœur un idolâtre convaincu il vivait cependant parmi ces chrétiens, portait les mêmes vêtements qu’eux et s’efforçait de parler leur charabia. C’est pourquoi il avait d’étranges manières bien qu’il eût quitté son pays depuis un certain temps déjà.\par
Je lui demandai allusivement s’il ne se proposait pas d’y retourner et de s’y faire couronner puisque selon les dernières nouvelles qu’il avait eues son père était bien vieux et affaibli, ce qui laissait à croire qu’il devait être mort à présent. Il me répondit que non, pas encore ; et il ajouta qu’il craignait que la chrétienté, ou plutôt les chrétiens, l’aient rendu indigne d’accéder à ce trône pur et sans tache où trente rois païens l’avaient précédé. Mais il s’en retournerait une fois ou l’autre, ajouta-t-il, dès qu’il se sentirait comme baptisé à nouveau. Pour le moment, il se proposait de naviguer et de jeter sa gourme dans les quatre océans. On avait fait de lui un harponneur, et son fer à présent lui tenait lieu de sceptre.\par
Je lui demandai quels seraient éventuellement ses projets immédiats de déplacement. Il me répondit : retourner à nouveau à la mer, sa vocation première. Sur ce, je lui déclarai que moi-même je souhaitais m’embarquer pour la pêche à la baleine, l’informai de mon intention de faire voile depuis Nantucket car c’était le port offrant le plus de ressources pour le départ d’un baleinier aventureux. Il résolut aussitôt de m’accompagner sur cette île, d’embarquer sur le même navire, de prendre les mêmes heures de quart, la même pirogue, de faire gamelle avec moi, bref de partager en tout mon sort et, mes deux mains dans les siennes, de plonger avec moi courageusement dans les hasards de ce monde et de l’autre. À tout cela je consentis avec joie, car, indépendamment de l’affection que j’éprouvais désormais pour Queequeg, il était un harponneur chevronné et, en tant que tel, ne pouvait manquer d’être du plus grand secours à quelqu’un qui, comme moi, ignorait tout des mystères de la pêche à la baleine, bien que la mer me fût une vieille connaissance à la façon dont peut la connaître un marin marchand.\par
Son histoire prit fin avec une mourante dernière bouffée de pipe, Queequeg me donna l’accolade, pressa son front contre \hspace{1em}le mien et souffla la lumière, nous nous retournâmes chacun vers les bords du lit et nous endormîmes aussitôt.
\chapterclose


\chapteropen
\chapter[{CHAPITRE XIII. La brouette}]{CHAPITRE XIII \\
La brouette}\renewcommand{\leftmark}{CHAPITRE XIII \\
La brouette}


\chaptercont
\noindent Le lendemain matin lundi, après avoir déposé la tête réduite chez un coiffeur en guise de tête à perruque, je réglai ma note et celle de mon camarade, avec l’argent de ce dernier toutefois. Le ricanant patron, ainsi que ses hôtes, paraissaient au comble de l’amusement devant la soudaine amitié qui était née entre Queequeg et moi, d’autant plus que les histoires à dormir debout que Peter Coffin m’avait contées à son sujet m’avaient fortement effrayé à l’égard de celui-là même dont je faisais à présent mon compagnon.\par
Nous empruntâmes une brouette dans laquelle nous mîmes nos affaires, dont mon pauvre sac de voyage, le sac de toile de Queequeg et son hamac, et nous voilà en route vers le {\itshape Varech}, la petite goélette faisant le service de Nantucket et qui se trouvait à quai. Tandis que nous nous y rendions, les gens se retournaient pour nous regarder, non point tant sur \hspace{1em}Queequeg\par
– car ils avaient l’habitude de voir des cannibales de son espèce dans leurs rues – mais parce que nous étions pareillement en confiance l’un envers l’autre. Mais nous n’y prêtions pas attention, roulant notre brouette à tour de rôle et Queequeg s’arrêtant parfois pour rajuster le fourreau sur les barbes de son harpon. Je lui demandai pourquoi il prenait à terre un objet aussi gênant et si les baleiniers ne fournissaient pas les harpons. Il répondit que ma question était des plus raisonnables mais qu’il avait une prédilection pour son propre harpon parce qu’il était d’excellente qualité, qu’il avait fait ses preuves en maints combats mortels et qu’il avait pénétré profond dans le cœur des baleines. En somme, comme nombre de moissonneurs et de faucheurs, bien qu’ils ne soient nullement tenus de le faire vont aux champs de leurs fermiers armés de leurs propres faux, Queequeg, pour des raisons de lui seul connues, préférait son propre harpon.\par
Prenant son tour de brouette, il me raconta l’histoire amusante qui lui était arrivée lorsqu’il en vit pour la toute première fois. Cela se passait à Sag Harbor. Les propriétaires de son bateau lui en avaient, semble-t-il, prêté une pour transporter son coffre pesant jusqu’à sa pension. Afin de ne pas paraître ignorer à quoi elle servait, ce qui était la vérité, et comment la manipuler, Queequeg posa son coffre dessus, l’attacha solidement, et quitta les quais la brouette sur le dos. « Comment, Queequeg, dis-je, on pourrait croire que vous auriez eu un peu plus de bon sens. Les gens n’ont-ils pas ri ? »\par
Cette question l’invita à me conter une autre histoire. Les habitants de son île de Rokovoko, lors des fêtes de mariage, versent le lait parfumé des jeunes noix de coco dans une vaste calebasse peinte semblable à un bol à punch, et celle-ci est toujours l’ornement central de la natte brodée sur laquelle le repas est servi. Or un grand navire marchand relâcha une fois à Rokovoko, et son commandant – d’après ce que je compris c’était un gentilhomme très digne et très protocolaire, du moins pour un capitaine marin – fut invité aux fêtes données en l’honneur du mariage d’une sœur de Queequeg, une jeune et jolie princesse ayant de peu dépassé ses dix ans d’âge. Eh bien ! lorsque tous les hôtes furent réunis dans la hutte de bambous de la mariée, ce capitaine auquel était assignée la place d’honneur entra et s’installa devant le bol à punch entre le grand prêtre et Sa Majesté le Roi, père de Queequeg. Lorsque fut dite la prière avant le repas – car ces gens rendent grâces tout comme nous – bien que Queequeg m’ait dit que, contrairement à nous qui tenons alors le nez dans nos écuelles, eux, à la façon des canards, lèvent les yeux vers Celui auquel on est redevable de toutes les fêtes – je dis donc que, les prières terminées, le grand prêtre ouvrit le banquet par la cérémonie immémoriale de l’île, c’est-à-dire qu’il trempa ses doigts consacrés et consacrants dans le bol avant de faire circuler ce breuvage saint. Se trouvant placé à côté du grand prêtre et ayant suivi son geste, pensant qu’il avait entière préséance – étant capitaine de navire – sur un simple roi insulaire, d’autant plus qu’il était reçu par lui – le capitaine se lava froidement les mains dans le bol à punch, le prenant, j’imagine, pour un gigantesque rince-doigts. « À présent, dit Queequeg, que pensez-vous ? À votre idée ? Nos gens ont-ils ri ? »\par
Enfin, trajet payé et bagages à l’abri, nous fûmes à bord de la goélette. Hissant sa voile elle descendit la rivière Acushnet. Sur une rive, New Bedford dressait les étages de ses rues dont les arbres givrés étincelaient dans l’air clair et froid. D’énormes collines, des montagnes de futaille s’empilaient sur ses quais et, côte à côte, les navires baleiniers, ces errants de toute la terre, dormaient dans le silence et la paix enfin retrouvée. Certains pourtant s’éveillaient déjà pour de nouveaux départs aux bruits des charpentiers et des tonneliers de bord ; et les feux et les forges, fondant le goudron, ronflaient ; à peine un long et périlleux voyage a-t-il pris fin qu’un second se prépare ; ce second se termine-t-il que s’apprête le troisième, et ainsi de suite pour toujours et à jamais. Ainsi en va-t-il, infiniment, intolérablement, de tout effort humain.\par
À mesure qu’on gagnait le large, le vent vivifiant fraîchissait ; le petit {\itshape Varech} secouait l’écume vive de son étrave à la façon dont s’ébroue une jeune pouliche. Ah ! comme j’aspirais cette brise intraitable ! – comme je refusais de tout mon être ces routes de la terre, cette grande route commune marquée tout au long des empreintes laissées par des talons et des sabots d’esclaves, comme je me tournais vers la mer magnanime qui ne permet à rien de demeurer inscrit !\par
Queequeg paraissait tituber avec moi et boire à cette même fontaine écumante, ses narines sombres dilatées, ses lèvres retroussées sur des dents aiguës et polies. Loin, loin, nous volions… et, comme nous avions pris le large, le {\itshape Varech} rendit hommage au vent, plongeant de l’étrave comme l’esclave devant le sultan. Penchés sur le côté, sur le côté nous filions ; chaque fil de caret tintait comme un métal ; les deux grands mâts s’arquaient comme des bambous dans la tornade terrestre. Nous étions si envoûtés par tant de tournoiements, debout à ce beaupré plongeant, que de longtemps nous ne prîmes pas garde aux coups d’œil railleurs des passagers, une réunion de terriens semblait-il, qui s’étonnaient de voir deux hommes pareillement en accord ; comme si un blanc était supérieur à un nègre passé à la chaux. Mais les rustres et les dadais qui étaient là étaient des fruits verts, d’une verdeur si intense qu’elle semblait tirée du cœur même de toute verdure. Queequeg surprit un de ces blancs-becs le singeant dans son dos. Je crus arrivée la dernière heure du péquenot. Lâchant son harpon, le sauvage musclé le prit dans ses bras et avec une adresse et une force miraculeuses il le projeta haut dans les airs et à son solstice il lui donna sur le cul une tape légère, le gars retomba sur ses pieds les poumons sur le point d’éclater, cependant que Queequeg, lui tournant le dos, alluma son calumet-tomahawk et me le passa pour que j’en tire une bouffée.\par
– Capiting ! Capiting ! hurla le rustre en courant vers ledit officier, Capiting, c’est le diable lui-même.\par
Celui-ci, un étique pilier de la mer, marcha sur Queequeg en lui criant :\par
– Holà, vous, monsieur, par tous les tonnerres à quoi pensez-vous ? Ignorez-vous que vous auriez pu tuer ce gaillard ?\par
– Y dit quoi ? questionna Queequeg en se tournant vers moi avec douceur.\par
– Il dit, répondis-je, que vous avez failli tuer cet homme-là, et je désignai du doigt le blanc-bec encore tout tremblant.\par
– Toué ! s’étonna Queequeg son visage tatoué se convulsant dans une expression intraduisible de dédain. Ah ! lui être petit poisson. Queequeg pas touer si pétit poisson ; Queequeg touer grande baleine !\par
– Attendez un peu ! rugit le capitaine, je vous touerai, vous, espèce de cannibale si vous jouez encore un de vos tours à bord ; alors prenez garde !\par
Mais il se trouva que ce fut au capitaine à prendre garde de toute urgence. La formidable poussée imposée à la grand-voile avait fait sauter l’écoute du vent et la bôme énorme volant de droite et de gauche opérait un balayage complet à l’arrière du pont. Le pauvre diable que Queequeg avait pareillement rudoyé fut expédié par-dessus bord, tout le monde était gagné par la panique et ce paraissait être une folie de tenter de se saisir de la bôme pour l’immobiliser. Elle allait de gauche à droite et de droite à gauche, comme le balancier d’une pendule et semblait à chaque instant sur le point de voler en éclats. On ne faisait donc rien et rien d’utile ne semblait pouvoir être fait ; ceux qui se trouvaient sur le pont s’étaient précipités à l’avant et fixaient des yeux la bôme comme s’ils étaient devant la mâchoire inférieure d’un cachalot furieux. Dans la confusion générale, Queequeg se jeta prestement à genoux et rampant sous le va-et-vient de la bôme, attrapa un filin au vol, en fixa une extrémité à la lisse tandis qu’il jetait l’autre bout, en lasso, autour de la bôme comme elle passait par-dessus sa tête ; à la saccade suivante, l’espar se trouva ainsi emprisonné, et tout rentra dans l’ordre. La goélette était montée au vent et cependant que l’équipage détachait le canot de l’arrière, Queequeg, nu jusqu’à la ceinture, bandé comme un arc, sauta à l’eau. Pendant trois minutes ou davantage il nagea à la façon d’un chien lançant tout droits devant lui ses longs bras, ses épaules musclées émergeant tour à tour de l’écume glacée. Je regardai ce superbe et glorieux personnage, mais ne voyais personne qui dût être sauvé. Le blancbec avait coulé. Se dressant sur l’eau, Queequeg jeta un regard circulaire autour de lui, et paraissant avoir vu de quoi il retournait, plongea et disparut. Quelques instants plus tard il réapparaissait, nageant d’un bras, tirant de l’autre une forme sans vie. Le canot les eut bientôt recueillis. Le pauvre blanc-bec fut ranimé. L’équipage déclara à l’unanimité que Queequeg était un noble cœur, le capitaine implora son pardon. Dès ce moment, je m’attachai à Queequeg comme une bernacle, oui jusqu’à ce que ce pauvre Queequeg eût fait son dernier grand plongeon.\par
Vit-on jamais pareille modestie ? Il n’avait pas l’impression qu’il méritait si peu que ce soit une médaille décernée par les sociétés humanitaires et de bienfaisance. Il demanda seulement de l’eau – de l’eau douce – de quoi se rincer de la saumure ; cela fait, il mit des vêtements secs, alluma sa pipe et appuyé à la rambarde, et regardant avec douceur ceux qui étaient autour de lui, il semblait se dire : « Ce monde est un capital social où le secours mutuel est nécessaire sous tous les méridiens. Nous autres, cannibales, devons aider ces chrétiens. »
\chapterclose


\chapteropen
\chapter[{CHAPITRE XIV. Nantucket}]{CHAPITRE XIV \\
Nantucket}\renewcommand{\leftmark}{CHAPITRE XIV \\
Nantucket}


\chaptercont
\noindent Le reste de la traversée se passa sans incident, de sorte qu’après un excellent voyage nous arrivâmes sains et saufs à Nantucket.\par
Nantucket ! Sortez votre carte et regardez-la. Regardez quel vrai coin du monde elle occupe. Comme elle se tient là, au large, plus solitaire que le phare d’Eddystone. Regardez-la, un simple tertre, un bras de sable tout en plage, sans toile de fond. Il y a là plus de sable que le monde n’en utiliserait, vingt ans durant, en guise de buvard. Les plaisantins vous diront qu’ils doivent y semer la mauvaise herbe qui n’y pousse pas toute seule, qu’ils importent les chardons du Canada, qu’ils doivent faire venir d’outre-mer un fausset pour boucher un trou dans un baril d’huile, et que tout bout de bois est solennellement porté à travers Nantucket, comme à Rome les morceaux de la vraie croix. Pour se procurer un peu d’ombre en été, les habitants plantent des faux agarics dans leurs jardins ; un brin d’herbe est une oasis, trois brins d’herbe découverts au cours d’une journée de promenade sont une prairie. Ils y portent des chaussures pour les sables mouvants assez semblables à celles que mettent les Lapons sur la neige. Ils y sont si bien enfermés, ceinturés, séquestrés de toute manière, entourés, l’Océan en fait si bien une île, que des grappes de coquillages adhèrent parfois jusqu’à leurs chaises et leurs tables comme aux carapaces de tortues de mer. Toutes ces divagations signifient seulement que Nantucket n’est pas l’Illinois !\par
Considérez à présent l’étonnante légende racontant comment les Peaux-Rouges vinrent à s’établir sur l’île. Ainsi va l’histoire : au temps jadis, un aigle s’abattit sur la côte de la Nouvelle-Angleterre et enleva dans ses serres un petit enfant indien. En se lamentant bien haut, les parents virent leur enfant emporté hors de vue au-dessus de la vaste mer. Ils décidèrent de suivre cette direction à sa recherche. Ayant mis à l’eau leurs pirogues, ils accomplirent une traversée périlleuse, découvrirent enfin l’île et y trouvèrent, coffret d’ivoire vide, le squelette du pauvre petit Indien.\par
Quoi d’étonnant dès lors à ce que ces Nantuckais nés sur la grève prennent la mer pour gagner leur vie ! Ils commencèrent par chercher dans le sable des crabes et des vénus ; s’enhardissant, ils pataugèrent avec des filets pour prendre les maquereaux ; leur expérience croissant, ils s’éloignèrent en bateau pour pêcher la morue ; enfin ils lancèrent une flotte de grands navires sur l’Océan afin d’explorer l’univers liquide ; ils le cernèrent d’incessantes circumnavigations, jetèrent un coup d’œil dans le détroit de Behring, et, en toutes saisons et sur toutes les mers, déclarèrent une guerre éternelle à l’être le plus puissant qui ait survécu au déluge, au plus monstrueux, au plus montagneux ! À cet Himalaya, à ce Mastodonte salé, de mauvais augure et dont la force brute rend ses paniques plus redoutables que la téméraire malveillance de son assaut.\par
C’est ainsi que ces Nantuckais nus, ces ermites de la mer, sortant de leur fourmilière océanique, parcoururent le monde liquide et le conquirent comme autant d’Alexandres, se partageant les océans Atlantique, Pacifique et Indien, comme les trois nations flibustières firent de la Pologne. Que l’Amérique ajoute le Mexique au Texas, empile Cuba sur le Canada, que les Anglais essaiment dans toute l’Inde et plantent leur drapeau resplendissant sur le soleil même, il n’en reste pas moins que les deux tiers du monde aqueux sont aux Nantuckais. Car la mer est à eux, elle leur appartient comme un empire à son empereur, les autres marins n’y ont qu’un droit de transit. Les navires marchands ne sont qu’un prolongement des ponts ; les bâtiments de guerre rien de plus que des forteresses flottantes ; les pirates et les corsaires, bien que sillonnant les mers, comme des voleurs de grand chemin parcourent les routes, ne font rien de plus que piller d’autres navires, fragments de terre comme eux, qui ne quêtent pas leur subsistance dans l’abîme sans fond. Seul le Nantuckais réside et pullule sur la mer ; lui seul vogue sur la mer, dans des navires – pour parler en termes bibliques – labourant, ici et là, sa plantation personnelle. C’est là son foyer. Là il mène ses affaires, à l’abri d’un nouveau déluge, quand bien même toute la population de la Chine en serait submergée. Il vit sur la mer comme le coq des landes sur la lande, se cache dans la vague, et l’escalade comme les Alpes le chasseur de chamois. Pendant des années il ne sait plus rien de la terre, et lorsqu’il y revient enfin, elle a pour lui un parfum d’autre monde, plus étrange que celui de la lune n’en aurait pour un terrien. Comme le goéland sans patrie replie ses ailes au coucher du soleil et s’abandonne à la berceuse des flots, ainsi à la tombée du soir le Nantuckais, loin de toute terre, ferle ses voiles et s’étend cependant que sous son oreiller même défilent des troupes de morses et de baleines.
\chapterclose


\chapteropen
\chapter[{CHAPITRE XV. Soupes de poissons}]{CHAPITRE XV \\
Soupes de poissons}\renewcommand{\leftmark}{CHAPITRE XV \\
Soupes de poissons}


\chaptercont
\noindent Il était bien tard dans la soirée quand le petit {\itshape Varech} mouilla confortablement l’ancre et que Queequeg et moi débarquâmes. Aussi nous ne pouvions vaquer à aucune affaire le soir même, du moins à aucune autre que de trouver à souper et à dormir. Le patron de l’Auberge du Souffleur nous avait recommandé d’aller chez son cousin Osée Hussey, propriétaire des Tâte-Pots, l’hôtel le mieux tenu de tout Nantucket, nous affirma-t-il ; de plus, nous avait-il assurés, ce cousin Osée – comme il l’appelait – était célèbre pour ses soupes de poissons. En somme, il déclarait carrément que nous ne pouvions faire mieux que de tâter la chance au Tâte-Pots. Mais le chemin à suivre qu’il nous avait indiqué, nous engageant à suivre, par tribord, un entrepôt jaune jusqu’à ce qu’on aperçoive une église blanche à bâbord, de rester alors à bâbord jusqu’à un carrefour à trois points de tribord, et ceci fait de demander où se trouvait l’auberge au premier homme que nous croiserions. Ces indications retorses nous déconcertèrent d’abord grandement, d’autant plus que Queequeg soutenait que l’entrepôt jaune, notre point de départ, devait se trouver à bâbord, tandis que j’avais compris que Peter Coffin parlait de tribord. Toutefois, à force de louvoyer dans l’obscurité et de frapper aux portes de paisibles habitants pour demander notre chemin, nous arrivâmes enfin à quelque chose ne permettant aucune méprise.\par
Deux énormes chaudières de bois, peintes en noir, étaient suspendues par leurs oreilles d’âne de part et d’autre des élongis d’un vieux mât de hune fiché en terre devant un antique portail, sciés d’un côté, ils donnaient à ce vieux mât l’allure d’un gibet. Peut-être étais-je, à ce moment-là, rendu trop vulnérable à certaines impressions, mais je ne pus m’empêcher de contempler cette potence avec de sombres pressentiments. J’attrapai le torticolis à lever la tête vers ces deux cornes restantes ; oui, deux ; l’une pour Queequeg, l’autre pour moi. C’est un mauvais présage, me dis-je. En arrivant dans le premier port baleinier, je trouvai un aubergiste du nom de Coffin\hyperref[_bookmark35]{\dotuline{2}}, puis le regard, posé sur moi, des plaques funéraires dans la chapelle, ici enfin un gibet et une paire de prodigieuses chaudières noires par-dessus le marché ! Serait-ce une allusion insidieuse à Topheth ?\par
Je fus tiré de mes réflexions par la vue d’une femme couverte de taches de rousseur, assortissant une robe jaune à des cheveux jaunes, debout sous le porche de l’auberge où se balançait une lampe rouge sans éclat évoquant singulièrement un œil injecté de sang. Elle tançait vivement un homme en chemise de laine pourpre.\par
– Filez de par là, disait-elle à l’homme, ou vous allez recevoir une peignée !\par
– Venez, Queequeg, tout va bien, voilà M\textsuperscript{me} Hussey. Il s’avéra que c’était elle en effet. M. Osée Hussey était absent, mais il avait laissé à M\textsuperscript{me} Hussey les pleins pouvoirs sur la conduite de ses affaires. Lorsque nous eûmes exprimé le désir d’un souper et d’un lit, M\textsuperscript{me} Hussey, renvoyant à plus tard ses semonces, nous précéda dans une petite pièce, nous fit asseoir à une table jonchée des reliefs d’un repas qui venait d’être terminé et, se tournant vers nous, questionna :\par
– Clovisses ou morue ?\par
– Comment se présente la morue, madame ? demandai-je avec une politesse extrême.\par
– Clovisses ou morue ? répéta-t-elle.\par
– Des clovisses pour le souper ? une froide clovisse. Est-ce bien là ce que vous voulez dire, madame Hussey ? C’est une coquille plutôt humide et glaciale où se retirer en hiver, ne trouvez-vous pas, madame Hussey ?\par
Mais impatiente de poursuivre ses criailleries attendues sous le porche par l’homme en chemise rouge et n’ayant paru saisir que le mot clovisse, M\textsuperscript{me} Hussey se précipita vers une porte donnant sur la cuisine et brailla : « Deux clovisses, deux ! » et disparut.\par
– Queequeg, dis-je, pensez-vous que nous pouvons souper à deux sur une clovisse ?\par
Pourtant, le parfum appétissant de la chaude vapeur, qui s’échappait de la cuisine, infligeait un démenti à un programme apparemment peu réjouissant. Lorsque la soupe fumante apparut, la clef du mystère nous fut délicieusement donnée. Ah ! mes bons amis, écoutez bien. De petites clovisses juteuses, à peine plus grosses qu’une noisette, mélangées à des biscuits de mer émiettés et à du porc salé finement émincé, composaient cette soupe enrichie de beurre et généreusement assaisonnée de sel et de poivre. Un voyage hivernal nous avait aiguisé l’appétit. Queequeg se trouvait devant son plat favori, la soupe était une réussite parfaite, aussi l’eûmes-nous expédiée promptement. Je m’appuyai au dossier de ma chaise et, songeant à la proposition de M\textsuperscript{me} Hussey : clovisses ou morue ? j’envisageai de tenter une petite expérience. Me dirigeant vers la porte de la cuisine, je prononçai le mot : morue avec grandiloquence et regagnai mon siège. Au bout d’un instant la vapeur appétissante renaissait, apportant un parfum différent, et, en temps voulu, on nous servit une magnifique soupe à la morue.\par
Nous nous y attaquâmes, et tandis que nous plongions nos cuillères dans nos écuelles, je me demandai si cela avait un effet sur la tête ? Une locution populaire ne veut-elle pas qu’on ait des têtes de poissons frits ? Mais, regardez, Queequeg, n’y a-t-il pas une anguille vivante dans votre écuelle ? Où est votre harpon ?\par
Ce Tâte-Pots était plus poissonneux que tous les endroits poissonneux, et il méritait bien son nom ! Dans toutes ses chaudières bouillaient des soupes de poissons, on vous servait de la soupe de poissons au petit déjeuner, de la soupe de poissons au repas de midi, de la soupe de poissons au souper, jusqu’à ce que les arêtes vous sortent de la peau et que vous en veniez à les chercher dans vos vêtements. L’entrée, devant la maison, était pavée de coquilles de clovisses. M\textsuperscript{me} Hussey portait un collier de vertèbres de morue polies et Osée Hussey avait des livres de comptes reliés en peau de chagrin, ancienne et de qualité. Le lait avait un arrière-goût de poisson, lui aussi. La raison m’en échappait tout à fait. Mais un matin, tandis que je me promenais le long de la plage parmi les barques de pêcheurs, je vis la vache bringée d’Osée, pâturant des restes de poissons, avançant sur le sable, ses quatre sabots chaussés d’une tête de morue ; croyez-moi, elle avait l’air d’une traîne-savates.\par
Le souper terminé, nous reçûmes une lampe et des renseignements de M\textsuperscript{me} Hussey sur le plus court chemin menant au lit, mais tandis que Queequeg s’apprêtait à me précéder dans l’escalier, la dame tendit le bras et réclama son harpon. Elle ne tolérait aucun harpon dans les chambres. « Pourquoi ? demandai-je, tout vrai harponneur dort avec son harpon… alors pourquoi non ? » – « Parce que c’est dangereux, dit-elle, depuis que le jeune Stiggs de retour de c’mal’reux voage, après quatre ans et demi d’absence, avec seulement trois barils \hspace{1em} d’haïle, depuis qu’il a été trouvé mort au premier dans la chambre de derrière, le harpon dans le cœur ; depuis, alors, je n’autorise aucun pensionnaire à emporter des armes aussi dangereuses dans leurs chambres la nuit. Auchi, monsieur Queequeg (car elle avait appris son nom) je vais juste prendre ce fer-ci et vous le garder jusqu’à demain matin. Mais à propos de la soupe, clovisses ou morue pour le petit déjeuner, les gars ? »\par
– Les deux, répondis-je, et donnez-nous aussi une paire de harengs fumés pour varier le menu.
\chapterclose


\chapteropen
\chapter[{CHAPITRE XVI. Le navire}]{CHAPITRE XVI \\
Le navire}\renewcommand{\leftmark}{CHAPITRE XVI \\
Le navire}


\chaptercont
\noindent Au lit, nous combinâmes un plan pour le lendemain, mais à ma surprise et pour ma plus grande inquiétude, Queequeg me donna alors à entendre qu’il avait sérieusement consulté Yoyo – ainsi s’appelait son petit dieu noir – que Yoyo lui avait répondu par deux ou trois fois, et fortement insisté de toutes les manières possibles, pour qu’au lieu d’aller ensemble au port et choisir d’un commun accord parmi la flottille des baleiniers le navire sur lequel nous embarquerions, au lieu de cela, dis-je, Yoyo ordonnait instamment que j’assumasse l’entière responsabilité de ce choix, attendu que Yoyo se proposait de nous protéger et qu’à cette fin il avait d’ores et déjà choisi lui-même ce navire que, livré à moi-même, moi Ismaël, je trouverai comme par hasard mais infailliblement, comme si la chance l’avait amené là exprès ; je devais m’engager aussitôt sur ce navire, indépendamment de Queequeg pour le moment.\par
J’ai oublié de dire que, en bien des cas, Queequeg accordait une grande confiance à la valeur des jugements de Yoyo et à l’étonnante sûreté de ses prédictions. Il tenait Yoyo en considérable estime, comme un dieu plutôt bon de nature aux intentions généralement amicales, mais dont les bienveillants desseins ne réussissaient pas toujours.\par
Cette décision de Queequeg, ou plutôt de Yoyo, je ne l’appréciai à aucun degré. Je m’en étais presque absolument remis à la perspicacité de Queequeg pour élire le mieux qualifié pour nous transporter sûrement, nous et notre sort. Mais toutes mes protestations restant sans effet, je fus contraint de céder et, en conséquence, je me préparai à prendre l’affaire en main avec une hâte décidée, une énergie vigoureuse, propre à bâcler promptement cette question futile. Le lendemain matin de bonne heure, je quittai Queequeg enfermé avec Yoyo dans notre petite chambre, car il semblait que ce fût le jour d’une espèce de carême ou de ramadan, ou un jour de jeûne, de mortification et de prières entre Queequeg et Yoyo. Ce qu’il en était précisément, je ne pus jamais le découvrir car bien que je me sois appliqué maintes fois à méditer sa liturgie et ses XXXIX Articles, je ne pus jamais en venir à bout ; dès lors, abandonnant Queequeg jeûnant avec son tomahawk, et Yoyo se chauffant au feu sacrificatoire des copeaux, je me mis en route parmi les bateaux. Après des déambulations interminables, des renseignements demandés à tort et à travers, j’appris que trois navires étaient en partance pour des voyages de trois ans : le {\itshape Diable-et-sa-mère}, la {\itshape Bonne-Bouche} et le {\itshape Péquod}. J’ignore l’origine du nom du {\itshape Diable-et-sa-mère}, celle de {\itshape Bonne Bouche} est évidente, quant à {\itshape Péquod}, vous vous souvenez sans doute que c’était le nom d’une célèbre tribu d’Indiens du Massachussetts aussi éteinte à présent que les anciens Mèdes. Je jetai un regard inquisiteur et fureteur sur l e{\itshape Diable-et-sa-mère} ; de là, je sautai dans la {\itshape Bonne- Bouche} et enfin, montant à bord du {\itshape Péquod}, je l’examinai un moment et décidai que c’était là le navire idéal pour nous.\par
Autant que je sache, vous avez sans doute vu, au cours de votre vie, bien des embarcations pittoresques : des lougres à bouts carrés, des jonques japonaises hautes comme des montagnes, des galiotes-caisses à beurre, je ne sais quoi encore mais, croyez-moi sur parole, vous n’avez jamais vu un vieux bâtiment aussi extraordinaire que cet extraordinaire vieux {\itshape Péquod}. C’était un navire de la vieille école, plutôt petit, qui avait la façon surannée des meubles à pieds de griffon. Longuement amarinée, colorée par tous les temps, des typhons aux calmes plats des quatre océans, sa vieille coque avait pris le teint basané d’un grenadier français qui aurait combattu en Égypte comme en Sibérie. Son étrave vénérable semblait barbue. Ses mâts – taillés quelque part sur la côte japonaise là où la tempête emporta ceux qu’il avait à l’origine – ses mâts avaient la raideur de l’épine dorsale des trois vieux rois de Cologne. Ses ponts antiques étaient usés et ridés comme la dalle vénérée des pèlerins où fut versé le sang de Becket dans la cathédrale de Cantorbéry. À ces pièces de musée, étaient venues s’ajouter des caractéristiques nouvelles et étonnantes qui racontaient les aventures sauvages qui furent les siennes pendant plus d’un demi-siècle. Le vieux capitaine Peleg, second à son bord pendant plusieurs années avant de commander son propre navire, qui était maintenant à la retraite et l’un des principaux propriétaires du {\itshape Péquod}, ce vieux Peleg avait, durant son règne de second, ajouté à son caractère grotesque primitif et l’avait pénétré de part en part d’une étrangeté, due à la fois au matériau et à son esprit inventif, qui n’avait sa pareille nulle part sauf peut-être sur le bouclier ou le châlit de Thorkill Hake. Il portait parures comme un barbare empereur d’Éthiopie au cou alourdi de pendentifs d’ivoire poli. C’était un reliquaire de trophées. Un cannibale de navire, se pavanant dans les ossements ciselés de ses ennemis. Ses pavois à découvert, sans jambettes, étaient ornés sur tout leur pourtour, sans interruption, telle une seule mâchoire, avec les longues dents aiguës du cachalot tenant lieu de cabillots pour amarrer ses muscles de chanvre et ses tendons. Ces filins ne couraient pas dans des poulies de vulgaire bois des forêts mais filaient prestement dans des réas creusés dans du morfil. Méprisant un gouvernail à tourniquet, il arborait une barre digne de respect, taillée d’une seule pièce dans la longue et étroite mâchoire inférieure de son ennemi héréditaire. L’homme de barre lorsqu’il gouvernait dans la tempête se sentait, cette barre en main, pareil au Tartare lorsqu’il retient par le mors son ardente monture. Un navire d’une vraie noblesse, mais aussi d’une certaine manière, d’une grande mélancolie ! Toute chose noble en est empreinte.\par
Lorsque j’en vins à chercher sur le gaillard d’arrière quelqu’un nanti d’autorité, afin de me porter candidat au voyage, je ne vis personne de prime abord mais ce que je ne pus manquer de voir c’était une singulière sorte de tente, ou plutôt de wigwam, dressée un peu en retrait du grand-mât. Elle semblait être une installation de fortune pour le temps passé au port, conique, de quelque dix pieds de haut, elle était construite avec les longs et immenses fanons noirs et souples, découpés en lanières et pris dans le centre et la partie la plus haute des mâchoires de la baleine franche. Leurs extrémités les plus larges s’appuyaient sur le pont ; lacées ensemble, ces lanières formaient un cercle décroissant qui se terminait en touffe serrée au sommet où ces crins flottants s’agitaient de-ci de-là telle une mèche de scalp mêlée de plumes sur la tête de quelque vieux sachem des Pottowottamie. Face à l’étrave, une ouverture triangulaire permettait à son hôte d’avoir une vue totale sur l’avant.\par
À demi dissimulé dans cette curieuse habitation, je trouvai enfin quelqu’un dont l’aspect semblait indiquer qu’il détenait un pouvoir et qui, tout travail étant interrompu à bord parce qu’il était midi, ayant déposé le fardeau du commandement, jouissait de son répit. Il était assis sur une chaise de chêne démodée, contorsionnée de toutes parts d’insolites sculptures et dont le siège était fait d’un puissant entrelacs de cette même matière élastique dont était construit le wigwam.\par
Peut-être l’aspect du vieil homme que je vis n’avait-il rien de si particulier ; il était tanné et musclé comme la plupart des vieux marins, lourdement enveloppé dans un manteau de pilote en drap brun, coupé à la mode quaker ; mais autour de ses yeux un fin et presque microscopique réseau de rides minuscules avait dû être tissé par un regard plissé sans cesse, fixé au vent, au cours d’incessantes navigations dans plus d’une rude tempête. De telles rides sont très utiles pour prendre un air menaçant.\par
– Ai-je l’honneur de parler au capitaine du {\itshape Péquod} ? demandai-je en m’approchant de l’entrée de la tente.\par
– En supposant que ce soit le capitaine du {\itshape Péquod}, que lui veux-tu ?\par
– Je pensais à embarquer.\par
– Tu y pensais, n’est-ce pas ? Je vois que tu n’es pas de Nantucket – déjà été dans un bateau défoncé ?\par
– Non, monsieur, jamais.\par
– Tu ne connais rien de rien en fait de pêche à la baleine, j’en jurerais, hein ?\par
– Rien, monsieur, mais je suis sûr que j’apprendrai vite. J’ai fait plusieurs voyages dans la marine marchande et je crois que…\par
– Le diable emporte la marine marchande. Pas de jargon avec moi. Vois-tu ce pied ? – Je te le flanquerai au cul si jamais tu reparles devant moi de la marine marchande. La marine marchande, sans blague ! Et je présume que tu en es même fier, d’avoir servi sur ces navires marchands. Mais bon sang, homme ! qu’est-ce qui te pousse à vouloir pêcher la baleine, hein ? – ça m’a l’air un peu suspect, non ? – Tu n’as pas été pirate des fois, non ? Tu n’as pas volé ton dernier capitaine, non ? Tu n’as pas pensé à assassiner les officiers une fois au large ?\par
Je protestai de mon innocence devant ces soupçons. Je comprenais que ce vieux marin, en tant que quaker de l’île de Nantucket, dissimulait, sous le masque de la facétie, ses préjugés d’insulaire, méfiant envers tout étranger à moins qu’il ne vienne du cap Cod ou de Vineyard.\par
– Mais qu’est-ce qu’il te prend de vouloir pêcher la baleine ? Je veux en connaître la vraie raison avant d’envisager de t’embarquer.\par
– Eh bien ! monsieur, je veux savoir ce que pêcher la baleine veut dire. Je veux voir le monde.\par
– Tu veux tâter de la pêche à la baleine, hein ? As-tu aperçu le capitaine Achab ?\par
– Qui est le capitaine Achab, monsieur ?\par
– Oui, oui, c’est bien ce que je me disais ! Le capitaine Achab est le capitaine de ce navire.\par
– Alors, je me trompe. Je croyais que je parlais au capitaine en personne.\par
– Tu parles au capitaine Peleg – voilà à qui tu parles, jeune homme. C’est à moi et au capitaine Bildad qu’il appartient de veiller à ce que le {\itshape Péquod} soit armé pour le voyage, approvisionné de tout le nécessaire, les hommes y compris. Nous avons des parts de propriété et nous sommes également armateurs. Mais j’étais en train de dire que si tu veux savoir ce que pêcher la baleine signifie, comme tu me dis le vouloir, je peux te donner le moyen de l’apprendre avant que tu ne sois engagé sans possibilité de te dédire. Va voir un peu le capitaine Achab, et tu te rendras compte qu’il n’a qu’une jambe.\par
– Que voulez-vous dire, monsieur ? Qu’il a perdu l’autre à cause d’une baleine ?\par
– À cause d’une baleine ! Jeune homme, viens plus près de moi : elle a été dévorée, mâchée, broyée par le plus monstrueux cachalot qui ait jamais fait voler en éclats un bateau !… ah ! \hspace{1em}ah !…\par
Je fus un peu effrayé par sa violence, peut-être un peu ému aussi par tout ce que sa dernière exclamation exprimait de sincère douleur, mais je répondis aussi calmement que je pus :\par
– Ce que vous dites n’est certainement que trop vrai, monsieur, mais comment aurais-je pu savoir que ce cachalot particulier était d’une si particulière férocité, encore qu’en vérité j’eusse bien pu en tirer cette déduction du simple fait de l’accident.\par
– Écoute à présent, jeune homme, tu as le poumon mou, tu comprends ? tu n’as pas le langage marin. Tu es bien sûr que tu as déjà pris la mer, tu es sûr de ça ?\par
– Monsieur, j’ai cru vous avoir dit que j’avais fait quatre voyages dans la marine…\par
– Ne la ramène pas avec ça ! Prends garde à ce que j’ai dit au sujet de la marine marchande, ne m’exaspère pas, je ne le permettrai pas. Mais essayons de nous comprendre. Je t’ai laissé entendre ce que pêcher la baleine signifiait, t’y sens-tu toujours disposé ?\par
– Oui, monsieur.\par
– Bon. Maintenant es-tu homme à jeter un harpon au fond de la gueule d’une baleine vivante et à filer à sa suite ? Réponds, vite !\par
– Je le suis, monsieur, s’il se trouve qu’il soit absolument nécessaire de le faire ; mais pas pour qu’on se débarrasse ainsi de moi, bien sûr, ce dont je ne parle pas comme d’une chose devant se produire.\par
– Bon, encore une fois. Alors, tu ne veux pas seulement partir pêcher la baleine pour savoir par expérience ce que c’est, mais encore tu souhaites voir le monde ? N’était-ce pas ce que tu disais ? C’est ce que j’ai cru comprendre. Eh bien ! avance seulement jusque-là, jette un coup d’œil du côté du vent, et reviens me dire ce que tu y auras vu.\par
Je restai quelque temps légèrement abasourdi par cette intimation curieuse, ne sachant pas très bien comment l’interpréter, que ce fût une plaisanterie ou un ordre sérieux. Mais ramenant toutes ses pattes d’oie en un froncement renfrogné, le capitaine Peleg m’invita à aller.\par
Je me rendis à l’avant, je regardai du côté du vent et je m’aperçus que le navire à l’ancre, balancé par la marée, pointait maintenant obliquement vers le large. La vue s’étendait à l’infini, un infini monotone et hostile à l’excès, je ne pus y discerner quoi que ce soit qui en rompît l’uniformité.\par
– Eh bien ! quel est le compte rendu ? me demanda Peleg lorsque je revins. Qu’as-tu vu ?\par
– Pas grand-chose… rien que de l’eau, un horizon immense toutefois et peut-être un grain qui se prépare…\par
– Alors, qu’est-ce que tu en penses maintenant de voir le monde ? As-tu envie de doubler le cap Horn pour ne rien voir de plus, hein ? Ne peux-tu voir le monde depuis là où tu es ?\par
J’étais un peu ébranlé, mais pêcher la baleine, je le devais, et j’irai ; et le {\itshape Péquod} était un bateau aussi bon qu’un autre – meilleur, pensais-je – pour embarquer. Tout cela, je l’exprimai à Peleg. Me voyant si décidé, il consentit à m’engager.\par
– Tu peux aussi bien signer les papiers tout de suite, ajouta-t-il, suis-moi. Ce disant, il me précéda jusqu’à la descente de la cabine.\par
Sur la barre d’arcasse était assis un personnage qui me parut extraordinaire et surprenant. Il se trouva que c’était le capitaine Bildad qui, avec le capitaine Peleg, possédait le plus grand nombre de parts du navire, les autres actions, comme c’est parfois le cas dans ces ports, appartenant à une foule de vieux rentiers, à des veuves et des orphelins, à des pupilles sous tutelle judiciaire ; chacun d’entre eux se trouvant propriétaire de la valeur d’une tête d’allonge, d’un pied de planche ou d’un ou deux clous à bord. Les gens de Nantucket placent leur argent sur les baleiniers, tout comme vous l’engagez en fonds d’État sûrs et d’un bon rapport.\par
Bildad, comme Peleg, et bien d’autres Nantuckais, était quaker, cette secte ayant fondé le premier établissement de l’île ; et ses habitants conservent en général et jusqu’à ce jour, dans une mesure très exceptionnelle, les particularités des quakers, nuancées seulement de façon variable et irrégulière par des éléments parfaitement étrangers et disparates. Car quelques-uns de ces mêmes quakers sont parmi les plus sanguinaires des marins et des chasseurs de baleines. Ce sont des quakers combattants, des quakers en furie.\par
De sorte qu’il y a parmi eux des exemples d’hommes portant des noms tirés des Écritures – une coutume très répandue sur l’île – et qui, assimilant avec le lait de l’enfance le tutoiement théâtral et majestueux du langage quaker, mêlent étrangement à ces particularités indéracinables, au cours de leurs vies d’aventures sans bornes, audacieuses et téméraires, mille traits de caractère d’effronterie, que ne renieraient pas un roi viking ou un conquérant romain. Lorsque ces contraires fusionnent dans un homme d’une force naturelle vraiment supérieure, au cerveau bien fait et au cœur grave, que tant de longues \hspace{1em}nuits de quart aussi, sur les plus lointaines eaux, sous les constellations que le Nord ne révèle jamais, amènent, par leur paix et leur solitude, à penser avec indépendance en marge des traditions, qui reçoit toutes les empreintes douces ou violentes à la source même du sein vierge, généreux et confiant de la nature, et qui se rend maître, dès lors, c’est l’un de ses profits accidentels, d’un langage hardi, énergique et hautain – cet homme – le recensement d’une nation n’en donnera qu’un – vous offre le spectacle de la grandeur de qui est promis aux nobles tragédies. Et il ne sera amoindri en rien, quant au pathétique, si de naissance ou à cause de quelque circonstance, une tristesse maladive domine le fond de sa nature, dans laquelle il semble se complaire à demi. Car c’est un certain goût morbide qui façonne tous les hommes d’une tragique grandeur. Ô jeune ambition, sache-le bien, toute grandeur humaine n’est que maladie. Mais pour le moment nous ne nous trouvons pas devant une nature de ce genre mais devant une autre, fort différente, avec un homme qui, pour singulier qu’il soit, n’exprime qu’un aspect du quakerisme modifié par sa personnalité.\par
Comme le capitaine Peleg, le capitaine Bildad était un baleinier à la retraite, fortune faite. Mais au contraire du capitaine Peleg, qui lui se moquait complètement de ce qu’on appelle les choses sérieuses et qui, à vrai dire, considérait ces mêmes choses sérieuses comme des broutilles, le capitaine Bildad non seulement avait été élevé selon les règles les plus strictes du quakerisme de la secte de Nantucket mais ni toute sa vie passée sur l’Océan, ni la vue, aux environs du cap Horn, de superbes créatures nues, ni rien de tout cela n’avait changé d’un iota ce quaker né comme cela, n’avait en rien adouci les angles de sa veste. En dépit de ces données inaltérables, l’honorable capitaine Bildad manquait passablement de simple logique. Bien que refusant, par scrupules de conscience, de porter les armes contre de terrestres envahisseurs, lui-même avait envahi sans que rien puisse le retenir, l’Atlantique et le Pacifique et, bien qu’ennemi juré du sang versé, il avait pourtant, dans son manteau \hspace{1em} étroit, répandu à flots celui du léviathan. Comment le pieux Bildad, au soir contemplatif de sa vie, réconciliait-il ces faits dans son souvenir, je ne sais ; mais cela ne semblait pas l’inquiéter outre mesure, et il en était très probablement venu à la conclusion sage et raisonnable que la religion d’un homme est une chose et que ce monde positif en est une autre. Ce monde paie des dividendes. Passant de mousse de carré, vêtu de court et de gris, au rang de harponneur habillé d’un large gilet en ventre d’alose, puis à celui de chef de pirogue, de second, de capitaine pour devenir enfin commanditaire, Bildad, je l’ai déjà dit, avait mis un terme à sa carrière aventureuse en abandonnant toute activité au bel âge de soixante ans, vouant le restant de ses jours à la douceur d’encaisser un revenu bien gagné.\par
Maintenant, j’ai le regret de le dire, Bildad avait la réputation d’être un vieil avare incorrigible et passait pour avoir été, du temps où il naviguait, un tyran implacable. On m’a raconté à Nantucket, bien que cela paraisse sans aucun doute une curieuse histoire, que lorsqu’il était maître à bord du vieux navire-baleinier Catgut, son équipage, en débarquant, avait été emmené presque au complet à l’hôpital, cruellement épuisé et usé. Pour un homme pieux, qui plus est pour un quaker, il avait plutôt le cœur dur, c’est le moins qu’on en puisse dire. Jamais il n’injuriait ses hommes, mais il avait une façon à lui de les soumettre aux travaux forcés, avec une cruelle démesure. Lorsque Bildad était second, sentir son œil terne vous fixer intensément vous rendait fou de nervosité jusqu’à ce que vous puissiez faire main basse sur n’importe quoi : un marteau ou un épissoir, et vous mettre au travail, comme un forcené, à une chose ou à une autre, peu importe quoi. L’indolence et la paresse tombaient mortes devant lui. Sa propre personne était l’incarnation parfaite de la parcimonie. Son long corps étique ne comportait aucune chair superflue, aucune barbe superflue, son menton s’ornant d’un poil doux ratissé à l’économie, semblable au velours élimé de son chapeau à larges bords.\par
Telle était la personne que je trouvai assise sur la barre d’arcasse après avoir suivi le capitaine Peleg dans la cabine. L’espace entre les ponts était étroit, et le vieux Bildad se tenait assis là, droit comme un i, car il s’asseyait toujours ainsi sans s’appuyer jamais comme s’il eût voulu ménager les basques de son vêtement. Son chapeau était posé à côté de lui, il tenait ses jambes croisées avec raideur, sa veste de drap était boutonnée jusqu’au cou et, les lunettes sur le nez, il paraissait absorbé dans la lecture d’un volume imposant.\par
– Bildad, s’écria le capitaine Peleg, je t’y reprends, Bildad, hein ? Voilà maintenant trente ans, à ma connaissance certaine, que tu étudies ces Écritures. Où en es-tu, Bildad ?\par
Comme s’il était habitué depuis longtemps à un langage aussi profane de la part de son vieux compagnon de bord, Bildad, sans relever son manque de respect, leva tranquillement les yeux et, me voyant, les reporta interrogativement sur Peleg.\par
– Il dit qu’il est notre homme, Bildad, dit Peleg. Il veut embarquer.\par
– Le veux-tu ? me demanda Bildad d’une voix caverneuse en se tournant vers moi.\par
– Si tu le veux bien, répondis-je, le tutoyant sans m’en rendre compte, tant il était quaker jusqu’au bout des ongles.\par
– Que penses-tu de lui, Bildad ? demanda Peleg.\par
– Il fera l’affaire, répondit Bildad, en m’examinant, et il se remit à déchiffrer son livre en marmottant de manière presque audible.\par
Je trouvais qu’il était bien le plus bizarre vieux quaker que j’aie jamais vu, d’autant plus que Peleg, son ami et \hspace{1em}compagnon de longue date, accusait le contraste, avec son genre de casseur d’assiettes. Mais je ne dis rien, me contentant de porter un regard perçant sur ce qui m’entourait. Peleg ouvrit un coffre et, sortant le rôle d’équipage, plaça devant lui une plume et de l’encre, en s’asseyant lui-même à une petite table. Je commençai à penser qu’il était grand temps que je me décide à connaître les conditions que j’accepterais pour mon engagement. Je savais déjà qu’il n’était pas question de salaire sur les baleiniers ; mais l’équipage, y compris le capitaine, recevait des parts sur le bénéfice appelées quantièmes d’une part proportionnelle au degré d’importance des tâches respectives des hommes. Je me rendais compte également qu’étant novice, ma part ne saurait être bien grande, mais du fait que je connaissais la mer, que j’étais capable de gouverner un navire et d’épisser un cordage, je ne doutais pas que, d’après ce que j’avais entendu dire, l’on m’offrirait un 275\textsuperscript{e} – c’est-à-dire le 275\textsuperscript{e} d’une part des bénéfices nets du voyage à quelque taux qu’ils se montassent. Bien que ce 275\textsuperscript{e} de part fût plutôt de ceux qu’ils appelaient à long terme, c’était toujours mieux que rien ; et si la chance favorisait notre voyage, cela paierait presque les vêtements que j’userais à bord, sans compter le lit et la table pour lesquels je ne débourserais pas un sou.\par
D’aucuns pourront trouver que c’est une piètre manière d’accumuler une fortune princière – et ce l’était, une bien piètre manière en vérité. Mais je ne suis pas de ceux qu’émeuvent les fortunes princières, et je suis bien content si le monde est disposé à me nourrir et à m’héberger lorsque je peux loger à l’enseigne menaçante du « Nuage d’orage ». Tout compte fait, je trouvai ce 275\textsuperscript{e} à peu près équitable, mais je n’aurais pas été surpris qu’on m’offrît le 200\textsuperscript{e}, vu la carrure de mes épaules.\par
Néanmoins, une chose me faisait douter de recevoir une part généreuse sur les bénéfices et c’était celle-ci : à terre, j’avais entendu parler du capitaine Peleg et de son inénarrable vieux compère Bildad et du fait qu’étant les plus gros propriétaires du {\itshape Péquod}, les autres actionnaires disséminés et de moindre importance laissaient la presque entière direction des affaires du navire à ces deux-là. Et je soupçonnai ce vieux pingre de Bildad d’avoir largement son mot à dire dans les engagements d’autant plus que je l’avais trouvé à bord, se sentant tout à fait chez lui dans cette cabine, et lisant sa Bible comme s’il se trouvait au coin de sa propre cheminée. Maintenant, tandis que Peleg essayait vainement de tailler une plume à l’aide de son couteau, le vieux Bildad, à ma grande surprise, étant donné qu’il était partie intéressée dans ces formalités, Bildad ne nous accorda aucune attention, mais continua à marmotter les phrases de son livre : « Ne vous amassez point de trésors sur la terre ou la mite… »\par
– Alors, capitaine Bildad, interrompit Peleg, qu’en dis-tu ? quelle part donnerons-nous à ce jeune homme ?\par
– Tu sais mieux… fut la sépulcrale réponse, la 777\textsuperscript{e} ne serait pas trop, n’est-ce pas ?… « où la mite et le ver consument… mais amassez-vous… »\par
Amassez ! vraiment ! pensai-je avec une part pareille ! la 777\textsuperscript{e} ! Eh bien ! vieux Bildad, vous avez résolu que moi le premier, je n’amasserai pas ici-bas, là où la mite et le ver consument. C’était une part à long terme, à un terme excessif en vérité, bien que l’amplitude du chiffre puisse, de prime abord, tromper un terrien, un peu de réflexion montrera que si 777 est un bien gros nombre, pourtant lorsqu’on lui ajoute la particule ème, force lui sera faite de constater que le 777\textsuperscript{e} d’un centime cela fait beaucoup moins que 777 doublons d’or. C’est bien ce que je pensais à ce moment-là.\par
– Que le diable t’emporte, Bildad, s’écria Peleg, tu ne veux quand même pas filouter ce jeune homme ! il faut lui donner plus que cela.\par
– Sept cent soixante-dix-septième, répéta Bildad sans lever les yeux ; et il se remit à marmonner « car où est ton trésor, là aussi sera ton cœur. »\par
– Je vais l’inscrire pour une trois centième, dit Peleg. Tu m’as bien entendu, Bildad. J’ai dit la trois centième.\par
Bildad posa son livre et se tournant vers lui avec solennité dit :\par
– Capitaine Peleg, tu as un cœur généreux, mais tu dois prendre en considération ce devoir qui est le tien envers les autres propriétaires de ce navire : des veuves, des orphelins pour bon nombre d’entre eux, et admettre que si nous rétribuons trop largement les services de ce jeune homme, nous enlèverons peut-être le pain de la bouche à ces veuves et à ces orphelins. La 777\textsuperscript{e}, capitaine Peleg.\par
– Toi, Bildad ! rugit Peleg en se levant et en s’agitant bruyamment dans la cabine. Maudit sois-tu, capitaine Bildad, si j’avais suivi ton avis dans ces domaines, j’aurais eu, avant ce jour, une conscience à traîner qui serait assez lourde pour envoyer par le fond le plus grand vaisseau qui ait jamais doublé le cap Horn.\par
– Capitaine Peleg, poursuivit Bildad fermement, ta conscience peut bien tirer dix pouces d’eau ou dix brasses, je n’en sais rien, mais étant donné que tu es toujours un pécheur impénitent, capitaine Peleg, je redoute beaucoup que ta conscience, elle, n’ait une voie d’eau, ne te coule et ne t’entraîne dans la fournaise de l’enfer, capitaine Peleg.\par
– La fournaise de l’enfer ! La fournaise de l’enfer ! tu m’insultes, homme, au-delà de ce qu’on peut humainement supporter, tu m’insultes. C’est un sacré outrage que de dire à n’importe quel être humain qu’il est voué à l’enfer. Par tous les tonnerres ! Bildad, dis-le moi encore une fois et fais-moi sortir de mes gonds, mais je… je… oui, j’avalerai une chèvre vivante, poils, cornes et tout. Hors d’ici, cafard décoloré, pistolet de bois… au large et plus vite que ça !\par
Tandis qu’il tempêtait de la sorte, il se précipita sur Bildad qui esquiva sur le côté avec une promptitude admirable et l’évita pour cette fois.\par
Alarmé par cette terrible algarade entre les deux principaux propriétaires et responsables du navire, me sentant à demi prêt à renoncer à toute idée d’embarquer sur un bateau en si discutable possession, et bien que l’autorité de ces capitaines ne fût que temporaire, je m’écartai de la porte pour laisser passer Bildad qui, je n’en doutai pas un instant, brûlait du désir de se soustraire à la colère éveillée en Peleg. Mais à mon étonnement, il se rassit tout à fait paisiblement et ne parut pas avoir la moindre intention de se retirer. Il paraissait fait au feu en ce qui concernait l’impénitent Peleg et ses habitudes. Quant à Peleg, après avoir donné libre cours à sa fureur, il semblait l’avoir totalement épuisée, et lui aussi, il s’assit comme un agneau, bien qu’il eût des crispations comme si ses nerfs vibraient encore.\par
– Pfuit ! siffla-t-il enfin… le grain a passé sous le vent, je crois. Bildad, autrefois tu étais adroit pour aiguiser les lances, taille-moi cette plume, veux-tu. Mon couteau a besoin de la meule. Je te reconnais là, merci, Bildad. À présent, mon garçon, Ismaël est ton nom, as-tu dit ? Eh bien, c’est en règle, Ismaël, pour la trois centième.\par
– Capitaine Peleg, dis-je, j’ai un ami avec moi qui voudrait embarquer aussi… puis-je l’amener demain ?\par
– Naturellement, dit Peleg, amène-le que nous le voyions.\par
– Quelle part veut-il ? gémit Bildad, levant les yeux de son livre dans lequel il s’était à nouveau absorbé.\par
– Oh ! Ne t’occupe pas de cela, Bildad, puis, se tournant vers moi, Peleg ajouta : A-t-il déjà chassé la baleine ?\par
– Tué plus de baleines que je ne saurais compter, capitaine Peleg.\par
– Bon, alors amène-le.\par
Les papiers signés, je partis, ne mettant pas en doute que j’avais fait du bon travail ce matin et que le {\itshape Péquod} était bel et bien le navire prévu par Yoyo pour nous transporter Queequeg et moi au-delà du cap Horn.\par
Mais je ne m’étais guère éloigné que je réalisai n’avoir pas encore vu le capitaine avec lequel nous devions partir, bien qu’à vrai dire, en maintes occasions, un baleinier puisse se trouver complètement armé, tout son équipage à bord, et que le capitaine fasse seulement son apparition à l’instant de prendre le commandement ; car les voyages se prolongent parfois si longuement, le temps passé à terre et à la maison est si éphémère que, si le capitaine a une famille, ou quelque attache de même nature, il ne se soucie guère de son navire à l’ancre, et l’abandonne aux propriétaires jusqu’à ce qu’il soit prêt à prendre la mer. Toutefois, il est toujours plus prudent d’avoir jeté un coup d’œil sur lui avant de se remettre irrévocablement entre ses mains. Retournant sur mes pas, j’abordai le capitaine Peleg pour lui demander où l’on pouvait trouver le capitaine Achab.\par
– Et qu’est-ce que tu lui veux, au capitaine Achab ? Tout est en ordre, tu es enrôlé.\par
– Oui, mais j’aimerais le voir.\par
– Je ne crois pas que tu le pourras pour l’instant. Je ne sais pas au juste ce qu’il a, mais il garde la chambre… une sorte de maladie, pourtant il n’a pas l’air malade. En fait il n’est pas malade, mais il n’est pas bien non plus. De toute façon, jeune homme, il n’est pas toujours d’accord de me voir moi, aussi je ne pense pas qu’il souhaite te voir, toi. C’est un homme étrange, le capitaine Achab – certains, du moins, le trouvent étrange – mais c’est un homme ! Oh ! tu l’aimeras, n’aie crainte, n’aie crainte. Il a de la grandeur, c’est un homme sans dieu, pareil à un dieu, le capitaine Achab, peu causant, mais quand il parle, alors il faut bien l’écouter. Prends note, je t’en avertis, le capitaine Achab est au-dessus du commun ; Achab a fréquenté tant les grandes écoles que les cannibales ; connu des prodiges plus profonds que la plus profonde vague ; plongé sa lance ardente dans des ennemis plus puissants et plus étranges que les baleines. Son harpon ! oui, le plus aigu et le plus sûr de toute notre île ! Oh ! ce n’est pas un capitaine Bildad, non, ni un capitaine Peleg ; il est Achab, mon fils ; et l’Achab de l’histoire, tu le sais, était un roi couronné !\par
– Et un roi très infâme. Lorsque ce roi pervers fut assassiné, les chiens ne léchèrent-ils pas son sang ?\par
– Viens près de moi, plus près, plus près, me dit Peleg et l’expression de son regard m’effraya presque. Écoute bien, mon gars, ne répète jamais cela à bord du {\itshape Péquod}. Ne le répète jamais nulle part. Le capitaine Achab n’a pas choisi son nom. C’est une lubie insensée de sa mère, une veuve folle et ignorante, qui mourut lorsqu’il n’avait que douze mois d’âge. Et pourtant la vieille squaw Tistig de Gayhead disait que, d’une façon ou d’une autre, ce nom se révélerait prophétique. Et peutêtre que d’autres imbéciles de la même trempe viendront te dire la même chose. Je souhaite t’en avertir, c’est un mensonge. Je connais bien le capitaine Achab, j’ai été son second il y a bien des années ; je sais qui il est, c’est un homme intègre, non point un brave homme bigot comme Bildad, mais un brave homme qui jure – un peu comme moi – seulement il y a en lui bien d’autres richesses. Oui, oui, je sais qu’il n’a jamais été très gai ; et je sais qu’au retour il a eu l’esprit un peu dérangé par un maléfice, mais la douleur aiguë, lancinante que lui infligeait son moignon sanglant en était la cause ; n’importe qui le comprendrait. Je sais aussi que, depuis qu’il a perdu sa jambe au dernier voyage à cause de cette maudite baleine, il a été d’humeur changeante, parfois désespéré, parfois furieux, mais tout cela passera. Et une fois pour toutes, permets-moi de te dire et de t’affirmer : mieux vaut naviguer avec un bon capitaine ombrageux qu’avec un mauvais capitaine hilare. Au revoir à toi ! et ne condamne pas le capitaine Achab parce qu’il se trouve qu’il porte un nom mauvais. D’autre part, mon fils, il a une femme – il n’y a pas trois voyages qu’il est marié – une douce fille résignée. Pense à ceci : de cette gentille fille, ce vieil homme a un enfant, penses-tu alors qu’un mal total et sans espoir possède Achab ? Non, non, mon gars, si frappé, si dévasté qu’il soit, Achab est humain !\par
J’étais en m’éloignant profondément songeur. Ce qui venait de m’être incidemment révélé sur le capitaine Achab m’emplissait d’une souffrance à la fois vague et violente à son égard. Et d’une certaine manière, à ce moment-là, j’éprouvai envers lui une douloureuse compassion, dont j’ignorais la raison, à moins que ce ne fût à cause de sa jambe perdue de si cruelle façon. Et pourtant je ressentais aussi à son égard une terreur respectueuse, mais cette sorte de terreur, que je serais bien en peine de décrire, n’était pas vraiment de la terreur, je ne sais pas ce que c’était. Elle me pénétrait et ne me rebutait pas. Mais le mystère qui m’enveloppait, alors que je le connaissais si peu encore, engendrait en moi une impatience. Enfin mes pensées prirent un autre cours de sorte que le sombre Achab les quitta.
\chapterclose


\chapteropen
\chapter[{CHAPITRE XVII. Le ramadan}]{CHAPITRE XVII \\
Le ramadan}\renewcommand{\leftmark}{CHAPITRE XVII \\
Le ramadan}


\chaptercont
\noindent Étant donné que le ramadan de Queequeg, ou son jeûne, ou sa macération devait se poursuivre toute la journée, je ne voulus pas le déranger jusque vers le soir, nourrissant le plus grand respect des obligations religieuses de chacun, si comiques qu’elles puissent être, et au fond de mon cœur ne pouvant même mépriser une congrégation de fourmis adorant un champignon vénéneux, ni ces autres créatures qui en certains lieux de notre globe, avec des manières de laquais sans exemple sur d’autres planètes, font des courbettes devant le buste d’un défunt propriétaire terrien sous le simple prétexte que des biens excessifs sont encore possédés et gérés en son nom.\par
Je dis que nous autres chrétiens presbytériens devrions nous montrer tolérants, et ne pas nous imaginer si immensément supérieurs aux autres mortels, païens ou autres, à cause de leurs conceptions saugrenues en ces domaines. Queequeg était en train sans aucun doute de cultiver les notions les plus absurdes au sujet de Yoyo et de son ramadan. Et puis après ? Queequeg pensait savoir ce qu’il faisait, je présume. Il paraissait heureux, laissons-le en paix. Tous les arguments que nous pourrions lui opposer seraient vains ; laissons-le libre, dis-je. Et que Dieu ait pitié de nous tous, tant presbytériens que païens, car nous avons tous la tête lamentablement fêlée d’une façon ou d’une autre, et nous aurions besoin de réparations.\par
Vers le soir, lorsque j’eus la certitude que ces rites et célébrations devaient être terminés, je montai à sa chambre et frappai à la porte. Pas de réponse. J’essayai de l’ouvrir mais elle était fermée de l’intérieur. « Queequeg », soufflai-je par le trou de la serrure – silence absolu. « Allons, Queequeg ! pourquoi ne répondez-vous pas ? C’est moi… Ismaël. » Mais rien ne bougeait. Je commençai à être soucieux ; lui ayant accordé un laps de temps si long, je crus que peut-être il avait eu une attaque d’apoplexie. Je regardai par le trou de la serrure, mais la porte ouvrant dans un coin écarté de la chambre, le trou de la serrure n’offrait d’autre horizon qu’un angle sinistre : un fragment du pied du lit et un pan de mur, mais rien de plus. Pourtant je fus surpris de voir, appuyé contre le mur, le manche de bois du harpon de Queequeg que la patronne lui avait enlevé le soir précédent, avant que nous montions dans la chambre. C’est étrange, pensai-je, mais en tout cas du moment que le harpon est là, et qu’il ne sort pour ainsi dire jamais sans lui, il doit être dans la chambre, sans erreur possible. « Queequeg ! Quee- queg ! » Rien. Il avait dû se passer quelque chose. Apoplexie ? J’essayai d’enfoncer la porte, mais elle résista avec opiniâtreté. Descendant en courant, je fis rapidement part de mes inquiétudes à la première personne que je rencontrai, la femme de chambre. – « Oh ! là ! là ! cria-t-elle. Je pensais bien qu’il était arrivé quelque chose. Je suis montée pour faire le lit après le petit déjeuner et la porte était fermée à clef, on aurait entendu voler une mouche, et rien n’a bougé depuis. Mais je me disais que, peut-être, vous étiez sortis tous les deux et que vous aviez fermé à cause de vos bagages. Oh ! là ! là ! M’ame ! Patronne ! au meurtre ! Madame Hussey ! Apoplexie ! » Poursuivant ses exclamations, elle courut vers la cuisine et moi sur ses talons.\par
M\textsuperscript{me} Hussey apparut aussitôt, un pot de moutarde dans une main, une burette de vinaigre dans l’autre, arrachée à son occupation qui consistait à repourvoir les huiliers tout en grondant son petit domestique noir.\par
– Le bûcher ! criai-je, où se trouve le bûcher ? Courez-y de grâce, et ramenez quelque chose pour enfoncer la porte – la hache ! – la hache ! il a eu une attaque, sa vie en dépend ! et, ce disant, je me précipitai en haut d’une manière désordonnée, les mains toujours vides, lorsque M\textsuperscript{me} Hussey s’interposa avec le pot de moutarde, la burette de vinaigre et son armature personnelle.\par
– Qu’est-ce qu’il vous prend, jeune homme ?\par
– Allez chercher la hache ! Pour l’amour de Dieu, que quelqu’un aille appeler un docteur, pendant que j’enfoncerai la porte !\par
– Doucement ! dit la patronne, posant promptement la burette de vinaigre, afin d’avoir une main libre : attendez un peu, vous parlez d’enfoncer mes portes ? – et elle m’empoigna le bras. Qu’est-ce qu’il vous prend ? qu’est-ce qu’il vous prend, matelot ?\par
Aussi calmement et aussi succinctement que je le pus, je lui donnai à comprendre toute l’affaire. Collant machinalement la burette à vinaigre sous sa narine, elle médita un moment, puis s’écria : « Non ! je ne l’ai pas revu depuis que je l’ai posé là. » Elle se précipita vers un réduit sous l’escalier, y jeta un coup d’œil, et me dit que le harpon de Queequeg n’y était plus. Elle hurla : « Il s’est tué, v’là que ça recommence comme avec ce mal’reux Stiggs – encore un couvre-lit de perdu – Dieu ait pitié de sa pauvre mère ! – ce sera la faillite de ma maison. Est-ce que le pauvre diable a une sœur ? Où est passée cette fille ? Ah ! vous voilà, Betty, courez chez Snarles, le peintre, et dites-lui de me faire l’écriteau suivant : Il est interdit de se suicider dans la maison et de fumer dans le salon. Autant faire d’une pierre deux coups. Tué ? Le Seigneur prenne soin de son âme ! Qu’est-ce que c’est que ce bruit ? Vous, jeune homme, baste !\par
Se lançant à ma poursuite, elle m’attrapa au moment où j’essayais à nouveau d’enfoncer la porte.\par
– Je ne le permettrai pas ! Je ne supporterai pas qu’on détériore l’immeuble. Allez chercher le serrurier, il y en a un à une lieue d’ici environ. Mais baste ! et plongeant la main dans sa poche, voilà une clef qui fera l’affaire, j’ai l’impression, voyons un peu. Sur ce, elle la tourna dans la serrure, mais hélas ! Queequeg avait mis le verrou de sûreté !\par
– Il faut que je la fasse sauter, dis-je, en prenant dans le corridor le recul nécessaire pour avoir un bon élan, lorsque la patronne m’empoigna, jurant ses grands dieux que je ne démolirais pas ses locaux ; mais je m’arrachai à sa poigne et bondissant, je sautai droit dans le but de toutes mes forces.\par
La porte céda avec un bruit retentissant, et le gond, giclant contre le mur, envoya du plâtre jusqu’au plafond. Et là… Dieu du ciel ! Là Queequeg était assis, froidement assis sur ses fesses, Yoyo en équilibre sur la tête. Il ne regardait ni d’un côté, ni de l’autre, mais ne donnait pas plus signe de vie qu’une statue.\par
– Queequeg, dis-je en allant vers lui, Queequeg, que vous arrive-t-il ?\par
– Y n’est quand même pas resté assis com’ça toute la journée, non ? dit la patronne.\par
Malgré tout ce que nous pouvions dire, il fut impossible d’en tirer un mot, j’eus presque envie de le bousculer pour lui faire perdre cette position qui paraissait intolérable tant elle semblait douloureuse et crispée de manière antinaturelle, d’autant plus que selon toute vraisemblance il était resté ainsi assis tout droit pendant huit ou dix heures et sans prendre ses repas.\par
– Madame Hussey, dis-je, en tout cas, il est vivant, aussi laissez-nous s’il vous plaît et j’éclaircirai cette étrange affaire moi-même.\par
Fermant la porte sur la patronne, j’entrepris d’imposer à Queequeg de prendre une chaise, mais en vain. Il était vissé là et tout ce que lui inspirèrent mes dépenses de politesses et de cajoleries fut qu’il ne bougea pas une oreille, ne dit pas un mot, ne me regarda même pas et ne prit d’aucune manière garde à ma présence.\par
Je me demandais si cela pouvait réellement faire partie de son ramadan ! Jeûnent-ils assis sur leurs fesses dans son île natale ? Il doit en être ainsi, oui, c’est inhérent à sa croyance, je suppose. En ce cas, laissons-le, il finira sans doute par se lever tôt ou tard. Cela ne peut pas durer toujours. Dieu merci, son ramadan ne revient probablement qu’une fois par an, et je présume qu’il n’est pas d’une absolue ponctualité.\par
Je descendis pour le souper, et restai longuement à écouter les histoires interminables de quelques marins rentrés la veille d’un voyage « plum-pudding » comme ils disaient (c’est une courte campagne de pêche à la baleine sur une goélette ou un brick, limitée au nord de la ligne et dans l’Atlantique seulement). Après avoir écouté ces plum-puddingers jusque vers onze heures, je montai dans l’intention d’aller me coucher, bien certain que cette fois Queequeg aurait mis le point final à son ramadan. Mais non ! il était tel que je l’avais laissé, n’ayant pas bougé d’un pouce. Je commençais à me sentir fâché contre lui ; cela semblait si intégralement dépourvu de sens et si démentiel de rester assis là tout le jour et la moitié de la nuit sur ses fesses dans une chambre glaciale, à tenir un bout de bois sur la tête.\par
– Pour l’amour du ciel, Queequeg, levez-vous et secouezvous ; allons, debout et allez souper. Vous allez mourir de \hspace{1em}faim, vous allez vous détruire, Queequeg. Mais pas un mot de réponse.\par
Désespérant de lui, je résolus dès lors de me mettre au lit et de dormir, sans aucun doute il me rejoindrait dans peu de temps. Mais avant de m’étendre je pris ma lourde veste de laine et la lui posai sur les épaules car la nuit s’annonçait très froide et il ne portait que son gilet. Pendant longtemps, j’eus beau faire, je ne pus trouver le sommeil. J’avais soufflé la chandelle, mais la seule pensée de Queequeg, assis à quatre pieds de moi, dans son inconfortable position, tout seul dans le froid et l’obscurité, cette pensée me rendait vraiment malheureux. Imaginez cela : dormir toute la nuit dans la même chambre qu’un païen bien éveillé, installé sur ses fesses, pour ce ramadan inexplicable et lugubre !\par
Je finis pourtant par m’endormir et je ne sus plus rien jusqu’à la pointe du jour, lorsque, me redressant, je vis Queequeg accroupi comme s’il était vissé au plancher. Mais dès que le premier rayon de soleil entra par la fenêtre, il se leva, les articulations raidies et grinçantes, mais le regard joyeux ; il vint jusqu’à moi en boitant, pressa son front contre le mien et m’annonça que son ramadan était terminé.\par
Je l’ai déjà dit, je n’ai d’objection contre la religion de personne, qu’elle soit ce qu’on voudra, pour autant qu’elle ne tue ni n’insulte un autre, sous le prétexte que sa conviction diffère. Mais quand la religion d’un homme devient vraiment forcenée, quand elle lui est une véritable torture et fait enfin de ce monde une mauvaise auberge, alors je crois qu’il est grand temps de prendre cet individu à part et de débattre la question avec lui.\par
C’est ce que je fis alors avec Queequeg. « Queequeg, dis-je, mettez-vous au lit, à présent, étendez-vous et écoutez-moi. » Je commençai mon discours à partir de la naissance et du progrès des religions primitives pour en arriver aux différentes religions des temps présents, m’évertuant en cours de route de démontrer à Queequeg que tous ces carêmes, ces ramadans, ces accroupissements sur les fesses dans le froid et dans des chambres lugubres étaient une pure sottise, que c’était mauvais pour la santé, inutile pour l’âme, bref contraire à toutes les lois évidentes de l’hygiène et du bon sens. Je lui dis également qu’étant donné qu’il se trouvait être, dans tant de domaines, un sauvage plein de raison et de sagesse, cela me peinait, cela me peinait affreusement, de le voir à présent si déplorablement absurde avec son ridicule ramadan. D’autre part je mis l’accent sur le fait que le jeûne mène le corps à l’effondrement et provoque dès lors l’effondrement de l’esprit, et que toutes les pensées nées en période de jeûne ne peuvent être que des pensées faméliques. C’est pour cette raison que la plupart des dévots dyspeptiques entretiennent des idées si mélancoliques sur l’au-delà. En un mot, Queequeg – et j’étais quelque peu hors du sujet – la notion de l’enfer est née d’une indigestion de chaussons aux pommes et s’est propagée depuis lors à cause de toutes les dyspepsies héréditaires engendrées par les ramadans.\par
Je demandai ensuite à Queequeg s’il n’avait jamais luimême été incommodé par la dyspepsie, lui posant ma question de manière très imagée afin qu’il pût la comprendre parfaitement. Il répondit que non, sauf une fois dans une circonstance mémorable. C’était lors d’une fête donnée par le roi son père en l’honneur d’une grande victoire au cours de laquelle cinquante ennemis avaient été tués, vers deux heures environ de l’aprèsmidi, rôtis et mangés le même soir.\par
– N’en dites pas davantage, Queequeg, dis-je en frissonnant, cela suffira. Je pouvais, en effet, tirer mes déductions sans qu’il fût besoin d’insister. J’avais rencontré un marin ayant visité cette même île et il m’avait raconté qu’à l’issue victorieuse d’une grande bataille la coutume voulait que tous les morts, se trouvant dans la cour ou le jardin du vainqueur, fussent rôtis tout entiers. Ensuite de quoi, on les disposait, un par un, sur de grandes planches à hacher, on les garnissait comme un pilaf avec des fruits de l’arbre à pain et des noix de coco, on leur mettait du persil dans la bouche et le vainqueur les distribuait en cadeau à tous ses amis, avec ses meilleurs compliments, comme autant de dindes de Noël.\par
Tout compte fait, je ne crois pas que mes réflexions métaphysiques firent grande impression à Queequeg. Parce que d’une part cette grave question semblait l’ennuyer et qu’il paraissait en avoir assez de ce sujet, sauf à le considérer de son propre point de vue ; et parce que d’autre part il n’avait pas compris le tiers de ce que je lui avais dit bien que j’eusse exprimé mes idées aussi simplement que possible ; parce que, enfin, il était bien sûr d’en savoir plus long que moi sur la vraie religion. Il me regardait avec une inquiétude et une compassion condescendantes, comme s’il pensait qu’il était fort dommage qu’un jeune homme aussi intelligent fût perdu sans espoir pour la cause de l’évangile païen.\par
Enfin nous nous levâmes et nous nous habillâmes ; Queequeg fit un petit déjeuner si prodigieusement copieux de soupes de poissons de toute espèce que la patronne ne tira pas grand profit de son ramadan puis nous sortîmes faire un tour à bord du {\itshape Péquod}, flânant et nous curant les dents avec des arêtes de flétan.
\chapterclose


\chapteropen
\chapter[{CHAPITRE XVIII. Sa marque}]{CHAPITRE XVIII \\
Sa marque}\renewcommand{\leftmark}{CHAPITRE XVIII \\
Sa marque}


\chaptercont
\noindent Tandis que nous nous dirigions vers l’estacade où se trouvait le navire, Queequeg arborant son harpon, le capitaine Peleg nous héla violemment de sa voix bourrue depuis son wigwam, en disant qu’il n’avait pas soupçonné mon ami d’être un cannibale, claironnant en outre qu’il ne tolérait pas la présence de cannibales à bord, à moins qu’ils ne montrassent au préalable leurs papiers.\par
– Qu’entendez-vous par là, capitaine Peleg ? demandai-je en sautant par-dessus la rambarde, laissant mon camarade debout sur le quai.\par
– J’entends qu’il doit montrer ses papiers.\par
– Oui, renchérit le capitaine Bildad de sa voix caverneuse, pointant sa tête derrière celle de Peleg, hors du wigwam. Il doit prouver qu’il est converti. Fils des ténèbres, ajouta-t-il en se tournant vers Queequeg, es-tu à présent communiant de quelque église chrétienne ?\par
– Comment ! dis-je, il est membre de la Première Église Congrégationaliste. Soit dit entre parenthèses, bien des sauvages tatoués enrôlés sur des bateaux de Nantucket finissent par être convertis à une confession ou à une autre.\par
– La Première Église Congrégationaliste ! s’écria Bildad. Comment ! celle qui se réunit dans le temple de l’Ancien Deutéronome de Coleman ? et ce disant il enleva ses lunettes, les essuya dans son grand mouchoir jaune, et les remettant avec un soin extrême, il sortit du wigwam, se pencha avec raideur pardessus le bastingage, et examina longuement Queequeg.\par
– Depuis combien de temps en es-tu membre ? demanda-til enfin, puis se tournant vers moi : pas très longtemps, j’ai l’impression, jeune homme.\par
– Non, dit Peleg, et il n’a pas non plus été baptisé comme il faut, sinon cela aurait lavé un peu de bleu du diable qu’il a sur la figure.\par
– Maintenant dis-moi, cria Bildad, est-ce que ce Philistin est un membre reconnu du temple de l’Ancien Deutéronome ? Je ne l’ai jamais rencontré y allant et je passe devant tous les jours que Dieu fait.\par
– J’ignore tout de l’Ancien Deutéronome et de son temple, dis-je, tout ce que je sais c’est que Queequeg ici présent est né membre de la Première Église Congrégationaliste. C’est un diacre lui-même, Queequeg est diacre.\par
– Jeune homme, dit Bildad sévèrement, vous me racontez des sornettes – explique-toi toi-même, jeune Hittite. De quelle église parle-t-il ? réponds-moi !\par
Me trouvant acculé, je répondis : « Je veux dire, monsieur, à la même ancienne Église catholique à laquelle vous et moi et le capitaine Peleg ici présent et Queequeg également présent, et nous tous, et le fils de chaque mère et l’âme de tout un chacun appartenons à la grande et éternelle première Congrégation de l’universelle adoration ; tous nous lui appartenons ; seulement certains d’entre nous nourrissent des préjugés mais ceux-ci ne portent nullement atteinte à la personne de Dieu, en qui nos mains à tous sont unies.\par
– Épissées, tu entends, des mains épissées, s’écria Peleg en se rapprochant. Jeune homme, tu ferais mieux de t’embarquer comme missionnaire plutôt que comme simple matelot. Je n’ai jamais entendu meilleur sermon. Ancien Deutéronome, le père Mapple lui-même n’aurait pas dit mieux et son éloquence est connue. Montez à bord, montez à bord, peu importe les papiers. Je dis, informez Quohog là, – comment est-ce qu’il s’appelle ? – dis à Quohog d’approcher. Pas la grande ancre ! Quel harpon il a là ! me paraît de bonne trempe, celui-là, et il a l’air de le manipuler comme il faut. Je disais donc Quohog, ou quel que soit votre nom, vous êtes-vous déjà tenu à l’avant d’une baleinière ? Avez-vous déjà frappé le poisson ?\par
Sans un mot, Queequeg, à sa manière farouche, sauta pardessus le pavois, et de là à l’avant d’une pirogue suspendue au flanc du navire, le jarret gauche tendu, il équilibra son harpon et s’époumona à peu près en ces mots :\par
– Cap’tain tu le vois petit goutte goudron sur l’eau là ? Tu vois lui ? ben s’pose lui œil baleine, alors ben ! et visant avec acuité, il darda le fer qui siffla juste au-dessus du chapeau à larges bords du vieux Bildad, droit à travers les ponts du navire et piqua la goutte brillante de goudron qui disparut.\par
– Alors, dit Queequeg, tranquillement, en rentrant sa ligne, s’pose-e-lui baleine-e-œil ; ben mort baleine, mort.\par
– Vite, Bildad, dit Peleg à son compagnon qui, épouvanté par le si proche voisinage d’un harpon en plein vol, avait battu en retraite vers la cabine. Vite, te dis-je, sacré Bildad, sors le rôle d’équipage. Il nous faut Hedgehog, je veux dire Quohog, dans une de nos baleinières. Écoute un peu, Quohog, nous te donnerons la 90\textsuperscript{e} part et c’est plus que nous n’avons jamais donné à un harponneur de Nantucket.\par
Nous descendîmes dans la cabine, et, à ma grande joie, Queequeg fut bientôt enrôlé dans ce même équipage auquel j’appartenais désormais.\par
Ces préliminaires terminés, Peleg prépara la plume et l’encre et dit en se tournant vers moi : « Je présume que ce Quohog ne sait pas écrire, n’est-ce pas ? Quohog, que le diable t’emporte ! signes-tu ton nom ou poses-tu ta marque ?\par
Mais cette question ne démonta nullement Queequeg qui avait déjà participé à deux ou trois cérémonies du même ordre, mais, prenant la plume qui lui était tendue, il traça sur le papier, à la juste place, la réplique parfaite d’un étrange motif arrondi qui ornait son bras tatoué, de sorte que, Peleg s’entêtant à déformer son nom, cela donnait quelque chose comme suit :\par
Quohog Sa * marque\par
Pendant ce temps le capitaine Bildad était assis, regardant longuement et obstinément Queequeg, puis il se leva avec solennité, farfouilla dans les immenses poches de son vaste manteau brun, en sortit une poignée de brochures, en choisit une intitulée : « L’approche du Dernier Jour » ou « Pas de temps à perdre », la mit dans les mains de Queequeg qu’il pressa sur le livre entre les siennes et, le regardant dans les yeux, lui dit : « Fils des ténèbres, je dois accomplir mon devoir à ton endroit, je suis en partie propriétaire de ce navire et je me sens responsable des âmes de tout l’équipage ; si tu te cramponnes à tes cultes païens, comme je le crains avec tristesse, je t’implore, ne demeure pas à jamais le serf de Bélial. Repousse l’idole Baal et le dragon hideux ; détourne de toi la colère future. Aie l’œil ! dis-je. Oh ! bonté divine ! Mets le cap loin de l’enfer ! »\par
Un peu du sel de la mer s’attardait encore dans le langage du vieux Bildad, formant un mélange hétéroclite entre un style scripturaire et le langage familier.\par
– Baste ! Baste maintenant, Bildad ! Assez gâché notre harponneur. Les harponneurs dévots n’ont jamais fait de bons marins, ça émousse l’émerillon en eux et un harponneur ne vaut pas un fétu s’il n’a pas le croc aigu. Il y a eu le jeune Nat Swaine, naguère le plus hardi chef de pirogue de tout Nantucket et de Vineyard. Depuis qu’il est allé au temple, on n’en a plus rien tiré de bon. Il tremblait tellement pour sa maudite âme qu’il reculait et prenait le large devant les baleines, craignant les derniers soubresauts risquant de défoncer la pirogue et de l’envoyer dans la grande Baille.\par
– Peleg ! Peleg ! dit Bildad en levant et les yeux et les mains, toi, toi-même tout comme moi, tu as connu bien des dangers, tu n’ignores pas, Peleg, ce que c’est que d’avoir peur devant la mort, comment peux-tu, dès lors, jacasser de cette manière impie. Tu démens ton propre cœur, Peleg. Dis-moi, lorsque ce même {\itshape Péquod} eut ses trois mâts balayés par ce typhon, au large du Japon, au cours de ce même voyage où tu étais second du capitaine Achab, n’as-tu pas alors pensé à la mort et au jugement ?\par
– Écoutez-le, non mais écoutez-le ! s’écria Peleg, traversant la cabine en enfonçant ses mains au plus profond de ses poches, écoutez-le, vous tous. Pensez un peu ! Quand à chaque minute nous pensions voir sombrer le navire ! La mort et le jugement, à un moment pareil ? Et comment ! Avec les trois mâts qui faisaient un bruit incessant de tonnerre contre les membrures ; et toutes les vagues de la mer qui nous passaient par-dessus, à l’avant et à l’arrière. À la vie, voilà à quoi le capitaine Achab et moi nous pensions, et comment sauver les hommes, comment équiper un mât de fortune, comment atteindre le port le plus proche, voilà à quoi je pensais !\par
Bildad ne pipa plus mot mais, boutonnant son manteau, il monta sur le pont où nous le suivîmes. Et il se tint là, surveillant impassiblement des marins qui raccommodaient un hunier sur l’embelle. De temps en temps il se baissait pour ramasser une pièce de toile, ou un bout de fil à voile goudronné, qui, sans cela, auraient risqué d’être gaspillés.
\chapterclose


\chapteropen
\chapter[{CHAPITRE XIX. Le prophète}]{CHAPITRE XIX \\
Le prophète}\renewcommand{\leftmark}{CHAPITRE XIX \\
Le prophète}


\chaptercont
\noindent « Matelots, vous êtes-vous enrôlés sur ce navire ? » Queequeg et moi venions tout juste de quitter \hspace{1em}le {\itshape Péquod}, nous éloignant nonchalamment du quai, absorbés chacun dans nos propres pensées, lorsqu’un étranger nous adressa en ces mots la parole en montrant d’un index massif le bateau en question. Il était misérablement vêtu d’un veston passé, de pantalons rapiécés, et d’un lambeau de mouchoir noir autour du cou. Une petite vérole confluente avait inondé son visage et l’avait abandonné, tel le lit d’un torrent d’où les eaux ruisselantes se sont retirées, couturé de nervures compliquées.\par
– Êtes-vous enrôlés à ce bord ? répéta-t-il.\par
– Vous entendez le {\itshape Péquod}, je présume ? dis-je en essayant de gagner du temps pour scruter sa physionomie.\par
– Oui, le {\itshape Péquod}, ce navire-là, dit-il, ramenant son bras en arrière et le détendant brusquement devant lui, armé de la baïonnette de son index pointée au cœur dudit objet.\par
– Oui, dis-je, nous venons de signer le rôle.\par
– Pas de clause concernant vos âmes ?\par
– Concernant quoi ?\par
– Oh ! peut-être que vous n’en avez pas, ajouta-t-il vivement. Aucune importance, je connais beaucoup de gars qui n’en ont pas… grand bien leur fasse et ils s’en sentent d’autant mieux. Une âme, c’est quelque chose comme la cinquième roue d’un char.\par
– Que radotez-vous, camarade ? dis-je.\par
– Lui, du moins, il en a assez pour suppléer à toutes les déficiences de cet ordre des autres gars, dit brusquement l’étranger, soulignant avec émotion le mot « lui ».\par
– Queequeg, dis-je, allons-y, ce personnage est un échappé de quelque part, il parle de quelque chose et de quelqu’un que nous ne connaissons pas.\par
– Un moment ! s’écria l’étranger. Vous dites vrai, vous n’avez pas encore vu le vieux Tonnerre, n’est-ce pas ?\par
– Qui est le vieux Tonnerre ? demandai-je subjugué encore par la conviction et la démence de ses manières.\par
– Le capitaine Achab.\par
– Comment ! le capitaine de notre navire, du {\itshape Péquod} ?\par
– Oui, quelques vieux marins d’entre nous l’appellent ainsi. Vous ne l’avez pas encore vu ?\par
– Non. Il est malade, à ce qu’ils disent, mais il se remet et sera tout à fait bien sous peu.\par
– Tout à fait bien sous peu ! s’esclaffa l’étranger avec un rire à la fois ironique et solennel. Écoutez-moi bien ! lorsque le capitaine Achab ira bien, alors mon bras gauche aussi. Pas avant.\par
– Que savez-vous de lui ?\par
– Que vous ont-ils raconté à son sujet ? Dites…\par
– Ils ne m’ont pas dit grand-chose sous aucun rapport à son sujet, j’ai seulement entendu dire qu’il était un excellent chasseur de baleines et un bon capitaine pour son équipage.\par
– C’est vrai, c’est vrai – oui, ces deux choses sont également vraies. Mais il s’agit de bondir lorsqu’il donne un ordre. Marche et grogne, grogne et marche, telle est la formule du capitaine Achab. Mais on ne vous a rien dit de ce qui lui était arrivé au large du cap Horn, voici longtemps, et comment il resta comme mort pendant trois jours et trois nuits ; rien non plus de cette lutte à mort menée devant l’autel avec l’Espagnol à Santa ? Vous n’avez rien entendu de tout cela, n’est-ce pas ? Rien non plus au sujet de la calebasse d’argent dans laquelle il a craché ? Rien encore de la perte de sa jambe conformément à la prophétie ? Avez-vous entendu dire quoi que ce soit de ces choses et de bien d’autres encore ? Non, je le pensais bien, comment les auriez-vous apprises ? Qui le sait ? Pas une âme à Nantucket, j’imagine. Peut-être pourtant avez-vous entendu parler de sa jambe et de la manière dont il l’a perdue ? Oui, vous en avez entendu parler, j’oserais l’affirmer. Oh ! oui, ça, tout le monde le sait – je veux dire que tout le monde sait qu’il n’a qu’une jambe et qu’un cachalot lui a arraché l’autre.\par
– Mon ami, dis-je, je ne sais pas à quoi rime tout votre charabia et je m’en soucie peu du reste, car il me semble que vous devez avoir le cerveau légèrement atteint. Mais si vous parlez du capitaine Achab, de ce navire-là, le {\itshape Péquod}, alors laissez-moi vous dire que je sais tout au sujet de la perte de sa jambe.\par
– Tout, hein ? Êtes-vous sûr de tout savoir ? Vraiment tout ?\par
– Joliment sûr.\par
L’index toujours pointé, les yeux levés vers le {\itshape Péquod}, l’étranger loqueteux sombra un moment dans une inquiète rêverie ; puis il se tourna en tressaillant et ajouta : « Vous vous êtes enrôlés, alors c’est vrai ? Vos noms figurent sur le rôle ? Eh bien ! Eh bien ! ce qui est signé est signé. Ce qui doit être sera. Et encore cela ne sera peut-être pas après tout… De toute façon tout est écrit, prévu et j’imagine qu’il faut bien qu’il y ait des marins pour partir avec lui, autant vous que d’autres, que Dieu les ait en pitié ! Bien le bonjour à vous, camarades, bien le bonjour. Que l’Ineffable vous bénisse ; je m’excuse de vous avoir retardés.\par
– Écoutez un peu, ami, dis-je, si vous avez quoi que ce soit d’important à nous dire, sortez-le ! Mais si vous essayez de nous embobiner, alors vous perdez votre temps, je n’ai rien à ajouter !\par
– Et c’est très bien dit, j’aime entendre des gars s’exprimer de cette manière-là ; vous êtes exactement les hommes qu’il lui faut… vous et vos pareils. Bien le bonjour, camarades, bien le bonjour ! Oh ! quand vous y serez, dites-leur bien que j’ai décidé de ne pas être des leurs…\par
– Ah ! mon vieux, vous ne pouvez pas nous berner de la sorte… vous ne pouvez pas nous duper ! Il n’y a rien au monde de plus facile à un homme que de faire croire qu’il détient un grand secret.\par
– Bien le bonjour, camarades, bonjour…\par
– Et c’est le jour en effet, dis-je, venez Queequeg, quittons cet homme fou à lier. Mais un instant, dites-moi comment vous vous appelez, voulez-vous ?\par
– Élie.\par
– Élie ! pensai-je. Et nous partîmes, commentant, chacun à notre façon, les propos de ce vieux marin en haillons, nous tombâmes d’accord pour trouver que c’était un donneur d’eau bénite jouant au loup-garou. Mais nous n’avions pas fait cent mètres que, venant à passer l’angle de rue et à me retourner, je vis Élie nous suivant, bien qu’à une certaine distance. Pour je ne sais quelle raison cela me frappa à tel point que je n’en informai pas Queequeg, mais allai de l’avant avec lui, soucieux de savoir si l’étranger allait prendre la même rue que nous. C’est ce qu’il fit. Il me sembla alors qu’il nous surveillait mais, sur ma vie, je ne pus deviner son intention. Ce fait, venant se greffer sur son discours voilé, ambigu, à demi allusif, à demi révélateur, engendrait à présent en moi toutes sortes de vagues points d’interrogation, de demi-appréhensions, tout cela lié avec le {\itshape Péquod}, le capitaine Achab, cette jambe qu’il avait perdue, cet accès qu’il avait eu au cap Horn, la calebasse d’argent, et ce que le capitaine Peleg m’en avait dit, lorsque j’avais quitté le navire la veille, et les prédictions de Tistig, la squaw, et ce voyage pour lequel nous nous étions liés, et cent autres choses sous le signe de l’ombre.\par
Que cet Élie déguenillé nous épiât ou non, j’étais décidé à en avoir le cœur net, et dans cette intention je traversai la rue avec Queequeg et rebroussai chemin. Mais Élie passa sans paraître nous voir, je me sentis soulagé. Et une fois de plus, et définitivement, du moins je le crus, je le taxai, en mon âme, de charlatan.
\chapterclose


\chapteropen
\chapter[{CHAPITRE XX. Grande animation}]{CHAPITRE XX \\
Grande animation}\renewcommand{\leftmark}{CHAPITRE XX \\
Grande animation}


\chaptercont
\noindent Un jour ou deux passèrent et une activité débordante régnait sur le {\itshape Péquod}. Non seulement on réparait les vieilles voiles, mais on en embarquait des neuves, ainsi que des pièces de toiles et des glènes de filin. Bref, tout disait que l’armement du navire tirait à sa fin. Le capitaine Peleg ne descendait pour ainsi dire jamais à terre, mais restait assis dans son wigwam, surveillant étroitement les hommes ; Bildad s’employait à tous les achats et à la fourniture du matériel ; les hommes occupés dans la cale ou au gréement travaillaient très avant dans la nuit.\par
Le lendemain du jour où Queequeg eut signé son engagement, toutes les auberges où étaient descendus les membres de l’équipage reçurent le mot d’ordre d’amener les coffres à bord avant la nuit car on ne pouvait prévoir quand le navire lèverait l’ancre. De sorte que Queequeg et moi embarquâmes nos atours, résolus toutefois à dormir à terre jusqu’à la dernière minute. Mais il semble que les avertissements soient donnés très longtemps à l’avance, en pareil cas, car le navire n’appareilla pas de plusieurs jours. Il n’y a rien là d’étonnant, il y avait beaucoup à faire et le nombre de choses auquel il convenait de penser pour que l’armement du {\itshape Péquod} fût complet est incalculable.\par
Tout un chacun sait quelle multitude d’objets s’avèrent indispensables à la tenue d’une maison : lits, casseroles, couteaux et fourchettes, pelles et pincettes, serviettes, casse-noix, que sais-je encore ? Il en va de même sur un baleinier qui réclame pendant trois ans la tenue d’un ménage au milieu du \hspace{1em}vaste Océan, loin des épiciers, des marchands des quatre saisons, des médecins, des boulangers et des banquiers. Et si cela est vrai pour l’équipement d’un navire marchand, il n’y a pas de commune mesure avec l’armement réclamé par un baleinier. Car indépendamment de la durée prolongée du voyage, le matériel de pêche à lui seul se compose de nombreux objets introuvables dans les ports lointains ordinairement fréquentés. De plus, il faut s’en souvenir, de tous les navires, les baleiniers sont les plus exposés aux accidents de toute nature, plus particulièrement à la destruction et à la perte des engins dont dépend le succès du voyage, c’est pourquoi il y faut des pirogues de rechange, des avirons de rechange, des lignes et des harpons de rechange et de tout de rechange ou presque, hormis un capitaine de rechange et un navire de rechange.\par
Au moment de notre arrivée dans l’île, le {\itshape Péquod} avait déjà embarqué le plus gros de son chargement, y compris son bœuf salé, son pain, son eau, ses feuillards en fer et ses futailles en botte. Mais, comme je l’ai déjà dit, un va-et-vient perpétuel se prolongea longuement entre la terre et le navire pour l’embarquement de choses et d’autres, tant grandes que petites.\par
La personne la plus affairée à ces transports était la sœur du capitaine Bildad, une vieille dame maigre, à l’esprit on ne peut plus résolu et infatigable, de bon cœur en outre, qui paraissait décidée, si c’était en son pouvoir, à ce que rien ne manquât à bord, une fois que le {\itshape Péquod} serait engagé en haute mer. Une fois, elle arrivait avec un pot de légumes marinés pour l’office du cambusier ; une autre, avec un bouquet de plumes pour le bureau où le second tenait son journal de bord ; une autre encore, avec une pièce de flanelle destinée au creux d’un dos rhumatisant. Jamais femme ne mérita si bien son nom de Charité, tante Charité comme chacun disait. Et telle une sœur de charité, cette charitable tante Charité allait et venait de-ci de-là, prête à se mettre à l’œuvre, corps et âme, pour tout ce qui pourrait assurer la sécurité, le confort et la consolation de tous sur un \hspace{1em} navire où son bien-aimé frère Bildad avait ses intérêts, et sur lequel elle-même possédait une ou deux vingtaines de dollars d’économies.\par
Il était saisissant de voir cette quakeresse au grand cœur monter à bord, comme elle le fit le dernier jour, avec une longue cuillère pour les pots dans une main et une lance plus longue encore dans l’autre. Bildad et le capitaine Peleg ne restaient pas à la traîne. Bildad se promenait avec une interminable liste des objets nécessaires et à chaque arrivage nouveau il mettait une coche face à l’objet désigné sur son papier. De temps à autre, Peleg se dégageait de son antre de fanons, rugissant contre les hommes dans les écoutilles, rugissant vers les gréeurs à la tête du mât, et terminant ses rugissements à l’intérieur de son wigwam.\par
Au cours de ces jours de préparatifs, Queequeg et moi montâmes souvent sur le navire, et chaque fois je demandais des nouvelles du capitaine Achab, et quand il rejoindrait le bord. À ces questions, il était répondu qu’il allait de mieux en mieux et qu’on l’attendait d’un jour à l’autre ; entre-temps, les deux capitaines Peleg et Bildad pouvaient s’occuper de tout ce qui était nécessaire au bateau pour prendre la mer. Si j’avais été parfaitement sincère avec moi-même, j’aurais vu très clairement dans mon cœur que cela ne me plaisait qu’à moitié d’être engagé de cette manière pour accomplir un aussi long voyage, sans avoir une seule fois aperçu l’homme qui serait maître absolu du navire dès que celui-ci aurait gagné la haute mer. Mais lorsqu’un homme pressent qu’il s’est embarqué dans une mauvaise affaire, il lui arrive de lutter inconsciemment pour se cacher à lui-même ses soupçons. C’était bien à peu près ce que je faisais. Je ne disais rien et j’essayais de ne penser à rien.\par
Enfin, on annonça que le navire appareillerait probablement le lendemain. Aussi Queequeg et moi nous nous levâmes de bonne heure ce jour-là.
\chapterclose


\chapteropen
\chapter[{CHAPITRE XXI. Nous embarquons}]{CHAPITRE XXI \\
Nous embarquons}\renewcommand{\leftmark}{CHAPITRE XXI \\
Nous embarquons}


\chaptercont
\noindent Il était près de six heures lorsque nous approchâmes du quai, mais l’aube grise était incertaine et brumeuse.\par
– Il y a des marins qui courent devant nous, si je vois bien, dis-je à Queequeg, ce ne peuvent pas être des ombres, je pense que nous allons lever l’ancre avec le soleil. Dépêchez-vous !\par
– Arrière ! s’écria une voix, et celui à qui elle appartenait fut en même temps dans notre dos et posa une main sur nos épaules respectives puis, se glissant entre nous, il s’arrêta, légèrement penché en avant et, dans cette lumière trouble, il nous regarda tour à tour d’étrange manière. C’était Élie.\par
– Vous embarquez ?\par
– Bas les pattes je vous prie, dis-je.\par
– Écoutez, vous, dit Queequeg en se secouant, vous, allez !\par
– Alors vous n’embarquez pas ?\par
– Oui, nous embarquons, mais en quoi cela vous regarde-til ? dis-je. Savez-vous, monsieur Élie, que je vous considère comme légèrement indiscret.\par
– Non, non, non, je ne m’en rendais pas compte, dit Élie, nous regardant alternativement Queequeg et moi d’un regard inexprimable, lent, étonné.\par
– Élie, dis-je, vous nous obligeriez, mon ami et moi, en vous retirant. Nous partons pour l’océan Indien et l’océan Pacifique et nous préférerions n’être pas retenus.\par
– Vous y allez vraiment ? Et vous revenez avant le petit déjeuner ?\par
– Il est toqué, Queequeg, dis-je, venez !\par
– Holà ! cria Élie, immobile, nous interpellant alors que nous avions fait quelques pas.\par
– Ne nous occupons pas de lui, lui dis-je, venez Queequeg.\par
Mais il nous rattrapa à nouveau et, me tapant soudain sur l’épaule, il dit : « N’avez-vous pas vu quelque chose qui ressemblait à des hommes se dirigeant vers le bateau il y a un mo- ment ? »\par
Frappé par cette question simple et terre à terre, je répondis : « Oui, j’ai cru voir quatre ou cinq hommes, mais il faisait trop sombre pour que j’en sois sûr. »\par
– Très sombre, très sombre, dit Élie. Bien le bonjour à vous !\par
Une fois de plus, nous le quittâmes, mais une fois de plus il vint en tapinois derrière nous, et me touchant une fois de plus à l’épaule, il dit :\par
– Vous tâcherez d’essayer de les retrouver à présent, n’estce pas ?\par
– Retrouvez qui ?\par
– Bien le bonjour à vous, bien le bonjour ! entonna-t-il une fois de plus, puis s’éloignant, il ajouta : Oh ! j’allais vous mettre en garde contre… mais peu importe, peu importe… c’est tout un, tout en famille aussi. Il gèle dur ce matin, n’est-ce pas ? Adieu à vous ! Je ne vous reverrai pas de sitôt, m’est idée, à moins que ce ne soit devant le Grand Tribunal. Sur ces divagations, il partit enfin, me laissant, sur l’instant, stupéfait d’une impudence aussi éhontée.\par
À bord du {\itshape Péquod}, régnait un profond silence, pas une âme ne bougeait. La porte de la chambre était fermée de l’intérieur, les panneaux d’écoutilles étaient tous en place et encombrés de glènes de filins. Nous dirigeant vers le gaillard d’avant, nous trouvâmes le panneau de l’écoutillon ouvert. Voyant une lumière, nous descendîmes et ne découvrîmes qu’un vieux gréeur, enveloppé dans une vareuse d’ordonnance en lambeaux. Couché de tout son long sur deux coffres, à plat ventre, le visage enfoui entre ses bras croisés, il dormait du plus profond sommeil.\par
– Ces marins que nous avons vus, Queequeg, où peuventils avoir passé ? demandai-je en regardant ce dormeur d’un air soupçonneux. Mais il apparut que, lorsque nous étions sur le quai, Queequeg n’avait rien remarqué de ce à quoi je faisais allusion à présent, aussi j’aurais cru à une illusion d’optique, n’eût été la question, inexplicable alors, posée par Élie. Mais je chassai cette pensée et, jetant un regard sur le dormeur, je suggérai facétieusement à Queequeg de veiller le corps, l’invitant à s’installer à cette fin. Il posa la main sur l’arrière-train de l’homme, comme pour tâter s’il était assez moelleux, puis, sans autre forme de procès, il s’y assit tranquillement.\par
– Miséricorde ! Queequeg ne vous asseyez pas là, lui dis-je.\par
– Oh ! joli bon siège, dit Queequeg, mon pays coutume. Fera pas mal figure lui.\par
– Sa figure ! dis-je, vous appelez ça sa figure ? Elle a une mine fort bienveillante alors ; mais comme il a de la peine à respirer ; levez-vous, Queequeg, vous êtes lourd, vous lui écrasez la figure, à ce pauvre. Debout, Queequeg ! Il va vous secouer de là, bientôt. Je m’étonne qu’il ne se réveille pas.\par
Queequeg s’installa alors juste derrière la tête du dormeur et alluma son tomahawk-pipe. Je m’assis aux pieds. Nous nous passions la pipe par-dessus le dormeur. Tout en fumant, répondant à sa manière incohérente aux questions que je lui avais posées, Queequeg me laissa entendre que, dans son pays, en l’absence de causeuses et de sofas d’aucune espèce, les rois, les chefs et tous les grands en général avaient coutume d’engraisser quelques individus des basses classes en guise de divans, et de meubler ainsi une maison très confortablement, vous n’aviez qu’à acheter huit ou dix paresseux et à les disposer dans les trumeaux et les alcôves. D’autre part, ils présentaient de gros avantages lors d’une promenade, ils étaient de beaucoup supérieurs à ces chaises de jardin qu’on peut replier afin qu’elles servent de cannes car, lorsque l’occasion s’en présente, le chef peut transformer son escorte en canapé s’il désire s’asseoir sous un arbre croissant peut-être dans un sol humide ou marécageux.\par
Tout en racontant son histoire, chaque fois que je lui repassais la pipe, Queequeg en brandissait le côté hache sur la tête du dormeur.\par
– Pourquoi faites-vous cela, Queequeg ?\par
– Joli facile touer lui, oh ! joli facile…\par
Il évoquait de sauvages souvenirs au sujet de sa pipetomahawk qui semblait avoir eu la double fonction de défoncer le crâne de ses ennemis et d’apaiser son âme, lorsque le gréeur endormi attira notre attention. La fumée acre avait complètement rempli ce trou resserré et commençait à agir sur lui ; il respirait comme s’il avait été bâillonné, puis il présenta des troubles nasaux, se tourna enfin une ou deux fois, puis s’assit et se frotta les yeux.\par
– Holà ! souffla-t-il finalement, qui êtes-vous, fumeurs ?\par
– Des enrôlés, répondis-je, quand est-ce qu’il part ?\par
– Oui, oui, vous partez avec, hé ? Il part aujourd’hui. Le capitaine a rejoint le bord la nuit dernière.\par
– Quel capitaine ? Achab ?\par
– Qui sinon lui ?\par
J’allais lui poser d’autres questions au sujet d’Achab lorsque nous entendîmes un bruit sur le pont.\par
– Holà ! Starbuck est debout, dit le gréeur. C’est un second actif, celui-là, bon diable et pieux, mais tout à fait actif à présent, faut que j’aille. Sur ce, il monta sur le pont où nous le suivîmes.\par
Le soleil se levait. Les hommes arrivaient par groupes de deux ou de trois, les gréeurs se démenaient, les seconds s’affairaient et plusieurs hommes à quai s’empressaient à embarquer encore les derniers colis. Pendant ce temps, le capitaine Achab demeurait invisible, enchâssé dans sa cabine.
\chapterclose


\chapteropen
\chapter[{CHAPITRE XXII. Joyeux Noël}]{CHAPITRE XXII \\
Joyeux Noël}\renewcommand{\leftmark}{CHAPITRE XXII \\
Joyeux Noël}


\chaptercont
\noindent Enfin, vers midi, après qu’on eut renvoyé les gréeurs à terre, après que le {\itshape Péquod} eut été remorqué loin de l’estacade, et que la prévenante Charité ait fait sa dernière apparition dans une baleinière avec son dernier présent : un bonnet de nuit pour Stubb, le deuxième second, son beau-frère, et une Bible de rechange pour le cambusier, après tout cela, les deux capitaines Peleg et Bildad sortirent de la cabine et Peleg demanda, en se tournant vers le second :\par
– À présent, monsieur Starbuck, êtes-vous sûr que tout soit en ordre ? Le capitaine Achab est parfaitement prêt, je viens de lui parler, plus rien à prendre à terre, hein ? Alors, appelez tout l’équipage, rassemblement à l’arrière… le diable les emporte !\par
– Inutile de blasphémer, même si le temps presse, Peleg, dit Bildad, mais va, ami Starbuck, et exécute notre ordre.\par
Quoi ! Jusqu’à l’extrême dernière minute avant le départ, le capitaine Peleg et le capitaine Bildad gardaient la haute main sur le commandement comme s’ils allaient être co-officiers en mer comme ils semblaient l’être au port. Quant au capitaine Achab, rien ne trahissait son existence, mais il était, disaient-ils, dans sa cabine. Tout donnait à penser que sa présence n’était à aucun égard nécessaire aux préparatifs d’appareillage, ni pour mener le navire en haute mer. En vérité, cela ne le concernait en rien, c’était l’affaire du pilote ; d’autre part, comme il n’était pas tout à fait remis – disaient-ils – le capitaine Achab restait dans sa cabine. Tout cela paraissait assez naturel. Dans la marine marchande en particulier, nombre de capitaines ne se montrent sur le pont que fort longtemps après avoir levé l’ancre, mais restent à la table de la cabine, célébrant leur départ avec leurs amis qui demeurent à terre avant qu’ils ne quittent tout de bon le navire avec le pilote.\par
Mais les chances de pouvoir méditer étaient minces car le capitaine Peleg débordait à présent d’activité ; il semblait que ce fût lui et non Bildad qui tînt le plus de discours et donnât le plus d’ordres.\par
– À l’arrière, eh vous fils de célibataires ! criait-il aux hommes qui s’attardaient au grand-mât. Monsieur Starbuck, expédiez-les à l’arrière. Abattez la tente, là ! fut l’ordre suivant. Comme je l’ai déjà dit, cette marquise en fanons n’était jamais dressée qu’au port, et à bord du {\itshape Péquod ;} depuis trente ans, on savait bien que l’ordre d’abattre la tente suivait immédiatement celui de lever l’ancre.\par
– Armez le cabestan ! Sang et tonnerre ! Sautez ! Et les hommes se ruèrent aux anspects.\par
Lorsqu’un navire fait effort pour déraper son ancre, la place généralement occupée par le pilote est à l’avant du navire. C’est là que se tenait Bildad qui, tout comme Peleg – qu’on se le dise – en plus de ses autres qualifications, était l’un des pilotes brevetés du port ; on le soupçonnait d’avoir acquis ce brevet à seule fin d’économiser la redevance qu’il aurait dû verser à un pilote de Nantucket pour tous les navires où il avait des intérêts, car on ne l’avait jamais vu piloter d’autres bâtiments. On pouvait donc voir Bildad très occupé à regarder par-dessus la proue pour surveiller l’apparition de l’ancre, chantant par moments un couplet de ce qui semblait être un psaume lugubre destiné à encourager les hommes au guindeau qui, eux, braillaient avec conviction un refrain où il était question des filles de Booble Alley. Il n’y avait pas trois jours, toutefois, que Bildad leur avait dit qu’aucune chanson profane ne serait tolérée à bord du {\itshape Péquod}, et moins que jamais au moment où l’on lèverait l’ancre ; Charité sa sœur avait déposé sur la couchette de chaque marin un fascicule de morceaux choisis de Watts.\par
Pendant ce temps, surveillant l’arrière du navire, Peleg se démenait et jurait de la plus terrifiante manière. Je fus sur le point de croire qu’il allait envoyer le bateau par le fond avant qu’on ait pu amener l’ancre ; involontairement j’immobilisai mon anspect et conseillai à Queequeg de faire de même, songeant aux dangers que nous courions tous deux avec un diable pareil comme pilote. Je me consolai cependant à la pensée que le pieux Bildad assurerait notre salut en dépit de sa 777\textsuperscript{e} part, lorsque je sentis soudain un coup brutal me défoncer le derrière, me retournant, je fus horrifié de surprendre le capitaine Peleg en flagrant délit de retirer son pied. Ce fut mon premier coup de pied au cul.\par
– C’est comme ça qu’ils virent sur la chaîne dans la marine marchande, hein ? rugit-il ; saute, tête de veau, saute, romps-toi l’échine ? Pourquoi ne sautez-vous pas, dis-je, vous tous – allons ! Quohog ! Saute. Toi et tes favoris rouges, saute ; toi, le béret écossais, que ça barde ; toi, remue tes pantalons verts ; sautez, vous tous, dis-je, et que les yeux vous sautent de la tête ! » Ce disant, il se promenait près du guindeau, en faisant ici et là un usage très libéral de son pied, tandis que l’imperturbable Bildad psalmodiait toujours. Le capitaine Peleg, me disais-je, a dû boire un coup aujourd’hui.\par
Enfin, l’ancre fut levée, les voiles hissées et nous glissâmes sur l’eau. C’était un jour de Noël aigre et froid, et tandis que la courte journée nordique se fondait déjà en nuit, nous nous trouvâmes presque en plein océan hivernal dont l’écume gelée nous saisissait dans une étincelante armure de glace. Les longues rangées de dents des pavois luisaient au clair de lune, et pareils aux défenses d’ivoire de quelque gigantesque éléphant, d’immenses glaçons se recourbaient à la proue.\par
En tant que pilote, le décharné Bildad était chef du premier quart et, de temps en temps, tandis que le vieux bâtiment plongeait profondément dans la verte mer et en faisait jaillir les paillettes d’une gerbe qui le givrait tout entier, et tandis que le vent mugissait et que vibraient les cordages, on l’entendait chanter assidûment :\par
Au-delà des flots qui enflent, les douces prairies Se tiennent prêtes en éclatantes robes vertes Ainsi apparut aux Juifs le vieux pays de Chanaan Dont les flots du Jourdain les séparaient encore… \par
Jamais douces paroles ne furent plus douces à mes oreilles qu’alors. Elles chantaient l’espérance et l’abondance. En dépit de cette glaciale nuit d’hiver sur le rude Atlantique, en dépit de mes pieds mouillés et de ma vareuse détrempée, il y avait pourtant, me sembla-t-il alors, plus d’un port accueillant à m’attendre ; et des prés et des clairières où un printemps éternel gardait intacte, jusqu’à la mi-été, la jeunesse d’une herbe fraîche et vierge de pas.\par
Enfin nous fûmes assez au large pour que les pilotes ne fussent plus nécessaires. La robuste embarcation à voiles qui avait navigué de conserve avec nous nous aborda.\par
Il était curieux et point déplaisant de voir à quel point ce moment critique affectait Peleg et Bildad, plus particulièrement le capitaine Bildad. Il avait le regret du départ, le regret très profond de quitter pour de bon un navire appareillant pour un si long, si périlleux voyage, au-delà des deux caps \hspace{1em} tempétueux, un bateau sur lequel étaient investis quelques milliers de ses dollars durement acquis, un bâtiment qui avait pour capitaine son ancien camarade de bord, un homme presque aussi vieux que lui, partant une fois de plus à la rencontre de toutes les terreurs d’une mâchoire sans merci, il avait le regret de dire au revoir à tout ce qui, pour lui, débordait d’intérêts de toute nature et le pauvre vieux Bildad s’attarda longuement ; il arpenta le pont à grands pas inquiets, descendit en courant jusqu’à la cabine pour y dire un dernier mot d’adieu, remonta sur le pont et regarda au vent, regarda vers l’étendue sans fin de la vaste mer que nulle terre ne limitait plus jusqu’aux lointains et invisibles continents de l’est ; regarda vers la côte et regarda le ciel ; regarda à droite et puis à gauche ; regarda partout et nulle part, amarra machinalement une manœuvre courante à son cabillot, saisit convulsivement la main du vigoureux Peleg, et levant un fanal, le regarda héroïquement pendant un moment en plein visage comme pour dire : « Pourtant, ami Peleg, je peux supporter cela, oui je le peux. »\par
Quant à Peleg, lui, il prenait la chose avec plus de sagesse, mais malgré toute sa philosophie, le fanal approché de trop près révéla une larme brillante. Lui aussi, il courut de la cabine au pont : un mot en bas, un mot à Starbuck le second.\par
Puis enfin, après un dernier regard circulaire, il se tourna vers son compère : « Capitaine Bildad, allons vieux camarade, il nous faut partir. Coiffez la grande vergue ! Ohé du canot ! Parez à l’accostage ! Doucement, doucement ! Allons, Bildad, mon garçon, dis ton dernier mot. Bonne chance Starbuck, bonne chance Monsieur Stubb, bonne chance Monsieur Flask – au revoir et bonne chance à vous tous – et dans trois ans jour pour jour un souper fumant vous attendra dans le vieux Nantucket. Hourra et en route ! – Dieu vous bénisse et vous ait en sa sainte garde, murmura le vieux Bildad de façon presque inintelligible, j’espère que vous aurez du beau temps afin que le capitaine Achab soit bientôt parmi vous ; un beau soleil, voilà tout ce dont il a besoin et vous en aurez largement dans les tropiques vers lesquels vous partez. Soyez prudents dans votre chasse, vous les seconds. Ne maltraitez pas inutilement les pirogues, vous les harponneurs, les bons bordés de cèdre blanc ont monté de trois pour cent au cours de l’année. N’oubliez pas non plus vos prières. Monsieur Starbuck, veillez à ce que le tonnelier ne gaspille pas les douvelles de rechange. Oh ! les aiguilles à voiles sont dans le coffre vert ! Ne chassez pas trop les jours du Seigneur, hommes, mais ne laissez pas non plus passer une belle occasion, ce serait repousser les dons du Ciel. Jetez un œil sur le tierçon de mélasse, Monsieur Stubb, il coule un peu, je crains. Si vous relâchez dans les îles, Monsieur Flask, méfiez-vous de la fornication. Au revoir, au revoir ! Ne gardez pas ce fromage trop longtemps dans la cale, Monsieur Starbuck, il s’abîmerait. Économisez le beurre, vingt cents la livre il coûte, et prenez garde, si…\par
– Allons, allons, capitaine Bildad assez palabré, en route ! sur ces mots, Peleg le pressa de passer la muraille et tous deux sautèrent dans l’embarcation.\par
Le navire et la chaloupe s’écartèrent et entre eux s’engouffra le vent humide et froid de la nuit, un goéland les survola en criant ; les deux coques roulèrent sauvagement. Le cœur lourd, nous poussâmes trois hourras et, pareils au destin, nous plongeâmes aveuglément dans la solitude océane.
\chapterclose


\chapteropen
\chapter[{CHAPITRE XXIII. Terre sous le vent}]{CHAPITRE XXIII \\
Terre sous le vent}\renewcommand{\leftmark}{CHAPITRE XXIII \\
Terre sous le vent}


\chaptercont
\noindent Dans un chapitre précédent, il a été question d’un dénommé Bulkington, un marin de haute stature, fraîchement débarqué et rencontré à l’auberge de New Bedford.\par
Tandis que, par cette nuit d’hiver, cassante de gel, le {\itshape Péquod} pourfendait d’une étrave vindicative la vague maligne et froide, qui vis-je à sa barre sinon Bulkington ! Je regardai avec une sympathie et une crainte respectueuse l’homme qui, en plein hiver, à peine rentré d’un dangereux voyage de quatre ans pouvait, sans répit, repartir aussitôt vers la tempête et de nouveaux dangers. La terre semblait lui brûler les pieds. Les plus étonnantes merveilles sont à jamais ineffables ; les souvenirs profonds ne demandent point d’épitaphes, ce chapitre, long de six pouces, est la tombe sans marbre de Bulkington. Laissez-moi dire seulement qu’il en alla pour lui comme pour le navire secoué par l’ouragan qui longe misérablement la terre sous le vent. Le port ne serait que trop heureux d’accorder son secours, le port est compatissant ; le port assure la sécurité, le confort, la pierre du foyer, le souper, la chaude couverture, des amis, tout ce qui dispense la douceur à notre faiblesse. Mais dans ce vent de tempête le port et la terre sont les pires dangers qui guettent ce navire, il lui faut fuir toute hospitalité ; sa quille viendrait-elle à effleurer la terre, qu’un grand frisson le secouerait de part en part. De toute sa puissance il doit forcer de voiles pour s’éloigner des rivages, et ce faisant lutter contre les vents mêmes qui voudraient le ramener vers eux, chercher la mer cinglante, vide de toute terre, et pour survivre se précipiter avec désolation dans le péril son seul ami, son plus implacable ennemi !\par
Le sais-tu, Bulkington ? Tu semblais avoir entrevu cette vérité, mortellement intolérable, qu’une pensée profonde et sincère n’est que l’intrépide effort de l’âme pour sauver l’indépendance sans limites de son propre océan cependant que les vents les plus furieux, soufflant de terre et de mer, conspirent pour la rejeter à la côte traîtresse et servile ?\par
Mais, comme loin de toute terre seulement, demeure la vérité la plus haute, sans rivages, et comme Dieu illimitée, mieux vaut périr dans cet infini hanté de clameurs, que d’échouer honteusement à la sécurité de la terre sous le vent ! Tout ver de terre que nous sommes, lequel d’entre nous se sentirait l’ardent désir de ramper ! Épouvante de l’épouvante ! tant d’agonie serait-elle vaine ? Courage, courage, ô Bulkington ! Bats-toi avec acharnement, demi-dieu ! Ton apothéose jaillit tout droit de l’écume soulevée par ta mort océane.
\chapterclose


\chapteropen
\chapter[{CHAPITRE XXIV. Le plaideur}]{CHAPITRE XXIV \\
Le plaideur}\renewcommand{\leftmark}{CHAPITRE XXIV \\
Le plaideur}


\chaptercont
\noindent Étant donné que Queequeg et moi nous sommes désormais embarqués pour la pêche à la baleine et parce que celle-ci est aujourd’hui considérée par les terriens comme une occupation honteuse et dépourvue de poésie, je désire vous convaincre, vous terriens, de l’injustice qui nous est ainsi faite à nous autres, chasseurs de baleines.\par
Bien qu’il paraisse presque superflu d’insister sur ce fait, il convient de relever tout d’abord que, parmi le grand public, la chasse à la baleine n’a pas place dans les professions libérales. Si un étranger se présentait dans une société mélangée de la métropole, il ne rehausserait guère l’opinion générale sur ses mérites, s’il se présentait en tant que harponneur, par exemple. Et si, à l’instar des officiers de marine, il faisait figurer sur sa carte de visite les initiales P. A. C. (pêche au cachalot), un tel procédé paraîtrait présomptueux et ridicule à l’extrême.\par
Sans doute la raison majeure pour laquelle le monde refuse de nous rendre hommage à nous baleiniers est-elle celle-ci : les gens pensent, en mettant les choses au mieux, que notre profession n’est qu’une sorte de tuerie, et qu’une fois engagés activement dans cette affaire, tout n’est plus que souillures autour de nous. Nous sommes des bouchers il est vrai. Mais furent aussi des bouchers, des bouchers portant les plus sanglantes médailles, tous les chefs militaires que le monde, invariablement, se délecte à honorer. Quant à la charge de saleté que l’on fait peser sur notre travail, vous serez bientôt éclairés sur certains faits, jusqu’ici à peu près généralement inconnus, et qui, tout compte fait, situeront, de façon éclatante, le navire baleinier parmi les choses les plus propres de cette proprette terre. Mais, même en acceptant comme véridique cette allégation, quel pont visqueux et en désordre d’un baleinier est comparable au charnier répugnant de ces champs de bataille dont tant de soldats reviennent pour boire aux applaudissements de toutes ces dames ? Et si la notion de danger ajoute un prestige populaire tel au métier de soldat, permettez-moi de vous affirmer que plus d’un vétéran qui s’était pourtant lancé d’un cœur léger à l’assaut d’une batterie, fuirait rapidement à l’apparition de la queue géante du cachalot battant l’air en tourbillons au-dessus de sa tête. Car, quel rapport peut-on établir entre les terreurs engendrées par l’homme et accessibles à son entendement et ces prodiges d’épouvante réalisés par Dieu !\par
Au reste, bien que le monde nous repousse avec mépris nous autres baleiniers, il nous rend, inconsciemment, le plus profond hommage et même une adoration généreuse ! En effet, comme sur autant d’autels, et d’un bout à l’autre de la terre, presque tous les cierges, les lampes et les bougies brûlent à notre gloire !\par
Mais considérez cette question sous d’autres angles, pesezla dans les plus diverses balances et voyez ce que nous sommes et ce que nous avons été, nous autres baleiniers.\par
Pourquoi la Hollande, sous l’administration des Witt, eutelle des amiraux pour ses flottilles de baleiniers ? Pourquoi Louis XVI de France arma-t-il à ses propres frais des navires baleiniers à Dunkerque, invitant courtoisement dans cette ville quelque vingt ou quarante familles de notre propre île de Nantucket ? Pourquoi, entre 1750 et 1788, la Grande-Bretagne versa-t-elle à ses baleiniers des primes se montant à 1 000 000 de livres ? Et finalement comment se fait-il que nous, baleiniers d’Amérique, nous surpassions à présent en nombre la totalité des autres baleiniers du monde ; que nous ayons une flotte de plus de 700 navires menés par 18 000 hommes, dépensant annuellement 4 000 000 de dollars, des navires valant à l’appareillage 20 000 000 de dollars et ramenant chaque année dans nos ports une moisson bien gagnée de 7 000 000 de dollars. Que signifie tout cela sinon la puissance inhérente à la pêche à la baleine ?\par
Et ce n’est pas même la moitié des arguments. Voyez encore.\par
Je soutiens qu’un philosophe universel ne peut pas, quand bien même il y irait de sa vie, relever une seule influence pacificatrice ayant eu, au cours des soixante dernières années, un pouvoir aussi étendu sur le monde entier, pris en bloc, que cette noble et grandiose chasse à la baleine. D’une façon ou d’une autre, elle a entraîné des événements si remarquables en euxmêmes, si importants dans leurs conséquences permanentes, qu’on peut bien la comparer à cette mère égyptienne qui mettait au monde des filles déjà gravides à leur naissance. Ce serait une besogne sans fin et sans espoir que d’établir un répertoire de ses mérites. Contentons-nous d’une poignée : pendant bien des années le navire baleinier a été le pionnier de la découverte des terres les plus lointaines et les moins connues. Il a exploré des mers et des mers et des archipels ne se trouvant sur aucune carte et où aucun Vancouver, aucun Cook n’étaient allés. Si les navires de guerre américains et européens entrent paisiblement dans des ports autrefois barbares, qu’ils tirent des salves en l’honneur et à la gloire du baleinier qui le premier leur a ouvert la route et leur a servi d’interprète auprès des sauvages. Qu’on célèbre tant qu’on voudra les héros des expéditions d’explorations, vos Cook, vos Krusenstern, moi je vous dis que bien des capitaines anonymes qui sont partis de Nantucket étaient aussi grands, et plus grands, que votre Cook et votre Krusenstern car, avec leurs mains vides et démunies, dans des eaux païennes infestées de requins, et sur les rivages d’îles inconnues où les attendaient des javelots, ils combattaient de neuves merveilles et des terreurs que Cook n’eût pas volontiers défiées avec tous ses marins et tous ses mousquets. Tout ce que montent en épingle les vieux récits de voyage dans les mers du Sud n’était que les banalités de la vie quotidienne de nos héroïques Nantuckais. Souvent des aventures auxquelles Vancouver consacre trois chapitres n’auraient pas paru à ces hommes dignes d’être mentionnées dans leur journal de bord. Ah ! le monde ! Oh ! le monde !\par
Aucun commerce, presque aucune communication ne s’établirent entre l’Europe et le long territoire des opulentes provinces espagnoles de la côte Pacifique jusqu’à ce que les baleiniers aient doublé le cap Horn. Ce fut le chasseur de baleine qui, le premier, fit une brèche dans la politique jalouse de la couronne d’Espagne envers ses colonies, et si l’espace ne m’était pas mesuré, je pourrais prouver clairement comment, grâce à ces baleiniers, s’est progressivement élaborée la libération du Pérou, du Chili, et de la Bolivie du joug de la vieille Espagne et comment une éternelle démocratie a pu, dès lors, lui succéder.\par
C’est un chasseur de baleines qui a donné au monde aux yeux dessillés cette grande Amérique des antipodes, l’Australie. Après la première découverte qu’en fit par erreur un Hollandais, tous les autres navires fuirent ses côtes barbares comme la peste, mais le baleinier y relâcha, et c’est lui qui a vraiment enfanté cette colonie à présent puissante. De plus, à l’aube des premiers établissements australiens, les émigrants furent plusieurs fois sauvés de la famine par le charitable biscuit du baleinier venant, par bonheur, mouiller l’ancre dans leurs eaux. Les îles innombrables de la Polynésie s’accordent à attester une même vérité et rendent hommage au commerce fait avec les baleiniers qui ouvrirent la voie aux marchands et aux missionnaires, et en bien des cas les amenèrent eux-mêmes à destination. Et si ce pays deux fois verrouillé qu’est le Japon devient une fois hospitalier, c’est au baleinier seul qu’en reviendra un mérite qu’il est sur le point de gagner.\par
Si, résistant à toutes ces preuves, vous persistez à déclarer que rien de noble, rien de beau n’est lié à la pêche à la baleine, alors je suis prêt à rompre ici cinquante lances avec vous et à vous désarçonner à chaque coup, le heaume fendu.\par
La baleine n’a pas son auteur célèbre, sa chasse pas de chroniqueur fameux, direz-vous.\par
{\itshape La baleine pas son auteur célèbre, sa chasse pas de chroniqueur fameux} ? Qui le premier a décrit le Léviathan ? Qui sinon le grand Job ? Qui a écrit le premier récit d’une campagne de pêche à la baleine ? Qui, sinon un roi non parmi les moindres, Alfred, surnommé le Grand, lequel transcrivit de sa plume royale l’histoire que lui en conta Other, le chasseur norvégien de baleines de cette époque ! Et qui a fait de nous un ardent panégyrique au Parlement. Qui sinon Edmund Burke !\par
C’est bien vrai, mais les baleiniers eux-mêmes sont de pauvres diables, ils n’ont pas de sang bleu dans les veines.\par
{\itshape Pas de sang bleu dans les veines} ? Ils ont dans les veines un sang meilleur que le sang royal. La grand-mère de Benjamin Franklin était Mary Morrel, puis, par alliance, Mary Folger, c’est le nom des premiers fondateurs de Nantucket, elle fut l’ancêtre d’une longue lignée de Folger et des harponneurs – tous parents du noble Benjamin – qui, en ce jour, lance son harpon d’un côté du monde à l’autre.\par
Soit encore ! mais cependant tout le monde trouve quelque chose de peu honorable à la pêche à la baleine.\par
Pas respectable, la chasse à la baleine ? La chasse à la baleine est impériale !\par
Une vieille loi statutaire anglaise la déclare « poisson royal ».\par
Oh ! ce n’est qu’un mot ! La baleine n’a jamais joué de rôle éclatant d’aucune manière.\par
La baleine n’a jamais joué de rôle éclatant ?\par
Lors de l’un des grands triomphes célébrés en l’honneur d’un général romain entrant dans la capitale du monde, les fanons d’une baleine, transportés depuis la côte de Syrie, furent la plus grande attraction d’une procession entraînée au son des cymbales.\par
Je le crois puisque vous le dites, mais dites tout ce que vous voulez, il n’y a pas de dignité véritable dans la chasse à la baleine.\par
{\itshape Pas de dignité dans la chasse à la baleine} ? La dignité de notre vocation, le ciel même l’atteste. La Baleine est une constellation australe. Il suffit ! Enfoncez votre chapeau devant le Tsar et tirez-le à Queequeg ! Il suffit ! Je connais un homme qui, au cours de sa vie, a pris trois cent cinquante baleines. Je tiens cet homme pour plus respectable que ce grand capitaine de l’Antiquité qui s’est vanté d’avoir pris autant de villes fortes.\par
Quant à moi, si, par chance, une puissance non encore révélée est en moi, si je mérite jamais quelque vraie réputation dans ce monde avare, à laquelle je puisse raisonnablement prétendre ; si j’accomplis dans l’avenir quelque chose qu’un homme, somme toute, préfère avoir fait plutôt que laisser à faire ; si, à ma mort, mes exécuteurs testamentaires, ou plus probablement mes créanciers, trouvent quelque précieux manuscrit dans mon bureau, j’en impute d’ores et déjà ici tout l’honneur et toute la gloire à la chasse à la baleine, car un navire baleinier fut mon Yale et mon Harvard.
\chapterclose


\chapteropen
\chapter[{CHAPITRE XXV. Post-scriptum}]{CHAPITRE XXV \\
Post-scriptum}\renewcommand{\leftmark}{CHAPITRE XXV \\
Post-scriptum}


\chaptercont
\noindent Pour prouver la dignité de la chasse à la baleine, j’aimerais n’avancer que des faits établis. Fort de ceux-ci, un avocat qui tairait tout à fait une conjecture raisonnable, représentant un atout majeur en faveur de sa cause, ne serait-il pas à blâmer ?\par
Il est de notoriété publique qu’au couronnement des rois et des reines, de notre temps encore, la remise de leurs charges s’accompagne d’un curieux assaisonnement. Il y a une salière dite d’État et peut-être un huilier d’État. Que fait-on exactement avec le sel, qui le sait ? Toutefois je suis absolument sûr que, lors du couronnement, la tête d’un roi est solennellement huilée comme une tête de salade. Pourtant serait-il concevable qu’on l’oigne dans l’intention de faire fonctionner au mieux ses rouages, comme on huile un mécanisme ? Il y a là matière à mainte méditation sur la dignité essentielle du procédé car, dans la vie courante nous avons une opinion piètre, voire méprisante, d’un individu qui se graisse les cheveux et dégage l’odeur consécutive à pareille onction. En vérité, un homme d’âge mûr qui utilise, si ce n’est médicalement, une huile pour les cheveux, cet homme a vraisemblablement en lui un point faible. Et en général il ne peut guère avoir d’envergure.\par
Mais la seule question qui m’intéresse ici : quelle sorte d’huile emploie-t-on lors des couronnements ? Sans aucun doute ce ne peut être de l’huile d’olive, ni de l’huile de Macassar, ni de l’huile de ricin, ni de la graisse d’ours, ni du thran, ni de l’huile de foie de morue. Quelle huile donc sinon l’huile de cachalot, vierge et pure, la plus douce de toutes les huiles ?\par
Pensez à cela, loyaux sujets de Grande-Bretagne ! Nous autres baleiniers, nous fournissons à vos rois et à vos reines la matière première de leur couronnement !
\chapterclose


\chapteropen
\chapter[{CHAPITRE XXVI. Chevaliers et écuyers}]{CHAPITRE XXVI \\
Chevaliers et écuyers}\renewcommand{\leftmark}{CHAPITRE XXVI \\
Chevaliers et écuyers}


\chaptercont
\noindent Le premier second du {\itshape Péquod} était Starbuck, natif de Nantucket et d’ascendance quaker. Il était grand, grave et, bien que né sur une côte glaciale, il paraissait bien adapté à supporter les latitudes brûlantes ; sa chair étant aussi dure qu’un biscuit deux fois cuit. Transporté aux Indes, son sang vif ne se gâterait pas comme la bière en boîte. Il avait dû naître en période de sécheresse ou lors d’une famine générale, ou l’un de ces jours de jeûne pour lesquels son État est célèbre. Il n’avait vu que quelque trente étés arides, mais ceux-ci avaient desséché tout superflu charnel. Mais cette maigreur, pour l’appeler ainsi, ne paraissait pas davantage provoquée par l’angoisse et le souci qu’elle ne paraissait révéler quelque maladie du corps. Elle n’était que condensation de l’homme. Il n’était pas désagréable à regarder, au contraire. Sa peau nette et serrée était parfaitement ajustée et l’enveloppait étroitement, sa force et sa santé y étaient embaumées et l’on eût dit une momie rendue à la vie ; ce Starbuck semblait bâti pour tout endurer tant dans le présent que pour de longues années à venir ; sa vitalité intérieure était une garantie de bon fonctionnement sous tous les climats, de la neige polaire au soleil torride, tel un chronomètre de précision. Regardant au fond de ses yeux, il semblait qu’on pût voir subsister les images des milliers de dangers qu’il avait calmement affrontés au cours de sa vie. Un homme réservé et ferme dont presque toute la vie était action éloquente au lieu d’être un morceau d’éloquence banale. Pourtant, malgré sa pondération, son courage, sa force d’âme, quelques traits de son caractère les altéraient parfois et en certains cas semblaient l’emporter sur ces vertus. Une conscience exceptionnelle pour un marin, un esprit naturellement religieux, la solitude de sa vie en mer l’avaient fortement incliné vers la superstition, mais vers cette sorte de superstition qui, chez certains, semble relever plus de l’intelligence que de l’ignorance. Les présages extérieurs comme ses pressentiments intérieurs étaient son fort. Et s’il arrivait qu’ils vinssent à courber l’acier bien trempé de son âme, combien les souvenirs lointains et tendres de sa jeune femme cap-codaise et ceux de son enfant tendaient à courber plus avant sa vigoureuse nature, le rendant ainsi plus accessible à ces influences secrètes qui, chez les moins hésitants parfois, retient l’élan de cette folle témérité, dont d’autres font souvent preuve face aux vicissitudes les plus périlleuses de la pêche : « Je ne veux pas, dans ma pirogue, d’un homme qui n’ait pas peur de la baleine », disait Starbuck. Il semble qu’il ait voulu signifier par là que le courage le plus efficace et le plus sûr découlait d’une juste estimation du danger et que, dès lors, un homme n’ayant absolument peur de rien est un compagnon beaucoup plus redoutable qu’un lâche.\par
– Oui, oui, disait Stubb, le deuxième second, Starbuck est l’homme le plus prudent que vous trouverez jamais parmi les chasseurs de baleines. Mais nous verrons ce que signifie exactement ce mot de « prudent » dans la bouche d’un homme comme Stubb, comme dans celle de presque n’importe quel chasseur de baleines.\par
Starbuck n’était pas un croisé en quête de périls ; le courage, chez lui, n’était pas affaire de sentiment, mais un instrument utile qu’il avait toujours à portée de main dans les circonstances où il y allait de la vie. D’autre part, il considérait peutêtre, dans cette affaire de pêche, que le courage était une matière première faisant partie de l’approvisionnement du navire, tout comme son bœuf salé et son pain et qu’il ne convenait pas de le gaspiller sottement. C’est la raison pour laquelle il répugnait à mettre les pirogues à la mer après le coucher du soleil, ou à s’acharner à combattre un poisson qui s’obstinait à lutter car, pensait Starbuck, je suis sur cet océan de dangers pour tuer des baleines afin de subvenir à mon existence et non pour subvenir à la leur, et qu’elles aient tué des centaines d’hommes, Starbuck le savait bien. Comment mourut son propre père ? Où, dans les abîmes sans fond, retrouverait-il les membres déchirés de son frère ?\par
Portant de tels souvenirs, enclin, je l’ai dit à une certaine superstition, le courage de ce Starbuck, qui pouvait néanmoins grandir encore, devait vraiment être remarquable. Mais il n’appartenait pas au domaine de la nature et de la raison le fait qu’un homme de cette trempe, marqué de souvenirs et d’expériences terribles, ait des défaillances permettant à un principe secret de rompre en lui les digues, lors d’une circonstance prédisposante, et de consumer d’un coup son courage. Si brave qu’il pût être, c’était de cette sorte de bravoure propre à certains hommes intrépides qui, demeurant fermes dans la lutte contre les océans, les vents, les baleines ou n’importe quelle horreur irrationnelle tangible, ne peuvent pas supporter ces terreurs plus épouvantantes parce que d’ordre spirituel qui s’amassent parfois sous les sourcils froncés d’un homme puissant et hors de lui.\par
Mais si la suite du récit devait révéler l’anéantissement total de la force morale de ce pauvre Starbuck, c’est à peine si j’aurais le cœur de l’écrire car rien n’est plus douloureux, non, révoltant, que d’amener au jour l’écroulement de la vaillance d’une âme. Considérés en tant que compagnies anonymes et en tant que nations, les hommes paraissent haïssables ; il peut bien y avoir des valets, des fous et des meurtriers ; les hommes peuvent bien avoir des visages mesquins et ingrats ; mais l’homme, dans l’idéal, est une créature si noble, si éclatante, si grande et si lumineuse que ses frères devraient courir jeter leurs manteaux les plus précieux sur la souillure d’une ignominie qui se fait jour en lui. Cette force virile et immaculée que nous sentons en nous, dans notre moi le plus inaccessible, si profond qu’elle demeure intacte alors que s’est effondré tout ce que nous voyons du caractère d’un homme, cette dignité saigne de la plus poignante angoisse devant une déchéance mise à nu. La piété elle-même, devant un si honteux spectacle, ne peut étouffer tout à fait ses reproches envers les astres qui l’ont permise. Mais cette noblesse dont je parle n’est pas celle des rois et des magistrats, c’est cette noblesse sans limites qui n’est pas investie par la robe. Tu la verras briller dans le bras qui lève la pioche ou qui plante un clou, cette dignité du peuple venue de Dieu, et qui irradie sans fin de toutes les mains. De Dieu lui-même. Le Grand, l’Absolu ! Le centre et la circonférence de toute démocratie. Son omniprésence, notre divine égalité !\par
Dès lors si, par la suite, je revêts les plus misérables marins, les renégats et les réprouvés de hautes vertus, fussent-elles sombres ; si je tisse autour d’eux des grâces tragiques ; si même le plus triste, peut-être le plus avili d’entre eux s’élève parfois jusqu’aux sommets les plus sublimes ; si je pose au bras d’un travailleur un rayon de lumière éthérée ; si je déploie un arc-enciel sur le désastre de leur soleil couchant ; toi, juste Esprit d’Égalité, soutiens-moi contre la critique des hommes, toi qui as étendu un seul manteau royal d’humanité sur tous mes semblables ! Soutiens-moi, Toi le Grand Être social qui n’a pas refusé au convict basané, Bunyan, la pâle perle de la poésie ; Toi qui as revêtu de l’or le plus fin en feuilles deux fois amincies le bras estropié et perdu du vieux Cervantès ; toi qui as tiré du ruisseau Andrew Jackson pour le hisser sur un cheval de guerre, qui l’a élevé de façon foudroyante, plus haut qu’un trône ! Toi qui dans tes puissantes démarches terrestres cueilles les maîtres de la supériorité parmi le peuple souverain, soutiens-moi, ô Dieu !
\chapterclose


\chapteropen
\chapter[{CHAPITRE XXVII. Chevaliers et écuyers (suite)}]{CHAPITRE XXVII \\
Chevaliers et écuyers (suite)}\renewcommand{\leftmark}{CHAPITRE XXVII \\
Chevaliers et écuyers (suite)}


\chaptercont
\noindent Stubb était deuxième second. Il était natif du cap Cod et dès lors, selon l’usage de l’endroit, on l’appelait un homme-ducap-Cod. Un Roger Bontemps ; ni poltron, ni brave ; prenant les dangers tels qu’ils venaient avec un air indifférent. Lorsqu’il se trouvait engagé dans le moment le plus décisif de la chasse, il travaillait avec calme et sang-froid comme un compagnon menuisier engagé à l’année. De belle humeur, à l’aise, insouciant, il présidait sa baleinière comme si la plus meurtrière rencontre n’était qu’un dîner et ses canotiers des convives. Il était aussi maniaque, en ce qui concernait les aménagements confortables dans sa partie de la pirogue, qu’un vieux cocher de diligence l’est pour son siège. Proche de la baleine, au moment crucial du combat, il maniait sa lance impitoyable avec une froide désinvolture, comme un chaudronnier-au-sifflet son marteau. Flanc contre flanc avec le monstre le plus furieux, il fredonnait ses vieux airs de rigodon. Une longue expérience avait pour ce Stubb transformé les mâchoires de la mort en chaise-longue. Ce qu’il pensait de la mort elle-même, on l’ignore. Qu’il y ait jamais pensé ou non, c’est une question que l’on pourrait se poser, mais s’il venait à y songer après un repas copieux, il ne fait pas de doute qu’en bon marin il la considérait comme un appel au quart à grimper au mât en trébuchant, et à s’agiter là-haut au sujet de quelque chose qui lui serait révélé après qu’il eut obéi à l’ordre et pas avant.\par
Ce qui, peut-être, entre autres, faisait de Stubb un homme insouciant et sans peur, clopinant si gaiement sous le fardeau de la vie dans un monde de graves porteurs de faix, tous ployés vers la terre sous leurs ballots, ce qui contribuait à provoquer cette bonne humeur presque sacrilège en lui, ce devait être sa pipe, car, tout comme son nez, sa courte petite pipe noire était un trait de son visage. Vous vous seriez plutôt attendu à le voir sauter de son hamac sans son nez que sans sa pipe. Il avait à portée de main, dans un râtelier, tout un bataillon de pipes déjà bourrées et, lorsqu’il allait se coucher, il les fumait l’une après l’autre, les allumant l’une à l’autre, jusqu’à extermination de la rangée, puis il les bourrait à nouveau afin qu’elles soient prêtes car, lorsque Stubb s’habillait, la première chose qu’il faisait n’était pas d’enfiler ses jambes dans son pantalon mais sa pipe dans sa bouche.\par
Je pense que dans le fait de fumer sans arrêt résidait l’une des raisons au moins de sa curieuse disposition d’esprit ; car chacun sait que l’air ambiant, que ce soit à terre ou sur mer, est effroyablement infesté par les misères sans nom que les mortels innombrables ont exhalées avec leur dernier souffle. Lors d’épidémies de choléra, il y a des gens pour se promener avec un mouchoir imprégné de camphre sur la bouche ; de même, contre toute épreuve mortelle, la fumée du tabac de Stubb devait servir d’agent désinfectant.\par
Le troisième second était Flask, natif de Tisbury, de Martha’s Vineyard. Un jeune gars, petit, râblé, haut en couleurs, très agressif envers les baleines et qui semblait penser que les grands léviathans lui avaient fait un affront personnel et héréditaire, de sorte qu’il se faisait un point d’honneur de détruire tous ceux qu’il rencontrait. Il était parfaitement réfractaire à tout respect pour les nombreuses merveilles de leur taille majestueuse et de leurs habitudes occultes, et complètement étranger à tout sentiment pouvant ressembler à l’appréhension d’un danger possible à les aborder ; selon sa chétive opinion, la prestigieuse baleine n’était rien de plus qu’une espèce de souris d’un fort grossissement, ou disons un rat d’eau, qui réclamait seulement une petite manœuvre, une quelconque dépense de temps et de peine, pour la tuer et la faire bouillir. Cette absence ignorante et inconsciente de toute crainte le rendait badin en matière de baleines, il poursuivait ces poissons pour l’amusement et un voyage de trois ans au-delà du cap Horn n’était qu’une joyeuse facétie qui durait ce laps de temps. De même que les clous de charpentier se divisent en clous forgés et en clous faits en série, l’humanité peut être soumise à pareille distinction. Le petit Flask faisait partie des clous forgés conçus pour river serré et durer longtemps. On l’appelait Cabrion à bord du {\itshape Péquod}, parce que sa forme rappelait ces bordages courts et carrés ainsi nommés par les baleiniers de l’Arctique, lesquels, pourvus d’allonges disposées en rayons, servent à protéger le navire contre le choc des glaces dans ces mers violentes.\par
Ces trois seconds étaient des hommes d’importance ; ce sont eux qui, selon l’usage universel, en tant que chefs de pirogue, avaient tout pouvoir sur trois de ces embarcations du {\itshape Péquod}. Ces trois chefs de baleinières étaient pareils à des capitaines de compagnies, dans cet ordre de bataille grandiose que le capitaine Achab allait sans doute ordonner pour aller sus à la baleine. Armés de leurs longues lances acérées, ils formaient un trio choisi de lanciers, tout comme les harponneurs figuraient des lanceurs de javelots.\par
Et parce que dans cette fameuse chasse, chaque second ou chef de baleinière, tel un chevalier goth de jadis, est accompagné de son timonier et harponneur qui, lorsque besoin en est, lui tend une nouvelle lance lorsque la première s’est gravement tordue ou coudée lors d’un premier assaut, parce qu’il s’établit de plus généralement, entre les deux hommes, une intimité et une amitié étroites, il n’est que séant que nous disions ici qui étaient les harponneurs du {\itshape Péquod} et à quel chef appartenait chacun d’entre eux.\par
Venait d’abord Queequeg que Starbuck, le premier second, avait choisi pour écuyer. Mais on connaît déjà Queequeg.\par
Venait ensuite Tashtego, un Indien pur sang de Gay Head, le promontoire situé le plus à l’ouest de Martha’s Vineyard, où l’on trouve encore les vestiges d’un village de Peaux-Rouges et qui a longuement fourni à l’île de Nantucket toute proche un grand nombre de ses plus intrépides harponneurs. Dans la baleinerie, on les appelle habituellement du nom générique de Gay-Headers. Tashtego avait de longs et fins cheveux noirs, les pommettes hautes, les yeux noirs n’étaient pas bridés comme ceux des Indiens, mais largement fendus comme ceux d’un Oriental et brûlaient d’un éclat polaire ; tout trahissait qu’il portait en ses veines, sans mélange, le sang de ces fiers guerriers chasseurs qui, à la poursuite du grand élan de la Nouvelle- Angleterre, avaient, l’arc à la main, parcouru les forêts primitives du continent. Ayant fini de suivre au flair les pistes des bêtes sauvages dans les bois, Tashtego, à présent, se lançait dans le sillage marin des grandes baleines ; l’infaillible harpon du fils remplaçait dignement la sûre flèche des ancêtres. À voir ses membres musclés, fauves, souples comme le serpent, on eût presque pu attacher foi aux superstitions de quelques premiers puritains et à croire à demi que ce farouche Indien était un fils du prince des Puissances de l’Air. Tashtego était l’écuyer de Stubb, le deuxième second.\par
Daggoo était le troisième harponneur. C’était un nègre géant, primitif, d’un noir de charbon, un Assuérus à la démarche de lion. Il portait aux oreilles deux boucles d’or si grandes que les marins les appelaient des chevilles à boucle et suggéraient d’y amarrer les drisses de hune. Daggoo s’était, dans son jeune âge, volontairement embarqué sur un baleinier en relâche dans une baie solitaire de son rivage natal. Il ne connaissait du monde que l’Afrique, Nantucket et les ports non civilisés les plus fréquentés des baleiniers ; depuis de nombreuses années, il menait cette vie téméraire de la chasse, à bord de navires dont les propriétaires se montraient circonspects à l’extrême sur le genre d’hommes qu’ils engageaient. Daggoo conservait intacte sa prestance barbare et, droit comme une girafe, il déplaçait sur les ponts la splendeur de ses six pieds cinq pouces, en chaussettes. En levant les yeux vers lui, une humilité physique vous envahissait. Un homme blanc, debout à ses côtés, faisait figure du drapeau blanc du parlementaire. Par un hasard curieux, ce nègre impérial, Assuérus Daggoo, était l’écuyer du petit Flask qui, près de lui, semblait un pion d’échecs. Quant au reste de l’équipage du {\itshape Péquod}, disons-le, pas un sur deux, jusqu’à ce jour, des milliers de matelots engagés dans la baleinerie américaine n’est américain, bien que presque tous les officiers le soient. Il en va des équipages baleiniers comme de l’armée, de la marine de guerre, de la marine marchande et du génie civil employé à la construction des canaux et des voies ferrées en Amérique. Je dis qu’il en va de même parce que, dans tous ces cas, le natif américain fournit aussi généreusement le cerveau que les autres nations les muscles. Un bon nombre de baleiniers viennent des Açores où les bâtiments en provenance de Nantucket font souvent escale dans le but d’augmenter leur équipage avec les soi-disant paysans de ces îles rocheuses. Les baleiniers groenlandais, partis de Hull ou de Londres, s’arrêtent de même aux Shetland, afin de recruter le complément d’hommes qu’ils y déposent à leur retour. Les insulaires semblent faire les meilleurs baleiniers, on ne sait trop pourquoi. À bord du {\itshape Péquod}, la majorité des hommes étaient des insulaires, des Isolatoes qui plus est, je dirais, car, non contents de ne rien connaître du continent commun à tous les hommes, chaque isolato vit à l’écart sur un continent à lui. Mais à présent, établis en communauté à bord, quel ensemble ils formaient ! Une délégation digne d’Anacharsis Clootz, recrutée dans toutes les îles de la mer, dans tous les coins de la terre, pour témoigner, avec le vieil Achab, de toutes les injustices du monde, à un banc dont bien peu sont revenus. Le petit Noir Pip n’est pas revenu, oh ! non ! il est parti avant. Pauvre gars d’Alabama ! Sur le lugubre gaillard d’avant du {\itshape Péquod}, vous le verrez sous peu battre son tambourin ; il jouait un prélude à l’éternité lorsqu’il fut mandé au gaillard d’arrière du ciel, pour se joindre aux anges et jouer du tambourin dans la gloire. Tenu pour un couard ici, salué en héros là-haut !
\chapterclose


\chapteropen
\chapter[{CHAPITRE XXVIII. Achab}]{CHAPITRE XXVIII \\
Achab}\renewcommand{\leftmark}{CHAPITRE XXVIII \\
Achab}


\chaptercont
\noindent Pendant plusieurs jours après le départ de Nantucket, aucun signe du capitaine Achab ne se manifesta au-dessus des écoutilles. Les seconds se relayaient régulièrement aux quarts et, rien qu’on pût voir ne prouvant le contraire, ils paraissaient seuls commander à bord ; pourtant, ils ressortaient parfois de la cabine avec des ordres si soudains et si péremptoires qu’après tout il était clair qu’ils ne commandaient que par procuration. Donc leur seigneur et maître était là, bien qu’invisible à tous les regards, hormis de ceux qui étaient autorisés à voir au-delà du seuil de cette retraite sacrée qu’était la cabine.\par
Chaque fois que je montais sur le pont en revenant de mes quarts en bas, je regardais intensément vers l’arrière pour voir si n’apparaissait pas un visage étranger, car l’isolement en mer avait rendu obsessionnel le premier malaise que j’avais vaguement éprouvé au sujet du capitaine inconnu. Cette obsession était étrangement exacerbée par ma mémoire lorsque je venais à songer aux propos diaboliques et incohérents du loqueteux Élie et bien que ces pensées fussent des intruses dont je n’aurais jamais imaginé qu’elles puissent avoir une aussi subtile violence. Je le supportais mal, alors que d’humeur différente, j’étais presque disposé à sourire des fantasques solennités de cet incongru prophète des quais. Que je ressentisse de l’appréhension ou de la gêne, pour l’appeler ainsi, rien ne semblait justifier le fait d’entretenir un tel trouble lorsque je regardais autour de moi. Bien que les harponneurs et la majorité de l’équipage formassent un ensemble bouffon, barbare et païen, tel qu’on n’en \hspace{1em} voit pas à bord des insipides navires marchands sur lesquels j’avais fait mes premières expériences, je ne pouvais mettre cette angoisse, et fort justement, qu’au compte de la nature unique et farouche de cette sauvage vocation de Viking à laquelle j’avais cédé de façon désordonnée. L’aspect des trois officiers supérieurs, les seconds, était, toutefois, des plus propres à dissiper énergiquement ces phantasmes blafards et à infuser la confiance et la gaieté dans les mauvais pressentiments. Comme officiers de marine et comme hommes, et chacun à leur manière, il n’eût pas été aisé d’en trouver trois meilleurs ; et tous trois étaient américains, un Nantuckais, un Vineyarder, et un Cap Codais.\par
Comme le navire avait quitté le port le jour de Noël, nous eûmes un temps polaire et un froid mordant sur un certain parcours, mais nous étions sans cesse en train de fuir devant lui vers le sud, et chaque minute et chaque degré de latitude nous éloignaient de cet hiver impitoyable et insupportable.\par
Ce fut par un de ces matins d’entre deux climats, moins menaçant mais encore bien gris et sombre, tandis que le navire courait vent arrière, bondissant de façon vengeresse, avec une hâte mélancolique, que, montant sur le pont à l’appel du quart du matin, et levant les yeux vers la lisse de couronnement, je fus aussitôt envahi d’un frisson prémonitoire. La réalité avait lutté de vitesse avec l’appréhension, le capitaine Achab se tenait sur le gaillard d’arrière.\par
Rien ne trahissait en lui une quelconque maladie physique, ni les signes d’une convalescence. Il avait l’air d’un homme qu’on eût arraché à un bûcher dont les flammes l’auraient de part en part dévasté sans le consumer, et sans altérer si peu que ce soit la force dont était pétri son vieil âge. Sa stature haute et large semblait coulée dans un bronze massif, et avoir pris forme dans un moule inaltérable, comme le Persée de Cellini. Une marque fine prenant sa source dans ses cheveux gris courait en un sillon blême sur un côté de son visage aduste, descendait dans son cou basané, pour venir se perdre dans ses vêtements. On eût dit la cicatrice verticale zébrant le fût altier et droit d’un grand arbre lorsque la foudre, précipitant sur lui sa flèche, n’arrachant nulle brindille mais creusant une cannelure dans l’écorce depuis le faîte jusqu’aux racines, avant de pénétrer dans la terre, laisse néanmoins l’arbre bien vivant dans sa verdure, mais marqué au fer. Personne ne peut savoir s’il était né avec sa marque ou si c’était la trace de quelque tragique blessure. Pendant tout le voyage et selon un accord tacite, il y fut peu ou il n’y fut pas fait allusion, surtout pas par les officiers. Mais une fois, un aîné de Tashtego, un vieil Indien de Gay Head qui faisait partie de l’équipage, affirma superstitieusement qu’Achab ne fut marqué de la sorte qu’après la quarantaine passée et non à la suite d’un combat furieux et meurtrier avec un homme mais lors d’une lutte surnaturelle en mer. Toutefois, cette suggestion insensée fut démentie par les déductions que donnait à entendre un vieux Mannois, un vieillard sépulcral qui, n’ayant jamais encore embarqué à Nantucket, posait pour la première fois son regard sur le farouche Achab. Néanmoins, les vieilles traditions marines, la crédulité immémoriale prêtaient communément à ce vieux Mannois des pouvoirs transcendants de clairvoyance. De sorte qu’il ne se trouvait pas un seul marin blanc pour lui opposer une contradiction sérieuse lorsqu’il disait que si jamais le capitaine Achab venait à être enseveli paisiblement – ce qui a peu de chances d’arriver, ajoutait-il dans un murmure – alors celui qui veillerait à sa toilette funèbre découvrirait une marque de naissance le parcourant de la tête aux pieds.\par
L’apparence sinistre d’Achab, sa marque blafarde, me troublaient si profondément qu’au premier instant je fus à peine sensible au fait que le sentiment envahissant de menace qu’il dégageait était partiellement dû à cette barbare jambe d’ivoire sur laquelle il s’appuyait à demi. Il m’était déjà revenu que ce pilon de morfil avait été façonné à la mer dans l’os poli d’une mâchoire de cachalot. « Oui, dit une fois le vieil Indien de Gay Head, il a démâté au large du Japon, mais comme pour son navire, il a remplacé le mât sans pour autant retourner chez lui à cette fin. Il n’est pas au bout de ses ressources. »\par
Je fus frappé de sa position singulière. De chaque côté du gaillard d’arrière, assez proche des haubans d’artimon, se trouvait un trou de tarière d’un demi-pouce de profondeur. Sa jambe d’ivoire immobilisée dans ce trou, agrippé d’une main à un hauban, le capitaine Achab se tenait droit, regardant fixement au-delà de la proue toujours plongeante du navire. Ce regard hardi tendu vers l’avant exprimait un infini de courage inébranlable, de volonté précise et irréductible. Il ne disait mot et ses officiers ne lui parlaient guère, mais leurs moindres gestes, leurs moindres expressions trahissaient ouvertement la conscience pénible, sinon douloureuse, qu’ils avaient de se trouver sous l’œil d’un maître tourmenté. De plus, cet Achab frappé et maussade avait un visage de crucifié, empreint d’une dignité indicible, royale et impérieuse et d’une douleur immense.\par
Ayant pris cette première gorgée d’air, il se retira rapidement dans sa cabine. Mais dès ce matin-là, l’équipage put l’apercevoir tous les jours, tantôt debout sur son pivot, ou assis sur son tabouret d’ivoire, tantôt encore arpentant lourdement le pont. À mesure que le ciel se faisait moins sombre, enclin à la clémence même, il se faisait de moins en moins reclus, comme si seul l’Océan du départ si morne, si mortellement hivernal, l’avait poussé à cette claustration. Et petit à petit, il fut sans cesse sur ce pont à présent presque ensoleillé. Toutefois, pour le peu qu’il disait et faisait, il y semblait aussi inutile qu’un mât de surcroît. Le {\itshape Péquod} faisait simplement route d’ailleurs et ne se trouvait pas en croisière régulière, les officiers avaient toute compétence pour surveiller les préparatifs de la chasse, de sorte qu’Achab n’était sollicité ni passionné par rien d’extérieur à luimême et qui fût susceptible de dissiper, pour l’heure, les nuages qui, couche sur couche, barraient son front, car toutes les nuées choisissent depuis toujours le plus haut sommet pour s’y amonceler. Bientôt pourtant, la force de persuasion d’un temps de vacances, chaud, doux, mélodieux, parut le tirer hors de sa morosité. Car, lorsque Avril et Mai, les joues en fleurs et la danse dans les reins, pénétrèrent au cœur misanthrope des bois de l’hiver, le vieux chêne le plus nu, le plus bourru, le plus frappé de foudre se met en peine, pour accueillir la joie de ces hôtes, de tendre quelque pousse verte. De même Achab céda-t-il enfin un peu aux joueuses séductions de ce temps d’enfance. Plus d’une fois son regard eut un épanouissement furtif qui, chez tout autre, fût devenu sourire.
\chapterclose


\chapteropen
\chapter[{CHAPITRE XXIX. Stubb affronte Achab}]{CHAPITRE XXIX \\
Stubb affronte Achab}\renewcommand{\leftmark}{CHAPITRE XXIX \\
Stubb affronte Achab}


\chaptercont
\noindent Quelques jours passèrent et le {\itshape Péquod}, toutes banquises et toutes glaces derrière lui, roulait à présent dans l’éblouissant printemps de Quito qui, sur mer, règne presque sans cesse au seuil de l’août éternel des tropiques. Ces jours chauds, nuancés de fraîcheur, clairs, vibrants, odorants, débordants, généreux étaient pareils à un sorbet persan emplissant jusqu’au bord une coupe de cristal des flocons d’une neige à la rose. Les nuits étoilées, majestueuses, semblaient les dames hautaines dont les bijoux illuminaient des robes de velours et qui, dans une orgueilleuse solitude, berçaient dans l’absence le souvenir de leurs princes conquérants : les soleils casqués d’or. Pour un homme qui ne peut se passer de sommeil, il était dur de choisir entre des jours aussi captivants et d’aussi séduisantes nuits. Mais la magie soutenue de ces beaux jours ne prêtait pas seulement de nouveaux pouvoirs et de nouveaux envoûtements au monde extérieur, elle pénétrait jusqu’à l’âme surtout aux heures tranquilles et douces de l’approche du soir ; alors naissaient les cristaux du souvenir comme naissent les glaces les plus pures dans les crépuscules de silence. Toutes ces influences subtiles agissaient sur la trame d’Achab.\par
L’âge avancé est toujours vigilant, comme si un lien plus ancien avec la vie faisait qu’un homme considère de moins en moins tout ce qui ressemble à la mort. Parmi les commandants de vaisseau, les grisons abandonnent plus d’une fois leurs couchettes pour le pont drapé dans le manteau de la nuit. Ainsi en allait-il d’Achab. Mais à présent il semblait tellement \hspace{1em}vivre au grand air qu’en vérité ses apparitions étaient plutôt destinées à la cabine qu’au pont. « On a l’impression de descendre dans sa tombe, se murmurait-il, pour un vieux capitaine comme moi, passer par un si étroit écoutillon, c’est aller à ma couche creusée dans la terre. »\par
Ainsi presque toutes les vingt-quatre heures, quand les quarts de nuit étaient établis, que la bordée du pont montait la garde pour la bordée endormie, s’il advenait qu’il fallût raidir une manœuvre sur le gaillard d’avant, les hommes ne la laissaient pas retomber rudement, comme dans la journée, mais l’amenaient avec précaution afin de ne pas troubler le sommeil de leurs camarades ; lorsque cette paix soutenue venait à régner, le timonier silencieux épiait, en général, l’écoutillon de la chambre et, sans tarder, le vieil homme émergeait, cramponné à la rampe de fer pour alléger sa marche d’infirme. Il montrait alors quelques égards humains, s’abstenant à ces heures de parcourir le gaillard d’arrière parce que ses officiers fatigués, cherchant le repos à six pouces au-dessous de son talon d’ivoire, eussent rêvé au grincement de dents des requins en entendant l’écho répété du claquement d’os de son pas. Mais une fois son humeur ombrageuse l’emporta sur toutes considérations et, d’un pas pesant et maladroit, il arpenta le pont, de la lisse de couronnement au grand mât. Stubb alors, ce deuxième second rassis, l’y rejoignit et, faisant plaisamment sentir sa désapprobation d’une façon mal assurée, lui dit que si tel était le bon plaisir du capitaine Achab de faire les cent pas sur les planches personne ne pouvait s’y opposer mais qu’il devait y avoir moyen d’étouffer le bruit, suggérant avec hésitation et d’une voix indistincte qu’on pourrait insérer le talon d’ivoire dans une boule d’étoupe… Ah ! Stubb, tu ne connaissais pas encore Achab !\par
– Suis-je un boulet de canon, Stubb, que tu veuilles me bourrer de cette façon ? Mais… va-t’en. J’avais oublié. Va à ta tombe nocturne où tes pareils dorment entre des linceuls afin de s’habituer à celui qui les attend à la fin. En bas, chien ! À la niche !\par
Stubb fut rendu muet un instant, surpris par cette exclamation finale, imprévisible, lourde de mépris du vieillard, puis révolté il rétorqua :\par
– Je n’ai pas l’habitude qu’on me parle ainsi, sir, cela ne me plaît qu’à moitié, sir.\par
– Baste ! grinça Achab entre ses dents serrées, reculant avec violence comme pour échapper à la tentation de la colère.\par
– Non, sir, pas encore, dit Stubb enhardi, je ne me laisserai pas servilement traiter de chien, sir.\par
– Alors que tu sois dix fois appelé âne, mule, baudet et vat’en ou je débarrasse le monde de ta présence !\par
En disant ces mots Achab marchait sur lui et son aspect était si terrifiant que Stubb s’écarta malgré lui.\par
– Je n’ai jamais été traité de la sorte sans riposter avec les poings, marmotta Stubb en descendant l’écoutillon. C’est très étrange. Un moment, Stubb ! comment se fait-il que je ne sache pas très bien si je dois retourner en arrière et le frapper ou – qu’est-ce que cela signifie ? – tomber à genoux et prier pour lui. Oui, telle est bien la pensée qui m’est venue, mais ç’aurait bien été la première fois que j’aurais jamais prié. C’est étrange, très étrange. Mais il est étrange lui aussi, oui, par quelque bout qu’on le prenne, c’est le plus étrange vieil homme avec lequel Stubb ait jamais navigué. Comme il m’a foudroyé ! ses yeux prenaient feu comme de la poudre ! Est-il fou ? En tout cas, il a quelque chose sur le cœur, aussi sûr qu’il y a quelque chose sur le pont lorsqu’il craque. Et maintenant il ne passe pas, non plus, plus de trois heures au lit sur vingt-quatre et même alors il ne dort pas. Qu’est-ce qu’il me disait le garçon, cette Pâte-Molle ? Qu’il trouve tous les matins les draps du hamac du vieil homme tout chiffonnés, en désordre, rejetés au pied et la couverture presque tordue en nœuds, et l’oreiller brûlant d’effrayante manière comme si l’on avait posé dessus une brique tirée du feu ? Un vieillard de flamme ! Je pense qu’il a ce que quelques gars à terre appellent une conscience ; c’est une sorte de tic douloureux à ce qu’ils disent, pire qu’une rage de dents. Eh bien ! eh bien ! je ne sais pas ce que c’est, mais Dieu me garde de l’attraper. Il est plein d’énigmes. Je me demande ce qu’il peut bien aller faire dans la cale arrière tous les soirs comme Pâte- Molle l’en soupçonne. Qu’est-ce que ça veut dire ? J’aimerais bien le savoir. Avec qui a-t-il rendez-vous dans la cale ? Alors ! Est-ce que ça n’est pas étrange, ça ? Mais on ne peut pas savoir… c’est un truc vieux comme le monde ! Allons dormir. Du diable ça vaut la peine d’être né, rien que pour s’endormir daredare. Et maintenant que j’y réfléchis, c’est bien la première chose que font les bébés, c’est étrange aussi, ça, d’une certaine façon. Du diable, tout vient à être étrange si l’on commence à y penser. Mais c’est contraire à mes principes. Ne pense pas ! C’est mon onzième commandement, et dors quand tu peux, c’est mon douzième. Alors, allons-y. Mais comment ? Ne m’a-til pas traité de chien ? Du diable ! Il m’a dit que j’étais dix fois une bourrique, et tout un monceau d’ânes bâtés, après « cela » ! Il aurait mieux fait de me cogner dessus et qu’on n’en parle plus. Peut-être qu’il m’a cogné dessus en effet et que je ne m’en suis pas aperçu, tellement j’ai été interloqué par son front ; il brillait comme un os blanchi. Que diable m’arrive-t-il ? Je ne tiens pas sur mes jambes. Affronter ce vieillard m’a mis les idées à l’envers. Par Dieu, je dois pourtant avoir rêvé quoique… Comment, ô comment ? Comment ? La seule chose à faire c’est de n’y plus penser, alors au hamac ! Et demain je verrai comment se présentent ces maudites jongleries à la lumière du jour.
\chapterclose


\chapteropen
\chapter[{CHAPITRE XXX. La pipe}]{CHAPITRE XXX \\
La pipe}\renewcommand{\leftmark}{CHAPITRE XXX \\
La pipe}


\chaptercont
\noindent Stubb parti, Achab resta un moment appuyé à la rambarde et, comme il avait coutume de le faire depuis quelque temps, appelant un homme de quart, il l’envoya quérir son tabouret et sa pipe. Il l’alluma à la lampe de l’habitacle, installa son tabouret au bord du vent, s’assit et se mit à fumer.\par
Au temps des Vikings, les rois du Danemark amoureux de la mer avaient des trônes faits en défenses de narvals, si l’on en croit l’histoire. Qui, voyant alors Achab, assis sur son trépied d’ivoire, n’eût pas évoqué la royauté dont il était le symbole ? Khan des bordages, roi de l’Océan, et grand seigneur des léviathans, tel était Achab.\par
Quelques instants passèrent, pendant lesquels il souffla une fumée épaisse en volutes rapides et incessantes que le vent rabattait sur son visage. Enfin, retirant de ses lèvres le tuyau, il entama un monologue : « Comment ! Fumer ne m’apaise plus. Oh ! ma pipe, cela va aller bien mal pour moi, si ton charme n’opère plus ! J’ai inconsciemment peiné, et non point pris du plaisir, certes, et fumer si longtemps au vent sans m’en rendre compte ; au vent, et par bouffées si nerveuses, comme si, pareil à celui de la baleine, mon dernier souffle était le plus fort et le plus angoissé. Qu’ai-je à faire de cette pipe ? Elle est faite pour la sérénité, et pour envoyer vers de doux cheveux blancs une fumée douce et blanche, mais elle n’a pas de raison de monter jusqu’à des mèches hirsutes, couleur d’acier comme les miennes. Je ne fumerai plus… »\par
Il jeta la pipe allumée dans la mer, sa braise siffla dans la vague et au même instant le navire effaça la bulle gonflée de sa chute. Le chapeau rabattu, Achab arpenta le pont en titubant.
\chapterclose


\chapteropen
\chapter[{CHAPITRE XXXI. La reine Mab}]{CHAPITRE XXXI \\
La reine Mab}\renewcommand{\leftmark}{CHAPITRE XXXI \\
La reine Mab}


\chaptercont
\noindent Le lendemain matin Stubb abordait Flask :\par
« Je n’ai de ma vie fait un rêve aussi étrange, Cabrion. Vous connaissez bien la jambe d’ivoire du vieux ; eh bien ! j’ai rêvé qu’il m’en frappait, et lorsque j’essayai de lui rendre ses coups, sur mon âme, mon petit bonhomme, ma jambe droite s’est détachée comme je faisais le geste ! Et alors, presto ! Achab se transforma en pyramide et comme un idiot complet, je persistai à lui envoyer des coups de pied. Mais ce qui est encore plus bizarre, Flask – vous savez à quel point tous les rêves sont bizarres – malgré la rage qui me possédait, je me disais obscurément qu’après tout, ce coup dont m’avait gratifié Achab n’était pas tellement une insulte. « Eh bien ! me disais-je, pourquoi cette colère ? Ce n’est pas une vraie jambe, mais seulement une fausse. » Et il y a une belle différence entre un coup envoyé par un pied vivant ou par une chose morte. C’est ce qui fait, Flask, qu’il est cinquante fois plus dur d’encaisser un coup porté avec la main qu’un coup porté avec une canne. Le membre vivant, voilà ce qui fait vivante l’insulte, mon petit bonhomme. Et pendant que j’usais stupidement mes orteils contre cette maudite pyramide, je ne cessais de me dire, tant je me trouvais dans un état déconcertant de contradiction, je ne cessais de me dire : « Qu’est donc sa jambe à présent, sinon une canne, une canne de morfil ? Mais oui, me disais-je, il jouait seulement au jeu du bâton, ce n’était qu’un coup d’os de baleine, et non un avilissant coup de pied. D’autre part, pensai-je, regarde bien comme la partie qui en constitue le pied est petite. Si un paysan aux grands pieds m’avait fait la même chose, ça ç’aurait été une puissante et maudite injure. Tandis que là, l’injure a la dimension d’un petit rond. Mais à présent Flask, c’est là que commence la grosse farce du rêve. Tandis que je continuais à batailler contre la pyramide, une sorte de vieux triton, poilu comme un blaireau, et nanti d’une bosse, me prit aux épaules et me fit pivoter. » « À quoi jouez-vous ? » me dit-il. Sacrebleu, quelle peur j’avais, mon gars ! Une trogne pareille ! Pourtant, l’instant d’après la frousse m’avait passé. « À quoi je joue ? » répondis-je enfin. « En quoi cela vous regarde-t-il, j’aimerais bien le savoir, monsieur du Bossu. Est-ce que vous voudriez un coup de pied ? » Juste ciel, Flask, à peine avais-je prononcé ces mots, qu’il se tourna, se pencha et soulevant un tas d’algues qu’il portait en guise de jupe, me montra son derrière – que croyez-vous que je vis ? – mille tonnerres, homme, son derrière était tout hérissé d’épissoirs, pointes en dehors. Réflexion faite, j’ajoutai : « Je crois que je ne vous enverrai pas un coup de pied, vieux camarade ! » « Sage Stubb, dit-il, sage Stubb » et il marmottait ça sans s’arrêter, en mâchonnant ses gencives comme une vieille sorcière. Sentant qu’il n’en finirait pas avec son « sage Stubb, sage Stubb », j’en conclus que je pourrais tout aussi bien recommencer à botter la pyramide. Dans cette intention, je levai le pied mais il se mit aussitôt à rugir : « Cessez ces manières ! » « Holà, dis-je, qu’est-ce qu’il y a à présent, vieux ? » « Écoutezmoi bien, vous, discutons l’affaire. Le capitaine Achab vous a cogné, n’est-ce pas ? » « Oui, c’est bien ce qu’il a fait et juste \hspace{1em}là. » « Bon ! Et il a utilisé sa jambe d’ivoire, n’est-ce pas ? » « En effet ! » « Eh bien alors ! sage Stubb, de quoi vous plaignezvous ? Ce coup n’était-il pas bien intentionné ? Il ne vous a pas touché avec une jambe de pitchpin, n’est-ce pas ? Non, Stubb, vous avez été frappé par un homme grand et avec une jambe d’un magnifique ivoire. C’est un honneur, je considère cela comme un honneur ! Écoutez-moi, sage Stubb. Dans la vieille Angleterre, les plus grands seigneurs tirent une très grande gloire d’avoir été souffletés par une reine, et ainsi investis de l’Ordre de la Jarretière. Ainsi, Stubb, glorifiez-vous d’avoir été botté par le vieil Achab qui vous a, ce faisant, infusé la sagesse. Souvenez-vous de ce que je vous dis : être frappé par lui, c’est un honneur insigne, et ne lui rendez ce coup sous aucun prétexte car vous êtes démuni. Ne voyez-vous pas cette pyra- mide ? » Sur ce, il parut tout soudain se dissoudre dans les airs. Je ronflais. Je me retournai… et je me retrouvai dans mon hamac ! Eh bien ! que pensez-vous de ce rêve, Flask ?\par
– Je ne sais pas. Pourtant il me paraît un peu bête.\par
– Peut-être, peut-être ! Mais il a fait de moi un homme sage, Flask. Voyez-vous Achab, debout, regardant par-dessus bord à la poupe ? Eh bien, la meilleure chose à faire, Flask, c’est de laisser seul le vieil homme, ne jamais lui répondre quoi qu’il dise. Holà ! Qu’est-ce qu’il crie ? Écoutez !\par
– Ohé, vigies ! Ouvrez l’œil, tous ! Il y a des baleines par là ! Si vous en voyez une blanche, donnez de la voix à vous faire sauter les cordes vocales.\par
– Que pensez-vous de cela, Flask ? N’y a-t-il pas l’ombre de quelque chose d’insolite là-dedans ? Hein ? Une baleine blanche, vous avez bien entendu ! Soyez sûr qu’il y a quelque chose de singulier dans le vent. Tenez-vous paré, Flask ! Achab est possédé par sa malédiction. Mais chut… il vient par ici.
\chapterclose


\chapteropen
\chapter[{CHAPITRE XXXII. Cétologie}]{CHAPITRE XXXII \\
Cétologie}\renewcommand{\leftmark}{CHAPITRE XXXII \\
Cétologie}


\chaptercont
\noindent Déjà nous voici lancés hardiment sur la mer profonde, mais nous allons bientôt nous perdre sur son immensité sans ports et sans rivages. Avant que le {\itshape Péquod} n’aborde flanc contre flanc, le léviathan incrusté de bernacles, il serait bon de se pencher sur une étude presque indispensable à la bonne compréhension des surprises qu’il nous réserve et aux allusions de toute nature qui vont suivre.\par
Je voudrais à présent vous exposer des vues systématiques sur la baleine et ses diverses espèces. Je ne tenterai rien moins que d’établir une classification d’éléments chaotiques, et cela n’est pas tâche facile. Voici ce qu’en disent les auteurs les plus récents et les plus compétents.\par
« Aucune branche de la zoologie n’est aussi confuse que celle que l’on nomme Cétologie », dit le capitaine Scoresby en 1820.\par
« Il n’entre pas dans mes intentions, en admettant que ce soit en mon pouvoir, d’approfondir quelle serait la juste méthode à employer pour diviser les cétacés en groupes et familles. Le désordre et l’obscurité sont le fait des historiens de cet animal (cachalot) », dit le chirurgien Beale, en 1839. « Incapable de poursuivre une étude dans ces incommensurables profon- deurs. » « Un voile opaque enveloppe notre connaissance des cétacés. » « Un chemin semé d’épines. » « Ces vagues notions ne font qu’aiguiser notre tourment à nous autres naturalistes. »\par
C’est en ces termes que s’expriment le grand Cuvier, John Hunter, et Lesson, ces lumières de la zoologie et de l’anatomie. Néanmoins, bien que la connaissance soit mince, les livres sur la question ne manquent pas ni, bien que pauvres d’apport, les passages sur la cétologie ou science des cétacés. En voici quelques-uns : les auteurs de la Bible, Aristote, Pline, Aldrovande, Sir Thomas Browne, Gesner, Ray, Linné, Rondelet, Willoughby, Green, Artédi, Sibbald, Brisson, Martens, Lacépède, Bonnaterre, Desmaret, le baron Cuvier, Frédéric Cuvier, John Hunter, Owen, Scoresby, Beale, Bennett, J. Ross Brown, l’auteur de « Miriam Coffin », Olmstead et le révérend T. Cheever. Dans quelles intentions de généralisation ils ont tous écrit, les citations qui précèdent le montrent bien.\par
Parmi ces auteurs, seuls ceux qui viennent après Owen ont vu des cétacés vivants, et un seul d’entre eux fut un baleinier et un harponneur professionnel, je veux parler du capitaine Scoresby. Mais Scoresby ne savait rien et ne fait pas même mention du cachalot, à côté duquel il ne vaut même pas la peine de parler de la baleine franche. Et disons ici que la baleine du Groenland usurpe la royauté des mers, car elle n’est à aucun égard le plus grand des cétacés. Pourtant, à cause de la longue revendication de ses droits, et de l’ignorance profonde où l’on était, jusqu’à ces soixante-dix dernières années, du cachalot inconnu et fabuleux, de cette ignorance qui sévit encore, en dehors de quelques rares milieux scientifiques et des ports baleiniers, cette usurpation put être absolue. Tous les grands poètes de l’Antiquité, lorsqu’ils font allusion au léviathan, vous diront que la baleine du Groenland est pour eux le roi incontesté des océans. Mais le temps de la justice est venu. C’est Charing Cross ; écoutez bien, bonnes gens… la baleine franche est déposée, le grand cachalot règne désormais.\par
Deux livres seulement prétendent vous montrer le cachalot vivant et y parviennent jusqu’à un certain point. Ce sont ceux de Beale et de Bennett qui furent tous deux chirurgiens de bord sur des navires anglais dans les mers du Sud. La substance à tirer de leurs ouvrages au sujet du cachalot est inévitablement mince, mais d’excellente qualité, bien que limitée presque uniquement à la description scientifique. Mais jusqu’à ce jour le cachalot, qu’il soit abordé par la science ou par la poésie, n’a vu retracer toute sa vie dans aucune littérature. Alors qu’on a parlé de bien d’autres chasses aux cétacés, l’épopée du cachalot n’est pas encore écrite.\par
Les différentes espèces de cétacés demandent une nomenclature accessible à tous, nous en ferons une ébauche qu’à l’avenir les chercheurs compléteront, chacun dans sa spécialité propre. Et puisqu’il ne se présente pas un homme plus qualifié pour empoigner cette affaire, je vais risquer ici mes modestes tentatives. Je ne proposerai pas un traité sans lacunes car, dans les entreprises humaines, tout ce qui se prétend intégral ne peut être qu’erroné pour cette raison même. Je ne prétends pas donner une description anatomique détaillée des différentes espèces, ni – ici du moins – une longue description quelle qu’elle soit. Je tends seulement à établir un schéma de systématisation en matière de cétologie, je serai l’architecte et non le bâtisseur.\par
Mais c’est une lourde tâche, un simple trieur de lettres à la poste n’en viendrait pas à bout. Tâtonner à leur suite au fond des mers, approcher les mains des indicibles fondations, des côtes, du bassin même de la terre, c’est une tentative effrayante. Qui suis-je pour prétendre « mettre un jonc dans les narines » du léviathan ! Les sarcasmes dont Job se vit accablé peuvent bien me glacer de terreur. « Fera-t-il (le léviathan) un accord avec toi ? Tout espoir de le prendre s’évanouit ! » Mais j’ai traversé à la nage les bibliothèques et fait voile sur les océans. J’ai eu affaire avec les baleines avec les mains que voici, je suis sincère, et je vais me risquer. Il y a toutefois quelques préliminaires à établir.\par
Tout d’abord : que la cétologie en soit encore au stade incertain et indéfini de son enfance, un seul fait suffirait à le prouver : on discute parfois encore la question de savoir s’il faut ou non classer les cétacés dans le genre des poissons. Dans son {\itshape Systema Naturæ}, 1776, Linné déclare : « Désormais je ne mentionnerai plus les cétacés parmi les poissons. » Mais, je le sais, jusque vers 1850, et malgré l’affirmation formelle de Linné, les requins, les différentes espèces d’aloses tout comme les harengs étaient considérés au même titre que le léviathan parce que habitants d’une même mer.\par
Linné donne les raisons pour lesquelles il s’appuie pour dire que les cétacés ne participent pas du seul Océan : « à cause de leur sang chaud, de leur cœur biloculaire, de leurs poumons, de leurs paupières mobiles, de leurs oreilles sans pavillon externe, {\itshape penem intrantem feminam mammis lactantem} » ? et enfin « {\itshape ex lege naturæ jure meritoque} ». J’ai soumis cette question à mes amis Simeon Macey et Charley Coffin de Nantucket, qui furent tous deux mes compagnons d’ordinaire lors d’un certain voyage, et tous deux s’accordèrent à reconnaître que ces arguments étaient insuffisants. Charley eut même l’impiété de dire que c’étaient des balivernes.\par
Qu’on sache donc que, faisant fi de tout argument, je vais tabler sur la bonne vieille croyance qui fait de la baleine un poisson et que j’implorai le soutien du saint Jonas. Ce point essentiel étant fixé, le suivant est celui-ci : quelles sont les caractéristiques physiques qui distinguent les cétacés des autres poissons ? Linné, dans le passage précité, les énumère, j’ajoute plus succinctement : des poumons, le sang chaud tandis que tous les autres poissons ont le sang froid et point de poumons.\par
Ensuite : quelle définition allons-nous donner de la baleine d’après son aspect extérieur, de façon que son signalement reste valable pour les temps à venir ? En bref, {\itshape la baleine est un poisson souffleur dont la queue est horizontale}. Tout est là ! Car, pour laconique que soit cette définition, elle est le fruit de longues méditations. Un morse souffle joliment comme une baleine, mais le morse n’est pas un poisson parce qu’il est amphibie. La seconde donnée de ma définition, liée à la première, est décisive. Presque tout un chacun a pu remarquer que les poissons familiers aux terriens n’ont pas la queue horizontale mais verticale, tandis que, d’une forme analogue, chez les poissons souffleurs elle est invariablement horizontale.\par
Par cette définition, je n’entends nullement exclure de la confrérie léviathanesque une quelconque créature marine jusqu’ici appelée cétacé par les Nantuckais les mieux informés, ni d’autre part les apparenter à des poissons jusqu’ici considérés par des gens compétents comme leur étant étrangers\footnote{Je sais que jusqu’à ce jour les poissons appelés Lamantins et Dugongs (cochons et truies de mer des Coffin de Nantucket) sont classés dans les cétacés. Mais je leur refuse ces lettres de crédit car ces cochons de mer sont bruyants, forment une société méprisable, se cachent à l’embouchure des rivières, broutent du foin mouillé et surtout ne sont pas des souffleurs. Je leur ai délivré leurs passeports afin qu’ils quittent le royaume de la Cétologie.} Dès lors tous les poissons de moindre dimension, souffleurs et plagiures entreront dans le plan ichnographique de la cétologie. Et maintenant établissons les grandes divisions dans la région des cétacés.\par
Tout d’abord : selon leur ordre de grandeur je sépare en VOLUMES (subdivisibles en chapitres) tous les baleinoptères tant petits que grand. I : LA BALEINE IN-FOLIO – II : LA BALEINE IN-OCTAVO – III : LA BALEINE IN-DOUZE.\par
Comme type de l’in-folio, je donnerai le cachalot, de l’inoctavo le grampus ou orque épaulard, de l’in-douze, le marsouin.\par
IN-FOLIO : j’y inclus les chapitres suivants : I : le \hspace{1em}cachalot\par
– II : la baleine franche – III : le rorqual commun – IV : la jubarte – V : la baleine à dos en rasoir – VI : le sulphur bottom.\par
Livre I (in-folio), Chapitre I : – le cachalot. Ce cétacé fut vaguement connu par les anciens Anglais sous le nom de Trumpo, physeter, et baleine à tête d’enclume ; c’est l’actuel cachalot des Français, le Pottwal des Allemands, et le Macrocéphalus de la terminologie recherchée. C’est, sans aucun doute, le plus grand habitant du globe, le cétacé le plus redoutable à affronter, le plus majestueux d’aspect et enfin, de loin, celui qui a la plus grande valeur commerciale car c’est la seule créature qui fournisse le spermaceti, substance d’un prix inestimable. Nous nous étendrons, au cours du récit, sur toutes ses autres singularités. Je traiterai surtout de son nom pour le moment. Considéré du point de vue philologique, il est absurde. Il y a de cela quelques siècles, alors que sa personnalité était parfaitement inconnue, qu’on n’en tirait par hasard de l’huile que s’il se trouvait échoué, il semble qu’on pensait généralement que le spermaceti provenait d’un cétacé identique à celui que l’on nomme en Angleterre : baleine du Groenland ou baleine franche. On croyait également que ce même spermaceti était la liqueur séminale de cette baleine franche ce que suggère littéralement la première moitié du mot. En ces temps-là encore, le spermaceti était d’une extrême rareté et on ne l’utilisait pas pour en faire des bougies mais uniquement comme onguent et médicament. On ne pouvait l’obtenir que chez l’apothicaire, comme de nos jours vous y allez quérir une once de rhubarbe. Je pense que plus tard, lorsque fut connue la nature véritable du spermaceti, les marchands lui conservèrent son nom primitif afin d’augmenter sa valeur grâce à la notion de rareté qu’il impliquait. Mais la dénomination de baleine à spermaceti est enfin revenue à qui de droit.\par
Livre I, Chapitre II (la baleine franche) : À certains égards elle est le plus vénérable des léviathans, ayant constitué le premier gibier poursuivi par les baleiniers. On lui doit ce qui est communément connu sous le nom de fanons ou de baleines, et l’huile dénommée « huile de baleine », peu cotée commercialement. Par les baleiniers, elle est indifféremment appelée : la baleine, la baleine du Groenland, la baleine noire, la grande baleine, la baleine vraie et la baleine franche. L’identité des espèces demeure très obscure sous ces noms de baptêmes si généreusement multipliés. Quelle est donc alors la baleine que je donne en deuxième exemple dans mes in-folios ? C’est la grande mysticetus des naturalistes anglais ; la baleine du Groenland des baleiniers britanniques, la baleine franche des baleiniers français, la Grönlandsval des Suédois. C’est la baleine qui, pendant plus de deux siècles, a été classée par les Hollandais et les Anglais dans les eaux arctiques ; c’est celle que les Américains ont longuement poursuivie dans l’océan Indien, sur les bancs du Brésil, sur la côte nord-ouest et en bien d’autres régions du monde, qu’ils appelaient les parages de chasse à la baleine franche.\par
Certains prétendent différencier la baleine du Groenland des Anglais de la baleine franche des Américains, pourtant ils s’accordent quant à ses caractéristiques majeures et n’ont pas encore avancé un seul fait précis sur lequel appuyer une distinction fondamentale. C’est en raison de subdivisions sans fin, fondées sur des différences peu probantes, que certaines branches des sciences naturelles sont devenues d’une complexité rebutante. Nous parlerons plus longuement de la baleine vraie ailleurs, dans le but de porter une lumière sur le cachalot.\par
Livre I (in-folio), Chapitre III (le dos-en-rasoir). Je désigne sous ce nom le monstre qui, sous des appellations diverses de rorqual, grand souffleur et longimane, a été vu dans presque toutes les mers. Son souffle élevé a été bien souvent décrit par les passagers traversant l’Atlantique sur le paquebot de New York. Quant à sa longueur et à ses fanons, le rorqual ressemble à la baleine vraie, mais il est plus élancé et d’une couleur plus claire tirant sur l’olivâtre. Ses grandes lèvres ont un aspect festonné dû aux plis traversés de larges sillons. Sa caractéristique essentielle est l’aileron auquel il doit son nom (« fin back ») et qui attire le regard. Cet aileron, qui se dresse à l’extrémité de son dos, mesure deux ou trois pieds de long et forme un triangle extrêmement aigu à sa pointe. Même lorsqu’on n’aperçoit rien de son corps, cet aileron émerge parfois très visiblement audessus de la surface. Lorsque la mer est suffisamment calme, gravée de fins cercles concentriques, que cet aileron se dresse pareil au style d’un cadran solaire et projette son ombre sur le froissement de l’eau on dirait qu’il indique des heures mouvantes et liquides. Sur ce cadran d’Achaz, l’ombre va souvent à rebours. Le rorqual n’est pas grégaire, il semble haïr ses semblables à l’instar de certains hommes. Très timide, toujours solitaire, il émerge de façon inattendue dans les eaux les plus lointaines et les plus mornes ; son souffle droit, haut, unique, s’élève comme le fût d’un arbre farouche, seul dans une plaine nue. Douée d’une force et d’une rapidité étonnantes, sa nage semble défier la poursuite de l’homme. Ce léviathan semble être le Caïn maudit et irréductible de sa race, marqué au fer sur le dos. Étant donné qu’il a la bouche garnie de fanons, le rorqual est souvent inclus dans le groupe des baleines vraies, dans l’espèce appelée théoriquement : cétacés à fanons, c’est-à-dire à « baleines ». Ces cétacés à fanons semblent comprendre des genres nombreux dont la plupart sont malheureusement peu connus. Les baleiniers leur donnent des noms divers, baleines à gros nez, à bec, à bonnet, à tuyaux, rostré…\par
Au sujet de cette nomenclature « à fanons », il est important de relever que, si elle est très pratique pour désigner certaines espèces de cétacés, il serait vain d’espérer qu’elle permette une classification précise du léviathan d’après ses fanons, ses bosses, ses ailerons ou ses dents, bien que ces traits caractéristiques semblent plus aptes à fournir les fondements d’une cétologie que toute autre partie de leur anatomie. Et ensuite ? Fanons, bosse, ailerons et dents, voilà qui est réparti sans discrimination parmi toutes sortes de cétacés, sans qu’ils aient pour autant des traits communs beaucoup plus essentiels. Ainsi le cachalot et la baleine à bosse sont tous deux bossus et là s’arrête leur ressemblance. Et cette même baleine à bosse, tout comme la baleine du Groenland, a des fanons et les similitudes s’arrêtent là. Il en va de même pour les dents, etc. dont nous venons de parler. Dans différentes espèces de baleines, ces éléments se combinent irrégulièrement, ou si l’on en isole un, on ne peut établir une méthodologie sur une base aussi peu sûre. C’est un écueil où échouèrent tous les naturalistes spécialisés en cétologie.\par
Mais on pourrait penser que l’anatomie interne d’un cétacé permet une plus juste classification. Non. Qu’est-ce qui, dans la baleine franche, paraît plus remarquable que ses fanons ? Or nous venons de voir qu’elle est impossible à classer d’après eux. Et si vous pénétrez jusque dans les entrailles des divers léviathans, vous n’y trouverez pas un cinquantième de caractéristiques plus utiles que les caractéristiques extérieures. Que reste-til ? Rien d’autre à faire qu’à empoigner les cétacés en masse et à les classer hardiment par ordre de grandeur. C’est le système bibliographique que j’adopte ici et le seul possible. Je continue :\par
Livre (in-folio), Chapitre IV (la jubarte ou baleine à bosse). On voit souvent cette baleine au large de la côte nord de l’Amérique où elle a souvent été prise et remorquée vers les ports. Elle porte un véritable ballot de colporteur sur le dos et vous pourriez aussi bien la nommer baleine-éléphant ou « howdah ». Quoi qu’il en soit, ce nom qu’on lui donne vulgairement ne suffit pas à la distinguer car le cachalot porte également une bosse, mais de moindre importance. L’huile qu’elle fournit n’a pas grande valeur. Elle est « à fanons ». C’est la plus joueuse \hspace{1em}et la plus gaie des baleines, elle soulève une écume plus pétillante et plus blanche que les autres cétacés.\par
Livre I (in-folio), Chapitre V (rorqual ou baleine à dos de rasoir). On sait peu de choses de cette baleine en dehors de son nom. Je l’ai aperçue au large du cap Horn. De nature repliée, elle évite également les chasseurs et les philosophes. Bien qu’elle ne soit pas lâche, elle ne montre jamais autre chose que son dos qui forme une longue arête tranchante. Laissons-la courir. Je ne sais rien de plus à son sujet et personne d’autre non plus d’ailleurs.\par
Livre I (in-folio), Chapitre VI (« sulphur bottom »). Autre gentilhomme-ermite dont le ventre a dû prendre sa couleur soufrée à racler les toits de l’enfer lorsqu’il sonde les plus grandes profondeurs. On le voit rarement, du moins je ne l’ai jamais vu ailleurs que dans les plus lointaines mers du Sud et point d’assez près pour étudier sa mine. On ne le chasse pas car il s’enfuirait avec la production de lignes de toute une corderie. On raconte à son sujet des merveilles. Adieu « sulphur bottom » ! Je ne puis rien dire de plus sur toi sans mentir, et nul des plus vieux Nantuckais n’en pourrait dire davantage.\par
Fin du livre I (in-folio). Début du Livre II (in-octavo).\par
IN-OCTAVO\footnote{La raison pour laquelle ce livre des cétacés n’est pas appelé inquarto est claire. Étant donné que les cétacés de cet ordre, bien que plus petits que ceux de l’ordre précédent, présentent toutefois avec eux une ressemblance relative quant à leurs formes, tandis que l’in-quarto du relieur ne conserve pas en réduction la forme de l’in-folio, ce qui est le cas pour l’in-octavo.}. Celui-ci embrasse les cétacés de taille moyenne parmi lesquels on compte à présent : I : le grampus – II : la baleine-pilote ou poisson noir – III : le narval – IV : le batteur – V : le tueur.\par
Livre II (in-octavo), Chapitre I (le grampus). Ce poisson, dont la respiration, ou plutôt le souffle d’une vigoureuse sonorité a suggéré un proverbe aux terriens, si bien qu’il soit un citoyen des profondeurs, n’est pas communément classé parmi les baleines. Mais étant donné qu’il a toutes les caractéristiques distinctives du léviathan, la plupart des naturalistes l’ont admis comme tel. Il compte parmi les in-octavo de taille moyenne, mesurant de quinze à vingt-cinq pieds de long et d’un tour de taille correspondant. Il voyage en troupes, on ne le chasse pas systématiquement, bien qu’il fournisse une quantité d’huile assez considérable et d’assez bonne qualité. Quelques baleiniers le considèrent comme annonçant à sa suite le grand cachalot.\par
Livre II (in-octavo), Chapitre II (le poisson noir). Je donne à tous ces poissons les noms que leur donnent familièrement les baleiniers, car ce sont en général les meilleurs. Lorsqu’un de ces noms me paraîtra vague ou inapproprié, j’en suggérerai un autre. C’est ce que je vais faire au sujet du poisson noir, ainsi appelé parce que la noirceur est le propre des cétacés. De sorte que, si vous voulez bien, nous l’appellerons baleine-hyène. Sa voracité est bien connue et les commissures internes de ses lèvres remontent de sorte qu’il paraît éternellement ricaner de manière méphistophélique. Il mesure en moyenne de seize à dix-huit pieds de long, on le trouve sous presque toutes les latitudes, il a une façon à lui de montrer en nageant son aileron dorsal courbe qui fait penser à un nez romain. Lorsqu’ils ne trouvent pas de meilleur gibier, les chasseurs de cachalots capturent parfois la baleine-hyène, afin de maintenir leur réserve d’huile inférieure pour les usages courants, tout comme les ménagères économes, lorsqu’elles se trouvent absolument seules à la maison brûlent un suif malodorant plutôt qu’une cire parfumée. Bien que leur couche de lard soit mince, quelques-unes de ces baleines donnent parfois jusqu’à trente gallons d’huile.\par
Livre II (in-octavo), Chapitre III (le narval), c’est-à-dire la baleine à nez. Voilà un exemple de dénomination assez curieux qui doit venir, je pense, de la singulière corne qui dut être prise pour un nez pointu. Cet animal mesure quelque seize pieds de long, tandis que sa corne en a cinq, atteignant même parfois dix pieds et même quinze. À vrai dire, cette corne n’est qu’une longue défense qui prend racine dans l’os de la mâchoire supérieure et se développe à gauche seulement, ce qui rend son propriétaire peu séduisant et lui donne un air emprunté de gaucher. À quoi sert précisément cette défense ou cette lance d’ivoire, il serait difficile de le dire, elle ne semble pas correspondre à l’usage que l’espadon et l’espadon à bec font de leurs épées, quoique des pêcheurs m’aient dit que le narval l’utilisait pour sarcler le fond de la mer, en quête de nourriture. Selon Charley Coffin, il s’en servirait pour percer la glace, car, venant à se trouver dans les mers arctiques sous une couche de glace et cherchant à émerger, il y vrillerait un trou avec sa défense et se libérerait de la sorte. Mais rien ne prouve que ces suppositions soient justes. Quant à moi, je pense que si le narval fait vraiment usage de son unicorne – de quelque façon que ce soit – ce doit être comme d’un coupe-papier lorsqu’il lit des pamphlets. J’ai entendu appeler le narval : baleine à défense, baleine à corne et licorne. C’est certainement l’un des plus curieux exemples d’unicornisme qu’on puisse trouver dans tout le règne animal. Certains auteurs monastiques anciens m’ont appris que cette défense de la licorne de mer était autrefois considérée comme un puissant antidote contre le poison et que les préparations qu’on en tirait atteignaient, dès lors, des prix exorbitants. On la distillait aussi en sels volatils pour dames défaillantes tout comme on en fait avec des cornes de cerf. À l’origine, elle était tenue pour une véritable curiosité en soi. Le recueil d’Hackluyt m’apprend que lorsque la reine Elisabeth fit élégamment un signe de sa main chargée de bagues, d’une fenêtre du Palais de Greenwich, à Sir Martin Frobisher de retour de son voyage, tandis que son hardi vaisseau descendait la Tamise « lorsque Sir Martin revint de ce voyage, dit ce recueil, les genoux ployés il présenta à Son Altesse une corne de narval prodigieusement longue qu’on put voir longtemps après au château de Wind- sor ». Un auteur irlandais déclare que le comte de Leicester s’agenouilla également pour faire présent à Son Altesse d’une autre corne provenant, celle-là, d’une licorne terrestre.\par
Le narval a une robe pittoresque, semblable à celle du léopard, dont le fond est d’une blancheur laiteuse et parsemée de taches noires rondes ou oblongues. Son huile est de qualité supérieure, claire et pure, mais il en fournit peu et il est rarement chassé. On le trouve principalement dans les mers polaires.\par
Livre II (in-octavo), Chapitre IV (le tueur). Les Nantuckais ne savent pas grand-chose de ce cétacé et les naturalistes rien du tout. Pour ce que j’en ai vu à distance, je dirai qu’il est à peu près de la taille d’un grampus. C’est une brute, un cannibale des Fidji ; il attrape parfois aux lippes les baleines du format infolio, s’y suspend comme une sangsue, jusqu’à ce que le monstre puissant soit harassé, jusqu’à la mort. On peut objecter que ce terme de tueur n’est pas distinctif car, tant sur mer que sur terre, nous sommes tous des tueurs : les Bonaparte comme les requins.\par
Livre II (in-octavo), Chapitre V (le batteur). Ce personnage est connu pour sa queue dont il se sert comme d’une férule pour fouetter ses ennemis. Il monte sur le dos des baleines in-folio et, tout en nageant, cingle sa monture et se fraye sa route comme certains maîtres d’école le font dans le monde. On en sait encore moins au sujet du batteur que du tueur. Tous deux sont des hors-la-loi, même dans une mer sans lois.\par
Fin du Livre II. Début du Livre III (in-douze).\par
IN-DOUZE : comprend les cétacés de moindre grandeur : \hspace{1em}I : le marsouin Hourra – II : le marsouin-pirate – III : le marsouin à bec blanc.\par
À ceux qui ne sont pas familiers du sujet, il peut paraître étrange que des poissons ne dépassant pas quatre à cinq pieds soient classés parmi les baleines, le mot évoquant, pour le commun des mortels, la notion d’énormité. Pourtant ces animaux, format in-douze, sont bel et bien des cétacés, selon les termes de ma propre définition : un souffleur ayant une queue horizontale.\par
Livre III (in-douze), Chapitre I (le marsouin hourra). Le marsouin commun fréquente toutes les mers du globe. Ce nom est de mon invention personnelle, car il y a plus d’une espèce de marsouins, et il convient de faire quelque chose pour les distinguer les uns des autres. Je l’ai appelé ainsi parce qu’il nage toujours en bandes joyeuses qui, sur la mer immense, font des bonds en l’air, pareils aux chapeaux de la foule du Quatre Juillet. Les marins saluent avec bonheur leur arrivée. Pleins de bonne humeur, ils viennent invariablement de la vague moutonnante au vent, ce sont des gars qui vivent toujours à vauvent ; ils passent pour être de bon augure. Et si vous-même vous résistez à pousser trois vivats à la vue de ces poissons pleins d’entrain, alors, que le ciel vous vienne en aide, l’esprit enjoué de la sainteté n’est pas en vous. Un marsouin hourra bien nourri et dodu vous donnera un bon gallon de bonne huile, et le liquide pur et délicat que l’on extrait de sa mâchoire a une valeur inestimable, il est très demandé des joailliers et des horlogers et les marins en huilent leur pierre à aiguiser. La chair du marsouin est un plat recherché. Il ne vous est peut-être pas venu à l’esprit qu’un marsouin souffle et à vrai dire son souffle est si faible qu’on le remarque difficilement, mais, à la prochaine occasion, regardez-le bien. Vous verrez alors le modèle réduit du grand cachalot lui-même.\par
Livre III (in-douze), Chapitre II (le marsouin-pirate). Un vrai pirate ! Très féroce. Je crois qu’on ne le trouve que dans le Pacifique. Il est un peu plus grand que le marsouin commun mais d’apparence assez semblable. Provoqué, il s’attaquera à un requin. J’ai souvent mis à la mer pour le chasser mais je n’en ai jamais vu capturer.\par
Livre III (in-douze), Chapitre III (le marsouin à bec blanc). La plus grande espèce de marsouin, pour autant qu’on le sache, ne fréquente que le Pacifique. Le seul nom anglais par lequel on le désigne est celui que lui donnent les pêcheurs : marsouinbaleine franc parce qu’on le trouve dans le voisinage de cet infolio. Il diffère quelque peu du marsouin commun, sa forme est moins rondelette, moins gironde ; en fait, il a une silhouette déliée et aristocratique ; il n’a pas de nageoire dorsale comme la plupart des marsouins, mais une queue ravissante et l’œil indien, sentimental et couleur noisette. Malheureusement, sa bouche enfarinée le défigure. Bien que son dos et sa nageoire caudale soient absolument noirs, il est délimité, sans transition, comme la coque d’un navire par la ligne de flottaison, par ce qu’on appelle la « ceinture claire » qui le sépare de la tête à la queue, noire dessus et blanc dessous. La partie inférieure de sa tête et sa bouche sont entièrement blanches, de sorte qu’il a l’air de sortir d’une coupable visite à un sac de farine, une pauvre mine de papier mâché. Son huile est sensiblement la même que celle du marsouin commun.\par
* * *\par
Il n’y a plus de formats inférieurs à l’in-douze, les marsouins sont les plus petits cétacés. Au-dessus, tous les léviathans marquants. Mais il y a aussi une foule indéterminée, fugace, de baleines à demi fabuleuses que je connais de réputation en tant que baleinier américain, mais pas personnellement. Je les passerai en revue selon les noms qu’on leur donne sur le gaillard d’avant, car une telle liste peut être utile à de futurs \hspace{1em}chercheurs qui pourront la compléter. Si l’on prend et marque les baleines dont les noms suivent, elles pourront aussitôt être cataloguées selon les formats : in-folio, in-octavo et in-douze : la baleine à nez de bouteille, la baleine à pâté, la Poeskop, la baleine du cap de Bonne Espérance, la baleine pilote, la baleine rugueuse, la baleine de Biscaye, la baleine à ventre lisse, la baleine éléphant, la Sei, la baleine grise, la baleine bleue, etc. D’après les autorités islandaises, hollandaises et anglaises, on pourrait dresser encore d’autres listes de baleines mal connues, baptisées de toutes sortes de noms barbares, je les passe sous silence car ce sont des termes inusités et je ne puis m’empêcher de le penser, des mots qui ne signifient rien bien que léviathanesques.\par
Pour finir, comme je l’avais dit au départ, ce système ne saurait trouver ici son aboutissement, vous ne pouvez manquer de reconnaître que j’ai tenu parole, et je laisse ma classification cétologique inachevée, comme la grande cathédrale de Cologne fut abandonnée, la grue demeurant sur la tour non terminée. Les petites constructions peuvent bien être menées à terme par leur premier architecte, mais la postérité doit parfaire les œuvres grandes et vraies. Dieu me garde de jamais achever quoi que ce soit. Tout ce livre n’est qu’une ébauche, non, l’ébauche d’une ébauche ! Ô Temps, Force, Argent et Patience !
\chapterclose


\chapteropen
\chapter[{CHAPITRE XXXIII. Le specksynder}]{CHAPITRE XXXIII \\
Le specksynder}\renewcommand{\leftmark}{CHAPITRE XXXIII \\
Le specksynder}


\chaptercont
\noindent Le lieu ne me paraît pas moins justifié pour aborder à présent la question de l’état-major d’un baleinier et pour parler d’une particularité interne provenant du rang d’officiers conféré aux harponneurs, titres, bien entendu, inconnus dans toute autre marine que baleinière.\par
L’importance majeure du métier de harponneur est prouvée par le fait qu’à l’origine, dans l’ancienne pêcherie hollandaise, il y a plus de deux siècles, le commandement d’un navire baleinier n’incombait pas entièrement à celui qu’on appelle à présent le capitaine, mais était partagé entre celui-ci et un officier dit le {\itshape Specksynder}. Littéralement le mot signifie : coupeur de lard ; toutefois, à la longue l’usage l’identifia à celui de chef harponneur. En ce temps-là, l’autorité du capitaine ne s’exerçait que sur la conduite du navire et son organisation, tandis que le specksynder ou chef harponneur régnait en maître sur tout ce qui concernait la chasse. Dans les pêcheries anglaises du Groenland, ce titre, maintenu, est devenu par corruption {\itshape Specksioneer} mais sa dignité ancienne est tristement diminuée. Il n’est plus, à présent, que le supérieur des harponneurs et, en tant que tel, il se trouve au rang le plus bas des subalternes du capitaine. Toutefois, comme le succès d’une campagne baleinière dépend largement de la compétence des harponneurs, et comme, dans la pêcherie américaine, ceux-ci sont non seulement timoniers et responsables des pirogues, mais encore, en certains cas (quart de nuit sur un terrain de chasse), commandants du pont, il s’ensuit que les règles politiques de la mer réclament qu’ils se distinguent nominalement des simples matelots et soient considérés comme leurs supérieurs, bien que ceux-ci les estiment toujours familièrement comme leurs égaux dans la société.\par
Ce qui établit à bord la grande distinction entre l’équipage et les officiers, c’est que les hommes vivent à l’avant et l’étatmajor à l’arrière du navire. Sur les navires baleiniers comme sur les navires marchands, les seconds ont leurs quartiers avec le capitaine et, sur la plupart des baleiniers américains, les harponneurs vivent également à l’arrière. C’est-à-dire qu’ils prennent leur repas dans la cabine du capitaine et que la chambre où ils dorment communique indirectement avec celle-ci.\par
Malgré la longue durée d’une expédition baleinière dans les mers du Sud (de beaucoup les voyages les plus longs qu’aient jamais entrepris les hommes), malgré les dangers qui lui sont inhérents, et la mise en commun par tous les hommes de leurs intérêts, les bénéfices dépendant, du premier au dernier d’entre eux, non de gages fixes mais de leur chance commune, de leur commune vigilance, de leur courage et de leur dur labeur, malgré toutes ces conditions qui entraînent une discipline moins rigoureuse, parfois, que sur les navires marchands en général, et malgré la vie primitive qui rend ces baleiniers semblables à quelque famille de l’antique Mésopotamie, le protocole, du moins sur le gaillard d’arrière, se relâche rarement et n’est jamais banni. En vérité, sur bien des navires de Nantucket, vous pourrez voir le capitaine parader avec une majesté délirante qui n’a pas sa pareille sur les vaisseaux de guerre, et extorquer les hommages comme s’il était vêtu de la pourpre impériale et non du drap de pilote le plus râpé.\par
Le ténébreux capitaine du {\itshape Péquod} était le dernier homme à avoir de si mesquines prétentions, le seul hommage qu’il exigeait était celui d’une obéissance absolue et instantanée. Il ne demandait pas aux hommes de se déchausser pour venir sur le gaillard \hspace{1em}d’arrière. \hspace{1em}Dans des circonstances \hspace{1em}particulières, que nous viendrons à raconter, il s’adressait à eux d’une manière insolite, tantôt avec condescendance, tantôt {\itshape in terrorem}. Pourtant, le capitaine Achab n’était nullement insoucieux de l’étiquette et des usages de la mer.\par
Mais peut-être pouvait-on supposer que ce respect des coutumes lui servait de masque car il s’en servait, à l’occasion, à des fins personnelles ou différentes de ce pour quoi elles avaient été établies. Son âme de sultan, qui ne se serait guère révélée autrement, filtrait à travers le protocole et sa volonté de puissance prenait corps dans une dictature qui n’autorisait pas la résistance. Car, si grande que soit la supériorité d’esprit d’un homme, il ne peut prétendre gagner une tangible suprématie, sans user de dissimulation et d’artifices toujours plus ou moins vils et indignes. Les princes du Saint Empire romain étaient préservés des campagnes électorales et les plus grands honneurs revenaient à ceux qui se rendirent fameux davantage grâce à leur infinie infériorité spirituelle qu’à leur supériorité réelle sur la masse. Mais les vanités détiennent un tel ascendant surtout lorsqu’elles sont doublées de superstitions politiques, que nous avons de royaux exemples où le pouvoir a été investi aux plus parfaits imbéciles. Mais lorsque la couronne encercle à la fois un cerveau impérial et un empire immense, comme celle du Tsar Nicolas, alors le troupeau de la plèbe rampe humblement devant une puissance aussi formidable. Et le poète tragique qui voudra peindre l’élan et l’envolée de la nature indomptable de l’homme fera bien, pour le bénéfice de son art, de se souvenir de l’exemple auquel nous venons de faire allusion.\par
Mais Achab, mon capitaine, m’est toujours présent dans sa mélancolique rudesse nantuckaise, et cette parenthèse sur les Empereurs et les Rois ne me dissimule pas qu’avec lui je n’ai affaire qu’à un pauvre vieux chasseur de baleines, et que me sont refusés dès lors tous les atours et tous les palais de la Majesté. Ô Achab ! pour révéler ta grandeur il faut l’aller \hspace{1em} arracher au\hspace{1em}ciel,\hspace{1em}l’aller\hspace{1em}cueillir\hspace{1em}dans\hspace{1em}les\hspace{1em}profondeurs\hspace{1em}et\hspace{1em}la\hspace{1em}modeler d’immatériel espace.
\chapterclose


\chapteropen
\chapter[{CHAPITRE XXXIV. La table de la chambre}]{CHAPITRE XXXIV \\
La table de la chambre}\renewcommand{\leftmark}{CHAPITRE XXXIV \\
La table de la chambre}


\chaptercont
\noindent Il est midi, et Pâte-Molle, le garçon, passant sa tête de miche mal cuite par l’écoutille de la chambre, annonce que le repas est servi à son seigneur et maître. Celui-ci, assis dans la baleinière sous le vent, vient d’observer le soleil et calcule en silence la latitude sur la tablette lisse et ronde ménagée pour cet usage quotidien sur la partie supérieure de sa jambe d’ivoire. Vu sa parfaite indifférence à l’invitation de son domestique, vous pourriez croire que le sombre Achab ne l’a pas entendue. Mais, s’agrippant bientôt aux haubans d’artimon, il se hisse sur le pont et d’une voix posée et sans joie il annonce : « Dîner, monsieur Starbuck » et disparaît dans la chambre.\par
Quand le dernier écho de son pas de sultan s’est évanoui et que Starbuck, son premier émir, a toute raison de supposer qu’il est assis à table, Starbuck émerge de sa torpeur, fait quelques pas sur le pont, jette un regard plein de gravité dans l’habitacle, dit, non sans humour : « Dîner monsieur Stubb », puis s’engouffre dans l’écoutille. Le deuxième émir flâne un instant autour du gréement, secoue légèrement le bras de grand hunier, pour s’assurer que tout va bien du côté de cet important filin, puis, comme les autres, reprend le vieux fardeau, et après un rapide : « Dîner, monsieur Flask » s’engage à la suite de ses prédécesseurs.\par
Mais le troisième émir, se retrouvant seul sur le gaillard d’arrière, semble soulagé d’un poids étrange, il lance de tous côtés des regards entendus, enlève ses souliers d’un coup de pied et se jette dans la bourrasque d’une matelote effrénée mais silencieuse juste au-dessus de la tête du Grand Turc. Par un habile tour de passe-passe, il envoie son couvre-chef s’accrocher à la hune d’artimon et se livre à ses ébats folâtres, pour autant qu’il soit visible du pont, et au rebours des autres cortèges, il ferme le sien avec la musique. Mais dès qu’il a posé pied sur le seuil de la chambre, il s’arrête, se compose un tout autre visage et alors le joyeux, l’indépendant petit Flask, affronte la présence du roi Achab dans le rôle de l’esclave abject.\par
Ce n’est pas la moindre étrangeté engendrée par le caractère artificiel du protocole à bord, que de voir les officiers, sur une quelconque provocation, tenir tête parfois avec hardiesse à leur commandant alors que, neuf fois sur dix, ces mêmes officiers, l’instant d’après, prenant leur repas habituel dans la cabine du capitaine, deviennent sur-le-champ des moutons humbles pour ne pas dire avilis, devant celui qui siège à la place d’honneur. C’est étonnant, et parfois du plus haut comique. À quoi tient ce changement ? Y a-t-il là un mystère ? Peut-être pas. Avoir été Balthazar, roi de Babylone, et avoir été Balthazar, sans morgue mais avec courtoisie, ne devait pas manquer de grandeur. Mais celui qui préside à la table où il reçoit ses hôtes avec tact et intelligence, sa puissance et sa supériorité personnelles sont alors incontestées, sa majesté l’emporte même sur celle de Balthazar car Balthazar ne fut pas le plus grand. Qui a reçu, ne fût-ce qu’une seule fois, des amis à sa table, a su ce que signifiait d’être un César. Il y a là un sortilège du tsarisme social qui est irrésistible. Si vous ajoutez à cette considération le prestige du maître à bord, vous comprendrez la raison de cette singulière solennité.\par
À sa table incrustée d’ivoire Achab présidait, comme un lion de mer, muet sous sa crinière, règne sur une grève de corail blanc, entouré de ses petits agressifs mais pleins de respect. Chaque officier attendait son tour d’être servi, tous étaient de petits enfants devant Achab qui pourtant ne donnait aucun signe de préjugé social. D’une seule âme, ils suivaient, avec une attention soutenue, le couteau du vieil homme s’attaquant au plat de résistance. Je pense que pour rien au monde ils n’eussent profané cet instant par la plus anodine des observations, fût-ce quelque lieu commun sur le temps qu’il faisait. Non ! Et lorsque, enlevant son couteau et sa fourchette entre lesquels était retenue la tranche de bœuf, il passait ensuite cette assiette à Starbuck, le second recevait sa viande comme une aumône, la découpait tendrement, sursautant si, par hasard, le couteau avait grincé sur l’assiette, la mâchait en silence et l’avalait précautionneusement. Comme au banquet du Couronnement, à Francfort, l’Empereur germanique dîne en secret avec les sept Électeurs, la gravité caractérise ces repas péniblement silencieux. Ce n’est pas qu’à sa table le vieil Achab interdise toute conversation mais lui-même est muet. Quel soulagement n’éprouva-t-il pas Stubb lorsque, au moment où il s’étranglait, un rat mena subitement un ramdam dans la cale. Le pauvre petit Flask était le benjamin, le petit garçon, de cette accablante réunion de famille. C’est à lui que revenait l’os du jarret de bœuf salé, à lui que serait revenu l’os du pilon s’il y avait eu du poulet. L’idée même de se servir tout seul à table eût paru à Flask au suprême degré l’équivalent d’un vol ; l’eût-il fait que, sans aucun doute, il n’eût plus jamais osé lever la tête devant les honnêtes gens ; toutefois, c’est étrange à constater, Achab ne le lui avait jamais interdit et l’eût-il fait, il y a fort peu de chances pour qu’Achab s’en fût jamais aperçu. C’est de beurre surtout que Flask n’aurait jamais osé se servir. Soit qu’il pensât que les propriétaires du navire le lui refusaient pour ne pas mouiller son teint clair et ensoleillé, soit qu’il jugeât qu’au cours d’un si long voyage sans possibilité d’achats le beurre fût une récompense qui n’était pas destinée à un subalterne comme lui, quelle qu’en fût la raison Flask était, hélas, un homme privé de beurre !\par
Autre chose. Flask, le dernier à descendre, est le premier à remonter. Songez-y ! Car de ce fait son repas devait être englouti à une triste vitesse. Starbuck et Stubb avaient tous deux de l’avance sur lui en outre le privilège de pouvoir s’attarder. Et s’il advenait que Stubb, de bien peu le supérieur de Flask, n’eût guère d’appétit et manifestât rapidement les symptômes de la satiété, alors Flask devait se démener et se contenter de trois bouchées ce jour-là car il était contre le sacro-saint usage que Stubb précédât Flask sur le pont. Le résultat, Flask le reconnut une fois dans l’intimité, fut qu’il vécut toujours plus ou moins affamé depuis qu’il avait accédé à la dignité d’officier. Car ce qu’il mangeait, loin de calmer sa faim, l’aiguisait éternellement. La paix et le contentement ont à jamais déserté mon estomac, pensait Flask. Je suis officier mais je voudrais pouvoir pêcher un bout du traditionnel bœuf salé dans la gamelle du gaillard d’avant comme je le faisais quand j’étais matelot. Voilà bien les fruits de l’avancement, la vanité de la gloire, l’insanité de la vie ! D’autre part, si le moindre matelot du {\itshape Péquod} avait une dent contre Flask, une rancune d’ordre professionnel, tout ce que ce matelot avait à faire pour obtenir une vengeance éclatante, c’était de se glisser jusqu’au gaillard d’arrière à l’heure du repas et de jeter un coup d’œil par la claire-voie de la cabine et regarder Flask assis là, stupide et ahuri devant le terrible Achab.\par
Achab et ses trois seconds formaient ce qu’on pourrait appeler le premier service dans le carré du {\itshape Péquod}. Après leur départ, dans l’ordre inverse de celui de leur arrivée, la nappe de toile était débarrassée, ou tout au moins vaguement mise en ordre, par le garçon blafard. Alors les trois harponneurs étaient conviés au festin dont ils étaient les légataires universels. Le lieu saint devenait par leur présence l’office des domestiques.\par
La démocratie forcenée de ces êtres inférieurs qu’étaient les harponneurs formait, grâce à son laisser-aller et ses aises insouciantes, le plus étrange contraste avec la contrainte presque insupportable et la tyrannie invisible et indicible qui régnait à la table du capitaine. Tandis que leurs officiers semblaient redouter le bruit de leurs propres mâchoires les harponneurs mastiquaient leur nourriture avec un plaisir spectaculaire. Ils mangeaient comme des rois et se remplissaient le ventre comme on remplit longuement de cargaisons d’épices les navires en partance des Indes. Queequeg et Tashtego avaient des appétits si pantagruéliques que, pour faire compensation au repas précédent, le pâle Pâte-Molle était contraint d’apporter un gros aloyau de bœuf salé qui paraissait avoir été taillé à l’instant dans l’animal vivant. Et s’il ne s’activait pas à l’aller quérir, s’il n’y allait pas presto en deux temps trois mouvements, alors Tashtego avait une façon très peu gentilhommesque de le presser en lui harponnant sûrement le derrière avec une fourchette. Et Daggoo, pris d’une soudaine colère, aidait Pâte-Molle à bien se souvenir de sa tâche, en l’empoignant et en lui maintenant la tête sur un grand tranchoir, cependant que Tashtego traçait le délimité du scalp. Il était par nature craintif et frissonnant, ce petit gars de mie de pain, le fruit d’un boulanger failli et d’une infirmière mais toute sa vie n’était plus qu’une lutte quotidienne contre les charmes, devant le sombre et terrible Achab et face aux visites répétées et tumultueuses de ces trois sauvages. Bien souvent, après avoir pourvu les harponneurs de tout ce qu’ils demandaient, il échappait à leurs serres dans le petit office adjacent et les épiait avec crainte, à travers les volets, jusqu’à ce qu’ils en aient terminé.\par
C’était un spectacle que de voir briller les dents aiguës de Queequeg en face des dents aiguës de l’Indien ; entre eux, Daggoo était assis à même le sol car, de la hauteur du banc, sa tête emplumée comme un corbillard aurait touché les hiloires renversées ; à chaque mouvement de ses membres colossaux, il ébranlait toute la charpente de la chambre, comme un éléphant d’Afrique embarqué comme passager. Malgré cela, le grand nègre était admirablement sobre, pour ne pas dire délicat. Il paraissait à peine possible qu’avec d’aussi minuscules bouchées il pût conserver toute sa vitalité diffuse dans une personne aussi large, seigneuriale et magnifique, mais sans doute ce noble sauvage se nourrissait-il copieusement et se désaltérait-il profondément de grand air, ses narines dilatées aspirant la vie sublime de l’univers. Ce n’est pas avec du bœuf ou du pain qu’on fait ou qu’on nourrit les titans. Queequeg lui, avait un claquement de lèvres bien humain et barbare – assez repoussant – à tel point que le frémissant Pâte-Molle en venait presque à examiner son bras mince pour voir si des marques de dents n’y étaient pas imprimées. Et lorsque Tashtego lui claironnait d’approcher afin qu’on lui ronge les os, ce garçon simplet entrait dans de telles transes qu’il manquait de faire choir toute la vaisselle traînant autour de lui dans l’office. Les pierres à aiguiser que les harponneurs transportaient dans leurs poches qu’ils sortaient ostensiblement au repas pour y aiguiser leurs couteaux, le bruit grinçant qui s’ensuivait, n’étaient pas pour tranquilliser si peu que ce soit le pauvre Pâte-Molle. Comment pouvait-il oublier que Queequeg entre autres, lorsqu’il vivait dans son île, s’était, sans aucun doute, rendu coupable de festins meurtriers et indiscrets. Hélas ! Pâte-Molle ! C’est une cruelle situation pour un blanc que celle qui le contraint à servir à table des cannibales. Ce n’est pas une serviette qu’il doit porter au bras mais un bouclier. Pourtant, le moment venu et à son grand soulagement, ces trois loups de mer guerriers se levaient pour partir ; à ses oreilles crédules aux fables médisantes leurs os tintinnabulaient de façon martiale à chaque pas comme des cimeterres mauresques cliquettent dans leurs fourreaux.\par
Mais bien que ses barbares dînassent dans le carré, et fussent censés y séjourner, n’étant nullement de mœurs sédentaires, ils n’y faisaient guère d’autre apparition qu’au moment des repas et juste avant d’aller se coucher quand ils la traversaient pour regagner leurs quartiers personnels.\par
À ce sujet, Achab ne différait pas des autres capitaines baleiniers américains qui considèrent que la chambre du navire leur appartient de droit et que c’est pure amabilité de leur part si, à l’occasion, ils y tolèrent qui que ce soit. De sorte que, on peut bien le dire en toute vérité, les seconds et les \hspace{1em}harponneurs du {\itshape Péquod} vivaient plutôt hors de la cabine qu’à l’intérieur. Car lorsqu’ils y entraient c’était joliment à la manière dont une porte d’entrée pénètre un instant dans une maison pour être rabattue à l’extérieur l’instant suivant et livrée aux intempéries tout le reste du temps. Ils n’y perdaient pas grand-chose, la fraternité ne régnait pas au carré, Achab était humainement inaccessible. Bien qu’il fît partie en titre de la chrétienté, il y était parfaitement étranger. Il vivait sur la terre comme le dernier grizzli vit dans le Missouri colonisé. Lorsque le printemps et l’été se sont enfuis, ce sauvage Logan des bois, s’enfouissant dans le cœur d’un arbre creux, y passe l’hiver à se pourlécher les pattes. Mêmement, en son vieil âge inclément et orageux, l’âme d’Achab, enfermée dans le tronc évidé de son corps, se nourrit aux mornes paumes de ses ténèbres !
\chapterclose


\chapteropen
\chapter[{CHAPITRE XXXV. La tête de mât}]{CHAPITRE XXXV \\
La tête de mât}\renewcommand{\leftmark}{CHAPITRE XXXV \\
La tête de mât}


\chaptercont
\noindent Il faisait un temps magnifique lorsque vint mon tour de monter, pour la première fois, à la tête du mât.\par
Sur la plupart des navires baleiniers américains, des guetteurs s’installent aux têtes de mâts presque dès l’instant où le navire quitte le port, même s’il a quinze mille milles à parcourir avant d’atteindre son lieu de pêche. Et si après trois, quatre ou cinq ans de voyage il rentre avec quoi que ce soit de vide – la moindre fiole peut-être – ses têtes de mâts sont armées jusqu’au dernier moment. Lorsque ses vergues de cacatois s’avancent parmi les mâtures du port, il abandonne enfin l’espoir d’attraper une dernière baleine.\par
Comme cette façon de guetter au sommet du mât, que ce soit à terre ou en mer, est très ancienne et très intéressante, qu’on nous permette ici une digression. Je pense que les premiers hommes à être en vigie à la hune furent les anciens Égyptiens, car je n’en ai pas trouvé de mention antérieure au cours de mes recherches, bien que leurs ancêtres, en édifiant la tour de Babel, n’aient eu, sans aucun doute, d’autre intention que celle d’élever le ton de mât la plus haute de toute l’Asie ou d’Afrique. Toutefois (avant que n’y fût établie la dernière plateforme) on peut dire que leur grand mât de pierre fut emporté par la lame soulevée par la tempête redoutable de la colère divine, c’est pourquoi nous ne saurions donner le pas aux bâtisseurs de Babel sur les Égyptiens. Pour dire que ceux-ci furent une nation de guetteurs nous nous fondons sur l’opinion répandue chez les archéologues voulant que les premières pyramides aient été des observatoires. Après une interminable ascension, ces anciens astronomes avaient coutume de se tenir à leur sommet et d’annoncer l’apparition de nouvelles étoiles comme, de nos jours, le guetteur d’un navire annonce à voix forte une voile ou une baleine. Parmi les saints stylites, un fameux ermite chrétien des temps anciens se construisit un éminent pilier de pierre dans le désert et passa, perché à son sommet, toute la dernière moitié de sa vie, hissant sa nourriture avec un palan ; il fournit un bon exemple d’intrépide guetteur de hune, inamovible malgré les brouillards ou les gels, la pluie, la grêle ou la neige, faisant face à tout jusqu’au dernier moment et mourant à son poste. Aujourd’hui nous n’avons plus que des guetteurs privés de vie : les hommes de pierre, de fer ou de bronze qui, bien que capables d’affronter une tempête drue, se montrent tout à fait incompétents pour annoncer une découverte insolite. Voici Napoléon ! les bras croisés, il se tient à la pointe de la colonne Vendôme, à quelque cent cinquante pieds en l’air, tout à fait indifférent désormais à qui commande sur le pont au-dessous de lui, que ce soit Louis-Philippe, Louis Blanc ou Louis le Diable. Le grand Washington est juché lui aussi à son grand mât dominant Baltimore et, comme celle d’Hercule, sa colonne témoigne d’une grandeur humaine que bien peu dépasseront. L’amiral Nelson, sur son cabestan de bronze, veille sur Trafalgar Square, et même lorsque l’épaisse fumée de Londres l’enveloppe, on sait qu’un héros caché est présent car il n’y a pas de fumée sans feu. Mais ni Washington, ni Napoléon, ni Nelson ne répondront jamais à un appel venu d’en bas, si éperdument que les êtres tourmentés qu’ils surplombent les supplient de leur venir en aide avec leurs conseils. Pourtant on peut présumer que leurs esprits pénètrent la brume opaque du futur et distinguent les écueils.\par
Il peut paraître immotivé d’associer si peu que ce soit les hommes de vigie à la hune à terre avec ceux qui le sont en mer, pourtant il n’en est rien, un détail rapporté par Obed Macy, le seul historien de Nantucket, suffit à le prouver. L’honorable Obed nous raconte qu’aux premiers temps de la chasse à la baleine, alors que les navires étaient régulièrement lancés à la poursuite du gibier, les habitants de l’île dressaient de hautes perches le long des côtes, où les guetteurs grimpaient grâce à de petites traverses de bois clouées, à peu près à la façon dont les poules regagnent leur poulailler. Il y a quelques années une tactique similaire fut utilisée par les pêcheurs en baie de Nouvelle- Zélande qui, ayant dépisté le gibier, en avertissaient les hommes des pirogues parées qui se trouvaient à la côte prêts à partir. Cette coutume étant tombée en désuétude, revenons-en au guetteur de hune d’un baleinier au large. Les postes de vigie sont armés aux trois tons de mâts du lever jusqu’au coucher du soleil, les marins y prennent leur quart comme à la barre et se relayent toutes les deux heures. Le sommet du mât est un lieu agréable à l’extrême sous le ciel tranquille des tropiques. Que dis-je ? pour un rêveur, pour un contemplatif, c’est le comble des délices. Vous êtes là, à cent pieds au-dessus des ponts silencieux, marchant sur l’Océan comme si les mâts vous servaient de gigantesques échasses, tandis qu’au-dessous, entre vos jambes, passent les monstres les plus énormes de la mer, à la manière dont les navires passèrent une fois entre les bottes du fameux colosse de Rhodes. Vous êtes là, perdu dans un infini marin que seule trouble la vague. Le navire roule dans une indolente extase. Les Alizés soufflent à demi assoupis, tout vous pénètre de langueur. Une grande partie du temps ces expéditions baleinières sous les tropiques vous accordent la splendeur d’une vie dépourvue de tout événement. Vous n’apprenez aucune nouvelle, vous ne lisez aucun journal, vous ne savez rien des malheurs domestiques, rien des valeurs qui font banqueroute, des produits en baisse, aucune édition spéciale ne vous annonce, sous des titres à sensation, ses trivialités, rien ne vous invite à une inutile agitation. Vous ne vous préoccupez jamais de savoir ce que vous aurez pour le repas car, pour trois ans et plus, tous vos repas sont douillettement en réserve dans des futailles et le menu est immuable.\par
Sur l’un de ces baleiniers des mers du Sud, au cours d’un si long voyage, le total des heures passées à la hune équivaudrait bien souvent à plusieurs mois et l’on peut sincèrement déplorer qu’un endroit auquel vous donnez une si grande partie de votre vie soit si tristement dépourvu de quoi que ce soit qui rappelle le confort, ou qui puisse vous donner le sentiment heureux d’un abri, comme le fait un lit, un hamac, un corbillard, une guérite, une chaire, un coche, ou n’importe laquelle de ces petites inventions que l’homme a réalisées pour savourer temporairement sa solitude. Votre perchoir habituel se trouve à la tête du mât de perroquet, où vous vous tenez debout sur deux minces bâtons parallèles (presque uniquement particuliers aux baleiniers) qui s’appellent barres de cacatois. Là, le novice, secoué par les flots, se sent à peu près aussi à l’aise que sur les cornes d’un taureau. Vous pouvez bien par temps froid emporter votre maison avec vous sous la forme d’un caban, mais à vrai dire le manteau le plus épais n’est pas davantage une maison qu’une peau toute nue, car de même que l’âme s’englue dans son tabernacle de chair et ne peut s’y mouvoir librement, ni s’en échapper sans courir le risque de périr (comme un pèlerin ignorant qui passerait, en hiver, les Alpes enneigées), ainsi un épais caban n’est pas tant une maison qu’une simple enveloppe, une peau supplémentaire adhérant à vous. Vous ne pouvez mettre une étagère ou une commode dans votre corps, vous ne pouvez pas davantage transformer votre manteau en chambre agréable.\par
À ce sujet, on peut bien déplorer que les tons de mâts d’un baleinier des mers du Sud ne soient pas nantis de ces enviables petites tentes, ou chaires, appelées nids-de-pie, qui protègent les guetteurs des baleiniers groenlandais contre les rigueurs des océans arctiques. Dans le livre à lire au coin du feu du capitaine Sleet, intitulé : « Un voyage parmi les icebergs, à la poursuite de la baleine du Groenland et à la re-découverte éventuelle des colonies islandaises perdues du vieux Groenland », dans cet admirable ouvrage, il est raconté en détail et de la façon la plus charmante comment les guetteurs en vigie à la hune sont pourvus, à bord du {\itshape Glacier}, le bon navire du capitaine Sleet, de ces nids-de-pie d’invention récente. Il les avait appelés Nids-de-pie de Sleet, car c’est lui qui les avait imaginés et réalisés et, libre de toute ridicule fausse modestie, jugeant que si nos enfants prennent notre nom (car nous, pères, sommes les inventeurs brevetés de nos enfants), tout dispositif que nous créons doit de même porter notre patronyme. Le nid-de-pie se présente comme un gros tuyau ou comme un fût, ouvert toutefois sur le dessus, où un écran mobile du côté du vent protège votre tête des tempêtes. Comme il est fixé au sommet du mât, on y accède par une trappe ménagée dans le fond. Sur le côté qui donne vers l’arrière du navire, se trouve un siège confortable, sous lequel un petit placard est destiné à ranger les parapluies, cache-nez et manteaux. Sur le devant, un casier de cuir permet de poser une pipe, le porte-voix, la longue-vue et les instruments nautiques. Lorsque le capitaine Sleet veillait en personne dans son sien nid-de-pie, il nous raconte qu’il avait toujours son fusil dans le casier, ainsi qu’une poire à poudre et des balles pour faire déguerpir les narvals, licornes vagabondes des mers, qui infestaient ces eaux, car on ne peut pas les tirer depuis le pont à cause de la résistance de l’eau, mais les tirer de cette altitude est une tout autre affaire. Ce fut nettement, pour le capitaine Sleet, une œuvre d’amour que de conter par le menu tous les agréments de son nid-de-pie, mais bien qu’il s’étende à perte de vue sur certains d’entre eux, et bien qu’il nous régale de rapports scientifiques sur ses expériences qu’il faisait, là-haut, avec un compas afin de corriger les erreurs provoquées par ce qu’on appelle « l’attraction locale » sur l’aiguille du compas de l’habitacle, attraction due au proche voisinage de fer dans les bordages, et peut-être dans le cas particulier du {\itshape Glacier} au fait qu’il y avait à son bord une équipe de dépannage si nombreuse de forgerons, je dis donc que, quoique le capitaine Sleet se montre si avisé et si scientifique dans toutes ses connaissances : « déviations de l’habitacle », « observations au compas azimutal » et « approximations d’erreurs », il sait très bien, le capitaine Sleet, qu’il n’était pas tellement plongé dans de profondes méditations magnétiques, qu’attiré, de temps en temps, vers le flacon carré fidèlement repourvu, et si gentiment installé, juste à portée de sa main, dans un coin de son nid-de-pie. Bien qu’en général, j’admire profondément et je dirais même que j’aime ce brave, honnête et savant capitaine, je trouve très laid de sa part d’avoir complètement passé sous silence ce flacon, conscient que je suis de l’ami fidèle et réconfortant qu’il dut être, tandis que, mitaines aux doigts et tête encapuchonnée, il se livrait à l’étude des mathématiques dans son altier nid d’oiseau à quelque 3 ou 4 perches du Pôle.\par
Mais si nous, baleiniers du Sud, nous ne sommes pas aussi confortablement logés dans les airs que le capitaine Sleet et ses Groenlandais, ce désavantage est largement compensé par la séduisante sérénité de ces mers si différentes sur lesquelles nous voguons le plus souvent. Le tout premier, je montai nonchalamment au gréement, m’attardant à tailler une bavette, avec Queequeg ou quiconque n’était pas de quart et que je venais à rencontrer, puis, poursuivant mon ascension, et jetant une jambe paresseuse par-dessus la vergue de hunier, je contemplai une première fois les pâturages marins, pour regagner enfin ainsi ma destination ultime.\par
Qu’on me permette ici de décharger ma conscience, et d’avouer franchement que j’étais un piteux guetteur. Ressassant le problème de l’univers, comment pouvais-je – entièrement livré à moi-même à une altitude si favorable à l’exercice de l’esprit – comment pouvais-je ne pas prendre à la légère mon devoir en obéissant aux ordres traditionnels à bord d’un baleinier : « Ouvrez bien l’œil au vent et donnez de la voix à chaque fois ! »\par
Et laissez-moi, ô vous armateurs nantuckais, vous faire ici une pathétique semonce ! Prenez garde de ne pas enrôler parmi vos pêcheurs vigilants un gars aux tempes creuses, aux yeux enfoncés, prédisposé à une méditation inopportune et qui propose sa candidature en ayant en tête Phédon plutôt que Bowditch. Prenez garde, vous dis-je, à un individu de ce genre ! Vos baleines doivent être vues avant d’être tuées, et ce jeune platonicien aux yeux caves vous entraînera dix fois dans son sillage autour du monde et ne vous enrichira jamais d’une pinte de spermaceti. Ces avertissements ne sont pas inutiles car, de nos jours, les navires baleiniers sont devenus des asiles pour bien des jeunes gens romantiques, mélancoliques et distraits, dégoûtés des soucis rongeurs du monde, en quête d’émotion dans le goudron et la graisse. Souvent Childe Harold monte à la tête de mât d’un baleinier malchanceux et, déçu, s’écrie sombrement :\par
Roule, profond Océan ton ténébreux azur, roule Dix mille chasseurs de lard te sillonnent en vain. \par
Bien souvent les capitaines de tels navires réprimandent des jeunes philosophes dans les nuages, leur reprochant de ne pas éprouver un « intérêt » suffisant pour le voyage, laissant entendre à demi qu’ils sont à jamais perdus pour toute noble ambition, car, dans le secret de leur cœur, ils souhaitent ne pas voir les baleines. Vaine leçon, car ces jeunes platoniciens ont dans l’idée qu’ils ont une mauvaise vue, qu’ils sont myopes et qu’il est dès lors inutile de se fatiguer le nerf optique. Ils ont laissé à la maison leurs jumelles de théâtre.\par
« Espèce de singe, disait un harponneur à l’un de ces gars, voilà bientôt trois ans que nous sommes en croisière, et tu n’as pas levé une seule baleine. Chaque fois que tu es là-haut, les baleines sont aussi rares que des dents aux poules. » Et peut-être l’étaient-elles. Ou peut-être qu’il s’en trouvait des troupes au lointain horizon. Mais cet adolescent, visionnaire, perdu jusqu’à l’indifférence dans une rêverie inconsciente, bercé comme dans les fumées de l’opium, par le rythme pareil des vagues et des songes, en vient à être dépossédé de lui-même ; le mystique Océan, déroulé à ses pieds, n’est plus pour lui que l’image révélée de l’âme bleue, profonde et insondable, diffuse dans l’humanité et dans la nature ; et chaque merveille entrevue étrange, prestigieuse le fuit en glissant, tout aileron, émergeant indistinct, lui semble l’incarnation de ces pensées évasives qui ne font que traverser l’esprit, et l’esprit lui-même, dans cet enchantement, reflue jusqu’au principe et s’épanche hors du temps et de l’espace, comme les cendres éparpillées du panthéiste Cranmer sont devenues partie infime de toutes les grèves du monde.\par
Tu n’as plus d’autre vie en toi, à présent, que cette vie balancée dévolue par le doux roulis du navire qui l’emprunte à la mer comme la mer le prend aux impénétrables marées de Dieu. Mais tandis que ce sommeil, ce songe t’enveloppe, si tu remues si peu que ce soit ton pied ou ta main, que tu lâches un peu ta prise, alors ton identité se rue à nouveau en toi dans l’épouvante. Tu planes vertigineusement au-dessus des tourbillons cartésiens. Et peut-être qu’au zénith, par un temps radieux, avec un cri étranglé, tu traverseras de ta chute cet air lumineux et la mer estivale se refermera sur toi pour jamais. Qu’il t’en souvienne, ô panthéiste !
\chapterclose


\chapteropen
\chapter[{CHAPITRE XXXVI. Le gaillard d’arrière}]{CHAPITRE XXXVI \\
Le gaillard d’arrière}\renewcommand{\leftmark}{CHAPITRE XXXVI \\
Le gaillard d’arrière}


\chaptercont
\noindent (Entre Achab, puis tous.) \par
Peu de temps après l’affaire de la pipe, Achab monta à son habitude, tôt après le petit déjeuner, de la cabine au pont. C’est là que la plupart des capitaines marins font d’ordinaire leur promenade matinale, comme les gentilshommes campagnards font quelques pas dans leurs jardins après cette première collation.\par
Bientôt on entendit son pas d’ivoire régulier aller et venir, tandis qu’il faisait sa tournée coutumière sur des planches si longuement martelées par sa marche qu’elles étaient partout creusées, comme les pierres d’âge géologique, par sa marque particulière. Et si vous aviez regardé intensément son front sillonné et ravagé vous y eussiez découvert les traces de pas plus étranges encore laissées par une pensée obsédante, toujours en marche et sans sommeil.\par
Ce jour-là, ces empreintes paraissaient plus profondes, comme si son pas nerveux eût été plus appuyé. Et Achab était si hanté par sa pensée qu’à chaque tour monotone qu’il accomplissait, tantôt au grand mât, tantôt à l’habitacle, vous auriez presque pu voir son idée fixe tourner avec lui, arpenter son âme comme il arpentait le pont, le possédant si absolument que tous ses mouvements visibles semblaient être le moule de sa réflexion.\par
– Le voyez-vous, Flask ? murmura Stubb, le poussin qu’il porte en lui frappe à la coquille. Il ne va pas tarder à sortir.\par
Les heures passèrent. Tantôt Achab s’enfermait dans la cabine, tantôt il mesurait le pont avec la même apparence de viser un but avec une ardeur fanatique.\par
Le jour tirait à sa fin. Soudain, il s’arrêta près de la lisse, fixa son pied d’ivoire dans le trou de tarière, s’accrocha d’une main à un hauban et ordonna à Starbuck de rassembler tout le monde à l’arrière.\par
– Sir ! s’écria le second surpris par un ordre rarement ou jamais donné à bord sauf dans des circonstances exceptionnelles.\par
– Tout le monde à l’arrière, répéta Achab. En bas, les hommes de vigie !\par
Lorsque tout l’équipage fut réuni, contemplant Achab avec un mélange de curiosité et de crainte tant il ressemblait à un horizon menaçant, prometteur d’une proche tempête, celui-ci, après avoir jeté un rapide coup d’œil par-dessus les pavois, puis transpercé les hommes du regard, recommença lourdement sa ronde sur le pont comme s’il n’y avait âme qui vive auprès de lui. La tête penchée, le chapeau rabattu à demi, il continua d’aller et de venir, indifférent aux chuchotements étonnés des hommes, jusqu’à ce que Stubb murmurât avec précaution à Flask qu’Achab les avait sans doute convoqués pour assister à une prouesse pédestre. Mais cela ne dura pas longtemps. Achab s’arrêta brutalement et s’écria :\par
– Hommes, que faites-vous lorsque vous voyez une baleine ?\par
– Nous donnons de la voix ! fut la réponse spontanée, une vingtaine de voix faisant chorus.\par
– Bien ! et l’intonation d’Achab trahissait un farouche accord avec l’entrain joyeux qui magnétisait les hommes à cette question inattendue.\par
– Et que faites-vous ensuite, hommes ?\par
– On met à la mer, et hardi après elle !\par
– Et quelle est votre devise quand vous la poursuivez ?\par
– Morte la baleine ou périsse la pirogue !\par
À chaque cri, l’attitude du vieil homme croissait en approbation maniaque, en joie cruelle, tant et si bien que les marins commencèrent à s’entre-regarder interrogativement, se demandant avec ébahissement si c’étaient bien eux qui s’étaient pareillement laissé exciter par des questions si inutiles, semblait-il.\par
Tous pourtant redevinrent attentifs lorsque Achab, venant à pivoter dans la tarière, une main haut levée, crispée sur un hauban, presque convulsivement serrée, commença à les haranguer :\par
« Vous tous, les guetteurs, m’avez, plus d’une fois, entendu donner des ordres au sujet d’une baleine blanche. Regardez bien ! Vous voyez ce doublon ? et il éleva dans le soleil une large pièce d’or – il vaut seize dollars, les gars ! Vous le voyez bien ? Monsieur Starbuck passez-moi la masse, là-bas…\par
Tandis que le second allait quérir le marteau, Achab, sans mot dire, frottait lentement la pièce d’or sur les pans de sa vareuse, comme pour en aviver l’éclat, en fredonnant à voix basse un air sans paroles, dont le son si étouffé, si indistinct semblait être le bourdonnement des rouages de sa vie intérieure.\par
Prenant le marteau des mains de Starbuck, il marcha sur le grand mât, le marteau levé dans une main, brandissant de l’autre le doublon, et s’écria à voix forte : « Celui d’entre vous, les gars, qui me lèvera une baleine à tête blanche, au front ridé et à la mâchoire de travers, celui d’entre vous qui me lèvera cette baleine à tête blanche dont la nageoire de la queue est percée de trois trous à tribord – écoutez bien ! celui d’entre vous qui me lèvera cette baleine-là, celui-là aura cette pièce d’or, les enfants ! » – Hourra ! Hourra ! crièrent les marins en agitant leurs suroîts pour saluer le clouement au mât du doublon.\par
– C’est un cachalot blanc, dis-je, résuma Achab, en jetant la masse au sol, un cachalot blanc ! que les yeux vous en sortent à l’épier, les gars ! regardez bien si l’eau blanchit, et si vous apercevez ne fût-ce qu’une bulle, donnez de la voix !\par
Pendant tout ce temps, Tashtego, Daggoo et Queequeg l’avaient, plus que tous les autres, fixé avec une surprise et un intérêt ardents, et à la mention du front ridé et de la mâchoire torve, ils avaient sursauté comme sous l’aiguillon d’un souvenir personnel précis.\par
– Capitaine Achab, dit Tashtego, ce cachalot blanc doit être celui que certains appellent Moby Dick.\par
– Moby Dick ? hurla Achab, alors vous connaissez le cachalot blanc, Tash ?\par
– Est-ce qu’il n’agite pas un peu bizarrement la queue comme un éventail avant de sonder ? demanda le Gay-Header délibérément.\par
– Et n’a-t-il pas un souffle étrange aussi ? demanda Daggoo, très épais même pour un spermaceti, puissant et rapide, capitaine Achab ?\par
– Et il a un, deux, trois… oh ! beaucoup de fers en dedans de lui, aussi, capitaine ! s’écria Queequeg de façon hachée, tous tortis, tortés, tortus, comme çui, çui… et il bégayait en cherchant ses mots, puis faisant un geste de tourner et tourner comme s’il débouchait une bouteille… comme çui… çui…\par
– Tire-bouchon ! s’exclama Achab. Oui, Queequeg, il porte en lui des harpons tout tire-bouchonnés et tordus, oui, Daggoo, et son jet est énorme, pareil à une meule de blé, et blanc comme un monceau de notre laine de Nantucket après la tonte annuelle des moutons, oui, Tashtego, il bat de la queue comme un foc déchiré dans la tempête. Mort et enfer ! Hommes, c’est bien Moby Dick que vous avez vu… Moby Dick… Moby Dick !\par
– Capitaine Achab, dit Starbuck qui, de pair avec Stubb et Flask, n’avait cessé de fixer leur supérieur avec une surprise grandissante, mais qui parut enfin frappé d’une idée qui lui donnait la clef du mystère. Capitaine Achab, j’ai entendu parler de Moby Dick… mais ne serait-ce pas Moby Dick qui vous a emporté la jambe ?\par
– Qui t’a dit ça ? s’écria Achab, puis il se tut un instant. Oui, Starbuck. Oui, mes braves, tous mes braves, c’est bien Moby Dick qui m’a démâté ; Moby Dick qui m’oblige à me tenir debout sur ce moignon mort. Oui, oui, hurla-t-il dans un sanglot terrible, violent, animal, le sanglot d’un élan frappé au cœur. Oui, oui, c’est cette maudite baleine blanche qui m’a rasé ; c’est lui qui a fait de moi un pauvre béquillard empoté pour toujours et à jamais ! Puis levant les bras au ciel, il hurla vers l’infini ses imprécations : Oui, oui ! Et je le poursuivrai au-delà du cap de Bonne-Espérance, au-delà du cap Horn, au-delà du maelström de Norvège, au-delà du brasier de l’enfer, mais je ne me rendrai pas ! Et c’est pour cela que vous êtes là, les gars ! Pour livrer la chasse à ce cachalot blanc dans les deux océans, d’un bout à l’autre de la terre, jusqu’à ce qu’il souffle du sang noir et roule sur le flanc. Qu’en dites-vous les gars, serrons-nous les mains à présent, voulez-vous ? Je trouve que vous avez l’air courageux.\par
– Oui, oui ! crièrent les harponneurs et les marins en se précipitant sur le vieil homme hors de lui. L’œil ouvert sur Moby Dick ! un harpon aiguisé pour Moby Dick !\par
– Dieu vous bénisse ! on ne savait si c’était un sanglot ou un hurlement, Dieu vous bénisse, hommes. Garçons, allez tirer la grande mesure de grog ! Mais pourquoi cette tête de six pieds de long, monsieur Starbuck ? Ne veux-tu pas pourchasser la baleine blanche ? N’es-tu pas en forme pour Moby Dick ?\par
– Je suis prêt à affronter sa mâchoire de travers, comme les mâchoires de la mort, capitaine Achab, si elles s’ouvrent honnêtement au cours du travail que nous avons à faire, mais je suis ici pour chasser les baleines, non pour assouvir la vengeance de mon commandant. Combien de barriques d’huile te rapportera ta revanche si tu la remportes, capitaine Achab ? elle ne te sera pas d’un grand profit sur le marché de Nantucket.\par
– Le marché de Nantucket ! Pouah ! Mais viens plus près, Starbuck. Il te faut atteindre une couche plus profonde. Si l’argent doit être la jauge, les comptables ont mesuré la planète en la ceinturant de guinées, une couvre un tiers de pouce, alors permets-moi de te dire que ma vengeance atteindra le gros prix, ici !\par
– Il se frappe la poitrine, chuchota Stubb, pourquoi ? à mon avis ça sonne vaste mais creux.\par
– Des représailles sur une brute muette ! Qui ne t’a frappé que par aveugle instinct ! Folie ! s’écria Starbuck. La fureur envers un animal, capitaine Achab, c’est un blasphème !\par
– Écoute encore… une couche plus profonde… Homme ! toutes choses visibles ne sont que des masques de carton-pâte. Mais dans chaque événement… dans l’acte vivant, le fait indubitable… quelque chose d’inconnu mais doué de raison porte, sous le masque dépourvu de raison, la forme d’un visage. Si l’homme frappe, qu’il frappe à travers ce masque ! Comment le prisonnier pourrait-il s’évader sans percer la muraille ? La baleine blanche est cette muraille dressée devant moi. Parfois je crois qu’il n’y a rien derrière. Mais il suffit. Elle me met à l’épreuve, elle m’accable. Je vois en elle une force révoltante, nourrie de vigoureuse malignité. Et c’est ce qui échappe à ma compréhension ce que je hais avant tout. Que la baleine blanche soit un agent ou qu’elle soit un principe, j’assouvirai sur elle ma haine. Ne me parle pas de blasphème, homme, je frapperais le soleil s’il m’insultait. Car si le soleil pouvait le faire, je pourrais aussi riposter, il y a une sorte d’équité dans la lutte, la jalousie a présidé à toute création. Mais je ne suis pas soumis aux règles du jeu, homme. Qui est au-dessus de moi ? La vérité est infinie. Détourne ton regard ! Le regard d’un imbécile est plus intolérable que l’œil furieux d’un démon ! Alors… tu rougis, tu pâlis, mon ardeur a embrasé ta colère. Mais écoute, Starbuck, les mots de l’emportement se dédisent eux-mêmes. Il y a des hommes dont les paroles enflammées ne sont qu’un léger affront. Je n’avais pas l’intention de te blesser. Laisse courir. Regarde, làbas, ces joues de Turcs aux taches fauves, ces images qui respirent et vivent peintes par le soleil, ces léopards païens à l’existence insouciante, dépourvue du sens du sacré, qui ne cherchent, ni n’attribuent aucune raison à leur vie brûlante toute de sensations ! L’équipage, homme, l’équipage ! Ces hommes ne font-ils pas qu’un avec Achab dans cette affaire de baleine ? Regarde Stubb ! Il rit ! Regarde le Chilien, il ronfle rien que d’y penser ! Ton seul jeune arbre ne saurait rester debout dans l’ouragan général, Starbuck ! Et de quoi s’agit-il ? \hspace{1em}penses-\par
– Rien de plus que d’aider à frapper une nageoire, ce n’est pas un haut fait pour Starbuck. Qu’est-ce d’autre ? Ce n’est pas pour une seule pauvre chasse que la meilleure lance de tout Nantucket va reculer alors que chaque matelot empoigne une pierre à aiguiser. Ah ! tu es dans l’étau d’une contrainte intérieure ! Je vois… la lame te soulève ! Mais parle, parle donc ! Oui, oui, ton silence alors parle pour toi. (À part.) Ses poumons ont bu le souffle de mes narines dilatées. Starbuck, désormais, m’appartient. Il ne peut plus me résister, sans révolte ouverte.\par
– Dieu me garde ! Qu’il nous garde tous ! murmura Starbuck à voix basse.\par
Tout à la joie de la soumission muette, envoûtée du second, Achab n’entendit pas cette invocation de mauvais augure, ni pour l’instant le rire étouffé qui monte de la cale, ni le sifflement prophétique du vent dans les cordages, ni le battement sourd des voiles contre les mâts comme si leur cœur défaillait tout à coup. Mais dans les yeux baissés de Starbuck se ralluma le feu opiniâtre de la vie, le rire souterrain s’éteignit, le vent souffla, les voiles se gonflèrent, le navire alla à nouveau de l’avant. Ah ! signes et présages, pourquoi n’insistez-vous jamais ? Mais ombres, vous êtes bien plus des prophéties que des avertisse- ments ! Mais vous n’êtes point tant des prédictions venues du dehors que la confirmation d’événements qui se préparent intérieurement. Car il n’est guère besoin de contraintes extérieures pour que nous obéissions aux appels de notre être le plus intime.\par
– La mesure ! la mesure ! cria Achab.\par
Lorsqu’il eut reçu le pot d’étain plein jusqu’au bord, il se tourna vers les harponneurs et leur ordonna de présenter leurs harpons, puis il les fit tenir devant lui, près du guindeau, armes en main, tandis qu’à ses côtés, les trois seconds portaient \hspace{1em}leurs lances et que l’équipage formait cercle autour d’eux, alors pendant un instant il scruta chaque homme du regard. Et ces yeux sauvages croisaient les siens comme les yeux injectés de sang des coyotes cherchent ceux de leur chef avant que celui-ci ne s’élance à leur tête sur la piste du bison… mais hélas ! seulement pour tomber dans le piège invisible de l’Indien.\par
– Buvez et faites passer, dit-il aux marins les plus proches en leur tendant le pot pesant. Seul l’équipage boit à présent. Que ça circule, que ça fasse le tour ! D’un seul trait, les gars, mais d’un bon coup de gosier. C’est brûlant comme les sabots de Satan. Alors, alors, ça fait le tour à la perfection. Ça vous vrille, ça arrache l’œil du serpent qui cherche à mordre. Très bien. Presque à sec. Il revient d’où il était parti. Donnez-le moi, les gars, il est vide. Hommes, vous êtes comme les années… une vie si débordante disparaît en une gorgée. Garçon, remplissez-le !\par
– Maintenant, mes braves, écoutez, je vous ai tous rassemblés autour du cabestan, vous, les seconds, formez mon flanc avec vos lances, vous, les harponneurs, tenez-vous là avec vos harpons, et vous, courageux matelots, mettez-vous en cercle autour de moi, que je puisse en quelque sorte faire revivre une noble coutume de mes ancêtres pêcheurs. Ô hommes, vous verrez que… Comment garçon, te voilà déjà ? Les pièces fausses ne reviennent pas plus vite. Donnez-le moi. Eh bien, maintenant que ce pot est plein à nouveau, on n’a plus besoin d’un possédé de la danse de saint Guy… file, trembleur !\par
– Avancez, seconds ! Croisez vos lances devant moi. Très bien ! Que j’en touche le faisceau. Ce disant, le bras tendu, il saisit en leur point de croisement les rayons formés par les trois lances et leur imprima nerveusement une saccade soudaine tout en regardant intensément Starbuck puis Stubb, Flask puis Starbuck. Il semblait qu’il voulût, par une volonté intérieure et sans nom, leur infuser la même émotion sauvage contenue dans la bouteille de Leyde qu’était sa propre vie magnétique. Les seconds fléchirent sous son regard violent, soutenu, surnaturel ; Stubb et Flask détournèrent les yeux cependant que l’honnête Starbuck baissait les siens aussitôt.\par
– En vain ! s’écria Achab. Mais c’est peut-être mieux. Car si vous trois aviez, une fois seulement, été ébranlés par la plénitude du choc, alors mon propre fluide, cela m’aurait peut-être abandonné. Il eût été possible aussi que vous fussiez tombés foudroyés. Peut-être d’ailleurs que vous n’en avez aucun besoin. Abaissez les lances ! Et maintenant, seconds, je vous investis échansons de mes trois partisans païens, ces trois très nobles gentilshommes, mes valeureux harponneurs. Vous méprisez cet honneur ? Comment ? Quand le grand Pape lui-même se servit de sa tiare en guise d’aiguière pour laver les pieds des mendiants. Oh ! mes doux cardinaux ! C’est votre propre vouloir qui vous y courbera, je ne vous en donne point l’ordre, c’est vous qui le voudrez. Enlevez la douille et démanchez vos fers, harponneurs !\par
Ayant obtempéré en silence, les trois harponneurs pointaient à présent devant lui le dard levé de leurs fers de quelque trois pieds de long.\par
– Ne me poignardez pas de cet acier cruel ! Retournez-les, pointe en bas ! Ignorez-vous le côté godet ? En haut le soc ! Alors, alors… Avancez, échansons ! Les fers ! Prenez-les, et tendez-les moi afin que je les remplisse ! Aussitôt, allant lentement de l’un à l’autre officier, il versa dans le godet des harpons, jusqu’au bord, le feu liquide du pot d’étain.\par
– À présent vous voici trois par trois. Investissez les harponneurs des calices meurtriers ! Et tendez-les, vous qui faites maintenant partie de cette indissoluble alliance. Ah ! Starbuck ! C’est accompli ! Le soleil attendait pour se coucher d’avoir donné son accord. Buvez, harponneurs ! Buvez et jurez, vous qui avez place à l’avant de la pirogue de la mort. Mort à Moby Dick !\par
Que Dieu nous donne à tous la chasse, si nous ne la donnons pas à Moby Dick jusqu’à sa mort !\par
Les longs godets à barbelures furent dressés, et aux cris de guerre et de malédiction, aux huées adressées à la baleine blanche, l’alcool fut avec un ensemble parfait sifflé d’un seul trait. Starbuck pâlit et se détourna en frissonnant. Une fois de plus, la dernière, le pot fut empli et fit le tour d’un équipage en transes, puis les hommes se dispersèrent, Achab leur fit un signe de la main et se retira dans sa cabine.
\chapterclose


\chapteropen
\chapter[{CHAPITRE XXXVII. Au coucher du soleil}]{CHAPITRE XXXVII \\
Au coucher du soleil}\renewcommand{\leftmark}{CHAPITRE XXXVII \\
Au coucher du soleil}


\chaptercont
\noindent (La cabine ; Achab, seul, est assis et regarde au-dehors par les fenêtres donnant vers l’arrière.) \par
Quel sillage blanc et trouble je laisse sur mon passage, de pâles eaux, de plus pâles joues. Les lames envieuses s’enflent derrière moi pour effacer ma trace. Qu’elles la fassent disparaître, j’aurai néanmoins passé le premier.\par
Là-bas déborde la coupe toujours pleine, la vague chaude rougit comme un vin. Le fil à plomb d’or sonde la mer. Le soleil qui, lentement, décline depuis le matin achève sa courbe plongeante, il descend cependant que s’élève mon âme ! Elle peine à cette montée sans fin. Serait-elle trop lourde, la couronne que je porte ? Cette couronne de fer des rois lombards ? Elle est pourtant sertie de pierres précieuses et moi qui la porte je ne puis voir l’éclat qu’elles jettent au loin, mais je sens obscurément que cet éblouissement engendre la confusion. C’est du fer – je le sais – non de l’or. Elle est brisée – je le sens. Ses bords déchirés me blessent si profond que mon cerveau semble palpiter dans un étau de métal. Oui, mon crâne est d’acier, point n’est besoin de casque dans cette lutte où la charge est lancée contre mon esprit !\par
La fièvre brûle mon front. Oh ! il fut un temps où le soleil levant m’était un noble aiguillon et un apaisement le soleil du soir. Rien ne m’est plus. Cette lumière adorable ne m’éclaire pas, toute beauté m’est une angoisse dont je ne peux tirer nulle joie. S’il m’est accordé de la percevoir à l’extrême, je suis privé de l’humble pouvoir d’y prendre plaisir. Je suis damné de la plus subtile, de la plus perverse façon ! Damné au cœur du paradis ! Bonne nuit… bonne nuit ! {\itshape (Il agite la main et s’écarte de la fenêtre.)}\par
Ce ne fut pas tâche si ardue. J’aurais cru trouver du moins un rebelle, mais ma roue dentée s’adapte à tous leurs engrenages et ils tournent. Ou bien, si l’on préfère, ils sont devant moi comme des tas de poudre et je suis l’étincelle. Mais le pire c’est qu’il faille consumer l’allumette pour communiquer aux autres la flamme ! Ce que j’ai osé, je l’ai voulu, ce que j’ai voulu, je le ferai ! Ils me croient fou – Starbuck le croit. Mais je suis satanique, je suis la folie elle-même déchaînée ! Cette folie furieuse qui n’a de lucidité que pour se comprendre elle-même ! La prophétie veut que je sois déchiqueté, eh… oui ! J’ai perdu cette jambe. Je prédis à présent que je démembrerai qui m’a démembré. Sois maintenant et le prophète et l’exécuteur. C’est plus que vous ne fûtes jamais, Dieux Grands. Je me ris de vous et je vous conspue, vous les joueurs de cricket, les pugilistes, les Burke sourds et les Bendigoe aveugles ! Je ne dirai pas ce que disent les écoliers aux brutes : « Trouvez quelqu’un à votre taille, ne me rossez pas ! » Non, vous m’avez abattu et je me suis relevé, mais vous vous êtes enfuis et cachés. Sortez de derrière vos ballots de coton ! Je n’ai pas de fusil pour vous atteindre. Venez, Achab vous présente ses compliments, venez voir si vous pouvez me détourner. Me détourner ? Vous ne le pouvez sans dévier vous-mêmes ! C’est là que je vous tiens ! M’écarter de ma voie, quand la route qui mène à mon but immuable est faite de rails d’acier et que les roues de mon âme sont creusées pour la suivre. Au-dessus de l’abîme des gorges, à travers le cœur transpercé des montagnes, sous le lit des torrents, je me rue tout droit devant moi ! Ni obstacle ni tournant à ma voie ferrée !
\chapterclose


\chapteropen
\chapter[{CHAPITRE XXXVIII. À la brune}]{CHAPITRE XXXVIII \\
À la brune}\renewcommand{\leftmark}{CHAPITRE XXXVIII \\
À la brune}


\chaptercont
\noindent (Starbuck est appuyé au grand mât.) \par
Mon âme est plus que circonvenue, elle est envahie, et par un fou ! Douleur cuisante et insupportable que le bon sens doive mettre bas les armes sur un tel champ de bataille ! Mais il a foré profond et anéanti ma raison ! Je crois deviner son but impie, pourtant je sens devoir l’aider à l’atteindre, que je le veuille ou non, l’indicible m’a lié à lui et me remorque avec un câble pour lequel il n’est point de couteau. Horrible vieillard ! Il s’écrie : « Qui est au-dessus de moi ? » – Oui, il est démocrate pour ce qui est au-dessus de lui mais c’est un féodal pour ce qui lui est inférieur ! Oh ! je vois clairement ma situation misérable, obéir avec la révolte au cœur et, qui pis est, haïr avec une pointe de pitié ! Car je lis en ses yeux une tragique douleur qui, si elle était mienne, me dessécherait. Pourtant, tout espoir n’est pas perdu. Le temps et la marée sont vastes. La baleine haïe a tout le globe liquide pour voyager, comme le petit poisson rouge son bocal de verre. Dieu peut faire échouer son dessein qui insulte au ciel. Je reprendrais du cœur, s’il n’était devenu de plomb, sa pendule s’en est arrêtée et je n’ai pas de clef pour en remonter les poids.\par
(Un bruit de bacchanale retentit sur le gaillard d’avant.) \par
Oh ! Seigneur ! Voyager avec un tel équipage de païens non enfantés par une mère humaine mais mis bas quelque part par la mer peuplée de requins. La baleine blanche est leur Démogorgon. Écoutez ! Les orgies de l’enfer ! Ce vacarme vient de l’avant ! À l’arrière… un silence sans faille ! Il me semble que c’est l’image de la vie. À l’avant, railleuse, effrontée, la proue fortifiée fait jaillir la mer étincelante, mais ce n’est que pour traîner à sa suite le sombre Achab, ruminant dans sa cabine, dressée sur le remous du sillage et poursuivie par ses gloussements voraces. Ce long mugissement me fait frémir ! Paix ! ripailleurs et prenez le quart ! Ô vie ! En une heure comme celleci l’âme est vaincue comme le sont par la faim les êtres sauvages… Ô vie, c’est maintenant que j’éprouve ta secrète horreur ! Mais ce n’est pas en moi qu’elle est, cette horreur, c’est en dehors de moi ! Et avec la douceur humaine qui demeure en moi, j’essayerai de te combattre encore, spectre sinistre de l’avenir ! Influences bienheureuses, soyez à mes côtés, soutenez-moi, étreignez-moi !
\chapterclose


\chapteropen
\chapter[{CHAPITRE XXXIX. Premier quart de nuit}]{CHAPITRE XXXIX \\
Premier quart de nuit}\renewcommand{\leftmark}{CHAPITRE XXXIX \\
Premier quart de nuit}


\chaptercont
\noindent (Stubb, seul, répare un bras de vergue.) \par
Ha ! ha ! ha ! ha ! hem ! que je me racle la gorge ! Depuis lors, je n’ai cessé d’y penser, et ce ha ! ha ! en est la conclusion. Pourquoi ? Parce que le rire est la réponse la plus sage et la plus facile à l’étrangeté ; et quoi qu’il arrive, il reste toujours ce réconfort infaillible : c’était écrit ! Je n’ai pas entendu ce qu’il a dit à Starbuck mais, à mon humble avis, Starbuck faisait joliment la même tête que moi l’autre soir. Pour sûr, le vieux Mogol a dû aussi le réduire à quia. Je l’ai compris, je le savais, si j’avais eu le don de prophétie, j’aurais pu le dire tout de suite, je l’avais vu en jetant un seul coup d’œil sur son crâne. Eh bien ! Stubb, sage Stubb – c’est mon qualificatif ! – eh bien ! Stubb, et puis après, Stubb ? On danse sur le volcan. Je ne sais pas tout ce qui va se passer, mais, que ce soit ce que ça voudra, j’irai au-devant en riant. Une facétie polissonne guigne sous toutes vos abominations ! Je me sens drôle… traderidera !… Que fait ma juteuse petite poire à la maison, pendant ce temps ? Elle pleure toutes les larmes de ses yeux ? Je penserais plutôt qu’elle régale les harponneurs qui viennent de débarquer, joyeuse comme un pavillon de frégate. Moi aussi je suis joyeux, traderidera !… Oh :\par
Nous boirons, ce soir, à la santé de l’amour D’un cœur léger, gai et volage Comme les bulles qui, au bord de la coupe Éclatent sur nos lèvres.\par
C’est un fameux couplet ! Qui appelle ? M. Starbuck ? Oui, oui, sir… ({\itshape à part}) c’est mon supérieur mais il a le sien si je ne m’abuse. Oui, oui, sir, j’ai tout de suite fini ce travail, j’arrive.
\chapterclose


\chapteropen
\chapter[{CHAPITRE XL. Minuit. Le gaillard d’avant}]{CHAPITRE XL \\
Minuit. Le gaillard d’avant}\renewcommand{\leftmark}{CHAPITRE XL \\
Minuit. Le gaillard d’avant}


\chaptercont

\labelblock{HARPONNEURS ET MATELOTS}

\noindent (La misaine se lève et l’on voit les hommes de quart debout, flânant, appuyés ou étendus dans des attitudes diverses, et chantant en chœur.) \par
Notre capitaine nous a donné l’ordre De vous dire adieu, dames espagnoles ! De vous dire adieu, dames d’Espagne !\par

\labelblock{LE PREMIER MATELOT NANTUCKAIS}

\noindent Oh ! les gars, pas de fadaises, c’est mauvais pour la digestion ! Quelque chose de plus stimulant, accompagnez-moi !\par
(Il chante, tous se joignent à lui.) \par
Notre capitaine était sur le pont Avec sa longue vue Il regardait les nobles baleines Qui soufflent et sondent Oh ! les bailles dans les pirogues, les gars ! Soyez prêts aux bras de vergues Et nous aurons une des ces belles baleines ! Halons main sur main, les gars !\par
Soyez gais et gardez le cœur en place Quand le hardi harponneur pique la baleine !\par

\labelblock{Voix du second sur le gaillard d’arrière}

\noindent Piquez huit, à l’avant !\par

\labelblock{LE DEUXIÈME MATELOT NANTUCKAIS}

\noindent En avant le chœur ! Huit coups là-bas ! Entends-tu, sonneur ? Toi, Pip, pique huit coups, noiraud ! J’appellerai la bordée, j’ai le gosier qu’il faut, une gueule de baril. Comme ça {\itshape (il passe sa tête dans l’écoutille).} Tri – booordais, en haut ! Huit coups, là en bas ! Grimpez !\par

\labelblock{LE MATELOT HOLLANDAIS}

\noindent Belle ronflée, ce soir, les amis, une nuit fertile pour ça. J’ai remarqué que le vin de notre vieux Mogol, s’il est un coup de fouet pour les uns, est un coup de massue pour les autres. Nous chantons, ils dorment, oui, comme des souches, un vrai premier plan d’arrimage. Là, appelez-les à travers cette pompe de cuivre. Dites-leur de cesser de rêver à leurs belles. Dites-leur que c’est le jour de la Résurrection, qu’ils doivent les embrasser une dernière fois et se présenter au Jugement. Voilà, très bien… parfait… le beurre d’Amsterdam ne t’a pas graissé la voix, à toi.\par

\labelblock{LE MATELOT FRANÇAIS}

\noindent Psitt, les gars, En avant pour une ou deux gigues avant de mouiller l’ancre dans la baie des Couvertures. Qu’en dites-vous ! Voici l’autre bordée qui monte. Attention ! Toutes les jambes prêtes ! Pip, petit Pip ! Hourra pour ton tambourin !\par

\labelblock{PIP {\itshape (Boudeur et somnolent.)}}

\noindent Sais pas où il est.\par

\labelblock{LE MATELOT FRANÇAIS}

\noindent Alors frappe-toi sur le ventre et remue les oreilles. Dansons la gigue, les hommes ; la vie est belle, hourra ! Le diable m’emporte ne danserez-vous pas ? En file indienne et une sabotière au galop ! Remuez-vous, en avant, les jambes !\par

\labelblock{LE MATELOT ISLANDAIS}

\noindent Je n’aime pas ton plancher, camarade, il est trop moelleux pour mon goût. J’ai l’habitude de parquets de glace. Je m’excuse de jeter un froid, mais excusez-moi.\par

\labelblock{LE MATELOT MALTAIS}

\noindent Excusez-moi aussi. Où sont les filles ? Qui, sinon un fou va prendre sa main gauche dans sa droite et se dire à lui-même : « Comment allez-vous ? » Des partenaires, il me faut des danseuses !\par

\labelblock{LE MATELOT SICILIEN}

\noindent Oui, des filles et de l’herbe ! Alors je sauterai avec vous, oui, je me transformerai en sauterelle !\par

\labelblock{LE MATELOT DE LONG ISLAND}

\noindent Bon, bon, les grognons, nous sommes assez nombreux sans vous. Sarclez le grain quand vous le pouvez, dis-je. Toutes les jambes iront bientôt à la moisson. Ah ! voilà la musique. Allonsy.\par

\labelblock{LE MATELOT AÇORIEN}

\noindent (En montant, il lance le tambourin par l’écoutillon.) \par
Voilà Pip, et voilà les bittes du guindeau. Grimpe dessus ! Allons, les gars ! {\itshape (La moitié des hommes dansent au son du tambourin, d’autres descendent, se couchent, d’autres encore dorment ou restent étendus sur des glènes de filins. Abondance de jurons.)}\par

\labelblock{LE MATELOT AÇORIEN {\itshape (Dansant.)}}

\noindent Vas-y Pip ! Frappe, sonneur ! Bing le, bang le, bong le, sonneur ! Tires-en des lucioles ! Brise les grelots !\par

\labelblock{PIP}

\noindent Les grelots, dites-vous ! En voilà encore un d’arraché tellement je cogne.\par

\labelblock{LE MATELOT CHINOIS}

\noindent Alors, claque des dents et cogne toujours ! Change-toi en pagode…\par

\labelblock{LE MATELOT FRANÇAIS}

\noindent Joyeuse folie ! Lève le cerceau, Pip, jusqu’à ce que je saute à travers ! Déhanchez vos tournures ! Fendez-vous !\par

\labelblock{TASHTEGO {\itshape (Fumant tranquillement.)}}

\noindent Ça c’est un homme blanc, il appelle ça s’amuser. Hum ! J’économise ma sueur, moi.\par

\labelblock{LE VIEUX MATELOT MANNOIS}

\noindent Je me demande si ces joyeux gaillards savent sur quoi ils dansent. Moi je danserai sur votre tombe, certes… c’est la plus cruelle menace de vos femmes nocturnes qui remontent dans le vent au coin des rues. Ô Christ ! Penser à ces flottes verdies, aux crânes rongés de mousse des équipages ! Que voulez-vous ? Vraisemblablement le monde n’est qu’une boule, à ce que disent les savants, alors autant en faire une salle de danse. Dansez, les gars, vous êtes jeunes, je l’ai été une fois !\par

\labelblock{LE TROISIÈME MATELOT NANTUCKAIS}

\noindent La relève, oh ! Ouf ! C’est pire que de ramer après les baleines par le calme plat… Laissez-nous tirer une bouffée, Tash.\par
(Ils s’arrêtent de danser, et se rassemblent en groupes. \par
Cependant que le ciel s’assombrit et que le vent se lève.) \par

\labelblock{LE MATELOT LASCAR}

\noindent Par Brahma ! les gars, il va bientôt falloir amener les voiles rondement. Le Gange du ciel a changé en vent ses grosses eaux ! Tu fronces les sourcils, Civa !\par

\labelblock{LE MATELOT MALTAIS {\itshape (Étendu, secouant son chapeau.)}}

\noindent Ce sont les vagues… ce sont les bonnets de neige qui dansent à présent. Bientôt, elles secoueront leurs pompons. Et si toutes les vagues étaient des femmes, alors je me noierai et je danserai avec elles à jamais. Il n’est rien de plus doux sur la terre – le ciel n’a pas sa pareille – que ces poitrines chaudes et sauvages, furtivement surprises dans la danse, lorsque la tonnelle d’une étreinte dissimule ces grappes mûres et pleines.\par

\labelblock{LE MATELOT SICILIEN {\itshape (Étendu.)}}

\noindent Ne m’en parle pas ! Écoute, le gars… ce rapide entrecroisement des membres… ces balancements souples… ces timidités… ces trémoussements ! ces lèvres ! ce cœur ! ces hanches ! ces frôlements… effleurer sans cesse et rien de plus ! Ne pas goûter, juste regarder, sans cela vient la satiété. Hein, païen ? {\itshape (Il lui envoie une bourrade.)}\par

\labelblock{LE MATELOT TAHITIEN {\itshape (Allongé sur une natte.)}}

\noindent Salut, sainte nudité de nos danseuses ! Les Hula-Hula ! Ah ! Tahiti, la voile aux reins, les palmes hautes ! Je demeure étendu sur ta natte mais elle ne repose plus sur ta douce terre ! Je t’ai vu tisser dans les bois, ô ma natte ! Je t’ai apportée jusqu’ici verte encore, mais à présent te voilà flétrie, usée. Ah ! ni toi, ni moi n’avons pu supporter cet exil ! Qu’en sera-t-il, si nous sommes transplantés au ciel ? N’entends-je pas gronder les torrents qui descendent des cimes aiguës de Pirohitee, bondissent dans les rochers escarpés et viennent noyer les villages ?… L’orage ! L’orage ! Redresse l’échine et va à sa rencontre ! {\itshape (Il saute sur ses pieds.)}\par

\labelblock{LE MATELOT PORTUGAIS}

\noindent Comme la mer éclate contre la coque ! Soyez prêts à prendre les ris, mes braves ! Les vents croisent le fleuret mais bientôt à la débandade ils vont se fendre.\par

\labelblock{LE MATELOT DANOIS}

\noindent Craque, craque, vieille carène ! Tant que tu craques, tu tiens bon ! Bravo ! Le second t’a durement en mains, il n’a pas plus peur que l’île du Cattegat, dont le bastion salé soutient le pilonnement furieux des tempêtes de la Baltique !\par

\labelblock{QUATRIÈME MATELOT NANTUCKAIS}

\noindent Notez qu’il a des ordres. J’ai entendu le vieil Achab lui dire qu’il doit toujours viser le grain, un peu comme on tire au pistolet sur un jet d’eau, faire feu du navire droit dessus !\par

\labelblock{LE MATELOT ANGLAIS}

\noindent Sacredieu ! Ce vieil homme est un grand type ! Et nous sommes ses hommes pour chasser sa baleine !\par

\labelblock{TOUS}

\noindent Oui !… Oui !…\par

\labelblock{LE VIEUX MATELOT MANNOIS}

\noindent Comme tremblent les trois pins, ce sont pourtant les arbres les plus durs à la vie dans tout autre sol ! Mais ici, ils n’ont que l’argile maudite du pont. Sois ferme à la barre, timonier ! C’est par un temps pareil que les cœurs courageux se brisent à terre et que les quilles se fendent à la mer. Notre capitaine a sa marque de naissance, voyez là-bas, les gars, il y a la même dans le ciel… blême comme la sienne… et tout le reste est ténèbres.\par

\labelblock{DAGGOO}

\noindent Et puis après ? Qui a peur des ténèbres a peur de moi ! Je suis taillé à même l’ombre !\par

\labelblock{LE MATELOT ESPAGNOL}

\noindent {\itshape (À part.)} Il veut faire le fendant, ah ! la vieille rancune me rend susceptible. {\itshape (S’avançant.)} Oui, harponneur, ta race est incontestablement le côté noir de l’humanité, diaboliquement noir. Soit dit sans offense.\par

\labelblock{DAGGOO {\itshape (Menaçant.)}}

\noindent Sans offense aucune.\par

\labelblock{LE MATELOT DE SAN JAGO}

\noindent Cet Espagnol est fou ou saoul. Pourtant c’est impossible ou alors, en son cas, l’eau-de-vie de notre vieux Mogol est lente à faire son effet.\par

\labelblock{CINQUIÈME MATELOT NANTUCKAIS}

\noindent Qu’est-ce que j’ai vu… des éclairs ? Oui.\par

\labelblock{LE MATELOT ESPAGNOL}

\noindent Non, c’est Daggoo qui a montré ses dents.\par

\labelblock{DAGGOO {\itshape (Bondissant.)}}

\noindent Je vais te faire avaler les tiennes, pantin ! Peau blanche ! Foie blanc !\par

\labelblock{LE MATELOT ESPAGNOL {\itshape (Allant à lui.)}}

\noindent Je te surinerai de bon cœur ! Grande carcasse, petite cervelle !\par

\labelblock{TOUS}

\noindent La bagarre ! La bagarre ! La bagarre !\par

\labelblock{TASHTEGO {\itshape (Tirant sur sa pipe.)}}

\noindent Bagarre en bas, bagarre en haut – les dieux et les \hspace{1em}hommes\par
– tous des brailleurs ! Pfuit !\par

\labelblock{LE MATELOT DE BELFAST}

\noindent Une bagarre ! Une vraie bagarre ! Bénie soit la Vierge, une bagarre ! Foncez les gars !\par

\labelblock{LE MATELOT ANGLAIS}

\noindent Que ce soit de bonne guerre ! Enlevez le couteau à l’Espagnol ! En cercle, en cercle !\par

\labelblock{LE VIEUX MATELOT MANNOIS}

\noindent Déjà fait. Là-bas… la courbe de l’horizon, Dans ce cercle Caïn frappa Abel. Honnête et joli travail ! Non ? Mais pourquoi, Dieu, as-tu alors créé ce cercle ?\par

\labelblock{VOIX DU SECOND AU GAILLARD D’ARRIÈRE}

\noindent Aux drisses ! Amenez les voiles de perroquet ! Soyez prêts à prendre un ris aux huniers !\par

\labelblock{TOUS}

\noindent Le grain ! Le grain ! Sautez, mes lurons ! {\itshape (Ils se dispersent.)}\par

\labelblock{PIP {\itshape (Se recroquevillant sous le guindeau.)}}

\noindent Des lurons ! Que le Seigneur vienne en aide à des lurons de cette espèce ! Cric, crac, voilà la draille de foc qui part. Bing, bang ! Seigneur ! Baisse-toi davantage, Pip, voilà la vergue de cacatois qui arrive ! C’est pire que d’être dans les bois quand le vent tourbillonne, au dernier jour de l’année ! Qui grimperait en quête des châtaignes à présent ? Mais ils sont tous en train de jurer, et pas moi. Ils se préparent un brillant avenir, ils sont sur la route du ciel. Tiens bon, de toutes tes forces ! Jimmini, quel grain ! Mais ces gars, eux, sont pires que les grains blancs. Les rafales blanches ? Ce cachalot blanc ? Br… Je viens d’entendre leurs bavardages, à l’instant, et la baleine blanche. Brr… On n’en a parlé qu’une fois et c’est justement ce soir ! Ça me fait frémir de la tête aux pieds comme mon tambourin. Ce vieil anaconda leur a fait prêter serment de la chasser ! oh ! toi, grand Dieu blanc, quelque part là-haut dans l’obscurité, aie pitié de ce petit garçon noir ici en bas, protège-le de tous ces hommes qui n’ont pas d’entrailles pour éprouver la peur !
\chapterclose


\chapteropen
\chapter[{CHAPITRE XLI. Moby Dick}]{CHAPITRE XLI \\
Moby Dick}\renewcommand{\leftmark}{CHAPITRE XLI \\
Moby Dick}


\chaptercont
\noindent Moi, Ismaël, je faisais partie de cet équipage. Mes cris s’étaient élevés avec ceux des autres, mon serment s’était joint aux leurs et plus je criais fort, plus la terreur qui habitait mon âme rivait ce serment à coups de marteau. Une sympathie occulte et farouche me possédait, la haine dévorante d’Achab devenait mienne. Je tendis une oreille avide à l’histoire de ce monstre sanguinaire, contre lequel j’avais, avec tous les autres, juré meurtre et vengeance.\par
Depuis assez longtemps, bien que par intermittence, la Baleine blanche, à l’écart et solitaire, avait hanté ces mers barbares fréquentées surtout par les pêcheurs de cachalots. Tous n’étaient pas au courant de son existence et seul un nombre relativement petit d’entre eux l’avaient vue sachant qui elle était. Plus petit encore le nombre de ceux qui l’avaient attaquée en toute connaissance de cause. Le grand nombre de navires qui partaient en campagne, la débandade qui les éparpillait sur tous les océans du globe, certains d’entre eux cherchant l’aventure sous des latitudes si désolées qu’au cours de douze mois ou plus ils ne rencontraient que rarement ou jamais un seul navire porteur de quelque message, la longueur excessive de chaque voyage, l’irrégularité des départs, bien d’autres raisons directes ou indirectes firent que les nouvelles portant sur la personnalité de Moby Dick furent longuement empêchées à se répandre parmi les flottes baleinières du monde entier. Sans doute, plusieurs navires rapportèrent avoir rencontré, à telle ou telle époque, ou sous tel ou tel méridien, un cachalot d’une grandeur \hspace{1em} et d’une malignité inusitées qui, après avoir mis à mal ses assaillants, leur avait complètement échappé. Quelques-uns pensaient qu’il y avait de fortes présomptions pour que ce fût Moby Dick en personne. Mais comme la pêche au cachalot a récemment offert des exemples divers et plutôt fréquents de férocité, de ruse et de méchanceté extrêmes du monstre poursuivi, les pêcheurs qui, par hasard, livrèrent bataille, sans le savoir, à Moby Dick, se contentèrent d’attribuer la terreur insolite qu’il provoquait aux dangers de la chasse au cachalot en général plutôt qu’à un individu particulier. C’est sous ce jour-là que la plupart voyaient la rencontre tragique d’Achab et de la baleine jusqu’à présent.\par
Quant à ceux qui, ayant ouï parler de la Baleine blanche l’aperçurent par hasard, presque tous, au départ, mirent sans crainte les pirogues à la mer comme pour n’importe quelle autre baleine de cette espèce. Mais ces attaques furent suivies de telles calamités, non pas limitées à des poignets ou à des chevilles foulés, à des membres brisés ou dévorés, mais fatales jusqu’à la suprême fatalité et ces échecs désastreux et répétés accumulèrent tant de terreurs sur Moby Dick, ils prirent de telles proportions qu’ils ébranlèrent le courage de bien de vaillants chasseurs à qui était revenue l’histoire de la Baleine blanche.\par
Des légendes insensées vinrent amplifier l’horreur vraie de ces rencontres meurtrières, car tout événement surprenant et terrible engendre naturellement le fabuleux comme l’arbre abattu des champignons, et bien plus encore que sur la terre ferme, les rumeurs fantastiques se répandent d’abondance sur la mer pour peu qu’elles trouvent un point d’appui dans la réalité. Si la vie sur mer l’emporte déjà sur la vie terrienne dans ce domaine, la pêche à la baleine, elle, l’emporte sur tout autre mode de vie maritime, en contes merveilleux et effrayants. Car non seulement les baleiniers n’échappent pas à l’ignorance et aux superstitions qui se transmettent de générations en générations, à tous les marins mais encore ce sont eux qui se trouvent, de par les circonstances mêmes de leur action, mis le plus brutalement en contact avec tout ce que la mer recèle de saisissantes épouvantes ; non seulement, ils voient face à face les merveilles les plus grandes mais ils leur livrent combat corps à corps. Seul, sur des mers si lointaines, qu’en ayant vogué des milliers de milles durant, côtoyé des milliers de rivages, jamais ne l’attendent la pierre ciselée d’une cheminée ni quoi que ce soit d’accueillant sous le soleil ; sous de telles latitudes et longitudes, faisant un tel métier, le baleinier est enveloppé d’influences qui toutes tendent à féconder son imagination pour de puissantes naissances.\par
Il n’est pas étonnant, dès lors, que les bruits qui couraient à travers la désolation océane au sujet de la Baleine blanche soient allés s’enflant de tant d’espace même et aient fini par faire corps avec toutes sortes d’insinuations délirantes, d’allusions avortées au sujet d’interventions surnaturelles, investissant Moby Dick d’un pouvoir de neuve terreur qu’aucune apparence ne justifiait. La panique qu’il semait retenait bien des chasseurs, auxquels ces bruits étaient parvenus, de souhaiter affronter les périls de sa mâchoire.\par
Mais des influences plus prosaïques étaient à l’œuvre. Le prestige originel du cachalot, en tant que monstre distinct de tout autre espèce de léviathan, demeure intact, maintenant encore, dans l’esprit des baleiniers. Et certains encore parmi eux, bien qu’intelligents et courageux lorsqu’il s’agit de chasser la baleine du Groenland ou baleine franche, refuseraient le combat avec le cachalot, soit par manque d’expérience professionnelle, soit par incompétence ou par timidité. En tout cas, il y a bien des baleiniers, surtout parmi ceux qui naviguent sous d’autres pavillons que le pavillon américain, qui n’ont jamais engagé la lutte contre le cachalot et dont la seule connaissance du léviathan se borne à celle du monstre ignoble primitivement poursuivi dans le Nord. Ces hommes-là, assis sur le panneau, pareils aux enfants auxquels on raconte une histoire au coin du feu, écoutent avec un intérêt respectueux les histoires sauvages et étranges des expéditions baleinières dans les mers du Sud. Nulle part, la souveraineté terrifiante du cachalot n’est ressentie avec autant de sensibilité qu’au gaillard d’avant de ces navires qui s’en écartent.\par
Sa puissance, maintenant prouvée, projetait autrefois une ombre légendaire jusque dans les récits de certains naturalis- tes ; Olafsen et Poaelsen déclarent que non seulement le cachalot est la terreur de tout ce qui vit dans la mer mais que son incroyable férocité lui donne une soif inextinguible de sang humain. De semblables idées n’avaient pas disparu au temps de Cuvier. Dans son {\itshape Histoire naturelle}, le baron affirme qu’à la vue d’un cachalot, tous les poissons, requins y compris, « sont frappés des plus vives terreurs » et que « souvent dans la hâte qu’ils mettent à fuir, ils se précipitent contre des rochers avec une violence telle que leur mort est instantanée ». L’expérience générale de la pêcherie peut bien compenser des remarques de cette nature, mais leur ampleur tragique confirme les baleiniers dans leurs croyances superstitieuses, même au sujet de la soif sanguinaire dont parle Powelsen, et surtout lorsque des vicissitudes les invitent à y penser.\par
De sorte que, submergés par la crainte entretenue par les histoires et les présages concernant Moby Dick, bien des pêcheurs rappelaient le temps où il était difficile de persuader des hommes ayant pourtant une longue expérience de la baleine franche de s’engager dans les dangers de cette chasse audacieuse, alors à ses débuts. Ces hommes prétendaient que, si l’on pouvait chasser avec espoir les autres léviathans, poursuivre et pointer des lances sur une apparition telle que le cachalot était une entreprise inhumaine et que s’y risquer c’était inévitablement se précipiter au-devant de l’éternité. On peut consulter sur ce thème des documents remarquables.\par
Pourtant il s’en trouvait qui, quoique informés, étaient prêts à livrer la chasse à Moby Dick et d’autres, plus nombreux, qui n’en ayant ouï parler que vaguement et avec réserve, sans détails précis d’une catastrophe donnée, sans surenchère de superstitions, étaient suffisamment téméraires pour ne pas esquiver la lutte si elle s’offrait.\par
L’une des idées les plus folles qui hantaient les esprits superstitieux au sujet de la Baleine blanche était la notion surnaturelle de son ubiquité, elle aurait été réellement rencontrée au même instant précis sous deux latitudes opposées.\par
Mais la crédulité et la superstition ne sont pas sans comporter quelque vraisemblance, car les courants marins n’ont pas à ce jour livré tous leurs secrets même à la plus érudite recherche, et les chemins mystérieux que le cachalot suit dans les profondeurs restent inconnus de ses poursuivants, donnant lieu, de temps à autre, à des conjectures aussi curieuses que contradictoires, surtout en ce qui concerne sa manière occulte, après avoir sondé dans l’abîme, de se déplacer à une vitesse hallucinante jusqu’à des lieux prodigieusement éloignés.\par
C’est un fait bien connu des baleiniers anglais et américains, et mentionné depuis des années dans le récit digne de foi de Scoresby, que des baleines prises tout au nord du Pacifique portaient dans leurs flancs des fers de harpons reçus dans les mers du Groenland. Et l’on ne peut nier que, dans certains cas, le laps de temps écoulé entre ces deux combats n’ait pu excéder quelques jours. Certains baleiniers en ont déduit que le Passage du Nord-Ouest qui a posé si longuement un problème à l’homme, n’en fut jamais un pour la baleine. On retrouve ainsi dans la réalité vécue par des hommes bien vivants des prodiges aussi fabuleux que ceux que l’on contait autrefois au sujet de la Serra da Estrella, montagne de l’intérieur du Portugal et dont un lac proche du sommet voit flotter sur ses eaux les épaves de navires perdus en mer, ou l’histoire plus merveilleuse encore de la fontaine Aréthuse, près de Syracuse, dont les eaux, croyaiton, venaient de Terre sainte par une voie souterraine. Ces fables ne surpassent guère les prodiges quotidiens de la chasse à la baleine. Accoutumés à semblables magies, sachant que la Baleine blanche avait échappé vivante à des assauts répétés et intrépides, il n’est pas surprenant que les baleiniers aient poussé leurs superstitions jusqu’à affirmer que Moby Dick était non seulement omniprésent mais qu’il était encore immortel (l’immortalité n’était que l’ubiquité dans le temps), que des forêts de lances dans les flancs le laissaient intact et que même si l’on parvenait jamais à lui faire souffler un sang épais, cette vue ne serait qu’une effroyable illusion, car, à des lieues de là, sur des lames non rougies, son souffle monterait, immaculé.\par
Mais, dépouillé de tout surnaturel, le caractère insolite du seul aspect physique et irrécusable du monstre suffit à frapper l’imagination. En effet, ce n’est pas tellement sa masse inusitée qui le distingue des autres cachalots mais, comme nous l’avons déjà laissé entendre, un front étrange, ridé, d’une blancheur de neige, et une haute bosse pyramidale, blanche elle aussi. Ces particularités essentielles révèlent à ceux qui le connaissaient, son identité jusque dans les mers infinies ne figurant sur aucune carte.\par
Son corps était également si rayé, tacheté, marbré de cette même blancheur spectrale qu’il méritait bien son nom de Baleine blanche, justifié par l’éclat de son apparition, lorsque à l’heure de midi, glissant sur le bleu sombre de l’océan, il y ouvrait le sillage d’une voie lactée, écume blanche constellée de paillettes d’or.\par
Et ce n’était pas tant sa taille exceptionnelle, ni sa robe insigne, ni même sa mâchoire inférieure difforme qui rendaient la Baleine blanche naturellement terrifiante, que la méchanceté concertée et sans exemple dont elle avait fait preuve à d’innombrables reprises dans ses attaques. Plus que tout, peutêtre, ses fuites traîtresses inspiraient l’épouvante, car lorsqu’elle s’éloignait de ses poursuivants triomphants, manifestant tous les signes de la peur, elle avait, plus d’une fois, fait brutalement volte-face et, fonçant sur eux, fait voler en éclats leurs pirogues, ou les avait contraints de regagner, en désarroi, leur navire.\par
Déjà plusieurs catastrophes marquaient l’histoire de sa chasse. De semblables désastres, dont il est peu fait état à terre, ne sont nullement rares dans les annales de la pêcherie mais, dans la plupart de ces cas, la préméditation criminelle et la férocité satanique de la Baleine blanche semblent telles que chaque démembrement ou chaque mort ne sauraient être imputées à une brute dépourvue de toute intelligence.\par
Imaginez dès lors dans quels brasiers de fureur démente étaient jetés ses chasseurs désespérés lorsque, parmi les éclats de leurs embarcations broyées, parmi les membres de leurs camarades déchiquetés, ils nageaient pour se dégager du lait caillé de la funeste colère de la baleine, et se retrouvaient devant le sourire serein et exaspérant du soleil, aussi immuable que s’il se fût agi d’une naissance ou d’une noce.\par
Ses trois pirogues défoncées autour de lui, les avirons et les hommes pris dans les remous, un capitaine, arrachant à sa proue brisée le couteau à trancher la ligne, avait bondi vers la baleine, et dans un corps à corps digne de l’Arkansas sur son adversaire, cherchait aveuglément à atteindre, avec une lame de six pouces, sa vie enfouie à une toise de profondeur. Achab fut ce capitaine. Et c’est alors que, glissant soudain sous lui la faucille de sa mâchoire, Moby Dick avait moissonné la jambe d’Achab, comme le faucheur une feuille d’herbe dans les champs. Aucun Turc enturbanné, aucun mercenaire vénitien ou malais, n’aurait pu le frapper avec une plus apparente malice. On ne peut guère douter que ce fut à partir de cette rencontre, presque fatale, qu’Achab ait nourri envers la baleine une fureur vengeresse. Sa frénésie maladive s’accrût encore du fait qu’il l’identifiait non seulement à toutes ses douleurs physiques mais encore à toutes ses révoltes de l’esprit. La Baleine blanche nageait devant lui, obsédante incarnation de ces puissances néfastes dont certaines natures profondes se sentent dévorées jusqu’à ce qu’elles ne leur laissent pour vivre qu’un demi-cœur et un demi-poumon. Devant ce mal spirituel originel auquel les chrétiens modernes reconnaissent la possession de la moitié des mondes et dont les anciens ophites avaient fait une idole à laquelle ils rendaient un culte… Achab ne s’inclinait pas comme eux pour l’adorer mais, dans son délire, l’esprit du mal prenait corps dans la Baleine blanche tant haïe et, infirme, il se mesurait à elle. Tout ce qui incline à la folie, tout ce qui torture, tout ce qui remue la vase, toute vérité entachée de venin, tout ce qui fissure les nerfs et encroûte le cerveau, toute intervention démoniaque subtile dans la vie et dans la pensée, tout le mal, pour le dément Achab, c’était l’être visible de Moby Dick à qui l’on pouvait livrer un tangible combat. Sur la bosse blanche de la baleine, il accumulait la révolte et la haine universelles éprouvées par l’humanité depuis Adam et il chargeait le mortier de sa poitrine du brûlant explosif de son cœur.\par
Il est peu probable que cette obsession ait brusquement germé en lui lors de son amputation ; à ce moment-là, se jetant sur le monstre, le couteau à la main, il avait seulement donné libre cours à une impulsion passionnée de haine charnelle et, lorsqu’il fut déchiré, il ne ressentit vraisemblablement que l’agonie de cette lacération, mais rien de plus. Mais, lorsqu’il fallut, à cause de cela, prendre la route du retour et que, pendant de longs mois faits de longues semaines et de longs jours, Achab se trouva étendu, côte à côte, avec son angoisse dans un même hamac, lorsqu’il fallut doubler en plein hiver le cap hurlant et lugubre de Patagonie, ce fut alors que son corps en lambeaux et son âme poignardée se mirent à saigner l’un dans l’autre et cette osmose le rendit fou. Ce fut alors seulement, lors de ce voyage de retour, après ce combat, qu’il devint la proie de sa monomanie, et par moments sa folie devenait furieuse. \hspace{1em}Bien que mutilé, une telle force vitale couvait dans sa poitrine égyptienne qu’elle s’amplifiait de son délire et ses seconds durent le ligoter dans son hamac. La berceuse démente des tempêtes le balançait dans sa camisole de force.\par
Lorsque le navire atteignit des latitudes plus clémentes et que ses bonnettes doucement gonflées le portèrent à travers les tranquilles tropiques, le vieil homme, selon toute apparence, avait abandonné sa folie avec les houles du cap Horn et de son antre obscur dans la lumière bénie.\par
Pourtant, alors même qu’il montrait un front résolu, rasséréné, pâle encore, donnait ses ordres à nouveau calmement et que ses seconds remerciaient Dieu d’avoir éloigné de lui cette fureur démente, alors même, dans le sanctuaire de son âme, Achab délirait. Souvent la folie des hommes est un félin rusé, lorsqu’on croit en avoir fini avec elle, elle est subtilement transfigurée ; celle d’Achab ne s’était pas apaisée, elle s’était resserrée en profondeur, pareille à l’Hudson, ce noble Nordique, dont les flots n’ont pas perdu leur violence lorsqu’ils courent insondables dans les étroites gorges des montagnes. Ainsi étranglée dans un défilé, la monomanie d’Achab n’avait rien perdu de sa démesure et rien n’avait sombré de sa grande intelligence naturelle. Son libre arbitre s’était fait instrument. Si l’on ose donner aussi audacieusement à la pensée le caractère qui lui est pro- pre : la folie particulière d’Achab avait pris d’assaut tout son jugement et concentrait toutes les batteries de son bon sens en les contraignant à servir son obsession, de sorte que, loin d’avoir décru, sa puissance s’était mille fois multipliée face à ce but et il y apportait une énergie dont il n’avait jamais fait usage en poursuivant un objectif raisonnable.\par
Tout cela est beaucoup… pourtant, les abîmes vastes et ténébreux d’Achab demeurent inviolés. Il est vain de divulguer les profondeurs car profonde est toute vérité. Si grandiose, si merveilleux que soit l’hôtel de Cluny à tourelles où vous vous \hspace{1em}trouvez, quittez-le, éloignez-vous de son intimité, âmes nobles et tristes, et descendez vers les grands thermes romains où, enfouie au-dessous des tours fantastiques que l’homme a érigées sur l’écorce de la terre, la racine de sa grandeur, toute son atroce essence demeure tel un patriarche, c’est l’ancien enseveli sous les antiquités et trônant sur des statues brisées ! Et les grands dieux se rient du roi captif et de son trône en ruine mais lui, patient, accroupi comme une cariatide, soutient de son front glacé les couches superposées des âges. Vous, les âmes les plus fières, les plus tristes, descendez jusqu’à lui et questionnez ce fier, ce triste souverain. Il vous ressemble comme un frère ! Oui, il vous a engendrées, jeunes royautés en exil, et seul ce sévère seigneur pourra vous révéler le secret de la création.\par
Achab entrevoyait en son cœur que si ses moyens étaient sains, ses mobiles et son but relevaient d’une démence à laquelle il ne pouvait se dérober, qu’il ne pouvait ni détruire ni modifier ; il savait aussi qu’il avait longuement dissimulé auprès des hommes, qu’il dissimulait encore. Il s’en rendait compte, mais sa volonté n’y pouvait rien changer. Toutefois, il avait si bien réussi à feindre que lorsqu’il avait enfin débarqué sur sa jambe d’ivoire, aucun Nantuckais ne devina en lui autre chose qu’une souffrance naturelle, l’ayant atteint dans ses œuvres vives, et dont le terrible accident qui s’était abattu sur lui était cause.\par
Ce que l’on avait appris de son indiscutable crise de folie en mer fut attribué à la même cause, ainsi que l’humeur sombre qui, depuis, obscurcissait régulièrement son front et cela jusqu’au jour même où appareilla le {\itshape Péquod}. Il est vraisemblable aussi que, loin de le croire, d’après d’aussi noirs symptômes, inapte à une nouvelle campagne, les propriétaires calculateurs de cette île économe, inclinèrent à penser que ces raisons mêmes le rendraient d’autant plus âpre à cette poursuite si furieuse, si sauvage, si sanguinaire qu’est la chasse à la baleine. Brûlé au-dehors, rongé en dedans par les crocs inexorables d’une incurable idée fixe, un tel homme, s’il existait, était tout désigné pour lever la lance et jeter le fer contre la plus repoussante des brutes. Ou si l’on présumait que son infirmité l’en empêcherait, du moins serait-il par excellence capable d’exciter ses subalternes au combat par ses encouragements et ses hurlements. Quoi qu’il en soit, il est certain que, le secret délirant d’une fureur toujours neuve enfermé et verrouillé en son cœur, Achab s’était embarqué pour le présent voyage avec le but unique, l’absorbant tout entier, de chasser la Baleine blanche. Si une seule de ses vieilles connaissances à terre avait pu imaginer ce qui couvait en lui, comme leurs âmes médusées et vertueuses eussent arraché le navire à cet homme satanique ! Mais elles ne pensaient qu’à une expédition fructueuse, à un profit en bel et bon argent, et lui n’avait d’autre dessein qu’une vengeance téméraire, implacable et surnaturelle. Voici donc ce vieil homme impie et grisonnant, poursuivant de sa malédiction la baleine de Job à travers tous les océans, à la tête d’un équipage composé en majeure partie de métis renégats, de parias et de cannibales, moralement affaibli encore par l’impuissance à laquelle étaient réduites la vertu et la droiture sans soutien de Starbuck, l’insouciance réjouie et inentamable de Stubb et la médiocrité totale de Flask. Qu’il eût sous ses ordres un tel équipage semble avoir été prévu et décidé par une fatalité infernale, complice de sa folie vengeresse. Comment se fait-il que tous se faisaient l’écho de la colère du vieillard, quel envoûtement néfaste subjuguait leurs âmes au point de leur faire croire parfois que sa haine était la leur, que la Baleine blanche leur était un ennemi aussi intolérable qu’à lui ? Ce serait plonger plus profond qu’Ismaël ne peut le faire que de tenter d’expliquer ce que la Baleine blanche représentait pour eux et pourquoi, dans leur inconscient, elle leur apparaissait, à leur insu, confusément, comme le grand démon qui sillonne les océans de la vie. Le mineur souterrain, à l’œuvre en chacun d’entre nous, sait-il où le conduit son pic dont le bruit sourd sans cesse se déplace ? Qui ne se sent pas tirer par une main irrésistible ? Quel esquif resterait immobile remorqué par un vaisseau de 74 ? Le tout premier je m’abandonnai au temps et au lieu mais, malgré ma brûlante impatience à rencontrer la baleine, je ne pouvais voir en cette brute autre chose que le mal le plus funeste.
\chapterclose


\chapteropen
\chapter[{CHAPITRE XLII. La blancheur de la baleine}]{CHAPITRE XLII \\
La blancheur de la baleine}\renewcommand{\leftmark}{CHAPITRE XLII \\
La blancheur de la baleine}


\chaptercont
\noindent J’ai suggéré ce que la Baleine blanche représentait pour Achab, ce qu’elle était, parfois, pour moi reste à dire.\par
Mis à part les raisons indiscutables pour lesquelles Moby Dick\hspace{1em}pouvait\hspace{1em}semer\hspace{1em}l’effroi\hspace{1em}dans\hspace{1em}l’âme\hspace{1em}de\hspace{1em}n’importe\hspace{1em}quel homme, une autre considération ou plutôt une horreur trouble et sans nom prenait, de temps en temps, en moi une intensité telle qu’elle submergeait toutes les autres ; sa nature ineffable et mystique me fait presque désespérer de pouvoir l’exprimer. La blancheur de la baleine par-dessus tout m’épouvantait. Comment espérer m’expliquer, pourtant il le faut, ne serait-ce que confusément et au hasard, faute de quoi tout ce récit serait vain.\par
Bien que la blancheur confère souvent à la beauté un raffinement singulier, comme si elle infusait aux choses sa vertu particulière, aux marbres, aux camélias, aux perles par exemple, bien que de nombreuses nations lui aient reconnu une suprématie royale, que les grands rois barbares de Pégu aient mis le titre de « Seigneur des Éléphants blancs » au-dessus de toute grandiloquente distinction, que les actuels rois du Siam ornent leur étendard de ce même quadrupède neigeux, que le drapeau hanovrien arbore un destrier immaculé, que le grand empire césarien d’Autriche, héritier de la suzeraineté romaine, ait choisi comme elle cette couleur impériale, bien que cette suprématie touche jusqu’à la race humaine, accordant idéalement à l’homme blanc une autorité sur toute tribu noire, bien que la blancheur ait même en été, en outre, synonyme d’allégresse, les Romains marquant d’une pierre blanche un jour heureux, bien que, dans les symboles, dans les tendresses humaines, cette même blancheur soit devenue l’emblème de choses nobles et touchantes : la pureté des épousées, la douceur de la vieillesse, bien que parmi les Peaux-Rouges d’Amérique il ne soit pas plus grand gage d’honneur que le don d’une ceinture blanche de wampoum, bien que sous des cieux divers la blancheur de l’hermine du juge symbolise la majesté de la Justice, que la dignité des rois et des reines soit quotidiennement soulignée par la couleur laiteuse de leurs coursiers, bien que dans les mystères les plus élevés des plus augustes religions elle signifie la puissance immaculée de la divinité, que les adorateurs persans du feu aient tenu pour sainte entre toutes la flamme blanche et fourchue de l’autel, que dans la mythologie grecque le grand Jupiter lui-même se soit incarné dans un taureau blanc, bien que chez les nobles Iroquois, le sacrifice hivernal du Chien blanc sacré ait été de loin la cérémonie la plus béatifique de leur religion, cette créature fidèle et sans tache étant le témoignage le plus pur qu’ils pussent offrir à l’Esprit suprême, de leur propre fidélité ; bien que tous les prêtres chrétiens empruntent directement au latin le nom de la tunique qu’ils portent sous la soutane, cette aube qui fait partie de leurs vêtements sacerdotaux, bien que le cérémonial de l’Église romaine ait choisi le blanc pour les ornements de la célébration de la Passion de Notre Seigneur, bien que dans une vision de saint Jean les élus soient vêtus de robes blanches, que les vingt-quatre vieillards, également vêtus de blanc, se tiennent devant le trône blanc où siège l’Agneau de laine blanche, eh bien, malgré toutes ces associations si nombreuses de la blancheur avec tout ce qui est doux, honorable et sublime, la notion la plus intime qu’elle sécrète est d’une nature insaisissable qui frappe l’esprit d’une terreur plus grande que la pourpre du sang.\par
Cette insaisissable nature, lorsqu’elle est dénuée de tout rapprochement bienveillant, et se trouve liée à un objet terrible en soi, porte la terreur à son comble. Voyez l’ours blanc des pôles, le requin blanc des tropiques, leur horreur transcendante ne vient-elle pas de leur douce blancheur neigeuse ? C’est leur effroyable blancheur qui investit leur muette avidité d’une si abominable suavité, plus repoussante que terrifiante. Et le tigre aux crocs cruels en sa robe héraldique n’ébranle pas autant le courage que l’ours ou le requin dans leurs blancs suaires\footnote{ \noindent Au sujet de l’ours polaire, celui qui voudrait approfondir la question pourrait alléguer que ce n’est pas la blancheur en soi qui accentue l’intolérable hideur de cet animal, car cette laideur poussée à l’extrême, si on l’analyse, vient de cette particularité : l’irresponsable férocité de cette créature est enveloppée dans une toison d’innocence céleste et d’amour, de sorte que l’ours polaire nous terrifie du fait qu’il provoque en nous deux émotions contraires dues à un contraste contre nature. Mais en admettant cela comme vrai il n’en reste pas moins que, n’était sa blancheur, il ne reproduit pas un effet aussi intense d’épouvante.\par
 Quant au requin blanc, le calme glissement de sa spectrale blancheur, lorsqu’on l’aperçoit dans ses évolutions normales, fait une impression semblable à celle de l’ours polaire, et pour les mêmes raisons. Cette singularité est mise en relief par le nom que porte ce poisson en français. La messe catholique des défunts commence par le {\itshape Requiem aeternam} (repos éternel) et {\itshape Requiem} est le mot qui qualifie cette messe elle-même et toute autre musique funèbre. Faisant allusion à son immobilité silencieuse et blanche de mort, et à l’apparente douceur de son caractère meurtrier, les Français l’appellent requin.
 }.\par
Pensez à l’albatros ! D’où vient qu’à travers les nuées de l’émerveillement spirituel et de la blafarde crainte, ce fantôme blanc vole dans toutes les imaginations ? Coleridge n’est pas le premier responsable de pareil envoûtement, mais la grande lauréate peu flatteuse pour Dieu : la Nature.\footnote{ \noindent Je me souviens du premier albatros que je vis. Ce fut lors d’une interminable tempête non loin des mers antarctiques. Pendant mon quart en bas du matin, j’étais allé sur le pont obscurci et là, collé contre le panneau de la grande écoutille, je vis un être royal, emplumé d’une blancheur immaculée, au bec d’une sublime courbe romane. De temps à autre, il abaissait l’ogive de ses grandes ailes d’archange comme pour embrasser une arche sainte. Il palpitait d’admirables et frissonnants battements. Quoique nullement blessé, il gémissait comme l’âme d’un roi affligé d’une détresse surnaturelle. Je croyais entrevoir dans ses étranges yeux, vides d’expression, des secrets volés à Dieu. Je m’inclinai comme Abraham devant les anges. Cet être blanc était d’une telle blancheur, si vastes étaient ses ailes, à jamais en exil dans ces eaux, que j’avais perdu le souvenir misérable des traditions et des villes. Longtemps je contemplai ce miracle de plumes. Je ne saurais dire, à peine suggérer, les sentiments qui me traversèrent alors. Mais enfin je repris conscience et, me tournant vers un matelot, je lui demandai quel était cet oiseau. Il me répondit : un goney ! Jamais encore je n’avais entendu ce nom. Est-il concevable que cet être glorieux soit totalement inconnu des terriens ? Non ! Mais j’appris plus tard que les marins appellent goney l’albatros. Le poème fantastique de Coleridge ne pouvait aucunement m’avoir incliné aux pensées mystiques qui m’envahirent à la vue de cet oiseau sur notre pont car je ne l’avais pas lu alors et j’ignorais qu’il s’agît d’un albatros. En l’avouant, je ne fais qu’aviver indirectement l’éclat et la noblesse du poème et du poète.\par
 J’affirme donc que le secret de l’envoûtement provoqué par cet oiseau tient à sa parfaite et merveilleuse blancheur, vérité mise en valeur par le fait qu’un solécisme veut qu’il y ait des « albatros gris » et j’en ai souvent vu, mais sans l’émotion que m’a donnée l’oiseau de l’Antarctique.\par
 Mais comment cet être mystique avait-il été pris ? Ne le répétez pas et je vous le dirai : avec une ligne et la traîtrise d’un hameçon alors qu’il était posé sur la mer. Enfin le capitaine le transforma en facteur, en lui attachant au cou une étiquette de cuir où il avait inscrit la position du navire avant de lui rendre la liberté. Mais je ne doute pas que ce message destiné aux hommes ait été emporté au ciel lorsque l’oiseau blanc s’envola pour rejoindre les chérubins aux ailes repliées dans leur adoration perpétuelle !
 }\par
Le Cheval blanc de la Prairie est célèbre dans nos annales de l’Ouest et dans les traditions indiennes : un magnifique étalon blanc, les yeux larges, la tête petite, le poitrail bombé, dont le port altier et méprisant a la dignité de mille monarques. Il était le Xerxès élu d’immenses troupeaux de chevaux sauvages dont les pâturages n’avaient alors d’autres limites que les Montagnes-Rocheuses et les Alleghanies. Flamboyant à leur tête, il les conduisait vers l’ouest, comme l’étoile choisie, chaque soir, dirige les phalanges de lumière. La cascade éclatante de sa crinière, la chevelure de comète de sa queue, lui faisaient un caparaçon tel que nul orfèvre ne l’eût pu façonner. Dans ce monde, non encore déchu, de l’Ouest, son apparition impériale, archangélique, recréait, aux yeux des vieux trappeurs et des chasseurs, la gloire des premiers âges lorsque Adam s’avançait dans sa divine majesté, le front haut, sans peur, semblable au puissant coursier. Qu’il marchât aux côtés de ses officiers et de ses maréchaux à la tête de ses cohortes innombrables qui déferlaient à travers les plaines pareilles aux flots d’un Ohio, ou qu’il se tînt au milieu de ses sujets broutant jusqu’aux confins de l’horizon, qu’il les passât en revue au galop, ses narines chaudes ouvrant des pétales roses dans sa fraîcheur de lait, de quelque manière qu’il se montrât, le Cheval blanc était toujours pour les plus intrépides des Indiens un objet de vénération et de terreur. D’après tout ce que raconte la légende sur ce noble coursier, on ne peut mettre en doute que ce fut sa blancheur spirituelle qui le revêtit d’un caractère divin, et ce même caractère, tout en forçant la dévotion, inspirait une indicible terreur.\par
Mais il est d’autres cas où cette blancheur perd l’étrange gloire qui auréole le cheval blanc et l’albatros.\par
Qu’est-ce qui inspire une répugnance si particulière et choque la vue chez un albinos au point que parfois cet homme en vient à être exécré de sa propre famille ! C’est de cette blancheur qu’il tire son nom. Un albinos est un homme aussi bien conformé que les autres, il n’est atteint d’aucune difformité véritable et pourtant cette blancheur diffuse qui est la sienne le rend plus singulièrement hideux que le plus laid des avortons. Pourquoi en est-il ainsi ?\par
Et la nature ne s’est pas fait faute d’introduire, de manière imperceptible et perverse, ce paroxysme de la terreur au sein des éléments. C’est à cause de sa neigeuse blancheur que le fantôme agressif des mers du Sud se nomme grain blanc. Et l’histoire fournit bien des exemples où l’art de la méchanceté humaine a veillé à s’assurer un aussi puissant allié. Quelle grandeur sauvage cela ne donne-t-il pas au passage de Froissart, le fait que les forcenés Blancs Chaperons n’ayant plus rien à perdre fussent masqués par leurs symboliques capuchons de neige lorsqu’ils assassinèrent leur bailli sur la place du Marché !\par
L’expérience commune et ancestrale du genre humain porte aussi témoignage du caractère surnaturel de la blancheur. La chose est certaine, ce qui épouvante le plus à la vue d’un mort, c’est sa marmoréenne blancheur, elle semble trahir l’effroi de se trouver dans l’autre monde comme elle trahit une mortelle émotion en celui-ci. À cette pâleur même des morts, nous empruntons la blancheur du suaire dont nous les enveloppons. Et dans nos superstitions, nous ne manquons pas de jeter un manteau de neige sur les épaules de nos fantômes, tous les esprits se lèvent à travers un blême brouillard… oui et tandis que les épouvantes nous empoignent, ajoutons que le prince de la terreur, dans le récit de l’évangéliste, monte un cheval blanc. Même si l’homme, dans un autre état d’âme, fait de la blancheur le symbole de la grandeur et de la grâce, il ne pourra cependant nier que, dans son sens le plus profond et dans l’idée qu’elle exprime, elle évoque pour l’âme une singulière apparition.\par
Comment rendre compte de ces constatations indéniables ? Il semble impossible de les analyser. Nous pourrions citer des cas où la blancheur, dépouillée de tout élément de terreur, n’en détient pas moins un pouvoir magique d’une autre essence, mais aurions-nous pour autant l’espoir de trouver le fil conducteur nous menant à la raison cachée dont nous sommes en quête ?\par
Essayons. Dans un domaine aussi délicat, la subtilité appelle la subtilité et un homme sans imagination ne peut en suivre un autre dans ces dédales, si, sans doute, la plupart des hommes ont éprouvé quelques-unes au moins des impressions occultes dont nous allons parler, bien peu durent en prendre véritablement conscience sur le moment et dès lors ils ne pourront plus à présent faire appel à leur souvenir.\par
Comment se fait-il que pour un esprit non averti, n’ayant qu’une connaissance vague du caractère particulier de ce jour, le simple mot de dimanche blanc de Pentecôte évoque de longues processions mornes et muettes de pèlerins abattus, marchant à pas lents et encapuchonnés de neige fraîche ? Pourquoi, lorsqu’on parle incidemment d’un frère blanc ou d’une nonne blanche devant un protestant candide et illettré des États centraux d’Amérique, une statue sans regard se dresse-t-elle dans son âme ?\par
Et si l’on fait abstraction de tous les rois et de tous les guerriers qui y furent emprisonnés (car telle n’est pas la seule raison de son prestige) pourquoi la Tour Blanche de Londres parle-telle tellement plus fortement à l’imagination d’un Américain qui n’a pas voyagé que ses voisines historiques : la Tour Byward ou même la Tour Sanglante ? Et ces tours plus sublimes encore, les Montagnes-Blanches du New Hampshire, pourquoi, à leur seul nom, selon notre humeur, projettent-elles sur notre âme un spectre géant, tandis que la Crête-Bleue de Virginie n’éveille qu’un songe suave, lointain, emperlé ? Et pourquoi, sans tenir compte des latitudes et des longitudes, le nom de mer Blanche éveille-t-il un écho sépulcral dans l’imagination tandis que celui de mer Jaune nous berce si humainement d’une évocation de longues après-midi paresseuses sur la laque de l’eau, finissant dans le faste d’un somnolent coucher de soleil ? Ou, pour prendre un exemple tout à fait immatériel et parlant à la seule imagination, pourquoi, en lisant les vieux contes de fées d’Europe centrale, le « grand homme pâle » des forêts du Hartz dont l’immuable blancheur froisse au passage les buissons verts, pourquoi ce fantôme est-il plus terrifiant que tous les turbulents diablotins du Brocken ?\par
Ce n’est pas non plus, assurément, le souvenir de ses tremblements de terre qui ont ébranlé les cathédrales, ni les raz de marée de ses mers frénétiques, ni l’inflexible aridité de ses deux sans larmes, ce ne sont pas ses étendues de clochers penchés, de murs tordus, de croix affaissées comme les vergues inclinées des navires à l’ancre, ni ses avenues de banlieues où les ruines s’entassent en jeux de cartes, ce ne sont pas toutes ces choses seulement qui font de Lima-la-Sèche la cité la plus insolite et la plus triste qu’on puisse voir. Car Lima porte le deuil en blanc et rien n’est plus atroce que sa blanche douleur. Aussi ancienne que le temps de Pizzare, la blancheur conserve à ses ruines une éternelle jeunesse, refuse la joyeuse verdure d’un déclin consenti et étend sur ses remparts brisés la rigide pâleur convulsée d’une mort foudroyante.\par
Je sais que pour un entendement commun, ce phénomène de blancheur ne passe pas pour être le facteur principal qui amplifie l’épouvante d’un objet en soi terrifiant, et que dans un esprit peu sensible les apparences ne sèment pas la crainte qui, dans un autre, deviennent redoutables à cause de ce même phénomène de blancheur, surtout lorsqu’il apparaît sous une forme silencieuse et absolue. Les exemples suivants éclaireront ma pensée.\par
D’abord : lorsqu’un marin approche dans la nuit de rivages inconnus et qu’il entend rugir les brisants, il se trouvera en éveil et ressentira juste ce qu’il faut d’émotion pour aiguiser ses facultés mais tirez-le de son hamac dans des circonstances identiques pour lui montrer le navire glissant à la minuit sur une mer d’une blancheur de lait, comme si, des promontoires qui l’encerclent, des ours blancs avaient déferlé pour nager autour de lui, alors il sera pétrifié par une terreur superstitieuse, le suaire des eaux blanches lui inspirera la même horreur que s’il enveloppait un revenant ; en vain la sonde lui assurera-t-elle qu’il est sur des fonds suffisants, en même temps que la barre, le cœur est mis dessous, et le marin n’a de répit que lorsque, \hspace{1em}sous lui, l’Océan est bleu à nouveau. Et pourtant quel est le marin qui te dira : « Sir, ce n’était point tant la peur de toucher des rochers invisibles, mais celle de cette hideuse blancheur qui m’a si fort bouleversé. » ?\par
Ensuite : Pour l’Indien né au Pérou, la vue continuelle des howdahs neigeuses des Andes n’engendre point de crainte sauf, peut-être, s’il rêve de la désolation des glaces éternelles qui emprisonnent ces altitudes et que lui vienne cette idée toute naturelle de l’horreur qu’il y aurait à se trouver perdu dans de si inhumaines solitudes. Il en va ainsi du coureur des bois de l’Ouest qui peut contempler, avec une relative indifférence, la prairie couverte de neige jusqu’à l’infini sans que l’ombre d’un arbre, fût-ce d’un rameau, vienne briser l’extase figée de sa blancheur. Il n’en va pas du tout de même du marin qui, dans les mers antarctiques, contemple parfois les puissances du gel et de l’air accomplissant d’une manière infernale quelque tour de magie blanche tandis que, grelottant et à demi naufragé, au lieu de l’arc-en-ciel de l’espérance et de la consolation, il ne voit que le mirage d’un cimetière sans limites dont les monuments de glace efflanqués et les croix brisées se rient de sa misère.\par
Mais, vas-tu penser, ce chapitre de céruse sur la blancheur n’est que le pavillon blanc hissé par une âme poltronne, tu rends les armes à l’hypocondrie, Ismaël.\par
Dites-moi pourquoi ce jeune et vigoureux poulain qui a vu le jour dans une vallée paisible du Vermont, loin de toutes bêtes de proie, si, au jour le plus radieux, vous agitez seulement derrière lui la peau d’un bison fraîchement tué, de façon à ce qu’il ne puisse la voir mais que seule lui parvienne sa sauvage odeur de musc, pourquoi tressaillira-t-il, renâclera-t-il, et, les yeux exorbités, grattera-t-il le sol dans les transes de la terreur ? Sa verte terre natale du nord ne lui accorde aucun souvenir de cornes meurtrières, et l’étrange parfum musqué ne peut lui \hspace{1em}rappeler aucun danger connu, que sait-il, en effet, ce poulain de Nouvelle-Angleterre, des bisons noirs du lointain Orégon ?\par
Il n’en sait rien, mais c’est là que vous découvrez dans l’ignorant animal l’instinctive connaissance de la démonologie. Bien qu’à des milliers de milles de l’Orégon, lorsqu’il respire cette sauvage odeur, les troupeaux de ces bisons éventreurs sont aussi présents dans sa prairie déserte et solitaire que dans celle où, au même instant, leurs sabots soulèvent la poussière.\par
Ainsi la houle étouffée d’une mer laiteuse, le bruissement dans le vent des festons de glace aux crêtes des montagnes, la neige emportée par la bourrasque dans la prairie désolée, tout cela est pour Ismaël cette peau d’un bison qui s’agite pour le poulain effrayé !\par
Pourtant ni l’un ni l’autre ne savent à quel être sans nom se réfère ce signe mystique ; pourtant, pour moi comme pour ce poulain, cet inconnu existe quelque part. Bien que ce monde visible semble, à bien des égards, inspiré par l’amour, les mondes invisibles furent créés dans l’épouvante.\par
Mais nous n’avons pas encore découvert le mystère incantatoire de la blancheur, ni pourquoi son appel a sur l’âme une telle puissance, ni, ce qui est plus étrange et d’une portée plus vaste, pourquoi, alors qu’elle est le symbole le plus vivant de la spiritualité, le voile même du Dieu des chrétiens, elle accroît le caractère repoussant de ce qui est objet de terreur pour l’homme.\par
Est-ce parce que sa nature indéfinissable projette les espaces cruels de l’infini et nous poignarde dans le dos avec le néant lorsque nous contemplons les blancs abîmes de la voie lactée ? Ou bien est-ce parce que le blanc n’est point tellement une couleur qu’une absence visible de couleur comme, en même temps, la fusion de toutes couleurs ? est-ce pour ces raisons que le silence vide, peuplé de sens, d’une vaste étendue de neige, nous fait reculer devant l’absence de Dieu faite de l’absence de toute couleur ou faite de toutes les couleurs fondues ensemble ?\par
Et si nous considérons la théorie des physiciens, voulant que toutes les autres couleurs de la terre – ces ornements d’apparat et de beauté – les nuances exquises des bois et des cieux du couchant, oui et les velours dorés des papillons et les joues de papillons des filles jeunes, ne soient qu’un raffinement de supercherie, nous voyons que les couleurs ne sont pas réellement inhérentes aux choses, mais seulement posées à leur surface. Et toute cette Nature que nous avons divinisée est maquillée comme la prostituée dont les séductions masquent le charnier intérieur.\par
Si nous allons plus loin, nous comprenons que le fard mystique dont la nature tire toutes ses nuances est le grand principe de la lumière qui est à jamais blanche ou incolore par elle-même et qui, si elle se posait sans intermédiaire sur les choses, neutraliserait aussi bien la teinte des tulipes que celles de nos pourpoints. Si nous méditons, l’univers se déploie alors à nos yeux comme une lèpre, et comme le voyageur entêté qui refuse, en Laponie, de mettre des lunettes noires ou de couleur, le malheureux mécréant s’aveugle à contempler l’immensité drapée dans un suaire blanc. La Baleine albinos est le symbole de toutes choses. Vous étonnerez-vous dès lors que lui soit livrée une chasse féroce ?
\chapterclose


\chapteropen
\chapter[{CHAPITRE XLIII. Écoutez !}]{CHAPITRE XLIII \\
Écoutez !}\renewcommand{\leftmark}{CHAPITRE XLIII \\
Écoutez !}


\chaptercont
\noindent – Chut ! Avez-vous entendu ce bruit, Cabaco ?\par
C’était pendant le quart de minuit, par un beau clair de lune. Les matelots faisaient la chaîne, se passant les barriques d’eau douce du passavant pour emplir le charnier situé près du couronnement de poupe. La plupart de ceux qui se trouvaient sur le territoire sacré du gaillard d’arrière prenaient garde de ne pas parler et de ne pas remuer les pieds. Les baquets passaient de mains en mains dans le silence le plus profond, rompu seulement parfois par le claquement d’une voile et l’incessant bruissement de la quille poursuivant sa route.\par
C’est dans ce calme qu’Archy, qui, dans la chaîne, se trouvait près des écoutilles de poupe, chuchota à son voisin, un Chole, ces mots :\par
– Chut ! Avez-vous entendu ce bruit, Cabaco ?\par
– Prenez le seau, voulez-vous, Archy ? de quel bruit parlezvous ?\par
– Voici qu’on l’entend à nouveau… sous les écoutilles… ne l’entendez-vous pas… une toux… on aurait dit une toux.\par
– Au diable la toux ! Faites suivre ce seau.\par
– Encore… on dirait à présent deux ou trois dormeurs qui se retournent !\par
– Caramba ! Suffit, camarade ! Ce sont les trois biscuits détrempés de votre souper qui se retournent dans votre estomac, c’est tout. Attention au seau !\par
– Dites ce que vous voudrez, camarade, j’ai l’oreille fine.\par
– Oui, c’est vous le gars qui, à cinquante milles au large de Nantucket, avez entendu cliqueter les aiguilles à tricoter de la vieille quakeresse, c’est bien vous !\par
– Riez toujours, on verra bien ce qu’il en sortira. Écoutezmoi, Cabaco, dans la cale arrière il y a quelqu’un qu’on n’a pas encore vu sur le pont, et je soupçonne le vieux Mogol d’en savoir quelque chose. J’ai entendu, pendant un quart du matin, Stubb raconter à Flask qu’il y avait une histoire de ce genre dans l’air.\par
– Pfuit ! le seau !
\chapterclose


\chapteropen
\chapter[{CHAPITRE XLIV. La carte}]{CHAPITRE XLIV \\
La carte}\renewcommand{\leftmark}{CHAPITRE XLIV \\
La carte}


\chaptercont
\noindent Si vous aviez accompagné le capitaine Achab dans sa cabine, après le grain qui s’était levé dans la nuit suivant l’approbation sauvage de son projet par son équipage, vous l’eussiez vu se diriger vers un coffre dans l’étambot, en sortir un grand rouleau froissé de vieilles cartes marines jaunies et les étaler devant lui sur sa table vissée. Puis, s’asseyant, il se mit à étudier avec attention leurs différentes lignes et hachures et à tracer d’un crayon lent et sûr de nouvelles routes dans des zones encore vierges. Parfois il consultait de vieux livres de bord empilés près de lui, où étaient notés les lieux et les saisons, ceux où divers navires avaient, lors de divers voyages, vu ou capturé des cachalots.\par
Tandis qu’il était ainsi absorbé, la lourde lampe d’étain suspendue par des chaînes au-dessus de sa tête se balançait sans arrêt au rythme du navire et projetait sur son front ridé des rais de lumière et d’ombres, si bien qu’on eût dit, cependant qu’il marquait des lignes et des routes sur les cartes ridées, qu’un crayon invisible dessinait aussi des lignes et des routes sur la carte ravagée de son front.\par
Ce n’était pas le premier soir que, dans la solitude de sa cabine, Achab méditait sur ces cartes. Il les sortait presque chaque nuit, presque chaque nuit il effaçait certains traits de crayon pour les remplacer par d’autres. La carte des quatre océans devant lui, Achab élaborait un labyrinthe de courants et de remous pour cerner la réalisation de son but insensé.\par
À quiconque n’a pas pleine connaissance des mœurs des léviathans, il peut paraître aussi absurde que sans espoir de rechercher une créature unique et solitaire à travers des océans sans limites. Ce n’était pas l’avis d’Achab, il savait la direction des courants et des marées et pouvait dès lors porter sur la carte les lieux où serait entraînée la nourriture du cachalot et, en se fondant sur la saison à laquelle on le chasse sous une latitude donnée, arriver à une conclusion raisonnable, voire presque certaine, sur le jour opportun où il faudrait se trouver à tel ou tel endroit à la recherche de sa proie.\par
Le retour périodique du cachalot dans certaines eaux est un fait si indubitable que de nombreux chasseurs pensent que si l’on pouvait l’observer et l’étudier de près dans toutes les mers du globe, si l’on pouvait confronter les livres de bord d’un seul voyage de toute la flotte baleinière, on verrait les migrations du cachalot se révéler aussi immuables que celles des bancs de harengs ou des vols d’hirondelles. Sur cette base, on a déjà tenté de dresser des cartes détaillées des migrations du cachalot\footnote{Depuis que ces lignes ont été écrites, cette assertion a été confirmée par un rapport officiel du lieutenant Maury, de l’Observatoire national, Washington, 16 avril 1851. Ce rapport précise qu’une telle carte est en voie d’achèvement et il en donne des fragments. Cette carte divise l’Océan en secteurs de 50 de latitude par 50 de longitude. Perpendiculaires à chacun de ces secteurs, douze colonnes représentent les douze mois de l’année, trois lignes horizontales les traversent : la première indique le nombre de jours passés chaque mois dans chaque secteur et les deux autres le nombre de jours pendant lesquels les cachalots et les baleines franches y ont été vus.}.\par
De plus, les cachalots suivent des veines, comme on les appelle, lorsqu’ils vont d’un territoire alimentaire à l’autre, guidés par quelque infaillible instinct – disons plutôt par l’intelligence secrète de la divinité – et ils tracent leur route selon une ligne droite qu’aucune carte ne permettrait à un navire de suivre avec le dixième de leur admirable précision. Bien que le trajet adopté par un cétacé ait la rigueur d’un tracé d’arpenteur, qu’il s’y conforme inéluctablement, la veine arbitraire dans laquelle il évolue, a généralement quelques milles de largeur, plus ou moins selon qu’elle s’élargit ou se rétrécit, mais elle ne s’étend jamais au-delà du champ de vision des hommes en vigie au grand mât du navire, lorsqu’il glisse avec circonspection sur cette voie magique. En conséquence, si l’on emprunte ce chemin aux saisons adéquates, on peut avoir l’assurance d’y rencontrer des cétacés en migration.\par
Ainsi Achab pouvait espérer trouver sa proie, non seulement grâce au juste choix de l’époque et du lieu de séjour du cachalot en des territoires alimentaires déterminés mais il pouvait même espérer l’y croiser, grâce à la subtilité de ses calculs, en traversant les vastes étendues qui séparaient ces zones.\par
Un fait semblait de prime abord devoir entraver son projet délirant mais toutefois méthodique. Bien que les cachalots soient grégaires et hantent des régions déterminées à des saisons régulières, on ne peut guère en conclure que la bande qui s’y trouvera cette année sous telle latitude et telle longitude sera formée des mêmes individus que l’année précédente, bien que des exemples indubitables appuient la vérité du contraire. Cela est généralement valable, de façon plus restreinte, pour les vieux cachalots qui vivent en solitaires.\par
De sorte que Moby Dick ayant, par exemple, été vu antérieurement dans ce qui s’appelle le territoire des Seychelles, dans l’océan Indien, ou dans la baie des Volcans au Japon, il ne s’ensuivait pas nécessairement que si le {\itshape Péquod} se rendait sur ces lieux à la même saison, il dût inévitablement l’y rencontrer. Il en va de même de ses autres habitats temporaires qui semblent n’être que ses haltes, ses auberges océaniques où il ne fait pas de séjour prolongé. En parlant des chances qu’Achab avait de réaliser son projet nous n’avons jusqu’ici parlé que des perspectives qui s’ouvraient à lui, s’il se trouvait à tel moment sur tel lieu où ses possibilités deviendraient probabilités ; or, comme il aimait à le penser, toute possibilité est proche d’une certitude. Ce temps et ce lieu tiennent dans une seule phrase technique : la Saison sur la Ligne. Car Moby Dick, pendant plusieurs années de suite, avait été aperçu, s’attardant à un endroit donné à une époque donnée, dans les mêmes eaux, comme le soleil, dans sa course d’une année, demeure, pendant un laps de temps connu, dans un certain signe du Zodiaque. C’était aussi dans ces eaux que les rencontres les plus meurtrières avec la Baleine blanche avaient eu lieu, là que les vagues racontaient ses exploits, c’était là le lieu tragique où le vieil homme obsédé avait reçu l’impulsion atroce de sa vengeance. Mais avec l’intelligence fine et prudente, la vigilance aiguisée qu’Achab mettait à jeter son âme sombre dans cette poursuite inflexible, il ne s’abandonnait pas à des espoirs fondés sur les raisons dont nous venons de parler, si séduisantes qu’elles pussent être ; et son désir toujours en éveil ne laissant aucun répit à son cœur tourmenté, lui interdisait de renoncer à chercher Moby Dick pendant les périodes intermédiaires.\par
Or le {\itshape Péquod} avait quitté Nantucket tout au début de la Saison sur la Ligne. Aucune tentative ne pouvait permettre à son capitaine d’atteindre le Sud, de doubler le cap Horn, de courir soixante degrés de latitude, pour arriver dans la région équatoriale du Pacifique à temps pour ouvrir la chasse. Il devait, dès lors, attendre la saison suivante. Que le {\itshape Péquod} ait appareillé si prématurément, son capitaine l’avait peut-être bien délibérément voulu, à cause de la complexité de la poursuite entreprise. Il avait ainsi devant lui trois cent soixante-cinq jours et trois cent soixante-cinq nuits qui l’eussent, à terre, rongé d’impatience et qu’il emploierait à des chasses combinées pour le cas où la Baleine blanche, séjournant par hasard dans des mers très éloignées de son territoire alimentaire périodique, viendrait à montrer son front ridé au large du golfe Persique, de la baie du Bengale, des mers de Chine ou de tout autre lieu hanté par ses pareils. De sorte que la mousson, le pampero, les vents de nord-ouest, le harmattan, les alizés, tous les vents sauf celui du levant et le simoun, pouvaient bien entraîner Moby Dick dans un tourbillon errant autour du monde à la suite du {\itshape Péquod}.\par
Tout cela étant admis, si on en juge avec une froide réserve, l’idée ne semble-t-elle pas folle que si l’on vient à rencontrer sur l’Océan infini une baleine solitaire, son poursuivant pourra la reconnaître en tant qu’individu précis, comme un mufti à barbe blanche dans les rues surpeuplées de Constantinople ? Certes. Car le front neigeux et la bosse d’albâtre de Moby Dick ne pouvaient prêter à confusion. Après s’être absorbé dans ses cartes longtemps après minuit, Achab, retombant dans ses rêves, se murmurait : « N’ai-je pas marqué la Baleine, ainsi entaillée, échapperait-elle à ma vue ? Ses nageoires sont percées et déchiquetées comme les oreilles d’une brebis perdue ! » Et sur ce, sa folie prenait un galop effréné jusqu’à ce que la lassitude lui vienne et que défaille sa pensée ! Alors, il montait sur le pont pour raffermir ses forces au grand air. Ah ! Seigneur ! quels frénétiques tourments ne doit-il pas endurer, l’homme que consume le désir insatisfait de la vengeance ? Il dort les poings serrés et s’éveille les clous sanglants de ses ongles transperçant ses paumes.\par
Souvent, les rêves de sa nuit, intolérables, épuisants, aux couleurs de la vie, le chassaient de son hamac, tant ils perpétuaient les pensées ardentes qu’il nourrissait pendant la journée, les entrechoquant avec fureur, les vrillant, les emportant en tourbillons toujours plus serrés dans son cerveau brûlant, jusqu’à ce que la pulsation même de la vie devînt en lui une angoisse insupportable ; alors parfois ces agonies spirituelles l’arrachaient à lui-même, un gouffre s’ouvrait en lui d’où montaient des flammes aiguës et des éclairs, d’où les démons maudits l’invitaient à se joindre à eux ; lorsque cet enfer béait sous lui, un cri sauvage retentissait sur le navire et, les yeux étincelants, Achab jaillissait de sa cabine comme s’il échappait à un lit en fer. Ce n’étaient pas là les symptômes mal réprimés d’une secrète faiblesse, ni d’une crainte que lui eût inspirée sa décision mais les manifestations extérieures de son intensité. Car, à de pareils moments, ce n’était pas Achab le fou, le chasseur inassouvi et rusé de la Baleine blanche qui, ayant regagné son hamac, devait en surgir en proie à l’horreur, celui qui se sauvait ainsi c’était sa propre âme, son principe de vie, son essence éternelle qui, dans le sommeil, dissociée de l’esprit critique, maître en d’autres temps, cherchait à fuir la promiscuité brûlante d’une frénésie à laquelle, momentanément, elle n’adhérait plus. Mais comme l’esprit n’existe que lié à l’âme, la crise d’Achab devait être provoquée par le fait que toute sa pensée, toute son imagination centrée sur un but suprême et unique, ce but absolu qu’il poursuivait avec une volonté libre et implacable, devait défier les dieux et les démons, le contraignant à devenir un être en soi, indépendant, par lui engendré. Il pouvait brûler sinistrement tandis que la vie normale qui était en lui s’enfuyait, frappée d’horreur devant cette naissance indésirable et sans père. Dès lors, l’esprit torturé qui brillait dans ses yeux et semblait être Achab tandis qu’il surgissait hors de sa cabine, n’était en fait qu’une forme vide, une apparition vague et somnambulique, un rayon de lumière vivante, sans doute, mais n’éclairant rien, et dès lors un néant en soi. Que Dieu te vienne en aide, vieil homme, car tes pensées ont engendré un étranger en toi, dont la ferveur a fait un Prométhée. Un vautour se nourrira à jamais de ce cœur, ce vautour qu’il crée lui-même.
\chapterclose


\chapteropen
\chapter[{CHAPITRE XLV. Témoignages}]{CHAPITRE XLV \\
Témoignages}\renewcommand{\leftmark}{CHAPITRE XLV \\
Témoignages}


\chaptercont
\noindent Ce chapitre est aussi important que n’importe quel autre de ce livre, sans faire à proprement parler partie du récit, car il traite de quelques habitudes intéressantes et singulières du cachalot ; cette question doit être développée pour élargir davantage les connaissances afin qu’elle soit bien comprise et, qui plus est, pour dissiper l’incrédulité qui guette certains esprits non avertis devant la véracité de cette histoire.\par
Je ne me soucierai pas d’observer une méthode pour m’acquitter de cette tâche mais j’essayerai d’atteindre mon but en citant quelques faits qu’en tant que baleinier je connais pour être authentiques par expérience personnelle, ou sur des rapports dignes de foi. Des citations entraîneront d’elles-mêmes la conclusion désirée.\par
D’abord : Je connais personnellement trois cas où une baleine harponnée a complètement échappé. Frappée par la même main, trois années plus tard, et tuée, elle portait dans ses flancs les deux fers à la marque personnelle de son agresseur. J’ai dit qu’en cette circonstance trois ans s’étaient écoulés entre les deux harponnages et je crois que ce fut davantage, car celui qui les avait faits était, entre-temps, parti pour un voyage en Afrique à bord d’un navire marchand. Lorsqu’il y débarqua il se joignit à une équipe d’explorateurs, pénétra très avant à l’intérieur des terres, le sillonna pendant une période de près de deux ans, exposé à tous les dangers que comporte cette incursion au cœur de régions inconnues, serpents, sauvages, tigres et miasmes empoisonnés. Cependant, la baleine frappée voyageait de son côté sans doute, sans doute avait-elle par trois fois accompli le tour du monde et balayé de ses flancs les rivages d’Afrique, mais sans intention. Cet homme et cette baleine vinrent à se retrouver et l’un vainquit l’autre. J’ai dit avoir été témoin moi-même de trois exemples identiques. Dans deux cas, j’étais présent quand les baleines furent harponnées et, lors de la seconde attaque, j’ai vu les fers qu’on retirait de leurs corps et qui portaient leurs marques respectives. Dans le cas de l’intervalle de trois ans, il advint que je me trouvais dans la pirogue et la première fois et la seconde, où je reconnus nettement sous l’œil de la baleine une énorme et singulière excroissance que j’avais déjà remarquée trois ans auparavant. Voilà trois cas dont j’affirme personnellement l’authenticité, mais j’ai entendu bien d’autres exemples rapportés par des personnes dont la véracité en ce domaine ne saurait être mise en doute.\par
Ensuite : Si les terriens sont ignorants de ces choses, des histoires mémorables, bien connues des chasseurs de cachalots, relatent qu’un cétacé particulier a été reconnu en des temps et en des lieux différents. Qu’un tel individu soit identifié ne tient pas exclusivement à des caractéristiques physiques qui le différencieraient de ses semblables car, quelle que soit l’originalité d’un cétacé dans ce domaine, elle a tôt fait d’être réduite à néant lorsqu’il est fondu en une huile de grande valeur. Non, la raison est la suivante : après des expériences fatales de la chasse, le danger auréolait un tel cachalot d’un prestige de terreur, semblable à celui qui entourait Rinaldo Rinaldi, prestige tel que bien des pêcheurs se contentaient de lui faire un signe de reconnaissance en portant la main à leur chapeau lorsqu’ils le rencontraient sans chercher à entrer avec lui en relations plus intimes. De la même manière que certains pauvres diables, à terre, connaissant quelque grand homme irascible, le saluent dans la rue, discrètement et de loin, craignant au cas où ils tenteraient de se rapprocher, que leur présomption soit aussitôt accueillie à coups de poing.\par
Non seulement ces baleines célèbres jouissaient d’une renommée individuelle, on pourrait dire d’une gloire à la dimension des océans, non seulement elles étaient réputées de leur vivant mais, après leur mort, elles étaient immortalisées par les histoires du gaillard d’avant. À leurs noms s’attachaient des droits, des privilèges et des distinctions tout comme à ceux de Cambyse ou de César. N’en est-il pas ainsi ô Tom Timor ! toi le fameux léviathan, couturé comme un iceberg, qui t’es si longuement attardé dans la mer qui porte ton nom, et dont le souffle fut si souvent vu de la plage aux palmiers d’Ombay ? N’en est-il pas ainsi, ô Jack de Nouvelle-Zélande, toi la terreur de tous ceux qui croisèrent au large de la terre des Tatouages ? N’en est-il pas ainsi, ô Morquan, roi du Japon, dont on dit que le souffle parfois s’élève pareil à une croix de neige dans le ciel ? N’en est-il pas ainsi, ô Don Miguel, toi le cachalot chilien buriné comme une vieille tortue d’hiéroglyphes mystiques ? Bref, voilà quatre cachalots qui sont aussi bien connus des étudiants en Histoire des Cétacés, que Marius et Sylla des étudiants en Histoire classique.\par
Mais ce n’est pas tout. Jack de Nouvelle-Zélande et Don Miguel, après avoir semé la destruction parmi les baleinières de différents navires, furent finalement poursuivis, chassés systématiquement et tués par d’intrépides capitaines qui avaient levé l’ancre dans ce dessein aussi nettement arrêté que celui, autrefois, du capitaine Butler, lorsqu’il se mit en route dans les forêts des Narragansetts afin de prendre le célèbre et meurtrier sauvage Annawon, le chef guerrier du roi indien Philippe.\par
Je ne saurais trouver meilleur endroit pour parler encore d’une ou deux choses qui me semblent importantes parce que, écrites, elles établissent à tous égards le caractère raisonnable de toute l’histoire de la Baleine blanche et plus encore celui de son épilogue catastrophique. Car c’est là un de ces cas désespérants où la vérité demande à être aussi solidement étayée que l’erreur. La plupart des terriens sont tellement ignorants des merveilles les plus évidentes et les plus manifestes de ce monde que sans quelques références à des faits précis, historiques ou autres, de la pêcherie, ils pourraient voir en Moby Dick une fable monstrueuse ou, ce qui serait détestable et plus grave, une allégorie hideuse et intolérable.\par
D’abord : Bien que la plupart des hommes aient une notion vague et sommaire des dangers qu’offre dans l’ensemble la grande pêche, ils n’ont pas une conception rigoureuse et vivante de ces dangers et de la fréquence avec laquelle ils se renouvellent. Cela tient peut-être en partie au fait que les journaux ne relatent pas un cinquantième des morts et des désastres survenus au cours des campagnes de pêches, si éphémères et vite oubliées que soient ces relations. Pensez-vous que le pauvre gars qui, en cet instant peut-être, est entraîné à l’abîme par la ligne à la suite du léviathan sondant au large de la Nouvelle-Guinée, pensez-vous que son nom figurera à la rubrique nécrologique du journal que vous ouvrirez à votre petit déjeuner ? En fait, avezvous jamais entendu quoi que ce soit qui ressemble à des nouvelles régulières de la Nouvelle-Guinée ? Pourtant je vous dirai que, lors d’un certain voyage que je fis dans le Pacifique, et au cours duquel nous entrâmes en rapport avec trente navires différents, chacun d’eux comptait une mort provoquée par un cétacé, quelques-uns en comptaient davantage et trois d’entre eux avaient perdu une baleinière avec tous ses hommes. Pour l’amour de Dieu, soyez économes de vos bougies et de l’huile de vos lampes, pour chaque gallon que vous brûlez, une goutte au moins de sang humain a été versée !\par
Ensuite : Les terriens ont une idée en vérité fort imprécise de la taille énorme de la baleine et de son énorme puissance, et lorsqu’il m’est arrivé de leur fournir des exemples formels de cette double énormité, je fus, de manière significative, félicité pour ma bonne plaisanterie, pourtant je jure sur mon âme qu’il n’entrait pas plus dans mes vues de me montrer facétieux que Moïse lorsqu’il écrivit l’histoire des plaies d’Égypte.\par
Mais heureusement le point particulier sur lequel je désire un témoignage convaincant sera établi indépendamment de moi. Ce point est le suivant : le cachalot est parfois assez puissant, assez intelligent, et peut faire preuve d’une méchanceté assez concertée, d’une préméditation même, semble-t-il, pour défoncer, détruire et couler un vaisseau de fort tonnage. Qui plus est, il l’a effectivement fait.\par
D’abord : en 1820, le navire l’{\itshape Essex}, commandé par le capitaine Pollard de Nantucket, se trouvait dans l’océan Pacifique. Il aperçut des souffles, mit les pirogues à la mer et donna la chasse à un groupe de cachalots. Bientôt, plusieurs cétacés furent blessés, lorsque soudain le plus gros d’entre eux s’écarta des baleinières, se détacha de son groupe et s’élança droit sur le navire. Il fracassa du front sa quille et, en moins de « dix minutes », le bâtiment fit eau et se coucha sur le côté. On n’en revit pas la moindre épave. Après les plus cruelles épreuves, une partie de l’équipage atteignit la terre dans les pirogues. Après avoir finalement regagné son pays, le capitaine Pollard, commandant un nouveau navire, appareilla pour le Pacifique, mais les dieux lui étaient contraires, il échoua sur des rocs inconnus et pour la seconde fois perdit son navire. Il jura dès lors de ne plus reprendre la mer et tint parole. Il réside actuellement à Nantucket. J’ai vu Owen Chace qui était second à bord de l’{\itshape Essex} au moment du drame, j’ai lu le récit simple et fidèle qu’il en fit, j’ai parlé à son fils et tout cela à quelques milles du lieu de la catastrophe\footnote{ \noindent Les extraits suivants sont tirés du récit de Chace : « Tous les faits me portent à conclure que le cachalot n’avait pas agi au hasard ; à de brefs intervalles, il mena deux attaques successives contre le navire, chacune dirigée de manière à nous faire le plus de tort possible, attaquant de front et calculant sa vitesse par des manœuvres précises pour en venir où il voulait. Il était horrible à voir, exprimant la rancune et la fureur. Il s’était détaché de la troupe au sein de laquelle nous venions de frapper trois de ses compagnons, comme s’il avait brûlé du désir de venger leurs souffrances. » Et encore : « En tout cas, je vis tout cela de mes propres yeux et l’ensemble de ces faits me fit sur le moment l’impression d’une malignité concertée de la part du cachalot (je ne puis me rappeler à présent toutes mes réactions) mais je suis sûr que je ne me trompe pas. »\par
 Et voici ses réflexions après avoir quitté le navire dans une pirogue ouverte par une nuit sombre et sur le point de désespérer de gagner une côte hospitalière : « Ni l’Océan noir, ni les vagues, ni la crainte d’être englouti par une effroyable tempête ou jeté sur des écueils, ni aucun des autres sujets d’effroi ne me paraissaient dignes de retenir un instant ma pensée, mais l’épave lugubre et l’horrible aspect vengeur du cachalot m’obsédèrent jusqu’au lever du jour. » Ailleurs – page 45 – il parle de « la mystérieuse et mortelle attaque de l’animal ».
 }.\par
Deuxièmement : le même accident arriva, en 1807, au navire {\itshape Union de Nantucket} qui fut perdu, corps et biens, au large des Açores. Je n’ai jamais pu avoir de détails précis sur cette catastrophe mais j’ai parfois entendu les baleiniers y faire diverses allusions.\par
Troisièmement : Il y a quelque dix-huit ou vingt ans, le capitaine J…, qui commandait une corvette de guerre américaine de première classe, dînait par hasard avec des capitaines baleiniers, à bord d’un navire de Nantucket en relâche à Oahu, l’une des îles Sandwich. La conversation tombant sur les baleines, le capitaine mit quelque vanité à se montrer sceptique sur la force étonnante que leur attribuaient les professionnels présents. Il nia de façon péremptoire, par exemple, qu’il pût se trouver un cétacé susceptible de porter à sa robuste corvette un coup assez violent pour l’amener à faire plus d’un dé à coudre d’eau. Très bien, mais attendons la suite. Quelques semaines plus tard, le capitaine mit le cap de son invincible navire sur Valparaiso. Il fut arrêté en route par un majestueux cachalot qui sollicita un court entretien confidentiel avec lui. Cette petite affaire consista à administrer au bâtiment du capitaine une si retentissante correction que, toutes pompes en action, il dut se hâter d’aller en radoub au port le plus proche. Je ne suis pas superstitieux au point de croire que cette entrevue entre la baleine et le capitaine fut providentielle. Saul de Tarse ne fut-il pas guéri de son incroyance par une semblable peur ? Je vous le dis, le cachalot ne tolère pas qu’on prenne des libertés avec lui.\par
Je vais maintenant me référer aux Voyages de Langsdorff à propos d’un fait particulièrement digne d’intérêt pour l’auteur de ce livre. Langsdorff, je vous le signale en passant, faisait partie de la célèbre expédition d’exploration de l’amiral russe Krusenstern au début de ce siècle. Le capitaine Langsdorff commence ainsi son dix-septième chapitre :\par
« Le 13 mai, notre navire fut prêt à partir et le jour suivant nous étions au large, faisant route vers Okhotsk. Le temps était clair et beau mais le froid était si intolérable que nous dûmes garder nos vêtements de fourrure. Pendant quelques jours, nous n’eûmes que peu d’air et ce ne fut que le dix-neuvième jour que se leva une forte brise soufflant du nord-ouest. Une baleine d’une taille inusitée, dont la largeur dépassait celle du navire, flottait presque en surface. Personne à bord ne la vit jusqu’au moment où, portant toutes les voiles, nous fûmes presque sur elle, de sorte qu’il était impossible de ne pas l’aborder. Nous courûmes ainsi un danger imminent pendant que cette créature gigantesque, en arrondissant le dos, souleva le navire de trois pieds au moins au-dessus de l’eau. Les mâts furent ébranlés, les voiles s’affaissèrent complètement tandis que tous ceux d’entre nous qui étaient en bas bondirent sur le pont croyant que nous avions touché un écueil, au lieu de cela nous vîmes le monstre s’éloigner avec une suprême solennité. Le capitaine D’Wolf eut aussitôt recours aux pompes pour se rendre compte si le choc avait causé quelque avarie au navire, mais nous nous rendîmes compte avec soulagement qu’il s’en était sorti sans dommage aucun. »\par
Or, le capitaine D’Wolf, officier sur le navire en question, est un homme de la Nouvelle-Angleterre, qui, après une longue carrière fort aventureuse en tant que capitaine de vaisseau, réside aujourd’hui à Dorchester, près de Boston. J’ai l’honneur d’être un sien neveu et je l’ai beaucoup questionné sur ce passage de Langsdorff, il m’en a confirmé chaque mot. Le bâtiment, toutefois, n’était pas grand, il avait été construit sur la côte de Sibérie et acquis par mon oncle après qu’il eut vendu celui à bord duquel il avait quitté son pays.\par
Dans ce livre démodé de mâles aventures et si plein de vraies merveilles – le voyage de Lionel Wafer, un vieux compagnon de Dampier – j’ai trouvé une petite histoire si semblable à celle que je viens de citer, que je ne puis résister au désir d’en faire état afin de corroborer la précédente, si tant est qu’il en soit besoin.\par
Lionel, semble-t-il, faisait route vers Juan Ferdinandino, comme il appelle l’actuelle Juan Fernandez : « En route, dit-il, vers quatre heures du matin, nous trouvant à cent cinquante milles du continent américain, notre navire accusa un choc terrible qui atterra nos hommes à tel point qu’ils ne savaient plus où ils étaient ni que penser et chacun se préparait à la mort. En vérité, il avait été si brusque et si violent que nous fûmes persuadés d’avoir heurté un écueil ; lorsque nous fûmes quelque peu revenus de notre stupeur, nous jetâmes la sonde qui ne rencontra pas de fond… La soudaineté du coup fit glisser les canons sur leurs affûts et jeta plusieurs hommes hors de leurs hamacs. Le capitaine Davis qui dormait la tête sur un sabord fut projeté hors de sa cabine ! » Lionel attribue ensuite cette secousse à un séisme et semble vouloir confirmer ses dires en déclarant qu’à peu près dans le même temps un grand tremblement de terre avait fait des ravages sur la côte d’Espagne. Mais je ne serais guère étonné si l’on me prouvait que le choc avait été provoqué par une baleine non repérée dans les ténèbres de cette heure matinale et qui aurait frappé la quille par-dessous.\par
Je pourrais donner d’autres exemples encore, recueillis ici ou là, de la puissance et de la malice que manifeste parfois le cachalot. Dans plus d’un cas non seulement il a pourchassé les pirogues de ses assaillants et les a renvoyées à leur navire mais encore il s’en est pris au navire même, résistant longuement aux lances dont on le harcelait du pont ; le navire anglais {\itshape Pussie Hall} pourrait raconter une histoire de ce genre. Quant à la force du cachalot, permettez-moi de vous dire que par temps calme, les lignes des baleinières ayant été transférées sur le navire et amarrées à celui-ci alors que la bête harponnée s’enfuyait, le cétacé entraîna la lourde quille à sa suite comme un cheval une carriole. On a également observé souvent que si on laisse à un cachalot frappé le temps de se ressaisir, il agit alors moins sous l’empire d’une rage aveugle qu’avec une volonté délibérée de détruire ses poursuivants. Ceci encore révèle éloquemment son caractère : lorsqu’il est attaqué il ouvre fréquemment la gueule et garde ses redoutables mâchoires écartées pendant plusieurs minutes de suite. Mais qu’il me suffise d’une seule et dernière illustration, aussi remarquable que lourde de sens, qui ne manquera pas de vous prouver que non seulement les événements les plus merveilleux de ce récit sont confirmés par des faits actuels et évidents, mais que ces merveilles (comme toutes merveilles) se répètent régulièrement au cours des âges, de sorte que pour la millionième fois nous disons amen avec Salomon : en vérité, il n’y a rien de nouveau sous le soleil.\par
Au VI\textsuperscript{e} siècle de notre ère, sous le règne de Justinien, alors que Bélisaire était général, Procope fut le sénateur chrétien de Constantinople. Comme chacun le sait, il écrivit l’Histoire de son temps, ouvrage à tous égards précieux. Toutes les autorités le considèrent comme un historien digne de foi et modéré, à une ou deux exceptions près, qui n’ont rien à voir avec le sujet que nous allons aborder.\par
Dans son Histoire, Procope raconte que du temps où il était préfet de Constantinople, un monstre marin géant fut pris dans la proche Propontide ou mer de Marmara, après avoir détruit à intervalles réguliers des navires dans ces eaux pendant une période de plus de cinquante ans. Un fait relaté dans une Histoire aussi sérieuse ne peut guère être démenti et il n’y a pas de raison qu’il le soit. Il n’est pas dit à quelle famille appartenait ce monstre marin mais, étant donné qu’il détruisait des navires et pour d’autres motifs encore, ce dut être un cétacé et tout m’incite à croire que c’était un cachalot. Je vais vous dire pourquoi. Pendant longtemps je m’imaginais que le cachalot avait toujours été inconnu en Méditerranée et dans les eaux avoisinantes. Or, j’apprends de source sûre que, sur les côtes de Barbarie, le capitaine Davis, de la marine anglaise, trouva le squelette d’un cachalot. Si un bateau de guerre peut franchir les Dardanelles, à plus forte raison un cachalot peut-il, par la même route, passer de la Méditerranée dans la Propontide.\par
Pour autant que je sache, il n’y a pas, dans la Propontide, de krill, nourriture de la baleine franche. Et j’ai tout lieu de croire que la nourriture du cachalot, les calmars ou les seiches, l’attend au fond de ces mers, puisqu’on a trouvé en surface des échantillons de grande taille, encore que ce ne soient pas les plus grands. Si donc vous réunissez ces faits et les méditez, vous verrez que tout raisonnement conduit à conclure que le monstre marin de Procope, qui, pendant un demi-siècle, défonça les navires d’un empereur romain, devait, selon toute probabilité, être un cachalot.
\chapterclose


\chapteropen
\chapter[{CHAPITRE XLVI. Conjectures}]{CHAPITRE XLVI \\
Conjectures}\renewcommand{\leftmark}{CHAPITRE XLVI \\
Conjectures}


\chaptercont
\noindent Bien que le brasier de son dessein le consumât, que toutes ses pensées et tous ses actes fussent centrés sur la prise finale de Moby Dick, bien qu’il parût prêt à sacrifier tous intérêts humains à sa passion unique, il se peut qu’Achab néanmoins, de par sa nature et de par une longue habitude passionnée de la chasse à la baleine, n’ait pas entièrement abandonné le but parallèle du voyage. Ou du moins, s’il n’en était pas ainsi, d’autres motifs nombreux ne manquaient pas qui l’influençaient. Peutêtre, même en tenant compte de sa monomanie, serait-ce aller trop loin que de suggérer que sa haine de la Baleine blanche ait pu s’étendre à un degré moindre à tous les cachalots et que plus il tuait de ces monstres plus il croirait multiplier les chances d’un prochain engagement avec celui qu’il poursuivait de sa vengeance. Si une telle hypothèse est discutable, d’autres considérations, sans être directement conjointes à la fureur de sa passion maîtresse, étaient susceptibles de l’influencer.\par
Pour arriver à ses fins, Achab avait besoin d’instruments ; et de tous les instruments les hommes sont les plus exposés à se détériorer. Il savait, par exemple, que son ascendant, à bien des égards magnétique, sur Starbuck n’agissait pas sur l’homme spirituel tout entier ; pas plus qu’une supériorité physique n’implique l’autorité intellectuelle, encore que, au contraire de la pure spiritualité, l’intellect demeure soumis à la matière. Le corps de Starbuck et sa volonté contrainte étaient possession d’Achab pour autant qu’il dirigeait sur le cerveau de Starbuck l’aimant de sa domination. Mais il savait qu’en dépit de cela, l’âme de son second abhorrait sa quête et que, s’il l’avait pu, il s’en serait désolidarisé avec joie et même l’aurait fait échouer. Il pouvait se passer bien du temps avant que la Baleine blanche ne fût en vue. Pendant cette période, Starbuck pouvait entrer en rébellion ouverte contre son commandement à moins qu’il ne fût prudemment circonvenu en créant des circonstances normales. Le raffinement de sa folie au sujet de Moby Dick se manifestait clairement dans la divination et la clairvoyance suprêmes qui l’invitaient à dépouiller momentanément cette chasse du caractère étrangement imaginaire et impie qu’elle revêtait tout naturellement, et à repousser dans l’ombre l’apogée de l’horreur de ce voyage, car il est peu d’hommes dont le courage résiste à une méditation prolongée que l’action ne vient pas interrompre. Il fallait à ses officiers et à ses hommes, au cours des longs quarts de nuit, un thème de réflexion plus banal que Moby Dick. Tous les marins, quels qu’ils soient, sont capricieux et instables, soumis à l’inconstance du temps, aussi, bien que l’équipage barbare ait salué l’annonce de sa poursuite avec une ardeur passionnée et même si cette poursuite promettait d’être animée et passionnée à son épilogue, sa ferveur ne saurait être longtemps soutenue par l’espoir d’atteindre un but lointain et vague, il lui fallait, avant tout, dans l’intervalle des intérêts et des occupations qui le garde sainement disponible pour l’assaut final.\par
Achab savait aussi que les hommes, sous l’empire d’une émotion violente, dédaignent toutes viles considérations, mais que de tels moments sont éphémères. Il était convaincu que l’état permanent de l’homme est la sordidité. En admettant que la Baleine blanche sollicite les cœurs de mon sauvage équipage, pensait-il, et même que sa sauvagerie flattée engendre en lui une générosité héroïque et prêt à donner gratuitement la chasse à Moby Dick, encore faut-il nourrir ses appétits quotidiens plus triviaux. Si hautains et chevaleresques qu’aient été les Croisés de jadis, ils ne traversèrent pas deux mille milles de terres à la conquête du saint sépulcre sans se livrer au pillage, au vol et à d’autres petits profits pieux en même temps. S’ils s’en étaient tenus à leur but final et romantique, beaucoup d’entre eux s’en seraient détournés avec dégoût. Je ne vais pas dépouiller ces hommes de tout espoir de gain, songeait Achab, oui, de gain. Ils peuvent bien mépriser l’argent aujourd’hui, mais que passent les mois sans leur en offrir la perspective, ils se mutineront et me casseront.\par
Un motif plus personnel invitait Achab à la prudence. Il est probable qu’il avait révélé le but majeur mais personnel du voyage, sous le coup d’une impulsion et peut-être prématurément ; il s’en rendait pleinement compte, à présent : ce faisant, il s’était exposé à l’accusation irréfutable d’être un usurpateur et, en toute impunité, tant morale que légale, son équipage, si tel était son bon plaisir, et en connaissance de cause, pouvait refuser de lui obéir et même lui arracher par la violence le commandement. Achab était naturellement très soucieux d’éviter que cette pensée ne germe et ne se répande. Il ne pouvait se prémunir que grâce à l’autorité de son cerveau, de son cœur, de sa main, appuyée sur une attention soutenue, calculée, vigilante de la moindre atmosphère à laquelle son équipage pouvait être sensible.\par
Pour toutes ces raisons, et d’autres peut-être trop profondes pour être abordées ici, Achab vit clairement qu’il fallait rester fidèle au but normal, présumé de la croisière du {\itshape Péquod}, s’en tenir aux us et coutumes, et même se contraindre à faire preuve de tout l’intérêt passionné qu’on lui savait pour son métier.\par
Quoi qu’il en soit, on l’entendait maintenant souvent héler les hommes de vigie aux trois mâts, leur recommander d’être bien à l’œil et de ne pas manquer de signaler fût-ce un marsouin. Cette vigilance ne tarda pas à être récompensée.
\chapterclose


\chapteropen
\chapter[{CHAPITRE XLVII. Le tisserand}]{CHAPITRE XLVII \\
Le tisserand}\renewcommand{\leftmark}{CHAPITRE XLVII \\
Le tisserand}


\chaptercont
\noindent C’était un après-midi nuageux et lourd, les matelots traînaient paresseusement sur le pont ou contemplaient distraitement les eaux couleur de plomb. Queequeg et moi nous étions paisiblement occupés à tresser ce qu’on appelle un paillet d’espade pour notre pirogue. La scène était si calme, d’une douceur étouffante, une telle joie incantatoire planait, que chaque homme silencieux semblait s’être dissous dans son moi invisible.\par
Dans ce travail de tissage, j’étais le serviteur ou le page de Queequeg. Tandis que je passais et repassais la trame de lusin entre les longs fils de grelin, tandis que Queequeg, debout, glissait encore et encore sa lourde épée de chêne entre les fils, regardant la mer avec indolence, et serrant les fils en bonne place d’un geste nonchalant. Une rêverie insolite régnait sur le navire et sur la mer, rompue seulement par le bruit sourd de l’épée. Il semblait que sur le métier du Temps je n’étais plus qu’une navette tissant et tissant encore machinalement entre les mains des Parques. Les fils tendus du grelin répétaient une vibration unique, toujours la même, une vibration juste suffisante pour laisser passer les fils qui, par-dessus et par-dessous, venaient se lier à eux. Ces fils immuables de la fatalité, pensais-je, voilà que j’y fais courir la navette qui tisse ma propre destinée, cependant que l’épée irréfléchie, indifférente de Queequeg, heurtait la trame tantôt de biais, tantôt de façon sinueuse, tantôt avec force, tantôt faiblement, et ce coup final différemment porté imposait ses contrastes à l’aspect final de l’œuvre. Cette épée sauvage, me disais-je, donne la forme finale tant à la trame qu’aux grelins, cette épée désinvolte et impassible doit être le hasard. Oui, hasard, libre arbitre et nécessité, aucunement incompatible, s’entrelacent pour agir ensemble. Les fils droits du destin ne sauraient être déviés de la ligne droite bien que chaque vibration s’y essaie, le libre arbitre est toujours libre de pousser sa navette entre ces fils imposés et le hasard, bien que contraint par les verticales de la nécessité et par les horizontales tirées par le libre arbitre, ordonné par l’un et l’autre, le hasard parfois est maître de l’un comme de l’autre, et modèle le visage ultime des événements.\par
Ainsi nous tissions et tissions encore lorsque je sursautai à un bruit étrange, prolongé, dont la sauvage musique était si surnaturelle que la pelote du libre arbitre s’échappa de mes mains et que je restai à fixer les nuées d’où cette voix tombait comme une aile. Haut perché dans les barres de hune, Tashtego, le fou de Gay Head, était passionnément tendu en avant, la main dressée comme une baguette de magicien et à intervalles courts et répétés il poussait ses cris. Sans doute au même moment toutes les mers retentissaient d’un cri semblable poussé par des centaines d’hommes en vigie perchés dans le ciel ; pourtant, ce vieux cri traditionnel ne pouvait pas sortir de beaucoup de poitrines avec le rythme miraculeux qui l’arrachait de celle de Tashtego l’Indien.\par
Ainsi accroché à même l’espace, dévorant l’horizon d’un regard ardent et farouche, on eût dit quelque voyant, quelque prophète apercevant les ombres du destin et annonçant leur venue par des cris frénétiques :\par
« Elle souffle ! là ! là ! là ! Elle souffle ! Elle souffle ! »\par
– Où donc ?\par
– Côté sous le vent, à deux milles environ ! toute une gamme !\par
L’effervescence régna aussitôt.\par
Le cachalot souffle avec une régularité d’horloge, avec la même constance jamais démentie. C’est à cela que les baleiniers le distinguent des autres tribus de son espèce.\par
– Voilà les queues ! cria encore Tashtego, et les cachalots disparurent.\par
– Vite, garçon, cria Achab. L’heure ! L’heure !\par
Pâte-Molle se hâta dans la cabine, regarda la montre et donna l’heure exacte à Achab.\par
On mit en panne et le navire tangua doucement pointe au vent. Tashtego signalant que les cétacés avaient sondé sous le vent, nous attendions avec confiance de les voir réapparaître devant nous, car le cachalot avec une ruse singulière, sonde dans une direction mais, lorsqu’il est dans les profondeurs, il fait brusquement volte-face. Toutefois cette supercherie n’avait pas dû être pratiquée car il n’y avait pas de raison de supposer que l’animal signalé par Tashtego fût si peu que ce soit inquiété ou même qu’il eût la moindre connaissance de notre présence. Un des hommes, désigné comme gardien, ce sont ceux qui ne descendent pas dans les baleinières, remplaça alors Tashtego, au grand mât. Les matelots des mâts de misaine et d’artimon étaient descendus, les bailles à ligne furent mises en place, les bossoirs d’embarcations parés, la grande voile coiffée, et les trois pirogues se balancèrent au-dessus de l’eau comme trois paniers de perce-pierre au-dessus de hautes falaises. Hors des pavois, les hommes impatients s’accrochaient d’une main à la lisse et posaient dans l’attente un pied sur le plat-bord, pareils aux marins de navires de guerre prêts pour l’abordage.\par
Mais à ce moment critique une exclamation détourna tous les regards de la baleine. Ébahis, tous fixaient le sombre Achab entouré de cinq fantômes sombres qui semblaient avoir pris corps de l’air.
\chapterclose


\chapteropen
\chapter[{CHAPITRE XLVIII. Première mise à la mer}]{CHAPITRE XLVIII \\
Première mise à la mer}\renewcommand{\leftmark}{CHAPITRE XLVIII \\
Première mise à la mer}


\chaptercont
\noindent Les fantômes, car c’est ce qu’ils semblaient être alors, glissèrent légèrement de l’autre côté du pont et, avec une silencieuse promptitude, larguèrent les palans et les coulisseaux de la pirogue qui se balançait là. Cette baleinière avait toujours été considérée comme étant de réserve, bien qu’on l’appelât baleine du capitaine parce qu’elle se trouvait à tribord arrière. Le personnage qui se tenait alors à sa proue était grand et basané, une dent blanche saillait cruellement entre ses lèvres d’acier. Il était funèbrement vêtu d’une veste chinoise de coton noir, froissée, et d’un large pantalon taillé dans la même étoffe sombre. Couronnant étrangement cette figure d’ébène, un turban d’une blancheur éblouissante coiffait sa tête de ses propres cheveux tressés et enroulés plusieurs fois. Ses compagnons, moins bistrés que lui, avaient ce teint haut en couleur, jaune tigre, particulier à quelques aborigènes de Manille, une race célèbre pour sa ruse satanique et tenue par certains braves marins blancs pour être composée de suppôts et d’agents secrets, envoyés sur l’eau par leur seigneur infernal, lequel, croient-ils, a ses assises ailleurs.\par
Tandis que l’équipage contemplait avec étonnement ces étrangers, Achab cria au vieil homme enturbanné de blanc, leur chef :\par
– Tout est paré, Fedallah ?\par
– Paré, fut la réponse à demi sifflée.\par
– Alors, mettez à la mer, vous entendez ? et hurlant à travers le pont : j’ai dit mettez à la mer, là-bas.\par
Le tonnerre de sa voix était tel que les hommes, malgré leur stupéfaction, bondirent par-dessus la lisse ; les réas tournèrent dans les poulies, les trois pirogues tombèrent à la mer avec un bruit mou, tandis qu’avec une adresse audacieuse et désinvolte, inconnue dans tout autre métier, les matelots sautaient comme des chèvres dans les baleinières secouées au flanc du navire.\par
À peine avaient-ils débordé sous le vent du navire, qu’une quatrième embarcation, venant du côté du vent, vira sous la poupe, les cinq inconnus étaient aux avirons cependant qu’Achab, debout à l’arrière dans une attitude rigide, criait fortement à Starbuck, Stubb et Flask de s’écarter largement les uns des autres afin de couvrir la plus grande surface possible. Mais rivés de tous leurs yeux sur le brun Fedallah et son équipage, les hommes des autres baleinières n’obéirent pas à l’ordre donné.\par
– Capitaine Achab ? dit Starbuck.\par
– Déployez-vous, cria Achab. Nagez partout, les quatre pirogues ! Toi, Flask, nage davantage au vent !\par
– Oui, oui, sir, répondit gaiement le petit Cabrion, en manœuvrant son grand aviron de queue. Nagez fort ! dit-il à ses hommes. Là !… là !… là ! encore ! La voilà qui souffle, droit devant, les gars ! Nagez, nagez ! Ne prenez pas garde à ces gars jaunes, Archy.\par
– Oh ! ils me sont indifférents, sir, dit Archy. Je savais déjà tout. Ne les ai-je pas entendus dans la cale ? Ne l’ai-je pas dit à Cabaco ? Qu’en dites-vous, Cabaco ? Ce sont des passagers clandestins, monsieur Flask.\par
– Nagez, nagez, mes jolis cœurs, souquez, mes enfants, souquez, mes poussins, soupirait Stubb à ses hommes d’une voix traînante et apaisante car quelques-uns montraient encore des signes d’inquiétude. Pourquoi ne rompez-vous pas l’échine, mes enfants ? Que regardez-vous ainsi ? Ces gars dans la pirogue ? Quelle bêtise ! Cela ne fait que des mains pour nous aider\par
– peu importe d’où elles viennent – plus on est de fous, plus on rit. Allons, nagez, nagez, de grâce ! Peu importe le soufre, les diables sont assez bons bougres. Bon, voilà qui va bien, c’est un coup qui vaut ses mille livres, c’est un coup qui ramasse les enjeux ! Hourra pour la coupe d’or de spermaceti, mes braves ! Trois vivats, hommes, et haut les cœurs ! Comme ça, comme ça, ne soyez pas pressés, ne soyez pas pressés. Pourquoi ne brisezvous pas vos avirons, canailles ? Un peu de mordant, chiens que vous êtes ! Bon, bon, là, comme ça, comme ça ! Voilà… comme ça, voilà ! Allongez la nage, tirez. Commencez à tirer ! Le diable vous emporte, gueux de propres à rien. Vous dormez ! Fini de ronfler, fainéants, et tirez. Nagez, voulez-vous ? Ne pouvez-vous pas nager ? Ne voulez-vous pas nager ? Je vous en donnerai des petits couteaux pour les perdre, pourquoi ne nagez-vous pas ? Nagez, et que ça barde ! Souquez, que les yeux vous en sortent ! Là ! Il tira brusquement de sa ceinture son couteau effilé. Que chaque fils de sa mère nage la lame entre les dents ! Voilà… très bien, mes tranchants de hache, voilà… Faites-la avancer, mes nés-coiffés ! Faites-la avancer, mes épissoirs !\par
Si l’exorde de Stubb est, ici, donnée en entier, c’est qu’il avait une manière à lui de s’adresser à ses hommes, tout particulièrement pour leur inculquer la religion de la rame. Que cet échantillon de sermon ne vous laisse pas à croire qu’il n’entrait jamais dans de furieuses colères contre ses ouailles. Au contraire, et c’est là qu’était son originalité. Il pouvait dire à ses hommes les choses les plus affreuses sur un ton qui alliait la plaisanterie à la rage, et sa fureur ne semblait être qu’un piment à la farce, de sorte qu’il ne se trouvait aucun canotier qui ne tirât sur les avirons comme si sa vie était en jeu et qui, en même temps, ne tirât pour la blague, en entendant ses invocations saugrenues. D’autre part, il avait toujours l’air à l’aise, si nonchalant, manœuvrant si paresseusement son aviron de queue, bâillant si largement, la bouche grande ouverte, que la seule vue d’un chef se décrochant pareillement la mâchoire, par le seul effet de contraste, agissait comme un charme sur ses hommes. De plus, Stubb était un de ces singuliers humoristes, dont la jovialité était parfois si curieusement ambiguë qu’elle mettait tous les subalternes sur la défensive lorsqu’il s’agissait d’obéir.\par
Manœuvrant sur un signe d’Achab, Starbuck avançait obliquement pour croiser l’avant de Stubb, et lorsque les deux pirogues furent assez proches l’une de l’autre pour un instant, Stubb héla le second :\par
– Monsieur Starbuck ! Ohé du canot à bâbord ! Juste un mot, s’il vous plaît, sir !\par
– Ohé ! répondit Starbuck sans faire mine de se retourner le moins du monde, pressant ses hommes à mi-voix, avec zèle, détournant de Stubb un front d’airain.\par
– Que pensez-vous de ces types jaunes, sir ?\par
– Embarqués en fraude, d’une façon ou d’une autre, avant de lever l’ancre. Tirez fort, les gars, fort ! chuchota-t-il à ses hommes, puis à haute voix de nouveau : une triste affaire, monsieur Stubb ! Faites bouillir l’eau, les gars ! Mais peu importe, monsieur Stubb, tout est pour le mieux. Que vos hommes nagent ferme et advienne que pourra. Sautez, les gars, sautez ! Il y a des barriques d’huile de cachalot en perspective et c’est pour cela que vous êtes là, monsieur Stubb, tirez, les gars ! L’huile, c’est le but ! Cela du moins c’est le devoir, le devoir et le gain vont de pair.\par
– Oui, oui, c’est bien ce que je pensais, soliloqua Stubb, lorsque les pirogues s’écartèrent, dès que je les ai vus, je me suis dit ça. Oui, et c’est la raison pour laquelle il descendait dans la cale si souvent, comme l’en soupçonnait Pâte-Molle. Ils étaient cachés en bas. Il y a la Baleine blanche là-dessous. Eh bien ! eh bien ! qu’il en soit ainsi ! On n’y peut rien ! Bon ! Ensemble, les gars. Ce n’est pas le jour de la Baleine blanche ! Ensemble !\par
La découverte de ces étrangers bizarres, à un moment aussi fatidique que la mise à la mer des pirogues, n’avait pas été sans éveiller un étonnement superstitieux chez quelques matelots, si Archy avait divulgué ses présomptions depuis un certain temps sans rencontrer alors de crédit, les hommes quelque peu préparés à cet événement éprouvaient une surprise émoussée ; d’autre part Stubb ne se formalisait pas de leur apparence et en parlait de façon rassurante, tout cela les libéra momentanément de leurs craintes superstitieuses, bien que le champ fût ouvert à toutes sortes de conjectures insensées sur le rôle véritable qu’avait joué le sombre Achab dès le début de cette affaire. Quant à moi, je me souvenais secrètement des ombres mystérieuses que j’avais vu se glisser à bord du {\itshape Péquod} dans l’eau trouble de Nantucket, ainsi que des sentences énigmatiques du bizarre Élie.\par
Pendant ce temps, Achab, hors de portée de voix de ses officiers, s’étant rangé sur le flanc extrême du côté du vent, était toujours en tête des autres pirogues, ce qui prouvait la puissance de ses rameurs. Ses créatures jaune tigre semblaient bâties d’acier et de fanons ; comme cinq marteaux à bascule, ils s’abaissaient et se levaient à coups réguliers dont la force lançait la pirogue sur l’eau comme le jet d’une chaudière horizontale d’un vapeur du Mississipi. Quant à Fedallah qui manœuvrait l’aviron de harponneur, il avait enlevé sa veste noire et sa poitrine se découpait, au-dessus du plat-bord, sur les ondulations de l’horizon liquide. À l’autre bout de la pirogue, Achab, d’un bras, déjeté en arrière comme celui d’un escrimeur pour conserver son équilibre, maniait avec fermeté l’aviron de queue comme il l’avait fait des milliers de fois avant d’avoir été déchiré par la Baleine blanche. Tout à coup son bras tendu fit un mouvement singulier et se figea tandis que les cinq avirons étaient matés. Pirogue et équipage furent immobiles sur la mer. Aussitôt les trois baleinières de l’arrière-garde s’arrêtèrent. Les cachalots avaient tous disparu dans les profondeurs, ne donnant à distance aucun mouvement indicateur, mais Achab, plus proche, les avait observés.\par
– Chacun prêt à ses avirons ! cria Starbuck. Toi, Queequeg, debout !\par
Sautant souplement dans la boîte triangulaire de l’avant, le sauvage s’y tint tout droit, fixant avec une intense avidité le lieu présumé de la chasse. À l’arrière de la pirogue, sur une identique plate-forme triangulaire à hauteur du plat-bord, Starbuck se balançait, avec une impassible adresse, aux oscillations de sa coquille de noix, et son regard plongeait silencieusement dans le regard bleu de la mer.\par
Non loin de là, Flask, insouciant, se tenait debout sur le taberin, un solide morceau de bois qui s’élève à quelque deux pieds au-dessus de la plate-forme arrière qui sert à retenir la ligne quand celle-ci est attachée au cétacé. Elle n’a pas plus que la largeur d’une main, et debout sur un si mince appui, Flask semblait perché sur la pomme du mât d’un navire naufragé dont rien d’autre n’émergerait. Mais le petit Cabrion était courtaud et râblé, et en même temps le petit Cabrion était plein d’une grande et haute ambition, de sorte que le socle de son taberin ne le satisfaisait nullement.\par
– Je n’y vois pas à trois vagues de là, tendez-moi un aviron que je grimpe dessus.\par
À ces mots, Daggoo, se tenant des deux mains au plat-bord pour assurer son équilibre, se glissa promptement à l’arrière et, se redressant, lui offrit le piédestal de ses épaules altières.\par
– Une pomme de mât qui en vaut une autre, sir. Voulezvous monter ?\par
– Je n’y manquerai pas et merci beaucoup, mon cher camarade, je voudrais seulement que vous mesuriez cinquante pieds de plus.\par
Calant alors fermement ses pieds contre les bordages opposés de la pirogue, le gigantesque nègre se baissa un peu, tendit sa paume au pied de Flask, lui prit la main pour la poser sur les plumes de corbillard de sa tête et le pria de sauter tandis que lui-même se relèverait. D’un coup adroit, il établit haut et sec le petit homme sur ses épaules. Et voilà que Flask y était à présent debout, Daggoo de son bras levé lui fournissant un parapet où s’appuyer.\par
C’est toujours un spectacle stupéfiant pour un novice que de voir l’adresse prestigieuse, rendue machinale par l’habitude, du baleinier capable de conserver sa position verticale alors même que sa pirogue tangue sur la mer la plus perfidement agitée. Mais c’était un spectacle encore plus étrange de le voir, en de telles circonstances, vertigineusement perché sur le taberin. Et la vue du petit Flask debout sur les épaules du géant Daggoo était encore plus curieuse car, avec une aisance tranquille, indifférente, inconsciente, le noble nègre, dans sa barbare majesté, accordait le rythme de son corps splendide à celui de la mer. Sur son large dos, Flask, avec ses cheveux de lin, semblait un flocon de neige. La monture avait plus de prestance que son cavalier. Vif, agité, ostentatoire, le petit Flask en venait parfois à frapper du pied avec impatience, mais la poitrine seigneuriale du nègre n’en respirait pas moins avec la même régularité, sans à-coups. J’ai vu ainsi la Passion et la Vanité taper du pied sur la terre vivante et magnanime sans qu’elle change pour autant le cours de ses marées et de ses saisons.\par
Cependant Stubb, le troisième second, ne trahissait pas semblable empressement à guetter l’horizon. Les cachalots avaient pu sonder normalement comme ils le font à intervalles réguliers et non par crainte. Si tel était le cas, Stubb, comme à l’accoutumée, était résolu à tromper l’ennui de l’attente avec sa pipe. Il la prit dans le ruban de son chapeau où il la tenait piquée comme une plume, la remplit et la bourra du pouce mais, à peine avait-il allumé son allumette au rude papier émeri de sa main, que Tashtego, son harponneur, dont les yeux étaient restés rivés au vent comme deux étoiles immobiles, tomba comme un éclair sur son banc, en criant avec une hâte frénétique :\par
– Assis, assis tous, et en avant partout ! Les voilà !\par
Un terrien en cet instant n’aurait pas vu davantage le signe d’un cachalot que celui d’un hareng ; on ne voyait rien qu’une eau verte, troublée d’un peu de blanc, et parsemée de légères touffes de vapeur que le vent emportait et prêtait à l’écume des vagues. Alentour, l’air vibrait et tintait comme s’il avait passé au-dessus de plaques de métal surchauffées. Sous ces remous et ces ondulations atmosphériques, et aussi sous une couche d’eau peu épaisse, les cétacés nageaient. Première manifestation de leur présence, ces bouffées de vapeur semblaient un détachement d’avant-garde chargé d’un message.\par
Les quatre baleinières s’étaient lancées âprement à la poursuite de cette tache d’air et d’eau troublés, mais elle les gagnait de vitesse, fuyant et fuyant encore devant eux, comme les bulles serrées qu’un torrent emporte vivement au flanc des montagnes.\par
– Nagez, nagez, mes braves garçons, disait Starbuck à ses hommes, sur le ton du chuchotement mais avec une pénétrante intensité tandis que la flèche de son regard était fichée droit audelà de la proue, semblable à l’aiguille qui ne dévie pas d’un compas d’habitacle. Il ne parlait guère à son équipage et son équipage ne lui parlait guère. Le silence de la pirogue n’était rompu à intervalles que par ses chuchotements singuliers, où la rudesse du commandement alternait avec la douce sollicitation.\par
Combien différent était le tapageur petit Flask :\par
– Donnez de la voix, dites quelque chose, mes jolis cœurs. Rugissez et nagez, mes tonnerres ! Débarquez-moi sur leurs dos noirs, les gars, faites ça pour moi, et je vous cède par écrit ma propriété de Martha’s Vineyard, femme et enfants compris. Débarquez-moi, oh ! débarquez-moi sur leur dos ! Ô Seigneur, Seigneur ! je vais devenir fou à lier ! Regardez ! Regardez cette eau blanche !\par
En hurlant de la sorte, il enleva son chapeau, le jeta, le piétina, puis enfin l’envoya au large en vol plané. Ensuite il se livra à force cabrioles dévergondées à l’arrière de la pirogue comme un poulain fou de la Prairie.\par
Stubb, qui suivait à peu de distance, sa courbe pipe non allumée, machinalement serrée entre les dents, émit philosophiquement d’une voix traînante :\par
– Regardez-moi ce type. Il a des crises, ce Flask. Des crises ? Oui, et qu’on lui en donne des crises – c’est bien le mot – des crises d’hystérie. Avec entrain, allez-y avec entrain, mes jolis cœurs. Il y aura du pudding pour le dessert, vous savez – avec entrain – c’est bien le mot – nagez, mes bébés… nagez, mes nourrissons… nagez tous. Pourquoi diable vous agitez-vous, comme ça, comme ça et fort. Tirez seulement et continuez de tirer. Rompez-vous l’échine et brisez la lame de votre couteau entre vos dents… c’est tout. Avec souplesse… doucement j’ai dit, et faites-vous sauter le foie et les poumons !\par
Mais ce que disait l’impénétrable Achab à son jaune équipage, mieux vaut ne pas le répéter ici car vous vivez dans la lumière bénie d’une terre chrétienne et seuls les requins mécréants des mers audacieuses peuvent prêter l’oreille à des mots pareils tandis qu’Achab bondissait après sa proie, l’ouragan aux sourcils, la pourpre du crime dans les yeux et l’écume aux lèvres.\par
Cependant toutes les baleinières s’arrachaient à l’eau. Les allusions précises et répétées de Flask à « cette baleine » comme il appelait le monstre imaginaire dont il disait que la queue aguichait la proue de sa pirogue, ces allusions étaient si vigoureusement vivantes, qu’elles amenaient un ou deux de ses hommes à lancer par-dessus son épaule un regard apeuré. C’était contraire à la règle qui veut que les canotiers gardent leur fixité comme s’ils avaient avalé un sabre, règle qui réclame qu’ils n’aient d’autre sens que l’ouïe, d’autres membres que des bras dans ces moments critiques.\par
Le spectacle inspirait un émerveillement aigu de crainte et de respect. Les vastes houles de la mer toute-puissante, leur grondement caverneux tandis qu’elles roulaient sur les huit plats-bords, jeu de quilles géant sur un terrain de jeu sans limites, la brève agonie de la pirogue suspendue, comme si elle allait basculer sur le tranchant de la vague, si effilé qu’on eût cru qu’il fallait couper en deux, la descente soudaine dans le ravin liquide, le coup d’éperon et d’aiguillon acerbe donné pour regrimper l’autre versant de la colline, la glissade tête première depuis ce sommet, les cris des chefs et des harponneurs, le halètement frémissant des canotiers, le prestigieux {\itshape Péquod} d’ivoire veillant sur ses baleinières les ailes ouvertes, semblable à une perdrix des neiges suivant sa piaillante progéniture, tout cela était poignant. Ni la nouvelle recrue arrachée à l’étreinte de sa femme pour être jetée dans la fièvre de sa première bataille, ni l’âme d’un mort confrontée au premier fantôme inconnu de l’autre monde, ni l’un ni l’autre ne peuvent éprouver des émotions plus violentes et plus étranges que celui qui se trouve pris pour la première fois dans le cercle magique et bouillonnant du cachalot pourchassé.\par
La danse de l’eau soulevée par la poursuite accusait une blancheur plus intense dans les ténèbres croissantes que les nuages noirs projetaient sur la mer. Les bouffées de vapeur n’allaient plus se confondant mais jouaient de droite et de gauche, les cachalots semblaient aller leur chemin séparément. Les pirogues s’écartèrent les unes des autres, Starbuck amena sur trois cachalots qui filaient sous le vent. Nous avions mis à la voile et le vent fraîchissant toujours nous emportait si follement que les avirons arrivaient à peine à suivre le mouvement, menaçant d’être arrachés à leurs dames de nage.\par
Bientôt nous courûmes dans un voile de brume largement déployé, dérobant à la vue baleinières et navire.\par
– En avant, hommes, chuchota Starbuck en bordant son écoute de voile et en serrant la toile, nous avons le temps de tuer un poisson avant le gros du grain. Voilà de nouveau l’eau blanche ! Près à toucher ! Sautez !\par
Peu après, deux cris retentirent presque simultanément de part et d’autre de notre pirogue, indiquant que les autres avaient foncé eux aussi. À peine nous étaient-ils parvenus que le chuchotement de Starbuck tomba dru comme grêle : « Debout » et Queequeg bondit le harpon à la main.\par
Aucun des canotiers ne pouvait voir de face le péril mortel qui était devant eux mais, les yeux rivés sur le visage tendu de leur chef, ils savaient que l’instant ultime était arrivé, ils entendirent aussi un bruit énorme semblable à celui qu’eussent fait cinquante éléphants en train de se vautrer. Cependant la pirogue bondissait toujours à travers la brume entre les vagues dressées et sifflantes comme autant de serpents furieux.\par
– Voilà sa bosse. Là, là ! murmura Starbuck.\par
Un sifflement bref jaillit de la pirogue, l’envol du fer de Queequeg. Alors l’arrière de la baleinière fut ébranlé par une poussée invisible, tandis que l’avant paraissait heurter un récif, la voile s’affaissa en claquant, un jet de vapeur brûlante fusa tout près de nous, un séisme roula son chaos au-dessous de nous. L’équipage à demi suffoqué fut lancé dans le tohu-bohu écumant du grain blanc. Le grain, le cachalot, le harpon n’avaient fait qu’un et la bête, à peine effleurée par le dard, s’était enfuie.\par
Bien qu’entièrement submergée, l’embarcation était presque indemne. Nous récupérâmes à la nage les avirons qui flottaient et après les avoir jetés par-dessus le plat-bord nous regagnâmes nos places en titubant. Nous nous trouvâmes assis là, de l’eau jusqu’aux genoux dans la pirogue quasi pleine et nos yeux baissés sur cette coque flottant à ras des flots nous offraient la vision d’un atoll de corail monté jusqu’à nous du fond de l’Océan.\par
Le vent s’enfla jusqu’au hurlement, les vagues entrechoquèrent leurs boucliers, le grain rugit, pétilla, s’embrasa tout autour de nous tel un feu blanc allumé sur la prairie, dans lequel nous brûlions sans être consumés, immortels entre ces mâchoires de mort ! En vain hélions-nous les autres pirogues. Autant s’adresser en hurlant aux charbons ardents d’un brasier que d’appeler dans pareille rafale. Cependant la tourmente et la brume s’assombrissaient avec la nuit tombante, le navire était devenu invisible ; la fureur de la mer interdisait toute tentative d’écoper ; les avirons devenus inutiles ne pouvaient plus servir que de moyens de sauvetage. Ouvrant le barillet étanche des allumettes, Starbuck, après de nombreux, efforts infructueux, parvint à allumer le fanal, l’amarra au manche d’une lance naufragée et le tendit à Queequeg qui devint porte-étendard de notre morne espoir. Et là il se trouvait assis, élevant cette chandelle imbécile au cœur de cet abandon infini. Et là il se trouvait assis, signe et symbole de l’homme sans foi élevant sans espoir l’espérance au sein de la désespérance.\par
Transpercés, grelottants, renonçant à apercevoir baleinières ou navire, nous levâmes les yeux vers l’aube naissante, la brume s’étendait sur la mer, le fanal brisé gisait au fond de la pirogue. Soudain Queequeg se dressa, portant sa main en cornet à son oreille. Nous perçûmes un léger craquement de gréement et de vergues jusqu’alors étouffé par la tempête. Le son se rapprocha toujours davantage et une forme immense et indistincte fendit l’épaisseur de la brume. Épouvantés, nous sautâmes à la mer tandis que le navire fonçait droit sur nous à une distance qui n’excédait guère sa longueur.\par
La pirogue abandonnée flottait sur les vagues, puis elle se retourna et fut avalée par la mer sous l’étrave du navire comme un copeau sous la chute d’une cascade. Engloutie sous la lourde quille, elle disparut à nos yeux jusqu’à ce qu’elle émergeât à nouveau à la poupe. Une fois de plus nous la regagnâmes à la nage, jetés vers elle par les lames, jusqu’à ce qu’enfin nous fûmes recueillis sains et saufs à bord. Lorsque le grain s’était rapproché, les autres baleinières avaient coupé leurs lignes et regagné le navire à temps ; celui-ci avait perdu tout espoir de nous retrouver mais croisait toujours, en quête d’un témoignage de notre naufrage, quelque rame ou quelque manche de lance.
\chapterclose


\chapteropen
\chapter[{CHAPITRE XLIX. L’hyène}]{CHAPITRE XLIX \\
L’hyène}\renewcommand{\leftmark}{CHAPITRE XLIX \\
L’hyène}


\chaptercont
\noindent Il est des moments et des circonstances dans cette affaire étrange et trouble que nous appelons la vie où l’univers apparaît à l’homme comme une farce monstrueuse dont il ne devinerait que confusément l’esprit tout en ayant la forte présomption que la plaisanterie se fait à ses dépens et à ceux de nul autre. Pourtant, rien ne l’abat, comme rien ne lui paraît valoir la peine de combattre. Il avale tous les événements, tous les credo, toutes les croyances, toutes les opinions, toutes les choses visibles et invisibles les plus indigestes, si coriaces soient-elles comme l’autruche à la puissante digestion engloutit les balles et les pierres à fusil. Car les petites difficultés et les soucis, les présages d’un proche désastre ne lui semblent que des traits sarcastiques décrochés par la bonne humeur, des bourrades joviales dans les côtes expédiées par un farceur invisible et énigmatique. Cette humeur insolite et fantasque ne s’empare d’un homme qu’au paroxysme de l’épreuve ; ce qui, l’instant d’avant, dans sa ferveur lui apparaissait si grave, ne lui semble plus qu’une scène de la farce universelle. Rien de tel que les dangers de la chasse à la baleine pour développer cette libre et insouciante cordialité, cette philosophie désespérée ! C’est sous ce jour que désormais, je vis la croisière du {\itshape Péquod} et son but : la grande Baleine blanche.\par
– Queequeg, dis-je, lorsqu’ils m’eurent repêché le dernier et tandis que je m’ébrouais sur le pont pour faire dégorger ma vareuse. Queequeg, mon bel ami, ces sortes de choses arriventelles souvent ?\par
Sans grande émotion, bien qu’aussi trempé que moi, il me donna à entendre qu’elles sont fort fréquentes.\par
– Monsieur Stubb, dis-je en me tournant vers ce digne personnage qui, son ciré dûment boutonné, fumait sa pipe sous la pluie, monsieur Stubb, je crois vous avoir entendu dire que de tous les baleiniers que vous avez rencontrés, M. Starbuck était de loin le plus prudent et le plus sage. Je pense dès lors que piquer droit sur une baleine en fuite, la voile hissée dans la rafale et dans la brume, représente le comble de la réserve pour un baleinier ?\par
– Certes. J’ai mis à la mer, avec une pirogue qui faisait eau, dans une tempête, au large du cap Horn.\par
– Monsieur Flask, dis-je alors en me tournant vers le petit Cabrion qui était à côté, vous avez l’expérience de ces choses et pas moi. Pouvez-vous me dire si c’est pour un canotier une loi inviolable de cette pêche que de se rompre l’échine pour se pousser plus avant entre les mâchoires de la mort ?\par
– Ne pourriez-vous pas vous exprimer d’une façon encore plus tarabiscotée ? Oui, c’est la loi. Je voudrais bien voir l’équipage d’une pirogue scier et aborder le cachalot face à face. Ha ! ha ! il vous rendrait œil pour œil, soyez-en sûr !\par
Trois témoins avisés et impartiaux m’avaient dès lors mis au fait. Considérant que les rafales, les embarcations renversées et les bivouacs dans les profondeurs étaient la monnaie courante de ce genre de vie, considérant qu’à l’instant décisif d’amener sur la baleine je devais remettre ma vie entre les mains du timonier, souvent un gars capable d’envoyer la pirogue par le fond à cause de son impétuosité et de ses piétinements frénétiques, considérant que notre désastre était principalement imputable à Starbuck amenant sur le cachalot au nez du grain qui arrivait sur nous, considérant que Starbuck était cependant connu pour être circonspect, considérant que j’appartenais à la baleinière de ce très prudent Starbuck, considérant enfin le caractère diabolique de cette chasse à la Baleine blanche dans laquelle j’étais engagé, considérant toutes choses, dis-je, je songeai que mieux valait descendre rédiger un brouillon rapide de mes dernières volontés.\par
– Venez, Queequeg, dis-je, vous serez mon notaire, mon exécuteur testamentaire, et mon héritier.\par
Il peut paraître étrange que de tous les hommes, les marins soient les plus occupés à bricoler leurs testaments et dernières volontés, pourtant personne au monde n’est plus féru de cet amusement. C’est la quatrième fois que je m’y surprenais au cours de ma vie maritime. Lorsque cette fois la cérémonie fut terminée, je me sentis bien aise, cela m’avait enlevé un poids du cœur. D’autre part, tous les jours à venir auraient la saveur de ceux que vécut Lazare après sa résurrection, un bénéfice net de tant de mois ou de semaines suivant le cas. Je me survivais. Ma mort et mes funérailles étaient enfermées à clef dans mon coffre. Je jetais autour de moi un regard paisible et satisfait, tel un calme fantôme à la conscience limpide, douillettement installé dans un caveau de famille.\par
Et maintenant, me disais-je, en remontant inconsciemment mes manches, en avant pour un plongeon glacé et de sang-froid dans la mort et la destruction. Chacun pour soi et Dieu pour tous.
\chapterclose


\chapteropen
\chapter[{CHAPITRE L. La pirogue d’Achab, son équipage. Fedallah}]{CHAPITRE L \\
La pirogue d’Achab, son équipage. Fedallah}\renewcommand{\leftmark}{CHAPITRE L \\
La pirogue d’Achab, son équipage. Fedallah}


\chaptercont
\noindent – Qui l’eût cru, Flask ! s’écria Stubb. Si je n’avais qu’une jambe, on ne me prendrait pas à descendre dans une pirogue, à moins que ce ne soit pour boucher son nable avec mon pilon. \hspace{1em}Ah ! c’est un étonnant vieil homme !\par
– Tout compte fait, je ne trouve pas ça si extraordinaire, répondit Flask. S’il était amputé depuis la hanche, ce serait une autre affaire, il serait hors de combat, mais il lui reste un genou et une bonne partie de l’autre, vous savez.\par
– Ça, je n’en sais rien, mon petit bonhomme, je ne l’ai encore jamais vu se mettre à genoux.\par
Les gens compétents ont souvent discuté pour savoir si, étant donné l’importance majeure d’un capitaine dans le succès d’une expédition baleinière, il est juste que celui-ci expose sa vie aux dangers de la chasse. Les larmes aux yeux, les soldats de Tamerlan se sont de même demandé si sa vie inestimable devait être risquée au plus fort de la bataille.\par
Avec Achab, la question se posait différemment. Étant donné qu’un homme qui a ses deux jambes n’est jamais qu’un boiteux dans le danger, étant donné que la chasse à la baleine est toujours faite de difficultés effrayantes et que chacun de ses instants comporte un péril, est-il sage, dès lors, \hspace{1em} qu’un estropié descende dans une baleinière au moment de la chasse ? Tel ne devait pas être l’avis des armateurs du {\itshape Péquod}.\par
Achab savait bien que ses amis n’auraient pas désapprouvé qu’il descendît dans une baleinière lorsque les dangers de la chasse étaient moindres, afin de se trouver dans le champ d’action et de donner personnellement des ordres, mais quant à avoir une pirogue à lui, à en être le chef, qui plus est, à disposer de cinq hommes n’appartenant pas à l’équipage enrôlé, il savait très bien que des points de vue aussi généreux n’auraient pas effleuré les propriétaires du {\itshape Péquod}. C’est pourquoi il ne leur avait jamais demandé de lui fournir des hommes à cet effet, pas plus qu’il avait fait d’allusions à ce sujet. Néanmoins, il avait, de sa propre initiative, pris des mesures à cet effet. Jusqu’à ce que la découverte de Cabaco se fût répandue, les matelots ne s’étaient doutés de rien, quoique, bien sûr, l’équipage une fois au large ait, comme il est de coutume, achevé de parer les baleinières pour les mettre en service. Alors la curiosité s’éveilla vivement du fait qu’Achab fut, de temps en temps, surpris à faire de ses propres mains des tolets pour ce qu’on tenait pour une baleinière de réserve, et même à tailler avec amour les épinglettes de bois que l’on fixe sur la rainure de proue lorsqu’on file la ligne. Tout cela fut remarqué et plus particulièrement le soin qu’il prit à ce qu’il y ait une épaisseur de natte supplémentaire au fond de la pirogue, comme si elle devait résister à la pression aiguë de sa jambe d’ivoire, et le souci qu’il eut de veiller à ce que la forme la plus parfaite fût donnée au bordé de cuisse, cette encoche creusée, à la proue, pour recevoir et tenir les jambes du harponneur lorsqu’il s’apprête à lancer le harpon ou à manier la lance. On remarqua encore combien souvent il se tenait dans la baleinière, son unique genou emboîté dans le creux arrondi du bordé de cuisse et avec la gouge du charpentier creusant un peu davantage ici, rectifiant un peu là. Tout cela, dis-je, piqua l’intérêt, mais presque tout le monde supposa que ces précautions n’étaient prises par Achab qu’en vue de la chasse de Moby Dick puisqu’il avait déjà manifesté son intention de chasser personnellement ce monstre. Ces suppositions n’entraînaient nullement le moindre soupçon qu’un équipage dût être affecté à cette baleinière.\par
L’étonnement eut tôt fait de s’évanouir avec l’apparition des subalternes-fantômes car, sur un baleinier, les effets de surprise durent peu. En effet, ici ou là, des rebuts indicibles surgissent de nations étrangères, de coins inconnus, de dépotoirs de toute la terre, pour équiper ce hors-la-loi flottant qu’est un navire baleinier, et ces navires mêmes repêchaient souvent des naufragés se débattant en pleine mer accrochés à des planches, des morceaux d’épave, des rames, des baleinières, des canots, des jonques japonaises brisées, que sais-je encore, si bien que Belzébuth en personne fût-il grimpé à bord et fût-il descendu dans la cabine du capitaine pour y faire un brin de causette, cela n’eût pas créé sur le gaillard d’avant une émotion fascinée.\par
Quoi qu’il en soit, il est certain que les fantômes eurent tôt fait de s’intégrer à l’équipage ; toutefois, comme s’il n’en faisait pas partie, Fedallah l’enturbanné demeura jusqu’au bout enveloppé de mystère. D’où venait-il dans un monde policé comme celui-ci, quel était le lien impénétrable qui l’unissait à la destinée d’Achab au point d’exercer sur lui un secret ascendant ? Dieu seul le sait, mais il semblait même qu’il eût sur lui de l’autorité, et quand bien même personne ne savait rien à ce sujet, on ne pouvait rester indifférent devant Fedallah. C’était un personnage que les habitants civilisés, domestiqués de la zone tempérée ne voient qu’en rêve et encore confusément, mais dont on voit les pareils se glisser dans les immuables communautés asiatiques, en particulier dans les îles orientales à l’est de ce continent, terres isolées, inviolées, immémoriales qui, maintenant encore, conservent intact le mystère primitif de l’homme à l’origine des temps, de ceux où le souvenir du premier homme était encore vivant et où ses descendants se regardaient les uns les autres comme des revenants, et demandaient à la lune et au soleil la raison pour laquelle ils avaient été créés et à quelle fin ; bien que, à en croire la Genèse, les anges aient eu des rapports avec les filles des hommes, et que, selon des rabbins non canoniques, les démons se soient également livrés à de terrestres amours.
\chapterclose


\chapteropen
\chapter[{CHAPITRE LI. Le souffle-esprit}]{CHAPITRE LI \\
Le souffle-esprit}\renewcommand{\leftmark}{CHAPITRE LI \\
Le souffle-esprit}


\chaptercont
\noindent Des jours, des semaines passèrent. Sous voilure réduite, le {\itshape Péquod} d’ivoire avait lentement traversé quatre parages de croisières différents, celui des Açores, celui du cap Vert, de la Plata, ainsi appelé parce qu’il se trouve au large de l’embouchure du Rio de la Plata, et celui du lieu-dit de Carroll, étendue de mer mal délimitée au sud de Sainte-Hélène.\par
Ce fut tandis que nous glissions dans ces dernières eaux par une sereine nuit de lune, tandis que les vagues s’enroulaient en volutes d’argent et que leur doux bouillonnement diffus transformait la solitude en silence argentin, ce fut par une telle nuit silencieuse que bien au-delà de la blanche écume de l’étrave nous vîmes un souffle argenté. La lumière de la lune le faisait paraître céleste ; on eût dit, sortant des flots, quelque dieu étincelant, paré de plumes. Fedallah fut le premier à l’apercevoir car c’était son habitude, par ces nuits lunaires, de monter au grand mât, et d’y guetter d’un œil aussi sûr que s’il eût fait grand jour. Pourtant, bien que des gammes de baleines soient parfois aperçues de nuit, il ne se trouvait pas un baleinier sur cent pour s’aventurer à mettre alors à la mer. Vous imaginez, dès lors, avec quelle émotion les matelots voyaient ce vieil oriental perché si haut, à des heures aussi indues, le ciel mariant son turban à la lune. Mais lorsqu’il eut passé plusieurs nuits de suite, pendant un même laps de temps, dans la mâture, sans proférer le moindre son, lorsque après tant de silence sa voix surnaturelle retentit signalant ce souffle argenté de lune, alors chaque homme bondit sur ses pieds comme si un esprit ailé avait illuminé le gréement et appelé cet équipage de mortels. « La voilà qui souffle ! » Leur frisson n’aurait pas été plus grand si la trompette du Jugement avait retenti, pourtant ils n’éprouvèrent aucune terreur, et au contraire un certain plaisir car, bien que l’heure fût malvenue, le cri était si solennel, si délirant, si émouvant, que chaque âme à bord eut le désir instinctif de mettre les pirogues à la mer.\par
Traversant le pont à enjambées rapides et obliques, Achab donna l’ordre de déployer les perroquets et les cacatois, ainsi que toutes les bonnettes. L’homme le plus qualifié dut prendre la barre, les trois postes de vigie furent parées, et le navire, sous toute sa voile, courut vent arrière. La brise gonflant tant de voiles donnait l’impression étrange de soulever le navire et de laisser l’air circuler librement sous un pont qui aurait flotté et pourtant, tandis qu’il courait ainsi, il semblait la proie de deux influences contraires, l’une tendant à l’emporter droit au ciel, l’autre à le faire embarder horizontalement. Et si vous aviez pu observer le visage d’Achab, cette nuit-là, vous auriez pensé que deux antagonistes se livraient combat en lui. Tandis que sa jambe vivante éveillait sur le pont des échos de la vie, chaque coup de sa jambe morte clouait un cercueil. C’est sur la vie et sur la mort que marchait ce vieil homme. Mais bien que le navire eût filé à vive allure, bien que les regards avides eussent lancé toutes leurs flèches sur la mer, le souffle d’argent ne devait pas être revu cette nuit-là. Chaque matelot affirma l’avoir aperçu une fois, mais une seule.\par
Ce souffle de minuit était presque oublié lorsque, quelques jours plus tard, voici que, à la même heure de silence, il fut à nouveau annoncé, voici que tous déclarèrent à nouveau l’avoir vu, mais lorsqu’on mit le cap sur lui, il disparut de même, comme s’il n’avait jamais été. Il en alla ainsi nuit après nuit, jusqu’à ce que chacun n’y prêta plus attention sinon pour s’en étonner. Jaillissant mystérieusement soit à la clarté de la \hspace{1em}lune, soit à celle des étoiles, disparaissant pendant un jour ou deux, ou trois, et paraissant chaque fois surgir à une plus grande distance, ce souffle solitaire semblait vouloir nous entraîner à jamais plus avant.\par
Soumis à la superstition de tout temps inhérente à leur race, sensibilisés par le caractère surnaturel du {\itshape Péquod}, certains matelots juraient qu’où que ce soit, en quelque temps que ce fût, si immémorial qu’il puisse être, sous quelque latitude, quelque longitude si lointaine qu’elle puisse être, ce souffle inaccessible était la respiration d’une seule et même baleine, et qu’elle avait nom Moby Dick. Pendant un certain temps, cette apparition fugitive engendra une singulière terreur, comme si elle nous invitait traîtreusement à la suivre, toujours plus loin, afin que le monstre pût faire volte-face et nous déchiqueter enfin dans des mers écartées et sauvages.\par
Ces appréhensions passagères, si vagues mais si atroces, gagnaient en puissance par leur contraste avec la sérénité du temps ; certains découvraient sous son azur caressant un maléfice satanique tandis que nous voguions, jour après jour, sur des mers solitaires d’une douceur obsédante, comme si l’espace, réprouvant notre quête vengeresse, se vidait de toute vie devant l’urne funéraire de notre proue.\par
Mais quand enfin nous fîmes route vers l’est, les vents du Cap commencèrent à hurler autour de nous et ses eaux tourmentées nous soulevèrent et nous abaissèrent longuement. Alors, quand le {\itshape Péquod} poignarda le vent de ses défenses d’ivoire et, dans sa folie, éventra les vagues sombres jusqu’à ce que les flocons d’écume s’envolassent au-dessus des pavois comme une averse de copeaux d’argent, alors s’évanouit la désolation vide de la vie mais pour faire place seulement à de plus lugubres spectacles.\par
Près de l’étrave, surgissaient ici et là des formes étranges tandis que nous suivaient des vols serrés d’impénétrables cormorans. Chaque matin, ils se perchaient en rang sur les étais et se cramponnaient aux filins longuement, obstinément, en dépit de nos huées, comme s’ils prenaient notre navire pour une épave abandonnée à la dérive, vouée à la désolation, et dès lors pour un perchoir idéal pour eux, les sans-asile. Et le sombre Océan se soulevait, respirait, se soulevait encore dans une incessante angoisse comme la marée d’une conscience, comme si l’âme du monde y exprimait le remords torturant d’avoir engendré de si longues souffrances et le si lourd péché.\par
On te nomme, n’est-ce pas, cap de Bonne-Espérance ? Plutôt cap des Tempêtes comme on t’appelait jadis. Longuement séduits par les silences perfides qui nous avaient entourés, nous nous trouvions jetés dans la tourmente où les âmes coupables incarnées en oiseaux, en poissons, semblaient condamnées à nager éternellement sans espoir d’un havre, ou à brasser de leurs ailes un air ténébreux dépourvu d’horizon. Mais paisible, d’une blancheur de neige, immuable, sa fontaine de plumes jaillissant toujours vers le ciel, nous invitant sans cesse, le souffle solitaire était aperçu parfois.\par
Pendant cette période de ténèbres, Achab, bien qu’assurant le commandement sur le pont glissant et dangereux montra une ombrageuse réserve, adressant moins que jamais la parole à ses seconds. Par de pareilles tempêtes, lorsque tout est assuré tant sur le pont que dans la mâture, il n’y a plus rien à faire qu’à attendre qu’elles passent. Le capitaine et son équipage devenaient fatalistes. De sorte que, sa jambe d’ivoire fixée dans son trou de tarière comme à l’accoutumée, une main agrippée à un hauban pendant des heures et des heures, Achab se tenait debout, le regard au vent, tandis que, parfois, une rafale de neige fondue soudait de gel ses cils. Pendant ce temps, les hommes chassés du gaillard d’avant par les paquets de mer qui passaient pardessus les pavois, s’alignaient le long du bastingage de la \hspace{1em} coursive et pour se mieux protéger des lames ils s’assuraient dans des boulines fixées aux râteliers des haubans qui leur faisaient une ceinture lâche. On ne disait mot ou presque et le navire réduit au silence, comme s’il était monté par des matelots de cire peinte, s’arrachait, jour après jour, à la folie et à la joie démoniaque des vagues. La nuit, les hommes opposaient un mutisme pareil aux cris aigus de l’Océan ; taciturnes, ils étaient bercés dans leurs boulines et sans mot dire, Achab faisait front au vent. Même lorsque la nature semblait réclamer une trêve, il n’allait pas chercher le repos de son hamac. Jamais Starbuck n’oublierait l’aspect du vieil homme, lorsqu’il descendit une nuit dans la cabine pour regarder le baromètre, et le vit assis tout droit sur sa chaise rivée au plancher, les yeux fermés, son chapeau et son caban qu’il n’avait pas ôtés dégouttant d’eau et de neige fondue ; sur la table, près de lui, le rouleau d’une carte des courants et marées dont nous avons parlé. Dans sa main crispée, son fanal se balançait. Bien qu’il fût assis avec raideur, sa tête était rejetée en arrière de sorte que ses yeux clos étaient dirigés vers l’aiguille de l’axiomètre pendu au plafond \footnote{Le compas renversé ou axiomètre permet au capitaine de savoir, depuis sa chambre, quel est le cap du navire sans avoir besoin d’aller consulter le compas de route.}.\par
« Terrible vieil homme ! pensa Starbuck en frissonnant, tu dors dans la tempête et pourtant ton regard reste implacablement fixé sur ton but ! »
\chapterclose


\chapteropen
\chapter[{CHAPITRE LII. L’Albatros}]{CHAPITRE LII \\
\textbf{{\itshape L}’Albatros}}\renewcommand{\leftmark}{CHAPITRE LII \\
\textbf{{\itshape L}’Albatros}}


\chaptercont
\noindent Au sud-est du Cap, au large des lointaines Crozets, un bon parage de pêche pour la baleine franche, un voilier se dessina devant nous. C’était le {\itshape Diomède} ({\itshape Albatros}). Tandis qu’il se rapprochait lentement, du haut de mon perchoir élevé au mât de misaine, m’apparut ce spectacle étonnant pour un novice de ces pêches de plein Océan : un navire-baleinier en haute mer et ayant depuis longtemps quitté son port d’attache.\par
Le foulon des vagues l’avait blanchi comme le squelette d’un morse échoué. De longues traînées de rouille sillonnaient de sang ses flancs fantomatiques, les branches épaisses de ses espars et de son gréement étaient enrobées de givre comme un arbre hivernal. Il ne portait que ses basses voiles. Ses hommes de vigie aux longues barbes, perchés dans ses trois mâts, offraient une étrange vision. Ils semblaient enveloppés de peaux de bêtes tant étaient déchirés et rapiécés leurs vêtements qui avaient survécu à quelque quatre ans de voyage. Debout dans les cerceaux de fer fixés au mât, ils oscillaient sur une mer sans fond et bien que, leur navire venant à se trouver proche de notre poupe, nous autres, hommes juchés dans les airs, eussions pu sauter de nos mâts sur les leurs, ces pêcheurs de si pitoyable apparence se contentèrent de nous regarder avec douceur sans mot dire à nos hommes de vigie, cependant qu’un appel partait du gaillard d’arrière :\par
– Ohé ! du navire ! Avez-vous vu la Baleine blanche ?\par
Mais tandis que le capitaine étranger, penché sur son pavois décoloré, s’apprêtait à élever jusqu’à ses lèvres son portevoix, il lui échappa des mains et tomba à la mer. Or, le vent ayant fraîchi à nouveau, ce fut en vain qu’il tenta de se faire entendre sans cornet, cependant que son navire s’éloignait du nôtre. Tandis que les marins, chacun à leur manière, interprétaient en silence cet incident comme un mauvais présage, provoqué par la seule mention de la Baleine blanche faite à un autre navire, Achab parut réfléchir et on eût presque dit qu’il envisageait de mettre à la mer pour aborder le navire étranger, n’eût été la violence du vent. Profitant de l’avantage de se trouver au vent, il saisit de nouveau son porte-voix et, reconnaissant à son aspect que le navire était de Nantucket et qu’il y serait bientôt de retour, il cria fortement :\par
– Ohé ! là-bas ! Ici, le {\itshape Péquod}, en route pour le tour du monde ! Dites-leur d’adresser nos lettres dans l’océan Pacifique ! Et si je ne suis pas de retour d’ici trois ans, qu’ils les adressent à…\par
À cet instant les deux sillages se croisèrent et aussitôt obéissant à leurs mœurs étranges, des bancs d’inoffensifs petits poissons qui des jours durant nous avaient tranquillement escortés, filèrent loin de nous, les nageoires toutes frissonnantes, semblait-il, pour s’aller serrer à l’avant et à l’arrière du navire étranger. Bien qu’au cours de ses innombrables voyages Achab ait dû plus d’une fois voir se produire pareil phénomène, l’obsession sait charger d’un sens fantastique la moindre vétille. Il regarda la mer et murmura :\par
– Vous vous éloignez de moi, n’est-ce pas ?\par
Ces mots en eux-mêmes étaient anodins, mais le timbre de sa voix trahissait un abandon, une tristesse profonde, tels que le vieil homme insensé n’en avait jamais encore manifestés. Mais se tournant vers le timonier qui jusqu’alors avait loffé pour ralentir le navire, il rugit de son habituelle voix de lion :\par
– La barre au vent ! Le cap sur le tour du monde !\par
Le tour du monde ! On peut tirer gloire des ces mots mais cette circumnavigation où mène-t-elle, à travers d’innombrables dangers, sinon au point dont nous sommes partis, ceux que nous avions laissés à l’abri derrière nous se trouvant toujours de ce fait devant nous.\par
Ce monde serait-il une plaine sans fin, qu’en faisant voile vers l’est, nous pourrions sans cesse atteindre des lieux nouveaux et découvrir des charmes plus étranges que ceux des Cyclades et des îles Salomon, le voyage alors serait une espérance. Mais lorsque nous poursuivons nos rêves autour d’un noyau de mystère, ou que nous livrons une chasse torturante à ce démon invisible qui, une fois ou l’autre, nage devant tout cœur humain, nous tournons en rond autour du monde et nous pénétrons dans de stériles labyrinthes, ou nous sombrons en chemin.
\chapterclose


\chapteropen
\chapter[{CHAPITRE LIII. La gamme}]{CHAPITRE LIII \\
La gamme}\renewcommand{\leftmark}{CHAPITRE LIII \\
La gamme}


\chaptercont
\noindent La prétendue raison pour laquelle Achab n’avait pas rendu visite au baleinier dont nous venons de parler était que le vent et la mer menaçaient. Si tel n’avait pas été le cas, peut-être qu’il ne l’eût pas davantage abordé pour autant, si l’on en juge par la façon dont il se conduisit ultérieurement dans des circonstances identiques, lorsque, après avoir hélé un navire, il obtenait une réponse négative à sa question. Car il se révéla qu’il ne tenait pas à avoir de rapport avec un capitaine étranger, fût-ce pour cinq minutes, sauf si celui-ci pouvait lui fournir quelques-uns des indices à l’affût desquels il était avec vigilance. Mais tout cela resterait peu clair si nous ne disions ici un mot des usages établis entre navires baleiniers lorsqu’ils se rencontrent en des mers étrangères et plus particulièrement dans des parages de croisière communs.\par
Si deux étrangers viennent à se rencontrer dans les Pine Barrens de l’État de New York ou dans la plaine non moins désolée de Salisbury en Angleterre, si les hasards les mettent face à face dans des solitudes aussi inhospitalières, ils peuvent difficilement, fût-ce au prix de leur vie, éviter de se saluer et de s’arrêter un instant pour se donner des nouvelles, peut-être même de s’asseoir un moment et de se reposer ensemble. Il est combien plus naturel dès lors que, sur ces landes illimitées de la mer, deux navires baleiniers s’entr’apercevant aux confins du monde, au large de la solitaire île Fanning ou des distantes îles Gilbert, combien plus naturel, dis-je, qu’en pareilles circonstances ces navires ne se bornent pas à échanger des salutations mais recherchent un contact plus intime, plus amical, plus fraternel. Il semble que cela coule de source lorsque deux navires appartiennent, de plus, au même port d’attache et que le capitaine, ses officiers et bon nombre de ses hommes se connaissent personnellement et ont, dès lors, à parler de toutes sortes de choses chères et familiales.\par
Celui qui est parti depuis peu de temps a peut-être à bord des lettres destinées au navire depuis longtemps en mer, il est, à coup sûr, en possession de journaux plus récents d’une année ou deux que les liasses maculées et écornées du navire parti le premier. En courtois échange, le navire sur le chemin du retour donnera des renseignements tout frais et de la plus haute importance sur les parages de pêche où l’autre se rend éventuellement. Cela reste vrai aussi jusqu’à un certain point, lorsque les navires sont en mer depuis le même laps de temps, se rencontrant sur le lieu de croisière même car l’un d’eux peut avoir du courrier reçu d’un troisième navire, maintenant à son tour fort éloigné, et peut-être destiné au bâtiment rencontré. D’autre part, ils peuvent échanger des nouvelles de la pêche et bavarder agréablement, étant non seulement unis par la sympathie des marins entre eux, mais par cette communauté de sentiments engendrée par une même poursuite, par des privations et des dangers semblables.\par
La différence de nationalité ne crée pas de distinctions essentielles tant que le langage demeure le même comme c’est le cas pour les Américains et les Anglais. En vérité, ces rencontres sont rares, vu le petit nombre de baleiniers anglais, et lorsqu’elles se produisent une contrainte règne en raison d’une sorte de timidité, car l’Anglais a une certaine réserve que le Yankee n’imagine pas chez d’autres que lui-même. En outre, les baleiniers anglais affichent parfois une forme de supériorité citadine sur les baleiniers américains, regardant le grand et maigre Nantuckais, et ses provincialismes indéfinissables, un peu comme un paysan de la mer. À quoi tient ce sentiment de supériorité éprouvé par les Anglais, il est difficile à dire, surtout si l’on considère que les Yankees tuent en un jour plus de baleines que les Anglais en dix ans. C’est là une innocente faiblesse des baleiniers britanniques dont le Nantuckais ne s’offusque guère, sans doute parce qu’il se reconnaît à lui-même quelques petites faiblesses.\par
Ainsi donc, parmi les navires qui se croisent, les baleiniers sont ceux qui ont le plus de raisons de se montrer sociables et ils le sont en effet. Tandis que les navires marchands se rencontrant en plein Atlantique poursuivront leur route sans faire le moindre signe de reconnaissance, se croisant en haute mer comme des dandies à Broadway, peut-être même critiquent-ils réciproquement leur gréement en passant. Quant aux navires de guerre, ils mettent en branle une telle série idiote de révérences, de courbettes, de plongeons de pavillons, que ces simagrées ne semblent dictées ni par une chaleureuse spontanéité ni par un amour fraternel. En ce qui concerne les négriers, ils sont tellement pressés qu’ils fuient au plus vite ; lorsque deux pirates croisent leurs os croisés, le premier interpelle l’autre : « Combien de crânes ? » de la même manière que les baleiniers demandent : « Combien de barils ? » Dès qu’ils ont reçu la réponse, ils filent aussitôt chacun de son côté, car ce sont d’infernaux bandits, qui n’aiment guère contempler leur vilenie incarnée en d’autres.\par
Mais regardez ce baleinier voguant à la bonne franquette, pieux, honnête, modeste, accueillant et sociable ! Que fait-il lorsque, par un temps à peu près convenable, il rencontre un autre baleinier ? Ils font une « gamme », chose si parfaitement inconnue aux autres navires qu’ils ignorent jusqu’à ce terme et si, par hasard, ils l’entendent prononcer, ils ricanent et font des gammes de plaisanteries à ce sujet en parlant de « souffleurs », de « bouilleurs de lard » et autres aménités du même genre. Pourquoi les hommes des navires marchands et des navires de guerre, aussi bien que les pirates et les négriers \hspace{1em}nourrissent-ils un tel mépris pour les baleiniers ? Il est difficile d’y répondre. Je voudrais bien savoir quelle gloire il peut y avoir à faire le métier de pirate qui mène, certes, à une élévation peu ordinaire, celle du gibet. Or un homme parvenu à cette singulière et haute situation n’a pas grande base sur quoi la reposer. J’en conclus que lorsqu’un pirate se rengorge devant un baleinier, c’est sans fondement.\par
Mais qu’est-ce qu’une gamme ? Vous pouvez bien vous user l’index à suivre les colonnes des dictionnaires sans trouver ce mot. Le Dr Johnson n’a jamais atteint pareille érudition et l’arche de Noé du Webster ne le contient pas. Et pourtant, ce terme si évocateur est d’un usage courant depuis bien des années parmi quelque quinze mille Yankees de bonne souche. Il réclame une définition, certes, et devrait figurer dans tout lexique. Aussi permettez-moi de le définir doctement.\par
GAMME. Substantif. – {\itshape Réunion amicale de deux (ou plusieurs) navires baleiniers, en général sur un lieu de croisière. Après avoir échangé des salutations, les hommes se rendent visite en pirogues, les deux capitaines passant un moment ensemble sur le même navire, tandis que les seconds se réunissent sur l’autre.}\par
Un petit détail au sujet de cette réunion ne doit pas être omis ici. Toutes les professions ont des visages propres dans les détails de leurs activités, la pêche à la baleine a le sien. Quand on conduit à rames quelque part le capitaine d’un pirate, d’un vaisseau de guerre ou d’un négrier, celui-ci s’assoit toujours confortablement à la poupe de l’embarcation sur un siège parfois capitonné, et souvent il tient lui-même une ravissante petite barre de modiste enjolivée de cordons et de rubans. Mais une baleinière n’a pas de siège à l’arrière, ni de canapé d’aucune sorte, et point de barre du tout. Il ferait beau voir en vérité qu’on véhicule en mer les capitaines baleiniers sur des fauteuils à roulettes comme de vieux magistrats goutteux. En ce qui concerne la barre, jamais une baleinière n’a eu d’engin d’une telle mollesse, aussi doit-elle quitter le navire avec son équipage au complet, timonier ou harponneur compris, car le harponneur est aussi timonier en pareille occasion, et le capitaine invité n’ayant pas d’endroit où s’asseoir, est entraîné vers l’objectif de sa visite tout debout, comme le fût d’un pin. Vous remarquerez souvent que, se sentant la cible de tous les regards des équipages des deux navires, ce capitaine debout est tout entier préoccupé de conserver sa dignité en maintenant son équilibre. Et cela n’est pas une mince affaire, car derrière lui l’immense aviron de queue vient de temps en temps lui cogner le creux des reins, tandis que l’aviron de baille le frappe aux genoux. Il se trouve ainsi complètement coincé, par devant et par derrière, et ne peut se mouvoir que sur les côtés en se rétablissant sur ses jambes écartées, mais un tangage soudain et violent le fera aisément chanceler car une base ne saurait être ferme si elle n’a pas d’appuis sur une largeur donnée. Essayez donc de faire tenir debout deux perches en opposant leurs extrémités supérieures. De plus, sous le feu de tant de regards, il ne serait pas de mise que ce capitaine aux jambes écarquillées soit surpris en train d’empoigner la moindre des choses pour se tenir d’aplomb ; en vérité, témoignant ainsi de son allègre maîtrise de soi, il met en général les mains dans les poches de son pantalon, peut-être que ses mains, très larges et très lourdes lui servent de lest. Néanmoins, il y a eu des cas dûment attestés, où l’on a vu le capitaine, en un moment critique, un grain par exemple, saisir à pleines mains les cheveux du rameur le plus proche et s’y accrocher comme la mort.
\chapterclose


\chapteropen
\chapter[{CHAPITRE LIV. L’histoire du Town-Howw (telle qu’elle fut racontée à l’Auberge Dorée)}]{CHAPITRE LIV \\
L’histoire du {\itshape Town-Howw} \\
(telle qu’elle fut racontée à l’Auberge Dorée)}\renewcommand{\leftmark}{CHAPITRE LIV \\
L’histoire du {\itshape Town-Howw} \\
(telle qu’elle fut racontée à l’Auberge Dorée)}


\chaptercont
\noindent Le cap de Bonne-Espérance et toute l’étendue liquide qui l’avoisine ressemble à un de ces célèbres carrefours de grandes routes où l’on rencontre plus de voyageurs que partout ailleurs.\par
Peu de temps après que nous eûmes parlé au Diomède, nous rencontrâmes un autre baleinier sur le chemin du retour, le {\itshape Town-Ho}\footnote{Cri que poussaient autrefois les hommes en vigie des baleiniers lorsqu’ils apercevaient un cétacé. Il est encore en usage chez les baleiniers qui chassent la fameuse terrapène des Galapagos.}. Son équipage était presque exclusivement composé de Polynésiens. Dans la courte gamme qui suivit, il nous donna de sérieuses nouvelles de Moby Dick. Pour certains, l’intérêt porté à la Baleine blanche s’accrut à la narration d’un fait qui semblait obscurément impliquer que la baleine était le surprenant instrument de ce qu’on appelle la colère de Dieu et dont on dit qu’elle s’abat parfois sur certains hommes. Ce fait et les circonstances particulières qui l’entourent, qu’on pourrait appeler le noyau occulte de la tragédie, et que nous allons relater, ne parvint jamais aux oreilles du capitaine Achab et de ses seconds, car le capitaine du {\itshape Town-Ho} n’en sut rien lui-même. Trois matelots blancs de ce navire en étaient dépositaires et l’un d’eux, semble-t-il, le transmit à Tashtego sous serment solennel de n’en rien répéter, mais la nuit suivante Tashtego divagua à haute voix et son rêve en révéla tellement qu’il lui fut difficile ensuite de dissimuler le reste. Néanmoins, ce secret prit un tel ascendant sur les hommes du {\itshape Péquod} qui en eurent pleine connaissance que, par une étrange délicatesse, pour lui donner ce nom, ils le gardèrent si bien que rien n’en transpira à l’arrière du grand mât. Je m’en vais narrer cette étrange histoire, en tissant à sa juste place ce fil noir dans la trame du récit tel qu’il courait à bord.\par
Mes propres dispositions m’invitent à la relater dans les termes mêmes par lesquels j’en fis part, à Lima, à des amis espagnols à moi avec lesquels je m’attardais, à la veille d’une fête, tandis que nous fumions sur la terrasse aux pavés vermeils de l’Auberge Dorée. Parmi ces galants gentilshommes, les jeunes don Pedro et don Sebastian étaient mes intimes ; c’est pourquoi ils m’interrompaient de questions auxquelles je répondis alors, comme de juste.\par
Quelque deux ans avant que j’apprenne les événements que je vais vous conter, messieurs, le {\itshape Town-Ho}, chasseur de cachalots de Nantucket, était en croisière dans votre Pacifique, à quelques jours de voile vers l’ouest de cette bonne Auberge Dorée, un peu au nord de la ligne. Un matin, tandis qu’on pompait, comme il était d’usage quotidien, on remarqua qu’il y avait plus d’eau que de coutume dans la cale. On crut, messieurs, qu’un espadon avait transpercé la quille. Mais le capitaine, ayant une raison insolite pour croire qu’une chose exceptionnelle l’attendait dans ces parages, répugnait à les quitter ; d’autre part, la voie d’eau n’était pas considérée comme dangereuse, bien qu’en vérité les hommes n’aient pu la découvrir si bas qu’ils l’eussent cherchée dans les petits fonds pour autant que la grosse mer le permettait. Le navire poursuivit donc sa croisière, les matelots se relayant aisément aux pompes à de longs intervalles. Mais la chance ne vint pas, les jours passèrent et non seulement on ne trouva pas la voie d’eau mais elle s’aggrava sensiblement. De sorte que le capitaine, maintenant inquiet, mit le cap, toute toile dessus, sur l’île la plus proche offrant un port où il pourrait mettre en cale sèche et radouber. Bien qu’il eût de la route à faire, avec un peu de chance, il ne craignait aucunement d’aller par le fond, ses pompes étaient des meilleures et les trente-six hommes de son équipage s’y relevant régulièrement pouvaient sans peine soulager le navire, même si la voie d’eau doublait. Et en vérité, les vents ne cessant pas d’être favorables, le {\itshape Town-Ho} aurait eu toutes les chances d’atteindre en toute sécurité un port, n’eût été un incident fatal, provoqué par l’arrogance brutale de Radney, de Vineyard, le second, qui suscita la vengeance amèrement attisée de Steelkilt, un homme des Lacs, une tête brûlée, venant de Buffalo.\par
– Un homme des Lacs ?… Buffalo ! Je vous en prie, qu’estce qu’un homme des Lacs et où se trouve Buffalo ? demanda don Sebastian, se soulevant dans le siège de rabane où il se balançait.\par
– Sur la rive est de notre lac Érié, señor, mais – je sollicite votre patience – car nous en reparlerons bientôt. Or, messieurs, sur les bricks à voiles carrées et sur les trois-mâts, les plus grands et les plus robustes qui aient fait voile de votre Callao vers les Philippines, personne plus que cet homme, venu d’un lac au cœur même de notre terre d’Amérique, n’avait reçu de ces impressions aventureuses que l’imagination populaire ne prête qu’à la mer libre. Car ces grandes mers d’eau douce des lacs Supérieur, Érié, Ontario, Huron, Michigan qui communiquent entre eux, ont une étendue océanique, de nombreuses caractéristiques de noblesse propres à l’Océan, et leurs rivages comportent la même diversité de races et de climats.\par
Ils ont, comme les eaux polynésiennes, de ronds archipels d’îles romantiques ; comme l’Atlantique, ils séparent deux grandes nations fort dissemblables ; ils sont une lointaine voie d’accès à la mer pour les nombreuses colonies de l’Est, établies un peu partout sur leurs rives ; ici ou là des forts les menacent, comme les canons juchés, pareils à des chèvres dans les escarpements de Mackinac ; ils ont entendu le grondement de victoires navales, ils cédèrent parfois leurs plages à des sauvages barbares dont les visages peints en rouge flamboyaient hors de leurs wigwams en peaux de bêtes ; sur des lieues et des lieues, ils sont flanqués d’antiques forêts inviolées où des pins décharnés s’alignent comme les rois d’une généalogie gothique, où habitent de sauvages bêtes de proie et des créatures soyeuses dont les peaux partaient vêtir les empereurs tartares. Leurs eaux mirent aussi bien les capitales pavées de Cleveland et de Buffalo que les villages des Winnebagos, ils sont également sillonnés par le navire marchand gréé en trois-mâts carrés, par le croiseur de la marine de guerre, par le steamer et le canoë ; ils sont balayés par des coups de vent du nord aussi funestes et destructeurs que ceux qui flagellent la mer, les naufrages ne leur sont pas étrangers car, hors de vue des terres qui les enferment, ils ont englouti la nuit bien des navires et leurs équipages hurlants.\par
C’est pourquoi, messieurs, bien que terrien, Steelkilt était né homme de l’Océan et fut nourri de la sauvagerie de la mer, marin audacieux s’il en fut. Quant à Radney, même si la mer l’avait nourri de son sein sur la grève désolée de Nantucket, même si ensuite il avait longuement parcouru notre austère Atlantique et votre méditatif Pacifique, il était tout aussi vindicatif et chicanier qu’un coureur des bois frais émoulu des régions où l’on manie le couteau de chasse en corne de cerf. Ce Nantuckais toutefois n’était pas sans bonté et cet homme des Lacs, une espèce de démon, à vrai dire, s’était longtemps montré inoffensif et docile parce qu’il avait été traité avec une inflexible fermeté tempérée par la simple politesse voulant qu’on reconnaisse dans l’autre un être humain, ce qui est le droit du plus humble esclave. Quoi qu’il en soit, il s’était montré jusqu’ici pacifique, mais Radney fut exaspéré et pris de folie, et Steelkilt… mais procédons par ordre :\par
Il n’y avait guère plus d’un jour ou deux que le {\itshape Town-Ho} avait mis le cap sur un port insulaire que sa voie d’eau parut s’accroître, ne requérant toutefois qu’un peu plus d’une heure de pompage par jour. Je dois vous dire, messieurs, que sur un Océan civilisé comme notre Atlantique, certains capitaines ne songent guère à pomper en cours de traversée, bien que, par une nuit calme et indolente, si l’officier de quart en venait à oublier son devoir dans ce domaine, il y ait de fortes chances que ni lui ni ses camarades de bord n’en gardassent le souvenir car ils seraient gentiment allés par le fond. Et même dans les mers solitaires et sauvages qui s’étendent loin à l’ouest de chez vous, messieurs, il est rare qu’un équipage se cramponne en chœur aux poignées des pompes, même lors d’un long voyage, pour autant que le navire ne soit pas trop loin d’une côte suffisamment accessible ou qu’il puisse trouver un refuge quelconque. C’est seulement lorsqu’il est en haute mer, sous des latitudes ne lui offrant aucune terre proche, que son capitaine commence à se sentir un peu soucieux.\par
Il en était allé à peu près ainsi avec le {\itshape Town-Ho}, de sorte que lorsqu’il commença à embarquer davantage, une inquiétude se répandit parmi quelques membres de l’équipage dont plus particulièrement Radney, le second. Il ordonna de hisser les voiles hautes, de les border à joindre, afin d’offrir le maximum de toile au vent. Je pense que ce Radney était aussi peu couard que possible, aussi peu enclin à craindre nerveusement pour sa personne que la créature la plus téméraire, la plus privée de discernement qu’il vous plaira d’imaginer, messieurs, tant sur mer que sur terre. De sorte que lorsqu’il manifesta une telle sollicitude pour la sécurité du navire, certains des matelots déclarèrent qu’elle provenait de ce qu’il en était en partie propriétaire. Aussi, tandis qu’ils travaillaient aux pompes ce soir-là, l’avaientils sournoisement pris comme tête de Turc, et s’en donnaient-ils à cœur joie, cependant que l’eau claire clapotait à leurs chevilles. Oui, messieurs, une eau claire comme une source de montagne bouillonnait au sortir des pompes, courait sur le pont et se déversait à jets réguliers par les dalots sous le vent.\par
Or, comme vous le savez, il n’est pas rare dans ce monde qui est le nôtre, qu’il soit liquide ou solide, que lorsqu’un homme, assumant le commandement sur ses semblables, découvre que l’un d’eux lui est supérieur en dignité humaine, il nourrisse aussitôt contre lui une aversion et une rancœur insurmontables ; et si la chance lui en est donnée, il abattra la forteresse de ce subalterne, la pulvérisera et la réduira à un petit tas de poussière. C’est peut-être une idée que je me fais, messieurs, toujours est-il que Steelkilt était un grand et noble animal, avec une tête de Romain et une barbe blonde qui flottait comme le harnachement à aigrettes du destrier de votre dernier vice-roi ; et il avait un cerveau, un cœur et une âme, messieurs, qui eût fait de Steelkilt Charlemagne s’il était né fils de Charlemagne. Cependant que Radney, le second, était laid comme une mule et mêmement résistant, entêté et mauvais. Il n’aimait pas Steelkilt et Steelkilt le savait.\par
Apercevant le second qui s’approchait tandis qu’il travaillait aux pompes avec les autres, l’homme des Lacs feignit de ne l’avoir point remarqué et poursuivit ses railleries irrespectueuses.\par
– Oui, oui, mes joyeux compères, voilà une vigoureuse voie d’eau, que l’un de vous prenne un bidon et qu’on y goûte. Seigneur, elle vaut d’être mise en bouteilles ! Je vais vous dire, les gars, les actions du vieux Rad vont en prendre un coup ! Il aurait mieux fait de couper sa part de quille et de la remorquer jusque chez lui. Le fait est que cet espadon a seulement ébauché le travail et qu’il est revenu avec une escorte de poissonscharpentiers, de poissons-scies, de poissons-limes et toute l’équipe. Elle est maintenant au travail au complet, elle coupe et taillade, à qui mieux mieux, par là-dessous, en vue d’améliorations, je présume. Si le vieux Rad était ici en ce moment, je lui conseillerais de se jeter à la mer et de les disperser. Ils sont en train d’envoyer au diable tout son bien, je le lui dirai aussi. Mais c’est une âme simple, Rad… et une beauté pardessus le marché. On dit, les gars, qu’il a placé le reste de ses actions sur des miroirs. Je me demande s’il donnerait à un pauvre diable comme moi le modèle de son nez.\par
– Que le diable vous emporte ! Pourquoi cette pompe s’arrête-t-elle ? rugit Radney, faisant comme s’il n’avait pas entendu le bavardage des matelots. Allons, que ça barde !\par
– Oui, oui, sir, répondit Steelkilt, joyeux comme un grillon. De l’entrain, les gars, de l’entrain ! Et là-dessus, les pompes retentirent comme cinquante pompes à incendie, les hommes repoussèrent leurs chapeaux et l’on entendit ce long halètement des poumons qui trahit la plus grande tension de l’énergie vitale.\par
Quittant les pompes enfin avec le reste des hommes, l’homme des Lacs s’avança tout essoufflé jusqu’au guindeau et s’y assit, cramoisi, les yeux injectés de sang et il essuya l’abondante sueur de son front. Quel insidieux démon poussa alors Rad à chercher noise à un homme dans un pareil état d’excitation physique, messieurs, je ne saurais le dire. Mais il en fut ainsi. Arpentant le pont d’une manière insupportable, le second lui intima l’ordre de prendre un balai, de nettoyer les planches et d’aller chercher une pelle pour ramasser les immondices d’un cochon qui avait été laissé en liberté.\par
Or, messieurs, balayer les ponts en mer est une besogne ménagère qui est faite tous les soirs, sauf en cas de tempêtes violentes. Il est des exemples où ils furent dûment balayés pendant que le navire sombrait. Telle est, messieurs, la rigueur des usages à la mer et l’amour instinctif que les marins portent à la propreté est tel que certains ne se noieraient pas volontiers sans s’être lavé la figure. Mais sur tous les navires cette histoire de balais est le partage des mousses, si mousses il y a. D’autre part, ce furent les hommes les plus robustes du {\itshape Town-Ho} qui avaient été répartis en équipes pour les pompes, et Steelkilt étant le plus athlétique d’entre eux, avait régulièrement été nommé chef d’une de ces équipes, aussi aurait-il dû être libéré de tout travail subalterne qui ne relevât pas directement du devoir nautique, ce qui était le cas pour ses camarades. Si je fais état de ces détails, c’est afin que vous compreniez mieux de quoi il en retournait entre ces deux hommes.\par
Qui plus est, l’ordre donné au sujet de la pelle visait aussi ouvertement à piquer au vif Steelkilt et à lui faire affront que si Radney lui avait craché au visage. Tout homme qui a été à bord d’un baleinier comprendra cela, et bien davantage encore. L’homme des Lacs en saisit toute la portée dès que le second eut formulé son ordre. Mais il resta tranquillement assis et regarda sans broncher au fond des yeux mauvais du second ; il y voyait une mèche brûler silencieusement en s’approchant des barils de poudre qui y étaient amoncelés. Comme il voyait instinctivement tout cela, une étrange indulgence l’envahit, une répugnance à attiser plus profondément la colère d’un être irrité ; lorsqu’elle est ressentie, cette répugnance révèle des hommes vraiment courageux même lorsqu’ils sont blessés et ce sentiment indéfinissable, messieurs, s’empara de Steelkilt.\par
C’est pourquoi, de sa voix habituelle, un peu brisée par la fatigue, il lui répondit que balayer le pont n’était pas son affaire et qu’il ne le ferait pas. Puis, sans aucune allusion à la pelle, il nomma les trois garçons dont c’était la charge coutumière et qui, n’ayant pas été affectés aux pompes, n’avaient pas fait grand-chose de toute la journée. Radney répliqua par un juron et réitéra son ordre de la manière la plus catégorique, la plus impérieuse et la plus insultante, tout en s’approchant de l’homme des Lacs resté assis. Il tenait à la main un cochoir de tonnelier qu’il avait empoigné sur un baril voisin.\par
Échauffé et énervé comme il l’était par son travail athlétique aux pompes, malgré le sentiment mystérieux \hspace{1em}d’indulgence qu’il avait tout d’abord éprouvé, encore en nage, Steelkilt supportait difficilement cette attitude du second ; pourtant il parvint à maîtriser en lui le feu de la colère et sans mot dire resta obstinément vissé à son siège jusqu’à ce qu’enfin Radney exaspéré brandit le cochoir à deux doigts de son visage, en lui intimant avec fureur d’obéir.\par
Steelkilt se leva, se retira lentement derrière le guindeau, talonné par le second armé du cochoir, et exprima délibérément son refus d’obéir. Comprenant toutefois que son indulgence n’avait pas le moindre effet, il fit un geste atroce et innommable de la main signifiant à cet homme insensé et dément de se retirer. Cela ne servit de rien. Et c’est ainsi que les deux hommes firent lentement le tour du guindeau quand enfin, résolu à l’affronter, trouvant qu’il avait patienté jusqu’aux limites de son tempérament, l’homme des Lacs s’immobilisa sur le panneau d’écoutille et s’adressa ainsi à l’officier :\par
– Monsieur Radney, je ne vous obéirai pas. Enlevez ce cochoir de là, ou gare à vous !\par
Mais prédestiné, le second se rapprocha encore de l’homme des Lacs figé et il secouait à présent le lourd cochoir à un doigt de ses dents en proférant un chapelet d’intolérables malédictions. Sans reculer d’un millième de pouce, le poignardant d’un regard impassible, fermant le poing droit et le ramenant doucement en arrière, Steelkilt avertit son persécuteur que si le cochoir effleurait seulement sa joue, lui, Steelkilt le tuerait. Mais, messieurs, ce fou était marqué au sceau du massacre par les dieux. Aussitôt le cochoir effleura la joue et à l’instant, la mâchoire défoncée, il s’effondra sur l’écoutille, crachant le sang comme une baleine.\par
Avant que l’alarme pût être donnée à l’arrière, Steelkilt secoua l’un des galhaubans qui montait jusqu’au poste de vigie où deux de ses camarades étaient appointés. Tous deux étaient des mariniers.\par
– Des mariniers ! s’écria don Pedro. Nous avons vu bien des navires baleiniers dans nos ports, mais nous n’avons jamais entendu parler de vos mariniers. Pardon… mais que sont-ils et qui sont-ils ?\par
– Señor, nos mariniers sont les bateliers de notre grand canal de l’Érié. Vous avez dû en entendre parler…\par
– Non, señor, ici, dans notre vieux pays engourdi, chaud et des plus paresseux, nous n’en savons guère sur votre Nord énergique.\par
– Oui ? Alors, señor, remplissez mon verre, votre chicha est excellente, je vais donc vous dire qui sont nos mariniers car cela éclairera la suite de mon histoire.\par
Sur trois cent soixante milles, messieurs, traversant dans toute sa largeur l’État de New York, de nombreuses cités populeuses et de très prospères villages, d’interminables et lugubres marécages, des champs cultivés, riches et d’une fertilité sans égale, passant à côté des salles de billard et des bars, traversant le saint des saints des vastes forêts, passant sur les arches romanes des ponts jetés sur les rivières indiennes, traversant l’ombre et le soleil, passant à côté des cœurs heureux et des cœurs brisés, parcourant l’immense région diverse et contrastée des nobles Mohawks, longeant particulièrement les rangées de chapelles blanches comme neige, dont les clochers se dressent comme des bornes milliaires, et où coulent un flot incessant de corruption toute vénitienne et une vie souvent sans lois. Là sont les vrais Achantis, messieurs, là rugissent les païens, là vous les trouverez à votre porte, dans la grande ombre portée et l’abri protecteur des églises. Car une curieuse fatalité veut que les flibustiers de vos villes aient toujours leurs campements aux alentours des palais de justice, messieurs, de même que les pécheurs foisonnent autour des lieux saints. – N’est-ce pas un frère qui passe ? demanda don Pedro, simulant avec humour l’inquiétude en regardant la foule sur la place.\par
– Heureusement pour notre ami nordique, l’Inquisition de dame Isabelle est sur le déclin à Lima, dit don Sebastian en riant. Continuez, je vous prie.\par
– Un instant !… s’écria un autre membre de la compagnie. En notre nom à tous, gens de Lima, je tiens à vous exprimer, monsieur du marin, que nous sommes sensibles au fait que vous n’avez pas cité l’actuelle Lima mais bien plutôt l’ancienne Venise pour illustrer la corruption. N’ayez pas l’air si étonné, vous connaissez le proverbe qui court au long de cette côte : « Corrompue comme Lima », il en va de même de Venise, j’y suis allé, la ville bénie du saint évangéliste Marc. Que saint Dominique la purge ! Votre verre ! Merci, le voilà plein, videz-le !\par
Le marinier, messieurs, est si pervers et si pittoresque, que, si on le saisissait sur le vif, on en ferait un excellent héros de tragédie. Tel Antoine, il vogue nonchalamment sur son Nil verdoyant et fleuri, jour après jour, lutinant ouvertement sa Cléopâtre aux joues rouges dorant au soleil ses cuisses couleur d’abricot. Cette mollesse s’évanouit aussitôt à terre. Le marinier arbore fièrement une allure de brigand, son chapeau joyeusement enrubanné, rabattu, accuse la beauté de ses traits. Il est la terreur de l’innocence souriante des villages où il flâne, son teint basané, son air avantageux ne passent pas inaperçus dans les villes. Une fois, vagabond sur son propre canal, un de ces mariniers me rendit service, je l’en remercie de tout cœur, je ne voudrais pas être ingrat, car c’est souvent une des plus grandes qualités qui rachètent ces hommes violents que d’avoir la poigne pour tirer d’embarras un pauvre étranger comme pour piller un riche. Bref, messieurs, tout cela vous prouve la vie déréglée qu’ils mènent ; notre rude pêche à la baleine accueille un grand nombre de leurs échantillons les plus parfaits et nos capitaines baleiniers ne redoutent aucune race humaine autant qu’eux, hormis les hommes de Sydney. Le fait que, pour tant de nos jeunes campagnards nés sur ses rives, l’apprentissage de la vie du canal constitue la seule transition entre la moisson paisible d’un champ de blé chrétien et le labourage téméraire des plus barbares océans, ne diminue en rien son étrangeté.\par
– Je vois ! je vois ! s’écria impétueusement don Pedro en renversant sa chicha sur son plastron. Inutile de voyager ! Le monde n’est qu’une Lima. J’aurais cru pourtant que dans votre Nord tempéré les générations se succédaient, froides et saintes comme les collines… Mais poursuivez…\par
– J’en étais resté, messieurs, au moment où l’homme des Lacs secouait le galhauban. À peine avait-il fait ce geste qu’il fut cerné et assiégé par les trois seconds et quatre harponneurs. Mais, glissant le long des haubans comme des comètes de malheur, les deux mariniers se ruèrent dans la mêlée et tentèrent de tirer leur homme sur le gaillard d’avant. D’autres matelots se joignirent à eux dans cet effort et une bagarre générale s’ensuivit, cependant que le courageux capitaine, se tenant à l’écart du danger, dansait d’un pied sur l’autre, une pique à la main. Il ordonna à ses officiers de maîtriser cet affreux scélérat et de le traîner jusqu’au gaillard d’arrière ; de temps à autre, il s’approchait du pourtour de ce cercle en effervescence et, cherchant son centre de la pointe de sa lance, il essayait d’en atteindre l’objet de son ressentiment afin de l’en faire sortir. Mais les risque-tout qui entouraient Steelkilt étaient trop nombreux, ils réussirent à gagner le gaillard d’avant, dressèrent à l’alignement du guindeau trois ou quatre gros barils qui se trouvaient là et, Parisiens de la mer, se retranchèrent derrière leurs barricades.\par
– Sortez de là, pirates, hurla le capitaine, les menaçant à présent d’un pistolet dans chaque main, pistolets que venait de lui apporter le garçon. Sortez de là, égorgeurs !\par
Steelkilt bondit sur la barricade et y fit les cent pas, défiant les pistolets et laissant entendre clairement que sa mort à lui, Steelkilt, serait parmi l’équipage le signal d’une sanglante mutinerie. Ne craignant que trop que cela ne se révélât vrai, le capitaine fléchit légèrement, mais ordonna toutefois aux insurgés de rejoindre leurs postes.\par
– Promettez-vous de ne pas nous toucher, si nous le faisons ? demanda le meneur.\par
– À vos postes ! dis-je, je ne promets rien, à vos postes ! Voulez-vous faire sombrer le navire en l’abandonnant dans un moment aussi critique ? À vos postes ! et il leva à nouveau un pistolet.\par
– Faire sombrer le navire ? s’écria Steelkilt, mais oui qu’il sombre ! Pas un seul d’entre nous ne reprendra le travail si vous ne jurez pas de ne pas lever le fouet sur nous. Qu’en dites-vous, les gars ? demanda-t-il à ses camarades. Ils répondirent par un sauvage vivat.\par
L’homme des Lacs patrouillait maintenant sur la barricade, sans perdre de l’œil le capitaine et s’exprimant par phrases décousues :\par
– Ce n’est pas de notre faute,… nous ne l’avons pas cherché… je lui avais bien demandé de laisser ce maillet… c’était le travail du mousse… il aurait dû mieux me connaître… je l’avais averti de ne pas me pousser à bout… je crois que je me suis cassé un doigt sur sa maudite mâchoire… les couteaux à émincer ne sont-ils pas dans le gaillard d’avant, les gars ? Prenez garde à ces anspects, mes jolis cœurs. Capitaine, par Dieu, méfiezvous… Jurez, ne soyez pas idiot… passez l’éponge, nous sommes prêts à retourner à nos postes… traitez-nous décemment et nous sommes vos hommes… mais nous refusons le fouet.\par
– À vos postes ! je ne promets rien, je le répète !\par
– Écoutez-moi bien, vous, cria l’homme des Lacs en étendant le bras, nous sommes quelques-uns ici, dont je suis, qui nous sommes embarqués pour la croisière, vous savez… comme vous le savez, sir, nous pouvons réclamer notre débarquement dès que nous aurons mouillé l’ancre… aussi nous ne voulons pas la bagarre, ce n’est pas dans notre intérêt, nous voulons la paix et nous sommes prêts à travailler, mais nous ne voulons pas du fouet.\par
– À vos postes ! vociféra le capitaine. Steelkilt réfléchit un instant et ajouta :\par
– Je vais vous dire, capitaine, plutôt que de vous tuer et d’être pendu honteusement, nous ne lèverons pas la main sur vous à moins que vous ne nous attaquiez, mais tant que vous n’aurez rien promis, nous ne bougerons pas.\par
– Alors en bas, en bas au gaillard d’avant et je vous y garderai jusqu’à ce que vous en ayez assez. Allons, en bas !\par
– Irons-nous ? demanda le meneur à ses hommes. La plupart n’étaient pas d’accord mais enfin, se soumettant à Steelkilt, ils le précédèrent dans l’antre obscur, disparaissant en grondant comme des ours dans une caverne.\par
À peine la tête de l’homme des Lacs se trouva-t-elle au niveau des planches, que le capitaine et les siens sautèrent pardessus la barricade, tirèrent promptement le panneau d’écoutille, s’y appuyèrent de toutes leurs forces, et hurlèrent au garçon d’apporter le lourd cadenas de cuivre du couloir. Alors le capitaine souleva légèrement le panneau, murmura quelque chose dans la fente, puis tourna la clef sur les hommes – au nombre de dix – laissant libres quelque vingt d’entre eux qui s’étaient montrés neutres jusqu’alors.\par
Toute la nuit tous les officiers montèrent un quart d’alerte à l’avant et à l’arrière et plus particulièrement près du panneau et des écoutilles du gaillard d’avant, de crainte que les insurgés, ayant brisé les cloisons d’entrepont, ne surgissent par là. Mais les heures nocturnes s’écoulèrent dans la paix, les hommes restés au travail eurent une rude tâche aux pompes dont le cliquetis rythmé résonnait lugubrement dans la morne nuit du navire.\par
Au lever du soleil, le capitaine se rendit à l’avant et, frappant sur le pont, somma les prisonniers de reprendre leur travail. Un hurlement de refus lui répondit. On leur fit descendre de l’eau et deux poignées de biscuits, puis le capitaine retourna sur le gaillard d’arrière après les avoir verrouillés à nouveau et empoché la clef. Cette cérémonie eut lieu deux fois par jour pendant trois jours, mais au matin du quatrième jour, lorsque l’injonction coutumière eut été prononcée, elle fut suivie d’un bruit confus de dispute et de bousculade, et soudain quatre hommes firent irruption sur le pont en se déclarant prêts à reprendre leurs postes. La puanteur d’un air raréfié, la faim, et peut-être la crainte d’un châtiment final les avaient poussés à se rendre. Enhardi par ce succès, le capitaine réitéra aux autres sa demande, mais Steelkilt lui intima, de façon terrifiante, d’en finir avec son bla-bla et de retourner d’où il venait. Au matin du cinquième jour, trois autres mutins s’arrachèrent aux bras désespérés qui cherchaient à les retenir et surgirent à l’air libre. Ils n’étaient plus que trois désormais.\par
– Vous ne préféreriez pas retourner à vos postes maintenant ? demanda le capitaine avec une ironie mauvaise.\par
– Bouclez-nous en vitesse, voulez-vous !\par
– Très volontiers, dit le capitaine et la clef tourna. C’est alors, messieurs, qu’exaspéré par la défection de sept de ses anciens partisans, piqué au vif par la voix gouailleuse qu’il venait d’entendre, et rendu fou par cette longue sépulture dans un endroit aussi sombre que les entrailles du désespoir, c’est alors que Steelkilt proposa aux deux mariniers qu’il croyait jusqu’ici ne faire qu’un avec lui, de bondir de leur trou au prochain appel de la garnison, et, armés de leurs tranchants couteaux à émincer (une longue lame cintrée munie d’une poignée à chaque bout) de courir comme des fous du Beaupré jusqu’à la lisse de couronnement, et en dernière extrémité, de se rendre maîtres du navire. Quant à lui, qu’ils soient ou non des siens, c’est ce qu’il ferait, c’était la dernière nuit qu’il passerait dans cet antre. Ce projet ne rencontra aucune opposition de la part des deux autres, ils jurèrent être prêts à le suivre en cela ou en toute autre idée folle qui pourrait lui venir, prêts à tout en somme, sauf à se rendre. Qui plus est, chacun d’eux insista pour être le premier à se précipiter sur le pont le moment venu. Contre quoi leur chef protesta avec véhémence, revendiquant ce privilège, d’autant plus que ni l’un ni l’autre de ses camarades ne voulant céder, ils ne pouvaient être le premier l’un et l’autre, l’échelle ne donnant passage qu’à un seul homme à la fois. Maintenant, messieurs, va se faire jour le vilain jeu de ces mécréants.\par
À l’énoncé du projet forcené de leur chef, la même traîtrise s’était allumée soudain dans l’âme de chacun d’eux, le projet d’être le premier en haut, premier des trois quoique le dernier des dix, pour se rendre et s’assurer de la sorte la petite chance de pardon qu’une telle conduite laissait subsister. Mais lorsque Steelkilt leur eut déclaré vouloir demeurer à leur tête jusqu’au bout, leur vilenie les poussa à user d’une subtile chimie et à mélanger leurs trahisons encore informulées ; lorsque le meneur s’endormit, ils s’épanchèrent l’un en l’autre en trois phrases, ligotèrent le dormeur avec des cordages, le bâillonnèrent et, au milieu de la nuit, se mirent à appeler en hurlant le capitaine.\par
Croyant au meurtre, flairant le sang dans les ténèbres, il se précipita avec ses seconds armés et ses harponneurs vers le gaillard d’avant. En quelques minutes le panneau fut ouvert et, lié de la tête aux pieds, le meneur qui se débattait toujours fut hissé sur le pont par ses perfides disciples qui, d’une seule voix, revendiquèrent l’honneur d’avoir réduit à l’impuissance un homme mûr pour le crime. Mais tous trois furent également empoignés, traînés sur le pont comme des pièces de boucherie et suspendus, côte à côte, attachés aux chaînes de haubans d’artimon, comme trois quartiers de viande, jusqu’au matin. Le capitaine passant et repassant devant eux leur cria :\par
– Soyez maudits ! les vautours eux-mêmes ne voudraient pas de vous, canailles !\par
Au lever du soleil, il rassembla l’équipage et sépara les mutins des autres, informant les premiers qu’il avait bien envie de les faire fouetter tous tant qu’ils étaient… que réflexion faite il le ferait peut-être… qu’il devrait le faire… que c’eût été justice… mais que pour le moment, vu leur reddition opportune, il se bornerait à leur adresser une semonce, qu’il administra aussitôt dans leur langue maternelle.\par
– Mais quant à vous, charognes de fourbes, dit-il en se tournant vers les trois hommes ficelés au gréement, quant à vous je me propose de vous émincer et de vous faire fondre dans les pots, et empoignant un cordage il en cingla de toutes ses forces le dos des deux traîtres jusqu’à ce qu’épuisés à force de crier, ils laissent aller leur tête sur leur épaule et restent inanimés, pareils à l’image des deux larrons en croix.\par
– Vous me valez un poignet foulé, mais il reste assez de corde pour vous qui ne vouliez pas vous rendre, mon beau coq de Bantam. Enlevez-lui ce bâillon et écoutons ce qu’il a à dire pour sa défense.\par
Pendant un moment, les mâchoires crispées du mutin à bout de forces se mirent à trembler puis, tournant douloureusement la tête, il dit dans un sifflement :\par
– Ce que j’ai à dire, souvenez-vous-en, c’est ce que si vous me fouettez, je vous tuerai.\par
– Vraiment ? alors voyez comme vous me faites peur… et le capitaine lança la corde en arrière pour frapper.\par
– Vaut mieux pas, siffla l’homme des Lacs.\par
– Mais je vais le faire, et de nouveau il s’y apprêta.\par
Alors, toujours dans un sifflement, entendu de personne hormis du capitaine, Steelkilt murmura quelque chose et, à l’étonnement de tout l’équipage, le capitaine fit un bond en arrière, arpenta fébrilement le pont à deux ou trois reprises puis, jetant soudain sa corde, il dit :\par
– Je ne le ferai pas… libérez-le… coupez ces liens, vous m’entendez ?\par
Mais tandis que les seconds se hâtaient dans le but d’exécuter cet ordre, un homme pâle, la tête bandée, s’interposa, c’était Radney le premier second. Il était resté couché depuis qu’il avait reçu le coup, mais ce matin-là, entendant ce remueménage sur le pont, il s’était glissé jusque-là et avait assisté à toute la scène. Il avait la mâchoire dans un tel état qu’il pouvait à peine parler, mais il marmotta quelque chose, proposant de mettre lui-même à exécution ce devant quoi le capitaine reculait, il s’empara de la corde et marcha sur son ennemi garrotté.\par
– Vous êtes un lâche ! siffla l’homme des Lacs.\par
– C’est juste, mais ramassez toujours ! Le second était sur le point de frapper, lorsqu’un nouveau sifflement immobilisa son bras. Il hésita, puis délibérément tint parole malgré la menace de Steelkilt, quelle qu’elle ait pu être.\par
Les trois marins furent relâchés, les matelots renvoyés à leurs postes et les pompes manœuvrées par des hommes maussades et ombrageux retentirent comme auparavant.\par
Le même jour, juste après la tombée de la nuit, tandis qu’une bordée venait de quitter le quart, une clameur s’éleva dans le poste d’équipage et les deux traîtres tout tremblants vinrent en courant assiéger la porte du capitaine en déclarant qu’ils n’osaient pas rester avec l’équipage. Prières, taloches et coups de pieds ne parvinrent pas à les déloger de sorte que, selon leur propre suggestion, ils descendirent chercher leur salut dans la coulée-arrière. Pourtant, aucun symptôme de mutinerie ne se manifesta chez les autres. Il semblait au contraire qu’à l’instigation de Steelkilt ils soient résolus à maintenir la paix la plus parfaite, à obéir jusqu’à la fin et à déserter d’un seul corps dès que le navire aurait touché un port. Mais dans le but d’accélérer le voyage, ils s’entendirent tous sur un autre point : à savoir ne pas donner de la voix au cas où des baleines seraient vues. Car, malgré sa voie d’eau, et tous les autres dangers qu’il courait, le {\itshape Town-Ho} gardait ses guetteurs aux postes de vigie et son capitaine était aussi disposé à mettre les pirogues à la mer à présent qu’au moment où il était arrivé sur les lieux de croisière. Radney, le second, était également prêt à troquer son hamac contre la baleinière et, le pansement sur la bouche, à chercher à bâillonner à mort la vivante mâchoire de la baleine.\par
Mais bien que l’homme des Lacs eût encouragé les matelots à adopter une attitude passive, il tenait un petit conseil privé (du moins jusqu’à ce que tout prît fin) avec lui-même au sujet de la vengeance personnelle qu’il allait exercer sur l’homme qui l’avait blessé en plein cœur. Il faisait partie de la bordée de Radney ; or celui-ci, malgré les conseils du capitaine, avait insisté pour reprendre son quart de nuit, après la scène du fouet, comme si, ensorcelé, il voulait parcourir plus de la moitié du chemin pour rencontrer son destin. Sur ce, et à cause d’une ou deux autres circonstances, Steelkilt échafauda méthodiquement son plan de vengeance.\par
Au cours de la nuit, Radney avait une manie contraire à tout sens marin. S’asseyant sur les pavois du gaillard d’arrière, il posait un bras sur le plat-bord d’une pirogue suspendue là, en dehors de la lisse. On savait qu’il lui arrivait de s’endormir dans cette position. L’espace séparant le navire de la baleinière était considérable et en dessous il y avait la mer. Steelkilt examina son horaire et nota que son tour de barre se situerait à deux heures du matin le troisième jour après qu’il eût été trahi. En attendant, il employa ses loisirs à tresser quelque chose avec un soin extrême pendant ses quarts d’en bas.\par
– Qu’est-ce que vous fabriquez là ? lui demanda un camarade.\par
– Que croyez-vous que ce soit ? À quoi cela ressemble-t-il ?\par
– À une aiguillette pour votre sac, pourtant elle me paraît un peu curieuse.\par
– Oui, plutôt curieuse, dit l’homme des Lacs en la tenant à bout de bras devant lui, mais je crois qu’elle fera l’affaire. Camarade, je n’ai pas assez de fil à voile, en auriez-vous ?\par
– Il n’y en a pas sur le gaillard d’avant.\par
– Alors il me faut aller en demander à ce vieux Rad, et il se leva pour se diriger vers le gaillard d’arrière.\par
– Vous n’avez quand même pas l’intention d’aller lui mendier quelque chose à lui ! s’écria le matelot.\par
– Pourquoi pas ? Croyez-vous qu’il va me refuser un service quand c’est pour lui rendre service à lui, en fin de compte, camarade ? et allant vers le second il le regarda tranquillement et lui demanda du fil à voile pour réparer son hamac. Il lui fut donné mais on ne devait plus revoir ni le fil ni l’aiguillette. La nuit suivante, une boule de fer, enfermée dans une résille tressée, faillit rouler hors de la poche de la vareuse de l’homme des Lacs, tandis qu’il l’arrangeait, en guise d’oreiller, dans son hamac. Vingt-quatre heures plus tard, son tour de barre était venu, l’heure fatale et silencieuse approchait où il se trouverait près de l’homme qui, parfois, s’endormait au-dessus de la tombe toujours ouverte pour le marin. Et dans l’esprit de Steelkilt, le second était déjà étendu roide, le crâne en éclats.\par
Toutefois, messieurs, un imbécile épargna au meurtrier en puissance de commettre l’acte sanglant qu’il avait prémédité. Pourtant il eut entière vengeance, sans être le vengeur car, par une mystérieuse fatalité, le Ciel lui-même parut intervenir pour lui prendre des mains l’acte qui l’eût conduit à la damnation, et l’assumer lui-même.\par
Entre l’aube et le lever du soleil, au matin du second jour, tandis que les hommes lavaient les ponts, un idiot venu de Ténériffe, tandis qu’il amenait de l’eau sous les porte-haubans, s’écria soudain :\par
– La voilà qui roule ! la voilà qui roule ! Jésus, quelle baleine !\par
C’était Moby Dick.\par
– Moby Dick ! s’écria don Sebastian. Par saint Dominique, monsieur du marin, les baleines ont-elles des noms de baptême ? Qui donc appelez-vous Moby Dick ?\par
– Un monstre très blanc, très célèbre, le plus implacable des monstres immortels, señor… mais ceci serait une trop longue histoire…\par
– Racontez-la, racontez-la, s’exclamèrent tous les jeunes Espagnols en se rapprochant.\par
– Non, señores, non, non ! je ne puis entrer dans un récit aussi détaillé maintenant. Laissez-moi respirer.\par
– La chicha ! la chicha ! s’écria don Pedro, notre énergique ami a l’air de se trouver mal, qu’on emplisse son verre !\par
– Ce n’est pas nécessaire, messieurs, un instant et je poursuis… Donc, messieurs, apercevant si brutalement, la Baleine blanche à cinquante mètres à peine du navire oublieux du pacte liant les membres de l’équipage, l’homme de Ténériffe avait instinctivement et involontairement poussé ce cri, alors que le monstre avait bel et bien été vu par les hommes renfrognés des trois têtes de mâts. Une agitation frénétique régna instantanément.\par
La Baleine blanche, la Baleine blanche ! s’écriaient à qui mieux mieux le capitaine, les seconds et les harponneurs, nullement ébranlés par les rumeurs inquiétantes qui couraient à son sujet et également anxieux de prendre un poisson si célèbre et si précieux, tandis que l’équipage maussade regardait de travers, et en la maudissant, la repoussante beauté de cette grande forme laiteuse qui, illuminée par les paillettes d’un soleil horizontal, se mouvait en scintillant, vivante opale dans le matin bleu de la mer. Messieurs, les événements se déroulèrent \hspace{1em}selon une étrange fatalité, comme s’ils avaient été tracés sur la carte du monde avant même la naissance de la terre. Le mutin était premier rameur à bâbord dans la baleinière du second, il était donc assis à ses côtés lorsqu’il se trouvait à l’avant avec sa lance et devait haler ou filer la ligne au commandement. De plus, lorsque les quatre pirogues furent mises à la mer, celle du second se trouva en tête, et personne n’eut un hurlement de joie aussi féroce que Steelkilt tandis qu’il peinait à son aviron. Après avoir nagé vigoureusement, leur harponneur piqua et, la lance à la main, Radney bondit à l’avant. Il entrait toujours en fureur, semblait-il, dès qu’il était dans une baleinière. Et le cri qui traversait son bandage demandait qu’on l’amenât sur le dos même du cachalot. Sans se faire prier, son tireur l’amena toujours plus près, à travers une écume aveuglante qui mêlait deux blancheurs, jusqu’à ce que soudain la pirogue donnât comme sur un récif, penchât et projetât à la mer le second qui était debout. Il tomba aussitôt sur le dos glissant de la baleine, la pirogue se redressa et fut jetée de côté par le remous, tandis que Radney se débattait dans la houle sur l’autre flanc du monstre. Il émergea de l’écume et on l’aperçut un instant indistinctement à travers ce voile, luttant sauvagement pour se dérober au regard de Moby Dick. Mais la baleine fit brutalement volte-face et, soulevant un soudain tourbillon, saisit le nageur dans ses mâchoires, se dressa sur l’eau en le brandissant, piqua à nouveau droit devant elle et sonda.\par
Cependant, au premier heurt de la pirogue, l’homme des Lacs avait filé la ligne de façon à se trouver à l’arrière du maelström, il observa tranquillement la scène, considérant ses propres pensées. Mais une secousse brusque et terrifiante fit plonger du nez la pirogue ; tirant rapidement son couteau, il coupa la ligne, et la baleine fut libre. Mais à quelque distance de là on vit réapparaître Moby Dick, des lambeaux rouges de la chemise de Radney accrochés dans les dents qui l’avaient dévoré. Les quatre baleinières continuèrent à lui livrer la chasse, mais le cachalot les évita puis enfin disparut tout à fait.\par
En temps voulu, le {\itshape Town-Ho} atteignit son port, un endroit sauvage et solitaire que n’habitait aucune créature civilisée. Là, mené par l’homme des Lacs, tout l’équipage, hormis cinq ou six matelots, déserta parmi les palmiers. Par la suite, ils s’emparèrent d’une grande double pirogue de guerre des sauvages et mirent à la voile vers un autre port.\par
L’équipage du navire étant réduit à une poignée d’hommes, le capitaine fit appel aux insulaires pour l’aider à la besogne ardue d’abattre en carène afin de réparer la voie d’eau. Mais ces quelques Blancs durent, de nuit et de jour, monter une garde incessante dans la crainte où ils étaient de leurs dangereux alliés. Le travail qu’ils eurent à fournir était si rigoureux que lorsque le navire fut prêt à reprendre la mer, les hommes étaient trop épuisés pour que le capitaine osât s’embarquer avec eux dans un navire d’un pareil tonnage. Après avoir tenu conseil avec ses officiers, il ancra le navire aussi loin que possible de la côte, mit ses canons en batterie, les chargea, entassa ses mousquets à la poupe, avertit les insulaires qu’ils n’approcheraient qu’à leurs risques et périls puis, prenant un homme avec lui, il hissa la voile sur sa meilleure baleinière, mit le cap, vent arrière, sur Tahiti, à cinq cents milles de là, afin d’y enrôler un équipage de renfort.\par
Quatre jours après leur départ, ils aperçurent un grand canoë qui semblait avoir abordé sur une basse île corallienne. Ils s’en éloignèrent mais la sauvage embarcation vint sur eux et bientôt la voix de Steelkilt les somma de mettre en panne, faute de quoi ils les couleraient. Le capitaine sortit un pistolet. Un pied sur l’une des proues, un pied sur l’autre des pirogues accouplées, l’homme des Lacs eut un rire méprisant et l’assura que, s’il entendait seulement le clic du verrou du pistolet, il l’enterrerait lui, le capitaine, parmi les bulles et l’écume.\par
– Que voulez-vous de moi ? s’écria le capitaine.\par
– Où allez-vous ? et pourquoi ? demanda Steelkilt. Et ne mentez pas !\par
– Je suis en route pour Tahiti afin de trouver des hommes.\par
– Bon. Laissez-moi vous aborder un moment. En toute paix. Sur ce, il plongea depuis la pirogue, gagna la baleinière à la nage puis, grimpant sur le plat-bord, il se tint face à face avec le capitaine.\par
– Croisez les bras, sir, et tenez la tête droite. Maintenant répétez après moi : dès que Steelkilt m’aura quitté, je jure de débarquer sur cette île et d’y demeurer six jours. Si je manque à ma parole, que je sois foudroyé !\par
– Quel bon élève ! ricana l’homme des Lacs. Adios, señor ! et il retourna à la nage vers ses camarades.\par
Ils surveillèrent la baleinière jusqu’à ce qu’elle fût tirée à sec sur les racines des cocotiers, puis Steelkilt remit à la voile et en temps voulu arriva à Tahiti sa destination à lui aussi. Là, la chance lui sourit, deux navires appareillaient pour la France et providentiellement ils avaient exactement besoin du nombre d’hommes qu’ils se trouvaient être. Ils s’embarquèrent et furent à jamais à l’abri de leur ancien capitaine si celui-ci avait eu des velléités de les poursuivre légalement.\par
Une dizaine de jours après le départ des navires français, la baleinière atteignit Tahiti, et le capitaine fut contraint d’enrôler deux des insulaires parmi les plus civilisés ayant quelque usage de la mer. Il fréta une petite goélette indigène et regagna son navire avec ses hommes. Il y retrouva tout en ordre et reprit sa croisière.\par
Où est Steelkilt à présent ? nul ne le sait, messieurs, mais sur l’île de Nantucket, la veuve de Radney implore encore la mer qui se refuse à rendre ses morts, et voit encore en rêve l’affreuse baleine blanche qui l’a tué.\par
– En avez-vous terminé ? demanda doucement don Sebastian.\par
– Oui, señor.\par
– Alors je vous en prie, dites-moi en toute sincérité si vousmême vous croyez que cette histoire est vraie en substance ? Elle est si étonnante ! La tenez-vous de source sûre ? Pardonnez-moi si j’ai l’air d’insister.\par
– Et pardonnez-nous également, monsieur du marin, car nous renchérissons sur la demande de don Sebastian, réclama l’assemblée avec un intérêt extrême.\par
– Y a-t-il un volume des Saints Évangiles à l’auberge Dorée, messieurs ?\par
– Non, répondit don Sebastian mais je connais un honorable prêtre tout près d’ici qui m’en procurera un sur-le-champ. Je vais le chercher, mais est-ce très judicieux ? la question risque de devenir trop sérieuse.\par
– Auriez-vous la bonté de ramener également le prêtre, señor ?\par
– Bien qu’il n’y ait plus d’autodafés à Lima, dit un membre de la compagnie à un autre, je crains que notre ami marin n’encoure des risques auprès de l’archiépiscopat. Retirons-nous hors de ce clair de lune. Je ne vois pas d’utilité à tout cela.\par
– Excusez-moi de vous courir après, don Sebastian, mais oserais-je vous demander de vous procurer les Évangiles dans le plus grand format possible.\par
– Voici le prêtre, il vous apporte les Évangiles, dit don Sebastian gravement en revenant avec un personnage aussi grand que solennel.\par
– Permettez-moi de me découvrir. Et maintenant, vénérable prêtre, avancez davantage à la lumière, je vous prie, et tenez le livre assez près de moi, pour que je puisse le toucher.\par
– Au nom du ciel et de mon honneur, messieurs, l’histoire que je viens de vous conter est vraie, en substance et dans ses grandes lignes. Je la sais vraie, elle s’est passée ici-bas, j’étais sur ce navire, j’ai connu l’équipage, j’ai vu Steelkilt et je lui ai parlé depuis la mort de Radney.
\chapterclose


\chapteropen
\chapter[{CHAPITRE LV. Monstrueuses représentations de baleines}]{CHAPITRE LV \\
Monstrueuses représentations de baleines}\renewcommand{\leftmark}{CHAPITRE LV \\
Monstrueuses représentations de baleines}


\chaptercont
\noindent Je vais peindre pour vous, aussi bien qu’on peut le faire sans toile, quelque chose comme la forme vraie de la baleine telle qu’elle apparaît actuellement aux yeux du baleinier lorsqu’elle est amarrée au flanc du navire et qu’on peut aisément poser son pied dessus. Il vaut peut-être la peine de parler auparavant des curieux portraits imaginaires qu’on en a faits et qui jusqu’à ce jour même ont été un défi à la crédulité des terriens. Il est temps de redresser l’opinion en ce domaine en prouvant que toutes ces représentations de la baleine sont fausses.\par
Peut-être que ces images erronées prennent leurs sources dans les plus anciennes sculptures hindoues, égyptiennes et grecques car, depuis ces temps riches en imagination et dépourvus de scrupules, on a représenté le dauphin couvert d’écailles, avec une cotte de mailles digne de Saladin et casqué comme saint Georges, tant sur le marbre des temples que sur le piédestal des statues, les boucliers, les médaillons, les coupes et les pièces de monnaies. Pareille licence s’est partiellement perpétuée depuis, non seulement dans la plupart des représentations populaires des cétacés, mais encore en bien des ouvrages scientifiques.\par
La plus ancienne des images qui prétendent représenter une baleine se trouve dans une grotte d’Elephanta. Les Brahmanes affirment que tous les commerces, tous les travaux, toutes les vocations possibles de l’homme furent préfigurées par les innombrables sculptures de ce temple immémorial, des siècles avant qu’elles ne deviennent réalité. Rien d’étonnant, dès lors, à ce que l’ombre de notre noble profession de baleiniers s’y dessinât déjà. La baleine hindoue dont nous parlons occupe une paroi isolée où se trouve représentée l’une des manifestations de Vishnu sous la forme du léviathan, doctement connue sous le nom de Matsya-avatâra. Mais bien que cette sculpture soit mihomme, mi-baleine, que cette partie de baleine soit la queue, cette petite partie se trouve encore être toute fausse. Elle ressemble davantage à l’extrémité effilée d’un anaconda qu’au majestueux éventail de vastes palmes de la queue de la baleine franche.\par
Mais parcourons les musées et regardons la toile d’un grand peintre chrétien ; il n’a pas mieux réussi que l’Hindou antédiluvien à représenter ce poisson. C’est le tableau du Guide où Persée sauve Andromède du monstre marin ou de la baleine. Où le Guide a-t-il trouvé le modèle d’une aussi étrange créa- ture ? Hogarth n’a pas fait mieux avec sa « Descente de Per- \hspace{1em}sée ». L’obèse monstre hogarthien ondule à la surface, tirant à peine un pouce d’eau. Il porte sur son dos une sorte de houddan et sa gueule distendue, ornée de défenses et dans laquelle semblent rouler les lames, pourrait être prise pour la Porte des Traîtres qui mène à la Tour par les eaux de la Tamise. Il y a encore les baleines-prémices du vieil écossais Sibbald, et la baleine de Jonas telle que la montrent les Bibles anciennes et les vieux abécédaires. Que dire de celles-là ? Quant à la baleine du relieur s’enroulant comme une tige de vigne autour d’un jas d’ancre, telle qu’elle est gravée et dorée au dos de bien des livres anciens ou modernes, elle est certes très pittoresque mais c’est une créature purement fabuleuse, imitée, je pense, de celles qui ornent les vases antiques. Bien que le poisson de ce relieur soit universellement connu pour être un dauphin, j’insiste pour y voir l’ébauche d’une baleine, car c’est bien ce qu’elle était censée représenter au départ ; c’est un vieil éditeur italien qui la mit à la mode au XV\textsuperscript{e} siècle, pendant la Renaissance, or, à cette époque, et même sensiblement plus tard, la croyance populaire voulait que les dauphins fussent une espèce de léviathan.\par
Dans les vignettes et autres ornementations de livres anciens, vous trouverez de temps en temps de curieux dessins de baleines où toutes sortes de souffles jaillissent en bouillonnant de son cerveau inépuisable sous formes de jets d’eau, de geysers chauds ou froids, de sources de Saratoga et de Baden-Baden. Le frontispice de l’édition originale du « Progrès de la Connaissance » représente d’étranges baleines.\par
Mais laissons-là ces essais d’amateurs pour jeter un coup d’œil aux représentations du léviathan qui ont la prétention d’être sérieuses, scientifiquement exécutées par ceux qui savent. Dans la collection de voyages du vieil Harris, on peut voir, tirées d’un ouvrage hollandais de 1671 ; « Campagne de pêche à la baleine au Spitzberg, à bord du {\itshape Jonas-dans-la-baleine}, capitaine Peter Peterson, de Frise » quelques planches de baleines. Sur l’une d’elles les baleines, pareilles à un train de flottage, sont étendues entre des îles de glace, tandis que des ours blancs courent sur leurs dos bien en vie. Sur une autre, bévue majeure, la baleine a une queue verticale.\par
Et dans un important in-quarto, écrit par un certain capitaine Colnett, capitaine de vaisseau dans la marine britannique, et intitulé « Voyage autour du Cap Horn dans les mers du Sud, ayant pour but d’étendre les pêcheries du cachalot » ; un horstexte prétend nous montrer « Le Physeter ou cachalot dessiné à l’échelle d’après celui qui fut tué sur les côtes du Mexique et halé à bord en août 1793. « Je ne doute pas que le capitaine ait fait faire ce portrait véridique pour instruire ses marins. Pour n’en mentionner qu’un détail, permettez-moi de vous dire que, mesuré à l’échelle qui l’accompagne, l’œil d’un cachalot adulte aurait les dimensions d’une baie vitrée de quelque cinq pieds de long. Ah, mon brave capitaine, pourquoi n’avez-vous pas mis Jonas regardant par cette fenêtre !\par
Les plus consciencieuses compilations d’Histoire Naturelle à l’usage de l’âge tendre ne laissent pas de commettre les mêmes erreurs énormes. Regardons cet ouvrage populaire : {\itshape La Nature Animée de Goldsmith}, dans l’édition abrégée, parue à Londres en 1807, on y trouve des planches qui sont censées représenter une « baleine » et un « narval ». Je ne voudrais pas paraître inélégant mais cette vilaine baleine ressemble beaucoup à une truie amputée ; quant au narval, il suffit d’un simple coup d’œil pour s’émerveiller de ce qu’au XIX\textsuperscript{e} siècle un pareil hippogriffe ait pu passer pour authentique auprès de n’importe quel écolier intelligent.\par
En 1825, Bernard Germain, comte de Lacépède, un grand naturaliste, a publié un ouvrage scientifique dans lequel il propose une nomenclature des baleines, où l’on trouve plusieurs représentations des différentes espèces de léviathans. Non seulement elles sont incorrectes mais encore, de la Mysticetus ou Baleine du Groenland (c’est-à-dire la baleine franche), Scoresby lui-même, qui avait une grande expérience de celle-ci, déclare qu’elle ne ressemble en rien à celles qu’il a vues.\par
La palme de la maladresse revient à Frédéric Cuvier, homme de science et frère du fameux Baron. En 1836, il a publié une Histoire Naturelle des Cétacés, dans laquelle il donne ce qu’il appelle une image du cachalot. Avant de la montrer à Nantucket, tenez-vous prêt à en déguerpir. En un mot, le cachalot de Cuvier n’est pas un cachalot, c’est une courge. Bien sûr, il n’a pas eu la chance de faire une expédition baleinière, de tels hommes ont rarement cette chance, mais nul ne sait ce qui lui a inspiré ce dessin, peut-être qu’à l’instar de son prédécesseur dans ce domaine, Desmaret, il a puisé ce monstre à la source des dessins chinois. Des tasses et sous-tasses suffisent à nous prouver la verve imaginative de ces Chinois lorsqu’ils ont un pinceau entre les mains.\par
Quant aux enseignes qui se balancent dans les rues devant les portes des marchands d’huile, qu’en dire ? Les baleines y ressemblent à Richard III, sont bossues comme des dromadaires, ont un air féroce, et déjeunent d’une tarte de matelots, c’està-dire d’une baleinière avec tout son équipage, croquent deux ou trois hommes et, difformes, barbotent dans des mers de sang et de peinture bleue.\par
Mais toutes ces erreurs ne sont pas étonnantes, après tout. Pensez-y ! La plupart de ces dessins ont été faits d’après des poissons échoués, ce qui équivaut à représenter un navire d’après une épave. Comment dès lors, l’échiné brisée, le noble animal aurait-il la fringante fierté de sa coque et de sa mâture ? Si les éléphants ont posé en pied, jusqu’à ce jour, le léviathan n’est pas encore sorti de l’eau pour faire faire son portrait. La baleine vivante, dans toute sa majesté et dans toute sa signification, ne peut être vue que dans les profondeurs insondables. Lorsqu’elle vient en surface, sa quille énorme demeure invisible comme celle d’un vaisseau de ligne dès qu’il a été lancé et, hors de cet élément il sera à jamais impossible à l’homme mortel de la hisser en l’air tout en lui conservant ses courbes puissantes et son rythme onduleux. Je ne parle pas de la différence de forme qui existe sans aucun doute entre un baleineau à la mamelle et un léviathan adulte et idéal ; d’ailleurs même lorsqu’un de ces nourrissons a été halé sur le pont d’un navire, sa forme est tellement insolite, souple et changeante que le diable lui-même ne saurait la définir.\par
On pourrait penser que d’après le squelette d’un cétacé échoué, sa forme exacte pourrait être évoquée avec justesse, il n’en est rien. Car c’est l’une des caractéristiques les plus curieuses du léviathan que son squelette ne donne qu’une faible idée de son aspect général. Si le squelette de Jérémie Bentham, qui fait office de candélabre dans la bibliothèque de l’un de ses exécuteurs testamentaires, donne une idée juste d’un vieux \hspace{1em} monsieur au front solide nourrissant des idées utilitaires, ainsi que toutes les caractéristiques marquantes dudit Jérémie, les articulations d’aucun léviathan ne sont susceptibles de pareilles révélations. En fait, comme le dit le grand Hunter, le seul squelette de la baleine a autant de rapport avec l’animal bien en chair que l’insecte avec la chrysalide ronde qui l’emmaillote. La tête le démontrera particulièrement bien comme nous le verrons incidemment au cours de ce récit. Ces nageoires pectorales en sont aussi une illustration intéressante, leurs os correspondant presque exactement à ceux de la main humaine, pouce en moins. Une nageoire pectorale a quatre doigts osseux, l’index, le majeur, l’annulaire et l’auriculaire, mais ils sont enfouis dans leur vêtement de chair comme les doigts des hommes dans des moufles. Comme le disait un jour Stubb, l’humoriste : « Si cavalièrement que la baleine nous traite parfois, on ne pourra pourtant jamais dire qu’elle n’y met pas de gants. »\par
Pour toutes ces raisons, sous quelque angle que vous envisagiez la question, vous êtes obligé d’arriver à cette conclusion que le grand léviathan est l’unique créature au monde dont personne ne fera jamais le portrait. Il est vrai qu’un portrait peut être plus ressemblant qu’un autre, mais aucun ne peut prétendre à une exactitude parfaite, de sorte qu’il n’y a aucun moyen de découvrir de quoi a réellement l’air une baleine. La seule manière de vous faire une idée convenable de sa forme vivante, c’est d’aller vous-même à la pêche, et, ce faisant, vous ne risquez pas peu d’être défoncé et envoyé par le fond par ses soins. Il me semble dès lors qu’il vaudrait mieux pour vous de n’avoir point une curiosité trop pointilleuse au sujet du léviathan.
\chapterclose


\chapteropen
\chapter[{CHAPITRE LVI. Représentations moins fausses de baleines et tableaux véridiques de scènes de pêche à la baleine}]{CHAPITRE LVI \\
Représentations moins fausses de baleines et tableaux véridiques de scènes de pêche à la baleine}\renewcommand{\leftmark}{CHAPITRE LVI \\
Représentations moins fausses de baleines et tableaux véridiques de scènes de pêche à la baleine}


\chaptercont
\noindent À propos des représentations monstrueuses des cétacés, je suis bien tenté d’aborder ici les histoires encore plus aberrantes, qu’on peut trouver à leur sujet dans des livres anciens et modernes, en particulier dans Pline, Purchas, Hackluyt, Harris, Cuvier, etc. Mais laissons.\par
Je ne connais que quatre dessins du grand cachalot qui aient été publiés, ceux de Colnett, Huggins, Frédéric Cuvier et Beale. Nous avons parlé dans le précédent chapitre de ceux de Colnett et de Cuvier. Celui d’Huggins est bien meilleur que les leurs, mais celui de Beale est de loin supérieur. Tous les dessins de ce cétacé par Beale sont bons, sauf la figure centrale de trois d’entre eux, en diverses attitudes en tête de son second chapitre. Son frontispice, des baleinières à l’assaut de cachalots, bien que visant sans doute à éveiller un scepticisme poli dans les salons est parfaitement exact et donne le sentiment de la réalité. Quelques-uns des dessins du cachalot de J. Ross Browne sont joliment corrects mais misérablement gravés. Il n’en peut mais toutefois.\par
Scoresby a donné les meilleurs dessins de la baleine franche mais ils sont à une trop petite échelle pour produire l’impression souhaitable. Triste lacune, il ne montre qu’une scène de chasse, or ce sont de telles figurations seules qui, si elles sont à peu près bien faites, peuvent donner quelque idée juste du cétacé vivant tel que le voient ses vivants chasseurs.\par
Mais, à tout prendre, et bien que certains détails laissent à désirer, les représentations de baleines et de scènes de chasse de loin les mieux réussies sont celles qu’en donnent deux grandes estampes françaises fort bien reproduites d’après les toiles d’un certain Garneray. L’une illustre l’attaque livrée à un cachalot, l’autre à une baleine franche. La première montre un cachalot dans toute la majesté de sa puissance déployée, émergeant des profondeurs de l’Océan juste sous la pirogue et soulevant dans les airs l’épave terrifiante de l’embarcation brisée ; la proue presque intacte se balance sur l’échine du monstre et, debout à cette étrave, pour cet unique et incalculable instant, un rameur à demi voilé par le bouillonnement furieux du souffle s’apprête à plonger comme du haut d’une falaise. L’action est admirablement rendue dans sa vérité. La baille à ligne flotte à demi vide sur les flots blanchis, les manches des harpons répandus se dressent sur l’eau, les visages des hommes nageant autour de la baleine expriment la terreur cependant que dans le lointain orageux et sombre s’avance le navire. On pourrait relever des fautes de détails anatomiques dans ce cachalot, mais passons-les sous silence car, de ma vie, je n’en pourrais dessiner un aussi bien.\par
La seconde estampe montre une baleinière en train d’aborder le flanc couvert de coronules d’une grande baleine franche qui roule sa masse noire et moussue dans la mer comme quelque roc descendant des falaises de Patagonie. Ses jets sont droits, épais, noirs comme de la suie, de sorte qu’une pareille fumée dans la cheminée inviterait à croire qu’un honnête souper est en train de cuire dans ses entrailles. Des oiseaux de mer picorent de petits crabes, des coquillages, et autres bonbons et macaronis marins, pestilence que la vraie baleine porte parfois sur le dos. Pendant tout ce temps, le léviathan aux lippes épaisses se rue à travers les flots, laissant un sillage violent et crémeux qui ballote la pirogue légère comme \hspace{1em} les roues à aubes d’un vapeur le ferait d’un esquif surpris à ses côtés. Ainsi le premier plan n’est que furieuse agitation mais au loin, par un contraste d’un art admirable, une mer encalminée délimite un horizon de verre, tandis que les voiles du navire impuissant sont mollement affaissées et que, forteresse conquise, la masse inerte d’une baleine morte porte le drapeau de la victoire flottant paresseusement à la lance plantée dans l’un de ses évents.\par
Qui est ou qui était le peintre Garneray ? je n’en sais rien. Mais, sur ma vie, ou bien il était familier du sujet, ou bien il fut merveilleusement instruit par un baleinier expérimenté… Les Français sont des gars faits pour peindre l’action. Allez voir toutes les peintures de l’Europe, et voyez où vous trouverez une galerie qui vous offre des toiles plus vivantes et animées que celles du triomphal palais de Versailles. Le spectateur s’y fraie son chemin parmi toutes les grandes batailles de France, toute épée y est aurore boréale, les rois et les empereurs en armes chargent comme des centaures couronnés. Et les scènes de combat de Garneray ne seraient pas indignes de figurer à leurs côtés.\par
Le don qu’ont les Français de saisir le pittoresque est prouvé par les peintures et les gravures qu’ils ont de leurs scènes de pêche à la baleine. Bien qu’ils n’aient pas le dixième de l’expérience des Anglais en ce domaine et pas le millième de celle des Américains, ils ont néanmoins donné à ces deux pays les seuls croquis achevés évoquant le véritable esprit de la chasse. La plupart du temps, les Anglais et les Américains se sont contentés de dessiner des contours extérieurs tels que la silhouette vide d’une baleine, dont l’effet pictural équivaut à peu près à celui que produirait le profil d’une pyramide. Scoresby lui-même, à juste titre célèbre comme chasseur de baleines franches, après nous avoir donné une fort raide image de la baleine franche de la tête à la queue, et trois ou quatre miniatures délicates de narvals et de marsouins, nous régale d’une série de gravures classiques de gaffes d’embarcation, de couteaux \hspace{1em}à émincer et de grappins puis, avec l’application minutieuse d’un Leuwenhoeck, il nous offre l’univers brisé de quatre-vingt-seize agrandissements de cristaux arctiques. Je rends hommage à ce pionnier et je n’ai aucune intention de dénigrer l’excellent voyageur mais ce fut certainement une négligence de sa part d’omettre, dans une affaire aussi importante, de pourvoir chaque cristal d’une attestation écrite par-devant un Juge de Paix du Groenland.\par
En plus de celles de Garneray, deux autres estampes françaises sont dignes d’attention, dont l’auteur signe « H. Du- rand ». L’une d’elles, bien qu’un peu hors de notre présent sujet, mérite mention à d’autres égards. La scène se passe aux tranquilles heures de midi dans les îles du Pacifique, un navire baleinier français, à l’ancre près de la côte, refait paresseusement ses réserves d’eau, ses voiles déferlées et les longues palmes de l’arrière-plan pendent avec la même mollesse dans l’air immobile. L’impression est très bien rendue de l’un des rares moments de repos oriental de ces pêcheurs téméraires. L’autre estampe traite d’une tout autre affaire. Le navire a mis en panne en pleine mer, au cœur même de la vie léviathanesque, une baleine franche amarrée à son flanc comme à une estacade, et l’on procède au dépècement. Fuyant cette scène de laborieuse activité, une pirogue s’apprête à donner la chasse à des cétacés dans le lointain. Harpons et lances reposent prêts à l’emploi, trois canotiers ajustent le mât dans son emplanture cependant qu’une houle soudaine soulève l’avant de la baleinière à demi dressée sur l’eau comme un cheval cabré. Du navire s’élève la fumée des tourments de la baleine en train de fondre, pareille à celle que ferait tout un village de forgerons, et au vent un nuage noir prometteur de grains et de pluies paraît inviter les marins impatients à se hâter.
\chapterclose


\chapteropen
\chapter[{CHAPITRE LVII. Baleines diversement représentées : en peinture, en ivoire, en bois, en tôle, en pierre… Dans les montagnes et dans le ciel}]{CHAPITRE LVII \\
Baleines diversement représentées : en peinture, en ivoire, en bois, en tôle, en pierre… Dans les montagnes et dans le ciel}\renewcommand{\leftmark}{CHAPITRE LVII \\
Baleines diversement représentées : en peinture, en ivoire, en bois, en tôle, en pierre… Dans les montagnes et dans le ciel}


\chaptercont
\noindent À Tower Hill, en descendant vers les docks de Londres, vous avez peut-être vu un mendiant estropié (un camelot comme disent les marins) tenant devant lui la planche peinte où est représentée la scène tragique qui lui valut la perte de sa jambe. On voit trois baleines et trois pirogues, l’une des baleinières (censée contenir la jambe manquante dans son intégrité originelle) est en train d’être broyée entre les mâchoires du cachalot du premier plan. Voilà dix ans, m’a-t-on dit, que cet homme tient sa pancarte et exhibe son moignon aux passants incrédules. Le temps est venu de présenter sa défense d’après la vérité des faits. Ses trois baleines sont en tout cas aussi bonnes que celles qui furent jamais vues à Wapping, et son moignon est aussi indiscutable que les souches des essartages de l’Ouest ; bien qu’à jamais perché sur une jambe, ce pauvre baleinier ne vous contraint pas par des harangues à faire le pied de grue mais, les yeux baissés, il contemple tristement son infirmité.\par
Partout dans le Pacifique, ainsi qu’à Nantucket, New Bedford, Sag Harbor, il vous arrivera de voir des croquis très vivants de baleines ou de scènes de pêche, gravés par les matelots eux-mêmes sur des dents de cachalots ou des baleines de corset, fanons de la baleine franche, et bien d’autre articles de paccotille, comme les marins appellent les nombreux et ingénieux ouvrages qu’ils sculptent minutieusement dans la matière brute pendant leurs heures de loisir en mer. Quelques-uns d’entre eux possèdent des petites boîtes d’un outillage qui rappelle celui des dentistes et destiné spécialement à cet usage mais, en général, ils ne travaillent qu’avec leur couteau de poche et grâce à cet outil d’usage quasiment universel pour le marin, ils vous feront tout ce que vous voudrez qui relève de l’imagination d’un marin.\par
Un long exil loin de la chrétienté et de la civilisation replace inévitablement un homme dans la condition première que Dieu lui avait assignée, un état qu’on appelle barbarie. Un vrai baleinier est un sauvage au même titre qu’un Iroquois. Je suis moimême un sauvage, ne devant allégeance qu’au Roi des Cannibales et toujours prêt à me révolter contre lui.\par
Or, l’une des caractéristiques du sauvage, lors de ses heures passées en famille, c’est sa patience merveilleuse et son application au travail. Un ancien casse-tête ou pagaie-sagaie hawaïenne, par la multiplicité et le raffinement de ses sculptures, est un aussi grand trophée de patience humaine qu’un dictionnaire latin. Car ce réseau miraculeux de dessins intriqués ne fut fait qu’avec un morceau de coquillage brisé ou une dent de requin et coûta des années assidues de persévérance.\par
Tel est le sauvage hawaïen, tel est le matelot-sauvage. Avec la même patience admirable, avec la même unique dent de requin, celle de son pauvre couteau de poche, il vous sculptera un os, non point tout à fait aussi habilement que ne le fut le bouclier du Grec sauvage Achille, mais aussi chargé et complexe quant aux motifs, et tout aussi empreint d’esprit barbare et de caractère suggestif que les gravures de ce magnifique sauvage germain, Albert Dürer.\par
Des baleines de bois, ou des silhouettes de baleines taillées dans de petits morceaux noirs de ce noble bois de tek des mers du Sud se trouvent fréquemment dans les gaillards d’avant des navires baleiniers américains et certaines sont exécutées avec beaucoup de justesse.\par
À la porte d’entrée de quelques vieilles maisons à pignons, à la campagne, vous verrez des baleines de bronze, pendues par la queue, servir de marteaux, et lorsque le portier dort la baleine à tête d’enclume semble la plus apte à l’éveiller. Mais ces baleines heurtoirs pèchent rarement par excès de fidélité. Sur les flèches d’églises à l’ancienne mode, des baleines de tôle font office de girouettes, mais elles sont si haut perchées et si bien protégées, à toutes fins utiles, par des écriteaux « Défense de toucher ! » qu’on ne peut les examiner d’assez près pour juger de leur valeur.\par
Dans des régions osseuses, squelettiques, de la terre, là où des groupes de rochers fantastiques s’éparpillent dans la plaine, sous les cassures de hautes falaises, vous verrez souvent des formes pétrifiées de léviathans à demi enfouies dans l’herbe qui, aux jours de vent, brise contre leurs flancs le ressac de sa houle verte.\par
Puis, dans les contrées montagneuses où des arènes rocheuses ceinturent presque sans cesse le voyageur, d’ici ou de là, d’un point de vue privilégié, vous apercevrez d’éphémères profils de baleines se dessinant sur les crêtes. Mais vous ne les verrez que si vous êtes un baleinier d’expérience, et qui plus est, si vous voulez retourner à ce point de vue, il faut vous assurer de l’intersection exacte de sa latitude et de sa longitude car, sur ces hauteurs, retrouver un tel endroit nécessiterait une véritable redécouverte, comme il a fallu redécouvrir les îles Salomon qui restèrent inconnues bien que Mendana, engoncé dans sa fraise, les ait parcourues et que le vieux Figueira les ait décrites.\par
Enfin lorsque l’enthousiasme baleinier vous emporte, vous ne manquerez pas d’en découvrir dans les cieux étoiles, poursuivies par des pirogues, comme des nations de l’Est, obsédées par la pensée de la guerre, virent dans les nuages des armées en ordre de bataille. C’est ainsi que j’ai pourchassé le léviathan dans le Nord, tournant encore et encore autour du Pôle avec les révolutions des points brillants qui l’avaient une fois délimité pour moi. Et sous les cieux rayonnants de l’Antarctique, je me suis embarqué sur le navire {\itshape Argo} pour livrer la chasse à la constellation de la baleine, bien au-delà de l’Hydre et du Poisson volant.\par
Des ancres de frégate pour mors de bride, des faisceaux de harpons pour éperons, j’aimerais pouvoir enfourcher cette baleine et bondir au-delà de la voûte céleste pour voir si les cieux fabuleux ont bien planté leurs tentes innombrables plus loin que ne peut porter mon regard humain !
\chapterclose


\chapteropen
\chapter[{CHAPITRE LVIII. Krill}]{CHAPITRE LVIII \\
Krill}\renewcommand{\leftmark}{CHAPITRE LVIII \\
Krill}


\chaptercont
\noindent Faisant route au nord-ouest des Crozets, nous nous trouvâmes dans de vastes prairies de krill, la manne minuscule et jaune dont la baleine franche se nourrit à peu près exclusivement. Cette nappe ondulait autour de nous des lieues durant, de sorte que nous semblions naviguer dans des champs de blé mûr.\par
Le second jour, nous vîmes un grand nombre de baleines franches. À l’abri des attaques d’un chasseur de cachalots tel que le {\itshape Péquod}, elles nageaient paresseusement, la bouche ouverte à travers le krill qui reste pris dans les fibres des fanons tandis que l’eau est expulsée entre les interstices de sa prodigieuse jalousie.\par
Comme au matin les faucheurs, côte à côte, poussent lentement leur faux chantante dans la longue herbe mouillée des marécageuses prairies, ainsi les monstres avançaient en faisant un étrange bruit d’herbe qu’on coupe laissant, derrière eux, d’immenses fauchées bleues sur la mer jaune\footnote{Cet endroit de la mer, connu par les baleiniers sous le nom de Bancs du Brésil, ne s’appelle pas ainsi à la façon dont on parle des Bancs de Terre-Neuve où se trouvent des hauts fonds et des sondes, mais à cause de l’aspect remarquable de prairie dû à la quantité de krill qui flotte sur les eaux sous ces latitudes où l’on chasse souvent la baleine franche.}.\par
Mais seul le bruit qu’ils faisaient en essorant le krill rappelait les faucheurs ; vus du haut des mâts, surtout lorsqu’ils demeuraient un moment immobiles, leurs gigantesques masses noires ressemblaient bien davantage à des rochers qu’à tout autre chose. C’est ainsi que, dans les vastes territoires de chasse de l’Inde, l’étranger traversant des plaines prendra des éléphants couchés pour des monticules de terre sombre et nue. Il en ira de même pour celui qui voit, pour la première fois, ces léviathans immobiles dans la mer. Et même lorsqu’ils les aura identifiés enfin, il aura grand-peine à croire qu’une masse aussi importante, aussi énorme, puisse être aussi pleine de vie qu’un chien ou un cheval.\par
À vrai dire, à d’autres égards, on peut difficilement considérer les créatures des grandes profondeurs avec les mêmes sentiments qu’on nourrit pour les animaux terrestres. Cela bien que d’anciens naturalistes aient affirmé que toute la faune de la terre ait son équivalent dans la mer. D’un point de vue général, c’est parfait, mais si l’on en vient aux détails, l’Océan nous offret-il jamais l’exemple d’un poisson doué de l’intelligence et de la bonté du chien ? Seul le requin maudit pourrait avoir quelque analogie avec lui au point de vue générique.\par
Mais bien que les terriens en général aient toujours considéré les habitants des mers avec une indicible émotion faite d’antipathie et de répulsion, bien que nous sachions que la mer cache une terre éternellement inconnue et que Colomb ait navigué au-dessus d’innombrables mondes mystérieux pour découvrir celui-là seul qui émergeait à l’ouest, bien que les plus terrifiants désastres se soient abattus sans discrimination et depuis des temps immémoriaux sur des centaines de milliers de ceux qui se sont aventurés sur les eaux, bien qu’un instant de réflexion nous apprenne que l’homme, encore en enfance, se vante de son adresse et de ses connaissances qu’un avenir prometteur augmentera encore, malgré tout cela pour toujours et à jamais jusqu’au jour du jugement, la mer l’insultera, l’assassinera et pulvérisera la frégate la plus majestueuse et la plus robuste qu’il puisse construire. Et cependant, ces \hspace{1em} impressions mêmes se répétant, la terreur primordiale inhérente à la mer s’est émoussée au cœur de l’homme.\par
Le premier navire dont nous parle l’histoire flotta sur un Océan dont la vengeance engloutit un univers entier sans même laisser une veuve. C’est le même Océan qui roule maintenant ses flots, c’est lui qui a détruit les vaisseaux naufragés de l’année précédente. Oui, mortels inconséquents, le déluge de Noé ne s’est pas encore asséché, il couvre encore les deux tiers de la douce terre.\par
En quoi la mer diffère-t-elle donc de la terre pour qu’un miracle sur l’une n’en soit point un sur l’autre ? Une terreur sacrée envahit les Hébreux lorsque la terre vivante s’ouvrit sous les pieds de Korah et de ses hommes, les engloutissant à jamais et pourtant le soleil pas un jour ne se lève que la mer vivante n’avale de la même manière navires et équipages.\par
Non seulement la mer est l’ennemie de cet homme qui lui est étranger mais encore elle est démoniaque envers ses propres enfants, plus fourbe que l’hôte persan qui assassine ses invités, n’épargnant pas ceux qu’elle a engendrés. Comme une tigresse sauvage étouffe en se retournant ses propres enfants, la mer jette aux rochers de la côte les plus puissantes baleines et les abandonne flanc à flanc avec les épaves des navires naufragés. Point de miséricorde, elle ne connaît d’autre maître que sa propre puissance. Haletant et renâclant comme un destrier affolé qui a perdu son cavalier, le libre Océan galope autour du globe.\par
Songez à la ruse de la mer et à la manière dont ses créatures les plus redoutables glissent sous l’eau, à peu près invisibles, traîtreusement cachée par les plus suaves tons d’azur. Songez à la beauté et à l’éclat satanique de ses plus impitoyables tribus, à la forme exquise de certains requins. Songez au cannibalisme universel qui règne dans la mer où les créatures de \hspace{1em}proie s’entre-dévorent, menant une guerre éternelle depuis l’origine du monde.\par
Songez à tout cela et tournez alors vos regards vers cette terre aimable et verte infiniment docile, songez à l’Océan et à la terre, ne retrouvez-vous pas en vous-même leurs pareils ? Car de même que cet océan de terreur entoure les verts continents, de même l’âme de l’homme enferme une Tahiti, île de paix et de joie, cernée par les horreurs sans nombre d’une vie à demi inconnue. Que Dieu te garde ! Ne pousse pas au large de cette île, tu n’y pourrais jamais revenir !
\chapterclose


\chapteropen
\chapter[{CHAPITRE LIX. Le calmar}]{CHAPITRE LIX \\
Le calmar}\renewcommand{\leftmark}{CHAPITRE LIX \\
Le calmar}


\chaptercont
\noindent Se frayant péniblement un passage à travers le krill, le {\itshape Péquod} maintenait son cap nord-ouest vers Java, une faible brise le poussait lentement, et dans cette sérénité, ses trois grands mâts effilés se balançaient mollement au gré d’une brise languissante avec la nonchalance de trois palmiers sur une plaine. Et toujours, à larges intervalles, dans la nuit argentée, le souffle solitaire et attirant était aperçu.\par
Mais dans la transparence bleue d’un matin, tandis qu’une tranquillité presque surnaturelle s’étalait sur la mer bien que ce ne fût pas le calme plat ; tandis que la clairière longuement polie que le soleil ouvrait sur les eaux semblait l’empreinte de son doigt d’or les invitant au silence ; tandis que les vagues feutrées couraient doucement en chuchotant, alors dans le profond silence de ce monde visible, un spectre étrange fut signalé par Daggoo, en vigie au grand mât.\par
Au loin, une grande masse blanche s’éleva paresseusement toujours plus haut et, se détachant sur l’azur, elle étincela enfin devant notre proue comme une avalanche fraîchement descendue des montagnes. Elle brilla ainsi un instant, puis s’abaissa et plongea. Puis elle s’éleva à nouveau, lumineuse et silencieuse. On n’eût point dit une baleine, pourtant Daggoo se demanda si ce n’était pas. Une fois de plus, le spectre plongea et réapparut. Alors, d’une voix aiguë dont la pointe tira chaque homme de sa somnolence, le nègre cria :\par
– Là ! Là ! encore. La voilà qui émerge ! Droit devant ! La Baleine blanche ! La Baleine blanche !\par
À ces mots les matelots se précipitèrent aux fusées de vergues, comme, au temps des essaims, les abeilles aux rameaux. Tête nue sous le soleil accablant, Achab, debout sur le beaupré, une main en arrière, prêt à donner par gestes ses ordres au timonier, regardait ardemment dans la direction du bras tendu que, là-haut, Daggoo tenait immobile.\par
Soit que la fugitive régularité du souffle tranquille et solitaire eût progressivement amené Achab à associer le calme et la paix à l’apparition de la baleine particulière qu’il poursuivait, soit qu’il eût d’autres raisons, ou que son impatience le dominât,\hspace{1em}à\hspace{1em}peine\hspace{1em}eût-il\hspace{1em}aperçu\hspace{1em}distinctement\hspace{1em}la\hspace{1em}masse\hspace{1em}blanche qu’avec une hâte passionnée il donna l’ordre de mettre à la mer.\par
Les quatre pirogues furent bientôt à l’eau et celle d’Achab, en tête, nageait vigoureusement vers sa proie. Celle-ci plongea bientôt. Tandis que, les avirons hauts, nous attendions qu’elle réapparaisse, voici qu’elle émergea lentement à l’endroit précis où elle s’était enfoncée. Oubliant presque pour l’instant, nous contemplions le plus prodigieux phénomène que les mers secrètes aient jusqu’à ce jour révélé aux hommes. Une vaste masse pulpeuse, d’une envergure fabuleuse, d’un blanc éblouissant, flottait à la surface de l’eau, des bras innombrables rayonnaient de son centre, se lovaient et se tordaient tel un nid d’anacondas, comme pour saisir aveuglément ce qui pourrait par malheur se trouver à leur portée. Masse uniforme, elle n’avait point de visage, rien qui pût permettre de lui prêter un instinct et des sensations mais elle ondulait là sur la houle, surnaturelle, informe image incompréhensible de la vie.\par
Tandis qu’elle disparaissait avec un faible bruit de siphon, Starbuck, les yeux encore fixés sur le bouillonnement de sa disparition, s’écria avec colère :\par
– J’aurais préféré rencontrer et lui livrer combat, plutôt que de t’avoir vu toi, fantôme blanc !\par
– Qu’était-ce, sir ? demanda Flask.\par
– Le calmar géant lui-même dont on dit que les rares baleiniers qui l’ont vu ne rentrèrent jamais au port pour en parler.\par
Mais Achab se tut ; faisant virer sa pirogue, il fit voile vers le navire et les autres suivirent en silence.\par
Quelles que soient les superstitions que les chasseurs de cachalots nourrissent envers cette chose, il est certain qu’il est tout à fait exceptionnel de l’apercevoir et que cette circonstance dès lors fut considérée comme de mauvais augure. Les marins prétendent tous que c’est la plus grande créature de l’Océan mais, parce qu’elle ne se montre presque jamais, bien rares sont ceux qui ont la plus vague idée de ce que peuvent être sa nature et sa forme véritables ; néanmoins, ils disent qu’elle fournit au cachalot son unique nourriture. En effet, les autres espèces de baleines trouvent leur subsistance en surface et on peut les voir se nourrir, mais le cachalot va chercher sa proie dans des profondeurs inconnues et c’est par déduction seulement qu’on peut dire de quoi il s’alimente. Parfois, harponné, il rejette ce qu’on pense être des bras du calmar géant, dont certains mesurent de vingt à trente pieds. Les hommes de la mer croient que ce monstre s’accroche au fond même de l’Océan par ses bras et que le cachalot étant pourvu de dents, contrairement aux autres espèces, est à même de l’attaquer et de le déchiqueter.\par
Il semble y avoir là matière à supposer que l’immense Kraken de l’évêque Pontoppidan pourrait bien être le calmar. La description de l’évêque relative à sa manière d’émerger et de disparaître tour à tour, comme certaines autres \hspace{1em}particularités, sont bien concordantes, mais il faut en rabattre quant aux dimensions incroyables qu’il lui attribue.\par
Quelques naturalistes, ayant entendu de vagues rumeurs concernant la mystérieuse créature dont nous parlons, la font entrer dans la famille des seiches, à laquelle en effet, certaines de ses caractéristiques semblent l’apparenter, mais seulement en tant que l’Anak de la tribu.
\chapterclose


\chapteropen
\chapter[{CHAPITRE LX. La ligne}]{CHAPITRE LX \\
La ligne}\renewcommand{\leftmark}{CHAPITRE LX \\
La ligne}


\chaptercont
\noindent En rapport avec la scène de pêche qui va bientôt être décrite, et dans le but de rendre plus claires les scènes de même nature qui suivront, il faut que je parle maintenant de la ligne à baleine, magique et parfois horrible.\par
La ligne employée à l’origine était en chanvre de première qualité, légèrement goudronnée mais non imprégnée comme les filins ordinaires, le goudron employé comme d’habitude permet au cordier une torsion plus aisée et rend le filin lui-même plus pratique pour son usage courant à bord ; mais un trop fort coaltarage rendrait la ligne à baleine trop rigide pour être lovée serrée comme elle doit souvent l’être, et la plupart des marins commencent à se rendre compte que le goudron n’ajoute rien à la durée ni à la solidité des filins, quelque épaisseur et brillant qu’il leur donne.\par
Depuis quelques années, le manille a supplanté tous les autres chanvres dans la fabrication des lignes de pêche américaines ; bien qu’il ne dure pas aussi longtemps que le chanvre ordinaire, il est plus solide, plus flexible et plus souple ; j’ajouterai (puisque l’esthétique a sa place en toute chose) qu’il est beaucoup plus beau et plus seyant à la pirogue. Le chanvre ordinaire est sombre et basané comme un Indien, le manille est un blond Circassien.\par
La ligne n’a que deux tiers de pouce d’épaisseur et, à première vue, on ne la croirait jamais aussi solide qu’elle n’est, chacun de ses cinquante et un fils est calculé pour une résistance de cent vingt livres, de sorte que toute la ligne supporte une traction de trois tonnes environ. La ligne employée ordinairement pour le cachalot a une longueur d’un peu plus de deux cents brasses. À l’arrière de la baleinière elle est enroulée dans une baille, non point à la manière d’un serpentin d’alambic mais en couches horizontales bien serrées qui ont l’aspect final d’un fromage et ne laissent d’autre vide que le « cœur », un tube minuscule formant l’axe du fromage. Comme le moindre nœud, la moindre coque entraîneraient à coup sûr quelque bras, quelque jambe ou un corps entier lorsqu’on file la ligne, elle est enroulée dans sa baille avec des précautions extrêmes. Certains harponneurs passeront une matinée presque entière à ce travail, montant la ligne en haut des mâts du navire et la faisant descendre en la passant à travers une cosse jusque dans la baille afin d’éviter le moindre faux pli et la moindre coque.\par
Dans les baleinières anglaises, il y a deux bailles au lieu d’une et la même ligne est enroulée de façon continue dans les deux. Ce procédé présente un avantage car, les deux bailles étant toutes petites, elles sont plus aisées à placer dans la pirogue et la pression sur le fond est répartie tandis que la baille américaine, qui a près de trois pieds de diamètre sur autant de haut, constitue une charge relativement lourde pour une embarcation dont les bordages n’ont qu’un demi-pouce d’épaisseur. Le fond d’une baleinière est pareil à une couche de glace fragile et dangereuse si le poids porte à un seul endroit alors qu’elle supportera un poids considérable si les charges sont également distribuées. Lorsque la baille américaine est revêtue de sa housse, la pirogue semble partir pour offrir aux baleines un énorme et prodigieux gâteau de mariage.\par
Chaque extrémité de la ligne est libre, l’une d’elles se termine par une boucle et remonte du fond de la baille pour pendre en dehors de celle-ci. Il est indispensable de disposer de cette façon l’extrémité inférieure pour deux raisons : d’abord pour pouvoir l’attacher plus facilement à la ligne d’une autre pirogue, au cas où le cétacé harponné sonde si profond qu’il menace d’emporter toute la longueur de la ligne fixée au harpon. En ces circonstances, la baleine est évidemment halée d’une pirogue à l’autre, un peu comme un pot de bière qu’on se passerait, quoique la première baleinière ne s’éloigne pas toujours, prête à seconder celle qui l’a remplacée. D’autre part, cet arrangement assure la sécurité commune car, si cette extrémité inférieure était amarrée à la pirogue et que la baleine vînt à entraîner la ligne en un instant fulgurant comme elle le fait parfois, les choses n’en resteraient pas là et la baleinière la suivrait dans les profondeurs de la mer, auquel cas aucun crieur des rues ne la retrouverait jamais.\par
Avant de mettre la pirogue à la mer pour l’attaque, la première extrémité de la ligne est tirée de la baille, glissée d’un demi-tour autour du taberin, amenée à travers toute la longueur de la baleinière et passée au-dessus des avirons de chaque canotier, à frôler leurs poignets quand ils nagent, elle file entre les hommes assis aux plats-bords opposés jusqu’à une engoujure pratiquée à la pointe de l’étrave, où un cabillot en bois de la grosseur d’une quille l’empêche de sortir de son logement. De là, elle festonne légèrement à l’extérieur de la proue, revient à l’intérieur de la pirogue, où quinze à vingt brasses de ligne sont lovées qui portent le nom de ligne de baille et, un peu plus à l’arrière du plat-bord, elle est attachée à une courte ligne fixée au harpon et coulisse au moyen d’une cosse sur la ligne principale, le détail des agencements de cette courte ligne serait fastidieux à décrire.\par
Ainsi la ligne à baleine enveloppe la pirogue de méandres compliqués évoluant dans presque toutes les directions. Les canotiers sont tous prisonniers de ce redoutable réseau de sorte qu’à l’œil timoré du terrien ils apparaissent comme ces charmeurs de serpents qui laissent ceux-ci s’enrouler folâtrement autour de leurs membres. Il n’y a pas un fils né de la femme qui puisse s’asseoir, pour la première fois, dans ce piège de chanvre, et tandis qu’il peine à l’aviron, sans penser qu’à un moment inconnu le harpon sera lancé, que ces anneaux souples entreront en action en sifflant comme l’éclair, sans qu’un frisson lui parcoure l’échine et que sa moelle épinière elle-même ne frémisse comme une gelée. Et pourtant l’habitude… étrange chose ! Que ne peut l’habitude ? L’acajou de votre salon n’a jamais entendu de plus joyeuses boutades, de rires plus francs, de meilleures plaisanteries, de plus vives reparties que le demi-pouce de cèdre blanc des bordages d’une pirogue ainsi suspendue dans le nœud coulant du bourreau et que, pareils aux six bourgeois de Calais devant le roi Édouard, les six hommes d’équipage rament dans les mâchoires de la mort, la corde au cou l’on peut dire.\par
Un instant de réflexion vous permettra de comprendre les malheurs répétés de la chasse – dont bien peu ont fait l’objet d’une relation –, les hommes arrachés à la pirogue par la ligne et perdus. Car lorsque la ligne file, être assis dans la baleinière reviendrait à se trouver dans le sifflement d’une machine à vapeur en pleine action, dont chaque balancier, chaque arbre, chaque rouage vous effleure. Et c’est pis encore car vous ne pouvez rester assis immobile au milieu de ces dangers, la pirogue se balançant comme un berceau et vous projetant d’un côté et de l’autre de façon inattendue. Seules une certaine souplesse, la volonté et l’action mises simultanément en œuvre peuvent vous épargner de devenir un Mazeppa et d’être expédié en des lieux où même le soleil qui voit tout ne vous découvrirait pas.\par
Et encore : l’apparence d’un calme profond qui précède l’orage est peut-être plus terrible que l’orage lui-même car, en vérité, ce calme n’est que la chrysalide qui enrobe la tempête ; comme le fusil, d’innocente allure, contient en lui la poudre fatale, la balle et l’explosion, de même l’aimable repos de la ligne, tandis qu’elle serpente silencieusement entre les canotiers avant d’entrer en action, est détenteur d’une terreur plus grande que quoi que ce soit d’autre dans cette périlleuse \hspace{1em}entreprise. Pourquoi en dire davantage ? L’humanité tout entière est cernée par une ligne à baleine. Tous les hommes naissent la corde au cou mais ce n’est qu’au moment où ils sont pris dans le tourbillon soudain et rapide de la mort qu’ils prennent conscience des dangers muets, subtils, toujours présents de la vie. Et si vous êtes un sage, l’effroi ne troublera pas davantage votre cœur si vous êtes assis dans une baleinière plutôt qu’au coin du feu avec votre tisonnier et non un harpon à vos côtés.
\chapterclose


\chapteropen
\chapter[{CHAPITRE LXI. Stubb tue un cachalot}]{CHAPITRE LXI \\
Stubb tue un cachalot}\renewcommand{\leftmark}{CHAPITRE LXI \\
Stubb tue un cachalot}


\chaptercont
\noindent Si l’apparition du calmar fut de mauvais augure, pour Starbuck, il n’en alla pas de même pour Queequeg.\par
– Quand vous voir lui camar, dit le sauvage en aiguisant son harpon sur le plat-bord de sa pirogue suspendue, vous bientôt voir lui perm-baleine.\par
Le jour suivant fut d’un calme écrasant. Rien ne sollicitant l’intérêt de l’équipage du {\itshape Péquod}, les hommes avaient peine à résister à l’envoûtement d’un sommeil distillé par une mer atone. Cette partie de l’océan Indien où nous croisions n’était pas ce que les baleiniers appellent un parage animé, c’est-à-dire qu’on y voit moins de marsouins, de dauphins, de poissons volants et de ces autres habitants pleins d’entrain peuplant des eaux plus agitées que celles du large du Rio de la Plata ou de la côte du Pérou.\par
Mon tour était venu d’être en vigie au mât de misaine et, les épaules appuyées contre les haubans relâchés du cacatois, je me balançais dans cet air enchanté. Aucun effort de volonté ne permettait d’y échapper et je perdis tout à fait conscience dans cette atmosphère de rêve, si bien que mon âme sortit de mon corps, sans qu’il ne cesse d’osciller comme le fait un pendule longtemps après que l’impulsion se soit immobilisée.\par
J’avais remarqué, avant d’être submergé par l’oubli total, que les hommes en vigie au grand mât et au mât d’artimon étaient déjà assoupis. De sorte que la mâture berçait trois êtres sans vie, chacun de leurs mouvements approuvés par un hochement de tête du timonier qui sommeillait à la barre. Les vagues, elles aussi, penchaient leurs crêtes engourdies et, à travers la vaste hypnose de la mer, l’est dodelinait vers l’ouest, et le soleil dormait dans l’espace.\par
Des bulles soudain semblèrent éclater sous mes paupières closes, mes doigts se refermèrent comme des étaux sur les haubans, quelque intermédiaire invisible et bienveillant m’avait soustrait au danger, un sursaut me rendit à la réalité. Que vis-je alors ? À moins de quarante brasses, sous notre vent, un cachalot gigantesque roulait dans l’eau comme la quille retournée d’une frégate ; le miroir de son vaste dos, d’un noir africain, brillait d’un éclat métallique aux rayons du soleil. Ondulant paresseusement au creux des vagues, lançant son jet à intervalles, le cachalot semblait un corpulent bourgeois en train de fumer sa pipe par une chaude après-midi. Pauvre baleine, ce fut ta dernière pipe ! Comme touchés par une baguette magique, le navire endormi et ses dormeurs s’éveillèrent tout soudain, plus de vingt voix jaillirent de toutes parts, faisant écho aux trois notes qui, du haut des mâts, égrenaient la formule consacrée, cependant que le grand poisson soufflait lentement et régulièrement son embrun scintillant.\par
– Dégagez les pirogues ! Lof ! cria Achab et, obéissant luimême à l’ordre qu’il venait de donner, il mit la barre dessous avant même que le timonier ait eu le temps de porter la main aux rayons de roue.\par
Les cris de l’équipage avaient dû alerter la baleine car, avant que les pirogues soient à la mer, elle vira majestueusement et se dirigea sous le vent mais avec une tranquillité tellement imperturbable, ridant à peine l’eau qu’Achab, se disant qu’après tout elle n’avait peut-être pas été inquiétée, donna l’ordre de ne pas se servir des avirons et de parler à voix basse ; de sorte que, pareils à des Indiens de l’Ontario sur les platsbords de la pirogue, nous avancions vivement mais silencieusement aux pagaies, le calme interdisant l’établissement de la voile. Bientôt, tandis que nous glissions à sa poursuite, le monstre leva perpendiculairement hors de l’eau sa queue de quarante pieds et disparut à la vue comme une tour engloutie.\par
– La voilà qui sonde ! Ce cri général annonçait un répit dont Stubb profita aussitôt pour allumer sa pipe. En temps voulu, la baleine émergea à nouveau et se trouva à l’avant de la pirogue du fumeur, de sorte que celui-ci, s’en trouvant plus proche que tous les autres, compta bien avoir l’honneur de sa capture. Il était évident à présent qu’elle se savait prise en chasse et toute précaution de silence devenait inutile ; les pagaies furent rentrées et les avirons entrèrent bruyamment en jeu. Tirant toujours sur sa pipe, Stubb encourageait ses hommes à l’assaut.\par
Oui, l’attitude du poisson avait changé du tout au tout. Sur le qui-vive, il allait la tête dressée de biais hors de la folle écume qu’il soulevait\footnote{Nous verrons plus loin quelle substance très légère contient la tête du cachalot. Malgré son apparence massive, c’est la partie de son corps qui flotte le mieux. Aussi n’a-t-il aucune peine à la tenir levée hors de l’eau, ce qu’il ne manque jamais de faire lorsqu’il nage à sa vitesse maximum. D’autre part, la largeur du sommet de son front est telle et s’amincit à tel point en taille-mer à sa partie inférieure, qu’en soulevant obliquement sa tête, on peut dire que la galiote paresseuse à l’avant renflé se transforme en fin bateau-pilote new-yorkais.}.\par
– Attrapez-la, attrapez-la, les gars ! Ne vous pressez pas, prenez votre temps, mais lancez-la comme le tonnerre, c’est tout, cria Stubb, postillonnant sa fumée en parlant. Allongez la nage, souquez fort, Tashtego. En avant, Tash, mon gars, mais restez froids, restez froids comme des concombres… comme ça, comme ça… allez-y comme la mort inexorable et les démons ricanants, et faites sortir les morts tout debout hors de la tombe, les gars ! Amenez dessus !\par
– Whoo-hoo ! Wa-hee ! s’écria en réponse le Gay-Header lançant au ciel quelque antique cri de guerre tandis que les hommes, dans la pirogue bondissante, étaient projetés en avant par le formidable coup d’aviron de l’ardent Indien.\par
Des cris non moins féroces répondaient aux siens.\par
– Kee-hee ! Kee-hee ! hurlait Daggoo, peinant d’avant en arrière sur son banc, pareil à un fauve en cage.\par
– Ka-la ! Koo-loo ! rugit Queequeg comme s’il claquait des lèvres devant quelque bouchée d’un bifteck de grenadier. Et les pirogues fendaient l’eau au bruit des cris et des avirons, cependant que Stubb, conservant sa place d’avant-garde, encourageait toujours ses hommes sans cesser d’émettre des nuages de fumée. Comme des forcenés, ils nageaient et peinaient de toutes leurs forces jusqu’à ce que le cri attendu jaillît :\par
– Debout, Tashtego ! Pique ! Le harpon fut lancé.\par
– Sciez !\par
Au moment où les canotiers faisaient culer la pirogue ils sentirent une chaleur passer en sifflant sur leurs poignets. C’était la ligne magique, à laquelle, l’instant d’avant, Stubb avait donné rapidement deux retours supplémentaires autour du taberin et qui, filant de plus en plus vite, dégageait une fumée bleue qui se mêlait à celle de sa pipe. Avant de passer autour du taberin, la ligne traversait de son éclair blessant les deux mains de Stubb qui avait laissé malencontreusement tomber ses manicles, sortes de carrés d’étoffe que les hommes portent parfois en pareil cas, et il était là comme un homme qui tiendrait à deux mains l’épée à double tranchant de son ennemi, un ennemi qui se serait débattu sans cesse pour l’arracher.\par
– Mouillez la ligne ! Mouillez la ligne ! cria Stubb au canotier de la baille (celui qui est assis près de la baille) qui, saisissant aussitôt son chapeau\footnote{Pour montrer à quel point cela est nécessaire, disons ici que les pêcheurs hollandais utilisaient un guipon pour mouiller la ligne tandis qu’elle filait. À bord de bien d’autres pirogues, un gamelot ou une sasse sont réservés à cet usage. Un chapeau, toutefois, est ce qu’il y a de plus pratique.}, y puisa de l’eau de la mer. La ligne fut encore halée de quelques tours, de sorte qu’elle se tendit ; la pirogue volait maintenant dans les bouillons d’écume comme un requin tout ailerons. Stubb et Tashtego changèrent de place, avant pour arrière, pérégrination titubante dans ce branle-bas.\par
La ligne vibrait sur toute sa longueur, aussi tendue qu’une corde de harpe et l’on eût dit que la pirogue avait deux quilles, l’une fendant l’eau, l’autre l’air, tandis qu’elle brassait ces deux éléments à la fois. Une cataracte jouait sans cesse à son étrave, un remous incessant tourbillonnait dans son sillage et au moindre mouvement, fût-ce un doigt remué, l’embarcation frémissante, craquante, pendait convulsivement ses plats-bords au ras de l’eau. Ainsi se déroulait cette course précipitée, chaque homme s’accrochant de toutes ses forces à son banc pour n’être pas jeté dans la mer écumante et la haute stature de Tashtego, à l’aviron de queue, se courbait en deux pour garder l’équilibre. Tous les Atlantiques, tous les Pacifiques semblaient frôlés par cette trajectoire de balle, jusqu’à ce qu’enfin la baleine ralentît un peu dans sa fuite.\par
– Amenez, amenez ! cria Stubb au canotier de l’avant et, tandis que l’on faisait front à la baleine, l’équipage entier amenait la pirogue qu’elle remorquait droit sur elle. Bientôt elle fut abordée et Stubb, calant fermement son genou dans l’évidement creusé à cet effet dans le tillac, darda encore et encore sa lance dans le poisson en fuite. Au commandement, la pirogue tantôt sciait pour s’écarter de la souille horrible de la baleine, tantôt se rapprochait pour porter un nouveau coup.\par
Des flots rouges ruisselaient maintenant aux flancs du monstre comme des ruisseaux jaillissant des collines. Il ne roulait plus dans l’écume, mais dans le sang, son corps à la torture, et jusqu’au loin le sillage n’était plus qu’un bouillonnement sanglant. Le soleil bas, illuminant cette mer cramoisie, posait sur le visage des hommes un masque de braise qui les transformait en Peaux-Rouges, cependant que l’évent de la bête à l’agonie laissait fuser à jets saccadés une vapeur blanche, au même rythme véhément des bouffées que le second enfiévré tirait de sa pipe. À chaque coup, il ramenait sa lance tordue, par la ligne qui lui était attachée, la redressait à coups rapides, encore et encore, sur le plat-bord, et encore et encore, il la replongeait dans la baleine.\par
– Embraquez ! Embraquez ! cria-t-il alors au fileur, lorsque la colère du cachalot affaibli désarma. Serrez… tout près ! Et la pirogue se rangea contre lui flanc à flanc. Penché à l’extrême à l’avant, Stubb fouilla lentement le poisson de sa longue lance aiguë, la maintenant profond, il la tournait et la retournait comme s’il cherchait méticuleusement quelque montre en or qu’elle eût avalée et qu’il eût craint de briser avant d’avoir pu l’extraire. Mais la montre d’or qu’il quêtait c’était le centre vital du poisson. Il fut atteint. Et la bête sortit brusquement de sa torpeur pour entrer dans cette phase inexprimable où l’on dit qu’elle « fleurit », se vautrant atrocement dans son sang, elle s’enveloppa de toutes parts d’un embrun furieux, impénétrable, bouillonnant, si bien que la pirogue en péril, reculant instantanément, eut bien à faire pour se dégager à l’aveugle de ce crépuscule délirant et retrouver la lumière du jour.\par
Son « flurry » s’affaiblit et la baleine réapparut à la vue, roulant d’un flanc sur l’autre, un spasme dilatant et contractant son évent dans un râle d’agonie aigu et fêlé. Enfin, après avoir laissé fuser, flot par flot, vers le ciel épouvanté, les caillots de son sang, pareil à la lie épaisse d’un vin pourpre, ils retombèrent et ruisselèrent sur son flanc immobile. Son cœur avait éclaté !\par
– Elle est morte, monsieur Stubb, dit Daggoo.\par
– Oui… nos deux pipes sont éteintes ! Et retirant la sienne de sa bouche, Stubb en secoua les cendres sur la mer et, pendant un moment, resta à contempler pensivement l’immense cadavre qu’il venait de faire.
\chapterclose


\chapteropen
\chapter[{CHAPITRE LXII. Le dard}]{CHAPITRE LXII \\
Le dard}\renewcommand{\leftmark}{CHAPITRE LXII \\
Le dard}


\chaptercont
\noindent Un mot au sujet d’un épisode du chapitre précédent. Selon l’usage immuable de la pêcherie, la baleinière déborde du navire avec son chef de pirogue, ou tueur de baleines, comme timonier temporaire, son harponneur, ou piqueur de baleines, nageant à l’aviron de pointe, appelé aviron du harponneur. Il faut un bras énergique et puissant pour lancer le premier fer dans le poisson, car souvent ce qu’on appelle le dard long, fer pesant, doit être lancé à une distance de vingt ou trente pieds. Pourtant, quelque longue et épuisante que puisse être la poursuite, on attend du harponneur qu’il ne cesse de tirer sur l’aviron pendant toute sa durée et, en vérité, on attend même de lui qu’il donne aux autres l’exemple d’une activité surhumaine, non seulement par son effort incroyable à l’aviron, mais encore en poussant des exclamations retentissantes et répétées d’encouragement. Ce que signifie crier continuellement à tue-tête lorsque tous les muscles sont tendus à craquer, nul ne le sait qui n’en a point fait l’expérience. Le tout premier je suis incapable de brailler de bon cœur et de travailler à corps perdu en même temps. C’est alors que, s’époumonnant et exténué, tournant le dos au gibier, le harponneur entend tout à coup le cri excitant : « Debout ! Pique ! » Il lui faut aussitôt laisser tomber son aviron, l’assurer, se tourner à demi, s’emparer de son harpon en le dégageant de la fourche et, avec le peu de force dont il dispose encore, essayer de le lancer tant bien que mal dans la baleine. Il n’y a rien d’étonnant, dès lors, à ce que, si l’on prend l’ensemble d’une flotte baleinière, sur cinquante occasions offertes de piquer, il n’y en ait pas cinq qui réussissent, rien d’étonnant à ce que tant de malheureux harponneurs soient voués aux malédictions et honnis, rien d’étonnant à ce que tant d’entre eux se fassent sauter les vaisseaux sanguins dans l’effort, rien d’étonnant à ce que des cachalotiers ne fassent que quatre barils d’huile en quatre ans, rien d’étonnant à ce que l’industrie baleinière soit une mauvaise affaire pour de nombreux armateurs, car c’est du harponneur que dépend le succès du voyage, et si on le prive de son souffle, comment attendre qu’il le retrouve au moment où le besoin en est le plus impérieux.\par
Si la baleine est par chance harponnée, l’instant qui suit est également critique, tandis qu’elle commence à fuir, le chef de pirogue et le harponneur commencent eux aussi à courir de l’avant à l’arrière, menacés par le danger et menace pour les autres. Car c’est à ce moment-là qu’ils changent de place, le chef de pirogue prenant son poste à l’étrave.\par
Il m’est indifférent qu’on vienne me soutenir le contraire, mais je prétends que cela est à la fois idiot et superflu. Le chef devrait être à l’avant depuis le début, il devrait lancer et le harpon et la lance, et ne devrait pas avoir à ramer du tout, sauf lors de circonstances particulières où tout pêcheur sait ce qu’il a à faire. Je sais que cela entraînerait parfois une légère perte de vitesse dans la chasse, mais une longue expérience des baleiniers de plus d’une nation m’a convaincu que la majorité des échecs de la pêche est due moins à la rapidité de la baleine qu’à l’épuisement du harponneur.\par
Pour avoir un maximum d’efficacité, les harponneurs de ce monde devraient bondir sur leurs pieds au sortir de l’oisiveté et non d’un dur effort.
\chapterclose


\chapteropen
\chapter[{CHAPITRE LXIII. La fourche}]{CHAPITRE LXIII \\
La fourche}\renewcommand{\leftmark}{CHAPITRE LXIII \\
La fourche}


\chaptercont
\noindent Du tronc croissent les branches et des branches les rameaux. De même les sujets fertiles font croître les chapitres.\par
La fourche mentionnée dans une page précédente mérite une description. C’est une tige de bois à encoche, de forme particulière, de quelque deux pieds de haut, encastrée verticalement dans le plat-bord de la pirogue, près de l’étrave, de manière à fournir un support au manche du harpon, dont la lame ressort sur l’avant afin qu’il soit aussi facile à saisir qu’un fusil pendu au mur pour le coureur des bois. Il est d’usage que deux harpons reposent sur la fourche, respectivement appelés premier et deuxième fers.\par
L’extrémité de la ligne est frappée sur l’estrope de ces deux harpons afin qu’ils puissent, dans la mesure du possible, être lancés presque instantanément l’un après l’autre dans la même baleine, de sorte qu’à la première résistance offerte, l’un reste ancré si l’autre cède. Les chances sont ainsi doublées. Mais il arrive souvent qu’ayant reçu son premier fer, la baleine fuie aussitôt avec de violents soubresauts, rendant impossible au harponneur, pourtant prompt comme l’éclair dans ses mouvements, de lancer son second fer. Toutefois, le second harpon étant déjà relié à la ligne et celle-ci se mettant à filer, il faut à tout prix, que d’une façon ou d’une autre, l’arme soit jetée, quelque part et de quelque manière que ce soit, hors de la baleinière, afin de préserver l’équipage du danger qu’il représente. En ce cas, il est simplement lancé à l’eau. La plupart du \hspace{1em} temps, cette mesure de prudence est réalisable grâce à ce qu’il reste de ligne en réserve dans la baille et dont il a été question au chapitre précédent. Mais cette action critique n’est pas toujours accomplie sans accidents des plus graves, voire mortels.\par
Qui plus est, il faut savoir que, lorsque ce second harpon a été jeté par-dessus bord, il devient une épée de Damoclès, terreur acérée qui se balance fantasquement entre la pirogue et le gibier, emmêlant les lignes ou les tranchant, semant l’émotion dans toutes les directions. En général, il est impossible de s’en rendre maître à nouveau avant que la baleine soit morte.\par
Songez, dès lors, à ce qu’est la situation lorsque quatre pirogues sont engagées ensemble sur un poisson particulièrement fort, agile et rusé, lorsque, outre ces dangers provoqués par ses qualités mêmes ajoutés à ceux que comporte cette entreprise audacieuse, huit ou dix seconds fers viennent à tournoyer autour de la bête. Car, bien entendu, chaque pirogue a plusieurs harpons fixés sur la ligne pour le cas où le premier, lancé inefficacement, ne pourrait être récupéré. Tous ces détails sont fidèlement rapportés ici, car ils ne manqueront pas d’éclaircir plusieurs passages très importants et compliqués des scènes suivantes.
\chapterclose


\chapteropen
\chapter[{CHAPITRE LXIV. Le souper de Stubb}]{CHAPITRE LXIV \\
Le souper de Stubb}\renewcommand{\leftmark}{CHAPITRE LXIV \\
Le souper de Stubb}


\chaptercont
\noindent Le cachalot de Stubb avait été tué à bonne distance du navire. Nous étions dans un calme, de sorte qu’attelant les trois pirogues en flèche, nous entreprîmes le lent travail de remorquer le trophée jusqu’au {\itshape Péquod}. Et tandis que les dix-huit hommes que nous étions avec nos trente-six bras, nos cent quatre-vingts doigts, pouces y compris, nous peinions heure après heure à tirer ce corps inerte et pesant qui semblait à peine bouger, nous eûmes ainsi une excellente preuve de l’énormité de la masse que nous déplacions. Sur le grand canal de Hanachow, quel que soit le nom qu’on lui donne, quatre ou cinq hommes, sur le sentier de halage, tireront une jonque lourdement frétée à la vitesse d’un mille à l’heure, mais cette grandiose caraque que nous remorquions lourdement semblait lestée de saumons de plomb.\par
La nuit tomba, mais trois feux dans le gréement du {\itshape Péquod} nous indiquaient faiblement notre route et en nous rapprochant nous vîmes Achab élever hors de la rambarde l’un des falots supplémentaires. Il regarda un instant d’un air absent la baleine qui flottait, donna les ordres habituels d’arrimage pour la nuit puis, tendant son falot à un matelot, il se dirigea vers sa cabine et ne reparut plus jusqu’au matin.\par
Bien que le capitaine Achab, en surveillant la chasse de ce cachalot, eût fait preuve de son activité habituelle, si l’on peut dire, maintenant qu’il était mort, quelque insatisfaction vague, ou quelque impatience, ou le désespoir semblaient le \hspace{1em}tenailler, comme si la vue de ce cadavre lui rappelait que Moby Dick était encore à mettre à mort et que quand bien même des milliers d’autres baleines seraient remorquées jusqu’à son navire, cela ne le distrairait en rien de son obsession unique. Bientôt on aurait pu croire, au bruit qui régnait sur le pont du {\itshape Péquod}, que l’équipage s’apprêtait à mouiller l’ancre en plein Océan. De lourdes chaînes étaient halées sur le pont et jaillissaient des sabords de charge avec fracas. Mais c’était l’immense cadavre, et non le navire, que ces chaînes cliquetantes amarraient. Liée par la tête à la poupe, par la queue à la proue, la baleine appuyait sa coque noire contre celle du navire et, entrevue dans les ténèbres qui dissimulaient la mâture et le gréement sur le ciel, on eût dit un attelage de bœufs géants dont l’un se serait couché et dont l’autre serait resté debout\footnote{Notons un petit détail : la prise la plus solide et la plus sûre que le cétacé offre au navire lorsque celui-ci est amarré à son flanc, c’est sa queue. Étant donné que la caudale est la partie la plus musclée du corps, c’est aussi la plus lourde, les nageoires pectorales exceptées. Sa souplesse, qu’elle conserve dans la mort, l’entraîne bien au-dessous de la surface de sorte qu’on ne peut pas la capeler depuis la pirogue, étant donné qu’il est impossible de le faire à la main. Cette difficulté a été résolue avec ingéniosité : une petite ligne résistante munie d’un flotteur de bois à un bout, d’un poids en son milieu, est assurée au navire à son autre extrémité. Le flotteur est adroitement amené de l’autre côté de la masse de manière à encercler le cétacé, la chaîne suit facilement et, glissée le long du corps, elle est amarrée à la partie la plus étroite de la caudale, à l’endroit où elle diverge en larges palmes ou ailerons.}.\par
Si le sombre Achab, pour autant qu’on pouvait le présumer, était maintenant au repos, Stubb, son second, enfiévré par sa victoire, montrait une animation inhabituelle mais toujours empreinte de bonhomie. Il déployait une activité si peu coutumière que son supérieur, le posé Starbuck, lui abandonna pour le moment avec tranquillité la direction des opérations. Une petite raison stimulait la pétulance de Stubb et se révéla bientôt étrangement. Amateur de bonne chère, Stubb avait une prédilection immodérée pour la baleine considérée comme un mets savoureux.\par
– Une tranche, une tranche, avant que j’aille dormir ! Vous, Daggoo !… par-dessus la rambarde et coupez-m’en une dans le small !\par
Sachez qu’en général ces sauvages pêcheurs ne font pas, selon l’usage militaire, payer à l’ennemi les frais de la guerre (du moins pas avant la réalisation des bénéfices de la campagne), mais de temps à autre certains de ces Nantuckais font leurs délices de ce morceau du cachalot désigné par Stubb et qui se trouve à l’extrémité fuselée du corps.\par
Vers minuit, ce steak était coupé et, à la lueur de deux falots où brûlait de l’huile de cachalot, Stubb prit vaillamment son repas de cachalot debout au chapeau de cabestan comme si c’eût été un buffet. Il ne fut pas le seul à festoyer de chair de baleine cette nuit-là. Mêlant leurs mâchonnements à sa mastication, des milliers de requins, en essaims autour du léviathan mort, faisaient entendre les clappements du banquet qu’ils faisaient de sa graisse. En bas, les rares hommes au repos sursautaient souvent aux vigoureux claquements de leurs queues frappant la coque du navire à quelques pouces de leurs cœurs de dormeurs. En se penchant par-dessus bord on pouvait les voir (de même qu’auparavant on les avait entendus), se vautrant dans les eaux mornes et noires, se tournant sur le dos pour creuser dans la baleine des trous ronds de la grosseur d’une tête d’homme. Cet exploit du requin semble quasiment miraculeux ; comment arrive-t-il, dans une surface apparemment si inattaquable, à arracher des bouchées aussi symétriques ? Cela fait partie de l’universel mystère. La marque qu’il laisse ainsi sur la baleine ne saurait être mieux comparée qu’au forage que fait un menuisier pour noyer la tête d’une vis.\par
Bien que dans la fumée et la diabolique horreur d’un combat naval, on puisse voir les requins fixer avec convoitise le pont des navires, – tels des chiens affamés autour de la table sur laquelle on découpe une viande saignante, – prêts à avaler avidement tout homme tué jeté jusqu’à eux, et bien que sur le pont les vaillants bouchers tranchent dans la chair humaine comme des cannibales, avec des lames dorées et parées de glands, les requins, eux aussi, avec leurs mâchoires ornant la garde de leurs poignards se disputent sous la table, taillant dans la chair morte, et bien que vous retourniez ce problème en tous sens, il reste que le résultat est joliment le même, c’est-à-dire que c’est une scandaleuse affaire de requins de part et d’autre. Les requins sont aussi les piqueurs immanquables de tous les négriers qui traversent l’Atlantique, trottant systématiquement à leurs côtés afin d’être disponibles dans le cas où il y aurait un colis à transporter où que ce soit ou un esclave à ensevelir décemment. Et bien qu’on puisse donner encore un ou deux exemples, relatifs au temps, au lieu et aux circonstances où les requins se réunissent fort amicalement et festoyent on ne peut plus joyeusement, il n’est cependant pas de moment ni de circonstance imaginables où on les trouvera en société aussi innombrable, animée d’une plus joviale gaieté qu’autour d’un cachalot mort, amarré de nuit au flanc d’un baleinier en pleine mer. Si vous n’avez jamais contemplé pareil spectacle, alors réservez votre jugement quant à l’opportunité d’adorer le diable et à l’à-propos de se le concilier.\par
Mais de même que jusqu’alors Stubb n’avait pas prêté attention au bruit de mâchoires du banquet qui avait lieu si près de lui, de même les requins n’avaient pas remarqué qu’il faisait claquer ses lèvres gourmandes :\par
– Coq ! Coq ! – où est ce vieux Toison ? cria-t-il enfin, en écartant encore un peu les jambes comme pour assurer une base plus solide à son repas, et piquant sa fourchette dans le plat comme s’il usait de sa lance. Coq ! Hé, coq ! Mettez le cap vers ici, coq !\par
Le vieux Noir qui ne jubilait pas précisément d’avoir déjà été tiré de la chaleur de son hamac à une heure aussi indue, se traîna hors de sa cuisine, péniblement car comme chez beaucoup de vieux Noirs, quelque chose clochait du côté de ses articulations qu’il ne pouvait astiquer aussi bien que ses casseroles ; ce vieux Toison, comme on l’appelait, arriva, traînant les pieds et boitant, assurant son pas avec ses pincettes grossièrement façonnées dans des cercles de barils redressés, ce vieil ébène trébucha de l’avant et, au commandement, s’arrêta net en face de Stubb, de l’autre côté de son buffet. Les deux mains jointes devant lui, appuyé sur sa canne double, il pencha un peu plus avant son dos voûté en inclinant sa tête de côté de façon à tendre sa meilleure oreille.\par
– Coq, dit Stubb, en portant vivement à sa bouche un morceau de viande plutôt rouge, coq, ne trouvez-vous pas que ce steak est un peu trop cuit ? Et vous l’avez trop battu, coq, il est trop tendre. N’ai-je pas toujours dit que pour être bonne la viande de baleine doit être coriace ? Ces requins, tout près, ne voyez-vous pas qu’ils la préfèrent dure et saignante ? Quel vacarme ils mènent ! Coq, allez leur dire deux mots, informez-les qu’on leur permet volontiers de se servir poliment et modérément, mais qu’ils ne fassent pas de bruit. Du diable si je m’entends parler. Allez, coq, et transmettez-leur mon message. Tenez, prenez ce falot, allez et faites-leur un sermon !\par
Maussade, le vieux Toison prit la lanterne et boitilla à travers le pont jusqu’aux pavois, puis abaissant d’une main sa lumière autant que possible au-dessus de l’eau, de façon à bien voir ses ouailles, de l’autre il brandit ses pincettes et se penchant largement par-dessus bord, il commença à haranguer les requins en marmottant, cependant que Stubb, se glissant silencieusement derrière lui, écoutait ce qu’il disait.\par
– Mes fè’es, m’vlà ave l’ode de vous di que vous devez ahéter ce damné buit là en bas. Vous entendez ? Ahétez ce damné claquement de lèves ! Missié Stubb dit que vous pouvez empi vos damnées panses jusqu’aux écoutilles, mais pa Dieu ! ahétez ce damné tintama !\par
À ces mots Stubb intervint en lui posant la main sur l’épaule.\par
– Coq ! Eh bien ! du diable ! Vous ne devez pas jurer de la sorte en prêchant. Ce n’est pas le bon moyen de convertir des pécheurs, coq !\par
– Qui est ? Alo, faites-y un semon vous-même, répondit-il renfrogné en se détournant pour partir.\par
– Non, coq, continuez, continuez !\par
– Non alo, fè’es benaimés…\par
– Très bien, approuva Stubb, amadouez-les ainsi, essayez ce moyen. Et Toison poursuivit :\par
– Pa natu vous autes equins êtes tes voaces, mais ze vous le dis, mes fè’es, que cet’voacité… ahétez ce damné claquement de queues ! Comment pouvez-vous entende si vous continuez ce damné vacame de mâchoi ?\par
– Coq, dit Stubb en l’empoignant au collet, je ne veux pas de ces jurons. Faites-leur une adresse courtoise.\par
– Vot voacité, fè’es, ze ne vous en blâme pas tant, c’est la natu et on y peut yen, mais contôler cette natu pevesse, ça c’est le bu. Su et cetain, vous êtes des equins, mais si vous épimez le equin en vous, vous seez des anzes, car tous les anzes ne sont que des equins qui se sont maîtisés. Allons, mes fè’es, essayez d’ête polis en vous sevant de cette baleine. N’ahachez pas les moceaux de gaisse de la bouce de vot voisin, ze vous dis. Est-ce qu’un equin n’en vaut pas un aut pou cet’baleine ? Et, pa Dieu, aucun de vous n’a de doit su cet’baleine. Cet’baleine appatient à quequ’un d’aut. Je sais que cetains d’ente vous ont des tes gandes gueules, plus gandes que les autes, mais des fois à gandes gueules petits ventes, aussi la gandeur de la gueule est pas pour s’empifTer mais pour pépaéer des moceaux pou les petits equins qui peuvent pas se sévi eux-mêmes.\par
– Bien dit, vieux Toison ! Ça, c’est vraiment chrétien, continuez.\par
– Inutile continuer, ces damnés scéléhats continuont à chapader et à se batte ente eux, missié Stubb, y n’écoutent pas un mot ! Ça se à yen de fé un semon à des maudits gloutons comme vous les avez appelés comme ça, zusque leu ventes soient pleins et leu ventes est sans fond et quand y zahivent à le empli, y vous écoutehons pas enco pace qu’alo y sombent zusqu’au fond de la mé, y s’endoment pofondément su le cohail et peuvent pu yen entende du tout, pu du tout pou touzou et à zamais.\par
– Sur mon âme, je suis à peu près du même avis, alors, donnez-leur votre bénédiction, Toison et je vais retourner à mon souper.\par
Alors Toison étendit ses deux mains sur la foule des poissons, éleva sa voix suraiguë et s’écria :\par
– Maudits fè’es ! Faites autant de vacame que vous pouvez, emplissez vos damnés ventes zusqu’à ce qui zéclatent et puis mouez !\par
– Maintenant, coq, dit Stubb, revenu à son souper au cabestan. Remettez-vous où vous étiez tout à l’heure juste en face de moi et prêtez une attention toute particulière à ce que je vais vous dire.\par
– Tout’o’eilles, répondit Toison, reprenant la position voulue, penché sur ses pincettes.\par
– Eh bien, dit Stubb, tout en se servant copieusement, je m’en vais en revenir au sujet de ce steak. D’abord, quel âge avez-vous, coq ?\par
– Qu’est ça affai av’le steak ? demanda le vieux Noir avec humeur.\par
– Silence ! Quel âge avez-vous, coq ?\par
– À peu pès quat’vingt-dix, y disent, grommela-t-il sombrement.\par
– Et vous avez vécu en ce monde près de cent ans, coq, et vous ne savez pas encore faire cuire une tranche de cachalot ? dit Stubb en engloutissant à ce mot une autre bouchée de sorte qu’elle parut être la continuation de la question.\par
– Où êtes-vous né, coq ?\par
– Déhiè une écoutille, dans un bac, su le’Oanoke.\par
– Né sur un bac. Voilà qui est curieux. Mais je voudrais savoir dans quel pays vous êtes né, coq.\par
– Est-ce que je n’ai pas dit le pays de’Ooanoke ? répondit-il sèchement.\par
– Non, coq, vous ne l’avez pas dit, mais moi je vais vous dire où je veux en venir. Il vous faut retourner dans votre pays et naître à nouveau, vous ne savez pas encore faire cuire une tranche de cachalot.\par
– Dieu ait mon âme si zen fais cui une aut’, gronda-t-il avec colère, se détournant pour partir.\par
– Revenez, coq. Là, donnez-moi ces pincettes ; maintenant goûtez ce bout de viande et dites-moi si vous trouvez qu’il est cuit comme il devrait l’être. Prenez, vous dis-je, ajouta-t-il en lui tendant les pincettes. Prenez-le et goûtez !\par
Serrant faiblement ses lèvres fanées sur le morceau, le vieux nègre marmotta : « Le steak le mieux cuit que zai zamais goûté, zuteux, tès zuteux. »\par
– Coq, poursuivit Stubb, reprenant sa position de combat, appartenez-vous à l’Église ?\par
– Passé devant une à Cape-Town, dit le vieil homme boudeur.\par
– Ainsi vous avez passé une fois dans votre vie devant une sainte église à Cape-Town, où vous avez sûrement entendu un saint pasteur s’adresser à ses auditeurs comme à des frères bien-aimés, n’est-ce pas, coq ? Et pourquoi vous voilà me racontant un affreux mensonge. Où pensez-vous que vous irez, coq ?\par
– Domi bientôt, marmotta-t-il, à demi détourné.\par
– Baste ! Virez ! Je veux dire quand vous mourrez, coq. C’est une question atroce. Et quelle est votre réponse ?\par
– Losque ce vieux homme noici moua, répondit lentement le nègre, en changeant tout à fait d’attitude et de ton, de luimême y ia nulle pat, mais un anze béni vienda le checher.\par
– Le chercher ? Comment ? Dans un carrosse à quatre chevaux, comme on est venu chercher Élie ? Et où le chercher ?\par
– Là-haut, dit Toison en élevant ses pincettes droit audessus de sa tête et en les y maintenant avec solennité.\par
– Ainsi, quand vous serez mort, vous espérez monter dans notre grand-hune, vraiment, coq ? Mais ne savez-vous pas que plus on monte plus il fait froid ? Dans la grand-hune, hein ?\par
– Zamais dit ça, répondit Toison, à nouveau boudeur.\par
– Vous avez bien dit là-haut, n’est-ce pas ? Regardez bien où vous pointez vos pincettes. Mais peut-être que vous espérez aller au ciel en grimpant par le trou du chat, coq, mais non, non vous n’y arriverez pas sans passer par la voie régulière qui contourne le gréement. C’est une affaire scabreuse, mais il faut en passer par là, il n’y a pas d’autre moyen. Mais nul d’entre nous n’est encore au ciel. Abaissez vos pincettes et écoutez mes ordres, coq. Vous m’entendez ? Prenez votre chapeau dans une main et serrez l’autre sur votre cœur lorsque je donne des ordres, coq ! Comment ! C’est là qu’est votre cœur ? Ça, c’est votre gésier. Plus haut ! plus haut ! ça y est, vous y êtes. Serrez-la bien, à présent et soyez toute attention !\par
– Tout’o’eilles, répondit le vieux Noir les deux mains dans la position demandée et tordant sa tête grise comme pour amener vainement ses deux oreilles en avant.\par
– Eh bien, coq, vous voyez, votre fameux steak de baleine était si affreusement mauvais que je l’ai fait disparaître au plus vite, vous le constatez, n’est-ce pas ? Alors, à l’avenir quand vous cuirez une autre tranche pour ma table personnelle, ici sur le cabestan, je vais vous indiquer comment vous y prendre pour qu’elle ne soit pas trop cuite. Tenez la tranche d’une main et de l’autre montrez-lui un tison ardent, ceci fait, elle est prête servez-la. Vous m’entendez ? Et à présent, quand demain nous ferons le dépeçage du gibier, ne manquez pas d’être prêt à prendre l’extrémité des nageoires et mettez-les mariner. Quant aux extrémités de la queue, mettez-les en saumure, coq. C’est tout, vous pouvez aller.\par
Mais Toison n’avait pas fait trois pas qu’il était rappelé :\par
– Coq, demain au quart de minuit, servez-moi des croquettes pour mon souper. Vous m’avez entendu ? Alors, mettez les voiles ! Holà, un instant ! Faites une courbette avant de partir ! Un instant ! Et des boulettes de baleine pour le petit déjeuner, n’oubliez pas !\par
– Pa Dieu ! je voudais que la baleine le manze plutôt que lui manze la baleine. Ze veux ben ête pendu si n’est pas pu equin que missié Equin lui-même, grommela le vieil homme en boitant et sur cette sage pensée il regagna son hamac.
\chapterclose


\chapteropen
\chapter[{CHAPITRE LXV. La baleine en tant que mets}]{CHAPITRE LXV \\
La baleine en tant que mets}\renewcommand{\leftmark}{CHAPITRE LXV \\
La baleine en tant que mets}


\chaptercont
\noindent Qu’un mortel se nourrisse de l’animal qui alimente sa lampe et que, comme Stubb il le mange à la lumière qu’il donne, pour ainsi dire, paraît chose si incongrue qu’il convient d’entrer un peu dans l’histoire et la philosophie de la question.\par
Selon la chronique, il y a trois siècles, la langue de la baleine franche était considérée en France comme un mets délicat et elle y atteignait des prix élevés. De même, au temps d’Henri VIII, un certain cuisinier de la cour obtint une belle récompense pour la création d’une sauce merveilleuse destinée à accompagner, rôtis tout entiers, des marsouins, lesquels, vous vous en souvenez, sont des espèces de baleines. En vérité, de nos jours encore, le marsouin est estimé. On en fait des croquettes rondes, à peu près de la dimension de boules de billard et, bien assaisonnées et épicées, on les prendrait pour un hachis de tortue ou de veau. Les anciens moines de Dunfermline en étaient très friands. La Couronne leur accorda une grande concession de marsouins.\par
Le fait est que, aux yeux de leurs chasseurs tout au moins, les cétacés seraient considérés comme un plat noble, n’était son excessive abondance. Lorsque vous vous asseyez devant un pâté de viande de près de cent pieds de long, cela vous coupe l’appétit. Seuls des hommes absolument détachés de tout préjugé, comme Stubb, mangent de la baleine cuite, mais les Esquimaux ne sont pas si difficiles. Nous savons tous qu’ils tirent leur subsistance de la baleine et qu’ils ont de vieux crus précieux thran. Zograndan, l’un de leurs plus célèbres médecins, préconise, pour les enfants, le lard de baleine comme étant extrêmement juteux et nourrissant. Cela me remémore que certains Anglais, accidentellement abandonnés il y a longtemps au Groenland par un navire baleinier, vécurent, des mois durant, de déchets moisis de baleine qui avaient été laissés à terre après la fonte de la graisse. Les pêcheurs hollandais les appellent des beignets, auxquels, en vérité, ils ressemblent fort, étant dorés et croustillants et, lorsqu’ils sont frais, leur odeur rappelle celle des pets-de-nonne que font les ménagères d’Amsterdam. Ils sont si appétissants que l’étranger le plus frugal a peine à se retenir d’y toucher.\par
Mais ce qui déprécie le plus la baleine dans notre cuisine de civilisés, c’est qu’elle est une nourriture trop riche. Elle est l’énorme bœuf gras de l’Océan, trop gras pour être délicieux. Prenez sa bosse par exemple, qui pourrait être un plat aussi excellent que l’est celle du bison (mets rare et recherché), n’était la pyramide de graisse consistante qu’elle forme. Le spermaceti même, si suave et crémeux, pareil à la chair transparente et à demi solide d’une noix de coco au troisième mois de sa croissance, est bien trop riche pour remplacer le beurre. Néanmoins, bien des baleiniers en mangent en le mêlant à quelque autre aliment. Lors des longs quarts de nuit, au cours de la fonte, les matelots plongent, fréquemment, leurs biscuits de mer dans les énormes chaudrons de graisse et les y laissent frire un moment. J’ai ainsi fait plus d’un bon souper.\par
En ce qui concerne un petit cachalot, sa cervelle est considérée comme un plat fin. La boîte crânienne est ouverte à la hache et l’on en retire les deux lobes dodus et blanchâtres (ressemblant très exactement à deux gros puddings), on les mélange alors à de la farine, on les cuit et c’est un mets tout à fait délicieux, dont le goût rappelle celui de la tête de veau appréciée de certains gastronomes. Et tout un chacun sait que quelques jeunes dandies parmi les gourmets, en se nourrissant sans cesse de cervelles de veau, finissent par acquérir assez de cervelle euxmêmes pour distinguer leurs propres têtes d’une tête de veau, la distinction réclamant d’ailleurs une perspicacité peu commune. C’est pourquoi un jeune dandy, attablé devant une tête de veau qui a l’air intelligente, est l’un des plus tristes spectacles que l’on puisse voir. Cette tête le contemple avec une sorte de reproche et une expression signifiant un « {\itshape Tu quoque, Brutus} ».\par
Ce n’est peut-être pas seulement à cause de sa trop grande onctuosité que les terriens paraissent avoir de l’aversion à consommer de la baleine, il semble que ce soit une conséquence des raisons dont nous avons précédemment parlé, à savoir cette obligation pour un homme de consommer une créature marine fraîchement assassinée à la lumière qu’elle lui fournit. Il ne fait pas de doute que le premier homme qui tua un bœuf fut considéré comme un meurtrier, peut-être fut-il pendu, et s’il avait passé en jugement devant un tribunal bovin, il l’aurait certainement été, et si un assassin mérite ce sort, il coule de source qu’il le mérite aussi. Allez aux halles des viandes un samedi soir et regardez la foule de ces bipèdes vivants regardant fixement les longs alignements de quadrupèdes morts. Cette vision n’enlève-t-elle pas une dent à la mâchoire du cannibale ? Les cannibales ? Qui n’est pas un cannibale ? Je vous le dis : au jour du Jugement, le Fidjien qui a mis au saloir un maigre missionnaire pour les jours de famine, ce prévoyant Fidjien suscitera plus d’indulgence que toi, gourmet conscient et civilisé, qui a cloué au sol des oies et festoyé de leurs foies hypertrophiés sous forme de pâtés.\par
Mais Stubb, lui, il mange de la baleine à la lumière de l’huile de baleine, n’est-ce pas ? Et c’est ajouter l’affront au tort, n’est-ce pas ? Regardez un peu le manche de votre couteau, vous mon gourmet cultivé et civilisé, tandis que vous coupez votre rosbif, en quoi ce manche est-il fait ? En quoi, sinon avec les os du frère du même bœuf que vous mangez ? Et avec quoi vous curez-vous les dents après avoir dévoré cette oie grasse ? Avec une plume de cette même volaille. Et avec quelle plume le secrétaire de la Société pour la répression de la cruauté envers les oies blanches a-t-il rédigé ses premières circulaires ? Il n’y a guère qu’un mois ou deux que cette société a délibéré pour accorder sa faveur aux seules plumes d’acier.
\chapterclose


\chapteropen
\chapter[{CHAPITRE LXVI. Le massacre des requins}]{CHAPITRE LXVI \\
Le massacre des requins}\renewcommand{\leftmark}{CHAPITRE LXVI \\
Le massacre des requins}


\chaptercont
\noindent Lorsque, dans la pêcherie du Sud, après une chasse longue et fatigante, un cachalot est ramené au navire tard dans la nuit, il n’est pas d’usage, du moins dans des circonstances ordinaires, de procéder aussitôt au dépècement. Car c’est un travail pénible et long qui réclame la collaboration de l’équipage entier. Dès lors la coutume veut qu’habituellement on cargue toutes les voiles, qu’on fixe la barre sous le vent puis qu’on envoie chacun à son hamac jusqu’à la pointe du jour, à cette réserve près que la garde au mouillage est maintenue, c’est-à-dire qu’à tour de rôle deux fois deux hommes d’équipage doivent veiller pendant une heure à ce que tout aille bien sur le pont.\par
Mais parfois, surtout sur la ligne dans le Pacifique, cette mesure est tout à fait insuffisante parce que de telles légions de requins s’attaquent au cadavre amarré que si on le laissait, disons six heures d’affilée sans surveillance, on ne trouverait plus guère au matin qu’un squelette. Cependant, presque partout ailleurs dans les autres océans, où ces poissons pullulent moins, on peut rabattre leur prodigieuse voracité en les chassant énergiquement avec de tranchantes pelles à baleines, procédé qui, toutefois, en certains cas, semble seulement stimuler davantage leur activité. Tel n’était pas le cas actuellement pour les requins du {\itshape Péquod}, quoique assurément tout homme à qui ce spectacle n’eût pas été familier, aurait pu croire que toute la mer environnante n’était qu’un vaste fromage dont ces requins eussent été les asticots.\par
Cependant, après que Stubb eut désigné la garde au mouillage, et terminé son souper, après qu’ensemble Queequeg et un matelot furent montés sur le pont, ce ne fut pas un mince remue-ménage, qui fut déclenché parmi les requins car les deux hommes, ayant descendu hors des pavois les établis de dépeçage et suspendu trois falots de manière à ce qu’ils projettent sur la mer troublée de longs rais de lumière, armés de leurs longues pelles\footnote{La pelle utilisée pour le dépeçage de la baleine est du meilleur acier, elle a à peu près la dimension d’une main ouverte et la forme d’une bêche de jardinage, avec cette seule différence qu’elle est tout à fait plate et sensiblement plus étroite à son sommet, On la maintient toujours parfaitement aiguisée et, en cours de travail, on l’affûte de temps à autre comme un rasoir. Elle est emmanchée à une hampe de bois de vingt à trente pieds.}, ils procédèrent à un massacre systématique des requins en leur assenant sur le crâne, apparemment le seul endroit où l’on ait une chance de les atteindre mortellement, des coups du tranchant effilé du louchet. Mais, dans la confusion de ce grouillement en lutte et dans l’écume, nos tireurs d’élite n’atteignaient pas toujours leur but ce qui apportait encore de nouvelles révélations sur l’incroyable férocité de l’ennemi. Non seulement, ils arrachaient furieusement les entrailles de leurs semblables mais encore, se courbant comme un arc souple, ils mordaient dans les leurs propres, tant et si bien que ces entrailles semblaient être avalées encore et encore par une gueule unique pour ressortir à l’autre bout d’une plaie béante. Et ce n’est pas tout. Ce serait risqué d’avoir affaire même aux cadavres et aux fantômes de ces créatures. Une sorte de vitalité générique et diabolique semble sourdre de leurs os mêmes après que les ait quittés ce qu’on pourrait appeler la vie individuelle. Tué, hissé à bord pour sa peau, l’un de ces requins emporta presque la main du pauvre Queequeg lorsque celui-ci essaya de fermer le couvercle mort de sa mâchoire meurtrière.\par
– Queequeg s’en moque savoir quel dieu a fait lui requin, dit le sauvage en abaissant et en levant, tour à tour, douloureusement sa main, si être le dieu de Fidji ou le dieu de Nantucket, mais le dieu qui faire requin être un damné faux jeton.
\chapterclose


\chapteropen
\chapter[{CHAPITRE LXVII. Dépeçage}]{CHAPITRE LXVII \\
Dépeçage}\renewcommand{\leftmark}{CHAPITRE LXVII \\
Dépeçage}


\chaptercont
\noindent C’était un samedi soir et quel sabbat suivit ! Tous les baleiniers sont des docteurs es violation du repos dominical. Le {\itshape Péquod} d’ivoire se métamorphosa en une sorte d’abattoir, chaque marin en boucher. On aurait cru que nous offrions aux dieux de la mer l’holocauste de dix mille bœufs rouges.\par
Tout d’abord, entre autres instruments pesants, les énormes caliornes furent mises en place, elles comportent un bouquet de poulies, généralement peintes en vert, qu’un homme ne pourrait jamais soulever ; cette gigantesque grappe de raisins fut hissée au capelage du grand mât et solidement aiguilletée au ton du mât de misaine, le point d’appui le plus résistant qu’il y ait au-dessus du pont d’un navire. Le bout de la haussière courant dans le jeu compliqué des caliornes fut alors amené au guindeau, et la géante poulie inférieure fut balancée au-dessus de la baleine sur l’estrope de laquelle est aiguilleté le grand croc pesant quelque cent livres. À présent les deux seconds, Starbuck et Stubb, suspendus sur les chaffauds, par-dessus bord et armés de leurs longues pelles, commencent à creuser un trou en avant des nageoires pectorales pour l’insertion du croc. Cela fait, une entaille en demi-lune est encore pratiquée autour de ce trou et le croc y est inséré. Alors le gros de l’équipage, massé au guindeau, commence à virer en entonnant un chant sauvage. Aussitôt le navire se couche, il tressaille de toutes ses chevilles comme de ses têtes de clous une vieille maison attaquée par le gel. Il vibre et tremble et ses mâts épouvantés saluent le ciel. De plus en plus, il s’incline vers la baleine, cependant qu’à \hspace{1em} chaque hoquet du guindeau, répond, secourable, un soulèvement de la houle. Enfin un claquement vif et brutal éclate, dans un grand clapotement le navire roule et s’écarte de la baleine, la caliorne triomphante s’élève, entraînant à sa suite la première bande de lard à partir de l’entaille en demi-lune. Comme le lard enveloppe la baleine à la manière dont l’écorce recouvre une orange, on la pèle souvent en spirale comme on épluche ce fruit. Suivant la traction constante du guindeau, la baleine vire et vire sur ellemême dans l’eau, et le lard se détache en bande régulière le long de la ligne appelée « la taillade » tracée simultanément par les louchets des seconds, Starbuck et Stubb. À mesure que le lard cède, la bande monte jusqu’à frôler la grande hune ; les hommes cessent alors de virer au guindeau et pendant un moment la masse prodigieuse ruisselante de sang se balance de-ci de-là, comme descendue du ciel, et chacun doit l’éviter avec soin s’il ne veut pas qu’elle lui frictionne les oreilles et l’expédie, tête première, par-dessus bord.\par
L’un des harponneurs s’avance alors avec une arme longue et tranchante qu’on appelle sabre d’abordage et, au moment propice, découpe adroitement un grand trou dans la partie inférieure de cette bande qui se balance, passe l’estrope de l’autre caliorne dans ce trou assurant ainsi une prise dans le lard pour la suite des opérations. Cela fait, notre épéiste, avertissant chacun de s’écarter, et d’un mouvement savamment porté, après quelques entailles obliques, sépare tout à fait cette bande en deux, de sorte que si la partie inférieure plus courte reste fixe, la longue partie supérieure se libère et se trouve prête à tomber dans l’entrepont. Au guindeau les hommes reprennent désormais leur chant et cependant qu’on vire sur la caliorne qui arrache et hisse une seconde bande de lard, l’autre est manœuvrée perpendiculairement au-dessus de la grande écoutille où la première bande descend dans un salon non meublé, appelé parc au gras. Dans cet appartement crépusculaire, de nombreuses et adroites mains enroulent la longue enveloppe comme si elle était un immense entrelacs de serpents vivants. Ainsi se poursuit le travail, les deux caliornes s’élèvent et s’abaissent simultanément, la baleine et le guindeau virent, les hommes chantent, ces messieurs du parc à gras enroulent, les seconds tailladent, le navire peine et tout l’équipage jure à l’occasion pour détendre la tension générale.
\chapterclose


\chapteropen
\chapter[{CHAPITRE LXVIII. La couverture}]{CHAPITRE LXVIII \\
La couverture}\renewcommand{\leftmark}{CHAPITRE LXVIII \\
La couverture}


\chaptercont
\noindent Je n’ai pas peu accordé d’attention à la peau de la baleine, ce sujet souvent débattu. J’en ai discuté, en mer, avec des baleiniers expérimentés, à terre avec de savants naturalistes. Mon opinion première reste inchangée mais ce n’est qu’une opinion.\par
La question est la suivante : qu’est-ce que la peau de la baleine et où se situe-t-elle ? Vous savez déjà ce qu’est son lard. Il a la consistance ferme, à grain serré, en plus dur, du bœuf ; il est plus élastique, plus dense et varie d’une épaisseur de huit ou douze jusqu’à quinze pouces.\par
Pour absurde qu’il semble de prime abord de parler de peau à propos d’une masse offrant une pareille résistance et d’un pareil volume, on ne peut alléguer aucun argument contre cette dénomination, car l’on ne peut peler la baleine d’aucune enveloppe autre que de ce lard, et l’enveloppe extérieure d’un animal quel qu’il soit, si elle présente un tant soit peu d’épaisseur, peut-elle être autre chose que sa peau ? En vérité, lorsque le corps de la baleine est encore intact, vous pouvez y gratter du doigt une matière infiniment mince, transparente, semblable, aux plus fines lamelle de mica, encore qu’elle ait la douceur et la souplesse du satin avant qu’elle ait séché, car alors elle se resserre, s’épaissit et devient plutôt dure et cassante. J’en ai plusieurs morceaux secs dont je me sers pour marquer les pages de mes livres se rapportant aux baleines. Comme je viens de le dire, c’est une matière transparente et, quand elle est posée sur la page imprimée, je me suis souvent plu à lui découvrir un pouvoir grossissant. Quoi qu’il en soit, c’est charmant de lire des histoires de baleines avec leurs propres lunettes, si l’on ose dire. Mais voici où je veux en venir. Cette pellicule, pareille à un mica extrêmement mince recouvre, je le reconnais, le corps entier de la baleine, mais on ne doit point tant le considérer comme sa peau que comme la peau de la peau, pour ainsi dire. Car il serait tout simplement ridicule d’affirmer que la vraie peau de la colossale baleine est plus fine et plus délicate que celle d’un nouveau-né. Mais en voilà assez sur ce thème !\par
En admettant que le lard soit la peau du cétacé, lorsque celle-ci, comme c’est le cas pour les grands cachalots, donnera cent barils d’huile et lorsqu’on considère qu’en quantité, ou plutôt en poids cette huile, pour parler en chiffres, ne représente que les trois quarts et non la totalité de l’enveloppe, on peut se faire une idée de l’énormité d’une masse vivante dont le simple tégument fournit un pareil lac de liquide. Si l’on compte dix barils à la tonne, on obtient dix tonnes, poids net, avec les trois quarts seulement de la peau de la baleine.\par
Ce n’est pas la moindre des merveilles offertes par le cachalot que la vue de sa peau quand il est vivant. Presque toujours, elle est sillonnée de toutes parts et en tout sens de lignes obliques sans nombre, droites et serrées comme celles des plus belles gravures italiennes. Mais ces cannelures ne semblent pas imprimées sur la matière transparente dont je viens de parler mais elles paraissent plutôt vues à travers elle, taillées à même le corps. Et ce n’est pas tout. En certains cas, pour un œil prompt et observateur, ces hachures, tout comme dans les véritables gravures, ne sont que le fond de bien d’autres dessins. Ce sont des hiéroglyphes, si vous appelez ainsi les signes mystérieux inscrits sur les pyramides. Ma mémoire sûre revoit les hiéroglyphes d’un cachalot, en particulier, sur lequel je fus très frappé de retrouver les mêmes caractères indiens que ceux des célèbres parois qui bordent le Mississipi supérieur. Les signes mystiques de la baleine demeurent tout aussi indéchiffrables que ceux de ces pierres mystérieuses et cette allusion aux roches des Indiens me rappelle autre chose… Entre autres phénomènes présentés par l’extérieur du cachalot, il n’est pas rare que les stries parallèles de son dos, et plus spécialement de ses flancs, soient à demi brouillées par des éraflures profondes, irrégulières et comme tracées au hasard. Je dirai que ces rocs de la Nouvelle Angleterre dont Agassiz pense qu’ils portent les marques du puissant raclement que leur faisaient subir les icebergs en dérive, j’oserais dire que ces rochers ne sont pas sans ressembler au cachalot sur ce point. Je pense aussi que ces stries doivent être les traces des combats que se livrent les cétacés, car je les ai remarquées, la plupart du temps, sur les grands mâles adultes.\par
Encore un ou deux mots au sujet de cette peau, ou de ce lard, de la baleine. Comme nous l’avons dit, il est pelé en longues bandes, appelées morceaux de couverture. Comme la plupart des termes marins, celui-ci est d’un choix heureux et imagé, car la baleine est en vérité enveloppée dans son lard comme dans une vraie couverture ou courtepointe, mieux encore comme dans un poncho indien qui passerait par-dessus sa tête pour s’allonger jusqu’à sa queue. C’est grâce à cette douillette que la baleine jouit d’un égal confort sous tous les climats, dans toutes les mers, en tout temps et à toutes les profondeurs. Que deviendrait la baleine du Groenland par exemple dans le frisson glacial de ces mers arctiques, si elle n’était pas pourvue de cet agréable pardessus ? Sans aucun doute, on trouve d’autres poissons tout à fait pleins d’entrain dans ces eaux hyperboréennes, soulignons toutefois que ce sont vos poissons à sang froid, sans poumons, dont les entrailles sont des frigorifiques, des créatures qui se chauffent à l’abri d’une banquise comme un voyageur lézarde un jour d’hiver au coin d’une cheminée d’auberge. Tandis que la baleine tout comme l’homme, a des poumons et le sang chaud. Gelez son sang, elle meurt. Aussi quelle merveille – sauf après qu’on en ait fourni l’explication – que ce monstre énorme à qui la chaleur corporelle est aussi \hspace{1em} indispensable qu’elle l’est à l’homme, quelle merveille qu’elle soit dans son élément, immergée jusqu’aux lèvres sa vie durant dans les eaux de l’Arctique, alors que les marins tombés par-dessus bord sont parfois retrouvés des mois plus tard, raides au cœur des glaces comme mouches dans l’ambre. Il est plus étonnant encore d’apprendre, l’expérience l’a prouvé, que la baleine du pôle a le sang plus chaud qu’un nègre de Bornéo en été.\par
Il me semble qu’ici se manifeste la rare vertu d’une forte vitalité individuelle, la rare vertu de murs épais, et la rare vertu de l’espace intérieur. Ô homme, admire la baleine, prends modèle sur elle ! Toi aussi, conserve ta chaleur au sein des glaces. Toi aussi, sache vivre en ce monde sans être de ce monde. Garde ta fraîcheur sous l’Équateur et la vivacité de ton sang au Pôle. Comme le grand dôme de St-Pierre, et comme la grande baleine, ô homme, sauve la température propre en toutes saisons.\par
Mais il est trop simple et combien sans espoir d’enseigner ces choses belles ! Combien peu d’édifices ont un dôme comme St-Pierre ? Parmi les créatures, combien peu ont la grandeur de la baleine !
\chapterclose


\chapteropen
\chapter[{CHAPITRE LXIX. Les funérailles}]{CHAPITRE LXIX \\
Les funérailles}\renewcommand{\leftmark}{CHAPITRE LXIX \\
Les funérailles}


\chaptercont
\noindent – Halez les chaînes en dedans ! Larguez la carcasse !\par
Les grandes caliornes ont fini leur travail. Le corps de la baleine décapitée, pelé jusqu’à la blancheur, brille comme un tombeau de marbre. Si sa masse a changé de couleur, elle semble n’avoir rien perdu de sa dimension. Elle est toujours colossale. Lentement, elle s’éloigne toujours davantage en flottant, sur une eau déchirée et éclaboussée par les requins insatiables, et le ciel est attristé par le vol avide des oiseaux dont les becs sont autant de poignards qui l’insultent. Toujours et toujours plus loin du navire, l’immense et blanc fantôme sans tête ! Et à chaque perche de sa dérive, le pré carré des requins et la ronde des oiseaux augmentent leur tumulte meurtrier. Aux hommes du navire presque encalminé ce hideux spectacle s’offre des heures durant. Sous la douceur azurée d’un ciel sans nuages, sur le clair visage d’une mer aimable, égayée par la brise, cette île de la mort va se perdre dans l’infini.\par
Tristesse et dérision suprême de ces funérailles ! Le deuil pieux de ces vautours marins, requins du ciel en noir de cérémonie ! J’imagine que bien peu d’entre eux eussent porté secours à la baleine lorsqu’elle était en vie et si, par hasard, elle avait eu besoin d’eux. Mais au banquet de sa mort, comme ils s’abattent dévotement ! Horrible curée de ce monde, à laquelle la puissante baleine elle-même n’échappe pas…\par
Mais ce n’est pas encore la fin… Pour profané que soit son corps, un esprit vengeur lui survit qui fait planer la terreur. Qu’il soit aperçu de loin par un timide bâtiment de guerre ou par un maladroit vaisseau d’exploration, tandis que la distance estompe le pullulement des oiseaux mais que le soleil livre néanmoins à la vue la masse blanche où vient briser la blanche écume, aussitôt l’inoffensif cadavre de la baleine sera signalé d’une main tremblante sur les livres de bord : {\itshape Haut-fonds, récifs et écueils dans ces parages, attention} ! Et pendant des années peut-être les navires éviteront ce lieu, tels des moutons stupides sautant par-dessus rien parce que leur chef a dû, à cet endroit, sauter par-dessus un bâton tendu. Telle est votre loi des précédents, l’utilité de vos traditions, telle est l’histoire de la survivance opiniâtre de vos croyances, qui n’ont jamais pris racine dans la terre et ne savent pas encore fleurir dans le ciel ! Telle est l’orthodoxie !\par
C’est ainsi que le corps du grand cachalot a pu être dans sa vie la terreur de ses ennemis et que son fantôme crée encore une panique injustifiée pour tout un monde.\par
Êtes-vous de ceux qui croient aux fantômes, mon ami ? Il est d’autres revenants que celui de Cok-Lane et des hommes beaucoup plus profonds que le docteur Johnson pour y ajouter foi.
\chapterclose


\chapteropen
\chapter[{CHAPITRE LXX. Le Sphinx}]{CHAPITRE LXX \\
Le Sphinx}\renewcommand{\leftmark}{CHAPITRE LXX \\
Le Sphinx}


\chaptercont
\noindent Il aurait fallu dire qu’avant de peler entièrement le léviathan, on le déjointait. Et la décollation du cachalot est anatomiquement un exploit scientifique dont les chirurgiens baleiniers d’expérience peuvent, non sans raison, s’enorgueillir.\par
En effet la baleine n’a rien de ce qu’on pourrait appeler un cou. Au contraire, c’est à l’endroit précis où semblent se réunir la tête et le corps que se situe sa partie la plus épaisse. Qu’on se souvienne aussi que le chirurgien doit intervenir à quelque huit ou dix pieds au-dessus d’un champ opératoire partiellement dissimulé par une houle terne, ou sur une mer souvent furieusement agitée. Qu’on se souvienne encore qu’en pareilles circonstances il doit cependant entamer la chair de plusieurs pieds de profondeur, travailler à l’aveugle dans cette entaille qui se resserre, éviter adroitement les parties adjacentes et détacher avec exactitude la colonne vertébrale au point précis de son insertion, tout près du crâne. N’y a-t-il pas, dès lors, matière à s’émerveiller de ce que Stubb se vantât qu’il ne lui fallait que dix minutes pour déjointer un cachalot ?\par
Sectionnée, la tête est amarrée à l’arrière jusqu’à ce que le reste du corps soit entièrement pelé. Cela fait, s’il s’agit de la tête d’une petite baleine, elle est hissée sur le pont afin que l’on en dispose sans hâte ; c’est chose impossible dans le cas d’un léviathan adulte car la tête du cachalot représente près d’un tiers de son volume total et suspendre un tel fardeau ; fût-ce à des caliornes aussi énormes que celles des navires baleiniers, serait une tentative aussi vaine que de vouloir peser une étable hollandaise dans une balance d’orfèvre.\par
La baleine du {\itshape Péquod}, ayant été déjointée et pelée, la tête fut élevée contre le flanc du navire, à hauteur voulue pour qu’elle puisse encore être soutenue par son élément naturel. Et là, le navire abruptement penché sur elle à cause de la traction énorme infligée au bas mât, et chaque bout de vergue s’étendant de ce côté au-dessus des vagues tels des mâts de charge, là cette tête ruisselante de sang est pendue à la taille du {\itshape Péquod} comme celle du géant Holopherne à la ceinture de Judith.\par
Lorsque cette dernière besogne fut accomplie, il était midi, les hommes descendirent pour leur repas et le silence se fit sur le pont déserté si bruyant l’instant d’auparavant. Un calme ardent, cuivré, comme un universel lotus jaune, s’épanouissait à l’infini sur la mer, pétale à pétale, sans bruit…\par
Un instant passa puis, dans toute cette immobilité, Achab sortit seul de sa cabine. Il fit quelques pas sur le gaillard d’arrière, s’arrêta pour regarder par-dessus bord et, se tenant aux cadènes, il prit la longue pelle de Stubb, abandonnée là depuis la décollation, l’enfonça dans la partie inférieure de la masse à demi suspendue, glissa sous son bras la béquille du manche et demeura ainsi penché en avant, les yeux attentivement fixés sur cette tête.\par
Elle portait un capuchon noir et pendue là, dans un calme aussi intense, on eût dit celle du Sphinx dans le désert.\par
« Parle, ô grande et vénérable tête, murmura Achab, si tu ne portes pas de barbe, tu parais çà et là blanchie de mousses, parle, puissante tête et livre-nous le secret qui est en toi. De tous les plongeurs, nul n’est descendu aussi profond que toi, cette tête sur laquelle à présent brille le soleil de midi a voyagé parmi les fondations du monde. Là où rouillent des flottes et des noms inconnus, où pourrissent les ancres et les espoirs inavoués, là où la frégate de la terre baigne sa cale meurtrière lestée des millions d’ossements de noyés, là, dans cet affreux pays liquide, tu avais ton domaine le plus cher. Tu es allée là où n’a jamais atteint une cloche de plongée, tu as dormi au flanc de plus d’un marin, là où bien des mères, au sommeil perdu, eussent donné leur vie pour s’étendre. Tu as vu les amants enlacés sauter de leur navire en feu, cœur contre cœur engloutis par la vague triomphante, fidèles l’un à l’autre alors que le ciel semblait les trahir. Tu as vu le second assassiné lorsque les pirates l’ont jeté par-dessus bord dans la nuit. Des heures durant, il est descendu dans la nuit plus profonde encore de cette panse insatiable, tandis que ses assassins continuaient à voguer sains et saufs, tandis que des éclairs rapides faisaient voler en éclats un navire proche qui aurait rendu à des bras impatiemment tendus un homme juste. Ô tête ! tu en as vu assez pour faire éclater les planètes, pour faire d’Abraham un infidèle et pourtant tu ne dis mot ! »\par
– Navire en vue ! cria l’homme en vigie avec jubilation.\par
– Oui ? Eh bien, c’est réjouissant ! s’écria Achab, se redressant soudain, le front dégagé de tous nuages. – Ce cri de vie dans ce calme de mort pourrait convertir un homme meilleur. À quelle distance ?\par
– Trois points tribord avant, sir, et venant sur nous à notre vent.\par
– De mieux en mieux ! Si saint Paul pouvait venir par le même chemin nourrir de son souffle mon étouffement. Ô Nature, ô âme de l’homme, indiciblement comparables ! Il n’est si petit atome vivant et se mouvant au sein de la matière qui n’ait dans l’esprit un double rusé.
\chapterclose


\chapteropen
\chapter[{CHAPITRE LXXI. L’histoire du Jéroboam}]{CHAPITRE LXXI \\
L’histoire du Jéroboam}\renewcommand{\leftmark}{CHAPITRE LXXI \\
L’histoire du Jéroboam}


\chaptercont
\noindent La main dans la main, le navire et la brise arrivaient sur nous, mais la brise était plus rapide que le navire et bientôt le {\itshape Péquod} se mit à rouler.\par
À la longue-vue, le navire étranger se révéla, aux guetteurs de ses mâts, être un baleinier. Mais il était si loin au vent, faisant apparemment cap sur d’autres parages, que le {\itshape Péquod} ne pouvait espérer le rejoindre ; aussi hissa-t-on le signal pour voir ce qu’il répondrait.\par
Il faut dire ici que, tout comme les vaisseaux de guerre, les navires de la flotte baleinière américaine ont tous leur signal distinctif, figurant, avec le nom du navire correspondant, dans un livre qui est en possession de chaque capitaine, de sorte que les capitaines baleiniers sont à même de se reconnaître en plein Océan, à de grandes distances et sans peine.\par
Le navire étranger répondit au signal du {\itshape Péquod} en hissant le sien et il s’avéra être le {\itshape Jéroboam} de Nantucket. Brassant carré, il laissa porter sur nous, se rangea par le travers sous le vent du {\itshape Péquod} et mit une pirogue à la mer.\par
Elle eût tôt fait de se rapprocher mais, tandis que sur l’ordre de Starbuck on apprêtait l’échelle de coupée pour recevoir la visite de son capitaine, celui-ci, à l’arrière de sa baleinière, fit un geste de la main signifiant que c’était tout à fait inutile. Il se trouva qu’une épidémie maligne sévissait à bord du {\itshape Jéroboam} et que Mayhew, son capitaine, craignait de contaminer l’équipage du {\itshape Péquod}. Si lui-même et les hommes de sa baleinière étaient saufs, si la mer incorruptible et l’air du large purifiaient l’espace d’une portée de fusil qui nous séparait de son navire, le capitaine toutefois observait consciencieusement la pusillanime quarantaine des terriens et refusa catégoriquement d’entrer en contact direct avec nous.\par
Mais cela ne nous empêcha en rien de communiquer. Se maintenant à quelques pieds de distance, de quelques coups de rames intermittents, la pirogue du {\itshape Jéroboam} parvint à rester en parallèle avec le {\itshape Péquod}, tandis que celui-ci courait sur son erre, son grand hunier coiffé, la brise ayant fortement fraîchi, et bien que la poussée brusque d’une grosse vague déportât parfois la pirogue vers notre avant, elle était ramenée adroitement en bonne place. Malgré ces interruptions et d’autres encore, la conversation suivait un cours décousu lorsqu’elle prit une orientation d’un tout autre genre.\par
L’un des canotiers de la baleinière du {\itshape Jéroboam} était un homme d’une apparence singulière, même dans cette faune sauvage des baleiniers qui recrute, en majeure partie, des individus des plus originaux. C’était un jeune homme, petit, râblé, au visage semé de taches de rousseur, qui avait une crinière de cheveux jaunes. Il portait une vareuse insolite, à longues basques, couleur de noix, passée et dont les manches trop longues étaient retroussées sur ses poignets. Une folie profonde, ancrée, fanatique allumait son regard.\par
À peine ce personnage eut-il été aperçu que Stubb s’était écrié :\par
– C’est lui ! C’est lui ! c’est le Scaramouche aux longues nippes dont ceux du {\itshape Town-Ho} nous ont parlé !\par
Stubb faisait allusion à une étrange histoire courant sur le {\itshape Jéroboam} et sur un certain membre de son équipage et qui nous avait été contée lorsque le {\itshape Péquod} avait rencontré le {\itshape Town-Ho} quelque temps auparavant. Selon ce récit et ce que l’on apprit plus tard, il apparaissait que le Scaramouche en question avait pris un ascendant étonnant sur presque tous les hommes du {\itshape Jéroboam}. Voici son histoire :\par
Élevé dans la secte toquée des trembleurs de Neskyeuna, il avait été un grand prophète ; au cours de leurs secrètes réunions de cinglés, il était plusieurs fois descendu du ciel par une trappe annonçant l’ouverture toute proche de la septième fiole qu’il portait dans la poche de son veston, et qui, en lieu et place de poudre à canon, était présumée contenir du laudanum. Une curieuse lubie apostolique s’étant emparée de lui, il avait quitté Neskyeuna pour Nantucket, où, avec la ruse particulière à la folie, il afficha un extérieur sévère et raisonnable, puis demanda à s’engager comme novice à bord du {\itshape Jéroboam}. On l’accepta mais, à peine le navire avait-il gagné la pleine mer que l’écluse de sa folie s’ouvrit. Il se présenta comme l’archange Gabriel et ordonna au capitaine de sauter par-dessus bord. Il proclama publiquement qu’il était envoyé pour délivrer les îles de la mer comme vicaire général de toute l’Océanie. La conviction inébranlable qu’il mettait à ses déclarations, le jeu hardi et ténébreux de son imagination exaltée et sans repos, toutes les terreurs surnaturelles qu’engendre une vraie démence, s’emparèrent à la fois des esprits de la plupart des ignorants qui composaient l’équipage, et investirent ce Gabriel d’un prestige sacré. Qui plus est, ils en avaient peur. Le capitaine sceptique se fût volontiers débarrassé de lui, car non seulement un tel homme n’était pas très utile à bord, mais encore il refusait de travailler quand tel n’était pas son bon plaisir. Apprenant que son intention était de le débarquer dans le premier port qui s’y prêterait, l’archange ouvrit incontinent tous ses sceaux et ses fioles, vouant le navire et son équipage à la perdition pure et simple au cas où ce projet serait mis à exécution. Il avait exercé une telle pression sur ses disciples parmi l’équipage qu’en bloc ils se rendirent auprès du capitaine et l’avertirent que si Gabriel était débarqué aucun d’entre eux ne resterait à bord. Il fut, dès lors, contraint de renoncer à son intention. Ils exigèrent également que Gabriel ne fût jamais molesté quoi qu’il pût dire ou faire de sorte que celui-ci finit par jouir d’une totale liberté. En conséquence, il se souciait bien peu de l’autorité du capitaine ou des seconds et, depuis que l’épidémie s’était déclarée, il avait plus que jamais main mise sur tous, déclarant que la peste, comme il l’appelait, sévissait sur son ordre et ne cesserait que lorsqu’il le souhaiterait. Les matelots, de pauvres diables pour la plupart, rampaient obséquieusement devant lui, allant parfois, pour lui obéir, jusqu’à lui rendre un culte personnel comme à une divinité. Pour incroyable et surprenant que paraisse un tel état de choses, il n’en est pas moins vrai. En soi, l’histoire de ces fanatiques n’est point si frappante que le pouvoir infini qu’ils ont de se duper eux-mêmes et l’infini pouvoir qu’ils ont de duper et d’ensorceler tant d’autres. Mais il est temps d’en revenir au {\itshape Péquod}.\par
– Je ne redoute pas votre épidémie, camarade, dit le capitaine Achab penché sur la rambarde, au capitaine Mayhew debout à la poupe de sa pirogue, montez à bord !\par
Mais Gabriel avait bondi sur ses pieds :\par
– Songe, songe aux fièvres, jaunes et bilieuses ! Craignez la peste horrible !\par
– Gabriel, Gabriel ! s’écria le capitaine Mayhew, tu dois ou…\par
Mais au même instant une vague impétueuse lança la pirogue en avant et ses bouillonnements noyèrent tout discours.\par
– As-tu vu la Baleine blanche ? demanda Achab lorsque la baleinière revint à portée.\par
– Songe, songe à ton navire défoncé, sombrant ! Crains l’horrible queue !\par
– Je te répète, Gabriel, que… mais la pirogue fut à nouveau arrachée vers l’avant, comme tirée par des démons. Un silence se fit, tandis que se pressaient des vagues tumultueuses, qui selon un caprice occasionnel de la mer, moutonnaient sans s’élever. Cependant la tête suspendue du cachalot était violemment secouée et Gabriel la lorgnait avec plus d’appréhension que ne l’eût justifié sa nature d’archange.\par
Lorsque cet intermède eut pris fin, le capitaine Mayhew entama une sombre histoire au sujet de Moby Dick, non sans être fréquemment interrompu par Gabriel, chaque fois que ce nom revenait, et par la mer qui semblait avoir partie liée avec lui.\par
Il en ressortait que le {\itshape Jéroboam} n’était pas en mer depuis longtemps lorsque au cours d’une gamme avec un autre navire baleinier, son équipage apprit l’existence de Moby Dick et les ravages qu’il avait exercés. Savourant avec gourmandise cette information, Gabriel avertit solennellement le capitaine de ne pas attaquer au cas où le monstre viendrait à être rencontré ; dans sa folie idiote, il affirma que la Baleine blanche n’était rien moins que l’incarnation du Dieu Trembleur, les trembleurs admettant la Bible. Mais lorsque, quelque deux ans plus tard, les hommes en vigie signalèrent Moby Dick, Macey le premier second, fut dévoré du désir de l’affronter, et le capitaine lui-même ne s’opposait pas à lui en laisser l’occasion, malgré toutes les prédictions et tous les avertissements de l’archange ; Macey réussit à persuader cinq hommes d’armer sa baleinière. Il poussa au large avec eux et, après une nage longue et fatigante, bien des tentatives dangereuses et inefficaces, il réussit enfin à planter un fer. Pendant ce temps, Gabriel était monté à la tête du grand mât et agitait frénétiquement un bras, hurlant à pleins poumons, prophétisant la perte rapide des agresseurs sacrilèges de sa divinité. Macey, le second, debout à l’avant de sa baleinière, avec toute l’énergie téméraire de sa race, invectivait la baleine à cris sauvages, tout en guettant l’instant propice pour jeter la lance qu’il tenait en position, mais voilà qu’une large ombre blanche monta de la mer dont l’éventail rapide coupa momentanément le souffle aux canotiers. L’instant suivant, le malheureux second, si débordant de vie, fut lancé en l’air, et, après avoir décrit une longue courbe, il retomba à une cinquantaine de mètres de là. La pirogue n’eut pas une égratignure, pas un cheveu des hommes ne fut effleuré, mais le second sombra pour toujours.\par
Il convient d’ajouter entre parenthèses que, dans la chasse au cachalot, ce genre d’accident mortel est peut-être aussi fréquent que n’importe quel autre. Parfois, comme dans ce cas particulier, un homme est anéanti alors que rien n’est endommagé, plus souvent l’avant de la baleinière est arraché, à moins que le bordé de cuisse où se tient le chef, soit enlevé avec l’homme. Mais le plus étrange est qu’à plus d’une reprise on repêcha le corps qui ne portait aucune trace de violence, l’homme ayant été foudroyé.\par
La scène de la catastrophe, la chute de Macey, avait été clairement vue du navire. Poussant un cri perçant : « La fiole ! la fiole ! » Gabriel détourna l’équipage frappé de terreur de poursuivre la baleine. Cette terrible aventure revêtit l’archange d’un surcroît de prestige car ses disciples crédules crurent qu’il avait annoncé cet événement particulier et non fait une prophétie d’ordre général, que n’importe qui aurait pu faire avec une large marge de chances de tomber juste. Il devint une terreur sans nom à bord.\par
Mayhew ayant terminé son récit, Achab lui posa de telles questions que le capitaine étranger ne put s’empêcher de lui demander s’il avait l’intention de chasser la Baleine blanche si l’occasion s’en présentait. À quoi Achab répondit affirmativement. Aussitôt Gabriel bondit à nouveau sur ses pieds, jeta un regard flamboyant sur le vieil homme et s’écria avec véhémence, pointant un index vers l’eau :\par
– Pense, pense au blasphémateur qui gît mort, là au fond ! Crains la fin des blasphémateurs !\par
Achab se détourna, impassible, puis s’adressant à Mayhew :\par
– Capitaine, je pense à mon sac à lettres, si je ne me trompe il y a une lettre pour un de tes officiers. Starbuck, jetez un coup d’œil dans le sac.\par
Tout navire baleinier emporte un bon nombre de lettres destinées à divers navires et dont la délivrance dépend de la seule chance des rencontres dans l’un des quatre océans. Aussi la plupart ne leur parviennent-elles jamais, et nombre d’entre elles arrivent à qui de droit au bout de deux ou trois ans sinon davantage.\par
Starbuck revint bientôt avec une enveloppe à la main. Elle était piteusement chiffonnée, détrempée, couverte de tristes taches de moisissure verte, attestant un long séjour dans un placard sombre de la cabine. De pareille lettre, la Mort en personne eût pu être le facteur.\par
– Tu n’arrives pas à la lire ? s’enquit Achab, passe-la moi, camarade. Oui, oui, ce n’est qu’un vague gribouillage… mais qu’est-ce cela ? Tandis qu’il l’examinait, Starbuck fendait de son couteau le manche d’une pelle à trancher juste assez pour y introduire la lettre et la tendre jusqu’à la pirogue, sans qu’elle ait besoin de s’approcher davantage du navire. Cependant Achab, tenant toujours la lettre, \hspace{1em} murmura :\par
– Har… oui Harry… (une fine écriture de femme, son épouse, j’imagine). Oui. M. Harry Macey, à bord du {\itshape Jéroboam} ; mais c’est pour Macey et il est mort !\par
– Pauvre diable ! pauvre diable ! et elle est de sa femme, soupira Mayhew mais passez-la moi.\par
– Non, garde-la toi-même, cria Gabriel à Achab, tu suivras bientôt le même chemin.\par
– Que le diable t’étrangle ! hurla Achab, capitaine Mayhew, préparez-vous à la recevoir et, prenant des mains de Starbuck la missive fatale, il l’inséra dans la fente du manche et la tendit vers la pirogue. Mais les canotiers avaient, dans cette attente, cessé de nager, la baleinière dériva légèrement vers la poupe du navire de sorte que, comme par magie, l’enveloppe se trouva devant la main de Gabriel qui s’en saisit, empoigna le couteau de pirogue, y ficha le lettre et l’expédia ainsi lestée sur le navire. Elle tomba aux pieds d’Achab, Gabriel cria à ses camarades de scier et cette baleinière de mutins s’éloigna rapidement du {\itshape Péquod}.\par
Lorsque après cet intermède, les matelots reprirent leur travail sur la robe du cachalot, bien des propos étranges circulèrent au sujet de cette ténébreuse affaire.
\chapterclose


\chapteropen
\chapter[{CHAPITRE LXXII. La corde à singe}]{CHAPITRE LXXII \\
La corde à singe}\renewcommand{\leftmark}{CHAPITRE LXXII \\
La corde à singe}


\chaptercont
\noindent Affairé à la besogne du dépeçage, l’équipage s’agite dans un va-et-vient incessant. On a besoin des hommes ici, on a besoin d’eux ailleurs. Personne ne peut rester à une place car il faut être partout à la fois, tout devant être fait partout à la fois. Il en va joliment de même pour qui veut décrire la scène, et nous devons revenir quelque peu en arrière. Dès qu’on descend sur le dos de la baleine, le croc à lard est inséré dans le trou préparé par les pelles des seconds. Mais comment ce même crochet a-til été fixé dans une masse si incommode et si pesante ? C’était l’affaire du harponneur, aussi fut-ce mon ami Queequeg qui descendit sur le dos du monstre. Souvent, les circonstances réclament que le harponneur reste sur le dos de l’animal jusqu’à ce que l’opération du dépeçage soit complètement terminée. Soulignons que le gibier est presque entièrement immergé sauf la partie sur laquelle on œuvre, aussi, à quelque dix pieds en dessous du niveau du pont, le malheureux harponneur patauge, à moitié sur la baleine, à moitié dans l’eau, cependant que, sous lui, l’énorme masse roule comme une roue de potier. En l’occurrence, Queequeg s’y trouvait vêtu à l’écossaise – en chemise et chaussettes – tenue qui, à mes yeux du moins, le faisait paraître particulièrement à son avantage, et personne plus que moi, on va le voir, n’était mieux placé pour l’observer.\par
Étant le premier rameur du sauvage, c’est-à-dire le second à partir de l’avant, l’aviron du harponneur étant le premier numériquement, j’avais pour heureuse mission de l’assister lors de sa pénible acrobatie sur le dos de la baleine. Vous avez vu sans doute quelque Italien, avec son orgue de Barbarie, tenant un singe cabriolant au bout d’une longue corde. C’est ainsi que du flanc abrupt du navire je tenais Queequeg en bas dans la mer, au bout de ce que les baleiniers appellent une corde à singe, fixée à une forte bande de toile lui enserrant la taille.\par
Cette situation comique était dangereuse pour tous les deux. Car la corde à singe était attachée à chaque extrémité, d’une part à la large ceinture de toile de Queequeg, d’autre part à mon étroite ceinture de cuir, de sorte que pour le meilleur et pour le pire nous étions tous deux momentanément unis. Que le pauvre Queequeg vînt à couler pour ne plus remonter, la coutume et l’honneur réclamaient à la fois qu’au lieu de couper la corde, je le suive dans la mort. Ainsi nous unissait un long lien siamois. Queequeg était mon inséparable jumeau et je ne pouvais me dégager d’aucune des périlleuses responsabilités qu’entraînait ce nœud de chanvre.\par
J’étais à ce point sensibilisé à ma situation que je l’envisageais sous un jour métaphysique et, tandis que j’épiais avec attention les moindres mouvements de Queequeg, il réapparaissait clairement que ma propre personnalité se fondait avec la sienne dans cette association, que mon libre arbitre avait reçu un coup mortel et que, si l’autre commettait une faute ou si le malheur s’abattait sur lui, je courrais, innocent, à un désastre et à une mort imméritée. Je compris alors que la Providence subissait une sorte d’interrègne, car sa justice impartiale n’aurait jamais pu approuver une aussi grossière injustice. Réfléchissant plus avant, tandis que, de temps en temps, d’un coup sec, je lui évitais d’être écrasé entre le navire et la baleine, méditant, dis-je, plus avant, je me rendis compte que ma situation était aussi l’exacte situation de tout être vivant, à cette différence près, dans la plupart des cas, que cette relation de Siamois joue d’une façon ou d’une autre avec un plus grand nombre d’individus. Que votre banquier saute, vous sauterez, que votre apothicaire mette par erreur du poison dans vos pilules, vous mourrez. Vous pourrez répondre qu’avec d’extrêmes précautions, vous pouvez peut-être échapper à ces mauvais sorts ainsi qu’à d’autres innombrables revers de la vie. Mais, quelle que soit ma vigilance à tenir Queequeg au bout de la corde à singe, il lui imprimait parfois de telles secousses que j’étais bien près de glisser par-dessus bord. Je ne pouvais oublier non plus que, quoi que je fisse, je n’étais maître que d’une de ses extrémités\footnote{Tous les baleiniers pratiquent le système de la corde à singe, ce n’est qu’à bord du {\itshape Péquod} que le singe et son montreur sont liés. Ce progrès apporté à l’usage courant fut introduit par Stubb afin d’accorder au harponneur exposé le maximum de certitude quant à la fidélité et à la vigilance de celui qui tient la corde.}.\par
J’ai déjà dit que je tirais souvent le pauvre Queequeg d’entre la baleine et le navire, espace dans lequel il tombait parfois à cause de l’incessant roulement qui les balançait l’un et l’autre. Mais ce n’était pas là le seul danger auquel il était exposé. Nullement rebutés par le massacre qu’on avait fait d’eux pendant la nuit, les requins, plus alléchés que jamais par le sang jusqu’alors enfermé dans la carcasse et qui maintenant coulait à flots, grouillaient en essaim vorace autour du cadavre, comme des abeilles dans une ruche.\par
Et Queequeg se trouvait en plein milieu de ces requins et les repoussait parfois de son pied trébuchant. Ceci peut paraître presque incroyable mais il est rare que ces carnassiers touchent à l’homme lorsqu’ils ont une proie telle qu’une baleine morte.\par
Néanmoins il est sage de surveiller de près, on le comprend, ces rapaces mêlés à l’affaire. Aussi, outre la corde à singe avec laquelle j’écartais le pauvre diable du voisinage d’un requin me paraissant particulièrement féroce, une autre protection lui était assurée. Tashtego et Daggoo, sur l’établi suspendu au flanc du navire, brandissaient sans cesse au-dessus de sa tête leurs tranchantes pelles à baleine, tuant autant de requins qu’ils en pouvaient atteindre. À n’en pas douter, leur activité était aussi désintéressée que charitable, je reconnais qu’ils ne voulaient que le bonheur de Queequeg mais, dans leur zèle précipité à le servir, et vu que les requins et lui étaient parfois à demi dissimulés par l’eau souillée de sang, leurs louchets imprudents étaient aussi près d’amputer une jambe qu’une queue. Mais je pense que le malheureux Queequeg, peinant et s’essoufflant avec son grand croc de fer, le malheureux Queequeg, je pense, priait seulement son Yoyo et remettait sa vie entre les mains de ses dieux.\par
Eh bien, eh bien, mon cher ami et frère jumeau, me disaisje, tout en donnant du mou ou en halant la corde à chaque mouvement de la mer, quelle importance cela a-t-il après tout ? N’êtes-vous pas la précieuse image de chacun et de tous dans ce monde baleinier ? Cet Océan insondable où vous haletez, c’est la Vie, ces requins vos ennemis, ces pelles vos amies, et entre les requins et les pelles vous êtes dans un bien dangereux pétrin, pauvre gars !\par
Mais courage ! une bonne surprise vous attend, Queequeg. Car à présent, les lèvres bleuies, les yeux injectés de sang, le sauvage épuisé grimpe enfin aux cadènes et se tient tout trempé, tremblant malgré lui, sur le pont ; le garçon s’approche et avec un regard consolateur et bienveillant lui tend… quoi ?… Un cognac chaud ? Non ! lui tend, ô Seigneur ! une tisane de gingembre tiède !\par
– Du gingembre ? Est-ce que je sens une odeur de gingembre ? demanda soupçonneusement Stubb en s’approchant. Oui, cela doit être du gingembre, ajouta-t-il en jetant un coup d’œil dans la tasse encore intacte.\par
Il resta un instant immobile comme s’il n’en croyait pas ses yeux, puis il s’avança calmement vers le garçon étonné et lui dit en détachant ses mots :\par
– Du gingembre ? du gingembre ? et voulez-vous avoir la bonté de me dire quelle est la vertu du gingembre, monsieur Pâte-Molle ? Du gingembre ! est-ce là la sorte de combustible que vous utilisez pour allumer un feu dans ce cannibale grelottant ? Du gingembre !… et que diable est le gingembre ?… de la houille ?… du bois de chauffage ?… des allumettes soufrées ?… de la mèche ?… de la poudre à canon ?… que diable est le gingembre, dis-je, pour que vous en offriez à ce pauvre Queequeg que voilà ?\par
– Il y a anguille sous roche avec la Société de tempérance, dans cette affaire, ajouta-t-il brusquement, se dirigeant à présent vers Starbuck qui venait d’arriver. Voulez-vous jeter un coup d’œil à ce flacon, sir, sentez-le, je vous prie. Puis épiant la réaction du second, il ajouta : Le garçon a eu le front d’offrir ce calomel, ce jalap, à Queequeg qui remonte à l’instant de la baleine. Le garçon est-il apothicaire, sir ? et oserais-je demander si telle est la sorte d’apéritif avec laquelle il compte rendre la vie à un homme à demi noyé ?\par
– J’espère que non, répondit Starbuck, c’est une assez misérable potion.\par
– Oui, oui, garçon, s’écria Stubb, nous allons vous apprendre à droguer un harponneur, pas de vos remèdes par-là, vous cherchez à nous empoisonner, n’est-ce pas ? Vous avez contracté sur nous des assurances sur la vie et vous voulez tous nous faire passer de vie à trépas pour empocher les bénéfices, n’est-ce pas ?\par
– Ce n’est pas moi, répondit Pâte-Molle, c’est tante Charité qui a apporté le gingembre à bord en me priant de ne jamais donner d’alcool aux harponneurs mais seulement cette gingembrade, comme elle l’appelle.\par
– Gingembrade ! gingembrée canaille ! emportez ça ! et volez à vos placards pour y trouver quelque chose de mieux. J’espère n’avoir pas tort, monsieur Starbuck. Ce sont les ordres du capitaine : un grog pour le harponneur qui est sur la baleine.\par
– Ça va, seulement ne le frappez plus, mais…\par
– Oh ! je ne fais jamais de mal quand je frappe, sauf quand c’est une baleine ou quelque chose du même genre, et ce gars est une belette. Qu’allez-vous dire, sir ?\par
– Seulement ceci : descends avec lui et cherche toi-même ce que tu veux.\par
Lorsque Stubb réapparut, il avait un flacon sombre dans une main et une sorte de boîte à thé dans l’autre. Le flacon contenait de l’alcool et fut tendu à Queequeg, l’autre, le cadeau de tante Charité, fut généreusement offert à la mer.
\chapterclose


\chapteropen
\chapter[{CHAPITRE LXXIII. Stubb et Flask tuent une baleine franche et en discutent}]{CHAPITRE LXXIII \\
Stubb et Flask tuent une baleine franche et en discutent}\renewcommand{\leftmark}{CHAPITRE LXXIII \\
Stubb et Flask tuent une baleine franche et en discutent}


\chaptercont
\noindent Pendant tout ce temps, il convient de s’en souvenir, une prodigieuse tête de cachalot était suspendue au flanc du {\itshape Péquod}. Laissons-la où elle est en attendant d’avoir l’occasion de nous en occuper. Nous avons pour le moment d’autres besognes urgentes et ce que nous pouvons faire de mieux, au sujet de cette tête, c’est de prier le ciel que les caliornes tiennent bon.\par
Au cours de la nuit et de la matinée, le {\itshape Péquod} avait progressivement dérivé dans des eaux jaunes de krill qui témoignaient de la présence inusitée de baleines franches, espèce de léviathans dont bien peu eussent soupçonné qu’ils fréquentassent ces parages en pareille saison. Et bien que les équipages dédaignent habituellement de livrer la chasse à des créatures aussi viles, et bien que le {\itshape Péquod} n’ait nullement été armé dans le but d’en pêcher, et bien qu’il en ait déjà rencontré un grand nombre près des Crozets sans mettre à la mer, maintenant surtout qu’un cachalot avait été pris et déjointé, on reçut, avec une surprise unanime, l’ordre de prendre ce jour-là une baleine franche, si l’occasion s’en présentait.\par
Il n’y eut pas longtemps à attendre. Sous le vent, s’élevèrent de hauts souffles et les baleinières de Stubb et de Flask furent envoyées en chasse. S’éloignant toujours davantage, elles furent bientôt hors de vue des hommes en vigie, mais ceux-ci aperçurent soudain au loin un grand remous d’eau blanche et annoncèrent peu après qu’une pirogue, sinon les deux, était tirée par le gibier. Un certain laps de temps s’écoula, puis on vit nettement deux pirogues remorquées droit sur le navire par la baleine. Le monstre approcha la quille de si près qu’on eût pu lui croire des intentions malignes lorsque soudain, à quelque trois perches des bordages, il sonda dans un maëlstrom et disparut comme s’il avait passé sous le navire.\par
– Coupez ! coupez ! cria-t-on du navire aux baleinières qui semblèrent un instant sur le point de s’écraser contre la coque du {\itshape Péquod}. Mais ayant encore bien des brasses de ligne dans la baille, la baleine ne sondant pas très rapidement, les hommes lui filèrent d’abondance la ligne et, nageant à toc d’avirons, cherchèrent à gagner l’avant du navire. Pendant un moment la lutte fut très critique car les hommes laissaient filer la ligne dans un sens et nageaient dans l’autre. Cette traction contraire menaçait de les envoyer par le fond. Mais ils ne cherchaient à gagner que quelques pieds et ils tinrent bon jusqu’à ce qu’ils y fussent parvenus ; aussitôt une trépidation courut comme l’éclair le long de la quille, tandis que la ligne tendue à l’extrême raclait le dessous du navire pour surgir à l’avant, claquante et frémissante, secouant si vivement l’eau dont elle ruisselait, que les gouttes retombèrent comme des éclats de verre, tandis que la baleine réapparaissait elle aussi et que les pirogues étaient une fois de plus libres de s’envoler. Mais la baleine, à bout de forces, ralentit et changeant à l’aveugle de direction, revint sur l’arrière du navire, entraînant à sa suite les deux pirogues de sorte qu’elles avaient accompli un circuit fermé.\par
Cependant les hommes avaient embraqué leurs lignes toujours davantage, jusqu’à flanquer le gibier de chaque côté, la lance de Stubb répondant à celle de Flask. Ainsi la bataille se déroulait en rond autour du {\itshape Péquod} tandis que les requins qui pullulaient auparavant autour du corps du cachalot se ruaient vers le sang fraîchement versé, buvant avidement à chaque nouvelle blessure, comme les Israélites altérés aux fontaines jaillies du rocher sous le bâton du patriarche.\par
Enfin le souffle s’épaissit, avec une convulsion et un vomissement effrayants, la baleine roula sur le dos, morte.\par
Tandis que les deux chefs de pirogue capelaient solidement la queue et préparaient le remorquage de cette masse, une conversation s’engagea entre eux.\par
– Je me demande bien où le vieux veut en venir avec ce monceau de lard infect, dit Stubb dégoûté d’avoir affaire à un léviathan aussi ignoble.\par
– Où il veut en venir ? dit Flask tout en lovant un surplus de ligne à l’avant de la pirogue, n’avez-vous jamais entendu dire qu’un navire qui a en même temps une tête de cachalot à tribord et une tête de baleine franche à bâbord, n’avez-vous jamais entendu dire, Stubb qu’il ne peut plus jamais chavirer ?\par
– Pourquoi pas ?\par
– Je n’en sais rien, mais j’ai entendu ce fantôme de gomme-gutte de Fedallah le dire et il a l’air très au fait en matière de sortilèges marins. Mais je me dis parfois qu’il attirera finalement le mauvais sort sur le navire. Je n’aime qu’à moitié ce gars-là, Stubb. Avez-vous remarqué que ce croc dépassant de ses lèvres est comme qui dirait sculpté en tête de serpent, Stubb ?\par
– Qu’il coule bas ! Je ne le regarde jamais, mais si une nuit sombre me fournit l’occasion de le trouver se tenant tout près de la rambarde et qu’il n’y ait personne alentour… regardez bien, Flask… et il désigna la mer en faisant des deux mains un geste singulier. Oui, je le ferais ! Flask, pour moi ce Fedallah est le diable déguisé. Croyez-vous cette histoire à dormir \hspace{1em} debout comme quoi il aurait embarqué clandestinement ? C’est le diable, je vous dis. La raison pour laquelle vous ne lui voyez pas la queue c’est qu’il la relève pour la cacher, je suppose qu’il la porte enroulée dans sa poche. Maudit soit-il ! Maintenant que j’y pense, il lui faut toujours de l’étoupe pour mettre dans le bout de ses bottes.\par
– Et il dort sans les enlever, n’est-ce pas ? Et il n’a pas de hamac, mais je l’ai plus d’une fois vu la nuit couché sur une glène de cordage.\par
– Sans aucun doute, et c’est à cause de sa maudite queue, vous voyez il la love dans l’œil du cordage.\par
– Qu’est-ce que le vieux a à faire avec lui ?\par
– Il conclut un marché ou un échange, je suppose.\par
– À quel sujet, un marché ?\par
– Eh bien, vous voyez, le vieux est acharné après cette Baleine blanche et le diable essaie de l’embobeliner pour lui soutirer sa montre en argent ou son âme, ou quelque chose de ce genre, contre.\par
– Peuh ! Stubb, vous radotez ! Comment Fedallah pourraitil faire cela ?\par
– Je n’en sais rien, Flask, mais le diable est un drôle de gars, et mauvais, croyez-moi. On raconte qu’il est allé flâner une fois sur le vieux vaisseau amiral, en agitant la queue avec une désinvolture satanique et distinguée, en demandant si le vieux patron était chez lui. Il y était justement et demanda au diable ce qu’il voulait, raclant des sabots, lui répondit : « Je veux John. » – « Pourquoi faire ? » demanda le vieux patron. – « En quoi cela vous regarde-t-il ? » répliqua le diable en se fâchant, j’ai besoin de lui. – « Prenez-le », dit le patron et par Dieu, Flask, si le diable n’a pas donné à John le choléra asiatique avant d’en finir avec lui, je veux bien ne faire qu’une bouchée de cette baleine. Mais ouvrez l’œil – n’êtes-vous pas prêt ? Alors, en avant, que nous amenions la baleine au navire.\par
– Je crois me souvenir d’une histoire du même genre, dit Flask, tandis que les deux baleinières avançaient lentement avec leur fardeau. Mais je n’arrive plus à me souvenir où je l’ai entendue.\par
– Trois Espagnols ? Les aventures de ces trois soldats sanguinaires ? C’est dans ce livre que vous l’aurez lue, Flask ? N’estce pas ?\par
– Non, je n’ai jamais eu ce livre en mains, mais j’en ai entendu parler. Dites-moi, Stubb, pensez-vous que ce diable dont vous venez de parler, est le même que celui dont vous dites qu’il est maintenant à bord du {\itshape Péquod} ?\par
– Suis-je le même homme que celui qui a contribué à tuer cette baleine ? Le diable n’est-il pas éternel ? Qui a jamais entendu dire que le diable fût mort ? Avez-vous jamais vu un pasteur porter le deuil du diable ? Et si le diable a un passe-partout pour entrer dans la cabine d’un amiral, ne croyez-vous pas qu’il peut se faufiler par un sabord ? Qu’avez-vous à répondre, monsieur Flask ?\par
– Quel âge donnez-vous à Fedallah, Stubb ?\par
– Vous voyez ce grand mât ? répondit-il en pointant vers le navire, eh bien, il servira de chiffre un ; maintenant, prenez tous les cercles de barils se trouvant dans la cale du {\itshape Péquod} et alignez-les derrière ce grand mât, en guise de zéros. Eh bien, cela commencera à donner une idée de l’âge de Fedallah. Et tous les tonneliers de la création ne pourraient fournir assez de cercles pour faire assez de zéros.\par
– Mais écoutez, Stubb, j’ai cru que vous vous vantiez un peu quand vous me disiez il y a un instant que vous balanceriez Fedallah à la mer si une bonne occasion se présentait. Alors s’il est aussi vieux que votre alignement de fameux cercles, et s’il est éternel, à quoi cela avancera-t-il de le jeter par-dessus bord, dites-moi ?\par
– À lui faire boire une bonne tasse, en tout cas.\par
– Il reviendra à la nage.\par
– On lui en fera boire une seconde, puis une troisième et ainsi de suite.\par
– En admettant qu’il se mette dans la tête de vous la faire boire à vous, oui, jusqu’à ce que ce soit la dernière alors quoi ?\par
– J’aimerais bien l’y voir, je lui collerais une telle paire d’yeux au beurre noir qu’il n’oserait pas se montrer dans la cabine de l’amiral de longtemps sans parler du faux-pont où il vit ni des ponts supérieurs où il rôde tout le temps en catimini. Maudit soit le diable, Flask, croyez-vous que j’aie peur de lui ? Qui a peur de lui, hormis le vieux patron qui n’ose pas l’attraper et lui mettre les menottes, comme il le mérite, mais le laisse enlever les gens, oui et qui signe un pacte avec lui s’engageant à lui rôtir tous ceux qu’il enlèverait ? En voilà un patron à la manque !\par
– Pensez-vous que Fedallah cherche à enlever le capitaine Achab ?\par
– Si je le pense ? Vous serez bientôt renseigné, Flask. Mais dorénavant je vais le surveiller étroitement et si je m’aperçois qu’il se passe des choses très suspectes, je le prendrai tout simplement au collet et je lui dirai : Écoutez-moi bien, Belzébuth, pas de ça, compris ? Et s’il fait des façons, par Dieu, je mettrai ma main dans sa poche, j’empoignerai sa queue et je la virerai au guindeau jusqu’à ce que, à force de l’enrouler et de la tirer, elle lâche à son emplanture, voyez-vous. Et quand il se verra écourté de cette belle manière, j’ai dans l’idée qu’il filera en douceur sans même cette pauvre satisfaction de partir la queue entre les jambes.\par
– Et de la queue, qu’est-ce que vous en ferez, Stubb ?\par
– Ce que j’en ferai ? Je la vendrai comme fouet pour les bœufs au retour… Quoi encore ?\par
– Allons, est-ce sérieux tout ce que vous venez de raconter, Stubb ?\par
– Sérieux ou non, nous voilà arrivés au navire.\par
Les deux pirogues reçurent l’ordre de remorquer la baleine à bâbord où les grelins et tout le dispositif d’amarrage étaient parés.\par
– Qu’est-ce que je vous avais dit ? s’écria Flask, oui vous verrez bientôt la tête de la baleine franche faire pendant à celle du cachalot.\par
Le temps prouva que Flask avait dit vrai. Le {\itshape Péquod} qui, jusqu’alors penchait fortement vers la tête de cachalot, se redressa sous l’équilibre des poids, bien qu’il peinât durement, comme vous pouvez le penser. Il en va de même si nous hissons d’un côté la tête de Locke, qui nous fait pencher de son bord, mais si de l’autre côté nous halons celle de Kant, nous retrouvons l’équilibre, mais dans quel piteux état ! C’est ainsi que certains esprits ont toujours à équilibrer le chargement de leur navire à égal tirant d’eau. Ô insensés ! jetez donc par-dessus bord toutes ces têtes menaçantes et alors vous flotterez d’aplomb et légèrement.\par
L’amarrage d’une baleine franche à bord se déroule de la même manière que celui d’un cachalot, sauf que l’on sectionne la tête du cachalot en entier tandis qu’on prélève les lippes et la langue de la baleine et qu’on les hisse à bord de même que le rocher faisant partie de la couronne. Mais cette fois-ci rien de pareil. Les deux carcasses furent larguées et le navire, ainsi chargé de ses têtes, ressemblait fort à un mulet trop pesamment bâté.\par
Cependant Fedallah contemplait calmement la tête de la baleine, son regard allant, de temps en temps, des rides profondes qui la sillonnaient à celles qui couraient dans sa paume. Or il advint qu’Achab se trouva placé de telle façon que le Parsi interceptait son ombre et, pour autant que le Parsi eût une ombre, celle-ci se confondait avec celle d’Achab et la prolongeait. Tout en travaillant, les hommes se renvoyaient les conjectures les plus lapones au sujet de ces événements.
\chapterclose


\chapteropen
\chapter[{CHAPITRE LXXIV. La tête du cachalot. Croquis de comparaison}]{CHAPITRE LXXIV \\
La tête du cachalot. Croquis de comparaison}\renewcommand{\leftmark}{CHAPITRE LXXIV \\
La tête du cachalot. Croquis de comparaison}


\chaptercont
\noindent Voici donc deux grands cétacés qui ont uni leurs têtes, faisons de même !\par
Parmi les léviathans in-folio, le cachalot et la baleine franche sont de loin les plus remarquables. Ce sont les deux seuls cétacés que l’homme chasse régulièrement. Le Nantuckais les considère comme les deux extrêmes de toutes les espèces connues. Ce sont leurs têtes qui différencient principalement, leur aspect extérieur aussi, puisque nous pouvons, simplement en traversant le pont, observer les têtes de l’un comme de l’autre cétacé, suspendues aux flancs du {\itshape Péquod}, j’oserai vous demander où nous trouverions une meilleure occasion de faire une étude pratique de cétologie ?\par
Ce qui frappe instantanément, c’est le contraste offert par ces deux têtes. En toute conscience, reconnaissons qu’elles sont toutes deux massives, mais celle du cachalot a une symétrie géométrique qui fait tristement défaut à celle de la baleine. Celle du cachalot a plus de caractère en la regardant vous lui accordez spontanément la supériorité à cause de la dignité dont elle est tout entière empreinte. Dans le cas particulier, cette dignité est encore rehaussée par le sommet poivre et sel de cette tête, témoignage d’un âge avancé et d’une expérience étendue. Bref, les pêcheurs appellent un tel cachalot une « tête grise ».\par
Relevons d’abord ce qui est le moins dissemblable dans ces deux têtes, soit les deux organes les plus importants : l’œil et l’oreille. Très en arrière et en bas sur le côté de la tête, près de l’angle de la mâchoire de l’un et l’autre cétacés, vous découvrirez en cherchant bien un œil sans cils qui pourrait être celui d’un jeune poulain tant il est hors de toute proportion avec la grandeur de la tête.\par
Il est évident que cette singulière position des yeux sur les côtés interdit à la baleine de voir jamais un objet se trouvant exactement en face d’elle pas plus qu’elle ne peut voir celui qui se trouve derrière elle. En un mot, les yeux de la baleine correspondent aux oreilles de l’homme et vous imaginez aisément quelle serait votre propre situation si vous deviez regarder avec les oreilles. Vous constateriez que vous avez un champ de vision de trente degrés seulement en avant de votre axe perpendiculaire et trente degrés en arrière. Si votre ennemi le plus acharné marchait droit sur vous, en plein jour, le poignard levé, vous ne seriez pas davantage à même de le voir que s’il vous attaquait par derrière. On pourrait dire que vous avez deux dos, mais également aussi deux fronts (situés sur le côté) car qu’est-ce qui fait le front de l’homme en vérité sinon ses yeux ?\par
De plus, tandis que dans la plupart des animaux auxquels je puis penser en ce moment, les yeux sont placés de façon à confondre les deux visions afin d’envoyer au cerveau une seule image et non deux, la position particulière de ceux de la baleine, séparés comme ils le sont par une tête de plusieurs mètres cubes s’élevant entre eux comme une haute montagne entre deux lacs de vallée doit naturellement lui fournir des impressions indépendantes pour chaque œil. La baleine doit voir, dès lors, une image donnée d’un côté et une image différente à l’autre, cependant que, droit devant elle, doivent s’étendre le néant et les ténèbres. On peut dire que l’homme voit le monde depuis une guérite ayant une fenêtre à double châssis, tandis que pour la baleine ces châssis s’ouvrent séparément, en formant deux fenêtres distinctes mais en amputant tristement la vision. Cette singulière disposition des yeux des cétacés doit toujours entrer en considération dans la pêcherie et le lecteur fera bien de s’en souvenir lors de quelques scènes qui vont suivre.\par
On pourrait soulever une question curieuse et tout à fait déconcertante au sujet de la vision du léviathan, mais il faudra se contenter de l’effleurer. Tant qu’un homme a les yeux ouverts à la lumière, sa vision est un réflexe, il ne peut s’empêcher de voir, machinalement, les objets qui se trouvent devant lui. Toutefois, la moindre expérience lui prouvera que s’il peut d’un seul coup d’œil embrasser un nombre de choses, il lui sera tout à fait impossible d’en examiner deux avec attention et en détail, qu’elles soient grosses ou petites, même si elles sont proches à se toucher. Si vous séparez ces deux objets et les placez au centre d’un cercle noir afin d’en regarder un avec concentration, l’autre échappera tout à fait à votre conscience. Qu’en est-il en l’occurrence pour la baleine ? Certes ses deux yeux doivent la servir simultanément mais a-t-elle un cerveau tellement plus apte à la prise de conscience, à l’accommodation et à la subtilité que l’homme qu’elle puisse examiner simultanément avec application deux perspectives distinctes et diamétralement oppo- sées ? Si elle en est capable, elle détient le pouvoir merveilleux qui serait celui d’un homme apte à démontrer en même temps deux problèmes d’Euclide différents. À la bien considérer, cette comparaison n’est pas saugrenue.\par
Ce n’est peut-être de ma part qu’une fantaisie oiseuse mais il m’a toujours semblé que les hésitations extraordinaires manifestées par certaines baleines lorsqu’elles sont prises en chasse par trois ou quatre pirogues, leur timidité, les craintes insolites qu’elles manifestent proviennent indirectement de l’impuissance due à la perplexité où la plonge une double vision diamétralement opposée.\par
L’oreille de la baleine est pour le moins aussi curieuse que son œil. Si la race des léviathans vous est tout à fait inconnue, vous pourrez bien la chercher des heures durant sur ces deux têtes et ne la découvrir jamais. L’oreille est sans pavillon externe d’aucune sorte et une Plume même entrerait avec peine dans l’orifice tant il est minuscule. Il s’ouvre un peu en arrière de l’œil. Il y a, entre l’oreille du cachalot et celle de la baleine franche, une différence essentielle ; celle du premier a une ouverture externe, celle de la seconde est recouverte tout entière et de façon égale par une membrane qui la rend tout à fait invisible.\par
N’est-ce pas étrange qu’une créature aussi énorme que la baleine voie le monde à travers un œil si petit et qu’elle entende le tonnerre d’une oreille moins grande que celle d’un lièvre ? Mais ses yeux auraient-ils la dimension des lentilles du puissant télescope de Herschel, ses oreilles celle des porches de cathédrales, sa vue serait-elle meilleure et plus aiguë son ouïe ? Pas du tout ! Pourquoi, dès lors, cherchez-vous à élargir votre esprit ? Affinez-le !\par
Maintenant, quels que soient les leviers ou les machines à vapeur dont nous disposions, retournons sens dessus dessous la tête du cachalot, et grimpons sur une échelle afin de pouvoir jeter un coup d’œil dans sa gueule. N’était que le corps s’en trouve à présent détaché, nous pourrions descendre dans ses entrailles pareilles à la grande caverne du Mammouth en Kentucky. Mais tenons-nous à une dent et contemplons les lieux. Comme elle est belle et chaste, cette bouche ! Revêtue, ou plutôt tapissée, du sol au plafond d’une membrane blanche, brillante et lustrée comme un satin nuptial.\par
Sortons à présent et regardons cette sinistre mâchoire inférieure qui semble le couvercle étroit et long d’une tabatière gigantesque dont la charnière serait à un bout et non sur le côté. L’ouvrant largement au-dessus de votre tête, vous verrez l’effrayante herse des dents qui a fait ses preuves sur plus d’un pauvre diable de pêcheur, l’empalant avec force sur ses pointes. Elle est encore plus terrible à voir, lorsque dans les profondeurs, un cachalot maussade, flottant immobile, laisse pendre sa \hspace{1em}mâchoire prodigieuse de quelque quinze pieds de long à angle droit avec son corps tel le bout-dehors d’un grand foc. Ce cachalot n’est pas mort, il est seulement découragé, mal dans sa peau, peut-être hypocondriaque, et si éteint que les jointures de sa mâchoire se sont relâchées, lui donnant cet air disgracieux et triste, véritable reproche adressé à sa tribu qui, sans doute, lui souhaite méchamment le trisme.\par
La plupart du temps, cette mâchoire inférieure est aisément dégondée par un artiste expérimenté, et hissée à bord pour l’extraction des dents d’ivoire, ce morfil blanc dont les marins font toutes sortes d’objets curieux, pommeaux de cannes, becs d’ombrelles, manches de cravaches, etc.\par
La mâchoire est péniblement amenée comme une ancre jusqu’à bord et, après quelques jours consacrés aux travaux plus pressants, le moment venu, Queequeg, Daggoo et Tashtego, tous dentistes accomplis, se mettront à l’extraction. Queequeg incise les gencives avec une pelle à découper tranchante, puis la mâchoire est amarrée en bas à des chevilles à boucle, une poulie est fixée au-dessus et ils arrachent ces dents tels les bœufs du Michigan attelés à l’arrachage des souches de vieux chênes dans les forêts inviolées. Un cachalot a généralement quarante-deux dents, fort usées chez les sujets âgés, mais saines et non plombées selon nos artifices. La mâchoire est ensuite sciée en plaques, lesquelles sont empilées telles des solives de construction.
\chapterclose


\chapteropen
\chapter[{CHAPITRE LXXV. La tête de la baleine franche. Croquis de comparaison}]{CHAPITRE LXXV \\
La tête de la baleine franche. Croquis de comparaison}\renewcommand{\leftmark}{CHAPITRE LXXV \\
La tête de la baleine franche. Croquis de comparaison}


\chaptercont
\noindent Regardons maintenant attentivement la tête de la baleine franche de l’autre côté du pont.\par
Tandis que la forme de la noble tête du cachalot peut être comparée à un char romain (surtout de face où elle présente un large arrondi), celle de la baleine franche, dans sa vue d’ensemble, n’est pas sans rappeler une galiote inélégante et gigantesque. Il y a deux siècles, un voyageur hollandais lui prêtait une ressemblance avec une forme de cordonnier. Dans ce soulier, on pourrait loger fort confortablement la vieille femme des contes de fées et son essaim de marmaille.\par
Mais si l’on s’approche de cette immense tête, elle prend des aspects différents suivant l’angle sous lequel on la regarde. Si l’on se tient debout sur le sommet et si l’on plonge sur les évents en forme de « f », toute la tête apparaît comme un énorme violoncelle dont les évents seraient l’ouverture de sa caisse de résonance. Puis, en observant l’étrange incrustation en forme de peigne qui crête le sommet de cette masse, verte coiffure de bernacles que les Groenlandais appellent la « cou- ronne » et que les pêcheurs du Sud nomment le « bonnet » et en ne considérant que cette partie, on prendrait cette tête pour un chêne géant portant un nid dans une enfourchure. Cette idée ne saurait manquer de vous venir à la vue des crabes vivants qui nichent dans le bonnet à moins que, subjugués par le mot de couronne, vous ne vous demandiez avec intérêt comment ce monstre puissant a été sacré roi de la mer et qui a façonné, de manière si merveilleuse, cette couronne verte. Mais s’il est roi, c’est un gars bien maussade pour honorer un diadème. Regardez cette lippe pendante ! Quelle prodigieuse moue boudeuse ! Une moue qui, à la mesure d’un charpentier, fait environ vingt pieds de long et cinq de profondeur, une moue qui donnera quelque cinq cents gallons d’huile et plus.\par
Il est regrettable que cette malheureuse baleine soit affligée d’un bec-de-lièvre d’environ un pied. Sa mère, sans doute, voyageait le long de la côte du Pérou lorsque les tremblements de terre firent bâiller la grève. En passant par cette lippe, comme sur un seuil glissant, on pénètre dans la bouche. Parole d’honneur, si j’étais à Mackinac, je la prendrais pour l’intérieur d’un wigwam indien. Seigneur, est-ce là le chemin que prit Jonas ? Le toit s’élève à une douzaine de pieds et court selon un angle assez aigu pour former un véritable faîte, tandis que les parois côtelées, cintrées, velues, offrent l’étonnante architecture, à demi verticale, courbée en cimeterre, des fanons, trois cents environ sur un côté qui, pendant de la partie supérieure de la tête ou os de la couronne, forment ces stores à l’italienne dont il a été brièvement question ailleurs. Les bords des fanons sont bordés de fibres à travers lesquelles la baleine filtre l’eau et dans lesquelles elle retient les petits poissons lorsqu’elle traverse, la bouche ouverte, les bancs de krill. Dans le rideau central des fanons, tels qu’ils se présentent naturellement, on relève de curieuses marques, courbes, trous, arêtes, d’après lesquels certains baleiniers calculent l’âge du sujet comme l’on compte l’âge d’un chêne d’après ses anneaux. Bien que ce critère soit loin d’avoir fait ses preuves, il présente, par analogie, une vraisemblance. De toute façon, si nous nous y fions, nous attribuons à la baleine franche un âge beaucoup plus avancé qu’il ne paraît raisonnable de le faire à première vue.\par
Autrefois, il semble que ces fanons aient donné naissance aux idées les plus fantaisistes. Dans Purchas, un certain voyageur les appelle les merveilleuses « moustaches » de l’intérieur de la bouche de la baleine\footnote{Ceci nous rappelle que la baleine franche a vraiment une sorte de moustache… quelques poils blancs clairsemés bordant sa lippe inférieure. Ces touffes ajoutent parfois un petit air de brigand à sa solennité naturelle.}, un autre les qualifie de « soies de cochon », un troisième vieux gentilhomme des voyages de Hackluyt emploie le langage élégant que voici : « Deux cent cinquante ébarbures environ croissent de chaque côté de sa babine supérieure, formant un arc au-dessus de sa langue… »\par
Comme chacun sait, ces « soies de cochon », ces « ébarbures », ces « moustaches », ces « stores », ou tout ce qu’il vous plaira, fournissent aux dames les buses de leurs corsets et autres artifices de maintien. C’est là une demande qui est en baisse depuis fort longtemps. Ce fut au temps de la reine Anne que les fanons connurent la gloire, les vertugadins étant le dernier cri. De même que ces dames du bon vieux temps évoluaient gaiement, dans les mâchoires de la baleine pour ainsi dire, nous autres, avec une semblable insouciance, nous nous réfugions aussi en cas d’averse sous ces mâchoires, le parapluie étant une tente tendue sur les mêmes fanons.\par
Mais laissons là stores et fanons pendant un moment et, debout dans la bouche de la baleine, jetons un coup d’œil autour de nous. À voir l’ordre si méthodique de ces colonnades, ne se croirait-on pas dans les grandes orgues de Harlem, contemplant leurs mille tuyaux ? Pour tapis d’orgue, la langue offre le plus moelleux tapis d’Orient, qu’on dirait collé au plancher de la bouche. Elle est très grasse et si délicate qu’on risque de la déchirer en la hissant à bord ; de prime abord, je dirai que celle-ci doit être une six-barils, c’est-à-dire qu’elle donnera environ cette quantité d’huile.\par
Vous devez avoir désormais réalisé que j’ai dit vrai quant à la différence presque totale qu’il y a entre \hspace{1em}la tête du cachalot et celle de la baleine franche. En résumé, celle de la baleine n’a pas de grande boîte de spermaceti, ni dents d’ivoire ni la mâchoire inférieure longue et étroite. Celle du cachalot n’a pas de fanons, pas d’énorme lippe inférieure et c’est à peine s’il a une langue. D’autre part, il n’a qu’un évent et la baleine en a deux.\par
Pendant qu’elles sont encore côte à côte, jetons un dernier regard à ces vénérables têtes encapuchonnées car l’une ira bientôt, sans épitaphe, à la mer et l’autre la suivra de près.\par
Saisissons l’expression du cachalot. Il a conservé celle qu’il eut dans la mort sauf que quelques-unes de ses longues rides se sont effacées sur son front, auquel je trouve la sérénité de la Prairie, due je crois à une indifférence philosophique devant la mort. Mais voyons l’expression de l’autre tête. Son étonnante lippe inférieure, serrée par hasard contre le flanc du navire, ferme solidement la mâchoire. Toute la tête semble trahir une acceptation résolue et farouche en affrontant la mort. Je serais porté à croire que cette baleine franche a été un stoïque et le cachalot un platonicien qui se serait mis à Spinoza sur ses vieux jours.
\chapterclose


\chapteropen
\chapter[{CHAPITRE LXXVI. Le bélier}]{CHAPITRE LXXVI \\
Le bélier}\renewcommand{\leftmark}{CHAPITRE LXXVI \\
Le bélier}


\chaptercont
\noindent Avant d’en finir avec la tête du cachalot, j’aimerais qu’en physiologiste sensé, vous remarquiez la densité de son front et l’examiniez dans le seul but d’estimer, d’une manière lucide et non exagérée, la puissance qu’il peut offrir en tant que bélier. C’est une question majeure car ou bien vous en jugerez équitablement ou bien vous resterez à jamais incrédule devant un fait terrifiant mais non moins véridique qui pourrait être relaté.\par
Dans sa position habituelle de nage, le front du cachalot présente un plan presque absolument vertical par rapport à l’eau et vous remarquerez que la partie inférieure s’incline fortement vers l’arrière de façon à fournir un abri à l’emboîture de la mâchoire inférieure en forme de bout-dehors. La gueule se trouve entièrement sous la tête, tout comme si vous aviez, en vérité, la bouche carrément sous le menton. De plus, le cachalot n’a pas d’appendice nasal, et son nez, son évent en l’occurrence, est situé sur le sommet de sa tête, ses yeux et ses oreilles sont sur les côtés, à près d’un tiers de la longueur totale de son front. Dès lors, vous vous en rendez compte, la partie antérieure de la tête du cachalot est un mur orbe dépourvu de tout organe sensible proéminent. Il faut savoir encore que sauf à l’extrême partie arrière et inférieure de cette tête, il n’y a pas trace d’os ; ce n’est qu’à vingt pieds du front que le crâne est complètement développé. De sorte que cette énorme masse sans os est comme un bouchon mais, vous le verrez bientôt, il contient l’huile la plus délicate. Il vous reste à apprendre de quelle nature est cette substance qui rend inattaquable cette tête apparemment molle. Je vous ai déjà parlé de la manière dont le lard enveloppe le corps de la baleine comme le zeste la pulpe de l’orange. Il en va de même avec la tête, à cette différence près que cette enveloppe, si elle n’y est pas aussi épaisse, offre une résistance dont ne peut se rendre compte qui n’y a eu affaire. Le dard de harpon le plus acéré, la lance la plus aiguë, lancés par le bras humain le plus fort, y rebondissent avec puissance, comme si le front du cachalot était pavé de sabots de cheval. Je ne crois pas qu’il soit le siège d’une quelconque sensibilité.\par
Réfléchissez aussi à une autre chose. Lorsque deux gros bateaux des Indes, lourdement chargés, serrés à quai risquent de s’écraser l’un contre l’autre, que font les marins ? Ils ne suspendent pas entre eux, à l’endroit où ils viennent à se toucher, une pièce de matière dure telle que le fer ou le bois, mais un parebattage de liège et de câble fourré du cuir le plus épais et le plus solide. Ce tampon amortit allègrement et sans dommage un choc qui aurait brisé tous les anspects de chêne et de fer. Cet exemple illustre logiquement ce à quoi je veux en venir. J’ajouterai encore une hypothèse qui m’est venue à l’esprit, à savoir que si tout poisson ordinaire possède une vessie natatoire se dilatant et se contractant à volonté, le cachalot, pour autant que je le sache, n’en a point mais son pouvoir d’enfoncer la tête sous l’eau comme de la dresser au-dessus de la surface, l’élasticité de son enveloppe, le caractère unique de l’intérieur de sa tête, ses mystérieuses alvéoles d’une nature semblable au poumon, me paraissent plaider en faveur d’une relation avec l’air ambiant et d’une capacité de dilatation ou de contraction suivant la pression, relation qui reste à ce jour inconnue et insoupçonnée. S’il en est ainsi, songez que le plus impalpable et le plus destructeur des éléments viendrait ajouter à cette puissance irrésistible.\par
Or notez qu’en poussant infailliblement devant soi ce rempart sans ouvertures et inexpugnable, contenant la matière la plus légère, le cachalot entraîne à sa suite la masse d’un corps dont la dimension ne peut être estimée qu’au cordeau, comme des piles de bois, animé d’une vie exceptionnelle et soumis à une volonté unique comme celui du plus petit insecte. Aussi, lorsque plus tard j’entrerai dans les détails au sujet du pouvoir concentré et particulier de ce monstre monumental, lorsque je vous narrerai des exploits, parmi les plus insignifiants, dont il est capable avec sa tête, j’espère que vous aurez renoncé à tout scepticisme, que vous vous inclinerez et que vous ne lèverez pas les sourcils lorsque je vous dirai que le cachalot a ouvert un passage dans l’isthme de Darien, mêlant ainsi l’Atlantique au Pacifique. Car si vous ne prenez pas en considération la baleine, vous ne serez en matière de vérité qu’un provincial et un sentimental. Mais seules les salamandres géantes sont mises face à face avec la Vérité pure, aussi les chances d’un provincial ne sont-elles pas bien minces. Quel fut le sort du faible adolescent qui souleva le voile de la redoutable déesse Saïs ?
\chapterclose


\chapteropen
\chapter[{CHAPITRE LXXVII. Le foudre d’Heidelberg}]{CHAPITRE LXXVII \\
Le foudre d’Heidelberg}\renewcommand{\leftmark}{CHAPITRE LXXVII \\
Le foudre d’Heidelberg}


\chaptercont
\noindent Le moment est maintenant venu de vider la boîte. Pour suivre le procédé il convient de savoir quelque chose de sa structure interne.\par
Considérant la tête du cachalot comme un volume rectangulaire, on peut, selon un plan incliné, la diviser de biais en deux coins\footnote{Coin n’est pas un terme euclidien. Il appartient aux seules mathématiques marines. Un coin diffère d’une cale, en ce sens qu’il est angulaire d’un seul côté et non des deux.} dont l’inférieur est constitué par l’ossature du crâne et des mâchoires, le supérieur par une masse molle sans aucun os, son côté large formant la partie verticale apparente du front. Prenez le milieu de ce front, divisez cet angle supérieur et vous obtiendrez deux parties à peu près égales que sépare naturellement une paroi interne épaisse et tendineuse.\par
La partie inférieure, dite le pâté, est une immense alvéole pleine d’huile, formée par le croisement et les entrelacs de dix mille cellules, fibres blanches et élastiques qui s’interpénètrent sur toute la superficie. La partie supérieure, connue sous le nom de boîte, peut être regardée comme le foudre d’Heidelberg du cachalot. De même que ce célèbre tierçon est mystérieusement sculpté sur le devant, le front du cachalot est ridé d’innombrables et étranges emblèmes ornant son étonnante cuve. De plus, tel le foudre d’Heidelberg, toujours rempli des meilleurs crus du Rhin, celui du cachalot contient la plus précieuse des huiles, le spermaceti à l’état pur, limpide, odorant, dont la valeur est inestimable et qu’on ne trouve dans aucune autre partie de la bête. Il est absolument fluide tant que l’animal vit, mais dès qu’il est exposé à l’air, l’animal étant mort, il se solidifie en cristaux admirables, comme la première glace fragile qui se forme sur l’eau. La boîte d’un grand cachalot donne en général cinq cents gallons de spermaceti, quand bien même il s’en perd inévitablement une grande quantité, soit qu’il en soit renversé, soit qu’il s’en écoule lors de la délicate opération de transvasement.\par
Je ne sais quelle matière belle et coûteuse revêt intérieurement le foudre d’Heidelberg mais, quelle que soit sa somptueuse richesse, elle ne saurait rivaliser avec la membrane soyeuse, couleur de perle, pareille à la doublure d’une luxueuse pelisse, qui tapisse la boîte du cachalot.\par
On a vu que le foudre d’Heidelberg du cachalot occupe toute la longueur de sa tête, soit le tiers de sa longueur totale qui atteint quatre-vingts pieds pour un animal de belle taille. La profondeur de la boîte amarrée de toute sa longueur à la verticale au flanc du navire sera donc de plus de vingt-six pieds.\par
L’homme qui a déjointé la baleine a approché son instrument tout près de l’endroit où l’on pratiquera ultérieurement l’ouverture du réservoir de spermaceti, aussi a-t-il dû agir avec un soin extrême de crainte qu’un geste maladroit ne perforât ce sanctuaire et que se perdît, dès lors, son onéreux contenu. Cette partie de la tête est enfin élevée au-dessus de l’eau et maintenue en position par les énormes apparaux dont les filins de chanvre forment une véritable jungle.\par
Cela dit, prêtez maintenant, je vous prie, votre attention à la merveilleuse mise en perce – elle faillit en l’occurrence avoir des conséquences fatales – du foudre d’Heidelberg du cachalot.
\chapterclose


\chapteropen
\chapter[{CHAPITRE LXXVIII. Seaux et citerne}]{CHAPITRE LXXVIII \\
Seaux et citerne}\renewcommand{\leftmark}{CHAPITRE LXXVIII \\
Seaux et citerne}


\chaptercont
\noindent Souple comme un chat, Tashtego grimpe dans la mâture et, toujours debout, rejoint en courant la fusée de la grande vergue là où elle surplombe exactement la tête du cachalot. Il porte un léger cartahu, c’est-à-dire un cordage passant par une poulie simple, aiguilleté de façon à pendre au bout de la vergue, et jette un brin du cordage à une main qui le tiendra solidement à bord. Puis, main sur main, l’Indien descend à travers les airs, jusqu’à ce qu’il atterrisse adroitement sur le sommet de la tête d’où il domine le reste de l’équipage qu’il interpelle avec entrain. On dirait quelque muezzin turc invitant les fidèles à la prière du haut d’un minaret. On lui a envoyé une pelle tranchante à manche court et il cherche avec application l’endroit précis où pratiquer l’ouverture de la boîte, travaillant prudemment, tel un chercheur de trésor sondant les murs d’une vieille maison pour découvrir où est caché l’or. Lorsqu’il a trouvé, un robuste seau de fer, pareil à celui d’un puits, est attaché à une extrémité du cartahu tandis que l’autre bout, se déroulant à travers le pont, est tenu par deux ou trois matelots attentifs. Ils hissent le seau à portée de l’Indien à qui un autre a tendu une longue perche. Tashtego, introduisant cette perche dans le seau, le guide dans la boîte jusqu’à ce qu’il y disparaisse complètement, puis, à son signal, les hommes au cartahu le hissent et le seau réapparaît tout écumant comme celui de la laitière à l’heure de la traite. Abaissé délicatement de ses hauteurs, le récipient débordant est saisi par l’homme affecté à le vider rapidement dans une grande cuve. Cette navette s’accomplit jusqu’à l’épuisement de la citerne. Vers la fin, Tashtego doit pousser sa longue perche toujours plus fort et toujours plus profond à l’intérieur de la boîte jusqu’à l’engloutir de près de vingt pieds.\par
Les hommes du {\itshape Péquod} avaient ainsi puisé depuis un bon moment, plusieurs cuves étaient pleines d’odorant spermaceti, lorsque se produisit un singulier accident. Ce sauvage Indien de Tashtego fut-il assez étourdi et insouciant pour lâcher un instant les filins des caliornes où il s’assurait d’une main, l’endroit où il se tenait était-il traîtreusement glissant, ou bien le Malin le voulut-il ainsi sans fournir ses raisons personnelles ? Comment cela se passa-t-il, on ne saurait le dire, mais soudain, alors que remontait pour la quatre-vingtième ou la quatre-vingt-dixième fois le seau pompeur, Seigneur ! le pauvre Tashtego, tel le seau jumeau de va-et-vient d’un véritable puits, tomba la tête la première dans cet immense foudre d’Heidelberg et disparut à la \hspace{1em}vue dans un horrible gargouillis d’huile !\par
– Un homme à la mer ! s’écria Daggoo qui fut le premier à reprendre ses esprits dans la consternation générale. Envoyez le seau par ici ! Il posa un pied dedans afin d’affermir sa prise sur le cartahu glissant, et les hommes le hissèrent au sommet de la tête presque avant que Tashtego ait eu le temps d’en atteindre le fond. Une grande agitation régna ; regardant par-dessus bord, les hommes virent la tête jusque-là inerte palpiter violemment et se soulever au-dessus de l’eau, comme traversée d’une idée importante, alors que ces soubresauts révélaient seulement à son insu à quelle profondeur dangereuse le pauvre Indien avait sombré.\par
Au même moment, tandis que Daggoo, au sommet de la tête dégageait le cartahu embarrassé dans les caliornes, on entendit un craquement sonore, et à l’inexprimable horreur de chacun, l’un des deux énormes crocs, auxquels était suspendue la tête, lâcha et avec une immense oscillation, celui-ci se balança latéralement tant et si bien que le navire ivre roula et trembla comme s’il avait été heurté par un iceberg. Le croc restant devait maintenant soutenir tout le poids et semblait devoir céder à chaque instant, d’autant plus que la tête accomplissait des mouvements désordonnés.\par
– Descendez ! descendez ! crièrent les matelots à Daggoo qui se cramponnait d’une main aux lourdes caliornes, de façon à rester suspendu si la tête venait à se décrocher, mais le nègre, ayant dégagé le filin, poussa le seau vers le fond affaissé du puits, dans l’espoir que le harponneur enseveli pourrait le saisir et être tiré du gouffre.\par
– Au nom du ciel, homme, s’écria Stubb, vous croyez-vous en train de bourrer une cartouche ? Baste ! À quoi lui servira-t-il qu’on lui cogne sur la tête avec un seau de fer ? Baste, compris !\par
Un hurlement fusa : – Gare la caliorne !\par
Presque au même instant, avec un fracas de tonnerre, la masse énorme tomba dans la mer, comme la table rocheuse du Niagara dans le tourbillon des chutes. La coque brutalement soulagée roula en s’en écartant jusqu’à montrer le doublage de cuivre brillant de la carène. Tous retinrent leur souffle, tandis qu’ils regardaient Daggoo qui, accroché à la caliorne balancée tantôt au-dessus de leurs têtes, tantôt au-dessus de l’eau, disparaissait à demi dans la brume épaisse de l’embrun, pendant que le pauvre Tashtego, enseveli vivant, coulait jusqu’au fin fond de la mer ! À peine ce brouillard s’était-il dissipé, qu’on vit le temps d’un éclair une silhouette nue, un sabre d’abordage à la main, bondir par-dessus bord. Puis un éclaboussement d’eau indiqua que mon brave Queequeg avait plongé ; tous se ruèrent de ce côté et tous les yeux se rivèrent sur le moindre frémissement de l’eau, mais le temps passait et rien ne trahissait la présence ni du noyé ni de son sauveteur. Quelques hommes sautèrent alors dans une pirogue et débordèrent quelque peu.\par
– Ha ! Ha ! s’écria tout à coup Daggoo du sommet de son perchoir maintenant immobile et, regardant plus au loin, nous vîmes un bras dressé, tout droit au-dessus des vagues bleues, vision aussi étrange qu’un bras qui aurait jailli de l’herbe d’une tombe.\par
– Deux ! deux ! il est deux ! fut son cri joyeux. Bientôt après, on vit Queequeg luttant hardiment d’une main de l’autre tirant l’Indien par ses longs cheveux. Repêchés par la baleinière, ils furent promptement hissés sur le pont, mais Tashtego fut long à revenir à lui et Queequeg n’était guère plus florissant.\par
Comment ce noble sauvetage avait-il été accompli ! Eh bien, plongeant après la tête qui coulait lentement Queequeg l’avait entamée par le fond sur les côtés afin d’y pratiquer avec son sabre d’abordage effilé une large ouverture puis, abandonnant sa lame, il y avait enfilé profondément son bras et avait tiré le pauvre Tash par les cheveux. Il déclara avoir tout d’abord saisi une jambe, mais que sachant bien que cela n’était pas ce qu’il fallait et pourrait occasionner de graves difficultés, il la repoussa à l’intérieur puis, d’une adroite secousse, il avait fait faire la culbute à l’Indien de sorte qu’à la tentative suivante il se présenta à la bonne vieille manière… par la tête. Quant à la grande tête elle-même, elle se portait aussi bien que possible.\par
Ainsi, grâce au courage et aux talents d’obstétricien de Queequeg, la délivrance, ou plutôt l’accouchement de Tashtego se passa fort bien, malgré les circonstances adverses et apparemment désespérées. C’est une leçon à ne jamais oublier. Le métier de sage-femme devrait être enseigné au même titre que l’escrime et la boxe, l’équitation et l’aviron.\par
Je sais que cette curieuse aventure survenue au {\itshape Gay- Header} paraîtra à coup sûr invraisemblable à certains terriens, bien qu’ils aient vu tomber de leurs propres yeux ou entendu parler de quelqu’un tombé dans un puits, accident qui n’est point si rare et se justifie moins aisément que celui de l’Indien si l’on pense à quel point est glissante la margelle de la citerne du cachalot.\par
Il se pourrait qu’on m’objectât avec perspicacité : « Comment cela se fait-il ? Nous avions cru comprendre que cette tête alvéolée et imprégnée d’huile du cachalot était sa partie la plus légère, flottant comme un bouchon, et maintenant tu la fais couler dans un élément d’une beaucoup plus grande densité que la sienne propre. Nous te tenons là ! » Pas du tout, c’est moi qui vous tiens. Car au moment où le pauvre Tashtego y était tombé, la boîte avait été presque entièrement vidée de son contenu le plus léger, ne laissant guère que la paroi tendineuse, forgée, soudée de la citerne qui, je l’ai dit antérieurement, est beaucoup plus lourde que l’eau de mer et dont un morceau coule comme du plomb ou presque. Mais la tendance de cette matière à aller rapidement au fond était, en l’occurrence, compensée par les autres parties de la tête qui y demeuraient attachées, de sorte qu’elle s’enfonçait très lentement et posément, accordant à Queequeg une bonne chance de réaliser son accouchement lestement au pas de course, si j’ose dire. Oui, ce fut bien un accouchement essoufflant !\par
Que Tashtego eût péri dans cette tête, quelle mort précieuse il aurait eue ! Enfoui dans la suprême blancheur, dans l’extrême raffinement de l’odorant spermaceti, ayant à la fois son cercueil, son corbillard et sa tombe dans la plus intérieure chambre secrète, dans le saint des saints du cachalot. Je ne connais qu’un exemple d’une fin plus douce encore, la mort délicieuse d’un chasseur d’abeilles de l’Ohio, qui en trouva une telle quantité dans la fourche d’un arbre creux, qu’il se pencha trop avant et y fut aspiré, mourant ainsi embaumé. Combien sont-ils, pensez-vous, à être ainsi tombés dans le miel logé dans la tête de Platon et à y avoir suavement péri ?
\chapterclose


\chapteropen
\chapter[{CHAPITRE LXXIX. La prairie}]{CHAPITRE LXXIX \\
La prairie}\renewcommand{\leftmark}{CHAPITRE LXXIX \\
La prairie}


\chaptercont
\noindent Examiner les lignes de sa face ou tâter les bosses de la tête de ce léviathan, voilà ce que n’ont jamais entrepris ni un physiognomoniste, ni un phrénologiste. Une telle tentative semblerait aussi pleine d’espoir que celle d’un Lavater scrutant les rides du rocher de Gibraltar ou de Gall montant sur une échelle afin de palper le dôme du Panthéon. Pourtant, dans son célèbre ouvrage, Lavater étudie non seulement le visage humain mais encore la physionomie des chevaux, des oiseaux, des serpents, et des poissons, il analyse même en détail toutes leurs expressions. Gall et son disciple Spurzheim ne manquent pas non plus d’évoquer des caractéristiques phrénologiques d’autres créatures que l’homme. Aussi, bien que je ne sois guère un pionnier qualifié, je me risquerai à tenter d’appliquer ces deux demisciences à la baleine. J’essaie tout, je réalise ce que je peux.\par
Du point de vue physiognomonique, le cachalot est une créature anormale. Il n’a pas de vrai nez. Or, le nez étant le trait le plus saillant, le plus central, modifie peut-être tous les autres et commande l’expression générale ; il semble dès lors que son absence totale, en tant qu’appendice, doive largement influer sur l’aspect de la baleine. L’art du jardinier-paysagiste veut qu’une flèche, une coupole, un monument ou une tour quelconque, soit presque indispensable à l’harmonie de la scène ; de même un visage n’a pas de physionomie sans le beffroi ajouré du nez. Brisez celui du Jupiter de Phidias, quel triste reste vous aurez ! Toutefois, le léviathan a une stature si puissante, ses proportions sont si majestueuses que ce manque, qui rendrait hideux Jupiter, ne lui nuit en rien. C’est, au contraire, un surcroît de grandeur. À la baleine un nez serait une impertinence. Tandis que pour ce voyage d’étude vous naviguez autour de cette immense tête dans votre petit canot, la noble opinion que vous vous faites d’elle n’est jamais entamée par l’idée qu’elle a un nez qu’on pourrait tirer. Exécrable trait d’esprit qui vous hantera souvent, même en contemplant le plus imposant fonctionnaire royal sur son trône.\par
À certains égards, la vue la plus imposante qu’on puisse avoir de la physionomie du cachalot, c’est absolument de face. Vu ainsi, il est sublime.\par
Un beau front humain méditant est pareil à l’est ému par le lever du jour. Dans le calme des pâturages, le front bouclé du taureau n’est pas sans grandeur. Le front de l’éléphant qui tire de lourds canons dans les défilés de montagne est empreint de majesté. Humain ou animal, le front mystique est pareil au grand sceau d’or que les empereurs du Saint Empire romain apposaient sur leurs ordonnances. Il signifie : « Fait de ma main ce jour, Dieu. » Mais, chez la plupart des créatures, qui plus est chez l’homme même, le front n’est très souvent qu’une simple bande de terre alpine à la limite des neiges. Il y a peu de fronts qui, tels ceux de Shakespeare et de Mélanchtoe, s’élèvent si haut et descendent si bas que les yeux semblent des lacs de montagne limpides, éternels, immobiles, et au-dessus d’eux, dans les rides du front, il semble que l’on puisse suivre la piste des pensées, descendues pour y boire, pareilles aux cerfs dont le chasseur des montagnes suit la trace sur la neige. Mais chez le grand cachalot cette dignité suprême et puissante du front presque divine, est si immensément amplifiée, qu’à le voir ainsi de face, s’impose à vous, avec plus de force que n’en saurait dispenser tout autre objet de la vivante nature, le sentiment de la divinité et des puissances redoutables. Car ce qu’on voit est indéfini et ne présente aucun trait distinct : ni nez, ni yeux, ni oreilles, ni bouche, ni visage car il n’en a pas à proprement parler, rien d’autre que ce vaste firmament du front, sillonné d’énigmes, silencieusement mis à la mer pour la perte des navires et des hommes. Vu de profil, ce front merveilleux ne perd pas sa grandeur mais elle est moins impérieuse, on découvre clairement dans son centre cette dépression horizontale en demi-lune qui est, chez l’homme, selon Lavater la marque du génie.\par
Mais quoi ! Du génie chez le cachalot ? A-t-il écrit un livre, prononcé un discours ? Non, son grand génie, accru par son silence pyramidal, se révèle à ce qu’il ne fait rien pour le prouver. Je pense que la religiosité naïve du jeune Orient eût divinisé le cachalot s’il l’avait connu. Le crocodile du Nil a été divinisé parce qu’il n’a pas de langue, or le cachalot n’a pas de langue ; du moins est-elle si menue qu’elle est incapable de protraction. Si, à l’avenir, une nation de haute culture, goûtant la poésie, devait rétablir le droit d’aînesse, les joyeuses divinités du printemps de jadis, leur rendre la vie et les introniser à nouveau dans ce ciel à présent si sectaire, sur cette colline à présent déserte, alors soyez sûrs qu’élevé sur le trône de Jupiter, le grand cachalot régnerait.\par
Champollion a déchiffré les rides de granit des hiéroglyphes mais il n’est pas de Champollion pour déchiffrer l’Égypte de chaque homme et le visage de chaque être. La physiognomonie, comme toute science humaine, n’est qu’une fable éphémère. Dès lors, si sir William Jones, qui lisait trente langues, ne pouvait lire le sens profond et subtil du plus simple visage de paysan, comment l’ignorant Ismaël pourrait-il espérer lire le terrible front chaldéen du cachalot ? Je vous l’offre, déchiffrez-le si vous le pouvez.
\chapterclose


\chapteropen
\chapter[{CHAPITRE LXXX. La noix}]{CHAPITRE LXXX \\
La noix}\renewcommand{\leftmark}{CHAPITRE LXXX \\
La noix}


\chaptercont
\noindent Si le cachalot est un sphinx pour le physiognomoniste, son cerveau représente pour le phrénologiste la quadrature du cercle.\par
Chez le sujet adulte, le crâne mesure au moins vingt pieds de long. Si l’on enlève sa mâchoire inférieure, son crâne vu de profil ressemblera à une varlope inclinée reposant sur une surface plane. Mais lorsqu’il est vivant comme nous l’avons déjà vu, ce plan incliné est angulairement rempli, de façon à devenir presque carré, par l’énorme masse superposée de pâté et de spermaceti à son sommet, le crâne forme un cratère pour les recevoir et, dans une autre cavité, sous le long plancher de ce cratère, dans une autre cavité dépassant rarement dix pouces de longueur sur autant de profondeur, se trouve le cerveau du monstre, pas plus gros que le poing. Il se situe à vingt pieds au moins du front, caché derrière ses vastes fortifications comme la citadelle de Québec à l’intérieur du labyrinthe de ses ouvrages de défense. Il est caché en lui comme un coffret précieux et j’ai connu des baleiniers pour dénier au cachalot tout autre cerveau que l’apparence fournie par la boîte à spermaceti. Dans leur idée, ses plis étranges et ses circonvolutions correspondent mieux à la notion de puissance qu’ils ont du cachalot et ils considèrent cette boîte mystérieuse comme le siège de son intelligence.\par
Il est certain que la tête du léviathan vivant est phrénologiquement trompeuse, rien ne signale son vrai cerveau. Comme toute puissance, le cachalot n’offre au commun qu’une façade.\par
Son crâne une fois délesté du spermaceti, si vous regardez son cerveau depuis l’arrière, vous serez frappé de sa ressemblance avec un cerveau humain vu sous le même angle. En vérité, si l’on plaçait ce crâne vu de dos, ramené à l’échelle de celui de l’homme, parmi des crânes humains, vous les confondriez et, notant les dépressions qu’il présente à son sommet, vous déclareriez en jargon de phrénologie : « Cet homme n’avait de luimême ni estime ni vénération exagérées. » Mise en parallèle avec la masse et la puissance du cachalot cette assertion vous donnera l’idée la plus vraie et la plus réjouissante de ce qu’est la suprême puissance.\par
Mais si vous estimez que vu ses dimensions relatives on ne peut situer sainement le cerveau du cachalot, je vous suggérerai autre chose. Regardez attentivement les vertèbres d’à peu près n’importe quel quadrupède, vous serez frappé de la ressemblance qu’elles présentent avec un collier de crânes en réduction, chacune d’elles offrant une analogie rudimentaire avec le crâne proprement dit. Une théorie allemande veut que les vertèbres soient des embryons de crânes ; je ne pense pas que les Allemands aient été les premiers à relever cette curieuse ressemblance. Un mien ami étranger me la fit une fois remarquer tandis qu’il travaillait en bas-relief à orner la proue relevée de son canoë des vertèbres d’un ennemi qu’il avait tué. C’est, à mon avis, un manque grave de la part des phrénologistes que de n’avoir pas poussé leurs investigations du cervelet jusqu’au cordon cérébro-spinal, car je crois qu’il y a beaucoup à découvrir sur le caractère d’un homme dans son épine dorsale et, qui que vous soyez, j’aimerais mieux vous palper la colonne que le crâne. Jamais frêle colonne n’a soutenu la charpente d’une âme mûre et noble. Je suis enchanté de mon épine dorsale, hampe fière et hardie de ce drapeau que je déploie à demi face au monde.\par
Appliquons cette partie de la phrénologie au cachalot. Sa première vertèbre cervicale continue sa boîte crânienne et le canal qui la traverse, affectant la forme d’un triangle pointe en haut, mesure dix pouces de large sur huit de haut, conserve une grande capacité assez longuement et va en se rétrécissant dans les dernières vertèbres. Le cordon cérébro-spinal est bien entendu rempli d’une substance assez analogue à celle du cerveau ; qui plus est, il garde, sur plusieurs pieds de longueur, la dimension qu’il avait au sortir du cerveau, à peu près égale à celle de ce dernier. Serait-il dès lors déraisonnable de dresser la carte de l’épine dorsale du cachalot, puisque, vue sous ce jour, l’étonnante petitesse de son cerveau proprement dit est plus que largement compensée par la merveilleuse ampleur de sa moelle épinière ?\par
Mais laissons aux phrénologistes le soin d’exploiter cette suggestion ; je n’en traiterai que par rapport à la bosse du cachalot. Si je ne me trompe, cette bosse auguste s’élève sur l’une des plus grosses vertèbres, formant en quelque sorte son relief extérieur. À cause de sa position, je dirai que cette haute bosse est le siège de la fermeté et du caractère indomptable du cachalot. Que le cachalot est indomptable, vous allez l’apprendre bientôt.
\chapterclose


\chapteropen
\chapter[{CHAPITRE LXXXI. Le Péquod rencontre la Vierge}]{CHAPITRE LXXXI \\
\textbf{{\itshape Le} Péquod {\itshape rencontre la} Vierge}}\renewcommand{\leftmark}{CHAPITRE LXXXI \\
\textbf{{\itshape Le} Péquod {\itshape rencontre la} Vierge}}


\chaptercont
\noindent Au jour voulu par le destin, nous rencontrâmes le baleinier {\itshape Jungfrau}, capitaine : Derick de Deer, de Brème.\par
Les Hollandais et les Allemands, autrefois les plus grands baleiniers du monde, sont devenus les moindres mais ici ou là, à de très grands écarts de longitude et de latitude, on rencontre parfois leur pavillon dans le Pacifique.\par
Pour une raison inconnue, la {\itshape Vierge} témoignait de beaucoup d’ardeur à vouloir nous présenter ses hommages. À bonne distance encore du {\itshape Péquod}, elle lofa, mit à la mer une pirogue dans laquelle son capitaine impatient se tenait à l’avant et non à l’arrière, comme il est d’usage.\par
– Qu’est-ce qu’il tient dans la main ? s’étonna Starbuck en désignant quelque chose que brandissait l’Allemand. Pas possible !… une burette pour les fanaux ?\par
– Non, intervint Stubb, non, non, c’est une cafetière, monsieur Starbuck, il vient nous faire le café, c’est l’homme à tout faire, ne voyez-vous pas ce grand bidon à côté de lui ? C’est son eau bouillante. Oh ! il est un peu là, cet homme à tout faire !\par
– Ne nous racontez pas d’histoires, dit Flask, c’est une burette pour les lampes et un bidon à huile, il n’en a plus et vient nous en mendier.\par
Pour curieux que paraisse le fait qu’un navire à huile soit contraint d’emprunter de l’huile sur un parage de chasse, et bien que cela fasse mentir le vieux dicton parlant de porter du charbon à Newcastle, le cas se produit parfois et le capitaine Derick de Deer venait bel et bien avec une burette, comme l’avait dit Flask.\par
Dès qu’il fut sur notre pont, le capitaine Achab l’aborda brusquement, sans prêter la moindre attention à ce qu’il tenait à la main, mais, dans son jargon hésitant, l’Allemand manifesta bientôt son ignorance complète de la Baleine blanche, et détournant aussitôt la conversation sur son bidon à huile, il fit quelques remarques relatives à l’obligation où il était de regagner son hamac dans la plus profonde obscurité, ayant épuisé jusqu’à la dernière goutte sa provision d’huile de Brème, et n’ayant même pas pris un poisson volant pour suppléer à ce manque ; il conclut en disant que son navire était vraiment ce qu’on appelle en terme de pêcherie un navire allège (c’est-à-dire vide), méritant bien son nom de {\itshape Jungfrau} ou {\itshape Vierge}.\par
Ses besoins pourvus, Derick s’en fut, mais il n’avait pas atteint le flanc de son navire que des baleines furent signalées simultanément par les hommes en vigie des deux bâtiments, et Derick était si acharné à leur poursuite que, sans perdre le temps de déposer à son bord son récipient, il fit virer sa pirogue et amena sur le léviathan réservoir d’huile.\par
Le gibier étant apparu sous le vent, sa baleinière et trois autres du navire allemand qui l’avaient suivie, prirent une bonne avance sur celles du {\itshape Péquod}. C’était une gamme de huit cachalots, ce qui est un groupe habituel. Alertés, ils fuyaient droit devant eux vent arrière, flancs contre flancs comme des chevaux attelés, laissant un large sillage blanc pareil à un parchemin sans cesse déroulé sur la mer.\par
Au milieu de ce sillage rapide, bien des brasses en arrière, suivait un vieux mâle énorme, bossu, dont la vitesse moins grande, les plaques jaunâtres dont il était couvert, le faisaient paraître atteint de la jaunisse ou de quelque autre infirmité. On pouvait se demander s’il faisait partie de la gamme qui le précédait car de tels vétérans ne sont ordinairement pas du tout sociables. Pourtant, il n’abandonnait pas le sillage, bien qu’en vérité ce remous dût le ralentir car il se brisait contre son large mufle avec la fougue de deux courants contraires qui s’affrontent. Son souffle court, lent et difficile, jaillissait comme suffoqué, s’épuisait en charpie, était suivi d’étranges commotions internes semblant avoir leur issue à son autre extrémité invisible car des bulles montaient de l’eau derrière lui.\par
– Qui a du parégorique ? demanda Stubb, il a mal au ventre, je crains. Seigneur, imaginez ce que ce doit être d’avoir la colique sur un parcours d’un demi-arpent ! Des vents contraires mènent le bal en lui, les gars. C’est la première fois que je vois un vent fétide soufflant à l’arrière, mais avez-vous jamais vu une baleine faire de pareilles embardées, elle doit avoir perdu son gouvernail.\par
Tel un navire trop chargé, descendant les côtes de l’Inde, portant un fret de chevaux épouvantés, donne de la bande, pique, roule et se vautre, de même cette vieille baleine soulevait son corps pesant, et, de temps à autre, se retournant lourdement sur le flanc, révélait la raison de sa démarche tortueuse due à sa nageoire de tribord réduite à l’état de moignon. Il est difficile de savoir si elle avait perdu sa nageoire au combat ou si c’était congénital.\par
– Attends un peu, ma vieille, et je vais te mettre ce bras blessé en écharpe, s’écria Flask avec cruauté en lui montrant la ligne à côté de lui.\par
– Prends garde qu’elle ne te mette toi-même en écharpe, répartit Starbuck. Nagez partout ! sinon c’est l’Allemand qui l’aura.\par
Les pirogues rivales amenaient toutes sur ce poisson, non seulement parce qu’étant plus gros il avait davantage de valeur, mais encore parce qu’il était plus proche d’eux. Et la gamme filait à une allure telle qu’elle défiait toute poursuite pour l’instant. Au moment critique, les baleinières du {\itshape Péquod} avaient dépassé les trois pirogues que les Allemands avaient mises à la mer après celle de Derick, mais celui-ci avait une telle avance qu’il tenait toujours la tête, bien que ses rivaux étrangers s’en rapprochassent toujours davantage. Ces derniers craignaient de ne pouvoir le gagner de vitesse avant que Derick ait jeté le fer à sa marque, ce dont il paraissait lui-même très sûr car il faisait de temps à autre un geste de dérision vers eux avec sa burette pour les lampes.\par
– Chien mal léché et ingrat ! s’écria Starbuck. Il me raille et me défie avec la sébile même que j’ai remplie pour lui il n’y a pas cinq minutes ! Puis, avec son habituel chuchotement intense, il ajouta : Nagez partout, lévriers ! Talonnez-le !\par
– Je vais vous dire, les gars, disait Stubb à son équipage, c’est contre ma religion de sortir de mes gonds, mais j’aimerais ne faire qu’une bouchée de cet infâme homme à tout faire. Nagez, voulez-vous ? Allez-vous vous laisser battre par ce scélérat ? Aimez-vous l’eau de vie ? Une barrique d’eau de vie au meilleur d’entre vous. Allons, l’un d’entre vous ne veut-il pas se faire sauter les artères ? Qui a jeté l’ancre, nous ne bougeons pas d’un pouce, nous sommes encalminés. Hello, voici l’herbe qui pousse au fond de la pirogue et, juste ciel, voici le mât qui bourgeonne. Ça ne va pas comme ça, les gars ! Regardez l’homme à tout \hspace{1em}faire ! Je vous le répète, en long et en large, les gars, voulez-vous oui ou non cracher du feu ?\par
– Oh ! voyez-vous cette eau de savon qu’elle fait ! disait Flask en trépignant. Quelle bosse ! Oh ! jetez-vous sur le bœuf, je vous en prie… il est là comme une souche ! Oh ! mes amis, que ça saute, je vous en prie… des crêpes et des clovisses, vous savez, mes amis, pour le souper… des coquillages cuits et des pains au lait… ho ! je vous en prie, je vous en prie, sautez… c’est un cent-barils, ne le perdez pas… oh ! non ne le perdez pas, je vous en prie… regardez cet homme à tout faire… Oh ! ne nagerez-vous pas pour du gâteau, mes amis… Vous vous esquivez ! Quels tire-au-flanc ! N’aimez-vous pas le spermaceti ? Ça représente trois mille dollars, les gars ! une banque ! toute une banque ! La banque d’Angleterre ! Oh ! je vous en prie, je vous en prie… Qu’est-ce qu’il fabrique à présent cet homme à tout faire ?\par
Derick était en train de jeter sa burette et son bidon aux pirogues qui avançaient, peut-être avec le double objectif de retarder ses rivaux tout en accélérant lui-même grâce à l’impulsion momentanée que cela provoquerait sur son arrière.\par
– Le rustre marin de dogre hollandais ! s’écria Stubb, nagez à présent, les hommes, comme cinquante mille vaisseaux de ligne avec un chargement de diables à cheveux rouges. Qu’en dites-vous, Tashtego ? Êtes-vous l’homme à vous rompre l’échine en vingt-deux morceaux en l’honneur du vieux Gay- Head ? Qu’en dites-vous ?\par
– Je dis que Dieu me damne si je peux tirer plus fort, répondit l’Indien.\par
Âprement et régulièrement aiguillonnées par les sarcasmes de l’Allemand, les trois baleinières du {\itshape Péquod} avançaient maintenant presque de front et, dans cet ordre, le rattrapèrent momentanément. Dans cette magnifique attitude souple et chevaleresque des chefs lorsqu’ils s’approchent de leur proie, les trois seconds encourageaient le dernier rameur d’exclamations stimulantes : « La voilà qui glisse ! Hourra pour la brise du frère blanc ! À bas l’homme à tout faire ! Dépassons-le ! »\par
Mais l’avance que Derick avait prise au départ était telle que, malgré leur vaillance, il eût été vainqueur de la course si la justice immanente n’était intervenue. Le rameur du milieu fit fausse rame ! Tandis que cet empoté luttait pour dégager son aviron, mettant ainsi la pirogue de Derick en danger de chavirer et provoquant chez lui des hurlements de fureur exacerbée, Starbuck, Stubb et Flask jubilaient. Avec un cri, ils se ruèrent brusquement en avant et se rangèrent obliquement près de l’Allemand. L’instant d’après, les quatre pirogues se trouvèrent en diagonale dans le sillage même de la baleine dont le remous écumeux s’étendait de part et d’autre de leurs flancs.\par
Ce spectacle était à la fois exaspérant, terrifiant et lamentable. La baleine fuyait maintenant, la tête dressée hors de l’eau, expulsant sans discontinuer un souffle torturé, tandis que sa pauvre et unique nageoire battait son flanc dans une agonie de terreur. Elle embardait tantôt d’un côté, tantôt de l’autre dans sa fuite incertaine et, à chaque lame qu’elle fendait, elle plongeait convulsivement dans la mer ou, roulant sur le côté, battait l’air de sa seule nageoire. J’ai vu ainsi un oiseau, l’aile pincée, décrire, épouvanté, des cercles brisés dans le ciel, dans l’effort désespéré d’échapper à la piraterie du faucon. Mais l’oiseau avait une voix et disait sa peur en cris plaintifs, tandis que celle de cette brute géante de la mer était enchaînée en elle par un envoûtement. Elle était sans voix, hormis cette respiration dont son évent trahissait l’étouffement. Tout cela la rendait indiciblement pitoyable et pourtant sa masse étonnante, la herse de sa mâchoire, sa queue toute-puissante suffisaient à glacer d’épouvante la pitié du plus courageux.\par
Réalisant que quelques instants de plus assureraient l’avantage aux baleinières du {\itshape Péquod}, et plutôt que d’être ainsi frustré du gibier, Derick choisit de risquer un lancer \hspace{1em}exceptionnellement long, de crainte que ne lui échappe à jamais sa dernière chance.\par
Mais son harponneur ne s’était pas plutôt dressé que les trois tigres, Queequeg, Tashtego et Daggoo, giclèrent instinctivement sur leurs pieds, et en rang oblique, pointèrent ensemble leurs fers et les lancèrent par-dessus la tête du harponneur allemand, et les trois marques de Nantucket pénétrèrent la baleine. Fusée d’écume aveuglante, flammes blanches ! Emportées par la fureur du bond en avant de la bête, les trois pirogues heurtèrent le flanc de l’Allemand avec une force telle que Derick et son harponneur déjoué furent jetés à l’eau tandis que passait sur eux le vol des trois quilles.\par
– N’ayez pas peur, mes boîtes à beurre, leur cria Stubb en leur jetant un regard au passage éclair, vous serez récoltés dans un instant… et très bien… j’ai vu des requins là derrière, ce sont de vrais saint-bernards vous savez… ils secourent les voyageurs en détresse. Hourra ! c’est ce que j’appelle naviguer. Chaque quille est un rayon de soleil ! Hourra ! Nous voilà comme trois casseroles à la queue d’un couguar enragé ! Cela m’invite à atteler, en plaine, un éléphant à un tilbury ; attaché de la sorte, il fait voler les rayons de roues, les gars ! et on court aussi le danger d’être projeté au-dehors quand on aborde une colline. Hourra, voilà ce qu’éprouve un gars quand il se rend dans la grande Baille, un galop descendant une pente ! Hourra ! cette baleine porte le courrier de l’éternité.\par
Mais la course du monstre fut brève. Avec une suffocation brusque, il sonda tumultueusement. Les trois lignes filèrent sur les taberins, avec un raclement si violent et une force telle qu’elles y creusèrent des sillons profonds. Les harponneurs, redoutant que cette plongée rapide n’épuise bientôt les lignes, donnèrent de toutes leurs forces et avec adresse, plusieurs tours à la ligne fumante, jusqu’à ce qu’ensuite de l’effort perpendiculaire sur les engoujures de plomb d’où les trois lignes filaient dans les profondeurs, les plats-bords de l’étrave se trouvassent presque au niveau de l’eau, tandis que les proues se dressaient vers le ciel. La baleine cessant bientôt de sonder, ils restèrent un moment dans cette posture délicate, craignant de donner davantage de ligne. Quoique des pirogues aient été tirées vers le fond et perdues de cette manière, cet amarrage comme on l’appelle, ces barbelures acérées plantées dans sa chair vive, sont bien souvent pour le léviathan l’épreuve torturante qui l’incite à remonter en surface pour retrouver la lance aiguë de ses ennemis. Indépendamment du danger que comporte ce procédé, on peut douter qu’il soit toujours le meilleur, néanmoins on peut conclure logiquement que plus la baleine harponnée reste longtemps sous l’eau, plus elle s’épuise, car, étant donné sa surface énorme – un cachalot adulte mesure à peine moins de 2 000 pieds carrés –, la pression exercée est également énorme. Nous savons tous sous quel poids étonnant d’atmosphère nous évoluons à ciel libre, combien plus lourd le fardeau que doit supporter la baleine sur laquelle pèse une colonne de deux cents brasses d’Océan ! Ce doit être au moins l’équivalent de cinquante atmosphères. Un baleinier en a estimé le poids à celui de vingt vaisseaux de guerre avec tous leurs canons, leur matériel et leurs hommes.\par
Tandis que les trois pirogues se berçaient sur la houle légère, que les regards fixaient l’éternel midi de ses profondeurs bleues, sans qu’un cri, un gémissement, sans qu’une ride, une bulle même ne montent de ses abîmes, quel terrien eût pu penser que sous tant de silence et de paix le géant des mers se tordait déchiré par les affres de l’agonie ! On ne voyait pas plus de huit pouces de ligne perpendiculaire à l’étrave, il semblait incroyable que le léviathan fût suspendu à trois fins fils de caret comme le gros poids d’une horloge à balancier. Suspendu ? Et à quoi ? À trois bouts de planches. Est-ce là la créature dont il fut une fois dit triomphalement : « Perceras-tu sa peau d’un dard ? ou sa tête d’un hameçon ? Quand on l’approche, l’épée ne sert à rien, pas plus que la lance, le dard ou la cuirasse, pour lui le \hspace{1em}fer est comme de la paille, la flèche ne le fait pas fuir, les pierres de la fronde sont pour lui comme du chaume, il se rit du frémissement des javelots ! » Est-ce là la même créature ? Est-ce bien elle ? Oh ! ces prophéties qui ne s’accomplissent pas ! Car le léviathan, dont la queue a plus de force que mille cuisses, a poussé sa tête sous les montagnes de la mer, afin de se cacher des javelots du {\itshape Péquod} !\par
À cette heure de l’après-midi où le soleil décline, les ombres que les trois pirogues plongeaient sous la surface de l’eau devaient être assez longues et assez larges pour envelopper la moitié de l’armée de Xerxès. Qui sait la terreur dont leurs fantômes mouvants pouvaient envahir la baleine blessée !\par
– Paré ! il bouge, les gars… cria Starbuck, tandis que les trois lignes vibraient soudain dans l’eau, conduisant jusqu’à eux, comme un fil magnétique, les pulsations de la vie et de la mort de la baleine dont la trépidation se transmettait jusqu’aux bancs des canotiers. L’instant d’après, les pirogues, soulagées de la traction vers le bas, se redressèrent d’un bond, comme un petit iceberg secoué par le saut apeuré d’une troupe serrée d’ours blancs.\par
– Rentrez la ligne ! cria à nouveau Starbuck, il remonte ! Les lignes dont on n’aurait pu, l’instant d’auparavant, gagner une largeur de main, étaient rentrées toutes ruisselantes dans les pirogues et lovées à la hâte, et bientôt la baleine émergea à une encablure des chasseurs.\par
Ses mouvements disaient son épuisement. Les veines de la plupart des animaux terrestres sont munies de valves ou de vannes, qui, lorsqu’ils sont blessés, empêchent le sang de s’épancher, du moins sur le moment. Il n’en va pas de même chez la baleine, dont l’une des caractéristiques est de n’avoir pas de système valvulaire de sorte que, lorsqu’elle est transpercée par une pointe, fût-elle aussi petite que celle d’un harpon, un écoulement mortel se déclenche dans tout son système artériel. Et lorsque cette hémorragie est accentuée par la formidable pression de l’eau à de grandes profondeurs, on peut dire que la vie s’échappe d’elle à flots continus. Néanmoins, elle a une telle quantité de sang, ses sources intérieures sont si nombreuses et si éloignées les unes des autres, qu’elle ira perdant son sang encore et encore pendant fort longtemps, comme en période de sécheresse continuera de couler la rivière dont la source jaillit dans de lointaines collines invisibles. Même à présent que les pirogues amenaient sur cette baleine, évitant le proche danger de sa queue battante, maintenant que de nouvelles lances ouvraient de nouvelles blessures jaillissantes, que son évent ne faisait plus fuser sa buée qu’à intervalles, d’ailleurs rapides, même alors le sang ne giclait pas de cet évent parce qu’aucun centre vital n’avait été encore atteint, sa vie, comme l’appellent les hommes de façon significative, restait intacte.\par
Comme les pirogues la serraient de plus près, la partie, de son corps, habituellement immergée, fut rendue visible. On vit ses yeux ou plutôt la place où elle avait eu des yeux ; comme d’étranges mousses s’insinuent dans les creux des nœuds des plus nobles chênes abattus, dans les orbites que les yeux de la baleine occupaient jadis saillaient des protubérances aveugles, affreusement pitoyables. Mais il n’y avait pas de pitié. Car malgré son grand âge, son unique bras, ses yeux aveugles, elle devait être mise à mort, être assassinée, afin d’illuminer de joyeuses noces et autres réjouissances humaines, briller dans les solennelles églises qui prêchent la non-violence de tous envers chacun. Roulant toujours dans son sang elle découvrit enfin un curieux bouquet décoloré, de la taille d’un boisseau, croissant bas sur son flanc.\par
– Belle cible ! dit Flask, qu’on me laisse la piquer là une fois.\par
– Baste ! s’écria Starbuck, c’est vraiment superflu !\par
Mais l’humain Starbuck n’avait pas été assez prompt. À l’instant où la lance l’atteignait le pus s’échappa de sa cruelle plaie et la baleine éveillée ainsi à une douleur intolérable, soufflant un sang épais, se lança dans une brusque et aveugle furie sur les pirogues, aspergeant de sang les hommes enorgueillis, chavirant la baleinière de Flask et endommageant ses avants. Ce fut le sursaut de la mort. Car elle était désormais si épuisée par son hémorragie qu’elle roula impuissante à l’écart du naufrage qu’elle avait provoqué, se coucha haletante sur le flanc, battit faiblement l’air de son moignon, puis lentement se retourna, petit à petit, comme un monde qui finit, révéla les blancs secrets de son ventre, resta immobile comme une souche et mourut. Rien n’avait été plus pitoyable que son dernier souffle, pareil à un jet d’eau dont des mains invisibles diminuent progressivement la puissance, et dont la colonne avec des gargouillements étouffés de mélancolie décline encore et encore vers le sol. Il en alla ainsi du long souffle d’agonie du cachalot.\par
Tandis que les pirogues attendaient l’arrivée du navire, le corps montra les symptômes de qui va couler, emportant ses trésors intacts. Aussitôt, sur l’ordre de Starbuck, il fut capelé en divers points, les pirogues faisant ainsi office de bouées, la carcasse suspendue à quelques pouces au-dessous d’elles. Lorsque le navire se fut approché, la baleine fut amenée à son flanc avec d’infinies précautions, amarrée par un grelin solide, car il était évident que si des précautions supplémentaires n’étaient pas prises elle irait aussitôt par le fond.\par
Il advint qu’au premier coup de pelle on trouva enchâssé dans sa chair, le fer entier d’un harpon rouillé un peu en dessous du bouquet que nous avons décrit précédemment. Mais il n’est pas rare qu’on trouve des pointes de harpons dans le corps des baleines, une chair parfaitement saine s’étant reformée sur eux et rien d’extérieur ne signalant leur présence. Aussi fallait-il qu’il y ait eu une raison inconnue à l’ulcération susdite. Plus singulière encore la découverte, près de ce fer enrobé, d’une tête de lance en pierre prisonnière de chairs intactes ! Qui avait jeté cette lance de pierre ? Et quand ? Il se pourrait que ce fût celle d’un Indien du Nord-Ouest bien avant que fût connue l’existence de l’Amérique.\par
En cherchant bien, quelles autres merveilles ce meuble monstrueux eût-il livrées ? On ne sait. Il fut brutalement mis fin à de nouvelles trouvailles, car le navire pencha de plus en plus dangereusement sur le côté, entraîné par la tendance croissante que la carcasse avait à s’enfoncer. Toutefois, Starbuck, qui avait le commandement, lutta jusqu’au bout avec une telle obstination, en vérité, que le navire aurait finalement chaviré s’il était resté jumelé à la carcasse, et lorsque l’ordre fut donné de s’en dégager, l’effort trop grand infligé aux têtes d’allonge où les grelins de capelage de la queue étaient amarrés, rendit impossible de les lâcher. Le {\itshape Péquod} engagea à tel point que passer de l’autre côté du pont, c’était marcher sur le pignon abrupt d’un toit. Le navire gémissait et haletait. Nombre des incrustations d’ivoire des bastingages et des cabines sautèrent, déboîtées de manière anormale. En vain essaya-t-on de soulager les chaînes inamovibles avec des leviers et des anspects pour les libérer des têtes d’allonge, et la baleine avait pour lors coulé si bas qu’on ne pouvait atteindre leurs extrémités cependant que des tonnes semblaient s’ajouter au poids de la masse en train de sombrer et que le navire paraissait sur le point de faire quartier.\par
Stubb harangua le cadavre :\par
– Tiens bon ! Tiens bon, veux-tu ? Ne sois pas si diablement pressé de couler ! Par tous les tonnerres, les hommes, il faut faire quelque chose ou suivre le même chemin. Les leviers ne servent à rien, baste, je vous dis, avec vos anspects, que l’un de vous coure chercher un livre de prières et un canif, et coupez les grosses chaînes.\par
– Un canif ? Oui, oui, s’écria Queequeg, et empoignant la lourde hache du charpentier, il se pencha hors d’un sabord et, acier contre fer, s’attaqua aux plus grosses chaînes. Il n’avait donné que quelques coups, fleuris d’étincelles, que la traction trop forte fit le reste. Toutes les amarres lâchèrent avec fracas, le navire se redressa et la carcasse coula.\par
Cette perte inhabituelle d’un cachalot fraîchement tué est un étrange accident dont aucun pêcheur n’a pu jusqu’à ce jour fournir une explication valable. D’ordinaire le cachalot flotte avec beaucoup de légèreté, son flanc ou son ventre largement soulevés au-dessus de l’eau. Si seules les baleines qui coulaient étaient vieilles, maigres et le cœur brisé, leurs rembourrages de lard perdus et les os alourdis de rhumatismes, on pourrait alléguer avec quelque raison que le fait est dû à une gravité spécifique accrue du poisson ayant perdu ses flotteurs de graisse. Mais ce n’est pas le cas. Car de jeunes baleines, resplendissantes de santé, débordantes de nobles aspirations, sont prématurément enlevées au printemps de la vie, dodues de tout leur lard. Pareils héros gras et musclés vont parfois par le fond.\par
Ajoutons encore que le cachalot est beaucoup moins sujet à cet accident que tout autre cétacé. Pour un cachalot, il sombre vingt baleines franches, leur ossature beaucoup plus importante en est sans doute responsable, leurs stores vénitiens à eux seuls pesant souvent plus d’une tonne ; le cachalot, lui, n’en est pas embarrassé. En certains cas, après quelques heures ou quelques jours, la baleine remonte en surface plus légère qu’au cours de sa vie, la raison en est évidente. Des gaz se sont formés qui la dilatent prodigieusement et elle devient une sorte de ballon animal. Un vaisseau de guerre ne suffirait pas alors à l’enfoncer. Dans les pêches côtières des baies de la Nouvelle-Zélande, lorsqu’une baleine franche menace de sombrer, on lui attache des flotteurs avec de bonnes longueurs de filins, afin de savoir où la chercher lorsqu’elle refera surface.\par
Peu de temps après que la carcasse eut coulé, les guetteurs du {\itshape Péquod} annoncèrent que la {\itshape Vierge} remettait les pirogues à la mer, bien que le seul souffle en vue fût celui d’un dos en rasoir, espèce impossible à capturer à cause de sa puissance de nage. Toutefois, le souffle du dos en rasoir ressemble tellement à celui du cachalot que les pêcheurs inexpérimentés les confondent souvent. Ainsi Derick et sa tribu se lançaient vaillamment à la chasse de cette brute inapprochable. Portant tout dessus, la {\itshape Vierge} se lança à la suite de ses enfants, et elles disparurent ensemble sous le vent, chassant encore avec audace et espoir.\par
Oh ! mon ami, ils sont nombreux les dos en rasoir et les Derick !
\chapterclose


\chapteropen
\chapter[{CHAPITRE LXXXII. Honneur et gloire de la chasse à la baleine}]{CHAPITRE LXXXII \\
Honneur et gloire de la chasse à la baleine}\renewcommand{\leftmark}{CHAPITRE LXXXII \\
Honneur et gloire de la chasse à la baleine}


\chaptercont
\noindent Dans certaines entreprises un désordre circonspect est la vraie méthode.\par
Plus je me plonge dans la question de la chasse à la baleine, plus je pousse mes recherches jusqu’à ses sources mêmes, plus je suis pénétré de son honorabilité et de son antiquité, en constatant que tant de grands demi-dieux, de héros, de prophètes de toute sorte s’y sont, d’une manière ou d’une autre illustrés. Je suis alors transporté à la pensée que, bien que d’une façon secondaire, j’appartiens à cette noble confrérie.\par
Le galant Persée, un fils de Jupiter, fut le premier baleinier et, à l’éternel honneur de notre métier, il faut le dire, la première baleine attaquée par notre confrérie fut tuée sans intention sordide. C’était le temps chevaleresque de notre profession, alors que nous prenions les armes à seule fin de secourir ceux qui étaient en détresse et non pour remplir les bidons d’huile destinés aux lampes des hommes. Tout le monde sait l’admirable histoire de Persée et d’Andromède, et comment la belle Andromède, fille d’un roi, fut attachée à un rocher au bord de la mer, et comment le léviathan étant sur le point de l’enlever, Persée, prince des baleiniers, s’avança courageusement, harponna le monstre, délivra la belle et l’épousa. Ce fut un exploit d’un art consommé, rarement accompli par les meilleurs harponneurs de ce jour, d’autant plus que ce léviathan mourut au tout premier dard. Que personne ne mette en doute non plus cette histoire phénicienne, qui dit que dans l’ancienne Joppé, l’actuelle Jaffa, sur la côte de Syrie, on vit, pendant des siècles, dans un temple païen, un vaste squelette de baleine, dont la légende et les habitants de l’endroit disent que ce sont là les os du monstre tué par Persée. Lorsque les Romains prirent Joppé, ce même squelette fut porté en triomphe jusqu’en Italie. Mais le plus singulier et le plus lourd de sens dans cette histoire c’est que Jonas s’embarqua de cette même Joppé.\par
Assez semblable à l’aventure de Persée et d’Andromède – certains vont jusqu’à penser qu’elle en est indirectement issue – celle de saint Georges et du dragon. Lequel dragon je soutiens avoir été une baleine, car dans bien des vieilles chroniques, il a été fait d’étranges confusions entre les baleines et les dragons. « Tu es comme un lion des eaux, et un dragon de la mer » dit Ézéchiel, ce qui signifie clairement une baleine et en vérité certaines versions de la Bible usent du mot même. D’autre part, la gloire de l’exploit de saint Georges serait gravement amoindrie s’il n’avait affronté qu’un vil reptile terrestre, au lieu de combattre le grand monstre des profondeurs. Tout homme est capable de tuer un serpent, mais seuls un Persée, un saint Georges, un Coffin ont le courage de marcher hardiment sur la baleine.\par
Que les représentations modernes de cette scène ne nous abusent pas. Car, bien que la créature rencontrée par ce vaillant baleinier des anciens âges soit vaguement représentée sous l’aspect d’un griffon, bien que le combat s’y déroule à terre, et que le saint soit à cheval, il faut prendre en considération l’ignorance extrême d’une époque où les artistes n’avaient aucune connaissance de la forme véritable de la baleine, et le \hspace{1em}fait, – ce fut le cas pour Persée – que la baleine de saint Georges avait très bien pu se traîner de la mer jusque sur la plage, pensons aussi que la monture de saint Georges a pu être simplement un grand phoque ou un morse. Si nous réfléchissons à tout cela, il n’apparaîtra pas comme contradictoire avec la légende sacrée ou avec les plus anciennes esquisses de cette scène, de tenir ce prétendu dragon pour le grand léviathan \hspace{1em} luimême. En fait, si l’on met cette histoire face à la stricte et éclatante vérité, il en adviendra comme de l’idole, poisson, chair et volaille des Philistins, dragon cheval et les palmes de ses mains, ne conservant que sa partie poisson. Ainsi, le sceau de notre noblesse est d’avoir un baleinier pour patron de l’Angleterre et, en bonne justice, nous, harponneurs de Nantucket, devrions recevoir l’ordre auguste de saint Georges. Dès lors, ne permettez jamais que les chevaliers de cet ordre honorable (dont aucun, j’oserai l’affirmer, n’a eu affaire à une baleine comme leur grand patron) ne permettez jamais qu’ils jettent un regard de dédain sur un Nantuckais, étant donné que même nos vareuses de laine et nos pantalons goudronnés sont beaucoup plus dignes qu’eux d’être reçus dans le très noble ordre de saint Georges.\par
Je me suis longuement demandé s’il fallait ou non admettre Hercule parmi nous. Bien que, selon la mythologie grecque, ce Crockett et Kit Carson de l’Antiquité, ce hardi champion de bonnes actions enchanteresses ait été avalé et rejeté par une baleine, que cela suffise à en faire un baleinier, peut être sujet à controverse. Il n’est dit nulle part qu’il ait harponné son poisson, à moins bien sûr qu’il ne l’ait fait de l’intérieur. Néanmoins, on peut dire de lui que c’est un baleinier malgré lui. En tout cas, la baleine l’a attrapé s’il n’a pas attrapé la baleine. Je réclame donc son appartenance à notre clan.\par
Mais de l’avis de contradicteurs les plus autorisés, cette histoire grecque d’Hercule et de la baleine dériverait de l’histoire hébraïque plus ancienne encore de Jonas, et viceversa ; il est vrai qu’elles sont très semblables. Si j’admets le demi-dieu, pourquoi pas le prophète ?\par
Mais les héros, saints, demi-dieux et prophètes ne sont pas seuls à composer la liste de notre ordre. Notre grand maître reste à nommer, comme dans les origines des rois de l’antiquité, nous découvrons que celles de notre confrérie n’est pas à \hspace{1em}court de grands dieux eux-mêmes, La merveilleuse histoire orientale de Çastra nous apprend que l’une des trois personnes de la trimouri indoue, le redoutable Vishnu, est notre Seigneur. Dans la première de ses dix manifestations terrestres, il a pour toujours donné une place particulière à la baleine et l’a sanctifiée. Lorsque Brahma, le Dieu des Dieux, dit le Çastra décida de recréer le monde après l’un de ses périodiques avatars, il engendra Vishnu afin de régner sur son œuvre ; mais les livres mystiques des Védas dont l’attentive lecture semble avoir été indispensable à Vishnu avant d’entreprendre la création, et qui semblent dès lors avoir dû contenir quelque chose comme des conseils pratiques aux jeunes architectes, ces Védas étaient au fond des eaux, aussi Vishnu prit-il la forme d’une baleine et, sondant jusque dans les grandes profondeurs, il sauva les livres sacrés. Ce Vishnu n’est-il pas dès lors un baleinier ? L’homme qui monte un cheval est bien appelé cavalier !\par
Persée, saint Georges, Hercule, Jonas et Vishnu, telle est la liste de nos membres ! Quel club, autre que celui des baleiniers, a de pareils titres de gloire ?
\chapterclose


\chapteropen
\chapter[{CHAPITRE LXXXIII. Jonas du point de vue historique}]{CHAPITRE LXXXIII \\
Jonas du point de vue historique}\renewcommand{\leftmark}{CHAPITRE LXXXIII \\
Jonas du point de vue historique}


\chaptercont
\noindent Dans le chapitre précédent, nous avons parlé de Jonas et de la baleine considérés historiquement. Mais certains Nantuckais n’y ajoutent pas foi. Il est vrai que des Grecs et des Romains sceptiques, hérétiques selon l’orthodoxie païenne de leur temps, mirent également en doute l’histoire d’Hercule et de la baleine et celle d’Arion et du dauphin ; pourtant, ce scepticisme devant ces traditions n’altère pas leur authenticité.\par
L’argument principal d’un vieux baleinier de Sag Harbor pour mettre en doute l’histoire hébraïque tient au fait qu’il possédait une pittoresque Bible à l’ancienne mode, illustrée de planches curieuses et peu scientifiques. L’une d’elles représentait la baleine de Jonas avec deux évents, ce qui n’est juste que pour une espèce de léviathan, la baleine franche et celles de la même famille, dont les pêcheurs disent qu’un petit pain d’un sou l’étoufferait tant elle a le gosier étroit. L’évêque Jebb avait répondu d’avance à cette objection. Il n’est pas nécessaire, disait-il, d’envisager Jonas englouti dans le ventre de la baleine, mais temporairement logé dans quelque coin de sa bouche. Le bon évêque semble passablement raisonnable car, en vérité, on pourrait dresser deux tables de whist dans la bouche du monstre et y installer confortablement tous les joueurs. Jonas aurait pu aussi se blottir dans une dent creuse, mais réflexion faite la baleine franche n’a pas de dents.\par
Une autre raison qu’avançait Sag Harbor (on le surnommait ainsi) pour justifier son incrédulité vis-à-vis du \hspace{1em}prophète, était une allusion nébuleuse à son corps baignant dans les sucs gastriques de la baleine. Cette opinion ne résiste pas à l’examen, car un exégète allemand suggère que Jonas a dû chercher refuge dans le corps flottant d’une baleine morte, tout comme les soldats français, au cours de la campagne de Russie, se glissaient dans le cadavre de leurs chevaux utilisés en guise de tentes. D’autre part, certains commentateurs européens ont émis l’idée que lorsque Jonas fut jeté par-dessus bord du navire de Joppé, il s’échappa aussitôt vers un vaisseau voisin qui avait une baleine en figure de proue, j’ajouterai que ce bâtiment pouvait s’appeler {\itshape la Baleine} comme d’autres, de nos jours, se nomment {\itshape Requin, Mouette} ou {\itshape Aigle.} Il ne manque pas non plus d’exégètes pour penser que la baleine dont il est question dans le livre de Jonas, n’est autre qu’une bouée de sauvetage, quelque vessie gonflée d’air, vers laquelle le prophète naufragé aurait nagé, évitant ainsi la noyade. Le pauvre Sag Harbor paraît battu à plates coutures. Mais il tenait en réserve un dernier argument. Si j’ai bonne mémoire, le voici : Jonas fut avalé par la baleine en Méditerranée et, au bout de trois jours, il fut rejeté quelque part à trois jours de Ninive cité qui se trouvait sur le Tigre, et dès lors à bien plus de trois jours de la côte méditerranéenne. Comment cela se peut-il ?\par
N’y avait-il pas un autre moyen grâce auquel la baleine aurait pu déposer le prophète à si peu de distance de Ninive ? Si. Elle a pu doubler le cap de Bonne-Espérance Mais, pour ne pas parler de la traversée de toute la Méditerranée que cela implique, ni celle du golfe Persique et de la mer Rouge, une telle supposition exigerait qu’une circumnavigation complète ait été accomplie en trois jours autour de l’Afrique, sans compter que, près de Ninive, les eaux du Tigre n’ont pas la profondeur voulue pour qu’y puisse nager une baleine. De plus, admettre que Jonas ait doublé le cap de Bonne-Espérance à une époque aussi reculée, frustrerait de l’honneur de sa découverte Barthélémy Diaz, le célèbre navigateur, et rendrait mensongère l’histoire moderne.\par
Mais tous ces arguments insensés du vieux Sag Harbor ne font que prouver son fol orgueil – d’autant plus répréhensible qu’il n’avait guère d’autre instruction que celle que lui avaient donnée le soleil et la mer. Je dis qu’ils prouvent seulement son orgueil stupide, impie, et sa révolte abominable et satanique contre le révéré clergé, Car, selon un prêtre catholique portugais, l’aventure de Jonas se rendant à Ninive par le cap de Bonne-Espérance, n’est qu’une image exaltant de tout ce miracle. Et ce fut bien cela. De plus, et jusqu’à ce jour, des Turcs remarquablement éclairés croient avec dévotion à l’historicité de l’aventure de Jonas. Il y a quelque trois cents ans, un voyageur anglais rapportait dans les {\itshape Voyages de Harris} que, dans une mosquée turque bâtie à l’honneur de Jonas, se trouvait une lampe miraculeuse qui brûlait sans huile.
\chapterclose


\chapteropen
\chapter[{CHAPITRE LXXXIV. Le javelot}]{CHAPITRE LXXXIV \\
Le javelot}\renewcommand{\leftmark}{CHAPITRE LXXXIV \\
Le javelot}


\chaptercont
\noindent Si l’on graisse les essieux des voitures, c’est afin de leur donner plus d’aisance et de rapidité ; dans un but à peu analogue, certains baleiniers soumettent leurs pirogues à un semblable traitement en graissant leurs quilles. Ce qui ne peut en rien leur nuire et comporte sans doute des avantages non négligeables si l’on tient compte du fait que l’huile et l’eau sont incompatibles, que l’huile est une matière glissante et que le but est de faire glisser bravement la baleinière. Queequeg croyait fermement à l’onction de sa pirogue et un matin, peu de temps après la disparition de la {\itshape Vierge}, il mit plus de soin que jamais à cet ouvrage et rampant sous la quille suspendue aux potences, il la frotta avec une application telle qu’on eût dit qu’il cherchait à faire pousser une toison de cheveux sur le crâne chauve de l’embarcation, semblant obéir à un pressentiment que les événements justifièrent.\par
Vers midi, des baleines furent signalées mais, dès que le navire eut amené sur elles, elles firent demi-tour et s’enfuirent précipitamment. Ce fut une débandade digne de celles des barges de Cléopâtre à Actium.\par
Toutefois les pirogues se lancèrent, celle de Stubb en tête. Après de pénibles efforts, Tashtego réussit enfin à planter un fer, mais la baleine piquée poursuivit sa fuite horizontale sans sonder, à une vitesse accrue. Une telle traction continue devait tôt ou tard arracher le dard. Il devenait urgent de l’attaquer à la lance si l’on ne voulait pas la perdre. L’aborder était impossible tant elle nageait rapidement et furieusement. Que faire dès lors ?\par
De tous les merveilleux stratagèmes, tours d’adresse ou de prestidigitation et autres innombrables subtilités auxquels les vétérans de la pêche à la baleine sont souvent contraints d’avoir recours, nul n’égale la magnifique performance faite avec la lance que l’on appelle javelot. Aucun exercice à l’arme blanche ne le surpasse. Cet exploit n’est indispensable que dans le cas d’une baleine fuyant avec acharnement et la prouesse réside dans l’étonnante distance de laquelle est jetée, avec précision, la longue lance, depuis une pirogue emportée violemment, bondissant et roulant. La lance, fer et hampe, mesure en tout quelque dix ou douze pieds, cette hampe est beaucoup plus fine que celle du harpon et d’un bois plus léger, du pin. Elle est munie d’une estrope ténue, un grelin très long, permettant de la reprendre en main après l’avoir lancée.\par
Mais avant d’aller plus loin, il est important de le dire, si le harpon peut être jeté de la même manière que la lance, cette manœuvre est rarement pratiquée, et plus rarement encore couronnée de succès, à cause du poids plus grand et de la longueur du harpon comparés à celui de la lance et constituant un handicap sérieux. Aussi convient-il en général d’avoir harponné la baleine avant d’entreprendre un lancer au javelot.\par
Regardez Stubb à présent ! C’est l’homme dont le sangfroid réfléchi et teinté d’humour, dont l’égalité d’humeur dans les situations les plus critiques, le désignaient pour exceller dans le lancer du javelot. Voyez-le, il se tient droit à l’avant secoué de sa pirogue en plein vol, enveloppé dans la laine de l’écume. La baleine qui le remorque est à quarante pieds devant lui. Manipulant légèrement la longue lance, il l’examine sur toute sa longueur pour vérifier qu’elle soit bien droite ; tout en sifflotant, il tient la glène de grelin d’une main de manière à en assurer l’extrémité libre dans sa prise, laissant courir le reste. Puis, tenant la lance droit devant lui, au niveau de sa taille, il vise la baleine ; cela fait, il en abaisse sans à-coups l’extrémité inférieure, élevant d’autant la pointe jusqu’à ce qu’enfin l’arme soit bien équilibrée sur sa paume, le fer dressé à quinze pieds en l’air. Il fait penser à un jongleur balançant une longue perche sur son menton. L’instant d’après, l’élan rapide et indicible est donné. Décrivant une courbe élevée et grandiose, l’acier brillant franchit la distance écumeuse et vibre enfin au centre vital de la baleine. Son souffle n’a plus son étincelante blancheur, un jet de sang fuse par son évent.\par
– Ça lui a ouvert la bonde ! s’écria Stubb, c’est aujourd’hui l’immortel 4 juillet et le vin doit couler des fontaines ! Dommage que ce ne puisse être du vieux whisky d’Orléans, ou du vieil Ohio, ou de l’ineffable vieux Monon-gahéla ! Alors Tashtego, ami, je vous aurais invité à tenir sous ce jet un pichet et nous aurions bu en chœur ! Oui, vraiment, mes jolis cœurs, nous aurions préparé un punch de choix dans le bol de cet évent et à cette coupe vivante nous aurions bu la vie !\par
Sans que Stubb interrompe son joyeux bavardage, le dard adroitement part et repart, la lance revenant chaque fois à son maître tel un lévrier tenu par une laisse habile. La haleine à l’agonie fleurit, on donne du mou à la touée, le lancier retournant à la poupe, joint les mains et regarde en silence mourir le monstre.
\chapterclose


\chapteropen
\chapter[{CHAPITRE LXXXV. La fontaine}]{CHAPITRE LXXXV \\
La fontaine}\renewcommand{\leftmark}{CHAPITRE LXXXV \\
La fontaine}


\chaptercont
\noindent Que pendant six mille ans – et nul ne sait combien de millions de siècles auparavant – les grandes baleines aient projeté leur souffle sur toutes les mers, arrosant et parant de mystère les jardins des profondeurs avec tant de jets d’eau et que, depuis des siècles, des milliers de chasseurs se soient approchés de la fontaine de la baleine, regardant cet arrosage et ce souffle, tout cela sans que, jusqu’en cette minute sainte (treize heures, quinze minutes, trente secondes du seizième jour de décembre 1851), la question ait été tranchée de savoir si ces souffles sont après tout de l’eau véritable, ou seulement de la vapeur, voilà qui est digne d’intérêt.\par
Examinons dès lors la chose en détail. Tout le monde sait que grâce à l’ingéniosité particulière de leurs ouïes, les tribus à nageoires respirent en général l’air combiné avec l’élément dans lequel elles évoluent, aussi un hareng ou une morue peuvent bien vivre un siècle sans jamais sortir la tête au-dessus de l’eau. Mais la constitution interne de la baleine, comportant de véritables poumons, pareils à ceux de l’homme, ne lui permet pas de vivre sans respirer l’air libre. C’est pourquoi lui sont nécessaires ses visites périodiques au monde supérieur. Mais elle ne peut en aucun cas respirer par la bouche, celle du cachalot, par exemple, se trouve à huit pieds au moins sous la surface ; de plus, sa trachée-artère est indépendante de sa bouche. Il ne respire que par l’évent situé au sommet de sa tête.\par
Je ne crois pas me tromper en affirmant que la respiration est une fonction vitale chez toute créature car elle tire de l’air un élément qui apporte au sang le principe de la vie. Je vais devoir peut-être utiliser des mots scientifiques superflus. Si le sang d’un homme pouvait être oxygéné par une seule inspiration, il pourrait se boucher le nez, et n’en point prendre d’autre pendant un long laps de temps, ce qui revient à dire qu’il vivrait sans respirer. Aussi anormal que cela puisse paraître, tel est bien le cas pour la baleine qui vit régulièrement, à intervalles, une heure pleine et davantage lorsqu’elle sonde profond, sans inspirer d’air si peu que ce soit, car ne l’oublions pas, elle n’a pas d’ouïes. Comment cela se fait-il ? Entre les côtes et de chaque côté de l’épine dorsale, elle possède un remarquable et complexe labyrinthe crétois de vaisseaux vermiculaires qui, lorsqu’elle quitte la surface, sont absolument gorgés de sang oxygéné. Aussi, pour une heure ou davantage, à mille brasses de profondeur, a-t-elle une réserve de vie, tout comme le chameau emporte à travers un désert sans eau sa provision de boisson dans quatre estomacs supplémentaires. La réalité anatomique de ce labyrinthe est indiscutable, la supposition qui en découle me paraît juste et raisonnable, si je considère l’obstination autrement inexplicable du léviathan à venir « discourir » comme disent les pêcheurs. C’est ce que je veux dire. S’il n’est pas troublé lorsqu’il émerge, le cachalot restera en surface pendant un laps de temps absolument identique chaque fois, disons qu’il y reste onze minutes et souffle soixante-dix fois, c’est-à-dire qu’il respire soixante-dix fois ; lorsqu’il ressortira, il aura à nouveau ses soixante-dix souffles. Mais s’il se trouve inquiété après quelques respirations seulement et qu’il sonde de ce fait, il remontera toujours pour prendre sa ration d’air nécessaire et ne restera son temps normal en profondeur que lorsqu’il aura eu ses soixante-dix souffles indispensables. Pourquoi dès lors la baleine s’obstinerait-elle à venir souffler en surface sinon pour remplir son réservoir d’air avant de sonder longuement ? Il est évident que ce besoin l’expose aux risques mortels de la chasse, car ce grand léviathan ne pourrait être pris ni par l’hameçon, ni par le filet, lorsqu’il voyage à des milliers de brasses dans les profondeurs où le soleil ne pénètre pas. Ainsi, ô chasseur, ce n’est point tant ton adresse qui te vaut la victoire que les impératifs de la vie !\par
Chez l’homme, la respiration suit un cours incessant, une inspiration alimentant deux ou trois pulsations, de sorte que, quoi qu’il fasse, qu’il veille ou qu’il dorme, respirer il lui faudra ou mourir. Mais le cachalot ne respire qu’un septième, le dimanche de son temps.\par
Nous avons dit que la baleine ne respire que par son évent, si nous pouvions ajouter avec certitude que ses souffles sont mêlés d’eau, je pense que nous saurions pourquoi le sens olfactif semble lui faire défaut, car ce même évent est le seul organe qui corresponde à un nez, et étant obturé par deux éléments, on ne peut s’attendre à ce qu’il ait l’odorat. Mais on n’a pas encore pu éclaircir le mystère de cet évent, à savoir s’il rejette de l’eau ou de la vapeur. Une chose est certaine néanmoins, le cachalot n’a pas d’organes olfactifs à proprement parler. Qu’aurait-il à en faire ? Le fond de la mer n’a ni roses, ni violettes, ni eau de Cologne.\par
Qui plus est, comme sa trachée-artère n’ouvre que dans le canal de son évent et que ce long canal – comme le grand canal de l’Érié – est pourvu de sortes d’écluses Rouvrant et se fermant, destinées à la rétention de l’air lorsque la baleine est en profondeur et à l’expulsion de l’eau lorsqu’elle est en surface, elle n’a dès lors point de voix, à moins que vous ne lui fassiez l’affront de dire qu’elle parle du nez lorsqu’elle gronde si étrangement. Là encore, qu’aurait-elle à dire ? J’ai rarement connu un être profond qui ait un mot à adresser au monde, sauf pour bégayer de force quelque chose afin de gagner sa vie. Oh ! il est heureux que le monde sache si bien écouter !\par
Le canal de l’évent du cachalot, prévu pour véhiculer l’air, situé horizontalement juste sous la surface supérieure de sa tête et légèrement de côté, ce curieux canal rappelle une conduite de gaz courant sous le sol d’une rue. La question reste de savoir si cette conduite de gaz est aussi une conduite d’eau, c’est-à-dire si le cachalot expire seulement de la vapeur ou si ce souffle est mêlé d’eau absorbée par la bouche et rejetée par l’évent. Il est certain que la bouche communique indirectement avec ce dernier, mais cela ne prouve pas que ce soit dans le but d’expulser l’eau par l’évent, car il n’y aurait, semble-t-il, à cela de nécessité urgente que si le cachalot venait à absorber de l’eau en se nourrissant, mais il va chercher sa subsistance dans les profondeurs où il ne pourrait souffler, le voudrait-il. D’autre part, en l’observant avec attention, on constate que, lorsqu’il n’est pas inquiété, la concordance entre les jets et les temps habituels de respiration est parfaitement immuable.\par
Mais pourquoi assommer quelqu’un avec tous ces raisonnements ? Parlez sans détours ! Vous l’avez vu souffler, alors dites-nous ce qu’est ce souffle, ne distinguez-vous pas l’air de l’eau ? Cher monsieur, il n’est pas aisé, en ce monde, de décider des choses simples. J’ai toujours trouvé vos choses simples épineuses entre toutes. Quant à cet évent du cachalot, vous pourriez presque vous y tenir debout et rester pourtant indécis quant à sa véritable nature.\par
Sa partie centrale est cachée par la brume neigeuse et étincelante qui l’enveloppe et comment pourriez-vous affirmer avec certitude que c’est de l’eau qui en retombe alors que, lorsque vous approchez un cachalot d’assez près pour voir cet évent, il s’agite tumultueusement et que l’eau cascade de toutes parts. Si alors vous sentez des goutté d’humidité, comment saurez-vous que ce n’est pas simplement de la condensation, ou de l’eau superficiellement logée dans la dépression de l’évent régulièrement submergée ? Car même lorsque le cachalot nage tranquillement dans le calme de midi, sa haute bosse aussi sèche que celle d’un dromadaire dans le désert, même alors il emporte toujours sur la tête un petit bassin qui brille au soleil comme une flaque de pluie dans le creux d’un rocher.\par
Il n’est d’ailleurs pas prudent du tout pour un chasseur de se montrer trop curieux de la nature exacte de ce souffle. Il ne lui conviendrait guère d’aller jeter un coup d’œil dedans ou d’y mettre le nez. C’est une fontaine à laquelle l’on ne peut aller remplir sa cruche pour la ramener pleine. Car pour peu que l’on entre en contact avec la vapeur extérieure au jet, ce qui arrive souvent, la peau vous cuira fiévreusement sous son âcreté. Je connais quelqu’un qui en approcha de très près, poussé par un intérêt scientifique ou autre, je ne sais, et dont la peau des joues et des bras pela. Aussi les baleiniers tiennent-ils ce jet pour empoisonné et font-ils leur possible pour l’éviter. D’autre part, je l’ai entendu dire et je n’en doute guère, si ce souffle vous atteint dans les yeux, il vous rend aveugle. Aussi la chose la plus sage que le chercheur ait à faire est de laisser tranquille ce jet mortel.\par
Si nous ne pouvons établir de preuves, nous pouvons toutefois émettre des hypothèses. Voici la mienne : ce souffle n’est qu’une condensation. Et entre autres raisons amenant à cette conclusion, j’en trouve une dans la dignité et le caractère sublime du cachalot que je ne considère pas comme un être commun et superficiel car il est indiscutable qu’il ne fréquente jamais les zones peu profondes, ni le voisinage des côtes comme d’autres cétacés. Et j’en suis convaincu, de toutes les têtes des penseurs profonds tels que Platon, Pyrrhon, le diable, Jupiter, Dante, etc., monte toujours un jet de vapeur à demi visible lorsqu’elles sont plongées dans leurs méditations. Tandis que j’écrivais un petit traité sur l’Éternité, j’eus la curiosité de placer un miroir devant moi et peu après je vis s’y réfléchir une curieuse spirale qui ondulait au-dessus de ma tête. La moiteur de mes cheveux, chaque fois que je suis absorbé par des pensées profondes après avoir bu six tasses de thé brûlant sous le toit mince de ma mansarde par un après-midi d’août, semble ajouter du poids à la supposition que je viens d’émettre.\par
L’idée que nous nous faisons du monstre puissant et brumeux s’élève et s’ennoblit à le voir voguer solennellement dans le calme des tropiques, sa vaste et douce tête surmonté d’un dais de vapeur engendrée par son recueillement incommunicable, cette vapeur que vous verrez souvent glorifiée par un arc-en-ciel comme si le ciel lui-même mettait son sceau sur ses pensées. Car, voyez-vous, les arcs-en-ciel ne fréquentent pas les ciels clairs, ils n’irradient que de la vapeur. De même, à travers les brumes denses de mes doutes obscurs, jaillissent parfois les divines intuitions qui illuminent mon brouillard d’un céleste rayon. J’en rends grâce à Dieu, car tous ont des doutes, beaucoup nient, mais à travers les doutes ou les refus bien peu ont des intuitions. Des doutes sur toutes choses de ce monde, des intuitions des choses du ciel, leur fusion ne donne pas la foi, pas plus qu’elle ne rend mécréant, mais elle accorde à l’homme de les considérer objectivement.
\chapterclose


\chapteropen
\chapter[{CHAPITRE LXXXVI. La queue}]{CHAPITRE LXXXVI \\
La queue}\renewcommand{\leftmark}{CHAPITRE LXXXVI \\
La queue}


\chaptercont
\noindent Des poètes ont chanté la douceur de l’œil de l’antilope ou le merveilleux plumage de l’oiseau qui jamais ne se pose. Plus prosaïque je célébrerai une queue.\par
Si l’on admet que la nageoire caudale du cachalot commence à cette partie du tronc où elle s’effile jusqu’à la seule épaisseur d’une taille d’homme, sa surface est d’au moins cinquante pieds carrés. Ronde et épaisse à la naissance, elle s’ouvre en deux larges palmes, denses et plates qui vont s’amincissant jusqu’à ne mesurer qu’un pouce d’épaisseur. À l’embranchement ou fourche, ces palmes se chevauchent légèrement puis elles s’écartent comme des ailes entre lesquelles un vide serait découpé. Aucune créature vivante n’offre une beauté de ligne aussi parfaite que le cachalot avec les croissants de sa queue. Chez le sujet adulte, l’envergure de la caudale dépassera largement vingt pieds.\par
Elle apparaît comme une couche serrée de muscles soudés, en y pratiquant une coupe, on voit qu’elle est formée de trois couches superposées. Les fibres des couches supérieures et inférieures sont longues et horizontales, celles de la couche médiane très courtes courant en croix entre les autres. Cette triple unité joue son rôle dans la puissance de la queue. Qui étudie les vieux murs romains verra une mince épaisseur de briques alterner avec la pierre et ce procédé a certainement assuré leur remarquable solidité à ces admirables vestiges du passé.\par
Toutefois, comme si la puissance musculaire de la nageoire caudale ne devait pas suffire, la masse entière du léviathan est enrobée d’une trame noueuse de fibres descendant de part et d’autre des lombes jusqu’aux palmes de la queue avec lesquelles elles se confondent, lui conférant un surcroît de force qui semble ainsi ramassée en cette partie de l’animal. Ce serait le fléau idéalement propre à réaliser une destruction totale.\par
Cette force étonnante n’infirme en rien la grâce des mouvements et de cette aisance, comme enfantine, d’un titan découle une terrifiante beauté. La force véritable n’altère jamais la beauté ni l’harmonie mais souvent les consacre et la puissance détient une magie. Privez la statue d’Hercule des nœuds de tendons qui semblent vouloir faire sauter le marbre, elle perdra son attrait. Lorsque le fervent Eckermann souleva le drap mortuaire qui recouvrait le corps nu de Goethe, il fut subjugué par la massive charpente de sa poitrine pareille à un arc romain. Lorsque Michel-Ange donne à Dieu le Père une forme humaine, il l’empreint de robustesse. Et les tableaux italiens où le Fils apparaît comme un androgyne doux et bouclé, s’ils traduisent au mieux l’amour divin qui était en Lui, en Le privant de toute vigueur ne suggèrent aucun pouvoir autre que celui, négatif et tout féminin, de soumission et de longanimité dont tout un chacun reconnaîtra qu’elles sont les vertus mêmes qu’il enseignait.\par
Pour en revenir à l’organe dont je parle, sa souplesse est si subtile que, se manifestant dans le jeu, dans le sérieux ou dans la colère, selon son humeur, ses mouvements sont toujours empreints d’une grâce si incomparable que le bras même d’une fée ne saurait la surpasser.\par
On lui reconnaît cinq mouvements distincts, celui de l’engin de propulsion, celui de la massue au combat, lorsqu’il « brise », lorsqu’il « mâte sa queue » et enfin quand il sonde.\par
Première position : la queue du léviathan, horizontale, se comporte différemment de toutes celles des autres créatures marines. Elle ne frétille jamais. Frétiller est un signe d’infériorité tant chez l’homme que chez le poisson. Elle est le seul organe de propulsion du cachalot, formant une volute avançant sous son corps, et rapidement rejetée en arrière, c’est elle qui donne au monstre cette singulière allure de bond en flèche lorsqu’il nage furieusement. Ses nageoires pectorales ne lui servent qu’à se diriger.\par
Deuxièmement et ceci est lourd de sens : tandis qu’un cachalot combat un autre cachalot de la tête et de la mâchoire, dans ses luttes avec l’homme, il fait principalement un usage méprisant de sa queue. Frappant une baleinière, il enroule promptement ses palmes et le coup est assené par leur seule détente. S’il est suffisamment hors de l’eau et droit au-dessus de son but, la volée sera tout simplement sans rémission, les côtes des hommes et celles de la pirogue voleront en éclats. Votre seule chance de salut, c’est de l’éviter, mais si le coup est porté de côté grâce à la résistance de l’eau, à la légèreté de la pirogue et à sa plasticité, le dégât le plus grave consistera en un bordage ou une ou deux membrures brisées, une sorte de point de côté en somme ! Des gifles de ce genre sont si souvent reçues au cours de la chasse que les pêcheurs les considèrent comme jeux d’enfants ; l’un deux enlève sa vareuse et bouche le trou.\par
Troisièmement : je ne puis le prouver, mais il me semble que le sens du toucher est concentré dans la caudale, sa sensibilité n’a d’égale que la délicatesse de la trompe de l’éléphant, elle se révèle surtout lorsque le cachalot balaie la surface de la mer et remue de droite et de gauche ses immenses palmes avec une douceur et une lenteur de jeune fille ; mais vient-il à sentir fûtce la moustache d’un marin, alors malheur au marin, aux moustaches et au reste ! Quelle tendresse dans cet effleurement premier ! Si cette queue était préhensible, elle me rappellerait aussitôt l’éléphant de Darmonode qui fréquentait le marché aux fleurs, offrait des bouquets aux demoiselles en les saluant bien bas et leur caressait la taille. Il est encore regrettable à d’autres égards que la queue du cachalot ne soit pas préhensible car j’ai entendu parler d’un autre éléphant qui, blessé au combat, arracha la flèche avec sa trompe.\par
Quatrièmement : si vous pouviez glisser inaperçu au milieu des océans solitaires, vous trouveriez le cachalot, se croyant en sécurité, dépouillé de la dignité que lui confère sa corpulence, jouant dans la mer comme un chaton dans le foyer. Au jeu, il fait encore usage de sa force il lance haut ses larges palmes et lorsqu’il en frappe l’eau la détonation retentit à des milles à la ronde, pareille à celle d’une grosse pièce dont on vient de faire feu, et le mince filet de vapeur de l’évent simule la fumée qui s’échappe alors de la bouche du canon.\par
Enfin : dans sa position habituelle de nage, ses palmes situées au-dessous du niveau de son dos sont submergées et invisibles mais lorsqu’il s’apprête à sonder, trente pieds de son corps et toute sa queue se dressent droit en l’air, vibrant un moment jusqu’à ce qu’il disparaisse dans les profondeurs. Hormis le saut sublime – qui sera décrit plus loin – ce piqué de la baleine est peut-être le spectacle le plus grandiose que puisse offrir la nature. De l’Océan sans fond, la queue gigantesque semble convulsivement saisir les hauteurs du ciel. C’est ainsi qu’en rêve j’ai vu Satan majestueux projeter hors des flammes de la Baltique de l’enfer sa griffe colossale et tourmentée. Mais en contemplant pareille scène, tout dépend de votre humeur. Si elle est dantesque, ce seront les démons qui viendront à vous, si vous êtes sous l’influence d’Isaïe, ce seront les archanges. Du poste de vigie, je vis une fois, tandis que l’aurore empourprait le ciel et la mer, une troupe nombreuse de baleines allant à la rencontre du soleil et dont les palmes vibrèrent toutes ensemble. Il me parut à ce moment-là que jamais les dieux n’avaient reçu pareil témoignage d’adoration, même en Perse, chez les adorateurs \hspace{1em}du \hspace{1em}feu. \hspace{1em}Comme \hspace{1em}Ptolémée \hspace{1em}Philopator \hspace{1em}est \hspace{1em}l’avocat de l’éléphant d’Afrique, je suis celui de la baleine que je déclare le plus pieux d’entre tous les êtres. Car, selon le roi Juba, les éléphants des armées de l’Antiquité saluaient souvent le matin, la trompe levée dans le plus profond silence.\par
Cette comparaison de la baleine et de l’éléphant, hasardeuse dans la mesure où il s’agit de la queue de l’une et de la trompe de l’autre, ne tend pas à mettre ces organes sur un plan d’égalité, pas plus que les créatures à qui ils appartiennent. Car, à côté du léviathan, le plus puissant éléphant n’est qu’un petit chien ; à côté de la queue de la baleine, sa trompe n’est qu’une tige de lys ; le coup le plus terrible qu’il puisse en porter n’est qu’un coup d’éventail joueur à côté du fracas d’écrasement soulevé par les palmes puissantes de la queue du cachalot qui, en de nombreuses occasions, ont jeté dans les airs, les uns\footnote{Bien que toute comparaison entre la taille de la baleine et celle de l’éléphant soit absurde, la baleine étant à l’éléphant ce que celui-ci est à un chien, ils n’en offrent pas moins de curieuses ressemblances. Leur jet entre autres. On sait que l’éléphant aspire souvent de l’eau avec la trompe et la rejette en souffle en la relevant.}\hyperref[_bookmark194]{\dotuline{19}}.\par
Plus je pense à la puissance de cette caudale, plus je déplore mon incapacité à la décrire. Elle fait parfois des gestes qui honoreraient une main d’homme mais qui restent inexplicables. Dans une troupe nombreuse, leurs mouvements mystérieux sont parfois si remarquables que certains chasseurs les disaient apparentés aux signes et symboles des francs-maçons et permettant de cette façon à la baleine de converser intelligemment avec le monde. Tout le corps du cachalot a d’ailleurs des mouvements pleins d’étrangeté, énigmatiques au chasseur le plus expérimenté. Je peux bien, pour ma part, le disséquer sans atteindre plus profond que sa peau, je ne puis le connaître et ne le connaîtrai jamais. Et si je ne sais rien de sa queue combien j’ignore encore plus sa tête ! Comment comprendrais-je un visage qu’il n’a point ? Il semble me dire : « Tu peux me voir de dos, ma face te restera cachée. » Mais je ne peux comprendre parfaitement ce dos, et il peut me dire ce qu’il veut de sa face. Je le répète, de face il n’en a point.
\chapterclose


\chapteropen
\chapter[{CHAPITRE LXXXVII. La grande armada}]{CHAPITRE LXXXVII \\
La grande armada}\renewcommand{\leftmark}{CHAPITRE LXXXVII \\
La grande armada}


\chaptercont
\noindent L’étroite et longue péninsule de Malacca, s’étendant au sud-est de la Birmanie, forme la pointe extrême sud du continent asiatique. De là s’échelonnent les longues îles de Sumatra, Java, Bali et Timor, et bien d’autres encore qui constituent un vaste môle, une digue reliant l’Asie à l’Australie divisant le vide de l’océan Indien du semis serré des archipels orientaux. Ce rempart est percé de plusieurs poternes de sortie pour la commodité des navires et des baleines, les plus importantes étant les détroits de la Sonde et de Malacca. Les vaisseaux qui vont en Chine depuis l’ouest gagnent les mers de Chine par le détroit de la Sonde.\par
Celui-ci sépare Sumatra de Java et se trouve à mi-chemin dans ce rempart d’îles ; il est flanqué d’un hardi promontoire de verdure, contrefort connu des marins sous le nom de cap de Java, et qui ressemble fort à la porte d’un immense empire fortifié. Si l’on pense à l’inépuisable richesse en épices, en soie, en bijoux, en or en ivoire, de ces milliers d’îles orientales, il semble que cette configuration géographique soit une mesure significative prise par la nature, même si elle n’est qu’apparence inefficace pour protéger de tels trésors contre la rapacité du monde occidental. Les côtes du détroit de la Sonde sont dépourvues de ces places fortes impérieuses qui gardent les entrées de la Méditerranée, de la Baltique et de la Propontide. À l’encontre des Danois, les Orientaux n’exigent pas l’obséquieux hommage du hunier abaissé de l’interminable procession de navires qui ont passé vent arrière nuit et jour, durant des siècles, entre les îles de Java et de Sumatra, frétés des plus précieux chargements de l’Orient. S’ils renoncent librement à pareil cérémonial, ils réclament en revanche une marque de respect plus substantielle.\par
Depuis des temps immémoriaux, les praos des pirates malais, tapis dans les anses ombreuses et les îlots de Sumatra ont fondu sur les vaisseaux traversant les détroits, réclamant férocement tribut à la pointe de leurs lances. Les châtiments sanglants et répétés que leur ont infligé les croiseurs européens ont récemment tempéré quelque peu leur audace, pourtant, aujourd’hui encore, on entend parler de navires anglais ou américains impitoyablement abordés et pillés dans ces eaux.\par
Avec un vent favorable et frais, le {\itshape Péquod} approchait à présent de ces détroits, Achab se proposant d’atteindre par là la mer de Java, puis de faire cap au nord vers les parages connus pour être fréquentés ici ou là par le cachalot, de longer les Philippines et de gagner la lointaine côte du Japon à temps pour la grande saison baleinière. Ainsi le {\itshape Péquod} aurait accompli un périple sur presque tous les parages de croisière au cachalot du monde entier avant de redescendre à l’équateur dans le Pacifique où Achab, eût-il été partout ailleurs frustré de sa proie, comptait fermement défier Moby Dick dans les eaux qu’il hantait le plus fréquemment et à la saison qui semblait être la plus propice.\par
Mais quoi ? Au cours de cette quête étendue, Achab n’abordait-il jamais ? Son équipage vivait-il de l’air du temps ? Sûrement il devait s’arrêter pour refaire sa réserve d’eau. Non ! Voici bien longtemps déjà que le soleil dans sa course embrasée ne se nourrit que de lui-même. Ainsi en va-t-il d’Achab. Et aussi du navire baleinier. Tandis que d’autres cales sont remplies de denrées étrangères destinées à des quais étrangers, le navire baleinier, ce vagabond du globe entier, n’a point d’autre charge que lui-même, son équipage, ses armes et de quoi pourvoir à ses besoins. Tout un lac est mis en bouteilles dans sa vaste cale. Il est lesté de nécessités premières, point ne lui faut des gueuses de plomb ou de fer. Il charrie dans ses flancs des années d’eau. De la bonne vieille eau de Nantucket qu’au long de ces trois ans de voyage dans le Pacifique un Nantuckais préfère au breuvage saumâtre fraîchement soustrait aux rivières indiennes ou péruviennes. Aussi, tandis que d’autres vaisseaux ont touché des vingtaines de ports en allant de New York en Chine et retour, le baleinier, pendant ce temps, peut n’avoir point aperçu un grain de sable et son équipage point d’autre visage humain que ceux de marins, comme lui, appartenant au large, de sorte que si vous les informiez d’un nouveau déluge, ils répondraient simplement : « Eh bien, les gars, voici l’arche ! »\par
Comme de nombreux cachalots avaient été pris au large de la côte ouest de Java, tout près du détroit de la Sonde, comme l’endroit était tenu par les pêcheurs pour un excellent lieu de croisière, les guetteurs du {\itshape Péquod} étaient interpellés sans cesse et invités à garder l’œil ouvert tandis que le navire se rapprochait toujours du cap de Java. Encore que les palmes vertes des falaises se fussent bientôt dessinées à tribord et que l’air ait apporté le parfum enchanteur de la cannelle fraîche, aucun souffle ne fut signalé. Le navire, ayant pénétré le détroit, avait renoncé à rencontrer une gamme dans les parages lorsque retentit le cri traditionnel et joyeux, tombé du ciel, et bientôt nous fûmes accueillis par un spectacle d’une singulière magnificence.\par
Disons d’abord que, vu la chasse acharnée qui leur a été livrée sur les quatre océans, les cachalots, au lieu de se déplacer en petits groupes, comme par le passé, forment maintenant des troupes si nombreuses, des foules telles parfois qu’on dirait presque que plusieurs nations d’entre eux ont signé solennellement un traité d’assistance et de protection mutuelles. C’est parce qu’ils voyagent en caravanes si immenses que vous pouvez parfois naviguer des semaines et des mois durant sans qu’un seul souffle vous salue et recevoir soudain l’ovation de milliers et de milliers d’entre eux, vous semblera-t-il.\par
À deux ou trois milles de part et d’autre du navire, en large demi-cercle embrassant le plat horizon, une chaîne continue de jets jouaient et étincelaient dans la lumière de midi. Contrairement aux jets jumeaux et perpendiculaires de la baleine franche qui, se divisant à leur sommet, retombent comme deux branches de saule, le souffle unique du cachalot, incliné vers l’avant, offre un bouquet d’épaisses boucles de brume blanche, s’élevant et retombant toujours sous le vent.\par
Vue du pont du {\itshape Péquod} perché sur une colline de la mer, cette armée de souffles vaporeux aux boucles isolées estompés par un voile bleuâtre, apparaissait comme les milliers de riantes cheminées d’une grande ville à un cavalier les apercevant d’une hauteur par une douce matinée d’automne.\par
Comme les armées en marche, affrontant l’hostilité d’un défilé de montagne, pressent le pas dans leur hâte d’avoir traversé ses dangers et de se déployer dans la sécurité relative de la plaine, l’armée de cachalots semblait se précipiter à travers le détroit, resserrant les ailes de son demi-cercle pour former une masse compacte toujours développée en forme de croissant.\par
Toutes voiles dehors, le {\itshape Péquod} les talonna et ses harponneurs, leurs armes à la main, poussaient des cris joyeux à l’avant de leurs pirogues suspendues encore à leurs potences. Ils n’en doutaient pas, si le vent ne faiblissait pas, cette armée ne s’élargirait dans les mers orientales que pour assister à la prise de quelques-uns d’entre eux. Et qui sait si, au sein de cette congrégation, Moby Dick lui-même ne se trouvait pas, pareil à l’éléphant blanc sacré des cortèges du couronnement au Siam ! Aussi, bonnette sur bonnette, chassions-nous ces léviathans devant nous, lorsque la voix retentissante de Tashtego attira notre attention dans notre sillage.\par
Si un croissant avançait devant nous un autre nous suivait paraissant élever des souffles de vapeur blanche, jaillissant et retombant quelque peu comme ceux des baleines. Toutefois, ils n’étaient pas intermittents et restaient suspendus sans jamais disparaître. Achab virevolta prestement dans son trou de tarière, leva sa longue-vue et s’écria : « Ohé de la hune ! Équipez les cartahus et les bailles pour mouiller les voiles. Les Malais sont à nos trousses ! »\par
Comme si, en attendant que le {\itshape Péquod} fût fermement engagé dans le détroit, ils n’avaient que trop tardé derrière leurs promontoires, ces bandits asiatiques nous serraient maintenant de près pour rattraper le temps perdu à la prudence. Mais le rapide {\itshape Péquod}, vent arrière soufflant frais, livrait lui-même une chasse ardente, et ces philanthropes basanés étaient fort charitables de l’aiguillonner à coups d’éperons ou de cravache ! Sa longue-vue sous le bras, Achab arpentait le pont, épiant à l’avant les monstres auxquels il livrait la chasse, à l’arrière les pirates sanguinaires qui la lui livraient à lui. Cette pensée avait dû lui venir ! Tandis qu’il regardait les vertes parois du défilé marin que traversait le navire et songeait que c’était la porte ouvrant sur la route de sa vengeance, et qu’il s’y trouvait à la fois chassant et chassé vers son but meurtrier, et que, de plus, une bande impitoyable de pirates sauvages, de démons inhumains et mécréants lui hurlaient la salutation infernale de leurs malédictions, tandis que ces réflexions l’habitaient, elles avaient laissé marqué et farouche, le front d’Achab, semblable à la plage de sable noir corrodée par la marée d’une tempête mais que rien pourtant ne peut arracher à ses assises.\par
L’insouciance de la plupart des hommes n’était pas troublée par de pareilles considérations et quand le {\itshape Péquod} eut laissé loin, et toujours plus loin, derrière lui les pirates et la verdure éclatante de la pointe de Cockatoo du côté de Sumatra, quand il eut gagné les vastes eaux qui s’ouvraient au-delà, alors les harponneurs semblèrent regretter l’avance prise par les rapides cachalots plus que se réjouir de la victoire remportée sur les Malais. Mais enfin le navire, toujours dans le sillage des baleines, s’en rapprocha comme elles semblaient ralentir, et le vent tombant, l’ordre fut donné de mettre les pirogues à la mer. À peine la troupe eut-elle senti la présence des trois baleinières à leur poursuite, bien qu’elle fût à un mille encore, avertie par ce merveilleux instinct que l’on prête au cachalot, elle se regroupa en rangs serrés, bataillon dont les souffles étaient un faisceau étincelant de baïonnettes et qui accélérait sa marche.\par
En pantalons et manches de chemise, nous sautâmes sur nos avirons de frêne et après plusieurs heures de nage nous hésitions à poursuivre la chasse lorsque les cachalots firent une halte, nous avertissant qu’ils étaient enfin pris de cette perplexité insolite qui leur donne inertie de l’irrésolution et fait dire aux pêcheurs qu’ils sont pétrifiés d’effroi. Une déroute démesurée brisa les colonnes martiales jusqu’alors denses et rapides ; comme les éléphants du roi Porus dans la bataille contre Alexandre, ils parurent saisis d’une folie atterrée. Ils se jetèrent dans toutes les directions, décrivant de vastes cercles, nageant sans but de-ci et de-là, leurs jets courts et épais trahissant ouvertement un trouble et une panique mis en relief par ceux d’entre eux qui, complètement paralysés, flottaient, désemparés, comme les épaves d’un navire naufragé. Ces léviathans eussent-ils été un simple troupeau de moutons, poursuivis à travers le pâturage par trois loups féroces, qu’ils n’auraient pas fait montre d’une plus excessive épouvante, mais cette timidité occasionnelle caractérise presque tous les êtres grégaires. On a vu des bisons de l’Ouest à crinière léonine qui, bien qu’en troupeaux de dizaines de milliers, prenaient la fuite devant un cavalier solitaire. Témoins aussi les êtres humains attroupés dans le parc à moutons d’un théâtre et qui, à la moindre alerte d’incendie, se ruent vers les sorties dans un sauve-qui-peut serré, se piétinent, s’écrasent et se jettent impitoyablement à terre les uns les autres jusqu’à ce que mort s’ensuive. Mieux vaut, dès lors, ne point trop s’étonner de l’étrange affolement des cachalots devant nous, car il n’est aucune bête sur la terre dont la démence ne soit infiniment surpassée par celle de l’homme.\par
Malgré l’agitation d’un certain nombre de cachalots, la troupe n’avançait ni ne reculait, mais restait groupée au même endroit. Comme il est de coutume en ce cas, les pirogues se séparèrent aussitôt, chacune amenant sur un gibier isolé aux lisières du banc. Au bout de trois minutes, Queequeg avait lancé son harpon, le poisson frappé nous aveugla d’écume, puis, filant comme l’éclair, nous entraîna droit au cœur de la troupe. Bien qu’en pareilles circonstances, un tel mouvement ne soit pas sans précédent et qu’on le prévoie plus ou moins, il n’en est pas moins l’une des plus dangereuses vicissitudes de la chasse. Car tandis que le monstre vous emporte promptement toujours plus avant au sein frénétique de la troupe, vous pouvez dire adieu à la prudente vie et n’être plus qu’une palpitation délirante.\par
Cependant qu’aveugle et sourd le cachalot piquait de l’avant comme pour se libérer, par le seul pouvoir de la vitesse, de la sangsue de fer rivée à lui, que nous ouvrions sur la mer une balafre blanche, que notre fuite était de toutes parts menacée par les bêtes en folie qui se ruaient sur nous, notre pirogue était comme le navire assailli par les icebergs dans la tempête, se débattant pour se frayer un passage dans le dédale de leurs détroits, ignorant à quel instant il sera coincé et écrasé.\par
Point du tout intimidé, Queequeg gouvernait intrépidement, tantôt débordant d’un monstre en travers de notre route, tantôt s’écartant doucement d’une queue colossale suspendue au-dessus de nos têtes, cependant que Starbuck, à l’avant, la lance à la main, éloignait de nous les cétacés à sa portée à traits courts, car il n’était pas question d’en envoyer de longs. Les canotiers n’étaient pas oisifs non plus, bien qu’ils n’eussent pas lieu de se livrer à leur tâche habituelle. Ils s’occupaient plutôt du côté hurlements de l’affaire : « Ôte-toi de là, commodore ! » cria l’un à un grand dromadaire qui menaçait de \hspace{1em}nous submerger. « Bas la queue, toi ! » fut l’invitation qu’un autre adressa à un monstre qui, tout près de notre plat-bord, semblait tranquillement se rafraîchir avec cet éventail.\par
Toutes les baleinières sont pourvues de curieux accessoires, appelés dragues, inventés à l’origine par les Indiens de Nantucket, et qui consistent en deux épaisses pièces de bois carrées, solidement chevillées ensemble, à contre-fil ; au centre de ce bloc est attachée une ligne d’une longueur considérable et, à l’une de ses extrémités, une estrope permet en un instant de frapper le harpon. On emploie surtout cette drague pour les baleines bilieuses, car elles sont alors autour de vous, trop nombreuses pour être prises en chasse, mais on ne rencontre pas des cachalots tous les jours, aussi faut-il les tuer quand on peut et si l’on ne peut les tuer tous il faut, au moins les darder pour pouvoir ensuite les tuer à loisir. Telle est l’utilité immédiate des dragues ; nous en avions trois à bord de notre pirogue. La première et la seconde furent piquées avec succès, et les cachalots s’enfuirent en zigzaguant, entravés par l’énorme résistance oblique de la drague qu’ils remorquaient, gênés comme des convicts par la chaîne et le boulet. Mais lorsque le troisième encombrant morceau de bois fut lancé par-dessus bord, il se prit sous l’un des bancs de la baleinière, l’arracha brutalement et l’emporta, laissant choir le canotier au fond de l’embarcation tandis que son siège se dérobait sous lui. Les bornages blessés laissèrent entrer l’eau des deux côtés, mais nous calfatâmes avec deux ou trois chemises et pantalons et la voie d’eau fut momentanément aveuglée.\par
Il aurait été quasiment impossible de lancer ces harpons à dragues si, en pénétrant le cœur de la troupe, notre cachalot remorqueur n’avait beaucoup ralenti ; de plus, tandis que nous nous écartions toujours plus du cercle qui circonscrivait le tumulte, ces désordres terribles semblaient se calmer de sorte que, lorsque le dernier fer se décrocha et que notre cachalot s’éloigna de côté, nous glissâmes, entre deux cachalots avec notre vitesse acquise, au sein même du banc, comme un torrent de montagne s’apaisant dans le lac serein d’un vallée. De là, la tempête rugissant dans les gorges qui séparaient les cachalots se trouvant à l’extérieur se laissait entendre mais ne se ressentait plus. En ce centre, la mer douce et satinée offrait ce miroir d’huile produit par la buée subtile que le cachalot émet en ses plus paisibles moments. Oui nous étions vraiment au cœur du calme enchanté dont on dit qu’il demeure au centre de toute confusion. Et, dans l’espace éperdu, nous voyions les cercles concentriques où des bandes de huit à dix cachalots tournaient rapidement comme des chevaux sur la piste d’un cirque et si serrés, épaule contre épaule, qu’un titanesque acrobate aurait aisément pu chevaucher ceux du milieu et poursuivre sur leurs dos cette ronde. Nous n’avions à présent aucune chance d’échapper tant la troupe immobile était dense aux abords immédiats de l’axe encapé de leur foule. Il nous fallait attendre l’ouverture d’une brèche dans ce mur vivant qui nous cernait, ce mur qui ne nous avait laissé pénétrer que pour mieux se refermer sur nous. Au centre de ce lac, de petites vaches peu craintives et leurs veaux venaient, de temps à autre, nous faire visite, les femmes et les enfants de l’armée en déroute.\par
En comptant les espaces vides séparant les cercles extérieurs et les intervalles entre les différents groupes du centre, la superficie que couvrait cette multitude était pour le moins de deux ou trois milles carrés. De notre pirogue basse, les souffles semblaient jouer presque jusqu’à l’horizon, bien qu’en vérité ç’ait été peut-être une illusion. Je mentionne ce fait pour montrer que ces vaches et ces veaux, comme s’ils avaient été volontairement placés au centre du parc et comme si la dispersion extrême de la troupe les eût empêchés de connaître les raisons précises de cette halte, à moins que ce ne fût à cause de leur extrême jeunesse, de leur naïveté, de leur innocence et de leur inexpérience totales, que ces petits baleines approchant de notre pirogue encalminée faisaient preuve d’une confiance \hspace{1em} et d’une absence de crainte étonnantes. Peut-être aussi étaientelles subjuguées ! Toujours est-il qu’il était difficile de ne point s’en émerveiller. Comme des chiens familiers, elles venaient nous toucher nos plats-bords, tant et si bien qu’un envoûtement semblait les avoir rendues amicales. Queequeg leur fessait le front, Starbuck leur grattait le dos de la pointe de sa lance mais, redoutant les conséquences, se gardait de les piquer.\par
Loin en dessous de ce monde surprenant de la surface, univers plus étrange encore s’offrait à notre regard penché. Sous ces voûtes liquides, on distinguait les formes des mères nourricières et celles dont la taille gonflée disait qu’elles allaient bientôt le devenir. Ce lac où nous étions était d’une transparence extrême jusqu’à de grandes profondeurs, et tout comme des enfants à la tétée fixent un regard calme loin du sein qui leur donne leur nourriture terrestre, un regard qui boit au festin spirituel de quelque souvenir céleste, les petites baleines semblaient lever les yeux vers nous, sans nous voir, comme si nous n’étions qu’un brin de sargasse à leur vue toute neuve. Flottant sur le flanc, les mères aussi nous contemplaient. L’un de ces nourrissons qui, à certains signes, ne devait pas avoir plus d’un jour, devait mesurer quelque quatorze pieds de long et six de tour de taille. Il était folâtre, bien que son corps parût encore modelé par la position ingrate qu’il occupait, il y a si peu de temps, dans le ventre de sa mère où, tête et queue rapprochées, la baleine à naître est courbée comme un arc tartare prêt à se détendre. Les délicates nageoires pectorales et les palmes de sa queue avaient l’aspect froissé des oreilles d’un enfant nouveauné fraîchement arrivé d’un monde étranger.\par
– La ligne ! la ligne ! cria Queequeg penché sur le platbord. Lui pris ! lui pris ! Qui ligne lui ? Qui frappé ? Deux baleines, une grosse, une petite !\par
– Qu’est-ce qui te prend, mon gars ? demanda Starbuck.\par
– Vous voir ici ! dit Queequeg en désignant les profondeurs.\par
De même que la baleine frappée a déroulé des brasses de ligne hors de la baille, et, après avoir sondé, remonté tandis que la ligne mollie flotte en boucles, le \hspace{1em}cordon \hspace{1em}ombilical \hspace{1em}de M\textsuperscript{me} Léviathan apparaissait à Starbuck longuement lové et le jeune veau semblait être encore rattaché à sa mère. Il advient plus d’une fois, dans les hasards de la chasse, que cette ligne naturelle, lorsqu’elle s’est détachée de la mère, vienne s’entortiller au filin de chanvre, faisant ainsi prisonnier le jeune. La mer, en cette mare enchantée, nous livrait quelques-uns de ses plus subtils secrets. Nous vîmes dans les profondeurs les intrigues galantes des jeunes léviathans\footnote{Le cachalot, à l’instar des autres espèces de léviathans, mais contrairement à la majorité des poissons, procrée indifféremment en toute saison après une gestation qu’on peut sans doute fixer à neuf mois. La femelle ne met bas qu’un seul petit, quoique en de rares cas on ait vu naître un Ésaü et un Jacob, auquel cas deux tétons apparaissent curieusement de part et d’autre de l’anus, les mamelles proprement dites étant situées sous le ventre. Lorsque par hasard ces précieux attributs d’une mère nourricière sont atteints par la lance du chasseur, son lait jaillissant et son sang s’unissent pour teinter la mer loin autour d’elle. Ce lait est très doux et très riche, l’homme y a goûté… il accompagnerait très bien des fraises. Lorsqu’elles débordent d’estime mutuelle, les baleines se le témoignent {\itshape mare hominum.}}.\par
Ainsi ces créatures impénétrables, bien qu’environnées de cercles concentriques d’épouvante et de consternation s’abandonnaient librement dans le centre à une sollicitude sans crainte, se livrant sereinement à un délicieux badinage. Ainsi, au cœur de l’ouragan atlantique de mon être, à jamais paisible, en mon centre je m’ébats dans un calme muet, et tandis que les graves planètes d’un malheur croissant refermant autour de moi leur course dans ma profondeur et dans ma largeur, je baigne encore dans l’éternelle douceur de la joie.\par
Tandis que nous demeurions dans l’extase, parfois dans le lointain, un spectacle forcené nous rappelait que les autres pirogues, à la frontière de la troupe, jetaient toujours les dragues, ou peut-être portaient la guerre au sein du premier cercle où ni l’espace ni les chances de retraite ne leur manquaient. Mais la vue des cachalots enragés par les dragues et qui de temps en temps s’élançaient aveuglément à travers les cercles, était peu de chose à côté du spectacle qui nous attendait. Il est de coutume parfois, lorsqu’on se trouve lié à une baleine d’une force et d’une vivacité peu ordinaires, de chercher à lui couper les jarrets si l’on ose dire, en tranchant ou en mutilant le gigantesque tendon de sa queue. On se sert à cette fin d’une pelle à découper à manche court que l’on lance et que l’on peut haler grâce au filin qui lui est attaché. Nous l’apprîmes plus tard, un cachalot ainsi atteint, mais semble-t-il inefficacement, débordant de la baleinière, avait emporté la moitié de la ligne du harpon et, dans sa douleur intolérable, il s’était rué à travers les cercles, semant sur son passage l’épouvante, tel le téméraire Arnold, cavalier seul à la bataille de Saratoga.\par
Si atroce que fût sa blessure, si effroyable le spectacle qu’elle offrait, l’horreur particulière que ce cachalot semblait inspirer au reste de la troupe avait une cause que la distance tout d’abord nous dissimula. Nous vîmes enfin que, par un de ces accidents inconcevables de la chasse, ce léviathan s’était empêtré dans la ligne de harpon qu’il remorquait et s’était également enfui, la pelle fichée en lui. Mais cependant que le bout libre aiguilleté à cette arme s’était pris dans la ligne qui lui entourait la queue, la pelle s’était arrachée à son corps, de sorte que, fou de douleur, il barattait l’eau, la battait avec violence de sa souple queue et, faisant voler autour de lui la pelle acérée, il blessait et tuait ses propres compagnons.\par
Devant cette terreur nouvelle, toute la troupe fut ressaisie de sa crainte figée. Tout d’abord, les cachalots qui formaient les rives de notre lac commencèrent à se rapprocher les uns des autres en se bousculant, comme portés par une houle de fond, puis le lac lui-même se mit à gonfler faiblement, les chambres nuptiales et les crèches sous-marines s’évanouirent, les cercles se resserrèrent et les baleines du centre s’ébranlèrent en grappes serrées. Oui, c’en était fini d’une longue paix. Un bourdonnement sourd approchait et, tel le roulement assourdissant des glaces au moment de la débâcle printanière de l’Hudson, la troupe toute entière des cachalots déferla vers le centre comme pour s’y empiler en une seule montagne. Aussitôt, Starbuck et Queequeg changèrent de place, Starbuck prenant l’aviron de queue.\par
– Lève rames ! lève rames ! chuchota-t-il ardemment, en empoignant l’aviron de queue. Agrippez vos rames, prenez en mains vos âmes ! Mon Dieu, homme, attention ! Vous, Queequeg, repoussez cette bête-là ! Piquez-la ! Frappez-la ! Debout ! Debout ! et restez debout ! Sautez, les hommes, nagez, peu importe leur dos… raclez-les… passez par-dessus !…\par
La pirogue était à présent étouffée entre deux énormes masses noires dans un détroit aussi resserré que celui des Dardanelles. Après un effort désespéré, nous atteignîmes enfin une zone momentanément libre et la traversant à force de rames, nous guettâmes en même temps avec anxiété une nouvelle issue. Après l’avoir plusieurs fois échappé belle, nous glissâmes enfin rapidement dans ce qui avait été l’un des cercles extérieurs mais qui n’était plus traversé que par les cachalots se précipitant vers le seul centre. Ce salut miraculeux fut acheté à bon marché par la seule perte du chapeau de Queequeg qui, debout à l’avant pour piquer les baleines en fuite, le vit rondement balayé de sa tête par l’éventail brusque de deux palmes toutes proches de lui.\par
Si tumultueuse et désordonnée qu’ait été cette débandade générale, elle ne tarda pas à s’organiser dans ce qui parut être une tactique réglée. En effet, s’étant enfin réunis en un seul corps, les cachalots reprirent la fuite à une vitesse redoublée. Il était vain de les poursuivre plus avant mais les pirogues s’attardèrent dans leurs sillages pour repérer ceux qui, grâce aux dragues, seraient restés à l’arrière, et aussi pour en amarrer un que Flask avait tué et sur lequel il avait planté pavillon car toute baleinière est parée de deux ou trois pavillons qui sont fichés dans une baleine morte si un autre gibier se trouve à portée afin de la retrouver en mer, et aussi en signe de propriété pour le cas où elle viendrait à être approchée par les pirogues d’un autre navire.\par
Le résultat de cette chasse illustre le sage dicton des pêcheurs : « Plus il y a de baleines, moins on en prend. » De tous les cachalots à dragues, un seul fut pris, les autres avaient pour l’instant réussi à s’échapper mais à seule fin d’être repris, comme nous l’allons voir, par un navire autre que le {\itshape Péquod.}
\chapterclose


\chapteropen
\chapter[{CHAPITRE LXXXVIII. Écoles et maîtres d’école}]{CHAPITRE LXXXVIII \\
Écoles et maîtres d’école}\renewcommand{\leftmark}{CHAPITRE LXXXVIII \\
Écoles et maîtres d’école}


\chaptercont
\noindent Le chapitre précédent parle de l’attroupement immense des cachalots et de ses raisons probables.\par
Bien qu’on rencontre occasionnellement des agglomérations aussi considérables, on peut, de temps à autre, observer des réunions de vingt à cinquante individus. Elles portent le nom d’écoles, il y en a de deux sortes, les unes presque uniquement composées de femelles, les autres ne rassemblant que de jeunes mâles vigoureux communément désignés sous le nom de taureaux.\par
Le chevalier servant d’une école de femmes est invariablement un mâle dans sa pleine maturité mais pas un vieux, qui prouve sa galanterie, à la moindre alarme, en fermant la garde pour protéger la fuite de ses dames. En fait, ce gentilhomme est un voluptueux oriental, voyageant dans le monde liquide, entouré par toutes les consolations et les caresses du harem. Le contraste est frappant entre ce pacha et ses concubines car, cependant qu’il est toujours de la taille maximum parmi les léviathans, les dames, même mûres, n’ont pas plus du tiers de la taille d’un mâle moyen. En vérité, elles ont une certaine finesse, oserais-je dire… elles n’ont guère plus d’une demi-douzaine de mètres de tour de taille. On ne saurait nier pourtant que, tout compte fait, elles ont un penchant héréditaire en l’en bon point.\par
Il est très curieux d’observer ce seigneur et son harem dans leurs indolents vagabondages. Tels des mondains, ils sont toujours à l’affût d’oisives nouveautés. Vous les rencontrez sur la ligne à la saison de l’épanouissement des nourritures équatoriales, au retour sans doute d’un été passé dans les mers du Nord, ayant échappé à ce que l’été apporte de fatigue et de chaleur déplaisantes. Après avoir flâné en long et en large sur la promenade de l’équateur, ils partent pour les eaux orientales afin d’y passer la saison fraîche et d’échapper aux températures excessives de certaines latitudes.\par
Au cours de ces voyages tranquilles, notre seigneur cachalot garde un œil vigilant sur son intéressante famille, pour le cas où se présenterait quoi que ce soit de suspect. Que quelque jeune léviathan gaillard ait l’inqualifiable présomption de serrer de trop près l’une de ces dames, avec quelle prodigieuse fureur ce pacha se jettera sur lui et le mettra en fuite ! Il ferait beau voir en vérité qu’un jeune libertin sans principes de son espèce ait licence de violer le sanctuaire du bonheur domestique ; mais quoi qu’il fasse, ce pacha ne peut tenir loin de sa couche le plus notoire des Don Juan car, hélas ! tous les poissons ont un lit commun. Tout comme à terre, les dames suscitent les duels les plus terribles entre leurs admirateurs rivaux, les cachalots en viennent parfois à des combats meurtriers, et tout cela au nom de l’amour. Ils tirent l’arme de leurs longues mâchoires inférieures, les emboîtant parfois l’une dans l’autre, luttant pour la souveraineté, tels des élans qui se battent les andouillers entrecroisés. Nombre de ceux qui ont été capturés portent les marques de ces rencontres : têtes labourées, dents cassées, nageoires découpées, parfois des gueules démises et tordues.\par
Si l’intrus de l’intimité domestique est mis en fuite à la première menace du seigneur, celui-ci est très divertissant à observer. Il insinue à nouveau doucement sa lourde masse parmi ses femmes plein de complaisance, infligeant le supplice \hspace{1em}de Tantale au jeune Don Juan encore proche, tel le pieux Salomon rendant un culte à ses mille concubines. S’il y a un autre gibier dans les parages, il est rare qu’un pêcheur livre la chasse à l’un de ces grands Turcs ; trop prodigues de leurs forces, ils n’ont guère de graisse. Quant aux filles et aux fils qui leur naissent, ma foi ! que ces fils et ces filles aient à s’occuper d’eux-mêmes car ils ne peuvent compter que sur l’appui maternel. À l’instar d’autres amants insatiables et vagabonds que l’on pourrait nommer, notre seigneur cachalot, n’a point le goût des pouponnières, si ardent que soit celui qu’il a des boudoirs. Ainsi, grand voyageur, il sème ses bébés anonymes de par le monde entier, chacun d’eux étant exotique Avec le temps toutefois le feu de la jeunesse décroît, tandis qu’avec les ans s’accumulent les idées noires et que la méditation accorde des césures solennelles. Bref, une lassitude générale submerge le grand Turc repu, l’amour de ses aises et de la vertu supplante l’amour des femmes et notre pacha entre dans la phase de la vie impotente repentante ; il fait des remontrances, il répudie et disperse son harem, et devenu une vieille barbe exemplaire et revêche, il va de méridiens en parallèles marmottant ses prières et mettant en garde les jeunes léviathans contre ses propres erreurs amoureuses.\par
Le harem étant appelé école par les pêcheurs, son seigneur et maître se nomme, en terme de métier, maître d’école. Dès lors, s’il n’est pas très logique il est toutefois admirablement ironique qu’après avoir été lui-même à l’école, il aille de par le monde enseignant non ce qu’il a appris mais l’insanité de ces leçons. Son titre semble venir naturellement du nom donné au harem lui-même mais certains pensent que le premier qui a ainsi baptisé ce pacha avait lu les mémoires de Vidocq, et songeait à la singulière sorte de maître d’école campagnard qu’avait été dans sa jeunesse ce fameux Français et de quelle nature étaient les leçons occultes qu’il inculquait à certains de ses élèves.\par
La même retraite et le même isolement dans lesquels se retire le cachalot-maître d’école sont également le fait de tous les autres cachalots d’un âge avancé. Presque universellement un solitaire – comme on nomme un léviathan isolé – s’avère être un ancien. Comme le vénérable Daniel Boone à la barbe moussue, il ne se laissera approcher de personne sinon de la Nature elle-même, c’est elle qu’il prend pour épouse dans le désert des profondeur et c’est la meilleure des femmes bien qu’elle garde tant de secrets ombrageux.\par
Les écoles formées seulement de mâles jeunes et vigoureux offrent un contraste saisissant avec les écoles-harems car, tandis que les femelles sont timides par nature, les jeunes mâles, ou taureaux quarante-barils, comme les appellent les pêcheurs, sont de loin les plus combatifs des léviathans, et de notoriété générale les plus dangereux à affronter, exception faite de ces étonnantes têtes-grises, les cachalots grisons qu’on rencontre parfois, et qui se battront avec un acharnement diabolique, exaspérés qu’ils sont par leurs perpétuels rhumatismes.\par
Les écoles de taureaux quarante-barils comportent plus d’individus que les écoles-harems. Comme une foule de jeunes étudiants, ils ont le goût de la bagarre, de la plaisanterie, ils sont méchants et mènent de par le monde un train de vie si désordonné, si insouciant, si tapageur qu’aucun membre de syndicat de garantie ne leur concéderait une assurance pas plus qu’il ne le ferait pour un gars de Yale ou de Harvard en rupture de ban. Cependant ils abdiquent bientôt cette turbulence et, lorsqu’ils sont aux trois quarts adultes, ils rompent et partent de leur côté en quête d’une situation, c’est-à-dire d’un harem.\par
Une autre différence s’affirme entre les écoles des mâles et celles des femelles, elle est caractéristique des sexes. Supposons que vous avez harponné un taureau quarante-barils… pauvre diable ! tous ses camarades l’abandonnent. Mais harponnez un membre d’une école-harem, toutes ses compagnes l’entoureront avec une inquiète sollicitude, s’attardant parfois si près et si longtemps qu’elles deviennent les victimes à leur tour.
\chapterclose


\chapteropen
\chapter[{CHAPITRE LXXXIX. Poissons amarrés et poissons perdus}]{CHAPITRE LXXXIX \\
Poissons amarrés et poissons perdus}\renewcommand{\leftmark}{CHAPITRE LXXXIX \\
Poissons amarrés et poissons perdus}


\chaptercont
\noindent L’allusion faite dans l’avant-dernier chapitre aux pavillons que l’on plante dans la baleine demande un commentaire des lois et règlements en usage dans la pêcherie dont on peut regarder ce pavillon comme le symbole majeur.\par
Il arrive souvent que lorsque plusieurs navires sont en croisière sur le même parage l’un d’eux tue une baleine, celle-ci lui échappant pour être tuée et prise par un autre, ce qui entraîne indirectement quantité de petits imprévus à partir d’une même cause. Par exemple, après la chasse fatigante, dangereuse et la capture d’une baleine, un violent orage peut arracher la carcasse au navire où elle est amarrée ; elle s’en ira sous le vent, dérivant fort loin, et sera reprise par un autre navire baleinier qui, le calme revenu pourra la remorquer confortablement sans risque aucun. Aussi, en découlerait-il souvent, entre pêcheurs, les querelles les plus fâcheuses et les plus violentes n’était l’existence d’une loi incontestée, universelle, écrite ou orale, applicable à tous les cas.\par
Peut-être le seul code baleinier enregistré par un texte législatif est-il celui de la Hollande. Il fut décrété par les États généraux en 1695. Bien qu’aucune autre nation n’ait eu de loi écrite, les pêcheurs américains ont été leurs propres législateurs et hommes de loi en ce domaine. Ils ont élaboré un système qui, dans sa concision et son intelligibilité, l’emporte sur les Pandectes de Justinien et les arrêtés municipaux de la Société chinoise pour la répression du goût de se mêler des affaires des autres. Oui, ces lois sont si succinctes qu’elles tiendraient gravées sur une pièce d’un sous du temps de la reine Anne ou sur une barbelure de harpon, portée en pendentif.\par
I : Un poisson amarré appartient à qui l’a amarré.\par
II : Un poisson perdu appartient au premier qui le prend.\par
L’ennui, c’est que l’interprétation de l’admirable brièveté de ce maître-code réclame un fort volume de commentaires.\par
D’abord : Qu’est-ce qu’un poisson amarré ? Vivant ou mort, un poisson est techniquement amarré quand il est relié à un navire ou à une pirogue non abandonnés par ses hommes et capable d’être remorqué par n’importe quel moyen par le ou les occupants du navire, que ce soit à l’aide d’un mât, d’un aviron, d’un câble de neuf pouces, d’un fil de télégraphe ou d’un fil d’araignée, tout est bon. De même, un poisson est techniquement amarré lorsqu’il porte un pavillon ou n’importe quelle autre marque distincte de propriété, pour autant que l’équipe qui y a planté son drapeau prouve clairement qu’elle est à même de le remorquer jusqu’au flanc de son navire n’importe quand, ainsi que son intention de le faire.\par
Mais c’est là vocabulaire scientifique. Les explications des baleiniers eux-mêmes consistent parfois en mots durs et en coups plus durs encore : le Coke-sur-Littleton du poing. À vrai dire, les plus honorables et les plus justes des baleiniers font des concessions dans les cas particuliers où il y aurait injustice morale révoltante à faire valoir des droits sur une baleine chassée et tuée par un autre navire que celui qui cherche à se l’approprier. D’autres ne sont à aucun égard aussi scrupuleux.\par
Il y a quelque cinquante ans fut intentée, en Angleterre une curieuse action en restitution au sujet d’une baleine : les plaignants faisaient valoir qu’après une rude chasse les mers du Nord, ayant réussi à harponner leur gibier, ils furent contraints pour sauver leurs propres vies d’abandonner non seulement leurs lignes mais encore leur baleinière. Les défendeurs (l’équipage d’un autre navire) s’emparèrent de cette baleine harponnée, tuée, capturée, sous les yeux mêmes des plaignants. Lorsque le reproche leur en fut fait, leur capitaine claqua ses doigts sous leur nez et les assura que, en guise de {\itshape Gloria Patri}, il s’approprierait encore leur ligne, leurs harpons et leur baleinière qui étaient restés amarrés à la baleine quand ils l’avaient prise. Aussi les plaignants entamèrent-ils une poursuite afin d’être indemnisés de la valeur de leur baleine, de leur ligne, de leurs harpons et de leur pirogue.\par
M. Erskine était avocat-conseil des défendeurs, lord Ellenborough jugeait. Au cours de la défense, le spirituel Erskine illustra sa position en faisant allusion à un cas récent d’adultère dans lequel un monsieur, après avoir vainement essayé de mettre un frein à la perversité de sa femme, l’avait, pour finir, abandonnée sur l’océan de la vie, mais, les années aidant, il regretta sa décision, et intenta une action pour rentrer en sa possession. Erskine plaida contre lui les arguments suivants, à savoir que si ce monsieur avait bel et bien harponné la dame au départ, l’avait amarrée, il l’avait ensuite abandonnée, vu le désarroi où le mettaient ses trop vicieuses plongées. L’abandon étant effectif, elle devenait un poisson perdu, de sorte que, reprise par un monsieur suivant, la dame devenait sa propriété, ainsi que tout harpon qu’il eût pu retrouver dans sa chair.\par
Le cas actuel, soutint Erskine, montre combien les exemples de la dame et de la baleine sont représentatifs l’un de l’autre.\par
Les exposés des deux parties ayant été dûment entendus, le très docte juge trancha ainsi : il adjugeait la pirogue aux plaignants vu qu’ils ne l’avaient abandonnée que pour sauver leur vie, mais la baleine objet du litige, la ligne et les harpons appartenaient aux défendeurs, la baleine, car elle était un poisson perdu au moment où ils la prirent, et les harpons et la ligne, parce que ceux-ci étaient propriété du poisson qui les avait emportés, de sorte que quiconque capturait ce poisson avait des droits sur lesdits engins.\par
Un homme ordinaire pourrait peut-être trouver à redire à cette décision du très docte juge. Mais labourée jusqu’au roc, cette affaire met en relief les deux grands principes baleiniers cités plus haut et appliqués par lord Ellenborough dans le cas susdit. À la réflexion, toute jurisprudence humaine repose sur ces deux lois concernant poisson amarré et le poisson perdu. Nonobstant les entrelacs de ses sculptures, le temple de la loi, comme celui des Philistins n’a que deux colonnes pour le soutenir.\par
Un adage courant ne dit-il pas, que la propriété fait la moitié de la loi, sans se soucier de la manière dont une chose a été acquise. Mais souvent la propriété fait toute la loi. Que sont les muscles et les âmes des serfs russes et les esclaves républicains sinon des poissons attachés dont la propriété fait loi absolue ? Qu’est le dernier denier de la veuve au rapace logeur sinon un poisson amarré ? Qu’est la maison de marbre de ce scélérat non démasqué à laquelle sa plaque sert de drapeau de repérage, sinon un poisson amarré ? Qu’est l’escompte que Mardochée, le courtier, retient d’avance à l’inconsolable, au pauvre failli sur le prêt qui doit empêcher les siens de mourir de faim, qu’est ce taux usuraire sinon un poisson amarré ? Qu’est le revenu de cent mille livres de l’archevêque de Sauvez-vos-âmes, provenant de la saisie du misérable pain et du fromage de centaines de milliers de travailleurs rompus (assurés du ciel sans le secours de Sauvez-vos-âmes, qu’est-ce donc ce revenu sinon un poisson amarré ? Que sont les villes et les hameaux héréditaires du duc de Lourdaud sinon un poisson amarré ? Qu’est l’Irlande pour le redoutable harponneur qu’est John Bull, sinon un poisson amarré ? Qu’est le Texas pour ce lancier apostolique frère Jonathan, sinon un poisson amarré ? Pour tous ceux-là la propriété n’a-t-elle pas force de loi ?\par
Si la doctrine du poisson amarré est joliment applicable en général, celle, proche parente, du poisson perdu, l’est encore davantage, internationalement et universellement.\par
Qu’était l’Amérique, en 1624, sinon un poisson perdu sur lequel Colomb planta les couleurs espagnoles afin de la repérer pour le bénéfice de ses souverains ? Que fut la Pologne pour le Tsar ? La Grèce pour les Turcs ? L’Inde pour l’Angleterre ? Que sera le Mexique pour les État-Unis pour finir ? Tous des poissons perdus.\par
Que sont les Droits de l’homme et la liberté sinon des poissons perdus ? Que sont les idées et les opinions de tous les hommes sinon des poissons perdus ? Quel est en eux le principe de la foi religieuse sinon un poisson perdu ? Que sont les pensées des philosophes aux yeux des critiques vaniteux et jugeant sur les seules apparences sinon des poissons perdus ? Qu’est la planète elle-même sinon un poisson perdu ? Et vous, lecteur, qu’êtes-vous sinon tout à la fois un poisson amarré et un poisson perdu ?
\chapterclose


\chapteropen
\chapter[{CHAPITRE XC. Têtes ou queues}]{CHAPITRE XC \\
Têtes ou queues}\renewcommand{\leftmark}{CHAPITRE XC \\
Têtes ou queues}


\chaptercont
\noindent De balena vero sufficit, si rex habeat caput, et regina caudam.\par

\bibl{BRACTON, 1.3, c. 3}
\noindent Prise avec son contexte, cette phrase latine des livres, de lois britanniques signifie que quiconque a capturé une baleine au large des côtes de ce pays, doit en remettre la tête au Roi en tant que grand harponneur honoraire, tandis que la queue doit en être respectueusement offerte à la Reine. Partage qui ressemble fort à celui d’une pomme vu qu’il n’y a rien entre deux. Sous une forme différente, cette loi est encore en vigueur actuellement en Angleterre, et comme, à bien des égards, elle est en contradiction avec le principe général de poisson attaché et de poisson perdu, elle fera l’objet de ce chapitre. Ne procède-t-elle pas des raisons qui veulent que les chemins de fer britanniques fassent courtoisement les frais d’un wagon spécial pour les aises de la royauté. Pour vous prouver d’abord que cette loi est encore en vigueur, je vous rapporterai un fait survenu au cours de ces deux dernières années.\par
Il arriva que quelques honnêtes marins de Douvres, de Sandwich ou de l’un des Cinq Ports aient, après une dure chasse, réussi à tuer et à échouer une superbe baleine qui avait été signalée depuis le rivage. Les Cinq Ports tombent partiellement sous la juridiction d’une sorte de policier ou d’un fonctionnaire appelé Gardien des Cinq Ports. Tenant son autorité directement de la Couronne, tous les profits royaux de ces territoires lui reviennent, je pense, par concession. Quelques auteurs prétendent que c’est une sinécure. Il n’en est rien, car parfois le Gardien a fort affaire pour filouter ses profits éventuels qui ne deviennent siens que précisément parce qu’il les filoute !\par
Or, quand ces pauvres marins hâlés, pieds nus, les pantalons roulés hauts sur leurs mollets d’anguilles, eurent péniblement tiré au sec leur gras poisson, se promettant un rapport de cent cinquante livres d’argent sonnant pour son huile précieuse et ses fanons, alors qu’en imagination ils sirotaient déjà un thé de choix avec leurs épouses, et une bonne bière avec les amis sur la foi de la part qui devait leur échoir à chacun, alors s’avança un gentilhomme très savant, très chrétien et très charitable, portant sous le bras un exemplaire de Blackstone et qui, l’ouvrant sur la tête de la baleine, leur dit : « Bas les pattes ! les patrons, ce poisson est un poisson amarré. Je le saisis au nom du Gardien. » À ces mots, les pauvres marins, dans un atterrement respectueux, si spécifiquement anglais, ne sachant que répondre, se mirent à se gratter vigoureusement la tête à la ronde, leurs regards allant lugubrement de la baleine à l’étranger. Cela n’arrangea pas l’affaire pas plus que cela n’attendrit le cœur de pierre du savant gentilhomme-àl’exemplaire-de-Blackstone. Enfin l’un d’eux, après un long grattage en quête d’idées, s’enhardit à parler :\par
– S’il vous plaît, sir, qui est le Gardien ?\par
– Le Duc.\par
– Mais le Duc n’a rien à voir avec la capture de cette baleine ?\par
– Elle est sienne.\par
– Elle nous a donné beaucoup de tracas, nous avons couru des dangers et dépensé de l’argent, tout cela doit-il être versé au bénéfice du Duc ? N’aurons-nous rien d’autre pour notre peine que des ampoules ?\par
– Elle est sienne.\par
– Le Duc est-il si affreusement pauvre qu’il en soit réduit à ces extrémités pour gagner sa vie ?\par
– Elle est sienne.\par
– Je pensais venir en aide à ma vieille mère infirme sur ma part de ce poisson.\par
– Il est sien.\par
– Le Duc ne se contenterait-il pas d’un quart ou d’une moitié ?\par
– Il est sien.\par
En un mot, la baleine fut saisie, vendue et M. le duc de Wellington encaissa l’argent. Pensant que, vu sous certains angles, le cas aurait une petite chance d’être un tantinet revu, étant donné les circonstances et à cause de sa rigueur, un honnête pasteur de la ville adressa une pétition au Duc, le priant respectueusement de prendre en considération le sort de ces pauvres marins. À quoi monseigneur le Duc répondit en substance (les deux lettres furent publiées) que c’était déjà fait, qu’il avait reçu l’argent, et qu’il serait très reconnaissant au révérend de bien vouloir désormais se mêler de ses affaires (à lui, révérend). N’est-ce pas là le vieillard toujours militant, debout au carrefour des trois royaumes pour arracher de toutes parts l’aumône aux mendiants ?\par
On aura tôt fait de comprendre qu’en ce cas le prétendu droit du duc sur la baleine lui était délégué par le souverain. Il faut nous demander, dès lors, au nom de quel principe le souverain détient lui-même ce droit. Nous avons déjà parlé de la loi, Plowden nous donne la raison de principe. Selon lui, la baleine ainsi capturée appartient au Roi et à la Reine « à cause de son excellence ». D’après les plus sains commentateurs, c’est là un argument convaincant.\par
Mais alors pourquoi le Roi aurait-il la tête et la Reine la queue ? Hommes de loi, fournissez à cela une raison !\par
Dans son traité sur « L’Or de la Reine », ou sur « L’Argent de poche de la Reine », un auteur du Banc du Roi, un certain William Prynne, s’exprime ainsi : « La queue est à la Reine, afin que ses toilettes puissent être munies de baleines. » Cela fut écrit à une époque où les fanons de la baleine franche étaient fort utilisés dans les corsages féminins. Or, les fanons ne se trouvent pas dans la queue mais bien dans la tête, et l’erreur est bien lamentable pour un homme de loi perspicace tel que Prynne. La Reine serait-elle une sirène pour qu’on lui offre la queue ? Il y a peut-être bien là quelque symbole caché.\par
Selon les juristes britanniques, il y a deux poissons royaux : la baleine et l’esturgeon ; ils sont tous deux propriété royale, certaines restrictions mises à part, et fournissant nominalement le dixième des revenus ordinaires de la Couronne. Je n’ai pas connaissance d’un autre auteur ayant abordé le sujet mais j’en déduis que l’esturgeon doit être partagé de la même manière que la baleine, le Roi recevant la tête, très stupide et très élastique, particulière à ce poisson et, considéré du point de vue symbolique, ce don pourrait bien être fondé, avec humour, sur des affinités présumées. Ainsi il semble y avoir une raison à toute chose, même aux lois.
\chapterclose


\chapteropen
\chapter[{CHAPITRE XCI. Le Péquod rencontre le Bouton-de-Rose}]{CHAPITRE XCI \\
\textbf{{\itshape Le} Péquod {\itshape rencontre le} Bouton-de-Rose}}\renewcommand{\leftmark}{CHAPITRE XCI \\
\textbf{{\itshape Le} Péquod {\itshape rencontre le} Bouton-de-Rose}}


\chaptercont
\noindent On racla en vain la panse de ce léviathan en quête d’ambre gris, l’effroyable puanteur ne décourageant pas cette recherche. \par

\bibl{Sir T. Browne, V. E.}
\noindent Une semaine ou deux après la dernière aventure de chasse dont j’ai parlé, tandis que nous glissions lentement sur la mer somnolente dans la buée de midi, plus encore que les trois paires d’yeux braquées du haut des mâts, les nombreux nez du {\itshape Péquod} se montrèrent prompts à la découverte. Une odeur singulière et fort peu agréable venait de la mer.\par
– Je parierais bien quelque chose, dit Stubb, qu’il y a non loin l’un de ces cachalots dragués que nous avons chatouillés l’autre jour. Je me disais bien qu’ils ne tarderaient pas à rouler sur leurs quilles.\par
Bientôt, les vapeurs se dissipèrent devant nous et nous aperçûmes à une certaine distance un navire dont les voiles ferlées disaient qu’il avait à son flanc un cétacé de quelque espèce. Tandis que nous approchions, le pavillon à la corne d’artimon nous apprit que l’étranger naviguait sous les couleurs françaises et le tournoiement épais des rapaces marins qui planaient et s’abattaient autour de lui nous rendit évident que le cétacé remorqué était ce que les pêcheurs appellent une baleine ballonnée, c’est-à-dire une bête trouvée morte, dont le cadavre flottant n’appartient à personne. On comprendra aisément quelle odeur nauséabonde peut s’exhaler d’une telle masse. C’est pire qu’une cité assyrienne lors d’une épidémie de peste, quand les vivants n’arrivent plus à enterrer les morts, Certains la trouvent si intolérable qu’aucun appât du gain ne saurait les persuader d’amarrer pareille épave à leur flanc ; pourtant, quelques-uns ne reculent pas, nonobstant le fait que l’huile obtenue est de qualité très inférieure et ne ressemble en rien à l’essence de roses.\par
Portés par une brise expirante, nous fûmes bientôt assez proches pour voir que le Français remorquait un second cétacé, dont le bouquet semblait encore plus odorant que celui du premier. Il se révéla être un de ces cachalots sujets à caution qui paraissent se dessécher et mourir d’une indigestion ou d’une dyspepsie prodigieuses, abandonnant une carcasse dépouillée de tout ce qui pourrait ressembler à de la graisse. Néanmoins, nous verrons en temps voulu qu’un pêcheur d’expérience ne fera pas fi d’un tel gibier même s’il évite en général tous les cachalots ballonnés.\par
Le {\itshape Péquod} était maintenant si près de l’étranger que Stubb jura reconnaître le manche de sa pelle à découpe pris dans les lignes emmêlées autour de la queue du cachalot.\par
– En voilà un joli collègue ! dit-il, goguenard, debout à l’avant, un vrai chacal ! Je sais bien que ces grenouilles de Français sont minables en matière de pêche et qu’ils mettent parfois à la mer pour des brisants les prenant pour des souffles de cachalots, oui, et qu’ils partent parfois la cale pleine de chandelles de suif et de caisses de bougie de cire, de crainte que l’huile qu’ils pourront récolter ne suffise pas à tremper la mèche du capitaine, tout cela nous le savons bien, mais attention cette grenouille-là se contente de nos restes, je veux dire de ce cachalot dragué, et elle se satisfait aussi à racler les os secs de ce précieux poisson qu’il a là. Pauvre diable ! Qu’on me passe un chapeau et je lui ferai un petit cadeau au nom de la sainte charité. Car l’huile qu’il tirera de ce cachalot dragué n’est même pas bonne à brûler au bagne, ni dans la cellule d’un condamné. Quant à son autre cachalot, je suis sûr qu’on tirerait plus d’huile à débiter nos mâts pour la fonte que son paquet d’os, quoiqu’à y réfléchir, il pourrait bien contenir quelque chose de plus précieux que l’huile, oui, de l’ambre gris. Je me demande si notre vieux y a pensé. Ça vaudrait la peine d’essayer. J’en suis ! et sur ce, il se dirigea vers le gaillard d’arrière.\par
La brise faible avait laissé place au calme plat, de sorte que le {\itshape Péquod} était maintenant pris au piège de cette odeur, sans espoir de délivrance si le vent ne venait pas à se lever à nouveau. Au sortir de la cabine, Stubb manda l’équipage de sa baleinière puis se mit en devoir d’aller rendre visite à l’étranger. En passant sous sa proue, il vit que, conformément au goût fantasque des Français, la partie supérieure en était sculptée en forme d’énorme tige inclinée, peinte en vert, et qu’en guise d’épines des pointes de cuivre en jaillissaient ici et là, le tout se terminant en un bourgeon replié d’un rouge vif. En lettres d’or, sur son pavois de poulaine, il put lire « Bouton-de-Rose », tel était le nom romantique porté par ce vaisseau parfumé.\par
Bien que Stubb ne comprît pas le sens du mot bouton, le mot rose et le bourgeon pris ensemble lui furent une explication suffisante.\par
– Un bouton de rose en bois, hein ! s’écria-t-il en portant une main à son nez, voilà qui est seyant, mais il pue !\par
Afin de pouvoir parler à ceux qui étaient sur le pont, il dut contourner l’étrave et se rendre à tribord du navire, devant ainsi entretenir la conversation par-dessus le cachalot ballonné.\par
De là, se tenant toujours le nez, il brailla :\par
– Ohé, du {\itshape Bouton-de-Rose}, y a-t-il parmi vous des boutons de rose qui parlent anglais ?\par
– Oui, répondit un homme de Guernesey penché au bastingage et qui se révéla être le premier second.\par
– Eh bien, ma fleur en bouton, avez-vous vu la Baleine blanche ?\par
– Quelle baleine ?\par
– La Baleine blanche, un cachalot, Moby Dick, l’avez-vous vu ?\par
– Jamais entendu parler d’une baleine pareille ! Cachalot blanc ! Baleine blanche ! Non !\par
– Bon, alors ! au revoir, je reviens dans une minute.\par
Il fit force de rames vers le {\itshape Péquod} et voyant Achab qui, penché sur la lisse du gaillard d’arrière, attendait sa réponse, il mit ses mains en porte-voix et hurla : non, sir non ! Sur quoi Achab se retira, et Stubb retourna vers le Français.\par
L’homme de Guernesey se trouvait dans les porte-haubans avec sa pelle à découper et Stubb remarqua qu’il s’était fourré le nez dans une sorte de sac.\par
– Qu’est-ce qu’il y a qui ne va pas avec votre nez ? Cassé ?\par
– Je voudrais bien, je préférerais encore n’en pas avoir du tout, répondit l’homme qui ne semblait pas faire ses délices de son travail. Mais pourquoi tenez-vous le vôtre ?\par
– Oh ! pour rien ! Il est en cire, je dois le tenir en place. Belle journée, n’est-ce pas ? Je dirais même qu’on se croirait dans un jardin… Envoyez-nous un bouquet de fleurs des champs, voulez-vous, bouton de rose ?\par
– Du diable, que nous voulez-vous ? rugit l’homme de Guernesey pris brutalement de colère.\par
– Oh ! restez froid, oui froid, c’est bien le mot ! Pourquoi n’emballez vous pas vos baleines dans de la glace pour y travailler ? Mais blague à part maintenant. Savez-vous, bouton de rose, qu’il est vain d’espérer de l’huile de telles baleines, c’est une sottise ! Quant au séchon, là, il n’en contient pas un gallon.\par
– Je sais bien, mais, voyez-vous, le capitaine ne veut rien croire, c’est son premier voyage, il était fabricant d’eau de Cologne auparavant. Venez donc à bord, il vous croira peut-être plus facilement que moi et je me tirerai de ce sale pétrin.\par
– Tout à votre service, mon doux et charmant ami dit Stubb en grimpant aussitôt sur le pont. Une scène curieuse s’offrit à ses yeux : les matelots en bonnets de laine à pompons rouges préparaient les lourdes caliornes pour hisser les baleines. Ils travaillaient lentement, parlaient vite et ne semblaient guère de charmante humeur. Leurs nez pointaient vers le ciel comme autant de bouts-dehors ; de temps en temps, quelquesuns, plantant là leur travail, grimpaient aux mâts en quête d’un air plus pur ; d’autres, pensant qu’ils allaient attraper la peste, trempaient de l’étoupe dans du goudron et la portaient à leurs narines. D’autres encore, ayant cassé leurs tuyaux de pipe au ras du fourneau, tiraient de vigoureuses bouffées de façon à être enfumés.\par
De la cabine du capitaine dans la dunette, Stubb fut frappé d’entendre sortir des malédictions et des clameurs d’indignation, il regarda dans cette direction et vit dans l’entrebâillement de la porte un visage enflammé. C’était le chirurgien du bord en proie à des tourments et qui, après avoir fait de vaines semonces contre les activités de la journée, s’était réfugié dans la dunette, qu’il appelait le cabinet, pour éviter la pestilence, mais il ne pouvait se retenir de hurler de temps à autre des supplications furieuses.\par
Prenant bonne note de tout cela, Stubb en augura fort bien pour ses projets et, se tournant vers l’homme de Guernesey, il fit avec lui un brin de causette, au cours de laquelle le second étranger exprima la haine qu’il portait à son capitaine, un âne bâté imbu de lui-même, qui les avait jetés dans une mélasse aussi puante que peu profitable. Tâtant prudemment le terrain, Stubb s’aperçut que l’homme de Guernesey n’avait pas été effleuré par la pensée de l’ambre gris. S’il évita ce sujet, Stubb se montra néanmoins sincère et confiant envers lui, de sorte qu’à eux deux ils eurent tôt fait de comploter un petit plan destiné à la fois à circonvenir et à tourner en dérision le capitaine sans qu’il puisse douter le moins du monde de leur franchise. Leur conspiration voulait que le Français, feignant de jouer le rôle d’interprète, dise au capitaine tout ce que bon lui semblerait comme venant de Stubb, tandis que Stubb déviderait le chapelet de sottises improvisées qui lui viendraient à l’esprit.\par
À ce moment-là, leur future victime sortit de sa cabine. Il était petit et brun, d’apparence plutôt frêle pour un marin, nanti toutefois de puissantes moustaches et de favoris, et portait une veste de velours de coton rouge et des breloques à sa montre au côté. Stubb fut courtoisement présenté à ce monsieur par l’homme de Guernesey qui prit aussitôt fonction avantageuse d’interprète.\par
– Que dois-je lui dire pour commencer ?\par
– Eh bien, répondit Stubb en jetant un œil sur la veste de velours, sur la montre et sur les breloques, vous feriez bien de lui dire d’abord qu’il me paraît un peu puéril quoique je n’ai pas la prétention d’en juger.\par
– Il dit, Monsieur, dit l’autre en se tournant vers le capitaine, que pas plus tard qu’hier son navire a rencontré un vaisseau dont le capitaine, le premier second et six matelots sont morts d’une fièvre provoquée par une baleine ballonnée qu’ils avaient amarrée.\par
Le capitaine tressaillit et manifesta son désir d’en savoir davantage.\par
– Et maintenant ?\par
– Du moment qu’il le prend ainsi, dites-lui que maintenant que l’ayant bien regardé, je suis tout à fait sûr qu’il n’est pas plus fait pour commander un navire baleinier que ne le serait un singe de l’île San Jago. En fait, dites-lui de ma part qu’il est un babouin.\par
– Il affirme sur l’honneur, Monsieur, que cette autre baleine, la sèche, est encore plus meurtrière que l’autre bref, Monsieur, il nous adjure, si nous tenons à nos vies de larguer ces poissons.\par
Le capitaine se précipita sur-le-champ à l’avant et donna, d’une voix forte, l’ordre à son équipage de cesser de hisser les caliornes et de couper instantanément les câbles et les chaînes qui amarraient les cachalots.\par
– Et maintenant ? demanda l’homme de Guernesey lorsque le capitaine fut revenu.\par
– Bon, laissez-moi réfléchir… oui, vous pouvez lui dire que… que… je l’ai roulé et… (à part) peut-être quelqu’un d’autre du même coup.\par
– Il dit, Monsieur, qu’il est très heureux d’avoir pu nous rendre service.\par
À ces mots, le capitaine jura que c’étaient eux (lui-même et le second) qui étaient heureux et reconnaissants et il conclut en invitant Stubb à venir boire une bouteille de bordeaux dans sa cabine.\par
– Il voudrait que vous alliez boire un verre de vin avec lui, transmit l’interprète.\par
– Remerciez-le chaleureusement mais dites-lui que c’est contraire à mes principes de trinquer avec un homme que je viens de rouler. Bref, dites-lui qu’il faut que je parte.\par
– Il dit, Monsieur, que ses principes lui interdisent de boire, mais que si Monsieur désire vivre un jour de plus pour boire, alors que Monsieur a meilleur temps de mettre à la mer ses quatre pirogues et de déborder au plus vite de ces baleines car, avec pareille accalmie, elles ne dériveront pas toutes seules.\par
Stubb avait déjà sauté par-dessus bord et, de sa baleinière, il héla l’homme de Guernesey, lui disant qu’ayant à son bord un long câble de remorquage, il ferait son possible pour les aider en tirant la plus légère des deux baleines. Cependant que les pirogues des Français remorquaient le navire dans un sens. Stubb tirait charitablement sa baleine dans l’autre sens, filant ostensiblement un câble d’une longueur exceptionnelle.\par
La brise se leva au même instant, Stubb feignit de larguer son cachalot, ayant hissé ses baleinières, le navire français prit rapidement de la distance, tandis que le {\itshape Péquod} se glissait entre lui et la proie de Stubb. Sur quoi, Stubb amena sur le cadavre flottant, cria ses intentions au {\itshape Péquod} et se mit aussitôt en devoir de cueillir les fruits de sa fourberie. Empoignant sa pelle d’embarcation, il commença de creuser un peu en arrière de la nageoire pectorale. On aurait dit qu’il ouvrait une fosse dans la mer, et lorsque sa pelle heurta enfin les côtes décharnées, on eût dit qu’il tirait d’antiques poteries romaines d’une terre grasse d’Angleterre. L’équipage de sa baleinière tout en l’aidant était sur des charbons ardents, aussi impatients que des chercheurs d’or.\par
Et pendant tout ce temps, les oiseaux sans nombre fondaient du ciel, plongeaient, criaient et se battaient autour d’eux. Stubb commençait à être déçu cependant que s’aggravait l’horrible bouquet lorsque soudain, du sein même de cette pestilence, s’échappa l’effluve d’un léger parfum qui traversa la marée de la puanteur sans se laisser submerger, comme au confluent d’un fleuve les eaux roulent pendant un certain temps ensemble, sans se mélanger.\par
– Je l’ai, je l’ai ! s’écria Stubb avec ravissement en frappant quelque chose dans ce souterrain. Un sac ! un sac !\par
Laissant tomber sa pelle, il y enfonça ses deux mains et sortit des poignées d’une matière ressemblant à un brun savon de Windsor, ou à un fromage marbré qui eût été, de plus, onctueux et odorant. On l’entamait aisément de l’ongle et il était d’une nuance allant du jaune au gris cendré. Et ce n’était, mes amis, ni plus ni moins que de l’ambre valant une guinée d’or l’once chez n’importe quel apothicaire. La récolte fut de six poignées, il s’en perdit inévitablement plus que cela dans la mer, peut-être qu’il y en aurait eu davantage encore à en sortir n’eût été l’impatience d’Achab qui ordonna à Stubb d’abandonner et de remonter à bord, faute de quoi le navire lui ferait ses adieux.
\chapterclose


\chapteropen
\chapter[{CHAPITRE XCII. Ambre gris}]{CHAPITRE XCII \\
Ambre gris}\renewcommand{\leftmark}{CHAPITRE XCII \\
Ambre gris}


\chaptercont
\noindent L’ambre gris est une curieuse matière et commercialement si importante qu’en 1791 un certain capitaine Coffin, originaire de Nantucket, fut interrogé à ce sujet à la Chambre des Communes. Car, à cette époque, et à vrai dire jusqu’à une date assez récente, l’origine de l’ambre gris, tout comme celle du succin, restait un problème pour les savants. Le mot ambre gris est un mot français passé dans la langue anglaise, et n’a rien de commun avec le succin, dit ambre tout court. Car celui-ci, bien qu’on le trouve parfois sur les rivages marins, est souvent tiré des profondeurs du sol au cœur des terres, cependant qu’on ne trouve l’ambre gris que dans la mer. D’autre part, c’est une matière dure, transparente, cassante et inodore dont on fait des tuyaux de pipes, des colliers et des colifichets tandis que l’ambre gris est malléable, cireux, et si hautement aromatique qu’on l’utilise en parfumerie, dans des pastilles à brûler, dans les bougies de luxe, les poudres à cheveux et les pommades. Les Turcs en épicent la nourriture, ils en emportent à la Mecque pour la même raison que l’encens va à St-Pierre à Rome. Quelques vignerons en mettent quelques grains dans le bordeaux pour le relever.\par
Qui irait penser, dès lors, que de si belles dames et de si beaux messieurs se régalent d’une essence tirée des tripes honteuses d’un cachalot malade ! Pourtant, c’est la vérité. Quelquesuns pensent que l’ambre gris est la cause, d’autres qu’il est l’effet de la dyspepsie chez le cachalot. Comment le guérir de pareille diarrhée, c’est difficile à dire, sinon peut-être en lui administrant la valeur de trois ou quatre pirogues de pilules de Brandreth, après quoi il conviendrait de fuir la zone dangereuse comme les ouvriers qui font sauter un rocher.\par
J’ai oublié de dire qu’on avait trouvé dans cet ambre gris des plaques rondes, osseuses ; Stubb pensa tout d’abord que c’étaient peut-être des boutons de culottes de marins, mais on comprit ensuite qu’elles n’étaient que les os de petits calmars ainsi embaumés.\par
Mais n’est-il pas surprenant de trouver cet ambre gris si pur et si odorant au cœur d’une telle pourriture ? Pense à ce que dit saint Paul dans une épître aux Corinthiens au sujet de la corruption et de la pureté et comment « semé dans l’ignominie, on ressuscite dans la gloire ». Qu’il vous souvienne aussi de l’endroit où Paracelse a dit qu’on trouvait le meilleur musc. N’oubliez pas non plus que de toutes choses malodorantes l’eau de Cologne, au stade premier de sa fabrication, se trouve être la pire.\par
J’aurais aimé terminer ce chapitre sur cette invitation, mais je ne le puis tant j’ai souci de réfuter une accusation souvent portée contre les baleiniers par certains esprits prévenus et qui peut être résumée par les qualificatifs appliqués aux deux cachalots du Français. Nous avons déjà prouvé, dans ce volume, la fausseté des noires calomnies voulant que le métier de baleinier soit un métier de souillon, un travail malpropre. Quelle est l’origine de cette odieuse flétrissure ?\par
Je suis d’avis qu’elle remonte à l’arrivée des premiers navires baleiniers groenlandais à Londres, il y a plus de deux siècles. Leurs équipages ne fondaient pas alors, pas plus que maintenant, leur graisse en mer comme l’ont toujours fait les pêcheurs des mers du Sud mais, coupant le lard frais en petits morceaux, ils le jetaient dans la bonde d’énormes barils et le transportaient ainsi jusque chez eux ; la brièveté de la saison dans ces mers glaciales, les orages aussi soudains que violents qui y sévissent, leur interdisaient tout autre procédé. D’où il suit qu’en ouvrant la cale et en déchargeant ce cimetière de baleines sur un quai du Groenland, il s’en exhale une odeur à peu près semblable à celle des déblais d’un vieux cimetière de ville, où l’on creuse les fondations d’une maternité.\par
Je conjecture aussi que cette accusation perverse portée contre les baleiniers peut être attribuée à l’existence autrefois, sur la côte du Groenland, d’un village hollandais nommé Schmerenburgh ou Smeerenberg, ce dernier nom étant celui qu’emploie le savant Fogo Von Slack dans son grand ouvrage sur les Odeurs et qui traite de ce sujet. Comme son nom l’indique (smeer, graisse, berg, montagne), ce village fut fondé pour amonceler la graisse ramenée par la flotte baleinière hollandaise afin d’y être fondue, sans être pour cela ramenée jusqu’en Hollande. Il n’était que fourneaux, chaudières et entrepôts d’huile et quand le travail battait son plein les exhalaisons n’en étaient guère suaves. Mais il en va tout différemment sur un cachalotier des mers du Sud qui, après un voyage de quatre ans peut-être, ayant rempli sa cale d’huile, n’aura pas plus de cinquante jours au travail de la fonte et dont l’huile ainsi préparée et mise en barils est pratiquement inodore. La vérité est que, vivantes ou mortes, si on les traite comme il faut, les baleines ne sont pas des êtres nauséabonds, pas plus que les baleiniers ne se signalent à leur odeur, à la façon dont les gens du Moyen Âge croyaient qu’on pouvait au flair découvrir un Juif dans une société. D’ailleurs le cachalot ne saurait être autrement que fragrant, puisqu’en général il jouit d’une excellente santé et prend un tel exercice toujours dehors, quoique rarement au grand air, il est vrai. J’affirme qu’un cachalot, lorsqu’il agite sa queue audessus de l’eau un répand un parfum, tout comme dans un salon bien chauffé, la robe bruissante d’une dame ointe de musc. Vu sa taille énorme, à quoi dès lors pourrais-je comparer le cachalot quant à sa bonne odeur, sinon au fameux éléphant, aux défenses incrustées de pierreries, frotté de myrrhe, qui \hspace{1em} fut conduit hors d’une ville des Indes à la rencontre d’Alexandre le Grand pour lui rendre honneur ?
\chapterclose


\chapteropen
\chapter[{CHAPITRE XCIII. Le naufragé}]{CHAPITRE XCIII \\
Le naufragé}\renewcommand{\leftmark}{CHAPITRE XCIII \\
Le naufragé}


\chaptercont
\noindent Peu de jours, après notre rencontre avec le navire français, un événement plein de signification survint au plus insignifiant des membres de l’équipage du {\itshape Péquod}, un événement tout à fait lamentable, dont les conséquences mirent les hommes de ce navire, parfois délirant de joie mais prédestiné, face à une vivante et constamment présente prophétie de la catastrophe qui le guettait.\par
L’équipage tout entier d’un baleinier ne prend pas la mer dans les pirogues ; quelques hommes, appelés gardiens du navire, restent à bord et gouvernent pendant que les baleinières sont en chasse. En général, ces gardiens sont des hommes aussi rudes que les canotiers mais qu’il se trouve un individu anormalement fluet et craintif, on en fera à coup sûr un gardien. Il en était ainsi à bord du {\itshape Péquod}, du petit nègre surnommé Pépin, Pip en diminutif. Pauvre Pip ! vous le connaissez déjà, vous vous souvenez de son tambourin lors de ce minuit dramatique, à la fois lugubre et joyeux.\par
D’aspect extérieur, Pip et Pâte-Molle faisaient la paire comme un poney noir et un poney blanc, d’un développement identique quoique de couleurs différentes et formant un attelage original. Mais tandis que le malheureux Pâte-Molle avait l’esprit lourd et paresseux, Pip, bien qu’hypersensible, était intelligent de nature, doué de cette vivacité charmante, bienveillante et gaie propre à sa race, qui, plus que tout autre, prend un plaisir plus pur et plus libre aux loisirs et aux fêtes. Pour les Noirs, le calendrier ne devrait compter que trois cent soixante cinq 4 juillet et Nouvel An. Ne souriez pas si je dis que ce petit noir était brillant, car la noirceur même à son éclat, pensez au lustre de l’ébène qui revêt un salon royal. Pip aimait la vie dans tout ce qu’elle pouvait apporter de sécurité paisible, de sorte que cette existence, propre à frapper de terreur, dans le piège de laquelle il était tombé on ne sait comment avait terni bien tristement ce qu’il avait de lumineux. Pourtant, nous le verrons bientôt, ce qui était passagèrement assourdi en lui, allait finalement s’embraser d’un feu sinistre, étrange, insensé, qui, tout en appartenant au domaine de l’imagination, allait décupler le rayonnement naturel qui était sien en son comté de Tolland dans son Connecticut natal, où il avait une fois animé de son violon de champêtres ébats, où, dans l’harmonie du soir avec un cri joyeux, il avait métamorphosé la rondeur de l’horizon en un tambourin serti de grelots d’étoiles. Ainsi dans le jour clair, suspendu à un cou veiné de bleu, l’eau pure d’une goutte de diamant brillera de santé mais pourtant, lorsque l’habile joaillier voudra vous montrer la pierre dans son plus émouvant éclat, il la posera sur un fond sombre et la fera briller non aux rayons du soleil mais sous une lumière artificielle lui arrachant ainsi la splendeur d’un feu d’une magnificence infernale. Alors, l’embrasement maléfique de ce diamant, auparavant le plus divin symbole des cieux de cristal, en fera une gemme volée à la couronne du Prince de l’Enfer. Mais revenons à notre histoire.\par
Il advint que dans cette affaire d’ambre gris, le rameur arrière de Stubb se foula le poignet et fut pendant quelque temps inapte, de sorte que Pip fut appelé à le remplacer.\par
Lors de la première mise à la mer de Stubb, Pip se montra fort nerveux mais par bonheur, cette fois-là, un contact proche avec le cachalot lui fut épargné et il s’en sortit sans trop de déshonneur. Stubb, l’ayant observé, prit soin par la suite de l’exhorter à s’armer le plus possible de courage, car il en aurait souvent besoin.\par
La seconde fois, la pirogue mena aux pagaies sur le cachalot ; en recevant le fer, le poisson donna l’habituelle secousse qui porta, cette fois, juste sous le banc du pauvre Pip. La surprise atterrée qu’il éprouva sur l’instant le fit sursauter à tel point que, pagaie à la main, il passa par-dessus bord, de telle manière qu’il entraîna avec lui la ligne qui se trouvait alors au niveau de sa poitrine, et une fois à l’eau il s’empêtra dedans. Au même instant, le cachalot blessé prit sa course effrénée, la ligne se tendit brusquement et presto le pauvre Pip émergea jusqu’aux tollets de la pirogue dans un bouillonnement d’écume, impitoyablement tiré jusque-là par la ligne qui lui entourait la poitrine et le cou de plusieurs tours.\par
Tashtego était debout à l’avant. L’ardeur de la chasse l’allumait. Il haïssait Pip pour sa poltronnerie. Tirant le couteau de pirogue de sa gaine, il en posa le tranchant sur la ligne et, se tournant vers Stubb, il interrogea : « Couper ? » tandis que le visage bleui par l’étouffement de Pip disait clairement : « Coupez, pour l’amour de Dieu ! » Tout se passa le temps d’un éclair, l’aventure ne dura pas plus d’une demi-minute.\par
– Qu’il soit maudit, coupez ! rugit Stubb. Ainsi le cachalot fut perdu et Pip sauvé.\par
À peine retrouvait-il ses esprits, que le pauvre petit nègre fut accablé par les malédictions et les vociférations de l’équipage. Stubb, lui, laissa libre cours à ces imprécations contraires aux règles puis, d’un ton neutre, comme on traite une affaire, mais plaisantant à demi, il maudit Pip officiellement et officieusement l’accabla de plus d’un salutaire conseil. Ses paroles pouvaient se résumer à ceci : « Ne saute jamais d’une pirogue, Pip, sauf… » mais tout le reste manquait de précision comme tout conseil judicieux. En général, la vraie devise de la chasse est : « Tiens bon à la pirogue » mais il arrive en certains cas que « Saute de la pirogue » convient mieux. Se rendant enfin compte qu’en mettant une trop grande conscience à donner des conseils à Pip, il lui laissait trop de champ pour renouveler son forfait, il renonça soudain à toute admonition et conclut par un ordre péremptoire : « Tenez ferme à la pirogue, Pip, ou, par Dieu, je ne vous repêcherai pas si vous sautez, tenez-le vous pour dit. Nous ne pouvons pas nous permettre de perdre des baleines pour des gars comme vous, un cachalot vaut trente fois ce qu’on vous vendrait en Alabama, Pip. Souvenez-vous bien de ça et ne sautez plus. » Peut-être que Stubb sous-entendait par là, que, quoiqu’il aime son semblable, un homme est un animal qui doit gagner de l’argent, penchant qui, trop souvent, entrave sa bienveillance.\par
Mais nous sommes tous entre les mains des dieux, et Pip sauta de nouveau. Les circonstances étaient à peu près celles de la première fois, sauf qu’il ne se prit pas dans la ligne et que lorsque le gibier se mit à fuir, Pip resta en arrière dans la mer, pareil à la malle d’un voyageur pressé. Hélas, Stubb ne fut que trop fidèle à sa promesse ! C’était un beau jour, bleu et généreux, le lamé d’une mer fraîche et calme s’étendait lisse jusqu’à l’horizon, tel une feuille d’or amincie à l’extrême. La tête d’ébène de Pip, s’enfonçant, surgissant, y semblait un clou de girofle. Il n’y eut point de couteau d’embarcation à brandir lorsqu’il tomba de la poupe. Le dos inexorable de Stubb ne se détourna pas et la baleine avait des ailes. En trois minutes, tout un mille d’Océan sans rivages sépara Pip de Stubb. Du centre de la mer, le pauvre Pip tourna sa tête noire, frisée dru, vers le soleil, cet autre naufragé solitaire malgré la hauteur éclatante de son abandon.\par
Par temps calme, nager au large est une affaire aussi simple pour un nageur entraîné que de voyager dans une voiture suspendue pour un terrien. Mais l’affreux sentiment d’abandon est intolérable. L’intensité avec laquelle l’être se ramasse en luimême au sein d’une aussi cruelle immensité, Seigneur, qui peut la dire ? Voyez comme les marins, lorsque, par calme plat, ils se baignent au large, voyez comme ils se serrent contre leur navire et se bornent à longer ses flancs.\par
Mais Stubb avait-il vraiment abandonné le pauvre petit nègre à son sort ? Non. Du moins telle n’était pas son intention. Car il y avait deux baleinières dans le sillage de la sienne et il pensa sans doute qu’elles auraient tôt fait de rejoindre Pip et de le repêcher, bien qu’en pareil cas les pêcheurs ne soient pas toujours disposés à de tels égards envers celui qui s’est mis en danger par sa propre pusillanimité quand bien même c’est une chose fréquente, car un couard est invariablement détesté par la pêcherie et poursuivi de cette même haine sans pitié que dans l’armée et dans la marine militaire.\par
Mais les circonstances voulurent que ces pirogues ne virent pas Pip. Absorbées par la présence proche de cachalots, elles virèrent de bord pour leur livrer la chasse. Quant à la baleinière de Stubb, elle était loin à présent et lui-même comme son équipage étaient tout à leur poursuite de sorte que l’anneau d’horizon s’élargit misérablement autour de Pip. Par un pur hasard, ce fut le navire qui le sauva enfin, mais, depuis ce moment-là, il déambula sur le pont comme un idiot, du moins c’est le terme qu’on lui appliquait. La mer moqueuse lui avait laissé son corps borné et noyé l’infini de son âme. Elle ne l’avait pas noyée tout à fait cependant, elle l’avait plutôt entraînée vive dans les profondeurs prodigieuses où les formes étranges du monde primordial encore intact glissaient ici et là devant son regard passif. La Sagesse, sirène avaricieuse, lui révélait ses trésors amassés et parmi les vérités éternelles, joyeuses, cruelles, jeunes à jamais, Pip voyait dans les innombrables insectes coralliens, l’omniprésence de Dieu qui, hors du firmament des eaux, tire les orbes immenses des atolls. Il voyait le pied de Dieu posé sur la pédale du métier à tisser et, parce qu’il le disait, ses compagnons l’appelaient fou. Fou aux yeux du monde, sage aux yeux de Dieu… et c’est en s’éloignant de la raison humaine que l’homme arrive enfin à l’esprit du ciel, pour qui la raison n’est que folie et frénésie. Devant le bonheur comme devant le malheur il n’éprouve plus que l’indifférence absolue qui est celle même de Dieu.\par
Pour le reste, ne blâmez pas trop sévèrement Stubb. C’est monnaie courante dans la pêcherie, et on verra par la suite que je fus victime d’un semblable abandon.
\chapterclose


\chapteropen
\chapter[{CHAPITRE XCIV. Mains serrées}]{CHAPITRE XCIV \\
Mains serrées}\renewcommand{\leftmark}{CHAPITRE XCIV \\
Mains serrées}


\chaptercont
\noindent Cette baleine de Stubb, si chèrement acquise, fut dûment amenée aux flancs du {\itshape Péquod}, et l’on procédait à toutes les opérations habituelles de dépeçage, de hissage, jusqu’à la vidange du foudre Heidelberg dont nous avons parlé précédemment.\par
Pendant que certains s’occupaient à ce dernier travail, d’autres emportaient les baquets plus grands dès qu’ils étaient pleins de spermaceti, et ce même spermaceti était soigneusement préparé avant d’aller aux chaudières dont il sera question plus loin.\par
Il s’était refroidi et cristallisé à tel point que, lorsque avec plusieurs marins je m’assis devant une immense baignoire de Constantin qui en était emplie, je le trouvai figé en mottes, roulant ça et là dans la partie restée liquide. Notre travail consistait à rendre leur fluidité à ces mottes. Doux et onctueux travail ! Rien d’étonnant à ce que jadis le spermaceti ait été un cosmétique si prisé. Un tel pouvoir détersif, un tel pouvoir émollient, un si délicieux adoucissant ! À peine mes mains y étaient-elles depuis quelques minutes, que mes doigts se sentirent devenir anguilles, serpents, spirales.\par
Tandis que j’étais confortablement assis à la turque sur le pont, après un rigoureux effort au guindeau, sous la paix d’un ciel bleu, le navire glissant sereinement sous ses voiles indolentes, tandis que mes mains baignaient dans ces globules de tissus si doux, coagulés sur l’heure, tandis que leur opulence fondait sous mes doigts comme la grappe mûre qui abandonne son vin, tandis que je respirais ce vierge parfum, en vérité tout pareil à celui des violettes au printemps, je vous le dis, pendant ce temps je vécus dans l’odeur fauve d’une prairie. J’avais tout oublié de notre serment atroce ; dans cet indicible spermaceti j’en lavais et mes mains et mon cœur. J’en venais presque à croire avec Paracelse que cette matière avait la précieuse vertu de tempérer l’ardeur de la colère. Plongé dans ce bain, je me sentais divinement libéré de toute malveillance, de toute irritabilité, de toute rancune de quelque nature que ce soit.\par
Serrer, presser, la matinée durant ! J’étreignais ce spermaceti jusqu’à m’y fondre, jusqu’à ce qu’enfin une étrange folie m’envahit et je me surpris à serrer involontairement les mains de mes camarades, les prenant pour des mottes douces. Ce travail faisait naître un tel débordement d’affection, de fraternité, d’amour que pour finir je continuai à étreindre leurs mains, les regardant tendrement dans les yeux comme pour leur dire : Oh ! mes bien-aimés semblables, pourquoi nourririons-nous des rancunes sociales, des humeurs acariâtres, de l’envie ? Allons, serrons-nous tous les mains, non, faisons davantage, fondons-nous les uns dans les autres, perdons-nous dans l’universel et devenons le lait et le spermaceti de la bonté.\par
Que n’ai-je pu presser à jamais ce spermaceti ! Car je sais à présent, par tant d’expériences prolongées, renouvelées, que l’homme doit abaisser ou du moins déplacer l’idée qu’il se faisait d’un bonheur accessible, qu’il ne doit pas le chercher dans l’intelligence ou l’imagination, mais dans la compagne, le cœur, le lit, la table, la selle du cheval, le coin du feu, le pays ; maintenant que j’ai compris cela, je suis prêt pour une éternelle étreinte. Dans mes nocturnes visions j’ai vu, au paradis, des anges défiler longuement, tenant entre leurs mains une jarre de spermaceti.\par
Dans une dissertation sur le spermaceti, il convient d’aborder les questions qui s’y rapportent, entre autres la manière de le préparer en vue de la fonte.\par
On extrait d’abord la crête blanche, puisque c’est ainsi qu’on la nomme, de la partie effilée du poisson ainsi que des parties les plus épaisses des palmes de la queue. Véritable pelote de muscles et de durs tendons, cette partie contient néanmoins un peu d’huile. Après avoir été détachée du corps, la crête blanche, avant d’être émincée, est débitée en rectangles maniables, ressemblant beaucoup à des blocs de marbre du Berkshire.\par
On appelle plum-pudding certains morceaux de chair adhérents à la couverture de lard et qui contribuent à lui donner son onctuosité. Rien n’est plus appétissant, plus rafraîchissant, plus beau à voir ! Comme son nom le laisse entendre, la couleur en est extrêmement riche, tachetée panachée de blanc pur et d’or, pointillée intensément de cramoisi et de pourpre. Prunes de rubis parmi des cédrats. En dépit de la raison, on a peine à se retenir d’en manger, j’avoue que je suis allé me cacher derrière le mât de misaine pour y goûter. Je pense que cette viande devait avoir le goût qu’aurait eu une tranche royale coupée dans la cuisse de Louis le Gros, en admettant qu’il ait été tué au jour de l’ouverture de la chasse et dans une année particulièrement bonne pour les vins de Champagne.\par
Une autre matière, qui se présente au cours de ce travail, très singulière, et que je trouve difficile à décrire correctement, c’est la vase, comme l’appellent les baleiniers, appellation justifiée par la nature de cette substance inexprimablement boueuse, filandreuse, qui se dépose dans les cuves de spermaceti après qu’il ait été longuement malaxé et qu’il ait décanté. Je pense que ce sont des membranes détachées de la paroi interne de la boîte à spermaceti, merveilleusement minces et qui se sont soudées.\par
Le rebut, terme qui appartient au vocabulaire des seuls chasseurs de la baleine franche et se trouve incidemment utilisé par les pêcheurs de cachalots, est une substance sombre, glutineuse qui est raclée sur le dos de la baleine du Groenland et qui couvre les ponts des navires de ces âmes basses qui chassent ce léviathan ignoble.\par
La canine. Ce mot n’est pas spécifique du vocabulaire des baleiniers. Pourtant, utilisé par eux, il le devient. La canine est une bande courte et ferme de tissus tendineux taillée dans la partie la plus étroite de la queue du léviathan ; elle a un pouce d’épaisseur et à peu près la dimension d’un fer de houe. Si l’on promène son tranchant sur le pont, elle fait office de balai de cuir et, grâce à un charme sans nom, comme par magie, elle amadoue toutes les saletés.\par
Mais pour tout apprendre sur ces mystérieuses questions, le mieux que vous puissiez faire est de descendre dans le parc au gras et de parler longuement avec ses hôtes. Nous l’avons déjà dit, c’est dans cette chambre que sont déposés les morceaux de l’enveloppe après le dépeçage. Lorsque le temps vient de les débiter, ce lieu devient un théâtre de terreur pour tous les novices, surtout de nuit ; d’un côté un espace faiblement éclairé d’un falot a été laissé libre pour les hommes qui doivent y peiner. Ils travaillent deux par deux en général, l’un armé d’une gaffe, l’autre d’une pelle. La gaffe est pareille à la pique d’abordage des frégates et porte d’ailleurs le même nom ; c’est une sorte de croc que le piqueur fiche dans une bande de lard, s’efforçant ensuite de l’empêcher de glisser tandis que le navire tangue et roule et que l’homme à la pelle, debout sur ce lard, le tranche en moellons. Cette pelle est aussi affûtée que possible, l’homme qui s’en sert est pieds nus et la matière sur laquelle il se tient fuit parfois irrésistiblement, comme un traîneau. S’il coupe un doigt de son pied, ou de celui d’un de ses aides, peut-on s’en étonner ? Les doigts de pied sont rares parmi les vétérans du parc au gras.
\chapterclose


\chapteropen
\chapter[{CHAPITRE XCV. La chasuble}]{CHAPITRE XCV \\
La chasuble}\renewcommand{\leftmark}{CHAPITRE XCV \\
La chasuble}


\chaptercont
\noindent Si vous étiez venu à bord du {\itshape Péquod} à un certain stade de l’autopsie du cachalot et si vous vous étiez approché du guindeau, je suis presque sûr que vous auriez examiné avec beaucoup de curiosité l’étrange et mystérieux objet sous le vent allongé que vous y auriez vu près des dalots. Ni la merveilleuse citerne de l’énorme tête, ni le prodige de la mâchoire inférieure déboîtée, ni le miracle de la queue symétrique, rien de tout cela ne vous aurait autant surpris qu’un demi-regard jeté sur ce cône énigmatique, plus long qu’un Kentuckais n’est haut, de près d’un pied de diamètre à la base et d’un noir de jais comme Yoyo, l’idole d’ébène de Queequeg. Et c’est en effet une idole, ou plutôt sa représentation en était une jadis. Une telle idole fut trouvée dans les bois secrets de Maaka, reine de Juda, et son fils, le roi Asa la déposa pour l’avoir adorée, l’abattit et la brûla dans la vallée du Cédron comme abomination ainsi que nous le dit ténébreusement le quinzième chapitre du premier livre des Rois.\par
Voyez le matelot qu’on nomme éminceur ; il s’avance aidé de deux hommes, charge péniblement sur son dos le grandissimus, comme l’appellent les pêcheurs et, les épaules courbées, l’emporte en chancelant tel un grenadier, sur le champ de bataille, un camarade mort. L’étendant sur le gaillard d’avant, il commence à le dépouiller de sa peau de bas en haut comme un chasseur africain ferait d’un boa. Puis il retourne la peau à la façon d’une jambe de pantalon, l’étire au point d’en doubler le diamètre, puis la met à sécher, bien étendue, dans le gréement. Bientôt il l’y reprend, en coupe la longueur de trois pieds à \hspace{1em}son extrémité pointue, pratique à l’autre bout des ouvertures pour les bras, puis s’y glisse tout entier. L’éminceur se tient maintenant devant vous dans les vêtements sacerdotaux de sa vocation. Seule cette investiture, de tradition immémoriale, le protégera tandis qu’il officie.\par
Cet office consiste à émincer le lard pour les chaudières Cette opération se fait sur un curieux chevalet de bois dont une extrémité est assurée contre la lisse et au-dessous duquel une grande baille est destinée à recevoir les morceaux émincés qui y tombent à la vitesse des feuilles du discours d’un orateur enthousiaste. Décemment vêtu de noir, occupant une chaire éminente, absorbé par ses feuillets de Bible, quel candidat à un archevêché, quel gaillard de Pape, cet éminceur ne serait-il pas\footnote{Feuillets de Bible ! feuillets de Bible ! tel est le cri immuable que les seconds jettent à l’éminceur. Ils lui signifient ainsi d’avoir soin couper les tranches aussi fines que possible, la fonte étant de ce fait accélérée, la quantité d’huile considérablement accrue et peut-être même sa qualité améliorée.} !
\chapterclose


\chapteropen
\chapter[{CHAPITRE XCVI. Les fourneaux}]{CHAPITRE XCVI \\
Les fourneaux}\renewcommand{\leftmark}{CHAPITRE XCVI \\
Les fourneaux}


\chaptercont
\noindent Outre ses pirogues hissées à leurs potences, un navire baleinier américain se reconnaît à ses fourneaux. C’est une singularité que cette solide construction de maçonnerie se mêlant, pour compléter le navire, à son chêne et à son chanvre, comme si un four à briques se dressait sur la rase campagne de son pont.\par
Les fourneaux occupent, entre le mât de misaine et le grand mât, la partie la plus spacieuse du pont, ils sont soutenus par des poutres assez fortes pour supporter une construction de briques et de ciment de quelque dix pieds sur huit de largeur sur cinq de hauteur. Celle-ci ne fait pas corps avec le pont, mais elle est solidement amarrée dessus par de lourdes équerres de fer l’enserrant de toutes parts et vissées dans les poutres. Elle est revêtue de bois sur les côtés et le dessus, plat, est entièrement ouvert par un grand panneau d’écoutille incliné et assujetti. Lorsqu’on l’enlève, on découvre les deux immenses chaudières, d’une capacité de plusieurs barils chacune. Elles sont remarquablement entretenues pendant tout le temps où l’on ne s’en sert pas, polies parfois avec de la stéatite jusqu’à ce qu’elles aient l’éclat d’un bol à punch en argent. Pendant les quarts de nuit, quelque vieux matelot railleur se roulera en boule à l’intérieur pour y faire un petit somme, et par-dessus ses rebords de fer, bien des confidences sont échangées par les hommes qui les astiquent, côte à côte, un dans chacune. C’est aussi un endroit propice à la réflexion mathématique la plus ardue. Ce fut dans la chaudière de gauche du {\itshape Péquod}, ma pierre de lard circulant diligemment en rond autour de moi, que je fus pour la première fois frappé indirectement par ce fait remarquable qu’en physique tout corps glissant sur une cycloïde, ma pierre de lard par exemple, tombe de n’importe quel point, d’une même hauteur pendant un intervalle de temps donné.\par
Lorsqu’on enlève la plaque qui protège le devant des fourneaux, on voit la maçonnerie, non revêtue de ce côté-là, percée par les deux bouches des foyers placées directement sous les chaudières et munies de lourdes portes de fer. Un vase rafraîchissant, peu profond, mais s’étendant sous toute la surface de l’ouvrage, empêche la chaleur intense de se communiquer au pont et un tuyau placé vers l’arrière permet de le remplir à mesure que l’eau s’évapore. Les cheminées, s’ouvrent directement dans le mur arrière. Et maintenant, rebroussons chemin un moment.\par
Il était près de neuf heures du soir lorsque les fourneaux du {\itshape Péquod} furent allumés pour la première fois au cours de cette croisière. Stubb avait la direction des opérations.\par
– Paré ? Alors enlevez le panneau, et commencez. Vous, coq, allumez les feux.\par
C’était chose aisée car depuis le départ le charpentier entassait ses copeaux dans le foyer ; il convient de dire que la première fois qu’on allume les fourneaux, lors d’une chasse à la baleine, on est obligé d’entretenir le feu au bois, puis il cesse d’être nécessaire sinon pour enflammer la matière première. Celle-ci est constituée par les beignets ou gratons, résidu croustillant et desséché de la fonte qui garde une teneur considérable en graisse. Tel un martyr pléthorique au bûcher, ou un homme consumé par sa misanthropie, une fois allumée, la baleine fournit son propre combustible, elle brûle par son propre corps. Que n’absorbe-t-elle pas sa propre fumée ! Horrible inhalation qu’il faut coûte que coûte inspirer, dans laquelle on est contraint \hspace{1em}de vivre momentanément, odeur innommable et forcenée pareille à celle que doivent répandre les bûchers funéraires des Indes ! L’odeur qui se dégagera, au jour du Jugement, du côté gauche, ce sera celui des boucs, est à elle seule un argument en faveur de l’enfer.\par
À minuit, le travail battait son plein. Nous avions largué la carcasse, nous avions fait de la toile, le vent fraîchissait et les ténèbres étreignaient l’Océan désert. Mais les flammes du brasier léchaient l’ombre et leurs fourches, jaillissant de temps à autre des cheminées encrassées de suie, illuminaient jusqu’à son faîte le gréement, comme un feu grégeois. Le navire embrasé courait comme impitoyablement délégué à l’accomplissement d’une vengeance. De même les bricks frétés de poix et de soufre de l’intrépide Kanaris, l’Hydriote, sortant, à la nuit, de leurs ports, portant de larges nappes de feu en guise de voiles, fondaient sur les frégates turques et y semaient l’incendie.\par
Le panneau supérieur des fourneaux ayant été enlevé, un grand foyer s’ouvrait à présent devant eux ; les silhouettes démoniaques des harponneurs païens, soutiers à bord des baleiniers, s’y dressaient ; armés de longues fourches, ils jetaient dans les chaudières brûlantes de gros morceaux de lard qui y sifflaient ou bien ils attisaient, au-dessous, les flammes jusqu’à ce que leurs serpents glissent leurs anneaux hors des portes afin de les saisir par les pieds. La fumée s’en allait au loin en rouleaux tristes. L’huile bouillante tanguait avec le navire comme si elle n’attendait que le moment de leur sauter à la figure En face des bouches des fourneaux, de l’autre côté du foyer, se trouvait le guindeau, vrai sofa marin. La bordée de quart s’y attardait lorsqu’elle était inoccupée, les yeux plongés dans la rougeur incandescente du foyer jusqu’à ce qu’ils leur flambassent dans la tête. Les visages fauve des hommes noircis de fumée, marqués de sueur, leurs barbes emmêlées, le contraste qu’y faisait l’éclat barbare de leurs dents, étaient étrangement mis en relief et hautement colorés par les caprices du feu. Tandis qu’ils se racontaient leurs aventures impies, des récits de terreur, sur le ton de l’allégresse, tandis que leurs rires barbares fusaient, fourchus comme les flammes de leurs fourneaux, tandis que, devant eux passaient et repassaient les harponneurs gesticulant avec leurs cuillères pour les pots, tandis que le vent hurlait, que bondissait la mer, que le navire gémissait et piquait, emportant toujours son enfer rouge, de plus en plus loin, au cœur noir de la mer et de la nuit, mâchant avec mépris les os blancs de la baleine, crachant rageusement de tous côtés, alors le fuyant {\itshape Péquod}, frété de sauvages, chargé de feu, brûlant un cadavre et plongeant dans les ténèbres, semblait la matérialisation de l’âme délirante de son capitaine.\par
C’est ainsi qu’il m’apparaissait tandis que, debout à la barre, je menais durant de longues heures silencieuses ce vaisseau-incendie sur la mer. Enveloppé d’ombre, je n’en voyais que mieux la rougeur, la folie et l’horreur peintes sur le visage des autres. La vue incessante de démons, faisant des entrechats, entre la fumée et le feu, engendra en mon âme d’identiques visions dès que j’eus cédé à l’indicible somnolence qui toujours m’envahit au gouvernail de minuit.\par
Mais cette nuit-là, une chose étrange m’advint que je ne pus jamais m’expliquer. M’éveillant en sursaut d’un bref sommeil tout debout, je pris une conscience horrible d’une erreur fatale. La barre en os contre laquelle j’étais appuyé me bourra les côtes, le bourdonnement sourd des voiles en train de faseyer siffla à mes oreilles, je crus avoir les yeux ouverts, je crus, à demi conscient porter mes doigts à mes paupières pour les écarquiller. Mais malgré cela je ne voyais nul compas pour guider ma route bien que, me semblait-il, il ne s’était pas écoulé plus d’une minute depuis que j’avais regardé la rose des vents éclairée par la lampe fixe de l’habitacle. Il me paraissait qu’il n’y avait devant moi qu’une obscurité de jais rendue parfois effrayante par des éclairs rouges. Par-dessus tout j’avais l’impression que, quelle que fût la chose ailée, rapide, sur laquelle je me tenais, elle n’allait point tant vers un havre qu’elle ne fuyait les ports derrière elle. Un égarement m’envahit, une rigidité pareille à celle de la mort. Mes mains serrèrent convulsivement la barre et l’idée folle me vint qu’un maléfice l’avait renversée. Seigneur ! que m’arrive-t-il ? pensais-je. Voilà que dans mon court sommeil je m’étais retourné, face à la poupe, dos à la proue et au compas. Je fis brusquement volte-face, juste à temps pour empêcher le navire d’abattre et sans doute de chavirer. Quelles ne furent pas ma joie et ma reconnaissance d’être libéré de cette hallucination nocturne et de la fatale éventualité d’empanner !\par
Ne regarde pas trop longtemps le visage du feu, ô homme ! Ne rêve jamais quand tu tiens la barre ! Ne tourne pas le dos au compas, accepte le premier avertissement de la barre défaillante, ne crois pas aux mensonges du feu lorsque sa rougeur revêt d’horreur toutes choses. Demain, à la lumière du soleil, les cieux seront clairs, ceux dans lesquels les flammes fourchues allumaient le regard menaçant des démons, le matin les modèlera tout autrement avec plus de douceur. La gloire, l’or et la joie du soleil, seule vraie lampe alors que trahissent toutes les autres !\par
Et pourtant le soleil ne cache pas le Dismal Swamp de Virginie, ni la campagne maudite de Rome, ni l’immense Sahara, ni les millions de milles de déserts et de douleurs du monde. Le soleil ne dissimule pas l’Océan qui est le côté sombre de cette terre dont il couvre les deux tiers, Aussi l’homme mortel qui a en lui plus de joie que de souffrance ne peut pas être vrai, ou il est insincère ou il n’a pas atteint sa plénitude. Il en va de même des livres. Le plus vrai d’entre les hommes fut l’Homme de Douleurs, le plus vrai de tous les livres est celui de Salomon et l’Ecclésiaste est l’acier le plus beau et le mieux trempé de la souffrance. Tout est vanité. Tout ! Ce monde obstiné n’a pas encore conquis la sagesse de Salomon, le non-chrétien. Mais celui qui esquive les hôpitaux et les prisons, celui qui se hâte à travers les cimetières et préfère parler opéras qu’enfer, taxe Cowper, Young, Pascal, Rousseau de pauvres diables de malades et passe sa vie insouciante à jurer par l’éphémère sagesse de Rabelais et par sa gaillardise, celui-là n’est pas digne de s’asseoir sur les pierres tombales et d’ouvrir la terre végétale humide et verte des abîmes de l’admirable Salomon.\par
Mais Salomon lui-même dit : « Celui qui abandonne les chemins de la droiture, sa route conduit chez les trépassés » et cela signifie qu’il sera un mort dans la vie. Dès lors ne t’abandonne pas au feu, de crainte qu’il ne te détourne et ne te prive momentanément de vie, comme il le fit pour moi. Il est une sagesse qui est souffrance ; mais il est une souffrance qui est folie. Un aigle de Catskill peut aussi bien plonger dans les gorges ténébreuses de certaines âmes que devenir invisible lorsqu’il en rejaillit à nouveau dans la lumière de l’espace. Et quand bien même il volerait à jamais dans ces gorges, il n’est pas de gorges qu’à l’altitude, de sorte que cet aigle des sommets volera toujours plus haut que les oiseaux des plaines, si haut qu’ils planent.
\chapterclose


\chapteropen
\chapter[{CHAPITRE XCVII. La lampe}]{CHAPITRE XCVII \\
La lampe}\renewcommand{\leftmark}{CHAPITRE XCVII \\
La lampe}


\chaptercont
\noindent Si quittant les fourneaux du {\itshape Péquod}, vous étiez descendu au gaillard d’avant, où dormait la bordée libre de quart, vous auriez pu un instant vous croire dans la chapelle de quelque saint roi ou conseiller. Ils gisaient là, entre les parois de chêne de leurs caveaux triangulaires, le ciseau du silence sculptant chaque visage, le feu d’une vingtaine de lampes sur leurs yeux clos.\par
À bord des navires marchands, l’huile est pour le matelot plus précieuse que le lait des reines. Se vêtir dans l’obscurité, manger dans l’obscurité et trébucher jusqu’à sa paillasse dans l’obscurité, tel est son lot habituel. Mais le baleinier vit dans la lumière comme il cherche l’aliment de la lumière. Il fait de sa couchette une lampe d’Aladin où s’étendre de sorte que, dans la plus profonde nuit, les fonds du navire sont toujours illuminés.\par
Voyez avec quelle entière liberté le baleinier apporte sa poignée de lampes – souvent rien de plus que de vieilles bouteilles et fioles – au rafraîchisseur de cuivre des fourneaux et les y remplit comme un pichet de bière à la cuve. L’huile qu’il brûle est aussi la plus pure, non traitée, à son état vierge, liquide inconnu des lampes solaires, lunaires ou astrales, à terre. Elle est douce comme le premier beurre de l’herbe d’avril. Il part chasser son huile pour être sûr de sa pureté et de son authenticité, comme le voyageur poursuit dans la prairie le gibier de son souper.
\chapterclose


\chapteropen
\chapter[{CHAPITRE XCVIII. Arrimage et nettoyage}]{CHAPITRE XCVIII \\
Arrimage et nettoyage}\renewcommand{\leftmark}{CHAPITRE XCVIII \\
Arrimage et nettoyage}


\chaptercont
\noindent Nous avons déjà raconté comment le grand léviathan est signalé à distance de la pointe des mâts, comment il est poursuivi sur les landes marines et mis à mort dans les vallées des profondeurs, comment il est remorqué au flanc du navire, puis déjointé, et comment (selon le principe qui accordait au bourreau de jadis les vêtements que portait le condamné au moment de sa mort) son grand pardessus feutré devient propriété de son exécuteur, comment, en temps voulu, il est voué aux chaudières et comment, tels Shadrak, Méshak et Abed Nego, ses os, son huile, son spermaceti traversent indemnes le feu. Il reste maintenant à terminer ces descriptions par un chapitre racontant, chantant si possible, le pittoresque transvasement de l’huile dans les barriques et leur arrimage dans la cale où le léviathan retourne à ses profondeur natales, glissant à nouveau sous la surface de la mer comme autrefois, mais hélas pour ne plus jamais remonter ni souffler.\par
Pendant qu’elle est encore chaude, l’huile, telle un punch brûlant, est mise en fûts de six barils et, pendant que peut-être le navire roule et tangue sur la mer nocturne, les barriques énormes font des tête-à-queue et parfois détalent dangereusement sur le pont glissant comme autant d’éboulements, jusqu’à ce qu’enfin des mains immobilisent leur course, et que sur les cercles retentissent autant de marteaux qu’il s’en trouve, car pour l’instant tout matelot est tonnelier.\par
Enfin, lorsque la dernière pinte est enfûtée, que tout est refroidi, on enlève les panneaux des grandes écoutilles, les entrailles du navire sont béantes et la futaille s’en va à son repos marin. Cela fait, les panneaux sont remis en place, hermétiquement fermés, comme une armoire murée.\par
C’est peut-être l’une des phases les plus remarquables des travaux de la pêche au cachalot. Un jour, une inondation de sang et d’huile ruisselle sur les ponts, sur le sacro-saint gaillard d’arrière d’énormes morceaux de la tête du cachalot sont entassés de façon impie, de grands barils rouillés traînent comme dans une cour de brasserie, la fumée des fourneaux a souillé de suie les pavois, les matelots sont imprégnés de graisse, le navire tout entier est l’image du léviathan lui-même tandis que règne partout un vacarme assourdissant.\par
Mais un jour ou deux plus tard, vous pouvez jeter les yeux autour de vous, tendre l’oreille sur ce même navire et, n’étaient les pirogues suspendues et les fourneaux qui sont la marque d’un baleinier, vous pourriez vous croire sur un tranquille navire marchand dont le commandant serait un homme plus que méticuleux. À l’état brut, l’huile du cachalot possède un pouvoir décapant extraordinaire. C’est pourquoi la blancheur des ponts n’est jamais aussi nette qu’après cette affaire d’huile, comme ils disent De plus, les cendres des déchets de la baleine fournissent une puissante lessive qui a tôt fait d’avoir raison de ce qui, de l’animal, aurait pu rester collé aux flancs du navire. Armés de seaux et de chiffons, les hommes s’activent aux bastingages et leur rendent leur propreté originelle. Ils brossent la suie du gréement, les nombreux ustensiles dont il a été fait usage sont également nettoyés avec soin et rangés. La grande écoutille est récurée et remise sur les fourneaux de manière à dissimuler complètement les chaudières, la futaille disparaît, les apparaux sont allés se lover dans des recoins invisibles. Et lorsque grâce au travail commun, accompli ensemble, de tout l’équipage, ce nettoyage consciencieux est enfin terminé, les hommes \hspace{1em}procèdent à leurs propres ablutions, se changent de la tête aux pieds et réapparaissent sur le pont immaculé, frais et pimpants comme des mariés tout droit sortis de drap de la plus fine Hollande.\par
Par groupes de deux ou de trois, ils arpentent le pont d’un pas joyeux. Ils discutent plaisamment de salons, de sofas, de tapis et de fine percale, ils se proposent d’étendre des nattes sur le pont, de suspendre aux mâts des tentures et n’ont aucune objection à prendre le thé au clair de lune sur la plaza du gaillard d’avant. Ce serait une impudence rare de venir parler, à ces marins odorants, d’huile, d’os et de lard, ils n’ont aucune notion de ce à quoi vous faites discrètement allusion. Allez ! loin, et apportez-nous des serviettes de table !\par
Mais attention, là-haut à la pointe des mâts, trois hommes debout épient ardemment d’autres baleines qui si on les prend, souilleront infailliblement le vieux mobilier de chêne et laisseront bien une petite tache d’huile au moins quelque part. Oui, et maintes fois, après les plus rudes travaux ininterrompus qui ne connaissent point de nuit et se poursuivent sur quatre-vingtseize heures, quand les hommes reviennent d’avoir ramé jusqu’à en avoir les poignets enflés, ils ne remontent sur le pont que pour transporter de lourdes chaînes, virer au pesant guindeau, et dépecer et tailler, oui, et être enfumés dans leur sueur même et brûlés à nouveau par les doubles feux du soleil équatorial et les tropiques des fourneaux, lorsque après tout cela, ils se sont évertués à polir le navire jusqu’à en faire une laiterie immaculée, bien souvent les pauvres diables n’ont pas fini d’agrafer le dernier bouton de leur vêtement propre qu’ils sont saisis par le cri : « La voilà qui souffle ! » Ils s’envolent à la poursuite d’une nouvelle baleine et recommencent à en passer par toutes les mêmes fatigues. Ah ! mes amis, c’est tuer son homme ! Et pourtant c’est la vie. Car à peine avons-nous, pauvres humains, soutiré à la masse imposante de ce monde, après de longs efforts, un tout petit peu de son précieux spermaceti, à peine, avec une épuisante patience, nous sommes-nous lavés de ses souillures, à peine avons-nous appris à vivre dans le pur tabernacle de l’âme, à peine en avons-nous fini que le cri : « La voilà qui souffle ! » fait lever le spectre, et nous voilà en route pour combattre quelque autre monde, et reprendre la vieille routine de la jeune vie une fois de plus.\par
Oh ! métempsychose ! Oh ! Pythagore, toi qui dans la Grèce éclatante, voici deux mille ans, mourus si bon, si sage, si doux, je fis voile avec toi au large du Pérou lors de mon dernier voyage et, si sot que je sois, je t’ai appris, novice, à faire une épissure.
\chapterclose


\chapteropen
\chapter[{CHAPITRE XCIX. Le doublon}]{CHAPITRE XCIX \\
Le doublon}\renewcommand{\leftmark}{CHAPITRE XCIX \\
Le doublon}


\chaptercont
\noindent Nous avons déjà dit comment Achab arpentait le gaillard d’arrière, allant et venant de l’une à l’autre de ses limites : de l’habitacle au grand mât. Mais, à tant de choses importantes à dire, il eût fallu ajouter qu’au cours de ses promenades, alors qu’il était particulièrement absorbé et ombrageux, il s’arrêtait à chacun de ces endroits et fixait étrangement l’objet qu’il avait sous les yeux. Lorsqu’il faisait une pause devant l’habitacle, son regard se rivait à la pointe de l’aiguille du compas, un regard aigu comme un javelot et qui avait l’intensité aiguisée du but qu’il poursuivait et lorsque, suspendant à nouveau sa marche, il se trouvait devant le grand mât, alors son œil se clouait sur la pièce d’or clouée là, et il avait la même expression de décision chevillée en lui, empreinte toutefois d’une nostalgie sauvage sinon désespérée.\par
Un matin, venant à passer devant le doublon, il parut attiré de façon neuve par les motifs étranges et les inscriptions qu’il portait, comme si, pour la première fois, en son délire, il commençait à y découvrir quelque sens caché. Car toutes choses seraient vaines si elles n’étaient chargées de quelque signe et notre monde rond ne serait alors rien de plus qu’un zéro, à vendre à la charretée, comme ils font des collines de Boston, afin de combler les fondrières de la Voie Lactée.\par
Or ce doublon était d’or vierge le plus pur, arraché quelque part au cœur des collines somptueuses du haut desquelles d’ouest en est, à travers des sables d’or, coulent les sources de plus d’un Pactole. Bien que fixé à présent entre des clous de fer rouillés et des pointes de cuivre vert-de-grisées, immaculé, impossible à souiller, il conservait son éclat de Quito. Bien que les mains âpres d’un âpre équipage l’effleurassent à toute heure, que les longues nuits l’enveloppassent d’opaques ténèbres propices au larcin, le soleil levant retrouvait le doublon là où le soleil couchant l’avait laissé. Car il était isolé et sanctifié ainsi en vue d’une fin redoutable et mystérieuse et, si licencieux que fussent les marins dans leurs façons, les uns et les autres le révéraient comme l’emblème évocateur de la Baleine blanche. Ils en parlaient parfois au cours du fatigant quart de nuit, se demandant à qui il reviendrait en fin de compte, et si celui-là vivrait pour le dépenser.\par
Ces nobles pièces de l’Amérique du Sud sont les médailles du soleil et le symbole des tropiques. Palmes, alpacas et volcans, disques solaires et étoiles, écliptiques, cornes d’abondance, riches bannières au vent s’y trouvent gravées avec une généreuse abondance, de sorte que cette fantaisie de la poésie espagnole semble rehausser la valeur et le prestige de l’or précieux.\par
Le doublon du {\itshape Péquod} illustrait, par hasard, cette opulence. Sa carnèle portait les mots : {\itshape Republica del Ecuador. Quito.} Ainsi cette pièce brillante venait d’un pays situé au milieu du monde, sous le grand équateur, et portait son nom, elle avait été frappée à mi-chemin des Andes, sous ce climat égal qui ne connaît point d’automne. Délimités par ces lettres, trois sommets des Andes y figuraient, l’un surmonté d’une flamme, l’autre d’une tour, le troisième d’un coq en train de chanter ; par-dessus, s’incurvait un fragment du zodiaque, chaque signe indiqué par son motif cabalistique, la clef de voûte du soleil entrant dans la ligne équinoxiale à la Balance.\par
Non sans être observé par les autres, Achab s’était arrêté devant cette pièce équatoriale.\par
« Il y a toujours quelque suffisance dans les sommets et dans les tours et dans toutes choses grandes et altières, voyez ces trois pics aussi orgueilleux que Lucifer. La tour inébranlable, c’est Achab, le volcan, c’est Achab, l’oiseau courageux, sans peur, victorieux, c’est aussi Achab, tous sont Achab, et ce disque d’or n’est que l’image d’un monde plus rond encore qui, tel le miroir d’un magicien, tour à tour, à chaque homme ne renvoie que l’image de son moi mystérieux. Grandes douleurs, petits profits à celui qui demande au monde des réponses alors qu’il ne sait ce qu’il est lui-même. Il me semble que ce soleil, devenu monnaie, rougeoie, mais voyez, il entre dans la ligne équinoxiale, dans le signe des tempêtes et, il n’y a que six mois, il sortait du Bélier du précédent équinoxe ! De tempête en tempête ! Qu’il en soit donc ainsi. Enfanté dans la douleur, il est juste que l’homme vive dans la souffrance et meure dans les affres ! Qu’il en soit donc ainsi ! Voilà une rude matière offerte à l’acharnement du malheur. Qu’il en soit donc ainsi. »\par
« Aucun doigt de fée n’a pu serrer l’or mais les griffes du diable ont dû hier y laisser leurs empreintes, se murmura Starbuck appuyé au bastingage. Le vieillard semble lire le terrible message du festin de Balthazar. Je n’ai jamais examiné cette pièce de près. Il s’en va, je vais aller la voir. Une sombre vallée entre trois sommets puissants griffant les cieux. On dirait un pâle symbole terrestre de la Trinité. Ainsi Dieu nous entoure dans cette vallée de la Mort et sur nos ténèbres resplendit le soleil de la Justice, son phare et son espérance. Si nous baissons les yeux, l’obscur vallon ne montre qu’une terre pourrie, mais que nous les levions et le soleil vient à la rencontre de notre regard pour nous réconforter. Pourtant la demeure du soleil n’est point fixe et si nous voulions à minuit lui arracher quelque douce consolation, nous le chercherions en vain ! Cette pièce a la voix de la sagesse, de la douceur, de la vérité mais pour moi celle de la tristesse aussi. Je la quitterai de crainte que la Vérité ne m’ébranle traîtreusement. »\par
« Voilà que le vieux Mogol, soliloquait Stubb près des fourneaux, a été la scruter, et voilà Starbuck qui a fait de même, et tous deux avec des têtes dont je dirais qu’elles ont neuf pieds de long. Et tout cela pour avoir regardé une pièce d’or que je ne contemplerais pas si longuement si je pouvais l’aller dépenser à Negro Hill ou à Corlaer’s Hook. Hum ! à mon humble avis, c’est bizarre. J’ai vu des doublons de la vieille Espagne, doublons du Pérou, doublons du Chili, doublons de Bolivie, doublons de Popayan, et en plus beaucoup de moïdores et de pistoles, de joes, de demi-joes et de quarts de joes. Qu’y aurait-il dès lors dans ce doublon de l’Équateur si séduisant ? Par tous les trésors de Golconde, il faut que j’y regarde de plus près ! Holà, en voilà en vérité des signes et des merveilles ! N’est-ce pas là ce que l’épitomé de Bowditch appelle le zodiaque et que mon almanach là en bas nomme de même. Je vais aller chercher cet almanach, j’ai entendu dire que par les calculs de Daboll on pouvait évoquer les démons, je vais m’essayer à évoquer ces êtres curvicaudes pour soutirer un sens avec l’aide de mon almanach du Massachussetts. Le voici. Voyons un peu. Signes et merveilles, et le soleil y est toujours au centre. Hem, hem, hem les voici, tous bien éveillés : Aries ou le Bélier, Taurus ou le Taureau, Gemini ou les Gémeaux ; eh bien, la roue du soleil au milieu, oui, sur la pièce il est en train de franchir le seuil entre deux des douze salles d’attente qui font cercle. Livre, ici, vous mentez, le fait est que, vous autres, livres, devriez rester à vos places. Vous feriez l’affaire pour nous donner les simples mots et les faits, mais nous intervenons avec nos pensées. C’est ce que ma petite expérience m’a appris, du moins en ce qui concerne l’almanach du Massachusetts, le voyageur Bowditch et l’arithmétique de Daboll. Signes et merveilles, hein ? Dommage qu’il n’y ait rien de merveilleux dans les signes, ni de significatif dans les merveilles ! Il y a une clef quelque part… attendez… chut, silence ! Par Jupiter, je la tiens ! Écoutez-moi bien. Doublon, votre zodiaque, c’est la vie d’un homme écrite en rond, et je vais la lire d’après le livre. Allons, almanach ! Pour commencer, il y a Aries ou le Bélier… chien lubrique, il nous enfante ; puis Taurus ou le Taureau qui nous donne le premier coup ; Gemini ou les Gémeaux, c’està-dire la Vertu et le Vice ; nous essayons d’atteindre la Vertu, quand voici qu’arrive le Cancer-l’Écrevisse qui nous tire en arrière, et voici, sortant de la Vertu, Leo, un lion rugissant ; couché sur le chemin, il donne quelques cuisantes morsures et quelques hargneux coups de patte, nous nous échappons et saluons Virgo, la Vierge ! C’est notre premier amour, nous nous marions et pensons être heureux quand survient Libra ou la Balance… le bonheur pesé et trouvé léger. Et tandis que nous nous attristons là-dessus, Seigneur, quel bond soudain nous faisons tandis que Scorpio, le Scorpion nous pique le derrière ; nous sommes en train de soigner la blessure, quand s’abat une grêle de flèches, c’est Sagittarius ou l’Archer qui s’amuse. Tandis que nous nous débarrassons de ces traits, attention, voici le Bélier, Capricorne ou le bouc, il arrive en courant, tête baissée, et nous piquons du nez à terre, lorsque Aquarius, le porteur d’eau déverse sur nous son déluge et nous noie, et pour finir avec Pisces, ou les Poissons nous dormons. En voilà un prône, écrit haut dans le ciel, et le soleil chaque année le traverse pourtant il en ressort toujours vivant et chaleureux. Là-haut, il traverse allègrement peines et malheurs, de même ici, en bas, fait le joyeux Stubb. Joyeux, c’est le mot. Adieu, Doublon ! Mais halte… voici venir le petit Cabrion, cachons-nous derrière les fourneaux et écoutons ce qu’il a à dire. Le voici devant et il va tout de suite raconter quelque chose. Oui, oui, il commence… »\par
« Je ne vois rien d’autre ici qu’une chose ronde en or, et à celui qui lèvera une certaine baleine, cette chose ronde appartiendra. De sorte qu’on se demande ce qu’ils ont tous à venir la contempler. Elle vaut seize dollars, c’est vrai, et à deux cents le cigare, ça fait neuf cent soixante cigares. Je ne voudrais pas fumer des sales pipes comme Stubb, mais j’aime les cigares, et en voici neuf cent soixante. Et Flask va grimper là pour les espionner.\par
Dirai-je que c’est sage ou absurde ? Si c’est réellement sage, ça a l’air absurde, mais si c’est réellement absurde, ça a un petit air de sagesse. Mais baste, voilà qu’arrive notre vieux Mannois, le cocher de corbillard, du moins ce devait être son métier avant qu’il ait pris la mer. Il fait une auloffée devant le doublon, holà, et contourne le mât, bien sûr, il y a un fer à cheval cloué de ce côté-là, il est déjà revenu… qu’est-ce que cela veut dire ? Chut… il marmonne… une voix de vieux moulin à café usé. Dressons l’oreille, et écoutons ! »\par
« Si la Baleine blanche doit être levée, elle le sera dans un mois et un jour, lorsque le soleil se trouvera dans l’un de ces signes. J’ai étudié les signes et je connais leur sens, une vieille sorcière de Copenhague me les a appris voici quarante ans. Voyons, dans quel signe sera le soleil à ce moment-là ? Dans le signe du fer à cheval car il se trouve exactement à l’opposé de la pièce. Et qu’est-ce que le signe du fer à cheval ? C’est le lion, le lion rugissant et dévorant. Navire, vieux navire ! Ma vieille tête tremble en pensant à toi. »\par
« Et voici une nouvelle interprétation, mais le texte reste le même. Des hommes divers dans un monde pareil, voyez-vous. Cachons-nous, c’est Queequeg, tout tatoué, il a lui-même l’air du Zodiaque. Que dit le cannibale. Aussi sûr que je vis, il fait une étude comparée, il regarde sa cuisse, il croit avoir le soleil dans la cuisse, ou dans le mollet, ou dans les intestins, je présume, comme les vieilles femmes des campagnes quand elles discutent l’astronomie de Surgeon. Et par Jupiter, il a découvert quelque chose sur sa cuisse… j’imagine que c’est le Sagittaire ou l’Archer. Non, il ne sait que penser du doublon, il le prend pour l’ancien bouton de culotte d’un roi. Mais, dissimulons-nous encore, voici ce diable fantôme de Fedallah, la queue enroulée hors de vue comme d’habitude, de l’étoupe dans la pointe de ses escarpins comme d’habitude. Que dit-il, avec cette mine qu’il a ? Ah ! il fait seulement un signe au signe et s’incline. Il y a un soleil sur la pièce… adorateur du feu, comptez dessus. Oh ! il en vient toujours. C’est Pip qui arrive… pauvre garçon ! il aurait mieux valu qu’il meure, ou moi, il me fait presque horreur. Lui aussi, il a épié tous ces interprètes – moi y compris – et voyez, il vient lire avec sa figure d’idiot de l’autre monde. Tenons-nous à l’écart et écoutons-le. Chut !\par
– Je regarde, tu regardes, il regarde, nous regardons, vous regardez, ils regardent.\par
– Sur mon âme, il a étudié la grammaire de Murray ! Pour s’enrichir l’esprit, pauvre gars ! Mais que dit-il…\par
– Je regarde, tu regardes, il regarde, nous regardons, vous regardez, ils regardent.\par
– Eh bien, il l’apprend par cœur… chut encore…\par
– Je regarde, tu regardes, il regarde, nous regardons, vous regardez, ils regardent.\par
– C’est pour le moins drôle.\par
– Et moi, vous et lui, et nous, vous et eux, tous des chauves-souris, et moi je suis un corbeau, surtout quand je suis perché dans ce pin qui est là. Croa, croa, croa, croa ! N’est-ce pas que je suis un corbeau ? Et où est l’épouvantail ? Il est là, deux os plantés dans une paire de vieux pantalons, et deux encore fichés dans les manches d’une vieille vareuse.\par
– Je me demande si je suis visé ? C’est flatteur ! pauvre type ! Pour un peu j’irais me pendre. De toute façon, pour le moment, je vais fuir le voisinage de Pip. Le reste je peux le supporter, car ils sont sains d’esprit, mais, lui, il a trop de sel de folie pour mon bon sens. Aussi je vais le laisser à ses radotages.\par
– Ce doublon-là, c’est le nombril du navire, et ils brûlent tous de le dévisser. Mais dévissez-vous le nombril, qu’en adviendra-t-il ? D’autre part, s’il reste en place, c’est laid aussi car, lorsque quelque chose est cloué au mât, c’est un signe qu’une affaire devient désespérée. Ah ! ah ! vieil Achab ! la Baleine blanche, elle te clouera ! Ceci est un pin. Mon père, au vieux pays de Tolland, a coupé un pin une fois et il a trouvé un anneau d’argent dans le tronc, l’anneau de mariage de quelque vieux Noir. Comment y était-il venu ? Ils se poseront la même question au jour de la résurrection, quand ils repêcheront ce vieux mât, et y trouveront un doublon, et que les huîtres lui auront fait une écorce rugueuse. Oh ! l’or, précieux, précieux or ! La verte avare le mettra dans son trésor ! Chut ! chut ! Dieu parcourt les mondes à la cueillette des mûres. Cop, coq, fais-nous cuire ! Jenny ! hé, hé, hé, Jenny, faites cuire votre galette sur la houe.
\chapterclose


\chapteropen
\chapter[{CHAPITRE C. Bras et jambe}]{CHAPITRE C \\
Bras et jambe}\renewcommand{\leftmark}{CHAPITRE C \\
Bras et jambe}


\chaptercont
\noindent Le {\itshape Péquod} de Nantucket rencontre le {\itshape Samuel-Enderby} de Londres.\par
– Ohé du navire ! As-tu vu la Baleine blanche ?\par
Une fois de plus, Achab hélait un navire avec ces mots. Le vaisseau battait pavillon britannique et nous croisait sur l’arrière. Son porte-voix à la bouche, le vieil homme était debout dans sa baleinière suspendue de sorte que sa jambe d’ivoire était bien visible pour le capitaine étranger négligemment penché à la proue de sa propre pirogue.\par
Ce dernier était un bel homme approchant de la soixantaine, tanné, solidement bâti, aimable ; il était vêtu d’un ample caban dont le drap bleu de pilote festonnait autour de lui et \hspace{1em}dont une manche vide flottait derrière lui comme la manche brodée d’un surcot de hussard.\par
– As-tu vu la Baleine blanche ?\par
– Voyez-vous cela ? Et, le sortant des plis qui le dissimulaient, il leva un bras blanc en os de cachalot se terminant par une tête de bois pareille à un maillet.\par
– Armez ma pirogue ! s’écria impétueusement Achab en mâtant les avirons qui se trouvaient près de lui. Paré à mettre à la mer !\par
En moins d’une minute, sans qu’il eût à quitter son esquif, lui et ses hommes furent déposés sur l’eau et se trouvèrent bientôt aux flancs de l’étranger. Mais là survint une curieuse difficulté. Tout à son impatience Achab avait oublié que, depuis la perte de sa jambe, il n’avait pas posé le pied sur un autre navire que le sien propre et qu’il y bénéficiait d’une invention ingénieuse et pratique particulière au {\itshape Péquod}, mais qu’il ne trouverait sur commande sur aucun autre navire. Or, il n’est aisé pour personne – hormis pour ceux qui, comme les baleiniers, l’ont fait à chaque heure de leur vie – de grimper au flanc d’un navire depuis une baleinière en pleine mer car les vagues puissantes la hissent jusqu’aux pavois, pour la faire retomber aussitôt à michemin de la contre-quille. De sorte que, privé d’une jambe, le navire étranger étant parfaitement dépourvu de système favorable, Achab se trouvait déchu, réduit à l’état de terrien emprunté et jetait un regard impuissant vers les hauteurs mouvantes qu’il ne pouvait espérer atteindre.\par
Peut-être a-t-il été déjà fait allusion au fait que, lorsque Achab se trouvait dans une difficulté provenant indirectement de son malheureux accident, il était presque invariablement irrité ou exaspéré. En l’occurrence, il était exacerbé par la vue de deux officiers du navire étranger qui, penchés sur la lisse, près de l’échelle perpendiculaire, balançaient à son intention une paire de tire-veilles ouvragées avec goût, car de prime abord il ne leur semblait pas être venu à l’idée qu’un unijambiste puisse être suffisamment infirme pour ne pouvoir utiliser ces rampes marines. Ce malaise ne dura qu’un instant car le capitaine étranger, ayant jugé d’un coup d’œil de quoi il en retournait, s’écria : « Je vois ! je vois ! assez hissé par là ! En vitesse, les gars, jetez le palan de dépeçage.\par
La chance voulut qu’ils aient eu une baleine amarrée deux ou trois jours auparavant et que les palans fussent encore en place et le grand croc à lard, propre et sec, encore suspendu à leur extrémité. Celui-ci étant rapidement descendu jusqu’à Achab, il comprit sur-le-champ, glissa sa cuisse unique dans la courbure du crochet (comme il se fût assis dans l’aile d’une ancre ou dans la fourche d’un pommier), donna l’ordre de le hisser, se cramponna fermement et aida à alléger son propre poids en tirant, main sur main, sur la manœuvre courante de la poulie. Il fut bientôt soigneusement rendu sur le pont et déposé avec douceur sur le chapeau de cabestan. Son bras d’ivoire franchement tendu en signe de bienvenue, l’autre capitaine s’avança et Achab levant sa jambe d’ivoire croisa le fer si l’on peut dire. On eût dit deux espadons, et Achab s’exclama à sa façon de morse : « Oui, oui, chaleureusement ! Serrons-nous les os ! Un bras et une jambe ! Un bras qui ne saurait plier jamais, voyezvous et une jambe qui jamais ne peut courir. Où as-tu vu la Baleine blanche ? Et il y a combien de temps ?\par
– La Baleine blanche, répondit l’Anglais, pointant vers l’est l’ivoire de son bras, en le balayant d’un long regard triste comme avec un télescope : C’est là que l’ai vue sur la ligne, la saison passée.\par
– Et elle a emporté ce bras, n’est-ce pas ? demanda Achab, se laissant glisser au bas du cabestan en prenant pour le faire appui sur l’épaule de l’Anglais.\par
– Oui, ou du moins elle fut la cause de sa perte. Et cette jambe aussi ?\par
– Racontez-moi toute l’histoire, dit Achab, comment cela s’est-il passé ?\par
– C’était ma première croisière sur la ligne, commença l’Anglais, j’ignorais tout de la Baleine blanche à cette époque-là. Un jour nous mîmes à la mer pour une gamme de quatre ou cinq cachalots et ma pirogue se piqua sur l’un d’eux. C’était un vrai cheval de cirque, il tournait comme un moulin tant et si bien que mon équipage dut s’asseoir sur le plat-bord opposé. Alors, du fond de la mer, une grande baleine fit brèche et s’élança. Elle avait un front d’une blancheur de lait, une bosse et n’était que rides et pattes d’oie.\par
– C’était elle, c’était elle ! s’écria Achab haletant.\par
– Elle avait des harpons fichés près de sa nageoire de tribord.\par
– Oui, oui… c’étaient les miens… mes fers, jubila Achab, mais poursuivez !\par
– Alors, laissez-moi la chance de le faire, dit l’Anglais avec bonne humeur. Eh bien, cette arrière-arrière-grand-mère, avec sa tête blanche et sa bosse, se rua dans la gamme, y soulevant une tempête d’écume, et se mit à mordre furieusement ma ligne de harpon.\par
– Oui, je vois ! elle voulait la couper pour libérer le poisson amarré… un de ses vieux trucs… je la connais.\par
– Ce qu’il en était exactement, je n’en sais rien, continua le capitaine manchot, mais en mordant la ligne, celle-ci se prit dans ses dents, sans que nous nous en rendions compte sur le moment, de sorte que lorsque nous halâmes, nous fîmes un bond sur sa bosse, tandis que l’autre baleine fuyait au vent, la queue battante. Prenant conscience de la situation, voyant à quelle grande et noble baleine nous avions affaire, la plus noble et la plus grande que j’ai jamais vue, sir, de ma vie, je décidai d’en faire une proie malgré la fureur écumante qui semblait la tenailler. Pensant que cette ligne qui la retenait par hasard se libérerait ou que sa dent lâcherait avec elle (car j’ai un équipage de tous les diables pour embraquer une ligne), voyant tout cela, dis-je, je sautai dans la pirogue de mon second, M. Mountopp ici présent (à propos, capitaine, Mountopp ; Mountopp, le capitaine) comme je disais, je sautai dans la pirogue de Mountopp qui, voyez-vous, se trouvait alors plat-bord contre plat-bord avec la mienne, et saisissant le premier harpon, j’en donnai à cette arrière-arrière-grand-mère. Mais, Seigneur, voyez-vous, sir, par mon âme, l’instant suivant, en cinq secs, j’étais aussi aveugle qu’une chauve-souris… des deux yeux… du brouillard partout, réduit à l’impuissance par une écume noire… la queue de la baleine dressée droite au milieu, perpendiculaire sur l’eau, comme un clocher de marbre. Il ne servait à rien de nager à culer, mais tandis qu’en plein midi, sous un soleil aveuglant, resplendissant de tous ses feux, je tâtonnais à la recherche du second fer pour m’en débarrasser, la queue s’abattit comme une tour lors d’un tremblement de terre à Lima, coupa ma baleinière en deux, réduisant chaque partie en éclats, et la queue la première, la bosse blanche recula à travers les épaves comme si elles n’étaient que copeaux. Nous lui portâmes tous des coups. Pour échapper à ce fléau en mouvement, je saisis le manche de mon harpon planté dans sa chair et je m’y cramponnai un moment, comme un rémora. Mais une vague me repoussa et, au même instant, la baleine se lança en avant et sonda comme l’éclair ; les barves de ce maudit second fer emporté tout contre moi se plantèrent ici (il serra sa main juste un peu en dessous de son épaule), oui, il m’attrapa juste là, dis-je, et m’entraîna dans les flammes de l’enfer, pensais-je alors ; lorsque… lorsque tout soudain, Dieu soit loué, les pointes me labourèrent les chairs tout le long du bras mais se dégagèrent au niveau du poignet et je revins en surface – et ce monsieur vous racontera le reste (à propos, capitaine, le Dr Bunger, chirurgien du bord, Bunger, ami, le capitaine). Et maintenant, Bunger, ami, racontez votre partie de l’histoire.\par
Rien ne révélait chez l’homme de l’art, ainsi familièrement interpellé, debout près d’eux pendant tout ce temps, le rang élevé qu’il occupait à bord. Il avait un visage tout rond mais austère, portait une chemise de laine d’un bleu fané, des pantalons rapiécés, et, jusque-là, il avait partagé son intérêt entre un épissoir qu’il tenait dans une main et une boîte de pilules qu’il avait dans l’autre, tout en jetant de temps à autre un regard critique sur les membres d’ivoire des deux capitaines. Mais quand son supérieur le présenta à Achab, il s’inclina courtoisement entreprit aussitôt le récit demandé :\par
– C’était une blessure affreusement mauvaise, commençat-il, et sur mon conseil, le capitaine Boomer mit le cap de notre vieux {\itshape Sammy…}\par
– {\itshape Samuel-Enderby} est le nom de mon navire, interrompit le capitaine manchot à l’intention d’Achab, continuez, mon garçon…\par
– Mit le cap de notre vieux {\itshape Sammy} sur le nord-ouest, pour échapper à la température brûlante de la ligne. Mais cela ne servait à rien… je fis ce que je pus, le veillai la nuit, très strict quant à son régime…\par
– Oh ! très strict ! s’accorda à reconnaître le patient luimême, puis changeant soudain de ton : il buvait de grogs au rhum avec moi toutes les nuits jusqu’à n’y plus voir pour faire les pansements, m’expédiant au lit, à moitié ivre, vers trois heures du matin. Oh ! par les étoiles, il veillait avec moi, en vérité et il était très strict en ce qui concernait mon régime. Oh ! un grand veilleur, certes, et un diététicien rigoureux, tel est le Dr Bunger. (Bunger coquin, riez donc ! pourquoi ne riez-vous pas ? Vous savez que vous êtes une jolie canaille.) Mais, allez de l’avant, mon garçon, j’aurais préféré être tué par vous que sauvé par n’importe qui d’autre.\par
– Vous avez dû remarquer, honorable sir, dit l’imperturbable Bunger, avec son air de sainte nitouche, et faisant un petit signe de tête à Achab, que mon capitaine a parfois l’esprit farceur, il nous débite souvent des plaisanteries de ce genre. Mais je peux bien vous le dire – en passant, comme disent les Français – moi-même je ne bois jamais… moi, Jack Bunger, ancien membre du révérend clergé, je suis un homme absolument, intégralement abstinent, je ne bois jamais…\par
– D’eau ! s’écria le capitaine, il n’en touche pas, elle lui donne des crises, l’eau fraîche lui donne l’hydrophobie, mais continuez… continuez avec cette histoire de bras.\par
– Oui, il vaudrait mieux, répartit froidement le chirurgien. Lors de l’interruption facétieuse du capitaine Boomer, j’étais en train de faire remarquer que malgré mes tentatives désespérées, l’état de la blessure empirait tous les jours. À la vérité, sir, c’était une plaie béante comme jamais chirurgien n’en vit de pire, plus de deux pieds et quelques pouces de long. Je l’ai mesurée avec la sonde. Bref, elle devenait noire, sachant où cela menait je l’amputai. Mais je n’ai rien à voir avec ce bras d’ivoire, c’est une chose contraire à toutes les règles, dit-il en le désignant avec son épissoir ; ça, c’est le travail du capitaine et non le mien ; il l’a fait exécuter par le charpentier, il y a fait mettre ce maillet au bout dans l’intention de faire éclater le crâne à quelqu’un, j’imagine, comme il a essayé faire du mien une fois. Il a parfois des colères diaboliques Voyez-vous ce creux, sir, et enlevant son chapeau et repoussant ses cheveux il exhiba une dépression en forme de bol sur son crâne qui ne portait pas la moindre trace de cicatrice, ni le moindre indice qu’elle eût été occasionnée par une blessure. Eh bien ! le capitaine pourrait vous dire à quoi c’est dû, il le sait.\par
– Non, je n’en sais rien, répondit ce dernier, mais sa mère, elle, le savait car il est né avec ce creux. Oh ! solennel pendard que vous êtes, vous, vous Bunger ! A-t-on jamais vu un pareil Bunger sous le soleil ? Bunger, quand vous mourrez, on devrait vous mettre au vinaigre, coquin, afin que vous vous conserviez pour les temps futurs, canaille.\par
– Que devint la Baleine blanche ? dit Achab qui avait jusqu’alors assisté avec impatience à cette comédie jouée par les deux Anglais.\par
– Oh ! dit le capitaine manchot. Oh ! oui ! En bien, après qu’elle eut sondé, nous ne la revîmes pas de quelque temps ; en fait, comme je vous le disais, je ne savais pas alors qu’elle était la baleine qui m’avait joué ce tour, jusqu’au moment où je revins plus tard sur la ligne et où nous entendîmes parler de Moby Dick – comme certains l’appellent – et où je sus que c’était bien elle.\par
– L’avez-vous rencontrée à nouveau ?\par
– Deux fois.\par
– Mais vous n’avez pas pu la harponner ?\par
– Je n’ai pas eu envie d’essayer ! Un membre, n’est-ce pas suffisant ? Que deviendrais-je sans cet autre bras ? Et je croirais volontiers que Moby Dick avale plutôt qu’il ne mord.\par
– En ce cas, interrompit Bunger, appâtez-la avec votre bras gauche pour récupérer le droit. Savez-vous, messieurs, ajouta-til en s’inclinant devant l’un et l’autre capitaines à tour de rôle, avec beaucoup de gravité et de componction, savez-vous, messieurs, que les organes digestif du cachalot sont si mystérieusement conçus par la Providence qu’il lui est impossible de digérer fût-ce un bras d’homme ? Et il le sait. Aussi ce que vous imputez à la méchanceté de la Baleine blanche n’est que maladresse de sa part. Elle n’a jamais eu l’intention d’avaler un seul membre mais ne pense qu’à susciter la terreur avec des simulacres. Elle ressemble parfois à ce vieux jongleur qui fut autrefois un de mes patients à Ceylan ; il feignait d’avaler des couteaux, en avala un tout de bon, un jour, et le conserva quelque douze mois ou davantage. Lorsque je lui eus donné un émétique, il rendit des clous de tapissier, voyez-vous. Il n’avait pas pu digérer ce couteau, ni l’assimiler. Oui, capitaine Boomer, si vous ne tardez pas et que vous soyez disposé à donner en gage votre bras afin que l’autre puisse être enterré décemment alors ce bras vous reviendra, il faut seulement donner sous peu une chance à la baleine, c’est tout.\par
– Non merci, Bunger, je lui laisse volontiers le bras qu’elle a, car je n’ai pas le choix, et je ne la connaissais pas encore, mais je ne lui offre pas l’autre. Plus de Baleine blanche pour moi, j’ai mis à la mer une fois pour elle, ça me suffit. Je sais qu’il y aurait grande gloire à la tuer et qu’elle représente une pleine cale de spermaceti mais écoutez-moi bien, mieux vaut la laisser tranquille. N’est-ce pas votre avis, capitaine ? dit-il en jetant un coup d’œil sur la jambe d’ivoire.\par
– Il vaudrait mieux en effet, mais cela n’empêche pas qu’elle sera chassée malgré cela. Ce qu’il vaut mieux laisser tranquille, cette chose maudite, n’est pas toujours le moins attirant. Elle est un véritable aimant ! Quand l’avez-vous vue pour la dernière fois ? Dans quelle direction ?\par
– Dieu me bénisse et que soit maudit le démon infect, s’écria Bunger, marchant penché autour d’Achab en reniflant étrangement comme un chien. Le sang de cet homme – qu’on apporte le thermomètre ! – est à son point d’ébullition ! Son pouls fait trembler ces planches ! Sir ! Il s’approcha du bras d’Achab armé d’un bistouri qu’il tira de sa poche.\par
– Arrière ! rugit Achab, en le lançant contre le bastingage. Armez la pirogue ! Dans quelle direction ?\par
– Bonté divine ! s’écria le capitaine à qui cette question s’adressait. Que vous arrive-t-il ? Elle se dirigeait vers l’est, je crois. Et il chuchota à Fedallah : Votre capitaine est-il fou ?\par
Mais Fedallah, un doigt sur les lèvres, enjamba la lisse pour prendre l’aviron de queue et Achab, amenant à lui le croc à lard, ordonna aux matelots de se tenir prêts à le descendre.\par
L’instant d’après il était debout à la proue de sa baleinière et les hommes des Philippines bondirent sur leurs avirons. En vain, le capitaine anglais le héla-t-il. Montrant le dos au navire étranger, tournant vers le sien un visage de silex, Achab se tint droit jusqu’à ce qu’il eût atteint le flanc du {\itshape Péquod.}
\chapterclose


\chapteropen
\chapter[{CHAPITRE CI. La carafe}]{CHAPITRE CI \\
La carafe}\renewcommand{\leftmark}{CHAPITRE CI \\
La carafe}


\chaptercont
\noindent Avant que le navire anglais disparaisse à nos yeux, il venait de Londres, disons-le ici, et portait le nom de feu Samuel Enderby, négociant en cette ville, fondateur de la maison d’industrie baleinière Enderby et fils, maison qui, selon ma modeste opinion de baleinier, n’a rien à envier aux maisons des Tudors et des Bourbons réunies quant à son réel intérêt historique. Ma nombreuse documentation sur la pêche ne me dit pas clairement depuis combien de temps cette maison existait avant l’année de notre Seigneur 1775 mais, cette année-là (1775), elle arma pour la pêche au cachalot les premiers navires anglais qui, depuis lors, chassèrent régulièrement, cela bien que, depuis une quarantaine d’années (dès 1726), nos vaillants Coffin, Maceys de Nantucket et du Vineyard, aient déjà eu des flottes nombreuses chassant le cachalot dans l’Atlantique nord et sud, mais pas ailleurs. Il faut faire remarquer ici que les Nantuckais furent les premiers à harponner le cachalot avec un acier civilisé, et que pendant un demi-siècle ils furent les seuls à le faire.\par
En 1778 un beau navire, l’{\itshape Amelia}, armé à cet effet aux seuls frais des énergiques Enderby, doubla le cap Horn et fut le premier à mettre à la mer une baleinière, de quelque nature qu’elle fût, dans la grande mer du Sud. Le voyage fut adroitement mené et fructueux, et, le navire étant rentré au bercail la cale pleine de précieux spermaceti, l’exemple de l’{\itshape Amelia} eut tôt fait d’être suivi par d’autres navires, tant anglais qu’américains, ainsi la pêche au cachalot se trouva portée dans l’océan Pacifique. Mais, non contente de cette bonne action, l’infatigable maison se \hspace{1em}démena à nouveau : Samuel et tous ses fils – seule leur mère sait combien ils étaient – persuadèrent le gouvernement anglais d’envoyer sous leurs auspices, et je pense partiellement à leurs frais, la corvette de guerre {\itshape Rattler}, en expédition baleinière de découverte dans les mers du Sud. Commandé par un capitaine de vaisseau, voyage du {\itshape Rattler} fit du bruit, et rendit quelques services, on voit mal lesquels. Mais ce n’est pas tout. En 1819, cette même maison arma un navire baleinier à elle pour une croisière d’exploration dans les eaux lointaines du Japon. Ce navire, la {\itshape Sirène}, le bien nommé – fit une noble croisière expérimentale et c’est depuis lors que furent connus les grands parages de pêche japonais. Lors de ce fameux voyage, la {\itshape Sirène} fut commandée par un certain capitaine Coffin, de Nantucket.\par
Honneur donc aux Enderby, dont la maison existe encore aujourd’hui, si je ne me trompe quoique, sans doute, son fondateur Samuel doive avoir depuis longtemps largué son amarre pour la grande mer du Sud de l’autre monde.\par
Le navire portant son nom était digne de cet honneur ; très rapide, c’était à tous égards un noble bâtiment. Je montai à son bord une fois, à minuit, au large de la Patagonie et j’y bus un bon flip dans son gaillard d’avant. Nous eûmes une belle gamme, chaque homme, à son bord était un brave cœur. Qu’ils aient une vie courte et une mort joyeuse ! Et cette gamme à laquelle je participai ; longtemps, bien longtemps après que le vieil Achab y eut posé son pied d’ivoire, me rappelle l’hospitalité généreuse, solide, teutonique de ce navire. Que mon pasteur m’oublie et que le diable se souvienne de moi si jamais je la perds de vue. Ai-je dit que nous y bûmes du flip ? Oui, du flip, au taux de dix gallons à l’heure et quand vint le grain (car il y a des grains au large de la Patagonie) et quand l’équipage, et les visiteurs aussi, furent appelés pour prendre des ris à la hune, nous étions tous si chargés dans les hauts, que nous dûmes grimper en nous aidant les uns les autres dans les boulines et que, sans nous en rendre compte, nous ferlâmes les pans de nos vareuses dans les voiles, de sorte que nous restâmes suspendus là, dûment attachés dans les hurlements de la tempête, avertissement exemplaire pour tous les mathurins saouls. Les mâts toutefois tinrent bon et, petit à petit, nous descendîmes à quatre pattes, si dégrisés que nous dûmes faire une nouvelle tournée de flip, bien que l’écume sauvage et salée giclant par l’écoutille de descente du gaillard d’avant l’eût par trop dilué et assaisonné pour mon goût.\par
Le bœuf était bon – coriace mais consistant. Ils disaient que c’était du bœuf de bœuf mais d’autres affirmaient que c’était du bœuf de dromadaire, je n’ai jamais très bien éclairci la question. Ils avaient aussi des boulettes, petites mais substantielles, parfaitement rondes et tout à fait indestructibles. J’eus l’impression qu’on pouvait, une fois avalées, les sentir rouler en soi et qu’en se penchant un peu trop en avant elles risquaient de ressortir comme des boules de billard. Le pain – mais cela on ne pouvait rien – était antiscorbutique, bref le pain était la seule nourriture fraîche qu’ils eussent. Mais il ne faisait pas très clair dans le gaillard d’avant et on pouvait toujours se réfugier dans un coin sombre pour le manger Mais en gros, pris de la quille à la pomme de mât, considération faite de la taille des marmites du coq et de ses vivantes marmites pansues, de l’avant à l’arrière, dis-je, le {\itshape Samuel Enderby} était bon vivant, bonne chère abondante, bon flip corsé, des gars d’élite, tous, du talon de leurs bottes à leur ruban de chapeau.\par
Mais vous vous demandez comment il se fait que le {\itshape Samuel Enderby} et d’autres navires baleiniers anglais de ma connaissance – pas tous, cependant – aient dispensé pareille hospitalité, et qu’aient pu y circuler ainsi le bœuf, le pain, la carafe et la plaisanterie, et que les hommes ne fussent pas tôt lassés de manger, de boire et de rire. Je vais vous le dire. Cette abondance est matière à recherche historique. Et je n’ai jamais été regardant en fait de recherche historique quand elle m’a paru nécessaire.\par
Les Hollandais, les Zélandais et les Danois précédèrent les Anglais dans la pêche à la baleine, on leur doit de nombreux termes de pêche encore en usage, et qui plus est, leurs vieilles coutumes de grasse mesure pour le boire et le manger. Car, en général, les navires marchands anglais lésinent pour leur équipage, ce qui n’est pas le cas des navires baleiniers, de sorte que cette opulence n’est pas, aux yeux des Anglais, normale et naturelle mais fortuite et particulière, et doit avoir, dès lors, une origine que nous relevons déjà ici et que nous éclaircirons plus loin.\par
Au cours de mes recherches léviathanesques, je suis tombé sur un ancien ouvrage hollandais dont les relents de baleine me disaient qu’il y était question de baleiniers. Je conclus d’après son titre de « {\itshape Dan Coopman »} que ce devaient être les mémoires inestimables de quelque tonnelier d’Amsterdam, puisque tout navire baleinier a son tonnelier de bord. Je fus confirmé dans mon opinion en voyant que l’auteur s’appelait Fitz Swackhammer. Mais mon ami, le Dr Snodhead, un très savant homme, professeur de bas allemand et de haut allemand à l’université de Saint-Nicolas et de Saint-Pott, à qui j’avais remis l’ouvrage aux fins de traduction, lui faisant cadeau pour peine d’une boîte de bougies de cire de spermaceti, ce Dr Snodhead, à peine eut-il aperçu le livre, m’assura que « Dan Coopman » ne voulait pas dire « Le Tonnelier » mais « Le Négociant ». Bref, ce docte et vieux livre traitait du commerce de la Hollande et abordait, entre autres sujets, des questions très intéressantes sur la pêche \hspace{1em}à \hspace{1em}la \hspace{1em}baleine. \hspace{1em}Au \hspace{1em}chapitre \hspace{1em}intitulé \hspace{1em}« Smeer » ou « Graisse », j’ai trouvé une liste détaillée des vivres embarqués par cent quatre-vingts navires baleiniers hollandais, je relève la suivante, traduite par le Dr Snodhead :\par
400.000 livres de bœuf salé\par
60.000 livres de porc salé de Frise\par
150.000 livres de morue sèche\par
550.000 livres de biscuit\par
72.000 livres de pain frais\par
2.800 livres de fréquins de beurre\par
20.000 livres de fromage de Texel et de Leyde\par
144.000 livres de fromage (probablement d’une qualité inférieure)\par
550 ankers de genièvre 10.800 muids de bière\par
Si la lecture de la plupart des tables statistiques est desséchante, ce n’est pas le cas dans ce livre où le lecteur est noyé dans la grande futaille, les barils, les quarts de gallons, les canons de bon genièvre, et la bonne chère.\par
À l’époque, j’ai consacré trois jours à la studieuse digestion de toute cette bière, de ce bœuf, de ce pain, au cours de laquelle bien des pensées profondes me vinrent, susceptibles d’une application transcendantale et platonicienne. En outre, j’ai consulté des tables, dressées par moi, des quantités probables de morue sèche etc. consommées par chaque harponneur de l’ancienne pêcherie au Groenland et au Spitzberg. Tout d’abord, les quantités de beurre et de fromage de Texel et de Leyde consommées paraissent surprenantes, je les impute, toutefois, à leur naturel lardeux, à un appétit de graisse augmenté par leur métier et plus spécialement au fait qu’ils poursuivent leur gibier dans ces glaciales mers polaires, et jusque sur les côtes du pays des Esquimaux dont les aborigènes trinquent fraternellement avec des rasades de thran.\par
La quantité de bière est également énorme : 10.800 barils. Or, comme, en ces régions arctiques, la pêche ne peut se poursuivre que lors de leur court été, la croisière complète d’un de ces navires baleiniers hollandais n’excédait guère trois mois, aller et retour, et si nous comptons un équipage de trente hommes pour chacun de leurs 180 navires, nous obtenons 5 450 marins hollandais en tout. J’en déduis que cela donne exactement deux barils de bière par homme pour une ration de douze semaines, sans compter la jolie part qui leur revient sur les 550 ankers de genièvre. Que ces harponneurs à la bière et au genièvre, pochardés comme on les imagine, eussent été les hommes qu’il fallait debout à l’avant d’une pirogue et lançant un dard sûr dans les rapides baleines, voilà qui paraît improbable. Pourtant ils lançaient leurs fers et prenaient du gibier. Mais il faut se souvenir que cela se passait dans le grand Nord où la bière convient parfaitement à l’organisme. Sous l’équateur, dans notre pêche du Sud, la bière endormirait le harponneur à la tête du mât et le trouverait pompette dans sa pirogue, d’où suivraient de lourdes pertes pour Nantucket et New Bedford.\par
Mais en voilà assez. Nous en avons dit suffisamment pour montrer que les baleiniers hollandais d’il y a deux ou trois siècles étaient de bons vivants et que les baleiniers anglais ne méprisaient pas un si excellent exemple car, disent-ils, si vous naviguez avec une cale vide d’huile et ne pouvez rien retirer du monde de mieux, prenez-lui au moins un bon repas. Cela déleste la carafe.
\chapterclose


\chapteropen
\chapter[{CHAPITRE CII. Un berceau de verdure aux îles des Arsacides}]{CHAPITRE CII \\
Un berceau de verdure aux îles des Arsacides}\renewcommand{\leftmark}{CHAPITRE CII \\
Un berceau de verdure aux îles des Arsacides}


\chaptercont
\noindent Jusqu’ici, en décrivant le cachalot, je me suis avant tout attardé à ses merveilles extérieures ou bien j’ai traité séparément de quelques-unes de ses structures internes. Mais pour avoir de lui une vue d’ensemble et une compréhension plus complète, il me faut maintenant le déboutonner plus avant, délacer son pourpoint, ouvrir les boucles de ses jarretières, détacher les agrafes et les crochets des jointures de ses os les plus secrets et vous le livrer dans son principe fondamental, c’est-à-dire son squelette.\par
Mais comment, Ismaël ? Comment se fait-il que vous, simple canotier, ayez la prétention de savoir quoi que ce soit du monde intérieur de la baleine ? L’érudit Stubb, perché sur le cabestan, vous aurait-il fait des cours sur l’anatomie des cétacés ? Vous aurait-il viré une côte au guindeau pour ses démonstrations ? Explique-toi, Ismaël. Pouvez-vous disposer sur le pont un cachalot adulte pour étudier, comme un cuisinier met un rôti de porc sur un plat ? Sûrement pas. Jusqu’ici, Ismaël, vous vous êtes montré un témoin authentique, mais prenez garde à présent de ne pas vous octroyer le privilège du seul Jonas, celui de discourir de poutres, de solives, de chevrons, de faîtage, de lambourdes, de chevillages, composant la charpente du léviathan, ainsi que des tonneaux de graisse des laiteries, des beurreries et des fromageries de ses entrailles.\par
J’avoue que, depuis Jonas, peu de baleiniers ont pénétré plus avant que l’épiderme d’un cachalot adulte, pourtant j’ai eu la chance bénie d’en disséquer un en miniature. À bord d’un navire auquel j’appartenais, un bébé cachalot fut hissé en entier sur le pont pour son estomac dont on fait des fourreaux pour les barbelures des harpons et pour les fers de lances. Pensez-vous que j’aie laissé passer l’occasion de me servir de ma hachette d’embarcation et de mon couteau de poche pour briser le sceau et lire le contenu de cet enfant ?\par
En ce qui concerne ma connaissance des os du léviathan parvenu à son gigantesque développement d’adulte, j’en suis redevable à feu mon royal ami Tranquo, roi de Tranque, l’une des Arsacides. En effet, me trouvant à Tranque, il y a des années, alors que j’appartenais au navire marchand {\itshape Dey-d’Alger}, je fus convié à passer une partie des vacances arsacidiennes chez le seigneur de Tranque, en sa villa de palmiers à Pupella, dans un vallon au bord de la mer, non loin de ce que nos marins appellent Bambbuville, la capitale.\par
Parmi nombre de belles qualités, mon royal ami Tranquo nourrissait un pieux amour pour tout objet rare artistique ou singulier, et il avait réuni à Pupella toutes les œuvres précieuses dues à l’imagination de ses gens, principalement des bois sculptés aux dessins merveilleux, des coquillages ciselés, des lances incrustées, des pagaies coûteuses, des canoës aromatiques, tout ceci voisinant avec les merveilles naturelles, tribut que les vagues détentrices de trésors avaient rejeté à la grève.\par
La principale d’entre elles était un grand cachalot échoué, après une tempête qui avait sévi durant un laps de temps inhabituel, et qui avait été trouvé mort la tête contre un cocotier dont la plumeuse aigrette semblait son jet verdoyant. Lorsque son corps immense fut dépouillé de toutes les épaisseurs qui l’enveloppaient et que ses os furent devenus d’une sécheresse crayeuse sous le soleil, l’on transporta délicatement son squelette dans le vallon de Pupella, où l’abrite à présent un vaste temple de palmes seigneuriales.\par
Aux côtés, on suspendit des trophées, l’histoire des Arsacides fut gravée sur les vertèbres, en hiéroglyphes étranges. Dans le crâne, les prêtres entretinrent une flamme perpétuelle et odorante, de sorte que la tête mystique laissait fuser à nouveau son souffle de vapeur tandis que, suspendue à un arbuste, la terrifiante mâchoire inférieure se balançait au-dessus des dévots comme l’épée qui effrayait tellement Damoclès.\par
C’était une vision étonnante. La forêt était verte comme les mousses de l’Icy Glen, les arbres sentaient monter la sève vivante dans leurs fûts altiers, à leur pied le métier à tisser de la terre besogneuse était tendu d’un tapis somptueux dont les vrilles de la vigne formaient la trame et les vivantes fleurs les ramages. Tous les arbres alourdis, les buissons, les fougères, les herbes et l’air messager menaient une activité incessante. À travers le galon des feuilles, la navette ailée du soleil tissait la verdure inlassable. Oh ! tisserand affairé, tisserand invisible, arrête !… un mot… où s’en va l’ouvrage ? Quel palais ornera-t-il ? Pourquoi ce labeur sans fin ? Parle, tisserand ! Que ta main se repose ! Je ne veux échanger qu’un seul mot avec toi ! Non… vole la navette… et les images s’échappent du métier, le courant du tapis ruisselle pour jamais au loin. Le tisserand-dieu tisse, le bruit de son travail l’assourdit et la voix humaine ne parvient pas jusqu’à lui et nous, qui contemplons le métier, nous voici à notre tour assourdis par son bourdonnement et c’est seulement lorsque nous nous en échappons que nous entendons les mille voix qui nous parlent au travers. Car il en est ainsi même dans les usines des hommes. Le vol des fuseaux couvre les mots, ces mêmes mots que l’on entend clairement depuis l’extérieur par les fenêtres ouvertes. C’est ainsi que furent surprises des vilenies. Ah mortel ! prends garde dès lors, car dans le vacarme du grand métier à tisser du monde, on peut surprendre au loin tes plus secrètes pensées.\par
Ainsi dans ce métier à tisser de la forêt arsacidienne, ce métier vert, trépidant de vie, gigantesque désœuvré, était étendu le blanc squelette vénéré et, tandis que la trame et la chaîne s’entremêlent en bruissant autour de lui, le puissant oisif semble être l’astucieux tisserand, tout tapissé de vignes, chaque mois lui apportant des verdures plus fraîches et plus vertes, à lui qui n’est que squelette pourtant La Vie enveloppant la Mort, la Mort y glissant la Vie, le dieu inexorable épousant la jeune Vie qui lui engendre des gloires aux têtes bouclées.\par
Or, quand je rendis visite, avec le royal Tranquo a cet étonnant cachalot et que je vis le crâne devenu autel et la fumée montant par où avait passé son souffle, je m’émerveillai de ce que le roi considérât une chapelle comme une curiosité. Il rit. Je m’émerveillai davantage encore de ce que les prêtres puissent jurer que son jet de fumée était authentique. Je passai et repassai devant le squelette, repoussai les vignes, me faufilai entre les côtes et, avec une pelote de fil arsacidien, je déambulai longuement dans ses méandres, dans l’ombre de ses colonnades et de ses mandrins. Mais bientôt ma pelote fut épuisée, je suivis le fil et je me retrouvai à l’entrée. Je n’avais rien vu de vivant à l’intérieur, rien que des os.\par
Après m’être taillé une baguette verte comme étalon de mesure, je replongeai à l’intérieur du squelette. De la fente en forme de flèche ouverte sur le crâne, les prêtres me surprirent en train de prendre la hauteur de la dernière côte. « Qu’est-ce que cela ? crièrent-ils. Tu oses mesurer notre dieu ! Cela nous incombe. »\par
– Bien, prêtres ! alors, combien lui donnez-vous de longueur ?\par
Là-dessus, une contestation sauvage surgit entre eux au sujet de pieds et de pouces ; ils s’entrebrisèrent le crâne avec leurs mesures de bois, le grand crâne en renvoyait l’écho et, saisissant cette heureuse opportunité, je terminai mes propres mesures.\par
Je me propose de vous les communiquer. Mais tout d’abord, il faut dire qu’en ce domaine je ne suis pas libre de donner des mesures fantaisistes selon mon bon plaisir car vous pourrez, pour vérifier mon exactitude vous référer à des autorités en matière de squelettes. Il y a, m’a-t-on dit, un musée du léviathan en Angleterre, à Hull, port baleinier de ce pays, où se trouvent de beaux échantillons de dos en rasoir et d’autres cétacés. J’ai également entendu dire que le Musée de Manchester, dans le New Hampshire, détient ce que ses propriétaires appellent : « le seul spécimen parfait d’une baleine du Groenland ou de Rivière, des États-Unis. » En outre, à Burton Constable, quelque part dans le Yorkshire, en Angleterre, un certain sir Clifford Constable possède le squelette d’un cachalot, mais d’une taille moyenne et qui n’est en rien comparable en dimensions à celui de mon ami le roi Tranquo.\par
Dans l’un et l’autre cas, les cétacés échoués auxquels appartenaient ces squelettes furent revendiqués par leurs propriétaires pour des motifs similaires. Le roi Tranquo s’en empara parce qu’il en avait envie et sir Clifford parce qu’il était seigneur de ces lieux. La baleine de sir Clifford avait été entièrement articulée de sorte que, telle immense commode, vous pouviez ouvrir et fermer tous ses tiroirs, déployer ses côtes comme un gigantesque éventail et vous balancer toute la journée sur sa mâchoire inférieure. Il est question de poser des serrures sur ses trappes et volets, de sorte qu’un valet en fera les honneurs aux visiteurs, un trousseau de clefs au côté. Sir Clifford pense demander deux pence pour jeter un coup d’œil dans la galerie des échos surnaturels de la colonne vertébrale, trois pence pour entendre l’écho dans la cavité du cervelet, et six pence pour la vue incomparable qu’on a de son front.\par
Les dimensions du squelette que je vais maintenant donner sont copiées textuellement d’après mon bras droit sur lequel je les avais fait tatouer, car mes vagabondages effrénés de cette époque ne m’assuraient aucun autre moyen sérieux de conserver ces précieuses statistiques. Mais, vu l’espace restreint et désireux de laisser en blanc les autres parties de mon corps pour un poème que j’étais alors en train de composer – sur les endroits encore non tatoués qui pouvaient me rester – je ne me mis pas en peine des pouces en plus ou en moins, aussi bien les pouces ne devraient-ils pas entrer en ligne de compte dans une mensuration convenable de la baleine.
\chapterclose


\chapteropen
\chapter[{CHAPITRE CIII. Mesures du squelette du cachalot}]{CHAPITRE CIII \\
\textbf{Mesures du squelette du cachalot}}\renewcommand{\leftmark}{CHAPITRE CIII \\
\textbf{Mesures du squelette du cachalot}}


\chaptercont
\noindent Tout d’abord, je souhaite vous faire un simple exposé sur la masse du léviathan vivant, dont nous allons brièvement aborder le squelette, car ce n’est pas inutile.\par
Selon un calcul minutieux que j’ai fait, partiellement fondé sur les estimations du capitaine Scoresby qui accorde un poids de soixante-dix tonnes à une baleine du Groenland d’une taille maximum de soixante pieds de long, selon mes calculs, dis-je, un cachalot des plus grands, de quatre-vingt-cinq à quatrevingt-dix pieds de long et d’un peu moins de quarante pieds de diamètre là où il est le plus épais, pèsera au moins quatre-vingtdix tonnes, de sorte qu’en comptant qu’une tonne représente le poids de treize hommes, le sien dépasserait largement celui de l’entière population d’un village de onze cent habitants.\par
Ne pensez-vous pas dès lors que pour s’imaginer une quelconque capacité de mouvement dans une masse pareille, un terrien doit lui prêter un cerveau gros comme un couple de bœufs sous le joug ?\par
Je vous ai déjà parlé d’une manière ou d’une autre de son crâne, de son évent, de sa mâchoire, de ses dents, de son front, de ses nageoires et des autres diverses parties de son anatomie, je me bornerai maintenant à relever ce qu’il y a de plus intéressant dans son ossature. Mais comme le crâne colossal fait la majeure partie de son squelette, qu’il en est de loin l’élément le plus complexe et que nous n’y reviendrons pas en ce chapitre, ne manquez pas de le garder soit en mémoire, soit sous le bras faute de quoi vous n’auriez pas une idée juste de la structure générale dont nous allons parler.\par
Le squelette du cachalot de Tranque mesurait soixantedouze pieds, de sorte que bien en chair et en vie, il devait en faire quatre-vingt-dix, car le squelette perd environ un cinquième de la longueur du corps vif. Sur ces soixante-douze pieds, le crâne et la mâchoire occupent quelque vingt pieds ; restent cinquante pieds d’épine dorsale. Sur le tiers environ de celle-ci venait s’attacher le coffre rond des côtes qui enfermait jadis ses organes vitaux. Ce poitrail immense aux côtes d’ivoire, avec sa longue colonne vertébrale formant une droite ininterrompue me rappelle la carène d’un grand navire récemment mis en chantier au moment où une vingtaine seulement de ses membrures nues sont posées et où la quille n’est encore qu’une longue poutre isolée.\par
Il y avait dix côtes de chaque côté ; la première, en commençant à partir du cou, avait près de six pieds de long ; les deuxième, troisième et quatrième étaient d’une longueur croissante ; la cinquième située au milieu atteignait la longueur maximum de huit pieds et quelques pouces ; les autres décroissaient depuis là jusqu’à la dixième et dernière qui ne fait que cinq pieds et quelques pouces. Leur épaisseur paraît proportionnelle à leur longueur. Les côtes centrales sont les plus arquées. Aux Arsacides, elles sont parfois utilisées comme arches de pont pour enjamber de petites rivières.\par
Considérant ces côtes, je fus une fois de plus frappé d’un fait, plus d’une fois mentionné en ce livre : le squelette du cachalot ne donne en aucune mesure le gabarit de son corps. La plus grande des côtes de Tranque, celle du milieu, se situe sur l’animal vivant à l’endroit où il est le plus épais, ayant seize pieds au moins ; or, la côte correspondante n’a guère plus de huit pieds et ne donne donc qu’à moitié l’idée de la taille du cachalot à cet endroit précis. D’autre part, là où je ne vois plus qu’une épine nue, elle portait un manteau fait de tonnes de chair, de muscles, de sang et d’entrailles. Qui plus est, là où se trouvait l’ample queue, ne restent que quelques os disjoints et, en lieu et place des lourdes et majestueuses calmes, il n’y a plus que le néant.\par
Comme il est vain et sot dès lors, pensai-je, de la part d’un homme craintif et qui n’a pas voyagé, de chercher à réaliser ce qu’est l’étonnant cachalot d’après ce squelette amoindri par la mort, étendu dans ce bois paisible. Non ! C’est seulement au cœur des dangers les plus aigus, au sein des tourbillons de ses palmes furieuses, sur l’infini de la mer profonde qu’on peut découvrir la baleine dans sa plénitude et dans sa vivante vérité.\par
Et l’épine dorsale ? La meilleure façon de s’en faire une idée juste, c’est d’empiler les vertèbres les unes sur les autres à l’aide d’une grue. Laborieuse entreprise. Mais cela fait, on croit voir la colonne de Pompée.\par
Il y a quarante et quelques vertèbres en tout qui sont soudées entre elles. Elles sont plutôt posées comme les blocs noueux d’un clocher gothique, formant les solides assises d’une lourde construction. La plus grande est large d’un peu moins de trois pieds, épaisse de plus de quatre ; la plus petite, là où l’épine dorsale s’amenuise à la naissance de la queue, n’a que deux pouces d’épaisseur et offre quelque ressemblance avec une boule blanche de billard. Je me suis laissé dire qu’il y en avait de plus petites encore, mais elles furent perdues par des cannibales polissons, les enfants du prêtre, qui les avaient volées pour jouer aux billes. À quoi on reconnaît que l’épine dorsale du plus grand des êtres vivants diminue jusqu’à devenir simple jouet entre des mains d’enfants.
\chapterclose


\chapteropen
\chapter[{CHAPITRE CIV. La baleine fossile}]{CHAPITRE CIV \\
La baleine fossile}\renewcommand{\leftmark}{CHAPITRE CIV \\
La baleine fossile}


\chaptercont
\noindent Vu sa masse imposante, la baleine est un sujet rêvé pour exagérer, et, d’une façon générale, discourir et s’étendre. Le voudriez-vous que vous ne la pourriez réduire. À tout seigneur, tout honneur, il ne faudrait en traiter que dans un in-folio impérial. Pour ne point reparler des dimensions qu’elle offre de l’évent à la queue, ni de son tour de taille, nous évoquerons seulement les circonvolutions gigantesques de ses intestins pareils aux glènes de cordages et de haussières dans la caverne du fauxpont d’un vaisseau de ligne.\par
Puisque j’ai entrepris de manier ce léviathan, il m’incombe de me montrer à la hauteur de ma tâche, de ne pas négliger la plus minuscule cellule de son sang et de raconter jusqu’au moindre repli de ses entrailles. Ayant déjà parlé de son habitat et de son anatomie, il me reste à l’exalter des points de vue archéologique, paléontologique et antédiluvien. Appliqués à toute autre créature que le léviathan – une fourmi ou une puce – pareils termes imposants pourraient à bon droit être considérés comme grandiloquents. Avec le léviathan pour sujet, ce n’est pas le cas. Pour accomplir cette prouesse, je suis trop heureux de chanceler sous les mots les plus pesants du dictionnaire. Il faut que vous le sachiez, chaque fois que j’ai eu besoin d’y recourir pour ces dissertations, je me suis invariablement servi d’une énorme édition in-quarto de Johnson, achetée expressément à cet effet, parce que l’énormité de ce célèbre grammairien en faisait le dictionnaire idéal d’un auteur, comme moi, traitant des baleines.\par
On entend souvent dire de certains auteurs qu’ils font mousser leur sujet et qu’ils le gonflent. Qu’en est-il alors de moi qui écris sur le léviathan ? Malgré moi, mon écriture s’enfle en caractères d’affiches. Qu’on me donne une plume de condor et le cratère du Vésuve pour l’y tremper ! Amis, retenez mes bras ! car le seul fait d’écrire mes pensées sur le léviathan m’accable de fatigue et me fait défaillir dès que je songe à l’envergure de mon étude, comme s’il fallait y faire entrer toutes les sciences, toutes les générations de baleines, d’hommes, de mastodontes passés, présents et à venir, de tous les panoramas des empires terrestres, à travers l’univers entier et ses banlieues aussi. Un thème si vaste et si généreux est exaltant ! On se dilate à sa dimension. Pour faire un livre puissant il convient de choisir un sujet puissant. On ne pourra jamais écrire une œuvre grande ni durable sur la puce, si nombreux que soient ceux qui s’y sont essayés.\par
Avant d’aborder le sujet des baleines fossiles, je présente mes lettres de créance de géologue, en spécifiant qu’au cours de ma vie mouvementée j’ai été tailleur de pierre, terrassier et grand fouilleur de fossés, de canaux et de puits, de caves à vin, de celliers et de citernes de toute espèce. Je désire aussi, en guise de préliminaire, rappeler au lecteur que, tandis que dans les couches géologiques les plus anciennes on trouve les fossiles de monstres actuellement disparus pour la plupart, ceux que l’on découvre dans les formations tertiaires semblent établir un lien, ou en tout cas former un chaînon, entre les créatures gigantesques et celles dont la lointaine postérité est, dit-on, entrée dans l’Arche. Toutes les baleines fossiles appartiennent à l’époque tertiaire, qui a précédé immédiatement la formation de la couche terrestre supérieure. Bien qu’aucune d’entre elles ne corresponde exactement à une espèce connue actuellement, elles offrent une parenté suffisante dans leurs caractéristiques générales pour prendre place parmi les cétacés fossiles.\par
Au cours de ces trente dernières années, on a trouvé des fragments de baleines fossiles, datant d’avant l’apparition de l’homme, des morceaux d’os et de squelettes, au pied des Alpes, en Lombardie, en France, en Angleterre, en Écosse, dans les États de la Louisiane, du Mississipi et de l’Alabama. L’un des plus curieux de ces restes est un fragment de crâne, mis à jour, en 1779, dans une cave de la rue Dauphine à Paris, une petite rue débouchant presque devant le Palais des Tuileries ; des os ont été également découverts lors du creusement des fossés de la citadelle d’Anvers au temps de Napoléon. Selon Cuvier, ces fossiles appartiennent à une espèce de léviathans tout à fait inconnue.\par
Mais le plus merveilleux vestige de cétacé est de loin le squelette immense et presque complet d’un monstre disparu et découvert en 1842 sur la plantation du juge Creagh, en Alabama. Les esclaves naïfs du voisinage, frappés d’une crainte respectueuse, le prirent pour les os de l’un des anges déchus. Les savants d’Alabama déclarèrent que c’était un grand saurien et le dénommèrent Basilosaurus. Mais des échantillons d’ossements ayant été envoyés à Owen, l’anatomiste anglais, il s’avéra que ce prétendu reptile était une baleine, encore que d’une espèce disparue. Cela illustre une fois de plus ce que je disais du squelette de la baleine incapable de fournir aucune idée de l’animal vivant. Il s’ensuivit qu’Owen rebaptisa le monstre Zeuglodon et, dans le rapport qu’il fit à la Société de géologie de Londres, déclara en substance que c’était là une des créatures les plus extraordinaires que les transformations terrestres aient détruites.\par
Quand je me trouve au milieu des restes de ces puissants léviathans, crânes, défenses, mâchoires, côtes et vertèbres offrant tous des ressemblances partielles avec les monstres marins actuels, et en même temps avec les léviathans antédiluviens disparus, leurs aînés d’un âge incalculable, je suis emporté moimême par un déluge jusqu’en ces époques merveilleuses où le temps n’avait pas encore commencé qui commença avec l’homme. Alors le morne chaos de Saturne me submerge et j’entrevois le sombre frisson des éternités polaires, au temps où des forteresses de glace serraient durement dans leur étau les tropiques d’aujourd’hui et où, sur les 25 000 milles de circonférence du globe, il n’y avait pas une largeur de main de terre habitable. Le monde entier appartenait alors à la baleine, souveraine de la création, elle laissait son sillage sur ce qui est maintenant les crêtes des Andes et de l’Himalaya. Qui a un arbre généalogique comparable à celui du léviathan ? Le harpon d’Achab a versé un sang plus ancien que celui des pharaons. Mathusalem fait figure d’écolier à côté de lui. Je cherche Sem des yeux. Je suis frappé d’horreur devant cette existence prémosaïque, dont on ne peut fixer le commencement et devant les épouvantes indicibles de la baleine qui, ayant été avant tous les temps, doit survivre lorsque seront révolus les âges de l’humanité.\par
Le léviathan a laissé sa trace non seulement sur les clichés que la nature a imprimés dans le calcaire et la marne, mais encore sur les tablettes égyptiennes auxquelles leur ancienneté semble donner le droit au titre de fossiles et qui portent la marque incontestable de sa nageoire. Il y a quelque cinquante ans, on découvrait dans une chambre du grand temple de Dendérah un zodiaque sculpté et peint sur une voûte de granit, où se trouvaient représentés d’innombrables centaures, griffons et dauphins, semblables aux figures grotesques qui ornent les mappemondes actuelles. Le vieux léviathan nageait au milieu d’eux comme par le passé, il nageait sur cette planisphère des siècles avant que fût bercé Salomon.\par
Une autre preuve étrange de l’ancienneté de la baleine en sa réalité postdiluvienne ne doit pas être passée sous silence, elle nous est fournie par le vénérable Jean Léon, le vieux voyageur de Barbarie.\par
« Non loin de la mer, ils ont un temple dont les poutres et les chevrons sont faits d’os de baleines, car souvent des baleines d’une taille monstrueuse sont rejetées mortes à la côte Les gens simples croient que Dieu a investi ce temple du pouvoir secret de frapper toute baleine qui passe devant lui d’une mort instantanée. Mais la vérité est que, de part et d’autre de ce temple, des récifs avancent à deux milles dans la mer, blessant les baleines imprudentes. Ils y conservent une côte de baleine d’une longueur incroyable qu’ils tiennent pour miraculeuse. Posée au sol, sa courbure dressée vers le haut, elle forme une arche d’une hauteur telle qu’un homme à dos de chameau ne l’atteint pas. » Au dire de Jean Léon, cette côte était là depuis un siècle. Leurs historiens affirment qu’un prophète annonciateur de Mahomet venait de ce temple et d’autres n’hésitent pas à assurer que le prophète Jonas fut rejeté par la baleine au pied de cet édifice.\par
En ce temple africain, lecteur, je vous laisse et, si vous êtes Nantuckais et baleinier, vous vous y recueillerez en silence.
\chapterclose


\chapteropen
\chapter[{CHAPITRE CV. La taille de la baleine diminue-t-elle ? Disparaîtra-t-elle ?}]{CHAPITRE CV \\
La taille de la baleine diminue-t-elle ? \\
\textbf{{\itshape Disparaîtra-t-elle} ?}}\renewcommand{\leftmark}{CHAPITRE CV \\
La taille de la baleine diminue-t-elle ? \\
\textbf{{\itshape Disparaîtra-t-elle} ?}}


\chaptercont
\noindent Puisque le léviathan vient vers nous des sources de l’éternité, on est en droit de se demander si, au long de tant de générations, il n’a pas perdu de la taille originelle de ses ancêtres.\par
La recherche nous apprend que non seulement les baleines actuelles sont plus grandes que celles dont les restes ont été trouvés dans les couches tertiaires, période géologique antérieure à l’homme, mais que ces dernières ont une taille supérieure à celles des formations plus anciennes.\par
L’une des plus grandes qui aient été mises au jour jusqu’à maintenant est celle de l’Alabama dont il a été question au chapitre précédent et dont le squelette avait un peu moins de soixante-dix pieds de longueur, alors que, nous l’avons vu, nous obtenons soixante-douze pieds avec une baleine actuelle. Des baleiniers autorisés m’ont dit avoir pris des cachalots de cent pieds de long.\par
Mais ne se pourrait-il que, même si les baleines actuelles ont des dimensions plus importantes que celles des époques géologiques, elles aient dégénéré depuis le temps d’Adam ?\par
Si nous en croyons des messieurs tels que Pline et les anciens naturalistes, nous ne pourrons que conclure en ce sens. Car Pline nous parle de baleines vivantes couvrant des acres, et Aldrovande d’autres cétacés qui mesuraient huit cents pieds de long. Des baleines-corderies et des baleines-tunnel-sous-la- Tamise ! Et même au temps de Banks et de Solander, les naturalistes de Cook, nous trouvons un Danois, membre de l’Académie des sciences, pour donner cent vingt yards de long à certaines baleines d’Islande, de l’espèce reydan-sîskur, à ventre plissé et cela fait trois cent soixante pieds. Lacépède, le naturaliste français, à la page 3 de son étude approfondie sur les cétacés, accorde cent mètres, c’est-à-dire trois cent vingt-huit pieds à la baleine franche. Et la publication de cet ouvrage ne date que de 1825.\par
Mais un baleinier croira-t-il ces histoires ? Non. La baleine du temps de Pline n’est pas plus grande que celle d’aujourd’hui, et si jamais je vais là où est Pline, moi, plus baleinier qu’il ne le fut, je me ferai fort de le lui dire, car je ne saurais comprendre que des momies égyptiennes ensevelies des milliers d’années avant la naissance de Pline ne mesurent pas plus dans leurs sarcophages qu’un Nantuckais actuel pieds nus, et que le bétail et autres animaux sculptés sur les plus anciennes tablettes d’Égypte et de Ninive, à en juger par les proportions du dessin, prouvent clairement que le bétail de race, primé et engraissé à l’étable de Smithfield non seulement égale mais surpasse en taille celui des pharaons. Je ne puis admettre dès lors que, seule entre tous les animaux, la baleine ait dégénéré.\par
Reste une autre question souvent débattue entre les plus profonds esprits de Nantucket. Étant donné la quasiomniscience des guetteurs en vigie aux têtes de mâts des navires baleiniers, alors qu’ils passent à présent même par le détroit de Behring et fouillent les tiroirs les plus secrets et les placards les plus lointains du monde, étant donné les milliers de lances et de harpons jetés au large de tous les continents, on se demande si le léviathan survivra longuement à une chasse aussi répandue, aussi impitoyable et destructrice, ou bien s’il sera finalement exterminé et si l’on verra la dernière baleine, comme on verra le dernier\hspace{1em}homme\hspace{1em}fumer\hspace{1em}sa\hspace{1em}dernière\hspace{1em}pipe\hspace{1em}et\hspace{1em}s’évanouir\hspace{1em}avec l’ultime bouffée.\par
Si l’on fait un parallèle entre les troupes bossues des baleines et les troupeaux bossus des bisons qui, il n’y a pas quarante ans, hantaient par dizaines de milliers les prairies de l’Illinois et du Missouri et secouaient leurs dures crinières sur un front d’orage en contemplant des lieux où sont nées depuis de populeuses capitales et où le courtier vous vend aimablement la terre à un dollar le pouce, on tient un argument de poids pour prouver que la baleine ne saurait échapper à une prompte destruction.\par
Mais il faut voir le problème sous tous ses angles. Bien qu’il y ait si peu de temps – la valeur d’une courte vie d’homme – la population des bisons en Illinois ait dépassé la population humaine actuelle de Londres, et bien qu’il n’en reste plus ni une corne ni un sabot, et bien que cette extermination soit due à la lance des chasseurs, la nature si différente de la chasse à la baleine interdit absolument une fin aussi indigne au léviathan. Quarante hommes chassant le cachalot pendant quarante-huit mois trouvent qu’ils ont fait une croisière fructueuse et rendent grâce à Dieu lorsqu’ils ramènent à leur port l’huile de quarante poissons. Tandis qu’au temps des chasseurs canadiens et indiens de jadis et des trappeurs de l’Ouest, cet Ouest lointain (dont la gloire se lève encore), désertique et inviolé, un même nombre d’hommes chaussés de mocassins, pendant un même nombre de mois, à cheval au lieu d’être à voiles, auraient tué non pas quarante mais quarante mille bisons au moins. Le fait pourrait être, le cas échéant, appuyé par les statistiques. Tout bien considéré, l’extermination progressive du cachalot n’est pas mieux démontrée par le fait qu’auparavant, disons au cours de la deuxième moitié du siècle précédent, on rencontrait plus fréquemment qu’à présent des gammes et que dès lors les voyages étaient à la fois plus courts et plus rémunérateurs. En effet, nous l’avons vu, les baleines, par souci de sécurité, forment d’immenses caravanes, de sorte qu’une grande partie des solitaires, des couples, des gammes, et des écoles de jadis font corps à présent avec des légions qu’on rencontre plus rarement et que séparent des espaces plus vastes. C’est tout. Il serait également faux de conclure que les baleines dites à fanons sont en voie de disparition parce qu’elles ne fréquentent plus certains parages où elles abondaient auparavant. Elles sont simplement chassées de promontoires en caps, et si leurs souffles n’animent plus certaines côtes, soyez sûrs qu’une autre rive plus écartée s’étonne de ce spectacle insolite.\par
En outre, les léviathans à fanons dont il vient d’être question possèdent des forteresses imprenables qui, à vues humaines, le resteront à jamais. De même que, lors de l’invasion de leurs vallées, les Suisses givrés se réfugièrent dans leurs montagnes, les baleines chassées des savanes et des clairières des mers centrales recourent à leurs citadelles polaires et, sondant sous les barrières et banquises ultimes, remontent parmi les champs et les îles de glace flottante et défient toute poursuite de l’homme dans le cercle enchanté d’un éternel décembre.\par
Mais comme l’on harponne un seul cachalot pour cinquante de ces baleines à fanons, quelques philosophes de gaillard d’avant en ont déduit que cette hécatombe avait déjà sérieusement décimé leurs bataillons. Bien que, depuis quelque temps, 13 000 baleines au moins aient été tuées annuellement sur la côte nord-ouest par les seuls Américains, ces considérations ne constituent pas une preuve.\par
Il est naturel de se montrer sceptique quant au foisonnement des plus immenses créatures qui soient, mais que répondrons-nous alors à Harto, l’historien de Goa, lorsqu’il nous raconte que le roi du Siam tua 4 000 éléphants au cours d’une même chasse ? sinon que les éléphants sont aussi nombreux en ces régions que les troupeaux domestiques dans les pays tempérés. Si ces éléphants ont été chassés pendant des \hspace{1em} milliers d’années par Sémiramis, Porus, Hannibal et tous les souverains d’Orient qui vinrent après lui, et qu’ils restent fort nombreux, il n’y a pas de raison de douter que la grande baleine survivra à toute chasse étant donné qu’elle a un pâturage très exactement deux fois aussi vaste que toute l’Asie, les deux Amériques, l’Europe et l’Afrique, la Nouvelle-Hollande et toutes les îles de la terre.\par
En outre, si nous songeons à la longévité présumée des baleines, qui atteindraient l’âge d’un siècle ou davantage, il s’en trouverait alors, n’importe quand, plusieurs générations adultes, distinctes et contemporaines. Nous acquerrons une idée de leur nombre en imaginant tous les cimetières, toutes les nécropoles, tous les caveaux de famille de la création rendant à la vie les corps de tous les hommes, femmes et enfants qui vivaient il y a soixante-quinze ans et en ajoutant cette foule sans nombre à la population actuelle du monde.\par
Toutes ces raisons font que nous tenons la baleine pour immortelle en tant qu’espèce, bien que périssable en tant qu’individu. Elle hantait les mers avant l’apparition des continents, elle nagea au-dessus de l’emplacement où se trouvent maintenant les Tuileries, le château de Windsor et le Kremlin. Lors du déluge de Noé, elle méprisa son arche et le monde viendrait-il à être à nouveau submergé, comme les Pays-Bas ont été inondés pour exterminer leurs rats, que l’immortelle baleine survivrait, dressée sur la suprême arête du flot équatorial défiant le ciel de son souffle écumeux.
\chapterclose


\chapteropen
\chapter[{CHAPITRE CVI. La jambe d’Achab}]{CHAPITRE CVI \\
La jambe d’Achab}\renewcommand{\leftmark}{CHAPITRE CVI \\
La jambe d’Achab}


\chaptercont
\noindent La précipitation avec laquelle le capitaine Achab avait quitté le {\itshape Samuel Enderby} de Londres n’avait pas été sans porter atteinte à sa personne. Il avait atterri avec une telle énergie sur un banc de sa pirogue que sa jambe d’ivoire s’était à demi fendue. Et lorsqu’il s’était retrouvé sur le pont de son navire, il s’était tourné avec tant de véhémence dans son trou de tarière afin de donner un ordre urgent au timonier (toujours pour sa distraction), que cet ivoire déjà fêlé fut tordu, sans se briser toutefois, et en conservant l’apparence de la solidité. Mais Achab ne s’y fia plus tout à fait.\par
Il n’y a pas lieu de s’étonner qu’Achab, malgré sa folle témérité, accordât parfois une attention jalouse à cet os mort qui le portait à demi. Peu de temps avant que le {\itshape Péquod} ne quittât Nantucket, on l’avait trouvé une nuit, gisant sans connaissance ; à la suite de quelque accident ignoré, inconcevable et inexplicable selon toute apparence, sa jambe d’ivoire violemment rompue lui avait transpercé l’aine comme un épieu et l’atroce blessure ne guérit qu’à grand-peine.\par
Il ne manqua pas sur le moment, dans sa démence, de considérer l’angoisse douloureuse du présent comme l’enchaînement d’un malheur plus ancien, et il comprit trop clairement que, semblables au plus venimeux serpent des marais perpétuant aussi irrévocablement sa race que le plus doux chanteur des bocages, la joie et le malheur engendrent leurs pareils. Oui, et non point à parts égales, pensa Achab, car l’ascendance et la descendance de la Douleur sont plus nombreuses que celles de la Joie. Cela sans parler d’un certain enseignement sacré voulant que les plaisirs naturels n’enfantent pas pour l’autre monde mais qu’ils soient suivis au contraire par la morne stérilité du désespoir de l’enfer, tandis que de coupables souffrances perpétuent pour l’éternité une race douloureuse par-delà la tombe, ce qui ne semble pas équitable si l’on creuse le problème En effet, pensait Achab, cependant que les plus grands bonheurs humains sont toujours empreints d’une secrète mesquinerie, les grandes douleurs ont un sens mystique qui donne parfois à l’homme la grandeur de l’archange. C’est une conclusion que ne saurait démentir aucune piste suivie. Refaire le chemin jusqu’à l’origine de toutes les misères humaines nous conduit pour finir aux dieux, ancêtres non engendrés et force nous est faite de comprendre alors, à la face des gais soleils de la fenaison et des douces cymbales des lunes rondes de la moisson que les dieux eux-mêmes ne sont pas heureux à jamais. Le sceau indélébile qui, dès sa naissance, marque de tristesse le front de l’homme n’est que la griffe de la douleur des signataires.\par
Un secret vient d’être involontairement dévoilé qui aurait peut-être dû être révélé plus tôt mais, parmi tant d’autres traits concernant Achab, le fait qu’avant même le départ du {\itshape Péquod} et un certain temps après encore, il se soit caché avec autant d’intransigeance qu’un grand lama, cherchant un temporaire refuge dans le silence et le monde marmoréen des morts, resta pour certains un mystère. Le capitaine Peleg en donna une raison pour le moins insuffisante quoique en vérité toute expression des profondeurs d’Achab participât davantage d’une obscurité incompréhensible que d’une claire explication. Pourtant tout vint enfin au jour, du moins en ce qui concerne cet épi- sode ; ce malheureux accident était la cause de sa claustration. Aux yeux du cercle toujours plus étroit, toujours plus fermé de ceux qui, à terre, avaient le privilège de pouvoir l’approcher quelque peu, aux yeux de ce cercle timoré, cet accident – sur lequel Achab garda sombrement le secret – revêtit un caractère de terreur surnaturelle. De sorte que tous, dans la ferveur qu’ils lui portaient, s’accordèrent, dans la mesure du possible, à taire ce qu’ils en savaient. C’est pourquoi il se passa longtemps avant qu’il s’en répandît quelque chose à bord du {\itshape Péquod.}\par
Quoi qu’il en soit, laissons là les conciles invisibles et ambigus des puissances de l’air, des princes vengeurs ou des potentats du feu, qu’ils aient eu ou non affaire au terrestre Achab ; dans la circonstance présente, il prit des mesures simples et pratiques et fit appel au charpentier.\par
Lorsque celui-ci se présenta devant lui, il lui ordonna de se mettre sans délai à la fabrication d’une nouvelle jambe et demanda aux seconds de veiller à lui procurer clous et chevilles en ivoire de cachalot et de le mettre à même de choisir soigneusement, dans tout ce qu’il en avait été accumulé à bord depuis le début de la campagne, la matière première la plus solide et la plus fine de grain. Cela fait, le charpentier reçut l’ordre de terminer la jambe pour la même nuit et de se procurer tous les accessoires nécessaires, indépendamment de ceux qu’on pourrait prélever sur la jambe suspecte encore en usage. En outre, ordre fut donné de monter la forge de la cale où elle dormait et, pour gagner du temps, le forgeron fut prié de se mettre sur-le-champ à forger toutes les pièces de fer paraissant nécessaires.
\chapterclose


\chapteropen
\chapter[{CHAPITRE CVII. Le charpentier}]{CHAPITRE CVII \\
Le charpentier}\renewcommand{\leftmark}{CHAPITRE CVII \\
Le charpentier}


\chaptercont
\noindent Assieds-toi en sultan parmi les lunes de Saturne et considère un homme seul, dans son essence, il t’apparaîtra merveille, grandeur, misère. Du même endroit, prends l’humanité dans son ensemble, et tu croiras voir un nombre incalculable de copies faites en série, répétitions dans le temps et dans l’espace, inutiles pour la plupart. Si humble qu’il fût, et loin d’être un exemple de l’esprit humain, le charpentier du {\itshape Péquod} n’était pas un duplicata, c’est pourquoi le voici qui entre en scène.\par
Comme tous les charpentiers de bord, et plus particulièrement ceux des baleiniers, il avait, sans autre formation, l’expérience pratique de nombreux métiers voisins du sien propre, le métier de charpentier était la souche et le tronc de l’adresse manuelle dans tout ce qui de près ou de loin faisait usage du bois comme matériau auxiliaire. Mais le charpentier du {\itshape Péquod} était de plus remarquablement adroit pour toutes les réparations de fortune délicates sans cesse nécessaires à bord d’un grand navire, au cours d’un voyage de trois ou quatre ans dans des mers sauvages et lointaines. Sans parler de la promptitude avec laquelle il exécutait les travaux lui incombant normalement : réparation des baleinières éventrées, des espars fendus, réfection d’avirons défectueux, insertion de verres-morts dans le pont, remplacement de gournables dans les bordages, et toutes ces autres besognes diverses relevant de son art, il avait de plus une sûre habileté et des dispositions à la fois utilitaires et permettant de satisfaire les caprices des autres dans des domaines parfaitement opposés.\par
La grande scène où il tenait ses rôles si divers c’était son établi, une longue table, lourde et grossière, garnie de plusieurs étaux de bois et de fer de tailles variées, et en tout temps, sauf quand on travaillait sur une baleine, amarrée par le travers contre des fourneaux.\par
Un cabillot est-il trop épais pour s’insérer aisément dans son trou, le charpentier le pince dans l’un de ses étaux toujours prêt et aussitôt le lime. Faisait-on prisonnier un oiseau de terre égaré à l’étrange plumage, des fanons polis deviennent des barreaux, de l’ivoire de cachalot des traverses et le charpentier lui fait une cage en forme de pagode. Un canotier se foule-t-il le poignet, le charpentier lui mixtionne une lotion adoucissante. Stubb désire-t-il ardemment que soient peintes des étoiles vermillon sur les pelles de tous ses avirons, le charpentier les serre l’un après l’autre dans son gros étau de bois et les orne d’une constellation symétrique. Prend-il à un matelot la fantaisie de porter des boucles en os de requin, le charpentier lui perce les oreilles. Un autre a-t-il mal aux dents, le charpentier sort ses tenailles et l’invite en y frappant de la main à s’asseoir sur son établi, mais le pauvre diable, indocile, grimace de douleur à l’avance quand le charpentier dévissant largement son étau lui signifie d’y immobiliser sa mâchoire pour l’extraction.\par
Ainsi ce charpentier était prêt à tout, avec une égale indifférence et un manque égal de respect pour tout. Il tenait les dents pour des morceaux d’ivoire, les têtes pour de simples poulies de guinderesse, les hommes eux-mêmes pour des cabestans. Des talents si variés, sur un éventail aussi vaste, une telle adresse et une telle activité laisseraient croire à une vivacité d’intelligence peu commune. Ce n’était pas précisément le cas. Cet homme se distinguait surtout par son flegme impersonnel, je dis bien impersonnel, car il se fondait si bien dans l’infini environnant des choses qu’il ne semblait faire qu’un avec l’impassibilité de tout le monde visible, ce monde qui \hspace{1em} conserve éternellement sa paix à travers son incessante et multiple activité et qui vous ignore alors même que vous creusez les fondations d’une cathédrale. Ce flegme à demi horrible semblait impliquer que le manque de cœur de cet homme devait s’étendre à tout, pourtant il était parfois curieusement injecté d’un vieil humour boiteux, antédiluvien, asthmatique non dépourvu d’un certain esprit geignard semblable à celui qu’il avait fallu pour faire passer le temps du quart de nuit sur le barbu gaillard d’avant de l’arche de Noé. Était-ce que ce vieux charpentier avait été, sa vie durant, un vagabond, qui roulant en long et en large, non seulement n’avait pas amassé de mousse mais semblait encore avoir élimé les petites touffes qui auraient pu à l’origine fleurir sur lui ? Il était une abstraction nue, un bloc indivisible, aussi intègre qu’un enfant nouveau-né, sans idées préconçues sur ce monde, ni sur l’autre. On pouvait presque dire que pareille étrange pureté comportait une sorte d’inintelligence car il ne paraissait pas tant accomplir ses divers travaux grâce à la raison, à l’intuition, ou simplement parce qu’il y avait été formé, ou grâce à un mélange plus ou moins dosé de ces trois éléments, mais bien tout simplement par un processus spontané, prosaïque, en quelque sorte sourd et muet. Il n’était qu’un automate, son cerveau, s’il en eut jamais un, avait dû s’infiltrer au bout de ses doigts. Il ressemblait à cette création de Sheffield qui, si elle ne raisonne pas, n’en n’est pas moins utile, ce {\itshape multum in parvo} qui a l’aspect – encore qu’un peu gonflé – d’un simple couteau de poche recelant non seulement des lames de grandeurs différentes, mais encore des tournevis, des tirebouchons, des brucelles, des poinçons, des plumes, des règles, des limes à ongles, des fraises. De sorte que, si ses supérieurs souhaitaient utiliser le charpentier comme tournevis, ils n’avaient qu’à l’ouvrir à l’endroit voulu et la vis était serrée ; quand ils voulaient des brucelles, il leur suffisait de le saisir par les jambes et le tour était joué.\par
Pourtant, comme nous l’avons déjà suggéré, ce charpentier dépliant, à outils multiples, n’était pas, somme toute, un simple automate. S’il n’avait pas d’âme, quelque chose de subtil et d’insolite la remplaçait en lui. Qu’était-ce ? esprit de mercure ou gouttes d’ammoniaque, il est impossible de le dire. Mais cela était présent et habitait le charpentier depuis quelque soixante ans ou plus. Et ce principe de vie, indicible et astucieux le poussait à soliloquer longuement, mais seulement à la manière d’une roue dont le bourdonnement est aussi un monologue. Ou plutôt, son corps était une guérite et la sentinelle qu’elle abritait ne cessait de se parler à elle-même pour se tenir éveillée.
\chapterclose


\chapteropen
\chapter[{CHAPITRE CVIII. Achab et le charpentier. Le pont. Premier quart de nuit}]{CHAPITRE CVIII \\
Achab et le charpentier \\
\textbf{{\itshape Le pont. Premier quart de nuit}}}\renewcommand{\leftmark}{CHAPITRE CVIII \\
Achab et le charpentier \\
\textbf{{\itshape Le pont. Premier quart de nuit}}}


\chaptercont
\noindent Debout à son établi, le charpentier lime activement, à la lumière de deux falots, le morceau d’ivoire, destiné à la jambe, serré dans son étau. Des plaques d’ivoire, des lanières de cuir, des mandrins, des vis et des outils de toute sorte sont étalés devant lui. Sur l’avant, à la flamme rouge de la forge, on voit le forgeron au travail. « Bon sang de lime et bon sang d’os ! C’est dur ce qui devrait être mou, et c’est mou ce qui devrait être dur. C’est ainsi que vont les choses pour nous qui limons de vieilles mâchoires et des tibias. Essayons-en une autre. Oui, elle va mieux (il éternue) oui, elle est… (il éternue). Dieu me bénisse elle ne me laissera pas parler ! C’est tout ce qu’un vieux gars retire à travailler du bois mort. Sciez un arbre vivant et vous n’aurez pas cette poussière, sciez un os vivant et vous ne l’aurez pas non plus (il éternue). Allons, allons, vieux Smut, aide-toi, prends cette virole et cette boucle à vis, je vais en avoir besoin tout de suite. C’est encore une chance… (il éternue) qu’il n’y ait pas de rotule à faire, ç’aurait été embarrassant, mais un simple tibia, ma foi c’est aussi facile à faire qu’une perche à houblon, seulement j’aimerais lui donner un bon fini. Du temps, du temps, si au moins j’avais du temps, je pourrais lui tourner une jambe aussi parfaite (il éternue) que le fut jamais une jambe qui ait fait une révérence devant une dame dans un salon. Il n’y aurait pas de comparaison possible avec ces jambes et ces mollets de peau de daim que j’ai vu dans les vitrines. Elles prennent l’eau, oui c’est tout ce qu’elles font et bien sûr elles attrapent des rhumatismes et il faut les soigner (il éternue) avec des lotions, tout à fait comme des jambes vivantes. Voilà, avant que je la scie, il faut que je demande Sa Seigneurie le vieux Mogol, pour voir si la longueur va bien, un rien trop court je crois, s’il y a quelque chose à reprendre. Ah ! j’entends le talon, nous avons de la veine, il arrive lui ou quelqu’un d’autre pour sûr. »\par

\labelblock{ACHAB S’AVANCE}

\noindent (Au cours de la scène suivante, le charpentier continue d’éternuer à intervalles.) \par
– Eh bien ! faiseur d’hommes !\par
– Juste à point, sir. Si le capitaine veut bien, je vais prendre la longueur. Permettez que je mesure, sir.\par
– La mesure… pour une jambe ! Bon. Ce n’est pas la première fois, après tout. Allons-y ! Ici, tiens ton doigt à l’endroit. C’est un puissant étau que tu as là, charpentier, laisse-moi tâter de sa poigne. Oui, oui, une fameuse pince.\par
– Oh ! sir, il réduirait des os en miettes… attention, attention !\par
– Ne crains rien, j’aime une poigne solide, quelque chose qui tienne dans ce monde glissant. Que fabrique Prométhée, làbas ? – Je veux dire le forgeron – qu’est-il en train de faire ?\par
– Je pense qu’en ce moment il doit commencer à forger la boucle, sir.\par
– Bien. C’est une collaboration, il fournit la partie muscle. Quelle ardente flamme rouge il excite là-bas !\par
– Oui sir, il doit pousser au blanc pour un travail aussi délicat.\par
– Hum… sans doute. Je réalise maintenant combien est chargé de sens le fait que ce vieux Grec, Prométhée, qui a créé les hommes, dit-on, ait été forgeron et leur ait donné la vie avec le feu, car ce qui vient du feu, appartient au feu. L’enfer est vraisemblable. Comme vole la suie ! Ce doit être avec ce résidu que le Grec a fait les Africains. Charpentier, quand il en aura fini avec cette boucle, dis-lui de forger deux omoplates d’acier. Il y a, à bord, un pèlerin qui porte un fardeau écrasant.\par
– Sir ?\par
– Attends… pendant qu’il y est, je vais lui commander un homme entier, à Prométhée, selon un modèle opportun. Tout d’abord, cinquante pieds de haut sans chaussures, puis la poitrine au gabarit du tunnel sous la Tamise, ensuite des jambes avec des racines pour qu’il demeure au même endroit, des bras avec un tour de poignets de trois pieds, point de cœur du tout, un front d’airain et environ un quart d’acre de bon cerveau, et que je réfléchisse… dois-je commander des yeux pour voir à l’extérieur ? Non, mais une lucarne sur le dessus de la tête afin que la lumière y pénètre. Voilà, prenez commande et file !\par
– Le charpentier, à part : Eh bien, de quoi parle-t-il à qui s’adresse-t-il ? J’aimerais bien savoir. Dois-je rester là ?\par
– Piètre architecture qu’un dôme aveugle, j’en ai un. Non, non, il me faut un falot.\par
– Ah ! Ah ! C’est ce qui le chicane ? En voici deux, sir, un me suffit.\par
– Pourquoi me brandis-tu sous le nez cet attrape-voleur, homme ? C’est pire de braquer une lumière qu’une paire de pistolets.\par
– J’avais cru, sir, que vous aviez parlé au charpentier.\par
– Au charpentier ? comment… mais non. Je dirai charpentier, que tu as là un travail très propre, extrêmement distingué… ou bien préférerais-tu travailler l’argile ?\par
– Sir ? l’argile ? l’argile, sir ? C’est de la boue, nous la laissons aux cureurs de fossés.\par
– Le gaillard blasphème ! Qu’as-tu à éternuer ?\par
– L’os est plutôt poussiéreux, sir.\par
– Prends-en de la graine, alors. Et quand tu seras mort, ne mets jamais ta dépouille sous le nez des vivants.\par
– Sir ? oh ! ah !… je pense bien… oui… oh ! mon Dieu…\par
– Écoute, charpentier, j’ose croire que tu te dis un bon ouvrier digne de son travail, n’est-ce pas ? Eh bien ce qui parlera en ta faveur, c’est que je continue à sentir une autre jambe là où tu auras ajusté celle que tu fais, c’est-à-dire, charpentier ma vieille jambe perdue de chair et de sang, veux-je dire. Ne peuxtu pas chasser le vieil Adam ?\par
– En vérité, sir, je commence à y comprendre quelque chose. Oui, j’ai entendu dire des choses curieuses à ce sujet, sir… et comment un homme démâté ne perd jamais entièrement le sentiment de son vieil espar, mais qu’il y sentira des picotements parfois. Puis-je humblement vous demander s’il en est ainsi, sir ?\par
– Il en est bien ainsi, homme. Tiens, mets ta jambe vivante là où autrefois se trouvait la mienne. Et maintenant l’œil n’y voit qu’une jambe, deux en voit l’âme toutefois. Là où tu sens vibrer la vie, là exactement là, à un cheveu près, je la sens. N’est-ce pas une énigme ?\par
– Je dirais humblement que c’est une question embarrassante, sir.\par
– Chut, alors. Comment saurais-tu si une chose vivante, entière, pensante ne se tient pas, invisible et autonome, précisément à l’endroit où tu te tiens à présent, et cela malgré toi ? Ainsi dans tes heures les plus solitaires, ne crains-tu pas qu’on t’écoute ? Non, ne dis rien ! Et si je sens encore la douleur cuisante de ma jambe écrasée bien qu’elle soit depuis si longtemps réduite à néant, pourquoi toi, charpentier, ne sentirais-tu pas à jamais le brasier de l’enfer te cuire sans que tu aies de corps ? Ah !\par
– Doux Seigneur ! En vérité, sir, si l’on en vient là, il faut que je refasse le calcul, je crois que j’ai oublié de porter un petit chiffre, sir.\par
– Écoute-moi bien, les nigauds ne devraient jamais poser de principes. Encore combien de temps avant que la jambe soit terminée ?\par
– Une heure peut-être, sir.\par
– Alors hâte-toi de faire le savetier et apporte-la-moi (il se détourne pour partir). Ô Vie ! Me voici, fier comme un dieu grec, et débiteur pourtant de ce lourdaud à cause d’un os sur lequel me tenir ! Maudite soit cette humaine dette réciproque qui ne s’inscrit dans aucun livre de compte. Je serais libre comme l’air et je figure dans tous les livres de comptes. Je suis si riche, j’aurais pu couvrir l’enchère des prétoriens les plus fortunés à l’adjudication de l’empire romain qui était aussi celui du monde et cependant je suis redevable pour la chair de cette langue avec laquelle je fanfaronne. Par le ciel, je vais chercher un creuset, m’y jeter et m’y fondre jusqu’à n’être plus qu’un petit abrégé de vertèbre. Oui. »\par
Le charpentier (se remettant à l’ouvrage)\par
« Eh bien, eh bien, eh bien ! Stubb le connaît mieux que personne et Stubb dit toujours qu’il est étrange, rien de plus que ce petit mot qui suffit : étrange. Il est étrange, étrange, il en rebat tout le temps les oreilles de M. Starbuck – étrange, sir – étrange, étrange, très étrange. Et voilà sa jambe ! Oui, maintenant que j’y pense, c’est sa compagne de lit ! il a un bâton en mâchoire de cachalot pour femme ! Et voilà sa jambe, il va se tenir debout dessus. Qu’est-ce que c’était déjà que cette histoire d’une jambe se tenant en trois endroits et de ces trois endroits se tenant dans un seul enfer… qu’est-ce que c’était ? Oh ! je ne m’étonne pas qu’il me regarde avec un pareil mépris ! Il me vient parfois des idées curieuses, qu’ils disent, mais c’est seulement par hasard. Aussi, un petit râblé comme moi ne devrait pas se risquer à s’aventurer en eau profonde avec de grands capitaines-hérons, l’eau vous arrive bien vite au menton et on crie bien fort pour le canot de sauvetage. Et voilà la jambe héronnière ! Pour sûr longue et fine ! Pour la plupart des gens, une paire de jambes dure une vie entière, et ce doit être parce qu’ils s’en servent avec clémence, comme une vieille dame au cœur tendre, se sert de ses vieux chevaux de carrosse rondelets. Mais Achab ne ménage pas sa monture. Voyez, il a mené une jambe à la mort et il a rendu l’autre boiteuse pour la vie, et maintenant il use jusqu’à la corde des jambes en os. Holà, dis donc, Smut, aide-toi un peu avec ces vis et finis-en avant que le gars de la résurrection ne vienne réclamer à coups de trompette toutes les jambes vraies ou fausses, comme les hommes des brasseries font la tournée de ramassage des vieux tonneaux pour les remplir à nouveau. En voilà une jambe ! Elle a l’air d’une jambe vivante limée jusqu’au noyau, il se tiendra debout dessus demain, il y mesurera la hauteur des astres et se mesurera à eux. Holà ! J’ai failli oublier la petite plaque ovale d’ivoire polie, sur laquelle il calcule la latitude. Oui, oui, à présent le ciseau, la lime et le papier de verre ! »
\chapterclose


\chapteropen
\chapter[{CHAPITRE CIX. Achab et Starbuck dans la cabine}]{CHAPITRE CIX \\
Achab et Starbuck dans la cabine}\renewcommand{\leftmark}{CHAPITRE CIX \\
Achab et Starbuck dans la cabine}


\chaptercont
\noindent Le lendemain matin, l’équipage pompait comme à l’accoutumée quand une quantité d’huile non négligeable vint avec l’eau. En bas, il devait y avoir une fuite sérieuse dans les barils. Chacun s’inquiéta et Starbuck descendit dans la cabine pour faire rapport de cette malencontreuse affaire\footnote{Lorsque la quantité d’huile est considérable à bord d’un baleinier, un devoir semi-hebdomadaire veut qu’on arrose les barils d’eau de mer au moyen d’un tuyau introduit dans la cale. On pompe ensuite cette eau à divers intervalles. C’est ainsi qu’on préserve l’étanchéité des barils et que les matelots reconnaissent à la qualité de l’eau pompée s’il y a une fuite sérieuse dans leur précieuse cargaison.}.\par
Par le sud-ouest, le {\itshape Péquod} approchait à présent de Formose et des îles Bachi entre lesquelles la mer de Chine débouche, au tropique, dans le Pacifique. C’est pourquoi Starbuck trouva Achab, une carte générale des archipels orientaux déployée devant lui, et une autre des longues côtes est des îles du Japon : Nippon, Matsmaï et Sikok. Sa nouvelle jambe d’ivoire, blanche comme la neige, appuyée contre la jambe vissée de sa table, un couteau de poche en forme de serpe à la main, l’étonnant vieillard, le dos tourné à la porte, fronçait le sourcil et retraçait une fois de plus sa route.\par
– Qui est là ? demanda-t-il en entendant le pas sur le seuil de la porte. Au pont ! Partez ! Le capitaine Achab se méprend. C’est moi. L’huile fuit dans la cale, sir. Il nous faut hisser les petits palans et décharger.\par
– Hisser les petits palans et décharger ? Maintenant que nous approchons du Japon, mettre en panne pour une semaine pour rafistoler un lot de vieux cerceaux ?\par
– Ou le faire, sir, ou perdre en un jour plus d’huile que nous n’en pouvons rattraper en une année. Ce que avons acquis en parcourant vingt mille milles vaut d’être sauvé, sir.\par
– Il en est ainsi, il en est ainsi, si nous rattrapons.\par
– Je parlais de l’huile dans la cale, sir.\par
– Et je n’en parlais pas, je n’y pensais même pas du tout. Partez ! Laissez-la couler. Moi-même je ne suis que fuites. Oui ! des fuites au sein de fuites ! Non seulement plein de barils qui fuient, mais ceux-ci encore dans un navire qui fait eau, c’est une situation bien pire que celle du {\itshape Péquod}, homme. Pourtant je ne m’arrête pas pour boucher ma fuite, qui la trouverait au fond d’une coque si lourdement chargée, et même viendrait-elle à être découverte, quel espoir de l’aveugler dans la tempête hurlante de cette vie ? Starbuck ! je ne ferai pas hisser les petits palans.\par
– Que diront les propriétaires, sir ?\par
– Que les propriétaires restent sur la plage de Nantucket et que leurs vociférations couvrent la voix des typhons. Qu’importe à Achab ? Les propriétaires, les propriétaires ? Tu es toujours à me débiter des niaiseries au sujet de ces avares, comme si ces propriétaires étaient ma conscience. Mais écoute bien, le seul vrai propriétaire de quoi que ce soit, c’est celui qui en a le commandement, et écoute encore, ma conscience est dans la quille de ce navire. Au pont !\par
– Capitaine Achab, répondit le second en rougissant et en pénétrant plus avant dans la cabine avec une audace si étrangement pleine de respect et de prudence qu’elle semblait éviter de se trahir si peu que ce fût extérieurement et qu’intérieurement elle semblait douter d’elle-même plus qu’à demi. Un homme meilleur que moi passerait sur ce qu’il prendrait en mauvaise part venant d’un homme plus jeune que toi, oui, et d’un homme plus heureux, capitaine Achab.\par
– Du diable ! Oses-tu aller jusqu’à me considérer d’un œil scrutateur ? Au pont !\par
– Non, sir, pas encore. Je vous implore. Et j’ose, sir, vous demander d’être indulgent ! Ne pouvons-nous nous comprendre l’un l’autre mieux que jusqu’à présent capitaine Achab ?\par
Achab arracha au râtelier (qui fait partie du mobilier de la plupart des cabines des navires des mers du Sud), un fusil chargé et le braquant sur Starbuck, il s’écria : il y a un seul Dieu qui est Seigneur de la terre et un seul qui est seigneur à bord du {\itshape Péquod.} Au pont !\par
On aurait pu croire un instant, au regard étincelant du second et à ses joues enflammées, qu’il avait reçu la décharge du canon levé. Dominant son trouble, il s’apprêta à partir presque calmement mais s’arrêta un instant sur le seuil et dit : « Tu m’as outragé, mais non insulté sir, je ne te demande point de te défier de Starbuck tu ne ferais qu’en rire, mais qu’Achab se défie d’Achab, défie-toi de toi-même, vieillard. »\par
– Il devient courageux mais n’en obéit pas moins, prudente bravoure que celle-là ! murmura Achab tandis que Starbuck disparaissait. Qu’a-t-il dit… qu’Achab se défie d’Achab… ce n’est pas sot ! Il y a quelque chose à dire sur ce point.\par
Puis, utilisant machinalement le fusil en guise de canne, il arpenta la petite cabine avec un front d’acier mais bientôt les barres s’effacèrent au-dessus de ses sourcils, il remit l’arme dans le râtelier et monta sur le pont.\par
– Tu n’es qu’un trop bon garçon, Starbuck, dit-il à voix basse au second puis élevant la voix il ordonna à l’équipage :\par
– Ferlez les voiles de perroquet. Au bas ris les huniers, à l’avant et à l’arrière, brassez la grande vergue, hissez les palans, et déchargez la cale.\par
Il est vain peut-être de se demander pour quelle raison exacte Achab agissait ainsi à l’égard de Starbuck. Ce pouvait être un sursaut d’honnêteté, ou simplement une politique de prudence qui, en cette circonstance, lui interdisait de montrer le moindre signe de désaffection fût-elle passagère, envers le premier officier de son bord. Quoi qu’il en fût, ses ordres furent exécutés et les palans hissés.
\chapterclose


\chapteropen
\chapter[{CHAPITRE CX. Queequeg dans son cercueil}]{CHAPITRE CX \\
Queequeg dans son cercueil}\renewcommand{\leftmark}{CHAPITRE CX \\
Queequeg dans son cercueil}


\chaptercont
\noindent En cherchant, on découvrit que les derniers barils arrimés étaient parfaitement sains et que la fuite devait se trouver plus loin. De sorte que, le temps étant calme, les hommes cherchèrent toujours plus avant, troublant le sommeil de l’énorme premier plan d’arrimage et tirant de leur sombre nuit ces géantes taupes pour les amener à la lumière du jour. Ainsi ils fouillèrent toujours plus avant. Au dernier plan, les tonneaux de cent gallons avaient un air si antique, si rongé, si moussu qu’on en venait presque à chercher un fût, pierre angulaire moisie, contenant les pièces de monnaie du capitaine Noé et les exemplaires des affiches en vain placardées pour avertir du déluge le vieux monde infatué. On hissa aussi, tierçon après tierçon, les provisions d’eau, de pain, de bœuf, la futaille en bottes et les cercles de fer de rechange, jusqu’à ce qu’il devînt difficile de passer sur le pont encombré. La coque vide sonnait sous les pas comme des catacombes et roulait en chancelant comme une damejeanne vide. Trop chargé dans les hauts, le navire semblait un étudiant à jeun dont la tête serait pleine de tout Aristote. Heureux que les typhons eussent alors épargné leur visite !\par
Or ce fut à ce moment-là que mon pauvre compagnon païen et fidèle ami de cœur Queequeg fut pris d’une fièvre qui l’amena près de son immortalité.\par
Ce métier de baleinier, il faut le dire, ne connaît aucune sinécure ; grade et danger vont la main dans la main jusqu’à ce qu’on parvienne au rang de capitaine ; plus on s’élève, plus on peine dur. Il en allait ainsi du pauvre Queequeg qui, en tant que harponneur, non seulement devait affronter la fureur de la baleine vivante, mais une fois morte, comme nous l’avons vu, se tenir sur son dos dans la houle. Enfin il lui fallait descendre dans les ténèbres de la cale et, suant amèrement tout le jour dans cette cellule souterraine, se charger résolument des plus lourds barils et veiller à leur arrimage. Bref les baleiniers appellent les harponneurs les gens de la cale.\par
Pauvre Queequeg ! lorsqu’on eût à demi extrait les entrailles du navire vous auriez dû, en vous penchant sur l’écoutille, jeter un coup d’œil sur lui tandis que, torse nu sur ses seuls pantalons de laine, le sauvage tatoué rampait dans la vase humide tel un lézard moucheté de vert au fond d’un puits. Et c’est bien un puits ou une glacière que s’avéra être la cale pour lui, le malheureux païen. Chose étrange, malgré ses suées brûlantes, il prit un refroidissement terrible qui dégénéra en fièvre et finalement l’amena, après quelques jours de souffrance sur son hamac, aux portes mêmes de la mort. Ces quelques jours qui se traînèrent avec lenteur le virent dépérir et dépérir encore, tant et si bien qu’il ne parut rester de lui que ses os et ses tatouages ; son visage s’affina au point que ses pommettes se firent aiguës tandis que ses yeux, s’élargissant toujours plus, prenaient un éclat d’une étrange douceur ; du fond de la maladie, ils vous regardaient avec bénignité et profondeur, merveilleux témoignage en lui d’une santé immortelle qui ne pouvait pas plus mourir que s’affaiblir. Et comme des cercles sur les eaux s’agrandissent en se perdant, ses yeux toujours plus vastes arrondissaient les anneaux de l’Éternité. Une terreur sacré indicible, vous envahissait auprès de ce sauvage déclinant, tandis qu’on voyait passer sur son visage d’étranges choses, telles qu’en virent ceux qui assistèrent à la mort de Zoroastre. Car l’effrayant mystère de l’homme n’a jamais été livré par les mots ou par les livres et l’approche de la mort, qui rend égal tout un chacun et le marque au sceau d’une ultime révélation, nul n’en pourrait rien dire sinon un auteur revenu d’entre les morts. De sorte que répétonsle, il n’y eut ni Chaldéen ni Grec qui, à sa mort, eût des pensées plus élevées et plus saintes que celles dont les ombres mystérieuses passaient sur le visage du pauvre Queequeg, tandis qu’il gisait paisible dans son hamac que les vagues marines semblaient doucement bercer pour son dernier sommeil et que la houle océane semblait le soulever toujours plus haut vers le ciel qui l’attendait.\par
Il n’était pas un homme de l’équipage qui ne le considérât perdu. Quant à ce que Queequeg lui-même pensait de son état, il le rendit évident par une curieuse faveur qu’il demanda. Dans la grisaille du premier quart, alors que le jour perçait à peine, il appela un homme, lui prit la main et lui dit que, se trouvant à Nantucket, il y avait vu par hasard de petits canoës de bois sombre, pareil à l’opulent bois de fer de son île natale et que, s’étant renseigné, il avait appris que tous les baleiniers venant à mourir à Nantucket étaient couchés dans ces mêmes canoës noirs et que l’idée d’être mêmement étendu lui avait beaucoup plu, car ce n’était pas sans ressembler aux coutumes de sa race lorsque, après avoir embaumé le corps d’un guerrier, on l’allonge dans son canoë, le laissant emporter par les flots jusqu’aux archipels d’étoiles. Car non seulement ils croient que les étoiles sont des îles mais encore que, bien au-delà de tout horizon, leur mer clémente et sans rivages se confond avec le bleu du ciel pour former les blancs brisants de la voie lactée. Il ajouta que l’idée lui donnait le frisson d’avoir, selon l’usage marin, son hamac pour linceul et d’être jeté comme une pâture infâme à la voracité des requins charognards. Non quand bien même cela impliquait une navigation incertaine et une longue dérive dans la confusion des âges, il souhaitait avoir un canoë comme ceux de Nantucket, Autant plus adéquat pour un baleinier qu’à la façon des pirogues, ces canoë-cercueils n’avaient pas de quille.\par
À peine cet étrange vœu vint-il à être connu à l’arrière que le charpentier reçut aussitôt l’ordre de le réaliser, quelle que fût la fantaisie de Queequeg. Il se trouvait à bord une vieille pièce de bois exotique, couleur de cercueil, qui, au cours d’un long voyage précédent, avait été coupée dans les forêts primitives des îles Laccadives et on lui recommanda de faire le cercueil avec ce bois sombre. À peine informé, le charpentier prit sa règle et, avec la promptitude indifférente qui le caractérisait, il se dirigea vers le gaillard d’avant et prit avec une précision méticuleuse les mesures de Queequeg, lui faisant des coches à la craie chaque fois qu’il déplaçait sa règle.\par
– Ah ! le pauvre gars ! le voilà obligé de mourir à présent ! s’écria le matelot de Long Island.\par
De retour à son établi, le charpentier, pour plus de facilité et pour mémoire, y fit au couteau deux entailles à chaque extrémité, d’après la longueur que devait avoir le cercueil. Cela fait, il disposa son bois et ses outils et se mit à l’ouvrage.\par
Le dernier clou planté, le couvercle dûment raboté et ajusté, il prit légèrement le cercueil sur son épaule, et, s’en allant vers le gaillard d’avant, il demanda si l’on était déjà prêt à s’en servir en ces lieux.\par
Surprenant les exclamations indignées, mais à demi amusées, avec lesquelles les hommes du pont essayaient de faire rebrousser chemin au cercueil, Queequeg, à la consternation générale, demanda que la chose lui soit apportée sans retard et personne ne s’y opposa car, de tous les mortels, certains moribonds sont les plus tyranniques et l’on cède à ces pauvres gars parce qu’ils en auront bientôt et à jamais fini de nous gêner.\par
Penché en dehors de son hamac, Queequeg regarda longuement et attentivement le cercueil. Puis il réclama son harpon, le fit démancher et en fit déposer le fer dans le cercueil avec une pagaie de sa baleinière. Sur sa demande aussi on en garnit le pourtour avec des biscuits, on déposa à la tête une bouteille d’eau douce et aux pieds un petit sac de sciure et de terre mêlées qu’on avait raclées dans la cale, et enfin, en guise d’oreiller, un morceau roulé de toile à voile. Queequeg pria alors qu’on le portât dans sa dernière couche afin qu’il juge de son confort si tant est qu’elle en eût. Il y resta étendu immobile quelques instants puis réclama qu’on aille chercher dans son sac son petit dieu Yoyo. Alors il croisa les bras sur Yoyo, demanda qu’on mît en place le panneau d’écoutille, comme il appelait le couvercle du cercueil ; le couvercle tourna sur sa charnière de cuir et voilà que Queequeg se trouva allongé dans son cercueil. Seul son visage serein était visible.\par
– {\itshape Rarmai} (ça ira, c’est confortable) murmura-t-il enfin en faisant signe qu’on le remette dans son hamac.\par
À peine y était-il installé que Pip, qui pendant tout ce temps avait traîné furtivement alentour, s’approcha de lui et lui prit la main en sanglotant doucement ; dans l’autre main, il tenait son tambourin.\par
– Pauvre vagabond ! n’en auras-tu jamais fini de cet épuisant vagabondage ? où vas-tu à présent ? Si les courants t’emportent jusqu’aux suaves Antilles aux plages battues des seuls nénuphars, y transmettras-tu pour moi un petit message ? Cherches-y un certain Pip, disparu depuis longtemps déjà ; je crois qu’il est dans ces îles lointaines. Si tu le trouves, console-le car il doit être bien triste car, vois, il a oublié son tambourin et je l’ai trouvé. Rig-a-dig, dig, dig ! Maintenant meurs, Queequeg et je rythmerai ta marche funèbre.\par
– J’ai entendu dire, murmura Starbuck, contemplant la scène, penché sur l’écoutille, que des hommes parfaitement ignorants venaient, sous le coup de fièvres violentes, à parler des langues anciennes et que l’on découvrait toujours en approfondissant ce mystère que, lors d’une enfance tout entière oubliée, ils les avaient effectivement entendu parler par des savants de leur entourage. Aussi j’ai la conviction que le pauvre Pip, dans l’étrange fraîcheur de sa folie, apporte les preuves divines de toutes nos demeures célestes. Où ailleurs aurait-il appris tout cela ? Chut ! il parle encore mais avec plus de frénésie.\par
– En rangs par deux ! Faisons de lui un général ! Oh ! où est son harpon ? Posons-le là. Rig-a-dig, dig, dig ! Hurrah ! Oh ! qu’un coq de combat vienne à s’asseoir sur sa tête pour chanter ! Queequeg meurt en héros, entendez bien, Queequeg meurt en héros ! Ne l’oubliez pas, Queequeg meurt en héros ! Je dis bien héros, héros ! mais ce vil petit Pip lui est mort en couard, il est mort tout frissonnant. Haro sur Pip ! Chut si vous trouvez Pip, informez toutes les Antilles que c’est un déserteur, un couard, un couard, un couard ! Dites-leur qu’il a sauté d’une baleinière ! Jamais je ne jouerais du tambourin pour l’ignoble Pip, ni le consacrerais général, s’il mourait ici encore une fois. Non, non ! honte à tous les couards, honte à eux ! qu’ils se noient comme Pip qui a sauté d’une baleinière. Honte ! Honte !\par
Pendant ce temps, Queequeg, immobile, les yeux clos, semblait rêver, puis Pip fut emmené et le malade fut remis dans son hamac.\par
Mais maintenant qu’il avait apparemment pris toutes ses mesures devant la mort, que son cercueil s’était révélé bien ajusté, Queequeg tout soudain revint à la vie, il s’avéra bientôt que la boîte du charpentier était inutile et, à ceux qui lui exprimaient leur ravissement étonné, il répondit en substance que la raison de cette brusque convalescence tenait au fait qu’au moment critique il s’était souvenu d’avoir négligé un petit devoir à terre. Aussi s’était-il ravisé quant à sa mort et il déclara qu’il ne pouvait mourir pour le moment. Ils lui demandèrent alors si vivre ou mourir dépendait de sa volonté souveraine et de son bon plaisir. Il répondit : certainement. En un mot, l’opinion de Queequeg était que si un homme était décidé à vivre, une simple maladie ne pouvait le tuer, ni rien, hormis une baleine, ou une tempête, ou quelque agent destructeur de même nature, violent, irrépressible, inintelligent.\par
Mais il y a une différence notoire entre un sauvage et un civilisé car, tandis qu’un civilisé malade fait souvent six mois de convalescence, un sauvage malade est en règle générale presque rétabli en un jour. Aussi mon Queequeg reprit-il des forces en temps voulu et enfin, après être indolemment resté assis pendant quelques jours sur le guindeau (mangeant toutefois avec un solide appétit) il gicla soudain sur ses pieds, s’étira en tous sens, bras et jambes, bâilla quelque peu et, sautant dans sa pirogue suspendue, et balançant un harpon, il se déclara prêt au combat.\par
Avec une sauvage fantaisie, il utilisa désormais son cercueil comme coffre, y versa le contenu de son sac de toile et y rangea ses vêtements. Il consacra de nombreuses heures de loisir à sculpter sur le couvercle des figures grotesques et des dessins et il semblait qu’il cherchât à sa manière primitive à y transcrire les tatouages compliqués de son corps. Tatouages qui étaient l’œuvre d’un défunt prophète et voyant de son île qui avait écrit, dans ces caractères hiéroglyphiques, une thèse complète sur les cieux et la terre et un traité mystique sur l’art d’atteindre à la vérité. La personne de Queequeg était dès lors une énigme à déchiffrer, une œuvre étonnante en un volume dont lui-même ne pouvait pas lire les mystères contre lesquels battait pourtant son cœur de chair. Ces mystères étaient donc destinés à tomber en poussière avec le parchemin vivant où ils s’inscrivaient et à demeurer à jamais impénétrés. C’est bien une telle pensée qui avait dû arracher à Achab cette exclamation farouche qu’il eut lorsque, se détournant un matin du pauvre Queequeg, il s’écria : « Oh ! satanique supplice de Tantale infligé par les dieux ! »
\chapterclose


\chapteropen
\chapter[{CHAPITRE CXI. Le Pacifique}]{CHAPITRE CXI \\
Le Pacifique}\renewcommand{\leftmark}{CHAPITRE CXI \\
Le Pacifique}


\chaptercont
\noindent Quand glissant le long des îles Bachi nous débouchâmes enfin dans la grande mer du Sud, j’aurais pu, en d’autres circonstances, accueillir mon cher Pacifique par des actions de grâce sans nombre car la longue prière de ma jeunesse était exaucée, cet Océan serein déroulait devant moi, vers l’est, des milliers de lieues d’azur.\par
On ne sait quel doux mystère dissimule cet Océan dont les mouvements, redoutables dans leur suavité, sont semblables à ceux des profondeurs d’une âme et à ceux de la terre d’Éphèse où saint Jean l’Évangéliste est enseveli. Il est juste que sur ces pâturages marins, sur la vaste houle de ces prairies liquides, sur ces champs du potier des quatre continents, les vagues se lèvent et s’abaissent dans un flux et reflux incessant, car, par millions, les ténèbres aux ombres mêlées, les songes engloutis, les somnambulismes, les rêveries, tout ce que nous appelons vies et âmes, sont étendus là rêvant encore, se tournant comme des dormeurs sur leurs couches, cependant que leur inquiétude communique aux vagues leur roulement sans fin.\par
De tout initié errant et contemplatif, ce paisible Pacifique, une fois aperçu, doit à jamais devenir l’Océan d’adoption au centre des eaux du monde dont il est le corps, l’océan Indien et l’Atlantique ne sont que les bras. Les mêmes vagues balaient les môles des cités nouvelles de Californie, bâties hier à peine par la plus jeune race d’hommes, et baignent les bords usés mais encore somptueux des terres asiatiques, plus vieilles qu’Abraham, les mêmes vagues portent des voies lactées d’îles de corail, de plats archipels interminables, inconnus, et des Japons impénétrables. Ce divin, mystérieux Pacifique répartit autour de lui le monde tout entier et transforme tout rivage en baie lui appartenant. La pulsation de sa marée est le cœur de la terre. Soulevé par sa houle éternelle, vous devez reconnaître et saluer le dieu de la séduction, le dieu Pan.\par
Le cerveau d’Achab ne devait guère être troublé par la pensée de Pan, tandis qu’il se tenait comme une statue de bronze à sa place accoutumée, à côté du gréement d’artimon, respirant d’une narine distraite le parfum musqué et sucré des îles Bachi (dont les bois devaient abriter la promenade de tendres amants), tandis que de l’autre il aspirait le souffle salé de cette mer nouvelle. Cette mer où la Baleine blanche tant haïe devait nager en ce même instant. Lancé enfin dans ces eaux qui marquaient un terme et glissant vers les parages de chasse du Japon, le vieillard durcissait sa détermination. Ses lèvres volontaires serraient leur étau, les veines de son front gonflaient leur delta accru de rivières débordantes et jusque dans son sommeil son cri faisait retentir la voûte de la coque : « À l’arrière tous ! la Baleine blanche souffle du sang épais ! »
\chapterclose


\chapteropen
\chapter[{CHAPITRE CXII. Le forgeron}]{CHAPITRE CXII \\
Le forgeron}\renewcommand{\leftmark}{CHAPITRE CXII \\
Le forgeron}


\chaptercont
\noindent Après en avoir terminé avec les accessoires de la jambe d’Achab, profitant de l’été doux et frais de ces latitudes, Perth, le vieux forgeron barbouillé de suie et couvert d’ampoules, dans l’attente d’une activité prochaine, n’avait pas redescendu sa forge dans la cale, mais l’avait gardée, solidement arrimée aux boucles d’amarrage près du mât de misaine ; chefs de pirogue et harponneurs faisaient sans cesse appel à lui pour quelque menue besogne, transformer ou réparer leurs armes diverses et le matériel de leurs baleinières. Il était souvent entouré d’un cercle impatient qui attendait ses services, tenant des pelles d’embarcation, des pointes de piques, des harpons et des lances, surveillant jalousement tous ses mouvements noircis de suie tandis qu’il travaillait. Toutefois, le vieil homme était un patient marteau entre des mains patientes. Il n’avait ni mouvement d’humeur, ni murmures, ni irritation. Lent, silencieux, solennel, courbant toujours plus avant son dos depuis longtemps brisé, il travaillait comme si le travail était sa vie et les coups lourds de son marteau les lourds battements de son cœur. Il en était ainsi. Infiniment misérable !\par
Tout au début du voyage, la curiosité des matelots avait été excitée par la démarche singulière du vieillard, une certaine façon qu’il avait, légère mais douloureuse, de tituber. Il avait fini par céder à l’assaut, importun et répété, de leurs questions, aussi advint-il que tout un chacun connaissait l’histoire honteuse de sa lamentable destinée.\par
Par une cinglante nuit d’hiver, il s’était attardé – et non innocemment – sur une route reliant deux villes campagnardes, et se sentant gagné, dans son hébétude par une torpeur mortelle, s’était réfugié dans une étable branlante, délabrée. Il en résulta qu’il perdit les extrémités de ses deux pieds. Petit à petit, après cette première révélation, on connut de sa vie les quatre actes du bonheur, puis le long cinquième acte inachevé de la souffrance.\par
C’était un vieil homme qui, sur le tard, vers l’âge de soixante ans, avait rencontré ce que le vocabulaire du chagrin appelle la ruine. Il avait été un artisan d’une supériorité reconnue. Ayant beaucoup de travail, il avait possédé une maison et un jardin, enlacé une femme jeune, aimante, qui aurait pu être sa fille, et trois enfants joyeux et roses ; chaque dimanche, il se rendait à une riante église au cœur d’un bosquet. Mais une nuit, à la faveur des ténèbres, et sous un déguisement plein de ruse, un voleur acharné s’était glissé dans cette maison de bonheur et les avait tous dépouillés de tout. Le plus triste à dire c’est que, par ignorance, le forgeron avait lui-même introduit ce malfaiteur au cœur de sa famille. C’était le Vase de l’Escamoteur ! À peine ôté le bouchon fatal, le démon jaillit du goulot et brûla son foyer. Pour des raisons très sages d’économie, de prudence, la forge se trouvait dans le sous-sol de la maison, mais elle avait une entrée séparée, de sorte que la jeune, robuste et tendre épouse entendait toujours, non seulement sans impatience chagrine, mais encore avec un sain plaisir, résonner les coups vigoureux de son vieux mari aux jeunes bras. Assourdis d’avoir traversé planchers et murs, ils parvenaient avec douceur jusqu’à la chambre des enfants qui s’endormaient au bruit de cette solide berceuse de fer forgé.\par
Ô malheur et infortune ! Ô Mort, pourquoi ne te trouves-tu pas parfois exacte au rendez-vous ? Si tu avais emporté le vieux forgeron avant que sa ruine fût consommée, sa jeune veuve aurait nourri un chagrin qui aurait eu ses charmes, ses orphelins auraient pu dans leurs années à venir rêver d’un héros vénérable, tout en ayant des moyens d’existence. Mais la Mort terrassa un vertueux frère aîné dont le chantant travail quotidien faisait vivre une autre famille et laissa debout le vieil homme néfaste, plus qu’inutile, afin que la hideuse pourriture de la vie en fît une moisson plus aisée.\par
Pourquoi tout raconter ? Le marteau du sous-sol espaça chaque jour ses coups devenus chaque jour plus faibles, la femme assise, gelée à la fenêtre, contemplait avec des yeux brillants mais sans larmes les visages en pleurs de ses enfants. Le soufflet se tut, la forge étouffa sous les cendres, la maison fut vendue, la mère plongea sous l’herbe haute du cimetière, deux de ses enfants l’y suivirent et le vieil homme, sans famille et sans toit, s’en fut titubant, vagabond en crêpe de deuil, cinglé de mépris à cause de sa douleur, de sa tête grise insultant aux boucles blondes !\par
La Mort seule paraît souhaitable au couronnement d’une telle carrière, mais la Mort n’est qu’un saut dans une région impénétrée, elle n’est que la salutation première à l’éventualité de l’immensité lointaine, désertique, liquide, sans rivages. À de tels hommes qui convoitent la mort d’un regard fervent, que des scrupules retiennent devant le suicide, l’Océan, tout offrande, tout compréhension, ouvre la séduction de sa vaste plaine aux terreurs fascinantes et inimaginables, aux aventures étonnantes d’une vie neuve et du sein des Pacifiques infinis, des milliers de sirènes lui chantent : « Viens, cœur brisé, voici une vie nouvelle sans le péché d’une mort dérobée, voici de surnaturelles merveilles, il n’est point besoin de mourir pour les posséder. Viens ! Ensevelis-toi dans une vie qui, face à ce monde de la terre qui désormais te hait comme tu le hais, apporte plus d’oubli que la mort. Viens ! Creuse ta tombe au cimetière et viens jusqu’à ce que nous t’épousions ! »\par
Entendant ces voix, venues de l’est et de l’ouest, au premier rayon du soleil, à la tombée de la nuit, l’âme du forgeron répondit : « Oui, je viens ! » Et Perth partit chasser la baleine…
\chapterclose


\chapteropen
\chapter[{CHAPITRE CXIII. La forge}]{CHAPITRE CXIII \\
La forge}\renewcommand{\leftmark}{CHAPITRE CXIII \\
La forge}


\chaptercont
\noindent La barbe en broussaille, enveloppé d’un tablier écailleux de peau de chagrin, Perth se trouvait debout, vers midi, entre sa forge et son enclume posée sur une traverse de bois de fer ; d’une main il tenait des pointes de fourche dans le feu, de l’autre il manœuvrait son soufflet lorsque arriva Achab, tenant un petit sac de cuir roux. À peu de distance de la forge, le sombre Achab s’immobilisa jusqu’à ce que Perth, ayant retiré son fer de la flamme, se fût mis à le battre sur l’enclume, tirant de la masse incandescente des vols épais d’étincelles qui vinrent tourbillonner jusqu’auprès d’Achab.\par
– Sont-ce là les poulets de la Mère Carey, Perth ? toujours à voler dans ton sillage, oiseaux de bon augure mais non point tous sans exception. Vois, ils s’enflamment, mais toi…, toi tu vis parmi eux sans qu’ils te brûlent.\par
– Parce que je suis brûlé par tous les bouts, capitaine Achab, répondit Perth, s’appuyant un instant sur son marteau, rien ne peut plus me brûler, le feu attaque difficilement les cicatrices.\par
– Bien, bien, n’en dis pas davantage. Ta voix amoindrie sonne trop calme, trop naturellement triste pour moi ; quand bien même je suis étranger à tout paradis, toute misère qui n’est point démente m’impatiente. Tu devrais devenir fou, forgeron ! dis-moi, pourquoi ne deviens-tu pas fou ? Comment peux-tu endurer sans l’être ? Le ciel te hait-il à ce point que la folie ne te gagne pas ? Et que fabriquais-tu là ?\par
– Je corroyais des pointes de piques, sir, elles avaient des veines et des brèches.\par
– Et peux-tu les rendre lisses à nouveau, forgeron, après qu’elles aient été si durement malmenées ?\par
– Je le crois, sir.\par
– Et je présume, forgeron que tu peux rendre lisses toutes veines et toutes brèches, si dur que soit le métal ?\par
– Oui, sir, je crois le pouvoir, toutes veines et toute brèches sauf une.\par
– Regarde ici alors, s’écria Achab avec passion, s’appuyant des deux mains sur les épaules de Perth, regarde ici – ici – peux-tu lisser une pareille brèche, forgeron ; et il passa la main sur les sillons de son front ? Si tu le peux, forgeron, je serai trop heureux de poser ma tête sur ton enclume et de sentir s’abattre entre mes deux yeux ton plus pesant marteau. Réponds ! Peuxtu effacer cette brèche ?\par
– Ah ! c’est là l’unique, sir ! N’ai-je pas dit toutes veines et brèches sauf une ?\par
– Oui, forgeron, et c’est bien celle-là, oui, homme, elle est ineffaçable car, alors que tu n’en vois la marque que dans ma chair, elle, elle a travaillé en profondeur jusqu’à l’os de mon crâne, tout y est brèches ! Mais assez d’enfantillages, plus de crochets, ni de piques pour aujourd’hui. Regarde ! » Et il agita le sac de cuir comme s’il eût été plein de pièces d’or. Moi aussi je veux un harpon, un harpon que mille démons ne puissent rompre, Perth, et qui se plantera dans la baleine à y tenir aussi ferme que sa propre nageoire. Voilà le matériau, et il lança le réticule sur l’enclume, écoute, forgeron, ce sont là des clous de fers de chevaux de course.\par
– Des clous de fers à cheval, sir ? Eh bien, capitaine Achab, tu as là ce que nous, forgerons, pouvons travailler de meilleur et de plus robuste.\par
– Je le sais, vieil homme, ces clous se souderont ensemble comme avec la colle provenant d’os fondus de meurtriers. Vite ! forge-moi ce harpon. Forge-moi d’abord douze tiges pour son dard puis enroule et tords et bats ces douze tiges ensemble comme les fils et torons d’une ligne. Vite ! Je soufflerai le feu.\par
Lorsque les tiges furent faites, Achab les éprouva une à une en les enroulant de sa propre main sur une longue et lourde goupille de fer : « Une paille ! » dit-il en rejetant la dernière. Retravaille celle-là, Perth. »\par
Perth s’apprêtait à souder ensemble les douze tiges, lorsque Achab arrêta sa main en disant qu’il ferait lui-même son fer. Tandis qu’il frappait sur l’enclume avec un halètement rythmé, que Perth lui passait les tiges chauffées au rouge et que la forge, durement poussée, lançait une haute flamme droite, le Parsi silencieux passa, et inclinant la tête vers le feu, parut attirer sur ce travail quelque malédiction ou quelque bénédiction. Mais comme Achab levait les yeux, il s’esquiva.\par
– Pourquoi ce bouquet d’allumettes soufrées vient-il traîner par-là autour ? murmura Stubb qui regardait la scène du gaillard d’avant. Ce Parsi sent le feu comme une amorce, et luimême dégage l’odeur du bassinet brûlant d’un mousquet.\par
La tige unique du dard fut terminée enfin et Perth pour la tremper, la plongea, sifflante, dans un baquet d’eau près de lui, la vapeur bouillante gicla au visage, penché d’Achab.\par
– Voudrais-tu me marquer, Perth ? demanda-t-il en grimaçant un instant de douleur. N’aurais-je fait que forger le fer destiné à me marquer ?\par
– Non certes, à Dieu ne plaise ! et pourtant je redoute quelque chose capitaine Achab. Ce harpon n’est-il pas pour la Baleine blanche ?\par
– Pour le diable blanc ! Mais les barbelures, c’est toi qui dois les faire, homme. Voici mes rasoirs… du meilleur acier, prends-les et fais-moi des barbelures aussi tranchantes que les aiguilles de glace d’une tourmente sur la mer polaire.\par
Le vieux forgeron regarda un instant les rasoirs comme s’il eût été heureux de ne point s’en servir.\par
– Prends-les, homme, je n’en ai aucun besoin car désormais je ne me raserai, ni ne souperai, ni ne prierai jusqu’à ce que… mais allons… au travail !\par
Façonné en forme de flèche, soudé à la tige, l’acier pointa bientôt à l’extrémité du fer et, tandis que le forgeron s’apprêtait à chauffer une dernière fois les barbelures avant de les tremper, il demanda à Achab d’approcher le baquet d’eau.\par
– Non, non, pas d’eau pour cela, je lui veux une vraie trempe de la mort. Ohé, vous ! Tashtego, Queequeg, Daggoo ! Qu’en dites-vous païens ? Me donnerez-vous ce qu’il faut de sang pour tremper ce dard ? dit-il en élevant son fer. Un chœur de sombres hochements de tête lui répondit oui. On fit trois entailles dans la chair païenne et les barbelures furent alors trempées pour la Baleine blanche.\par
– « {\itshape Eho non baptizo te in nomine patris, sed in nomine diaboli ! »} hurla Achab en délire, tandis que le fer maléfique buvait de son feu le sang baptismal.\par
Puis, passant en revue les manches de rechange. Achab en choisit un de noyer blanc auquel tenait encore l’écorce et en adapta l’extrémité à la douille du harpon. Une glène de ligne fut déroulée et quelques brasses fixées au guindeau afin d’être rigoureusement tendues. Posant le pied dessus jusqu’à ce qu’elle vibrât comme une corde de harpe, se penchant ardemment et n’y voyant point de torons rompus, Achab s’écria : « Bien ! Et maintenant l’aiguilletage ! »\par
La douille du harpon fut garnie en fil de caret, il y fut frappé une aiguillette en ligne de pêche au moyen d’un tour mort, puis ce nœud fut fortement souqué dans la douille ; cette aiguillette fut alors allongée jusqu’à mi-longueur du manche puis solidement fixée avec des fils de caret de ligne. Cela fait, manche, fer et ligne – telles les trois Parques – devinrent inséparables et Achab s’en fut sombrement avec l’arme. Sa jambe d’ivoire et le manche de noyer tirèrent de chaque planche un son caverneux. Mais avant qu’il n’eût atteint sa cabine, un bruit se fit entendre, léger, surnaturel, à demi-railleur et cependant très pitoyable. Oh ! Pip, ton rire misérable, ton regard vide et inquiet pourtant, toutes tes étranges pantomimes se confondaient d’une façon chargée de sens avec la noire tragédie de ce navire de mélancolie et s’en moquaient !
\chapterclose


\chapteropen
\chapter[{CHAPITRE CXIV. Le doreur}]{CHAPITRE CXIV \\
Le doreur}\renewcommand{\leftmark}{CHAPITRE CXIV \\
Le doreur}


\chaptercont
\noindent Pénétrant toujours plus avant au cœur des parages de croisière japonais, le {\itshape Péquod} fut bientôt tout voué à la pêche. Souvent, par un temps doux et agréable, les hommes lançaient les pirogues à la poursuite des baleines, soit aux avirons, soit à la voile ou aux pagaies, pendant douze, quinze, dix-huit et vingt heures de suite et attendant tranquillement qu’elles remontent pendant soixante ou soixante-dix minutes, peu de succès pourtant couronnait leurs peines.\par
En de tels moments, sous un soleil sans défaillance, flottant tout le jour sur des vagues lentes et légères, assis dans une embarcation aussi aérienne qu’un canoë de bouleau et si intimement liée à la langueur des lames qu’elles ronronnaient contre les plats-bords comme des chats sur la pierre de l’âtre, les hommes goûtaient une quiétude rêveuse lorsqu’en contemplant la peau éclatante et lisse de l’Océan, ils oubliaient le cœur de tigre qui battait dessous et se refusaient à se souvenir que cette patte de velours cachait des griffes impitoyables.\par
En de tels moments, le vagabond dans sa baleinière éprouve envers la mer un sentiment tendre, filial, confiant, assez semblable à celui qu’il porte à la terre, il la regarde comme un parterre de fleurs et le navire qui, au loin, ne laisse voir que la pointe de ses mâts, semble se frayer sa route non à travers le roulement de vagues mais à travers l’herbe haute d’une prairie ondulante, tels les chevaux des émigrants de l’ouest dont seules pointaient les oreilles tandis que leurs corps avançaient péniblement dans une étonnante verdure.\par
Vierges vallons longuement étirés, collines bleues et douces, sur lesquels glisse un silence murmurant… et l’on pourrait presque jurer que des enfants, las de leurs jeux, sont étendus endormis dans ces solitudes quand, en un mai joyeux, sont cueillies les fleurs des bois. Dans la magie de votre humeur, rêve et réalité se rencontrent à mi-chemin, s’interpénètrent et ne forment plus qu’un.\par
Ces scènes apaisantes, pour passagères qu’elles aient été, n’étaient pas sans avoir, passagèrement, un effet apaisant sur Achab, mais si ces secrètes clefs d’or semblaient livrer l’or de son trésor secret, son haleine toutefois en ternissait l’éclat.\par
Ô herbeuses clairières ! ô printemps éternel et infini des paysages de l’âme, en vous – bien que depuis longtemps calcinés par la sécheresse mortelle de cette vie terrestre – en vous, les hommes peuvent encore se rouler comme un jeune cheval dans le trèfle matinal et, pour de rares et éphémères instants, sentir en eux la fraîche rosée de la vie éternelle. Plût à Dieu que durent ces calmes bénis ! Mais les fils de la vie emmêlés, confondus, sont tissés de chaîne et de trame : les calmes traversés d’orages, un orage pour chaque calme. La marche de la vie n’est pas un chemin qu’on ne rebrousse jamais, nous n’avançons pas selon une progression constante jusqu’à l’ultime arrêt, à travers l’enchantement innocent du premier âge, la foi naïve de l’enfance, à travers la condamnation commune du doute de l’adolescence, puis le scepticisme, l’incroyance, pour trouver enfin le repos méditatif du « Si » de l’âge d’homme. Mais, le parcours terminé, nous recommençons la ronde et sommes à nouveau des enfants, des adolescents et des « Si », éternellement. Où est le port ultime dont nous ne lèverons plus l’ancre ? Sous quelle voûte céleste extasiée navigue le monde dont les plus harassés ne se lasseront pas ? Où se cache le père \hspace{1em} de l’enfant trouvé ? Nos âmes sont les orphelines de filles-mères mortes en leur donnant le jour et le secret de notre paternité demeure dans leurs tombes, là où il nous faut aller pour l’apprendre.\par
Et ce même jour aussi, Starbuck contemplant de sa pirogue les profondeurs dorées de ce même Océan murmura :\par
« Insondable beauté, telle que jamais amant n’en vit dans les yeux de sa jeune épouse ! Ne me dis rien de tes requins aux dents serrées, ni de tes mœurs de cannibale, ni de tes rapts. Que la foi évince les faits et les chimères le souvenir, je regarde au plus profond et je crois. »\par
Et Stubb bondit dans cette même lumière dorée, tel un poisson aux brillantes écailles :\par
– « Je suis Stubb et Stubb a son histoire, mais Stubb prête ici serment qu’il a toujours été joyeux ! »
\chapterclose


\chapteropen
\chapter[{CHAPITRE CXV. Le Péquod rencontre le Célibataire}]{CHAPITRE CXV \\
\textbf{{\itshape Le} Péquod {\itshape rencontre le} Célibataire}}\renewcommand{\leftmark}{CHAPITRE CXV \\
\textbf{{\itshape Le} Péquod {\itshape rencontre le} Célibataire}}


\chaptercont
\noindent Joyeuse aussi la vision et joyeux les sons que le vent nous apporta quelques semaines après que le harpon d’Achab eut été forgé.\par
C’était le {\itshape Célibataire}, navire de Nantucket. Il venait à peine de caler son dernier baril d’huile et de verrouiller ses écoutilles pleines à craquer. Maintenant il paradait gaiement, en tenue de fête et non sans quelque gloriole, parmi les navires que de grandes distances séparaient sur les parages de pêche, avant de pointer sa proue vers son pays.\par
Les trois hommes en vigie au sommet de ses mâts portaient à leurs chapeaux d’étroites et longues banderoles rouges, aux potences de la proue, une baleinière était suspendue, la quille \hspace{1em}en bas et, pendue au beaupré, on voyait la longue mâchoire inférieure du dernier cachalot qu’ils avaient tué. Les pavillons de poupe et de beaupré, les signaux de toutes couleurs flottaient dans son gréement, deux barils de spermaceti flanquaient de part et d’autre ses trois postes de vigie et plus haut, dans les barres traversières, de petits barils de galère contenaient eux aussi de ce précieux liquide tandis qu’une lampe de bronze était fixée à la pomme de son grand mât.\par
Comme on l’apprit par la suite, le {\itshape Célibataire} avait eu un succès d’autant plus surprenant que nombre d’autres navires, croisant dans les mêmes eaux, avaient passé des mois entiers sans prendre un seul poisson, non seulement il avait vidé des barils de bœuf et de pain pour faire de la place à l’huile tellement plus précieuse mais encore il avait acheté des tonneaux aux vaisseaux rencontrés et les avait arrimés le long du pont et dans les cabines du capitaine et des officiers. La table même de la cabine avait été réduite en petit bois et les repas y étaient servis sur une barrique d’huile fixée au plancher. Au gaillard d’avant les matelots étaient allés jusqu’à remplir leurs coffres qu’ils avaient calfatés et goudronnés ; on ajoutait en plaisantant que le coq avait mis un couvercle à sa plus grande marmite et l’avait remplie, que le garçon avait mis un tampon à sa cafetière et l’avait remplie, que les harponneurs avaient rempli les douilles de leurs fers, que tout en vérité était rempli de spermaceti hormis les poches de pantalon du capitaine dans lesquelles il se réservait d’enfoncer les mains avec complaisance, pour témoigner de sa pleine satisfaction.\par
Tandis que cet heureux navire de la chance se rapprochait du sombre {\itshape Péquod}, sur son gaillard d’avant résonnait le battement barbare d’énormes tambours. Lorsqu’il fut plus près encore, on vit que les chaudières géantes avaient été recouvertes de la peau de l’estomac tendu du poisson noir, semblable à du parchemin, et qu’elles grondaient sous les poings fermés des hommes. Sur le gaillard d’arrière, les seconds et les harponneurs dansaient avec les filles olivâtres qui s’étaient laissé enlever par eux aux îles polynésiennes cependant que, suspendus dans une pirogue décorée, solidement amarrée entre le mât de misaine et le grand mât, trois nègres de Long Island présidaient à cette gigue joyeuse avec d’étincelants archets en ivoire de cachalot. En même temps, d’autres membres de l’équipage s’affairaient à grand fracas à détruire la maçonnerie des fourneaux d’où avaient été retirées les chaudières. On aurait pu croire qu’ils tiraient bas une bastille maudite tant étaient sauvages les hurlements qu’ils poussaient tandis que volaient pardessus bord les briques et le mortier devenus inutiles.\par
Seigneur et maître du spectacle, le capitaine se tenait debout bien droit sur la demi-dunette afin que se déroulât pleinement sous ses yeux cette scène qui semblait avoir été montée pour son seul divertissement personnel.\par
Achab, lui aussi, se tenait debout sur son gaillard d’arrière, hirsute et noir, avec une tristesse opiniâtre, et les deux navires se croisant – l’un tout réjouissance en quittant le passé, l’autre tout pressentiment de malheur face à l’avenir – les deux capitaines incarnèrent ce contraste frappant.\par
– Venez à bord, venez à bord ! cria le joyeux commandant du {\itshape Célibataire} en élevant un verre et une bouteille.\par
– As-tu vu la Baleine blanche ? grinça Achab en réponse.\par
– Non, seulement entendu parler, mais n’en crois pas un mot, dit l’autre avec bonne humeur. Venez à bord !\par
– Tu es trop diablement gai. Va ton chemin. As-tu perdu des hommes ?\par
– Pas qui vaillent la peine d’en parler… deux Islandais, c’est tout. Mais venez à bord, vieux frère, venez. J’aurai vite fait d’effacer cette ombre à votre front. Venez, voulez-vous, c’est fête, un navire plein et en route pour le pays.\par
– La familiarité des imbéciles est étonnante ! marmonna Achab, puis à voix haute : Tu dis que tes cales sont pleines et que tu rentres au pays, appelle-moi un navire vide en partance. Aussi va ton chemin, j’irai le mien. Ohé, à l’avant. Tout dessus et au plus près !\par
Ainsi tandis qu’un navire était emporté, guilleret, vent arrière, l’autre, têtu, luttait contre la brise, et ils se séparèrent. L’équipage du {\itshape Péquod} regarda longuement, gravement le \hspace{1em}{\itshape Céli}{\itshape bataire} qui s’éloignait, mais les hommes de ce dernier ne détournèrent pas le regard de leur bacchanale. Cependant Achab, penché sur la lisse de couronnement, les yeux fixés sur le navire rentrant au pays, sortit de sa poche une petite fiole et, regardant alternativement le vaisseau et le flacon, parut réunir deux souvenirs lointains, car celui-ci contenait du sable de Nantucket.
\chapterclose


\chapteropen
\chapter[{CHAPITRE CXVI. Le cachalot agonisant}]{CHAPITRE CXVI \\
Le cachalot agonisant}\renewcommand{\leftmark}{CHAPITRE CXVI \\
Le cachalot agonisant}


\chaptercont
\noindent Tandis qu’on est encalminé et qu’un navire plus favorisé par la fortune fait voile à proximité, il n’est pas rare qu’on bénéficie de cette risée et nous la sentîmes gonfler joyeusement nos voiles. Il parut en aller ainsi du {\itshape Péquod.} Le lendemain de sa rencontre avec le {\itshape Célibataire}, des baleines furent signalées et quatre d’entre elles furent tuées dont une par Achab.\par
Il était tard dans l’après-midi lorsque le pourpre combat eut pris fin et que, flottant dans la beauté de la mer et du ciel du couchant, le soleil et le cachalot silencieusement agonisaient ensemble. Dans l’air couleur de rose montèrent alors une si plaintive douceur, et une telle guirlande d’oraisons qu’on eût dit que, du fond des couvents perdus dans les vertes vallées des Philippines, la brise de terre espagnole étourdiment avait pris la mer, emportant ces hymnes du soir.\par
Apaisé à nouveau, mais seulement pour s’ouvrir à une tristesse plus profonde, Achab, qui s’était écarté du cachalot, regardait avec une attention soutenue s’affaiblir ses soubresauts, depuis sa pirogue à présent tranquille. Tous les cachalots mourants offrent ce même et étrange spectacle d’un être se tournant vers le soleil pour expirer et dans un soir si calme cette vision revêtait pour Achab un caractère d’émerveillement inconnu jusqu’alors.\par
– Il tourne et se retourne vers lui – combien lentement mais combien obstinément son front lui rend hommage \hspace{1em}et l’invoque dans les dernières convulsions de la mort. Lui aussi adore le feu, fidèle et noble vassal du soleil ! Oh ! que ces yeux pleins de préjugés voient ces spectacles justifiant ces préjugés. Vois, prisonnier de lointaines eaux, au-delà du moindre bourdonnement de bonheur ou de malheur humain, dans ces mers sincères et justes où il n’est point de rocher où puissent s’inscrire les traditions, où, depuis des âges vieux comme la Chine les vagues muettes ont roulé sans que non plus il leur fût parlé, telles les étoiles qui brillent sur la source inconnue du Niger, ici aussi la vie meurt, pleine de foi, tournée vers le soleil, mais vois ! à peine la mort a-t-elle fait son œuvre qu’elle retourne le corps et l’oriente d’un autre côté.\par
Oh ! toi, moitié de la nature sombre comme une Indienne, qui t’es construit un trône solitaire avec les os des noyés, quelque part au cœur de ces mers sans verdures, tu es la reine infidèle qui ne me parles que trop de vérité dans la vaste étreinte du typhon et l’ensevelissement silencieux du calme qui le suit. Et ce n’est pas non plus sans me servir de leçon que ta baleine tourne vers le soleil sa tête mourante pour se retourner encore.\par
Oh ! flancs puissants trois fois cerclés de fer ! Oh ! souffle d’arc-en-ciel qui monte vers les nues ! et celui qui peine et celui qui s’élance sont également vains ! En vain, ô baleine, quêtes-tu l’intercession du soleil, là-haut, fécondateur qui appelle à la vie, mais ne la rends pas. Mais toi, moitié sombre, tu me berces d’une foi plus noire et plus fière. Tout ce que tu as confondu indiciblement flotte ici sous moi, je suis soutenu au-dessus de la surface par les souffles d’êtres autrefois vivants, exhalés en tant que souffles, mais devenus eau à présent.\par
C’est pourquoi je te salue, je te salue à jamais, ô Mer dont les flots éternels sont le seul repos de l’oiseau sauvage. Né de la terre, allaité cependant par l’Océan, les collines et les vallons m’ont servi de mère mais la boule pourtant est ma sœur adoptive.
\chapterclose


\chapteropen
\chapter[{CHAPITRE CXVII. La veillée de la baleine}]{CHAPITRE CXVII \\
La veillée de la baleine}\renewcommand{\leftmark}{CHAPITRE CXVII \\
La veillée de la baleine}


\chaptercont
\noindent Les quatre baleines tuées ce soir-là étaient mortes à grande distance les unes des autres, l’une bien loin au vent, l’une plus proche sous le vent, l’une des deux autres à l’avant, l’autre à l’arrière. Ces trois dernières furent amenées au navire avant la tombée de la nuit, mais celle qui se trouvait au vent ne put être atteinte avant le matin et la pirogue qui l’avait tuée resta à son côté toute la nuit ; c’était la baleinière d’Achab.\par
Le pavillon qui servait à la repérer fut fiché droit dans l’évent du cachalot et la lanterne qui y était suspendue jetait au loin une lueur trouble et clignotante sur le dos noir et luisant des vagues nocturnes qui se frottaient doucement comme un paisible ressac sur une plage contre le vaste flanc de la baleine.\par
Achab et ses hommes semblaient dormir, hormis le Parsi qui, assis accroupi à l’avant, suivait le jeu fantomatique des requins autour de la baleine tandis que leurs queues frappaient les légers bordages de cèdre. De frissonnantes lamentations couraient dans l’air, pareilles à celles des armées des spectres impardonnés de Gomorrhe sur le Lac Asphaltites.\par
Tiré brusquement de son sommeil, Achab vit le Parsi face à face ; pris dans le cercle des ténèbres de la nuit, ils semblaient les seuls survivants d’un déluge.\par
– J’ai refait le même rêve, dit-il.\par
– Des corbillards ? Ne t’ai-je pas dit, vieillard, que tu ne pourras avoir ni corbillard ni cercueil ?\par
– Et qui a un corbillard qui meurt en mer ?\par
– Mais je te l’ai dit, vieillard, avant que tu puisses mourir en ce voyage, tu dois, en vérité, voir deux corbillards sur la mer, le premier n’aura pas été fait par des mains mortelles et le bois tangible du second proviendra des forêts d’Amérique.\par
– Oui, oui ! étrange spectacle que celui-là, Parsi, un corbillard et ses panaches flottant sur l’Océan, les cordons du poêle tenus par les vagues. Ah ! c’est un spectacle que nous ne verrons pas de sitôt.\par
– Crois-le ou non, vieillard, tu ne saurais mourir que tu ne l’aies vu.\par
– Et en ce qui te concerne, que disait-il ce songe ?\par
– À l’apparition du second corbillard, je serai encore devant toi, comme ton pilote.\par
– Ainsi, quand tu seras parti le premier, si cela arrive jamais, avant que je puisse suivre, tu dois m’apparaître encore, pour encore me conduire ? N’est-ce pas ce qui était dit ? Alors, si je crois tout ce que tu dis, oh ! mon pilote, voilà deux preuves que je tuerai Moby Dick et que je survivrai.\par
– En voici une autre, vieillard, dit le Parsi, tandis que ses yeux s’allumaient comme des lucioles dans les ténèbres, seul le chanvre peut te tuer.\par
– Le gibet, veux-tu dire. Alors je suis immortel et sur terre et sur mer, dit Achab avec un rire de dérision. Immortel et sur terre et sur mer !\par
Ils se turent en même temps. L’aube grise se levait l’équipage assoupi se redressa et avant midi le cachalot mort fut remorqué jusqu’au navire.
\chapterclose


\chapteropen
\chapter[{CHAPITRE CXVIII. Le sextant}]{CHAPITRE CXVIII \\
Le sextant}\renewcommand{\leftmark}{CHAPITRE CXVIII \\
Le sextant}


\chaptercont
\noindent La saison approchait enfin où il faudrait regagner la Ligne et chaque jour, lorsque Achab, sortant de sa cabine, levait les yeux, le timonier vigilant tenait ostensiblement la barre et les matelots impatients couraient aux bras de vergues et s’y tenaient, le regard fixé sur le doublon cloué, ardents à attendre l’ordre de mettre le cap sur l’équateur. En temps voulu, cet ordre vint. Il était bientôt midi et Achab, assis à l’avant de sa baleinière haut hissée sur ses potences, s’apprêtait, comme à l’accoutumée, à faire son observation quotidienne du soleil.\par
Les jours d’été, dans la mer du Japon, sont pareils à la splendeur d’un fleuve de lumière. Ce fixe et tranchant soleil japonais semble être le foyer incandescent de la loupe unie et incommensurable de l’Océan. Point de nuages dans la laque du ciel, à l’horizon mouvant la nudité de ce rayonnement inflexible est l’image de la gloire insupportable du trône de Dieu. Heureusement, le sextant d’Achab avait des verres fumés pour contempler cet incendie solaire. Balancé au rythme du bateau, l’œil à son instrument astronomique, il resta quelques instants dans cette position à guetter le moment précis où le soleil atteindrait son méridien. Tandis qu’il était ainsi absorbé, le Parsi, audessous de lui, était agenouillé sur le pont, la tête levée comme celle d’Achab pour regarder avec lui le même soleil, les paupières à demi baissées et son sauvage visage adouci d’une terrestre impassibilité. L’observation faite, le crayon sur sa jambe d’ivoire, Achab eut tôt fait de calculer quelle devait être en ce moment précis sa latitude, puis une rêverie passagère s’empara de lui et levant à nouveau les yeux vers le soleil il se murmura :\par
« Ô toi, amer ! Toi, haut et puissant pilote ! Tu me dis en vérité où je suis, mais peux-tu si peu que ce soit me suggérer où je serai ? Peux-tu me dire où se trouve en ce moment un autre être que moi ? Où est Moby Dick ? Tu le regardes sans doute en cet instant. Mes yeux plongent dans l’œil même qui le voit maintenant, oui, dans cet œil qui voit maintenant toutes choses et même l’inconnu, de l’autre côté de toi, ô soleil ! »\par
Puis tournant son regard vers son sextant et manipulant l’un après l’autre ses dispositifs cabalistiques, il médita encore et ajouta : « Absurde jouet ! hochet pour amiraux hautains, pour commodores et capitaines, le monde se gausse de ton astuce et de ton pouvoir, mais après tout, tu ne peux rien de plus qu’indiquer le pauvre, le pitoyable point de la vaste planète où toi-même tu te trouves avec la main qui te tient, et rien, pas un iota de plus ! Tu ne saurais dire où seront, demain à midi, une goutte d’eau ou un grain de sable et pourtant ton impuissance insulte au soleil ! Science ! Maudit sois-tu, inutile jouet, et maudites soient toutes choses qui font lever les yeux de l’homme vers ce ciel dont l’éclat de vie ne peut que le brûler, comme ta lumière, ô soleil, brûle ces vieillissantes prunelles ! Le regard de l’homme, la nature l’a mis au niveau de l’horizon et au sommet de sa tête comme si Dieu eût voulu qu’il contemplât son firmament. Maudit sois-tu, sextant ! » Et le jetant sur le pont, il ajouta : « Je ne te laisserai plus le soin de me guider sur mon terrestre chemin, le compas à niveau d’homme, et l’estime faite à son niveau par le loch et la ligne me conduiront désormais et me diront où je suis sur la mer. » Et Achab sauta de la pirogue sur le pont : « Oui, je te piétine, chétif objet qui pointe faiblement vers les hauteurs, je te brise et je te détruis ! »\par
Tandis que le frénétique vieillard parlait ainsi et écrasait l’instrument, et de son pied vivant et de son pilon, un \hspace{1em}triomphe sarcastique qui paraissait destiné à Achab un désespoir fataliste qui semblait s’adresser à lui-même, passèrent sur le visage immobile et muet du Parsi. Il se leva et s’esquiva sans être vu cependant que, frappés de terreur par l’aspect de leur capitaine, les hommes se serraient sur le gaillard d’avant, jusqu’au moment où Achab arpentant anxieusement le pont cria à voix forte :\par
– Aux vergues ! Barre dessus ! Brassez carré !\par
En un instant, les vergues pivotèrent et le navire évita, ses trois mâts gracieux et robustes sur sa longue coque ridée semblèrent les trois Horaces pirouettant sur une unique monture.\par
Debout entre les apôtres, Starbuck épiait l’embardée du navire et celle d’Achab tandis qu’il titubait sur le pont.\par
– Je me suis assis devant un feu de charbon intense et j’en ai contemplé l’ardeur, et ses flammes tourmentées par la vie, et je l’ai vu décliner enfin, toujours plus bas jusqu’à n’être plus qu’une poussière inerte. Vieil homme des océans ! De toute cette vie embrasée qui est tienne que restera-t-il pour finir sinon un petit tas de cendres !\par
– Oui, intervint Stubb, mais des cendres de houille, notezle monsieur Starbuck, non des cendres de charbon ordinaire, oui, oui, j’ai entendu Achab murmurer : « Quelqu’un a mis entre mes vieilles mains ces cartes, jurant que je devais les jouer et non point d’autres. Et que je sois damné, Achab, mais tu as agi justement. Vivre le jeu et y mourir ! »
\chapterclose


\chapteropen
\chapter[{CHAPITRE CXIX. Les bougies}]{CHAPITRE CXIX \\
Les bougies}\renewcommand{\leftmark}{CHAPITRE CXIX \\
Les bougies}


\chaptercont
\noindent Aux climats les plus chauds les crocs les plus cruels : le tigre du Bengale se tapit dans les futaies capiteuses aux verdures immuables. Les cieux les plus éblouissants bercent les foudres les plus meurtrières, la somptueuse Cuba connaît des cyclones qui jamais ne balaient nos fades terres nordiques. C’est dans les resplendissantes mers du Japon que le marin rencontre le typhon, la plus néfaste des tempêtes. Il éclatera parfois dans un ciel sans nuages comme une bombe explose sur une ville endormie et médusée.\par
Vers le soir de ce jour-là, le {\itshape Péquod} eut ses voiles arrachées et dut combattre, à sec de toile, un typhon qui l’attaquait de front. Lorsque vint la nuit, le ciel et la mer rugirent et furent déchirés par la foudre, et l’illumination des éclairs révélait ses mâts désemparés balançant les haillons que la première fureur de la tempête leur avait laissés afin de s’en divertir plus tard.\par
Sur le gaillard d’arrière, Starbuck se tenait à un hauban pour voir, à chaque fulguration, quel nouveau désastre avait pu frapper, là-haut, le gréement, tandis que Flask et Stubb veillaient à ce que les hommes hissent plus haut les baleinières et les amarrent plus solidement. Mais tous leurs efforts furent vains. Bien que montée au sommet de la potence, la pirogue au vent (celle d’Achab) n’échappa point. Un énorme paquet de mer déferlant haut contre le flanc chancelant du navire défonça le fond de la baleinière puis l’abandonna dégouttant comme un tamis.\par
– Vilain travail, vilain travail ! monsieur Starbuck, dit Stubb, devant ces dégâts, mais la mer ne fait que selon son bon plaisir, et Stubb le tout premier ne peut lutter contre elle. Vous voyez, monsieur Starbuck, une vague fait le tour du monde pour prendre son élan avant de sauter, aussi quelle détente ! Tandis que moi, je n’ai que la largeur du pont pour l’affronter. Mais tant pis, tout cela n’est qu’une plaisanterie, c’est ce que dit la vieille chanson, {\itshape (il chante)}\par


\begin{verse}
Oh ! gaillarde est la tempête et farceur le cachalot\\
un panache sa queue\\
\end{verse}

\noindent C’est un gars si drôle, blagueur, badin, joueur, plaisantin que l’Océan !\par


\begin{verse}
Et vole l’embrun\\
Ce n’est qu’une pichenette d’écume quand il se tourne et se retourne\\
\end{verse}

\noindent C’est un gars si drôle, blagueur, badin, joueur, plaisantin que l’Océan !\par


\begin{verse}
La foudre fend les bateaux mais il claque seulement des lèvres\\
pour goûter son petit verre\\
\end{verse}

\noindent C’est un gars si drôle, blagueur, badin, joueur, plaisantin que l’Océan !\par
– Baste, Stubb ! Laisse chanter le typhon, laisse-le jouer de la harpe dans le gréement, mais si tu es un homme brave, taistoi, dit Starbuck.\par
– Mais je ne suis pas brave, jamais dit que je l’étais, je suis un lâche et je chante pour me donner du courage. Je vais vous dire, monsieur Starbuck, il n’y a pas d’autre moyen au monde de m’empêcher de chanter que de me trancher la gorge. Cela fait, dix chances contre une, que je vous chante le « Gloria Patri » en guise d’envoi.\par
– Insensé ! Regarde avec mes yeux si tu n’en as point toimême.\par
– Quoi ! Comment pourriez-vous y voir mieux qu’un autre, si sot soit-il, dans une nuit aussi sombre ?\par
– Là ! dit Starbuck en saisissant l’épaule de Stubb et en désignant de la main le côté du vent, n’as-tu pas remarqué que la tempête vient de l’est, la direction même qu’Achab doit prendre pour chasser Moby Dick ? Le cap même qu’il a pris aujourd’hui à midi. Maintenant regarde sa baleinière, où a-t-elle été défoncée ? À l’arrière, où il a coutume de se tenir, c’est sa position qui est détruite, homme ! Maintenant, saute par-dessus bord et chante tout ton saoul s’il le faut absolument.\par
– Je ne comprends pas la moitié de ce que vous dites qu’est-ce qui se passe ?\par
– Oui, oui, poursuivit Starbuck, monologuant soudain, sans prêter attention à la question de Stubb. Le cap de Bonne- Espérance est le plus court chemin pour Nantucket. La tempête qui nous frappe à présent pour nous détruire, nous pouvons l’utiliser comme un vent favorable pour nous ramener chez nous. Là-bas au vent, ce sont les ténèbres de la perdition, mais sous le vent, vers le pays, je vois s’allumer une lueur, et ce n’est pas la foudre.\par
En cet instant, dans un de ces intervalles d’obscurité profonde qui suivaient les éclairs, il entendit une voix près de lui et en même temps le roulement d’un feu de salve du tonnerre.\par
– Qui est là ?\par
– Le vieux Tonnerre ! dit Achab, tâtonnant le long du pavois jusqu’à son trou de tarière, et trouvant brusquement son chemin illuminé par les lances brisées de la foudre.\par
Un clocher porte un paratonnerre destiné à conduire dans la terre les dangereuses décharges, de même certains navires en ont au sommet de chaque mât pour les diriger vers l’eau. Mais, sur un vaisseau, le conducteur doit descendre à une grande profondeur, afin de n’avoir aucun contact avec la coque ; de plus, s’il était constamment à la remorque, il entraînerait de nombreux accidents, sans compter qu’il se mêlerait aux manœuvres et entraverait la bonne marche du navire ; c’est pourquoi les extrémités des paratonnerres de bord ne plongent pas toujours dans l’eau mais consistent en légers maillons qu’on rajoute aux chaînes extérieures ou qu’on jette à la mer lorsque besoin en est.\par
– Les paratonnerres ! les paratonnerres ! cria Starbuck à l’équipage, alerté par les flambeaux éblouissants qui avaient éclairé le chemin d’Achab. Sont-ils par-dessus bord ? Jetez-les à l’avant et à l’arrière ! Vite !\par
– Non ! s’écria Achab. Jouons franc jeu bien que nous soyons les plus faibles. Je participerais bien à l’érection des paratonnerres sur les Himalayas et sur les Andes pour mettre le monde entier en sécurité. Mais pas de privilèges ! Pas de paratonnerres ici, sir !\par
– Regardez là-haut ! Le Corpo Santo ! le Corpo Santo, dit Starbuck.\par
Sur tous les bouts de vergues un feu pâle s’était posé et au trident de chaque paratonnerre pointaient trois fines flammes blanches, chacun des trois grands mâts brûlait silencieusement dans cet air soufré, comme trois cierges gigantesques devant un autel.\par
– Le diable emporte cette maudite pirogue ! s’exclama Stubb à ce moment-là, comme un paquet de mer soulevait sa baleinière dont le plat-bord lui écrasa la main tandis qu’il y frappait une amarre. Maudite soit-elle ! Mais tandis qu’il reculait sur le pont, ses yeux levés rencontrèrent les flammes et, changeant de ton, il ajouta : Que le Corpo Santo ait pitié de nous tous !\par
Les matelots jurent comme ils respirent, ils jureront aussi bien dans les calmes extatiques qu’entre les dents de la tempête, ils enverront imprécations et malédictions du bout des vergues de hune alors même qu’ils seront le plus rudement secoués sur une mer bouillonnante, mais au cours de mes voyages j’ai rarement entendu s’échapper un gros mot lorsque le doigt brûlant de Dieu se posait sur le navire, quand son Mané, Thécel, Pharès se tissait dans les haubans et les cordages.\par
Les hommes ensorcelés n’échangèrent que peu de mots tandis que cette pâleur brûlait au sommet des mâts, ils se serraient étroitement sur le gaillard d’avant et leurs yeux brillaient dans cette phosphorescence blême comme une lointaine constellation. Se découpant contre cette lumière spectrale, la silhouette du nègre géant Daggoo triplait sa taille réelle et on eût dit qu’il était le nuage noir porteur de la foudre. Les lèvres entrouvertes de Tashtego diraient ses blanches dents de requin luisant étrangement comme si le Corpo Santo s’y était également posé, cependant que, sous cette lumière surnaturelle, les tatouages de Queequeg flambaient d’une satanique flamme bleue.\par
La scène s’effaça en même temps que disparut, là-haut cette pâle lueur, le {\itshape Péquod} et ses hommes furent à nouveau enveloppés dans un drap de catafalque. Un moment s’écoula puis Starbuck, qui se dirigeait vers l’avant se heurta à quelqu’un. C’était Stubb.\par
– Qu’en penses-tu à présent, homme ? J’ai entendu ton appel, ce n’était pas le même que dans la chanson.\par
– Non, non, ce n’était pas le même. J’ai demandé au Corpo Santo d’avoir pitié de nous tous et j’espère qu’il en aura, en effet, pitié. Mais ne doit-il l’avoir que des longues figures ? N’a-t-il pas d’entrailles devant un rire ? Regardez, monsieur Starbuck, mais il fait trop noir pour regarder, alors écoutez plutôt, je prends ces flammes au sommet des mâts pour un heureux présage, car ceux-ci sont enracinés dans une cale qui va être bourrée de spermaceti, de sorte qu’il va monter dans les mâts comme la sève dans un arbre. Oui, nos trois mâts seront comme trois cierges de spermaceti… ce que nous avons vu était une promesse de chance.\par
Au même instant, Starbuck vit le visage de Stubb commencer à luire dans l’ombre. Levant les yeux, il s’écria : « Vois, vois ! » Les hautes flammes effilées étaient réapparues et leur pâleur avait un éclat plus surnaturel encore.\par
– Que le Corpo Santo ait pitié de nous tous, répéta Stubb.\par
Au pied du grand mât, droit sous le doublon et sous la flamme, devant Achab, mais sa tête inclinée détournée de lui, le Parsi était agenouillé, cependant que tout près, dans l’arceau du gréement qui les surplombait, un groupe d’hommes qui venaient d’assurer un espar, figés par la lueur, se serraient en grappe suspendue, pareils à un essaim de guêpes engourdies alourdissant une ramille au verger. Dans diverses postures saisies par un enchantement, tels les squelettes d’Herculanum \hspace{1em}figés debout, en marche ou à la course, le reste de l’équipage avait pris racine sur le pont, tous les yeux levés.\par
– Oui, oui, hommes ! s’écria Achab. Regardez là-haut, regardez bien ! La flamme blanche est celle qui éclaire la route menant à la Baleine blanche ! Passez-moi les chaînons du grand mât que je leur tâte le pouls et que le mien batte contre le fer, sang contre feu ! Ainsi !\par
Puis se tournant, serrant fortement le dernier chaînon dans sa main gauche, il posa le pied sur le Parsi, et les yeux rivés au sommet des mâts, le bras droit levé au-dessus de la tête, il se tint roide devant le trident élevé qui portait une trinité de flamme.\par
– Oh ! toi glorieux esprit du feu clair que j’ai adoré sur ces mers, comme un Parsi, jusqu’à ce que dans l’acte sacramentel tu me brûles si profondément que j’en porte aujourd’hui encore la marque, je te connais à présent, glorieux esprit, et je sais qu’il faut te défier pour te rendre hommage. Tu n’as d’indulgence ni pour l’amour ni pour la vénération. Devant la haine même tu ne peux que tuer et tous sont tués. Ce n’est pas ici un insensé téméraire qui t’affronte. Je reconnais ta puissance sans demeure et sans voix, mais jusqu’au dernier souffle de ma vie catastrophée, je contesterai en moi ta domination absolue et universelle. Au centre de cette personnification d’une force impersonnelle, une personnalité ici se dresse. Aussi d’où que je vienne, où que j’aille, tandis que je vis ici-bas, une personnalité royale est en moi sensible à ses droits royaux. Mais la guerre est souffrance et la haine tristesse. Viens sous la forme du plus humble amour et agenouillé je t’embrasserai, mais sous ta forme la plus haute viens simplement comme puissance divine, et bien que tu lances des flottes de mondes lourdement frétés, une indifférence est au cœur de tout cela. Oh ! toi, glorieux esprit, ton feu m’a engendré et en vrai fils du feu, j’exhale vers toi mon souffle.\par
(Éclairs soudains et répétés, les neuf flammes triplent de hauteur, comme tous les autres, Achab ferme les yeux et appuie fortement sa main droite contre ses paupières.) \par
{\itshape –} J’ai reconnu, ne l’ai-je pas dit, ta puissance sans demeure et sans voix. Ce ne me fut pas arraché par la force, et je ne renie pas en cet instant ces liens. Tu peux m’aveugler, mais je peux aller à tâtons. Tu peux me consumer, mais je peux être cendres. Reçois l’hommage de ces pauvres yeux et l’écran de ces mains. Je n’en voudrais pas.\par
– La foudre me traverse le crâne, mes yeux sont douloureux, mon cerveau martelé me semble le fruit d’une décollation qui roule sur un sol frappé de stupeur. Oh ! Aveugle je te parlerai toutefois encore. Tout lumière que tu sois, tu as jailli des ténèbres, mais je suis les ténèbres qui jaillissent de la lumière, de toi ! Les javelots ne tombent plus, ouvres les yeux, y vois-tu ou non ? Là-haut brûlent les flammes. Oh ! toi, magnanime ! Je glorifie à présent de mon ascendance. Tu es mon père ardent et je ne connais pas ma douce mère. Oh ! cruel, qu’as-tu fait d’elle ? C’est là qu’est mon énigme, mais la tienne est plus grande. Tu ne sais d’où tu viens puisque tu te dis non engendré, tu ne connais certes pas ton origine puisque tu te dis sans commencement. Ce que tu ignores de toi-même je le sais de moi, ô tout-puissant ! Au-delà de toi, glorieux esprit, il y a une force devant laquelle ton éternité n’est que temps, ton pouvoir créateur mécanique. À travers ta flamme mes yeux brûlés l’aperçoivent faiblement. Oh ! toi, feu enfant trouvé, toi immémorial ermite, toi aussi tu es devant ton énigme incommunicable, ta douleur non partagée. Auprès de toi encore dans une hautaine angoisse, je devine un père. Bondis et lèche le ciel, je bondis avec toi, je brûle avec toi, je voudrais être fondu en toi, je t’adore en te défiant !\par
– La baleinière ! la baleinière ! regarde ta baleinière, vieil homme ! s’écria Starbuck.\par
Le harpon d’Achab qu’avait forgé Perth demeurait aiguilleté dans sa fourche bien en évidence, de sorte qu’il faisait saillie à l’avant de la baleinière, dont la lame avait arraché le fond, avait enlevé son fourreau de cuir et des pointes acérées s’élevait maintenant la lance bifide d’une flamme pâle. Tandis que le harpon brûlait silencieusement tel une langue de serpent, Starbuck saisit Achab par le bras :\par
– Dieu, Dieu est contre toi, vieillard. Renonce ! C’est un mauvais voyage ! mal commencé, mal poursuivi ; pendant qu’il est encore temps, laisse-moi brasser carré, vieillard, et me servir du vent pour nous ramener au pays, pour un voyage meilleur que celui-ci.\par
En entendant Starbuck, l’équipage, frappé de terreur, courut aux bras, bien que toute voile eût été emportée. Pendant un instant, ils firent leurs les pensées du second atterré et poussèrent à demi le cri de la mutinerie. Mais jetant sur le pont les chaînes cliquetantes des paratonnerres, et empoignant le harpon qui flambait, Achab le brandit au milieu d’eux, jurant de transpercer le premier matelot qui toucherait seulement une manœuvre. Pétrifiés par son aspect, reculant devant le dard embrasé qu’il tenait, les hommes battirent en retraite avec épouvante, et Achab éleva à nouveau la voix :\par
– Tous vos serments de chasser la Baleine blanche sont le même lien qui est mien, et cœur, âme, corps, poumons et vie, le vieil Achab est lié. Et afin que vous sachiez selon quelle humeur bat ce cœur, voyez tous, je souffle ainsi la dernière crainte !\par
Et d’un seul souffle il éteignit la flamme.\par
Lorsque l’ouragan balaie la plaine, les hommes fuient voisinage d’un orme géant et solitaire, dont la hauteur et la puissance ne sont que gages de danger, qu’attirance pour la \hspace{1em}foudre. De même, aux derniers mots d’Achab, les matelots le fuirent dans la terreur et le désarroi.
\chapterclose


\chapteropen
\chapter[{CHAPITRE CXX. Le pont vers la fin du premier quart de nuit}]{CHAPITRE CXX \\
Le pont vers la fin du premier quart de nuit}\renewcommand{\leftmark}{CHAPITRE CXX \\
Le pont vers la fin du premier quart de nuit}


\chaptercont
\noindent (Achab debout près de la barre. Starbuck s’approche de lui.) \par
– Il faut amener la vergue de grand hunier, sir. Le collier s’est relâché et la balancine sous le vent a des torons coupés. Dois-je l’amener, sir ?\par
– N’amenez rien, aiguilletez. Si j’avais des bouts-dehors de bonnettes je les ferais hisser à présent.\par
– Sir ? – au nom du ciel ! – Sir ?\par
– Eh bien ?\par
– Les ancres fatiguent, sir. Dois-je les faire hisser à bord ?\par
– N’amenez rien, ne hissez rien, mais amarrez tout. Le vent fraîchit, mais il n’a pas encore atteint mon plateau élevé. Vite, veillez-y. Par les mâts et les quilles ! il me prend pour le patron bossu de quelque barque de pêche. Amener ma vergue de grand hunier ! Marins d’eau bénite ! Les pommes de mât les plus hautes sont faites pour les vents les plus sauvages, et la pomme de mât de mon cerveau fend maintenant l’écume des nuages. L’amènerai-je ? Oh ! il n’y a que les lâches pour amener les voiles de sa pensée au moment de la tempête. Quel boucan là- haut ! Je le considérerais même comme sublime, si je ne savais que la colique est une maladie bruyante. Oh ! prenez une purge, prenez une purge !
\chapterclose


\chapteropen
\chapter[{CHAPITRE CXXI. Minuit. Les pavois du gaillard d’avant}]{CHAPITRE CXXI \\
Minuit. Les pavois du gaillard d’avant}\renewcommand{\leftmark}{CHAPITRE CXXI \\
Minuit. Les pavois du gaillard d’avant}


\chaptercont
\noindent (Stubb et Flask, à cheval sur les rambardes, assujettissent les ancres suspendues là avec des aiguillettes supplémentaires.) \par
– Non, Stubb, vous pouvez taper sur ce nœud autant que vous voudrez, jamais vous ne ferez entrer dans ma tête de la même manière ce que vous venez de dire. Il y a peu de temps vous me disiez juste le contraire ? Ne m’avez-vous pas dit une fois que tout navire commandé par Achab devrait payer une surprime d’assurance tout comme s’il était chargé de barils de poudre à l’arrière et d’allumettes à l’avant ? N’ajoutez rien ! N’est-ce pas ce que vous m’avez dit ?\par
– Eh bien, en admettant que je l’aie dit ? Et puis après… J’ai presque changé de peau depuis alors, pourquoi pas d’idée ? En outre, en supposant que nous soyons chargés de barils de poudre à l’arrière et d’allumettes à l’avant, comment diable prendraient-elles feu sous pareil déluge ? Mon petit bonhomme, vous avez les cheveux joliment rouges, mais vous seriez inenflammable actuellement. Secouez-vous ! Vous êtes le Verseau, ou le porteur d’eau, Flask, on pourrait remplir des cruches au col de votre caban. Ne comprenez-vous pas dès lors que, pour des risques plus grands, les compagnies d’assurances maritimes veulent des garanties plus grandes ? Voici les bouches d’incendie, Flask. Mais ne dites rien encore et je répondrai à l’autre question. Mais tout d’abord enlevez votre jambe de dessus la croisée d’ancre, afin que je puisse passer la saisine, et maintenant écoutez-moi. Y a-t-il une différence majeure \hspace{1em} entre tenir les chaînons du paratonnerre d’un mât dans la tempête et se tenir tout près d’un mât dépourvu de tout paratonnerre lors d’une même tempête ? Ne comprenez-vous pas, tête de bois, que celui qui tient les chaînons ne peut être frappé à moins que le mât ne soit frappé d’abord ? Dès lors, de quoi parlez-vous ? Il n’y a pas un navire sur cent qui ait des paratonnerres et Achab – oui, homme, et nous tous avec lui – nous ne courrions pas un danger plus grand, à mon humble avis, que les équipages des dix mille navires qui sont sur mer en ce moment. Allons, allons, Cabrion, vous voudriez peut-être que chaque homme se promène sur la terre avec un petit paratonnerre fiché dans son chapeau comme la plume piquée dans la coiffure d’un officier de milice et traînant derrière lui comme une ceinture. Pourquoi ne pouvez-vous pas être raisonnable, Flask ? C’est facile d’être raisonnable. Pourquoi, alors ne l’êtes-vous pas ? Tout homme n’ayant que la moitié d’un œil peut l’être.\par
– Je n’en sais rien, Stubb. Vous-même, vous trouvez ça plutôt difficile parfois.\par
– Oui, quand un gars est trempé jusqu’aux os, il a de la peine à être raisonnable, c’est un fait. Et je le suis, mais qu’importe, attrapez ce tour et passez-le. Il me semble que nous sommes en train d’aiguilleter ces ancres comme si elles ne devaient plus jamais servir. Attacher là ces deux ancres, Flask, c’est comme lier les mains d’un homme dans son dos. Et quelles grandes mains généreuses. Elles sont une poigne de fer, non ? Et quelle prise elles ont aussi ! Je me demande, Flask, si le monde est à l’ancre quelque part. S’il l’est, il se balance au bout d’un câble d’une longueur inusitée. Aplatissez ce nœud-là et nous aurons terminé. Voilà. Après le fait de mettre pied à terre, le plus satisfaisant est de le poser sur le pont. Dites-moi, voulezvous essorer mes basques… Merci. On se moque des frusques longues comme celles-là, Flask, mais il me semble que, lors d’un orage en mer, on devrait toujours porter un manteau à queue ; ces queues qui vont en s’amincissant font office de chéneaux, voyez-vous ; c’est la même chose avec les tricornes, leurs pignons servent de gouttières, Flask. Plus de jaquettes de singe ni de cirés pour moi, je vais porter l’habit et mettre un chapeau de castor, oui. Holà ! Fichtre ! voilà mon suroît qui passe pardessus bord. Seigneur, Seigneur, que ces vents qui viennent du ciel puissent manquer à ce point de courtoisie ! C’est une vilaine nuit, ami.
\chapterclose


\chapteropen
\chapter[{CHAPITRE CXXII. Minuit dans la mâture Tonnerre et éclairs}]{CHAPITRE CXXII \\
Minuit dans la mâture Tonnerre et éclairs}\renewcommand{\leftmark}{CHAPITRE CXXII \\
Minuit dans la mâture Tonnerre et éclairs}


\chaptercont
\noindent (La vergue de la grand-hune. Tashtego y passe de nouvelles amarres.) \par
{\itshape –} Mm, mm, mm. Assez de ce tonnerre ! Trop, beaucoup trop de tonnerre, ici en haut. Cela sert à quoi, le tonnerre ? Mm, mm, mm. Nous ne voulons pas de tonnerre, nous voulons du rhum, qu’on nous donne un verre de rhum, mm, mm, mm !
\chapterclose


\chapteropen
\chapter[{CHAPITRE CXXIII. Le mousquet}]{CHAPITRE CXXIII \\
Le mousquet}\renewcommand{\leftmark}{CHAPITRE CXXIII \\
Le mousquet}


\chaptercont
\noindent Sous les plus furieux assauts du typhon, l’homme à la barre d’ivoire du {\itshape Péquod} avait été, plusieurs fois, jeté au pont par les coups de mer, malgré les sauvegardes de gouvernail, assez lâches pour donner son jeu à la barre.\par
Dans une tempête aussi rude où le navire n’est plus qu’un volant que se renvoient les coups de vent, il n’est pas rare que les aiguilles du compas s’affolent. Il en allait ainsi à bord du {\itshape Péquod}, le timonier n’avait pas manqué de remarquer dans quelle ronde rapide elles étaient entraînées sur les cartes à chaque secousse. C’est un spectacle que presque personne ne peut considérer sans quelque émotion.\par
Quelques heures après minuit, le typhon avait suffisamment diminué d’intensité pour que, grâce aux efforts de Starbuck et de Stubb – l’un engagé à l’avant, l’autre à l’arrière – les lambeaux du foc, de la misaine et de la grand-hune puissent être libérés des espars et s’en aillent tourbillonner sous le vent, pareils aux plumes que le vent arrache parfois aux albatros qui volent dans la tempête.\par
Les trois voiles neuves correspondantes furent alors enverguées et arisées, tandis qu’une voile d’étai de cape était établie à l’arrière. Ainsi le navire reprit sa route avec quelque précision. Le cap donné au timonier était, pour le moment et dans la mesure du possible, est-sud-est, car au gros du typhon il n’avait gouverné que tant bien que mal. Mais tandis que maintenant \hspace{1em}il orientait au mieux le navire tout en regardant le compas, voici que, par bonheur le vent parut venir de l’arrière, oui le vent debout malveillant devenait vent de poupe !\par
Aussitôt on brassa carré, au chant joyeux des matelots  : « Oh ! le bon vent ! Oh ! yé oh ! saluons-le, les gars ! » car le promesse d’un tel événement démentait les funestes présages qui l’avaient précédé.\par
À peine avait-il orienté les voiles – malgré une sombre répugnance – que Starbuck, s’apprêtant à obéir à la consigne de son capitaine lui enjoignant de faire rapport à n’importe quelle heure du jour ou de la nuit sur tout arrangement survenu, se dirigea machinalement vers la cabine d’Achab pour l’informer.\par
Il s’arrêta involontairement devant sa porte avant de heurter. La lampe du carré avait des oscillations longues qui rendaient sa lumière capricieuse et jetaient des ombres changeantes sur la porte fermée du vieillard, une porte mince dont les panneaux supérieurs étaient remplacés par des stores. En raison de l’isolement souterrain de l’endroit il y régnait un bourdonnement silencieux encore qu’il fût enserré de toutes parts par le rugissement des éléments. Contre la cloison de l’avant, les mousquets chargés, debout dans le râtelier, brillaient, attirant le regard. Starbuck était un homme droit et honnête, mais au moment où il aperçut les armes, une pensée mauvaise germa étrangement en son cœur, si mêlée encore à sa pacifique bonté qu’il n’en prit pas conscience sur-le-champ.\par
– Il m’aurait bien descendu, une fois, murmura-t-il, oui, voilà le mousquet qu’il a braqué sur moi, celui-là avec la crosse ferrée. Que je le touche, que je le soulève. C’est étrange que moi qui ai tenu en main tant de lances meurtrières, c’est étrange que je tremble ainsi. Chargé ? Il faut que je le sache. Oui, oui… et la poudre dans le bassinet… ce n’est pas bien. Il vaudrait mieux la vider… Attendons, il faut que je me guérisse de cela. Je vais \hspace{1em} tenir l’arme fermement tout en réfléchissant. Je suis venu lui faire rapport d’un vent favorable. Mais favorable à quoi ? À la mort et à la perte… il n’est favorable que pour Moby Dick. C’est un vent favorable pour ce seul maudit poisson. Le canon même qu’il a levé sur moi ! celui-là même… je le tiens en ce moment, il m’aurait tué avec l’arme même que je serre. Oui, et il tuerait volontiers tout son équipage. N’a-t-il pas dit qu’aucune tempête ne lui ferait amener les voiles ? N’a-t-il pas brisé son sextant céleste ? Et dans ces mers périlleuses, ne cherche-t-il pas sa route à l’estime douteuse du loch ? Et dans ce typhon même, n’a-t-il pas juré qu’il ne voulait point de paratonnerres ? Doit-on souffrir docilement que ce vieil homme fou entraîne tout l’équipage d’un navire à sa perte avec lui ? Oui, si ce navire va à la mort, il deviendra de propos délibéré le meurtrier de plus de trente hommes, et sur mon âme je peux jurer que ce navire ira à la mort, si on laisse Achab faire ce qu’il veut. Ce crime lui serait épargné si, en cet instant, il disparaissait. Ah ! il marmonne en dormant ? Oui, il dort là… tout près. Il dort ? Oui, mais il vit et se réveillera bientôt. Et alors, je ne pourrai m’opposer à toi, vieil homme. Tu n’écoutes ni la voix de la raison, ni les arguments, tout cela tu le méprises. Tu n’aspires qu’à voir obéis catégoriquement tes ordres catégoriques. Oui et tu dis que les hommes ont prêté ce serment qui est le tien, tu dis que nous sommes tous des Achab. Dieu m’en préserve ! Mais n’y a-t-il pas d’autre moyen ? Un moyen légal ? Le faire prisonnier pour le ramener de force ? Comment espérer arracher aux mains vivantes de ce vieillard sa vivante puissance ? Seul un sot le tenterait. En admettant qu’il soit garrotté, lié de cordes et de haussières de la tête aux pieds, enchaîné aux bagues d’amarrage sur le sol même de sa cabine, il serait alors plus hideux qu’un tigre en cage. Je ne pourrais en supporter la vue, je ne pourrais fuir ses hurlements. Tout réconfort et jusqu’au sommeil et la raison sans prix me seraient ôtés pendant ce long, cet intolérable voyage. Que restet-il alors ? La terre est à des centaines de lieues, le Japon interdit est la plus proche. Je suis seul, ici, en pleine mer, deux océans et un continent tout entier se tiennent entre la loi et moi. Oui, oui, il en est bien ainsi. Le ciel est-il meurtrier lorsque sa foudre frappe dans son sommeil un meurtrier en puissance consumant ensemble ses draps et sa chair ? Et serais-je, alors, un meurtrier si… Et lentement, furtivement, il appuya le canon du mousquet contre la porte.\par
– Au niveau de cette arme, le hamac d’Achab se balance dans la cabine, sa tête est de ce côté. Une simple pression et Starbuck vivrait pour étreindre sa femme et son enfant. Oh ! Mary ! Mary ! mon garçon ! mon garçon ! Mais si je ne t’appelle pas à la mort, vieillard, qui peut dire jusqu’à quelles profondeurs insondables sombrera le corps de Starbuck, dans une semaine aujourd’hui et avec l’équipage tout entier ? Dieu grand, où es-tu ? Le ferai-je ? Le ferai-je ? Le vent a faibli et tourné, sir, la voile de misaine et celle de grand-hunier sont établies et arisées, le navire va sa route.\par
– À l’arrière tous ! Oh ! Moby Dick, j’empoigne enfin ton cœur !\par
Tel fut le cri qui retentit dans le sommeil tourmenté du vieil homme comme si la voix de Starbuck avait donné la parole au long rêve muet.\par
Le canon, toujours levé, de l’arme trembla comme le bras d’un homme ivre contre la porte, Starbuck semblait livrer combat avec l’ange, puis se détournant il remit le mousquet dans le râtelier et s’en fut.\par
– Il dort trop profondément, monsieur Stubb, va, descends, réveille-le et préviens-le. J’ai affaire sur le pont. Tu sais ce qu’il y a à dire.
\chapterclose


\chapteropen
\chapter[{CHAPITRE CXXIV. L’aiguille}]{CHAPITRE CXXIV \\
L’aiguille}\renewcommand{\leftmark}{CHAPITRE CXXIV \\
L’aiguille}


\chaptercont
\noindent Le\hspace{1em}lendemain\hspace{1em}matin,\hspace{1em}la\hspace{1em}mer\hspace{1em}encore\hspace{1em}inapaisée\hspace{1em}roulait d’énormes lames, lentes et lourdes qui, poursuivant le filage gargouillant du {\itshape Péquod}, le poussaient comme les mains grandes ouvertes d’un géant. La forte brise ne désarçonnait pas et transformait l’air et le ciel en voiles immensément gonflées, le monde entier bondissait devant le vent. Voilé dans la pleine lumière matinale, le soleil invisible révélait sa présence par sa seule intensité diffuse d’où rayonnaient les faisceaux de ses épées. Tout était couronné d’un faste babylonien. La mer était un creuset d’or fondu qui débordait, bouillonnant de lumière et de chaleur.\par
Achab se tenait à l’écart, longuement enfermé dans un silence enchanté ; chaque fois que le navire enfonçait son beaupré dans les profondeurs, il se tournait pour regarder, à l’avant, flamboyer les rayons du soleil, et chaque fois que le vaisseau plongeait lourdement de l’arrière, il se retournait pour voir l’emplacement de l’astre dont les rayons jaunes se fondaient dans son sillage implacable.\par
– Ah ! ah ! mon bateau ! On te prendrait en cet instant pour le char marin du soleil. Oh ! vous, toutes les nations vers lesquelles pointe ma proue, je vous apporte le soleil ! Houles lointaines attelez-vous en flèche à mon navire car je suis le maître de la mer !\par
Une pensée contraire arrêta soudain son monologue intérieur, il se hâta vers le timonier et lui demanda d’une voix altérée quel était le cap du navire.\par
– Est-sud-est, répondit le timonier effrayé.\par
– Tu mens ! Il le frappa de son poing fermé. Le cap à l’est à cette heure du matin, et le soleil en poupe ?\par
Tout un chacun se trouva confondu, car le phénomène que venait de constater Achab avait inexplicablement échappé à tous, peut-être parce qu’il était d’une évidence beuglante.\par
Introduisant à demi sa tête dans l’habitacle, Achab jeta un coup d’œil aux compas, son bras levé retomba lentement, il parut chanceler un instant. Debout derrière lui, Starbuck regardait aussi et voici que les deux compas indiquaient l’est tandis que le {\itshape Péquod} faisait incontestablement route vers l’ouest.\par
Mais avant que la première alerte ait pu se répandre follement parmi l’équipage, le vieil homme s’écria, avec un rire dur : « Je comprends ! C’est déjà arrivé. Monsieur Starbuck, la foudre de la nuit passée a faussé les compas… c’est tout. Je pense que tu as déjà entendu parler de ce genre de choses ?\par
– Oui, mais cela ne m’était jamais encore arrivé à moi répondit le second, très pâle, d’un air lugubre.\par
Des accidents de cette nature, il faut le dire, sont arrivés plus d’une fois à des navires pris dans de violents orages La force magnétique d’une aiguille de compas est, comme chacun sait, de même nature que l’électricité de l’air, aussi n’y a-t-il rien de très étonnant à ce que de pareils phénomènes se produisent. En certains cas, lorsque la foudre a frappé les espars et le gréement mêmes du navire, l’effet produit sur l’aiguille a été plus désastreux encore ! Sa vertu magnétique étant détruite, son utilité n’était pas plus grande que celle de l’aiguille à tricoter d’une vieille femme. Mais dans tous les cas, un aimant ne retrouve jamais son pouvoir dévié ou perdu et tous ceux qui se trouveraient ailleurs à bord subissent le même sort, même celui qui est inséré au plus profond de la contre-quille.\par
Campé avec intention devant l’habitacle, regardant les compas affolés, le vieil homme du tranchant de sa main tendue prit la position exacte du soleil, et, satisfait de ce que les aiguilles fussent inversées précisément, il donna ses ordres en conséquence pour changer le cap du navire. Les vergues furent brassées sous le vent et une fois de plus l’intrépide {\itshape Péquod} fit front à la brise puisque celle qu’il avait crue favorable l’avait seulement berné.\par
Cependant, quelles que fussent ses pensées secrètes, Starbuck se tut, transmettant calmement les ordres voulus, tandis que Stubb et Flask – qui semblaient partager quelque peu ses sentiments – se soumettaient eux aussi sans murmurer. Quant aux hommes, bien que certains grommelassent à voix basse, la peur que leur inspirait Achab l’emportait sur celle qu’ils avaient du destin. Quant aux harponneurs païens, ils restèrent comme toujours inentamés. Si quelque chose se gravait en eux, ce n’était que l’effet d’un fluide issu du cœur inflexible d’Achab pénétrant leurs cœurs aimables.\par
Plongé dans une houleuse rêverie, le vieil homme arpenta un moment le pont. Mais son talon d’ivoire venant à glisser, il aperçut les tubes de cuivre écrasés du sextant qu’il avait brisé la veille.\par
– Pauvre badaud orgueilleux du ciel, pilote du soleil ! Hier, je t’ai fait échouer, et aujourd’hui les compas auraient aimé me rendre la pareille. Oui, oui. Mais Achab est encore seigneur de l’aimant. Monsieur Starbuck… une lance sans hampe, un maillet et la plus petite aiguille à voiles. Vite !\par
Peut-être qu’indépendamment de l’impulsion qui lui dictait ce qu’il était sur le point de faire, il obéissait à certains mobiles de prudence dont l’objet était de rassurer les esprits des hommes par un tour d’adresse lors d’un événement aussi stupéfiant que l’affolement des aiguilles. Il savait d’autre part que si l’on pouvait tant bien que mal maintenir un cap avec des aiguilles inversées, ce n’était pas chose que les superstitieux matelots prendraient à la légère. Ils trembleraient devant les mauvais présages qu’ils lui attribuaient.\par
– Hommes, dit-il en se tournant fermement vers l’équipage, tandis que le second lui tendait les objets demandés. Mes hommes, la foudre a inversé les aiguilles du vieil Achab mais, de ce morceau d’acier, Achab fera un compas qui indiquera la direction aussi infailliblement que l’autre.\par
Des regards interloqués de servile étonnement furent échangés par les matelots. Fascinés, ils attendaient l’opération magique qui allait suivre. Mais Starbuck se détourna.\par
D’un coup de maillet, Achab détacha la pointe de la lance et, tendant au second la longue tige de fer, il le pria de la tenir droite en évitant qu’elle n’entre en contact avec le pont. Puis, après avoir assené des coups répétés sur cette tige, il y posa l’aiguille émoussée, de champ, et la martela, moins fort, à plusieurs reprises, le second tenant toujours la tige. Puis il lui imprima de légers et singuliers mouvements – il eût été difficile de dire s’ils étaient indispensables pour magnétiser l’acier ou s’ils avaient simplement pour but d’accroître la crainte respectueuse de l’équipage – demanda du fil et alla ensuite vers l’habitacle, il y prit les deux aiguilles affolées et suspendit par son centre, à l’horizontale, l’aiguille à voiles au-dessus de la rose des vents. Pour commencer, l’aiguille tourna en rond, vibrante et frémissante aux deux extrémités, pour se stabiliser ensuite. \hspace{1em} Alors Achab, qui avait attendu avec anxiété, s’écarta franchement de l’habitacle et, tendant le bras vers l’aiguille, il dit :\par
– Voyez de vos propres yeux si Achab n’est pas seigneur de l’aimant ! Le soleil est à l’est, ce compas vous le jure !\par
Ils regardèrent, l’un après l’autre, car seuls leurs propres yeux pouvaient avoir raison de leur ignorance, et l’un après l’autre ils s’éloignèrent furtivement.\par
Les yeux brûlants de mépris et de triomphe, Achab apparut alors dans tout son funeste orgueil.
\chapterclose


\chapteropen
\chapter[{CHAPITRE CXXV. Le loch et la ligne}]{CHAPITRE CXXV \\
Le loch et la ligne}\renewcommand{\leftmark}{CHAPITRE CXXV \\
Le loch et la ligne}


\chaptercont
\noindent Depuis le temps si long que le {\itshape Péquod} condamné tenait la mer, le loch et la ligne avaient été rarement utilisés. Se fiant à d’autres moyens pour estimer leur position, quelques navires marchands et de nombreux navires baleiniers, surtout sur les parages de croisière négligent tout à fait de jeter le loch, bien qu’en même temps, et souvent simplement pour la forme, la route du navire soit inscrite sur le traditionnel routier, aussi bien que sa vitesse moyenne présumée à l’heure. Il en était allé ainsi pour le {\itshape Péquod.} Le touret de bois et le secteur de cercle, depuis longtemps inemployés, étaient suspendus sous la lisse des pavois de poupe. Les pluies et l’écume les avaient détrempés, le soleil et le vent déformés, tous les éléments s’étaient concertés pour pourrir un si vain objet. Une lubie s’empara d’Achab, insoucieux de ces détails, comme son regard tombait sur le touret, quelques heures après la scène de l’aimant, et que lui revenait en mémoire la destruction du sextant et son serment frénétique de se servir du loch et de la ligne. Le navire tanguait, les lames se soulevaient tumultueusement à l’arrière.\par
– Hé ! ceux de l’avant ! Jetez le loch !\par
Deux matelots s’avancèrent, le Tahitien doré et le Mannois grisonnant.\par
– Que l’un de vous prenne le touret, et je jetterai le loch.\par
Ils se dirigèrent vers l’extrême arrière, sous le vent, là où le pont, sous la poussée vigoureuse de la brise, rasait l’écume qui courait à son flanc.\par
Le Mannois prit le touret et l’éleva aussi haut que possible par la poignée du fuseau autour duquel s’enroulait la ligne et d’où pendait, à la verticale, le secteur de ce jusqu’à ce qu’Achab se fût approché.\par
Achab, debout devant lui, déroulait légèrement trente ou quarante tours pour former une première glène à jeter pardessus bord, quand le vieux Mannois, qui fixait ardemment et Achab et la ligne, s’enhardit à parler :\par
– Sir, je ne m’y fierais pas, cette ligne a l’air d’avoir du mal. La chaleur et l’humidité prolongées l’ont gâtée.\par
– Elle tiendra, vénérable monsieur. Est-ce que la chaleur et l’humidité prolongées t’ont gâté, toi ? Tu me parais tenir. Il serait peut-être plus vrai de dire que c’est la vie qui te tient et non toi qui tiens la vie.\par
– Je tiens le fuseau, sir. Mais, comme le dit mon capitaine, avec mes cheveux gris, il est inutile de discuter, surtout avec un supérieur qui ne conviendra jamais qu’il a tort.\par
– Comment ? En voilà un professeur rapiécé du Collège de dame Nature aux assises de granit ! Mais il me semble qu’il est trop servile. Où es-tu né ?\par
– Dans la petite île rocheuse de Man, sir.\par
– Parfait ! Tu donnes ainsi une juste image du monde.\par
– Je n’en sais rien, sir, mais c’est là que je suis né.\par
– Dans l’île de Man, hein ? Eh bien ! en inversant les choses, c’est bien aussi. Voilà un homme de l’Homme, un homme né sur la terre autrefois indépendante de l’homme, et à présent émasculée, et qui est maintenant absorbée… par quoi ? Plus haut, le fuseau ! Pour finir, tous les fronts obsédés de questions vont donner contre un mur aveugle. Plus haut ! oui, ainsi.\par
Le loch fut jeté. Les glènes lâches se tendirent rapidement en une longue ligne traînante sur l’arrière et aussitôt le fuseau se dévida. Les vagues abaissant et élevant le loch tour à tour par saccades, la traction imposée fit étrangement chanceler le vieil homme.\par
– Tiens ferme !\par
La ligne trop tendue fit guirlande et le loch fut arraché.\par
– J’ai écrasé le sextant, la foudre a affolé les compas et voilà que la mer en folie coupe la ligne du loch. Mais Achab peut tout réparer. Hale en dedans, Tahitien, enroule, Mannois. Et toi, va dire au charpentier de faire un nouveau loch, pendant que tu répareras la ligne. Veilles-y.\par
– Le voilà qui s’en va, pour lui c’est comme si rien ne s’était passé, mais pour moi c’est comme si l’axe du monde se relâchait en son centre. Halez, halez, Tahitien ! Les lignes filent tout entières mais rentrent lentement et rompues. Ah ! Pip, tu es venu aider ? hein, Pip ?\par
– Pip ? Qui appelez-vous Pip ? Pip a sauté de la baleinière. Pip manque à l’appel. Laisse-moi regarder si tu ne l’as pas remonté au bout de cette ligne, pêcheur. Elle se tend fortement, je pense qu’il s’y cramponne. Secouez-le, Tahitien ! Faites-le lâcher prise, nous ne voulons pas hisser à bord des couards. Oh ! voilà son bras qui sort de l’eau. Une hache ! une hache ! coupezle lui… nous ne voulons pas hisser à bord des lâches. Capitaine Achab ! sir, sir ! voilà Pip qui essaie à nouveau de remonter à bord !\par
– La paix, chenapan toqué, cria le Mannois en le saisissant par le bras. File du gaillard d’arrière !\par
– C’est toujours le plus grand de deux imbéciles qui tance l’autre, murmura Achab en s’avançant. Ôtez vos mains de dessus cet être saint ! Où disais-tu que se trouvait Pip, mon garçon ?\par
– À l’arrière, là, sir, à l’arrière ! Le voilà, le voilà !\par
– Et qui es-tu, mon garçon ? Je ne vois pas mon image dans ton regard absent. Oh ! Seigneur, que l’homme puisse être le crible qui laisse fuir les âmes immortelles ! Qui es-tu, mon garçon ?\par
– Le sonneur, sir, le tambourineur : ding, dong, ding ! Pip ! Pip ! Cent livres d’argile en récompense à qui donnera Pip ; cinq pieds de haut… l’air d’un lâche… on le reconnaît aussitôt à cela ! Ding, dong, ding ! Qui a vu Pip, le couard ?\par
– Il ne saurait y avoir de cœurs au-dessus de la limite des neiges. Oh ! ciel de glace, abaisse ici ton regard. Vous avez engendré ce malheureux enfant et vous l’avez abandonné, créateurs débauchés. Viens, enfant, la cabine d’Achab sera ta demeure tant qu’Achab vivra. Tu m’émeus, enfant, au plus intime de mon être, tu es lié à moi par les fibres mêmes de mon cœur. Viens, descendons.\par
– Qu’est-ce que cela ? Voilà le velours d’une peau de chagrin, dit Pip en regardant attentivement la main d’Achab et en la palpant. Ah ! si le pauvre Pip avait tenu une main aussi douce, peut-être \hspace{1em}qu’il \hspace{1em}n’aurait \hspace{1em}jamais \hspace{1em}été \hspace{1em}perdu ! \hspace{1em}Elle m’apparaît comme une sauvegarde, sir, à quoi peuvent s’agripper les âmes faibles. Oh ! sir, que le vieux Perth vienne river ensemble ces deux mains, la noire avec la blanche, car je ne la lâcherai plus…\par
– Ni moi, ô enfant, à moins que, ce faisant, je doive t’entraîner vers des horreurs pires que celles qui nous entourent. Viens dans ma cabine. Vous tous qui croyez en des dieux parfaitement bons et en la malignité totale de l’homme, voyez ! Voyez les dieux omniscients oublie la souffrance de l’homme, et l’homme, pour idiot qu’il soit et ne sachant ce qu’il fait, envahi cependant par la douceur de l’amour et de la reconnaissance. Viens ! Je me sens plus fier à guider ta main noire qu’à tenir celle d’un empereur !\par
– Les deux toqués partent à présent ensemble, marmonna le Mannois. Un toqué par sa force, l’autre par sa faiblesse. Mais voici le bout de cette corde pourrie… et tout dégoulinant. La réparer, hein ? Je crois que mieux vaudrait une nouvelle ligne. Je vais en parler à M. Stubb.
\chapterclose


\chapteropen
\chapter[{CHAPITRE CXXVI. La bouée de sauvetage}]{CHAPITRE CXXVI \\
La bouée de sauvetage}\renewcommand{\leftmark}{CHAPITRE CXXVI \\
La bouée de sauvetage}


\chaptercont
\noindent Cap au sud-est, dirigé par l’aiguille d’Achab et sa progression déterminée seulement par le loch, le {\itshape Péquod} faisait route vers l’équateur. Un si long voyage, sur des mers pareillement désertes, où l’on ne croisait nul navire, et jusqu’alors retardé par des vents alizés réguliers soulevant des vagues d’une douceur monotone, était pénétré d’un calme étrange, prélude à quelque scène de violence et de désespoir.\par
Enfin le navire atteignit les franges, si l’on peut dire, des parages de pêche équatoriaux et, dans les profondes ténèbres qui précèdent l’aube, côtoya un groupe d’îlots rocheux ; la bordée de Flask sursauta en entendant un cri si sauvagement plaintif, si surnaturel, pareil aux gémissements confus des fantômes du massacre des Innocents réclamé par Hérode, que tout un chacun fut tiré de sa rêverie et demeura un instant figé dans sa position, debout, assise ou couchée, écoutant, le regard fixe, semblables à la sculpture des esclaves romains, tant que dura cette folle lamentation. Les chrétiens et les civilisés d’entre les matelots l’attribuèrent aux sirènes et frissonnèrent, les harponneurs païens demeurèrent impassibles. Mais le Mannois grison – l’aîné de tous – déclara que ce cri troublant et éperdu était la voix d’hommes qui venaient de se noyer.\par
Achab ne sut rien de cet événement avant de sortir de son hamac à l’aube grise. Il lui fut alors rapporté par Flask, non sans être commenté en un sens funeste. Il eut un rire sourd et expliqua le mystère.\par
Les récifs, au large desquels avait passé le navire, étaient le lieu de séjour de troupeaux de phoques, quelques jeunes avaient perdu leurs mères, quelques mères avaient égaré leurs petits dont certains avaient dû émerger près du vaisseau et le suivre en criant et pleurant leur plainte humaine. Cette explication ne fit qu’aggraver la chose pour la plupart des hommes car les marins nourrissent des superstitions envers les phoques, non seulement à cause de leurs intonations lorsqu’ils sont en détresse, mais encore parce que leurs têtes rondes et à demi intelligentes, apparaissant à la surface de l’eau offrent une ressemblance avec le visage de l’homme et plus d’une fois, en de semblables circonstances, on les a pris, en mer, pour des hommes.\par
Ces pressentiments de l’équipage devaient, le matin même, recevoir une confirmation plus plausible dans le sort dont un des leurs fut victime. Au lever du soleil cet homme quitta son hamac pour prendre son poste de vigie au mât de misaine. N’était-il pas encore bien réveillé, parfois les marins montent aux mâts en somnambules, ou fut-ce pour une autre raison, nul ne sait. Quoi qu’il en soit, à peine était-il sur son perchoir, qu’on entendit un cri – un cri puis une chute – et, levant les yeux, les hommes virent une forme fendre l’air et, la suivant du regard, un bouquet de bulles blanches sur le bleu de la mer.\par
La bouée de sauvetage était un tonneau allongé ; elle fut larguée de l’arrière où elle était toujours suspendue à un croupiat ingénieux mais aucune main ne se leva pour la saisir, et le soleil avait si longuement séché le fût, qu’il eut tôt fait de se remplir, absorbant l’eau par tous les pores de son bois, de sorte que le tonneau cerclé de fer suivit le matelot dans les profondeurs comme pour lui servir d’oreiller, bien dur en vérité.\par
Ainsi le premier homme du {\itshape Péquod} qui monta au mât pour guetter la Baleine blanche dans son domaine propre fut avalé par la mer. Pourtant, peu y pensèrent sur le moment. À vrai dire, cet accident n’affecta guère les hommes, du moins en tant qu’augure, car ils ne le considéraient pas comme présage d’un mal à venir mais comme l’accomplissement d’un malheur déjà prédit. Ils affirmèrent qu’ils connaissaient dès lors la raison des plaintes sauvages qu’ils avaient entendues la nuit précédente. Mais une fois de plus le vieux Mannois fut d’un autre avis.\par
Il fallait à présent remplacer la bouée de sauvetage perdue ; Starbuck reçut l’ordre de s’en occuper, mais l’on ne trouva pas de barrique suffisamment légère et, dans l’impatience fiévreuse suscitée par ce qui semblait être l’approche d’une crise, et du dénouement à leur voyage, les hommes s’énervaient devant tout travail qui n’était pas en rapport avec cette fin, quelle qu’elle pût être, aussi allaient-ils laisser le navire sans nouvelle bouée, lorsque Queequeg, à grand renfort de signes étranges et d’allusions, suggéra, à mots couverts, quelque chose au sujet de son cercueil.\par
– Un cercueil comme bouée ! sursauta Starbuck.\par
– Ce serait plutôt singulier, je dois dire, ajouta Stubb.\par
– Le charpentier l’arrangera facilement de façon à en faire une fort bonne, dit Flask.\par
– Amenez-le, il n’y a rien d’autre dont nous puissions tirer parti, ajouta Starbuck, après un silence mélancolique. Équipezle, charpentier, ne me regardez pas ainsi, le cercueil, veux-je dire. M’entends-tu ? Équipe-le !\par
– Dois-je clouer le couvercle, sir ? questionna-t-il en faisant le geste de frapper d’un marteau.\par
– Oui.\par
– Et calfater les joints, sir ? Il faisait le geste de promener un calfait.\par
– Oui.\par
– Et dois-je l’enduire de goudron, sir ? En faisant le geste de tenir un pot à brai.\par
– File ! Qu’est-ce qui te prend ? Transforme le cercueil en bouée et c’est tout ! Monsieur Stubb, monsieur Flask, venez avec moi !\par
– Il s’en va en ayant pris la mouche. Il supporte tout et puis un rien le contrarie. Ça ne me plaît pas. Je fais une jambe pour le capitaine Achab et il la porte en galant homme, mais je fais une boîte à chapeau pour Queequeg et il ne veut pas y mettre la tête. Dois-je m’être donné tellement de peine pour que ce cercueil ne serve à rien ? Et maintenant on m’ordonne d’en faire une bouée. C’est comme de retourner un vieux manteau, on va mettre le bon côté à l’endroit à présent. Je n’aime pas du tout ce travail de rapetassage, cela manque de dignité, ce n’est pas dans mon rôle. Qu’on laisse les rafistolages aux enfants de gitans, nous avons mieux à faire. Je n’aime entreprendre que des travaux propres, vierges, honnêtes, précis, qui commencent franchement au commencement, dont le milieu se trouve à michemin, et qui finissent par la fin. Pas du travail de savetier, qui est terminé au milieu et commence par la fin. C’est un tour de vieille femme que de vous donner à faire des rafistolages. Seigneur ! quelles amours ont les vieilles femmes des gitans ? Je connais une vieille de soixante-cinq ans qui a filé avec un jeune gitan chauve. Et c’est la raison pour laquelle je n’ai jamais voulu, à terre, travailler pour des vieilles veuve esseulées, du temps que j’avais ma boutique dans le Vineyard ; elles auraient pu se mettre dans leur vieille tête solitaire l’idée de filer avec moi. Ce que c’est que de nous ! En mer, il n’est de bonnets que d’écume. Voyons. Clouer le couvercle, calfater les joints, l’enduire de goudron, l’assujettir fermement, et le suspendre au croupiat, pardessus bord, à l’arrière. A-t-on jamais fait chose pareille avec un cercueil ? Il y a bien des vieux charpentiers superstitieux qui se laisseraient ligoter dans le gréement plutôt que de faire ce travail. Mais je suis taillé dans un noueux sapin-ciguë d’Aroostock, je ne bronche pas. Monter en croupe sur un cercueil ! Porté sur l’eau sur un plateau de cimetière ! Qu’importe. Nous autres qui travaillons le bois, nous faisons aussi bien des lits nuptiaux que des tables de jeux, des cercueils que des corbillards. Nous travaillons au mois, à la tâche ou à la part sur les bénéfices et non pour nous poser des pourquoi et des comment au sujet de notre travail, à moins que ce ne soit des satanés rafistolages, et alors nous nous en débarrassons si possible. Hm ! je vais me mettre à ce travail tendrement. En tout – voyons – combien d’hommes sommes-nous à bord ? J’ai oublié. Ça ne fait rien, je vais mettre trente lignes de sauvetage séparées, avec des nœuds de bonnet turc, chacune de trois pieds de long et pendues tout autour du cercueil. De sorte que si nous allons par le fond, il y aura trente gaillards bien en vie à se disputer un cercueil, spectacle qui ne se voit pas souvent sous le soleil ! Allons, marteau, calfait, pot à brai et épissoir ! En avant !
\chapterclose


\chapteropen
\chapter[{CHAPITRE CXXVII. Le pont}]{CHAPITRE CXXVII \\
Le pont}\renewcommand{\leftmark}{CHAPITRE CXXVII \\
Le pont}


\chaptercont
\noindent Le cercueil repose sur deux bailles à ligne, entre l’établi et l’écoutille ouverte, le charpentier le calfate, le fil tordu de l’étoupe se déroule sortant d’une grosse bobine dans sa poche. Achab sort lentement du passavant et entend que Pip le suit.\par
– Retourne, enfin. Je vais revenir vers toi tout suite. Il s’en va ! Ma propre main ne fait pas mieux ce que je veux que cet enfant. Une nef d’église ! Qu’est-ce que cela ?\par
– Une bouée de sauvetage, sir. Ordres de monsieur Starbuck. Oh ! attention à l’écoutille, sir !\par
– Merci. Ton cercueil est en effet près du sépulcre !\par
– Sir ? L’écoutille ? Oui, en effet, en effet…\par
– N’es-tu pas le fabricant de jambes ? Regarde, ce pilon ne vient-il pas de ta boutique ?\par
– Je pense que oui, sir, la virole est-elle solide ?\par
– Suffisamment. Mais n’es-tu pas également l’entrepreneur des pompes funèbres ?\par
– Oui, sir. J’avais arrangé cette chose-là comme cercueil pour Queequeg, mais on m’a chargé de le transformer en quelque chose d’autre à présent.\par
– Alors dis-moi, n’es-tu pas une vraie canaille, un accapareur, un vieux chenapan qui se mêle de ce qui ne le regarde pas, païen que tu es, pour faire un jour des jambes, et le lendemain des cercueils pour les y enfermer et de plus transformer en bouées de sauvetage ces mêmes cercueils. Tu manques autant de principes que les dieux. Un touche-à-tout !\par
– Mais je n’ai aucune intention, sir. Je fais comme je fais.\par
– Encore comme les dieux. Écoute, ne chantes-tu jamais quand tu travailles à un cercueil ? On dit que les Titans fredonnaient tout en creusant les cratères des volcans, et le fossoyeur dans la pièce de théâtre chante en creusant la tombe de sa bêche. Toi, ne chantes-tu jamais ?\par
– Chanter, sir ? Si je chante ? Oh ! ce genre de choses me laisse assez indifférent, sir, mais la raison pour laquelle un fossoyeur chante doit être que sa bêche n’est point musicale. Mais le calfait est plein de musique, écoutez-le.\par
– Oui, c’est parce que le couvercle sert de table d’harmonie, et ce qui fait toutes les tables d’harmonie, c’est le vide qui se trouve au-dessous. Pourtant un cercueil qui contient un corps offre à peu près la même résonance, charpentier. N’as-tu jamais aidé à transporter une bière et ne l’as-tu jamais entendue sonner contre le portail du cimetière, en entrant ?\par
– Ma foi, sir… j’ai…\par
– Ta foi ? Qu’est-ce que c’est que ça ?\par
– Eh bien ! sir, c’est seulement une sorte d’exclamation… c’est tout.\par
– Hm, hm, poursuis…\par
– J’allais dire, sir, que…\par
– Es-tu un ver de soie ? Files-tu ton propre suaire ? Regarde ta poche ! Dépêche-toi ! Et mets tous ces traquenard à l’abri du regard.\par
– Il part vers le gaillard d’arrière. C’est brusque, mais les grains viennent brutalement sous les tropiques. J’ai entendu dire que l’île d’Albermale, une des Galapagos, est coupée en plein milieu par l’équateur. Il me semble qu’une sorte d’équateur coupe aussi le vieil homme en plein milieu. Il est toujours en dessous de la ligne, embrasé, brûlant, je vous dis ! Il regarde de ce côté… allons, étoupe, vite. Ce maillet de bois est un bouchon et moi je suis le professeur d’harmonica… tap, tap !\par
Achab, pour lui-même :\par
– En voilà un spectacle ! En voilà un bruit ! Le pic à tête grise tape sur l’arbre creux ! On pourrait bien envier les aveugles et les sourds ! Voyez ! cette chose repose sur deux bailles contenant les lignes. Ce gars est un farceur Tic, tac ! les secondes des hommes ! Oh ! combien immatériels sont tous les matériaux ! Qu’est-ce que la réalité sinon un impondérable ? Voilà le symbole redouté de la mort inexorable mais, par un pur hasard, il s’est transformé en expression d’espoir et de secours pour la vie la plus exposée. Une bouée d’un cercueil ? Cela aurait-il un sens plus profond ? Un sens spirituel qui voudrait qu’un cercueil ne soit, après tout, qu’une sauvegarde en vue de l’éternité ! J’y penserai. Mais non. Je suis engagé si loin dans le côté sombre de la terre que celui qui est censé être de lumière ne me paraît qu’un incertain crépuscule. N’en finiras-tu pas, charpentier, avec ce maudit bruit ? Je descends, et que je ne sois pas obligé de revoir cette chose-là en remontant. Et maintenant, Pip, nous allons en discuter, tu me nourris de si surprenantes philosophies ! Des mondes inconnus doivent se déverser en toi !
\chapterclose


\chapteropen
\chapter[{CHAPITRE CXXVIII. Le Péquod rencontre la Rachel}]{CHAPITRE CXXVIII \\
\textbf{{\itshape Le Péquod rencontre la} Rachel}}\renewcommand{\leftmark}{CHAPITRE CXXVIII \\
\textbf{{\itshape Le Péquod rencontre la} Rachel}}


\chaptercont
\noindent Le lendemain, venant droit sur le {\itshape Péquod}, un grand navire, la {\itshape Rachel}, fut aperçu, ses vergues portant des grappes serrées d’hommes. À ce moment-là le {\itshape Péquod} filait à bonne allure, mais tandis que se rapprochait l’étranger, les vastes ailes de ses voiles gonflées de vent, ses ailes fanfaronnes, s’affaissèrent toutes comme des ballons éclatés et toute vie se retira de sa coque frappée.\par
– Mauvaises nouvelles, il apporte de mauvaises nouvelles, murmura le vieux Mannois. Mais avant que son capitaine, debout dans sa pirogue, le porte-voix à la bouche, ait pu lancer un appel plein d’espoir, la voix d’Achab retentit :\par
– As-tu vu la Baleine blanche ?\par
– Oui, hier. As-tu vu une baleinière à la dérive ?\par
Achab jugula sa joie pour répondre négativement à cette question inattendue, et il serait volontiers monté à bord du navire étranger si le capitaine lui-même ne descendait déjà à son flanc après avoir mis en panne. Quelques vigoureux coups d’avirons eurent tôt fait de l’amener au {\itshape Péquod} et sa gaffe d’embarcation d’en accrocher les cadènes. Il sauta sur le pont et Achab le reconnut aussitôt pour un Nantuckais de ses relations. Mais ils n’échangèrent aucune salutation conventionnelle.\par
– Où était-elle ?… pas tuée !… pas tuée !… s’écria Achab en avançant davantage. Que s’est-il passé ?\par
Il apprit qu’assez tard dans l’après-midi de la veille, tandis que trois baleinières du navire étranger livraient la chasse à une troupe de baleines qui les avaient emmenées à quelque quatre ou cinq milles du navire et pendant qu’ils étaient engagés au vent, la bosse blanche et la tête de Moby Dick avaient soudain émergé, non loin sous le vent. La quatrième baleinière, celle de réserve, avait aussitôt été mise à la mer. Après qu’elle eut filé à la voile avec le vent, cette pirogue, qui était la plus rapide, parut avoir réussi à piquer un fer, du moins pour autant qu’en pouvait juger l’homme en vigie. Il vit la baleinière s’amenuiser au loin puis une gerbe étincelante d’écume, puis plus rien, d’où l’on en conclut que la baleine frappée avait indéfiniment entraîné au loin ses poursuivants, comme il arrive souvent. Pourtant, malgré quelque anxiété, on ne s’alarma pas outre mesure sur le moment. Les signaux de rappel furent hissés dans le gréement, la nuit tomba et, contraint de rechercher ses trois pirogues loin au vent, non seulement le navire dut abandonner à son sort, jusqu’à minuit, sa quatrième baleinière mais encore accroître momentanément la distance qui l’en séparait, car elle se trouvait dans la direction opposée. Mais l’équipage des trois premières enfin en sécurité à bord, le navire fit force de voiles, bonnettes sur bonnettes, pour chercher la pirogue manquante, allumant ses fourneaux en guise de feu d’alarme et tous ses hommes guettant aux mâts. Mais bien qu’il eût parcouru une distance suffisante pour se trouver à l’endroit présumé où l’embarcation perdue avait été vue pour la dernière fois, bien qu’il ait pris la panne pour mettre à la mer toutes ses pirogues et prospecter tout autour de lui, ne trouvant rien, il avait repris sa course, avait à nouveau mis en panne, à nouveau mis à la mer mais, bien qu’il eût persévéré ainsi jusqu’au jour, on ne retrouva pas la moindre trace de la baleinière manquante.\par
Lorsqu’il eut conté les faits, le capitaine étranger révéla aussitôt quelle était son intention en abordant le {\itshape Péquod.} Il souhaitait que celui-ci s’unisse à ses recherches sillonnant la mer, une distance de quatre à cinq milles séparant les deux vaisseaux, et sur des lignes parallèles de façon à scruter un double horizon pour ainsi dire.\par
– Je parierais bien quelque chose, chuchota Stubb à Flask, que l’un des hommes de la baleinière manquante portait la meilleure vareuse du capitaine, ou peut-être sa montre, tant il a l’air sacrement anxieux de la retrouver. Qui a jamais entendu dire que deux honnêtes navires baleiniers voyagent de conserve à la recherche d’une pirogue en pleine saison de pêche ? Voyez, Flask, voyez comme il est pâle, pâle jusqu’à l’iris de ses yeux… voyez… ce n’était pas une question de vareuse… ce devait être !\par
– Mon fils, mon propre fils est avec l’équipage. Pour l’amour de Dieu, je vous supplie, je vous conjure… s’écria le capitaine étranger à l’intention d’Achab qui, jusqu’ici, avait accueilli sa requête d’un air glacial.\par
– Laissez-moi noliser votre navire pour quarante-huit heures, je paierai avec joie, et je paierai largement… s’il n’y a point d’autre moyen… pour quarante-huit heures seulement… pas davantage… il le faut, oh ! il le faut et vous allez le faire !…\par
– Son fils ! dit Stubb, oh ! c’est son fils qu’il a perdu. Je retire ce que j’ai dit au sujet de la vareuse et de la montre. Que répond Achab ? Nous devons sauver ce garçon.\par
– Il s’est noyé avec tous les autres, la nuit dernière, dit le vieux Mannois debout derrière eux. J’ai entendu, vous avez tous entendu leurs esprits.\par
Ce qui rendait plus poignant encore l’accident de la {\itshape Rachel}, on l’apprit bientôt, c’est que non seulement un des fils de son capitaine appartenait à l’équipage de la baleinière perdue, mais encore qu’au même moment son autre fils avait été séparé du navire par les vicissitudes de la chasse, ce qui avait plongé le malheureux père dans la plus cruelle perplexité dont seule l’avait tiré la décision instinctive du second qui, adoptant l’attitude habituelle des baleiniers en des circonstances où des pirogues se trouvant également en danger mais éloignées les unes des autres, on se porte toujours au secours du groupe le plus nombreux. Mais le capitaine, retenu par quelque raison inconnue, inhérente à sa nature, s’était abstenu de raconter ces faits et n’avait parlé de son fils perdu que forcé par la froideur d’Achab. Ce gamin n’avait que douze ans et son père, dans son amour paternel de Nantuckais fervent mais téméraire, avait tôt cherché à l’initier aux dangers et aux merveilles d’une vocation qui était la destinée presque immémoriale de sa race. Il n’est pas rare non plus que des capitaines nantuckais éloignent d’eux un enfant d’un âge aussi tendre en le confiant à un autre navire pour un voyage de trois ou quatre ans, afin que son premier apprentissage ne soit débilité par aucun favoritisme naturel de la part d’un père rongé par la crainte et l’anxiété.\par
L’étranger poursuivait auprès d’Achab sa pitoyable requête, mais Achab était une enclume sur laquelle les coups portaient sans l’ébranler le moins du monde.\par
– Je ne partirai pas avant que vous m’ayez dit oui. Faites pour moi ce que vous voudriez que je fasse pour vous en pareil cas. Car vous aussi vous avez un fils, capitaine Achab, un tout jeune enfant blotti en sécurité à la maison – un enfant de votre vieillesse aussi. Oui, oui, vous vous laissez fléchir, je le vois, courez, courez les hommes et tenez-vous prêts à brasser carré.\par
– Arrière ! cria Achab, ne touchez pas une manœuvre ! Puis d’une voix qui martelait chaque mot il ajouta : Capitaine Gardiner, je ne le ferai pas. Même en cet instant, je perds mon temps. Adieu. Adieu. Dieu vous bénisse, homme, et puis-je me le pardonner à moi-même mais il faut que je parte. Monsieur Starbuck, regardez la montre de l’habitacle, et que, dans trois minutes, pas plus, les étrangers à ce navire soient avertis qu’ils doivent quitter le bord. Ensuite, mêmes amures et même cap que jusqu’ici.\par
Il se détourna brusquement, sans regarder personne, et descendit à sa cabine, laissant le capitaine étranger pétrifié devant un refus aussi tranchant et inconditionnel à son ardente supplique. Puis il tressaillit et sortant de cet envoûtement, se hâta en silence vers la rambarde, tomba plutôt qu’il ne mit pied dans sa pirogue et retourna à son navire.\par
Les deux vaisseaux eurent tôt fait de reprendre leurs routes divergentes. Tant que la {\itshape Rachel} fut en vue, on put remarquer qu’elle faisait des embardées de-ci, de-là, vers chaque tache noire, si petite qu’elle pût être, sur la mer. Tantôt elle amurait à bâbord, tantôt elle virait sur tribord, tantôt elle luttait contre la mer, tantôt elle laissait porter grand largue ; ses vergues inclinées tantôt d’un bord, tantôt de l’autre chargées tout le temps de grappes serrées d’hommes, semblables à trois grands cerisiers, au temps où les gamins vont marauder.\par
Par sa course hésitante et louvoyante, à son allure douloureuse, on sentait clairement que ce navire qui versait tant de larmes d’écume, restait sans consolation. C’était Rachel pleurant ses enfants perdus.
\chapterclose


\chapteropen
\chapter[{CHAPITRE CXXIX. La cabine}]{CHAPITRE CXXIX \\
La cabine}\renewcommand{\leftmark}{CHAPITRE CXXIX \\
La cabine}


\chaptercont
\noindent (Achab s’apprête à monter sur le pont ; Pip lui prend la main pour le suivre.) \par
– Enfant, enfant, je te le dis, tu ne dois pas suivre Achab maintenant. L’heure vient où Achab ne t’éloignera pas de lui dans l’épouvante, tout en préférant ne pas t’avoir auprès de lui. Pauvre enfant, tu portes en toi un remède trop efficace à mon mal. Les semblables guérissent les semblables, et pour cette chasse mon mal est la santé que je souhaite le plus ardemment. Demeure donc ici, où ils te serviront, comme si tu étais le capitaine. Oui, enfant, tu t’asseoiras ici dans ma propre chaise vissée et tu seras comme une vis de plus.\par
– Non, non ! vous n’avez pas un corps entier, sir, que mon pauvre être remplace votre jambe perdue, appuyez-vous seulement sur moi, sir. Je ne demande rien de plus afin de rester une partie de vous-même.\par
– Oh ! en dépit de millions de scélérats, ces mots feraient de moi un fanatique croyant en l’impérissable fidélité de l’homme !… et ils viennent d’un Noir !… et d’un fou ! mais je crois que c’est à lui que s’applique la formule les semblables guérissent les semblables, il redevient trop raisonnable.\par
– On m’a dit, sir, que Stubb avait une fois abandonné le pauvre petit Pip dont les os sont, au fond de la mer, d’une blancheur de neige malgré la noirceur de sa peau quand il \hspace{1em} vivait. Mais je ne vous abandonnerai jamais, sir, comme l’a abandonné Stubb. Si, il faut que je vienne avec vous.\par
– Si tu me parles encore longuement ainsi, la volonté d’Achab défaillera en lui. Je te dis non, cela ne peut être.\par
– Oh ! bon maître, maître, maître !\par
– Pleure ainsi, et je te tue ! Prends garde car Achab aussi est fou. Prête l’oreille, et tu entendras souvent résonner sur le pont mon talon d’ivoire, et tu sauras que je suis toujours là. Et maintenant je te quitte. Ta main ! Donne ! Tu es aussi fidèle que la circonférence l’est à son centre. Dieu te bénisse à jamais et, si nous en venons là, Dieu te sauve à jamais quoi qu’il advienne.\par
(Achab part, Pip ébauche un pas en avant.) \par
{\itshape –} Il se tenait là, il y a un instant, je respire l’air qu’il respirait… mais je suis seul. Si le pauvre Pip au moins était là, je pourrais supporter son absence, mais il est perdu. Pip ! Pip ! Ding, dong, ding ! Qui a vu Pip ? Il doit être en haut, ici, essayons la porte. Comment ? Ni serrure, ni verrou, aucun obstacle et pourtant elle ne s’ouvre pas. Ce doit être de la magie. Il m’a dit de rester ici. Oui, il m’a dit que cette chaise vissée était mienne. Alors je vais m’asseoir là, contre la barre d’arcasse, en plein centre du navire, avec devant moi toute sa quille et ses trois mâts. C’est ici que, disent nos vieux marins, les grands amiraux dans leurs noirs vaisseaux de 74 canons s’assoient à table, et y président devant les capitaines et les lieutenants en rang. Ah ! qu’est-ce là ? des épaulettes ! des épaulettes ! les épaulettes arrivent en foule. Faites passer les carafons, heureux de vous voir, remplissez les grès, messieurs ! Quel étrange sentiment éprouve un garçon noir quand il accueille des Blancs galonnés d’or ! Messieurs, avez-vous vu un certain Pip ?… un petit nègre, de cinq pieds de haut, l’air d’un chien battu, et lâche ! Il a sauté d’une baleinière une fois… L’avez-vous vu ? Non ! Eh bien, remplissez à nouveau vos verres, capitaines, et buvons à la honte des lâches ! Je ne donnerai pas de noms. Honte à eux ! Mettez les pieds sur la table, là. Honte à tous les lâches ! Chut, là-haut, j’entends sonner l’ivoire… Oh ! maître ! maître ! En vérité, je suis abattu quand vous marchez sur moi. Mais ici je resterai, bien même la poupe donnerait sur des écueils qui la transperceraient, quand bien même les huîtres viendraient me rejoindre ici.
\chapterclose


\chapteropen
\chapter[{CHAPITRE CXXX. Le chapeau}]{CHAPITRE CXXX \\
Le chapeau}\renewcommand{\leftmark}{CHAPITRE CXXX \\
Le chapeau}


\chaptercont
\noindent Et maintenant qu’à l’heure et sur les lieux prédestinés après le préambule d’un si long voyage, sur un aussi vaste parcours, Achab – ayant fréquenté tous les parcours de pêche – semblait avoir traqué son ennemi dans un repli de l’Océan, afin de l’y abattre plus sûrement. Maintenant il se trouvait proche de la latitude et de la longitude mêmes où lui avait été infligée sa suppliciante blessure ; maintenant on avait parlé à un navire qui avait rencontré Moby Dick la veille même, maintenant toutes les rencontres avec divers navires avaient prouvé l’indifférence satanique avec laquelle la Baleine blanche avait déchiqueté ses chasseurs innocents ou coupables, maintenant on voyait poindre dans le regard du vieillard une expression que les âmes faibles avaient peine à supporter. Pareille à la fixité de l’étoile polaire qui transperce d’un regard inébranlable le centre de la longue nuit polaire, l’intention d’Achab, immuable, étincelait sur le minuit perpétuel du sombre équipage ; elle le dominait si bien que tous leurs pressentiments, leurs doutes, leurs inquiétudes, leurs terreurs enfouies au plus profond de leurs âmes ne laissaient pas croître la moindre tige, ni s’ouvrir la moindre feuille.\par
Au cours de cet intermède lourd d’un ténébreux avenir, toute bonne humeur, feinte ou naturelle, s’évanouit. Stubb ne luttait plus pour obtenir un sourire, ni Starbuck pour y mette obstacle. Joie et tristesse, espérance et crainte, semblaient également réduites à la poudre de la plus fine poussière dans le pesant mortier de l’âme de fer d’Achab. Les hommes allaient et venaient sur le pont comme des automates, sentant le regard despotique du vieil homme qui ne les quittait pas.\par
Mais si vous aviez sondé le secret de ses heures les plus intimes, celles où il pensait n’être vu de personne, sauf d’un seul, vous auriez découvert que, tout comme son regard emplissait l’équipage d’une terreur sacrée, il redoutait mêmement le regard insondable du Parsi, qui le troublait, parfois d’une façon insensée. Une étrangeté accrue, voilée, investissait désormais le maigre Fedallah, secoué à tel point d’incessants frissons que les hommes lui jetaient un regard inquisiteur, se demandant, semblait-il, s’ils avaient affaire à un être de chair ou à l’ombre tremblante qu’eût projetée sur le pont quelque corps invisible. Et cette ombre rôdait sans cesse, car même à la nuit, personne n’eût pu dire que Fedallah dormait ou qu’il fût même descendu. Il restait immobile des heures durant, mais jamais assis ni appuyé, ses yeux merveilleux et tristes disaient clairement : nous sommes les deux veilleurs qui jamais ne reposons.\par
De même, à quelque moment que ce fût du jour ou de la nuit, les matelots ne pouvaient faire un pas sur le pont, qu’Achab ne se tînt devant eux, soit debout dans son trou de tarière, soit arpentant le pont entre les deux bornes toujours les mêmes, du grand mât et du mât d’artimon. Ou bien ils le trouvaient dans l’ouverture de l’écoutillon de la cabine, son pied vivant avancé sur le pont comme prêt à aller de l’avant, le chapeau profondément enfoncé sur les yeux et, malgré son attitude figée, malgré les jours et les nuits qui s’étaient ajoutés les uns aux autres sans qu’il se fût étendu dans son hamac et à cause de ce chapeau rabattu si bas, ils ne pouvaient savoir si ses yeux se fermaient parfois ou si son regard les scrutait. Et tandis qu’il était là, debout dans l’écoutille, pendant une heure et plus, peu lui importait que la nuit insouciante emperlât de rosée un manteau et un chapeau qui paraissaient de pierre. Les vêtements que la nuit avait détrempés, le soleil du lendemain les séchait sur lui, et ainsi, jour après jour, nuit après nuit. Il ne descendait plus dans sa cabine mais y envoyait chercher ce dont il avait besoin.\par
C’est en plein vent aussi qu’il prenait ses deux seuls repas, son petit déjeuner et son repas de midi, car il ne touchait pas au souper, il ne se rasait plus non plus et sa barbe poussait en broussailles sombres pareilles à ces racines terreuses d’un arbre déraciné qui vivent encore faiblement tandis que meurent les verdures du faîte. Cependant que toute sa vie n’était plus qu’une veille sur le pont, et que la vigilance mystique du Parsi était elle aussi ininterrompue, ces deux hommes ne semblaient jamais s’adresser la parole sauf, de loin en loin, pour quelque nécessité sans poids. Quand bien même la puissance d’un envoûtement semblait les unir secrètement l’un à l’autre, ils apparaissaient à l’équipage frappé de crainte aussi distant l’un de l’autre que les pôles. Si le jour il leur arrivait d’échanger un mot, la nuit les trouvait plongés dans leur mutisme. Ils se tenaient parfois, durant de longues heures, sans échanger un seul signe, loin l’un de l’autre à la clarté des étoiles, Achab à l’écoutille, le Parsi près du grand mât, mais se contemplant fixement l’un l’autre comme si Achab voyait son ombre projetée dans le Parsi et le Parsi sa substance transférée à Achab.\par
Pourtant, d’une certaine manière, au tréfonds de son moi qui prouvait son autorité de chaque jour, de chaque heure, de chaque instant, face à ses subordonnés, Achab apparaissait comme un libre seigneur et le Parsi rien de plus que son esclave. Cependant tous deux semblaient, sous le même joug, menés par un invisible tyran, l’ombre efflanquée contre le torse solide. Car, quoi que fût le Parsi, la membrure et la quille c’était Achab.\par
À la première et faible lueur du jour, sa voix de métal sonna à l’arrière :\par
– Armez les postes de vigie !\par
Et pendant toute la durée du jour, après le coucher du soleil, après le crépuscule, la même voix, à chaque heure au coup de la cloche du timonier, criait :\par
– Que voyez-vous ? Ayez l’œil ! ayez l’œil !\par
Mais lorsque trois ou quatre jours se furent écoulés après la rencontre avec la {\itshape Rachel} chercheuse d’enfants et qu’aucun souffle n’eut été signalé, le vieillard obsédé parut douter de la loyauté de son équipage, hormis de celle des harponneurs païens. Il paraissait même se demander si Stubb et Flask ne fermeraient pas volontairement les yeux au cas où ce qu’ils attendaient s’offrirait à leur vue. Si ces soupçons étaient réellement siens, il se gardait sagement de les exprimer par des mots quand bien même ses actes paraissaient les trahir.\par
– C’est moi qui signalerai la Baleine blanche, dit-il, \hspace{1em} oui ! C’est à Achab que doit revenir le doublon !\par
Il gréa de ses propres mains une chaise de mâture et, envoyant un homme au mât avec une estrope pour l’assurer au grand mât et un cartahu pour la hisser, puis recevant les deux extrémités du filin il arrima sa chaise avec l’une et prépara un cabillot traversant le râtelier pour l’y amarrer. Cela fait, tenant toujours le filin et debout dedans le cabillot, son regard fit le tour de son équipage, s’attardant longuement sur Daggoo, Queequeg, Tashtego, évitant Fedallah, pour se fixer fermement sur son premier second, et il dit :\par
– Prends l’estrope, sir, je la remets entre tes mains, Starbuck.\par
Puis, s’installant dans sa chaise, il donna l’ordre de la hisser. À Starbuck incomberait de l’arrimer pour demeurer ensuite au cabillot. Ainsi, enserrant d’une main le mât de cacatois, Achab voyait au loin la mer sur des milles et des milles, \hspace{1em}à l’avant, à l’arrière, par tribord et bâbord, tout le vaste cercle qui s’épanouissait depuis une telle hauteur.\par
Lorsqu’un matelot doit travailler de ses mains, à quelque endroit élevé et presque isolé du gréement, qui n’offre point de prise pour les pieds, on le hisse ainsi dans la mâture, suspendu par une estrope dont l’extrémité amarrée au pont est toujours confiée à un homme qui en a la rigoureuse surveillance, parce que dans l’enchevêtrement des manœuvres courantes on ne distingue pas infailliblement du pont à quoi elles correspondent à une pareille hauteur, et que celles-ci, étant à tout instant larguées, il pourrait arriver que, par quelque inadvertance d’un matelot, l’homme ainsi suspendu soit balancé à la mer si l’estrope qui l’assure n’était l’objet d’un contrôle absolu. Aussi les dispositions d’Achab n’eurent-elles rien que d’ordinaire, hormis le fait étrange qu’il eût choisi pour ce faire Starbuck, le seul homme qui eût jamais osé lui faire front avec un tant soit peu de fermeté et dont il pût douter quant à sa vigilance de guetteur. Il était, certes, insolite qu’il ait choisi cet homme pour assurer sa sécurité, remettant librement sa vie entre les mains de celui dont par ailleurs il se méfiait tant.\par
Or, la première fois qu’Achab se trouva à ce poste élevé, il y était depuis six minutes à peine que l’une de ces sauvages frégates à bec jaune qui viennent si souvent tournoyer malencontreusement autour des hommes en vigie sous ces latitudes, vint décrire en criant des cercles serrés et rapides autour de sa tête, puis s’éleva à quelque mille pieds pour redescendre en spirales et recommencer son tourbillon autour de la tête du capitaine.\par
Le regard perdu au loin vers l’horizon incertain, Achab ne parut pas prêter attention au farouche rapace et personne en vérité ne l’eût remarqué, en d’autres temps, tant c’était une circonstance banale ; pourtant, à présent, l’œil le plus insouciant semblait découvrir un sens malin à tout incident.\par
– Votre chapeau, votre chapeau, sir ! cria soudain le matelot sicilien, qui occupait le poste de vigie au mât d’artimon et se tenait juste derrière lui, bien qu’un peu en dessous, et séparé de lui par un gouffre d’air.\par
Mais déjà l’aile noire passait devant les yeux du vieillard, le long bec courbe frôlait sa tête, et la sombre frégate enleva dans un cri son trophée et s’envola.\par
Un aigle avait volé trois fois autour de la tête de Tarquin, lui avait enlevé sa coiffure pour la lui remettre, ce qui fit dire à sa femme Tanaquil qu’il serait roi de Rome. Mais c’était seulement au cas où la coiffure venait à être replacée que l’augure prenait un sens favorable. Le chapeau d’Achab ne lui fut jamais rendu, l’aigle des mers s’enfuit à tire-d’aile en l’emportant, bien loin au-devant de la proue, pour disparaître enfin puis, à l’endroit où l’on le perdit de vue, on discerna faiblement un minuscule point noir tombant dans la mer de ces hauteurs vertigineuses.
\chapterclose


\chapteropen
\chapter[{CHAPITRE CXXXI. Le Péquod rencontre la Joie}]{CHAPITRE CXXXI \\
\textbf{{\itshape Le} Péquod {\itshape rencontre la} Joie}}\renewcommand{\leftmark}{CHAPITRE CXXXI \\
\textbf{{\itshape Le} Péquod {\itshape rencontre la} Joie}}


\chaptercont
\noindent L’ardent {\itshape Péquod} poursuivait sa route, les vagues et les jours passaient, la bouée-cercueil se balançait toujours mollement quand fut signalé un navire de misère dérisoirement nommé la {\itshape Joie.} Tandis qu’il approchait, tous les yeux se fixèrent sur ces larges portiques qui, sur certains navires baleiniers, dominent de huit ou neuf pieds le gaillard d’arrière et servent de support aux pirogues de réserve ou celles qui sont dégréées ou à réparer.\par
On apercevait sur les portiques du navire étranger les membrures blanches brisées et les éclats de bordages de ce qui avait été une baleinière, mais à travers l’épave on voyait comme à travers le squelette décoloré et désarticulé d’un cheval.\par
– As-tu vu la Baleine blanche ?\par
De la lisse de couronnement le capitaine aux joues caves répondit : Regarde, en désignant l’épave de son porte-voix.\par
– L’as-tu tuée ?\par
– Le harpon qui la tuera n’est pas encore forgé, répondit l’autre, en regardant avec tristesse un hamac gonflé, posé sur le pont et dont des matelots étaient occupés à coudre ensemble les deux bords.\par
– Pas encore forgé ! et arrachant à sa fourche le fer de Perth, Achab le tendit et s’écria :\par
– Vois, Nantuckais, en cette main je tiens sa mort ! Ces barbelures ont été trempées par le sang et la foudre et je jure de les tremper une troisième fois dans cette place chaude sous la nageoire pectorale là où la vie maudite de la Baleine blanche bat le plus intensément !\par
– Alors Dieu te garde, vieillard… vois-tu cela ? – et il désigna le hamac – j’ai perdu déjà quatre hommes sur cinq qui, hier encore, étaient bien vivants mais qui moururent avant la nuit. Il m’est donné de confier celui-là seul à la mer, les autres furent engloutis avant de mourir, vous voguez sur leur tombe. Puis se tournant vers son équipage il demanda : « Prêts ? posez la planche sur la lisse et soulevez le corps ; oui… et puis… Oh ! Seigneur – et il s’avança vers le hamac, les mains levées : Puisse la résurrection et la vie…\par
– En avant toute ! Barre dessus ! cria Achab, rapide comme l’éclair, à ses hommes.\par
Mais le bond du {\itshape Péquod} ne fut pas assez rapide pour qu’on n’entendît pas le bruit d’éclaboussement du corps tombant à la mer, pas assez rapide en vérité pour éviter la gerbe d’écume qui aspergea sa coque d’un baptême funèbre.\par
Comme Achab s’éloignait de la triste {\itshape Joie}, l’étrange bouée de sauvetage suspendue à la poupe du {\itshape Péquod} prit un relief insolite et une voix retentit dans son sillage, avertissant :\par
– Ah ! regardez, hommes, regardez là-bas ! C’est en vain, ô étrangers, que vous fuyez le chagrin de nos funérailles, car votre navire ne s’éloigne que pour nous montrer votre propre cercueil !
\chapterclose


\chapteropen
\chapter[{CHAPITRE CXXXII. La symphonie}]{CHAPITRE CXXXII \\
La symphonie}\renewcommand{\leftmark}{CHAPITRE CXXXII \\
La symphonie}


\chaptercont
\noindent C’est un jour clair d’un bleu d’acier. Le double firmament de la mer et du ciel se confondait dans cet azur partout répandu ; toutefois, la transparence douce et pure du ciel pensif avait un air féminin, tandis qu’une respiration lente et puissante, pareille à celle de Samson endormi, soulevait la mer robuste et virile.\par
Ici et là, dans les hauteurs, glissaient les ailes blanches de menus oiseaux immaculés, tendres pensées de ce ciel féminin ; dans l’abîme bleu et sans fond s’agitaient de puissants léviathans, des espadons et des requins, pensées vigoureuses, inquiètes, meurtrières du viril Océan. Si le contraste existait en profondeur, il n’était extérieurement qu’ombres et nuances, ils semblaient ne faire qu’un, le sexe seul les distinguait l’un de l’autre.\par
Tel un souverain absolu, le haut soleil semblait marier ce ciel tendre à l’Océan téméraire et mouvant, l’épouse à l’époux ; et à l’horizon une douce palpitation, plus sensible sous les tropiques, révélait la confiance passionnée et frémissante, l’amoureuse inquiétude avec laquelle la pauvre épousée s’abandonnait.\par
Noué, tordu, noueux, ravagé de rides, obstiné, inflexible, hagard, les yeux pareils à des braises ardentes sous les cendres du désastre, Achab, debout dans la clarté du matin, ne chancelait pas et levait le casque meurtri de son front vers le beau visage féminin du ciel.\par
Oh ! impérissable enfance, innocence de l’azur ! Créatures ailées et invisibles qui jouez autour de nous ! Tendre enfance du ciel ! Combien vous étiez indifférents à la glène serrée de la douleur d’Achab ! Ainsi j’ai vu Marie et Marthe, elfes aux yeux rieurs, gambader insouciantes autour de leur vieux père et jouer avec le cercle de boucles roussies qui croissaient autour du cratère éteint de sa tête.\par
Traversant lentement le pont depuis l’écoutillon, Achab se pencha bientôt sur la lisse et regarda son ombre s’enfoncer toujours et toujours davantage sous son regard tandis qu’il s’efforçait d’en transpercer la profondeur. Mais le parfum suave de cet air enchanté parut enfin éloigner de son âme la gangrène qui la rongeait. La joie et la séduction de ce ciel parvinrent enfin jusqu’à lui comme une caresse, le monde marâtre si longuement menaçant et féroce, nouait à présent autour de son cou têtu des bras aimants et semblait verser sur lui des larmes de joie comme sur un qui, malgré son obstination et son égarement, peut être encore sauvé par les ressources d’un cœur capable de bénédiction. Sous son chapeau rabattu, Achab versa une larme dans la mer et le Pacifique ne contint rien de plus précieux que cette seule petite goutte d’eau.\par
Starbuck vit le vieil homme, il vit combien lourdement il se penchait par-dessus la rambarde et il lui sembla entendre, au plus intime de son propre cœur, l’infini sanglot arraché au sein de la sérénité environnante. Attentif à ne pas le troubler, à n’être pas vu de lui, il s’approcha toutefois et se tint à ses côtés.\par
Achab se retourna :\par
– Starbuck !\par
– Sir !\par
– Oh ! Starbuck ! la douceur du vent est si tendre, le ciel si clément. C’est par un tel jour – d’une pareille bénignité – que, harponneur de dix-huit ans, j’ai piqué ma première baleine. Il y a quarante ans… quarante ans… passés, passés ! Quarante ans d’incessante chasse à la baleine ! Quarante ans de privations, de dangers, de tempêtes ! Quarante ans sur la mer impitoyable ! depuis quarante ans, Achab a abandonné la terre paisible pour mener la guerre contre les horreurs des abîmes marins ! Oui, Starbuck, sur ces quarante ans je n’en ai pas passé à terre plus de trois. Quand je pense à ma vie, à la solitude désolée qu’elle a été, à cette citadelle qu’est l’isolement d’un capitaine qui admet si peu, en ses murs, la sympathie de la campagne verdoyante du dehors… Oh ! lassitude, oh ! fardeau ! Noir esclavage d’un commandement solitaire !… quand je pense à tout cela que je n’avais fait qu’entrevoir et qui s’impose à moi aujourd’hui… et comment pendant quarante ans je n’ai vécu que d’une nourriture salée, symbole de l’aride nourriture de mon âme !… quand le plus pauvre des terriens a chaque jour un fruit frais à portée de main, et peut rompre le pain frais du monde au lieu de mes quignons moisis… loin, si loin… des océans entiers me séparant de cette femme-enfant que j’ai épousée à passés cinquante ans, faisant voile le lendemain pour le cap Horn, ne laissant que l’empreinte de ma tête sur l’oreiller de mes noces… une épouse ? une femme ?… non, une veuve plutôt dont le mari est vivant ! Oui, j’ai fait une veuve de cette pauvre fille quand je l’ai épousée, Starbuck ! Et puis la folie, la frénésie, le sang qui bout et le front qui fume, avec lesquelles des milliers de fois le vieil Achab a mis à la mer, pour chasser furieusement sa proie à travers l’écume – moins homme que démon ! – oui, oui, pendant quarante ans le vieil Achab a été fou, fou, un vieux fou ! Et pourquoi le combat de cette poursuite ? À quoi cela a-t-il servi de se rompre les bras de fatigue aux avirons, au fer, à la lance ? Achab en est-il plus riche ou meilleur maintenant ? Regarde. Oh ! Starbuck ! Ne fut-ce pas cruel, qu’accablé d’un lourd fardeau, une pauvre jambe m’ait été arrachée ? Allons, repoussons ces vieux cheveux, ils m’aveuglent et je semble pleurer. Jamais chevelure si grise n’a germé ailleurs que sur des cendres. Mais ai-je l’air si vieux, si terriblement vieux, Starbuck ? Je me sens mortellement faible, courbé, voûté, comme si j’étais Adam titubant sous le poids des siècles accumulés depuis le temps du Paradis. Seigneur ! Seigneur !… brisez mon cœur ! écrasez mon cerveau ! Dérision ! dérision, amère dérision, cruelle dérision des cheveux gris, ai-je vécu assez de joie pour vous porter, pour me sentir si intolérablement vieux ? Plus près ! tiens-toi près de moi, Starbuck ! Permets-moi de regarder dans un regard d’homme, c’est plus salutaire que de contempler la mer ou le ciel, ou de lever les yeux vers Dieu. Par la terre verte et le foyer brillant, c’est là, la boule des magiciens, je vois ma femme et mon enfant dans ton œil. Non, non, reste à bord, à bord ! ne mets pas à la mer avec lui quand Achab marqué au feu donnera la chasse à Moby Dick. Tu ne courras pas ce risque. Non, non ! pas avec le foyer lointain que je vois dans tes yeux !\par
– Oh ! mon capitaine ! mon capitaine ! âme noble ! vieux grand cœur, après tout ! Pourquoi quiconque livrerait-il la chasse à ce poisson maudit ? Partons ! Fuyons ces eaux meurtrières ! Retournons au pays ! Une femme, un enfant sont aussi le lot de Starbuck, un enfant de sa jeunesse fraternelle et joueuse, comme les tiens, sir, sont femme et enfant de ta vieillesse paternelle, aimante et nostalgique ! Partons ! Partons ! Permets-moi de changer le cap aussitôt ! Avec quelle heureuse gaieté, ô mon capitaine, ne voguerions-nous pas pour revoir notre vieux Nantucket ! Je crois qu’ils ont à Nantucket des jours aussi doux et bleus que celui-ci.\par
– En effet, ils en ont. J’ai vu des matins d’été… À peu près à cette heure, oui, l’heure de la sieste de l’enfant ; il s’éveille brusquement, il s’assied dans son lit et sa mère lui parle de moi, de ce vieux cannibale que je suis, elle lui raconte que je suis en mer et que je reviendrai pour le faire sauter sur mon genou.\par
– C’est ma Mary, ma Mary même ! Elle a promis que mon fils serait chaque matin conduit au sommet de la colline pour être le premier à apercevoir les voiles de son père ! Oui, oui ! N’en disons pas davantage ! les dés en sont jetés, nous allons faire route vers Nantucket ! Allons, mon capitaine, étudie la carte, et partons ! vois, vois ! le visage de l’enfant à la fenêtre, le salut de sa main sur la colline !\par
Mais le regard d’Achab se détourna, il frémit comme un arbre flétri et laissa tomber au sol son ultime fruit de cendre.\par
– Qu’est-ce ? quelle est cette chose surnaturelle, insondable et sans nom ? Qui est le seigneur et maître caché, cruel, impitoyable qui use d’artifices pour m’amener à lui obéir ? tant et si bien que contre toute nostalgie et tout amour humain je me fasse violence et me pousse et me presse, et me rendant prêt à faire ce que mon propre cœur, mon cœur humain, n’oserait même envisager ? Achab est-il Achab ? Est-ce moi, Seigneur, ou qui d’autre qui lève ce bras ? Mais si le grand soleil lui-même ne se meut pas de lui-même, s’il n’est qu’un messager dans ciel, s’il n’est point une seule étoile pour accomplir sa révolution sans une invisible puissance, comment ce cœur chétif battrait-il, ce cerveau débile penserait-il, si Dieu n’en est point le battement, la pensée et la vie, et non moi. Par le ciel, homme, nous sommes virés et virés encore en ce moment comme le guindeau là-bas dont le Destin est l’anspect. Et pendant tout ce temps, voilà que sourit le ciel et que la mer est insondable ! Vois ! vois cet Albacore ! Qui l’a amené à chasser et à accrocher ce poisson volant ? Où vont les meurtriers, homme ! Quand le juge lui-même est traîné en prison, qui portera la condamnation ? Mais c’est un vent si doux et un ciel d’une telle douceur, l’air embaume comme s’il avait passé sur de lointaines prairies, ils ont fait les foins quelque part sous les pentes des Andes, Starbuck, et les faucheurs dorment dans l’herbe fraîchement coupée. Dorment ? Oui, peinons tant que nous voudrons, nous nous endormons tous pour finir dans le champ. Dormir ? Oui, et rouiller dans la verdure, telles les faucilles abandonnées de l’année précédente. Starbuck !\par
Mais mortellement pâle de désespoir, le second s’était enfui.\par
Achab traversa le pont pour regarder par-dessus l’autre bord. Il tressaillit en voyant, dans l’eau, le reflet de deux yeux fixes. Fedallah était penché, immobile, à la même rambarde.
\chapterclose


\chapteropen
\chapter[{CHAPITRE CXXXIII. La chasse. Premier jour}]{CHAPITRE CXXXIII \\
La chasse. Premier jour}\renewcommand{\leftmark}{CHAPITRE CXXXIII \\
La chasse. Premier jour}


\chaptercont
\noindent Cette nuit-là, lors du quart de minuit à quatre heures, lorsque le vieillard – comme il le faisait parfois – sortit de l’écoutille où il s’appuyait et regagna son trou de tarière, il leva soudain le visage avec âpreté en humant l’air du large comme un chien de bord perspicace à l’approche d’une île barbare. Il déclara qu’il devait y avoir une baleine non loin. Toute la bordée sentit bientôt cette odeur particulière que le cachalot vivant répand souvent à une grande distance et aucun matelot ne fut surpris lorsque après avoir consulté le compas, puis le penon, et s’être assuré aussi précisément que possible de la direction d’où venait l’odeur, Achab donna rapidement l’ordre de changer légèrement le cap et de diminuer de voiles.\par
La ligne de conduite avisée, dictant cette manœuvre, se justifia pleinement au lever du jour qui révéla, à l’avant, une longue bande lisse horizontale, onctueuse comme de l’huile, et qui ressemblait, entourée comme elle l’était de risées, au revolin rapide, poli comme un métal, qui se forme à l’embouchure des grands fleuves.\par
– Aux postes de vigie ! Tout le monde sur le pont !\par
Sur le pont du gaillard d’avant, Daggoo fit un tel tonnerre avec le bout de trois anspects réunis, il éveilla les dormeurs avec un tel fracas de jugement dernier qu’ils semblèrent rejetés par l’écoutille tant ils apparurent avec promptitude, leurs vêtements à la main.\par
– Que voyez-vous ? cria Achab le visage renversé vers le ciel.\par
– Rien, rien, sir ! fut la réponse qui tomba d’en haut.\par
– Les perroquets !… les bonnettes ! Hautes et basses sur les deux bords !\par
Toute la toile établie, il largua la sauvegarde destinée à le hisser au mât de grand cacatois. Il était au deux tiers de son ascension lorsque, regardant dans l’espace vide séparant le grand perroquet du grand hunier, il poussa un cri de goéland :\par
– La voilà qui souffle ! La voilà qui souffle ! Une bosse comme une colline neigeuse ! C’est Moby Dick !\par
Enflammés par le cri qui parut poussé par les trois hommes de vigie ensemble, les matelots se ruèrent dans le gréement pour voir la fameuse baleine si longuement poursuivie. Achab avait maintenant atteint le sommet de son perchoir dominant les autres postes de vigie, Tashtego se trouvant juste au-dessous de lui, au chouquet du mât de grand perroquet de sorte que sa tête était au niveau du talon d’Achab. De cette hauteur, on apercevait le cachalot à un mille environ à l’avant, chaque vague dévoilant sa haute bosse étincelante cependant que son souffle montait régulièrement et silencieusement dans les airs. Aux yeux des crédules matelots c’était là le souffle silencieux qu’ils avaient vu sous la lune dans l’Atlantique et l’océan Indien.\par
– Et nul d’entre vous ne l’avait-il repéré avant moi ? cria Achab aux hommes perchés tout autour de lui.\par
– Je l’ai vu presque au même moment, sir, que le capitaine Achab, et j’ai donné de la voix, dit Tashtego.\par
– Mais pas au même instant, non, pas au même instant… le doublon est à moi, le Destin m’avait réservé le doublon. À moi seul ! Aucun d’entre vous n’aurait pu signaler la Baleine blanche le premier. La voilà qui souffle ! La voilà qui souffle ! Encore et encore ! répéta-t-il plusieurs fois avec des intonations longues, lentes, mesurées, accordées au rythme prolongé du souffle de la baleine :\par
– Elle va sonder ! Rentrez les bonnettes, amenez les perroquets ! Parez trois baleinières ! Monsieur Starbuck, souvenezvous, vous restez à bord et vous gardez le navire. Vous, à la barre. Loffe, loffe un point ! Doucement, homme, doucement ! Voilà la queue ! Non, non, ce n’est que la noirceur de l’eau ! Parées, les pirogues ? Paré ! paré ! Descendez-moi, monsieur Starbuck, descendez-moi, plus vite… vite ! et il traversa l’air jusqu’au pont.\par
– Elle file droit sous le vent, sir, dit Stubb, droit devant nous, elle n’a pas encore pu voir le navire.\par
– Silence, homme ! Paré aux bras ! Barre dessous toute ! Brassez ! Ralinguez ! Ralinguez ! C’est ça… bien ! Les pirogues à la mer !\par
Toutes les baleinières, hormis celle de Starbuck, furent bientôt à la mer, les voiles établies, toutes les pagaies maniées vigoureusement, soulevant des ondulations rapides, elles se ruaient sous le vent, celle d’Achab en tête. Dans les yeux caves de Fedallah s’alluma une pâle lueur de mort, un rictus hideux tordit sa bouche.\par
Telles de silencieuses coquilles de nautiles, leurs proues légères fendaient la mer, mais elles ne purent approcher leur ennemi que lentement, car à mesure qu’elles avançaient, l’Océan se fit plus calme encore, il paraissait étaler un tapis sur les vagues et sa sérénité en faisait une prairie matinale. Enfin le chasseur haletant fut si près de sa proie, apparemment sans méfiance, que sa bosse éblouissante fut tout entière visible, glissant comme une île solitaire sans cesse sertie de l’anneau mouvant d’une écume verdâtre, légère et floconneuse. Il vit les grandes rides indiquées qui barraient son front soulevé hors de l’eau à l’avant et, projetée loin en avant sur le moelleux tapis d’Orient des eaux, la chatoyante ombre blanche de ce large front laiteux qu’accompagnait, joueuse, la musique des vaguelettes, cependant que, derrière lui, la mer bleue roulait dans la vallée mouvante de son sillage, et que de part et d’autre de ses flancs des bulles brillantes jaillissaient en dansant. Les pattes légères de centaines d’oiseaux joyeux les faisaient aussitôt éclater, dont les plumes posaient leur douceur sur la mer au gré de leur vol capricieux. Pareil au mât de pavillon d’une caraque dressé sur sa coque peinte, la haute hampe brisée d’une lance, récemment reçue, se dressait sur le dos blanc de la baleine. Par moments, s’isolant du dais léger tendu par le nuage des oiseaux qui planaient au-dessus du poisson, l’un d’eux se perchait et se balançait sur la hampe, les longues plumes de sa queue flottant comme des banderoles.\par
Une joie paisible, une souveraine sérénité dans l’élan même enveloppaient le glissement de la baleine. Jupiter, taureau blanc emportant à la nage Europe accrochée à ses cornes gracieuses, coulant ses beaux yeux malicieux vers la jeune fille, filant, avec une vitesse ensorcelante, vers la demeure nuptiale de Crète, Jupiter ne surpassait pas, en sa majesté suprême, la glorieuse Baleine blanche en sa nage divine.\par
De chaque côté de son flanc éclatant, le flot partagé s’évasait largement, et la baleine soulevait une vague de séduction. Il n’est pas étonnant, dès lors, que certains chasseurs, indiciblement transportés et attirés par tant de sérénité se soient aventurés à l’attaquer, découvrant pour leur malheur que cette quiétude n’était qu’apparence et cachait des ouragans. Ainsi tu voguais, ô baleine, calme si calme aux yeux de ceux qui te voyaient pour la première fois, sans souci de tous ceux que tu avais déjà pris à ce piège pour les tromper et les détruire.\par
Ainsi à travers la tranquillité de la mer tropicale dont les vagues, au comble de l’extase, taisaient leurs applaudissements, Moby Dick avançait, cachant encore l’épouvante détenue par son corps, et dissimulant la hideur de sa mâchoire torve. Mais bientôt il se leva sur l’eau, et pendant un instant le marbre de son corps s’arqua, pareil au pont naturel de Virginie, il agita, en signe d’avertissement, l’étendard de sa queue, et le dieu révéla en entier sa grandeur, sonda et disparut. Les oiseaux blancs planèrent et plongèrent, puis s’attardèrent longuement sur le lac agité qu’il avait laissé.\par
Les avirons matés, les pagaies baissées, leurs écoutes choquées, les trois baleinières flottaient en silence, attendant que réapparût Moby Dick.\par
– Une heure, dit Achab, planté debout à l’arrière de sa pirogue, et il regarda loin au-delà de l’endroit où la baleine avait sondé vers les vastes et pâles espaces bleus qui laissaient sous le vent des vides caressants. Mais cela ne dura qu’un instant, ses yeux parurent se révulser tandis qu’ils embrassaient la mer du regard. La brise fraîchissait et la mer commençait à enfler.\par
– Les oiseaux !… les oiseaux ! cria Tashtego.\par
En longue file indienne, tels des hérons prenant leur vol, les oiseaux blancs se dirigeaient tous vers la pirogue d’Achab, et lorsqu’ils s’en furent approchés ils commencèrent à battre des ailes au-dessus de l’eau en tournoyant, avec des cris joyeux d’attente. Leur œil était plus aigu que celui de l’homme, Achab ne voyait rien sur la mer. Mais soudain, tandis qu’il scrutait encore et encore les profondeurs, il y discerna un point blanc pas plus gros qu’une hermine et qui montait et augmentait de volume à une vitesse surprenante, jusqu’à ce que, se retournant, il montrât brusquement les deux longues rangées crochues de ses dents éblouissantes remontant des abîmes indiscernables. C’était la gueule ouverte de Moby Dick et sa mâchoire tordue. Sa masse énorme, ombrée, était encore à demi dissimulée dans l’azur marin. Cette bouche éclatante bâillait juste sous la baleinière telle la porte ouverte d’une tombe de marbre. D’un long coup de son aviron de queue, Achab écarta l’embarcation de cette effrayante apparition. Puis, ordonnant à Fedallah de changer de place avec lui, il passa à l’avant, et, saisissant le harpon de Perth, il dit à ses hommes d’empoigner leurs avirons et d’être prêts à culer.\par
Grâce à cet opportun mouvement de rotation sur son axe, la proue est amenée à l’avance face à la tête de la baleine alors que celle-ci est encore sous l’eau. Mais comme s’il avait compris ce stratagème, avec l’intelligence maligne qu’on lui attribuait, Moby Dick se rejeta aussitôt sur le côté et plaça sa tête ridée par le travers sous la baleinière.\par
De part en part, l’embarcation frémit de tous ses bordés, tandis que la baleine, renversée obliquement sur le dos à la manière du requin qui mord, prenait lentement et délibérément l’étrave à pleine gueule de sorte que l’étroite et longue mâchoire inférieure, en volute de violon, se courba en l’air et que l’une de ses dents se prit dans une dame de nage. La blancheur nacrée tapissant l’intérieur de la mâchoire se trouvait à six pouces de la tête d’Achab et la surplombait. Dans cette posture, la Baleine blanche secouait maintenant le mince bois de cèdre à la manière précautionneuse dont un chat cruel secoue une souris. Fedallah, les bras croisés, contemplait la scène sans étonnement, mais les hommes couleur de tigre se bousculaient, cul par-dessus tête, pour se réfugier le plus possible à l’extrême arrière.\par
Les plats-bords souples plièrent en dedans et en dehors, cependant que la baleine folâtrait de cette façon diabolique avec l’embarcation en perdition, et comme son corps se trouvait sous la baleinière, on ne pouvait le harponner de l’avant, la proue étant, si l’on peut dire, en lui, et tandis les autres pirogues s’étaient involontairement arrêtées dans l’attente d’un dénouement rapide et inévitable, alors le dément Achab, rendu furieux par le contact tentateur avec son ennemi qui le tenait, tout vif et impuissant, entre ces mâchoires mêmes qu’il haïssait, comme un forcené empoigna de ses mains nues le long os et sauvagement pour lui faire lâcher prise. Tandis qu’il s’efforçait ainsi en vain, la mâchoire glissa loin de lui, les frêles plats-bords se courbèrent et s’effondrèrent en craquant, comme se refermaient, telles d’énormes cisailles, les mâchoires qui, mordant un peu plus en arrière, coupèrent complètement l’embarcation en deux et se verrouillèrent dans l’eau entre les deux épaves flottantes. Celles-ci s’écartèrent l’une de l’autre, les extrémités brisées abaissées, et dans la moitié arrière de la pirogue l’équipage s’agrippait aux plats-bords et tentait de se cramponner aux rames pour s’éloigner.\par
Dans l’instant qui précéda le happement de la baleinière, Achab devinant le premier l’intention de la baleine à la façon adroite dont elle dressa la tête, eut sa prise arrachée et fit, de sa main, un ultime effort pour soustraire l’embarcation à ses mâchoires. Mais celle-ci s’inclina sur le côté en pénétrant plus avant dans la bouche de la baleine et comme il se penchait pour pousser, il fut projeté à l’eau la tête la première.\par
Se retirant, dans un clapotis, à distance de sa proie, Moby Dick leva à la verticale sa tête blanche allongée, et l’abaissa tour à tour dans les lames, tournant en même temps avec lenteur son corps fuselé, de sorte que lorsque son vaste front ridé émergea – à quelque vingt pieds ou plus hors de l’eau – les vagues qui, maintenant, enflaient se pressèrent et se brisèrent en gerbes étincelantes autour de lui et jetèrent vindicativement plus \hspace{1em}haut leur écume tremblante\footnote{Ce mouvement est particulier au cachalot. Il tire son nom de ce balancement assez semblable au balancement du javelot, précédemment décrit. Ce geste permet à la baleine de mieux voir et de mieux comprendre ce qui l’entoure.}, de même que, lors d’une tempête, les vagues déroutées de la Manche, ne se retirent du pied du phare d’Eddystone que pour le couronner d’écume.\par
Mais reprenant bientôt sa position horizontale, Moby Dick mena une ronde rapide autour des hommes naufragés, brassant l’eau dans son sillage vengeur comme s’il se préparait à un assaut plus redoutable encore. La vue de la pirogue en éclats parut l’enrager, comme le sang des raisins et des mûres jetés devant les éléphants d’Antiochus dans le livre des Macchabées. Pendant ce temps Achab, à demi étouffé par l’écume soulevée par la queue insolente de la baleine, trop infirme pour nager put se maintenir en surface, même au cœur d’un tel tourbillon ; sa tête apparaissait comme une bulle ballottée que le moindre heurt pouvait faire éclater. De la poupe brisée, Fedallah le regardait avec une paisible indifférence. À l’autre extrémité en dérive, l’équipage cramponné ne pouvait lui être d’un quelconque secours, il suffisait aux hommes d’avoir à s’occuper d’eux-mêmes. La révolution terrifiante de la Baleine blanche, la rapidité planétaire avec laquelle elle resserrait ses anneaux était telle qu’elle semblait vouloir fondre sur eux. Et bien que les autres baleinières fussent intactes et se trouvassent à peu de distance, elles n’osaient pénétrer au cœur du tourbillon pour frapper, de crainte de donner le signal de destruction des naufragés en péril, tant d’Achab que des autres, et de perdre tout espoir d’échapper eux-mêmes. Sans la perdre un instant des yeux, les hommes restèrent, dès lors, au bord de cette zone lugubre dont la tête d’Achab était devenue le centre.\par
Cependant cette scène avait été depuis le début suivie du haut des mâts du navire. Brassant carré il s’était approché tant et si bien qu’Achab le héla : « Mettez le cap sur la… » mais à ce moment, il fut submergé par une vague soulevée par Moby Dick, il se débattit et soulevé par hasard sur la crête d’une lame, il hurla :\par
– Mettez le cap sur la baleine ?… Chassez-la.\par
La proue du {\itshape Péquod} pointa et, brisant le cercle enchanté elle sépara en effet la baleine blanche de sa victime. Tandis qu’elle s’éloignait, maussade, les baleinières s’élancèrent au secours.\par
Hissé dans la pirogue de Stubb, les yeux aveuglés, injectés de sang, la saumure séchant dans ses rides, Achab sentit ses forces, mises à rude épreuve, l’abandonner et son corps anéanti gisait impuissant au fond de l’embarcation, comme s’il eût été foulé aux pieds par des troupes d’éléphants. D’indicibles plaintes montaient de ses profondeurs aussi désolées que celles qu’on entend dans les gorges des torrents. Sa prostration fut si profonde qu’elle ne dura pas. En un instant, la souffrance de certains grands cœurs peut être aussi intense que la somme de toutes les souffrances plus superficielles éprouvées pendant leur vie entière par des âmes moins fortes. En de tels cœurs, une seule douleur est brève, mais si tel est le destin des dieux, leur vie faite d’instants fulgurants compte pour des siècles de douleur de toute l’humanité, car ces nobles natures sont le centre dont la circonférence enferme les âmes qui leur sont inférieures.\par
– Le harpon, dit Achab se levant à demi sur coude, le harpon est-il sauvé ?\par
– Oui, sir, puisqu’il n’a pas été jeté, le voici, répondit Stubb en le montrant.\par
– Posez-le devant moi… aucun homme ne manquera-t-il ?\par
– Un, deux, trois, quatre cinq… cinq avirons, sir et voilà leurs cinq hommes.\par
– Heureusement. Aide-moi, homme, je veux me mettre debout. Ainsi je la vois ! là ! là ! filant toujours sous le vent, quel jet élevé ! Ôtez vos mains de dessus moi ! La sève éternelle court à nouveau dans les os d’Achab ! Établissez la voile, levez les avirons, l’aviron de queue !\par
Lorsqu’une baleinière a été défoncée, il arrive souvent que son équipage, recueilli par une autre pirogue, contribue à la manœuvre de cette dernière et la chasse se poursuit avec ce qu’on appelle des avirons doublés. Il en allait ainsi maintenant. Mais cette puissance accrue de la baleinière n’égalait pas la puissance accrue de la baleine qui paraissait avoir triplé toutes ses nageoires et pris une telle vitesse qu’il était évident que la poursuite serait interminable sinon désespérée. Un équipage ne pouvait pas non plus fournir pendant un temps si long un effort aussi pénible et ininterrompu aux avirons, effort qui n’est supportable que lors d’un accident et pendant peu de temps. C’est alors le navire, cela arrive parfois, qui offre un moyen favorable de rejoindre la proie. Aussi les baleinières se dirigèrent-elles vers le {\itshape Péquod} et bientôt elles furent hissées à leurs pentoires, tandis que la pirogue brisée était également amenée à bord. Toute toile dessus, les bonnettes largement déployées de chaque côté comme des ailes d’albatros, le navire se lança, sous le vent, dans le sillage de Moby Dick. Le souffle lumineux de la baleine, aux intervalles bien connus et précis, fut régulièrement annoncé depuis les postes de vigie et lorsqu’il était signalé qu’elle avait sondé, Achab regardait l’heure et arpentait le pont, la montre de l’habitacle en main. Dès que s’était écoulée la dernière seconde du temps prévu, sa voix se faisait entendre :\par
– À qui le doublon, cette fois-ci ? La voyez-vous ? Si la réponse était négative, il donnait aussitôt l’ordre qu’on le hissât à son perchoir. Ce manège se répéta toute la journée, \hspace{1em} tantôt Achab restait immobile dans ses hauteurs, tantôt il arpentait le pont nerveusement.\par
Tandis qu’il allait et venait ainsi, sans mot dire, sauf pour interpeller les hommes en vigie ou pour ordonner qu’une voile fut hissée plus haut ou qu’on en larguât une autre, allant et venant ainsi, son chapeau rabattu, il passait chaque fois devant sa baleinière brisée gisant renversée sur le gaillard d’arrière, la proue broyée, la poupe déchirée. Enfin il s’arrêta devant elle, et comme de nouveaux nuages envahissent parfois un ciel nuageux, sur le visage du vieil homme passa une ombre nouvelle.\par
Stubb le vit et, dans l’intention peut-être, sans bravade toutefois, de prouver son courage invaincu et de garder ainsi une place d’honneur dans l’esprit du capitaine, il s’avança et regardant l’épave, il s’écria :\par
– L’âne a refusé le chardon, il lui piquait trop la bouche, sir, ah ! ah !\par
– Quel est l’être sans âme qui rit devant une épave ? Homme, homme ! Si je ne te savais pas courageux comme le feu intrépide et tout aussi naturellement, je jurerais que tu es un poltron. Une épave ne doit arracher ni une plainte ni un rire.\par
– Oui, sir, dit Starbuck, s’approchant à son tour, c’est une vision solennelle, un augure et un mauvais augure.\par
– Augure ? augure ?… le dictionnaire ! Si les dieux ont l’intention de parler ouvertement, honorablement aux hommes, ils leur parleront ouvertement, ils ne secoueront pas leurs têtes en faisant de sombres allusions de vieilles femmes. Allez ! Vous deux, vous êtes les deux pôles d’une seule et même chose. Starbuck c’est Stubb à rebours, et Stubb l’opposé de Starbuck et tous deux vous représentez l’humanité, cependant qu’Achab est seul entre tous, sans dieux, ni hommes pour voisins ! J’ai froid, je frissonne ! Et maintenant, là-haut ? La voyez-vous ? Donnez de la voix à chaque souffle, même si elle souffle dix fois par seconde !\par
Le jour s’en allait, seul bruissait encore l’ourlet de sa robe dorée, bientôt l’ombre fut là, mais les hommes guettaient encore aux mâts.\par
– On ne peut plus voir le souffle, à présent, sir… Il fait trop sombre ! dit une voix venue d’en haut.\par
– Quelle direction la dernière fois ?\par
– La même, sir, droit sous le vent.\par
– Bon ! Elle ira moins vite maintenant qu’il fait nuit. Amenez les cacatois et les bonnettes de perroquet, monsieur Starbuck. Il ne nous faut pas le devancer avant le matin, il va passéger à présent, et même mettre en panne un certain temps. Gouvernez plein vent arrière, là-haut descendez ! Monsieur Stubb, envoyez un nouveau guetteur au mât de misaine, et veillez à ce qu’il y reste jusqu’au matin. Puis s’avançant vers le doublon au grand mât :\par
– Hommes, cet or est le mien car je l’ai gagné, mais je le laisserai là jusqu’à ce que la Baleine blanche soit morte et cet or reviendra à celui qui la lèvera le jour où elle sera tuée. Et si ce jour-là, c’est encore moi qui la signale, alors dix fois sa valeur sera répartie entre vous tous ! Partez à présent ! Le pont est à toi, sir.\par
Ce disant, il se plaça à demi dans l’écoutille, rabattit son chapeau, et resta là debout jusqu’à l’aube, ne bougeant que pour regarder parfois où en était la nuit.
\chapterclose


\chapteropen
\chapter[{CHAPITRE CXXXIV. La chasse. Second jour}]{CHAPITRE CXXXIV \\
La chasse. Second jour}\renewcommand{\leftmark}{CHAPITRE CXXXIV \\
La chasse. Second jour}


\chaptercont
\noindent À l’aube, les postes de vigies des trois mâts furent ponctuellement occupés à nouveau.\par
– La voyez-vous ? cria Achab, après avoir laissé à la lumière un moment pour se répandre.\par
– On ne voit rien, sir.\par
– Tout le monde sur le pont et toute la toile dessus ! Elle va plus vite que je ne l’aurais cru… les cacatois ! oui, on aurait dû les laisser toute la nuit. Mais tant pis, ce n’est qu’un repos avant la ruée.\par
Soit dit en passant, cette poursuite opiniâtre d’une baleine particulière, poursuivie de l’aube à la nuit et de la nuit à l’aube, n’est à aucun degré insolite dans la pêcherie des mers du Sud. Car telle est l’étonnante compétence, la prescience fondée sur l’expérience, la confiance invincible acquises par certains capitaines nantuckais aux aptitudes exceptionnelles que sur la simple observation du comportement d’une baleine au dernier moment où elle a été vue, ils peuvent, dans des circonstances données, prédire avec assez de précision à la fois la direction qu’elle adoptera momentanément et sa vitesse probable. En ces cas-là, comme un pilote, sur le point de perdre de vue une côte familière qu’il veut retoucher en un point plus éloigné, restera près de son compas, prendra le relevé du cap actuellement en vue afin d’atteindre plus sûrement celui qu’il ne voit pas encore, ainsi fait le pêcheur avec la baleine car, après l’avoir chassée et assidûment repérée pendant plusieurs heures de jour, lorsque la nuit vient à la cacher, son sillage à venir dans l’obscurité est presque tracé pour l’esprit perspicace du chasseur, tout comme la côte dans celui du pilote. L’habileté surprenante du chasseur fait ainsi mentir le proverbe voulant que soit évanescent ce qui est écrit sur de l’eau, et il se fie à un sillage aussi sûrement qu’à la terre ferme. Et comme les hommes, montre en main, suivent à chaque pas la vitesse du léviathan de fer des trains modernes, à la façon dont les médecins prennent le pouls d’un bébé et disent légèrement : « le train qui vient ou qui part arrivera à tel endroit à telle heure », de même ces Nantuckais mesurent la vitesse du léviathan des profondeurs en l’accordant à celle qu’ils ont observée et se disent que dans tant d’heures, il aura parcouru deux cents milles et atteint tel ou tel degré de latitude ou de longitude. Pour qu’une telle précision soit en fin de compte utile, il faut que le vent et la mer se fassent les alliés du baleinier car à quoi sert au marin encalminé ou retenu par des vents contraires de savoir qu’il se trouve à trois lieues un quart du port ? Tant les impondérables entrent en ligne de compte dans la chasse à la baleine.\par
Le navire filait, laissant un sillon pareil à celui qu’un boulet de canon perdu creuse dans un champ plat.\par
– Par le sel et le chanvre ! s’écria Stubb, cette vitesse vous monte du pont dans les jambes et vous fait tinter le cœur. Ce bateau et moi nous sommes deux gars courageux ! Ha ! Ha ! Que quelqu’un me jette sur le dos à la mer car, par tous les chênes, mon échine est une quille. Ha ! Ha ! Nous filons à l’allure qui ne soulève pas de poussière !\par
– La voilà qui souffle… elle souffle !… elle souffle !… droit devant ! fut le cri qui tomba du sommet du mât.\par
– Oui, oui, dit Stubb, je le savais… tu n’échapperas pas… souffle toujours, crève-toi l’évent, ô Baleine ! le diable enragé en personne est à tes trousses ! Fais-toi sauter le cornet… Gonfle tes poumons ! Achab arrêtera le cours de ton sang comme le meunier qui ferme la vanne de l’eau du moulin !\par
– Et Stubb n’était que le porte-parole de presque tout l’équipage. La frénésie de la chasse les avait pour lors travaillés jusqu’à l’effervescence comme le renouveau un vin vieux. Quelles qu’eussent été leurs appréhensions indéfinies et leurs pressentiments, ceux-ci ne se manifestaient pas en raison de la terreur respectueuse accrue qu’inspirait Achab et parce qu’ils étaient mis en déroute, tels des lièvres craintifs de la prairie devant la charge du bison. La main du Destin avait dérobé leur âme. Les dangers de la veille les avaient attisés, la tension de la nuit précédente leur avait tordu les nerfs, la façon téméraire, aveugle dont leur navire fou poursuivait à corps perdu sa proie fuyante, tout contribuait à faire de leurs cœurs la boule d’un jeu de quilles. Le vent qui gonflait leurs voiles ventrues et dont les bras invisibles poussaient irrésistiblement le navire paraissait le symbole tangible de la volonté inconnue qui les asservissait à cette course.\par
Ils n’étaient qu’un seul homme et non trente. Tout comme le navire unique, qui les portait tous, alliait : chêne, érable, pin, fer, goudron et chanvre, pour ne former qu’une seule coque taillant sa route équilibrée et dirigée par la longue quille centrale, les particularités des hommes, la vaillance de l’un, la crainte de l’autre, l’offense de l’un, la culpabilité de l’autre, fusionnaient dans l’unité et les menait tous vers le but fatal vers lequel tendait Achab, à la fois leur seul seigneur et leur quille.\par
Le gréement vivait. Bras et jambes fleurissaient les hauts des mâts comme la touffe de palmes un grand palmier. Cramponnés aux espars d’une main, certains tendaient l’autre en des gestes impatients, d’autres, bercés au bout des vergues, abritaient leurs yeux de la vive lumière du soleil. Tous les espars portaient des hommes mûrs pour leur destin. Ah ! et comme ils transperçaient du regard cet infini bleu, en quête de l’instrument possible de leur destruction !\par
– Pourquoi ne donnez-vous pas de la voix, si vous la voyez ? cria Achab, lorsque plus rien ne se fit entendre pendant quelques minutes après la première annonce. Hissez-moi, hommes. Vous avez été trompés, Moby Dick n’envoie pas un seul jet de cette façon pour disparaître aussitôt.\par
Achab avait raison, dans l’impétuosité de leur impatience, les hommes avaient pris autre chose pour un souffle, comme on l’allait voir, car à peine Achab avait-il atteint son perchoir, à peine l’estrope était-elle amarrée au cabillot du pont qu’il donna l’accent tonique d’un orchestre qui retentit comme une décharge d’artillerie. Trente poumons de cuir lancèrent ensemble un cri de triomphe cependant que – beaucoup plus près du navire que le souffle imaginaire, à moins d’un mille sur l’avant – Moby Dick en personne apparut ! Et ce n’était pas par des souffles calmes et indolents, ni par le paisible jaillissement de sa course mystique, que la Baleine blanche révéla sa présence, mais par le phénomène beaucoup plus étonnant du saut. Montant des profondeurs à sa plus extrême vitesse, le cachalot projette ainsi sa masse tout entière à l’air libre, et la montagne d’écume éblouissante qu’il a soulevée le dénonce à une distance de sept milles et plus. En de pareils moments, sur les vagues furieusement arrachées qu’il secoue, ce saut parfois est un défi.\par
– La voilà qui saute ! la voilà qui saute ! fut le cri qui accompagna les incommensurables bravades de la Baleine blanche se lançant vers le ciel comme un saumon. Si brusquement surgie dans la plaine bleue de la mer et projetée sur le bleu encore plus intense du ciel, l’écume qu’elle avait soulevée brilla de façon insupportable, aussi aveuglante qu’un glacier, puis son éblouissante intensité s’atténua progressivement jusqu’à n’être plus que la brume indistincte qui, dans la vallée, annonce une averse.\par
– Oui, accomplis ton dernier saut vers le soleil, Moby Dick ! s’écria Achab, ton heure est venue et ton harpon est prêt ! Tous en bas, sauf un homme au mât de misaine. Les baleiniè- res ! Parés !\par
Dédaigneux des fastidieuses échelles de corde, les hommes, tels des étoiles filantes, glissèrent au pont par les galhaubans et les drisses, cependant que de manière moins foudroyante, quoique prompte, Achab était descendu de son perchoir.\par
– Les pirogues à la mer ! cria-t-il dès qu’il eut atteint la sienne, une baleinière de rechange gréée l’après-midi précédent. Monsieur Starbuck, le navire est à toi… tiens-toi à l’écart des pirogues, mais ne t’en éloigne pas non plus. Débordez tous !\par
Comme pour les frapper d’une terreur immédiate. Moby Dick, faisant volte-face, attaqua le premier en fonçant sur les trois équipages. La baleinière d’Achab était au centre et, encourageant ses hommes, il leur dit qu’il comptait prendre la baleine de front, c’est-à-dire ramer droit sur son front, manœuvre assez courante qui permet, à une certaine distance d’éviter l’assaut du monstre dont la vision est latérale. Mais avant que cette distance fût franchie et tandis que la baleine pouvait encore voir aussi clairement les trois pirogues que les trois mâts du navire, la Baleine blanche, battant l’eau à une vitesse furieuse, rua, le temps d’un éclair, parmi les pirogues, les mâchoires ouvertes et la queue cinglante, mena des deux côtés une effrayante bataille et, indifférente aux dards jetés des trois baleinières, sembla seulement animée de l’intention de pulvériser chaque bordé des embarcations. Habilement manœuvrées, pirouettant comme des chevaux de combat dressés, celles-ci l’évitèrent pendant un moment, à un cheveu près bien souvent, tandis que le surnaturel cri de guerre d’Achab déchiquetait tout autre cri.\par
Mais finalement, dans ses évolutions complexes, la Baleine blanche croisa et recroisa de mille manières les lignes qui lui étaient attachées, les raccourcit halant les pirogues condamnées vers les fers fichés dans sa chair bien qu’elle s’écartât un instant comme pour prendre l’élan nécessaire à une charge plus épouvantable encore. Saisissant cette occasion, Achab donna du mou à la ligne puis embraqua en la secouant, espérant ainsi la désenchevêtrer, lorsque apparut une vision plus sauvage que les dents en créneaux des requins !\par
Pris et emmêlés, tirebouchonnant dans l’inextricable fouillis de la ligne, les harpons et les lances, hérissant leurs pointes et leurs barbes, vinrent gicler en bouquet contre la galoche d’avant de la baleinière d’Achab. Il n’y avait qu’une chose à faire. Saisissant son couteau d’embarcation, il coupa et en dedans, et en travers, et dehors de ce faisceau d’acier, tira sur la ligne qui passait derrière, la tendit à son premier rameur et la tranchant au ras des plats-bords, la fit tomber à la mer et fut à nouveau libre. Au même instant la Baleine blanche fonça soudain dans l’enchevêtrement des autres lignes, en attirant irrésistiblement, les pirogues de Stubb et de Flask qui étaient les plus empêtrées, vers les palmes de sa caudale. Elle jeta les baleinières l’une contre l’autre comme deux coques roulées sur une plage battue des brisants puis, sondant, elle disparut dans un maelström bouillonnant, où les éclats odorants de cèdre dansèrent une ronde effrénée pendant un moment comme une noix de muscade râpée dans un bol de punch vivement remué.\par
Tandis que les deux équipages se débattaient, cherchant les bailles à ligne renversées, les avirons et tout le matériel flottant, tandis que le petit Flask sautait obliquement comme une fiole vide, levant les jambes par saccades pour échapper aux redoutables mâchoires des requins, et que Stubb hurlait à pleine poitrine qu’on vienne le ramasser à la cuillère, tandis que la ligne rompue du vieillard lui permettait désormais de ramer jusqu’à cet étang mousseux pour y sauver ce qu’il pourrait, au sein de ce sauvage amoncellement de mille dangers, la baleinière d’Achab, encore indemne, parut soulevée vers le ciel par des fils invisibles. Jaillissant des profondeurs à la verticale, telle une flèche, la Baleine blanche la bouta de son large front et l’envoya tournoyer en l’air, jusqu’à ce qu’elle retombât sens dessus dessous. Achab et ses hommes luttèrent pour s’en dégager comme des phoques sortant d’une caverne au bord de la mer.\par
L’élan de la baleine au moment où elle émergea modifia quelque peu sa direction à la surface, la lançant, malgré elle, de côté à une petite distance du centre de la destruction qu’elle avait accomplie ; lui tournant le dos, elle resta un moment à tâtonner délicatement autour d’elle avec la queue et chaque fois qu’elle palpait un aviron, un bout de planche, le plus petit copeau, la moindre miette des pirogues, elle la redressait promptement et l’abattait par le travers, fouettant la mer. Mais bientôt, satisfaite de la besogne accomplie, elle poussa dans la vague son front ridé, et entraînant à sa suite les lignes emmêlées, elle poursuivit sa route sous le vent, à l’allure régulière d’un voyageur.\par
Comme la veille, le navire vigilant avait suivi la scène et une fois de plus il vint au secours, mit une embarcation à la mer et ramassa les hommes qui nageaient, les bailles, les avirons, tout ce qui était à portée et les déposa en sûreté sur le pont. Quelques épaules luxées, quelques poignets et quelques chevilles foulées, des contusions livides, des harpons et des lances tordues, un écheveau inextricable de lignes, des avirons et des bordés brisés, tout était là ! Mais il n’y avait point de mort et les blessures étaient sans gravité. Comme Fedallah la veille, Achab était cramponné à la moitié intacte de sa baleinière, qui lui assurait une assez bonne flottabilité, et il n’était pas dans l’état d’épuisement du jour précédent.\par
Lorsqu’on l’eut aidé à prendre pied sur le pont, tous les yeux se rivèrent sur lui. Au lieu de se tenir seul, il était à demi pendu à l’épaule de Starbuck qui avait été le premier à se porter à son aide. Sa jambe d’ivoire avait été arrachée et il n’en restait qu’un court éclat aigu.\par
– Oui, oui, Starbuck, il est doux de s’appuyer parfois, quel que soit celui qui s’appuie. Pourquoi le vieil Achab ne s’est-il pas plus souvent appuyé ?\par
– La virole n’a pas tenu, sir, dit le charpentier en s’avançant. J’avais fait cette jambe, avec du bon bois.\par
– Mais pas d’os cassés, j’espère sir, dit Stubb avec une inquiétude sincère.\par
– Oui ! Brisés en éclats, Stubb ! voyez-vous. Mais même avec un os rompu, le vieil Achab est intact, et mes os vivants ne sont pas plus à moi que cet os mort que j’ai perdu. Il n’est point de Baleine blanche, d’homme, ni de démon, capable d’effleurer même le vieil Achab dans son être intérieur inaccessible. Existet-il un plomb qui atteigne les profondeurs, un mât pour égratigner le ciel ? Là-haut ! Quelle direction ?\par
– Droit sous le vent, sir.\par
– Alors, barre dessus, gardiens du navire, à foc de voiles une fois de plus ! Alpalez les autres baleinières de rechange et gréez-les. Monsieur Starbuck faites l’appel de l’équipage des pirogues.\par
– Permettez-moi de vous soutenir d’abord jusqu’à la rambarde, sir.\par
– Oh, oh, oh ! cette esquille me blesse à présent ! Destin maudit ! qui veut que l’âme aussi invincible du capitaine ait un second aussi lâche.\par
– Sir ?\par
– Mon corps, homme, pas toi. Donne-moi quelque chose en guise de canne, là cette lance brisée fera l’affaire. Fais l’appel des hommes. Il est sûr que je ne l’ai point encore vu. Par le ciel, cela ne peut pas être !… manquant ?… vite, rassemblez-les tous.\par
Le soupçon du vieil homme se confirma. Le Parsi manquait à l’appel.\par
– Le Parsi ! s’écria Stubb, il a dû être pris dans…\par
– Que la fièvre jaune t’étrangle ! Courez, vous tous… en haut, en bas la cabine, le gaillard d’avant… trouvez-le, trouvezle… il n’est pas parti… pas parti !\par
Mais ils revinrent bientôt lui annoncer qu’ils ne le trouvaient nulle part.\par
– Oui, sir, dit Stubb… pris dans les nœuds de votre ligne… il m’a semblé le voir entraîné au fond…\par
– Ma ligne ! ma ligne ! Parti ?… Parti ?… Que signifie ce petit mot ? Quel glas funèbre ébranle-t-il pour que le vieil Achab tremble comme s’il était le clocher ? Le harpon aussi !… cherchez-le dans ce désordre, là… le voyez-vous ?… le fer forgé, homme, celui de la Baleine blanche… non ; non, fou stupide ! C’est cette main même qui l’a lancé il est dans le poisson ! Ohé, là-haut, ayez-la à l’œil ! Vite ! Tout le monde au gréement des pirogues… réunissez les avirons… harponneurs ! les fers ! les fers !… Hissez à bloc les cacatois… serrez toutes les écoutes !… La barre ! doucement doucement… Je ferai dix fois le tour du globe démesuré… oui, et je plongerai droit en son travers, mais je la tuerai !\par
– Grand Dieu ! pour un instant si bref, montre-toi, s’écria Starbuck, jamais, jamais, tu ne l’attraperas, vieillard. Au nom de Jésus, n’insiste pas davantage, c’est pire qu’une satanique folie. Deux jours de chasse, deux jours d’épaves, ta propre jambe arrachée une fois de plus… ton mauvais ange perdu… tous les bons anges en foule t’avertissant… Que veux-tu de plus ?… Chasserons-nous ce poisson meurtrier jusqu’à ce qu’il ait englouti le dernier homme ? Devons-nous être entraînés par lui jusque dans les abîmes de la mer ?… jusque dans les profondeurs de l’enfer ? Oh ! oh ! C’est impiété et blasphème que de le poursuivre plus longtemps !\par
– Starbuck, dernièrement j’ai été étrangement ému par toi, dès l’instant où nous avons vu ce que tu sais dans les yeux l’un de l’autre. Mais dans cette affaire de baleine, que ton visage soit comme la paume de cette main… lisse et dépourvu de lèvres, vide. Achab est Achab pour jamais, homme ! Tout cela est écrit irrémédiablement. Nous en avons fait la répétition générale, toi et moi, un billion d’années avant que roulât cet Océan. Fou ! Je suis au service des Parques, j’agis selon des ordres. Écoute, subalterne ! tu dois obéir aux miens. Rassemblez-vous autour de moi, hommes. Vous voyez un vieil homme rompu, appuyé sur une lance brisée, soutenu par un pied solitaire. C’est Achab… le corps d’Achab… mais l’âme d’Achab est une scolopendre animée de mille pieds. Comme les câbles qui remorquent les frégates démâtées dans la tempête, je me sens tendu, les torons à demi coupés, et telle peut être mon apparence. Mais avant que je casse, vous m’entendrez craquer, et avant que vous entendiez cela, sachez que l’amarre d’Achab remorque toujours son intention. Croyez-vous, hommes à ce qu’on appelle des augures ? Alors riez à gorge déployée et criez « encore » ! Car avant de couler, les choses qui vont se perdre remontent deux fois à la surface, et remontent encore avant d’être englouties à jamais. Il en va ainsi de Moby Dick… pendant deux jours, elle est remontée… demain sera la troisième. Oui, homme, elle remontera une fois encore… mais seulement pour son dernier souffle ! Vous sentez-vous le courage, hommes ?\par
– Aussi intrépide que le feu, dit Stubb.\par
– Et tout aussi naturellement, murmura Achab, et tandis que les hommes s’en allaient vers l’avant, il continua à chuchoter : « Ce qu’on appelle augures ! Et hier j’ai parlé de même à Starbuck, au sujet de ma baleinière brisée. Oh ! avec quelle ardeur je cherche à arracher du cœur des autres ce qui est si solidement rivé dans le mien ! Le Parsi… le Parsi… parti, parti ? Et il devait partir le premier… mais je devais le revoir avant que je périsse… Comment cela peut-il être ? Cette énigme déconcerterait tous les hommes de loi soutenus par les fantômes de tous les juges qui ont vécu… elle picore mon cerveau comme un bec de vautour… Mais je la résoudrai, je la résoudrai cependant ! »\par
Lorsque l’ombre vint, on voyait toujours la baleine sous le vent.\par
De sorte qu’une fois de plus on diminua la toile et tout se passa à peu près comme la nuit précédente, mais les marteaux et le bourdonnement de la meule peuplèrent le silence jusqu’au lever du jour, tandis que les hommes peinaient à la lueur des falots pour gréer et parer soigneusement les baleinières de rechange et aiguiser les armes pour la chasse du lendemain. Pendant ce temps, le charpentier tirait de la quille brisée de la pirogue d’Achab une nouvelle jambe à son intention. Achab, immobile, se tient dans l’écoutille. Pareil au style du cadran solaire attendant l’aube, son regard d’héliotrope, sous son chapeau rabattu, fixe l’orient.
\chapterclose


\chapteropen
\chapter[{CHAPITRE CXXXV. La chasse. Troisième jour}]{CHAPITRE CXXXV \\
La chasse. Troisième jour}\renewcommand{\leftmark}{CHAPITRE CXXXV \\
La chasse. Troisième jour}


\chaptercont
\noindent Le matin du troisième jour se leva clair et frais. Une fois de plus, l’homme de vigie solitaire au mât de misaine fut relevé par des grappes de guetteurs qui, de jour, étaient suspendues à chaque mât et presque à chaque espar.\par
– La voyez-vous ? cria Achab, mais la baleine n’était pas encore en vue.\par
– Nous sommes cependant infailliblement dans son sillage, et il n’y a qu’à le suivre. Timonier, maintiens régulière la barre, comme tu le fais et l’as fait jusqu’à maintenant. Quel beau jour encore ! Serait-ce le jour de la création, ce monde serait-il fait pour la demeure estivale des anges, ce matin serait-il celui où on leur en ouvrirait les portes que cette aube ne serait pas plus belle. Il y aurait là matière à méditation si Achab avait le temps de penser, mais Achab ne pense jamais, il ne fait que ressentir, ressentir, ressentir et c’est assez éprouvant pour un être humain ! Penser est une audace. Dieu seul en a le droit et le privilège. La pensée est, ou devrait être, impassible et sereine, et nos pauvres cœurs sont trop palpitants, et nos pauvres cerveaux trop inquiets, pour lui donner une place. Pourtant j’ai cru, parfois, que ma tête était paisible, d’un calme glacé, tant mon crâne, est plein de craquements comme un verre brisé par l’eau devenue glace. Cependant mes cheveux poussent, ils poussent en cet instant même, et c’est la chaleur qui dicte leur croissance, mais non, ils ne doivent être que cette herbe commune qui pousse partout, aussi bien dans les fissures laissées par les glaces du Groenland que sur la lave du Vésuve. Comme le vent sauvage les malmène, il les fouette à mon front telle la fibre des voiles déchirées qui cingle le navire secoué. Un vent abject, il a dû traverser les corridors et les cellules des prisons, les salles d’hôpitaux et vient souffler ici d’innocents nuages. Honte à cet air corrompu ! Si j’étais le vent, je ne soufflerais plus sur un monde aussi mauvais et misérable. Je ramperais jusqu’à une cave et je m’y glisserais furtivement. Pourtant le vent est héroïque et noble ! Qui l’a jamais conquis ? Dans tout combat, c’est lui qui porte le dernier coup et le plus rigoureux. Courez une lance contre lui et vous ne faites que le traverser. Ah ! c’est un vent lâche que celui qui glace les hommes nus mais se dérobe au moindre coup. Achab même a plus de courage et de noblesse que cela. Si seulement le vent avait un corps, mais toutes choses qui insultent à l’homme et l’exacerbent sont immatérielles, mais seulement en tant qu’objets et non en tant qu’agents spirituels. Et c’est une différence lourde de ruse méchante à l’extrême ! Pourtant je le redis, et je le jure maintenant, le vent est toute gloire et toute grâce. Ces chauds alizés du moins, qui soufflent droit devant eux dans les ciels clairs, avec une douceur ferme, forte, vigoureuse et ne se détournent pas de leur voie quels que soient les tournoiements et les louvoiements des plus vils courants marins et les incartades et la rapidité des plus puissants Mississipis de la terre qui ne savent pour finir où aller. Et par les pôles éternels, ces mêmes alizés emportent mon bon navire, ces alizés – ou quelque chose qui leur ressemble – quelque chose d’aussi immuable, et avec la même énergie, emporte la quille de mon âme ! Allons donc !… Ohé, là-haut ! Que voyezvous ?\par
– Rien sir.\par
– Rien ! et il est bientôt midi ! Le doublon n’a-t-il pas d’amateur ? Voyez le soleil ! Oui, oui, il doit en être ainsi. J’ai dû le dépasser ! Comment ai-je pris de l’avance ? Oui, c’est lui qui mène la chasse à présent et non plus moi… Mauvaise affaire. J’aurais dû m’en douter aussi. Imbécile ! Les lignes… les harpons qu’elle entraîne avec elle. Oui, oui, je l’ai dépassée la nuit dernière. Parez à virer. Tous en bas ! sauf les hommes aux postes habituels de vigie ! Parez les vergues !\par
Courant comme il l’avait fait, le vent sur la hanche le {\itshape Péquod}, contre-brassé, marchait au plus près en barattant l’écume de son propre sillage.\par
– Il met le cap contre le vent sur la mâchoire béante murmura Starbuck, tout en levant sur la lisse le bras bordé de la grande vergue. Que Dieu nous garde, mais déjà je me sens depuis le dedans trempé jusqu’aux os et ma chair en moisit. Je crains de désobéir à Dieu en lui obéissant !\par
– Parés à me hisser ! s’écria Achab en se dirigeant vers sa chaise de chanvre. Nous ne devrions pas tarder à le rencontrer.\par
– Oui, oui, sir, Starbuck obtempéra aussitôt et une fois encore Achab s’éleva dans la mâture.\par
Une heure s’écoula, feuille d’or battu jusqu’aux confins des siècles. Le temps lui-même retenait son souffle dans son anxieuse attente. Mais enfin, à trois points environ du côté du vent, Achab signala à nouveau le souffle et des trois mâts jaillirent, aussitôt, trois cris qu’on eût dit envoyés par les langues de feu.\par
– Je te ferai face front contre front, cette troisième fois, Moby Dick ! Ohé sur le pont ! Orientez les voiles de très près, forcez de toile dans l’épi du vent. Elle est encore trop loin pour que nous mettions à la mer, à présent, monsieur Starbuck. Les voiles faseyent ! Surveillez le timonier avec un maillet ! Cela va mieux, nous marchons vite et je dois descendre. Mais que depuis là-haut mon regard embrasse encore une fois la mer, ce temps, je peux le prendre. Un vieux, vieux spectacle, pourtant si jeune, oui, il n’a changé en rien depuis le temps où je le contemplais, enfant, des dunes de Nantucket ! Le même… toujours le même, le même pour Noé et pour moi. Il y a une douce averse sous le vent.\par
Ils sont si beaux les espaces sous le vent ! Ils doivent conduire quelque part – à quelque chose de plus que la simple terre, à des palmes plus sublimes que les palmes des arbres. Sous le vent ! c’est là que va la Baleine blanche ; alors, regardons du côté du vent, le meilleur mais le plus déchirant. Mais adieu, adieu vieille tête de mât ! Qu’est-ce là ? De la verdure ? Oui, de menues mousses dans les courbures fendues du bois. Il n’est point sur la tête d’Achab de ces taches vertes laissées par les saisons ! C’est là la différence entre la vieillesse de l’homme et celle de la matière. Mais, oui, vieux mât, nous prenons de l’âge ensemble, mais nos quilles sont également saines, n’est-ce pas, mon bateau ? Oui, moins une jambe, c’est tout. Par le ciel, ce bois mort, à tous égards, a le meilleur de ma chair. Incomparablement… et j’ai connu des navires tirés d’arbres morts survivant à des hommes débordant d’une vie transmise par des pères vigoureux. Qu’est-ce qu’il avait dit ? Qu’il irait toujours devant moi, mon pilote, et que je devais le revoir ? Mais où ? En admettant que je descende ces escaliers sans fin, aurai-je des yeux au fond de la mer ? Et la nuit durant, je me suis éloigné de l’endroit où il fut englouti. Oui, oui, comme tant d’autres, tu as dit, en ce qui te concernait, l’affreuse vérité, ô Parsi, mais tu as visé Achab trop bas. Adieu, tête de mât, garde à l’œil la Baleine blanche pendant que je serai en bas. Nous parlerons demain, non, cette nuit, quand la Baleine blanche sera là, amarrée par la tête et par la queue.\par
Il donna l’ordre qu’on le descendît et, regardant toujours autour de lui cet air bleu qu’il fendait, il fut doucement amené sur le pont.\par
En temps voulu, les baleinières furent mises à la mer, mais tandis qu’il était debout à l’arrière de sa chaloupe, sur le point d’être descendue à l’eau, Achab fit un signe de la main à Starbuck qui, sur le pont, tenait une des manœuvres de palan de bossoir et lui demanda d’interrompre la manœuvre.\par
– Starbuck !\par
– Sir ?\par
– Pour la troisième fois, le navire de mon âme repart pour ce voyage, Starbuck.\par
– Oui, sir, tu l’auras voulu.\par
– Il est des navires qui quittent leur port et sont à jamais perdus, Starbuck !\par
– C’est la vérité, sir, la plus triste des vérités.\par
– Il est des hommes qui meurent au jusant, d’autres à la marée basse, d’autres au paroxysme de la marée haute… et je me sens pareil à la vague ourlée d’écume et qui va déferler, Starbuck. Je suis vieux. Serrons-nous la main, homme.\par
Leurs mains se rencontrèrent, ils plongèrent dans les yeux l’un de l’autre, les larmes s’amoncelèrent dans les yeux de Starbuck.\par
– Oh ! mon capitaine, mon capitaine !… noble cœur… ne pars pas… ne pars pas ! Vois, c’est un homme brave qui pleure tant son effort de persuasion est une agonie !\par
– Débordez ! cria Achab en secouant l’étreinte de son second. Paré, l’équipage ! En un instant la pirogue eut débordé près de la poupe.\par
– Les requins ! les requins ! cria une voix du hublot bas de la cabine, ô maître, mon maître, revenez !\par
Mais Achab n’entendit rien car à ce moment-là il élevait la voix et sa baleinière fila.\par
Pourtant la voix disait vrai. Á peine la pirogue s’était-elle éloignée du navire, que de nombreux requins, émergeant semblait-il des eaux sombres sous la quille, vinrent méchamment mordre les pelles des avirons, chaque fois qu’elles plongeaient dans l’eau et c’est ainsi qu’ils accompagnèrent de leurs dents la baleinière. Ce n’est pas chose inhabituelle dans ces mers qui en sont infestées ; les requins parfois semblent avoir la même prescience que les vautours qui planent au-dessus des étendards des armées de l’Est en marche. Mais c’étaient les premiers squales vus par le {\itshape Péquod} depuis que la Baleine blanche avait été signalée. Était-ce dû à l’équipage des barbares à peau jaune tigre de l’équipage d’Achab, leur odeur musquée flattait-elle davantage l’odorat des requins, il est reconnu qu’ils y sont sensibles, étaitce pour une autre raison, toujours est-il qu’ils suivaient sa seule pirogue sans s’attaquer aux autres.\par
– Cœur d’acier forgé ! murmura Starbuck, penché au bastingage, suivant des yeux la baleinière qui s’éloignait, peux-tu rester inébranlable devant pareille vision, ta quille mise à la mer parmi les requins voraces, et suivi par eux, la gueule ouverte pour accompagner ta chasse, et au moment crucial du troisième jour ? Car lorsque trois jours s’écoulent en une poursuite incessante et ardente, le premier est, sans doute aucun, le matin, le deuxième le milieu du jour, et le troisième le soir et la fin quelle qu’elle doive être. Oh ! mon Dieu ! qu’est-ce qui me transperce et me laisse si mortellement calme, et pourtant en attente, figé au sommet d’un frisson d’horreur ! Les choses de l’avenir passent \hspace{1em}devant \hspace{1em}moi, \hspace{1em}formes \hspace{1em}vides, \hspace{1em}squelettes. \hspace{1em}Tout \hspace{1em}le passé s’estompe. Mary, mon amie ! Une pâle gloire t’efface derrière moi. Mon fils, il me semble ne voir que tes yeux devenus d’un bleu étonnant. Les plus obscurs problèmes de la vie semblent s’éclaircir mais des nuages se glissent entre eux. Serais-je à la fin du voyage ? Je me sens les jambes faibles comme celui qui a marché tout le jour. Sens ton cœur !… Bat-il encore ? Remuetoi, Starbuck ! chasse cela… remue, remue ! Hausse la voix !… Tête de mât, vois-tu mon enfant qui agite la main sur la colline ? Insensé… Ohé, là-haut !… Ne perdez pas de vue les baleinières… signalez la baleine ! Ohé, encore ! chassez cet aigle ! Voyez, il donne des coups de bec, il réduit en lambeaux le penon, dit-il en désignant le drapeau rouge qui flottait au grand mât… Ah ! il l’emporte ! Où est le vieil homme à présent ? Vois-tu ce spectacle, oh ! Achab… tremble, tremble !\par
Les baleinières s’étaient peu éloignées, lorsqu’un signe des hommes de vigie, pointant leur bras vers la mer, apprit à Achab que la baleine avait sondé, mais soucieux d’en être proche lorsqu’elle émergerait à nouveau, il maintenait sa route légèrement oblique par rapport à celle du navire. L’équipage envoûté gardait le plus profond silence cependant que les vagues martelaient l’étrave à battements réguliers.\par
– Enfoncez vos clous, ô vagues, enfoncez-les jusqu’à leurs têtes ! Mais vous frappez sur une chose qui n’a point de couvercle, et ni cercueils ni corbillards ne sauraient être miens… seule la corde sera ma mort ! Ha ! ha !\par
Soudain les eaux autour d’eux s’élargirent en vastes cercles, puis se soulevèrent rapidement comme si elles glissaient autour d’un iceberg immergé montant rapidement à la surface. Un grondement sourd se fit entendre, un bourdonnement souterrain, et tous suspendirent leurs souffles. Une vaste forme dépenaillée par les lignes, les harpons et les lances qu’elle remorquait, jaillit obliquement de la mer. Enveloppée d’un mince voile de brume, elle plana un instant dans l’air irisé, puis retomba lourdement dans la mer. Les eaux giclèrent à trente pieds de hauteur comme autant de fontaines, puis se brisèrent en une averse, encerclant la baleine marmoréenne d’une mousse de lait frais.\par
– En avant ! cria Achab aux canotiers et les pirogues foncèrent à l’attaque. Mais Moby Dick, enragé par les fers de la veille qui lui fouillaient la chair, semblait possédé par l’armée des anges déchus. Sous la peau transparente de son front, ses tendons saillaient en nœuds épais ; il s’élança, la tête la première, et la queue fouettante parmi les baleinières, les séparant une fois de plus, et jeta à l’eau les harpons et les lances des deux seconds comme il frappait l’extrême de leur avant. Il laissa celle d’Achab sans une égratignure.\par
Cependant que Daggoo et Queequeg aveuglaient les interstices entre les bordés et que la baleine en s’éloignant d’eux révélait l’un de ses flancs en se retournant, un cri bref s’éleva. Lié au dos du poisson, garrotté par les tours innombrables de la ligne que la baleine avait enroulée durant la nuit, on vit le corps à demi déchiqueté du Parsi, son vêtement noir en lambeaux et ses yeux exorbités fixés droit sur le vieil Achab.\par
Le harpon lui en tomba des mains.\par
– Trahi ! J’ai été trahi ! dit-il en prenant faiblement un long souffle. Oui, Parsi ! Je te revois… Oui, et tu pars le premier… et ceci, ceci alors est le corbillard que tu as annoncé. Mais je te tiens au dernier mot de ta prophétie. Où est le second corbillard ? Seconds, retournez au navire ! Ces baleinières sont maintenant hors d’usage, réparez-les à temps si vous le pouvez et rejoignez-moi. Sans cela, il suffit qu’Achab meure… Assis, hommes ! À la première velléité de sauter de la baleinière, je vous harponne. Vous n’êtes plus des hommes, vous êtes mes bras et mes jambes, aussi obéissez-moi… Où est la baleine ? A-telle sondé à nouveau ?\par
Mais il regardait trop près de la pirogue. Moby Dick maintenant nageait régulièrement en avant, comme s’il n’avait d’autre souci que de s’enfuir avec le cadavre qu’il portait, comme si cette dernière rencontre n’était qu’une halte de son voyage sous le vent. Il avait presque dépassé le navire – qui jusqu’alors suivait le cap opposé – et maintenant avait mis en panne. Il semblait nager aussi vite que possible dans la seule intention de poursuivre son chemin dans l’Océan.\par
– Oh ! Achab, s’écria Starbuck, il n’est pas trop tard, même en ce troisième jour, pour renoncer. Vois, Moby Dick ne te cherche pas ! C’est toi, toi seul qui le cherches dans ta folie !\par
Établissant ses voiles dans le vent qui fraîchissait, l’embarcation solitaire fut rapidement entraînée, à la voile et aux avirons, sous le vent. Et enfin, tandis qu’elle glissait au flanc du navire, Achab se trouva assez proche pour distinguer clairement le visage de Starbuck penché sur la rambarde. Il lui cria de virer et de le suivre, point trop vite, à une distance convenable. Jetant un regard vers les têtes de mâts, il vit Tashtego, Queequeg et Daggoo qui y grimpaient hardiment, tandis que les canotiers se balançaient encore dans les baleinières endommagées que l’on venait juste de hisser et que des hommes s’affairaient déjà à réparer. Par les sabords, il aperçut une seconde Stubb et Flask occupés à choisir de nouveaux fers dans un fagot de lances et de harpons. Il vit tout cela. Il entendit les marteaux résonner sur les pirogues brisées, bien d’autres marteaux rivaient un clou dans son cœur. Mais il se ressaisit. Remarquant que le grand mât n’avait plus de penon, il hurla à Tashtego, qui venait d’en atteindre la pointe, de redescendre chercher un autre drapeau, un marteau et des clous et de le clouer au mât même.\par
Qu’elle fût exténuée par ces trois jours de chasse et par l’effort qu’elle devait fournir dans son gréement de lignes, ou que ce fût une supercherie dissimulée ou malice pure, quoi que ce fût, la Baleine blanche ralentit, si bien que la pirogue la rejoignait à nouveau rapidement, bien qu’en vérité elle n’eût pas pris une distance comparable à celle qu’elle avait auparavant. Et Achab avançait toujours sur les vagues, accompagné par les requins impitoyables obstinément attachés à sa baleinière, mordant sans cesse les avirons, dont les pelles mâchées, déchiquetées, abandonnaient à chaque mouvement des éclats sur la mer.\par
– N’en ayez cure ! ces dents ne font que donner de nouveaux tolets à nos avirons. Souquez ! la mâchoire du requin est un meilleur appui que l’eau fuyante.\par
– Mais sir, à chaque morsure, les pelles amincies deviennent de plus en plus courtes.\par
– Elles dureront bien assez ! Souquez ! Oui, de toutes vos forces à présent. Nous l’approchons. L’aviron de queue ! Prenez l’aviron de queue ! Laissez-moi passer !\par
Sur ce, deux des canotiers l’aidèrent à se frayer sa voie dans l’embarcation toujours en pleine envolée. Enfin, elle fut lancée de côté et aborda le flanc de la Baleine blanche, celle-ci parut étrangement oublier d’avancer, comme il arrive parfois aux baleines, et Achab se trouva presque au sein de la brume fumant à son sommet et qui, s’échappant de son évent, ourlait le grand Monadnock de sa bosse. Si près d’elle qu’il était ! Alors, le corps bandé en arrière, les deux bras haut levés pour balancer son fer, il jeta son harpon féroce avec une malédiction plus féroce encore dans la baleine haïe. L’acier et la malédiction s’enfoncèrent jusqu’à la garde comme absorbés par un marécage. Moby Dick se tordit de côté, roula convulsivement son flanc contre l’étrave, et sans l’endommager, fit chavirer si brutalement la pirogue, que si Achab n’avait pas été cramponné à la partie surélevée du plat-bord, il eût été une fois de plus jeté à la mer. Trois des canotiers qui ignoraient l’instant précis où le fer allait être lancé et n’avaient pu dès lors se préparer à en subir les contrecoups, passèrent par-dessus bord mais de telle manière qu’en un clin d’œil, deux d’entre eux avaient pu s’agripper au plat-bord et, soulevés par une vague se hisser à nouveau dans la baleinière ; le troisième homme resta, impuissant, à l’arrière, mais il nageait encore.\par
Presque aussitôt, avec une volonté puissante, spontanée, rapide, la Baleine blanche fila comme une flèche dans la mer tumultueuse. Mais lorsque Achab cria à ses canotiers de prendre de nouveaux tours au taberin, de maintenir la ligne tendue, et de se tourner sur leurs bancs pour haler l’embarcation à son but, la ligne traîtresse céda sous la double traction et fouetta le vide !\par
– Qu’est-ce qui s’est brisé en moi ? Un nerf craque ! elle est libre à nouveau. Aux avirons ! aux avirons ! Sautez-lui dessus !\par
Entendant la formidable poussée de la pirogue qui faisait retentir la mer, la baleine aux abois fit volte-face et présenta son front aveugle à ses agresseurs, mais ce faisant elle aperçut la quille noire du navire qui approchait ; elle parut reconnaître en lui le responsable de ses persécutions, et peut-être un ennemi plus puissant et plus noble. Tout soudain elle chargea contre sa proue qui avançait, claquant des mâchoires dans une sauvage averse d’écume.\par
Achab chancela et porta la main à son front :\par
– Je deviens aveugle… hommes, tendez-moi la main que je puisse tâtonner encore pour trouver mon chemin. Fait-il nuit ?\par
– La baleine ! Le navire ! s’écrièrent les rameurs en se dérobant craintivement.\par
– Aux avirons ! aux avirons ! Abaisse-toi jusqu’en tes profondeurs, ô mer, qu’avant qu’il ne soit trop tard, Achab puisse glisser jusqu’à son but pour la dernière, l’ultime fois ! Je vois : navire ! le navire ! Souquez, hommes souquez ! Ne sauverezvous pas mon navire ?\par
Mais tandis que les canotiers forçaient aux avirons à travers le martèlement de la mer, le bordé de l’étrave déjà frappé par la baleine céda et presque aussitôt la pirogue désemparée se trouva presque au niveau des vagues ; son équipage s’efforçait d’aveugler la voie d’eau et d’écoper tout en pataugeant et s’éclaboussant.\par
Pendant ce temps, l’instant d’un regard, Tashtego suspendit le geste de sa main qui tenait le marteau et le drapeau rouge s’enroula à demi autour de lui comme un plaid, puis se mit à flotter droit devant lui comme si son propre cœur battait dans l’espace, cependant que Starbuck et Stubb, debout au-dessous de lui sur le beaupré, virent en même temps que lui, le monstre arrivant sur eux.\par
– La baleine ! la baleine ! Barre dessus toute ! Barre dessus toute ! Ô vous, toutes les douces puissances de l’air, serrez-moi dans vos bras ! Si Starbuck doit mourir, qu’il ne meure pas défaillant comme une femme. Barre dessus, j’ai dit !… Imbéciles, la mâchoire, la mâchoire ! Est-ce là la réponse à toutes mes véhémentes prières ? À une fidélité de toute ma vie ! Oh ! Achab, Achab, voilà ton œuvre ! Redresse timonier… Non, non ! Barre dessus encore ! Elle s’est tournée pour nous aborder ! Oh ! son front insatiable se dirige sur l’homme à qui le devoir ordonne de ne point fuir. Mon Dieu, soyez à mes côtés en cet instant !\par
– Ne soyez pas à mes côtés, mais au-dessous de moi, qui que vous soyez si vous souhaitez aider Stubb, car Stubb, lui aussi, est rivé là. Je me moque de toi, ricanante baleine ! Qui a jamais soutenu Stubb, tenu Stubb en éveil, sinon son œil qui ne cille pas ? Et maintenant, le pauvre Stubb va se coucher sur \hspace{1em} un matelas qui n’est que trop moelleux, que n’est-il fourré de broussailles. Je me moque de toi, ricanante baleine ! Voyez, vous, soleil, lune, étoiles ! Je vous accuse d’être les assassins du meilleur garçon qui ait jamais rendu l’âme. C’est la raison pour laquelle je trinquerai quand même avec vous, si vous consentez à me tendre la coupe ! Oh ! oh ! toi, ricanante baleine, tu pourras bientôt te remplir le gosier. Pourquoi ne fuyez-vous pas, ô Achab ! Quant à moi, je quitte mes souliers et ma vareuse, que Stubb meure dans ses seuls pantalons ! Trop moisie, trop salée, cette mort, pourtant… des cerises ! des cerises ! Oh ! Flask, une seule cerise avant que nous ne mourions !\par
– Des cerises ? Je ne souhaiterais qu’être au pays où elles poussent. Oh ! Stubb, j’espère que ma pauvre mère a touché ma part de paie avant ce moment, sinon elle ne verra pas un sou, car le voyage est fini.\par
À l’avant du navire, les hommes à présent sont pétrifiés, le dernier geste qu’ils étaient en train de faire a figé dans leurs mains les marteaux, les morceaux de bordés, les lances et les harpons. Envoûtés, ils fixent la baleine, ils la regardent balancer de droite et de gauche son front, porteur du destin et contemplent le vaste demi-cercle d’écume que son élan soulève devant elle. Elle est la vision même du Jugement dernier, de la vengeance immédiate, de l’éternelle malice devant l’impuissance humaine. Le solide contrefort de son front blanc frappa la proue par tribord, faisant rouler les pièces de construction et les hommes. Quelques-uns s’abattirent face au sol. Comme des pommes de mât déboîtées, les têtes des harponneurs en vigie furent ébranlées sur leurs cous de taureaux. Par la brèche, ils entendirent s’engouffrer l’eau comme celle d’un torrent de montagne dans une rayère.\par
– Le navire ! Le corbillard !… le second corbillard ! s’écria Achab de sa baleinière, et dont le bois ne serait qu’un bois d’Amérique !\par
Plongeant sous le navire qui enfonçait, la baleine courut en frémissant le long de la quille, mais se retournant dans l’eau, elle réapparut promptement à la surface, loin de l’étrave, par bâbord, mais à peu de distance de la baleinière d’Achab. Et là, elle se tint immobile un moment.\par
– Je me détourne du soleil. Ohé, Tashtego, fais-moi entendre ton marteau. Oh ! vous mes trois mâts invincibles… toi, contre-quille intacte… coque par Dieu seul intimidée… toi, ferme pont, barre fière, proue pointée vers le Pôle… navire à la mort glorieuse, devras-tu périr sans moi ? Suis-je frustré de la dernière satisfaction d’orgueil du plus misérable des capitaines ? Oh ! solitaire mort après une vie solitaire ! Oh ! je sens à présent que mon extrême grandeur est dans ma douleur extrême. Oh ! accourez des plus lointains rivages pour gonfler, ô vagues intrépides de toute ma vie passée, cette lame unique de ma mort qui va déferler ! Vers toi je roule, baleine destructrice qui ne récolte que le néant, je suis aux prises avec toi jusqu’au dernier instant, du cœur de l’enfer je te frappe, au nom de la haine je crache contre toi mon dernier souffle. Sombrez tous cercueils, tous corbillards dans la mare commune puisque nuls ne peuvent êtres miens, que je sois déchiqueté et lié à toi en te chassant, baleine maudite ! C’est ainsi que je rends les armes !\par
Et le harpon fut lancé, la baleine frappée chargea, la ligne courut dans son engoujure en s’enflammant, puis se noua. Achab se pencha pour la démêler et il y parvint, mais le nœud coulant en plein vol lui enserra le cou et sans voix, comme la victime des bourreaux muets des sultans, il fut emporté hors de la baleinière avant que les hommes aient eu le temps de s’en apercevoir. L’instant d’après, la lourde épissure à œil de l’extrémité de la ligne gicla hors de la baille vide, renversa les canotiers et, frappant la mer, disparut dans les profondeurs.\par
L’équipage pétrifié resta un moment immobile, puis se retourna : « Le navire ? Grand Dieu, où est le navire ? »\par
Bientôt ils le virent dans une atmosphère trouble, bouleversante, fantôme évanescent, vu comme à travers les brouillards de la fée Morgane. Seuls les trois mâts émergeaient encore et soit aberration, fidélité, ou destin, au sommet de leurs perchoirs élevés, les harponneurs païens guettaient toujours la mer. Maintenant, les cercles concentriques se resserrèrent autour de la baleinière esseulée et de son équipage, saisissant chaque aviron qui flotte, chaque hampe de lance, les êtres animés, les objets inanimés, les emportant en rond dans un unique maelström, leur dérobant la vue de la plus petite épave du {\itshape Péquod.}\par
Mais tandis que les derniers tourbillons se refermaient sur la tête de l’Indien au grand mât, laissant encore émerger sa flèche ainsi que le penon qui flottait paisiblement de toute sa longueur, la dérision d’une coïncidence voulut qu’au-dessus des lames destructrices qui le touchaient presque, un bras rouge tenant un marteau sortit de l’eau et d’un geste large, se mit à clouer plus fort et toujours plus fort le drapeau à l’espar qui pointait encore. Un aigle de mer avait suivi, provoquant, la descente du grand mât loin de sa vraie demeure parmi les étoiles, harcelant Tashtego en piquant du bec le drapeau ; son aile se mit à battre entre le marteau et le bois et, sentant aussitôt ce frisson éthéré, le sauvage noyé, dans la convulsion de son agonie, le cloua. Ainsi l’oiseau du ciel au cri d’archange, le bec impérial levé, le corps captif du drapeau d’Achab, sombra avec son navire qui, tel Satan, ne descendit pas en enfer sans avoir entraîné à sa suite une vivante part de ciel pour s’en casquer.\par
Et maintenant de petits oiseaux volaient en criant audessus du gouffre encore béant, une blanche et morne écume battait ses flancs escarpés, puis tout s’affaissa, et le grand linceul de la mer roula comme il roulait il y a cinq mille ans.
\chapterclose

\section[{ÉPILOGUE}]{ÉPILOGUE}\renewcommand{\leftmark}{ÉPILOGUE}


\begin{quoteblock}
\noindent … Et je me suis échappé, moi seul, pour t’apporter cette nouvelle\end{quoteblock}


\bibl{JOB.}
\noindent Le drame est joué. Pourquoi, dès lors, quelqu’un s’avancera-t-il ? Parce que quelqu’un survécut au naufrage.\par
Il arriva qu’après la disparition du Parsi le destin me désigna pour premier rameur de la baleinière d’Achab, Achab prenant la place du disparu. C’est moi qui me trouvai abandonné à l’arrière lorsque trois hommes furent jetés à l’eau. Ainsi, surnageant loin de cette scène, pourtant tout entière livrée à mon regard, l’aspiration du navire qui sombrait m’attira lentement vers son tourbillon. Quand je l’atteignis, ce n’était plus qu’un étang mousseux. Et je tournai et tournai, me rapprochant sans cesse de la bulle noire au centre de cette roue en mouvement. Je tournai, tel un nouvel Ixion, jusqu’à ce qu’ayant atteint ce point vital, la bulle noire gonfla et creva entraînant dans son effort, et grâce à sa flottabilité, le cercueil-bouée qui, projeté avec force, bondit sur l’eau, se renversa et vint en surface à mes côtés. Soutenu par ce cercueil pendant près d’un jour et d’une nuit, je flottai sur un Océan qui chantait doucement un hymne funèbre. Les requins inoffensifs glissaient, muselés, autour de moi et les aigles sauvages de la mer avaient le bec au fourreau. Le second jour, une voile approcha, toujours plus près et me recueillit enfin. C’était l’errante {\itshape Rachel} qui, rebroussant chemin, en quête de ses enfants perdus, n’avait trouvé qu’un autre orphelin.
 


% at least one empty page at end (for booklet couv)
\ifbooklet
  \pagestyle{empty}
  \clearpage
  % 2 empty pages maybe needed for 4e cover
  \ifnum\modulo{\value{page}}{4}=0 \hbox{}\newpage\hbox{}\newpage\fi
  \ifnum\modulo{\value{page}}{4}=1 \hbox{}\newpage\hbox{}\newpage\fi


  \hbox{}\newpage
  \ifodd\value{page}\hbox{}\newpage\fi
  {\centering\color{rubric}\bfseries\noindent\large
    Hurlus ? Qu’est-ce.\par
    \bigskip
  }
  \noindent Des bouquinistes électroniques, pour du texte libre à participation libre,
  téléchargeable gratuitement sur \href{https://hurlus.fr}{\dotuline{hurlus.fr}}.\par
  \bigskip
  \noindent Cette brochure a été produite par des éditeurs bénévoles.
  Elle n’est pas faîte pour être possédée, mais pour être lue, et puis donnée.
  Que circule le texte !
  En page de garde, on peut ajouter une date, un lieu, un nom ; pour suivre le voyage des idées.
  \par

  Ce texte a été choisi parce qu’une personne l’a aimé,
  ou haï, elle a en tous cas pensé qu’il partipait à la formation de notre présent ;
  sans le souci de plaire, vendre, ou militer pour une cause.
  \par

  L’édition électronique est soigneuse, tant sur la technique
  que sur l’établissement du texte ; mais sans aucune prétention scolaire, au contraire.
  Le but est de s’adresser à tous, sans distinction de science ou de diplôme.
  Au plus direct ! (possible)
  \par

  Cet exemplaire en papier a été tiré sur une imprimante personnelle
   ou une photocopieuse. Tout le monde peut le faire.
  Il suffit de
  télécharger un fichier sur \href{https://hurlus.fr}{\dotuline{hurlus.fr}},
  d’imprimer, et agrafer ; puis de lire et donner.\par

  \bigskip

  \noindent PS : Les hurlus furent aussi des rebelles protestants qui cassaient les statues dans les églises catholiques. En 1566 démarra la révolte des gueux dans le pays de Lille. L’insurrection enflamma la région jusqu’à Anvers où les gueux de mer bloquèrent les bateaux espagnols.
  Ce fut une rare guerre de libération dont naquit un pays toujours libre : les Pays-Bas.
  En plat pays francophone, par contre, restèrent des bandes de huguenots, les hurlus, progressivement réprimés par la très catholique Espagne.
  Cette mémoire d’une défaite est éteinte, rallumons-la. Sortons les livres du culte universitaire, cherchons les idoles de l’époque, pour les briser.
\fi

\ifdev % autotext in dev mode
\fontname\font — \textsc{Les règles du jeu}\par
(\hyperref[utopie]{\underline{Lien}})\par
\noindent \initialiv{A}{lors là}\blindtext\par
\noindent \initialiv{À}{ la bonheur des dames}\blindtext\par
\noindent \initialiv{É}{tonnez-le}\blindtext\par
\noindent \initialiv{Q}{ualitativement}\blindtext\par
\noindent \initialiv{V}{aloriser}\blindtext\par
\Blindtext
\phantomsection
\label{utopie}
\Blinddocument
\fi
\end{document}
