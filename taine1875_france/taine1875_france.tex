%%%%%%%%%%%%%%%%%%%%%%%%%%%%%%%%%
% LaTeX model https://hurlus.fr %
%%%%%%%%%%%%%%%%%%%%%%%%%%%%%%%%%

% Needed before document class
\RequirePackage{pdftexcmds} % needed for tests expressions
\RequirePackage{fix-cm} % correct units

% Define mode
\def\mode{a4}

\newif\ifaiv % a4
\newif\ifav % a5
\newif\ifbooklet % booklet
\newif\ifcover % cover for booklet

\ifnum \strcmp{\mode}{cover}=0
  \covertrue
\else\ifnum \strcmp{\mode}{booklet}=0
  \booklettrue
\else\ifnum \strcmp{\mode}{a5}=0
  \avtrue
\else
  \aivtrue
\fi\fi\fi

\ifbooklet % do not enclose with {}
  \documentclass[french,twoside]{book} % ,notitlepage
  \usepackage[%
    papersize={105mm, 297mm},
    inner=12mm,
    outer=12mm,
    top=20mm,
    bottom=15mm,
    marginparsep=3pt,
    marginpar=7mm,
  ]{geometry}
  \usepackage[fontsize=9.5pt]{scrextend} % for Roboto
\else\ifav
  \documentclass[french,twoside]{book} % ,notitlepage
  \usepackage[%
    a5paper,
    inner=25mm,
    outer=15mm,
    top=15mm,
    bottom=15mm,
    marginparsep=0pt,
  ]{geometry}
  \usepackage[fontsize=12pt]{scrextend}
\else% A4 2 cols
  \documentclass[twocolumn]{report}
  \usepackage[%
    a4paper,
    inner=15mm,
    outer=10mm,
    top=25mm,
    bottom=18mm,
    marginparsep=0pt,
  ]{geometry}
  \setlength{\columnsep}{20mm}
  \usepackage[fontsize=9.5pt]{scrextend}
\fi\fi

%%%%%%%%%%%%%%
% Alignments %
%%%%%%%%%%%%%%
% before teinte macros

\setlength{\arrayrulewidth}{0.2pt}
\setlength{\columnseprule}{\arrayrulewidth} % twocol
\setlength{\parskip}{0pt} % 1pt allow better vertical justification
\setlength{\parindent}{1.5em}

%%%%%%%%%%
% Colors %
%%%%%%%%%%
% before Teinte macros

\usepackage[dvipsnames]{xcolor}
\definecolor{rubric}{HTML}{800000} % the tonic 0c71c3
\def\columnseprulecolor{\color{rubric}}
\colorlet{borderline}{rubric!30!} % definecolor need exact code
\definecolor{shadecolor}{gray}{0.95}
\definecolor{bghi}{gray}{0.5}

%%%%%%%%%%%%%%%%%
% Teinte macros %
%%%%%%%%%%%%%%%%%
%%%%%%%%%%%%%%%%%%%%%%%%%%%%%%%%%%%%%%%%%%%%%%%%%%%
% <TEI> generic (LaTeX names generated by Teinte) %
%%%%%%%%%%%%%%%%%%%%%%%%%%%%%%%%%%%%%%%%%%%%%%%%%%%
% This template is inserted in a specific design
% It is XeLaTeX and otf fonts

\makeatletter % <@@@

\usepackage{alphalph} % for alph couter z, aa, ab…
\usepackage{blindtext} % generate text for testing
\usepackage[strict]{changepage} % for modulo 4
\usepackage{contour} % rounding words
\usepackage[nodayofweek]{datetime}
\usepackage{enumitem} % <list>
\usepackage{etoolbox} % patch commands
\usepackage{fancyvrb}
\usepackage{fancyhdr}
\usepackage{float}
\usepackage{fontspec} % XeLaTeX mandatory for fonts
\usepackage{footnote} % used to capture notes in minipage (ex: quote)
\usepackage{framed} % bordering correct with footnote hack
\usepackage{graphicx}
\usepackage{lettrine} % drop caps
\usepackage{lipsum} % generate text for testing
\usepackage{manyfoot} % for parallel footnote numerotation
\usepackage[framemethod=tikz,]{mdframed} % maybe used for frame with footnotes inside
\usepackage[defaultlines=2,all]{nowidow} % at least 2 lines by par (works well!)
\usepackage{pdftexcmds} % needed for tests expressions
\usepackage{poetry} % <l>, bad for theater
\usepackage{polyglossia} % bug Warning: "Failed to patch part"
\usepackage[%
  indentfirst=false,
  vskip=1em,
  noorphanfirst=true,
  noorphanafter=true,
  leftmargin=\parindent,
  rightmargin=0pt,
]{quoting}
\usepackage{ragged2e}
\usepackage{setspace} % \setstretch for <quote>
\usepackage{scrextend} % KOMA-common, used for addmargin
\usepackage{tabularx} % <table>
\usepackage[explicit]{titlesec} % wear titles, !NO implicit
\usepackage{tikz} % ornaments
\usepackage{tocloft} % styling tocs
\usepackage[fit]{truncate} % used im runing titles
\usepackage{unicode-math}
\usepackage[normalem]{ulem} % breakable \uline, normalem is absolutely necessary to keep \emph
\usepackage{xcolor} % named colors
\usepackage{xparse} % @ifundefined
\XeTeXdefaultencoding "iso-8859-1" % bad encoding of xstring
\usepackage{xstring} % string tests
\XeTeXdefaultencoding "utf-8"

\defaultfontfeatures{
  % Mapping=tex-text, % no effect seen
  Scale=MatchLowercase,
  Ligatures={TeX,Common},
}
\newfontfamily\zhfont{Noto Sans CJK SC}

% Metadata inserted by a program, from the TEI source, for title page and runing heads
\title{\textbf{ Les origines de la France contemporaine. L’Ancien Régime. Tomes I et II }\par
}
\date{1875}
\author{Taine}
\def\elbibl{Taine. 1875. \emph{Les origines de la France contemporaine. L’Ancien Régime. Tomes I et II}}
\def\elsource{}
\def\eltitlepage{%
{\centering\parindent0pt
  {\LARGE\addfontfeature{LetterSpace=25}\bfseries Taine\par}\bigskip
  {\Large 1875\par}\bigskip
  {\LARGE
\bigskip\textbf{Les origines de la France contemporaine. L’Ancien Régime. Tomes I et II}\par

  }
}

}

% Default metas
\newcommand{\colorprovide}[2]{\@ifundefinedcolor{#1}{\colorlet{#1}{#2}}{}}
\colorprovide{rubric}{red}
\colorprovide{silver}{lightgray}
\@ifundefined{syms}{\newfontfamily\syms{DejaVu Sans}}{}
\newif\ifdev
\@ifundefined{elbibl}{% No meta defined, maybe dev mode
  \newcommand{\elbibl}{Titre court ?}
  \newcommand{\elbook}{Titre du livre source ?}
  \newcommand{\elabstract}{Résumé\par}
  \newcommand{\elurl}{http://oeuvres.github.io/elbook/2}
  \author{Éric Lœchien}
  \title{Un titre de test assez long pour vérifier le comportement d’une maquette}
  \date{1566}
  \devtrue
}{}
\let\eltitle\@title
\let\elauthor\@author
\let\eldate\@date




% generic typo commands
\newcommand{\astermono}{\medskip\centerline{\color{rubric}\large\selectfont{\syms ✻}}\medskip\par}%
\newcommand{\astertri}{\medskip\par\centerline{\color{rubric}\large\selectfont{\syms ✻\,✻\,✻}}\medskip\par}%
\newcommand{\asterism}{\bigskip\par\noindent\parbox{\linewidth}{\centering\color{rubric}\large{\syms ✻}\\{\syms ✻}\hskip 0.75em{\syms ✻}}\bigskip\par}%

% lists
\newlength{\listmod}
\setlength{\listmod}{\parindent}
\setlist{
  itemindent=!,
  listparindent=\listmod,
  labelsep=0.2\listmod,
  parsep=0pt,
  % topsep=0.2em, % default topsep is best
}
\setlist[itemize]{
  label=—,
  leftmargin=0pt,
  labelindent=1.2em,
  labelwidth=0pt,
}
\setlist[enumerate]{
  label={\arabic*°},
  labelindent=0.8\listmod,
  leftmargin=\listmod,
  labelwidth=0pt,
}
% list for big items
\newlist{decbig}{enumerate}{1}
\setlist[decbig]{
  label={\bf\color{rubric}\arabic*.},
  labelindent=0.8\listmod,
  leftmargin=\listmod,
  labelwidth=0pt,
}
\newlist{listalpha}{enumerate}{1}
\setlist[listalpha]{
  label={\bf\color{rubric}\alph*.},
  leftmargin=0pt,
  labelindent=0.8\listmod,
  labelwidth=0pt,
}
\newcommand{\listhead}[1]{\hspace{-1\listmod}\emph{#1}}

\renewcommand{\hrulefill}{%
  \leavevmode\leaders\hrule height 0.2pt\hfill\kern\z@}

% General typo
\DeclareTextFontCommand{\textlarge}{\large}
\DeclareTextFontCommand{\textsmall}{\small}

% commands, inlines
\newcommand{\anchor}[1]{\Hy@raisedlink{\hypertarget{#1}{}}} % link to top of an anchor (not baseline)
\newcommand\abbr[1]{#1}
\newcommand{\autour}[1]{\tikz[baseline=(X.base)]\node [draw=rubric,thin,rectangle,inner sep=1.5pt, rounded corners=3pt] (X) {\color{rubric}#1};}
\newcommand\corr[1]{#1}
\newcommand{\ed}[1]{ {\color{silver}\sffamily\footnotesize (#1)} } % <milestone ed="1688"/>
\newcommand\expan[1]{#1}
\newcommand\foreign[1]{\emph{#1}}
\newcommand\gap[1]{#1}
\renewcommand{\LettrineFontHook}{\color{rubric}}
\newcommand{\initial}[2]{\lettrine[lines=2, loversize=0.3, lhang=0.3]{#1}{#2}}
\newcommand{\initialiv}[2]{%
  \let\oldLFH\LettrineFontHook
  % \renewcommand{\LettrineFontHook}{\color{rubric}\ttfamily}
  \IfSubStr{QJ’}{#1}{
    \lettrine[lines=4, lhang=0.2, loversize=-0.1, lraise=0.2]{\smash{#1}}{#2}
  }{\IfSubStr{É}{#1}{
    \lettrine[lines=4, lhang=0.2, loversize=-0, lraise=0]{\smash{#1}}{#2}
  }{\IfSubStr{ÀÂ}{#1}{
    \lettrine[lines=4, lhang=0.2, loversize=-0, lraise=0, slope=0.6em]{\smash{#1}}{#2}
  }{\IfSubStr{A}{#1}{
    \lettrine[lines=4, lhang=0.2, loversize=0.2, slope=0.6em]{\smash{#1}}{#2}
  }{\IfSubStr{V}{#1}{
    \lettrine[lines=4, lhang=0.2, loversize=0.2, slope=-0.5em]{\smash{#1}}{#2}
  }{
    \lettrine[lines=4, lhang=0.2, loversize=0.2]{\smash{#1}}{#2}
  }}}}}
  \let\LettrineFontHook\oldLFH
}
\newcommand{\labelchar}[1]{\textbf{\color{rubric} #1}}
\newcommand{\lnatt}[1]{\reversemarginpar\marginpar[\sffamily\scriptsize #1]{}}
\newcommand{\milestone}[1]{\autour{\footnotesize\color{rubric} #1}} % <milestone n="4"/>
\newcommand\name[1]{#1}
\newcommand\orig[1]{#1}
\newcommand\orgName[1]{#1}
\newcommand\persName[1]{#1}
\newcommand\placeName[1]{#1}
\newcommand{\pn}[1]{\IfSubStr{-—–¶}{#1}% <p n="3"/>
  {\noindent{\bfseries\color{rubric}   ¶  }}
  {{\footnotesize\autour{#1}}}}
\newcommand\reg{}
% \newcommand\ref{} % already defined
\newcommand\sic[1]{#1}
\newcommand\surname[1]{\textsc{#1}}
\newcommand\term[1]{\textbf{#1}}
\newcommand\zh[1]{{\zhfont #1}}


\def\mednobreak{\ifdim\lastskip<\medskipamount
  \removelastskip\nopagebreak\medskip\fi}
\def\bignobreak{\ifdim\lastskip<\bigskipamount
  \removelastskip\nopagebreak\bigskip\fi}

% commands, blocks

\newcommand{\byline}[1]{\bigskip{\RaggedLeft{#1}\par}\bigskip}
% \setlength{\RaggedLeftLeftskip}{2em plus \leftskip}
\newcommand{\bibl}[1]{{\smallskip\RaggedLeft\normalsize\normalfont #1\par\medskip}}}
\newcommand{\biblitem}[1]{{\noindent\hangindent=\parindent   #1\par}}
\newcommand{\castItem}[1]{{\noindent\hangindent=\parindent #1\par}}
\newcommand{\dateline}[1]{\medskip{\RaggedLeft{#1}\par}\bigskip}
\newcommand{\docAuthor}[1]{{\large\bigskip #1 \par\bigskip}}
\newcommand{\docDate}[1]{#1 \ifvmode\par\fi }
\newcommand{\docImprint}[1]{\ifvmode\medskip\fi #1 \ifvmode\par\fi }
\newcommand{\labelblock}[1]{\medbreak{\noindent\color{rubric}\bfseries #1}\par\mednobreak}
\newcommand{\salute}[1]{\bigbreak{#1}\par\medbreak}
\newcommand{\signed}[1]{\medskip{\RaggedLeft #1\par}\bigbreak} % supposed bottom
\newcommand{\speaker}[1]{\medskip{\Centering\sffamily #1\par\nopagebreak}} % supposed bottom
\newcommand{\stagescene}[1]{{\Centering\sffamily #1\par}\bigskip}
\newcommand{\stagesp}[1]{\begingroup\leftskip\parindent\noindent\it\sffamily #1\par\endgroup} % left margin, better than list envs
\newcommand{\spl}[1]{\noindent\hangindent=2\parindent  #1\par} % sp/l
\newcommand{\trailer}[1]{{\Centering\bigskip #1\par}} % sp/l

% environments for blocks (some may become commands)
\newenvironment{borderbox}{}{} % framing content
\newenvironment{citbibl}{\ifvmode\hfill\fi}{\ifvmode\par\fi }
\newenvironment{msHead}{\vskip6pt}{\par}
\newenvironment{msItem}{\vskip6pt}{\par}


% environments for block containers
\newenvironment{argument}{\itshape\parindent0pt}{\bigskip}
\newenvironment{biblfree}{}{\ifvmode\par\fi }
\newenvironment{bibitemlist}[1]{%
  \list{\@biblabel{\@arabic\c@enumiv}}%
  {%
    \settowidth\labelwidth{\@biblabel{#1}}%
    \leftmargin\labelwidth
    \advance\leftmargin\labelsep
    \@openbib@code
    \usecounter{enumiv}%
    \let\p@enumiv\@empty
    \renewcommand\theenumiv{\@arabic\c@enumiv}%
  }
  \sloppy
  \clubpenalty4000
  \@clubpenalty \clubpenalty
  \widowpenalty4000%
  \sfcode`\.\@m
}%
{\def\@noitemerr
  {\@latex@warning{Empty `bibitemlist' environment}}%
\endlist}
\newenvironment{docTitle}{\LARGE\bigskip\bfseries\onehalfspacing}{\bigskip}
% leftskip makes big bugs in Lexmark printing \sffamily
\newenvironment{epigraph}{\begin{addmargin}[2\parindent]{0em}\sffamily\large\setstretch{0.95}}{\end{addmargin}\bigskip}
\newenvironment{quoteblock}% may be used for ornaments
  {\begin{quoting}}
  {\end{quoting}}
\newenvironment{titlePage}
  {\Centering}
  {}






% table () is preceded and finished by custom command
\newcommand{\tableopen}[1]{%
  \ifnum\strcmp{#1}{wide}=0{%
    \begin{center}
  }
  \else\ifnum\strcmp{#1}{long}=0{%
    \begin{center}
  }
  \else{%
    \begin{center}
  }
  \fi\fi
}
\newcommand{\tableclose}[1]{%
  \ifnum\strcmp{#1}{wide}=0{%
    \end{center}
  }
  \else\ifnum\strcmp{#1}{long}=0{%
    \end{center}
  }
  \else{%
    \end{center}
  }
  \fi\fi
}


% text structure
\newcommand\chapteropen{} % before chapter title
\newcommand\chaptercont{} % after title, argument, epigraph…
\newcommand\chapterclose{} % maybe useful for multicol settings
\setcounter{secnumdepth}{-2} % no counters for hierarchy titles
\setcounter{tocdepth}{5} % deep toc
\renewcommand\tableofcontents{\@starttoc{toc}}
% toclof format
% \renewcommand{\@tocrmarg}{0.1em} % Useless command?
% \renewcommand{\@pnumwidth}{0.5em} % {1.75em}
\renewcommand{\@cftmaketoctitle}{}
\setlength{\cftbeforesecskip}{\z@ \@plus.2\p@}
\renewcommand{\cftchapfont}{}
\renewcommand{\cftchapdotsep}{\cftdotsep}
\renewcommand{\cftchapleader}{\normalfont\cftdotfill{\cftchapdotsep}}
\renewcommand{\cftchappagefont}{\bfseries}
\setlength{\cftbeforechapskip}{0em \@plus\p@}
% \renewcommand{\cftsecfont}{\small\relax}
\renewcommand{\cftsecpagefont}{\normalfont}
% \renewcommand{\cftsubsecfont}{\small\relax}
\renewcommand{\cftsecdotsep}{\cftdotsep}
\renewcommand{\cftsecpagefont}{\normalfont}
\renewcommand{\cftsecleader}{\normalfont\cftdotfill{\cftsecdotsep}}
\setlength{\cftsecindent}{1em}
\setlength{\cftsubsecindent}{2em}
\setlength{\cftsubsubsecindent}{3em}
\setlength{\cftchapnumwidth}{1em}
\setlength{\cftsecnumwidth}{1em}
\setlength{\cftsubsecnumwidth}{1em}
\setlength{\cftsubsubsecnumwidth}{1em}

% footnotes
\newif\ifheading
\newcommand*{\fnmarkscale}{\ifheading 0.70 \else 1 \fi}
\renewcommand\footnoterule{\vspace*{0.3cm}\hrule height \arrayrulewidth width 3cm \vspace*{0.3cm}}
\setlength\footnotesep{1.5\footnotesep} % footnote separator
\renewcommand\@makefntext[1]{\parindent 1.5em \noindent \hb@xt@1.8em{\hss{\normalfont\@thefnmark . }}#1} % no superscipt in foot
\patchcmd{\@footnotetext}{\footnotesize}{\footnotesize\sffamily}{}{} % before scrextend, hyperref
\DeclareNewFootnote{A}[alph] % for editor notes
\renewcommand*{\thefootnoteA}{\alphalph{\value{footnoteA}}} % z, aa, ab…

% poem
\setlength{\poembotskip}{0pt}
\setlength{\poemtopskip}{0pt}
\setlength{\poemindent}{0pt}
\poemlinenumsfalse

%   see https://tex.stackexchange.com/a/34449/5049
\def\truncdiv#1#2{((#1-(#2-1)/2)/#2)}
\def\moduloop#1#2{(#1-\truncdiv{#1}{#2}*#2)}
\def\modulo#1#2{\number\numexpr\moduloop{#1}{#2}\relax}

% orphans and widows, nowidow package in test
% from memoir package
\clubpenalty=9996
\widowpenalty=9999
\brokenpenalty=4991
\predisplaypenalty=10000
\postdisplaypenalty=1549
\displaywidowpenalty=1602
\hyphenpenalty=400
% report h or v overfull ?
\hbadness=4000
\vbadness=4000
% good to avoid lines too wide
\emergencystretch 3em
\pretolerance=750
\tolerance=2000
\def\Gin@extensions{.pdf,.png,.jpg,.mps,.tif}

\PassOptionsToPackage{hyphens}{url} % before hyperref and biblatex, which load url package
\usepackage{hyperref} % supposed to be the last one, :o) except for the ones to follow
\hypersetup{
  % pdftex, % no effect
  pdftitle={\elbibl},
  % pdfauthor={Your name here},
  % pdfsubject={Your subject here},
  % pdfkeywords={keyword1, keyword2},
  bookmarksnumbered=true,
  bookmarksopen=true,
  bookmarksopenlevel=1,
  pdfstartview=Fit,
  breaklinks=true, % avoid long links, overrided by url package
  pdfpagemode=UseOutlines,    % pdf toc
  hyperfootnotes=true,
  colorlinks=false,
  pdfborder=0 0 0,
  % pdfpagelayout=TwoPageRight,
  % linktocpage=true, % NO, toc, link only on page no
}
\urlstyle{same} % after hyperref



\makeatother % /@@@>
%%%%%%%%%%%%%%
% </TEI> end %
%%%%%%%%%%%%%%


%%%%%%%%%%%%%
% footnotes %
%%%%%%%%%%%%%
\renewcommand{\thefootnote}{\bfseries\textcolor{rubric}{\arabic{footnote}}} % color for footnote marks

%%%%%%%%%
% Fonts %
%%%%%%%%%
\usepackage[]{roboto} % SmallCaps, Regular is a bit bold
% \linespread{0.90} % too compact, keep font natural
\newfontfamily\fontrun[]{Roboto Condensed Light} % condensed runing heads
\setsansfont{Roboto Light} % seen, problem if not set
\ifav
  \setmainfont[
    ItalicFont={Roboto Light Italic},
  ]{Roboto}
\else\ifbooklet
  \setmainfont[
    ItalicFont={Roboto Light Italic},
  ]{Roboto}
\else
\setmainfont[
  ItalicFont={Roboto Italic},
]{Roboto Light}
\fi\fi
\renewcommand{\LettrineFontHook}{\bfseries\color{rubric}}
% \renewenvironment{labelblock}{\begin{center}\bfseries\color{rubric}}{\end{center}}

%%%%%%%%
% MISC %
%%%%%%%%

\setdefaultlanguage[frenchpart=false]{french} % bug on part


\newenvironment{quotebar}{%
    \def\FrameCommand{{\color{rubric!10!}\vrule width 0.5em} \hspace{0.9em}}%
    \def\OuterFrameSep{0pt} % séparateur vertical
    \MakeFramed {\advance\hsize-\width \FrameRestore}
  }%
  {%
    \endMakeFramed
  }
\renewenvironment{quoteblock}% may be used for ornaments
  {%
    \savenotes
    \setstretch{0.9}
    \normalfont
    \begin{quotebar}
  }
  {%
    \end{quotebar}
    \spewnotes
  }


\renewcommand{\headrulewidth}{\arrayrulewidth}
\renewcommand{\headrule}{{\color{rubric}\hrule}}
\renewcommand{\lnatt}[1]{\marginpar{\sffamily\scriptsize #1}}

% delicate tuning, image has produce line-height problems in title on 2 lines
\titleformat{name=\chapter} % command
  [display] % shape
  {\vspace{1.5em}\centering} % format
  {} % label
  {0pt} % separator between n
  {}
[{\color{rubric}\huge\textbf{#1}}\bigskip] % after code
% \titlespacing{command}{left spacing}{before spacing}{after spacing}[right]
\titlespacing*{\chapter}{0pt}{-2em}{0pt}[0pt]

\titleformat{name=\section}
  [display]{}{}{}{}
  [\vbox{\color{rubric}\large\raggedleft\textbf{#1}}]
\titlespacing{\section}{0pt}{0pt plus 4pt minus 2pt}{\baselineskip}

\titleformat{name=\subsection}
  [block]
  {}
  {} % \thesection
  {} % separator \arrayrulewidth
  {}
[\vbox{\large\textbf{#1}}]
% \titlespacing{\subsection}{0pt}{0pt plus 4pt minus 2pt}{\baselineskip}

\ifaiv
  \fancypagestyle{main}{%
    \fancyhf{}
    \setlength{\headheight}{1.5em}
    \fancyhead{} % reset head
    \fancyfoot{} % reset foot
    \fancyhead[L]{\truncate{0.45\headwidth}{\fontrun\elbibl}} % book ref
    \fancyhead[R]{\truncate{0.45\headwidth}{ \fontrun\nouppercase\leftmark}} % Chapter title
    \fancyhead[C]{\thepage}
  }
  \fancypagestyle{plain}{% apply to chapter
    \fancyhf{}% clear all header and footer fields
    \setlength{\headheight}{1.5em}
    \fancyhead[L]{\truncate{0.9\headwidth}{\fontrun\elbibl}}
    \fancyhead[R]{\thepage}
  }
\else
  \fancypagestyle{main}{%
    \fancyhf{}
    \setlength{\headheight}{1.5em}
    \fancyhead{} % reset head
    \fancyfoot{} % reset foot
    \fancyhead[RE]{\truncate{0.9\headwidth}{\fontrun\elbibl}} % book ref
    \fancyhead[LO]{\truncate{0.9\headwidth}{\fontrun\nouppercase\leftmark}} % Chapter title, \nouppercase needed
    \fancyhead[RO,LE]{\thepage}
  }
  \fancypagestyle{plain}{% apply to chapter
    \fancyhf{}% clear all header and footer fields
    \setlength{\headheight}{1.5em}
    \fancyhead[L]{\truncate{0.9\headwidth}{\fontrun\elbibl}}
    \fancyhead[R]{\thepage}
  }
\fi

\ifav % a5 only
  \titleclass{\section}{top}
\fi

\newcommand\chapo{{%
  \vspace*{-3em}
  \centering\parindent0pt % no vskip ()
  \eltitlepage
  \bigskip
  {\color{rubric}\hline\par}
  \bigskip
  {\Large TEXTE LIBRE À PARTICIPATIONS LIBRES\par}
  \centerline{\small\color{rubric} {\href{https://hurlus.fr}{\dotuline{hurlus.fr}}}, tiré le \today}\par
  \bigskip
}}

\newcommand\cover{{%
  \thispagestyle{empty}
  \centering\parindent0pt
  \eltitlepage
  \vfill\null
  {\color{rubric}\setlength{\arrayrulewidth}{2pt}\hline\par}
  \vfill\null
  {\Large TEXTE LIBRE À PARTICIPATIONS LIBRES\par}
  \centerline{\href{https://hurlus.fr}{\dotuline{hurlus.fr}}, tiré le \today}\par
}}

\begin{document}
\pagestyle{empty}
\ifbooklet{
  \cover\newpage
  \thispagestyle{empty}\hbox{}\newpage
  \cover\newpage\noindent Les voyages de la brochure\par
  \bigskip
  \begin{tabularx}{\textwidth}{l|X|X}
    \textbf{Date} & \textbf{Lieu}& \textbf{Nom/pseudo} \\ \hline
    \rule{0pt}{25cm} &  &   \\
  \end{tabularx}
  \newpage
  \addtocounter{page}{-4}
}\fi

\thispagestyle{empty}
\ifaiv
  \twocolumn[\chapo]
\else
  \chapo
\fi
{\it\elabstract}
\bigskip
\makeatletter\@starttoc{toc}\makeatother % toc without new page
\bigskip

\pagestyle{main} % after style
\setcounter{footnote}{0}
\setcounter{footnoteA}{0}
  
\section[{Préface}]{Préface}
\renewcommand{\leftmark}{Préface}

\noindent En 1849, ayant vingt et un ans, j’étais électeur et fort embarrassé ; car j’avais à nommer quinze ou vingt députés, et de plus, selon l’usage français, je devais non seulement choisir des hommes, mais opter entre des théories. On me proposait d’être royaliste ou républicain, démocrate ou conservateur, socialiste ou bonapartiste : je n’étais rien de tout cela, ni même rien du tout, et parfois j’enviais tant de gens convaincus qui avaient le bonheur d’être quelque chose. Après avoir écouté les diverses doctrines, je reconnus qu’il y avait sans doute une lacune dans mon esprit. Des motifs valables pour d’autres ne l’étaient pas pour moi ; je ne pouvais comprendre qu’en politique on pût se décider d’après ses préférences. Mes gens affirmatifs construisaient une constitution comme une maison, d’après le plan le plus beau, le plus neuf ou le plus simple, et il y en avait plusieurs à l’étude, hôtel de marquis, maison de bourgeois, logement d’ouvriers, caserne de militaires, phalanstère de communistes, et même campement de sauvages. Chacun disait de son modèle : « Voilà la vraie demeure de l’homme, la seule qu’un homme de sens puisse habiter ». À mon sens l’argument était faible : des goûts personnels ne me semblaient pas des autorités. Il me paraissait qu’une maison ne doit pas être construite pour l’architecte, ni pour elle-même, mais pour le propriétaire qui va s’y loger. — Demander l’avis du propriétaire, soumettre au peuple français les plans de sa future habitation, c’était trop visiblement parade ou duperie : en pareil cas, la question fait toujours la réponse, et d’ailleurs, cette réponse eût-elle été libre, la France n’était guère plus en état que moi de la donner : dix millions d’ignorances ne font pas un savoir. Un peuple consulté peut à la rigueur dire la forme de gouvernement qui lui plaît, mais non celle dont il a besoin ; il ne le saura qu’à l’usage : il lui faut du temps pour vérifier si sa maison politique est commode, solide, capable de résister aux intempéries, appropriée à ses mœurs, à ses occupations, à son caractère, à ses singularités, à ses brusqueries. Or, à l’épreuve, nous n’avons jamais été contents de la nôtre : treize fois en quatre-vingts ans, nous l’avons démolie pour la refaire, et nous avons eu beau la refaire, nous n’avons pas encore trouvé celle qui nous convient. Si d’autres peuples ont été plus heureux, si, à l’étranger, plusieurs habitations politiques sont solides et subsistent indéfiniment, c’est qu’elles ont été construites d’une façon particulière, autour d’un noyau primitif et massif, en s’appuyant sur quelque vieil édifice central plusieurs fois raccommodé, mais toujours conservé, élargi par degrés, approprié par tâtonnements et rallonges aux besoins des habitants. Nulle d’entre elles n’a été bâtie d’un seul coup, sur un patron neuf, et d’après les seules mesures de la raison. Peut-être faut-il admettre qu’il n’y a pas d’autre moyen de construire à demeure, et que l’invention subite d’une constitution nouvelle, appropriée, durable, est une entreprise qui surpasse les forces de l’esprit humain.\par
En tout cas, je concluais que, si jamais nous découvrons celle qu’il nous faut, ce ne sera point par les procédés en vogue. En effet, il s’agit de la {\itshape découvrir}, si elle existe, et non de la mettre aux voix. À cet égard, nos préférences seraient vaines ; d’avance la nature et l’histoire ont choisi pour nous ; c’est à nous de nous accommoder à elles, car il est sûr qu’elles ne s’accommoderont pas à nous. La forme sociale et politique dans laquelle un peuple peut entrer et {\itshape rester} n’est pas livrée à son arbitraire, mais déterminée par son caractère et son passé. Il faut que, jusque dans ses moindres traits, elle se moule sur les traits vivants auxquels on l’applique ; sinon elle crèvera et tombera en morceaux. C’est pourquoi, si nous parvenons à trouver la nôtre, ce ne sera qu’en nous étudiant nous-mêmes, et plus nous saurons précisément ce que nous sommes, plus nous démêlerons sûrement ce qui nous convient. On doit donc renverser les méthodes ordinaires et se figurer la nation avant de rédiger la constitution. Sans doute, la première opération est beaucoup plus longue et plus difficile que la seconde. Que de temps, que d’études, que d’observations rectifiées l’une par l’autre, que de recherches dans le présent et dans le passé, sur tous les domaines de la pensée et de l’action, quel travail multiplié et séculaire, pour acquérir l’idée exacte et complète d’un grand peuple qui a vécu âge de peuple et qui vit encore ! Mais c’est le seul moyen de ne pas construire à faux après avoir raisonné à vide, et je me promis que, pour moi du moins, si j’entreprenais un jour de chercher une opinion politique, ce ne serait qu’après avoir étudié la France.\par
Qu’est-ce que la France contemporaine ? Pour répondre à cette question, il faut savoir comment cette France s’est faite, ou, ce qui vaut mieux encore, assister en spectateur à sa formation. À la fin du siècle dernier, pareille à un insecte qui mue, elle subit une métamorphose.. Son ancienne organisation se dissout ; elle en déchire elle-même les plus précieux tissus et tombe en des convulsions qui semblent mortelles. Puis, après des tiraillements multipliés et une léthargie pénible, elle se redresse. Mais son organisation n’est plus la même : par un sourd travail intérieur, un nouvel être s’est substitué à l’ancien. En 1808, tous ses grands traits sont arrêtés et définitifs : départements, arrondissements, cantons et communes, rien n’a changé depuis dans ses divisions et sutures extérieures : Concordat, Code, Tribunaux, Université, Institut, Préfets, Conseil d’État, impôts, percepteurs, Cour des Comptes, administration uniforme et centralisée, ses principaux organes sont encore les mêmes ; noblesse, bourgeoisie, ouvriers, paysans, chaque classe a dès lors la situation, les intérêts, les sentiments, les traditions que nous lui voyons aujourd’hui. Ainsi la créature nouvelle est à la fois stable et complète ; partant, sa structure, ses instincts et ses facultés marquent d’avance le cercle dans lequel s’agitera sa pensée ou son action. Autour d’elle, les autres nations, les unes précoces, les autres tardives, toutes avec des ménagements plus grands, quelques-unes avec succès meilleur, opèrent de même la transformation qui les fait passer de l’état féodal à l’état moderne ; l’éclosion est universelle et presque simultanée. Mais, sous cette forme nouvelle comme sous la forme ancienne, le faible est toujours la proie du fort. Malheur à ceux que leur évolution trop lente livre au voisin qui subitement s’est dégagé de sa chrysalide et sort le premier tout armé ! Malheur aussi à celui dont l’évolution trop violente et trop brusque a mal équilibré l’économie intérieure, et qui, par l’exagération de son appareil directeur, par l’altération de ses organes profonds, par l’appauvrissement graduel de sa substance vivante, est condamné aux coups de tête, à la débilité, à l’impuissance, au milieu de voisins mieux proportionnés et plus sains ! Dans l’organisation que la France s’est faite au commencement du siècle, toutes les lignes générales de son histoire contemporaine étaient tracées, révolutions politiques, utopies sociales, divisions des classes, rôle de l’Église, conduite de la noblesse, de la bourgeoisie et du peuple, développement, direction ou déviation de la philosophie, des lettres et des arts. C’est pourquoi, lorsque nous voulons comprendre notre situation présente, nos regards sont toujours ramenés vers la crise terrible et féconde par laquelle l’Ancien Régime a produit la Révolution, et la Révolution le Régime nouveau.\par
\par
Ancien Régime, Révolution, Régime nouveau, je vais tâcher de décrire ces trois états avec exactitude. J’ose déclarer ici que je n’ai point d’autre but ; on permettra à un historien d’agir en naturaliste. J’étais devant mon sujet comme devant la métamorphose d’un insecte. D’ailleurs, l’événement par lui-même est si intéressant, qu’il vaut la peine d’être observé pour lui seul, et l’on n’a pas besoin d’effort pour exclure les arrière-pensées. Dégagée de tout parti pris, la curiosité devient scientifique et se porte tout entière vers les forces intimes qui conduisent l’étonnante opération. Ces forces sont la situation, les passions, les idées, les volontés de chaque groupe, et nous pouvons les démêler, presque les mesurer. Elles sont sous nos yeux ; nous n’en sommes pas réduits aux conjectures, aux divinations douteuses, aux indications vagues. Par un bonheur singulier, nous apercevons les hommes eux-mêmes, leurs dehors et leur dedans. Les Français de l’Ancien Régime sont encore tout près de nos regards. Chacun de nous, dans sa jeunesse, a pu fréquenter quelques-uns des survivants de ce monde évanoui. Plusieurs de leurs hôtels subsistent encore, avec leurs appartements et leurs meubles intacts. Au moyen de leurs tableaux et de leurs estampes, nous les suivons dans leur vie domestique ; nous voyons leurs habillements, leurs attitudes et leurs gestes. Avec leur littérature, leur philosophie, leurs sciences, leurs gazettes et leurs correspondances, nous pouvons reconstituer toute leur pensée et jusqu’à leur conversation familière. Une multitude de Mémoires, sortis depuis trente ans des archives publiques ou privées, nous conduisent de salon en salon, comme si nous y étions présentés. Des lettres et journaux de voyageurs étrangers contrôlent et complètent, par des peintures indépendantes, les portraits que cette société a tracés d’elle-même. Elle a tout dit sur son propre compte, sauf ce qu’elle supposait banal et familier aux contemporains, sauf ce qui lui semblait technique, ennuyeux et mesquin, sauf ce qui concernait la province, la bourgeoisie, le paysan, l’ouvrier, l’administration et le ménage. J’ai voulu suppléer à ces omissions, et, outre le petit cercle des Français bien élevés et lettrés, connaître la France. Grâce à l’obligeance de M. Maury et aux précieuses indications de M. Boutaric, j’ai pu dépouiller une multitude de documents manuscrits, la correspondance d’un grand nombre d’intendants, directeurs des aides, fermiers généraux, magistrats, employés et particuliers, de toute espèce et de tout degré pendant les trente dernières années de l’Ancien Régime, les Rapports et Mémoires sur les diverses parties de la maison du roi, les procès-verbaux et cahiers des États généraux en cent soixante-seize volumes, la correspondance des commandants militaires en 1789 et 1790, les lettres, mémoires et statistiques détaillées contenus dans les cent cartons du Comité ecclésiastique, la correspondance en quatre-vingt-quatorze liasses des administrations de département et de municipalité avec les ministres de 1790 à 1799, les rapports des conseillers d’État en mission à la fin de 1801, la correspondance des préfets sous le Consulat, sous l’Empire et sous la Restauration jusqu’en 1825, quantité d’autres pièces si instructives et si inconnues, qu’en vérité l’histoire de la Révolution semble encore inédite. Du moins il n’y a que ces documents pour nous montrer des figures vivantes, petits nobles, curés, moines et religieuses de province, avocats, échevins et bourgeois des villes, procureurs de campagne et syndics de villages, laboureurs et artisans, officiers et soldats. Il n’y a qu’eux pour nous faire voir en détail et de près la condition des hommes, l’intérieur d’un presbytère, d’un couvent, d’un conseil de ville, le salaire d’un ouvrier, le produit d’un champ, les impositions d’un paysan, le métier d’un collecteur, les dépenses d’un seigneur ou d’un prélat, le budget, le train et le cérémonial d’une cour. Grâce à eux, nous pouvons donner des chiffres précis, savoir, heure par heure, l’emploi d’une journée, bien mieux, dire le menu d’un grand dîner, recomposer une toilette d’apparat. Nous avons encore, piqués sur le papier et classés par dates, les échantillons des robes que la reine Marie-Antoinette a portées et, d’autre part, nous pouvons nous figurer l’habillement d’un paysan, décrire son pain, nommer les farines dont il le composait, marquer en sous et deniers ce que lui en coûtait une livre. Avec de telles ressources, on devient presque le contemporain des hommes dont on fait l’histoire, et plus d’une fois, aux Archives, en suivant sur le papier jauni leurs vieilles écritures, j’étais tenté de leur parler tout haut.\par

\dateline{Menthon-Saint-Bernard, août 
\docDate{1875}
.}

\chapteropen

\part[{Livre premier. La structure de la société.}]{Livre premier. \\
La structure de la société.}
\renewcommand{\leftmark}{Livre premier. \\
La structure de la société.}


\chaptercont

\chapteropen

\chapter[{Chapitre I. Origine des privilèges.}]{Chapitre I. \\
Origine des privilèges.}


\chaptercont
\noindent En 1789, trois sortes de personnes, les ecclésiastiques, les nobles et le roi, avaient dans l’État la place éminente avec tous les avantages qu’elle comporte, autorité, biens, honneurs, ou, tout au moins, privilèges, exemptions, grâces, pensions, préférences et le reste. Si depuis longtemps ils avaient cette place, c’est que pendant longtemps ils l’avaient méritée. En effet, par un effort immense et séculaire, ils avaient construit tour à tour les trois assises principales de la société moderne.\par

\section[{I. Services et récompense du clergé.}]{I. Services et récompense du clergé.}

\noindent Des trois assises superposées, la plus ancienne et la plus profonde était l’ouvrage du clergé : pendant douze cents ans et davantage, il y avait travaillé comme architecte et comme manœuvre, d’abord seul, puis presque seul. Au commencement, pendant les quatre premiers siècles, il avait fait la religion et l’Église : pesons ces deux mots pour en sentir tout le poids. D’une part, dans un monde fondé sur la conquête, dur et froid comme une machine d’airain, condamné par sa structure même à détruire chez ses sujets le courage d’agir et l’envie de vivre, il avait annoncé « la bonne nouvelle », promis « le royaume de Dieu », prêché la résignation tendre aux mains du père céleste, inspiré la patience, la douceur, l’humilité, l’abnégation, la charité, ouvert les seules issues par lesquelles l’homme étouffé dans l’ergastule romain pouvait encore respirer et apercevoir le jour : voilà la religion. D’autre part, dans un État qui peu à peu se dépeuplait, se dissolvait et fatalement devenait une proie, il avait formé une société vivante ; guidée par une discipline et des lois, ralliée autour d’un but et d’une doctrine, soutenue par le dévouement des chefs et l’obéissance des fidèles, seule capable de subsister sous le flot de barbares que l’Empire en ruine laissait entrer par toutes ses brèches : voilà l’Église. — Sur ces deux premières fondations, il continue à bâtir, et, à partir de l’invasion, pendant plus de cinq cents ans, il sauve ce qu’on peut encore sauver de la culture humaine. Il va au-devant des barbares, ou les gagne aussitôt après leur entrée ; service énorme ; jugeons-en par un seul fait : dans la Grande-Bretagne, devenue latine comme la Gaule, mais dont les conquérants demeurèrent païens pendant un siècle et demi, arts, industries, société, langue, tout fut détruit ; d’un peuple entier massacré ou fugitif, il ne resta que des esclaves ; encore faut-il deviner leurs traces ; réduits à l’état de bêtes de somme, ils disparaissent de l’histoire. Tel eût été le sort de l’Europe, si le clergé n’eût promptement charmé les brutes farouches auxquelles elle appartenait.\par
Devant l’évêque en chape dorée, devant le moine « vêtu de peaux, maigre », hâve, « plus souillé et plus couvert de taches qu’un caméléon\footnote{Comte de Montalembert, {\itshape les Moines d’Occident}, I, 277. Saint Lupicin devant le roi burgonde Chilpéric, II, 416. Saint Karileff devant le roi Childebert. Cf. passim Grégoire de Tours et la collection des Bollandistes.} », le Germain converti a peur comme devant un sorcier. Aux heures calmes, après la chasse ou l’ivresse, la divination vague d’un {\itshape au-delà} mystérieux et grandiose, le sentiment obscur d’une justice inconnue, le rudiment de conscience qu’il avait déjà dans ses forêts d’outre-Rhin, se réveille en lui par des alarmes subites, en demi-visions menaçantes. Au moment de violer un sanctuaire, il se demande s’il ne va pas tomber sur le seuil, frappé de vertige et le col tordu\footnote{Rien de plus fréquent que cette légende ; on la trouve jusqu’au-delà du douzième siècle.}. Convaincu par son propre trouble, il s’arrête, épargne la terre, le village, la cité qui vit sous la sauvegarde du prêtre. Si la fougue animale des colères ou des convoitises primitives l’a poussé au meurtre et au vol, plus tard, après l’assouvissement, aux jours du malheur ou de maladie, sur les conseils de sa concubine ou de sa femme, il se repent ; il restitue au double, au décuple et au centuple, il prodigue les donations et les immunités\footnote{Par exemple Chilpéric, sur les conseils de Frédégonde, après la mort de tous leurs enfants.}. Ainsi, sur tout le territoire, le clergé garde et agrandit ses asiles pour les vaincus et pour les opprimés. — D’autre part, parmi les chefs de guerre aux longs cheveux, à côté des rois vêtus de fourrures, l’évêque mitré et l’abbé au front tondu siègent aux assemblées ; ils sont les seuls qui tiennent la plume, qui sachent discourir. Secrétaires, conseillers, théologiens, ils participent aux édits, ils ont la main dans le gouvernement, ils travaillent par son entremise à mettre un peu d’ordre dans le désordre immense, à rendre la loi plus raisonnable et plus humaine, à rétablir, ou à maintenir la piété, l’instruction, la justice, la propriété et surtout le mariage. Certainement on doit à leur ascendant la police telle quelle, intermittente, incomplète, qui a empêché l’Europe de devenir une anarchie mongole. Jusqu’à la fin du douzième siècle, si le clergé pèse sur les princes, c’est surtout pour refréner en eux et au-dessous d’eux les appétits brutaux, les rébellions de la chair et du sang, les retours et les accès de sauvagerie irrésistible qui démolissaient la société. — Cependant, dans ses églises et dans ses couvents, il conservait les anciennes acquisitions du genre humain, la langue latine, la littérature et la théologie chrétiennes, une portion de la littérature et des sciences païennes, l’architecture, la sculpture, la peinture, les arts et les industries qui servent au culte, les industries plus précieuses qui donnent à l’homme le pain, le vêtement et l’habitation, surtout la meilleure de toutes les acquisitions humaines et la plus contraire à l’humeur vagabonde du barbare pillard et paresseux, je veux dire l’habitude et le goût du travail. Dans les campagnes dépeuplées par le fisc romain, par la révolte des Bagaudes, par l’invasion des Germains, par les courses des brigands, le moine bénédictin bâtit sa cabane de branchages parmi les épines et les ronces\footnote{Montalembert, {\itshape ib.}, t. II, liv. 8, et surtout Alfred Maury, {\itshape les Forêts de la France dans l’antiquité et au moyen âge. Spinæ et vepres}, ce mot revient sans cesse dans les Vies des saints.} ; autour de lui de grands espaces jadis cultivés ne sont plus que des halliers déserts. Avec ses compagnons, il défriche et construit ; il domestique les animaux demi-sauvages, établit une ferme, un moulin, une forge, un four, des ateliers de chaussure et d’habillement. Selon sa règle, chaque jour il lit pendant deux heures ; sept heures durant, il travaille de ses mains, et il ne mange, il ne boit que le strict nécessaire. Par son travail intelligent, volontaire, exécuté en conscience et conduit en vue de l’avenir, il produit plus que le laïque. Par son régime sobre, concerté, économique, il consomme moins que le laïque. C’est pourquoi là où le laïque avait défailli\footnote{De même aujourd’hui les colonies des Trappistes en Algérie.}, il se soutient et même il prospère. Il recueille les misérables, les nourrit, les occupe, les marie ; mendiants, vagabonds, paysans fugitifs affluent autour du sanctuaire : Par degrés leur campement devient un village, puis une bourgade : l’homme laboure dès qu’il peut compter sur la récolte et devient père de famille sitôt qu’il se croit en état de nourrir ses enfants. Ainsi se forment de nouveaux centres d’agriculture et d’industrie qui deviennent aussi des centres nouveaux de population\footnote{{\itshape Polyptique d’Irminon} par Guérard ; on y verra la prospérité des domaines de l’abbaye de Saint-Germain-des-Prés à la fin du huitième siècle. D’après les statistiques de M. Guérard, les paysans de Palaiseau au temps de Charlemagne étaient à peu près aussi aisés qu’aujourd’hui.}.\par
Au pain du corps ajoutez celui de l’âme, non moins nécessaire ; car, avec les aliments, il fallait encore donner à l’homme la volonté de vivre, ou tout au moins la résignation qui lui fait tolérer la vie, et le rêve touchant ou poétique qui lui tient lieu du bonheur absent. Jusqu’au milieu du treizième siècle, le clergé s’est trouvé presque seul à le fournir. Par ses innombrables légendes de saints, par ses cathédrales et leur structure, par ses statues et leur expression, par ses offices et leur sens encore transparent, il a rendu sensible « le royaume de Dieu », et dressé le monde idéal au bout du monde réel, comme un magnifique pavillon d’or au bout d’un enclos fangeux\footnote{ \noindent Du sixième au dixième siècle, il y a vingt-cinq mille Vies de saints rassemblées par les Bollandistes. — Les dernières vraiment inspirées sont celles de saint François d’Assise et de ses compagnons au commencement du quatorzième siècle. Le même sentiment vif se prolonge jusqu’à la fin du quinzième siècle dans les peintures de Beato Angelico et de Hans Memling. — La Sainte Chapelle de Paris, l’église supérieure d’Assise, le paradis de Dante, les {\itshape Fioretti} peuvent donner une idée de ces visions. En fait d’œuvres littéraires modernes, l’état de l’âme croyante au moyen âge a été parfaitement peint par Henri Heine dans le {\itshape Pèlerinage à Kevlaar}, et par Tourguenef dans {\itshape les Reliques vivantes.}
 }. C’est dans ce monde doux et divin que se réfugie le cœur attristé, affamé de mansuétude et de tendresse. Là les persécuteurs, au moment de frapper, tombent sous une atteinte invisible ; les bêtes sauvages deviennent dociles ; les cerfs de la forêt viennent chaque matin s’atteler d’eux-mêmes à la charrue des saints ; la campagne fleurit pour eux comme un nouveau paradis ; ils ne meurent que quand ils veulent. Cependant ils consolent les hommes ; la bonté, la piété, le pardon coulent de leurs lèvres en suavités ineffables ; les yeux levés au ciel, ils voient Dieu et, sans effort, comme en un songe, ils montent dans la lumière pour s’asseoir à sa droite. Légende divine, d’un prix inestimable sous le règne universel de la force brutale, quand, pour supporter la vie, il fallait en imaginer une autre et rendre la seconde aussi visible aux yeux de l’âme que la première l’était aux yeux du corps. Pendant plus de douze siècles, le clergé en a nourri les hommes et, par la grandeur de sa récompense, on peut estimer la profondeur de leur gratitude : Ses papes ont été pendant deux cents ans les dictateurs de l’Europe. Il a fait des croisades, détrôné des rois, distribué des États. Ses évêques et ses abbés sont devenus ici princes souverains, là patrons et véritables fondateurs de dynasties. Il a tenu dans ses mains le tiers des terres, la moitié du revenu, les chaux tiers du capital de l’Europe. Ne croyons pas que l’homme soit reconnaissant à faux et donne sans motif valable ; il est trop égoïste et trop envieux pour cela. Quel que soit l’établissement, ecclésiastique ou séculier, quel que soit le clergé, bouddhiste ou chrétien, les contemporains qui l’observent pendant quarante générations ne sont pas de mauvais juges ; ils ne lui livrent leurs volontés et leurs biens qu’à proportion de ses services, et l’excès de leur dévouement peut mesurer l’immensité de son bienfait.

\section[{II. Services et récompense des nobles.}]{II. Services et récompense des nobles.}

\noindent Jusqu’ici, contre la force des francisques et des glaives, on n’a trouvé de secours que dans la persuasion et dans la patience. Les États qui, d’après l’exemple de l’ancien Empire, ont tenté de s’élever en édifices compacts et d’opposer une digue à l’invasion incessante, n’ont pas tenu sur le sol mouvant ; après Charlemagne, tout s’effondre. Il n’y a plus d’hommes de guerre à partir de la bataille de Fontanet ; pendant un demi-siècle des bandes de quatre ou cinq cents brigands viennent impunément tuer, brûler, dévaster dans tout le pays. — Mais, par contre-coup, à ce moment même, la dissolution de l’État suscite une génération militaire. Chaque petit chef a planté solidement ses pieds dans le domaine qu’il occupe ou qu’il détient ; il ne l’a plus en prêt ou en usage, mais en propriété et en héritage. C’est sa manse, sa bourgade, sa comté, ce n’est plus celle du roi ; il va combattre pour la défendre. À cet instant, le bienfaiteur, le sauveur est l’homme qui sait se battre et défendre les autres, et tel est effectivement le caractère de la nouvelle classe qui s’établit. Dans la langue du temps, le noble est l’homme de guerre, le soldat ({\itshape miles}), et c’est lui qui pose la seconde assise de la société moderne.\par
Au dixième siècle, peu importe son extraction. Souvent c’est un comte carlovingien, un bénéficier du roi, le hardi propriétaire d’une des dernières terres franches. Ici c’est un évêque guerrier, un vaillant abbé, ailleurs un païen converti, un bandit devenu sédentaire, un aventurier qui a prospéré, un rude chasseur qui s’est nourri longtemps de sa chasse et de fruits sauvages\footnote{Par exemple Tertulle, souche des Plantagenets : Rollon, duc de Normandie ; Hugues, abbé de Saint-Martin de Tours et de Saint-Denis.}. Les ancêtres de Robert le Fort sont inconnus et l’on contera plus tard que les Capétiens descendent d’un boucher de Paris. En tout cas, le noble alors c’est le brave, l’homme fort et expert aux armes, qui, à la tête d’une troupe, au lieu de s’enfuir et de payer rançon, présente sa poitrine, tient ferme et protège par l’épée un coin du sol. Pour faire cet office, il n’a pas besoin d’ancêtres, ne lui faut que du cœur, il est lui-même un ancêtre ; on est trop heureux du salut présent qu’il apporte pour le chicaner sur son titre. — Enfin, après tant de siècles, voici dans chaque canton des bras armés, une troupe sédentaire, capable de résister à l’invasion nomade ; on ne sera plus en proie à l’étranger ; au bout d’un siècle, cette Europe que saccageaient des flottilles de barques à deux voiles, va jeter deux cent mille hommes armés sur l’Asie, et désormais, au Nord, au Midi, en face des Musulmans, en face des païens, au lieu d’être conquise, elle conquiert. Pour la seconde fois, une figure idéale se dégage\footnote{Au dixième siècle, dans les « Cantilènes » qui ébauchent les chansons de Geste.} après celle du saint, celle du héros, et le nouveau sentiment, aussi efficace que l’ancien, groupe aussi les hommes en une société stable. Celle-ci est une gendarmerie à demeure où, de père en fils, on est gendarme. Chacun y naît avec son grade héréditaire, son poste local, sa solde en biens-fonds, avec la certitude de n’être jamais abandonné par son chef, avec l’obligation de se faire tuer au besoin pour son chef. En ce temps de guerre permanente, un seul régime est bon, celui d’une compagnie devant l’ennemi, et tel est le régime féodal ; par ce seul trait, jugez des périls auxquels il pare et du service auquel il astreint. « En ce temps-là, dit la chronique générale d’Espagne, les rois, comtes, nobles et tous les chevaliers, afin d’être prêts à toute heure, tenaient leurs chevaux dans la salle où ils couchaient avec leurs femmes. » Le vicomte dans la tour qui défend l’entrée de la vallée ou le passage du gué, le marquis jeté en enfant perdu sur la frontière brûlée, sommeille la main sur son arme, comme le lieutenant américain dans un blockhaus du Far-West, au milieu des Sioux. Sa maison n’est qu’un camp et un refuge ; on a mis de la paille et des tas de feuilles sur le pavé de la grande salle ; c’est là qu’il couche avec ses cavaliers, ôtant un éperon quand il a chance de dormir ; les meurtrières laissent à peine entrer le jour ; c’est qu’il s’agit avant tout de ne pas recevoir des flèches. Tous les goûts, tous les sentiments sont subordonnés au service ; il y a tel point de la frontière européenne où l’enfant de quatorze ans est tenu de marcher, où la veuve jusqu’à soixante ans est forcée de se remarier. Des hommes dans les rangs pour combler les vides, des hommes dans les postes pour monter la garde, voilà le cri qui sort à ce moment de toutes les institutions, comme l’appel d’une voix d’airain. Grâce à ces braves, le paysan\footnote{ Villanus.
 } est à l’abri ; on ne le tuera plus, on ne l’emmènera plus captif avec sa famille, par troupeaux, la fourche au cou. Il ose labourer, semer, espérer en sa récolte ; en cas de danger, il sait qu’il trouvera un asile pour lui, pour ses grains et pour ses bestiaux, dans l’enclos de palissades au pied de la forteresse. Par degrés, entre le chef militaire du donjon et les anciens colons de la campagne ouverte, la nécessité établit un contrat tacite qui devient une coutume respectée. Ils travaillent pour lui, cultivent ses terres, font ses charrois, lui payent des redevances, tant par maison, tant par tête de bétail, tant pour hériter ou vendre : il faut bien qu’il nourrisse sa troupe. Mais, ces droits acquittés, il a tort, si, par orgueil ou avidité, il leur prend quelque chose de plus. — Quant aux vagabonds, aux misérables qui, dans le désordre et la dévastation universelle, viennent se réfugier sons sa garde, leur condition est plus dure : la terre est à lui, puisque sans lui elle serait inhabitable ; s’il leur en accorde une parcelle, si même il leur permet seulement d’y camper, s’il leur donne du travail ou des semailles, c’est aux conditions qu’il édicte. Ils seront ses serfs ; ses mainmortables ; quelque part qu’ils aillent, il aura le droit de les ressaisir et ils seront, de père en fils, ses domestiques-nés, applicables au métier qu’il lui plaira, taillables et corvéables à sa merci, ne pouvant rien transmettre à leur enfant que si celui-ci, « vivant à leur pot », peut après leur mort continuer leur service. « Ne pas être tué, dit Stendhal, et avoir l’hiver un bon habit de peau, tel était pour beaucoup de gens le suprême bonheur au dixième siècle » ; ajoutons-y pour une femme celui de ne pas être violée par toute une bande. Quand on se représente un peu nettement la condition des hommes en ce temps-là, on comprend qu’ils aient accepté de bon cœur les pires droits féodaux, même celui de marquette ; ce qu’on subissait tous les jours était pire encore\footnote{Voir dans {\itshape les Voyages de Caillaud} en Nubie et en Abyssinie les razzias d’esclaves faites par les armées du pacha ; tel était à peu près le spectacle que donnait l’Europe de 800 à 900.}. La preuve en est qu’on accourait dans l’enceinte féodale, sitôt qu’elle était faite ; en Normandie, par exemple, dès que Rollon eut divisé les terres au cordeau et pendu les voleurs, les gens des provinces voisines affluèrent pour s’établir ; un peu de sécurité suffisait pour repeupler un pays.\par
On vit donc, ou plutôt on recommence à vivre sous la rude main gantée de fer qui vous rudoie, mais qui vous protège. Souverain et propriétaire, à ce double titre le seigneur garde pour lui la lande, la rivière, la forêt, toute la chasse ; le mal n’est pas grand, puisque le pays est à demi désert et qu’il emploie tout son loisir à détruire les grandes bêtes fauves. Ayant seul des avances, il est le seul qui puisse construire le moulin, le four et le pressoir, établir le bac, le pont ou la route, endiguer l’étang, élever ou acquérir le taureau ; pour se dédommager, il en taxe ou en impose l’usage. S’il est intelligent et bon fermier d’hommes, s’il veut tirer meilleur profit de sa terre, il relâche ou laisse se relâcher par degrés les mailles du rets où ses vilains et ses serfs travaillent mal parce qu’ils sont trop serrés. L’habitude, la nécessité, l’accommodation volontaire ou forcée font leur effet ; à la fin, seigneurs, vilains, serfs et bourgeois, adaptés à leur condition, reliés par un intérêt commun, font ensemble une société, un véritable corps. La seigneurie, la comté, le duché deviennent une {\itshape patrie} que l’on aime d’un instinct aveugle et pour laquelle on se dévoue. Elle se confond avec le seigneur et sa famille ; à ce titre, on est fier de lui, on conte ses grands coups d’épée ; on l’acclame quand sa cavalcade passe dans la rue ; on jouit par sympathie de sa magnificence\footnote{Voir dans les historiens du moyen âge, le zèle des sujets pour leur seigneur : Gaston Phoebus, comte de Foix, et Guy, comte de Flandre, dans Froissart ; Raymond de Béziers et Raymond de Toulouse, dans la chronique de Toulouse. Ce vif sentiment de la petite patrie locale apparaît à chaque réunion de province, Normandie, Bretagne, Franche-Comté, etc.}. Lorsqu’il est veuf et sans enfants, on députe auprès de lui pour qu’il se remarie et que sa mort ne livre pas le pays à la guerre des prétendants ou aux convoitises des voisins. — Ainsi renaît, après mille ans, le plus puissant et le plus vivace des sentiments qui soutiennent la société humaine. Celui-ci est d’autant plus précieux qu’il peut s’élargir : pour que la petite patrie féodale devienne la grande patrie nationale, il suffit maintenant que toutes les seigneuries se réunissent entre les mains d’un seul seigneur, et que le roi, chef des nobles, pose sur l’œuvre des nobles la troisième assise de la France.

\section[{III. Services et récompense du roi.}]{III. Services et récompense du roi.}

\noindent Il a édifié toute cette assise, pierre à pierre. Hugues Capet pose la première ; avant lui, la royauté ne donnait pas au roi une province, pas même Laon ; c’est lui qui ajoute au titre son domaine. Pendant  huit cents ans, par mariage, conquête, adresse, héritage, ce travail d’acquisition se poursuit ; même sous Louis XV, la France s’accroît de la Lorraine et de la Corse. Parti du néant, le roi a fait un État compact qui renferme vingt-six millions d’habitants, et qui est alors le plus puissant de l’Europe. — Dans tout l’intervalle, il a été le chef de la défense publique, le libérateur du pays contre les étrangers, contre le pape au quatorzième siècle, contre les Anglais au quinzième, contre les Espagnols au seizième. Au dedans, dès le douzième siècle, le casque en tête et toujours par chemins, il est le grand justicier, il démolit les tours des brigands féodaux, il réprime les excès des forts, il protège les opprimés\footnote{Suger, {\itshape Vie de Louis} VI.}, il abolit les guerres privées, il établit l’ordre et la paix : œuvre immense qui, de Louis le Gros à saint Louis, de Philippe le Bel à Charles VII et à Louis XI, de Henri IV à Louis XIII et à Louis XIV, se continue sans s’interrompre jusqu’au milieu du dix-septième siècle, par l’édit contre les duels et par les Grands Jours\footnote{{\itshape Les Grands Jours d’Auvergne} par Fléchier, éd. Chéruel. Sous Louis XV le dernier brigand féodal, le marquis de Pleumartin, en Poitou, est pris, jugé et décapité (1756).}. Cependant toutes les choses utiles exécutées par son ordre ou développées sous son patronage, routes, ports, canaux, asiles, universités, académies, établissements de piété, de refuge, d’éducation, de science, d’industrie et de commerce, portent sa marque et le proclament bienfaiteur public. — De tels services appellent une récompense proportionnée : on admet que, de père en fils, il contracte mariage avec la France, qu’elle n’agit que par lui, qu’il n’agit que pour elle, et tous les souvenirs anciens, tous les intérêts présents viennent autoriser cette union. L’Église la consacre à Reims par une sorte de huitième sacrement accompagné de légendes et de miracles ; il est l’oint de Dieu\footnote{Sous Louis XV encore, on envoya le procès-verbal des écrouelles guéries.}. Les nobles, par un vieil instinct de fidélité militaire, se considèrent comme sa garde, et viendront jusqu’au 10 août se faire tuer pour lui dans son escalier ; il est leur général-né. Le peuple, jusqu’en 1789, verra en lui le redresseur des torts, le gardien du droit, le protecteur des faibles, le grand aumônier, l’universel refuge. Au commencement du règne de Louis XVI, « les cris de vive le Roi, qui commençaient à six heures du matin, n’étaient presque point interrompus jusqu’après le coucher du soleil\footnote{\href{http://gallica.bnf.fr/ark:/12148/bpt6k2050396/f139}{\dotuline{{\itshape Mémoires} de Mme Campan, I, 89}} [\url{http://gallica.bnf.fr/ark:/12148/bpt6k2050396/f139}] ; II, 215.} ». Quand naquit son dauphin, la joie de la France fut celle d’une famille, « on s’arrêtait dans les rues, on se parlait sans se connaître, on embrassait tous les gens que l’on connaissait\footnote{ \noindent En 1785, un Anglais venu en France vante la liberté politique dont on jouit dans son pays. En revanche, les Français reprochent aux Anglais d’avoir décapité Charles I\textsuperscript{er} et « se glorifient d’avoir toujours gardé à leur propre roi un attachement inviolable, une fidélité, un respect que nul excès ou sévérité de sa part n’a pu ébranler ». ({\itshape A comparative view of the French and of the English nation}, by John Andrews, 257.)
 } ». Tous, par une vague tradition, par un respect immémorial, sentent que la France est un vaisseau construit par ses mains et par les mains de ses ancêtres, qu’à ce titre le bâtiment est à lui, qu’il y a droit comme chaque passager à sa pacotille, et que son seul devoir est d’être expert et vigilant pour bien conduire sur la mer le magnifique navire où toute la fortune publique vogue sous son pavillon. — Sous l’ascendant d’une pareille idée, on l’a laissé tout faire ; de force ou de gré, il a réduit les anciennes autorités à n’être plus qu’un débris, un simulacre, un souvenir. Les nobles ne sont que ses officiers ou ses courtisans. Depuis le concordat, il nomme les dignitaires de l’Église. Les États généraux n’ont pas été convoqués depuis cent soixante-quinze ans ; les États provinciaux qui subsistent ne font que répartir les impôts ; les Parlements sont exilés quand ils hasardent des remontrances. Par son Conseil, ses intendants, ses subdélégués, il intervient dans la moindre affaire locale. Il a quatre cent soixante-dix-sept millions de revenu\footnote{{\itshape Mémoires} d’Augeard, secrétaire des commandements de la reine et ancien fermier général.}. Il distribue la moitié de celui du clergé. Enfin il est maître absolu et le déclare\footnote{Réponse de Louis XV au Parlement de Paris, le 3 mars 1766, dans un lit de justice : « C’est en ma personne seule que réside l’autorité souveraine… C’est à moi seul qu’appartient le pouvoir législatif sans dépendance et sans partage. L’ordre public tout entier émane de moi ; j’en suis le gardien suprême. Mon peuple n’est qu’un avec moi ; les droits et les intérêts de la nation, dont on ose faire un corps séparé du monarque, sont nécessairement unis avec les miens et ne reposent qu’entre mes mains ».}. — Ainsi des biens, des exemptions d’impôt, des agréments d’amour-propre, quelques restes de juridiction ou d’autorité locale, voilà ce qui reste à ses anciens rivaux ; en échange, ils ont ses préférences et ses grâces. — Telle est en abrégé l’histoire des privilégiés, clergé, noblesse et roi ; il faut se la rappeler pour comprendre leur situation au moment de leur chute ; ayant fait la France, ils en jouissent. Voyons de près ce qu’ils sont devenus à la fin du dix-huitième siècle, quelle portion ils ont gardée de leurs avantages, quels services ils rendent encore et quels services ils ne rendent pas.
\chapterclose


\chapteropen

\chapter[{Chapitre II. Les privilèges.}]{Chapitre II. \\
Les privilèges.}


\chaptercont

\section[{I. Nombre des privilégiés.}]{I. Nombre des privilégiés.}

\noindent Ils sont environ 270 000 : dans la noblesse 140 000 ; dans le clergé 130 000\footnote{Voir note 1 [’p. 299’].}. Cela fait de 25 000 à 30 000 familles nobles, 23 000 religieux en 2 500 monastères, 57 000 religieuses en 1 500 couvents, 60 000 curés et vicaires dans autant d’églises et chapelles. Si l’on veut se les représenter un peu nettement, on peut imaginer, dans chaque lieue carrée de terrain et pour chaque millier d’habitants, une famille noble et sa maison à girouette, dans chaque village un curé et son église, toutes les six ou sept lieues une communauté d’hommes ou de femmes. Voilà les anciens chefs et fondateurs de la France : à ce titre, ils ont encore beaucoup de biens et beaucoup de droits.

\section[{II. Leurs biens, capital et revenu.}]{II. Leurs biens, capital et revenu.}

\noindent Souvenons-nous toujours de ce qu’ils ont été pour comprendre ce qu’ils sont encore. Si grands que soient leurs avantages, ils ne sont que les débris d’avantages plus grands. Tel évêque ou abbé, tel comte ou duc dont les successeurs font la révérence à Versailles, fut jadis l’égal des Carlovingiens et des premiers Capétiens. Un sire de Montlhéry a tenu en échec le roi Philippe I\textsuperscript{er}\footnote{Suger, {\itshape Vie de Louis} VI, chap. VIII. — Philippe I\textsuperscript{er} ne s’était rendu maître du château de Montlhéry qu’en mariant un de ses fils à l’héritière du fief. Il disait à son successeur : « Enfant, sois bien attentif à conserver cette tour dont les vexations m’ont fait vieillir, et dont les fraudes et les trahisons ne m’ont jamais donné paix ni trêve. »}. L’abbé de Saint-Germain-des-Prés a possédé quatre cent trente mille hectares de terre, l’étendue d’un département presque entier. Il ne faut pas s’étonner s’ils sont restés puissants et surtout riches ; rien de plus stable qu’une forme de société. Après huit cents ans, malgré tant de coups de la hache royale et l’immense changement de la culture sociale, la vieille racine féodale dure et végète toujours. On s’en aperçoit d’abord à la distribution de la propriété\footnote{Léonce de Lavergne, {\itshape les Assemblées provinciales}, 19. — Cf. les procès-verbaux imprimés de ces assemblées provinciales, notamment dans les chapitres qui traitent des vingtièmes.}. Un cinquième du sol est à la couronne et aux communes, un cinquième au tiers état, un cinquième au peuple des campagnes, un cinquième à la noblesse, un cinquième au clergé. Ainsi, si l’on défalque les terres publiques, les privilégiés possèdent la moitié du royaume. Et ce gros lot est en même temps le plus riche ; car il comprend presque toutes les grandes et belles bâtisses, palais, châteaux, couvents, cathédrales, et presque tout le mobilier précieux, meubles, vaisselle, objets d’art, chefs-d’œuvre accumulés depuis des siècles. — On peut en juger par l’estimation de la part du clergé. Ses biens valent en capital près de 4 milliards\footnote{Rapport de Treilhard au nom du comité ecclésiastique ({\itshape Moniteur}, 19 décembre 1789). Les maisons religieuses à vendre dans la seule ville de Paris étaient estimées cent cinquante millions. Plus tard (séance du 13 février 1791), Amelot estimait les biens vendus et à vendre, non compris les bois, à 3 700 millions. M. de Bouillé estime le revenu du clergé à cent quatre-vingts millions. ({\itshape Mémoires}, 44.)}, ils rapportent de 80 à 100 millions, à quoi il faut joindre la dîme, 125 millions par an, en tout 200 millions, somme qu’il faudrait doubler pour en avoir l’équivalent aujourd’hui ; outre cela, le casuel et les quêtes\footnote{Rapport de Chasset sur les dîmes, avril 1790. Sur les 123 millions, 23 passent en frais de perception ; mais, quand on compte le revenu d’un particulier, on n’en défalque pas ce qu’il paye à ses intendants, régisseurs et caissiers. — Talleyrand (10 octobre 1789) estime le revenu des biens-fonds à 70 millions et leur valeur à 2 100 millions ; mais, à l’examen, le capital et le revenu se sont trouvés notablement plus grands qu’au premier aperçu. (Rapports de Treilhard et de Chasset.) D’ailleurs, dans son évaluation, Talleyrand laissait à part les maisons et enclos d’habitation, ainsi que le quart de réserve des forêts. Il faut en outre compter dans le revenu avant 1789 les droits seigneuriaux dont jouissait l’Église. Enfin, d’après Arthur Young, la rente foncière perçue par le propriétaire en France était non de 2 1/2 pour 100 comme aujourd’hui, mais de 3 3/4 pour 100. — Quant à la nécessité de doubler les chiffres pour avoir leur valeur en monnaie actuelle, elle est établie par quantité de preuves, entre autres par le prix de la journée de travail, qui était alors de dix-neuf sous. (Arthur Young.)}. Pour mieux sentir la largeur de ce fleuve d’or, regardons quelques-uns de ses affluents. Les 399 Pré-montrés estiment leur revenu à plus d’un million et leur capital à 45 millions. Le provincial des Dominicains de Toulouse accuse, pour ses 236 religieux, « plus de 200 000 livres de rentes de revenu net, non compris leurs couvents et leurs enclos, et, dans les colonies, des biens-fonds, des nègres et autres effets, évalués à plusieurs millions ». Les Bénédictins de Cluny, au nombre de 298, ont un revenu de 1 800 000 livres. Ceux de Saint-Maur, au nombre de 1 672, estiment à 24 millions le mobilier de leurs églises et maisons, et à 8 millions leur revenu net, « sans compter ce qui retourne à MM. les abbés et prieurs commendataires », c’est-à-dire autant et peut-être davantage. Dom Rocourt, abbé de Clairvaux, a de 300 000 à 400 000 livres de rente ; le cardinal de Rohan, évêque de Strasbourg, plus d’un million\footnote{{\itshape Archives nationales}, papiers du comité ecclésiastique ; cartons 10, 11, 13, 25. \href{http://gallica.bnf.fr/ark:/12148/bpt6k37337f/f51.table}{\dotuline{Beugnot, {\itshape Mémoires}, I, 49}} [\url{http://gallica.bnf.fr/ark:/12148/bpt6k37337f/f51.table}], \href{http://gallica.bnf.fr/ark:/12148/bpt6k37337f/f82.table}{\dotuline{79}} [\url{http://gallica.bnf.fr/ark:/12148/bpt6k37337f/f82.table}]. Delbos, {\itshape l’Église de France}, I, 399. Duc de Lévis, {\itshape Souvenirs et portraits}, 156.}. Dans la Franche-Comté, l’Alsace et le Roussillon, le clergé possède la moitié des terres ; dans le Hainaut et l’Artois, les trois quarts ; dans le Cambrésis, 1 400 charrues sur 1 700\footnote{Léonce de Lavergne, \href{http://gallica.bnf.fr/ark:/12148/bpt6k836438/f31}{\dotuline{{\itshape Économie rurale en France}, 24}} [\url{http://gallica.bnf.fr/ark:/12148/bpt6k836438/f31}]. Périn, {\itshape la Jeunesse de Robespierre} (doléances des cahiers de l’Artois), 517.}. Le Velay presque entier appartient à l’évêque du Puy, à l’abbé de la Chaise-Dieu, au chapitre noble de Brioude et aux seigneurs de Polignac. Les chanoines de Saint-Claude, dans le Jura, sont propriétaires de 12 000 serfs ou mainmortables\footnote{Boiteau, {\itshape État de la France en} 1789, 47. — Voltaire, {\itshape Politique et législation}, supplique des serfs de Saint-Claude.}. — Par cette fortune du premier ordre, nous pouvons nous figurer celle du second. Comme avec les nobles il comprend les anoblis, et que depuis deux siècles les magistrats, depuis un siècle les financiers ont acquis ou acheté la noblesse, il est clair qu’on y trouve presque toutes les grandes fortunes de France, anciennes ou nouvelles, transmises par héritage, obtenues par des grâces de cour, acquises dans les affaires ; quand une classe est au sommet, elle se recrute de tout ce qui monte ou grimpe. Là aussi il y a des richesses colossales. On a calculé que les apanages des princes de la famille royale, comtes d’Artois et de Provence, ducs d’Orléans et de Penthièvre, couvraient alors le septième du territoire\footnote{Necker, \href{http://gallica.bnf.fr/ark:/12148/bpt6k429493}{\dotuline{{\itshape De l’administration des finances}}} [\url{http://gallica.bnf.fr/ark:/12148/bpt6k429493}], II, 272.}. Les princes du sang ont ensemble un revenu de 24 à 25 millions ; le duc d’Orléans, à lui seul, possède 11 500 000 livres de rente\footnote{ \noindent Marquis de Bouillé, {\itshape Mémoires}, 41. — Notez toujours qu’il faut au moins doubler ces chiffres pour avoir ceux qui leur correspondraient aujourd’hui. 10 000 livres de rente en 1766 en valaient 20 000 en 1825. (Mme de Genlis, {\itshape Mémoires}, chap. IX.)\par
 Arthur Young, visitant un château de Seine-et-Marne, écrit : « J’ai interrogé Mme de Guerchy ; il résulte de cette conversation que pour habiter un château comme celui-ci, avec six domestiques mâles, cinq servantes, huit chevaux, entretenir un jardin, etc., tenir table ouverte, recevoir quelque société, sans jamais aller à Paris, il faut environ 1 000 louis de revenu ; il en faudrait 2 000 en Angleterre ». Aujourd’hui en France, au lieu de 24 000 francs, ce serait 50 000 et davantage. Arthur Young ajoute : « Il y a ici des gentilshommes qui vivent pour 6 000 ou 8 000 livres avec deux domestiques, deux servantes, trois chevaux et un cabriolet ». Aujourd’hui il leur en faudrait 20 000 ou 25 000. — En province surtout, par l’effet des chemins de fer, la vie est devenue beaucoup plus chère. « Selon mes amis du Rouergue, dit-il encore, je pourrais vivre à Milhau avec ma famille dans la plus grande abondance pour 100 louis ; il y a là des familles nobles vivant d’un revenu de 50 et même de 25 louis. » Aujourd’hui à Milhau les prix sont triplés et même quadruplés. — À Paris, telle maison dans la rue Saint-Honoré, louée 6 000 francs en 1789, est louée 16 000 fr. aujourd’hui.
 }. — Ce sont là des vestiges du régime féodal ; on en trouve aujourd’hui de semblables en Angleterre, en Autriche, en Prusse, en Russie ; en effet la propriété survit longtemps aux circonstances qui la fondent. La souveraineté l’avait faite ; séparée de la souveraineté, elle est restée aux mains jadis souveraines. Dans l’évêque, l’abbé ou le comte, le roi a respecté le propriétaire en renversant le rival, et, dans le propriétaire subsistant, cent traits indiquent encore le souverain détruit ou amoindri.

\section[{III. Leurs immunités.}]{III. Leurs immunités.}

\noindent Telle est l’exemption d’impôt totale ou partielle. Les collecteurs s’arrêtent devant aux, parce que le roi sent bien que la propriété féodale a la même origine que la sienne ; si la royauté est un privilège, la seigneurie en est un autre ; le roi n’est lui-même que le plus privilégié des privilégiés. Le plus absolu, le plus infatué de son droit, Louis XIV a eu des scrupules lorsque l’extrême nécessité l’a contraint à mettre sur tous l’impôt du dixième\footnote{{\itshape Rapports de l’agence du clergé de} 1780 à 1785. À propos des droits féodaux dont le livre de Boncerf demandait l’abolition, l’avocat général Séguier disait en 1775 : « Nos rois ont déclaré eux-mêmes qu’ils sont dans l’heureuse impuissance de porter atteinte à la propriété. »}. Des traités, des précédents, une coutume immémoriale, le souvenir du droit antique retiennent encore la main du fisc. Plus le propriétaire ressemble à l’ancien souverain indépendant, plus son immunité est large. — Tantôt il est couvert par un traité récent, par sa qualité d’étranger, par son origine presque royale. « En Alsace, les princes possessionnés étrangers, les ordres de Malte et Teutonique jouissent de l’exemption de toute contribution personnelle et réelle. » — « En Lorraine, le chapitre de Remiremont a le privilège de se cotiser lui-même dans toutes les impositions de l’État\footnote{ \noindent Léonce de Lavergne, {\itshape les Assemblées provinciales}, 296. Rapport de M. Schwendt sur l’Alsace en 1787. — Waroquier, {\itshape État de la France en} 1789, I, 541. — Necker, \href{http://gallica.bnf.fr/ark:/12148/bpt6k429493}{\dotuline{{\itshape De l’administration des finances}}} [\url{http://gallica.bnf.fr/ark:/12148/bpt6k429493}], I, 19, 102. — Turgot (collection des économistes), {\itshape Réponse aux observations du garde des sceaux sur la suppression des corvées}, I, 559.
 }. » Tantôt il a été protégé par le maintien des États provinciaux et par l’incorporation de la noblesse à la terre : en Languedoc et en Bretagne, les biens roturiers payent seuls la taille. — Partout d’ailleurs, sa qualité l’en a préservé, lui, son château et les dépendances de son château ; la taille ne l’atteint que dans ses fermiers. Bien mieux, il suffit qu’il exploite lui-même ou par un régisseur, pour que son indépendance originelle se communique à sa terre ; dès qu’il touche le sol, par lui-même ou par son commis, il en abrite quatre charrues, trois cents arpents, qui, dans les mains d’un autre, payeraient deux mille francs d’impôt, et en outre « les bois, les prairies, les vignes, les étangs, les terres encloses qui tiennent au château, de quelque étendue qu’elles soient ». Par suite, en Limousin et ailleurs, dans les pays dont la principale production est en prairies ou en vignes, il a soin de régir lui-même ou de faire régir une notable portion de son domaine ; il l’affranchit ainsi du collecteur\footnote{Comte de Tocqueville, \href{http://classiques.uqac.ca/classiques/De\_tocqueville\_alexis/ancien\_regime/ancien\_regime.html}{\dotuline{l’{\itshape Ancien Régime et la Révolution}}} [\url{http://classiques.uqac.ca/classiques/De\_tocqueville\_alexis/ancien\_regime/ancien\_regime.html}], 406. « Les habitants de Montbazon avaient porté à la taille les régisseurs du duché que possédait le prince de Rohan. Ce prince fait cesser cet abus et obtient de rentrer dans une somme de 5 344 livres qu’on lui avait fait payer indûment de ce chef. »}. Il y a plus : en Alsace, par convention expresse, il ne paye pas un sou de taille. Ainsi, après quatre cent cinquante ans d’assaut, la taille, ce premier engin du fisc, le plus lourd de tous, a laissé presque intacte la propriété féodale\footnote{ \noindent Necker, {\itshape De l’administration des finances}, la taille rapportait 91 millions, les vingtièmes 76 500 000, la capitation 41 500 000.
 }  Depuis un siècle, deux nouvelles machines, la capitation et les vingtièmes, semblent plus efficaces et ne le sont guère davantage  D’abord, par un chef-d’œuvre de diplomatie ecclésiastique, le clergé a détourné, émoussé leur choc. Comme il fait corps et qu’il a des assemblées, il a pu traiter avec le roi, se racheter, éviter d’être taxé par autrui, se taxer lui-même, faire reconnaître que ses versements ne sont pas une contribution imposée, mais un « don gratuit », obtenir en échange une foule de concessions, modérer ce don, parfois ne pas le faire, en tout cas le réduire à 16 millions tous les cinq ans, c’est-à-dire à un peu plus de 3 millions par an ; en 1788, c’est seulement 1 800 000 livres, et il le refuse pour 1789\footnote{Raudot, {\itshape la France avant la Révolution}, 51. — Marquis de Bouillé, {\itshape Mémoires}, 44. — Necker, {\itshape De l’administration des finances}, II, 181. Il s’agit ici du clergé dit de France (116 diocèses). Le clergé dit étranger était celui des Trois-Évêchés et des pays conquis depuis Louis XIV ; il avait un régime à part et payait à peu près comme les nobles. — Les décimes que le clergé de France levait sur ses biens faisaient une somme de 10 500 000 livres.}. Bien mieux, comme il emprunte pour y fournir, et que les décimes qu’il lève sur ses biens ne suffisent pas pour amortir le capital et servir les intérêts de sa dette, il a eu l’adresse de se faire allouer en outre par le roi et sur le trésor du roi, chaque année, 2 500 000 livres, en sorte qu’au lieu de payer il reçoit ; en 1787, il touche ainsi 1 500 000 livres. — Quant aux nobles, ne pouvant se réunir, avoir des représentants, agir par voie publique, ils ont agi par voie privée, auprès des ministres, des intendants, des subdélégués, des fermiers généraux et de toutes les personnes revêtues d’autorité ; on a pour leur qualité des égards, des ménagements, des complaisances. D’abord cette qualité les exempte, eux, leurs gens et les gens de leurs gens, du tirage à la milice, du logement des gens de guerre, de la corvée pour les routes. Ensuite, la capitation étant fixée d’après la taille, ils payent peu, puisque leur taille est peu de chose. De plus, chacun d’eux a réclamé de tout son crédit contre sa cote : « Votre cœur sensible, écrit l’un d’eux à l’intendant, ne consentira jamais à ce qu’un père de mon état soit taxé à des vingtièmes stricts comme un père du commun\footnote{ \noindent Tocqueville, {\itshape ib..} 104, 381, 407. — Necker, {\itshape ib.}, I, 102. — Boiteau, {\itshape ib..} 362. — Bouillé, {\itshape ib.}, 26, 41, et suivantes. — Turgot, {\itshape ib.}, passim. — Cf. t. II, livre v, ch. 2, sur les impôts du taillable.
 } ». D’autre part, comme le contribuable paye la capitation au lieu de son domicile effectif, souvent fort loin de ses terres, et sans qu’on sache rien de ses revenus mobiliers, il peut ne verser que ce que bon lui semble. Nulle recherche contre lui, s’il est noble ; « on est infiniment circonspect envers les personnes d’un rang distingué » ; en province, dit Turgot, « la capitation des privilégiés s’est successivement réduite à un objet excessivement modique, tandis que la capitation des taillables est presque égale au principal des tailles ». Enfin, « les percepteurs se croient obligés d’observer des ménagements à leur égard », même quand ils doivent ; « ce qui fait, dit Necker, qu’il subsiste sur leur capitation et sur leurs vingtièmes des restes très anciens et beaucoup trop considérables ». Ainsi, n’ayant pu repousser de front l’assaut du fisc, ils l’ont esquivé ou atténué jusqu’à le rendre presque inoffensif. En Champagne, « sur près de 1 500 000 livres fournis par la capitation, ils ne payent que 14 000 livres », c’est-à-dire « 2 sous et 2 deniers pour le même objet qui coûte 12 sous par livre au taillable ». Selon Calonne, « si l’on eût supprimé les concessions et privilèges, les vingtièmes auraient rapporté le double ». À cet égard, les plus opulents étaient les plus habiles à se défendre. « Avec les intendants, disait le duc d’Orléans, je m’arrange ; je paye à peu près ce que je veux », et il calculait que les administrations provinciales, le taxant à la rigueur, allaient lui faire perdre 300 000 livres de rentes. On a vérifié que les princes du sang, pour leurs deux vingtièmes, payaient 188 000 livres, au lieu de 2 400 000  Au fond, dans ce régime, l’exemption d’impôt est un dernier lambeau de souveraineté ou tout au moins d’indépendance. Le privilégié évite ou repousse la taxe, non seulement parce qu’elle le dépouille, mais encore parce qu’elle l’amoindrit ; elle est un signe de roture, c’est-à-dire d’ancienne servitude, et il résiste au fisc autant par orgueil que par intérêt.

\section[{IV. Leurs droits féodaux. — Ces avantages sont des débris de la souveraineté primitive.}]{IV. Leurs droits féodaux. — Ces avantages sont des débris de la souveraineté primitive.}

\noindent Suivons-le chez lui dans son domaine. Un évêque, un abbé, un chapitre, une abbesse a le sien, comme un seigneur laïque ; car jadis le monastère et l’Église ont été de petits États, comme le comté et le duché. — Intacte de l’autre côté du Rhin, presque ruinée en France, la bâtisse féodale laisse partout apercevoir le même plan. En certains endroits mieux abrités ou moins assaillis, elle a gardé tous ses anciens dehors. À Cahors, l’évêque-comte de la ville a le droit, quand il officie solennellement, « de faire mettre sur l’autel le casque, la cuirasse, les gantelets et l’épée\footnote{Voir, pour tous ces détails, {\itshape la France ecclésiastique de} 1788.} ». À Besançon, l’archevêque-prince a six grands officiers qui doivent lui faire hommage de leurs fiefs, assister à son intronisation et à ses obsèques. À Mende\footnote{Procès-verbaux et cahiers manuscrits des États généraux de 1789. {\itshape Archives nationales}, t. LXXXVIII, 23, 85, 121, 122, 152. Procès-verbal du 12 janvier 1789.}, l’évêque, seigneur suzerain du Gévaudan depuis le onzième siècle, choisit les conseils, les juges ordinaires et d’appel, les commissaires et syndics du pays », dispose de toutes les places « municipales et judiciaires », et, prié de venir à l’assemblée des trois ordres de la province, « répond que sa place, ses possessions et son rang le mettant au-dessus de tous les particuliers de son diocèse, il ne peut être présidé par personne, qu’étant seigneur suzerain de toutes les terres et particulièrement des baronnies, il ne peut céder le pas à ses vassaux et arrière-vassaux », bref qu’il est roi ou peu s’en faut dans sa province. À Remiremont, le chapitre noble des chanoinesses a « la basse, haute et moyenne justice dans cinquante-deux bans de seigneuries », présente à soixante-quinze cures, confère dix canonicats mâles, nomme dans la ville les officiers municipaux, outre cela trois tribunaux de première instance et d’appel et partout les officiers de gruerie. Trente-deux évêques, sans compter les chapitres, sont ainsi seigneurs temporels, en tout ou en partie, de leur ville épiscopale, parfois du district environnant, parfois, comme l’évêque de Saint-Claude, de tout le pays. Ici la tour féodale a été préservée  Ailleurs elle est recrépie à neuf, notamment dans les apanages. Dans ces domaines qui comprennent plus de douze de nos départements, les princes du sang nomment aux offices de judicature et aux bénéfices. Substitués au roi, ils ont ses droits utiles et honorifiques. Ce sont presque des rois délégués et à vie ; car ils touchent, non seulement tout ce que le roi toucherait comme seigneur, mais encore une portion de ce qu’il toucherait comme monarque\footnote{Necker, {\itshape De l’administration des finances}, II, 271, 272. « La maison d’Orléans, dit-il, est en possession des aides. » Il évalue cet impôt à 51 millions pour tout le royaume.}. Par exemple la maison d’Orléans perçoit les aides, c’est-à-dire les droits sur les boissons, sur les ouvrages d’or et d’argent, sur la fabrication du fer, sur les aciers, sur les cartes, le papier et l’amidon, bref tout le montant d’un des plus lourds impôts indirects. Rien d’étonnant, si, rapprochés de la condition souveraine, ils ont, comme les souverains, un conseil, un chancelier, une dette constituée, une cour\footnote{Beugnot, \href{http://gallica.bnf.fr/ark:/12148/bpt6k37337f}{\dotuline{{\itshape Mémoires}}} [\url{http://gallica.bnf.fr/ark:/12148/bpt6k37337f}], I, 77. Notez le cérémonial chez le duc de Penthièvre, chap. I, III. Le duc d’Orléans institue un chapitre et des cordons de chanoinesses. La place de chancelier chez le duc d’Orléans vaut 100 000 livres par an ({\itshape Gustave III et la cour de France}, par Geffroy, I, 410).}, un cérémonial domestique, et si l’édifice féodal revêt entre leurs mains le décor luxueux et compassé qu’il a pris aux mains du roi.\par
Venons-en à des personnages moindres, à un seigneur de dignité moyenne, dans sa lieue carrée de pays, au milieu des mille habitants qui jadis ont été ses vilains ou ses serfs, à portée du monastère, du chapitre ou de l’évêque dont les droits s’entremêlent à ses droits. Quoi qu’on ait fait pour l’abaisser, sa place est toujours bien haute. Il est encore, disent les intendants, « le premier habitant » ; c’est un prince qu’ils ont peu à peu dépouillé de ses fonctions publiques et relégué dans ses droits honorifiques et utiles, mais qui demeure prince\footnote{Tocqueville, {\itshape ib.}, 40  Renauldon, avocat au bailliage d’Issoudun, {\itshape Traité historique et pratique des droits seigneuriaux}, 1765, 8, 10, 81 et passim  {\itshape Cahier d’un magistrat du Châtelet} sur les justices seigneuriales, 1789  Duvergier, {\itshape Collection des lois.} Décrets du 15-28 mars 1790 sur l’abolition du régime féodal, Merlin de Douai, rapporteur, t. I, 114 ; Décrets du 19-23 juillet 1790 (I, 293) ; Décrets du 13-20 avril 1791 (I, 295).}  À l’église il a son banc et droit de sépulture dans le chœur ; les tentures portent ses armoiries ; on lui donne l’encens, « l’eau bénite par distinction ». Souvent, ayant fondé l’église, il en est le patron, choisit le curé, prétend le conduire ; dans les campagnes, on le voit avancer ou reculer à sa fantaisie l’heure des messes paroissiales. S’il est titré, il est haut justicier, et il y a des provinces entières, par exemple le Maine et l’Anjou, où il n’y a pas de fief sans justice. En ce cas, il nomme le bailli, le greffier et autres gens de loi et de justice, procureurs, notaires, sergents seigneuriaux, huissiers à verge ou à cheval, qui instrumentent ou jugent en son nom, au civil et au criminel, par première instance. De plus, il institue un gruyer, ou juge des délits forestiers, et perçoit les amendes que cet officier inflige. Pour les délinquants de diverses sortes, il a sa prison, parfois ses fourches patibulaires. D’autre part, en dédommagement de ses frais de justice, il reçoit les biens de l’homme condamné à mort et à la confiscation dans son domaine ; il succède au bâtard né et décédé dans sa seigneurie sans testament ni enfants légitimes ; il hérite du régnicole, enfant légitime, décédé chez lui sans testament ni héritiers apparents ; il s’approprie les choses mobilières, vivantes ou inanimées, qui se trouvent égarées et dont on ignore le propriétaire ; il prélève le tiers ou la moitié des trésors trouvés, et, sur la côte, il prend pour lui les épaves des naufrages ; enfin, ce qui est plus fructueux en ces temps de misère, il devient possesseur des biens abandonnés qu’on a cessé de cultiver depuis dix ans. — D’autres avantages attestent plus clairement encore que jadis il eut le gouvernement du canton. Tels sont, en Auvergne, en Flandre, en Hainaut, dans l’Artois, dans la Picardie, l’Alsace et la Lorraine, les droits de poursoin ou de sauvement qu’on lui paye pour sa protection générale ; ceux de guet et de garde qu’il réclame pour sa protection militaire ; l’afforage qu’il exige de ceux qui vendent de la bière, du vin et autres boissons en gros ou en détail ; le fouage, en argent ou en grains, que, dans plusieurs coutumes, il perçoit sur chaque feu, maison ou famille ; le pulvérage, fort commun en Dauphiné et en Provence, sur les troupeaux de moutons qui passent ; les lods et ventes, droit presque universel, qui est le prélèvement d’un sixième, parfois d’un cinquième ou même d’un quart sur le prix de toute terre vendue et de tout bail qui excède neuf ans ; le droit de rachat ou relief, équivalent à une année de revenu et qu’il reçoit des héritiers collatéraux, parfois des héritiers directs ; enfin un droit plus rare, mais le plus lourd de tous, celui d’acapte ou de plaît-à-merci, qui est un cens double ou une année des fruits, payable aussi bien au décès du seigneur qu’à celui du censitaire. Ce sont là de véritables impôts, fonciers, mobiliers, personnels, de patente, de circulation, de mutation, de succession, établis jadis à condition d’un service public dont aujourd’hui il n’est plus chargé.\par
D’autres redevances sont aussi d’anciens impôts, en échange desquels il s’acquitte encore du service qu’ils défrayent. À la vérité, le roi a supprimé quantité de péages, douze cents en 1724, et on en supprime incessamment ; mais il en reste beaucoup au profit du seigneur, sur des ponts, sur des chemins, sur des bacs, sur les bateaux qui montent ou descendent, à charge pour lui d’entretenir le pont, le chemin, le bac, la route de halage, plusieurs fort lucratifs, tel rapportant quatre-vingt-dix mille livres\footnote{{\itshape Archives nationales}, G, 300 (1787). « M. de Boullongne, seigneur de Montereau (y) a un droit de péage qui consiste en 2 deniers par bœuf, vache, veau ou porc ; 1 par mouton ; 2 par bête chargée ; 1 sou 8 deniers par voiture à 4 roues ; 5 deniers pour celles à 2 roues, et 10 deniers par voiture attelée de 3, 4, 5 chevaux ; en outre, un droit de 10 deniers par coche, bateau ou bachot qui remonte la rivière ; le même droit par couple de chevaux qui remontent les bateaux ; 1 denier par futaille vide qui remonte. » Droits analogues à Varennes au profit de M. le duc du Châtelet, seigneur de Varennes.}. Pareillement, à condition d’entretenir la halle et de fournir gratis les poids et mesures, il prélève un droit sur les denrées et les marchandises apportées à sa foire ou à son marché : à Angoulême le quarante-huitième des grains vendus ; à Combourg, près de Saint-Malo, tant par tête de bétail ; ailleurs, tant sur les vins, les comestibles et le poisson\footnote{{\itshape Archives nationales}, K, 1453, n° 1448 : Lettre du 12 juin 1789 de M. de Meulan ; ce droit sur les grains appartenait alors au comte d’Artois. — Chateaubriand, \href{http://gallica.bnf.fr/ark:/12148/bpt6k1013503}{\dotuline{{\itshape Mémoires}}} [\url{http://gallica.bnf.fr/ark:/12148/bpt6k1013503}], I, 73.}. Ayant jadis bâti le four, le pressoir, le moulin, la boucherie, il oblige les habitants à s’en servir ou à s’y fournir, et il démolit les constructions qui lui feraient concurrence\footnote{Renauldon, {\itshape ib.}, 249, 258. « Il n’y a guère de villes seigneuriales qui n’aient des boucheries banales. Le boucher doit obtenir la permission expresse du seigneur. » — Sur la mouture, le droit était en moyenne de 1/16. En plusieurs provinces, Anjou, Berry, Maine, Bretagne, il y avait un moulin banal à draps ou à écorces.}. Visiblement, ce sont là encore des monopoles et des octrois qui remontent au temps où il avait le pouvoir public.\par
Non seulement il avait alors le pouvoir public, mais il possédait le sol et les hommes. Propriétaire des hommes, il l’est encore, du moins à plusieurs égards et en plusieurs provinces. « Dans la Champagne propre, dans le Sénonais, la Marche, le Bourbonnais, le Nivernais, la Bourgogne, la Franche-Comté, il n’y a point ou très peu de terres où il ne reste des marques de l’ancienne servitude… On y trouve encore quantité de serfs personnels ou constitués tels par leurs reconnaissances ou par celles de leurs auteurs\footnote{ \noindent Renauldon, {\itshape ib.}, 181, 200, 203 ; notez qu’il écrit en 1765. Louis XVI supprima le servage dans ses domaines en 1778 ; et plusieurs seigneurs, en Franche-Comté notamment, suivirent son exemple.\par
 Beugnot, \href{http://gallica.bnf.fr/ark:/12148/bpt6k37337f}{\dotuline{{\itshape Mémoires}}} [\url{http://gallica.bnf.fr/ark:/12148/bpt6k37337f}], I, 142. — Voltaire, {\itshape Mémoire} au {\itshape roi sur les serfs du Jura. —} \href{http://gallica.bnf.fr/ark:/12148/bpt6k204248h}{\dotuline{{\itshape Mémoires}}} [\url{http://gallica.bnf.fr/ark:/12148/bpt6k204248h}] de Bailly, II, 214, d’après le procès-verbal de l’Assemblée nationale du 7 août 1789. Je me suis reporté à ce procès-verbal et au livre de M. Clerget, curé d’Ornans en Franche-Comté, qui s’y trouve mentionné. M. Clerget y dit, en effet, qu’il y a encore en ce moment (1789) 1 500 000 sujets du roi soumis à la servitude, mais il n’apporte aucune preuve à l’appui de ce chiffre. Néanmoins, il est certain que le nombre des serfs et mainmortables est encore très grand. {\itshape Archives nationales}, H, 723, mémoires sur les mainmortables de la Franche-Comté en 1788 ; II, 200, mémoires par M. Amelot sur la Bourgogne en 1785. « Dans la subdélégation de Charolles, les habitants semblent à un siècle du temps actuel ; soumis aux droits féodaux, tels que la mainmorte, leur esprit et leur corps ne peuvent prendre aucun essor. Le rachat de la mainmorte, dont le roi a lui-même donné l’exemple, a été mis à un prix si exorbitant par les laïques, que les malheureux mainmortables ne peuvent ni ne pourront y atteindre. »
 }. » Là, l’homme est serf tantôt par le fait de sa naissance, tantôt par le fait de la terre. Mortaillables, mainmortables, bordeliers, d’une façon ou d’une autre, quinze cent mille personnes, dit-on, ont au col un morceau du collier féodal ; rien d’étonnant, puisque de l’autre côté du Rhin presque tous les paysans le portent encore. Maître et propriétaire autrefois de tout leur bien et de tout leur travail, le seigneur peut encore exiger d’eux dix à douze corvées par an et une taille fixe annuelle. Dans la baronnie de Choiseul, près de Chaumont en Champagne, « les habitants sont tenus de labourer ses terres, de les semer, de les moissonner pour son compte, d’en amener le produit dans ses granges ; chaque pièce de terre, chaque maison, chaque tête de bétail lui paye une redevance ; les enfants ne succèdent aux parents qu’à condition de demeurer avec eux ; s’ils sont absents à l’époque du décès, c’est lui qui hérite ». Voilà ce qu’en langage du temps on appelait une terre ayant « de beaux droits ». — Ailleurs le seigneur hérite des collatéraux, frères ou neveux, s’ils n’étaient pas en communauté avec le défunt au moment de sa mort, et cette communauté n’est valable que par sa permission. Dans le Jura et le Nivernais, il peut poursuivre les serfs qui se sont enfuis, et réclamer à leur mort, non seulement le bien qu’ils ont laissé chez lui, mais encore le pécule qu’ils ont acquis ailleurs. À Saint-Claude, il acquiert ce droit sur quiconque a passé un an et un jour dans une maison de la seigneurie. — Quant à la propriété du sol, on voit plus nettement encore que jadis il l’avait tout entière. Dans le district soumis à sa juridiction, le domaine public demeure son domaine privé ; les chemins, rues et places publiques en font partie ; il a le droit d’y planter des arbres et de revendiquer les arbres qui s’y trouvent. En plusieurs provinces, par le droit de blairie, il fait payer aux habitants la permission de paître leurs bestiaux dans les champs après la récolte, et dans les « terres vaines et vagues ». Les rivières non navigables sont à lui, ainsi que les îlots et atterrissements qui s’y forment et le poisson qui s’y rencontre. Il a droit de chasse dans toute l’étendue de sa juridiction et l’on a vu tel roturier obligé de lui ouvrir son parc enclos de murs.\par
Encore un trait pour achever de le peindre. Ce chef d’État, propriétaire des hommes et du sol, était jadis un cultivateur résidant sur sa métairie propre au milieu d’autres métairies sujettes, et, à ce titre, il se réservait des avantages d’exploitation dont il a conservé plusieurs. Tel est le droit de banvin, encore très répandu, et qui est le privilège pour lui de vendre son vin, à l’exclusion de tout autre, pendant les trente ou quarante jours qui suivent la récolte. Tel est, en Touraine, le droit de préage, c’est-à-dire la faculté pour lui d’envoyer ses chevaux, vaches et bœufs « paître à garde faite dans les prés de ses sujets ». Tel est enfin le monopole du grand colombier à pied, d’où ses pigeons par milliers vont pâturer en tout temps et sur toutes les terres, sans que personne puisse les tuer ni les prendre. — Par une autre suite de la même qualité, il perçoit des redevances sur tous les biens que jadis il a donnés à bail perpétuel, et, sous les noms de cens, censives, carpot, champart, agrier, terrage, parcière, ces perceptions en argent ou en nature sont aussi diverses que les situations, les accidents, les transactions locales ont pu l’être. Dans le Bourbonnais, il a le quart de la récolte ; dans le Berry, douze gerbes sur cent. Parfois son débiteur ou locataire est une communauté : un député à l’Assemblée nationale avait un fief de deux cents pièces de vin sur trois mille propriétés particulières\footnote{ Boiteau, {\itshape ib..} 25 (avril 1790). — Beugnot, \href{http://gallica.bnf.fr/ark:/12148/bpt6k37337f}{\dotuline{{\itshape Mémoires}}} [\url{http://gallica.bnf.fr/ark:/12148/bpt6k37337f}], I, 142.
 }. Ailleurs, par le retrait censuel, il peut « garder pour son compte toute propriété vendue, à charge de rembourser l’acquéreur, mais en prélevant à son profit le droit des lods et ventes ». — Remarquez enfin que tous ces assujettissements de la propriété forment, pour le seigneur, une créance privilégiée tant sur les fruits que sur le prix du fonds, et, pour les censitaires, une dette imprescriptible, indivisible, irrachetable. — Voilà les droits féodaux : pour nous les représenter par une vue d’ensemble, figurons-nous toujours le comte, l’évêque ou l’abbé du dixième siècle, souverain et propriétaire de son canton. La forme dans laquelle s’enserre alors la société humaine est construite sous les exigences du danger incessant et proche, en vue de la défense locale, par la subordination de tous les intérêts au besoin de vivre, de façon à sauvegarder le sol en attachant au sol, par la propriété et la jouissance, une troupe de braves sous un brave chef. Le péril s’est évanoui, la construction s’est délabrée. Moyennant argent, les seigneurs ont permis au paysan économe et tenace d’en arracher beaucoup de pierres. Par contrainte, ils ont souffert que le roi s’en appropriât la portion publique. Reste l’assise primitive, la structure ancienne de la propriété, la terre enchaînée ou épuisée pour le maintien d’un moule social qui s’est dissous, bref un ordre de privilèges et de sujétions dont la cause et l’objet ont disparu\footnote{Voir la note 2 [’p. 300’].}.

\section[{V. Ils peuvent être justifiés par des services locaux et généraux.}]{V. Ils peuvent être justifiés par des services locaux et généraux.}

\noindent Cela ne suffit par pour que cet ordre soit nuisible ou même inutile. En effet, le chef local qui ne remplit plus son ancien office peut remplir en échange un office nouveau. Institué pour la guerre quand la vie était militante, il peut servir dans la paix quand le régime est pacifique, et l’avantage est grand pour la nation en qui cette transformation s’accomplit ; car, gardant ses chefs, elle est dispensée de l’opération incertaine et redoutable qui consiste à s’en créer d’autres. Rien de plus difficile à fonder que le gouvernement, j’entends le gouvernement stable : il consiste dans le commandement de quelques-uns et dans l’obéissance de tous, chose contre nature. Qu’un homme dans son cabinet, parfois un vieillard débile, dispose des biens et des vies de vingt ou trente millions d’hommes dont la plupart ne l’ont jamais vu ; qu’il leur dise de verser le dixième ou le cinquième de leur revenu et qu’ils le versent ; qu’il leur ordonne d’aller tuer ou se faire tuer et qu’ils y aillent ; qu’ils continuent ainsi pendant dix ans, vingt ans, à travers toutes les épreuves, défaites, misères, invasions, comme les Français sous Louis XIV, les Anglais sous M. Pitt, les Prussiens sous Frédéric II, sans séditions ni troubles intérieurs : voilà certes une merveille, et, pour qu’un peuple demeure indépendant, il faut que tous les jours il soit prêt à la faire. Ni cette fidélité, ni cette concorde ne sont les fruits de la raison raisonnante ; elle est trop vacillante et trop faible pour produire un effet si universel et si énergique. Livré à lui-même et ramené subitement à l’état de nature, le troupeau humain ne saura que s’agiter, s’entre-choquer, jusqu’à ce qu’enfin la force pure prenne le dessus comme aux temps barbares, et que, parmi la poussière et les cris, surgisse un conducteur militaire, lequel d’ordinaire est un boucher. En fait d’histoire, il vaut mieux continuer que recommencer. — C’est pourquoi, surtout quand la majorité est inculte, il est utile que les chefs soient désignés d’avance par l’habitude héréditaire qu’on a de les suivre, et par l’éducation spéciale qui les a préparés. En ce cas le public n’a pas besoin de les chercher pour les trouver. Ils sont là, dans chaque canton, visibles, acceptés d’avance ; on les reconnaît à leur nom, à leur titre, à leur fortune, à leur genre de vie, et la déférence est toute prête pour leur autorité. Cette autorité, le plus souvent ils la méritent ; nés et élevés pour l’exercer, ils trouvent dans la tradition, dans l’exemple et dans l’orgueil de famille des cordiaux puissants qui nourrissent en eux l’esprit public ; il y a chance pour qu’ils comprennent les devoirs dont leur prérogative les charge. — Tel est le renouvellement que comporte le régime féodal. L’ancien chef peut encore autoriser sa prééminence par ses services, et rester populaire sans cesser d’être privilégié. Jadis capitaine du district et gendarme en permanence, il doit devenir propriétaire résidant et bienfaisant, promoteur volontaire de toutes les entreprises utiles, tuteur obligé des pauvres, administrateur et juge gratuit du canton, député sans traitement auprès du roi, c’est-à-dire conducteur et protecteur comme autrefois, par un patronage nouveau accommodé aux circonstances nouvelles. Magistrat local, représentant au centre, voilà ses deux fonctions principales, et, si l’on regarde au-delà de la France, on découvre qu’il remplit l’une ou l’autre, ou toutes les deux.
\chapterclose


\chapteropen

\chapter[{Chapitre III. Services locaux que doivent les privilégiés.}]{Chapitre III. \\
Services locaux que doivent les privilégiés.}


\chaptercont

\section[{I. Exemples en Allemagne et en Angleterre. — Les privilégiés ne rendent pas ces services en France.}]{I. Exemples en Allemagne et en Angleterre. — Les privilégiés ne rendent pas ces services en France.}

\noindent Considérons la première, le gouvernement local. À la porte de la France, il y a des contrées où la sujétion féodale, plus pesante qu’en France, semble plus légère, parce que, dans l’autre plateau de la balance, les bienfaits contrepèsent les charges. À Munster en 1809, Beugnot trouve un évêque souverain, une ville de couvents et de grands hôtels seigneuriaux, quelques marchands pour les objets indispensables, peu de bourgeoisie, alentour tous les paysans colons ou serfs. Le seigneur prélève une part de tous leurs produits, denrées ou bestiaux, et à leur mort une portion de leur héritage ; s’ils s’en vont, leur bien lui revient. Ses domestiques sont châtiés comme des moujiks, et, dans chaque remise, il y a un chevalet à cet usage, « sans préjudice de peines plus graves », probablement la bastonnade et le reste. Mais « jamais il n’est venu au condamné la moindre idée de réclamation ni d’appel ». Car, si le seigneur les frappe en père de famille, il les protège « en père de famille, il accourt quand il y a un malheur à réparer, il les soigne dans leurs maladies », il leur fournit un asile dans leur vieillesse ; il pourvoit leurs veuves et se réjouit quand ils ont beaucoup d’enfants ; il est en communauté de sympathies avec eux ; ils ne sont ni misérables ni inquiets ; ils savent que, dans tous leurs besoins extrêmes ou imprévus, il sera leur refuge\footnote{Beugnot, \href{http://gallica.bnf.fr/ark:/12148/bpt6k37337f/f295.table}{\dotuline{{\itshape Mémoires}, I, 292}} [\url{http://gallica.bnf.fr/ark:/12148/bpt6k37337f/f295.table}]. — Tocqueville, \href{http://classiques.uqac.ca/classiques/De\_tocqueville\_alexis/ancien\_regime/ancien\_regime.html}{\dotuline{{\itshape l’} {\itshape Ancien Régime et la Révolution}}} [\url{http://classiques.uqac.ca/classiques/De\_tocqueville\_alexis/ancien\_regime/ancien\_regime.html}], 34, 60.}  Dans les États prussiens, et d’après le code du grand Frédéric, une servitude plus dure encore est compensée par des obligations égales. Sans la permission du seigneur, les paysans ne peuvent aliéner leur champ, l’hypothéquer, le cultiver autrement, changer de métier, se marier. S’ils quittent la seigneurie, il peut les poursuivre en tout lieu et les ramener de force. Il a droit de surveillance sur leur vie privée et les châtie s’ils sont ivrognes ou paresseux. Adolescents, ils sont pendant plusieurs années domestiques dans son manoir ; cultivateurs, ils lui doivent des corvées, en certains lieux trois par semaine. Mais, de par l’usage et la loi, il doit « veiller à ce qu’ils reçoivent l’éducation, les secourir dans l’indigence, leur procurer, autant que possible, les moyens de vivre ». Il a donc les charges du gouvernement dont il a les profits, et, sous la lourde main qui les courbe, mais qui les soutient, on ne voit pas que les sujets regimbent  En Angleterre, la haute classe arrive au même effet par d’autres voies. Là aussi la terre paye encore la dîme ecclésiastique, le dixième strict, bien plus qu’en France\footnote{Arthur Young, \href{http://gallica.bnf.fr/ark:/12148/bpt6k102003v/f474}{\dotuline{{\itshape Voyages} {\itshape en France}, II, 456}} [\url{http://gallica.bnf.fr/ark:/12148/bpt6k102003v/f474}]. En France, dit-il, elle est du onzième au trente-deuxième. « Mais on ne connaît rien de tel que les énormités commises en Angleterre, où l’on prend réellement le dixième. »} ; le squire, le nobleman possède une part du sol encore plus large que celle de son voisin français, et, de fait, exerce sur son canton une autorité plus grande. Mais ses tenanciers, locataires et fermiers ne sont plus ses serfs ni même ses vassaux ; ils sont libres. S’il gouverne, c’est par influence, non par commandement. Propriétaire et patron, on a de la déférence pour lui ; lord-lieutenant, officier de la milice, administrateur, {\itshape justice}, il est visiblement utile. Surtout, de père en fils, il réside, il est du canton, en communication héréditaire et incessante avec le public local, par ses affaires et par ses plaisirs, par la chasse et par le bureau des pauvres, par ses fermiers qu’il admet à sa table, par ses voisins qu’il rencontre au comité ou à la {\itshape vestry.} Voilà comment les vieilles hiérarchies se maintiennent : il faut et il suffit qu’elles changent en cadre civil leur cadre militaire, et trouvent un emploi moderne au chef féodal.

\section[{II. Seigneurs qui résident. — Restes du bon esprit féodal. — Ils ne sont point durs avec leurs tenanciers, mais ils n’ont plus le gouvernement local. — Leur isolement. — Petitesse ou médiocrité de leur aisance. — Leurs dépenses. — Ils ne sont pas en état de remettre les redevances. — Sentiments des paysans à leur endroit.}]{II. Seigneurs qui résident. — Restes du bon esprit féodal. — Ils ne sont point durs avec leurs tenanciers, mais ils n’ont plus le gouvernement local. — Leur isolement. — Petitesse ou médiocrité de leur aisance. — Leurs dépenses. — Ils ne sont pas en état de remettre les redevances. — Sentiments des paysans à leur endroit.}

\noindent Lorsqu’on remonte un peu plus haut dans notre histoire, on y rencontre çà et là de pareils nobles\footnote{Saint-Simon, \href{http://gallica.bnf.fr/ark:/12148/bpt6k70363}{\dotuline{{\itshape Mémoires}}} [\url{http://gallica.bnf.fr/ark:/12148/bpt6k70363}], éd. Chéruel, t. I. — Lucas de Montigny, {\itshape Mémoires de Mirabeau}, t. I, de 53 à 182. — Le maréchal Marmont, \href{http://gallica.bnf.fr/ark:/12148/bpt6k284352}{\dotuline{{\itshape Mémoires}}} [\url{http://gallica.bnf.fr/ark:/12148/bpt6k284352}], I, 9, 11. — Chateaubriand, {\itshape Mémoires}, I, 17. — Comte de Montlosier, {\itshape Mémoires}, 2 vol. passim. — Mme de la \href{http://gallica.bnf.fr/ark:/12148/bpt6k292274}{\dotuline{Rochejaquelein}} [\url{http://gallica.bnf.fr/ark:/12148/bpt6k292274}], {\itshape Souvenirs}, passim. On trouvera dans ces passages des détails sur les types énergiques de l’ancienne noblesse. — Ils sont peints avec force et justesse dans deux romans de Balzac : \href{http://gallica.bnf.fr/ark:/12148/bpt6k1012740}{\dotuline{{\itshape Béatrix}}} [\url{http://gallica.bnf.fr/ark:/12148/bpt6k1012740}] (le baron de Guénic) et \href{http://gallica.bnf.fr/ark:/12148/bpt6k101421r}{\dotuline{{\itshape le Cabinet des antiques}}} [\url{http://gallica.bnf.fr/ark:/12148/bpt6k101421r}] (le marquis d’Esgrignon).}. Tel était le duc de Saint-Simon, père de l’écrivain, vrai souverain dans son gouvernement de Blaye, respecté du roi lui-même. Tel fut le grand-père de Mirabeau, dans son château de Mirabeau en Provence, le plus hautain, le plus absolu, le plus intraitable des hommes, « exigeant que les officiers qu’il présente pour son régiment soient agréés du roi et des ministres », ne souffrant les inspecteurs de revue que pour la forme, mais héroïque, généreux, dévoué, distribuant la pension qu’on lui offre à six capitaines blessés sous ses ordres, s’entremettant pour les pauvres plaideurs de la montagne, chassant de sa terre les procureurs ambulants qui viennent y apporter leur chicane, « protecteur naturel des hommes », jusque contre les ministres et contre le roi. Des gardes du tabac ayant fait une descente chez son curé, il les poursuivit à cheval si rudement qu’ils se sauvèrent à grand’peine en guéant la Durance, et là-dessus « il écrivit pour demander la révocation de tous les chefs, assurant que sans cela tous les employés des aides iraient dans le Rhône ou dans la mer ; il y en eut de révoqués, et le directeur du tripot vint lui-même lui faire satisfaction ». Voyant son canton stérile et ses colons paresseux, il les enrégimente, hommes, femmes, enfants, et, par les plus mauvais temps, lui-même à leur tête, avec ses vingt-sept blessures, le col soutenu par une pièce d’argent, il les fait travailler en les payant, défricher des terres qu’il leur donne à bail pour cent ans, enclore d’énormes murs et planter d’oliviers une montagne de roches. « Nul n’eût pu, sous aucun prétexte, se dispenser de travailler qu’il ne fût malade, et en ce cas secouru, ou occupé à travailler sur son propre bien, article sur lequel mon père ne se laissait pas tromper, et nul ne l’eût osé. » Ce sont là les derniers troncs de la vieille souche, noueux, sauvages, mais capables de fournir des abris. On en trouverait encore quelques-uns dans les cantons reculés, en Bretagne, en Auvergne, vrais commandants de district, et je suis sûr qu’au besoin leurs paysans les suivront autant par respect que par crainte. La force du cœur et du corps donne l’ascendant qu’elle justifie, et la surabondance de sève, qui commence par des violences, finit par des bienfaits.\par
Moins indépendant et moins âpre, le gouvernement paternel subsiste ailleurs, sinon dans la loi, du moins dans les mœurs. En Bretagne, près de Tréguier et de Lannion, dit le bailli de Mirabeau\footnote{Lettre du bailli de Mirabeau, 1760, publiée par M. de Loménie dans {\itshape le Correspondant}, t. XLIX, 132.}, « tout l’état-major de la garde-côte est composé de gens de qualité et de races de mille ans. Je n’en ai pas encore vu un s’échauffer contre un soldat-paysan, et j’ai vu en même temps un air de respect filial de la part de ces derniers… C’est le paradis terrestre pour les mœurs, la simplicité, la vraie grandeur patriarcale : des paysans dont l’attitude devant les seigneurs est celle d’un fils tendre devant son père, des seigneurs qui ne parlent à ces paysans dans leur langage grossier et rude que d’un air bon et riant ; on voit un amour réciproque entre les maîtres et les serviteurs »  Plus au sud, dans le Bocage, pays tout agricole et sans routes, où les dames voyagent à cheval et dans des voitures à bœufs, où le seigneur n’a pas de fermiers, mais vingt-cinq à trente petits métayers avec lesquels il partage, la primauté des grands ne fait point de peine aux petits. On vit bien ensemble, quand on vit ensemble depuis la naissance jusqu’à la mort, familièrement, avec les mêmes intérêts, les mêmes occupations et les mêmes plaisirs : tels des soldats avec leurs officiers, en campagne, sous la tente, subordonnés quoique camarades, sans que la familiarité nuise au respect. « Le seigneur les visite souvent dans leurs métairies, cause avec eux de leurs affaires, du soin de leur bétail, prend part à des accidents et à des malheurs qui lui portent aussi préjudice. Il va aux noces de leurs enfans et boit avec les convives. Le dimanche on danse dans la cour du château, et les dames se mettent de la partie\footnote{Mme de la Rochejaquelein, {\itshape ib.}, I, 84. « Comme M. de Marigny avait quelques connaissances de l’art vétérinaire, les paysans du canton venaient le chercher quand ils avaient des bestiaux malades. »}. » Quand il chasse le loup et le sanglier, le curé en fait l’annonce au prône ; les paysans avec leur fusil viennent joyeusement au rendez-vous, trouvent le seigneur qui les poste, observent strictement la consigne qu’il leur donne : voilà des soldats et un capitaine tout préparés. Un peu plus tard et d’eux-mêmes, ils vont le choisir pour commandant de la garde nationale, pour maire de la commune, pour chef de l’insurrection, et, en 1792, les tireurs de la paroisse marcheront sous lui contre les bleus, comme aujourd’hui contre le loup. — Tels sont les derniers restes du bon esprit féodal, semblables aux sommets épars d’un continent submergé. Avant Louis XIV, le spectacle était pareil dans toute la France. « La noblesse campagnarde d’autrefois, dit le marquis de Mirabeau, buvait trop longtemps, dormait sur de vieux fauteuils ou grabats, montait à cheval, allait à la chasse de grand matin, se rassemblait à la Saint-Hubert et ne se quittait qu’après l’octave de la Saint-Martin… Cette noblesse menait une vie gaie et dure, volontairement, coûtait peu de chose à l’État, et lui produisait plus par sa résidence et son fumier que nous ne lui valons aujourd’hui par notre goût, nos recherches, nos coliques et nos vapeurs… On sait à quel point était l’habitude, et, pour ainsi dire, la manie des présents continuels que les habitants faisaient à leurs seigneurs. J’ai vu de mon temps cette habitude cesser partout et à bon droit… Les seigneurs ne leur sont plus bons à rien ; il est tout simple qu’ils en soient oubliés comme ils les oublient… Personne ne connaissant plus le seigneur dans ses terres, tout le monde le pille, et c’est bien fait\footnote{Marquis de Mirabeau, \href{http://gallica.bnf.fr/ark:/12148/bpt6k89089c}{\dotuline{{\itshape Traité} {\itshape de la} {\itshape population}}} [\url{http://gallica.bnf.fr/ark:/12148/bpt6k89089c}], 57.[‘p. 217’, etc..]}. » Partout, sauf en des coins écartés, l’affection, l’union des deux classes a disparu ; le berger s’est séparé du troupeau, et les pasteurs du peuple ont fini par être considérés comme ses parasites.\par
Suivons-les d’abord en province. On n’y voit que la petite noblesse et une partie de la moyenne ; le reste est à Paris\footnote{Tocqueville, {\itshape ib.}, 180. Ceci est prouvé par les registres de la capitation, qu’on payait au domicile réel.}. Même partage dans l’Église : les abbés commendataires, les évêques et archevêques ne résident guère ; les grands vicaires et chanoines sont dans les grandes villes ; il n’y a que les prieurs et les curés dans les campagnes ; à l’ordinaire, tout l’état-major ecclésiastique ou laïque est absent ; les résidents ne sont fournis que par les grades secondaires ou inférieurs. — Comment ceux-ci vivent-ils avec le paysan ? Un point est sûr, c’est que le plus souvent ils ne sont pour lui ni durs ni même indifférents. Séparés par le rang, ils ne le sont point par la distance ; or le voisinage est à lui seul un lien entre les hommes. J’ai eu beau lire, je n’ai point trouvé en eux les tyrans ruraux que dépeignent les déclamateurs de la Révolution. Hautains avec le bourgeois, ils sont ordinairement bons avec le villageois. « Qu’on parcoure dans les provinces, dit un avocat contemporain, les terres habitées par les seigneurs ; entre cent, on en trouvera peut-être une ou deux où ils tyrannisent leurs sujets ; tous les autres y partagent patiemment la misère de leurs justiciables… Ils attendent les débiteurs, leur font des remises, leur procurent toute facilité pour payer. Ils adoucissent, ils tempèrent les poursuites parfois trop rigoureuses des fermiers, des régisseurs, des gens d’affaires\footnote{Renauldon, {\itshape ib.}, préface, 5. — Anne Plumptree, {\itshape a Narrative of three years’ residence in France from} 1802 {\itshape to} 1805, II, 357. — Baronne d’Oberkirch, \href{http://gallica.bnf.fr/ark:/12148/CadresFenetre?O=NUMM-204843\&M=tdm}{\dotuline{{\itshape Mémoires}}} [\url{http://gallica.bnf.fr/ark:/12148/CadresFenetre?O=NUMM-204843\&M=tdm}], II, 389. — {\itshape De l’état religieux}, par les abbés de Bonnefoi et Bernard (1784), 295. — Mme Vigée-Lebrun, {\itshape Souvenirs}, 171.}. » — Une Anglaise qui les voit en Provence au sortir de la Révolution dit que, détestés à Aix, ils sont très aimés sur leurs terres. « Tandis que devant les premiers bourgeois ils passent la tête haute, avec un air de dédain, ils saluent les paysans avec une courtoisie et une affabilité extrêmes. » Un d’eux distribue aux femmes, enfants, vieillards de son domaine de la laine et du chanvre pour filer pendant la mauvaise saison, et, à la fin de l’année, il donne un prix de cent livres aux deux meilleures pièces de toile. En nombre de cas, les paysans acquéreurs leur ont volontairement restitué leurs terres au prix d’achat  Autour de Paris, près de Romainville, après le terrible orage de 1788, on prodigue les aumônes ; « un homme fort riche distribue aussitôt pour son compte quarante mille francs aux malheureux qui l’entourent » ; pendant l’hiver, en Alsace, à Paris, tout le monde donne ; « devant chaque hôtel d’une famille connue brûle un vaste bûcher, où nuit et jour les pauvres viennent se chauffer »  En fait de charité, les moines qui résident et sont témoins de la misère publique restent fidèles à l’esprit de leur institut. À la naissance du Dauphin, les Augustins de Montmorillon en Poitou ont payé de leurs deniers les tailles et corvées de dix-neuf pauvres familles. En 1781, en Provence, les Dominicains de Saint-Maximin ont nourri leur district où l’ouragan avait détruit les vignes et les oliviers. « Les Chartreux de Paris donnent aux pauvres 1 800 livres de pain par semaine. Pendant l’hiver de 1784, les aumônes sont augmentées dans toutes les maisons religieuses : leurs fermiers distribuent des secours aux habitants pauvres des campagnes, et, pour fournir à ces besoins extraordinaires, plusieurs communautés ajoutent à la rigueur de leurs abstinences. » — Quand, à la fin de 1789, il s’agit de les supprimer, je rencontre en leur faveur nombre de réclamations écrites par des officiers municipaux, par les notables, par une foule d’habitants, artisans, paysans, et ces colonnes de signatures rustiques sont vraiment éloquentes. Sept cents familles de Cateau-Cambrésis\footnote{{\itshape Archives nationales}, D, XIX, cartons 14, 15, 25. Cinq dossiers sont remplis de ces pétitions.} dressent une supplique pour garder les dignes abbés et religieux de l’abbaye de Saint-André, leurs pères communs et bienfaiteurs, qui les ont nourris pendant la grêle ». Les habitants de Saint-Savin, dans les Pyrénées, « peignent avec des larmes de douleur leur consternation » à l’idée qu’on va supprimer leur abbaye de Bénédictins, seule fondation de charité dans ce pays pauvre. À Sierk, près de Thionville, « la Chartreuse, disent les notables, est à tous égards pour nous l’arche du Seigneur ; c’est la principale ressource de plus de douze à quinze cents personnes qui viennent tous les jours de la semaine. Cette année les moines leur ont distribué leur propre provision de grain à 16 livres au-dessous du cours ». Les chanoines réguliers de Domièvre en Lorraine nourrissent soixante pauvres deux fois par semaine ; il faut les conserver, dit la supplique, « par pitié et compassion pour le pauvre peuple dont la misère est au-dessus de l’imagination ; où il n’y a pas de couvents réguliers et de chanoines de leur dépendance, les pauvres crient misère\footnote{ \noindent {\itshape Ibid.}, D, XIX, carton 11. Très belle lettre de Joseph de Saintignon, abbé de Domèvre, général des chanoines réguliers de Saint-Sauveur et résident. Il a 23 000 livres de rente, dont 6 066 livres de pension donnée par le gouvernement en récompense de ses services. Sa dépense personnelle n’étant que de 5 000 livres, « il a été en état de verser entre les pauvres et les ouvriers, dans l’espace de onze ans, plus de 250 000 livres ».
 } ». À Moutiers-Saint-Jean, près de Semur en Bourgogne, les Bénédictins de Saint-Maur font vivre tout le village et l’ont nourri cette année dans la disette. Près de Morley en Barrois, l’abbaye d’Auvey, ordre de Cîteaux, « a toujours été, pour tous les villages qui l’avoisinent, un bureau de charité ». À Airvault, dans le Poitou, les officiers municipaux, le colonel de la garde nationale, quantité de « manants et habitants », demandent à conserver les chanoines réguliers de Saint-Augustin. « Leur existence, dit la pétition, est absolument essentielle tant pour notre ville que pour les campagnes, et nous ferions une perte irréparable par leur suppression. » La municipalité et le conseil permanent de Soissons écrivent que la maison de Saint-Jean-des-Vignes « a toujours réclamé avec empressement sa part dans les charges publiques. C’est elle qui, dans les calamités, recueille les citoyens sans asile et leur fournit la subsistance. C’est elle qui a porté seule la charge de l’assemblée du bailliage, lors de l’élection des députés à l’Assemblée nationale. C’est elle qui loge actuellement une compagnie du régiment d’Armagnac. C’est elle qu’on trouve partout, lorsqu’il y a des sacrifices à faire ». — En vingt endroits, on déclare que les religieux sont « les pères des pauvres ». Dans le diocèse d’Auxerre, pendant l’été de 1789, les Bernardins de Rigny « se sont dépouillés, en faveur des habitants des villages voisins, de tout ce qu’ils possédaient : pain, grains, argent et autres secours, tout a été prodigué envers douze cents personnes qui, pendant plus de six semaines, n’ont cessé de venir se présenter chaque jour à leur porte… Emprunts, avances prises sur les fermiers, crédit chez les fournisseurs de la maison, tout a concouru à leur faciliter les moyens de soulager le peuple ». — J’omets beaucoup d’autres traits aussi forts ; on voit que les seigneurs ecclésiastiques ou laïques ne sont point de simples égoïstes quand ils résident. L’homme compatit aux maux dont il est le témoin ; il faut l’absence pour en émousser la vive impression ; le cœur en est touché quand l’œil les contemple. D’ailleurs la familiarité engendre la sympathie ; on ne peut guère rester froid devant l’angoisse d’un pauvre homme, à qui, depuis vingt ans, l’on dit bonjour en passant, dont on sait la vie, qui n’est pas pour l’imagination une unité abstraite, un chiffre de statistique, mais une âme en peine et un corps souffrant. — D’autant plus que, depuis les écrits de Rousseau et des économistes, un souffle d’humanité chaque jour plus fort, plus pénétrant, plus universel, est venu attendrir les cœurs. Désormais on pense aux pauvres, et l’on se fait honneur d’y penser. Il suffit de lire les cahiers des États généraux\footnote{Sur la conduite et sur les sentiments des seigneurs ecclésiastiques et laïques, cf. Léonce de Lavergne, {\itshape les Assemblées provinciales}, 1 vol  Legrand, {\itshape l’Intendance du Hainaut}, 1 vol  Hippeau, {\itshape le Gouvernement de Normandie}, 9 vol.} pour voir que, de Paris, l’esprit philanthropique s’est répandu jusque dans les châteaux et les abbayes de province. Je suis persuadé que, sauf des hobereaux écartés, chasseurs et buveurs, emportés par le besoin d’exercice corporel et confinés par leur rusticité dans la vie animale, la plupart des seigneurs résidents ressemblaient, d’intention ou de fait, aux gentilshommes que, dans ses contes moraux, Marmontel mettait alors en scène ; car la mode les poussait de ce côté, et toujours en France on suit la mode. Leur caractère n’a rien de féodal ; ce sont des gens « sensibles », doux, très polis, assez lettrés, amateurs de phrases générales, et qui s’émeuvent aisément, vivement, volontiers, comme cet aimable raisonneur le marquis de Ferrières, ancien chevau-léger, député de Saumur à l’Assemblée nationale, auteur d’un écrit sur le {\itshape Théisme}, d’un roman moral, de mémoires bienveillants et sans grande portée ; rien de plus éloigné de l’ancien tempérament âpre et despotique. Ils voudraient bien soulager le peuple, et chez eux ils l’épargnent autant qu’ils peuvent\footnote{« La pitié la plus active remplissait les âmes : ce que craignaient le plus les hommes opulents, c’était de passer pour insensibles. » (Lacretelle, {\itshape Histoire de France au XVIII}\textsuperscript{e} {\itshape siècle}, V, 2.)}. On les trouve nuisibles sans qu’ils soient méchants ; le mal vient de leur situation, non de leur caractère. En effet, c’est leur situation qui, leur laissant les droits sans les services, leur interdit les offices publics, l’influence utile, le patronage effectif par lesquels ils pourraient justifier leurs avantages et s’attacher leurs paysans.\par
Mais sur ce terrain le gouvernement central a pris leur place. Depuis longtemps, ils sont bien faibles contre l’intendant, bien impuissants à protéger leur paroisse. Vingt gentilshommes ne peuvent se réunir et délibérer sans une permission expresse du roi\footnote{Floquet, \href{http://gallica.bnf.fr/ark:/12148/bpt6k945663/f701}{\dotuline{{\itshape Histoire du Parlement de Normandie}, VI, 696}} [\url{http://gallica.bnf.fr/ark:/12148/bpt6k945663/f701}]. En 1772, vingt-cinq gentilshommes sont emprisonnés ou exilés pour avoir signé une protestation contre les ordres de la cour.}. Si ceux de Franche-Comté viennent une fois l’an dîner ensemble et entendre une messe, c’est par tolérance, et encore cette innocente confrérie ne doit s’assembler qu’en présence de l’intendant. — Séparé de ses égaux, le seigneur est encore séparé de ses inférieurs. L’administration du village ne le regarde pas, il n’en a pas même la surveillance : répartir l’impôt et le contingent de la milice, réparer l’église, rassembler et présider l’assemblée de la paroisse, faire des routes, établir des ateliers de charité, tout cela est l’affaire de l’intendant ou des officiers communaux que l’intendant nomme ou dirige\footnote{Tocqueville, {\itshape ib.}, 19, 39, 56, 75, 184. Il a développé ce point avec une force et une profondeur admirables.}. Sauf par son droit de justice si écourté, le seigneur est oisif en matière publique\footnote{Tocqueville, {\itshape ib.}, 376. Plaintes de l’assemblée provinciale de la Haute-Guyenne. « On se plaint tous les jours qu’il n’y ait aucune police dans les campagnes. Comment y en aurait-il ? {\itshape Le noble ne se mêle de rien}, excepté quelques seigneurs justes et bienfaisants qui profitent de leur ascendant sur leurs vassaux pour prévenir les voies de fait. »}. Si, par hasard, il voulait intervenir à titre officieux, réclamer pour la communauté, les bureaux le feraient taire bien vite. Depuis Louis XIV, tout a ployé sous les commis ; toute la législation et toute la pratique administrative ont opéré contre le seigneur local pour lui ôter ses fonctions efficaces et le confiner dans son titre nu. Par cette disjonction des fonctions et du titre, il est devenu d’autant plus fier qu’il devenait moins utile. Son amour-propre, n’ayant plus la grande pâture, se rabat sur la petite ; désormais il recherche les distinctions, non l’influence, et songe à primer, non à gouverner\footnote{Cahiers des États généraux de 1789. Quantité de cahiers de la noblesse demandent pour les nobles, hommes et femmes, une marque distinctive honorifique, par exemple une croix ou un cordon qui les fasse reconnaître.}. En effet, le gouvernement local, aux mains de rustres brutalisés par des plumitifs, est devenu une chose roturière, paperassière, et cette chose lui semble sale. « On blesserait son orgueil en l’invitant à s’y livrer. Asseoir les taxes, lever la milice, régler les corvées, actes serviles, œuvres de syndic. » — Il s’abstient donc, reste isolé dans son manoir, laisse à d’autres une besogne dont on l’exclut et qu’il dédaigne. Loin de défendre ses paysans, c’est à peine s’il peut se défendre lui-même, maintenir ses immunités, faire réduire sa capitation et ses vingtièmes, obtenir pour ses domestiques l’exemption de la milice, préserver sa personne, sa demeure, ses gens, sa chasse et sa pêche de l’usurpation universelle qui met aux mains de « Monseigneur l’intendant » et de MM. les subdélégués tous les biens et tous les droits  D’autant plus que bien souvent il est pauvre. Bouillé estime que toutes les vieilles familles, sauf deux ou trois cents, sont ruinées\footnote{Bouillé, {\itshape Mémoires}, 50. — Tocqueville, {\itshape ibid.}, 118, 119. — Loménie, {\itshape les Mirabeau}, 132. (Lettre du bailli de Mirabeau, 1760.) Chateaubriand, {\itshape Mémoires}, I, 14, 15, 29, 76, 80, 125. — Lucas de Montigny, {\itshape Mémoires de Mirabeau}, I, 160. — Comptes rendus de la société du Berry : {\itshape Bourges en} 1753 {\itshape et} 1754, d’après un journal à la main (aux {\itshape Archives nationales}), écrit par un des parlementaires exilés, 273.}. Dans le Rouergue, plusieurs vivent sur un revenu de cinquante et même de vingt-cinq louis. En Limousin, dit un intendant au commencement du siècle, sur plusieurs milliers, il n’y en a pas quinze qui aient vingt mille livres de rente. En Berry, vers 1754, « les trois quarts meurent de faim ». En Franche-Comté, la confrérie dont nous parlions tout à l’heure est un spectacle comique : « après la messe, ils s’en retournent chacun chez eux, les uns à pied, les autres sur leurs Rossinantes ». En Bretagne, « il y a un tas de gentilshommes rats de cave, dans les fermes, dans les plus vils emplois ». Un M. de la Morandais s’est fait régisseur d’une terre. Telle famille a pour tout bien une métairie « qui n’atteste sa noblesse que par un colombier ; elle vit à la paysanne et mange du pain bis ». Un autre gentilhomme veuf passe ses jours à boire, vit dans le désordre avec ses servantes, et met les plus beaux titres de sa maison à couvrir des pots de beurre ». « Tous les chevaliers de Chateaubriand, dit le père, ont été des ivrognes et des fouetteurs de lièvres. » Lui-même vivote tristement et pauvrement, avec cinq serviteurs, un chien de chasse et deux vieilles juments, « dans un château qui aurait tenu cent seigneurs et leur suite ». Çà et là, dans les Mémoires, on voit passer quelques-unes de ces étranges figures surannées, par exemple, en Bourgogne, « des gentilshommes chasseurs, en guêtres, en souliers ferrés, portant sous le bras une vieille épée rouillée, mourant de faim et refusant de travailler\footnote{ \href{http://gallica.bnf.fr/ark:/12148/bpt6k894586}{\dotuline{{\itshape La vie de mon père}}} [\url{http://gallica.bnf.fr/ark:/12148/bpt6k894586}], par Rétif de la Bretonne, I, 146.
 } » ; ailleurs, « M. de Pérignan, en habit, perruque et figure rousses, faisant travailler à des murs de pierre sèche dans sa terre, et s’enivrant avec le maréchal-ferrant du lieu » ; parent du cardinal Fleury, on fit de lui le premier duc de Fleury  Tout contribue à cette décadence, la loi, les mœurs, et d’abord le droit d’aînesse. Institué pour que la souveraineté et le patronage ne soient pas divisés, il ruine les nobles, depuis que la souveraineté et le patronage n’ont plus de matière propre. « En Bretagne\footnote{La règle est analogue dans les autres coutumes, notamment dans celle de Paris. (Renauldon, {\itshape ib..} 134.)}, dit Chateaubriand, les aînés nobles emportaient les deux tiers des biens, et les cadets se partageaient entre eux tous un tiers de l’héritage paternel. » Par suite, « les cadets des cadets arrivaient promptement au partage d’un pigeon, d’un lapin, d’une canardière et d’un chien de chasse. Toute la fortune de mon aïeul ne dépassait pas cinq mille livres de rente, dont l’aîné de ses fils emportait les deux tiers, trois mille trois cents livres ; restait mille six cent soixante-six livres pour les trois cadets, sur laquelle somme l’aîné prélevait encore le préciput »  Cette fortune qui s’émiette et s’anéantit, ils ne savent ni ne veulent la refaire par le négoce, l’industrie ou l’administration : ce serait déroger. « Hauts et puissants seigneurs d’un colombier, d’une crapaudière et d’une garenne », plus la substance leur manque, plus ils s’attachent au nom  Joignez à cela le séjour d’hiver à la ville, la représentation, les dépenses que comportent la vanité et le besoin de société, les visites chez le gouverneur et l’intendant : il faut être Allemand ou Anglais pour passer les mois tristes et pluvieux dans son castel ou dans sa ferme, seul, en compagnie de rustres, au risque de devenir aussi emprunté et aussi hétéroclite qu’eux\footnote{Mme d’Oberkirch, \href{http://gallica.bnf.fr/ark:/12148/bpt6k2048423/f388.table}{\dotuline{{\itshape Mémoires}, I, 395}} [\url{http://gallica.bnf.fr/ark:/12148/bpt6k2048423/f388.table}].}. Par suite, ils s’endettent, ils s’obèrent, ils vendent un morceau de leur terre, puis un autre morceau : beaucoup ont tout aliéné, sauf leur petit manoir et les droits seigneuriaux, cens, lods et ventes, droit de chasse et de justice sur le territoire dont jadis ils étaient les propriétaires\footnote{Bouillé, \href{http://gallica.bnf.fr/ark:/12148/bpt6k467396/f59.table}{\dotuline{{\itshape Mémoires}, 50}} [\url{http://gallica.bnf.fr/ark:/12148/bpt6k467396/f59.table}]. Selon lui, toutes les vieilles familles nobles, « sauf deux ou trois cents au plus, étaient ruinées. La plus grande partie des grandes terres titrées étaient devenues l’apanage des financiers, des négociants et de leurs descendants. Les fiefs, pour la plupart, étaient entre les mains des bourgeois des villes ». — Léonce de Lavergne, \href{http://gallica.bnf.fr/ark:/12148/bpt6k836438/f34}{\dotuline{{\itshape Economie} {\itshape rurale} {\itshape en} rend="internet\_link"{\itshape France}, 26}} [\url{http://gallica.bnf.fr/ark:/12148/bpt6k836438/f34}]. « La plupart végétaient pauvrement dans de petits fiefs de campagne qui ne valaient pas souvent plus de 2 000 ou 3 000 francs de rente. » — Dans la répartition de l’indemnité, en 1825, plusieurs reçoivent moins de 1 000 francs. Le plus grand nombre des indemnités ne dépasse pas 50 000 francs. — « Le trône, disait Mirabeau, n’est entouré que de nobles ruinés. »}. Puisqu’ils vivent de ces droits, il faut bien qu’ils les exercent, même quand le droit est lourd, même quand le débiteur est pauvre. Comment lui remettraient-ils la redevance en grains et en vin, quand elle est pour eux le pain et le vin de l’année ? Comment le dispenser du quint et du requint, quand c’est le seul argent qu’ils perçoivent ? Comment, étant besogneux, ne seraient-ils pas exigeants   Les voilà donc, vis-à-vis du paysan, à l’état de simples créanciers ; c’est à cela qu’aboutit le régime féodal transformé par la monarchie. Autour du château je vois les sympathies baisser, l’envie s’élever, les haines se grossir. Écarté des affaires, affranchi de l’impôt, le seigneur reste isolé, étranger parmi ses vassaux ; son autorité anéantie et ses privilèges conservés lui font une vie à part. Quand il en sort, c’est pour ajouter forcément à la misère publique. Sur ce sol ruiné par le fisc, il vient prendre une part du produit, tant de gerbes de blé, tant de cuvées de vin. Ses pigeons et son gibier mangent la récolte. Il faut aller moudre à son moulin et lui laisser un seizième de la farine. Un champ vendu six cents livres met cent livres dans sa poche. L’héritage du frère n’arrive au frère que rogné par lui d’une année de revenu. Vingt autres redevances, jadis d’utilité publique, ne servent plus qu’à nourrir un particulier inutile  Le paysan, tel alors que nous le voyons aujourd’hui, âpre au gain, décidé et habitué à tout souffrir et tout faire pour épargner ou gagner un écu, finit par jeter en dessous des regards de colère sur la tourelle qui garde les archives, le terrier, les détestables parchemins, en vertu desquels un homme d’une autre espèce, avantagé au détriment de tous, créancier universel, et payé pour ne rien faire, tond sur toutes les terres et sur tous les produits. Vienne une occasion qui mette le feu à toutes ces convoitises : le terrier brûlera, avec lui la tourelle, et, avec la tourelle, le château.

\section[{III. Seigneurs qui ne résident pas. — Énormité de leurs fortunes et de leurs droits. — Ayant des avantages plus grands, ils doivent de plus grands services. — Raisons de leur absence. — Effet de leur éloignement. — Apathie dans les provinces. — État de leurs terres. — Ils ne font pas l’aumône. — Misère de leurs tenanciers. — Exactions de leurs fermiers. — Exigences de leurs dettes. — État de leurs justices. — Effets de leur droit de chasse. — Sentiments des paysans à leur endroit.}]{III. Seigneurs qui ne résident pas. — Énormité de leurs fortunes et de leurs droits. — Ayant des avantages plus grands, ils doivent de plus grands services. — Raisons de leur absence. — Effet de leur éloignement. — Apathie dans les provinces. — État de leurs terres. — Ils ne font pas l’aumône. — Misère de leurs tenanciers. — Exactions de leurs fermiers. — Exigences de leurs dettes. — État de leurs justices. — Effets de leur droit de chasse. — Sentiments des paysans à leur endroit.}

\noindent Le spectacle est plus triste encore lorsque, des terres où les seigneurs résident, on passe aux terres où les seigneurs ne résident pas. Nobles ou anoblis, ecclésiastiques et laïques, ceux-ci sont privilégiés entre les privilégiés et forment une aristocratie dans une aristocratie. Presque toutes les familles puissantes et accréditées en sont\footnote{Bouillé, {\itshape Mémoires}, 50. — Chérin, {\itshape Abrégé chronologique des édits} (1788). « De cette multitude innombrable qui compose l’ordre des privilégiés, à peine un vingtième peut-il prétendre véritablement à la noblesse immémoriale et d’ancienne date. » — 4 070 charges de finances, administration, judicature, conféraient la noblesse. — Turgot, {\itshape Collection des Économistes}, II, 276. « Au moyen de la facilité qu’on a d’acquérir la noblesse à prix d’argent, il n’est aucun homme riche qui, sur-le-champ, ne devienne noble. » — Marquis d’Argenson, \href{http://gallica.bnf.fr/ark:/12148/bpt6k27678n}{\dotuline{{\itshape Mémoires}}} [\url{http://gallica.bnf.fr/ark:/12148/bpt6k27678n}], III, 402.}, quelle que soit leur origine et leur date. Par leur résidence habituelle ou fréquente au centre, par leurs alliances ou leurs visites mutuelles, par leurs mœurs et leur luxe, par l’influence qu’ils exercent et les inimitiés qu’ils soulèvent, ils forment un groupe à part, et ce sont eux qui ont les plus vastes terres, les premières suzerainetés, les plus larges et les plus complètes juridictions. Noblesse de cour et haut clergé, ils sont peut-être un millier dans chaque ordre, et leur petit nombre ne fait que mettre en plus haut relief l’énormité de leurs avantages. On a vu que les apanages des princes du sang comprennent un septième du territoire ; Necker\footnote{Necker, {\itshape De l’administration des finances.} II, 271. — Legrand, {\itshape l’Intendance du Hainaut}, 104, 118, 152, 412.} estime à deux millions le revenu des terres dont jouissent les deux frères du roi. Les domaines des ducs de Bouillon, d’Aiguillon et de quelques autres occupent des lieues entières, et par l’immensité, par la continuité, rappellent ceux que le duc de Sutherland, le duc de Bedford possèdent aujourd’hui en Angleterre. Rien que par ses bois et par son canal, le duc d’Orléans, avant d’épouser sa femme aussi riche que lui, se fait près d’un million de rente. Telle seigneurie, le Clermontois, appartenant au prince de Condé, renferme quarante mille habitants ; c’est l’étendue d’une principauté allemande ; « de plus tous les impôts ou subsides qui ont lieu dans le Clermontois sont perçus au profit de Son Altesse Sérénissime, le roi n’y perçoit absolument aucune chose\footnote{Même après l’échange de 1784, le prince garde pour lui « toutes les impositions personnelles, ainsi que la subvention sur les habitants », sauf une somme de 6 000 livres pour les routes. {\itshape Archives nationales}, G, 192, Mémoires du 14 avril 1781 sur la situation du Clermontois. — {\itshape Procès-verbaux de l’assemblée provinciale des Trois-Évêchés} (1787), 380.} »  Naturellement, autorité et richesse vont ensemble, et, plus une terre rapporte, plus son propriétaire ressemble à un souverain. L’archevêque de Cambray, duc de Cambray, comte de Cambrésis, a la suzeraineté de tous les fiefs dans un pays qui compte soixante-quinze mille habitants ; il choisit la moitié des échevins à Cambray et toute l’administration du Cateau ; il nomme à deux grandes abbayes, il préside les États provinciaux et le bureau permanent qui leur succède ; bref, sous l’intendant et à côté de lui, il garde une prééminence, bien mieux, une influence à peu près semblable à celle que conserve aujourd’hui sur son domaine tel grand-duc incorporé dans le nouvel empire allemand. Près de lui, dans le Hainaut, l’abbé de Saint-Amand possède les sept huitièmes du territoire de la prévôté et perçoit sur le dernier huitième les rentes seigneuriales, des corvées et la dîme ; de plus, il nomme le prévôt et les échevins, en sorte, disent les doléances, « qu’il compose tout l’État, ou plutôt qu’il est lui seul tout l’État\footnote{La ville de Saint-Amand, à elle seule, contient aujourd’hui 10 210 habitants.} ». — Je ne finirais pas, si j’énumérais tous ces gros lots. Ne prenons que celui des prélats, et par un seul côté, celui de l’argent. Dans l’{\itshape Almanach royal} et dans {\itshape la France ecclésiastique} de 1788, nous lisons leur revenu avoué ; mais le revenu véritable est de moitié en sus pour les évêchés, du double et du triple pour les abbayes, et il faut encore doubler ce revenu véritable pour en avoir la valeur en monnaie d’aujourd’hui\footnote{Voir la note 3 [’p. 305’].}. Les cent trente et un évêques et archevêques ont ensemble 5 600 000 livres de revenu épiscopal et 1 200 000 livres en abbayes, en moyenne 50 000 livres par tête dans l’imprimé, 100 000 en fait : aussi bien aux yeux des contemporains, au dire des spectateurs qui savaient la vérité vraie, un évêque était « un grand seigneur ayant 100 000 livres de rente\footnote{Marquis de Ferrières, {\itshape Mémoires}, II, 57 : « Tous en avaient 100 000, quelques-uns 200, 300 et jusqu’à 800 000. »} ». Quelques sièges importants sont dotés magnifiquement. Sens rapporte 70 000 livres, Verdun 74 000, Tours 82 000, Beauvais, Toulouse et Bayeux 90 000, Rouen 100 000, Auch, Metz, Albi 120 000, Narbonne 160 000, Paris et Cambrai 200 000 en chiffres officiels, et probablement moitié en sus en sommes perçues. D’autres sièges, moins lucratifs, sont en proportion mieux traités encore. Figurez-vous une petite ville de province, qui souvent n’est pas même une mince sous-préfecture de notre temps, Couserans, Mirepoix, Lavaur, Rieux, Lombez, Saint-Papoul, Comminges, Luçon, Sarlat, Mende, Fréjus, Lescar, Belley, Saint-Malo, Tréguier, Embrun, Saint-Claude, alentour moins de deux cents, moins de cent, parfois moins de cinquante paroisses, et, pour exercer cette petite surveillance ecclésiastique, un prélat qui touche de 25 000 à 70 000 livres en chiffres officiels, de 37 000 à 105 000 livres en chiffres réels, de 74 000 à 210 000 livres en argent d’aujourd’hui. Quant aux abbayes, j’en compte trente-trois qui rapportent de 25 000 à 120 000 livres à l’abbé, vingt-sept qui rapportent de 20 000 à 100 000 livres à l’abbesse ; pesez ces chiffres de l’Almanach, et songez qu’il faut les doubler et au-delà pour avoir le revenu réel, les quadrupler et au-delà pour avoir le revenu actuel. Il est clair qu’avec de tels revenus et les droits féodaux de police, de justice, d’administration qui les accompagnent, un grand seigneur ecclésiastique ou laïque est, de fait, une sorte de prince dans son district, qu’il ressemble trop à l’ancien souverain pour avoir le droit de vivre en particulier ordinaire, que ses avantages privés lui imposent un caractère public, que son titre supérieur et ses profits énormes l’obligent à des services proportionnés, et que, même sous la domination de l’intendant, il doit à ses vassaux, à ses tenanciers, à ses censitaires, le secours de son intervention, de son patronage et de ses bienfaits.\par
Pour cela il faudrait résider, et le plus souvent il est absent. Depuis cent cinquante ans, une sorte d’attraction toute-puissante retire les grands de la province, les pousse vers la capitale, et le mouvement est irrésistible, car il est l’effet des deux forces les plus grandes et les plus universelles qui puissent agir sur les hommes, l’une qui est la situation sociale, l’autre qui est le caractère national. Ce n’est pas impunément qu’on retranche à un arbre ses racines. Instituée pour gouverner, une aristocratie se détache du sol lorsqu’elle ne gouverne plus, et elle a cessé de gouverner depuis que, par un empiètement croissant et continu, presque toute la justice, toute l’administration, toute la police, chaque détail du gouvernement local ou général, toute initiative, collaboration ou contrôle en matière d’impôts, d’élections, de routes, de travaux et de charités, a passé dans les mains de l’intendant et du subdélégué, sous la direction suprême du contrôleur général et du Conseil du roi\footnote{Tocqueville, {\itshape ib.}, liv. 2, chap. 2, 182. — Lettre du bailli de Mirabeau du 25 août 1770. « Cet ordre féodal n’était que fort, et ils l’ont appelé barbare, parce que la France, qui avait les vices de la force, n’a plus que ceux de la faiblesse et que le troupeau, qui était autrefois dévoré par les loups, l’est aujourd’hui par les poux… Trois ou quatre coups de pied ou de bâton ne nuisent pas tant à la famille d’un pauvre homme, ni à lui-même, que six rôles d’écritures qui le dévorent. » — « La noblesse, disait déjà Saint-Simon, est devenue un autre peuple qui n’a d’autre choix que de croupir dans une mortelle et ruineuse oisiveté qui la rend à charge et méprisée, ou d’aller se faire tuer à la guerre à travers les insultes des commis, des secrétaires d’État et des secrétaires des intendants. » Voilà les réclamations des âmes féodales  Tous les détails qui suivent sont tirés de Saint-Simon, Dangeau, Luynes, Argenson et autres historiens de la cour.}. Des commis, des gens « de plume et de robe », des roturiers sans consistance font la besogne ; nul moyen de la leur disputer. Même avec la délégation du roi, un gouverneur de province, fût-il héréditaire et prince du sang comme les Condés en Bourgogne, doit s’effacer devant l’intendant ; il n’a pas d’office effectif ; ses emplois publics consistent à faire figure et à recevoir. Du reste, il remplirait mal les autres : la machine administrative, avec ses milliers de rouages durs, grinçants et sales, telle que Richelieu et Louis XIV l’ont faite, ne peut fonctionner qu’aux mains d’ouvriers congédiables à volonté, sans scrupules et prompts à tout plier sous la raison d’État ; impossible de se commettre avec ces drôles. Il s’abstient, leur abandonne les affaires. Désœuvré, amoindri, que ferait-il maintenant sur son domaine où il ne règne plus et où il s’ennuie ? Il vient à la ville, surtout à la cour. — D’ailleurs il n’y a plus de carrière que par cette issue : pour parvenir, on est tenu d’être courtisan. Le roi le veut, il faut que vous soyez de son salon pour obtenir ses grâces ; sinon, à la première demande, il répondra : « Qui est-ce ? C’est un homme que je ne vois pas ». L’absence, à ses yeux, n’a pas d’excuse, même quand elle a pour cause une conversion, et pour motif la pénitence ; on lui a préféré Dieu, c’est une désertion. Les ministres écrivent aux intendants pour savoir si les gentilshommes de leur province « aiment à rester chez eux » et s’ils « refusent de venir rendre leurs devoirs au roi ». Songez à la grandeur d’un pareil attrait : gouvernements, commandements, évêchés, bénéfices, charges de cour, survivances, pensions, crédits, faveurs de toute espèce et tout degré pour soi et pour les siens, tout ce qu’un État de vingt et vingt-cinq millions d’hommes peut offrir de désirable à l’ambition, à la vanité et à l’intérêt se trouve rassemblé là comme en un réservoir. On y accourt, et l’on y puise  D’autant plus que l’endroit est agréable, disposé à souhait et de parti pris pour convenir aux aptitudes sociables du caractère français. La cour est un grand salon permanent, où « l’accès est libre et facile des sujets au prince », où ils vivent avec lui « dans une société douce et honnête, nonobstant la distance presque infinie du rang et du pouvoir », où le monarque se pique d’être un parfait maître de maison\footnote{{\itshape Œuvres de Louis} XIV ; ce sont là ses propres paroles. — Mme Vigée-Lebrun, {\itshape Souvenirs}, I, 71 : « J’ai vu la reine (Marie-Antoinette), faisant dîner Madame, alors âgée de six ans, avec une petite paysanne dont elle prenait soin, vouloir que cette petite fût servie la première, en disant à sa fille : " Vous devez lui faire les honneurs ”. »}. De fait, il n’y eut jamais de salon si bien tenu, ni si propre à retenir ses hôtes par les plaisirs de toute sorte, par la beauté, la dignité et l’agrément du décor, par le choix de la compagnie, par l’intérêt du spectacle. Il n’y a que Versailles pour se montrer, faire figure, se pousser, pour s’amuser, converser ou causer, au centre des nouvelles, de l’action et des affaires, avec l’élite du royaume et les arbitres du ton, de l’élégance et du goût. « Sire, disait M. de Vardes à Louis XIV, quand on est loin de Votre Majesté, non seulement on est malheureux, mais encore on est ridicule. » Il ne reste en province que la noblesse pauvre et rustique ; pour y vivre, il faut être arriéré, dégoûté ou exilé. Quand le roi renvoie un seigneur dans ses terres, c’est la pire disgrâce ; à l’humiliation de la déchéance s’ajoute le poids insupportable de l’ennui. Le plus beau château dans un site agréable est un affreux « désert » ; on n’y peut voir personne, sauf des grotesques de petite ville ou des rustres de village\footnote{Molière, {\itshape Misanthrope} ; c’est là le « désert » où Célimène refuse de s’ensevelir avec Alceste. Voyez aussi dans le {\itshape Tartufe} la peinture que Dorine fait d’une petite ville. — Arthur Young, {\itshape Voyages en France}, I, 78.}. « L’exil seul, dit Arthur Young, force la noblesse de France à faire ce que les Anglais font par préférence : résider sur leurs domaines pour les embellir. » Dix fois Saint-Simon et les autres historiens de la cour disent en parlant d’une cérémonie : « Toute la France était là » ; en effet, tout ce qui compte en France est là, et ils se reconnaissent à cette marque. Paris et la cour deviennent donc le séjour obligé de tout le beau monde. Dans une telle situation, les départs entraînent les départs ; plus la province est délaissée, plus on la délaisse. « Il n’y a pas dans le royaume, dit le marquis de Mirabeau, une seule terre un peu considérable dont le propriétaire ne soit à Paris, et conséquemment ne néglige ses maisons et ses châteaux\footnote{{\itshape Traité de la population}, 108 (1756).}. » Les grands seigneurs laïques ont leur hôtel dans la capitale, leur entresol à Versailles, leur maison de plaisance dans un cercle de vingt lieues ; si de loin en loin ils visitent leurs terres, c’est pour y chasser. Les quinze cents abbés et prieurs commendataires jouissent de leurs bénéfices comme d’une ferme éloignée. Les deux mille sept cents grands vicaires et chanoines de chapitre se visitent et dînent en ville. Sauf quelques hommes apostoliques, les cent trente et un évêques résident le moins qu’ils peuvent ; presque tous nobles, tous gens du monde, que feraient-ils loin du monde, confinés dans une ville de province ? Se figure-t-on un grand seigneur, jadis abbé brillant et galant, maintenant évêque avec cent mille livres de rente et qui volontairement s’enterre pour toute l’année à Mende, à Condom, à Comminges, dans une bicoque ? La distance est devenue trop grande entre la vie élégante, variée, littéraire du centre, et la vie monotone, inerte, positive de la province. C’est pourquoi le grand seigneur qui sort de la première ne peut entrer dans la seconde ; il reste absent, au moins de cœur.\par
Sombre aspect que celui d’un pays où le cœur cesse de pousser le sang dans les veines. Arthur Young, qui parcourut la France de 1787 à 1789, s’étonne d’y trouver à la fois un centre aussi vivant et des extrémités aussi mortes. Entre Paris et Versailles, la double file de voitures qui vont et reviennent\footnote{Je tiens ce détail de vieillards qui l’ont vu avant 1789.} se prolonge pendant cinq lieues et sans interruption depuis le matin jusqu’au soir. Le contraste est grand sur les autres chemins. « Sortis de Paris par la route d’Orléans, dit Arthur Young, pendant dix milles nous n’avons pas rencontré une diligence, rien que des messageries et des chaises de poste en petit nombre, pas la dixième partie de ce que nous aurions trouvé près de Londres en une heure. » Sur la grande route, près de Narbonne, « pendant trente-six milles, dit-il, je n’ai croisé qu’un cabriolet, une demi-douzaine de charrettes et quelques bonnes femmes menant leur âne ». Ailleurs, près de Saint-Girons, il note qu’en deux cent cinquante milles il a rencontré en tout « deux cabriolets et trois misérables choses semblables à notre vieille chaise de poste anglaise à un cheval, pas un gentilhomme ». Dans toute cette contrée, auberges exécrables ; impossible d’y louer une voiture, tandis qu’en Angleterre, même dans une ville écartée de deux mille à quinze cents âmes, on trouve des hôtels confortables et tous les moyens de transport ; c’est la preuve qu’en France « la circulation est nulle ». Il n’y a de civilisation et de bien-être que dans les très grandes villes. « À Nantes, superbe salle de spectacle, deux fois plus grande que celle de Drury-Lane et cinq fois plus magnifique. Bon Dieu, m’écriai-je intérieurement, est-ce à un tel spectacle que mènent les garennes, les landes, les déserts que j’ai traversés pendant trois cents milles   D’un bond vous passez de la misère à la prodigalité. La campagne est déserte, et si quelque gentilhomme l’habite, c’est dans quelque triste bouge, pour épargner cet argent qu’il vient ensuite jeter dans la capitale. » — « Un coche\footnote{{\itshape Mémoires} de M. de Montlosier, I, 161.}, dit M. de Montlosier, partait toutes les semaines des principales villes de province pour Paris, et n’était pas toujours plein : voilà pour le mouvement des affaires. On avait une seule gazette, appelée {\itshape Gazette de France}, qui paraissait deux fois par semaine, voilà pour le mouvement des esprits. » Des magistrats de Paris, exilés à Bourges en 1753 et 1754, en font le tableau suivant : « Une ville où l’on ne trouve personne à qui parler à son aise de quoi que ce soit de sensé et de raisonnable ; des nobles qui meurent les trois quarts de faim, entichés de leur origine, tenant à l’écart la robe et la finance, et trouvant singulier que la fille d’un receveur des tailles, devenue la femme d’un conseiller au Parlement de Paris, se permette d’avoir de l’esprit et du monde ; des bourgeois de l’ignorance la plus crasse, seul appui de l’espèce de léthargie où sont plongés les esprits de la plupart des habitants ; des femmes bigotes et prétentieuses, fort adonnées au jeu et à la galanterie\footnote{Comptes rendus de la société du Berry, {\itshape Bourges en} 1753 {\itshape et} 1754, 273.} » ; dans ce monde étriqué et engourdi, parmi ces MM. Tibaudier le conseiller et Harpin le receveur, parmi ces vicomtes de Sotenville et ces comtesses d’Escarbagnas, l’archevêque, cardinal de La Rochefoucauld, grand aumônier du roi, pourvu de quatre grosses abbayes, ayant cinq cent mille livres de revenu, homme du monde, le plus souvent absent, et, quand il réside, s’amusant à embellir ses jardins et son palais ; bref, un faisan doré de volière dans une basse-cour d’oies\footnote{{\itshape Ib.}, 271. Un jour, le cardinal, montrant à des hôtes son palais qu’il venait d’achever, les conduisit au fond d’un corridor où il avait installé des lieux à l’anglaise, chose nouvelle alors. M. Boutin de la Coulommière, fils d’un receveur général des finances, se récria à la vue de ce mécanisme ingénieux dont il se plaisait à faire jouer les ressorts, et se tournant vers l’abbé de Canillac : « Cela, dit-il, est admirable sans doute ; mais ce qui me semble plus admirable encore, c’est que Son Éminence, étant au-dessus des faiblesses humaines, veuille bien s’y accommoder. » Mot précieux et seul capable de montrer le rang, la position d’un prélat grand seigneur en province.}. Naturellement, toute pensée politique manque. « On ne peut imaginer, dit le manuscrit, personne plus indifférente pour toutes les affaires publiques. » Plus tard, au plus fort des événements les plus graves et qui les touchent par l’endroit le plus sensible, même apathie. À Château-Thierry, le 4 juillet 1789\footnote{Arthur Young, II, 230 et suivantes.}, pas un café où l’on puisse trouver un journal ; il n’y en a qu’un à Dijon ; à Moulins, le 7 août, « dans le meilleur café de la ville, où il y a au moins vingt tables, on m’aurait aussi tôt donné un éléphant qu’un journal ». Entre Strasbourg et Besançon, pas une gazette ; « à Besançon, il n’y a que la {\itshape Gazette de France}, pour laquelle un homme qui a le sens commun ne donnerait pas un sou dans le moment actuel, et le {\itshape Courrier militaire}, vieux de quinze jours ; des gens bien mis parlent des choses qui sont arrivées il y a deux ou trois semaines, et leurs discours démontrent qu’ils ne savent rien de ce qui se passe aujourd’hui ». À Clermont, « je dînai ou soupai cinq fois à table d’hôte avec vingt ou trente négociants, marchands, officiers, etc. ; à peine un mot de politique dans un moment où tous les cœurs devraient battre de sensations politiques ; l’ignorance ou la stupidité de ces gens-là est incroyable. Il ne se passe pas de semaine ou leur pays ne produise une multitude d’événements\footnote{L’abolition des dîmes, des droits féodaux, la permission de tuer le gibier, etc.} qui sont analysés et discutés même par les charpentiers et les serruriers de l’Angleterre ». La cause de cette inertie est manifeste ; interrogés sur leur opinion, tous répondent : « Nous sommes de la province, il nous faut attendre pour savoir ce que l’on fait à Paris ». N’ayant jamais agi, ils ne savent pas agir ; mais, grâce à leur inertie, ils se laisseront pousser. La province est une mare immense, stagnante, qui, par une inondation terrible, peut se déverser toute d’un côté et tout d’un coup ; c’est la faute de ses ingénieurs qui n’y ont fait ni digues ni conduites.\par
Telle et la langueur ou plutôt l’anéantissement où tombe la vie locale lorsque les chefs locaux lui dérobent leur présence, leur action ou leur sympathie. Je ne vois pour y prendre part que trois ou quatre grands seigneurs, philanthropes pratiques et guidés par l’exemple des nobles anglais, le duc d’Harcourt qui arrange les procès de ses paysans, le duc de La Rochefoucault-Liancourt qui a fondé dans ses terres une ferme modèle et une école des arts et métiers pour les enfants des militaires pauvres, le comte de Brienne dont trente villages viendront demander la liberté à la Convention\footnote{Loménie, {\itshape les Mirabeau}, 134. (Lettre du bailli du 25 septembre 1760) : « Je suis à Harcourt où j’admire la bonne et honnête grandeur du maître. Tu ne saurais penser le plaisir que j’ai eu les jours de fête de voir le peuple entier partout dans le château, et de bons petits paysans et petites paysannes venir regarder le bon patron sous le nez et presque lui tirer sa montre pour voir les breloques, tout cela avec l’air de fraternité sans familiarité. Le bon duc ne laisse point plaider ses vassaux, il les écoute et les juge en les accommodant avec une patience admirable. » — Lacretelle, \href{http://gallica.bnf.fr/ark:/12148/bpt6k467841/f63.table}{\dotuline{{\itshape Dix ans d’épreuve}, 58}} [\url{http://gallica.bnf.fr/ark:/12148/bpt6k467841/f63.table}].}. Les autres, pour la plupart libéraux, se contentent de raisonner sur le bien public et sur l’économie politique. En effet, la différence des manières, la séparation des intérêts, la distance des idées sont si grandes, qu’entre les plus exempts de morgue et leurs tenanciers directs, les contacts sont rares et lointains. Chez le duc de La Rochefoucauld-Liancourt lui-même, Arthur Young ayant besoin de renseignements, on lui envoie le régisseur. « Chez un noble de mon pays, on eût invité à dîner trois ou quatre fermiers qui se seraient assis à table à côté des dames du premier rang. Je n’exagère pas en disant que cela m’est arrivé cent fois dans les premières maisons du Royaume-Uni. C’est cependant une chose qu’on ne verrait pas en France de Calais à Bayonne, excepté, par hasard, chez quelque grand seigneur ayant beaucoup voyagé en Angleterre, et encore à condition qu’on le demandât. La noblesse française n’a pas plus l’idée de se livrer à l’agriculture ou d’en faire un sujet de conversation, sauf en théorie, et comme on parlerait d’un métier ou d’un engin de marine, que de toute autre chose contraire à ses habitudes et à ses occupations journalières. » Par tradition, mode et parti pris, ils ne sont et ne veulent être que gens du monde ; leur seule affaire est la causerie et la chasse. Jamais conducteurs d’hommes n’ont tellement désappris l’art de conduire les hommes, art qui consiste à marcher sur la même route, mais en tête, et à guider leur travail en y prenant part  Notre Anglais, témoin oculaire et compétent, écrit encore : « Un grand seigneur eût-il des millions de revenu, vous êtes sûr de trouver ses terres en friches. Celles du prince de Soubise et celles du duc de Bouillon sont les plus grandes de France, et tous les signes que j’ai aperçus de leur grandeur sont des bruyères, des landes, des déserts, des fougeraies. Visitez leur résidence où qu’elle soit, et vous les verrez au milieu des forêts très peuplées de cerfs, de sangliers et de loups »  « Les grands propriétaires, dit un autre contemporain\footnote{ {\itshape De l’état religieux}, par les abbés de Bonnefoi et Bernard (1784), 287, 291.
 }, attirés et retenus dans nos villes par les jouissances du luxe, ne connaissent rien de leurs terres », sauf « leurs fermiers qu’ils foulent pour fournir à un faste ruineux. Comment attendre des améliorations de ceux qui se refusent même à l’entretien et aux réparations les plus indispensables ? » Une preuve sûre que leur absence est la cause du mal, c’est la différence visible du domaine affermé par l’abbé commendataire absent et du domaine surveillé par les religieux présents. « Un voyageur instruit les reconnaît » tout d’abord à l’état des cultures. « S’il rencontre des champs bien environnés de fossés, plantés avec soin et couverts de riches moissons, ces champs, dit-il, appartiennent à des religieux. Presque toujours à côté de ces plaines fertiles, une terre mal entretenue et presque épuisée présente un contraste affligeant ; cependant la nature du sol est égale, ce sont deux parties du même domaine ; il voit que cette dernière est la portion de l’abbé commendataire. » — « La manse abbatiale, disait Lefranc de Pompignan, a souvent l’air du patrimoine d’un dissipateur ; la manse monacale est comme un patrimoine où l’on n’omet rien pour améliorer », en sorte que les « deux tiers » dont l’abbé jouit lui rapportent moins que le tiers réservé à ses moines. — Ruine ou détresse de l’agriculture, voilà encore un des effets de l’absence ; il y avait peut-être un tiers du sol en France qui, déserté comme l’Irlande, était aussi mal soigné, aussi peu productif que l’Irlande aux mains des riches {\itshape absentees}, évêques, doyens et nobles anglais.\par
Ne faisant rien pour la terre, comment feraient-ils quelque chose pour les hommes   Sans doute, de temps en temps, surtout quand les fermages ne rentrent pas, le régisseur écrit, allègue la misère du fermier. Sans doute aussi, et notamment depuis trente années, ils veulent être humains ; ils dissertent entre eux sur les droits de l’homme ; ils souffriraient de voir la face pâle d’un paysan qui a faim. Mais ils ne la voient pas, songeront-ils à la deviner sous la phrase maladroite et complimenteuse de leur homme d’affaires ? D’ailleurs, savent-ils ce que c’est que la faim ? Lequel d’entre eux a l’expérience de la campagne ? Et comment pourraient-ils se représenter la misère du misérable ? Ils sont trop loin de lui pour cela, trop étrangers à sa vie. Le portrait qu’ils s’en font est imaginaire ; jamais on ne s’est représenté plus faussement le paysan ; aussi le réveil sera-t-il terrible. C’est le bon villageois, doux, humble, reconnaissant, simple de cœur et droit d’esprit, facile à conduire, conçu d’après Rousseau et les idylles qui se jouent en ce moment même sur tous les théâtres de société\footnote{Voir à ce sujet \href{http://gallica.bnf.fr/ark:/12148/bpt6k879792}{\dotuline{{\itshape La partie de chasse de Henri IV}}} [\url{http://gallica.bnf.fr/ark:/12148/bpt6k879792}], par Collé. Cf. Berquin, Florian, Marmontel, etc., et aussi les estampes de l’époque.}. Faute de le connaître, ils l’oublient ; ils lisent la lettre de leur régisseur, puis aussitôt le tourbillon du beau monde les ressaisit, et, après un soupir donné à la détresse des pauvres, ils songent que cette année ils ne toucheront pas leurs rentes. — Ce n’est pas là une bonne disposition pour faire l’aumône. Aussi, c’est contre les absents, non contre les résidents que les plaintes s’élèvent\footnote{Boivin-Champeaux, {\itshape Notice historique sur la Révolution dans le département de l’Eure}, 61, 63.}. « Les biens de l’Église, dit un cahier, ne servent qu’à nourrir les passions des titulaires. » « Suivant les canons, dit un autre cahier, tout bénéficière doit donner le quart de son revenu aux pauvres ; cependant, dans notre paroisse, il y a pour plus de douze mille livres de revenu, et il n’en est rien donné aux pauvres, sinon quelque faible chose de la part du sieur curé. » — « L’abbé de Conches touche la moitié des dîmes et ne contribue en rien au soulagement de la paroisse. » Ailleurs, « le chapitre d’Ecouis, qui possède le bénéfice des dîmes, ne fait aucun bien aux pauvres et ne cherche qu’à augmenter son revenu ». Près de là, l’abbé de la Croix-Leufroy, « gros décimateur, et l’abbé de Bernay, qui touche cinquante-sept mille livres de son bénéfice et ne réside pas, gardent tout et donnent à peine à leurs curés desservants de quoi vivre ». — « J’ai dans ma paroisse, dit un curé du Berry\footnote{Archives {\itshape nationales}, Procès-verbaux des États généraux de 1789, t. XXXIX, 111 : Lettre du 6 mars 1789 du curé de Saint-Pierre de Ponsigny, en Berry  Marquis d’Argenson, 6 juillet 1756. « On a trouvé au feu cardinal de Soubise trois millions d’argent comptant et il ne donnait rien aux pauvres. »}, six bénéfices simples dont les titulaires sont toujours absents, et ils jouissent ensemble de neuf mille livres de revenu ; je leur ai fait par écrit les plus touchantes invitations dans la calamité de l’année dernière ; je n’ai reçu que deux louis d’un seul, et la plupart ne m’ont pas même répondu. » — À plus forte raison faut-il compter qu’en temps ordinaire ils ne feront point remise de leurs droits. D’ailleurs, ces droits, censives, lods et ventes, dîmes et le reste, sont entre les mains d’un régisseur, et un bon régisseur est celui qui fait rentrer beaucoup d’argent. Il n’a pas le droit d’être généreux aux dépens de son maître, et il est tenté d’exploiter à son profit les sujets de son maître. En vain la molle main seigneuriale voudrait être légère ou paternelle, la dure main du mandataire pèse sur les paysans de tout son poids, et les ménagements d’un chef font place aux exactions d’un commis. — Qu’est-ce donc lorsque, sur le domaine, au lieu d’un commis, on trouve un fermier, un adjudicataire qui, moyennant une somme annuelle, a acheté du seigneur l’exploitation de ses droits ? Dans l’élection de Mayenne\footnote{Tocqueville, {\itshape ib.}, 405  Renauldon, {\itshape ib.}, 628.}, et certainement aussi dans beaucoup d’autres, les principaux domaines sont affermés de la sorte. D’ailleurs il y a nombre de droits, comme les péages, la taxe des marchés, le droit du troupeau à part, le monopole du four et du moulin banal, qui ne peuvent guère être exercés autrement ; il faut au seigneur un adjudicataire qui lui épargne les débats et les embarras de la perception\footnote{L’exemple est donné par le roi, qui vend aux fermiers généraux, moyennant une somme annuelle, l’exploitation des principaux impôts indirects.}. En ce cas si fréquent, toute l’exigence et toute la rapacité de l’entrepreneur, décidé à gagner ou tout au moins à ne pas perdre, s’abattent sur les paysans : « C’est un loup ravissant, dit Renauldon, que l’on lâche sur la terre, qui en tire jusqu’aux derniers sous, accable les sujets, les réduit à la mendicité, fait déserter les cultivateurs, rend odieux le maître qui se trouve forcé de tolérer ses exactions, pour le faire jouir. » Imaginez, si vous pouvez, le mal que peut faire un usurier de campagne armé contre eux de droits si pesants ; c’est la seigneurie féodale aux mains d’Harpagon ou plutôt du père Grandet. En effet, lorsqu’un droit devient insupportable, on voit, par les doléances locales, que presque toujours c’est un fermier qui l’exerce\footnote{Voltaire, {\itshape Politique et Législation, La voix du curé} (à propos des serfs de Saint-Claude)  Discours du duc d’Aiguillon, le 4 août 1789, à l’Assemblée nationale : « Les propriétaires des fiefs, des terres seigneuriales, ne sont que bien rarement coupables des excès dont se plaignent leurs vassaux ; mais leurs gens d’affaires sont souvent sans pitié. »} : c’est un fermier de chanoines qui revendique l’héritage paternel de Jeanne Mermet, sous prétexte qu’elle a passé chez son mari la première nuit de ses noces. On trouverait à peine des exactions égales dans l’Irlande de 1830, sur ces domaines où, le fermier général louant à des sous-fermiers, et ceux-ci à d’autres moindres, le petit colon, placé au bas de l’échelle, portait à lui seul tout le poids de l’échelle entière, d’autant plus foulé que son créancier, foulé lui-même, mesurait les exigences qu’il pratiquait aux exigences qu’il subissait.\par
Supposons que, voyant cet abus de son nom, le seigneur veuille ôter à ces mains mercenaires l’administration de son domaine ; le plus souvent il ne le pourrait pas : il est trop endetté, il a délégué à ses créanciers telle portion de sa terre, telle branche de ses revenus. Depuis des siècles, la haute noblesse s’obère par son luxe, par sa prodigalité, par son insouciance, et par ce faux point d’honneur qui consiste à regarder le soin de compter comme une occupation de comptable. Elle est fière de sa négligence, elle appelle cela vivre noblement\footnote{Beugnot, {\itshape Mémoires}, I, 136  Duc de Lévis, {\itshape Souvenirs et portraits}, 156  {\itshape Moniteur}, séance du 22 novembre 1872, Discours de M. Bocher : « D’après l’état dressé par ordre de la Convention, la fortune du duc d’Orléans se composait de soixante-quatorze millions de dettes et de cent quatorze d’actif. » Le 8 janvier 1792, il avait abandonné à ses créanciers trente-huit millions de ses biens pour se libérer.}. « Monsieur l’archevêque, disait Louis XVI à M. de Dillon, on prétend que vous avez des dettes, et même beaucoup. — Sire, répondit le prélat avec une ironie de grand seigneur, je m’en informerai à mon intendant, et j’aurai l’honneur d’en rendre compte à Votre Majesté. » — Le maréchal de Soubise a cinq cent mille livres de rente qui ne lui suffisent pas. On sait les dettes du cardinal de Rohan, du comte d’Artois ; leurs millions de revenu se perdaient en vain dans ce gouffre. Le prince de Guéméné vient de faire une faillite de trente-cinq millions. Le duc d’Orléans, le plus riche propriétaire du royaume, devait à sa mort soixante-quatorze millions. Quand, sur les biens des émigrés, il fallut payer leurs créanciers, il fut avéré que la plupart des grandes fortunes étaient vermoulues d’hypothèques\footnote{En 1785, le duc de Choiseul évaluait dans son testament ses biens à quatorze millions et ses dettes à dix. (Comte de Tilly, {\itshape Mémoires}, II, 215.)}. Quiconque a lu les mémoires sait que depuis deux cents ans, pour boucler leurs vides, il a fallu des mariages d’argent et les bienfaits du roi  C’est pourquoi, à l’exemple du roi lui-même, ils ont fait argent de tout, notamment des places dont ils disposent, et, lâchant l’autorité pour les profits, ils ont aliéné le dernier lambeau de gouvernement qui leur restait. Ainsi partout ils ont dépouillé le caractère vénéré de chef pour revêtir le caractère odieux de trafiquant. « Non seulement, dit un contemporain\footnote{Renauldon, {\itshape ib.}, 45, 52, 628  Duvergier, {\itshape Collection des lois}, t. II, 391. Loi du 31 août-18 octobre 1792  {\itshape Cahier d’un magistrat du Châtelet} sur les justices seigneuriales (1789), 29  Legrand, {\itshape l’Intendance du Hainaut}, 110.}, ils ne donnent pas de gages à leurs officiers de justice, ou les prennent au rabais ; mais ce qu’il y a de pis, c’est que la plupart aujourd’hui vendent leurs offices. » Malgré l’édit de 1693, les juges ainsi nommés ne se font point recevoir aux justices royales et ne prêtent pas serment. « Qu’arrive-t-il alors ? La justice, trop souvent exercée par des fripons, dégénère en brigandage, ou en une impunité affreuse. » — Ordinairement le seigneur qui a vendu la charge moyennant finance perçoit en outre le centième, le cinquantième, le dixième du prix lorsqu’elle passe en d’autres mains ; d’autres fois il en vend la survivance. Charges et survivances, il en crée pour en vendre. « Toutes les justices seigneuriales, disent les cahiers, sont infestées d’une foule d’huissiers de toute espèce, sergents seigneuriaux, huissiers à cheval, huissiers à verge, gardes de la prévôté des monnaies, gardes de la connétablie. Il n’est pas rare d’en trouver jusqu’à dix dans un arrondissement qui pourrait à peine en faire vivre deux, s’ils se renfermaient dans les limites de leurs charges. » Aussi « sont-ils en même temps juges, procureurs, procureurs fiscaux, greffiers, notaires », chacun dans un lieu différent, chacun exerçant dans plusieurs seigneuries et sous divers titres, tous ambulants, tous s’entendant comme fripons en foire, et se réunissant au cabaret pour y instrumenter, plaider et juger. Parfois, pour faire une économie, le seigneur confire le titre à l’un de ses fermiers : « À Hautemont, dans le Hainaut, c’est un domestique qui est procureur fiscal. » Plus souvent il commet quelque avocat famélique de la petite ville voisine, avec des gages « qui ne suffiraient pas à le faire vivre une semaine ». Celui-ci se dédommage sur les paysans. Rôles de chicane, longueurs et complications voulues de la procédure, vacations à trois livres l’heure pour l’avocat, à six livres l’heure pour le bailli : l’engeance noire des sangsues judiciaires suce d’autant plus âprement qu’elle est plus nombreuse sur une proie plus maigre, et qu’elle a payé le privilège de sucer\footnote{{\itshape Archives nationales}, H, 614 (Mémoire par René de Hauteville, avocat au Parlement, Saint-Brieuc, 5 octobre 1776). En Bretagne, le nombre des justices seigneuriales est immense, et les plaideurs sont obligés de passer par quatre ou cinq juridictions avant d’arriver au Parlement. « Où exerce-t-on la justice ? C’est au cabaret, à la taverne, où, dans le sein de l’ivresse et de la crapule, le juge vend la justice à qui paye plus. »}  On devine l’arbitraire, la corruption, la négligence d’un pareil régime. « L’impunité, dit Renauldon, n’est nulle part plus grande que dans les justices seigneuriales… Il ne s’y fait aucune recherche des crimes les plus atroces » ; car le seigneur craint de fournir aux frais d’un procès criminel, et ses juges ou procureurs ont peur de n’être pas payés de leurs procédures. Au reste, sa geôle est souvent une cave du château ; « sur cent justices, il n’y en a pas une qui soit en règle du côté des prisons » ; ses gardiens ferment les yeux ou tendent la main. C’est pourquoi ses terres deviennent l’asile de tous les scélérats du canton »  Terrible effet de son indifférence et qui va se retourner contre lui-même : demain, au club, les procureurs qu’il a multipliés demanderont sa tête, et les bandits qu’il a tolérés la mettront au bout d’une pique.\par
Reste un point, la chasse, où sa juridiction est encore active et sévère, et c’est justement le point où elle se trouve le plus blessante. Jadis, quand la moitié du canton était en forêts ou en friches et que les grosses bêtes ravageaient l’autre moitié, il avait raison de s’en réserver la poursuite ; cela rentrait dans son office de capitaine local. Il était le grand gendarme héréditaire, toujours armé, toujours à cheval, aussi bien contre les sangliers et les loups que contre les rôdeurs et les brigands. À présent que du gendarme il n’a plus que le titre et les épaulettes, il maintient par tradition son privilège et d’un service il fait une vexation. Il faut qu’il chasse et soit seul à chasser ; c’est pour lui un besoin du corps et en même temps un signe de race. Un Rohan, un Dillon courent le cerf même quand ils sont d’Église, malgré les édits et malgré les canons. « Vous chassez beaucoup, Monsieur l’Évêque, disait Louis XV\footnote{Beugnot, {\itshape Mémoires}, I, 35.} à ce dernier ; j’en sais quelque chose. Comment voulez-vous interdire la chasse à vos curés, si vous passez votre vie à leur en donner l’exemple   Sire, pour mes curés la chasse est leur défaut ; pour moi, c’est le défaut de mes ancêtres. » — Lorsque l’amour-propre de caste monte ainsi la garde autour d’un droit, c’est avec une vigilance intraitable. À cet effet leurs capitaines de chasse, veneurs, gardes forestiers, gruyers, protègent les bêtes comme si elles étaient des hommes, et poursuivent les hommes comme s’ils étaient des bêtes. Dans le bailliage de Pont-l’Évêque, en 1789, on cite quatre exemples « d’assassinats récents commis par les gardes-chasses de Mme d’A., de Mme N., d’un prélat et d’un maréchal de France sur des roturiers pris en délit de chasse ou de port d’arme. Tous les quatre jouissent publiquement de l’impunité ». Dans l’Artois, une paroisse déclare que, « sur le territoire de la châtellenie, le gibier dévore tous les avêtis et que les cultivateurs se verront forcés d’abandonner leur exploitation ». Près de là, à Rumancourt, à Bellone, « les lièvres, les lapins, les perdrix dévorent entièrement les avêtis, le comte d’Oisy ne chassant pas et ne faisant pas chasser ». Dans vingt villages circonvoisins d’Oisy où il chasse, c’est à cheval et à travers les récoltes. « Ses gardes toujours armés ont tué plusieurs personnes, sous prétexte de veiller à la conservation des droits de leur maître… Le gibier, qui excède de beaucoup celui des capitaineries royales, mange chaque année l’espoir de la récolte, vingt mille razières de blé et autant d’autres grains. » Dans le bailliage d’Évreux, « le gibier vient tout détruire jusqu’au pied des maisons… À cause du gibier, le citoyen n’est pas même libre dans le cours de l’été d’aller retirer les mauvaises herbes qui étouffent le grain et qui gâtent les semences… Combien de femmes restées sans mari et d’enfants sans père pour un malheureux lièvre ou lapin ! » Les gardes de la forêt de Gouffern en Normandie « sont si terribles, qu’ils maltraitent, insultent et tuent les hommes… Je connais des fermiers qui, ayant plaidé contre la dame pour se faire indemniser de la perte de leurs blés, ont perdu leur temps, leur moisson, et les frais du procès… On voit des cerfs et des biches errer auprès de nos maisons en plein jour ». Dans le bailliage de Domfront, « les habitants de plus de dix paroisses sont obligés de veiller la nuit entière pendant plus de six mois de l’année pour la conservation de leurs moissons\footnote{ \noindent Boivin-Champeaux, {\itshape ib.}, 48. — Renauldon, 26, 416. — Procès-verbaux manuscrits des États généraux ({\itshape Archives nationales}), t. CXXXII, 896 et 901  Hippeau, {\itshape le Gouvernement de Normandie}, VII, 61, 74  Paris, {\itshape la Jeunesse de Robespierre}, 314 à 324  {\itshape Essai sur les capitaineries royales et autres} (1789), passim  L. de Loménie, \href{http://gallica.bnf.fr/ark:/12148/bpt6k2026918/f138.table}{\dotuline{{\itshape Beaumarchais et son} rend="internet\_link"{\itshape temps.} I. 125}} [\url{http://gallica.bnf.fr/ark:/12148/bpt6k2026918/f138.table}]. Beaumarchais, ayant acheté la charge de lieutenant général des chasses aux bailliages de la garenne du Louvre (douze à quinze lieues de rayon), jugeait à ce titre les délinquants. Le 15 juillet 1766, il condamne Ragondet, fermier, à cent livres d’amende et à démolir ses murs de clôture et son hangar, nouvellement bâtis sans autorisation, comme pouvant gêner les plaisirs du roi.
 } »  Voilà l’effet du droit de chasse en province. Mais c’est dans l’Ile-de-France, où les capitaineries abondent et vont s’élargissant, que le spectacle en est le plus lamentable. Un procès-verbal prouve que dans la seule paroisse de Vaux, près de Meulan, les lapins des garennes voisines ont ravagé huit cents arpents cultivés et détruit une récolte de deux mille quatre cents setiers, c’est-à-dire la nourriture annuelle de huit cents personnes. Près de là, à la Rochette, des troupes de biches et de cerfs, pendant le jour, dévorent tout dans les champs et, la nuit, viennent jusque dans les petits jardins des habitants manger les légumes et briser les jeunes arbres. Impossible dans un territoire soumis à la capitainerie de récolter des légumes, sauf dans des jardins clos de hautes murailles. À Farcy, de cinq cents pêchers plantés dans une vigne et broutés par les cerfs, il n’en reste pas vingt au bout de trois ans. Sur tout le territoire de Fontainebleau, les communautés, pour sauver leurs vignes, sont obligées d’entretenir, et encore sauf l’agrément de la capitainerie, des messiers qui, avec des chiens autorisés, veillent et font tintamarre, du soleil couchant au soleil levant, et du 1\textsuperscript{er} mai à la mi-octobre. À Chartrettes, les bêtes fauves, traversant la Seine, viennent détruire chez la comtesse de La Rochefoucauld toutes les plantations de peupliers. Un domaine, affermé deux mille livres, n’est plus loué que quatre cents livres depuis l’établissement de la capitainerie de Versailles. Bref, onze régiments de cavalerie ennemie, cantonnés dans les onze capitaineries voisines de la capitale, et allant tous les matins au fourrage, ne feraient pas plus de dégâts  Il ne faut pas s’étonner si, aux approches de ces repaires, on se dégoûte de la culture\footnote{ \noindent Marquis d’Argenson, {\itshape Mémoires}, éd. Rathery, 21 janvier 1757. « Le sieur de Montmorin, capitaine des chasses de Fontainebleau, tire de sa place des sommes immenses et se conduit en vrai brigand. Les habitants de plus de cent villages voisins ne sèment plus leurs terres, les fruits et graines étant mangés par les biches, cerfs et autre gibier. Ils ont seulement quelques vignes, qu’ils gardent six mois de l’année en faisant des factions et gardes jour et nuit avec tambours et charivari pour faire fuir les bêtes destructives. »\par
 23 janvier 1753  « M. le prince de Conti s’est fait une capitainerie de onze lieues autour de l’Isle-Adam où tout le monde est vexé. »\par
 25 septembre 1753  « Depuis que M. le duc d’Orléans jouit de Villers-Cotterets, il en a fait revivre la capitainerie, et il y a plus de soixante terres à vendre à cause de ces vexations de princes. »
 }. Près de Fontainebleau et de Melun, à Bois-le-Roi, les trois quarts du territoire restent en friche ; presque toutes les maisons de Brolle sont en ruines, on n’y voit plus que des pignons demi-écroulés ; aux Coutilles et à Chapelle-Rablay, cinq fermes sont abandonnées ; à Arbonne, quantité de champs sont délaissés ; à Villiers et à Dame-Marie, où il y avait quatre corps de ferme et nombre de cultures particulières, huit cents arpents demeurent incultes  Chose étrange, à mesure que le siècle va s’adoucissant, le régime de la chasse empire ; les officiers de la capitainerie font du zèle, parce qu’ils travaillent sous les yeux et pour les « plaisirs » du maître. En 1789, cent huit remises viennent d’être plantées dans un seul canton de la capitainerie de Fontainebleau et malgré les propriétaires. Par le règlement de 1762, il est interdit à tout particulier domicilié dans l’étendue d’une capitainerie d’enclore son héritage et tout terrain quelconque de murs, haies ou fossés, sans une permission spéciale\footnote{Les vieux paysans avec qui j’ai causé autrefois dans le pays ont gardé la vive impression de ces vexations et de ces ravages  Dans le Clermontois, ils racontent que les gardes du prince de Condé au printemps prenaient des portées de loups et nourrissaient les jeunes loups dans les fossés du château. On les lâchait au commencement de l’hiver, et l’équipage du loup leur donnait la chasse. Mais ils mangeaient les moutons, et, par-ci par-là, un enfant.}. En cas de permission, il doit laisser dans sa clôture un large espace vide et uni pour que la chasse puisse passer à son aise. Il ne peut avoir chez lui aucun furet, aucune arme à feu, aucun engin propre à la chasse, ni se faire suivre d’un chien même impropre à la chasse, à moins que ce chien ne soit tenu en laisse ou n’ait un billot au cou. Bien mieux, on lui défend de faucher son pré ou sa luzerne avant la Saint-Jean, d’entrer dans son propre champ du 1\textsuperscript{er} mai au 24 juin, d’aller dans les îles de la Seine, d’y couper de l’herbe ou de l’osier, même si l’herbe et l’osier sont à lui ; c’est qu’à ce moment les perdrix couvent, et que le législateur les protège ; il aurait moins d’égards pour une femme en couches ; les vieux chroniqueurs diraient de lui comme de Guillaume Rufus que ses entrailles sont paternelles seulement pour les bêtes. Or il y a en France quatre cent lieues carrées de pays soumises au régime des capitaineries\footnote{Le domaine du roi comprend en bois un million d’arpents, sans compter les bois situés dans les apanages ou affectés aux usines et aux salines. (Necker, {\itshape Compte rendu}, II, 56.)}, et, par toute la France, le gibier, grand ou petit, est le tyran du paysan. Concluez ou plutôt écoutez comment conclut le peuple. « Chaque fois, dit M. de Montlosier en 1789\footnote{ Montlosier, {\itshape Mémoires}, I, 175.
 }, qu’il m’arrivait de rencontrer des troupeaux de cerfs ou de daims sur ma route, mes guides de s’écrier aussitôt : Voilà la noblesse ! par allusion aux ravages que ces animaux faisaient dans leurs terres. » Ainsi, aux yeux de leurs sujets, ils sont des bêtes fauves. — Voilà où conduit le privilège détaché du service ; c’est ainsi qu’un devoir de protection dégénère en un droit de dévastation, et que des gens humains et raisonnables agissent, sans y penser, en gens déraisonnables et inhumains. Séparés du peuple, ils abusent de lui ; chefs nominaux, ils ont désappris l’office de chefs effectifs ; ayant perdu leur caractère public, ils ne rabattent rien de leurs avantages privés. C’est tant pis pour le canton et tant pis pour eux-mêmes. Les trente ou quarante braconniers qu’ils poursuivent aujourd’hui sur leurs terres marcheront demain contre leur château à la tête de l’émeute. — Absence des maîtres, apathie des provinces, mauvais état des cultures, exactions des fermiers, corruption des justices, vexations des capitaineries, oisiveté, dettes et exigences du seigneur, abandon, misère, sauvagerie et hostilité des vassaux, tout cela vient de la même cause et aboutit au même effet. Quand la souveraineté se transforme en sinécure, elle devient lourde sans rester utile, et, quand elle est lourde sans être utile, on la jette à bas.
\chapterclose


\chapteropen

\chapter[{Chapitre IV. Services généraux que doivent les privilégiés.}]{Chapitre IV. \\
Services généraux que doivent les privilégiés.}


\chaptercont

\section[{I. Exemple en Angleterre. — Les privilégiés ne rendent pas ces services en France. — Influence et droits qui leur restent. — Ils ne s’en servent que pour eux-mêmes.}]{I. Exemple en Angleterre. — Les privilégiés ne rendent pas ces services en France. — Influence et droits qui leur restent. — Ils ne s’en servent que pour eux-mêmes.}

\noindent Inutiles dans le canton, ils pourraient être utiles au centre, et, sans prendre part au gouvernement local, servir dans le gouvernement général. Ainsi fait un lord, un baronnet, un squire, même lorsqu’il n’est pas {\itshape justice} dans son comté ou membre d’une commission dans sa paroisse. Député élu à la chambre basse, membre héréditaire de la chambre haute, il tient les cordons de la bourse publique et empêche le prince d’y puiser trop avant. Tel est le régime dans les pays où les seigneurs féodaux, au lieu de laisser le roi s’allier contre eux avec les communes, se sont alliés avec les communes contre le roi. Pour mieux défendre leurs propres intérêts, ils ont défendu les intérêts des autres, et, après avoir été les représentants de leurs pareils, ils sont devenus les représentants de la nation. — Rien de semblable en France. Les États généraux sont tombés en désuétude, et le roi peut avec vérité se dire l’unique représentant du pays. Pareils à des arbres étouffés par l’ombre d’un chêne gigantesque, les autres pouvoirs publics ont péri de sa croissance ; ce qu’il en reste encombre aujourd’hui la place et forme autour de lui un cercle de broussailles rampantes ou de troncs desséchés. L’un d’eux, le Parlement, simple rejeton sorti du grand chêne, a cru parfois posséder une racine propre ; mais sa sève était trop visiblement empruntée pour qu’il pût se tenir debout par lui-même et fournir au peuple un abri indépendant. D’autres corps, survivants quoique rabougris, l’Assemblée du clergé et les États provinciaux, protègent encore un ordre et quatre ou cinq provinces ; mais cette protection ne couvre que l’ordre ou la province, et, si elle défend un intérêt partiel, c’est d’ordinaire contre un intérêt général.

\section[{II. Assemblées du clergé. — Elles ne servent que l’intérêt ecclésiastique. — Le clergé exempté de l’impôt. — Sollicitations de ses agents. — Son zèle contre les protestants.}]{II. Assemblées du clergé. — Elles ne servent que l’intérêt ecclésiastique. — Le clergé exempté de l’impôt. — Sollicitations de ses agents. — Son zèle contre les protestants.}

\noindent Regardons le plus vivace et le mieux enraciné de ces corps, l’Assemblée du clergé. Tous les cinq ans elle se réunit, et, dans l’intervalle, deux agents choisis par elle veillent aux intérêts de l’ordre. Convoquée par le gouvernement, dirigée par lui, contenue ou interrompue au besoin, toujours sous sa main, employée par lui à des fins politiques, elle reste néanmoins un asile pour le clergé qu’elle représente. Mais elle n’est un asile que pour lui, et, dans la série de transactions par lesquelles elle se défend contre le fisc, elle ne décharge ses épaules que pour rejeter un fardeau plus lourd sur les épaules d’autrui. On a vu comment sa diplomatie a sauvé les immunités du clergé, comment elle l’a racheté de la capitation et des vingtièmes, comment elle a changé sa part d’impôt en un « don gratuit », comment chaque année elle applique ce don au remboursement des capitaux empruntés pour son rachat, par quel art délicat elle est parvenue, non seulement à n’en rien verser dans le Trésor, mais encore à soutirer chaque année du Trésor environ 1 500 000 livres ; c’est tant mieux pour l’Église, mais tant pis pour le peuple. — Maintenant parcourez la file des in-folios où se suivent de cinq ans en cinq ans les rapports des agents, hommes habiles et qui se préparent ainsi aux plus hauts emplois de l’Église, les abbés de Boisgelin, de Périgord, de Barrai, de Montesquiou ; à chaque instant, grâce à leurs sollicitations auprès des juges et du Conseil, grâce à l’autorité que donne à leurs plaintes le mécontentement de l’ordre puissant que l’on sent derrière eux, quelque affaire ecclésiastique est décidée dans le sens ecclésiastique ; quelque droit féodal est maintenu en faveur d’un chapitre ou d’un évêque ; quelque réclamation du public est rejetée\footnote{{\itshape Rapport de l’agence du clergé de} 1775 {\itshape à} 1780, 31 {\itshape et} 34. — {\itshape Id.} de 1780 à 1783, 257.}. En 1781, malgré un arrêté du Parlement de Rennes, les chanoines de Saint-Malo sont maintenus dans le monopole de leur four banal, au détriment des boulangers qui voudraient cuire à domicile et des habitants qui payeraient moins cher le pain cuit chez les boulangers. En 1773, Guénin, maître d’école, destitué par l’évêque de Langres et vainement soutenu par les habitants, est forcé de laisser sa place au successeur que le prélat lui a nommé d’office. En 1770, Rastel, protestant, ayant ouvert une école publique à Saint-Affrique, est poursuivi à la demande de l’évêque et des agents du clergé ; on ferme son école et on le met en prison. — Quand un corps a gardé dans sa main les cordons de sa bourse, il obtient bien des complaisances ; elles sont l’équivalent de l’argent qu’il accorde. Le ton commandant du roi, l’air soumis du clergé ne changent rien au fond des choses ; entre eux, c’est un marché\footnote{Lanfrey, \href{http://gallica.bnf.fr/ark:/12148/bpt6k656362}{\dotuline{{\itshape l’Église et les philosophes}}} [\url{http://gallica.bnf.fr/ark:/12148/bpt6k656362}], passim.} : donnant, donnant ; telle loi contre les protestants, en échange d’un ou deux millions ajoutés au don gratuit. C’est ainsi que graduellement s’est faite, au dix-septième siècle, la révocation de l’édit de Nantes, article par article, comme un tour d’estrapade après un autre tour d’estrapade, chaque persécution nouvelle achetée par une largesse nouvelle, en sorte que, si le clergé aide l’État, c’est à condition que l’État se fera bourreau. Pendant tout le dix-huitième siècle, l’Église veille à ce que l’opération continue\footnote{Boiteau, {\itshape État de la France en} 1789, 205, 207. — Marquis d’Argenson, {\itshape Mémoires}, 5 {\itshape mai} 1752, 3, 22, 25 septembre 1753, 17 {\itshape octobre} 1753, 26 octobre 1755. — Prudhomme, {\itshape Résumé général des cahiers des États généraux}. 1789 ({\itshape Cahiers du clergé}). — {\itshape Histoire des églises du désert, par} Charles Coquerel, I, 151 {\itshape et} suivantes.}. En 1717, une assemblée de soixante-quatorze personnes ayant été surprise à Anduze, les hommes vont aux galères et les femmes en prison. En 1724, un édit déclare que tous ceux qui assisteront à une assemblée et tous ceux qui auront quelque commerce direct ou indirect avec les ministres prédicants, seront condamnés à la confiscation des biens, les femmes rasées et enfermées pour la vie, les hommes aux galères perpétuelles. En 1745 et 1746, dans le Dauphiné, deux cent soixante-dix-sept protestants sont condamnés aux galères et nombre de femmes au fouet. De 1744 à 1752, dans l’Est et le Midi, six cents protestants sont enfermés et huit cents condamnés à diverses peines. En 1774, les deux enfants de Roux, calviniste à Nîmes, lui sont enlevés. Jusqu’aux approches de la Révolution, dans le Languedoc, on pend les ministres et l’on envoie des dragons contre les congrégations qui se rassemblent au désert pour prier Dieu ; la mère de M. Guizot y a reçu des coups de feu dans ses jupes ; c’est qu’en Languedoc, par les États provinciaux, « les évêques sont maîtres du temporel plus que partout ailleurs, et que leur sentiment est toujours de dragonner, de convertir à coups de fusil ». En 1775, au sacre, l’archevêque Loménie de Brienne, incrédule connu, dit au jeune roi : « Vous réprouverez les systèmes d’une tolérance coupable… Achevez l’ouvrage que Louis le Grand avait entrepris. Il vous est réservé de porter le dernier coup au calvinisme dans vos États ». En 1780, l’Assemblée du clergé déclare « que l’autel et le trône seraient également en danger, si l’on permettait à l’hérésie de rompre ses fers ». Même en 1789, le clergé dans ses cahiers, tout en consentant à tolérer les non-catholiques, trouve l’édit de 1788 trop libéral ; il veut qu’on les exclue des charges de judicature, qu’on ne leur accorde jamais l’exercice public de leur culte, et qu’on interdise les mariages mixtes ; bien plus, il demande la censure préalable de tous les ouvrages de librairie, un comité ecclésiastique pour les dénoncer, et des peines infamantes contre les auteurs de livres irreligieux ; enfin, il réclame pour lui-même la direction des écoles publiques et la surveillance des écoles privées  Rien d’étrange dans cette intolérance et dans cet égoïsme. Un corps, comme un individu, pense d’abord et surtout à lui. Si parfois il sacrifie quelque chose de son privilège, c’est pour s’assurer l’alliance des autres corps. En ce cas, qui est celui de l’Angleterre, tous ces privilèges qui transigent entre eux et se soutiennent les uns les autres composent par leur réunion les libertés publiques  Ici, un seul corps étant représenté, ses députés ne sont ni chargés, ni tentés de rien concéder aux autres ; son intérêt est leur seul guide ; ils lui subordonnent l’intérêt général, et le servent à tout prix, même par des attentats publics.

\section[{III. Influence des nobles. — Règlements en leur faveur. — Préférence qu’ils obtiennent dans l’Église. — Distribution des évêchés et des abbayes. — Préférence qu’ils obtiennent dans l’État. — Gouvernements, offices, sinécures, pensions, gratifications. — Au lieu d’être utiles, ils sont à charge.}]{III. Influence des nobles. — Règlements en leur faveur. — Préférence qu’ils obtiennent dans l’Église. — Distribution des évêchés et des abbayes. — Préférence qu’ils obtiennent dans l’État. — Gouvernements, offices, sinécures, pensions, gratifications. — Au lieu d’être utiles, ils sont à charge.}

\noindent Ainsi travaillent les corps quand, au lieu d’être associés, ils sont séparés. Même spectacle, si l’on regarde les castes et les coteries ; leur isolement fait leur égoïsme. Du bas en haut de l’échelle, les pouvoirs légaux ou moraux qui devraient représenter la nation ne représentent qu’eux-mêmes, et chacun d’eux s’emploie pour soi au détriment de la nation  À défaut du droit de s’assembler et de voter, la noblesse a son influence, et, pour savoir comment elle en use, il suffit de lire les édits de l’almanach. Un règlement imposé au maréchal de Ségur\footnote{Comte de Ségur, \href{http://gallica.bnf.fr/ark:/12148/bpt6k29204g}{\dotuline{{\itshape Mémoires}, I}} [\url{http://gallica.bnf.fr/ark:/12148/bpt6k29204g}], 16, 41. — Bouillé, {\itshape Mémoires}, 34. — Mme Campan, {\itshape Mémoires}, I, 237 (détails à l’appui).} vient de relever la vieille barrière qui excluait les roturiers des grades militaires, et désormais, pour être capitaine, il faudra prouver quatre degrés de noblesse. Pareillement, dans les derniers temps, il faut être noble pour être reçu maître des requêtes, et l’on décide secrètement qu’à l’avenir « tous les biens ecclésiastiques, depuis le plus modeste prieuré jusqu’aux plus riches abbayes, seront réservés à la noblesse ». — De fait, toutes les grandes places, ecclésiastiques ou laïques, sont pour eux ; toutes les sinécures, ecclésiastiques ou laïques, sont pour eux, ou pour leurs parents, alliés, protégés et serviteurs. La France ressemble à une vaste écurie où les chevaux de race auraient double et triple ration pour être oisifs ou ne faire que demi-service, tandis que les chevaux de trait font le plein service avec une demi-ration qui leur manque souvent. Encore faut-il noter que, parmi ces chevaux de race, il est un troupeau privilégié qui, né auprès du râtelier, écarte ses pareils et mange à pleine bouche, gras, brillant, le poil poli et jusqu’au ventre en la litière, sans autre occupation que de toujours tirer à soi. Ce sont les nobles de cour, qui vivent à portée des grâces, exercés dès l’enfance à demander, obtenir et demander encore, uniquement attentifs aux faveurs et aux froideurs royales, pour qui l’Œil-de-bœuf compose l’univers, « indifférents aux affaires de l’État comme à leurs propres affaires, laissant gouverner les unes par les intendants de province, comme ils laissent gouverner les autres par leurs propres intendants ».\par
Voyons-les à l’œuvre sur le budget. On sait combien celui de l’Église est large ; j’estime qu’ils en prélèvent au moins la moitié. Dix-neuf chapitres nobles d’hommes, vingt-cinq chapitres nobles de femmes, deux cent soixante commanderies de Malte, sont à eux par institution. Ils occupent par faveur tous les archevêchés, et, sauf cinq, tous les évêchés\footnote{{\itshape La France ecclésiastique}, 1788.}. Sur quatre abbés commendataires et vicaires généraux, ils en fournissent trois. Si, parmi les abbayes de femmes à nomination royale, on relève celles qui rapportent 20 000 livres et au-delà, on trouve qu’elles ont toutes pour abbesses des demoiselles. Un seul détail pour montrer l’étendue des grâces : j’ai compté quatre-vingt-trois abbayes d’hommes possédées par des aumôniers, chapelains, précepteurs ou lecteurs du roi, de la reine, des princes et princesses ; l’un d’eux, l’abbé de Vermond, a 80 000 livres de rente en bénéfices. Bref, grosses ou petites, les quinze cents sinécures ecclésiastiques à nomination royale sont une monnaie à l’usage des grands, soit qu’ils la versent en pluie d’or pour récompenser l’assiduité de leurs familiers et de leurs gens, soit qu’ils la gardent en larges réservoirs pour soutenir la dignité de leur rang. Du reste, selon la coutume de donner plus à qui plus a, les plus riches prélats ont, par-dessus leurs revenus épiscopaux, les plus riches abbayes. D’après l’almanach, M. d’Argentré, évêque de Séez\footnote{Granier de Cassagnac, {\itshape Des causes de la Révolution française}, III, 58.}, se fait ainsi en supplément 34 000 livres de rente ; M. de Suffren, évêque de Sisteron, 36 000 ; M. de Girac, évêque de Rennes, 40 000 ; M. de Bourdeille, évêque de Soissons, 42 000 ; M. d’Agout de Bonneval, évêque de Pamiers, 45 000 ; M. de Marbeuf, évêque d’Autun, 50 000 ; M. de Rohan, évêque de Strasbourg, 60 000 ; M. de Cicé, archevêque de Bordeaux, 63 000 ; M. de Luynes, archevêque de Sens, 82 000 ; M. de Bernis, archevêque d’Alby, 100 000 ; M. de Brienne, archevêque de Toulouse, 106 000 ; M. de Dillon, archevêque de Narbonne, 120 000 ; M. de La Rochefoucauld, archevêque de Rouen, 130 000 : c’est-à-dire le double et parfois le triple en sommes perçues, le quadruple et parfois le sextuple en valeurs d’aujourd’hui. M. de Rohan tirait de ses abbayes, non pas 60 000 livres, mais 400 000, et M. de Brienne, le plus opulent de tous après M. de Rohan, le 24 août 1788, au moment de quitter le ministère\footnote{Marmontel, {\itshape Mémoires}, II, liv. XIII, 221.}, envoyait prendre au « Trésor les 20 000 livres de son mois qui n’était pas encore échu, exactitude d’autant plus remarquable, que, sans compter les appointements de sa place et les 6 000 livres de pension attachées à son cordon bleu, il possédait en bénéfices 678 000 livres de rente, et que, tout récemment encore, une coupe de bois dans une de ses abbayes lui avait valu un million ».\par
Passons au budget laïque ; là aussi les sinécures abondent et sont presque toutes à la noblesse. De ce genre, sont en province les trente-sept grands gouvernements généraux, les sept petits gouvernements généraux, les soixante-six lieutenances générales, les quatre cent sept gouvernements particuliers, les treize gouvernements de maisons royales, et nombre d’autres, tous emplois vides et de parade, tous entre des mains nobles, tous lucratifs, non seulement par les appointements du Trésor, mais aussi par les profits locaux. Ici encore la noblesse s’est laissé dérober l’autorité, l’action, l’utilité de sa charge, à condition d’en garder le titre, la pompe et l’argent\footnote{ \noindent Boiteau, {\itshape État de la France en} 1789, 55, 248. — Marquis d’Argenson, {\itshape Considérations sur le gouvernement de la France}, 177. — Duc de Luynes, {\itshape Journal}, XIII, 226 ; XIV, 287 ; XIII, 33, 158, 162, 218, 233, 237 ; XV, 268 ; XVI, 304. — Le gouvernement de Ham vaut 11 250 livres, celui d’Auxerre 12 000, celui de Briançon 12 000, celui des îles Sainte-Marguerite 16 000, celui de Schelestadt 15 000, celui de Brisach de 15 à 16 000, celui de Gravelines 18 000. — L’ordonnance de 1776 avait réduit ainsi ces diverses places (Waroquier, II, 467) : 18 gouvernements généraux à 60 000 livres, 21 à 30 000, 114 gouvernements particuliers, dont 25 à 12 000 livres, 25 à 10 000, 64 à 8 000, 176 lieutenants et commandants de villes, places, etc., dont 35 de 6 000 à 16 000, et 141 de 2 000 à 6 000. L’ordonnance de 1788 établit en outre 17 commandants en chef ayant de 20 000 à 30 000 livres de fixe, et de 4 000 à 6 000 par mois de résidence, et des commandants en second.
 }. C’est l’intendant qui gouverne ; « le gouverneur en titre ne peut remplir aucune fonction sans lettres particulières de commandement » ; il n’est là que pour donner à dîner ; encore lui faut-il pour cela une permission, « la permission d’aller résider dans son gouvernement ». Mais la place est fructueuse : le gouvernement général du Berry vaut 35 000 livres de rente, celui de la Guyenne 120 000, celui du Languedoc 160 000 ; un petit gouvernement particulier, comme celui du Havre, rapporte 35 000 livres, outre les accessoires ; une médiocre lieutenance générale, comme celle du Roussillon, 13 000 à 14 000 livres ; un gouvernement particulier, de 12 000 à 18 000 livres ; et notez que, dans la seule Ile-de-France, il y en a trente-quatre, à Vervins, Senlis, Melun, Fontainebleau, Dourdan, Sens, Limours, Etampes, Dreux, Houdan et autres villes aussi médiocres que pacifiques ; c’est l’état-major des Valois qui depuis Richelieu a cessé de servir, mais que le Trésor paye toujours  Considérez ces sinécures dans une seule province, en Languedoc, pays d’États, où il semble que la bourse du contribuable doive être mieux défendue. Il y a trois sous-commandants à Tournon, Alais et Montpellier, « chacun payé 16 000 livres, quoiqu’ils soient sans fonctions, puisqu’ils n’ont été établis que dans un temps de troubles et de guerres de religion, pour contenir les protestants ». Douze lieutenants du roi sont également inutiles et pour la montre. De même les trois lieutenants généraux : chacun d’eux « reçoit, à tour de rôle et tous les trois ans, une gratification de 30 000 livres, pour services rendus à cette même province, lesquels sont vains et chimériques, et qu’on ne spécifie pas » ; car aucun deux ne réside, et, si on les paye, c’est pour avoir leur appui en cour. « Ainsi, M. le comte de Caraman, qui a plus de 600 000 livres de rente comme propriétaire du canal du Languedoc, reçoit 30 000 livres tous les trois ans sans cause légitime, et indépendamment des dons fréquents et abondants que la province lui fait pour les réparations de son canal. » — La province donne aussi au commandant comte de Périgord une gratification de 12 000 livres en sus de ses appointements, et à sa femme une autre gratification de 12 000 livres, lorsque pour la première fois elle honore les États de sa présence. Elle paye encore au même commandant quarante gardes, « dont vingt-quatre seulement servent pendant sa courte présence aux États », et qui, avec leur capitaine, coûtent par an 15 000 livres. Elle paye de même au gouverneur de quatre-vingts à cent gardes « qui reçoivent chacun 300 ou 400 livres, outre beaucoup d’exemptions, et ne sont jamais en fonctions puisque le gouverneur ne réside jamais » ; pour ces fainéants subalternes la dépense est de 24 000 livres, outre 5 000 à 6 000 pour leur capitaine, à quoi il faut ajouter 7 500 pour les secrétaires du gouverneur, outre 60 000 livres d’appointements et des profits infinis pour le gouverneur lui-même. Je vois partout des oisifs secondaires pulluler à l’ombre des oisifs en chef et puiser leur sève dans la bourse publique qui est la commune nourrice. Tout ce monde parade, boit et mange copieusement, en cérémonie : tel est leur principal emploi, et ils s’en acquittent en conscience. Les tenues d’États sont des bombances de six semaines, où l’intendant dépense 25 000 livres en dîners et réceptions\footnote{{\itshape Archives nationales}, H, 944, 25 avril et 20 septembre 1780, Lettres et Mémoires de M. Furgole, avocat à Toulouse.}.\par
Aussi lucratives et aussi inutiles sont les charges de cour\footnote{{\itshape Archives nationales}, O1, 738 (Rapports faits au bureau général des dépenses de la maison du roi en mars 1780, par M. Mesnard de Chouzy)  Augeard, \href{http://gallica.bnf.fr/ark:/12148/bpt6k46720s}{\dotuline{{\itshape Mémoires}}} [\url{http://gallica.bnf.fr/ark:/12148/bpt6k46720s}], 97  Mme Campan, {\itshape Mémoires}, I, 291  Marquis d’Argenson, {\itshape Mémoires}, 10 février, 9 décembre 1751  {\itshape Essai sur les capitaineries royales et autres} (1789), 80  Waroquier, {\itshape État de la France en} 1789, I, 266.}, sinécures domestiques dont les profits et accessoires dépassent de beaucoup les émoluments. Je trouve dans l’état imprimé 295 officiers de bouche sans compter les garçons pour la table du roi et de ses gens, et « le premier maître d’hôtel jouit de 84 000 livres par an en billets et en nourritures », sans compter ses appointements et les « grandes livrées » qu’il touche en argent. Les premières femmes de chambre de la reine inscrites sur l’Almanach pour 150 livres et payées 12 000 francs, se font en réalité 50 000 francs par la revente des bougies allumées dans la journée ; Augeard, secrétaire des commandements et dont la place est marquée 900 livres par an, avoue qu’elle lui en vaut 200 000. Le capitaine des chasses, à Fontainebleau, vend à son profit chaque année pour 20 000 francs de lapins. « Dans chaque voyage aux maisons de campagne du roi, les dames d’atour, sur les frais de déplacement, gagnent 80 pour 100 ; on dit que le café au lait avec un pain à chacune de ces dames coûte 2 000 francs par an, et ainsi du reste. » — « Mme de Tallard s’est fait 115 000 livres de rente dans sa place de gouvernante des enfants de France, parce que, à chaque enfant, ses appointements augmentent de 35 000 livres. » Le duc de Penthièvre, en qualité de grand-amiral, perçoit sur tous les navires « qui entrent dans les ports et embouchures de France » un droit d’ancrage, dont le produit annuel est de 91 484 francs. Mme de Lamballe, surintendante, inscrite pour 6 000 francs, en touche 150 000\footnote{{\itshape Marie-Antoinette}, par Arneth et Geffroy, II, 377.}. Sur un seul feu d’artifice, le duc de Gesvres gagne 50 000 écus par les débris et charpentes qui lui appartiennent en vertu de sa charge\footnote{Mme Campan, {\itshape Mémoires}, I, 296, 298, 300, 301 ; III, 78  Hippeau, {\itshape le Gouvernement de Normandie.} IV, 171 (Lettre de Paris, du 13 décembre 1780)  Marquis d’Argenson, {\itshape Mémoires}, 5 septembre 1755  Bachaumont, 16 janvier 1758  {\itshape Mémoire sur l’imposition territoriale}, par M. de Calonne (1787), 54.}  Grands officiers du palais, gouverneurs des maisons royales, capitaines des capitaineries, chambellans, écuyers, gentilhommes servants, gentilshommes ordinaires, pages, gouverneurs, aumôniers, chapelains, dames d’honneur, dames d’atour, dames pour accompagner, chez le roi, chez la reine, chez Monsieur, chez Madame, chez le comte d’Artois, chez la comtesse d’Artois, chez Mesdames, chez Madame Royale, chez Madame Élisabeth, dans chaque maison princière et ailleurs, des centaines d’offices pourvus d’appointements et d’accessoires sont sans fonctions ou ne servent que pour le décor. « Mme de la Borde vient d’être nommée garde du lit de la reine avec 12 000 francs de pension sur la cassette du roi ; on ignore quelles sont les fonctions de cette charge, qui n’a pas existé depuis Anne d’Autriche. » Le fils aîné de M. de Machault est nommé intendant des classes. C’est un de ces emplois dits gracieux : cela vaut « 18 000 livres de rente pour signer son nom deux fois par an ». De même la place de secrétaire général des Suisses valant 30 000 livres de rente et donnée à l’abbé Barthélemy ; de même la place de secrétaire général des dragons, valant 20 000 livres par an, occupée tour à tour par Gentil Bernard et par Laujon, deux petits poètes de poche  Il serait plus simple de donner l’argent sans la place ; en effet on n’y manque pas ; quand on lit jour par jour les Mémoires, il semble que le Trésor soit une proie. Assidus auprès du roi, les courtisans le font compatir à leurs peines. Ils sont ses familiers, les hôtes de son salon, des gens de race comme lui, ses clients naturels, les seuls avec lesquels il cause et qu’il ait besoin de voir contents ; il ne peut s’empêcher de les assister. Il faut bien qu’il contribue à doter leurs enfants, puisqu’il signe au contrat ; il faut bien qu’il les enrichisse eux-mêmes, puisque leur luxe sert à la décoration de sa cour. La noblesse étant un ornement du trône, c’est au possesseur du trône à le redorer aussi souvent qu’il le faudra\footnote{ Marquis d’Argenson, {\itshape Mémoires}, 9 décembre 1751.\par
 « La dépense que font les gens de cour pour avoir deux habits neufs et magnifiques, chacun pour les deux jours de fête et cela par ordre du roi, achève de les ruiner. »
 }. Là-dessus quelques chiffres et anecdotes pris, entre mille, sont d’une rare éloquence\footnote{ \noindent Duc de Luynes, {\itshape Journal}, XIV, 147, 295 ; XV, 36, 119. — Marquis d’Argenson, {\itshape Mémoires}, 8 avril 1752, 30 mars et 28 juillet 1753, 23 juin 1755. — Hippeau, {\itshape ib}., IV, 153 (Lettre du 15 mai 1780). — Necker, {\itshape De l’administration des finances}, II, 265, 269, 270, 271, 282. — Augeard, {\itshape Mémoires}, 249.
 }  « M. le prince de Pons avait 25 000 livres de pension des bienfaits du roi, sur quoi Sa Majesté avait bien voulu en donner 6 000 à Mlle de Marsan, sa fille, chanoinesse de Remiremont. La famille a représenté au roi le mauvais état des affaires de M. le prince de Pons, et Sa Majesté a bien voulu accorder à M. le prince Camille, son fils, 15 000 livres de la pension vacante par la mort de son père, et 5 000 livres d’augmentation à Mme de Marsan. » — M. de Conflans épouse Mlle Portail : « En faveur de ce mariage, le roi a bien voulu que, sur la pension de 10 000 livres accordée à Mme la présidente Portail, il en passât 6 000 à M. de Conflans après la mort de Mme Portail. » — M. de Séchelles, ministre qui se retire, « avait 12 000 livres d’ancienne pension que le roi lui conserve ; il a, outre cela, 20 000 livres de pension comme ministre ; et le roi lui donne encore outre cela 40 000 livres de pension »  Parfois les motifs de la grâce sont admirables. Il faut consoler M. Rouillé de n’avoir pas participé au traité de Vienne ; c’est pourquoi « on donne une pension de 6 000 livres à sa nièce, Mme de Castellane, et une autre de 10 000 à sa fille, Mme de Beuvron, fort riche »  « M. de Puisieux jouit d’environ 76 ou 77 000 livres de rente des bienfaits du roi ; il est vrai qu’il a un bien considérable ; mais le revenu de ce bien est incertain, étant pour la plupart en vignes. » — « On vient de donner une pension de 10 000 livres à la marquise de Lède parce qu’elle a déplu à Madame Infante et pour qu’elle se retire. » — Les plus opulents tendent la main et prennent. « On a calculé que, la semaine dernière, il y eut pour 128 000 livres de pension données à des dames de la cour, tandis que depuis deux ans on n’a pas donné la moindre pension à des officiers : 8 000 livres à la duchesse de Chevreuse dont le mari a de 4 à 500 000 livres de rente, 12 000 livres à Mme de Luynes pour qu’elle ne soit pas jalouse, 10 000 à la duchesse de Brancas, 10 000 à la duchesse douairière de Brancas, mère de la précédente, etc. » En tête de ces sangsues sont les princes du sang. « Le roi vient de donner un million cinq cent mille livres à M. le prince de Conti pour payer ses dettes, dont un million sous prétexte de le dédommager du tort qu’on lui a fait par la vente d’Orange, et 500 000 livres de grâce. » « M. le duc d’Orléans avait ci-devant 50 000 écus de pension comme pauvre et en attendant la succession de son père. Étant devenu par cet événement riche de plus de trois millions de rente, il a remis sa pension. Mais depuis il a représenté qu’il dépenserait par-delà son revenu, et le roi lui a rendu ses 50 000 écus. » — Vingt ans plus tard, en 1780, quand Louis XVI, voulant soulager le Trésor, signe « la grande réforme de la bouche », « on donne à Mesdames 600 000 livres pour leur table » ; rien qu’en dîners, voilà ce que trois vieilles dames, en se retranchant, coûtent au public. Pour les deux frères du roi, 8 300 000 livres, outre deux millions de rente en apanages ; pour le Dauphin, Madame Royale, Madame Élisabeth et Mesdames, 3 500 000 ; pour la reine, quatre millions ; voilà le compte de Necker en 1784. Joignez à cela les dons de la main à la main avoués ou déguisés : 200 000 francs à M. de Sartine pour l’aider à payer ses dettes, 200 000 à M. de Lamoignon, garde des sceaux, 600 000 francs à M. de Miromesnil pour frais d’établissement, 166 000 à la veuve de M. de Maurepas, 500 000 au prince de Salm, 1 200 000 au duc de Polignac pour l’engagement du comté de Fenestranges, 754 337 à Mesdames pour payer Bellevue\footnote{Nicolardot, {\itshape Journal de Louis XVI}, 228. Sommes ordonnancées dans le Livre Rouge de 1774 à 1789 : 227 985 716 livres, dont 80 millions en acquisitions et dons à la famille du roi. — Entre autres, 14 450 000 livres à Monsieur, 14 600 000 au comte d’Artois. — 7 726 253 pour Saint-Cloud donné à la reine. — 8 700 000 pour acquisition de l’Isle-Adam.}. « M. de Calonne, dit Augeard, témoin compétent\footnote{Cf. {\itshape Compte général des revenus et dépenses fixes au 1\textsuperscript{er} mai} 1789. (Imprimerie royale, 1789, in-4.) « Terre de l’Ile-Dieu, acquise en 1783 du duc de Mortemart, 1 million. — Terre de Viviers, acquise du prince de Soubise en 1784, 1 500 000. — Terres de Saint-Priest et de Saint-Étienne, acquises en 1787 de M. Gilbert des Voisins, 1 335 935. — Forêts de Camors et de Floranges, acquises du duc de Liancourt en 1785, 1 200 000. — Comté de Montgommery, acquis de M. Clément de Barville en 1785, 3 306 604. »}, fit, à peine entré, un emprunt de cent millions, dont un quart n’est pas entré au Trésor royal : le reste a été dévoré par les gens de la cour ; on évalue ce qu’il a donné au comte d’Artois à cinquante-six millions, la part de Monsieur à vingt-cinq millions ; il a donné au prince de Condé, en échange de 300 000 livres de rente, douze millions une fois payés et 600 000 livres de rentes viagères, et il fait faire à l’État les acquisitions les plus onéreuses, des échanges dont la lésion était de plus de 500 pour 100. » N’oublions pas qu’au taux actuel tous ces dons, pensions, appointements valent le double. — Tel est l’emploi des grands auprès du pouvoir central : au lieu de se faire les représentants du public, ils ont voulu être les favoris du prince, et ils tondent le troupeau qu’ils devraient préserver.

\section[{IV. Isolement des chefs. — Sentiments des subordonnés. — La noblesse de province. — Les curés.}]{IV. Isolement des chefs. — Sentiments des subordonnés. — La noblesse de province. — Les curés.}

\noindent À la fin le troupeau écorché découvrira ce qu’on fait de sa laine. « Tôt ou tard\footnote{{\itshape Le président de Brosses}, par Foisset. (Remontrances au roi par le Parlement de Dijon, le 19 janvier 1764.)}, dit un Parlement dès 1764, le peuple apprendra que les débris de nos finances continuent d’être prodigués en dons si souvent peu mérités, en pensions excessives et multipliées sur les mêmes têtes, en dots et assurances de douaires, en places et appointements inutiles. » Tôt ou tard, il repoussera « ces mains avides qui toujours s’ouvrent et ne se croient jamais pleines, ces gens insatiables qui ne semblent nés que pour tout prendre et ne rien avoir, gens sans pitié comme sans pudeur ». — Et ce jour-là les écorcheurs se trouveront seuls. Car le propre d’une aristocratie qui ne songe qu’à soi est de devenir une coterie. Ayant oublié le public, elle néglige par surcroît ses subordonnés ; après s’être séparée de la nation, elle se sépare de sa suite. C’est un état-major en congé qui fait bombance et ne prend plus soin des sous-officiers ; vienne un jour de bataille, personne ne marche après lui, on cherche des chefs ailleurs. Tel est l’isolement des seigneurs de cour et des prélats au milieu de la petite noblesse et du bas clergé ; ils se font la part trop grosse, et ne donnent rien ou presque rien aux gens qui ne sont pas de leur monde. Contre eux, depuis un siècle, un long murmure s’élève et va s’enflant jusqu’à devenir une clameur où l’esprit ancien et l’esprit nouveau, les idées philosophiques grondent à l’unisson. « Je vois, disait le bailli de Mirabeau\footnote{Lucas de Montigny, {\itshape Mémoires de Mirabeau.} Lettre du bailli du 26 mai 1781. — Marquis d’Argenson, {\itshape Mémoires}, IV, 156, 157, 160, 176 ; VI, 320. — Maréchal Marmont, \href{http://gallica.bnf.fr/ark:/12148/bpt6k284352}{\dotuline{{\itshape Mémoires}}} [\url{http://gallica.bnf.fr/ark:/12148/bpt6k284352}], I, 9. — Marquis de Ferrières, \href{http://gallica.bnf.fr/ark:/12148/bpt6k46765r}{\dotuline{{\itshape Mémoires}}} [\url{http://gallica.bnf.fr/ark:/12148/bpt6k46765r}], préface. — Voir sur cette difficulté de parvenir, les \href{http://gallica.bnf.fr/ark:/12148/bpt6k205962c}{\dotuline{{\itshape Mémoires}}} [\url{http://gallica.bnf.fr/ark:/12148/bpt6k205962c}] de Dumouriez. Le père de Chateaubriand est aussi un de ces mécontents, « frondeur politique et grand ennemi de la cour ». (I, 206). — {\itshape Cahiers des États généraux de} 1789, résumé général par Prudhomme, t. II, passim.}, que la noblesse s’avilit et se perd. Elle s’étend sur tous les enfants de sangsues, sur la truandaille de finance, introduits par la Pompadour, sortie elle-même de ces immondices. Une partie va s’avilir dans la servitude de cour ; l’autre se mélange à la canaille plumière qui change en encre le sang des sujets du roi ; l’autre périt étouffée par de viles robes, ignobles atomes de la poussière de cabinet qu’une charge tire de la crasse » ; et tout cela, parvenus d’ancienne ou de nouvelle race, fait une bande qui est la cour. — « La cour ! s’écrie d’Argenson, dans ce mot est tout le mal. La cour est devenue le sénat de la nation ; le moindre valet de Versailles est sénateur ; les femmes de chambre ont part au gouvernement, sinon pour ordonner, du moins pour empêcher les lois et les règles ; et, à force d’empêcher, il n’y a plus ni lois, ni ordres, ni ordonnateurs… Sous Henri IV, les courtisans demeuraient chacun dans leur maison, ils n’étaient point engagés dans des dépenses ruineuses pour être de la cour ; ainsi les grâces ne leur étaient pas {\itshape dues} comme aujourd’hui… La cour est le tombeau de la nation. » — Quantité d’officiers nobles, voyant que les hauts grades ne sont que pour les courtisans, quittent le service et vont porter leur mécontentement dans leurs terres. D’autres, qui ne sont point sortis de leur domaine, y couvent dans la gêne, l’oisiveté et l’ennui leurs ambitions aigries par l’impuissance. En 1789, dit le marquis de Ferrières, la plupart sont « si las de la cour et des ministres qu’ils sont presque des démocrates ». Du moins « ils veulent retirer le gouvernement à l’oligarchie ministérielle entre les mains de laquelle il est concentré ». Point de grands seigneurs pour députés ; ils les écartent et « les rejettent absolument, disant qu’ils trafiqueraient des intérêts de la noblesse » ; eux-mêmes, dans leurs cahiers, ils insistent pour qu’il n’y ait plus de noblesse de cour.\par
Mêmes sentiments dans le bas clergé, et encore plus vifs ; car il est exclu des hautes places, non seulement comme inférieur, mais encore comme roturier\footnote{{\itshape Éphémérides du citoyen}, II, 200, 203. — Voltaire, \href{http://www.voltaire-integral.com/Html/18/cure\_de\_campagne.htm}{\dotuline{{\itshape Dictionnaire philosophique}, article {\itshape Curé de campagne}}} [\url{http://www.voltaire-integral.com/Html/18/cure\_de\_campagne.htm}]. — L’abbé Guettée, {\itshape Histoire de l’Église de France}, XII, 130.}. Déjà en 1766, le marquis de Mirabeau écrivait : « Ce serait faire injure à la plupart de nos ecclésiastiques à prétentions que de leur proposer une cure. Les revenus et les distinctions sont pour les abbés commendataires, pour les bénéficiers à simple tonsure, pour les nombreux chapitres ». Au contraire, « les vrais pasteurs des âmes, les coopérateurs dans le saint ministère ont à peine une subsistance ». La première classe, « tirée de la noblesse et de la bonne bourgeoisie, n’a que les prétentions sans vrai ministère. L’autre, n’ayant que des devoirs à remplir sans espoir et presque sans revenu…, ne peut se recruter que dans les derniers rangs de la société civile, et les parasites qui dépouillent les travailleurs affectent de les subjuguer et de les avilir de plus en plus »  « Je plains, disait Voltaire, le sort d’un curé de campagne obligé de disputer une gerbe de blé à son malheureux paroissien, de plaider contre lui, d’exiger la dîme des pois et des lentilles, de consumer sa misérable vie en querelles continuelles… Je plains encore davantage le curé à portion congrue à qui des moines, nommés gros décimateurs, osent donner un salaire de quarante ducats pour aller faire, pendant toute l’année, à deux ou trois milles de sa maison, le jour, la nuit, au soleil, à la pluie, dans les neiges, au milieu des glaces, les fonctions les plus pénibles et les plus désagréables. » — Depuis trente ans, on a tâché d’assurer et de relever un peu leur salaire ; en cas d’insuffisance, le bénéficier, collateur ou décimateur de la paroisse, doit y ajouter jusqu’à ce que le curé ait 500 livres (1768), puis 700 livres (1785), le vicaire 200 livres (1768), puis 250 (1778), et à la fin 350 (1785). À la rigueur, au prix où sont les choses\footnote{Aujourd’hui le traitement d’un curé est au minimum de 900 francs, plus le logement et le casuel.}, un homme peut s’entretenir là-dessus. Mais il vit parmi les misérables auxquels il doit l’aumône, et il garde au fond du cœur une amertume secrète contre le richard oisif qui, les poches pleines, l’envoie faire, avec des poches vides, un ministère de charité. À Saint-Pierre de Barjouville, dans le Toulousain, l’archevêque de Toulouse prend la moitié des dîmes et fait par an 8 livres d’aumône ; à Bretx, le chapitre de l’Isle-Jourdain qui perçoit la moitié de certaines dîmes et les trois quarts des autres, donne 10 livres ; à Croix-Falgarde, les Bénédictins, à qui la moitié de la dîme appartient, donnent 10 livres par an\footnote{Théron de Montaugé, {\itshape l’Agriculture et les classes rurales dans le pays Toulousain}, 86.}. À Sainte-Croix de Bernay en Normandie\footnote{Périn, {\itshape la Jeunesse de Robespierre}, doléances des paroisses rurales de l’Artois, 320. Boivin-Champeaux, {\itshape ib.}, 65, 68. — Hippeau, {\itshape ib.}, VI, 79, et VII, 177. — Lettre de M. Sergent, curé de Villers, du 27 janvier 1790. ({\itshape Archives nationales}, DXIX, carton 24.) Lettre de M. Briscard, curé de Beaumont-le-Roger, diocèse d’Évreux, du 19 décembre 1789. ({\itshape Ib}., DXIX, carton 6.) — Tableau moral du clergé de France (1789), 2.}, l’abbé non résident, qui touche 57 000 livres, donne 1 050 livres au curé qui n’a pas de presbytère et dont la paroisse contient 4 000 communiants. À Saint-Aubin-sur-Gaillon, l’abbé, gros décimateur, donne 350 livres au vicaire, qui est obligé d’aller dans le village quêter du blé, du pain, des pommes. À Plessis-Hébert, « le desservant déportuaire, n’ayant pas de quoi vivre, est forcé d’aller chercher ses repas chez les curés voisins ». Dans l’Artois, où souvent la dîme prélève 7 1/2 et 8 pour 100 du produit de la terre, nombre de curés sont à la portion congrue et sans presbytère ; leur église tombe en ruines et le bénéficier ne donne rien aux pauvres. « À Saint-Laurent, en Normandie, la cure ne vaut pas plus de 400 livres que le curé partage avec un obitier, et il y a 500 habitants, dont les trois quarts à l’aumône. » — Comme les réparations du presbytère et de l’église sont d’ordinaire à la charge d’un seigneur ou d’un bénéficier souvent éloigné, obéré ou indifférent, il arrive parfois que le prêtre ne sait ni où loger, ni où dire la messe. « J’arrivai, dit un curé de Touraine, au mois de juin 1788… Le presbytère ressemblerait à un souterrain hideux s’il n’était ouvert à tous les frimas et à tous les vents » : en bas, deux chambres carrelées sans portes ni fenêtres, hautes de quatre pieds et demi, une troisième haute de six pieds, carrelée, servant de salon, de salle, de cuisine, de buanderie, de boulangerie et d’égout pour les eaux de la cour et du jardin ; au-dessus trois pièces semblables, « le tout absolument lézardé, crevé, menaçant ruine, sans portes ni croisées qui tiennent », et, en 1790, les réparations ne sont pas encore faites. — Voyez par contraste le luxe des prélats qui ont un demi-million de rente, la pompe de leurs palais, les équipages de chasse de M. de Dillon, évêque d’Evreux, le confessionnaux garnis de satin de M. de Barrai, évêque de Troyes, l’innombrable batterie de cuisine en argent massif de M. de Rohan, évêque de Strasbourg. — Tel est le sort des curés à portion congrue, et il y en a beaucoup qui n’ont pas la portion congrue, que la mauvaise volonté du haut clergé en exclut, qui, avec leur casuel, ne touchent que 400 à 500 livres, qui réclament en vain la maigre pitance à laquelle ils ont droit par le dernier édit. « Une pareille demande, dit un curé, ne devrait-elle pas être acceptée de bon gré par MM. du haut clergé qui souffrent des moines jouir de 5 à 6 000 livres de rente par chaque individu, tandis qu’ils voient les curés, au moins aussi nécessaires, réduits à la mince portion, tant pour eux que pour la paroisse ? » — Et, sur cette mince pitance, on rogne encore pour payer le don gratuit. En ceci comme pour le reste, les pauvres sont chargés pour décharger les riches. Dans le diocèse de Clermont, « les curés, même à simple portion congrue, sont imposés à 60, 80, 100, 120 livres et plus ; les vicaires qui ne subsistent que du fruit de leurs sueurs, sont taxés à 22 livres ». Au contraire, les prélats payent peu de chose, et « encore est-on dans l’usage de présenter aux évêques la quittance de leur taxe, aux étrennes du premier de l’an\footnote{{\itshape Doléances} sur les surcharges que supportent les gens du Tiers-État, par Gaultier de Biauzat (1788), 237.} ». — Nulle issue pour les curés. Sauf trois ou quatre petits évêchés « de laquais », toutes les dignités de l’Église sont réservées à la noblesse ; « pour être évêque aujourd’hui, dit l’un d’entre eux, il faut être gentilhomme ». Je vois en eux des sergents qui, comme leurs pareils dans l’armée, ont perdu l’espoir de jamais devenir officiers. — C’est pourquoi il y en a chez qui la colère déborde : « Nous, malheureux curés à portions congrues ; nous, chargés communément des plus fortes paroisses, telles que la mienne qui a, jusqu’à deux lieues dans les bois, des hameaux qui en feraient une autre ; nous dont le sort fait crier jusqu’aux pierres et aux chevrons de nos misérables presbytères », nous subissons des prélats « qui feraient encore quelquefois faire par leurs gardes un procès au pauvre curé qui couperait dans leurs bois un bâton, son seul soutien dans ses longues courses par tous chemins ». À leur passage, le pauvre homme « est obligé de se jeter à tâtons le long d’un talus, pour se garantir des pieds et des éclaboussures de leurs chevaux, comme aussi des roues et peut-être du fouet d’un clocher insolent », puis « tout crotté, son chétif bâton d’une main et son chapeau, tel quel, de l’autre, de saluer humblement et rapidement, à travers la portière du char clos et doré, le hiérarque postiche ronflant sur la laine du troupeau que le pauvre curé va paissant et dont il ne lui laisse que la crotte et le suint ». Toute la lettre est comme un long cri de rage ; ce sont des rancunes semblables qui feront les Joseph Lebon et les Fouché. — Dans cette situation et avec ces sentiments, il est manifeste que le bas clergé traitera ses chefs comme la noblesse de province a traité les siens\footnote{Hippeau, {\itshape ib.}, VI, 164. (Lettre du curé de Marolles et de treize autres. Lettre de l’évêque d’Évreux du 20 mars 1789. Lettre de l’abbé d’Osmond du 2 avril 1789.) — {\itshape Archives nationales}, Procès-verbaux manuscrits des États généraux, t. 148, 245 et 257, Mémoires des curés de Toulouse ; t. 150, 282, représentations par le chapitre de Dijon.}. Il ne choisira pas « pour représentants ceux qui nagent dans l’opulence et qui l’ont vu toujours souffrir avec tranquillité ». De toutes parts les curés se confédèrent » pour n’envoyer aux États généraux que des curés, et pour exclure, « non seulement les chanoines, les abbés, les prieurs et tous autres bénéficiers, mais encore les premiers supérieurs, les chefs de la hiérarchie », c’est-à-dire les évêques. En effet, sur trois cents députés du clergé, on compte aux États généraux deux cent huit curés, et, comme la noblesse de province, ils apportent avec eux la défiance et le mauvais vouloir qu’ils nourrissent depuis si longtemps contre leurs chefs. On s’en apercevra tout à l’heure à l’épreuve. Si les deux premiers ordres sont contraints de se réunir aux communes, c’est qu’au moment critique les curés font défection. Si l’institution d’une chambre haute est repoussée, c’est que la plèbe des gentilshommes ne veut pas souffrir aux grandes familles une prérogative dont elles ont abusé.

\section[{V. Le roi. — Son privilège est le plus énorme de tous. — Ayant accaparé tous les pouvoirs, il s’est chargé de toutes les fonctions. — Pesanteur de cette tâche. — Il s’y dérobe, ou n’y suffit pas. — Sécurité de sa conscience. — La France est sa propriété. — Comment il en abuse. — La royauté centre des abus.}]{V. Le roi. — Son privilège est le plus énorme de tous. — Ayant accaparé tous les pouvoirs, il s’est chargé de toutes les fonctions. — Pesanteur de cette tâche. — Il s’y dérobe, ou n’y suffit pas. — Sécurité de sa conscience. — La France est sa propriété. — Comment il en abuse. — La royauté centre des abus.}

\noindent Reste un dernier privilège, le plus énorme de tous, celui du roi ; car, dans cet état-major de nobles héréditaires, il est le général héréditaire. À la vérité son office n’est pas une sinécure comme leur rang ; mais il comporte des inconvénients aussi graves et des tentations pires. Deux choses sont pernicieuses à l’homme, le manque d’occupation et le manque de frein ; ni l’oisiveté, ni la toute-puissance ne sont conformes à sa nature, et le prince absolu qui peut tout faire, comme l’aristocratie désœuvrée qui n’a rien à faire, finit par devenir inutile et malfaisant. — Insensiblement, en accaparant tous les pouvoirs, le roi s’est chargé de toutes les fonctions ; tâche immense et qui surpasse le forces humaines. Car ce n’est point la Révolution, c’est la monarchie qui a implanté en France la centralisation administrative\footnote{\href{http://classiques.uqac.ca/classiques/De\_tocqueville\_alexis/ancien\_regime/ancien\_regime.html}{\dotuline{Tocqueville, liv. II}} [\url{http://classiques.uqac.ca/classiques/De\_tocqueville\_alexis/ancien\_regime/ancien\_regime.html}]. Cette vérité capitale a été établie par M. de Tocqueville avec une perspicacité supérieure.}. Sous la direction du Conseil du roi, trois fonctionnaires superposés, au centre le contrôleur général, dans chaque généralité l’intendant, dans chaque élection le subdélégué, mènent toutes les affaires, fixent, répartissent et lèvent l’impôt et la milice, tracent et font exécuter les routes, emploient la maréchaussée, distribuent les secours, réglementent la culture, imposent aux paroisses leur tutelle, et traitent comme des valets les magistrats municipaux. « Un village, dit Turgot\footnote{Remontrances de Malesherbes, Mémoire de Turgot, Mémoire de Necker au roi. (Laboulaye, {\itshape De l’administration française sous Louis XVI. Revue des cours littéraires.} IV, 423, 759, 814.)}, n’est qu’un assemblage de maisons, de cabanes et d’habitants aussi passifs qu’elles… Votre Majesté est obligée de décider tout par elle-même ou par ses mandataires… Chacun attend vos ordres spéciaux, pour contribuer au bien public, pour respecter les droits d’autrui, quelquefois même pour user des siens propres. » Par suite, ajoute Necker, « c’est du fond des bureaux que la France est gouvernée… Les commis, ravis de leur influence, ne manquent jamais de persuader au ministre qu’il ne peut se détacher de commander un seul détail ». — Bureaucratie au centre, arbitraire, exceptions et faveurs partout, tel est le résumé du système. « Subdélégués, officiers d’élections, directeurs, receveurs et contrôleurs des vingtièmes, commissaires et collecteurs des tailles, officiers des gabelles, voituriers-buralistes, huissiers, piqueurs des corvées, commis aux aides, au contrôle, aux droits réservés, tous ces hommes de l’impôt, chacun selon son caractère, assujettissent à leur petite autorité et enveloppent de leur science fiscale des contribuables ignorants et inhabiles à reconnaître si on les trompe\footnote{« On a entendu le financier dire au citoyen : Il faut que la ferme ait des grâces à vous accorder, il faut que vous soyez obligé de venir les demander. — Celui qui paye ne peut jamais savoir ce qu’il doit. Le fermier est souverain législateur dans les matières qui font l’objet de son intérêt personnel. Toute requête, dans laquelle les intérêts d’une province ou ceux de la nation entière sont stipulés, est regardée comme une témérité punissable si elle est signée d’un seul particulier, et comme une association illicite si elle est signée de plusieurs. » (Malesherbes, {\itshape ibid.})}. » Une centralisation grossière, sans contrôle, sans publicité, sans uniformité, installe sur tout le territoire une armée de petits pachas qui décident comme juges les contestations qu’ils ont comme parties, règnent par délégation, et, pour autoriser leurs grappillages ou leurs insolences, ont toujours à la bouche le nom du roi, qui est obligé de les laisser faire. — En effet, par sa complication, son irrégularité et sa grandeur, la machine échappe à ses prises. Un Frédéric II levé à quatre heures du matin, un Napoléon qui dicte une partie de la nuit dans son bain et travaille dix-huit heures par jour, y suffiraient à peine. Un tel régime ne va point sans une attention toujours tendue, sans une énergie infatigable, sans un discernement infaillible, sans une sévérité militaire, sans un génie supérieur ; à ces conditions seulement on peut changer vingt-cinq millions d’hommes en automates, et substituer sa volonté partout lucide, partout cohérente, partout présente, à leurs volontés que l’on abolit. Louis XV laisse « la bonne machine » marcher toute seule, et se cantonne dans son apathie. « Ils l’ont voulu ainsi, ils ont pensé que c’était pour le mieux »\footnote{Mme Campan, {\itshape Mémoires}, I, 13. — Mme du Hausset, \href{http://gallica.bnf.fr/ark:/12148/bpt6k36431d}{\dotuline{{\itshape Mémoires}}} [\url{http://gallica.bnf.fr/ark:/12148/bpt6k36431d}]. 114.}, telle est sa façon de parler « quand les opérations des ministres n’ont pas réussi ». — « Si j’étais lieutenant de police, disait-il encore, je défendrais les cabriolets. » Il a beau sentir que la machine se disloque, il n’y peut rien, il n’y fait rien. En cas de malheur, il a sa réserve privée, sa bourse à part. « Le roi, disait Mme de Pompadour, signerait sans y songer pour un million, et donnerait avec peine cent louis sur son petit trésor. » — Louis XVI essaye pendant un temps de supprimer plusieurs rouages, d’en introduire de meilleurs, d’adoucir les frottements du reste ; mais les pièces sont trop rouillées, trop pesantes ; il ne peut les ajuster, les accorder, les maintenir en place ; sa main retombe impuissante et lassée. Il se contente d’être économe pour lui-même ; il inscrit sur son journal un raccommodage de montre, et laisse la voiture publique, aux mains de Calonne, se charger d’abus nouveaux pour rentrer dans l’ancienne ornière, d’où elle ne sortira qu’en se disloquant.\par
Sans doute le mal qu’ils font ou qu’on fait en leur nom leur déplaît et les chagrine ; mais au fond leur conscience n’est pas inquiète. Ils peuvent avoir compassion du peuple, mais ils ne se sentent pas coupables envers lui ; car ils sont ses souverains et non ses mandataires. La France est à eux comme tel domaine est à son seigneur, et un seigneur ne manque pas à l’honneur parce qu’il est prodigue et négligent. C’est son bien qu’il dissipe, et personne n’a le droit de lui demander des comptes. Fondée sur la seigneurie féodale, la royauté est comme elle une propriété, un héritage, et ce serait infidélité, presque trahison chez un prince, en tout cas faiblesse et bassesse, que de laisser passer entre des mains de sujets quelque portion du dépôt qu’il a reçu intact de ses pères pour le transmettre intact à ses enfants. Non seulement, par la tradition du moyen âge, il est commandant-propriétaire des Français et de la France, mais encore, par la théorie des légistes, il est, comme César, l’unique et perpétuel représentant de la nation, et, par la doctrine des théologiens, il est, comme David, le délégué sacré et spécial de Dieu lui-même. À tous ces titres, ce serait merveille s’il ne considérait pas le revenu public comme son revenu privé, et si, maintes fois, il n’agissait pas en conséquence. En ceci notre point de vue est si opposé, que nous avons de la peine à nous mettre au sien ; mais le sien était alors celui de tout le monde. En ce temps-là il semblait aussi étrange de s’ingérer dans les affaires du roi que dans celles d’un particulier. C’est seulement à la fin de 1788\footnote{{\itshape Gustave III et la cour de France}, par Geffroy, II, 474. (Archives de Dresde, correspondance de France, 20 novembre 1788.)} que le fameux salon du Palais-Royal, « avec une hardiesse et une déraison inimaginables, prétend que, dans une véritable monarchie, les revenus de l’État ne doivent pas être à la disposition du souverain, qu’il doit seulement lui être accordé une somme assez considérable pour les charges de sa maison, ses dons et les grâces de ses serviteurs, ainsi que pour ses plaisirs, que le surplus doit être déposé au Trésor royal pour n’y être employé qu’aux objets sanctionnés par l’Assemblée de la Nation ». Réduire le prince à une liste civile, mettre la main sur les neuf dixièmes de son revenu, lui interdire les acquits au comptant, quel attentat ! La surprise ne serait pas plus grande, si aujourd’hui l’on proposait de faire deux parts dans le revenu de chaque millionnaire, de lui en accorder la plus mince pour son entretien, de mettre la plus grosse à la caisse des consignations pour ne la dépenser qu’en œuvres d’utilité publique. Un ancien fermier général, homme d’esprit et sans préjugés, écrit sérieusement pour justifier l’achat de Saint-Cloud : « C’était une bague au doigt de la reine ». À la vérité, la bague coûtait 7 700 000 francs. Mais « le roi de France avait alors 477 millions de rente. Que dirait-on d’un particulier qui aurait 477 000 livres de rente, et qui, une fois dans sa vie, donnerait à sa femme pour 7 000 ou 8 000 livres de diamants ?\footnote{Agear, {\itshape Mémoires}, 135.} » On dirait que le don est modeste et que le mari est raisonnable  Pour bien comprendre l’histoire de nos rois, posons toujours en principe que la France est leur terre, une ferme transmise de père en fils, d’abord petite, puis arrondie peu à peu, à la fin prodigieusement élargie, parce que le propriétaire, toujours aux aguets, a trouvé moyen de faire de beaux coups aux dépens de ses voisins ; au bout de huit cents ans, elle comprend 27 000 lieues carrées. Certainement, en plusieurs points, son intérêt et son amour-propre sont d’accord avec le bien public ; en somme il n’a pas mal géré, et puisqu’il s’est toujours agrandi, il a mieux géré que beaucoup d’autres. De plus, autour de lui, nombre de gens experts, vieux conseillers de famille, rompus aux affaires et dévoués au domaine, bonnes têtes et barbes grises, lui font respectueusement des remontrances quand il dépense trop ; souvent ils l’engagent dans des œuvres utiles, routes, canaux, hôtels d’invalides, écoles militaires, instituts de science, ateliers de charité, limitation de la mainmorte, tolérance des hérétiques, recul des vœux monastiques jusqu’à vingt et un ans, assemblées provinciales, et autres établissements ou réformes par lesquels un domaine féodal se transforme en un domaine moderne. Mais, féodal ou moderne, le domaine est toujours sa propriété, dont il peut abuser autant qu’user ; or qui use en toute liberté finit par abuser avec toute licence. Si, dans sa conduite ordinaire, les motifs personnels ne l’emportaient pas sur les motifs publics, il serait un saint comme Louis IX, un stoïcien comme Marc-Aurèle, et il est un seigneur, un homme du monde semblable aux gens de sa cour, encore plus mal élevé, plus mal entouré, plus sollicité, plus tenté et plus aveuglé. À tout le moins, il a comme eux son amour-propre, ses goûts, ses parents, sa maîtresse, sa femme, ses familiers, tous solliciteurs intimes et prépondérants qu’il faut d’abord satisfaire ; la nation ne vient qu’ensuite  En effet, pendant cent ans, de 1672 à 1774, toutes les fois qu’il fait une guerre, c’est par pique de vanité, par intérêt de famille, par calcul d’intérêt privé, par condescendance pour une femme. Louis XV conduit les siennes encore plus mal qu’il ne les entreprend\footnote{« Mme de Pompadour, écrivant au maréchal d’Estrées à l’armée sur les opérations de la campagne, et lui traçant une espèce de plan, avait marqué sur le papier avec des {\itshape mouches} les différents lieux qu’elle lui conseillait d’attaquer ou de défendre. » (Mme de Genlis, 329, {\itshape Souvenirs de Félicie}, Récit de Mme de Puisieux, belle-mère du maréchal d’Estrées.)}, et Louis XVI, dans toute sa politique extérieure, trouve pour entrave le rets conjugal  À l’intérieur, il vit comme les autres seigneurs, mais plus grandement, puisqu’il est le plus grand seigneur de France ; je décrirai son train tout à l’heure, et l’on verra plus tard par quelles exactions ce faste est défrayé. En attendant, marquons deux ou trois détails. D’après des relevés authentiques, Louis XV a dépensé pour Mme de Pompadour 36 millions, au moins 72 millions d’aujourd’hui\footnote{D’après le registre manuscrit des dépenses de Mme de Pompadour, aux {\itshape Archives de la préfecture de Versailles}, elle avait dépensé 36 327 268 livres. (Granier de Cassagnac, I, 91.)}. Selon d’Argenson\footnote{Marquis d’Argenson, {\itshape Mémoires}. VI, 398 (24 avril 1751). — M. du Barry avouait hautement qu’il avait mangé 18 millions à l’État. » ({\itshape Correspondance} par Metra, I, 27.)}, en 1751, il a dans ses écuries 4 000 chevaux, et l’on assure que sa seule maison ou personne « a coûté cette année 68 millions », près du quart du revenu public. Quoi d’étonnant, lorsqu’on considère le souverain à la manière du temps, c’est-à-dire comme un châtelain qui jouit de son bien héréditaire ? Il bâtit, il reçoit, il donne des fêtes, il chasse, il dépense selon sa condition  De plus, étant maître de son argent, il donne à qui lui plaît, et tous ses choix sont des grâces. « Votre Majesté sait mieux que moi, écrit l’abbé de Vermond à l’impératrice Marie-Thérèse\footnote{ {\itshape Marie-Antoinette}, par Arneth et Geffroy, II, 168 (5 juin 1774).
 }, que, d’usage immémorial, les trois quarts des places, des honneurs, des pensions sont accordés non aux services, mais à la faveur et au crédit. Cette faveur est originairement motivée par la naissance, les alliances et la fortune ; presque toujours elle n’a de véritable fondement que dans la protection et l’intrigue. Cette marche est si fort établie, qu’elle est respectée comme une sorte de justice par ceux mêmes qui en souffrent le plus ; un bon gentilhomme, qui ne peut éblouir par des alliances à la cour, ni par une dépense d’éclat, n’oserait prétendre à un régiment, quelque anciens et distingués que puissent être ses services et sa naissance. Il y a vingt ans, les fils des ducs, des ministres, des gens attachés à la cour, les parents et protégés des maîtresses, devenaient colonels à seize ans ; M. de Choiseul fit jeter les hauts cris en rejetant cette époque à vingt-trois ; mais, pour dédommager la faveur et l’arbitraire, il a remis à la pure grâce du roi, ou plutôt des ministres, la nomination des lieutenances-colonelles et des majorités qui jusqu’alors allaient de droit à l’ancienneté du service, les gouvernements et les commandements des provinces et des villes. Vous savez, Monsieur l’ambassadeur, qu’on a fort multiplié ces places, et qu’elles se donnent par crédit et faveur, comme les régiments. Le cordon bleu, le cordon rouge sont dans le même cas, quelquefois même la croix de Saint-Louis. Les évêchés et les abbayes sont encore plus constamment au régime du crédit. Les places de finances, je n’ose en parler. Les charges de judicature sont les plus assujetties aux services rendus ; et cependant combien le crédit et la recommandation n’influent-ils pas sur la nomination des intendants, des premiers présidents », et des autres   Necker, entrant aux affaires, trouve 28 millions de pensions sur le Trésor royal, et, sitôt qu’il tombe, c’est une débâcle d’argent déversé par millions sur les gens de cour. Même de son temps, le roi s’est laissé aller à faire la fortune des amies et des amis de sa femme : à la comtesse de Polignac 400 000 francs pour payer ses dettes, 800 000 francs pour la dot de sa fille, en outre, pour elle-même, la promesse d’une terre de 35 000 livres de rente, et, pour son amant, le comte de Vaudreuil, 30 000 livres de pension ; à la princesse de Lamballe, 100 000 écus par an, tant par la charge de surintendante qu’on rétablit en sa faveur, que pour une pension à son frère\footnote{{\itshape Marie-Antoinette, ib.}, II, 377 ; III, 391.}. Mais c’est sous Calonne que la prodigalité devient folle. On a fait honte au roi de sa parcimonie ; pourquoi serait-il ménager de sa bourse ? Lancé hors de sa voie, il donne, il achète, il bâtit, il échange, il vient en aide aux gens de son monde, le tout en grand seigneur, c’est-à-dire en jetant l’argent à pleines mains. Qu’on en juge par un seul exemple : pour secourir les Guéméné faillis, il leur achète moyennant 12 500 000 livres trois terres qu’ils viennent d’acheter 4 millions ; de plus, en échange de deux domaines en Bretagne qui rapportent 33 758 livres, il leur cède la principauté de Dombes rapportant près de 70 000 livres de rente\footnote{{\itshape Archives nationales}, II, 1456. Mémoire pour M. Bouret de Vézelay, syndic des créanciers.}  Lorsqu’on lira plus tard le {\itshape Livre Rouge}, on y trouvera 700 000 livres de pensions pour la maison de Polignac, la plupart réversibles d’un membre à l’autre, et près de deux millions de bienfaits annuels à la maison de Noailles  Le roi a oublié que toutes ses grâces sont meurtrières ; car « le courtisan qui obtient 6 000 livres de pension reçoit la taille de six villages\footnote{Marquis de Mirabeau, \href{http://gallica.bnf.fr/ark:/12148/bpt6k89089c}{\dotuline{{\itshape Traité de la population}}} [\url{http://gallica.bnf.fr/ark:/12148/bpt6k89089c}], 81.} ». En l’état où est l’impôt, chaque largesse du monarque est fondée sur le jeûne des paysans, et le souverain, par ses commis, prend aux pauvres leur pain pour donner des carrosses aux riches  Bref le centre du gouvernement est le centre du mal ; toutes les injustices et toutes les misères en partent comme d’un foyer engorgé et douloureux ; c’est ici que l’abcès public a sa pointe, et c’est ici qu’il crèvera.

\section[{VI. Désorganisation latente de la France.}]{VI. Désorganisation latente de la France.}

\noindent Juste et fatal effet du privilège que l’on exploite à son profit au lieu de l’exercer au profit d’autrui. Qui dit sire ou seigneur, dit « le protecteur qui nourrit, l’ancien qui conduit\footnote{{\itshape Lord}, en vieux saxon, signifie « celui qui nourrit ». {\itshape Seigneur}, en latin du moyen âge, signifie « l’ancien », le chef du troupeau.} » ; à ce titre et pour cet emploi, on ne peut lui donner trop, car il n’y a pas d’emploi plus difficile et plus haut. Mais il faut qu’il le remplisse ; sinon, au jour du danger, on le laisse là. Déjà, et bien avant le jour du danger, sa troupe n’est plus à lui ; si elle marche, c’est par routine ; elle n’est qu’un amas d’individus, elle n’est plus un corps organisé. Tandis qu’en Allemagne et en Angleterre le régime féodal conservé ou transformé compose encore une société vivante, en France son cadre mécanique n’enserre qu’une poussière d’hommes. On trouve encore l’ordre matériel ; on ne trouve plus l’ordre moral. Une lente et profonde révolution a détruit la hiérarchie intime des suprématies acceptées et des déférences volontaires. C’est une armée où les sentiments qui font les chefs et les sentiments qui font les subordonnés ont disparu ; les grades sont marqués sur les habits et ne le sont plus dans les consciences ; il lui manque ce qui fait une armée solide, l’ascendant légitime des officiers, la confiance justifiée des soldats, l’échange journalier des dévouements mutuels, la persuasion que chacun est utile à tous et que les chefs sont les plus utiles de tous. Comment trouverait-on cette persuasion dans une armée dont l’état-major, pour toute occupation, dîne en ville, étale ses épaulettes et touche double solde ? Déjà avant l’écroulement final, la France est dissoute, et elle est dissoute parce que les privilégiés ont oublié leur caractères d’{\itshape hommes publics.}
\chapterclose

\chapterclose


\chapteropen

\part[{Livre deuxième. Les mœurs et les caractères.}]{Livre deuxième. \\
Les mœurs et les caractères.}
\renewcommand{\leftmark}{Livre deuxième. \\
Les mœurs et les caractères.}


\chaptercont

\chapteropen

\chapter[{Chapitre I. Principe des mœurs sous l’Ancien Régime.}]{Chapitre I. \\
Principe des mœurs sous l’Ancien Régime.}


\chaptercont
\noindent Un état-major en vacances pendant un siècle et davantage, autour du général en chef qui reçoit et tient salon : voilà le principe et le résumé des mœurs sous l’ancien régime. C’est pourquoi, si l’on veut les comprendre, il faut d’abord considérer leur centre et leur source, je veux dire la cour. Comme l’ancien régime tout entier, elle est la forme vide, le décor survivant d’une institution militaire ; quand les causes ont disparu, les effets subsistent, et l’usage survit à l’utilité. Jadis, aux premiers temps féodaux, dans la camaraderie et la simplicité du camp et du château fort, les nobles servaient le roi de leurs mains, celui-ci pourvoyant à son logis, celui-là apportant le plat sur sa table, l’un le déshabillant le soir, l’autre veillant à ses faucons et à ses chevaux. Plus récemment, sous Richelieu et pendant la Fronde\footnote{{\itshape Mémoires de Laporte} (1632). « M. d’Épernon vint à Bordeaux, où il trouva Son Éminence fort malade. Il l’alla voir soigneusement tous les matins avec 200 gardes qui l’accompagnaient jusqu’à la porte de la chambre. » — {\itshape Mémoires de Retz.} « Nous vînmes à l’audience, M. de Beaufort et moi, avec un corps de noblesse qui pouvait faire 300 gentilshommes ; MM. les princes avaient près de 1 000 gentilshommes avec eux. » — Tous les Mémoires du temps montrent à chaque instant ces escortes qui étaient nécessaires pour faire ou repousser un coup de main.}, parmi les coups de main et les exigences brusques du danger continu, ils étaient la garnison de son hôtel, ils l’escortaient en armes, ils lui faisaient un cortège d’épées toujours prêtes. Maintenant encore ils sont comme autrefois assidus autour de lui, l’épée au côté, attendant un mot, empressés sur un signe, et les plus qualifiés d’entre eux font chez lui un semblant de service domestique. Mais la parade pompeuse a remplacé l’action efficace ; ils ne sont que de beaux ornements, ils ne sont plus des instruments utiles ; ils représentent autour du roi qui représente, et, de leurs personnes, ils contribuent à son décor.\par

\section[{I. Aspect physique et caractère moral de Versailles.}]{I. Aspect physique et caractère moral de Versailles.}

\noindent Il faut dire que le décor est réussi, et que, depuis les fêtes de la Renaissance italienne, on n’en a pas vu de plus magnifique. Suivons la file de voitures qui, de Paris à Versailles, roule incessamment comme un fleuve. Des chevaux qu’on nomme « des enragés » et qu’on nourrit d’une façon particulière\footnote{Mercier, \href{http://gallica.bnf.fr/ark:/12148/bpt6k89045r}{\dotuline{{\itshape Tableau de Paris}}} [\url{http://gallica.bnf.fr/ark:/12148/bpt6k89045r}], IX, 3.} y vont et en reviennent en trois heures. Au premier coup d’œil, on se sent dans une ville d’espèce unique, bâtie subitement et tout d’une pièce, comme une médaille d’apparat frappée à un seul exemplaire et tout exprès : sa forme est une chose à part, comme aussi son origine et son usage. Elle a beau compter 80 000 âmes\footnote{Leroi, {\itshape Histoire de Versailles}, II, 21 (70 000 âmes de population fixe, et 10 000 de population flottante, d’après les registres de la mairie).}, être l’une des plus vastes cités du royaume, elle est remplie, peuplée, occupée par la vie d’un seul homme ; ce n’est qu’une résidence royale, arrangée tout entière pour fournir aux besoins, aux plaisirs, au service, à la garde, à la société, à la représentation du roi. Çà et là, dans les recoins et le pourtour, sont des auberges, des échoppes, des cabarets, des taudis pour les ouvriers, les hommes de peine, pour les derniers soldats, pour la valetaille accessoire ; il faut bien qu’il y ait de ces taudis, puisque la plus belle apothéose ne peut se passer de manœuvres. Mais le reste n’est qu’hôtels et bâtisses somptueuses, façades sculptées, corniches et balustres, escaliers monumentaux, architectures seigneuriales, espacées et ordonnées régulièrement comme un cortège autour du palais immense et grandiose où tout aboutit. Les premières familles ont ici leur résidence fixe : à droite du palais, hôtel de Bourbon, hôtel d’Ecquevilly, hôtel de la Trémoille, hôtel de Condé, hôtel de Maurepas, hôtel de Bouillon, hôtel d’Eu, hôtel de Noailles, hôtel de Penthièvre, hôtel de Livry, hôtel du comte de la Marche, hôtel de Broglie, hôtel du prince de Tingry, hôtels d’Orléans, de Châtillon, de Villeroy, d’Harcourt, de Monaco ; à gauche, pavillon d’Orléans, pavillon de Monsieur, hôtels de Chevreuse, de Balbelle, de l’Hospital, d’Antin, de Dangeau, de Pontchartrain : l’énumération ne finirait pas. Ajoutez-y tous ceux de Paris, tous ceux qui, à dix lieues à la ronde, à Sceaux, à Gennevilliers, à Brunoy, à l’Isle-Adam, au Raincy, à Saint-Ouen, à Colombes, à Saint-Germain, à Marly, à Bellevue, en cent endroits, forment une couronne de fleurs architecturales d’où s’élancent chaque matin autant de guêpes dorées pour briller et butiner à Versailles, centre de toute abondance et de tout éclat. On en « présente » chaque année une centaine, hommes et femmes\footnote{ \noindent Waroquier, {\itshape État de la France} (1789). Liste des personnes présentées de 1779 à 1789, 453 hommes et 414 femmes, t. II, 515.
 }, cela fait en tout deux ou trois mille : voilà la société du roi, les dames qui lui font la révérence, les seigneurs qui montent dans ses carrosses ; leurs hôtels sont tout près ou à portée pour remplir à toute heure son antichambre ou son salon.\par
Un pareil salon comporte des dépendances proportionnées ; c’est par centaines qu’il faut compter les hôtels et bâtiments occupés à Versailles pour le service privé du roi et des siens. Depuis les Césars, aucune vie humaine n’a tenu tant de place au soleil. Rue des Réservoirs, l’ancien hôtel et le nouvel hôtel du gouverneur de Versailles, l’hôtel du gouverneur des enfants du comte d’Artois, le garde-meuble de la couronne, le bâtiment pour les loges et foyers des acteurs qui jouent au Palais, les écuries de Monsieur  Rue des Bons-Enfants, l’hôtel de la garde-robe, le logement des fontainiers, l’hôtel des officiers de la comtesse de Provence  Rue de la Pompe, l’hôtel du grand-prévôt, les écuries du duc d’Orléans, l’hôtel des gardes du comte d’Artois, les écuries de la reine, le pavillon des Sources. — Rue Satory, les écuries de la comtesse d’Artois, le jardin anglais de Monsieur, les glacières du roi, le manège des chevau-légers de la garde du roi, le jardin de l’hôtel des trésoriers des bâtiments. — Par ces quatre rues, jugez des autres. — On ne peut faire cent pas dans la ville sans y rencontrer un accessoire du palais : hôtel de l’état-major des gardes du corps, hôtel de l’état-major des chevau-légers, hôtel immense des gardes du corps, hôtel des gendarmes de la garde, hôtels du grand-louvetier, du grand-fauconnier, du grand-veneur, du grand-maître, du commandant du canal, du contrôleur général, du surintendant des bâtiments, hôtel de la chancellerie, bâtiments de la fauconnerie et du vol de cabinet, bâtiments du vautrait, grand chenil, chenil-dauphin, chenil dit des chiens verts, hôtel des voitures de la cour, magasin des bâtiments et menus-plaisirs, ateliers et magasins pour les menus-plaisirs, grande écurie, petite écurie, autres écuries dans la rue de Limoges, dans la rue Royale et dans l’avenue de Saint-Cloud, potager du roi comprenant vingt-neuf jardins et quatre terrasses, grand-commun habité par deux mille personnes, maisons et hôtels dits des {\itshape Louis} où le roi assigne des logements à temps ou à vie : avec des mots sur du papier, on ne rend point l’impression physique de l’énormité physique. — Aujourd’hui, de cet ancien Versailles mutilé et approprié à d’autres usages, il ne reste plus que des morceaux ; allez le voir pourtant. Considérez ces trois avenues qui se réunissent sur la grande place, larges de quarante toises, longues de quatre cents, et qui n’étaient point trop vastes pour la multitude, le déploiement, la vitesse vertigineuse des escortes lancées à fond de train et des carrosses courant « à tombeau ouvert\footnote{Il y avait alors, presque chaque jour, des passants roués à Paris par les voitures à la mode, et c’était l’habitude chez les grands d’aller très vite.} » ; voyez, en face du château, les deux écuries, avec leurs grilles de trente-deux toises, ayant coûté, en 1682, trois millions, c’est-à-dire quinze millions d’aujourd’hui, si amples et si belles que, sous Louis XIV lui-même, on en faisait tantôt un champ de cavalcades pour les princes, tantôt une salle de théâtre, et tantôt un salle de bal ; suivez alors du regard le développement de la gigantesque place demi-circulaire, qui, de grille en grille et de cour en cour, va montant et se resserrant, d’abord entre les hôtels des ministres, puis entre les deux ailes colossales, pour s’achever par le fastueux encadrement de la Cour de Marbre, où les pilastres, les statues, les frontons, les ornements multipliés et amoncelés d’étage en étage portent jusque dans le ciel la raideur majestueuse de leurs lignes et l’étalage surchargé de leur décor. D’après un manuscrit relié aux armes de Mansart, le palais a coûté 153 millions, c’est-à-dire environ 750 millions d’aujourd’hui\footnote{153 282 827 livres 10 sous 3 deniers. (\href{http://gallica.bnf.fr/ark:/12148/bpt6k49961n/f159.table}{\dotuline{{\itshape Souvenirs d’un page de la cour de Louis XVI}}} [\url{http://gallica.bnf.fr/ark:/12148/bpt6k49961n/f159.table}], par le comte d’Hézecques, 142.) — En 1690, avant la construction de la chapelle et de la salle de spectacle, il coûtait déjà 100 millions. (\href{http://gallica.bnf.fr/ark:/12148/bpt6k7047p/f517}{\dotuline{Saint-Simon, XII, 514}} [\url{http://gallica.bnf.fr/ark:/12148/bpt6k7047p/f517}]. Mémoire de Marinier, commis des bâtiments du roi.)} ; quand un roi veut représenter, c’est à ce prix qu’il se loge  Jetez maintenant les yeux de l’autre côté, vers les jardins, et cette représentation vous deviendra plus sensible. Les parterres et le parc sont encore un salon en plain air ; la nature n’y a plus rien de naturel ; elle est tout entière disposée et rectifiée en vue de la société ; ce n’est point là un endroit pour être seul et se détendre, mais un lieu pour se promener en compagnie et saluer. Ces charmilles droites sont des murailles et des tentures. Ces ifs tondus figurent des vases et des lyres. Ces parterres sont des tapis à ramages. Dans ces allées unies et rectilignes, le roi, la canne à la main, groupera autour de lui tout son cortège. Soixante dames, en robes lamées et bouffantes sur des paniers qui ont vingt-quatre pieds de circonférence, s’espaceront sans peine sur les marches de ces escaliers. Ces cabinets de verdure pourront abriter une collation princière\footnote{Cabinet des Estampes, {\itshape Histoire de France par estampes}, passim, notamment plans et vues de Versailles par Aveline, « et dessin de la collation donnée par M. le Prince dans le milieu du Labyrinthe de Chantilly, le 29 août 1687 ».}. Sous ce portique circulaire, tous les seigneurs qui ont l’entrée de la chambre pourront assister ensemble au jeu d’un nouveau jet d’eau. Ils retrouvent leurs pareils jusque dans les figures de marbre et de bronze qui peuplent les allées et les bassins, jusque dans la contenance digne d’un Apollon, dans l’air théâtral d’un Jupiter, dans l’aisance mondaine et dans la nonchalance voulue d’une Diane ou d’une Vénus. Les dieux eux-mêmes sont de leur monde  Enfoncée par l’effort de toute une société et de tout un siècle, l’empreinte de la cour est si forte, qu’elle s’est gravée dans le détail comme dans l’ensemble et dans les choses de la matière comme dans les choses de l’esprit.

\section[{II. La maison du roi. — Personnel et dépenses. — Sa maison militaire, son écurie, sa vénerie, sa chapelle, sa faculté, sa bouche, sa chambre, sa garde-robe, ses bâtiments, son garde-meuble, ses voyages.}]{II. La maison du roi. — Personnel et dépenses. — Sa maison militaire, son écurie, sa vénerie, sa chapelle, sa faculté, sa bouche, sa chambre, sa garde-robe, ses bâtiments, son garde-meuble, ses voyages.}

\noindent Ceci n’est que le cadre ; avant 1789, il était rempli. « On n’a rien vu, dit Chateaubriand, quand on n’a pas vu la pompe de Versailles, même après le licenciement de l’ancienne maison du roi ; Louis XIV était toujours là\footnote{{\itshape Mémoires}, I, 221. Il avait été présenté le 19 février 1787.}. » C’est un fourmillement de livrées, d’uniformes, de costumes et d’équipages, aussi brillant et aussi varié que dans un tableau ; j’aurais voulu vivre huit jours dans ce monde ; il est fait à peindre, arrangé exprès pour le plaisir des yeux, comme une scène d’opéra. Mais comment nous figurer aujourd’hui des gens pour qui la vie était un opéra ? En ce temps-là, il faut à un grand un grand état de maison ; son cortège et son décor font partie de sa personne il se manque à lui-même s’il ne les a pas aussi amples et aussi beaux qu’il le peut ; il serait choqué d’un vide dans sa maison comme nous d’un trou dans notre habit. S’il se retranche, il déchoit ; quand Louis XVI fait des réformes, la cour dit qu’il agit en bourgeois. Dès qu’un prince ou une princesse est d’âge, on lui forme une maison ; dès qu’un prince se marie, on forme une maison à sa femme ; et par maison entendez une représentation à quinze ou vingt services distincts, écurie, vénerie, chapelle, faculté, chambre, garde-robe, chambre aux deniers, bouche, paneterie-bouche, cuisine-bouche, échansonnerie, fruiterie, fourrerie, cuisine-commun, cabinet, conseil\footnote{Pour tous les détails suivants, cf. Waroquier, t. I, passim. {\itshape Archives nationales}, O1, 710 {\itshape bis.} Maison du roi, dépenses de 1771  Marquis d’Argenson, 25 février 1752  En 1771, on dépense 3 millions pour l’installation de la comtesse d’Artois. Un simple appartement pour Madame Adélaide coûte 800 000 livres.} ; elle ne se sent point princesse sans cela. Il y a 274 charges chez le duc d’Orléans, 210 chez Mesdames, 68 chez Madame Élisabeth, 239 chez la comtesse d’Artois, 256 chez la comtesse de Provence, 496 chez la reine. Lorsqu’il s’agit de former une maison à Madame Royale, âgée d’un mois, « la reine, écrit l’ambassadeur d’Autriche, veut supprimer une mollesse nuisible, une affluence inutile de gens de service, et tout usage propre à faire naître des sentiments d’orgueil. Malgré le retranchement susdit, la maison de la jeune princesse se montera encore à près de 80 personnes destinées au service unique de sa personne royale\footnote{Marie-Antoinette, {\itshape Correspondance secrète}, par Arneth et Geffroy, III, 292. Lettre de Mercy, du 25 janvier 1779  Waroquier, en 1789, ne mentionne que 15 charges dans la maison de Madame Royale. Ceci, outre beaucoup d’autres indices, montre combien les chiffres officiels sont insuffisants.} ». La maison civile de Monsieur en comprend 420 et sa maison militaire 179 ; celle du comte d’Artois 237 et sa maison civile 456  Les trois quarts sont pour la montre ; avec leurs broderies et leurs galons, avec leur contenance dégagée et polie, leur air attentif et discret, leur belle façon de saluer, de marcher, de sourire, ils font bien, alignés dans une antichambre ou espacés par groupes dans une galerie ; j’aurais même voulu contempler les escouades des écuries et des cuisines ; ce sont les figurants qui remplissent le fond du tableau  Par cet éclat des astres secondaires, jugez de la splendeur du soleil royal. Il faut au roi une garde, infanterie, cavalerie, gardes du corps, gardes françaises, gardes suisses, Cent-Suisses, chevau-légers de la garde, gendarmes de la garde, gardes de la porte, 9 050 hommes\footnote{C’est le chiffre auquel on arrive après les réductions de 1775 et de 1776, avant celles de 1787. Voyez Waroquier, t. I  Necker, {\itshape Administration des finances.} II, 119.}, coûtant chaque année 7 681 000 livres. Quatre compagnies des gardes françaises et deux des gardes suisses font tous les jours la parade dans la cour des ministres, entre les deux grilles, et le spectacle est magnifique quand le roi sort en carrosse pour aller à Paris ou à Fontainebleau. Quatre trompettes sonnent à l’avant et quatre en arrière. Les gardes suisses d’un côté, les gardes françaises de l’autre\footnote{Cabinet des Estampes, {\itshape La maison du roi en} 1786 (estampes coloriées).} font la haie aussi loin qu’elle peut s’étendre. Devant les chevaux marchent les Cent-Suisses en costume du quinzième siècle, avec la pertuisane, la fraise, le chapeau à panache, l’ample pourpoint bariolé de couleurs mi-parties, à côté d’eux les gardes de la prévôté, à brandebourgs d’or et parements d’écarlate, avec des hoquetons tout hérissés de bouillons d’orfèvrerie. Dans tous les corps, les officiers, les trompettes, les musiciens, chamarrés de passementeries d’or et d’argent, sont éblouissants à voir ; la timbale pendue à l’arçon de la selle, toute brodée et surchargée d’ornements peints et dorés, est une pièce à mettre dans un garde-meuble ; le cymbalier nègre des gardes françaises ressemble à un soudan de féerie. — Derrière le carrosse et sur les flancs courent les gardes du corps, avec l’épée et la carabine, en culottes rouges, grandes bottes noires, habit bleu couturé de broderies blanches, tous gentilshommes vérifiés ; il y en a 1 200, choisis à la noblesse et à la taille ; parmi eux sont les gardes de la manche, plus intimes encore, qui, à l’église, aux cérémonies, en hoqueton blanc étoilé de papillotes d’argent et d’or, ayant en main leur pertuisane damasquinée, sont toujours debout et tournés vers le roi « pour avoir de toutes parts l’œil sur sa personne ». Voilà pour sa sûreté. — Étant gentilhomme, il est cavalier, et il lui faut une écurie proportionnée\footnote{{\itshape Archives nationales.} O1, 738. Rapport de M. Teissier (1780) sur la grande et la petite écurie. — L’écurie de la reine comprend 75 voitures et 330 chevaux. Ce sont là les chiffres véritables, extraits des rapports secrets et manuscrits ; ils montrent l’insuffisance des chiffres officiels. Par exemple, l’Almanach de Versailles de 1775 compte seulement 335 hommes dans les écuries, et l’on voit qu’effectivement le nombre était quadruple ou quintuple. — « Avant toutes les réformes, dit un témoin, je crois que le nombre des chevaux du roi allait bien à 3 000. » (Comte d’Hézecques, \href{http://gallica.bnf.fr/ark:/12148/bpt6k49961n/f138}{\dotuline{{\itshape Souvenirs} {\itshape d’} {\itshape un page} {\itshape de Louis XVI}, 121}} [\url{http://gallica.bnf.fr/ark:/12148/bpt6k49961n/f138}].)}, 1 857 chevaux, 217 voitures, 1 458 hommes qu’il habille et dont la livrée coûte 540 000 francs par an ; outre cela, 38 écuyers de main, cavalcadours et ordinaires ; outre cela, 20 gouverneurs, sous-gouverneurs, aumôniers, professeurs, cuisiniers et valets pour gouverner, instruire et servir les pages ; outre cela, une trentaine de médecins, apothicaires, garde-malades, intendants, trésoriers, ouvriers, marchands brevetés et payés pour les accessoires de ce service : en tout plus de 1 500 hommes. On achète pour 250 000 francs de chevaux par an, et il y a des haras en Limousin et en Normandie pour la remonte. 287 chevaux sont exercés tous les jours dans les deux manèges ; il y a 443 chevaux de selle dans la petite écurie, 437 dans la grande, et cela ne suffit pas à la « vivacité du service ». Le tout coûte 4 600 000 livres en 1775 et monte à 6 200 000 livres en 1787\footnote{{\itshape La maison du roi justifiée par un soldat citoyen} (1786), d’après les comptes publiés par le gouvernement. — {\itshape La future maison du roi} (1790). « Les deux écuries ont dépensé en 1786, la grande 4 207 606 livres, la petite 3 509 402 livres, total 7 717 008 livres, dont 486 546 livres en achats de chevaux. »}. Encore un spectacle qu’il faudrait voir avec les yeux de la tête, pages\footnote{« À mon arrivée à Versailles (1786) on y comptait 150 pages, sans compter ceux des princes du sang qui résidaient à Paris. Un seul habit de page de la chambre coûtait 1 500 livres (velours cramoisi brodé d’or sur toutes les tailles, chapeau garni d’un plumet et d’un large point d’Espagne). » (Comte d’Hézecques, \href{http://gallica.bnf.fr/ark:/12148/bpt6k49961n/f129}{\dotuline{{\itshape ib.}, 112}} [\url{http://gallica.bnf.fr/ark:/12148/bpt6k49961n/f129}]).}, piqueurs, élèves galonnés, élèves à boutons d’argent, garçons de la petite livrée en soie, joueurs d’instruments, chevaucheurs de l’écurie. C’est un art féodal que l’emploi du cheval ; il n’y a pas de luxe plus naturel à un homme de qualité ; pensez aux écuries de Chantilly, qui sont des palais. Pour dire un homme bien élevé et distingué, on disait alors « un cavalier accompli » ; en effet, il n’avait toute sa prestance qu’en selle et sur un cheval de race comme lui. — Autre goût de gentilhomme, qui est une suite du précédent : la chasse. Elle coûte au roi de 1 100 000 à 1 200 000 livres par an\footnote{{\itshape Archives nationales}, O1, 778. Mémoire sur la vénerie de 1760 à 1792 et notamment rapport de 1786.} et occupe 280 chevaux, outre ceux des deux écuries. On ne saurait imaginer un équipage plus varié ni plus complet : meute pour le sanglier, meute pour le loup, meute pour le chevreuil, vol pour corneille, vol pour pie, vol pour émerillon, vol pour lièvre, vol pour les champs. On dépense, en 1783, 179 194 livres pour la nourriture des chevaux et 53 412 livres pour celle des chiens. Tout le territoire, à dix lieues de Paris, est chasse gardée ; « on n’y saurait tirer un coup de fusil\footnote{Mercier, {\itshape Tableau de Paris}, I, 11 ; {\itshape V}, 62. — Comte d’Hézecques, \href{http://gallica.bnf.fr/ark:/12148/bpt6k49961n/f270}{\dotuline{{\itshape ib.}, 253}} [\url{http://gallica.bnf.fr/ark:/12148/bpt6k49961n/f270}]. — {\itshape Journal de Louis XVI}, publié par Nicolardot, passim.}, aussi voyez-vous dans toutes les plaines les perdrix, familiarisées avec l’homme, becqueter le grain tranquillement et ne point s’écarter quand il passe ». Joignez-y les capitaineries des princes jusqu’à Villers-Cotterets et Orléans ; cela fait, autour de Paris, un cercle presque continu, ayant trente lieues de rayon, où le gibier, protégé, remisé, multiplie, fourmille pour les plaisirs du roi. Le seul parc de Versailles est une enceinte close de plus de dix lieues. La forêt de Rambouillet comprend 25 000 arpents. On rencontre autour de Fontainebleau des bandes de soixante-dix à quatre-vingts cerfs. En lisant les carnets des chasses, il n’y a pas de vrai chasseur qui n’éprouve un mouvement d’envie. L’équipage du loup court toutes les semaines et prend 40 loups par an. De 1743 à 1774, Louis XV force 6 400 cerfs. Louis XVI écrit le 31 août 1781 : « Aujourd’hui tué 460 pièces ». En 1780, il abat 20 534 pièces ; en 1781, 20 291 ; en quatorze ans, 189 251 pièces, outre 1 254 cerfs ; les sangliers, les chevreuils, sont en proportion ; et notez que tout cela est sous sa main, puisque ses parcs confinent à ses maisons  Tel est en effet le caractère d’une « maison montée », c’est-à-dire munie de ses dépendances et de ses services ; tout y est à portée : c’est un monde complet qui se suffit à lui-même. Une grande vie se rattache et rassemble autour d’elle, avec une prévoyance universelle et un détail minutieux, tous les appendices dont elle use ou dont elle pourrait user  Ainsi chaque prince, chaque princesse a sa faculté, sa chapelle\footnote{Waroquier, t. I, passim. Maison de la reine, chapelle 22 personnes, faculté 6. Maison de Monsieur, chapelle 22, faculté 21. Maison de Madame, chapelle 20, faculté 9. Maison du comte d’Artois, chapelle 20, faculté 28. Maison de la comtesse d’Artois, chapelle 19, faculté 17. Maison du duc d’Orléans, chapelle 6, faculté 19.}, il ne convient pas que l’aumônier qui leur dit la messe, que le chirurgien qui les soigne, soient d’emprunt. À plus forte raison faut-il au roi les siens : pour sa chapelle, 75 aumôniers, chapelains, confesseurs, maîtres de l’oratoire, clercs, avertisseurs, sommiers de chapelle, chantres, noteurs, compositeurs de musique sacrée ; pour sa faculté, 48 médecins, chirurgiens, apothicaires, oculistes, opérateurs, renoueurs, distillateurs, pédicures et spagiriques. Notez encore sa musique profane, 128 chanteurs, danseurs, instrumentistes, maîtres et surintendants ; son cabinet de livres, 43 conservateurs, lecteurs, interprètes, graveurs, médaillistes, géographes, relieurs, imprimeurs ; le personnel qui orne ses cérémonies, 62 hérauts, porte-épées, introducteurs et musiciens ; le personnel qui pourvoit à ses logements, 68 maréchaux des logis, guides et fourriers. J’omets d’autres services, j’ai hâte d’arriver au centre, la bouche ; c’est à la table qu’on reconnaît une grande maison.\par
Il y a trois divisions de la bouche\footnote{{\itshape Archives nationales}, O1, 738. Rapports par M. Mesnard de Chouzy (mars 1780). — Là-dessus une réforme suivit (17 août 1780). — {\itshape La maison du roi justifiée} (1789), 24. En 1788, la dépense de bouche est réduite à 2 870 000 livres, dont 600 000 livres données à Mesdames pour leur bouche.} : la première pour le roi et ses enfants en bas âge ; la seconde, nommée petit commun, pour la table du grand maître, pour celle du grand chambellan et pour celle des princes et princesses qui logent chez le roi ; la troisième, nommée grand commun pour la seconde table du grand maître, pour celle des maîtres d’hôtel, pour celle des aumôniers, pour celle des gentilshommes servants et pour celle des valets de chambre : en tout 383 officiers de bouche, 103 garçons et 2 177 771 livres de dépense ; outre cela 389 173 livres pour la bouche de Madame Élisabeth, et 1 093 547 livres pour celles de Mesdames, total 3 660 491 livres pour la table. Le marchand de vin fournit par an pour 300 000 francs de vin et le pourvoyeur pour un million de gibier, viande et poisson. Rien que pour aller à Ville-d’Avray chercher l’eau, et pour voiturer les officiers, garçons et provisions, il faut 50 chevaux loués 70 591 francs par an. Les princes et princesses du sang, ayant le droit « d’envoyer prendre du poisson à la recette les jours maigres, quand ils ne font pas à la cour de résidence suivie », ce seul article revient, en 1778, à 175 116 livres. Lisez dans l’Almanach les titres des offices, et vous verrez se développer devant vous une fête de Gargantua, la solennelle hiérarchie des cuisines, grands officiers de la bouche, maîtres d’hôtel, contrôleurs, contrôleurs-élèves, commis, gentilshommes panetiers, échansons et tranchants, écuyers et huissiers de cuisine, chefs, aides et maîtres-queux, enfants de cuisine et galopins ordinaires, coureurs de vins et hâteurs de rôts, potagers, verduriers, lavandiers, pâtissiers, serdeaux, porte-tables, gardes-vaisselle, sommiers des broches, maître d’hôtel de la table du premier maître d’hôtel, toute une procession de dos amples et galonnés, de ventres majestueux et rebondis, de figures sérieuses qui, devant les casseroles, autour des buffets, officient avec ordre et conviction  Encore un pas et nous entrons dans le sanctuaire, l’appartement du roi. Deux dignitaires principaux y président, et chacun d’eux a sous ses ordres une centaine de subordonnés : d’un côté le grand chambellan avec les premiers gentilshommes de la chambre, avec les pages de la chambre, leurs gouverneurs et précepteurs, avec les huissiers de l’antichambre, avec les quatre premiers valets de chambre ordinaires, avec les seize valets de chambre par quartier, avec les porte-manteaux ordinaires et par quartier, avec les barbiers, tapissiers, horlogers, garçons et porteurs ; de l’autre côté, le grand-maître de la garde-robe, avec les maîtres de la garde-robe, avec les valets de la garde-robe ordinaires et par quartier, avec le porte-malle, le porte-mail, les tailleurs, les lavandiers, l’empeseur et les garçons ordinaires, avec les gentilshommes ordinaires, les huissiers et secrétaires de cabinet, en tout 198 personnes pour le service intime, comme autant d’ustensiles domestiques pour tous les besoins de la personne ou de meubles somptueux pour la décoration de l’appartement. Il y en a pour aller chercher le mail et les boules, pour tenir le manteau et la canne, pour peigner le roi et l’essuyer au bain, pour commander les mulets qui transportent son lit, pour gouverner les levrettes de sa chambre, pour lui plier, passer et nouer sa cravate, pour enlever et rapporter sa chaise percée\footnote{Comte d’Hézecques, \href{http://gallica.bnf.fr/ark:/12148/bpt6k49961n/f229}{\dotuline{{\itshape ib.} . 212}} [\url{http://gallica.bnf.fr/ark:/12148/bpt6k49961n/f229}]. Sous Louis XVI, il y avait deux porte-chaises du roi, qui tous les matins, en habit de velours, l’épée au côté, venaient vérifier et vider, s’il y avait lieu, l’objet de leurs fonctions ; cette charge valait à chacun d’eux 20 000 livres par an.}. Il y en a surtout dont tout l’office est d’être là et de remplir un coin qui ne doit pas rester vide. Certainement, pour le port et l’aisance, ils sont les premiers de tous ; si proches du maître, ils y sont obligés ; dans un tel voisinage, leur tenue ne doit pas faire disparate  Telle est la maison du roi, et je n’ai décrit qu’une de ses résidences ; il y en a une douzaine, outre Versailles, grandes ou petites, Marly, les deux Trianon, la Muette, Meudon, Choisy, Saint-Hubert, Saint-Germain, Fontainebleau, Compiègne, Saint-Cloud, Rambouillet\footnote{En 1787, Louis XVI démolit ou ordonne de vendre Madrid, la Muette, Choisy ; mais ses acquisitions, Saint-Cloud, l’Isle-Adam, Rambouillet, ont de beaucoup surpassé ses réformes.}, sans compter le Louvre, les Tuileries et Chambord, avec leurs parcs et territoires de chasse, avec leurs gouverneurs, inspecteurs, contrôleurs, concierges, fontainiers, jardiniers, balayeurs, frotteurs, taupiers, gruyers, gardes à cheval et à pied, plus de 1 000 personnes. Naturellement il entretient, plante et bâtit ; à cela il dépense 3 ou 4 millions par année\footnote{Necker, {\itshape Compte rendu}, II, 452  {\itshape Archives nationales}, O1, 738, p. 62 et 64 ; O1, 2805 ; O1, 736  {\itshape La maison du roi justifiée} (1789). Bâtiments en 1775, 3 924 400 l. ; en 1786, 4 000 000 l. ; en 1788, 3 077 000 livres. — Garde-meuble en 1788, 1 700 000 livres.}. Naturellement aussi il répare et renouvelle ses ameublements ; en 1778, qui est une année moyenne, cela lui coûte 1 936 853 livres. Naturellement aussi il y mène ses hôtes et les y défraye, eux et leurs gens : à Choisy, en 1780, outre les distributions, il y a 16 tables et 345 couverts ; à Saint-Cloud, en 1785, il y a 26 tables ; « un voyage à Marly de 21 jours est un objet de 120 000 livres de dépense extraordinaire » le voyage à Fontainebleau a coûté jusqu’à 400 000 et 500 000 livres. En moyenne, ses déplacements exigent par an un demi-million et davantage\footnote{Voici quelques autres dépenses accidentelles ({\itshape Archives nationales}, O1, 2805). Pour la naissance du duc de Bourgogne, en 1751, 604 477 l. Pour le mariage du dauphin, en 1770, 1 267 770 l. Pour le mariage du comte d’Artois, en 1773, 2 016 221 l. Pour le sacre, en 1775, 835 862 l. Pour les comédies, bals et concerts, en 1778, 481 744 l. ; en 1779, 382 986 l.}  Pour achever de concevoir ce prodigieux attirail, songez que « des artisans et marchands de tous les corps d’état sont obligés, par leur privilège, de suivre la cour » dans ses voyages, afin de la fournir sur place : « apothicaires, armuriers, arquebusiers, bonnetiers-vendeurs de bas de soie et de laine, bouchers, boulangers, brodeurs, cabaretiers, carreleurs de souliers, ceinturiers, chandeliers, chapeliers, charcutiers, chirurgiens, cordonniers, corroyeurs-baudroyeurs, cuisiniers, découpeurs-égratigneurs, doreurs et graveurs, éperonniers, épiciers-confituriers, fourbisseurs, fripiers, gantiers-parfumeurs, horlogers, libraires, lingers, marchands-vendeurs de vin en gros et en détail, menuisiers, merciers-joailliers-grossiers, orfèvres, parcheminiers, passementiers, poulaillers-rôtisseurs et poissonniers, proviseurs de foin, paille et avoine, quincailliers, selliers, tailleurs, vendeurs de pain d’épice et d’amidon, verduriers-fruitiers, verriers et violons\footnote{Waroquier, I, {\itshape ib. — Marie-Antoinette}, par Arneth et Geffroy. Lettre de Mercy du 16 septembre 1773. « La multitude du service qui suit le roi dans ses voyages ressemble à la marche d’une armée. »} ». On dirait d’une cour d’Orient qui, pour se mouvoir, entraîne tout un monde : « quand elle va s’ébranler, il faut, si l’on veut passer, prendre la poste d’avance ». Au total, près de 4 000 personnes pour la maison civile du roi, 9 000 à 10 000 pour sa maison militaire, 2 000 au moins pour celles de ses proches, en tout près de 15 000 personnes avec une dépense de 40 à 45 millions, qui en vaudraient le double aujourd’hui et qui sont alors le dixième du revenu public\footnote{Maison civile du roi, de la reine, de Madame Élisabeth, de Mesdames, de Madame Royale, 25 700 000 l  Aux frères et belles-sœurs du roi, 8 040 000 l  Maison militaire du roi, 7 681 000 l. (Necker, {\itshape Compte rendu}, II, 119)  De 1774 à 1788, la dépense des maisons du roi et de sa famille flotte entre 32 et 36 millions, non compris la maison militaire ({\itshape La maison du roi justifiée}). En 1789, la maison du roi, de la reine, du Dauphin, des enfants de France, de Mesdames coûte 25 millions  Celles de Monsieur et de Madame, 3 656 000 l. ; celles du comte et de la comtesse d’Artois, 3 656 000 l. ; ducs de Berry et d’Angoulême, 700 000 l. ; les traitements conservés aux personnes qui ont servi les princes montent à 228 000 l. Total 33 240 000 livres  À quoi il faut ajouter la maison militaire du roi et les 2 millions en apanage des princes. ({\itshape Compte général des revenus et dépenses fixes au 1\textsuperscript{er} mai} 1789, {\itshape remis par M. le premier ministre des finances à MM. du Comité des finances de l’Assemblée nationale.})}. Voilà la pièce centrale du décor monarchique. Si grande et si dispendieuse qu’elle soit, elle n’est que proportionnée à son usage, depuis que la cour est une institution publique et que l’aristocratie, occupée à vide, s’emploie à remplir le salon du roi.

\section[{III. La société du roi. — Officiers de sa maison. — Invités de son salon.}]{III. La société du roi. — Officiers de sa maison. — Invités de son salon.}

\noindent Deux causes y maintiennent cette affluence : l’une qui est la forme féodale conservée, l’autre qui est la nouvelle centralisation introduite ; l’une qui met le service du roi entre les mains des nobles, l’autre qui change les nobles en solliciteurs  Par les charges du palais, la première noblesse vit chez le roi, à demeure : grand aumônier, M. de Montmorency-Laval, évêque de Metz ; premier aumônier, M. de Bessuéjouls de Roquelaure, évêque de Senlis ; grand maître de France, le prince de Condé ; premier maître d’hôtel, le comte des Cars ; maître d’hôtel ordinaire, le marquis de Montdragon ; premier panetier, le duc de Brissac ; grand échanson, le marquis de Verneuil ; premier tranchant, le marquis de la Chesnaye ; premiers gentilshommes de la chambre, les ducs de Richelieu ; de Durfort, de Villequier, de Fleury ; grand maître de la garde-robe, le duc de La Rochefoucauld-Liancourt ; maîtres de la garde-robe, le comte de Boisgelin et le marquis de Chauvelin ; capitaine de la fauconnerie, le chevalier de Forget ; capitaine du vautrait, le marquis d’Ecquevilly ; surintendant des bâtiments, le comte d’Angiviller ; grand écuyer, le prince de Lambesc ; grand veneur, le duc de Penthièvre ; grand maître des cérémonies, le marquis de Brézé ; grand maréchal des logis, le marquis de la Suze ; capitaines des gardes, les ducs d’Ayen, de Villeroy, de Brissac, d’Aiguillon et de Biron, les princes de Poix, de Luxembourg et de Soubise ; prévôt de l’hôtel, le marquis de Tourzel ; gouverneurs des résidences et capitaines des chasses, le duc de Noailles, le marquis de Champcenetz ; le baron de Champlost, le duc de Coigny, le comte de Modène, le comte de Montmorin, le duc de Laval, le comte de Brienne, le duc d’Orléans, le duc de Gesvres\footnote{Waroquier, {\itshape ibid.} (1789), t. I, passim.}. Tous ces seigneurs sont pour le roi des familiers obligés, des hôtes perpétuels et le plus souvent héréditaires, logés chez lui, en société intime et quotidienne avec lui, puisqu’ils sont « ses gens\footnote{Mot du comte d’Artois en présentant à sa femme les officiers de sa maison.} » et font le service domestique de sa personne. Ajoutez-y leurs pareils, aussi nobles et presque aussi nombreux chez la reine, chez Mesdames, chez Madame Élisabeth, chez le comte et chez la comtesse de Provence, chez le comte et chez la comtesse d’Artois  Et ce ne sont là que les chefs d’emploi ; si, au-dessous d’eux, dans les offices, je compte les titulaires nobles, j’y trouve, entre autres, 68 aumôniers ou chapelains, 170 gentilshommes de la chambre ou servants, 117 gentilshommes de l’écurie et de la vénerie, 148 pages, 114 dames de compagnie titrées, en outre tous les officiers jusqu’au plus petit de la maison militaire, sans compter 1 400 simples gardes qui, vérifiés par le généalogiste\footnote{Le nombre des chevau-légers et des gendarmes a été réduit en 1775 et 1776 ; les deux corps sont supprimés en 1787.}, sont admis sur ce titre à faire leur cour. Telle est la recrue fixe des réceptions royales ; c’est le trait distinctif de ce régime que les serviteurs y sont des hôtes, et que l’antichambre y peuple le salon.\par
Non que le salon ait besoin de cela pour se remplir. Étant la source de tout avancement et de toute grâce, il est naturel qu’il regorge ; dans notre société égalitaire, celui d’un mince député, d’un médiocre journaliste, d’une femme à la mode, est plein de courtisans sous le nom de visiteurs et d’amis. — D’ailleurs ici la présence est d’obligation ; on pourrait dire qu’elle est une continuation de l’ancien hommage féodal ; l’état-major des nobles est tenu de faire cortège à son général-né. Dans le langage du temps, cela s’appelle « rendre ses devoirs au roi ». Aux yeux du prince, l’absence serait une marque d’indépendance autant que d’indifférence, et la soumission, aussi bien que l’empressement, lui est due. — À cet égard, il faut voir l’institution dès son origine. Du regard, à chaque instant Louis XIV faisait sa ronde, « à son lever, à son coucher, à ses repas, en passant dans ses appartements, dans ses jardins… : aucun ne lui échappait, jusqu’à ceux qui n’espéraient pas même être vus ; c’était un démérite aux uns et à tout ce qu’il y avait de plus distingué de ne pas faire de la cour son séjour ordinaire, aux autres d’y venir rarement, et une disgrâce sûre pour qui n’y venait jamais ou comme jamais\footnote{Saint-Simon, {\itshape Mémoires}, XVI, 456. — Ce besoin d’être entouré dure jusqu’à la fin ; en 1791, la reine disait amèrement en parlant de la noblesse : « Quand on obtient de nous une démarche qui la blesse, je suis boudée ; personne ne vient à mon jeu ; le coucher du roi est solitaire, on nous punit de nos malheurs ». (Mme Campan, II, 177.)} ». Dorénavant pour les premiers personnages du royaume, hommes et femmes, ecclésiastiques et laïques, la grande affaire, le principal emploi de la vie, le vrai travail, sera d’être à toute heure, en tout lieu, sous les yeux du roi, à portée de sa parole ou de son regard. « Qui considérera, dit La Bruyère, que le visage du prince fait toute la félicité du courtisan, qu’il s’occupe et se remplit toute sa vie de le voir et d’en être vu, comprendra un peu comment voir Dieu fait toute la gloire et toute la félicité des saints. » Il y eut alors des prodiges d’assiduité et d’assujettissement volontaire. Tous les matins à sept heures, en hiver comme en été, le duc de Fronsac, par ordre de son père, se trouvait au bas du petit escalier qui conduit à la chapelle, uniquement pour donner la main à Mme de Maintenon qui partait pour Saint-Cyr\footnote{Duc de Lévis, {\itshape Souvenirs et portraits.} 29. — Mme de Maintenon, \href{http://gallica.bnf.fr/ark:/12148/bpt6k206443s}{\dotuline{{\itshape Correspondance}}} [\url{http://gallica.bnf.fr/ark:/12148/bpt6k206443s}].}. « Pardonnez-moi, Madame, lui écrivait le duc de Richelieu, l’extrême liberté que je prends d’oser vous envoyer la lettre que j’écris au roi, par où je le prie à genoux qu’il me permette de lui aller faire de Ruel quelquefois ma cour ; car {\itshape j’aime autant mourir que d’être deux mois sans le voir.} » Le vrai courtisan suivait le prince comme l’ombre suit le corps ; tel fut sous Louis XIV le duc de La Rochefoucauld, grand veneur. « Le lever, le coucher, les deux autres changements d’habit tous les jours, les chasses et les promenades du roi tous les jours aussi, il n’en manquait jamais, quelquefois dix ans de suite sans découcher d’où était le roi, et sur pied de demander un congé, non pour découcher, car en plus de quarante ans il n’a jamais couché vingt fois hors de Paris, mais pour aller dîner hors de la cour et ne pas être de la promenade. » — Si plus tard, sous des maîtres moins exigeants et dans le relâchement général du dix-huitième siècle, cette discipline se détend, l’institution subsiste\footnote{M. de V., qui avait la promesse d’une lieutenance du roi ou d’un commandement, la cède à l’un des protégés de Mme de Pompadour, et obtient en échange le rôle d’{\itshape exempt} dans {\itshape Tartuffe} que des seigneurs de la cour jouaient dans les petits cabinets devant le roi. (Mme de Dausset, 168) « M. de V. remercia Madame comme si elle l’eût fait duc. »}, à défaut de l’obéissance, la tradition, l’intérêt et l’amour-propre suffiraient pour peupler la cour. Approcher du roi, être domestique dans sa maison, huissier, porte-manteau, valet de chambre, est un privilège qu’on achète, même en 1789, trente, quarante et cent mille livres ; à plus forte raison sera-ce un privilège, et le plus honorable, le plus utile, le plus envié de tous, de faire partie de sa société  D’abord, c’est une preuve de race. Un homme, pour suivre le roi à la chasse, une femme pour être présentée à la reine, doit établir au préalable, devant le généalogiste et par pièces authentiques, que sa noblesse remonte à l’an 1400  Ensuite c’est une certitude de fortune ; il n’y a que ce salon pour être à portée des grâces ; aussi bien, jusqu’en 1789, les grandes familles ne bougent pas de Versailles, et, nuit et jour, sont à l’affût. Le valet de chambre du maréchal de Noailles lui disait un soir en fermant ses rideaux : « À quelle heure Monseigneur veut-il que je l’éveille demain  À dix heures, s’il ne meurt personne cette nuit\footnote{{\itshape Paris, Versailles et les provinces au dix-huitième siècle}, II, 160, 168. — Mercier, {\itshape Tableau de Paris}, IV, 150. — Comte de Ségur, {\itshape Mémoires}, I, 16.} ». On trouve encore de ces vieux courtisans, qui « âgés de quatre-vingts ans, en ont bien passé quarante-cinq sur leurs pieds dans l’antichambre du roi, des princes et des ministres »  « Vous n’avez que trois choses à faire, disait l’un d’eux à un débutant : dites du bien de tout le monde, demandez tout ce qui vaquera, et asseyez-vous quand vous pourrez. »\par
C’est pourquoi, autour du prince, il y a toujours foule. Le 1\textsuperscript{er} août 1773, la comtesse du Barry présentant sa nièce, « le cortège est si nombreux, partout où cette présentation passe, qu’on peut à peine traverser les antichambres\footnote{ \noindent {\itshape Marie-Antoinette}, par Arneth et Geffroy. II, 27, 255, 281. — {\itshape Gustave III}, par Geffroy, novembre 1786. Bulletin de Mme de Staël. — Comte d’Hézecques, \href{http://gallica.bnf.fr/ark:/12148/bpt6k49961n/f268}{\dotuline{{\itshape ib.}, 251}} [\url{http://gallica.bnf.fr/ark:/12148/bpt6k49961n/f268}]. — {\itshape Archives nationales.} O1, 736. Lettre de M. Amelot, du 23 septembre 1780. — Duc de Luynes, XV, 260, 367 ; XVI, 268. 163 dames, dont 42 de service, viennent faire la révérence au roi. 160 hommes et plus de 100 dames viennent rendre leurs devoirs au dauphin et à la dauphine.
 } ». En décembre 1774, à Fontainebleau, où tous les soirs la reine tient son jeu, « l’appartement, quoique vaste, ne désemplit pas… La presse est telle, qu’on ne peut parler qu’aux deux ou trois personnes avec lesquelles on joue ». Aux réceptions d’ambassadeurs, les quatorze appartements sont pleins et combles de seigneurs et de femmes parées. Le 1\textsuperscript{er} janvier 1775 la reine « a compté au-delà de deux cents femmes qui se sont présentées pour lui faire leur cour ». En 1780, à Choisy, il y a tous les jours une table de trente couverts pour le roi, une autre de trente couverts pour les seigneurs, outre quarante couverts pour les officiers de garde et les écuyers, outre cinquante couverts pour les officiers de la chambre. J’estime qu’à son lever, à son coucher, dans ses promenades, à sa chasse, à son jeu, le roi a toujours autour de lui, outre les gens de service, quarante ou cinquante seigneurs au moins, plus souvent une centaine, et autant de dames ; à Fontainebleau, en 1756, quoiqu’il n’y eût « cette année-là ni fêtes ni ballets, on comptait cent six dames ». Quand le roi tient « grand appartement », lorsqu’il donne à jouer ou à danser dans la galerie des glaces, quatre ou cinq cents invités, l’élite de la noblesse et de la mode s’ordonnent sur les banquettes ou se pressent autour des tables de cavagnole et de tri\footnote{Cochin. Estampes, bal masqué, bal paré, jeu du roi et de la reine, salle de spectacle (1745). Costumes de Moreau (1777). — Mme de Genlis, {\itshape Dictionnaire des Étiquettes}, article {\itshape Parure}.}. Voilà le spectacle qu’il faudrait voir, non par l’imagination et d’après des textes incomplets, mais avec les yeux et sur place, pour comprendre l’esprit, l’effet, le triomphe de la culture monarchique ; dans une maison montée, le salon est la pièce principale ; et il n’y en eut jamais de plus éblouissant que celui-ci. De la voûte sculptée et peuplée d’amours folâtres, descendent, par des guirlandes de fleurs et de feuillage, les lustres flamboyants dont les hautes glaces multiplient la splendeur ; la lumière rejaillit à flots sur les dorures, sur les diamants, sur les têtes spirituelles et gaies, sur les fins corsages, sur les énormes robes enguirlandées et chatoyantes. Les paniers des dames rangées en cercle ou étagées sur les banquettes « forment un riche espalier couvert de perles, d’or, d’argent, de pierreries, de paillons, de fleurs, de fruits avec leurs fleurs, groseilles, cerises, fraises artificielles » ; c’est un gigantesque bouquet vivant dont l’œil a peine à soutenir l’éclat  Point d’habits noirs comme aujourd’hui pour faire disparate. Coiffés et poudrés, avec des boucles et des nœuds, en cravates et manchettes de dentelle, en habits et vestes de soie feuille morte, rose tendre, bleu céleste, agrémentés de broderies et galonnés d’or, les hommes sont aussi parés que les femmes. Hommes et femmes, on les a choisis un à un ; ce sont tous des gens du monde accomplis, ornés de toutes les grâces que peuvent donner la race, l’éducation, la fortune, le loisir et l’usage ; dans leur genre, ils sont parfaits. Il n’y a pas une toilette ici, pas un air de tête, pas un son de voix, pas une tournure de phrase qui ne soit le chef-d’œuvre de la culture mondaine, la quintessence distillée de tout ce que l’art social peut élaborer d’exquis. Si polie que soit la société de Paris, elle n’en approche pas\footnote{« Il y avait à peu près une différence aussi sensible entre le ton, le langage de la cour et celui de la ville qu’entre Paris et les provinces. » (Comte de Tilly, {\itshape Mémoires}, I, 153.)} ; comparée à la cour, elle semble provinciale. Il faut cent mille roses, dit-on, pour faire une once de cette essence unique qui sert aux rois de Perse ; tel est ce salon, mince flacon d’or et de cristal ; il contient la substance d’une végétation humaine. Pour le remplir, il a fallu d’abord qu’une grande aristocratie, {\itshape transplantée} en serre chaude et désormais stérile de fruits, ne portât plus que des fleurs, ensuite que, dans l’alambic royal, toute sa sève épurée se concentrât en quelques gouttes d’arôme. Le prix est excessif, mais c’est à ce prix qu’on fabrique les très délicats parfums.

\section[{IV. Les occupations du roi. — Lever, messe, dîner, promenades, chasse, souper, jeu, soirées. — Il est toujours en représentation et en compagnie.}]{IV. Les occupations du roi. — Lever, messe, dîner, promenades, chasse, souper, jeu, soirées. — Il est toujours en représentation et en compagnie.}

\noindent Une opération semblable engage celui qui la fait comme ceux qui la subissent. Ce n’est point impunément qu’on transforme une noblesse d’utilité en une noblesse d’ornement\footnote{Exemple du désœuvrement imposé à la noblesse, dîner de la reine Marie Leczinska à Fontainebleau. « J’arrive dans une salle superbe où je vois une douzaine de courtisans qui se promenaient, et une table d’au moins douze couverts, qui pourtant n’était préparée que pour une seule personne… La reine s’assit et aussitôt les douze courtisans se placèrent en demi-cercle à dix pas de la table ; je me tins auprès d’eux, imitant leur respectueux silence. Sa Majesté commence à manger fort vite, sans regarder personne, tenant les yeux baissés sur son assiette. Ayant trouvé à son goût un mets qu’on lui avait servi, elle y revint, et alors elle parcourut des yeux le cercle devant elle… et dit : « M. de Lowendal ? » À ce nom, je vois un superbe homme qui s’avance en inclinant la tête, et dit : « Madame ? » — « Je crois que ce ragoût est une fricassée de poulet »  « Je suis de cet avis, Madame ». Après cette réponse faite du ton le plus sérieux, le maréchal reprend sa place à reculons ; la reine acheva de dîner sans dire un mot de plus, et rentra dans son appartement comme elle était venue. » (Casanova, {\itshape Mémoires.})}, on tombe soi-même dans la parade qu’on a substituée à l’action. Le roi a une cour, il faut qu’il la tienne. Tant pis si elle absorbe son temps, son esprit, son âme, tout le meilleur de sa force active et de la force de l’État. Ce n’est pas une petite besogne que d’être maître de maison, surtout quand, à l’ordinaire, on reçoit cinq cents personnes ; on est obligé de passer sa vie en public et en spectacle. À parler exactement, c’est le métier d’un acteur qui toute la journée serait en scène. Pour soutenir ce fardeau et travailler d’ailleurs, il a fallu le tempérament de Louis XIV, la vigueur de son corps, la résistance extraordinaire de ses nerfs, la puissance de son estomac, la régularité de ses habitudes ; après lui, sous la même charge, ses successeurs se lassent ou défaillent. Mais ils ne peuvent s’y soustraire ; la représentation incessante et journalière est inséparable de leur place et s’impose à eux comme un habit de cérémonie lourd et doré. Le roi est tenu d’occuper toute une aristocratie, par conséquent de se montrer et de payer de sa personne à toute heure, même aux heures les plus intimes, même en sortant du lit, même au lit. Le matin, à l’heure qu’il a marquée d’avance\footnote{« Sous Louis XVI, qui quittait son lit à sept ou huit heures du matin, le lever était à onze heures et demie, à moins que des chasses ou des cérémonies n’en avançassent l’instant. » — Même cérémonial à onze heures du soir pour le coucher, et dans la journée pour le débotté. (Comte d’Hézecques, \href{http://gallica.bnf.fr/ark:/12148/bpt6k49961n/f178}{\dotuline{{\itshape ib}., 161}} [\url{http://gallica.bnf.fr/ark:/12148/bpt6k49961n/f178}].)}, le premier valet de chambre l’éveille : cinq séries de personnes entrent tour à tour pour lui rendre leurs devoirs, et « quoique très vastes, il y a des jours où les salons d’attente peuvent à peine contenir la foule des courtisans »  D’abord on introduit « l’entrée familière », enfants de France, princes et princesses du sang, outre cela le premier médecin, le premier chirurgien et autres personnages utiles\footnote{Waroquier, I, 94. Comparez le détail correspondant sous Louis XIV, dans Saint-Simon, XIII, 88.}  Puis on fait passer la « grande entrée » ; elle comprend le grand chambellan, le grand maître et le maître de la garde-robe, les premiers gentilshommes de la chambre, les ducs d’Orléans et de Penthièvre, quelques autres seigneurs très favorisés, les dames d’honneur et d’atour de la reine, de Mesdames et des autres princesses, sans compter les barbiers, tailleurs et valets de plusieurs sortes. Cependant on verse au roi de l’esprit-de-vin sur les mains dans une assiette de vermeil, puis on lui présente le bénitier ; il fait le signe de croix et dit une prière. Alors, devant tout ce monde, il sort de son lit, chausse ses mules. Le grand chambellan et le premier gentilhomme lui présentent sa robe de chambre ; il l’endosse et vient s’asseoir sur le fauteuil où il doit s’habiller  À cet instant, la porte se rouvre ; un troisième flot pénètre, c’est « l’entrée des brevets » ; les seigneurs qui la composent ont en outre le privilège précieux d’assister au petit coucher, et du même coup arrive une escouade de gens de service, médecins et chirurgiens ordinaires, intendants des menus-plaisirs, lecteurs et autres, parmi ceux-ci le porte-chaise d’affaires : la publicité de la vie royale est telle, que nulle de ses fonctions ne s’accomplit sans témoins  Au moment où les officiers de la garde-robe s’approchent du roi pour l’habiller, le premier gentilhomme, averti par l’huissier, vient dire au roi les noms des grands qui attendent à la porte : c’est la quatrième entrée, dite « de la chambre », plus grosse que les précédentes ; car, sans parler des porte-manteaux, porte-arquebuse, tapissiers et autres valets, elle comprend la plupart des grands officiers, le grand aumônier, les aumôniers de quartier, le maître de chapelle, le maître de l’oratoire, le capitaine et le major des gardes du corps, le colonel général et le major des gardes françaises, le colonel du régiment du roi, le capitaine des Cent-Suisses, le grand veneur, le grand louvetier, le grand prévôt, le grand maître et le maître des cérémonies, le premier maître d’hôtel, le grand panetier, les ambassadeurs étrangers, les ministres et secrétaires d’État, les maréchaux de France, la plupart des seigneurs de marque et des prélats. Des huissiers font ranger la foule et au besoin faire silence. Cependant le roi se lave les mains et commence à se dévêtir. Deux pages lui ôtent ses pantoufles ; le grand maître de la garde-robe lui tire sa camisole de nuit par la manche droite, le premier valet de garde-robe par la manche gauche, et tous deux le remettent à un officier de garde-robe, pendant qu’un valet de garde-robe apporte la chemise dans un surtout de taffetas blanc  C’est ici l’instant solennel, le point culminant de la cérémonie ; la cinquième entrée a été introduite, et, dans quelques minutes, quand le roi aura pris la chemise, tout le demeurant des gens connus et des officiers de la maison qui attendent dans la galerie apportera le dernier flot. Il y a tout un règlement pour cette chemise. L’honneur de la présenter est réservé aux fils et aux petits-fils de France, à leur défaut aux princes du sang ou légitimés, au défaut de ceux-ci au grand chambellan ou au premier gentilhomme ; notez que ce dernier cas est rare, les princes étant obligés d’assister au lever du roi, comme les princesses à celui de la reine\footnote{{\itshape Marie-Antoinette}, par Arneth et Geffroy, II, 217.}. Enfin voilà la chemise présentée ; un valet de garde-robe emporte l’ancienne ; le premier valet de garde-robe et le premier valet de chambre tiennent la nouvelle, l’un par la manche gauche, l’autre par la manche droite\footnote{Dans tous les changements d’habits, le côté gauche du roi est dévolu à la garde-robe, et le côté droit à la chambre.} et, pendant l’opération, deux autres valets de chambre tendent devant lui sa robe de chambre déployée, en guise de paravent. La chemise est endossée, et la toilette finale va commencer. Un valet de chambre tient devant le roi un miroir, et deux autres, sur les deux côtés, éclairent, si besoin est, avec des flambeaux. Des valets de garde-robe apportent le reste de l’habillement ; le grand maître de garde-robe passe au roi la veste et le justaucorps, lui attache le cordon bleu, lui agrafe l’épée ; puis un valet préposé aux cravates en apporte plusieurs dans une corbeille, et le maître de garde-robe met au roi celle que le roi choisit. Ensuite un valet préposé aux mouchoirs en apporte trois dans une soucoupe, et le grand maître de garde-robe offre la soucoupe au roi, qui choisit. Enfin le maître de garde-robe présente au roi son chapeau, ses gants et sa canne. Le roi vient alors à la ruelle de son lit, s’agenouille sur un carreau et fait sa prière, pendant qu’un aumônier à voix basse prononce l’oraison {\itshape Quæsumus, Deus omnipotens.} Cela fait, le roi prescrit l’ordre de la journée, et passe avec les premiers de sa cour dans son cabinet, où parfois il donne des audiences. Cependant tout le reste attend dans la galerie, afin de l’accompagner à la messe quand il sortira.\par
Tel est le lever, une pièce en cinq actes  Sans doute on ne peut mieux imaginer pour occuper à vide une aristocratie : une centaine de seigneurs considérables ont employé deux heures à venir, à attendre, à entrer, à défiler, à se ranger, à se tenir sur leurs pieds, à conserver sur leurs visage l’air aisé et respectueux qui convient à des figurants de haut étage, et tout à l’heure les plus qualifiés vont recommencer chez la reine\footnote{La reine déjeune dans son lit, et « il y a dix ou douze personnes à cette première entrée… ». Les grandes entrées faisaient leur cour à l’heure de la toilette. « Cette entrée comprenait les princes du sang, les capitaines des gardes, et la plupart des grandes charges. » — En tout trois entrées le matin chez la reine. — Même cérémonial que pour le roi au sujet de la chemise. Un jour d’hiver, Mme Campan présentait la chemise à la reine ; la dame d’honneur entre, ôte ses gants, prend la chemise. On gratte à la porte, c’est la duchesse d’Orléans ; elle ôte ses gants, reçoit la chemise. On gratte encore, c’est la comtesse d’Artois qui, par privilège, prend la chemise. Cependant la reine grelottait, les bras croisés sur sa poitrine, et murmurait : « C’est odieux ! quelle importunité ! » (Mme Campan, II, 217 ; III, 309-316).}. Mais par contre-coup le roi a subi la gêne et le désœuvrement qu’il imposait. Lui aussi, il a joué un rôle ; tous ses pas et tous ses gestes ont été réglés d’avance ; il a dû compasser sa physionomie et sa voix, ne jamais quitter l’air digne et affable, distribuer avec réserve ses regards et ses signes de tête, ne rien dire ou ne parler que de chasse, éteindre sa propre pensée s’il en a une. On ne peut pas rêver, méditer, être distrait quand on est en scène ; il faut être à son rôle. D’ailleurs, dans un salon, on n’a que des conversations de salon, et l’attention du maître, au lieu de se ramasser en un courant utile, s’éparpille en eau bénite de cour. Or toutes les heures de sa journée sont semblables, sauf trois ou quatre dans la matinée pendant lesquelles il est au conseil ou à son bureau : encore faut-il observer que, les lendemains de chasse, quand il revient de Rambouillet à trois heures du matin, il doit dormir pendant ce peu d’heures libres. Pourtant l’ambassadeur Mercy\footnote{{\itshape Marie-Antoinette}, par Arneth et Geffroy, II, 223 (15 août 1774).}, homme fort appliqué, semble trouver que cela est suffisant ; du moins il juge que Louis XVI « a beaucoup d’ordre, qu’il ne perd pas de temps aux choses inutiles » ; en effet son prédécesseur travaillait beaucoup moins, à peine une heure par jour  Ainsi les trois quarts de son temps sont livrés à la parade  Le même cortège est autour de lui, au botté, au débotté, quand il s’habille de nouveau pour monter à cheval, quand il rentre pour prendre l’habit de soirée, quand il revient dans sa chambre pour se mettre au lit. « Tous les soirs pendant six ans, dit un page\footnote{Comte d’Hézecques, \href{http://gallica.bnf.fr/ark:/12148/bpt6k49961n/f24}{\dotuline{{\itshape ib.}, 7}} [\url{http://gallica.bnf.fr/ark:/12148/bpt6k49961n/f24}].}, moi ou mes camarades nous avons vu Louis XVI se coucher en public », avec le cérémonial décrit tout à l’heure. « Je ne l’ai pas vu suspendre dix fois, et alors c’était toujours par accident ou pour cause d’indisposition. » L’assistance est plus nombreuse encore quand il dîne et soupe ; car, outre les hommes, il y a les femmes, les duchesses sur des pliants, les autres debout autour de la table. Je n’ai pas besoin de dire que le soir, à son jeu, à son bal, à son concert, la foule afflue et s’entasse. Lorsqu’il chasse, outre les dames à cheval et en calèche, outre les officiers de vénerie, le officiers des gardes, l’écuyer, le porte-manteau, le porte-arquebuse, le chirurgien, le renoueur, le coureur de vin, et je ne sais combien d’autres, il a pour invités à demeure tous les gentilshommes présentés. Et ne croyez pas que cette suite soit mince\footnote{Duc de Lauzun, {\itshape Mémoires}, 51. — Mme de Genlis, {\itshape Mémoires}, ch. XII : « Tous nos maris allaient régulièrement coucher ce jour-là (le samedi) à Versailles pour chasser le lendemain avec le roi. »} : le jour où M. de Chateaubriand est présenté, il y en a quatre nouveaux, et « très exactement » tous les jeunes gens de grande famille viennent deux ou trois fois par semaine se joindre au cortège du roi. — Non seulement les huit ou dix scènes qui composent chacune de ses journées, mais encore les courts intervalles qui séparent une scène de l’autre, sont assiégés et accaparés. On l’attend, on l’accompagne et on lui parle au passage, entre son cabinet et la chapelle, entre la chapelle et son cabinet, entre sa chambre et son carrosse, entre son carrosse et sa chambre, entre son cabinet et son couvert. — Bien mieux, les coulisses de sa vie appartiennent au public. S’il est indisposé et qu’on lui apporte un bouillon, s’il est malade et qu’on lui présente une médecine, « un garçon de chambre appelle tout de suite la grande entrée ». Véritablement le roi ressemble à un chêne étouffé par les innombrables lierres qui, depuis la base jusqu’à la cime, se sont collés autour de son tronc. — Sous un pareil régime, l’air manque ; il faut trouver une échappée : Louis XV avait ses petits soupers et la chasse ; Louis XVI a la chasse et la serrurerie. Et je n’ai pas décrit le détail infini de l’étiquette, le cérémonial prodigieux des grands repas, les quinze, vingt et trente personnes occupées autour du verre et de l’assiette du roi, les paroles sacramentelles du service, la marche du cortège, l’arrivée de « la nef », « l’essai des plats » ; on dirait d’une cour byzantine ou chinoise\footnote{Le grand couvert a lieu tous les dimanches. La {\itshape nef} est une pièce d’orfèvrerie placée au centre de la table et contenant, entre des coussins de senteur, les serviettes à l’usage du roi  {\itshape L’essai} est l’épreuve que les gentilshommes servants et les officiers de bouche font de chaque plat avant que le roi en mange. De même pour la boisson  Il faut quatre personnes pour servir au roi un verre d’eau et de vin.}. Le dimanche tout le public, même ordinaire, est introduit, et cela s’appelle le « grand couvert », aussi solennel et aussi compliqué qu’une grand’messe. Aussi bien, pour un descendant de Louis XIV, manger, boire, se lever, se coucher, c’est officier\footnote{Quand les dames de la cour et surtout les princesses passent devant le lit du roi, elles doivent faire la révérence. Quand les officiers du palais passent devant la nef, ils doivent faire le salut  De même le prêtre ou le sacristain qui passe devant l’autel.}. Frédéric II, s’étant fait expliquer cette étiquette, disait que, s’il était roi de France, son premier édit serait pour faire un autre roi qui tiendrait la cour à sa place ; en effet, à ces désœuvrés qui saluent, il faut un désœuvré qu’il saluent. Il n’y aurait qu’un moyen de dégager le monarque : ce serait de refondre la noblesse française et de la transformer, d’après le modèle prussien, en un régiment laborieux de fonctionnaires utiles. Mais, tant que la cour reste ce qu’elle est, je veux dire une escorte d’apparat et une parure de salon, le roi est tenu d’être comme elle un décor éclatant qui sert peu ou qui ne sert pas.

\section[{V. Divertissements des personnes royales et de la cour. — Louis XV. — Louis XVI.}]{V. Divertissements des personnes royales et de la cour. — Louis XV. — Louis XVI.}

\noindent En effet, quelle est l’occupation d’un maître de maison qui sait vivre ? Il s’amuse et amuse ses hôtes ; chez lui, c’est tous les jours une nouvelle partie de plaisir. Comptez celles d’une semaine. « Hier dimanche, dit le duc de Luynes, je trouvai en chemin le roi qui allait tirer dans la plaine Saint-Denis, et qui a été coucher à la Muette, où il compte continuer à tirer aujourd’hui et demain, et revenir ici mardi ou mercredi matin pour courre le cerf ce même jour mercredi\footnote{Duc de Luynes, IX, 75, 79, 105 (août 1748, octobre 1748).}. » Deux mois plus tard, « le roi, dit encore M. de Luynes, a été tous les jours de la semaine dernière et de celle-ci à la chasse, hors aujourd’hui et les dimanches, et tué, depuis le commencement des perdreaux, trois mille cinq cents pièces ». Il est toujours en route et en chasse, passant d’une résidence à l’autre, de Versailles à Fontainebleau, à Choisy, à Marly, à la Muette, à Compiègne, à Trianon, à Saint-Hubert, à Bellevue, à Rambouillet, et, le plus souvent, avec toute sa cour\footnote{Le roi étant à Marly, liste des voyages qu’il fera avant d’aller à Compiègne (duc de Luynes, XIV, 163, mai 1755) : « Le dimanche 1\textsuperscript{er} juin, à Choisy jusqu’au lundi soir. — Le mardi 3, à Trianon jusqu’au mercredi. — Le jeudi 5, retourne à Trianon, où il restera jusqu’à samedi après souper. — Le lundi 9, à Crécy jusqu’au vendredi 13. — Retourne à Crécy le 16 jusqu’au 21. — Le 1\textsuperscript{er} juillet, à la Muette ; le 2, à Compiègne. »}. À Choisy, notamment, et à Fontainebleau, tout ce monde est en liesse. À Fontainebleau, « dimanche et vendredi, jeu ; lundi et mercredi, concert chez la reine ; mardi et jeudi, les comédiens français ; samedi, ce sont les Italiens » : il y en a pour tous les jours de la semaine. À Choisy, écrit la Dauphine\footnote{{\itshape Marie-Antoinette}, par Arneth et Geffroy, I, 19 (12 juillet 1770) ; I, 265 (janvier 1772) ; I, 111 (18 octobre 1770).}, « depuis une heure où l’on dîne, on est jusqu’à une heure du matin sans rentrer chez soi… Après le dîner, l’on joue jusqu’à six heures, que l’on va au spectacle qui dure jusqu’à neuf heures et demie, et ensuite le souper ; de là encore jeu jusqu’à une heure et même la demie quelquefois ». À Versailles, où l’on est plus modéré, il n’y a que deux spectacles et un bal par semaine ; mais tous les soirs il y a appartement et jeu chez le roi, chez ses filles, chez sa maîtresse, chez sa bru, outre les chasses et trois petits voyages par semaine. On a compté que telle année Louis XV ne coucha que cinquante-deux nuits à Versailles, et l’ambassadeur d’Autriche dit très bien que « son genre de vie ne lui laisse pas une heure dans la journée à s’occuper des affaires sérieuses »  Quant à Louis XVI, on a vu qu’il dégage quelques heures dans la matinée ; mais la machine est montée et l’entraîne. Comment se dérober à ses hôtes, et comment ne pas faire les honneurs de chez soi ? Les convenances et l’usage sont aussi des despotes ; ajoutez-en un troisième, plus absolu encore, la vivacité impérieuse et folâtre d’une jeune reine qui ne peut supporter une heure de lecture. À Versailles, trois spectacles et deux bals par semaine, deux grands soupers, le mardi et le jeudi ; et, de temps en temps, l’Opéra à Paris\footnote{{\itshape Ib.}, II, 270 (18 octobre 1774) ; II, 395 (15 novembre 1775) ; II, 295 (20 février 1775) ; III, 25 (11 février 1777) ; III, 119 (17 octobre 1777) ; III, 409 (18 mars 1780).}. À Fontainebleau, trois spectacles par semaine, les autres jours jeu et souper. L’hiver suivant, la reine donne chaque semaine bal masqué, où la « composition des habillements, les contredanses figurées en ballets et les répétitions journalières prennent tant de temps que toute la semaine y passe ». Pendant le carnaval de 1777, la reine, outre ses propres fêtes, a les bals du Palais-Royal et les bals masqués de l’Opéra ; un peu plus tard, chez la comtesse Diane de Polignac, j’en trouve un autre où elle assiste avec toute la famille royale, sauf Mesdames, et qui dure depuis onze heures et demie du soir jusqu’à onze heures du matin. Cependant, les jours ordinaires, le pharaon fait rage ; dans son salon, « le jeu n’a plus de bornes » ; en une soirée, le duc de Chartres y perd huit mille louis. Véritablement cela ressemble au carnaval italien ; rien n’y manque, ni les masques, ni la comédie de société : on joue, on rit, on danse, on dîne, on écoute de la musique, on se costume, on fait des parties champêtres, on dit des galanteries et des médisances. « La chanson nouvelle\footnote{Mme Campan, I, 147.}, dit une femme de chambre instruite et sérieuse, le bon mot du jour, les petites anecdotes scandaleuses formaient les seuls entretiens du cercle intime de la reine. » — Pour le roi, qui est un peu lourd et qui a besoin d’exercice corporel, la chasse est sa grande affaire. De 1775 à 1789\footnote{Nicolardot, {\itshape Journal de Louis XVI}, 129.}, récapitulant lui-même ce qu’il a fait, il trouve « cent quatre chasses au sanglier, cent trente-quatre au cerf, deux cent soixante-six au chevreuil, trente-trois hourailleries, mille vingt-cinq tirés », en tout quinze cent soixante-deux jours de chasse, c’est-à-dire une chasse au moins tous les trois jours ; outre cela, cent quarante-neuf voyages sans chasse, et deux cent vingt-trois promenades à cheval ou en voiture. « Pendant quatre mois de l’année\footnote{Comte d’Hézecques, \href{http://gallica.bnf.fr/ark:/12148/bpt6k49961n/f270}{\dotuline{{\itshape ib.}, 253}} [\url{http://gallica.bnf.fr/ark:/12148/bpt6k49961n/f270}], et Arthur Young, I, 215.} il va à Rambouillet deux fois par semaine et n’en revient qu’après avoir soupé, c’est-à-dire à trois heures du matin »  Cette habitude invétérée finit par se tourner en manie et même en quelque chose de pis. « Il n’y a pas d’exemple, écrit Arthur Young, le 26 juin 1789, d’une nonchalance et d’une stupidité pareilles à celles de la cour ; le moment demanderait la plus grande décision, et hier, pendant qu’on discutait s’il serait doge de Venise ou roi de France, le roi était à la chasse. » Son journal semble celui d’un piqueur. Lisez-le aux dates les plus importantes, et vous serez stupéfait de ce qu’il y note. Il écrit {\itshape rien} aux jours où il n’a pas chassé ; c’est que pour lui ces jours-là sont vides. « 11 juillet 1789, {\itshape rien}, départ de M. Necker. 12, vêpres et salut, départ de MM. de Montmorin, de Saint-Priest et de la Luzerne. 13, {\itshape rien ;} 14 {\itshape juillet, rien ;} 29 juillet, {\itshape rien}, retour de M. Necker., 4 {\itshape août}, chasse au cerf à la forêt de Marly, pris un, aller et revenir à cheval… 13 août, audience des États dans la galerie, {\itshape Te Deum} pendant la messe en bas ; l’équipage a pris un cerf à Marly… 26 août, audience de compliment des États, grand’messe avec les cordons rouges, serment de M. Bailly, vêpres et salut, grand couvert… 5 {\itshape octobre}, tiré à la porte de Châtillon, tué quatre-vingt-une pièces, interrompu par les événements ; aller et retour à cheval. 6 {\itshape octobre}, départ pour Paris à midi et demi, visite à l’hôtel de ville, soupé et couché aux Tuileries. 7 octobre, {\itshape rien}, mes tantes sont venues dîner. 8, {\itshape rien…} 12, {\itshape rien}, le cerf chassait à Port-Royal. » — Enfermé à Paris, captif de la multitude, son cœur suit toujours sa meute. Vingt fois, en 1790, on lit sur son journal que tel jour le cerf chasse à tel endroit ; il souffre de n’y pas être. Nulle privation plus intolérable ; on retrouve la trace de son chagrin jusque dans la protestation qu’il rédigera avant de partir pour Varennes : transporté dans Paris, sédentaire aux Tuileries, « où, loin de trouver les commodités auxquelles il était accoutumé, il n’a pas même rencontré les agréments que se procurent les personnes aisées », il lui semblera que sa couronne a perdu son plus beau fleuron.

\section[{VI. Autres vies analogues. — Princes et princesses. — Seigneurs de la cour. — Financiers et parvenus. — Ambassadeurs, ministres, gouverneurs, officiers généraux.}]{VI. Autres vies analogues. — Princes et princesses. — Seigneurs de la cour. — Financiers et parvenus. — Ambassadeurs, ministres, gouverneurs, officiers généraux.}

\noindent Tel général, tel état-major ; les grands imitent le monarque. Comme une colossale effigie de marbre précieux érigée au centre de la France, et dont les copies réduites se répandent par milliers d’exemplaires dans toutes les provinces, ainsi la vie royale se répète, en proportions moindres, jusque dans la gentilhommière la plus reculée. On représente et on reçoit ; on fait figure et on passe son temps en compagnie. Je vois d’abord, autour de la cour, une douzaine de cours princières chaque prince ou princesse du sang a, comme le roi, sa maison montée, payée en tout ou en partie sur le Trésor, distribuée en services distincts, avec gentilshommes, pages, dames pour accompagner, bref cinquante, cent, deux cents et jusqu’à cinq cents charges. Il y a une maison de ce genre pour la reine, une pour Madame Victoire, une pour Madame Adélaïde, une pour Madame Élisabeth, une pour Monsieur, une pour Madame, une pour le comte d’Artois, une pour la comtesse d’Artois il y en aura une pour Madame Royale, une pour le petit Dauphin, une pour le duc de Normandie, tous les trois enfants du roi ; une pour le duc d’Angoulême, une pour le duc de Berry, tous les deux fils du comte d’Artois : dès six ou sept ans, les enfants représentent et reçoivent. Si je prends une date précise, en 1771\footnote{ \noindent État des pensions payées aux personnes de la famille royale en 1771. Duc d’Orléans, 150 000 ; Prince de Condé, 100 000 ; Comte de Clermont, 70 000 ; Duc de Bourbon, 60 000 ; Prince de Conti, 60 000 ; Comte de la Marche, 60 000 ; Douairière de Conti, 60 000 ; Duc de Penthièvre, 50 000 ; Princesse de Lamballe, 50 000 ; Duchesse de Bourbon, 50 000. ({\itshape Archives nationales}, O1, 710 {\itshape bis.})
 }, j’en trouve encore une pour le duc d’Orléans, une pour le duc de Bourbon, une pour la duchesse de Bourbon, une pour le prince de Condé, une pour le comte de Clermont, une pour la princesse douairière de Conti, une pour le prince de Conti, une pour le comte de la Marche, une pour le duc de Penthièvre  Chacun de ces personnages, outre son appartement chez le roi, a son château et son palais où il tient cercle, la reine à Trianon et à Saint-Cloud, Mesdames à Bellevue, Monsieur au Luxembourg et à Brunoy, le comte d’Artois à Meudon et à Bagatelle, le duc d’Orléans au Palais-Royal, à Monceau, au Raincy, à Villers-Cotterets, le prince de Conti au Temple et à l’Isle-Adam, les Condés au Palais-Bourbon et à Chantilly, le duc de Penthièvre à Sceaux, Anet et Châteauvilain : j’omets la moitié de ces résidences. Au Palais-Royal, toutes les personnes présentées peuvent venir souper les jours d’opéra. À Châteauvillain, tous ceux qui viennent faire leur cour sont invités à dîner, les nobles à la table du duc, les autres à la table de son premier gentilhomme. Au temple, les soupers du lundi rassemblent cent cinquante convives. Quarante ou cinquante personnes, disait la duchesse du Maine, sont « le particulier d’une princesse\footnote{Beugnot, I, 77. — Mme de Genlis, {\itshape Mémoires}, ch. XVII. — E. et J. de Goncourt, {\itshape la Femme au dix-huitième siècle}, 52. — Chamfort, \href{http://gallica.bnf.fr/ark:/12148/bpt6k878444}{\dotuline{{\itshape Caractères et anecdotes}}} [\url{http://gallica.bnf.fr/ark:/12148/bpt6k878444}].} ». Le train des princes est si inséparable de leur personne, qu’il les suit jusque dans les camps. « M. le prince de Condé, dit M. de Luynes, part demain pour l’armée avec une grande suite : il a deux cent vingt-cinq chevaux, et M. le comte de la Marche cent. M. le duc d’Orléans part lundi ; il a trois cent cinquante chevaux pour lui et sa suite\footnote{Duc de Luynes, XVI, 57 (mai 1757). — À l’armée de Westphalie, le général en chef, comte d’Estrées, avait vingt-sept secrétaires et Grimm fut le vingt-huitième. — Quand le duc de Richelieu partit pour son gouvernement de Guyenne, il lui fallut sur toute la route des relais de cent chevaux.}. » — Au-dessous des parents du roi, tous les grands qui figurent à la cour figurent aussi chez eux, dans leur hôtel de Paris ou de Versailles, et dans leur château à quelques lieues de Paris. De tous côtés, dans les Mémoires, on aperçoit en raccourci quelqu’une de ces vies seigneuriales. Telle est celle du duc de Gesvres, premier gentilhomme de la chambre, gouverneur de Paris et de l’Ile-de-France, ayant en outre les gouvernements particuliers de Laon, de Soissons, de Noyon, de Crépy en Valois, la capitainerie de Mousseaux et vingt mille livres de pension, véritable homme de cour, sorte d’exemplaire en haut relief des gens de sa classe, et qui, par ses charges, sa faveur, son luxe, ses dettes, sa considération, ses goûts, ses occupations et son tour d’esprit, nous représente en abrégé tout le beau monde\footnote{ Duc de Luynes, XVI, 186 (octobre 1757).
 }. Sa mémoire est étonnante pour les parentés et les généalogies ; il possède à fond la science précieuse de l’étiquette ; à ces deux titres, il est un oracle et très consulté. « Il a beaucoup augmenté la beauté de sa maison et de ses jardins à Saint-Ouen. » — « Au moment de mourir, dit M. de Luynes, il venait d’y ajouter vingt-cinq arpents qu’il avait commencé à faire enfermer dans une terrasse revêtue… Il avait une maison considérable en gentilshommes, pages, domestiques de toute espèce, et faisait une dépense prodigieuse… Il avait tous les jours un grand dîner… Il donnait presque tous les jours des audiences particulières. Il n’y avait personne à la cour ni à la ville qui ne lui rendît des devoirs. Les ministres, les princes du sang eux-mêmes lui en rendaient. Il recevait du monde pendant qu’il était encore dans son lit. Il écrivait, dictait au milieu d’une compagnie nombreuse… Sa maison à Paris et son appartement à Versailles ne désemplissaient point depuis qu’il était éveillé jusqu’à ce qu’il se couchât. » — Deux ou trois cents maisons à Paris, à Versailles et aux environs présentent un spectacle semblable. Jamais de solitude ; c’est l’usage en France, dit Horace Walpole, « de brûler jusqu’au lumignon sa chandelle en public ». L’hôtel de la duchesse de Gramont est assiégé dès le matin par les plus grands seigneurs et les plus grandes dames. Cinq fois par semaine, chez le duc de Choiseul, à dix heures du soir, le maître d’hôtel vient jeter un coup d’œil dans les salons, dans l’immense galerie pleine, et, au juger, fait mettre cinquante, soixante, quatre-vingts couverts\footnote{E. et J. de Goncourt, {\itshape ibid.}, 73, 75.} ; bientôt, sur cet exemple, toutes les riches maisons se font gloire de tenir table ouverte à tous venants  Naturellement, les parvenus, les financiers qui achètent ou se donnent un nom de terre, tous ces traitants et fils de traitants qui, depuis Law, frayent avec la noblesse, copient ses façons. Et je ne parle pas ici des Bouret, des Beaujon, des Saint-James, et autres bourreaux d’argent dont l’attirail efface celui des princes. Considérez un simple associé des fermes, M. d’Epinay, dont la femme modeste et fine se refuse à tant d’étalage\footnote{Mme d’Épinay, {\itshape Mémoires}, éd. Boiteau, I, 306 (1751).}. Il vient de « compléter son domestique », et aurait voulu qu’elle prît une seconde femme de chambre ; mais elle a tenu bon ; pourtant, dans cette maison écourtée, « les officiers, les femmes et les valets se montent au nombre de seize… Lorsque M. d’Epinay est levé, son valet se met en devoir de l’accommoder. Deux laquais sont debout à attendre ses ordres. Le premier secrétaire vient avec l’intention de rendre compte des lettres qu’il a reçues et qu’il est chargé d’ouvrir ; mais il est interrompu deux cents fois dans cette opération par toutes sortes d’espèces imaginables. C’est un maquignon qui a des chevaux uniques à vendre… Ensuite c’est un polisson qui vient brailler un air et à qui on accorde sa protection pour le faire entrer à l’Opéra, après lui avoir donné quelques leçons de bon goût et lui avoir appris ce que c’est que la propreté du chant français. C’est une demoiselle qu’on fait attendre pour savoir si je suis encore là… Je me lève et je m’en vais. Les deux laquais ouvrent les deux battants pour me laisser sortir, moi qui passerais alors par le trou d’une aiguille, et les deux estafiers crient dans l’antichambre : « Madame, Messieurs, voilà Madame ! » Tout le monde se range en haie, et ces messieurs sont des marchands d’étoffes, des marchands d’instruments, des bijoutiers, des colporteurs, des laquais, des décrotteurs, des créanciers, enfin tout ce que vous pouvez imaginer de plus ridicule et de plus affligeant. Midi ou une heure sonne avant que cette toilette soit achevée, et le secrétaire, qui sans doute sait par expérience l’impossibilité de rendre un compte détaillé des affaires, a un petit bordereau qu’il remet entre les mains de son maître pour l’instruire de ce qu’il doit dire à l’assemblée des fermiers. » — Oisiveté, désordre, dettes, cérémonial, ton et façons de protecteur, tout cela semble une parodie du vrai monde ; c’est que nous sommes au dernier étage de l’aristocratie. Et cependant la cour de M. d’Épinay ressemble en petit à celle du roi.\par
À plus forte raison faut-il que les ministres, ambassadeurs, officiers généraux, qui tiennent la place du roi, représentent d’une façon grandiose. Aucune circonstance n’a rendu l’ancien régime aussi éclatant et plus onéreux ; en ceci, comme dans tout le reste, Louis XIV est le principal auteur du mal comme du bien. La politique qui avait établi la cour prescrivait le faste. « C’était lui plaire, que de s’y jeter en habits, en tables, en équipages, en bâtiments, en jeu ; c’étaient là des occasions pour qu’il parlât aux gens\footnote{Saint-Simon, XII, 457, et Dangeau, VI, 408. Chez le maréchal de Boufflers, au camp de Compiègne (septembre 1698), il y avait tous les soirs et tous les matins deux tables de 20 à 25 couverts, outre les tables supplémentaires, 72 cuisiniers, 340 domestiques, 400 douzaines de serviettes, 80 douzaines d’assiettes d’argent, 6 douzaines d’assiettes de vermeil. 14 chevaux en relais apportaient tous les jours de Paris les liqueurs et les fruits ; chaque jour des exprès apportaient poisson, volaille et gibier de Gand, Bruxelles, Dunkerque, Dieppe et Calais. Dans les jours ordinaires on buvait 50 douzaines de bouteilles, et 80 douzaines pendant la visite du roi et des princes.}. De la cour, la contagion avait passé dans la province et aux armées, où les gens en quelque place n’étaient comptés qu’à proportion de leur table et de leur magnificence. » Pendant l’année que le maréchal de Belle-Isle passa à Francfort pour l’élection de Charles VI, il dépensa 750 000 livres en voyages, transports, fêtes, dîners, construction d’une salle à manger et d’une cuisine, outre cela 150 000 livres en boîtes, montres et autres présents ; par l’ordre du cardinal Fleury, si économe, il avait 101 officiers dans ses cuisines\footnote{Duc de Luynes, XIV, 149.}. À Vienne, en 1772, l’ambassadeur prince de Rohan avait deux carrosses coûtant ensemble 40 000 livres, 40 chevaux, 7 pages nobles, 6 gentilshommes, 5 secrétaires, 10 musiciens, 12 valets de pied, 4 coureurs dont les habits chamarrés avaient coûté chacun 4 000 livres, et le reste à proportion\footnote{L’abbé Georgel, {\itshape Mémoires}, 216.}. On sait le luxe, le bon goût, les dîners exquis, l’admirable représentation du cardinal de Bernis à Rome. « On l’appelait le roi de Rome, et il l’était en effet par sa magnificence et par la considération dont il jouissait… Sa table donnait l’idée des possibles… Dans les fêtes, les cérémonies, les illuminations, il était toujours au-dessus de toute comparaison. » Il disait lui-même en souriant : « Je tiens l’auberge de France dans un carrefour de l’Europe\footnote{Sainte-Beuve, {\itshape Causeries du lundi}, VIII, 63, textes de deux témoins, Mme de Genlis et Roland.} »  Aussi bien leurs traitements et indemnités sont-ils deux ou trois fois plus amples qu’aujourd’hui. « Le roi donne 50 000 écus pour les grandes ambassades. M. le duc de Duras a eu jusqu’à 200 000 livres par an pour celle de Madrid, outre cela 100 000 écus de gratification, 50 000 livres pour affaires secrètes, et on lui a prêté 400 000 ou 500 000 livres de meubles ou effets dont il a gardé la moitié\footnote{Duc de Luynes, XV, 455 et XVI, 219 (1757). « Le maréchal de Belle-Isle avait 1 200 000 livres de dettes contractées, un quart pour ses bâtisses de plaisir, le reste pour le service du roi. Le roi, pour le dédommager, lui donne 400 000 livres sur le produit des salines, et 80 000 livres de rente sur la compagnie qui a le privilège d’affiner les métaux précieux. »}. » — Les dépenses et les traitements des ministres sont pareils. En 1789, le chancelier a 120 000 livres d’appointements, le garde des sceaux 135 000 ; « M. de Villedeuil, comme secrétaire d’État, devait avoir 180 670 livres, mais il a représenté que cette somme ne couvrait pas ses dépenses, et son traitement a été porté à 226 000 livres tout compris\footnote{{\itshape Compte général} des revenus et dépenses fixes au 1\textsuperscript{er} mai 1789, 633  Notez qu’il faut doubler tous ces chiffres pour avoir leur équivalent actuel.} ». D’ailleurs la règle est que, lorsqu’ils se retirent, le roi leur fait une pension de 20 000 livres et donne 200 000 francs de dot à leur fille  Ce n’est pas trop pour leur train. « Ils sont obligés de tenir un si grand état de maison, qu’ils ne peuvent guère s’enrichir dans leur place ; ils ont tous table ouverte à Paris au moins trois fois par semaine, et à Versailles, à Fontainebleau, table ouverte tous les jours\footnote{Mme de Genlis, {\itshape Dictionnaire des Étiquettes}, I, 349.}. » M. de Lamoignon étant nommé chancelier avec 100 000 livres d’appointements, on juge tout de suite qu’il se ruinera\footnote{Barbier, {\itshape Journal}, III, 211 (déc. 1750.)} ; « car il a pris tous les officiers de cuisine de M. d’Aguesseau, dont la table seule allait à 80 000 livres. Le repas qu’il a donné à Versailles au premier conseil qu’il a tenu a coûté 6 000 livres, et il lui faut toujours à Versailles et à Paris une table d’environ vingt couverts »  À Chambord\footnote{Aubertin, {\itshape l’Esprit public au dix-huitième siècle}, 255.} le maréchal de Saxe a tous les jours deux tables, l’une de 60, l’autre de 80 couverts, 400 chevaux dans ses écuries, une liste civile de plus de 100 000 écus, un régiment de hulans pour sa garde, un théâtre qui a coûté plus de 600 000 livres, et la vie qu’il mène ou qu’on mène autour de lui ressemble à une bacchanale de Rubens  Quant aux gouverneurs généraux ou particuliers en province, on a vu que, lorsqu’ils y résident, ils n’ont d’autre emploi que de recevoir ; à côté d’eux, l’intendant qui fait seul les affaires, reçoit aussi et magnifiquement, surtout dans les pays d’États. Commandants, lieutenants généraux, partout les envoyés du centre sont conduits de même, par les mœurs, par les convenances et par leur propre désœuvrement, à tenir salon ; ils apportent avec eux les élégances et l’hospitalité de Versailles. Si leur femme les a suivis, elle s’ennuie et « végète au milieu de cinquante personnes, ne disant que des lieux communs, faisant des nœuds ou jouant au loto, et passant trois heures à table ». Mais « tous les militaires, tous les gentilshommes des environs, toutes les dames de la ville », s’empressent à ses bals et célèbrent à l’envi sa grâce, sa politesse, son égalité\footnote{Mme de Genlis, {\itshape Adèle et Théodore}, III, 54.} ». Jusque dans les grades secondaires, on retrouve ces habitudes somptueuses. En vertu de l’usage établi, les colonels et même les capitaines traitent leurs {\itshape officiers} et dépensent ainsi « beaucoup au-delà de leurs appointements\footnote{Duc de Lévis, 68. De même, avant la dernière réforme, les grades dans l’armée anglaise  Cf. Voltaire, {\itshape Entretiens entre A, B, C}, 15\textsuperscript{r} Entretien. « Un régiment n’est point le prix des services, c’est le prix de la somme que les parents d’un jeune homme ont déposée pour qu’il aille, trois mois de l’année, tenir table ouverte dans une ville de province. »} ». C’est même là une des raisons qui font réserver les régiments aux fils de bonne maison, et les compagnies aux gentilshommes riches  Du grand arbre royal opulemment épanoui à Versailles, partent des rejets qui s’étendent par milliers sur toute la France, pour s’y épanouir, comme à Versailles, en bouquets de gala et d’appartement.

\section[{VII. Prélats, seigneurs et petite noblesse en province. — L’aristocratie féodale est devenue une société de salon.}]{VII. Prélats, seigneurs et petite noblesse en province. — L’aristocratie féodale est devenue une société de salon.}

\noindent Sur ce modèle, et par l’effet même de la température, on voit, jusque dans les provinces reculées, toutes les tiges aristocratiques aboutir à une floraison mondaine. Faute d’un autre emploi, les nobles se visitent, et le principal office d’un seigneur notable est de faire dignement les honneurs de sa maison ; je parle ici des ecclésiastiques aussi bien que des laïques  Les cent trente et un évêques et archevêques, les sept cents abbés commendataires sont gens du monde ; ils ont de bonnes façons, ils sont riches, ils ne sont pas austères, et leur palais épiscopal ou leur abbaye est pour eux une maison de campagne qu’ils restaurent ou embellissent en vue de la résidence qu’ils y font et de la compagnie qu’ils y accueillent. À Clairvaux\footnote{Beugnot, I, 79.}, Dom Rocourt, très poli envers les hommes, et encore plus galant envers les femmes, ne marche qu’en voiture à quatre chevaux avec un piqueur en avant ; il se fait donner du Monseigneur par ses moines et tient une vraie cour. La chartreuse du Val-Saint-Pierre est un somptueux palais au milieu d’un immense domaine, et le père procureur Dom Effinger passe ses journées à recevoir les hôtes\footnote{Merlin de Thionville, {\itshape Vie et correspondance}  Récit de sa visite à la chartreuse de Val-Saint-Pierre en Thiérache.}. Au couvent d’Origny, près de Saint-Quentin\footnote{Mme de Genlis, {\itshape Mémoires}, ch. 7.}, « l’abbesse a des domestiques, une voiture, des chevaux, reçoit en visite et à dîner les hommes dans son appartement. » — La princesse Christine, abbesse de Remiremont, et ses dames chanoinesses sont presque toujours en route ; et pourtant « on s’amuse à l’abbaye », on y reçoit quantité de monde « dans les appartements particuliers de la princesse et dans les appartements des étrangers\footnote{Mme d’Oberkirch, I, 15.} ». Les vingt-cinq chapitres nobles de femmes et les dix-neuf chapitres nobles d’hommes sont autant de salons permanents et de rendez-vous incessants de belle compagnie qu’une mince barrière ecclésiastique sépare à peine du grand monde où ils sont recrutés. Au chapitre d’Alix, près de Lyon, les chanoinesses vont au chœur en paniers, « habillées comme dans le monde », sauf que leur robe est de soie noire et que leur manteau est doublé d’hermine\footnote{Mme de Genlis, ch. 1. — Mme d’Oberkirch, I, 62.}. Au chapitre d’Ottmarsheim en Alsace, « nos huit jours, dit une visiteuse, se passèrent à nous promener, à visiter le tracé des voies romaines, à rire beaucoup, à danser même, car il venait beaucoup de monde à l’abbaye, et surtout à parler de chiffons ». Près de Sarrelouis, les chanoinesses de Loutre dînent avec les officiers et ne sont rien moins que prudes\footnote{Duc de Lauzun, {\itshape Mémoires}, 257.}. Quantité de couvents sont des asiles agréables et décents pour des dames veuves, pour de jeunes femmes dont les maris sont à l’armée, pour des filles de condition, et la supérieure, qui le plus souvent est demoiselle, tient avec aisance et dextérité le sceptre de ce joli monde féminin  Mais nulle part la pompe, l’hospitalité, la foule ne sont plus grandes que dans les palais épiscopaux. J’ai décrit la situation des évêques : si opulents, possesseurs de pareils droits féodaux, héritiers et successeurs des anciens souverains de la contrée, outre cela, gens à la mode et habitués de Versailles, comment n’auraient-ils pas une cour ? Un Cicé, archevêque de Bordeaux, un Dillon, archevêque de Narbonne, un Brienne, archevêque de Toulouse, un Castellane, évêque de Mende et seigneur suzerain de tout le Gévaudan, un archevêque de Cambray, duc de Cambray, seigneur suzerain de tout le Cambrésis et président-né des États provinciaux, la plupart sont des princes ; ne faut-il pas qu’ils représentent en princes ? C’est pourquoi ils chassent, ils bâtissent, ils ont des clients, des hôtes, un lever, une antichambre, des huissiers, des officiers, une table ouverte, une maison montée, des équipages, et le plus souvent des dettes, dernier point qui achève le grand seigneur. Dans le palais presque royal que les Rohan, évêques héréditaires de Strasbourg et cardinaux d’oncle en neveu, se sont bâti à Saverne\footnote{Marquis de Valfons, {\itshape Souvenirs}, 60. — Duc de Lévis, 156. — Mme d’Oberkirch, I, 127 ; II, 360.}, il y a 700 lits, 180 chevaux, 14 maîtres d’hôtel, 25 valets de chambre. « Toute la province s’y rassemble » ; le cardinal a logé à la fois jusqu’à deux cents invités, sans les valets ; en tout temps on trouve chez lui « de vingt à trente femmes des plus aimables de la province, et souvent ce nombre est augmenté par celles de la cour et de Paris »  « Le soir à neuf heures tout le monde soupait ensemble, ce qui avait toujours l’air d’une fête », et le cardinal lui-même en était le plus bel ornement. Superbement vêtu, beau, galant, d’une politesse exquise, le moindre de ses sourires était une grâce. « Son visage toujours riant inspirait la confiance ; il avait la vraie physionomie de l’homme destiné à représenter. »\par
Telle est aussi l’attitude et l’occupation des principaux seigneurs laïques, chez eux, en été, lorsque le goût de la chasse et l’attrait de la belle saison les ramènent sur leurs terres. Par exemple, Harcourt en Normandie et Brienne en Champagne sont deux des châteaux les mieux habités. « Il y vient de Paris des personnes considérables, des hommes de lettres distingués, et la noblesse du canton y fait une cour assidue\footnote{Beugnot, I, 71. — Hippeau, {\itshape le Gouvernement de Normandie}, passim.}. » Il n’y a pas de résidence où des volées de beau monde ne viennent s’abattre à demeure pour dîner, danser, chasser, causer, parfiler, jouer la comédie. On peut suivre à la trace ces brillants oiseaux, de volière en volière ; ils restent une semaine, un mois, trois mois, étalant leur ramage et leur plumage. De Paris à l’Isle-Adam, à Villers-Cotterets, au Frétoy, à la Planchette, à Soissons, à Reims, à Grisolles, à Sillery, à Braine, à Balincourt, au Vaudreuil, le comte et la comtesse de Genlis promènent ainsi leur loisir, leur esprit, leur gaieté, chez des amis qu’à leur tour ils reçoivent à Genlis  Un coup d’œil jeté sur les dehors de ces maisons suffirait pour montrer que le premier devoir en ce temps-là est d’être hospitalier, comme le premier besoin est d’être en compagnie\footnote{Mme de Genlis, {\itshape Mémoires}, passim. — {\itshape Dictionnaire des Étiquettes.} I, 348.}. En effet leur luxe diffère du nôtre. Sauf en quelques maisons princières, il n’est pas grand en meubles de campagne : on laisse cet étalage aux financiers. « Mais il est prodigieux en toutes les choses qui peuvent donner des jouissances à autrui, en chevaux, en voitures, en tables ouvertes, en logements donnés à des gens qui ne sont point attachés à la maison, en loges aux spectacles qu’on prête à ses amis, enfin en domestiques beaucoup plus nombreux qu’aujourd’hui. » — Par ce frottement mutuel et continu, les nobles les plus rustiques perdent la rouille qui encroûte encore leurs pareils d’Allemagne ou d’Angleterre. On ne trouve guère en France de squires Western et de barons de Thundertentrunck ; une dame d’Alsace, qui voit à Francfort les hobereaux grotesques de la Westphalie, est frappée du contraste\footnote{Mme d’Oberkirch, I, 395. — Le baron et la baronne de Sotenville, dans Molière, sont des gens bien élevés, quoique provinciaux et pédants.}. Ceux de France, même dans les provinces éloignées, ont fréquenté les salons du commandant ou de l’intendant, et rencontré en visite quelques dames de Versailles ; c’est pourquoi « Ils ont toujours quelque habitude des grandes manières, et sont à peu près instruits des vicissitudes de la mode et du costume ». Le plus sauvage descend, le chapeau à la main, jusqu’au bas de son perron pour reconduire ses hôtes en les remerciant de la grâce qu’ils lui ont faite. Le plus rustre, auprès d’une femme, retrouve au fond de sa mémoire quelques débris de la galanterie chevaleresque. Le plus pauvre et le plus retiré ménage son habit bleu-de-roi et sa croix de Saint-Louis pour pouvoir, à l’occasion, présenter ses devoirs au grand seigneur son voisin ou au prince qui est de passage  Ainsi l’état-major féodal s’est transformé tout entier, depuis ses premiers jusqu’à ses derniers grades. Si l’on pouvait embrasser du regard ses trente ou quarante mille palais, hôtels, manoirs, abbayes, quel décor avenant et brillant que celui de la France ! Elle est un salon et je n’y vois que des gens de salon. Partout les chefs rudes ayant autorité sont devenus des maîtres de maison ayant des grâces. Ils appartiennent à cette société où, avant d’admirer tout à fait un grand général, on demandait « s’il était aimable ». Sans doute ils portent encore l’épée, ils sont braves par amour-propre et tradition, ils sauront se faire tuer, surtout en duel et dans les formes. Mais le caractère mondain a recouvert l’ancien fond militaire ; à la fin du dix-huitième siècle, leur grand talent est le savoir-vivre, et leur véritable emploi consiste à recevoir ou à être reçus.
\chapterclose


\chapteropen

\chapter[{Chapitre II. La vie de salon.}]{Chapitre II. \\
La vie de salon.}


\chaptercont

\section[{I. Elle n’est parfaite qu’en France. — Raisons tirées du caractère français. — Raisons tirées du ton de la cour en France. — Cette vie devient de plus en plus agréable et absorbante.}]{I. Elle n’est parfaite qu’en France. — Raisons tirées du caractère français. — Raisons tirées du ton de la cour en France. — Cette vie devient de plus en plus agréable et absorbante.}

\noindent D’autres aristocraties en Europe ont été conduites par des circonstances à peu près pareilles vers des mœurs à peu près semblables. Là aussi la monarchie a produit la cour, qui a produit la société polie ; mais la jolie plante ne s’est développée qu’à demi. Le sol était défavorable, et les graines n’étaient pas de la bonne espèce. En Espagne, le roi demeure enfermé dans l’étiquette comme une momie dans sa gaine, et l’orgueil trop raide, incapable de se détendre jusqu’aux aménités de la vie mondaine, n’aboutit qu’à l’ennui morne et au faste insensé\footnote{ \noindent L. de Loménie, {\itshape Beaumarchais et son temps}, I, 403. Lettre de Beaumarchais (24 décembre 1764). — {\itshape Voyage} de Mme d’Aulnoy, et \href{../Villars}{\dotuline{{\itshape Lettres}}} [\url{../Villars}] de Mme de Villars. — Pour l’Italie, voir Stendhal ({\itshape Rome, Naples et Florence}). — Pour l’Allemagne, voir les {\itshape Mémoires} de la margrave de Bareith et du chevalier Lang. — Pour l’Angleterre, on trouvera les textes dans les tomes III et IV de mon {\itshape Histoire de la littérature anglaise.}
 }. En Italie, sous de petits princes despotes et la plupart étrangers, le danger continu et la défiance héréditaire, après avoir lié les langues, tournent les cœurs vers les jouissances intimes de l’amour ou vers les jouissances muettes des beaux-arts. En Allemagne et en Angleterre, le tempérament froid, lourd et rebelle à la culture retient l’homme, jusqu’à la fin du dernier siècle, dans les habitudes germaniques de solitude, d’ivrognerie et de brutalité. Au contraire en France, tout concourt à faire fleurir l’esprit de société ; en cela le génie national est d’accord avec le régime politique, et il semble que d’avance on ait choisi la plante pour le terrain.\par
Par instinct, le Français aime à se trouver en compagnie, et la raison en est qu’il fait bien et sans peine toutes les actions que comporte la société. Il n’a pas la mauvaise honte qui gêne ses voisins du Nord, ni les passions fortes qui absorbent ses voisins du Midi. Il n’a pas d’effort à faire pour causer, point de timidité naturelle à contraindre, point de préoccupation habituelle à surmonter. Il cause donc, à l’aise et dispos, et il éprouve du plaisir à causer. Car ce qu’il lui faut, c’est un bonheur d’espèce particulière, fin, léger, rapide, incessamment renouvelé et varié, où son intelligence, son amour-propre, toutes ses vives et sympathiques facultés trouvent leur pâture ; et cette qualité de bonheur, il n’y a que le monde et la conversation pour la fournir. Sensible comme il est, les égards, les ménagements, les empressements, la délicate flatterie sont l’air natal hors duquel il respire avec peine. Il souffrirait d’être impoli presque autant que de rencontrer l’impolitesse. Pour ses instincts de bienveillance et de vanité, il y a de charmantes douceurs dans l’habitude d’être aimable, d’autant plus qu’elle est contagieuse. Quand nous plaisons, on veut nous plaire, et ce que nous donnons en prévenances, on nous le rend en attentions. En pareille compagnie, on peut causer ; car causer, c’est amuser autrui en s’amusant soi-même, et il n’y a pas de plus vif plaisir pour un Français\footnote{Volney, {\itshape Tableau du climat et du sol des États-Unis d’Amérique}  Selon lui, le trait caractéristique du colon français comparé à ceux des autres nations, c’est le besoin de voisiner et de causer.}. Agile et sinueuse, la conversation est pour lui comme le vol pour un oiseau : d’idées en idées, il voyage, alerte, excité par l’élan des autres, avec des bonds, des circuits, des retours imprévus, au plus bas, au plus haut, à rase terre ou sur les cimes, sans s’enfoncer dans les trous, ni s’empêtrer dans les broussailles, ni demander aux mille objets qu’il effleure autre chose que la diversité et la gaieté de leurs aspects. Ainsi doué et disposé, il était fait pour un régime qui, dix heures par jour, mettait les hommes ensemble : le naturel inné s’est trouvé d’accord avec l’ordre social pour rendre les salons parfaits. En tête de tous, le roi donnait l’exemple. Louis XIV avait eu toutes les qualités d’un maître de maison, le goût de la représentation et de l’hospitalité, la condescendance et la dignité, l’art de ménager l’amour-propre des autres et l’art de garder sa place, la galanterie noble, le tact et jusqu’à l’agrément de l’esprit et du langage. « Il parlait parfaitement bien\footnote{Mme de Caylus, {\itshape Souvenirs}, 108.} ; s’il fallait badiner, s’il faisait des plaisanteries, s’il daignait faire un conte, c’était avec des grâces infinies, un tour noble et fin que je n’ai vu qu’à lui. » — « Jamais homme si naturellement poli\footnote{ Saint-Simon, XII, 461.
 }, ni d’une politesse si mesurée, si fort par degrés, ni qui distinguât mieux l’âge, le mérite, le rang, et dans ses réponses et dans ses manières… Ses révérences, plus ou moins marquées, mais toujours légères, avaient une grâce et une majesté incomparables… Il était admirable à recevoir différemment les saluts à la tête des lignes de l’armée et aux revues. Mais surtout pour les femmes, rien n’était pareil… Jamais il n’a passé devant la moindre coiffe sans ôter son chapeau, je dis aux femmes de chambre et qu’il connaissait pour telles… Jamais il ne lui arriva de dire rien de désobligeant à personne… Jamais devant le monde rien de déplacé ni de hasardé, mais jusqu’au moindre geste, son marcher, son port, toute sa contenance, tout mesuré, tout décent, noble, grand, majestueux et toutefois très naturel. » — Voilà le modèle, et, de près ou de loin, jusqu’à la fin de l’ancien régime, il est suivi. S’il change un peu, ce n’est que pour devenir plus sociable. Au dix-huitième siècle, sauf dans les jours de grand apparat, on le voit, degré à degré, descendre de son piédestal. Il ne se fait plus autour de lui de « ces silences à entendre marcher une fourmi ». — « Sire, disait à Louis XVI le maréchal de Richelieu, témoin des trois règnes, sous Louis XIV, on n’osait dire mot ; sous Louis XV, on parlait tout bas ; sous Votre Majesté, on parle tout haut. » — Si l’autorité y perd, la société y gagne ; l’étiquette, insensiblement relâchée, laisse entrer l’aisance et l’agrément. Désormais les grands, ayant moins souci d’imposer que de plaire, se dépouillent de la morgue comme d’un costume gênant et « ridicule, et recherchent moins les respects que les applaudissements. Il ne suffit même plus d’être affable, il faut à tout prix paraître aimable à ses inférieurs comme à ses égaux\footnote{Duc de Lévis, 321.} ». — « Les princes français, dit encore une dame contemporaine, meurent de peur de manquer de grâces\footnote{Mme de Genlis, {\itshape Souvenirs de Félicie}, 160. — Il faut noter pourtant, sous Louis XV et même sous Louis XVI, le maintien de l’ancienne attitude royale. « Quoique je fusse prévenu, dit Alfieri, que le roi ne parlait pas aux étrangers ordinaires, je ne pus digérer le regard de Jupiter Olympien avec lequel Louis XV toisait de la tête aux pieds l’homme présenté, d’un air impassible, tandis que si l’on présentait une fourmi à un géant, le géant, l’ayant regardée, sourirait ou dirait peut-être : Oh, quel petit animalcule ! Du moins, s’il se taisait, son visage dirait cela pour lui. » (Alfieri, {\itshape Memorie}, I, 138. — 1768.) Voir dans les {\itshape Mémoires} de Mme \href{http://gallica.bnf.fr/ark:/12148/bpt6k204843g}{\dotuline{d’Oberkirch (II}} [\url{http://gallica.bnf.fr/ark:/12148/bpt6k204843g}], 349) la leçon donnée par Madame Royale, âgée de sept ans et demi, à une dame présentée.}. » Jusques autour du trône, « le ton est libre, enjoué », et, sous le sourire de la jeune reine, la cour sérieuse et disciplinée de Louis XVI se trouve à la fin du siècle le plus engageant et le plus gai des salons. Par cette détente universelle, la vie mondaine est devenue parfaite. « Qui n’a pas vécu avant 1789, disait plus tard M. de Talleyrand, ne connaît pas la douceur de vivre. » — Elle était trop grande, on n’en goûtait plus d’autre, elle prenait tout l’homme. Quand le monde a tant d’attraits, on ne vit que pour lui.

\section[{II. Subordination des autres intérêts et devoirs. — Indifférence aux affaires publiques. — Elles ne sont qu’une matière à bons mots. Négligence dans les affaires privées. — Désordre du ménage et abus de l’argent.}]{II. Subordination des autres intérêts et devoirs. — Indifférence aux affaires publiques. — Elles ne sont qu’une matière à bons mots. Négligence dans les affaires privées. — Désordre du ménage et abus de l’argent.}

\noindent On n’a point de loisir ni de goût pour autre chose, même pour les choses qui touchent l’homme de plus près, les affaires publiques, le ménage, la famille. — J’ai déjà dit que, sur le premier article, ils s’abstiennent et sont indifférents ; locale ou générale, l’administration est hors de leurs mains et ne les intéresse plus. Quand on en parle, c’est pour plaisanter ; les plus graves événements ne sont que des matières à bons mots. Après l’édit de l’abbé Terray qui fait une banqueroute de moitié sur la rente, un spectateur trop serré au théâtre s’écrie : « Ah ! quel malheur que notre bon abbé Terray ne soit pas ici pour nous réduire de moitié ! » Et l’on rit, l’on applaudit ; le lendemain tout Paris, en répétant la phrase, se console de la ruine publique. — Alliances, batailles, impôts, traités, ministères, coups d’État, on a toute l’histoire du siècle en épigrammes et en chansons. Un jour\footnote{Chamfort, 26, 55. — \href{http://gallica.bnf.fr/ark:/12148/bpt6k87380m}{\dotuline{Bachaumont}} [\url{http://gallica.bnf.fr/ark:/12148/bpt6k87380m}] [‘p. 124’], I, 136 (7 sept. 1762). Un mois après l’arrêt du Parlement contre les jésuites, paraissent de petits jésuites en cire ayant pour base un escargot. « À l’aide d’une ficelle on fait sortir et rentrer le jésuite dans la coquille. C’est une fureur, il n’y a pas de maison qui n’ait son jésuite. »}, dans une assemblée de jeunes gens de la cour, comme on répétait le mot de la journée, l’un d’eux, ravi de plaisir, dit en levant les mains : « Comment ne serait-on pas charmé des grands événements, des bouleversements même qui font dire de si jolis mots ! » Là-dessus, on repasse les mots, les chansons faites sur tous les désastres de la France. La chanson sur la bataille d’Hochstædt fut trouvée mauvaise, et quelques-uns dirent à ce sujet : « Je suis fâché de la perte de cette bataille ; la chanson ne vaut rien\footnote{En revanche, la chanson sur la bataille de Rosbach est charmante.} ». — Même en défalquant de ce trait ce que l’entraînement de la verve et la licence du paradoxe y ont mis d’énorme, il reste la marque d’un siècle où l’État n’était presque rien et la société presque tout. Sur ce principe, on peut deviner le genre de talent que le monde demande aux ministres. M. Necker, ayant donné un souper splendide avec opéra sérieux et opéra bouffon, « il se trouve que cette fête lui a valu plus de crédit, de faveur et de stabilité que toutes ses opérations financières… On n’a parlé qu’un jour de sa dernière disposition concernant le vingtième, tandis qu’on parle encore en ce moment de la fête qu’il a donnée, et qu’à Paris comme à Versailles on en détaille tous les agréments, et que l’on dit tout haut : Ce sont des gens admirables que M. et Mme Necker, ils sont délicieux pour la société\footnote{{\itshape Correspondance secrète}, par Metra, Imbert, etc. V, 277 (17 novembre 1777). — Voltaire, {\itshape la Princesse de Babylone.}} ». La bonne compagnie qui s’amuse impose aux gens en place l’obligation de l’amuser. Elle dirait presque d’un ton demi-sérieux, demi-badin, avec Voltaire, « que les dieux n’ont établi les rois que pour donner tous les jours des fêtes, pourvu qu’elles soient diversifiées ; que la vie est trop courte pour en user autrement ; que les procès, les intrigues, la guerre, les disputes des prêtres, qui consument la vie humaine, sont des choses absurdes et horribles, que l’homme n’est né que pour la joie », et que, parmi les choses nécessaires, il faut mettre au premier rang « le superflu ».\par
À ce compte, on peut prévoir qu’ils seront aussi insouciants dans leurs affaires privées que dans les affaires publiques. Ménage, administration des biens, économie domestique, à leurs yeux tout cela est bourgeois, et de plus insipide, affaire d’intendant et de maître d’hôtel. À quoi bon des gens, si l’on doit prendre ce soin ? La vie n’est plus une fête dès qu’on est obligé d’en surveiller les apprêts. Il faut que la commodité, le luxe, l’agrément coulent de source et viennent d’eux-mêmes se placer à portée des lèvres. Il faut que, naturellement et sans qu’il s’en mêle, un homme de ce monde trouve de l’or dans ses poches, un habit galant sur sa toilette, des valets poudrés dans son antichambre, un carrosse doré à sa porte, un dîner délicat sur sa table, et qu’il puisse réserver toute son attention pour la dépenser en grâces avec les hôtes de son salon. Un pareil train ne va pas sans gaspillage, et les domestiques, livrés à eux-mêmes, font leur main. Qu’importe, s’ils font leur service ? D’ailleurs, il faut bien que tout le monde vive, et il est agréable d’avoir autour de soi des visages obséquieux et contents. — C’est pourquoi les premières maisons du royaume sont au pillage. Un jour à la chasse\footnote{Baron de Besenval, {\itshape Mémoires}, II, 206. Anecdote racontée par le duc de Choiseul.}, Louis XV, ayant avec lui le duc de Choiseul, lui demanda combien il croyait que coûtait le carrosse où ils étaient assis. M. de Choiseul répondit qu’il se ferait bien fort d’en avoir un pareil pour 5 000 ou 6 000 livres, mais « que Sa Majesté, payant en roi et ne payant pas toujours comptant, devait le payer 8 000. — Vous êtes loin de compte, répartit le roi, car cette voiture, telle que vous la voyez, me revient à 30 000 francs… Les voleries dans ma maison sont énormes, mais il est impossible de les faire cesser ». — En effet, les grands tirent à eux comme les petits, soit en argent, soit en nature, soit en services. Il y a chez le roi cinquante-quatre chevaux pour le grand écuyer ; il y en a trente-huit pour Mme de Brionne qui gère une charge d’écurie pendant la minorité de son fils ; il y a deux cent quinze palefreniers d’attribution et à peu près autant de chevaux entretenus aux frais du roi pour diverses autres personnes toutes étrangères au département\footnote{{\itshape Archives nationales.} Rapport de M. Teissier (1780). Rapport de M. Mesnard de Chouzy (O1, 758).}. Sur cette seule branche de l’arbre royal, quelle nichée de parasites   Ailleurs je vois que Madame Élisabeth, si sobre, consomme par an pour 30 000 francs de poisson, pour 70 000 francs de viande et gibier, pour 60 000 francs de bougies ; que Mesdames brûlent pour 215 068 francs de bougie blanche et jaune ; que le luminaire chez la reine revient à 157 109 francs. On montre encore à Versailles la rue, jadis tapissée d’échoppes, où les valets du roi venaient, moyennant argent, nourrir Versailles de sa desserte. — Il n’y a point d’article sur lequel les insectes domestiques ne trouvent moyen de gratter et grappiller. Le roi est censé boire chaque année pour 2 190 francs d’orgeat et de limonade ; « le grand bouillon du jour et de nuit », que boit quelquefois Madame Royale âgée de deux ans, coûte par an 5 201 livres. Vers la fin du règne précédent\footnote{{\itshape Marie-Antoinette}, par Arneth et Geffroy, I, 277 (29 fév. 1772).}, les femmes de chambre comptent en dépense à la Dauphine « quatre paires de souliers par semaine, trois aunes de ruban par jour pour nouer son peignoir, deux aunes de taffetas par jour pour couvrir la corbeille où l’on dépose les gants et l’éventail ». — Quelques années plus tôt, en café, limonade, chocolat, orgeat, eaux glacées, le roi payait par an 200 000 francs ; plusieurs personnes étaient inscrites sur l’état pour dix ou douze tasses par jour, et l’on calculait que le café au lait avec un petit pain tous les matins coûtait pour chaque dame d’atour 2 000 francs par an\footnote{Duc de Luynes, XVII, 37 (août 1758). — Marquis d’Argenson, 11 février 1753.}. On devine qu’en des maisons ainsi gouvernées les fournisseurs attendent. Ils attendent si bien que parfois, sous Louis XV, ils refusent de fournir et « se cachent ». Même le retard est si régulier, qu’à la fin on est obligé de leur payer à 5 pour 100 l’intérêt de leurs avances ; à ce taux, en 1778, après toutes les économies de Turgot, le roi doit encore près de 800 000 livres à son marchand de vin, près de trois millions et demi à son pourvoyeur\footnote{{\itshape Archives nationales}, O1, 738. Les intérêts payés sont de 12 969 francs pour le boulanger, de 39 631 francs pour le marchand de vins, de 173 899 francs pour le pourvoyeur.}. Même désordre dans les maisons qui entourent le trône. « Mme de Guémené doit 60 000 livres à son cordonnier, 16 000 à son colleur de papiers, et le reste à proportion. » Une autre, à qui le marquis de Mirabeau voit des chevaux de remise, répond en voyant son air étonné : « Ce n’est pas qu’il n’y en ait 70 dans nos écuries ; mais il n’y en a point qui ait pu aller aujourd’hui\footnote{Marquis de Mirabeau, {\itshape Traité de la population}, 60. — {\itshape Le Gouvernement de Normandie.} par Hippeau, II, 204 (30 sept. 1780).} ». Mme de Montmorin, voyant que son mari a plus de dettes que de biens, croit pouvoir sauver sa dot de 200 000 francs ; mais on lui apprend qu’elle a consenti à répondre pour un compte de tailleur, et ce compte\footnote{Mme de la Rochejaquelein, {\itshape Mémoires}, 30. — Mme d’Oberkirch, II, 66.} « chose incroyable et ridicule à dire, s’élève au chiffre de 180 000 livres »  Une des manies les plus tranchées de ce temps-ci, dit Mme d’Oberkirch, est de se ruiner en tout et sur tout. » — « Les deux frères Villemur bâtissent des guinguettes de 500 000 à 600 000 livres ; l’un d’eux a 40 chevaux pour monter quelquefois à cheval au bois de Boulogne\footnote{Marquis d’Argenson, 26 janvier 1753.}. » En une nuit, M. de Chenonceaux, fils de M. et de Mme Dupin, perd au jeu 700 000 livres. « M. de Chenonceaux et M. de Francueil ont mangé 7 ou 8 millions d’alors\footnote{George Sand, {\itshape Histoire de ma vie}, I, 78.}. » — « Le duc de Lauzun, à l’âge de 26 ans, après avoir mangé le fonds de 100 000 écus de rente, est poursuivi par ses créanciers pour près de 2 millions de dettes\footnote{{\itshape Marie-Antoinette}, par Arneth et Geffroy, I, 61 (18 mars 1777).} »  « M. le prince de Conti manque de pain et de bois, quoiqu’il ait 600 000 livres de rente » ; c’est qu’il « achète et fait bâtir follement de tous côtés\footnote{Marquis d’Argenson, 26 janvier 1753.} ». Où serait l’agrément, si l’on était raisonnable ? Qu’est-ce qu’un seigneur qui regarde au prix des choses ? Et comment atteindre à l’exquis, si l’on plaint l’argent   Il faut donc que l’argent coule, et coule à s’épuiser, d’abord par les innombrables saignées secrètes ou tolérées de tous les abus domestiques, puis en larges ruisseaux par les prodigalités du maître en bâtisses, en meubles, en toilettes, en hospitalité, en galanteries, en plaisirs. Le comte d’Artois, pour donner une fête à la reine, fait démolir, rebâtir, arranger et meubler Bagatelle de fond en comble par neuf cents ouvriers employés jour et nuit ; et, comme le temps manque pour aller chercher au loin la chaux, le plâtre et la pierre de taille, il envoie sur les grands chemins des patrouilles de la garde suisse qui saisissent, payent et amènent sur-le-champ les chariots ainsi chargés\footnote{{\itshape Marie-Antoinette}, III, 135 (19 nov. 1777).}. Le maréchal de Soubise, recevant un jour le roi à dîner et à coucher dans sa maison de campagne, dépense à cela 200 000 livres\footnote{Barbier, IV, 155. — Le maréchal de Soubise avait un rendez-vous de chasse où le roi venait de temps en temps manger une omelette d’œufs de faisans, coûtant 157 livres 10 sous. (Mercier, XII, 192, d’après le cuisinier qui préparait l’omelette.)}. Mme de Matignon fait un marché de 24 000 livres par an pour qu’on lui fournisse tous les jours une coiffure nouvelle. Le cardinal de Rohan a une aube brodée en point à l’aiguille qu’on estime à plus de 100 000 livres, et sa batterie de cuisine est en argent massif\footnote{Mme d’Oberkirch, I, 129 ; II, 257.}. — Rien de plus naturel avec l’idée qu’on se faisait alors de l’argent ; épargné, entassé, au lieu d’un fleuve, c’était une mare inutile et qui sentait mauvais. La reine, ayant donné au Dauphin une voiture dont les encadrements en vermeil étaient ornés de rubis et de saphirs, disait naïvement : « Le roi n’a-t-il pas augmenté ma cassette de 200 000 livres ? ce n’est pas pour que je les garde\footnote{Mme de Genlis, {\itshape Souvenirs de Félicie}, 80 ; et {\itshape Théâtre d’Éducation}, II, 367. Une jeune femme honnête fait, en 10 mois, 70 000 francs de dettes : « Pour une petite table 10 louis, pour une chiffonnière 15 louis, pour un bureau 800 fr., pour une petite écritoire 200 fr., pour une grande écritoire 300 fr. Bagues de cheveux, montre de cheveux, chaîne de cheveux, bracelets de cheveux, cachet de cheveux, collier de cheveux, boîte de cheveux, 9 900 fr., etc. ».} ». On les jetterait plutôt par la fenêtre. Ainsi fit le maréchal de Richelieu d’une bourse qu’il avait donnée à son petit-fils et que le jeune garçon, n’ayant su la dépenser, rapportait pleine. Du moins l’argent, cette fois, servit au balayeur qui passait et le ramassa. Mais, faute d’un passant pour le ramasser, on l’eût jeté dans la rivière. Un jour, devant le prince de Conti, Mme de B. laissa soupçonner qu’elle voudrait avoir la miniature de son serin dans une bague. Le prince s’offrit ; on accepta, mais à condition que la miniature serait très simple et sans brillants. En effet, ce ne fut qu’un petit cercle d’or ; mais, pour recouvrir la peinture, un gros diamant aminci servait de glace. Mme de B. ayant renvoyé le diamant, « M. le prince de Conti le fit broyer, réduire en poudre et s’en servit pour sécher l’encre du billet qu’il écrivit à ce sujet à Mme de B. ». La pincée de poudre coûtait quatre ou cinq mille livres, mais on devine le tour et le ton du billet. Il faut l’extrême profusion à la suprême galanterie, et l’on est d’autant plus un homme du monde que l’on est moins un homme d’argent.

\section[{III. Divorce moral des époux. — La galanterie. Séparation des parents et des enfants. — L’éducation, ses lacunes et son objet. — Ton des domestiques et des fournisseurs. — L’empreinte mondaine est universelle.}]{III. Divorce moral des époux. — La galanterie. Séparation des parents et des enfants. — L’éducation, ses lacunes et son objet. — Ton des domestiques et des fournisseurs. — L’empreinte mondaine est universelle.}

\noindent Dans un salon, la femme dont un homme s’occupe le moins, c’est la sienne, et à charge de retour ; c’est pourquoi, en un temps où l’on ne vit que pour le monde et dans le monde, il n’y a pas place pour l’intimité conjugale. — D’ailleurs, quand les époux sont haut placés, l’usage et les bienséances les séparent. Chacun a sa maison, ou tout au moins son appartement, ses gens, son équipage, ses réceptions, sa société distincte, et, comme la représentation entraîne la cérémonie, ils sont entre eux, par respect pour leur rang, sur le pied d’étrangers polis. Ils se font annoncer l’un chez l’autre ; ils se disent « Madame, Monsieur », non seulement en public, mais en particulier ; ils lèvent les épaules quand à soixante lieues de Paris, dans un vieux château, ils rencontrent une provinciale assez mal apprise pour appeler son mari « mon ami » devant tout le monde\footnote{Mme de Genlis, {\itshape Adèle et Théodore}, III, 14.}. — Déjà divisées au foyer, les deux vies divergent au-delà par un écart toujours croissant. Le mari a son gouvernement, son commandement, son régiment, sa charge à la cour, qui le retiennent hors du logis ; c’est seulement dans les dernières années que sa femme consent à le suivre en garnison ou en province\footnote{Mme d’Avaray donna la première cet exemple, et fut d’abord très blâmée.}. D’autant plus qu’elle est elle-même occupée, et aussi gravement que lui, souvent par une charge auprès d’une princesse, toujours par un salon important qu’elle doit tenir. En ce temps-là, la femme est aussi active que l’homme\footnote{« Lorsque j’arrivai en France, le règne de M. de Choiseul venait seulement de finir. La femme qui pouvait lui paraître aimable, ou seulement plaire à la duchesse de Gramont, sa sœur, était sûre de faire tous les colonels et tous les lieutenants généraux qu’elle voulait. Les femmes avaient de l’importance, même aux yeux de la vieillesse et du clergé ; elles étaient familiarisées d’une manière étonnante avec la marche des affaires ; elles savaient par cœur le caractère et les habitudes des ministres et des amis du roi. Un d’eux qui revenait de Versailles dans son château parlait à sa femme de tout ce qui l’avait occupé ; chez nous, il lui dit un mot sur ses dessins à l’aquarelle, ou reste silencieux, pensif, à rêver à ce qu’il vient d’entendre au parlement. Nos pauvres ladies sont abandonnées à la société de ces hommes frivoles qui, par leur peu d’esprit, se sont trouvés au-dessous de toute ambition et, par là, de tout emploi (les dandies). » (Stendhal, {\itshape Rome, Naples et Florence}, 377. Récit du colonel Forsyth.)}, dans la même carrière, et avec les mêmes armes, qui sont la parole flexible, la grâce engageante, les insinuations, le tact, le sentiment juste du moment opportun, l’art de plaire, de demander et d’obtenir ; il n’y a point de dame de la cour qui ne donne des régiments et des bénéfices. À ce titre, la femme a son cortège personnel de solliciteurs et de protégés, et, comme son mari, ses amis, ses ennemis, ses ambitions, ses mécomptes et ses rancunes propres ; rien de plus efficace pour disjoindre un ménage que cette ressemblance des occupations et cette distinction des intérêts. — Ainsi relâché, le lien finit par se rompre sous l’ascendant de l’opinion. « Il est de bon air de ne pas vivre ensemble », de s’accorder mutuellement toute tolérance, d’être tout entier au monde. En effet, c’est le monde qui fait alors l’opinion, et, par elle, il pousse aux mœurs dont il a besoin.\par
Vers le milieu du siècle, le mari et la femme logeaient dans le même hôtel ; mais c’était tout. « Jamais ils ne se voyaient, jamais on ne les rencontrait dans la même voiture, jamais on ne les trouvait dans la même maison, ni, à plus forte raison, réunis dans un lieu public. » Un sentiment profond eût semblé bizarre et même « ridicule », en tout cas inconvenant : il eût choqué comme un {\itshape a parte} sérieux dans le courant général de la conversation légère. On se devait à tous, et c’était s’isoler à deux ; en compagnie, on n’a pas droit au tête-à-tête\footnote{Besenval, 49, 60  « Sur vingt seigneurs de la cour, il y en a quinze qui ne vivent point avec leurs femmes et qui ont des maîtresses. Rien même n’est si commun à Paris entre particuliers. » (Barbier, IV, 496.)}. À peine si, pour quelques jours, il était permis à deux amants\footnote{ Ne soyez point époux, ne soyez point amant,\par
 Soyez l’homme du jour et vous serez charmant.
 }. Encore était-il mal vu : on les trouvait trop occupés l’un de l’autre. Leur préoccupation répandait autour d’eux « la contrainte et l’ennui ; il fallait s’observer, se retenir en leur présence ». On les « craignait ». Le monde avait les exigences d’un roi absolu et ne souffrait pas de partage. « Si les mœurs y perdaient, dit un contemporain, M. de Besenval, la société y gagnait infiniment ; débarrassée de la gêne et du froid qu’y jette toujours la présence des maris, la liberté y était extrême ; la coquetterie des hommes et des femmes en soutenait la vivacité et fournissait journellement des aventures piquantes. » Point de jalousie, même dans l’amour. « On se plaît, on se prend ; s’ennuie-t-on l’un avec l’autre, on se quitte avec aussi peu de peine qu’on s’est pris. Revient-on à se plaire, on se reprend avec autant de vivacité que si c’était la première fois qu’on s’engageât ensemble. On se quitte encore, et jamais on ne se brouille. Comme on s’est pris sans s’aimer, on se sépare sans se haïr, et l’on retire au moins du faible goût qu’on s’est inspiré l’avantage d’être toujours prêts à s’obliger\footnote{Crébillon fils, {\itshape la Nuit et le Moment}, IX, 14.}. » — D’ailleurs les apparences sont gardées ; un étranger non averti n’y démêlerait rien de suspect. « Il faut, dit Horace Walpole\footnote{Horace Walpole, {\itshape Letters} (25 janvier 1766). — Le duc de Brissac, à Louveciennes, amant de Mme du Barry, et passionnément épris, n’avilit devant elle que l’attitude d’un étranger poli. (Mme Vigée-Lebrun, {\itshape Souvenirs}, I, 165.)}, une curiosité extrême ou une très grande habitude pour découvrir ici la moindre liaison entre les deux sexes. Aucune familiarité n’est permise, sauf sous le voile de l’amitié, et le vocabulaire de l’amour est aussi prohibé que ses rites au premier aspect semblent l’être. » — Même chez Crébillon fils, même chez Laclos, même aux moments les plus vifs, les personnages ne parlent qu’en termes mesurés, irréprochables. L’indécence qui est dans les choses n’est jamais dans les mots, et le langage des convenances s’impose, non seulement aux éclats de la passion, mais encore aux grossièretés de l’instinct  Ainsi les sentiments les plus naturellement âpres ont perdu leurs pointes et leurs épines ; de leurs restes ornés et polis, on a fait des jouets de salon que des mains blanches lancent, se renvoient et laissent tomber comme un joli volant. Il faut entendre à ce sujet les héros de l’époque, leur ton leste, dégagé, est inimitable, et les peint aussi bien que leurs actions. « J’étais, dit le duc de Lauzun, d’une manière fort honnête et même recherchée avec Mme de Lauzun ; j’avais très publiquement Mme de Cambis, dont je me souciais fort peu ; j’entretenais la petite Eugénie, que j’aimais beaucoup ; je jouais gros jeu, je faisais ma cour au roi, et je chassais très exactement avec lui\footnote{Duc de Lauzun, 51. — Chamfort, 39. — « Le duc de…, dont la femme venait de faire un scandale, s’est plaint à sa belle-mère ; celle-ci lui a répondu avec le plus grand sang-froid : Eh ! Monsieur, vous faites bien du bruit pour peu de chose. Votre père était de bien meilleure compagnie. » (Mme d’Oberkirch, II, 135, 241.) — « Un mari disait à sa femme : Je vous permets tout, hors les princes et les laquais. Il était dans le vrai, ces deux extrêmes déshonorent par leur scandale. » (Sénac de Meilhan, {\itshape Considérations sur les mœurs.}) — Un mari surprenant sa femme lui dit simplement : « Quelle imprudence, madame ! si c’était un autre que moi ! » (E. et J. de Goncourt, {\itshape la Femme au dix-huitième siècle}, 201.)}. » Du reste, il avait pour autrui l’indulgence dont il avait besoin lui-même. « On lui demandait ce qu’il répondrait à sa femme (qu’il n’avait pas vue depuis dix ans), si elle lui écrivait : Je viens de découvrir que je suis grosse. Il réfléchit et répondit : Je lui écrirais : Je suis charmé que le ciel ait enfin béni notre union ; soignez votre santé, j’irai vous faire ma cour ce soir. » — Il y a vingt réponses semblables, et j’ose dire qu’avant de les avoir lues on n’imagine pas à quel point l’art social peut dompter l’instinct naturel.\par
« Ici, à Paris, écrit Mme d’Oberkirch, je ne m’appartiens plus, j’ai à peine le temps de causer avec mon mari et de suivre mes correspondances. Je ne sais comment font les femmes dont c’est la vie habituelle ; elles n’ont donc ni famille à entretenir, ni enfants à élever ? » — Du moins elles font comme si elles n’en avaient pas, et les hommes de même. Des époux qui ne vivent pas ensemble ne vivent guère avec leurs enfants, et les causes qui ont défait le mariage défont aussi la famille  Il y a d’abord la tradition aristocratique qui, entre les parents et les enfants, met une barrière pour mettre une distance. Quoique affaiblie et en voie de disparaître\footnote{Voir à ce sujet les types un peu anciens, surtout en province. « Ma mère, ma sœur et moi, transformés en statues par la présence de mon père, nous ne recouvrions qu’après son départ les fonctions de la vie. » (Chateaubriand, {\itshape Mémoires}, I, 17, 28, 130.) — {\itshape Mémoire de Mirabeau}, I, 53. Le marquis disait de son père Antoine : « Je n’ai jamais eu l’honneur de toucher la joue de cet homme vénérable… À l’Académie, étant à 200 lieues de lui, son seul souvenir me faisait craindre toute partie de jeunesse qui pouvait avoir des suites un peu dangereuses »  L’autorité paternelle semble presque aussi âpre dans la bourgeoisie et dans le peuple. ({\itshape Beaumarchais et son temps}, par L. de Loménie, I, 23  {\itshape Vie de mon père}, par Rétif de la Bretonne, passim.)}, cette tradition subsiste. Le fils dit « Monsieur » à son père ; la fille, respectueusement, vient baiser la main de sa mère à sa toilette. Une caresse est rare et semble une grâce ; d’ordinaire, en présence des parents, les enfants sont muets, et le sentiment habituel qui les pénètre est la déférence craintive. Jadis ils étaient des sujets ; jusqu’à un certain point, ils le sont encore, et les exigences nouvelles de la vie mondaine achèvent de les mettre ou de les tenir à l’écart. M. de Talleyrand disait qu’il n’avait jamais couché sous le même toit que ses père et mère. S’ils y couchent, ils n’en sont pas moins négligés. « Je fus confié, dit le comte de Tilly, à des valets et à une espèce de précepteur qui leur ressemblait à beaucoup d’égards. » Pendant ce temps son père courait. « Je lui ai connu, ajoute le jeune homme, des maîtresses jusqu’à un âge avancé ; il les adorait toujours et les quittait sans cesse. » Le duc de Biron juge embarrassant de trouver un bon gouverneur à son fils : « c’est pourquoi, écrit celui-ci, il en confia l’emploi à un laquais de feu ma mère, qui savait lire et passablement écrire, et qu’on décora du titre de valet de chambre pour lui donner plus de considération. On me donna d’ailleurs les maîtres les plus à la mode ; mais M. Roch (c’était le nom de mon mentor) n’était pas en état de diriger leurs leçons ni de me mettre en état d’en profiter. J’étais d’ailleurs comme tous les enfants de mon âge et de ma sorte : les plus jolis habits pour sortir, nu et mourant de faim à la maison\footnote{Sainte-Beuve, {\itshape Nouveaux lundis}, XII, 13  Comte de Tilly, {\itshape Mémoires}, I, 12  Duc de Lauzun, 5. — {\itshape Beaumarchais}, par L. de Loménie, II, 289.} », non par dureté, mais par oubli, dissipation, désordre du ménage ; l’attention est ailleurs. On compterait aisément les pères qui, comme le maréchal de Belle-Isle, surveillent de leurs yeux et conduisent eux-mêmes avec méthode, sévérité et tendresse toute l’éducation de leurs fils  Quant aux filles, on les met au couvent ; délivrés de ce soin, les parents en sont plus libres. Même quand ils en gardent la charge, elle ne leur pèse guère. La petite Félicité de Saint-Aubin\footnote{Mme de Genlis, {\itshape Mémoires}, ch. 2 et 3.} ne voit ses parents « qu’un moment à leur réveil et aux heures des repas » ; c’est que leur journée est toujours prise ; la mère fait ou reçoit des visites ; le père est dans son cabinet de physique ou à la chasse. Jusqu’à sept ans, l’enfant passe sa vie avec des femmes de chambre qui ne lui apprennent qu’un peu de catéchisme « avec un nombre infini d’histoires de revenants ». Vers ce temps-là on prend soin d’elle, mais d’une façon qui peint bien l’époque. La marquise sa mère, auteur d’opéras mythologiques et champêtres, a fait bâtir un théâtre dans le château : il y vient de Bourbon-Lancy et de Moulins un monde énorme ; après douze semaines de répétitions, la petite fille, avec un carquois et des ailes bleues, joue le rôle de l’Amour, et le costume lui va si bien qu’on le lui laisse encore pendant neuf mois à l’ordinaire et toute la journée. Pour l’achever, on fait venir un danseur maître d’armes, et, toujours en costume d’Amour, elle prend des leçons de maintien et d’escrime. « Tout l’hiver se passe à jouer la comédie, la tragédie. » Renvoyée après le dîner, on ne la fait revenir que pour jouer du clavecin ou déclamer le monologue d’Alzire, devant une nombreuse assemblée. — Sans doute de tels excès ne sont pas ordinaires ; mais l’esprit de l’éducation est partout le même : je veux dire qu’aux yeux des parents il n’y a qu’une vie intelligible et raisonnable, celle du monde, même pour les enfants, et qu’on ne s’occupe d’eux que pour les y conduire ou pour les y préparer.\par
Jusqu’aux dernières années de l’ancien régime\footnote{Mme d’Oberkirch, II, 35. Cette mode ne cesse qu’en 1783. — E. et J. de Goncourt, {\itshape la Femme au dix-huitième siècle}, 415. — {\itshape Les petits Parrains}, estampe par Moreau. — Berquin, {\itshape l’Ami des enfants}, passim. — Mme de Genlis, {\itshape Théâtre d’éducation}, passim.}, les petits garçons sont poudrés à blanc, « avec une bourse, des boucles, des rouleaux pommadés » ; ils portent l’épée, ils ont le chapeau sous le bras, un jabot, un habit à parements dorés ; ils baisent les mains des jeunes demoiselles avec une grâce de petits-maîtres. Une fillette de six ans est serrée dans un corps de baleine ; son vaste panier soutient une robe couverte de guirlandes ; elle porte sur la tête un savant échafaudage de faux cheveux, de coussins et de nœuds, rattaché par des épingles, couronné par des plumes, et tellement haut que souvent « le menton est à mi-chemin des pieds » ; parfois on lui met du rouge. C’est une dame en miniature ; elle le sait, elle est toute à son rôle, sans effort ni gêne, à force d’habitude ; l’enseignement unique et perpétuel est celui du maintien ; on peut dire avec vérité qu’en ce siècle la cheville ouvrière de l’éducation est le maître à danser\footnote{Lesage, {\itshape Gil Blas}, discours du maître à danser chargé de l’éducation du fils du comte d’Olivarès.}. Avec lui, on pouvait se passer de tous les autres ; sans lui, tous les autres ne servaient de rien. Car, sans lui, comment faire avec aisance, mesure et légèreté les mille actions les plus ordinaires de la vie courante, marcher, s’asseoir, se tenir debout, offrir le bras, relever l’éventail, écouter, sourire, sous des yeux si exercés et devant un public si délicat ? Pour les hommes et les femmes ce sera plus tard la grande affaire ; c’est pourquoi c’est déjà la grande affaire pour les enfants. Avec les grâces de l’attitude et du geste, ils ont déjà celles de l’esprit et de la parole. À peine leur langue est-elle déliée, qu’ils parlent le langage poli, celui de leurs parents. Ceux-ci jouent avec eux et en font des poupées charmantes ; la prédication de Rousseau qui, pendant le dernier tiers du siècle, remet les enfants à la mode, n’a guère d’autre effet. On leur fait réciter leur leçon en public, jouer dans des proverbes, figurer dans des pastorales. On encourage leurs saillies. Ils savent tourner un compliment, inventer une répartie ingénieuse ou touchante, être galants, sensibles et même spirituels. Le petit duc d’Angoulême reçoit Suffren un livre à la main, et lui dit : « Je lisais Plutarque et ses hommes illustres, vous ne pouviez arriver plus à propos\footnote{{\itshape Correspondance}, par Metra, XIV, 212 ; XVI, 109. — Mme d’Oberkirch, II, 302.} ». Les enfants de M. de Sabran, fille et garçon, âgés de huit et neuf ans, ayant reçu des leçons des comédiens Sainval et Larive, viennent à Versailles jouer devant la reine et le roi l’Oreste de Voltaire, et le petit garçon qu’on interroge sur ses auteurs classiques « répond à une dame mère de trois charmantes demoiselles : Madame, je ne puis me souvenir ici que d’Anacréon ». Un autre, du même âge, réplique à une question du prince Henri de Prusse par un agréable impromptu en vers\footnote{ Comte de Ségur, I, 297 :\par
  \begin{poem}
Ma naissance n’a rien de neuf, \\
 J’ai suivi la commune règle ; \\
 Mais c’est vous qui sortez d’un œuf, \\
 Car vous êtes un aigle. \\-
\end{poem}
 \par
 \noindent Mme de Genlis, {\itshape Mémoires}, chap. IV. Mme de Genlis faisait des vers de ce genre à douze ans.
 }. Faire germer des bons mots, des fadeurs, de petits vers dans un cerveau de huit ans, quel triomphe de la culture mondaine ! C’est le dernier trait du régime qui, après avoir dérobé l’homme aux affaires publiques, à ses affaires propres, au mariage, à la famille, le prend avec tous ses sentiments et toutes ses facultés, pour le donner au monde, lui et tous les siens  Au-dessous de lui, les belles façons et la politesse obligatoire gagnent jusqu’à ses gens, jusqu’à ses fournisseurs. Un Frontin a la désinvolture galante et tourne le compliment\footnote{Déjà, dans {\itshape les Précieuses} de Molière, le marquis de Mascarille et le vicomte de Jodelet  De même, Marivaux, \href{http://gallica.bnf.fr/ark:/12148/bpt6k722980}{\dotuline{{\itshape l’É} {\itshape preuve}}} [\url{http://gallica.bnf.fr/ark:/12148/bpt6k722980}], \href{http://gallica.bnf.fr/ark:/12148/bpt6k101465n}{\dotuline{{\itshape les Jeux de l’amour et du hasard}}} [\url{http://gallica.bnf.fr/ark:/12148/bpt6k101465n}], etc  Lesage, \href{http://gallica.bnf.fr/ark:/12148/bpt6k70764k}{\dotuline{{\itshape Crispin rival de son maître}}} [\url{http://gallica.bnf.fr/ark:/12148/bpt6k70764k}]  Laclos, {\itshape les Liaisons dangereuses}, l’ lettre.}. Une soubrette n’a besoin que d’être entretenue pour devenir une dame. Un cordonnier est un « Monsieur en noir », qui dit à la mère en saluant la fille : « Madame, voilà une charmante demoiselle, et je sens mieux que jamais le prix de vos bontés » ; sur quoi la jeune fille, qui sort du couvent, le prend pour un épouseur et devient toute rouge. — Sans doute, entre ce louis de similor et un louis d’or pur, des yeux moins novices auraient démêlé la différence. Mais leur ressemblance suffit pour montrer l’action universelle du balancier central qui frappait tout à la même effigie, le métal vulgaire et l’or affiné.

\section[{IV. Attrait de cette vie. — Le savoir-vivre au dix-huitième siècle. — Sa perfection et ses ressources. — Autorité des femmes pour l’enseigner et le prescrire.}]{IV. Attrait de cette vie. — Le savoir-vivre au dix-huitième siècle. — Sa perfection et ses ressources. — Autorité des femmes pour l’enseigner et le prescrire.}

\noindent Pour que le monde ait tant d’empire, il faut qu’il ait bien de l’attrait ; en effet, dans aucun pays et dans aucun siècle, un art social si parfait n’a rendu la vie si agréable. Paris est l’école de l’Europe, une école d’urbanité, où, de Russie, d’Allemagne, d’Angleterre, les jeunes gens viennent se dégrossir. Lord Chesterfield, dans ses lettres, ne se lasse point de le répéter à son fils, et de le pousser dans ces salons qui lui ôteront « sa rouille de Cambridge ». Quand on les a connus, on ne les quitte plus, ou, si on est obligé de les quitter, on les regrette toujours. « Rien n’est comparable\footnote{Voltaire, \href{http://gallica.bnf.fr/ark:/12148/bpt6k720733}{\dotuline{{\itshape Princesse de Babylone}}} [\url{http://gallica.bnf.fr/ark:/12148/bpt6k720733}].}, dit Voltaire, à la douce vie qu’on y mène au sein des arts et d’une volupté tranquille et délicate ; des étrangers, des rois ont préféré ce repos si agréablement occupé et si enchanteur à leur patrie et à leur trône… Le cœur s’y amollit et s’y dissout, comme les aromates se fondent doucement à un feu modéré et s’exhalent en parfums délicieux. » Gustave III, battu par les Russes, dit qu’il ira passer ses vieux jours à Paris dans un hôtel sur les boulevards ; et ce n’est pas là une simple politesse ; il se fait envoyer des plans et des devis\footnote{{\itshape Gustave III}, par Geffroy, II, 37  Mme Vigée-Lebrun, I, 81.}. Pour être d’un souper, d’une soirée, on fait deux cents lieues. Des amis du prince de Ligne « partaient de Bruxelles après leur déjeuner, arrivaient à l’Opéra de Paris tout juste pour voir lever la toile, et, le spectacle fini, retournaient aussitôt à Bruxelles, courant toute la nuit »  De ce bonheur tant recherché, nous n’avons plus que des copies informes, et nous en sommes réduits à le reconstruire par raisonnement. Il consiste d’abord dans le plaisir de vivre avec des gens parfaitement polis ; nul plaisir plus pénétrant, plus continu, plus inépuisable. L’amour-propre humain étant infini, des gens d’esprit peuvent toujours inventer quelque raffinement d’égards qui le satisfasse. La sensibilité mondaine étant infinie, il n’y a pas de nuance imperceptible qui la laisse indifférente. Après tout, l’homme est encore la plus grande source de bonheur comme de malheur pour l’homme, et, dans ce temps-là, la source toujours coulante, au lieu d’amertumes, n’apportait que des douceurs. Non seulement il fallait ne pas heurter, mais encore il fallait plaire ; on était tenu de s’oublier pour les autres, d’être toujours pour eux empressé et dispos, de garder pour soi ses contrariétés et ses chagrins, de leur épargner les idées tristes, de leur fournir des idées gaies. « Est-ce qu’on était jamais vieux en ce temps-là ! C’est la Révolution qui a amené la vieillesse dans le monde. Votre grand-père\footnote{George Sand, I, 58-60. Récit de sa grand’mère qui, à trente ans, avait épousé M. Dupin de Francueil, âgé de soixante-deux ans.}, ma fille, a été beau, élégant, soigné, gracieux, parfumé, enjoué, aimable, affectueux et d’une humeur égale, jusqu’à l’heure de sa mort… On savait vivre et mourir alors ; on n’avait pas d’infirmités importunes. Si on avait la goutte, on marchait quand même, et sans faire la grimace ; on se cachait de souffrir par bonne éducation. On n’avait pas de ces préoccupations d’affaires qui gâtent l’intérieur et rendent l’esprit épais. On savait se ruiner sans qu’il y parût, comme de beaux joueurs qui perdent sans montrer d’inquiétude et de dépit. On se serait fait porter demi-mort à une partie de chasse. On trouvait qu’il valait mieux mourir au bal ou à la comédie que dans son lit entre quatre cierges et de vilains hommes noirs. On était philosophe ; on ne jouait pas l’austérité, on l’avait parfois sans en faire montre. Quand on était sage, c’était par goût et sans faire le pédant ou la prude. On jouissait de la vie, et, quand l’heure était venue de la perdre, on ne cherchait pas à dégoûter les autres de vivre. Le dernier adieu de mon vieux mari fut de m’engager à lui survivre longtemps et à me faire une vie heureuse. »\par
Avec les femmes surtout, c’est peu d’être poli, il faut être galant. Chez le prince de Conti, à l’Isle-Adam, chaque dame invitée « trouve une voiture et des chevaux à ses ordres ; elle est maîtresse de donner tous les jours à dîner dans sa chambre à sa société particulière\footnote{Mme de Genlis, {\itshape Souvenirs de Félicie}, 77. — Mme Campan, III, 74. — Mme de Genlis, {\itshape Dictionnaire des Étiquettes}, I, 348.} ». Mme de Civrac étant obligée d’aller aux eaux, ses amis entreprennent de la distraire pendant le voyage ; ils la devancent de quelques postes, et, dans tous les endroits où elle vient coucher, ils lui donnent une petite fête champêtre, déguisés en villageois, en bourgeois, avec bailli, tabellion et autres masques qui chantent et disent des vers  Une dame, la veille de Longchamps, sachant que le vicomte de V… a deux calèches, lui en fait demander une ; il en a disposé, mais il se garde bien de s’excuser, et sur-le-champ il en fait acheter une de la plus grande élégance, pour la prêter trois heures : il est trop heureux qu’on veuille bien lui emprunter quelque chose, et sa prodigalité paraît aimable, mais n’étonne pas. C’est que les femmes alors sont des reines\footnote{Voir sur cette royauté une anecdote dans Mme de Genlis ({\itshape Adèle et Théodore}, I, 69)  Mme Vigée-Lebrun, I, 156 : « Les femmes régnaient alors, la Révolution les a détrônées… Cette galanterie dont je vous parle a totalement disparu. »} ; en effet, dans un salon elles ont le droit de l’être ; voilà pourquoi, au dix-huitième siècle, en toutes choses, elles donnent la règle et le ton\footnote{« Les femmes en France dictent en quelque sorte tout ce qui est à dire et prescrivent tout ce qui est à faire dans le beau monde. » ({\itshape A comparative View}, by John Andrews, 1785.)}. Ayant fait le code des usages, il est tout naturel que ce soit à leur profit, et elles tiennent la main à ce que toutes les prescriptions en soient suivies. À cet égard, tel salon « de la très bonne compagnie » est un tribunal supérieur où l’on juge en dernier ressort\footnote{Mme d’Oberkirch, I, 299  Mme de Genlis, {\itshape Mémoires}, chap. XI.}. La maréchale de Luxembourg est une autorité ; point de bienséance qu’elle ne justifie par une raison ingénieuse. Sur un mot, sur un manque d’usage, sur la moindre apparence de prétention ou de fatuité, on encourt sa désapprobation qui est sans appel, et l’on est perdu à tout jamais dans le beau monde. Sur un trait fin, sur un silence, sur un « oh ! » dit à propos au lieu d’un « ah ! » on reçoit d’elle, comme M. de Talleyrand, le brevet de parfait savoir-vivre qui est le commencement d’une renommée et la promesse d’une fortune. — Sous une telle « institutrice », il est clair que le maintien, le geste, le langage, toute action ou omission de la vie mondaine devient, comme un tableau ou un poème, une œuvre d’art véritable, c’est-à-dire infinie en délicatesses, à la fois aisée et savante, si harmonieuse dans tous ses détails que la perfection y cache la difficulté.\par
Une grande dame « salue dix personnes en se ployant une seule fois, et en donnant, de la tête et du regard, à chacun ce qui lui revient\footnote{Comte de Tilly, I, 24.} », c’est-à-dire la nuance d’égards appropriée à chaque variété d’état, de considération et de naissance. « C’est à des amours-propres faciles à s’irriter qu’elle a toujours affaire, en sorte que le plus léger défaut de mesure serait promptement saisi\footnote{Necker, {\itshape Œuvres complètes}, XV, 259.} » ; mais jamais elle ne se trompe, ni n’hésite dans ces distinctions subtiles ; avec un tact, une dextérité, une flexibilité de ton incomparables, elle met des degrés dans son accueil. Elle en a un « pour les femmes de condition, un pour les femmes de qualité, un pour les femmes de la cour, un pour les femmes titrées, un pour les femmes d’un nom historique, un autre pour les femmes d’une grande naissance personnelle, mais unies à un mari au-dessous d’elles, un autre pour les femmes qui ont changé par leur mariage leur nom commun en un nom distingué, un autre encore pour les femmes d’un bon nom dans la robe, un autre enfin pour celles dont le principal relief est une maison de dépense et de bons soupers ». Un étranger reste stupéfait en voyant de quelle démarche adroite et sûre elle circule parmi tant de vanités en éveil, sans jamais donner ni recevoir un choc. « Elle sait tout exprimer par le mode de ses révérences, mode varié qui s’étend par nuances imperceptibles, depuis l’accompagnement d’une seule épaule qui est presque une impertinence, jusqu’à cette révérence noble et respectueuse que si peu de femmes, même à la cour, savent bien faire, ce plié lent, les yeux baissés, la taille droite, et une manière de se relever en regardant alors modestement la personne et en jetant avec grâce tout le corps en arrière : tout cela plus fin, plus délicat que la parole, mais très expressif comme moyen de respect. » — Ce n’est là qu’une action et très ordinaire ; il y en a cent autres et d’importance : imaginez, s’il est possible, le degré d’élégance et de perfection auquel le savoir-vivre les avait portées. J’en prends une au hasard, un duel entre deux princes du sang, le comte d’Artois et le duc de Bourbon ; celui-ci étant l’offensé, l’autre, son supérieur, était tenu de lui offrir une rencontre\footnote{ Récit de M. de Besenval, témoin du duel.
 }. « Dès que M. le comte d’Artois l’a vu, il a sauté à terre, et, allant droit à lui, il lui a dit d’un air souriant : — Monsieur, le public prétend que nous nous cherchons. M. le duc de Bourbon a répondu en ôtant son chapeau : — Monsieur, je suis ici pour recevoir vos ordres. — Pour exécuter les vôtres, a reparti M. le comte d’Artois, il faut que vous me permettiez d’aller jusqu’à ma voiture. » Il revient avec une épée, le combat commence ; au bout d’un temps, on les sépare, les témoins jugent que l’honneur est satisfait. « Ce n’est pas à moi d’avoir un avis, a repris M. le comte d’Artois ; c’est à M. le duc de Bourbon de dire ce qu’il veut ; je suis ici pour recevoir ses ordres. —  « Monsieur », a répliqué M. le duc de Bourbon en adressant la parole à M. le comte d’Artois et en baissant la pointe de son épée, « je suis pénétré de reconnaissance de vos bontés, et je n’oublierai jamais l’honneur que vous m’avez fait. » Se peut-il un plus juste et plus fin sentiment des rangs, des positions, des circonstances, et peut-on entourer un duel de plus de grâces   Il n’y a pas de situation épineuse qui ne soit sauvée par la politesse. Avec de l’usage et le tour convenable, même en face du roi, on concilie la résistance et le respect. Lorsque Louis XV, ayant exilé le Parlement, fit dire tout haut par Mme du Barry que son parti était pris et qu’il ne changerait jamais : « Ah ! madame, répondit le duc de Nivernais, quand le roi a dit cela, il vous regardait ». — « Mon cher Fontenelle », lui disait une de ses amies en lui mettant la main sur le cœur, « c’est aussi de la cervelle que vous ayez là. » Fontenelle souriait et ne disait pas non : voilà comment, même à un académicien, on faisait avaler ses vérités, une goutte d’acide dans un bonbon, le tout si bien fondu que la saveur piquante ne faisait que relever la saveur sucrée. Tous les soirs, dans chaque salon, on servait des bonbons de cette espèce, deux ou trois avec la goutte d’acide, tous les autres non moins exquis, mais n’ayant que de la douceur et du parfum. — Tel est l’art du monde, art ingénieux et charmant qui pénètre dans tous les détails de la parole et de l’action pour les transformer en grâces, qui impose à l’homme, non la servilité et le mensonge, mais le respect et le souci des autres, et qui en échange extrait pour lui de la société humaine tout le plaisir qu’elle peut donner.

\section[{V. Le bonheur au dix-huitième siècle. — Agrément du décor et de l’entourage. — Oisiveté, passe-temps, badinage.}]{V. Le bonheur au dix-huitième siècle. — Agrément du décor et de l’entourage. — Oisiveté, passe-temps, badinage.}

\noindent On peut bien comprendre en gros ce genre de plaisir ; mais comment le rendre visible ? Pris en eux-mêmes, les passe-temps du monde ne se laissent pas décrire ; ils sont trop légers ; leur charme leur vient de leurs accompagnements. Le récit qu’on en ferait serait un résidu insipide ; est-ce que le libretto d’un opéra donne l’idée de cet opéra   Si vous voulez retrouver ce monde évanoui, cherchez-le dans les œuvres qui en ont conservé les dehors ou l’accent, d’abord dans les tableaux et dans les estampes, chez Watteau, Fragonard et les Saint-Aubin, puis dans les romans et dans les comédies, chez Voltaire et Marivaux, même chez Collé et chez Crébillon fils\footnote{Voir notamment : Saint-Aubin, {\itshape le Bal paré, le Concert.} Moreau, {\itshape les Élégantes, la Vie d’un seigneur à la mode}, les vignettes de {\itshape la Nouvelle Héloïse.} Baudouin, {\itshape la Toilette, le Coucher de la mariée.} Lauwrence, {\itshape Qu’en dit l’abbé}   Watteau, le premier en date et en talent, {\itshape transpose} ces mœurs, et les peint d’autant mieux qu’il les rend plus poétiques. — Relire entre autres : {\itshape Marianne}, par Marivaux ; {\itshape la Vérité dans le vin}, par Collé ; {\itshape le Coin du feu, la Nuit et le Moment}, par Crébillon fils, et, dans la {\itshape Correspondance inédite} de Mme du Deffand, deux lettres charmantes, l’une de l’abbé Barthélemy, l’autre du chevalier de Boufflers (I, 258, 341).} ; alors seulement on revoit les figures, on entend les voix. Quelles physionomies fines, engageantes et gaies, toutes brillantes de plaisir et d’envie de plaire ! Que d’aisance dans le port et dans la démarche ! Quelle grâce piquante dans la toilette et le sourire, dans la vivacité du babil, dans le manège de la voix flûtée, dans la coquetterie des sous-entendus ! Comme on s’attarde involontairement à regarder et à écouter ! Le joli est partout, dans les petites têtes spirituelles, dans les mains fluettes, dans l’ajustement chiffonné, dans les minois et dans les mines. Leur moindre geste, un air de tête boudeur, ou mutin, un bras mignon qui sort de son nid de dentelles, une taille ployante qui se penche à demi sur le métier à broder, le froufrou preste d’un éventail qui s’ouvre, tout ici est un régal pour les yeux et pour l’esprit. En effet ici tout est friandise, caresse délicate pour des sens délicats, jusque dans le décor extérieur de la vie, jusque dans les lignes sinueuses, dans la parure galante, dans la commodité raffinée des architectures et des ameublements. Remplissez votre imagination de ces alentours et de ces figures, et vous trouverez alors à leurs amusements l’intérêt qu’ils y prenaient eux-mêmes. En pareil lieu et en pareille compagnie, il suffit d’être ensemble pour être bien. Leur oisiveté ne leur pèse pas, ils jouent avec la vie  À Chanteloup, où le duc de Choiseul en disgrâce voit affluer tout le beau monde, on ne fait rien, et il n’y a pas dans la journée une heure vide\footnote{{\itshape Correspondance inédite de Mme du Deffand}, publiée par M. de Saint-Aulaire, I, 235, 258, 296, 302, 363.}. « La duchesse n’a que deux heures de temps à elle, et ces deux heures sont pour sa toilette et ses lettres ; le calcul en est simple : elle se lève à onze heures ; à midi, déjeuner suivi d’une conversation qui dure jusqu’à trois ou quatre heures ; le dîner à six, ensuite le jeu et la lecture des Mémoires de Mme de Maintenon. » Ordinairement « on reste en compagnie jusqu’à deux heures du matin ». La liberté d’esprit est parfaite ; nul tracas, nul souci ; le whist et le trictrac l’après-midi, le pharaon le soir. « On fait aujourd’hui ce qu’on a fait hier, et ce qu’on fera demain ; on s’occupe du dîner-souper comme de l’affaire la plus importante de la vie, et l’on ne se plaint de rien au monde que de son estomac. Le temps nous emporte si vite, que je crois toujours être arrivé depuis hier au soir. » Parfois on arrange une petite chasse et les dames veulent bien y assister ; « car elles sont toutes fort lestes et en état de faire tous les jours à pied cinq ou six fois le tour du salon ». Mais elles aiment mieux l’appartement que le grand air ; en ce temps-là le vrai soleil, c’est la clarté des bougies, et le plus beau ciel est un plafond peint ; y en a-t-il un moins sujet aux intempéries, plus commode pour causer, badiner   On cause donc et l’on badine, en paroles avec les amis présents, par lettres avec les amis absents. On sermonne la vieille Mme du Deffand, qui est trop vive et qu’on nomme « la petite fille » ; la jeune duchesse, tendre et sensée, est « sa grand’maman ». Quant au « grand-papa », M. de Choiseul, « comme un petit rhume le tient au lit, il se fait lire des contes de fées toute la journée : c’est une lecture à laquelle nous nous sommes tous mis ; nous la trouvons aussi vraisemblable que l’histoire moderne. Ne pensez pas qu’il soit sans occupations : il s’est fait dresser dans le salon un métier à tapisserie, auquel il travaillait, je ne puis dire avec la plus grande adresse, du moins avec la plus grande assiduité… Maintenant, c’est un cerf-volant qui fait notre bonheur ; le grand-papa ne connaissait pas ce spectacle, il en est ravi »  En lui-même, un passe-temps n’est rien ; selon l’occasion ou le goût du moment, on le prend, on le laisse, et bientôt l’abbé écrit : « Je ne vous parle plus de nos chasses parce que nous ne chassons plus, ni de nos lectures parce qu’on ne lit plus, ni de nos promenades parce que nous ne sortons point. Que faisons-nous donc ? Les uns jouent au billard, d’autres aux dominos, d’autres au trou-madame. Nous défilons, effilons, parfilons. Le temps nous pousse et nous le lui rendons bien ».\par
Même spectacle dans les autres compagnies. Toute occupation étant un jeu, il suffit d’un caprice, d’un souffle de la mode pour en mettre une en honneur. À présent, c’est le parfilage, et, à Paris, dans les châteaux, toutes les mains blanches défont les galons, les épaulettes, les vieilles étoffes, pour en retirer les fils d’or et d’argent. Elles trouvent à cela un semblant d’économie, une apparence d’occupation, en tout cas une contenance. À peine un cercle de femmes est-il formé, qu’on pose sur la table un gros sac à parfiler en taffetas vert ; c’est celui de la maîtresse du logis ; toutes les dames aussitôt demandent leurs sacs « et voilà les laquais en l’air\footnote{Mme de Genlis, {\itshape Dictionnaire des Étiquettes}, II, 38. — George Sand, {\itshape Histoire de ma vie.} 1, 228. — Mme de Genlis, {\itshape Adèle et Théodore}, I, 312 ; II, 350. — E. et J. de Goncourt, III.} ». C’est une fureur ; on parfile tous les jours et plusieurs heures par jour ; telle y gagne cent louis par an. Les hommes sont tenus de fournir les matériaux de l’ouvrage : à cet effet, le duc de Lauzun donne à Mme de V… une harpe de grandeur naturelle recouverte de fils d’or ; un énorme mouton d’or apporté en cadeau par le comte de Lowendal a coûté deux ou trois mille francs et rapportera, effiloché, 500 ou 600 livres. Mais on n’y regarde pas de si près : il faut bien un emploi aux doigts oisifs, un débouché manuel à l’activité nerveuse ; la pétulance rieuse éclate au milieu du prétendu travail. Un jour, au moment de sortir pour la promenade avec un gentilhomme, Mme de R… remarque que les franges d’or de son habit seraient excellentes à parfiler, et, d’un élan soudain, elle coupe une des franges. À l’instant dix femmes entourent l’homme aux franges, lui arrachent son habit et mettent toutes ses franges et ses galons dans leurs sacs ; on dirait d’une volée de mésanges hardies qui, bruissant, caquetant, s’abattent à la fois sur un geai pour lui dérober son plumage, et désormais, quand un homme entre dans un cercle de femmes, il court risque d’être plumé vif. — Tout ce joli monde a les mêmes passe-temps, et les hommes aussi bien que les femmes. Il n’est guère d’homme qui n’ait quelque talent de salon, quelque petit moyen d’occuper son esprit ou ses mains, de remplir les heures vides : presque tous riment et sont acteurs de société ; beaucoup sont musiciens, peintres de nature morte ; tout à l’heure M. de Choiseul faisait de la tapisserie ; d’autres brodent ou font des nœuds. M. de Francueil est bon violon et fabrique ses violons lui-même, outre cela « horloger, architecte, tourneur, peintre, serrurier, décorateur, poète, compositeur de musique et brodant à merveille\footnote{ George Sand, I, 59.
 } ». Dans cette oisiveté générale, il faut bien « savoir s’occuper d’une manière agréable pour les autres autant que pour soi-même ». Mme de Pompadour est musicienne, actrice, peintre et graveur ; Madame Adélaïde apprend l’horlogerie et joue de tous les instruments, depuis le cor jusqu’à la guimbarde, pas très bien, à la vérité, à peu près comme la reine, dont la jolie voix n’est qu’à demi juste. Mais on n’y met pas de prétentions ; il s’agit de s’amuser, rien de plus ; l’entrain, l’aménité couvrent tout. Lisez plutôt ce haut fait de Mme de Lauzun à Chanteloup : « Savez-vous, écrit l’abbé, que personne ne possède à un plus haut degré une qualité que vous ne lui connaissez pas, celle de faire les œufs brouillés ? C’était un talent enfoui ; elle ne se souvient pas du temps où elle l’a reçu ; je crois que c’est en naissant. Le hasard l’a fait connaître, aussitôt on l’a mis à l’épreuve. Hier matin, époque à jamais mémorable dans l’histoire des œufs, on apporte tous les instruments nécessaires à cette grande opération, un réchaud, du bouillon, du sel, du poivre, des œufs ; et voilà Mme de Lauzun qui d’abord tremble et rougit, et qui ensuite, avec un courage intrépide, casse les œufs, les écrase dans la casserole, les tourne à droite, à gauche, dessus, dessous, avec une précision et un succès dont il n’y a pas d’exemple ; on n’a jamais rien mangé de si excellent. » Que de rires aimables et légers autour de cette seule petite scène ! Et, plus tard, que de madrigaux et d’allusions ! La gaieté ressemble alors à un rayon dansant de lumière ; elle voltige au-dessus de toute chose et pose sa grâce sur le moindre objet.

\section[{VI. La gaieté au dix-huitième siècle. — Ses causes et ses effets. — Tolérance et licence. — Bals, fêtes, chasses, festins, plaisirs. — Libertés des magistrats et des prélats.}]{VI. La gaieté au dix-huitième siècle. — Ses causes et ses effets. — Tolérance et licence. — Bals, fêtes, chasses, festins, plaisirs. — Libertés des magistrats et des prélats.}

\noindent « Être toujours gai, dit un voyageur anglais en 1785\footnote{{\itshape A comparative View}, etc. by John Andrews (1785).}, voilà le propre du Français », et il remarque que cela est d’obligation, parce qu’en France tel est le ton du monde et la seule façon de plaire aux dames, souveraines de la société et arbitres du bon goût. Ajoutez l’absence des causes qui font la tristesse moderne et mettent au-dessus de nos têtes un pesant ciel de plomb. Point de travail âpre et précoce en ce temps-là ; point de concurrence acharnée ; point de carrières indéfinies ni de perspectives infinies. Les rangs sont marqués, les ambitions sont bornées, l’envie est moindre. L’homme n’est pas habituellement mécontent, aigri, préoccupé comme aujourd’hui. On souffre peu des passe-droits là où il n’y a pas de droits ; nous ne songeons qu’à avancer, ils ne songent qu’à s’amuser. Au lieu de maugréer sur l’Annuaire, un officier invente un travestissement de bal masqué ; au lieu de compter les condamnations qu’il a obtenues, un magistrat donne un beau souper. À Paris, dans l’allée de gauche du Palais-Royal, toutes les après-midi, « la bonne compagnie en fort grande parure se réunit sous les grands arbres » ; le soir, « au sortir de l’Opéra, à huit heures et demie, on y revient, et l’on y reste souvent jusqu’à deux heures du matin ». On y fait de la musique en plein air, au clair de lune, Garat chante et le chevalier de Saint-Georges joue du violon\footnote{ Mme Vigée-Lebrun, I, 15, 154.
 }. À Morfontaine, « le comte de Vaudreuil, Lebrun le poète, le chevalier de Coigny, si aimable et si gai, Brongniart, Robert, font toutes les nuits des charades et se réveillent pour se les dire ». À Maupertuis chez M. de Montesquiou, à Saint-Ouen chez le duc de Nivernais, à Saint-Germain chez le maréchal de Noailles, à Gennevilliers chez le comte de Vaudreuil, au Raincy chez le duc d’Orléans, à Chantilly chez le prince de Condé, ce ne sont que fêtes. On ne peut lire une biographie, un document de province, un inventaire du temps, sans entendre tinter les grelots de l’universel carnaval. À Monchoix\footnote{Chateaubriand, I, 34  {\itshape Mémoires de Mirabeau}, passim  George Sand, I, 59, 76.}, chez le comte de Bédée, oncle de Chateaubriand, « on faisait de la musique, on dansait, on chassait, on était en liesse du matin jusqu’au soir, on mangeait son fonds et son revenu ». À Aix et Marseille, dans tout le beau monde, chez le comte de Valbelle, je ne vois que concerts, divertissements, bals, galanteries, théâtres de société avec la comtesse de Mirabeau pour première actrice. À Châteauroux, M. Dupin de Francueil entretient « une troupe de musiciens, de laquais, de cuisiniers, de parasites, de chevaux et de chiens, donnant tout à pleines mains, au plaisir et à la bienfaisance, voulant être heureux et que tout le monde le soit autour de lui », sans vouloir compter et jusqu’à se ruiner le plus aimablement du monde. Rien n’étouffe cette gaieté, ni l’âge, ni l’exil, ni le malheur ; en 1793, elle durait encore dans les prisons de la République  Un homme en place n’est point alors gêné par son habit, raidi par son emploi, obligé de garder l’air important et digne, astreint à cette gravité de commande que l’envie démocratique nous impose comme une rançon. En 1753\footnote{{\itshape Comptes rendus de la Société du Berry} (1863-1864).}, les parlementaires, qu’on vient d’exiler à Bourges, arrangent trois théâtres de société, jouent la comédie, et l’un d’eux, M. Dupré de Saint-Maur, trop galant, se bat à l’épée contre un rival. En 1787\footnote{{\itshape Histoire de Troyes pendant la Révolution}, par Albert Babeau, I, 46.}, quand tout le Parlement est relégué à Troyes, l’évêque, M. de Barral, revient exprès de son château de Saint-Lye pour le recevoir et préside tous les soirs à un dîner de quarante couverts. « C’étaient, dans toute la ville, des fêtes et des repas sans fin ; les présidents tenaient table ouverte » ; la consommation des traiteurs en fut triplée, et l’on brûla tant de bois dans les cuisines, que la ville fut sur le point d’en manquer. En temps ordinaire, la bombance et la joie ne sont guère moindres. Un parlementaire, comme un seigneur, doit se faire honneur de sa fortune ; voyez dans les lettres du président de Brosses la société de Dijon ; elle fait penser à l’abbaye de Thélème ; puis mettez en regard la même ville aujourd’hui\footnote{Foisset, {\itshape Le président de Brosses}, 65, 69, 70, 346. — {\itshape Lettres du président de Brosses} (Éd. Coulomb), passim. — Piron étant inquiété pour son {\itshape Ode à Priape}, le président Bouhier, « homme de haute et fine érudition et le moins gourmé des doctes », fit venir le jeune homme et lui dit : « Vous êtes un imprudent ; si l’on vous presse trop fort pour savoir l’auteur du délit, vous direz que c’est moi ». (Sainte-Beuve, {\itshape Nouveaux lundis}, VII, 414.)}. En 1744, à propos de la guérison du roi, M. de Montigny, frère du président de Bourbonne, invite à souper tous les ouvriers, marchands et artisans qu’il emploie, au nombre de quatre-vingts, avec une seconde table pour ses commis, secrétaires, médecins, chirurgiens, procureurs et notaires ; le cortège s’assemble autour d’un char de triomphe couvert de bergères, de bergers et de divinités champêtres en costume d’opéra ; des fontaines laissent couler le vin « comme s’il était de l’eau », et, après le souper, on jette toutes les confitures par les fenêtres  Autour de celui-ci, chaque parlementaire « a son petit Versailles, un grand hôtel entre cour et jardin ». La ville, silencieuse aujourd’hui, retentit toute la journée du bruit des beaux équipages. La prodigalité des tables est étonnante, « non pas seulement aux jours de gala, mais dans les soupers de chaque semaine, j’ai presque dit de chaque jour »  Au milieu de tous ces donneurs de fêtes, le plus illustre de tous, le président de Brosses, si grave sur les fleurs de lys, si intrépide dans ses remontrances, si laborieux\footnote{Foisset, {\itshape ibid.}, 185. Six audiences par semaine, et souvent deux par jour, outre ses travaux d’antiquaire, d’historien, de linguiste, de géographe, d’éditeur et d’académicien.}, si érudit, est un boute-en-train merveilleux, un vrai Gaulois, d’une verve étincelante, intarissable en plaisanteries salées : devant ses amis, il ôte sa perruque, sa robe et même quelque chose de plus. Nul ne songe à s’en scandaliser : personne n’imagine qu’un habit doive être un éteignoir, et cela est vrai de tous les habits, en premier lieu de la robe. « Quand je suis entré dans le monde, en 1785, écrit un parlementaire\footnote{{\itshape Souvenirs manuscrits}, par le chancelier Pasquier.}, je me suis vu présenter en quelque sorte parallèlement chez les femmes et chez les maîtresses des amis de ma famille, passant la soirée du lundi chez l’une, celle du mardi chez l’autre. Et je n’avais pas dix-huit ans ! Et j’étais d’une famille magistrale ! » À Basville, chez M. de Lamoignon, pendant les vacances de la Pentecôte et de l’automne, il y a chaque jour trente personnes à table ; on chasse trois et quatre fois par semaine, et les plus illustres magistrats, M. de Lamoignon, M. Pasquier, M. de Rosambo, M. et Mme d’Aguesseau, jouent {\itshape le Barbier de Séville} sur le théâtre du château.\par
Quant à la soutane, elle a les mêmes libertés que la robe. À Saverne, à Clairvaux, au Mans et ailleurs, les prélats la portent aussi gaillardement qu’un habit de cour. Pour la leur coller au corps, il a fallu la tourmente révolutionnaire, puis la surveillance hostile d’un parti organisé et la menace d’un danger continu. Jusqu’en 1789, le ciel est trop beau, l’air est trop tiède, pour qu’on se résigne à se boutonner jusqu’au cou. « Liberté, facilité, monsieur l’abbé, disait le cardinal de Rohan à son secrétaire ; sans cela nous ferions de ceci un désert\footnote{Marquis de Valfons, {\itshape Souvenirs}, 60.}. » C’est de quoi le bon cardinal s’était bien gardé ; tout au contraire il avait fait de Saverne un monde enchanté d’après Watteau, presque « un embarquement pour Cythère ». Six cents paysans et les gardes rangés en file forment le matin une chaîne longue d’une lieue et battent la campagne environnante ; cependant les chasseurs, hommes et femmes, sont postés. « De crainte que les dames n’eussent peur seules, on leur laissait toujours l’homme qu’elles haïssaient le moins, pour les rassurer », et, comme il était défendu de quitter son poste avant le signal, « il devenait impossible d’être surpris »  Vers une heure après midi, « la compagnie se rassemblait sous une belle tente, au bord d’un ruisseau ou dans quelque endroit délicieux ; on servait un dîner exquis, et, comme il fallait que tout le monde fût heureux, chaque paysan recevait une livre de viande, deux de pain, une demi-bouteille de vin, et ne demandait qu’à recommencer, ainsi que les dames ». Certainement, aux gens scrupuleux l’obligeant prélat eût répondu avec Voltaire « qu’il n’est jamais de mal en bonne compagnie ». De fait, il le disait, et en propres termes. Un jour, une dame accompagnée d’un jeune officier étant venue en visite, comme il les retenait à coucher, son valet de chambre « vient l’avertir tout bas qu’il n’a plus de place  Est-ce que l’appartement des bains est plein   Non, Monseigneur  N’y a-t-il pas deux lits   Oui, Monseigneur, mais ils sont dans la même chambre, et cet officier..  Eh bien ! ne sont-ils pas venus ensemble ? Les gens bornés comme vous voient toujours en mal. Vous verrez qu’ils s’accommoderont très bien ; il n’y a pas la plus petite réflexion à faire. » Effectivement il paraît que personne n’en fit, ni l’officier ni la dame  À Granselve\footnote{Montgaillard (témoin oculaire), {\itshape Histoire de France.} II, 246.}, dans le Gard, les bernardins sont encore plus hospitaliers ; on y vient de quinze ou vingt lieues, pour la fête de saint Bernard qui dure deux semaines ; pendant tout le temps, on danse, on chasse, on joue la comédie, « les tables sont servies à toute heure ». Le quartier des dames est pourvu de tout ce qu’il faut pour la toilette ; rien ne leur manque, et l’on dit même qu’aucune d’elles n’a besoin d’amener son officier  Je citerais vingt prélats non moins galants, le second cardinal de Rohan, héros du collier, M. de Jarente, évêque d’Orléans, qui tient la feuille des bénéfices, le jeune M. de Grimaldi, évêque du Mans, M. de Breteuil, évêque de Montauban, M. de Cicé, archevêque de Bordeaux, le cardinal de Montmorency, grand aumônier, M. de Talleyrand, évêque d’Autun, M. de Conzié, évêque d’Arras\footnote{ \noindent M. de Conzié est surpris à quatre heures du matin par son rival, officier aux gardes : « Point de bruit, lui dit-il, on va m’apporter un habit pareil au vôtre, je me ferai faire une queue et nous serons de niveau ». Un valet de chambre lui apporte son équipage de guerre et de bonne fortune. Il descend dans le jardin de l’hôtel, se bat avec l’officier et le désarme. ({\itshape Correspondance}, par Metra, XIV, 20 mai 1783.) — {\itshape Le comte de Clermont}, par Jules Cousin, passim  {\itshape Journal de Collé}, III, 232 (juillet 1769).
 }, au premier rang l’abbé de Saint-Germain des Prés, comte de Clermont, prince du sang, qui, ayant trois cent soixante-dix mille livres de rente, trouve moyen de se ruiner deux fois, joue la comédie chez lui à la ville et à la campagne, écrit à Collé en style de parade, et, dans sa maison abbatiale de Berny, installe une danseuse, Mlle Leduc, pour faire les honneurs de sa table. — Nulle hypocrisie : chez M. Trudaine, quatre évêques assistent à une pièce de Collé, intitulée {\itshape les Accidents ou les Abbés}, et dont le fond, dit Collé lui-même, est si libre qu’il n’a pas osé la faire imprimer avec les autres. Un peu plus tard, Beaumarchais, lisant chez la maréchale de Richelieu son {\itshape Mariage de Figaro}, non expurgé, bien plus vert et bien plus cru qu’aujourd’hui, a pour auditeurs des évêques et des archevêques, et ceux-ci, dit-il, « après s’en être infiniment amusés, m’ont fait l’honneur de m’assurer qu’ils publieraient qu’il n’y avait pas un seul mot dont les bonnes mœurs pussent être blessées\footnote{L. de Loménie, {\itshape Beaumarchais et son temps}, II, 304.} » : c’est ainsi que la pièce passa, contre la raison d’État, contre la volonté du roi, par la complicité de tous, même des plus intéressés à la supprimer. « Il y a quelque chose de plus fou que ma pièce, disait l’auteur lui-même, c’est son succès. » L’attrait était trop fort ; des gens de plaisir ne pouvaient renoncer à la comédie la plus gaie du siècle ; ils vinrent applaudir leur propre satire ; bien mieux, ils la jouèrent eux-mêmes  Quand un goût est régnant, il aboutit, comme une grande passion, à des extrémités qui sont des folies ; à tout prix, il lui faut la jouissance offerte. Devant la satisfaction du moment, il est comme un enfant devant un fruit, et rien ne l’arrête, ni le danger puisqu’il l’oublie, ni les convenances puisqu’il les fait.

\section[{VII. Principal divertissement, la comédie de société. — Parades et excès.}]{VII. Principal divertissement, la comédie de société. — Parades et excès.}

\noindent Se divertir, c’est se détourner de soi, s’en déprendre, en sortir ; et, pour en bien sortir, il faut se transporter dans autrui, se mettre à la place d’un autre, prendre son masque, jouer son rôle. Voilà pourquoi le plus vif des divertissements est la comédie où l’on est acteur. C’est celui des enfants qui, tout le long du jour, auteurs, acteurs, spectateurs, improvisent et représentent de petites scènes. C’est celui des peuples que leur régime politique exclut des soucis virils et qui jouent avec la vie à la façon des enfants. À Venise, au dix-huitième siècle, le carnaval dure six mois ; en France, sous une autre forme, il dure toute l’année. Moins familier et moins pittoresque, plus raffiné et plus élégant, il a quitté la place publique où le soleil lui manque, pour s’enfermer dans les salons où les lustres lui conviennent mieux. De la grande mascarade populaire, il ne garde qu’un lambeau, le bal de l’Opéra, magnifique d’ailleurs et fréquenté par les princes, par les princesses, par la reine. Mais ce lambeau, si brillant qu’il soit, ne lui suffit point, et, dans tous les châteaux, dans tous les hôtels, à Paris, en province, il installe les travestissements de société et la comédie à domicile  Pour accueillir un grand personnage, pour célébrer la fête du maître ou de la maîtresse de la maison, ses hôtes ou ses invités lui jouent une opérette improvisée, quelque pastorale ingénieuse et louangeuse, tantôt habillés en Dieux, en Vertus, en abstractions mythologiques, en Turcs, en Lapons, en Polonais d’opéra, et pareils aux figures qui ornent alors le frontispice des livres ; tantôt en costumes de paysans, de magisters, de marchands forains, de laitières, de rosières, et semblables aux villageois bien appris dont le goût du temps peuple alors le théâtre. Ils chantent, ils dansent, et viennent tour à tour débiter de petits vers de circonstance qui sont des compliments bien tournés\footnote{Duc de Luynes, XVI, 161 (septembre 1757). Fête villageoise donnée au roi Stanislas par Mme de Mauconseil, à Bagatelle. — Bachaumont, III, 247 (7 septembre 1767). Fête donnée par la prince de Condé.}. — À Chantilly, « la jeune et charmante duchesse de Bourbon, parée en voluptueuse Naïade, conduit le comte du Nord, dans une gondole dorée, à travers le grand canal, jusqu’à l’île d’Amour » ; de son côté, le prince de Conti sert de pilote a la grande-duchesse ; les autres seigneurs et les dames, « chacun sous des vêtements allégoriques », font l’équipage\footnote{{\itshape Correspondance}, par Metra, XIII, 97 (15 juin 1782), et V, 232 (24 et 25 juin 1777). — Mme de Genlis, {\itshape Mémoires}, chap. XIV.}, et, sur ces belles eaux, dans ce nouveau jardin d’Alcine, le riant et galant cortège semble une féerie du Tasse  Au Vaudreuil, les dames, averties qu’on veut les enlever pour le sérail, s’habillent en vestales, et le grand prêtre, avec de jolis couplets, les reçoit dans son temple au milieu du parc ; cependant plus de trois cents Turcs arrivent, forcent l’enceinte au son de la musique, et emportent les dames sur des palanquins le long des jardins illuminés  Au Petit Trianon, le parc représente une foire, les dames de la cour y sont les marchandes, « la reine tient un café comme limonadière », çà et là sont des parades et des théâtres ; la fête coûte, dit-on, quatre cent mille livres, et l’on va recommencer à Choisy sur plus grands frais.\par
À côté de ces déguisements qui s’arrêtent au costume et ne prennent qu’une heure, il est une distraction plus forte, la comédie de société qui transforme l’homme tout entier, et qui, pendant six semaines, pendant trois mois, l’occupe tout entier aux répétitions. Vers 1770\footnote{Bachaumont, 17 novembre 1770. — {\itshape Journal de Collé}, III, 136 (20 avril 1767). — Comte de Montlosier, {\itshape Mémoires}, I, 43. « Chez M. le Commandant (à Clermont) on voulut bien m’enrôler pour les comédies de société. »}, « c’est une fureur incroyable ; il n’est pas de procureur dans sa bastide qui ne veuille avoir des tréteaux et une troupe ». Un bernardin, qui vit en Bresse au milieu des bois, écrit à Collé qu’il va jouer avec ses confrères {\itshape la Partie de chasse de Henri IV}, et faire construire un petit théâtre « à l’insu des cagots et des petits esprits ». Des réformateurs, des moralistes font entrer l’art théâtral dans l’éducation des enfants ; Mme de Genlis compose des comédies à leur usage et juge que cet exercice est excellent pour donner une bonne prononciation, l’assurance convenable et les grâces du maintien. En effet le théâtre alors prépare l’homme au monde, comme le monde prépare l’homme au théâtre ; dans l’un et dans l’autre, on est en spectacle, on compose son attitude et son ton de voix, on joue un rôle ; la scène et le salon sont de plain-pied  Vers la fin du siècle, tout le monde devient acteur ; c’est que tout le monde l’était déjà\footnote{{\itshape Correspondance}, par Metra, II, 245 (18 novembre 1775).}. « On n’entend parler que de petits théâtres montés dans la campagne autour de Paris. » Depuis longtemps, les plus grands donnaient l’exemple. Sous le roi Louis XV, les ducs d’Orléans, de Nivernais, d’Ayen, de Coigny, les marquis de Courtenvaux et d’Entraigues, le comte de Maillebois, la duchesse de Brancas, la comtesse d’Estrades forment avec Mme de Pompadour la troupe « des petits cabinets » ; le duc de la Vallière en est le directeur : quand la pièce renferme un ballet, le marquis de Courtenvaux, le duc de Beuvron, les comtes de Melfort et de Langeron sont les danseurs en titre\footnote{Julien, {\itshape Histoire du théâtre de Mme de Pompadour.} Ces représentations durent sept ans et coûtent, pendant le seul hiver de 1749, 300 000 livres. — Duc de Luynes, X, 45. — Mme du Musset, 230.}. « Ceux qui sont dans l’usage de ces spectacles, écrit le sage et pieux duc de Luynes, conviennent qu’il serait difficile que des comédiens de profession jouassent mieux et avec plus d’intelligence. » — À la fin l’entraînement gagne encore plus haut et jusqu’à la famille royale. À Trianon, d’abord devant quarante personnes, puis devant un public fort étendu, la reine joue Colette dans {\itshape le Devin de village}, Gotte dans {\itshape la Gageure imprévue}, Rosine dans {\itshape le Barbier de Séville}, Pierrette dans {\itshape le Chasseur et la Laitière}\footnote{Mme Campan, I, 130. Cf. avec précaution les {\itshape Mémoires} très arrangés et très suspects de Fleury. — E. et J. de Goncourt, 114.}, et les autres comédiens sont les principaux de la cour, le comte d’Artois, les comtes d’Adhémar et de Vaudreuil, la comtesse de Laguiche, la chanoinesse de Polignac. On trouve un théâtre chez Monsieur ; il y en a deux chez le comte d’Artois, deux chez le duc d’Orléans, deux chez le comte de Clermont, un chez le prince de Condé. Le comte de Clermont tient les rôles « à manteaux sérieux » ; le duc d’Orléans représente avec rondeur et naturel les paysans et les financiers ; M. de Miromesnil, garde des sceaux, est le Scapin le plus fin et le plus délié ; M. de Vaudreuil semble un rival de Molé ; le comte de Pons joue {\itshape le Misanthrope} avec une perfection rare\footnote{Jules Cousin, {\itshape Le comte de Clermont}, 21  Mme de Genlis, {\itshape Mémoires}, chap. 3 et 11  E. et J. de Goncourt, 114.}. « Plus de dix de nos femmes du grand monde, écrit le prince de Ligne, jouent et chantent mieux que tout ce que j’ai vu de mieux sur tous nos théâtres. » — Par leur talent, jugez de leurs études, de leur assiduité et de leur zèle ; il est évident que, pour beaucoup d’entre eux, cette occupation était la principale. Il y avait tel château, celui de Saint-Aubin, où la dame du logis, pour avoir une troupe suffisante, enrôlait ses quatre femmes de chambre, faisait jouer {\itshape Zaïre} à sa fille âgée de dix ans, et, pendant plus de vingt mois, ne faisait pas relâche. Après sa banqueroute et dans son exil, le premier soin de la princesse de Guéméné fut de mander les tapissiers pour leur faire dresser un théâtre. Bref, de même qu’à Venise on ne sortait plus qu’en masque, de même ici l’on ne comprenait plus la vie qu’avec les travestissements, les métamorphoses, les exhibitions et les succès de l’histrion.\par
Dernier trait, plus significatif encore, je veux parler de la petite pièce. Véritablement, pour ce beau monde, la vie est un carnaval aussi libre et presque aussi débraillé qu’à Venise. D’ordinaire le spectacle finit par une parade empruntée aux contes de La Fontaine ou aux farces des bouffons italiens, non seulement vive, mais plus que leste, et parfois si crue, « qu’on ne peut la jouer que devant de grands princes ou des filles\footnote{Bachaumont, III, 343 (23 février 1768) et III, 232 ; IV, 174  {\itshape Journal de Collé}, passim. — Collé, Laujon, Poinsinet sont les principaux fournisseurs de ces parades ; la seule bonne est {\itshape la Vérité dans le vin.} Dans cette dernière pièce, au lieu de « Mylord » il {\itshape y} avait d’abord « l’évêque d’Avranches », et la pièce fut jouée ainsi à Villers-Cotterets, chez le duc d’Orléans.} » ; en effet, un palais blasé se dégoûte de l’orgeat et demande du rogomme. Le duc d’Orléans chante sur la scène les chansons les plus épicées, joue Bartholin dans {\itshape Nicaise} et Blaise dans {\itshape Jaconde, le Mariage sans curé, Léandre grosse, l’Amant poussif, Léandre étalon}, voilà des titres de parades « composées par Collé pour les plaisirs de Son Altesse et de la cour ». Contre une qui a du sel, il y en a dix bourrées de gros poivre. À Brunoy, chez Monsieur, elles sont si grivoises\footnote{Mme d’Oberkirch, II, 82  Sur le ton des meilleures sociétés, voir {\itshape Correspondance}, par Metra, I, 20 ; III, 68  et Besenval (Ed. Barrière), 387 à 394.} que le roi se repent d’y être venu ; « on n’avait pas l’idée d’une telle licence ; deux femmes qui étaient dans la salle sont obligées de se sauver, et, chose énorme, on avait osé inviter la reine ». La gaieté est une sorte d’ivresse qui puise jusqu’au dernier fond du tonneau, et, après le vin, boit la lie. Non seulement dans leurs petits soupers et avec des filles, mais dans le beau monde et avec des dames, ils font des folies de guinguette. Tranchons le mot, ce sont des polissons, et ils ne reculent pas plus devant le mot que devant la chose. « Depuis cinq ou six mois, écrit une dame en 1782\footnote{Mme de Genlis, {\itshape Adèle et Théodore}, II, 362.}, les soupers sont suivis d’un colin-maillard ou d’un traîne-ballet et finissent par {\itshape une polissonnerie générale.} » On y invite les gens quinze jours d’avance. « Cette fois, on renversa les tables, les meubles ; on jeta dans la chambre vingt carafes d’eau ; enfin je me retirai à une heure et demie, excédée de fatigue, assommée de coups de mouchoir, et laissant Mme de Clarence avec une extinction de voix, une robe déchirée en mille morceaux, une écorchure au bras, une contusion à la tête, mais s’applaudissant d’avoir donné un souper d’une telle gaieté et se flattant qu’il ferait la nouvelle du lendemain. » — Voilà où conduit le besoin d’amusement. Sous sa pression, comme sous le doigt d’un sculpteur, le masque du siècle se transforme par degrés et perd insensiblement son sérieux : la figure compassée du courtisan devient d’abord la physionomie enjouée du mondain ; puis, sur cette bouche souriante dont les contours s’altèrent, on voit éclater le rire effronté et débridé du gamin\footnote{G. Sand, I, 85. Chez ma grand’mère, « j’ai trouvé des cartons pleins de couplets, de madrigaux, de satires sanglantes… J’en ai brûlé de tellement obscènes que je n’aurais osé les lire jusqu’au bout, et celles-là écrites de la main d’abbés que j’avais connus dans mon enfance, et sortant du cerveau de marquis de bonne race. » Entre autres spécimens adoucis, on peut lire dans la {\itshape Correspondance}, par Metra, les chansons sur l’Oiseau et sur la Bergère.}.
\chapterclose


\chapteropen

\chapter[{Chapitre III. Inconvénients de la vie de salon.}]{Chapitre III. \\
Inconvénients de la vie de salon.}


\chaptercont

\section[{I. Elle est artificielle et sèche. — Retour à la nature et au sentiment.}]{I. Elle est artificielle et sèche. — Retour à la nature et au sentiment.}

\noindent À la longue, le simple plaisir cesse de plaire, et, si agréable que soit la vie de salon, elle finit par sembler vide. Quelque chose manque, sans qu’on puisse encore dire clairement ce que c’est ; l’âme s’inquiète, et peu à peu, avec l’aide des écrivains et des artistes, elle va démêler la cause de son malaise et l’objet de son secret désir. Artificiel et sec, voilà les deux traits du monde, d’autant plus marqués qu’il est plus parfait, et, dans celui-ci, poussés à l’extrême, parce qu’il est arrivé au suprême raffinement. — D’abord le naturel en est exclu ; tout y est arrangé, apprêté, le décor, le costume, l’attitude, le son de voix, les paroles, les idées et jusqu’aux sentiments. « La rareté d’un sentiment vrai est si grande, disait M. de V., que, lorsque je reviens de Versailles, je m’arrête quelquefois dans les rues à regarder un chien ronger un os\footnote{Chamfort, 110.}. » L’homme, s’étant livré tout entier au monde, n’avait gardé pour soi aucune portion de sa personne, et les convenances, comme autant de lianes, avaient enlacé toute la substance de son être et tout le détail de son action. Il y avait alors, dit une personne qui a subi cette éducation\footnote{George Sand, V, 59 : « On me reprenait sur tout, et je ne faisais pas un mouvement qui ne fût critiqué. »}, une manière de marcher, de s’asseoir, de saluer, de ramasser son gant, de tenir sa fourchette, de présenter un objet, enfin une mimique complète qu’on devait enseigner aux enfants de très bonne heure, afin qu’elle leur devînt par l’habitude une seconde nature, et cette convention était un article de si haute importance dans la vie des hommes et des femmes de l’ancien beau monde que les acteurs ont peine aujourd’hui, malgré toutes leurs études, à nous en donner une {\itshape idée} ». — Non seulement le dehors, mais encore le dedans était factice ; il y avait une façon obligée de sentir, de penser, de vivre et de mourir. Impossible de parler à un homme sans se mettre à ses ordres, et à une femme sans se mettre à ses pieds. Le bon ton avait réglé d’avance toutes les grandes et petites démarches, la manière de se déclarer à une dame et de rompre avec elle, d’engager et de conduire un duel, de traiter un égal, un subordonné, un supérieur. Si l’on manquait en quoi que ce fût à ce code universel de l’usage, on était « une espèce ». Tel homme de cœur et de talent, d’Argenson, fut surnommé « la bête », parce que son originalité dépassait le cadre convenu. « Cela n’a pas de nom, cela ne ressemble à rien », tel est le blâme le plus fort. Dans la conduite comme dans la littérature, tout ce qui s’écarte d’un certain modèle est rejeté. Le nombre des actions permises s’est restreint comme le nombre des mots autorisés. Le même goût épuré appauvrit l’initiative en même temps que la langue, et l’on agit comme on écrit, selon des formes apprises, dans un cercle borné. À aucun prix, l’excentrique, l’imprévu, le vif élan spontané ne sont de mise. — Entre vingt exemples qui se pressent, je choisis le moindre, puisqu’il s’agit d’un simple geste : de là on peut conclure aux autres choses. Mlle de…, par le crédit de sa famille, obtient une pension pour Marcel, célèbre maître à danser, accourt chez lui toute joyeuse et lui présente le brevet. Marcel le prend et le jette à terre : « Est-ce ainsi, Mademoiselle, que je vous ai enseigné à présenter quelque chose ? Ramassez ce papier, et rapportez-le-moi comme vous le devez. » Elle reprend le brevet, et le lui présente avec toutes les grâces voulues. « C’est bien, Mademoiselle, dit Marcel, je le reçois, quoique votre coude n’ait pas été assez arrondi, et vous remercie\footnote{ \noindent {\itshape Paris, Versailles et les provinces}, I, 162. — « Le roi de Suède est ici, il a des rosettes à ses culottes, tout est fini, c’est un homme ridicule et un roi de province. » ({\itshape Le Gouvernement de Normandie}, par Hippeau, IV, 237, 4 juillet 1784.)
 }. » — Tant de grâces finissent par lasser ; après n’avoir mangé pendant des années que d’une cuisine savante, on demande du lait et du pain bis.\par
Entre tous ces assaisonnements mondains, il en est un surtout dont on abuse, et qui, employé sans relâche, communique à tous les mets sa saveur piquante et froide : je veux dire le badinage. Le monde ne souffre pas la passion, et en cela il est dans son droit. On n’est pas en compagnie pour se montrer véhément ou sombre ; l’air concentré ou tendu y ferait disparate. La maîtresse de maison a toujours droit de dire à un homme que son émotion contenue réduit au silence : « Monsieur un tel, vous n’êtes pas aimable aujourd’hui ». Il faut donc être toujours aimable, et, à ce manège, la sensibilité qui se disperse en mille petits canaux ne peut plus faire un grand courant. « On avait cent amis, et sur cent amis, il y en a chaque jour deux ou trois qui ont un chagrin vif : mais on ne pouvait longtemps s’attendrir sur leur compte, car alors on eût manqué d’égards envers les quatre-vingt-dix-sept autres\footnote{Stendhal, {\itshape Rome, Naples. Florence}, 379. Récit d’un seigneur anglais.} » ; on soupirait un instant avec quelques-uns des quatre-vingt-dix-sept, et puis c’était tout. Mme du Deffand, ayant perdu son plus ancien ami, le président Hénault, venait le jour même souper en grande compagnie : « Hélas ! disait-elle, il est mort ce soir à six heures ; sans cela, vous ne me verriez pas ici. » Sous ce régime continu de distractions et d’amusements, il n’y a plus de sentiments profonds ; on n’en a que d’épiderme ; l’amour lui-même se réduit à « l’échange de deux fantaisies »  Et, comme on tombe toujours du côté où l’on penche, la légèreté devient une élégance et un parti pris\footnote{ \noindent Marivaux, {\itshape le Petit-maître corrigé. —} Gresset, {\itshape le Méchant. —} Crébillon fils, {\itshape la Nuit et le Moment} (notamment la scène de Clitandre avec Lucinde). — Collé, {\itshape la Vérité dans le vin} (rôle de l’abbé avec la présidente). — Besenval, 79 (Le comte de Frise et Mme de Blot). — {\itshape Vie privée du maréchal de Richelieu} (scènes avec Mme Michelin). — E. et J. de Goncourt, 167 à 174.
 }. L’indifférence du cœur est à la mode ; on aurait honte d’être vraiment ému. On se pique de jouer avec l’amour, de traiter une femme comme une poupée mécanique, de toucher en elle un ressort, puis l’autre, pour en faire sortir à volonté l’attendrissement ou la colère. Quoi qu’elle fasse, on ne se départ jamais avec elle de la politesse la plus insultante, et l’exagération même des respects faux qu’on lui prodigue est une ironie par laquelle on achève de lui montrer son détachement  On va plus loin, et, dans les âmes foncièrement sèches, la galanterie tourne à la méchanceté. Par ennui et besoin d’excitation, par vanité et pour se prouver sa dextérité, on se plaît à tourmenter, à faire pleurer, à déshonorer, à tuer longuement. À la fin, comme l’amour-propre est un gouffre sans fond, il n’y a pas de « noirceurs » dont ces bourreaux polis ne soient capables, et les personnages de Laclos ont eu leurs originaux\footnote{Laclos, {\itshape les Liaisons dangereuses.} Mme de Merteuil était copiée d’après une marquise de Grenoble. — Notez les différences entre Lovelace et Valmont, l’un qui est conduit par l’orgueil, l’autre qui n’a que de la vanité.}  Sans doute, ces monstres sont rares ; mais l’on n’a pas besoin d’avoir affaire à eux pour démêler ce que la galanterie du monde renferme d’égoïsme. Les femmes qui l’ont érigée en obligation sont les premières à en sentir le mensonge, et à regretter, parmi tant de froids hommages, la chaleur communicative d’un sentiment fort. Le caractère du siècle reçoit alors son trait final, et « l’homme sensible » apparaît.

\section[{II. Trait final qui achève la physionomie du siècle, la sensibilité de salon. — Date de son avènement. — Ses symptômes dans l’art et la littérature. — Son ascendant dans la vie privée. — Ses affectations. — Sa sincérité. — Sa délicatesse.}]{II. Trait final qui achève la physionomie du siècle, la sensibilité de salon. — Date de son avènement. — Ses symptômes dans l’art et la littérature. — Son ascendant dans la vie privée. — Ses affectations. — Sa sincérité. — Sa délicatesse.}

\noindent Ce n’est pas que le fond des mœurs devienne différent ; elles restent aussi mondaines, aussi dissipées jusqu’au bout. Mais la mode autorise une affectation nouvelle, des effusions, des rêveries, des attendrissements qu’on n’avait point encore connus. Il s’agit de revenir à la nature, d’admirer la campagne, d’aimer la simplicité des mœurs rustiques, de s’intéresser aux villageois, d’être humain, d’avoir un cœur, de goûter les douceurs et les tendresses des affections naturelles, d’être époux et père, bien plus d’avoir une âme, des vertus, des émotions religieuses, de croire à la providence et à l’immortalité, d’être capable d’enthousiasme. On veut être ainsi, ou du moins on a la velléité d’être ainsi. En tout cas, si on le veut, c’est à la condition sous-entendue qu’on ne sera pas trop dérangé de son train ordinaire et que les sensations de cette nouvelle vie n’ôteront rien aux jouissances de l’ancienne. Aussi l’exaltation qui commence ne sera guère qu’une ébullition de la cervelle, et l’idylle presque entière se jouera dans les salons  Voici donc la littérature, le théâtre, la peinture et tous les arts qui entrent dans la voie sentimentale pour fournir à l’imagination échauffée une pâture factice\footnote{L’avènement de la sensibilité est marqué par les dates suivantes : Rousseau, {\itshape Sur l’influence des lettres et des arts}, 1749 ; {\itshape Sur l’inégalité}, 1753 ; {\itshape Nouvelle Héloïse}, 1759. — Greuze, {\itshape le Père de famille lisant la Bible}, 1755 ; {\itshape l’Accordée de village}, 1761. — Diderot, \href{http://gallica.bnf.fr/ark:/12148/bpt6k881731}{\dotuline{{\itshape le Fils naturel}}} [\url{http://gallica.bnf.fr/ark:/12148/bpt6k881731}], 1757 ; \href{http://gallica.bnf.fr/ark:/12148/bpt6k88175q}{\dotuline{{\itshape le Père de famille}}} [\url{http://gallica.bnf.fr/ark:/12148/bpt6k88175q}], 1758.}. Rousseau prêche en périodes travaillées le charme de la vie sauvage, et les petits-maîtres, entre deux madrigaux, rêvent au bonheur de coucher nus dans la forêt vierge. Les amoureux de {\itshape la Nouvelle Héloïse} échangent, pendant quatre volumes, des morceaux de style, et là-dessus une personne, « non seulement mesurée, mais compassée », la comtesse de Blot, dans un cercle chez la duchesse de Chartres, s’écrie « qu’à moins d’une vertu supérieure une femme vraiment sensible ne pourrait rien refuser à la passion de Rousseau\footnote{Mme de Genlis, {\itshape Mémoires}, chap. XVII. — G. Sand, I, 72. La jeune Mme de Francueil, voyant Rousseau pour la première fois, fond en larmes.} ». On s’étouffe au Salon autour de {\itshape l’Accordée de village}, de {\itshape la Cruche cassée}, du {\itshape Retour de nourrice}, et autres idylles rustiques et domestiques de Greuze ; la pointe de volupté, l’arrière-fond de sensualité provocante qu’il laisse percer dans la naïveté fragile de ses ingénues est une friandise pour les goûts libertins qui durent sous les aspirations morales\footnote{Ce point a été développé avec autant de finesse que de justesse par MM. de Goncourt ({\itshape l’Art au dix-huitième siècle}, I, 433-438).}. Après eux, Ducis, Thomas, Parny, Colardeau, Roucher, Delille, Bernardin de Saint-Pierre, Marmontel, Florian, tout le troupeau des orateurs, des écrivains et des politiques, le misanthrope Chamfort, le raisonneur Laharpe, le ministre Necker, les faiseurs de petits vers, les imitateurs de Gessner et de Young, les Berquin, les Bitaubé, tous bien peignés, bien attifés, un mouchoir brodé dans la main pour essuyer leurs larmes, vont conduire l’églogue universelle jusqu’au plus fort de la Révolution. En tête du {\itshape Mercure} de 1791 et 1792 paraissent des {\itshape Contes moraux} de Marmontel\footnote{Numéro d’août 1792 : « les Rivaux d’eux-mêmes »  Autres pièces insérées vers le même temps dans {\itshape le Mercure} : « Pacte fédératif entre l’hymen et l’amour, le Jaloux, Romance pastorale, Ode anacréontique à Mlle S. D., etc. »}, et le numéro qui suit les massacres de septembre s’ouvre par des vers « aux mânes de mon serin ».\par
Par suite, dans tous les détails de la vie privée, la sensibilité étale son emphase. On bâtit dans son parc un petit temple à l’Amitié. On dresse dans son cabinet un petit autel à la Bienfaisance\footnote{Mme de Genlis, {\itshape Adèle et Théodore.} I, 312   E. et J. de Goncourt, {\itshape la Femme au dix-huitième siècle}, 318   Mme d’Oberkirch, I, 56. — Description du pouf au sentiment de la duchesse de Chartres (E. et J. de Goncourt, 311) : « Au fond est une femme assise dans un fauteuil et tenant un nourrisson, ce qui représente M. le duc de Valois et sa nourrice ; à droite on voit un perroquet becquetant une cerise, à gauche un petit nègre, les deux bêtes d’affection de la duchesse : le tout est entremêlé de mèches de cheveux de tous les parents de Mme de Chartres, cheveux de son mari, cheveux de son père, cheveux de son beau-père. »}. On porte des robes à la Jean-Jacques Rousseau « analogues aux principes de cet auteur ». On choisit pour coiffure « des poufs au sentiment », dans lesquels on place le portrait de sa fille, de sa mère, de son serin, de son chien, tout cela garni des cheveux de son père ou d’un ami de cœur ». On a des amies de cœur pour qui « on éprouve quelque chose de si vif et de si tendre que véritablement c’est de la passion », et qu’on ne peut se passer de voir trois fois par jour. « Toutes les fois que des amies se disent des choses {\itshape sensibles}, elles doivent subitement prendre une petite voix claire et traînante, se regarder tendrement en penchant la tête, et s’embrasser souvent », sauf à bâiller tout bas au bout d’un quart d’heure et à s’endormir de concert parce qu’elles n’ont plus rien à se dire. L’enthousiasme est d’obligation. À la reprise du {\itshape Père de famille}, l’on compte autant de mouchoirs que de spectateurs, et des femmes s’évanouissent. « Il est d’usage, surtout pour les jeunes femmes, de s’émouvoir, de pâlir, de s’attendrir, et même en général de se trouver mal en apercevant M. de Voltaire ; on se précipite dans ses bras, on balbutie, on pleure, on est dans un trouble qui ressemble à l’amour le plus passionné\footnote{Mme de Genlis, {\itshape les Dangers du monde}, I, scène VII ; II, scène IV   {\itshape Adèle et Théodore}, I, 312   {\itshape Souvenirs de Félicie}, 199  Bachaumont, IV, 320.}. » — Quand un auteur de société vient lire sa pièce dans un salon, la mode veut qu’on s’exclame, qu’on sanglote, et qu’il y ait quelque belle évanouie à délacer. Mme de Genlis, qui raille ces affectations, n’est pas moins affectée que les autres. Tout à coup, au milieu d’une compagnie, on l’entend dire à la jeune orpheline qu’elle exhibe : « Paméla, faites Héloïse ! » Et Paméla, défaisant ses cheveux, s’agenouille, les yeux au ciel, d’un air inspiré, aux applaudissements de l’assistance\footnote{Mme de La Rochejaquelein, {\itshape Mémoires.}}. — La sensibilité devient une institution. La même Mme de Genlis fonde l’ordre de la Persévérance, qui compte bientôt « jusqu’à quatre-vingt-dix chevaliers du plus grand monde ». Pour y être admis, il faut deviner une énigme, répondre à une question morale, faire un discours sur une vertu. Toute dame ou chevalier qui découvre et vient annoncer « trois actions vertueuses bien constatées », reçoit une médaille d’or. Chaque chevalier a son « frère d’armes », chaque dame a son amie, chaque membre a sa devise, et chaque devise, encadrée dans un petit tableau, va figurer dans « le Temple de l’Honneur », sorte de tente très galamment décorée et que M. de Lauzun a fait dresser au milieu d’un jardin\footnote{Mme de Genlis, {\itshape Mémoires}, chap. XX  Duc de Lauzun, 270.}. — La parade sentimentale est complète, et, jusque dans cette chevalerie restaurée, on retrouve une mascarade de salon.\par
Néanmoins la mousse de l’enthousiasme et des grands mots laisse au fond des cœurs un résidu de bonté active, de bienveillance confiante, et même de bonheur, à tout le moins d’expansion et de facilité. Pour la première fois, on voit des femmes accompagner leur mari en garnison ; des mères veulent nourrir, des pères s’intéressent à l’éducation de leurs enfants. La simplicité rentre dans les manières. On ne met plus de poudre aux petits garçons ; nombre de seigneurs quittent les galons, puis les broderies, les talons rouges et l’épée, sauf lorsqu’ils sont en grand habit. On en rencontre dans les rues « vêtus à la Franklin, en gros drap, avec un bâton noueux et des souliers épais\footnote{Mme d’Oberkirch, II, 35 (1783-1784). — Mme Campan, III, 371. — Mercier, {\itshape Tableau de Paris}, passim.} ». Le goût n’est plus aux cascades, aux statues, aux décorations raides et pompeuses ; on n’aime que les jardins anglais. La reine s’arrange un village à Trianon, où, « vêtue d’une robe de percale blanche et d’un fichu de gaze, coiffée d’un chapeau de paille », elle pêche dans le lac et voit traire ses vaches. L’étiquette tombe par lambeaux, comme un fard qui s’écaille, et laisse reparaître la vive couleur des émotions naturelles. Madame Adélaïde prend un violon et remplace le ménétrier absent pour faire danser des paysannes\footnote{{\itshape Correspondance}, par Metra, XVII, 55 (1784). — Mme d’Oberkirch, II, 234. — {\itshape Marie-Antoinette}, par Arneth et Geffroy, II, 29, 63.}. La duchesse de Bourbon sort le matin incognito pour faire l’aumône et « chercher des pauvres dans leurs greniers ». La Dauphine se jette à bas de son carrosse pour secourir un postillon blessé, un paysan que le cerf a renversé. Le roi et le comte d’Artois aident un charretier embourbé à dégager sa charrette. On ne songe plus à se composer et à se contraindre, à garder sa dignité en toute circonstance, à soumettre les faiblesses de la nature aux exigences du rang. À la mort du premier Dauphin\footnote{{\itshape Le Gouvernement de Normandie}, par Hippeau, IV, 387 (Lettres du 4 juin 1789, par un témoin oculaire).}, pendant que les gens de la chambre se jettent au-devant du roi pour l’empêcher d’entrer, la reine se précipite à genoux contre ses genoux, et lui crie en pleurant : « Ah ! ma femme, notre cher enfant est mort puisqu’on ne veut pas que je le voie ». Et le narrateur ajoute avec admiration : « Il me semble toujours voir un bon cultivateur et son excellente compagne en proie au plus affreux désespoir de la perte de leur fils chéri ». On ne cache plus ses larmes, on tient à honneur d’être homme ; on est humain, on se familiarise avec ses inférieurs. Un prince, passant une revue, dit aux soldats en leur présentant la princesse : « Mes enfants, voici ma femme ». On voudrait rendre les hommes heureux et jouir délicieusement de leur reconnaissance. Être bon, être aimé, voilà l’objet d’un chef d’État, d’un homme en place  Cela va si loin qu’on se figure Dieu sur ce modèle. On interprète « les harmonies de la Nature » comme des attentions délicates de la Providence ; en instituant l’amour filial, le Créateur a « daigné nous choisir pour première vertu notre plus doux plaisir\footnote{Florian, {\itshape Ruth.}} »  À l’idylle qu’on imagine au ciel, correspond l’idylle qu’on pratique sur la terre. Du public aux princes, et des princes au public, en prose, en vers, par les compliments de fête, par les réponses officielles, depuis le style des édits royaux jusqu’aux chansons des dames de la halle, c’est un échange continuel de grâces et de tendresses. Des applaudissements éclatent au théâtre lorsqu’un vers fait allusion à la vertu des princes, et, un instant après, quand une tirade exalte les mérites du peuple, les princes prennent leur revanche de politesse en applaudissant à leur tour\footnote{ \noindent Hippeau, IV, 86 (23 juin 1773), représentation du {\itshape Siège de Calais à} la Comédie-Française : « Au moment où Mlle Vestris a prononcé ces vers :\par
\begin{poem}
Le Français dans son prince aime à trouver un frère, \\
Qui, né fils de l’État, en devienne le père. \\?
 \noindent De longs et unanimes applaudissements ont accueilli l’actrice, qui s’était tournée vers M. le Dauphin. Dans un autre endroit se trouvaient ces vers :\par
\* {\itshape Quelle leçon pour vous, superbes potentats} ! \\
Veillez sur vos sujets dans le rang le plus bas \\
Tel, loin de vos regards. dans la misère expire. \\
Qui, quelque jour peut-être, eût sauvé votre empire. \\-
\end{poem}
 \noindent M. le Dauphin et Mme la Dauphine ont pris leur revanche et vivement applaudi la tirade. Cette marque de sensibilité de leur part a été accueillie par de nouveaux transports de tendresse et de reconnaissance. »
 }  De toutes parts, au moment où ce monde finit, une complaisance mutuelle, une douceur affectueuse vient, comme un souffle tiède et moite d’automne, fondre ce qu’il y avait encore de dureté dans sa sécheresse, et envelopper dans un parfum de roses mourantes les élégances de ses derniers instants. On rencontre alors des actions, des mots d’une grâce suprême, uniques en leur genre, comme une mignonne et adorable figurine de vieux Sèvres. Un jour que la comtesse Amélie de Boufflers parlait un peu légèrement de son mari, sa belle-mère lui dit : « Vous oubliez que vous parlez de mon fils  Il est vrai, maman, je croyais ne parler que de votre gendre ». C’est elle encore qui, au jeu du bateau, obligée de choisir entre cette belle-mère bien-aimée et sa mère qu’elle connaissait à peine, répondit : « Je sauverais ma mère et je me noierais avec ma belle-mère\footnote{Mme de Genlis, {\itshape Souvenirs de Félicie}, 76, 161.} ». La duchesse de Choiseul, d’autres encore, sont des miniatures aussi exquises. Quand le cour et l’esprit réunissent leurs délicatesses, ils font des chefs-d’œuvre, et ceux-ci, comme l’art, comme la politesse, comme la société qui les entoure, ont un charme que rien ne surpasse, si ce n’est leur fragilité.

\section[{III. Insuffisance du caractère ainsi formé. — Adapté à une situation, il n’est pas préparé pour la situation contraire. — Lacunes dans l’intelligence. — Lacunes dans la volonté. — Ce caractère est désarmé par le savoir-vivre.}]{III. Insuffisance du caractère ainsi formé. — Adapté à une situation, il n’est pas préparé pour la situation contraire. — Lacunes dans l’intelligence. — Lacunes dans la volonté. — Ce caractère est désarmé par le savoir-vivre.}

\noindent C’est que, plus les hommes se sont adaptés à une situation, moins ils sont préparés pour la situation contraire. Les habitudes et les facultés qui leur servaient dans l’état ancien leur nuisent dans l’état nouveau. En acquérant les talents qui conviennent aux temps de calme, ils ont perdu ceux qui conviennent aux temps de trouble, et ils atteignent l’extrême faiblesse en même temps que l’extrême urbanité. Plus une aristocratie se polit, plus elle se désarme, et, quand il ne lui manque plus aucun attrait pour plaire, il ne lui reste plus aucune force pour lutter  Et cependant, dans ce monde, on est tenu de lutter si l’on veut vivre. L’empire est à la force dans l’humanité comme dans la nature. Toute créature qui perd l’art et l’énergie de se défendre devient une proie d’autant plus sûre que son éclat, son imprudence et même sa gentillesse la livrent d’avance aux rudes appétits qui rôdent à l’entour. Où trouver la résistance dans un caractère formé par les mœurs qu’on vient de décrire   Avant tout, pour se défendre, il faut regarder autour de soi, voir et prévoir, se munir contre le danger. Comment le pourraient-ils, vivant comme ils font ? Leur cercle est trop étroit et trop soigneusement clos. Enfermés dans leurs châteaux et leurs hôtels, ils n’y voient que les gens de leur monde, ils n’entendent que l’écho de leurs propres idées, ils n’imaginent rien au-delà ; deux cents personnes leur semblent le public  D’ailleurs, dans un salon, les vérités désagréables ne sont point admises, surtout quand elles sont personnelles, et une chimère y devient un dogme parce qu’elle y devient une convention. Les voilà donc qui, déjà abusés par l’étroitesse de leur horizon ordinaire, fortifient encore leur illusion par l’illusion de leurs pareils. Ils ne comprennent rien au vaste monde qui enveloppe leur petit monde ; ils sont incapables d’entrer dans les sentiments d’un bourgeois, d’un villageois ; ils se figurent le paysan, non pas tel qu’il est, mais tel qu’ils voudraient le voir. L’idylle étant à la mode, nul n’ose y contredire ; toute autre supposition est fausse parce qu’elle serait pénible, et, les salons ayant décidé que tout ira bien, tout ira bien. — Jamais aveuglement ne fut plus complet et plus volontaire. Le duc d’Orléans offrait de parier cent louis que les États généraux s’en iraient sans avoir rien fait, sans avoir même aboli les lettres de cachet. Quand la démolition sera commencée, bien mieux, quand elle sera faite, ils ne jugeront pas plus juste. Ils n’ont aucune notion de l’architecture sociale ; ils n’en connaissent ni les matériaux, ni les proportions, ni l’équilibre ; ils n’y ont jamais mis la main, ils n’ont point de pratique. Ils ignorent la structure de la vieille fabrique\footnote{M. de Montlosier, à l’Assemblée constituante, est presque le seul qui sache le droit féodal.} dont ils occupent le premier étage. Ils n’en savent calculer ni les poussées, ni les résistances\footnote{« L’homme instruit et impartial qui soumettrait au calcul les probabilités du succès de la Révolution trouverait qu’il y avait plus de chances contre elle que contre le quine à la loterie ; mais le quine est possible et malheureusement cette fois il fut gagné. » (Duc de Lévis, {\itshape Souvenirs.} 328.)}. Ils finissent par s’imaginer que le mieux est de laisser l’écroulement s’achever, que l’édifice se reconstruira pour eux de lui-même, qu’ils vont rentrer dans leur salon rebâti exprès et redoré à neuf, pour y recommencer l’aimable causerie qu’un accident, un tumulte de rue vient d’interrompre\footnote{{\itshape Corinne}, par Mme de Staël : Caractère du comte d’Erfeuil. — {\itshape Mémoires} de Malouet, II, 297. (Exemple mémorable de niaiserie politique.)}. Si clairvoyants dans le monde, leurs yeux sont obtus en politique. Ils démêlent tout à la lumière artificielle des bougies ; ils se troublent et s’éblouissent à la clarté naturelle du grand jour. C’est que le pli est trop ancien et trop fort. L’organe, appliqué si longtemps sur les minces détails de la vie élégante, n’embrasse plus les grandes masses de la vie populaire, et, dans le milieu nouveau où subitement il est plongé, sa finesse fait son aveuglement.\par
Il faut agir cependant, car le danger est là qui les prend à la gorge. Mais c’est un danger d’espèce ignoble, et, contre ses prises, leur éducation ne leur fournit pas les armes appropriées. Ils ont appris l’escrime, et non la savate. Ils sont toujours les fils de ceux qui, à Fontenoy, au lieu de tirer les premiers, mettaient le chapeau à la main, et, courtoisement, disaient aux Anglais : « Non, Messieurs, tirez vous-mêmes ». Assujettis, aux bienséances, ils sont gênés dans leurs mouvements. Nombre d’actions et des plus nécessaires, toutes celles qui sont brusques, fortes et crues, sont contraires aux égards qu’un homme bien élevé doit aux autres, ou du moins aux égards qu’il se doit à lui-même  Ils ne se les permettent pas ; ils ne songent pas à se les permettre, et, plus ils sont haut placés, plus ils sont bridés par leur rang. Quand la famille royale part pour Varennes, les retards accumulés qui la perdent sont un effet de l’étiquette. Mme de Tourzel a réclamé sa place dans la voiture, et elle y avait droit, comme gouvernante des Enfants de France. Le roi voulait, en arrivant, donner à M. de Bouillé le bâton de maréchal, et, pour avoir un bâton, il a dû, après diverses allées et venues, emprunter celui du duc de Choiseul. La reine ne pouvait se passer d’un nécessaire de voyage, et il a fallu en fabriquer un énorme qui contient tous les meubles imaginables, depuis une bassinoire jusqu’à une écuelle d’argent ; outre cela, d’autres caisses et, comme s’il n’y avait pas de chemises à Bruxelles, un trousseau complet pour elle et ses enfants\footnote{Mme Campan, II, 140, 313. — Duc de Choiseul, {\itshape Mémoires.}}  La dévotion étroite, l’humanité {\itshape quand même}, la frivolité du petit esprit littéraire, l’urbanité gracieuse, l’ignorance foncière\footnote{{\itshape Journal} de Dumont d’Urville, commandant du navire qui transportait Charles X en 1830  Voir note 4 [’p. 306’].}, la nullité ou la rigidité de l’intelligence et de la volonté sont encore plus grandes chez les princes que chez les nobles  Contre l’émeute sauvage et grondante, tous sont impuissants. Ils n’ont pas l’ascendant physique qui la maîtrise, le charlatanisme grossier qui la charme, les tours de Scapin qui la dépistent, le front de taureau, les gestes de bateleur, le gosier de stentor, bref les ressources du tempérament énergique et de la ruse animale, seules capables de détourner la fureur de la bête déchaînée. Pour trouver de ces lutteurs, ils font chercher trois ou quatre hommes de race ou d’éducation différente, tous ayant roulé et pâti, un plébéien brutal comme l’abbé Maury, un satyre colossal et fangeux comme Mirabeau, un aventurier audacieux et prompt comme ce Dumouriez qui, à Cherbourg, lorsque la faiblesse du duc de Beuvron a livré les blés et lâché l’émeute, lui-même hué et sur le point d’être mis en pièces, aperçoit tout d’un coup les clés du magasin dans les mains d’un matelot hollandais, crie au peuple qu’on le trahit et qu’un étranger lui a pris ses clés, saute à bas du perron, saisit le matelot à la gorge, arrache les clés et les remet à l’officier de garde en disant au peuple : « Je suis votre père, c’est moi qui vous réponds des magasins\footnote{ Dumouriez, {\itshape Mémoires}, III, chap. III (21 juillet 1789).
 } ». Se commettre avec des crocheteurs et des harengères, se colleter au club, improviser dans les carrefours, aboyer plus haut que les aboyeurs, travailler de ses poings et de son gourdin, comme plus tard la jeunesse dorée, sur les fous et les brutes qui n’emploient pas d’autres arguments et auxquels il faut répondre par des arguments de même nature, monter la garde autour de l’Assemblée, se faire constable volontaire, n’épargner ni sa peau ni la peau d’autrui, être peuple en face du peuple, voilà des procédés efficaces et simples, mais dont la grossièreté leur semble dégoûtante. Il ne leur vient pas à l’idée d’y avoir recours ; ils ne savent ni ne veulent se servir de leurs mains, surtout pour cette besogne\footnote{« Toutes ces belles dames et ces beaux messieurs qui savaient si bien marcher sur les tapis et faire la révérence ne savaient pas faire trois pas sur la terre du bon Dieu sans être accablés de fatigue. Ils ne savaient pas même ouvrir ou fermer une porte ; ils n’avaient pas la force de soulever une bûche pour la mettre dans le feu : il leur fallait des domestiques pour leur avancer un fauteuil ; ils ne pouvaient pas entrer et sortir tout seuls. Qu’auraient-ils fait de leurs grâces, sans leurs valets pour leur tenir lieu de mains et de jambes ? » (G. Sand. V, 61.)}. Elles ne sont exercées qu’au duel, et, presque tout de suite, la brutalité de l’opinion va, par des voies de fait, barrer le chemin aux combats polis. Contre le taureau populaire, leurs armes sont des traits de salon, épigrammes, bons mots, chansons, parodies et autres piqûres d’épingle\footnote{« Quand Mme de F. a dit joliment une chose bien pensée, elle croit avoir tout fait. M… disait que, quand elle a dit une jolie chose sur l’émétique, elle est toute surprise de n’être pas purgée. » (Chamfort, 107)}. Le fonds et la ressource manquent à ce caractère ; à force de s’affiner, il s’est étiolé, et la nature, appauvrie par la culture, est incapable des transformations par lesquelles on se renouvelle et on se survit  L’éducation toute-puissante a réprimé, adouci, exténué l’instinct lui-même. Devant la mort présente, ils n’ont pas le soubresaut de sang et de colère, le redressement universel et subit de toutes les puissances, l’accès meurtrier, le besoin irrésistible et aveugle de frapper qui les frappe. Jamais on ne verra un gentilhomme arrêté chez lui casser la tête du jacobin qui l’arrête\footnote{ \noindent Exemple de ce qu’aurait pu faire la résistance armée de chacun chez soi et pour soi. Un gentilhomme de Marseille, retiré dans sa bastide et proscrit, se munit d’un fusil, d’une paire de pistolets et d’un sabre, ne sortit plus sans cet attirail, et déclara qu’on ne l’aurait point vivant. Personne n’osa exécuter le mandat d’arrêt. (Anne Plumptree, {\itshape a Residence of three years in Fronce} 1802-1805), II, 115.
 }. Ils se laisseront prendre, ils iront docilement en prison ; faire du tapage serait une marque de mauvais goût, et, avant tout, il s’agit pour eux de rester ce qu’ils sont, gens de bonne compagnie. En prison, hommes et femmes s’habilleront avec soin, se rendront des visites, tiendront salon ; ce sera au fond d’un corridor, entre quatre chandelles ; mais on y badinera, on y fera des madrigaux, on y dira des chansons, on se piquera d’y être aussi galant, aussi gai, aussi gracieux qu’auparavant : faut-il devenir morose et mal appris parce qu’un accident vous loge dans une mauvaise auberge   Devant les juges, sur la charrette, ils garderont leur dignité et leur sourire ; les femmes surtout iront à l’échafaud avec l’aisance et la sérénité qu’elles portaient dans une soirée. Trait suprême du savoir-vivre qui, érigé en devoir unique et devenu pour cette aristocratie une seconde nature, se retrouve dans ses vertus comme dans ses vices, dans ses facultés comme dans ses impuissances, dans sa prospérité comme dans sa chute, et la pare jusque dans la mort où il la conduit.
\chapterclose

\chapterclose


\chapteropen

\part[{Livre troisième. L’esprit et la doctrine.}]{Livre troisième. \\
L’esprit et la doctrine.}
\renewcommand{\leftmark}{Livre troisième. \\
L’esprit et la doctrine.}


\chaptercont

\chapteropen

\chapter[{Chapitre I. Composition de l’esprit révolutionnaire, premier élément, l’acquis scientifique.}]{Chapitre I. \\
Composition de l’esprit révolutionnaire, premier élément, l’acquis scientifique.}


\chaptercont
\noindent Lorsque nous voyons un homme un peu faible de constitution, mais d’apparence saine et d’habitudes paisibles, boire avidement d’une liqueur nouvelle, puis tout d’un coup, tomber à terre, l’écume à la bouche, délirer et se débattre dans les convulsions, nous devinons aisément que dans le breuvage agréable il y avait une substance dangereuse ; mais nous avons besoin d’une analyse délicate pour isoler et décomposer le poison. Il y en a dans la philosophie du dix-huitième siècle, et d’espèce étrange autant que puissante : car, non seulement il est l’œuvre d’une longue élaboration historique, l’extrait définitif et condensé auquel aboutit toute la pensée du siècle ; mais encore ses deux principaux ingrédients ont cela de particulier qu’étant séparés ils sont salutaires et qu’étant combinés ils font un composé vénéneux.\par

\section[{I. Accumulation et progrès des découvertes dans les sciences de la nature. — Elles servent de point de départ aux nouveaux philosophes.}]{I. Accumulation et progrès des découvertes dans les sciences de la nature. — Elles servent de point de départ aux nouveaux philosophes.}

\noindent Le premier est l’acquis scientifique, celui-ci excellent de tous points et bienfaisant par sa nature ; il se compose d’un amas de vérités lentement préparées, puis assemblées tout d’un coup ou coup sur coup. Pour la première fois dans l’histoire, les sciences s’étendent et s’affermissent au point de fournir, non plus comme autrefois, sous Galilée ou Descartes, des fragments de construction ou quelque échafaudage provisoire, mais un système du monde définitif et prouvé : c’est celui de Newton\footnote{{\itshape Philosophies naturalis principia}, 1687 ; {\itshape Optique}, 1704.}. Autour de cette vérité capitale se rangent comme compléments ou prolongements presque toutes les découvertes du siècle : — Dans les mathématiques pures, le calcul de l’infini inventé en même temps par Leibnitz et Newton, la mécanique ramenée par d’Alembert à un seul théorème, et cet ensemble magnifique de théories qui, élaborées par les Bernoulli, par Euler, Clairaut, d’Alembert, Taylor, Maclaurin, s’achèvent à la fin du siècle aux mains de Monge, de Lagrange et de Laplace\footnote{ \noindent Voir sur ce développement, Comte, {\itshape Philosophie positive}, t. I\textsuperscript{er}. — Au commencement du dix-neuvième siècle, le perfectionnement de l’instrument mathématique est si grand, qu’on croit pouvoir soumettre à l’analyse tous les phénomènes physiques, lumière, électricité, son, cristallisation, chaleur, élasticité, cohésion et autres effets des forces moléculaires. — Sur les progrès des sciences physiques, voir Whewell, {\itshape History of the inductive sciences}, t. Il et III.
 }. Dans l’astronomie, la suite des calculs et des observations qui, de Newton à Laplace, transforment la science en un problème de mécanique, expliquent et prédisent tous les mouvements des planètes et de leurs satellites, indiquent l’origine et la formation de notre système solaire, et débordent au-delà par les découvertes d’Herschel, jusqu’à nous faire entrevoir la distribution des archipels stellaires et les grandes lignes de l’architecture des cieux. — Dans la physique, la décomposition du rayon lumineux et les principes de l’optique trouvés par Newton, la vitesse du son, la forme de ses ondulations, et, depuis Sauveur jusqu’à Chladni, depuis Newton jusqu’à Bernoulli et Lagrange, les lois expérimentales et les théorèmes principaux de l’acoustique, les premières lois de la chaleur rayonnante par Newton, Kraft et Lambert, la théorie de la chaleur latente par Black, la mesure du calorique par Lavoisier et Laplace, les premières idées vraies sur l’essence du feu et de la chaleur, les expériences, les lois, les machines par lesquelles Dufay, Nollet, Franklin et surtout Coulomb expliquent, manient et utilisent pour la première fois l’électricité. — En chimie, tous les fondements de la science, l’oxygène, l’azote, l’hydrogène isolés, la composition de l’eau, la théorie de la combustion, la nomenclature chimique, l’analyse quantitative, l’indestructibilité de la matière et du poids, bref les découvertes de Scheele, de Priestley, de Cavendish et de Stahl, couronnées par la théorie et la langue définitives de Lavoisier. — En minéralogie, le goniomètre, la fixité des angles et les premières lois de dérivation par Romé de Lisle, puis la découverte des types et la déduction mathématique des formes secondaires par Haüy. — En géologie, les suites et la vérification de la théorie de Newton, la figure exacte de la terre, l’aplatissement des pôles, le renflement de l’équateur\footnote{ Voyages de La Condamine au Pérou et de Maupertuis en Laponie.
 }, la cause et la loi des marées, la fluidité primitive de la planète, la persistance de la chaleur centrale ; puis, avec Buffon, Desmarets, Hutton, Werner, l’origine aqueuse ou ignée des roches, la stratification des terrains, la structure fossile des couches, le séjour prolongé et répété de la mer sur les continents, le lent dépôt des débris animaux et végétaux, la prodigieuse antiquité de la vie, les dénudations, les cassures, les transformations graduelles du relief terrestre\footnote{Buffon, {\itshape Théorie de la terre}, 1749 ; {\itshape Époques de la nature}, 1788. — {\itshape Carte géologique de l’Auvergne}, par Desmarets, 1766.}, et à la fin le tableau grandiose où Buffon trace en traits approximatifs l’histoire entière de notre globe, depuis le moment où il n’était qu’une masse de lave ardente jusqu’à l’époque où notre espèce, après tant d’autres espèces détruites ou survivantes, a pu l’habiter  Sur cette science de la matière brute, on voit en même temps s’élever la science de la matière organisée. Grew, puis Vaillant viennent de démontrer les sexes et de décrire la fécondation des plantes ; Linné invente la nomenclature botanique et les premières classifications complètes ; les Jussieu découvrent la subordination des caractères et la classification naturelle. La digestion est expliquée par Réaumur et Spallanzani, la respiration par Lavoisier ; Prochaska constate le mécanisme des actions réflexes ; Haller et Spallanzani expérimentent et décrivent les conditions et les phases de la génération. On pénètre dans le bas-fond du règne animal ; Réaumur publie ses admirables mémoires sur les insectes, et Lyonnet emploie vingt ans à figurer la chenille du saule ; Spallanzani ressuscite ses rotifères, Trembley découpe son polype d’eau douce, Needham fait apparaître ses infusoires. De toutes ces recherches se dégage la conception expérimentale de la vie. Déjà Buffon et surtout Lamarck, dans leurs ébauches grandioses et incomplètes, esquissent avec une divination pénétrante les principaux traits de la physiologie et de la zoologie modernes. Des molécules organiques partout répandues ou partout naissantes, des sortes de globules en voie de déperdition et de réparation perpétuelles, qui, par un développement aveugle et spontané, se transforment, se multiplient, s’associent, et qui, sans direction étrangère, sans but préconçu, par le seul effet de leur structure et de leurs alentours, s’ordonnent pour composer ces édifices savants que nous appelons des animaux et des plantes ; à l’origine, les formes les plus simples, puis l’organisation compliquée et perfectionnée lentement et par degrés ; l’organe créé par les habitudes, par le besoin, par le milieu ; l’hérédité transmettant les modifications acquises\footnote{Voir une leçon de M. de La Caze-Duthiers sur Lamarck, {\itshape Revue scientifique}, III, 276-311.} : voilà d’avance, à l’état de conjectures et d’approches, la théorie cellulaire de nos derniers physiologistes\footnote{Buffon, {\itshape Histoire naturelle}, II, 340 : « Tous les êtres vivants contiennent une grande quantité de molécules vivantes et actives. La vie du végétal ou de l’animal ne paraît être que le {\itshape résultat des actions de toutes les petites vies particulières} de chacune des molécules actives dont la vie est primitive. » — Cf. Diderot, {\itshape Rêve de d’Alembert.}} et les conclusions de Darwin. Dans le tableau que l’esprit humain fait de la nature, la science du dix-huitième siècle a dessiné le contour général, l’ordre des plans et les principales masses en traits si justes, qu’aujourd’hui encore toutes les grandes lignes demeurent intactes. Sauf des corrections partielles, nous n’avons rien à effacer.\par
C’est cette vaste provision de vérités certaines ou probables, démontrées ou pressenties, qui a donné à l’esprit du siècle l’aliment, la substance et le ressort. Considérez les chefs de l’opinion publique, les promoteurs de la philosophie nouvelle : à divers degrés, ils sont tous versés dans les sciences physiques et naturelles. Non seulement ils connaissent les théories et les livres, mais encore ils touchent les choses et les faits. Non seulement Voltaire expose, l’un des premiers, l’optique et l’astronomie de Newton\footnote{{\itshape Philosophe de Newton}, 1738, et {\itshape Physique}, par Voltaire. — Cf. Bois-Raymond, {\itshape Voltaire physicien} ({\itshape Revue des cours scientifiques}, V, 539), et Saigey, {\itshape la Physique de Voltaire. —} « Voltaire, écrit lord Brougham, en continuant de s’occuper de physique expérimentale, aurait sans doute inscrit son nom parmi ceux des grands inventeurs de son siècle. »}, mais encore il calcule, il observe et il expérimente lui-même. Il adresse à l’Académie des Sciences des mémoires « sur la mesure de la force motrice », « sur la nature et la propagation de la chaleur ». Il manie le thermomètre de Réaumur, le prisme de Newton, le pyromètre de Muschenbroek, Il a dans son laboratoire de Cirey tous les appareils alors connus de physique et de chimie. Il fait de ses mains des expériences sur la réflexion de la lumière dans le vide, sur l’augmentation du poids dans les métaux calcinés, sur la renaissance des parties coupées dans les animaux, et cela en véritable savant, avec insistance et répétition, jusqu’à couper la tête à quarante escargots et limaces, pour vérifier une assertion de Spallanzani. — Même curiosité et préparation dans tous ceux qui sont imbus du même esprit. Dans l’autre camp, parmi les cartésiens qui vont finir, Fontenelle est un mathématicien excellent, le biographe compétent de tous les savants illustres, le secrétaire autorisé et le véritable représentant de l’Académie des Sciences. — Ailleurs, à l’Académie de Bordeaux, Montesquieu lit des discours sur le mécanisme de l’écho, sur l’usage des glandes rénales ; il dissèque des grenouilles, essaye l’effet du chaud et du froid sur les tissus vivants, publie des observations sur les plantes et sur les insectes. — Rousseau, le moins instruit de tous, suit les cours du chimiste Rouelle, herborise, et s’approprie, pour écrire son {\itshape Émile}, tous les éléments des connaissances humaines. — Diderot a enseigné les mathématiques, dévoré toute science, tout art et jusqu’aux procédés techniques des industries. D’Alembert est au premier rang parmi les mathématiciens. Buffon a traduit la théorie des fluxions de Newton, la statique des végétaux par Hales ; il devient à la fois ou tour à tour métallurgiste, opticien, géographe, géologue et à la fin anatomiste. Condillac, pour expliquer l’usage des signes et la filiation des idées, écrit des abrégés d’arithmétique, d’algèbre, de mécanique et d’astronomie\footnote{Voir sa {\itshape Langue des calculs} et son {\itshape Art de raisonner.}}. Maupertuis, Condorcet et Lalande sont mathématiciens, physiciens, astronomes ; d’Holbach, La Mettrie, Cabanis sont chimistes, naturalistes, physiologistes, médecins. — Grands ou petits prophètes, maîtres ou élèves, savants spéciaux ou simples amateurs, ils puisent tous directement ou indirectement à la source vive qui vient de s’ouvrir. C’est de là qu’ils partent pour enseigner à l’homme ce qu’il est, d’où il vient, où il va, ce qu’il peut devenir, ce qu’il doit être. Or un nouveau point de départ mène à un nouveau point de vue ; c’est pourquoi l’idée qu’on se fait de l’homme va changer du tout au tout.

\section[{II. Changement du point de vue dans la science de l’homme. — Elle se détache de la théologie et se soude comme un prolongement aux sciences de la nature.}]{II. Changement du point de vue dans la science de l’homme. — Elle se détache de la théologie et se soude comme un prolongement aux sciences de la nature.}

\noindent Car supposez un esprit tout pénétré des vérités nouvelles ; mettez-le spectateur sur l’orbite de Saturne et qu’il regarde\footnote{Pour l’exposition populaire de ces idées, voir Voltaire, passim, surtout {\itshape Micromégas} et {\itshape les Oreilles du comte de Chesterfield.}}. Au milieu de ces effroyables espaces et de ces millions d’archipels solaires, quel petit canton que le nôtre et quel grain de sable que la terre ! Quelle multitude de mondes au-delà de nous, et, si la vie s’y rencontre, que de combinaisons possibles autres que celles dont nous sommes l’effet ! Qu’est-ce que la vie, qu’est-ce que la substance organisée, dans ce monstrueux univers, sinon une quantité négligeable, un accident passager, une moisissure de quelques grains de l’épiderme ? Et, si telle est la vie, qu’est-ce que l’humanité qui en est un si mince fragment   Tel est l’homme dans la nature, un atome, un éphémère ; n’oublions pas cela dans les systèmes que nous faisons sur son origine, sur son importance, sur sa destinée. Une mite serait grotesque, si elle se considérait comme le centre des choses, et il ne faut pas « qu’un insecte presque infiniment petit montre un orgueil presque infiniment grand\footnote{Cf. Buffon, {\itshape ibid.}, I, 31 : « Ceux qui croient répondre par les causes finales ne font pas attention qu’ils prennent l’effet pour la cause. Le rapport que les choses ont avec nous n’influant point du tout sur leur origine, la convenance morale ne peut jamais être une raison physique. » — Voltaire, {\itshape Candide} : « Quand Sa Hautesse envoie un vaisseau en Égypte, s’embarrasse-t-elle si les souris qui sont dans le vaisseau sont à leur aise ou non ? »} ». Sur ce globe lui-même, combien son éclosion a été tardive ! Quelles myriades de siècles entre le premier refroidissement et les commencements de la vie\footnote{Buffon, {\itshape ib., supplément}, II, 513 ; {\itshape Époques de la nature}, IV, 65, 167. D’après ses expériences sur le refroidissement d’un boulet, il établit les périodes suivantes. Depuis la fluidité ardente de la planète jusqu’à la chute des eaux vaporisées, trente-cinq mille ans. Depuis le commencement de la vie jusqu’à l’état actuel, quarante mille ans. Depuis l’état actuel jusqu’à la congélation totale et l’extinction de la vie, quatre-vingt-treize mille ans. Au reste, il ne donne ces chiffres que comme des minima. On pense aujourd’hui qu’ils sont beaucoup trop faibles.} ! Qu’est-ce que le tracas de notre fourmilière à côté de cette tragédie minérale à laquelle nous n’avons pas assisté, combats de l’eau et du feu, épaississement de la croûte, formation de l’océan universel, construction et séparation des continents ? Avant notre histoire, quelle longue histoire animale et végétale, quelle succession de flores et de faunes, que de générations d’animaux marins pour former les terrains de sédiment, que de générations de plantes pour former les dépôts de houille, quels changements de climat pour chasser du pôle les grands pachydermes   Enfin voici l’homme, le dernier venu, éclos comme un bourgeon terminal à la cime d’un grand arbre antique, pour y végéter pendant quelques saisons, mais destiné comme l’arbre à périr après quelques saisons, lorsque le refroidissement croissant et prévu qui a permis à l’arbre de vivre forcera l’arbre à mourir. Il n’est pas seul sur la tige : au-dessous de lui, autour de lui, presque à son niveau, sont d’autres bourgeons nés de la même sève ; qu’il n’oublie jamais, s’il veut comprendre son être, de considérer, en même temps que lui-même, les autres vivants ses voisins, échelonnés jusqu’à lui et issus du même tronc. S’il est hors ligne, il n’est pas hors cadre, il est un animal parmi les animaux\footnote{Buffon, {\itshape ib.}, I, 12 : « La première vérité qui sort de cet examen sérieux de la nature est une vérité peut-être humiliante pour l’homme, c’est qu’il doit se ranger lui-même dans la classe des animaux. »} en lui et chez eux, la substance, l’organisation, la naissance, la formation, le renouvellement, les fonctions, les sens, les appétits, sont semblables, et son intelligence supérieure, comme leur intelligence rudimentaire, a pour organe indispensable une matière nerveuse dont la structure est la même chez eux et chez lui. — Ainsi enveloppé, produit, porté par la nature, peut-on supposer qu’il soit dans la nature comme un empire dans un empire ? Il y est comme une partie dans un tout, à titre de corps physique, à titre de composé chimique, à titre de vivant, à titre d’animal sociable, parmi d’autres corps, d’autres composés, d’autres animaux sociables, tous analogues à lui, et, à tous ces titres, il est comme eux soumis à des lois  Car, si nous ignorons le principe de la nature et si nous nous disputons pour savoir ce qu’il est, intérieur ou extérieur, nous constatons avec certitude la manière dont il agit, et il n’agit que selon des lois générales et fixes. Tout événement, quel qu’il soit, a des conditions, et, ces conditions données, il ne manque jamais de suivre. Des deux anneaux qui forment le couple, le premier entraîne toujours après soi le second. Il y a de ces lois pour les nombres, les figures et les mouvements, pour la révolution des planètes et la chute des corps, pour la propagation de la lumière et le rayonnement de la chaleur, pour les attractions et les répulsions de l’électricité, pour les combinaisons chimiques, pour la naissance, l’équilibre et la dissolution du corps organisé. Il y en a pour la naissance, le maintien et le développement des sociétés humaines, pour la formation, le conflit et la direction des idées, des passions et des volontés de l’individu humain\footnote{Voltaire, {\itshape Philosophie, Du principe d’action} : « Que tous les êtres, sans exception, sont soumis à des lois invariables. »}. En tout ceci l’homme continue la nature ; d’où il suit que, pour le connaître, il faut l’observer en elle, après elle, et comme elle, avec la même indépendance, les mêmes précautions et le même esprit  Par cette seule remarque, la méthode des sciences morales est fixée. En histoire, en psychologie, en morale, en politique, les penseurs du siècle précédent, Pascal, Bossuet, Descartes, Fénelon, Malebranche, La Bruyère, partaient encore du dogme ; pour quiconque sait les lire, il est clair que d’avance leur siège était fait. La religion leur fournissait une théorie achevée du monde moral ; d’après cette théorie latente ou expresse, ils décrivaient l’homme et accommodaient leurs observations au type préconçu. Les écrivains du dix-huitième siècle renversent ce procédé : c’est de l’homme qu’ils partent, de l’homme observable et de ses alentours à leurs yeux, les conclusions sur l’âme, sur son origine, sur sa destinée, ne doivent venir qu’ensuite, et dépendent tout entières, non de ce que la révélation, mais de ce que l’observation aura fourni. Les sciences morales se détachent de la théologie et se soudent comme un prolongement aux sciences physiques.

\section[{III. Transformation de l’histoire  Voltaire  La critique et les vues d’ensemble  Montesquieu  Aperçu des lois sociales.}]{III. Transformation de l’histoire  Voltaire  La critique et les vues d’ensemble  Montesquieu  Aperçu des lois sociales.}

\noindent Par ce déplacement et par cette soudure, elles deviennent des sciences. En histoire, tous les fondements sur lesquels nous construisons aujourd’hui sont posés. Que l’on compare le {\itshape Discours} de Bossuet {\itshape sur l’Histoire universelle}, et l’{\itshape Essai} de Voltaire {\itshape sur les mœurs}, on verra tout de suite combien ces fondements sont nouveaux et profonds  Du premier coup, la critique a trouvé son principe : considérant que les lois de la nature sont universelles et immuables, elle en conclut que, dans le monde moral, comme dans le monde physique, rien n’y déroge, et que nulle intervention arbitraire et étrangère ne vient déranger le cours régulier des choses, ce qui donne un moyen sûr de discerner le mythe de la vérité\footnote{{\itshape Essai sur les mœurs}, chap. CXLVII, résumé. « Un lecteur sage s’apercevra aisément qu’il ne doit croire que les grands événements qui ont quelque vraisemblance, et regarder en pitié toutes les fables dont le fanatisme, l’esprit romanesque, la crédulité ont chargé dans tous les temps la scène du monde. »}. De cette maxime naît l’exégèse biblique, non seulement celle que fait Voltaire, mais encore celle qu’on fera plus tard. En attendant, il court en sceptique à travers les annales de tous les peuples, tranche et retranche légèrement, trop vite, avec excès, surtout lorsqu’il s’agit des anciens, parce que son expédition historique n’est qu’un voyage de reconnaissance, mais avec un coup d’œil si juste que, de sa carte sommaire, nous pouvons garder presque tous les contours. L’homme primitif ne fut point un être supérieur, éclairé d’en haut, mais un sauvage grossier, nu, misérable, lent dans sa croissance, tardif dans son progrès, le plus dépourvu et le plus nécessiteux de tous les animaux, à cause de cela sociable, né comme l’abeille et le castor avec l’instinct de vivre en troupe, outre cela imitateur comme le singe, mais plus intelligent, capable de passer par degrés du langage des gestes au langage articulé, ayant commencé par un idiome de monosyllabes, qui peu à peu s’est enrichi, précisé et nuancé\footnote{\href{http://www.voltaire-integral.com/Html/22/12\_Metaphysique.html\#CHAPITRE\%20I.}{\dotuline{{\itshape Traité de métaphysique}, chap. I}} [\url{http://www.voltaire-integral.com/Html/22/12\_Metaphysique.html\#CHAPITRE\%20I.}]. « Descendu sur ce petit amas de boue et n’ayant pas plus de notion de l’homme que l’homme n’en a des habitants de Mars et de Jupiter, je débarque sur les côtes de l’océan dans le pays de la Cafrerie, et d’abord je me mets à chercher un homme. Je vois des singes, des éléphants et des nègres qui me semblent tous avoir quelque lueur d’une raison imparfaite, etc. » — On voit ici très nettement et en exercice la méthode nouvelle.}. Que de siècles pour atteindre à ce premier langage ! Combien d’autres siècles ensuite pour l’invention des arts les plus nécessaires, pour l’usage du feu, la fabrication des « haches de silex et de jade », la fonte et l’affinage des métaux, la domestication des animaux, l’élevage et l’amélioration des plantes comestibles, pour l’établissement des premières sociétés policées et durables, pour la découverte de l’écriture, des chiffres, des périodes astronomiques\footnote{\href{http://www.voltaire-integral.com/Html/11/04INT\_10.html\#i7}{\dotuline{{\itshape Introduction à l’Essai sur les mœurs : Des sauvages}}} [\url{http://www.voltaire-integral.com/Html/11/04INT\_10.html\#i7}]. — Buffon, {\itshape Époques de la nature}, septième époque. Sur l’amélioration des espèces utiles, il énonce d’avance les idées de Darwin.} ! Alors seulement, après un crépuscule d’une longueur indéfinie et énorme, on voit en Chaldée, en Chine, commencer l’histoire certaine et datée. Il y a cinq ou six de ces grands centres de civilisation spontanée, la Chine, Babylone, l’ancienne Perse, l’Inde, l’Égypte, la Phénicie, les deux empires d’Amérique. Ramassons leurs débris, lisons ceux de leurs livres qui ont subsisté et que les voyageurs nous rapportent, les cinq Kings des Chinois, les Védas des Hindous, le Zend-Avesta des anciens Perses, et nous y trouverons des religions, des morales, des philosophies, des institutions aussi dignes d’attention que les nôtres. Encore aujourd’hui trois de ces Codes, ceux de l’Inde, de la Chine et des musulmans, gouvernent des contrées aussi vastes que notre Europe et des peuples qui nous valent bien. N’allons pas, comme Bossuet, « oublier l’univers dans une histoire universelle », et subordonner le genre humain à un petit peuple confiné dans un canton pierreux auprès de la mer Morte\footnote{{\itshape Remarques de l’Essai sur les mœurs} : « On peut parler de ce peuple en théologie, mais il mérite peu de place en histoire. » — {\itshape Entretiens entre A, B, C}, 7{\itshape \textsuperscript{e} entretien.}}. L’histoire humaine est chose naturelle comme le reste ; sa direction lui vient de ses éléments ; il n’y a point de force extérieure qui la mène, mais des forces intérieures qui la font ; elle ne va pas vers un but, elle aboutit à un effet. Et cet effet principal est le progrès de l’esprit humain. « Au milieu de tant de saccagements et de destruction, nous voyons un amour de l’ordre qui anime en secret le genre humain et qui a prévenu sa ruine totale. C’est un des ressorts de la nature qui reprend toujours sa force ; c’est lui qui a formé le code des nations, c’est par lui qu’on révère la loi et les ministres de la loi dans le Tunquin et dans l’île Formose comme à Rome. » Ainsi il y a dans l’homme « un principe de raison », c’est-à-dire un « instinct de mécanique » qui lui suggère les idées utiles\footnote{Franklin définissait l’homme : « un animal qui fait des outils ».}, et un instinct de justice qui lui suggère les idées morales. Ces deux instincts font partie de sa constitution ; il les a de naissance, « comme les oiseaux ont leurs plumes, et comme les ours ont leur fourrure ». C’est pourquoi il est perfectible par nature et ne fait que se conformer à la nature lorsqu’il améliore son esprit et sa condition. Le sauvage, « le Brasilien est un animal qui n’a pas encore atteint le complément de son espèce ; c’est une chenille enfermée dans sa fève et qui ne sera papillon que dans quelques siècles ». Poussez plus loin cette idée avec Turgot et Condorcet\footnote{Condorcet, \href{http://classiques.uqac.ca/classiques/condorcet/esquisse\_tableau\_progres\_hum/esquisse.html}{\dotuline{{\itshape Esquisse d’un tableau historique des progrès de l’esprit humain}}} [\url{http://classiques.uqac.ca/classiques/condorcet/esquisse\_tableau\_progres\_hum/esquisse.html}].}, et, à travers des exagérations, vous verrez naître, avant la fin du siècle, notre théorie moderne du progrès, celle qui fonde toutes nos espérances sur l’avancement indéfini des sciences, sur l’accroissement du bien-être que leurs découvertes appliquées apportent incessamment dans la condition humaine, et, sur l’accroissement du bon sens que leurs découvertes vulgarisées déposent lentement dans l’esprit humain.\par
Reste un second principe à poser pour achever la fondation de l’histoire. Découvert par Montesquieu, aujourd’hui encore il nous sert d’appui pour construire, et, si nous devons reprendre en sous-œuvre l’édifice du maître, c’est seulement parce que l’érudition accrue a mis entre nos mains des matériaux plus solides et plus nombreux. Dans une société humaine, toutes les parties se tiennent ; on n’en peut altérer une sans introduire par contre-coup dans les autres une altération proportionnée. Les institutions, les lois, les mœurs n’y sont point juxtaposées comme dans un amas, par hasard ou caprice, mais liées entre elles, par convenance ou nécessité, comme dans un concert\footnote{ \noindent {\itshape Esprit des Lois}, préface : « J’ai d’abord examiné les hommes et j’ai cru que, dans cette infinie diversité de lois et de mœurs, ils n’étaient pas uniquement conduits par leurs fantaisies. J’ai posé les principes et j’ai vu les cas particuliers s’y plier comme d’eux-mêmes, les histoires de toutes les nations n’en être que les suites, et chaque loi particulière liée à une autre loi ou dépendre d’une autre plus générale. »
 }. Selon que l’autorité est aux mains de tous, ou de plusieurs, ou d’un seul, selon que le prince admet ou n’admet pas au-dessus de lui des lois fixes et au-dessous de lui des pouvoirs intermédiaires, tout diffère ou tend à différer dans un sens prévu et d’une quantité constante, l’esprit public, l’éducation, la forme des jugements, la nature et le degré des peines, la condition des femmes, l’institution militaire, la nature et la grandeur de l’impôt. Du grand rouage central dépendent une multitude de rouages secondaires. Car, si l’horloge marche, c’est par l’accord de ses diverses pièces, d’où il suit que, si l’accord cesse, l’horloge se détraquera. Mais, outre le ressort principal, il y en a d’autres qui, agissant sur lui ou combinant leur action avec la sienne, impriment à chaque horloge un tour propre et une marche particulière. Tel est d’abord le climat, c’est-à-dire le degré du chaud et du froid, du sec et de l’humide, avec ses conséquences infinies sur le physique et sur le moral de l’homme, par suite sur la servitude ou la liberté politique, civile et domestique. Tel est aussi le terrain, selon sa fertilité, sa position et sa grandeur. Tel est le régime physique, selon que le peuple est chasseur, pasteur ou agriculteur. Telle est la fécondité de la race, par suite la multiplication lente ou rapide de la population, et aussi le nombre excessif tantôt des mâles, tantôt des femelles. Tels sont enfin le caractère national et la religion. — Toutes ces causes ajoutées l’une à l’autre ou limitées l’une par l’autre contribuent ensemble à un effet total, qui est la société. Simple ou compliquée, stable ou changeante, barbare ou civilisée, cette société a en elle-même ses raisons d’être. On peut expliquer sa structure, si bizarre qu’elle soit, ses institutions, si contradictoires qu’elles paraissent. Ni la prospérité, ni la décadence, ni le despotisme, ni la liberté ne sont des coups de dés amenés par les vicissitudes de la chance, ou des coups de théâtre improvisés par l’arbitraire d’un homme. Elles ont des conditions auxquelles nous ne pouvons nous soustraire. En tout cas, il nous est utile de connaître ces conditions, soit pour améliorer notre état, soit pour le prendre en patience, tantôt pour exécuter les réformes opportunes, tantôt pour renoncer aux réformes impraticables, tantôt pour avoir l’habileté qui réussit, tantôt pour acquérir la prudence qui s’abstient.

\section[{IV. Transformation de la psychologie  Condillac  Théorie de la sensation et des signes.}]{IV. Transformation de la psychologie  Condillac  Théorie de la sensation et des signes.}

\noindent Nous voici arrivés au centre des sciences morales, il s’agit de l’homme en général. Nous avons à faire l’histoire naturelle de l’âme, et nous la ferons comme toutes les autres, en écartant les préjugés, en ne tenant compte que des faits, en prenant pour guide l’analogie, en commençant par les origines, en suivant pas à pas le développement qui, d’un enfant, d’un sauvage, d’un homme inculte et primitif, fait l’homme raisonnable et cultivé. Considérons les débuts de la vie, l’animal au plus bas degré de l’échelle, l’homme à l’instant qui suit sa naissance. Ce que nous trouvons d’abord en lui, c’est la sensation, de telle ou telle espèce, agréable ou pénible, par suite un besoin, tendance ou désir, par suite enfin, grâce à un mécanisme physiologique, des mouvements volontaires ou involontaires, plus ou moins exactement et plus ou moins vite appropriés et coordonnés. Et ce fait élémentaire n’est pas seulement primitif ; il est encore incessant et universel, puisqu’on le rencontre à chaque moment de chaque vie, dans la plus compliquée comme dans la plus simple. Cherchons donc s’il n’est pas le fil dont toute notre trame mentale est tissée, et si le déroulement spontané qui le noue maille à maille n’aboutit pas à fabriquer le réseau entier de nos pensées et de nos passions  Sur cette idée, un esprit d’une précision et d’une lucidité incomparables, Condillac donne à presque toutes les grandes questions les réponses que le préjugé théologique renaissant et l’importation de la métaphysique allemande devaient discréditer chez nous au commencement du dix-neuvième siècle, mais que l’observation renouvelée, la pathologie mentale instituée et les vivisections multipliées viennent aujourd’hui ranimer, justifier et compléter\footnote{ \noindent Pinel (1791), Esquirol (1838), sur les maladies mentales  Prochaska, Le Gallois (1812), puis Flourens pour les vivisections  Hartley et James Mill, à la fin du dix-huitième siècle, suivent en psychologie la même voie que Condillac ; aujourd’hui toute la psychologie contemporaine y est rentrée. (Wundt, Helmholtz, Fechner en Allemagne, Bain, Stuart Mill, Herbert Spencer, Carpenter en Angleterre.)
 }. Déjà Locke avait dit que toutes nos idées ont pour source première l’expérience externe ou interne. Condillac montre en outre que toute perception, souvenir, idée, imagination, jugement, raisonnement, connaissance, a pour {\itshape éléments actuels} des sensations proprement dites ou des sensations renaissantes ; nos plus hautes idées n’ont pas d’autres matériaux ; car elles se réduisent à des signes qui sont eux-mêmes des sensations d’un certain genre. Ainsi les sensations sont la substance de l’intelligence humaine comme de l’intelligence animale ; mais la première dépasse infiniment la seconde, en ceci que, par la création des signes, elle parvient à isoler, extraire et noter des fragments de ses sensations, c’est-à-dire à former, combiner et manier des notions générales  Cela posé, nous pouvons vérifier toutes nos idées ; car nous pouvons toutes les refaire, et reconstruire avec réflexion ce que sans réflexion nous avions construit. Au commencement, point de définitions abstraites : l’abstrait est ultérieur et dérivé ; ce qu’il faut mettre en tête de chaque science, ce sont des exemples, des expériences, des faits sensibles ; c’est de là que nous extrairons notre idée générale. Pareillement, de plusieurs idées générales du même degré, nous en extrairons une autre plus générale, et ainsi de suite, pas à pas, en cheminant toujours selon l’ordre naturel, par une analyse continue, avec des notations expressives, à l’exemple des mathématiques qui passent du calcul par les doigts au calcul par les chiffres, puis de là au calcul par les lettres, et qui, appelant les yeux au secours de la raison, peignent l’analogie intime des quantités par l’analogie extérieure des symboles. De cette façon la science parfaite s’achèvera par une langue bien faite\footnote{Condillac, passim, et notamment dans ses deux derniers ouvrages, la {\itshape Logique} et la {\itshape Langue des calculs.}}  Grâce à ce renversement du procédé ordinaire, nous coupons court à toutes les disputes de mots, nous échappons aux illusions de la parole humaine, nous simplifions l’étude, nous refaisons l’enseignement, nous assurons la découverte, nous soumettons toute assertion au contrôle, et nous mettons toute vérité à la portée de tout esprit.

\section[{V. Méthode analytique. — Son principe. — Conditions requises pour qu’elle soit fructueuse  Ces conditions manquent ou sont insuffisantes au dix-huitième siècle. — Vérité et survivance du principe.}]{V. Méthode analytique. — Son principe. — Conditions requises pour qu’elle soit fructueuse  Ces conditions manquent ou sont insuffisantes au dix-huitième siècle. — Vérité et survivance du principe.}

\noindent C’est ainsi qu’il faut procéder dans toutes les sciences, et notamment dans les sciences morales et politiques. Considérer tour à tour chaque province distincte de l’action humaine, décomposer les notions capitales sous lesquelles nous la concevons, celles de religion, de société et de gouvernement, celles d’utilité, de richesse et d’échange, celles de justice, de droit et de devoir ; remonter jusqu’aux faits palpables, aux expériences premières, aux événements simples dans lesquels les éléments de la notion sont inclus ; en retirer ces précieux filons sans omission ni mélange ; recomposer avec eux la notion, fixer son sens, déterminer sa valeur ; remplacer l’idée vague et vulgaire de laquelle on est parti par la définition précise et scientifique à laquelle on aboutit et le métal impur qu’on a reçu par le métal affiné qu’on obtient : voilà la méthode générale que les philosophes enseignent alors sous le nom d’analyse et qui résume tout le progrès du siècle  Jusqu’ici et non plus loin ils ont raison : la vérité, toute vérité est dans les choses observables et c’est de là uniquement qu’on peut la tirer ; il n’y a pas d’autre voie qui conduise aux découvertes. — Sans doute l’opération n’est fructueuse que si la gangue est abondante et si l’on possède les procédés d’extraction ; pour avoir une notion juste de l’État, de la religion, du droit, de la richesse, il faut être au préalable historien, jurisconsulte, économiste, avoir recueilli des myriades de faits et posséder, outre une vaste érudition, une finesse très exercée et toute spéciale. Sans doute encore, si ces conditions ne sont qu’à demi remplies, l’opération ne donne que des produits incomplets ou d’aloi douteux, des ébauches de sciences, les rudiments de la pédagogie avec Rousseau, de l’économie politique avec Quesnay, Smith et Turgot, de la linguistique avec le président de Brosses, de l’arithmétique morale et de la législation pénale avec Bentham. Sans doute enfin, si aucune de ces conditions n’est remplie, la même opération, exécutée par des spéculatifs de cabinet, par des amateurs de salon et par des charlatans de place publique, n’aboutit qu’à des composés malfaisants et à des explosions meurtrières. — Mais une bonne règle demeure bonne, même après que l’ignorance et la précipitation en ont fait mauvais usage, et, si aujourd’hui nous reprenons l’œuvre manquée du dix-huitième siècle, c’est dans les cadres qu’il nous a transmis.
\chapterclose


\chapteropen

\chapter[{Chapitre II. Deuxième élément, l’esprit classique.}]{Chapitre II. \\
Deuxième élément, l’esprit classique.}


\chaptercont
\noindent Ce grand et magnifique édifice de vérités nouvelles ressemble à une tour dont le premier étage, subitement achevé, devient tout d’un coup accessible au public. Le public y monte, et les constructeurs lui disent de regarder, non pas au ciel et dans les espaces, mais devant lui, autour de lui, du côté de la terre, pour connaître enfin le pays qu’il habite. Certainement, le point de vue est bon et le conseil est judicieux. Mais on en conclurait à tort que le public verra juste ; car il reste encore à examiner l’état de ses yeux, s’il est presbyte ou myope, si, par habitude ou par nature, sa rétine n’est pas impropre à sentir certaines couleurs. Pareillement il nous reste à considérer les Français du dix-huitième siècle, la structure de leur œil intérieur, je veux dire la forme fixe d’intelligence qu’ils emportent avec eux, sans le savoir et sans le vouloir, sur leur nouvelle tour.\par

\section[{I. Ses indices, sa durée, sa puissance. — Ses origines et son public. — Son vocabulaire, sa grammaire, son style. — Son procédé, ses mérites, ses défauts.}]{I. Ses indices, sa durée, sa puissance. — Ses origines et son public. — Son vocabulaire, sa grammaire, son style. — Son procédé, ses mérites, ses défauts.}

\noindent Cette forme fixe est l’esprit classique, et c’est elle qui, appliquée à l’acquis scientifique du temps, a produit la philosophie du siècle et les doctrines de la Révolution. On reconnaît sa présence à divers indices, notamment au règne du style oratoire, régulier, correct, tout composé d’expressions générales et d’idées contiguës. Elle dure deux siècles, depuis Malherbe et Balzac jusqu’à Delille et M. de Fontanes ; pendant cette période si longue, nulle intelligence, sauf deux ou trois, et encore dans des mémoires secrets comme Saint-Simon, dans des lettres familières comme le marquis et le bailli de Mirabeau, n’ose et ne peut se soustraire à son empire. Bien loin de finir avec l’ancien régime, elle est le moule d’où sortent tous les discours, tous les écrits, jusqu’aux phrases et au vocabulaire de la Révolution. Or, quoi de plus efficace qu’un moule préalable, imposé, accepté, dans lequel, en vertu du naturel, de la tradition et de l’éducation, tout esprit s’enferme pour penser ? Celui-ci est donc une force historique et de premier ordre. Pour le bien connaître, voyons-le se former. — Il s’établit en même temps que la monarchie régulière et la conversation polie, et il les accompagne non par accident, mais par nature. Car il est justement l’œuvre de ce public nouveau que forment alors le nouveau régime et les nouvelles mœurs : je veux dire de l’aristocratie désœuvrée par la monarchie envahissante, des gens bien nés, bien élevés, qui, écartés de l’action, se rejettent vers la conversation et occupent leur loisir à goûter tous les plaisirs sérieux ou délicats de l’esprit\footnote{Voltaire, \href{http://www.voltaire-integral.com/Html/19/langues.htm}{\dotuline{{\itshape Dictionnaire philosophique}, article {\itshape Langues}}} [\url{http://www.voltaire-integral.com/Html/19/langues.htm}]. « De toutes les langues de l’Europe, la française doit être la plus générale, parce qu’elle est la plus propre à la conversation. Elle a pris son caractère dans celui du peuple qui la parle. Les Français ont été, depuis plus de cent cinquante ans, le peuple qui a le plus connu la société et qui en a le premier écarté toute gêne… C’est une monnaie plus courante que les autres, quand même elle manquerait de poids. »}. À la fin, ils n’auront plus d’autre emploi ni d’autre intérêt : causer, écouter, s’entretenir avec agrément et avec aisance de tous les sujets, graves ou légers, qui peuvent intéresser des hommes ou même des femmes du monde, voilà leur grande affaire. Au dix-septième siècle, on les appelle « les honnêtes gens », et c’est à eux désormais que s’adresse l’écrivain, même le plus abstrait. « L’honnête homme, dit Descartes, n’a pas besoin d’avoir lu tous les livres ni d’avoir appris soigneusement tout ce qu’on enseigne dans les écoles » ; et il intitule son dernier traité « Recherche de la vérité selon les lumières naturelles qui, à elles seules et sans le secours de la religion et de la philosophie, déterminent les opinions que doit avoir un {\itshape honnête homme} sur toutes les choses qui doivent faire l’objet de ses pensées\footnote{Descartes, éd. Cousin, \href{http://gallica.bnf.fr/ark:/12148/bpt6k942726/f347}{\dotuline{t. XI, 333}} [\url{http://gallica.bnf.fr/ark:/12148/bpt6k942726/f347}], \href{http://gallica.bnf.fr/ark:/12148/bpt6k942726/f354}{\dotuline{340 }} [\url{http://gallica.bnf.fr/ark:/12148/bpt6k942726/f354}]; \href{http://gallica.bnf.fr/ark:/12148/bpt6k94258n/f128}{\dotuline{I, 121}} [\url{http://gallica.bnf.fr/ark:/12148/bpt6k94258n/f128}]. Descartes déprime « les simples connaissances qui s’acquièrent sans le secours du raisonnement, telles que les langues, l’histoire, la géographie, et en général tout ce qui ne dépend que de l’expérience… Il n’est pas plus du devoir d’un honnête homme de savoir le grec et le latin que le langage suisse et le bas-breton, ni l’histoire de l’empire romano-germanique que celle du plus petit État qui se trouve en Europe ».} ». En effet, d’un bout à l’autre de sa philosophie, pour toute préparation il ne demande à ses lecteurs que « le bon sens naturel », joint à cette provision d’expérience courante que donne la pratique du monde. — Comme ils sont l’auditoire, ils sont les juges. « C’est le goût de la cour qu’il faut étudier, dit Molière\footnote{Molière, \href{http://gallica.bnf.fr/ark:/12148/bpt6k89116f}{\dotuline{{\itshape les Femmes savantes}}} [\url{http://gallica.bnf.fr/ark:/12148/bpt6k89116f}] et {\itshape la Critique de l’Ecole des femmes.} Rôles de Dorante en face de Lysidas et de Clitandre en face de Trissotin.}, il n’y a point de lieu où les décisions soient si justes… Du simple bon sens naturel et du commerce de tout le beau monde, on s’y fait une manière d’esprit qui, sans comparaison, juge plus finement les choses que tout le savoir enrouillé des pédants. » — À partir de ce moment, on peut dire que l’arbitre de la vérité et du goût n’est plus, comme auparavant, l’érudit, Scaliger par exemple, mais l’homme du monde, un La Rochefoucauld, un Tréville\footnote{Le docte Huet (1630-1721), qui en était resté au goût du seizième siècle, décrit ce changement très bien et à son point de vue. « Quand je suis entré dans le monde des lettres, elles étaient encore florissantes ; de grands personnages en soutenaient la gloire. J’ai vu les lettres décliner et tomber enfin dans une décadence presque entière. Car je ne connais presque personne aujourd’hui qu’on puisse véritablement appeler savant. » — Du Cange, quelques bénédictins comme Mabillon, plus tard l’académicien Fréret, Bréquigny, le président Bouhier, à Dijon, bref les vrais érudits, restent sans influence.}. Le pédant, et à sa suite le savant, l’homme spécial est écarté. « Les vrais honnêtes gens, dit Nicole d’après Pascal, ne veulent point d’enseigne. On ne les devine point, ils parleront des choses dont on parlait quand ils sont entrés. Ils ne sont point appelés poètes ni géomètres, mais ils jugent de tous ceux-là\footnote{Nicole, {\itshape Œuvres morales}, second traité de la charité et de l’amour-propre, 142.}. » — Au dix-huitième siècle, leur autorité est souveraine. Dans la grande foule composée « d’imbéciles » et parsemée de cuistres, il y a, dit Voltaire, « un petit troupeau séparé qu’on appelle {\itshape la bonne compagnie ;} ce petit troupeau, étant riche, bien élevé, instruit, poli, est comme la fleur du genre humain ; c’est pour lui que les plus grands hommes ont travaillé ; c’est lui qui donne la réputation\footnote{Voltaire, \href{http://www.voltaire-integral.com/Html/24/38\_Grizel.html}{\dotuline{{\itshape Dialogues, L’intendant des menus et l’abbé Grizel}}} [\url{http://www.voltaire-integral.com/Html/24/38\_Grizel.html}] [‘troupeau’],129.} ». L’admiration, la faveur, l’importance appartiennent, non à ceux qui en sont dignes, mais à ceux qui s’adressent à lui. « En 1789, disait l’abbé Maury, l’Académie française était seule considérée en France et donnait réellement un état. Celle des Sciences ne signifiait rien dans l’opinion, non plus que celle des Inscriptions… Les langues sont la science des sots. D’Alembert avait honte d’être de l’Académie des Sciences. Un mathématicien, un chimiste, etc., ne sont entendus que par une poignée de gens ; le littérateur, l’orateur s’adressent à l’univers\footnote{Maury ajoutait avec sa brutalité habituelle : « À l’Académie française, nous regardions les membres de celle des Sciences comme nos valets ». — Ces valets étaient alors Lavoisier, Fourcroy, Lagrange, Laplace, etc. (Récit du comte Joseph de Maistre cité par Sainte-Beuve, {\itshape Causeries du lundi}, IV, 283.)}. » — Sous une pression si forte, il faut bien que l’esprit prenne le tour oratoire et littéraire, et s’accommode aux exigences, aux convenances, aux goûts, au degré d’attention et d’instruction de son public. De là le moule classique : il est formé par l’habitude de parler, d’écrire et de penser en vue d’un auditoire de salon.\par
La chose est visible, et du premier coup d’œil, pour la langue et le style. Entre Amyot, Rabelais, Montaigne d’un côté, et Chateaubriand, Victor Hugo, Honoré de Balzac de l’autre, naît et finit le français classique. Dès l’origine il a son nom : c’est la langue des honnêtes gens ; il est fait, non seulement pour eux, mais par eux\footnote{Vaugelas, \href{http://gallica.bnf.fr/ark:/12148/bpt6k509950}{\dotuline{{\itshape Remarques sur la langue française} }} [\url{http://gallica.bnf.fr/ark:/12148/bpt6k509950}]: « C’est la façon de parler de la plus saine partie de la cour conformément à la façon d’écrire de la plus saine partie des auteurs du temps… Il vaut mieux consulter les femmes et ceux qui n’ont point étudié que ceux qui sont bien savants en la langue grecque et en la latine. »}, et Vaugelas, leur secrétaire, ne s’applique pendant trente ans qu’à enregistrer les décisions « du bon usage ». C’est pourquoi, dans toutes ses parties, vocabulaire et grammaire, la langue se réforme et se reforme sur le modèle de leur esprit, qui est l’esprit régnant  En premier lieu, le vocabulaire s’allège. On exclut du discours la plupart des mots qui servent à l’érudition spéciale et à l’expérience technique, les expressions trop latines ou trop grecques, les termes propres d’école, de science, de métier, de ménage, tout ce qui sent de trop près une occupation ou profession particulière et n’est pas de mise dans la conversation générale. On en ôte quantité de mots expressifs et pittoresques, tous ceux qui sont crus, gaulois ou naïfs, tous ceux qui sont locaux et provinciaux ou personnels et forgés, toutes les locutions familières et proverbiales\footnote{Une des causes de la chute et de la défaveur du marquis d’Argenson, au dix-huitième siècle, fut l’habitude qu’il avait d’employer ces sortes de locutions.}, nombre de tours familiers, brusques et francs, toutes les métaphores risquées et poignantes, presque toutes ces façons de parler inventées et primesautières qui, par leur éclair soudain, font jaillir dans l’imagination la forme colorée, exacte et complète des choses, mais dont la trop vive secousse choquerait les bienséances de la conversation polie. « Il ne faut qu’un mauvais mot, disait Vaugelas, pour faire mépriser une personne dans une compagnie », et, à la veille de la Révolution, un mauvais mot dénoncé par Mme de Luxembourg rejette encore un homme au rang des « espèces », parce que le bon langage est toujours une partie des bonnes façons  Par ce grattage incessant la langue se réduit et se décolore : Vaugelas juge déjà qu’on a retranché la moitié des phrases et des mots d’Amyot\footnote{ \noindent Vaugelas, {\itshape ib.} « Bien que nous ayons retranché la moitié de ses phrases et de ses mots, nous ne laissons pas de trouver dans l’autre moitié presque toutes les richesses dont nous nous vantons et dont nous faisons parade. » — Comparez le lexique de deux ou trois écrivains du seizième siècle et de deux ou trois écrivains du dix-septième. J’essaye de donner ici en abrégé les résultats de la comparaison ; si on la fait, la plume à la main, sur cent pages de deux textes, on sera étonné de la différence. Prendre pour exemple deux écrivains du même genre et de second ordre, Charron et Nicole.
 }. Sauf chez La Fontaine, un génie spontané et isolé qui rouvre les sources anciennes, sauf chez La Bruyère, un chercheur hardi qui ouvre une source nouvelle, sauf chez Voltaire, un démon incarné qui, dans ses écrits anonymes ou pseudonymes, lâche la bride aux violences et à la crudité de sa verve\footnote{Par exemple, l’article \href{http://www.voltaire-integral.com/Html/19/ignorance.htm}{\dotuline{{\itshape Ignorance}}} [\url{http://www.voltaire-integral.com/Html/19/ignorance.htm}] dans le {\itshape Dictionnaire philosophique.}}, les mots propres tombent en désuétude. Un jour, à l’Académie, Gresset, dans un discours, en osa lâcher cinq ou six\footnote{Laharpe, {\itshape Cours de littérature}, éd. Didot, II, 142.} : il s’agissait, je crois, de voitures et de coiffures ; des murmures éclatèrent ; pendant sa longue retraite, il était devenu provincial et avait perdu le ton  Par degrés, on en vient à ne plus composer le discours que « d’expressions générales ». Même, selon le précepte de Buffon, on les emploie pour désigner les choses particulières. Cela est plus conforme à l’urbanité, qui efface, qui atténue, qui évite les accents brusques et familiers, à qui nombre d’idées sembleraient grossières ou triviales, si on ne les enveloppait d’un demi-voile. Cela est plus commode pour l’attention paresseuse ; il n’y a que les termes généraux de la conversation pour réveiller à l’instant les idées courantes et communes ; tout homme les entend par cela seul qu’il est du salon ; au contraire, des termes particuliers demanderaient un effort de mémoire ou d’imagination ; si, à propos des sauvages ou des anciens Francs, je dis « la hache de guerre », tous comprennent du premier coup ; si je dis « le tomahawk », ou « la francisque », plusieurs supposeront que je parle teuton ou iroquois\footnote{ \noindent Pour prendre un exemple au hasard, je trouve dans {\itshape l’Optimiste} (1788), de Colin d’Harleville, l’indication suivante : « La scène représente un bosquet rempli d’arbres odoriférants. » — Il eût été contraire à l’esprit classique de dire quels étaient ces arbres, lilas, tilleuls, aubépines, etc. — De même dans les paysages peints, les arbres ne sont d’aucune espèce connue : ce sont des arbres en général.
 }. À cet égard, plus le genre est élevé, plus le scrupule est fort ; tout mot propre est banni de la poésie ; quand on en rencontre un, il faut l’esquiver ou le remplacer par une périphrase. Un poète du dix-huitième siècle n’a guère à sa disposition que le tiers environ du dictionnaire, et la langue poétique à la fin sera si restreinte que, lorsqu’un homme aura quelque chose à dire, il ne pourra plus le dire en vers.\par
En revanche, plus on élague et plus on éclaircit. Réduit à un vocabulaire de choix, le français dit moins de choses, mais il les dit avec plus de justesse et d’agrément. « Urbanité, exactitude », ces deux mots qui naissent en même temps que l’Académie française sont l’abrégé de la réforme dont elle est l’organe et que les salons, par elle et à coté d’elle, imposent au public. De grands seigneurs retirés, de belles dames oisives s’amusent à démêler les nuances des termes pour en composer des maximes, des définitions et des portraits. Avec un scrupule admirable et une délicatesse de tact infinie, écrivains et gens du monde s’appliquent à peser chaque mot et chaque locution, pour en fixer le sens, pour en mesurer la force et la portée, pour en déterminer les affinités, l’usage et les alliances, et ce travail de précision se poursuit depuis les premiers académiciens, Vaugelas, Chapelain et Conrart, jusqu’à la fin de l’âge classique, par les {\itshape Synonymes} de Beauzée et de Girard, par les {\itshape Remarques} de Duclos, par le {\itshape Commentaire} de Voltaire sur Corneille, par le {\itshape Lycée} de Laharpe\footnote{Voir dans le {\itshape Lycée} de Laharpe, après l’analyse de chaque pièce, les remarques de détail sur le style.}, par l’effort, l’exemple, la pratique et l’autorité des grands et petits écrivains qui sont tous corrects. Jamais architectes, obligés de n’employer pour bâtir que les pavés de la grande route publique, n’ont si bien connu chacune de leurs pierres, ses dimensions, sa coupe, sa résistance, ses attaches possibles, sa place convenable  Cela fait, il s’agit de construire avec le moins de peine et le plus de solidité qu’il se pourra, et la grammaire se réforme en même temps et dans le même sens que le dictionnaire. Elle ne permet plus aux mots de se suivre selon l’ordre variable des impressions et des émotions ; elle les dispose régulièrement et rigoureusement selon l’ordre immuable des idées. L’écrivain perd le droit de mettre en tête et en vedette l’objet ou le trait qui le frappe le plus vivement et d’abord : le cadre est fait, les places sont désignées d’avance. Chaque partie du discours a la sienne : défense d’en omettre ou d’en transposer une seule, comme on faisait au seizième siècle\footnote{Omission des pronoms {\itshape je, il, nous, vous, ils}, des articles {\itshape le, la, les}, du verbe et notamment du verbe {\itshape est}. — Quant aux transpositions, il suffit de lire une page de Rabelais, Amyot ou Montaigne, pour voir combien alors elles étaient nombreuses et variées.} ; il les faut toutes et aux endroits marqués, d’abord le sujet avec ses appendices, puis le verbe, puis le régime direct, enfin le complément indirect. De cette façon, la phrase est un échafaudage gradué, où l’esprit place d’abord la substance, puis la qualité, puis les manières d’être de la qualité, comme un bon architecte qui pose en premier lieu le fondement, puis la bâtisse, puis les accessoires, par économie et par prudence, afin de préparer dans chaque morceau de son édifice un support pour le morceau qui suit. Il n’y a pas de phrase qui demande une moindre dépense d’attention, ni où l’on puisse, à chaque pas, constater plus sûrement l’attache ou l’incohérence des parties\footnote{Vaugelas, {\itshape ib.} « Il n’y a pas de langue qui soit plus ennemie des équivoques et de toutes sortes d’obscurités. »}  La méthode qui arrange la phrase simple arrange aussi la période, le paragraphe et la série des paragraphes ; elle fait le style, comme elle a fait la syntaxe. Dans le grand édifice total, il y a, pour chaque petit édifice partiel, un lieu distinct, et il n’y en a qu’un. À mesure que le discours avance, chaque emplacement doit se remplir à son tour, jamais avant, jamais après, sans que jamais un membre parasite soit introduit, sans que jamais un membre légitime usurpe sur son voisin ; et tous ces membres, liés entre eux par leur position même, doivent concourir de toutes leurs forces à un seul objet. Enfin, pour la première fois, voici dans un écrit des groupes naturels et distincts, des ensembles clos et complets, dont aucun n’empiète ni ne subit d’empiètement. Il n’est plus permis d’écrire au hasard et selon le caprice de la verve, de jeter ses idées par paquets, de s’interrompre par des parenthèses, d’enfiler l’enfilade interminable des citations et des énumérations. Un but est donné : il y a quelque vérité à prouver, quelque définition à trouver, quelque persuasion à produire ; pour cela, il faut marcher toujours, et toujours droit. Ordonnance, suite, progrès, transitions ménagées, développement continu, tels sont les caractères de ce style. Cela va si loin qu’à l’origine\footnote{ \noindent Voir les principaux romans du dix-septième siècle, \href{http://gallica.bnf.fr/ark:/12148/bpt6k2027898}{\dotuline{{\itshape le Roman bourgeois}}} [\url{http://gallica.bnf.fr/ark:/12148/bpt6k2027898}] de Furetière, \href{http://gallica.bnf.fr/ark:/12148/bpt6k575827}{\dotuline{{\itshape la Princesse de Clèves}}} [\url{http://gallica.bnf.fr/ark:/12148/bpt6k575827}] par Mme de la Fayette, la \href{http://gallica.bnf.fr/ark:/12148/bpt6k108429q}{\dotuline{{\itshape Clélie}}} [\url{http://gallica.bnf.fr/ark:/12148/bpt6k108429q}] de Mlle de Scudéry, et même \href{http://gallica.bnf.fr/ark:/12148/bpt6k27787w}{\dotuline{{\itshape le Roman comique}}} [\url{http://gallica.bnf.fr/ark:/12148/bpt6k27787w}] de Scarron. — Voir les lettres de Balzac, de Voiture et de leurs correspondants, le {\itshape Récit des grands jours d’Auvergne} par Fléchier, etc. Sur le caractère oratoire de ce style, cf. Sainte-Beuve, {\itshape Port-Royal}, 2\textsuperscript{e} éd., I, 515.
 } les lettres familières, les romans, les plaisanteries de société, les pièces de galanterie et de badinage sont des morceaux d’éloquence méthodique. À l’hôtel de Rambouillet, la période explicative s’étale avec autant d’ampleur et de raideur que chez Descartes lui-même. Un des mots les plus fréquents chez Mlle de Scudéry est la conjonction {\itshape car.} On déduit sa passion en raisonnements bien liés. Des gentillesses de salon s’allongent en phrases aussi concertées qu’une dissertation académique. L’instrument à peine formé manifeste déjà ses aptitudes ; on sent qu’il est fait pour expliquer, démontrer, persuader et vulgariser ; un siècle plus tard, Condillac aura raison de dire qu’il est par lui-même un procédé systématique de décomposition et de recomposition, une méthode scientifique analogue à l’arithmétique et à l’algèbre. À tout le moins, il a cet avantage incontesté, qu’en partant de quelques termes usuels il conduit aisément et promptement tout lecteur, par une série de combinaisons simples, jusqu’aux combinaisons les plus hautes\footnote{Voltaire, \href{http://www.voltaire-integral.com/Html/08/21\_Epique.html\#ESSAI}{\dotuline{{\itshape Essai sur le poème épique}}} [\url{http://www.voltaire-integral.com/Html/08/21\_Epique.html\#ESSAI}] [‘légère’]. « Notre nation, regardée comme si légère par les étrangers, est de toutes les nations la plus sage, la plume à la main. La méthode est la qualité dominante de nos écrivains. »}. À ce titre, en 1789, la langue française est la première de toutes. L’Académie de Berlin propose en concours l’explication de sa prééminence. On la parle dans toute l’Europe. On ne parle qu’elle dans la diplomatie. Elle est internationale comme autrefois le latin, et il semble qu’elle soit désormais l’organe préféré de la raison.\par
Elle n’est que l’organe d’une certaine raison, la raison raisonnante, celle qui veut penser avec le moins de préparation et le plus de commodité qu’il se pourra, qui se contente de son acquis, qui ne songe pas à l’accroître ou à le renouveler, qui ne sait pas ou ne veut pas embrasser la plénitude et la complexité des choses réelles. Par son purisme, par son dédain pour les termes propres et les tours vifs, par la régularité minutieuse de ses développements, le style classique est incapable de peindre ou d’enregistrer complètement les détails infinis et accidentés de l’expérience. Il se refuse à exprimer les dehors physiques des choses, la sensation directe du spectateur, les extrémités hautes et basses de la passion, la physionomie prodigieusement composée et absolument personnelle de l’individu vivant, bref cet ensemble unique de traits innombrables, accordés et mobiles, qui composent, non pas le caractère humain en général, mais tel caractère humain, et qu’un Saint-Simon, un Balzac, un Shakespeare lui-même ne pourraient rendre, si le langage copieux qu’ils manient et que leurs témérités enrichissent encore, ne venait prêter ses nuances aux détails multipliés de leur observation\footnote{Les œuvres de Milton contiennent environ 8 000 mots. « Shakespeare, chez qui la variété de l’expression est probablement plus grande que dans tout autre écrivain de quelque langue que ce soit, a composé toutes ses pièces avec 15 000 mots environ. » (Max Müller, {\itshape Lectures on the science of language}, I, 309.) — Il serait curieux d’établir en regard le compte si restreint du vocabulaire de Racine. Celui des romans de Mlle de Scudéry est extrêmement limité. Dans le plus beau roman du dix-septième siècle, {\itshape la Princesse de Clèves}, le nombre des mots est réduit au minimum  Le {\itshape Dictionnaire} de l’ancienne Académie française contient 29 712 mots ; le {\itshape Thesaurus grec} de H. Estienne environ 150 000.}. Avec ce style, on ne peut traduire ni la Bible, ni Homère, ni Dante, ni Shakespeare\footnote{Comparez les traductions de la Bible par M. de Sacy et par Luther, celles d’Homère par M. Dacier, Bitaubé, etc., et par Leconte de Lisle, celles d’Hérodote par Larcher et par Courier, les {\itshape Contes populaires} de Perrault et ceux de Grimm, etc.} ; lisez le monologue d’Hamlet dans Voltaire, et voyez ce qu’il en reste, une déclamation abstraite, à peu près ce qui reste d’Othello dans son Orosmane. Regardez dans Homère, puis dans Fénelon, l’île de Calypso : l’île rocheuse, sauvage, où nichent les mouettes et les autres oiseaux de mer aux longues ailes », devient dans la belle prose française un parc quelconque arrangé « pour le plaisir des yeux ». Au dix-huitième siècle, des romanciers contemporains, et qui sont eux-mêmes de l’âge classique, Fielding, Swift, Defoe, Sterne, Richardson, ne sont reçus en France qu’avec des atténuations et après des coupures ; ils ont des mots trop francs, des scènes trop fortes ; leurs familiarités, leurs crudités, leurs bizarreries feraient tache ; le traducteur écourte, adoucit, et parfois, dans sa préface, s’excuse de ce qu’il a laissé. Il n’y a place dans cette langue que pour une portion de la vérité, portion exiguë, et que l’épuration croissante rend tous les jours plus exiguë encore. Considéré en lui-même, le style classique court toujours risque de prendre pour matériaux des {\itshape lieux communs} minces et sans substance. Il les étire, il les entrelace, il les tisse ; mais, de son engrenage logique, il ne sort qu’un filigrane fragile ; on en peut louer l’élégant artifice, mais, dans la pratique, l’œuvre est d’usage petit, nul, ou dangereux.\par
D’après ces caractères du style, on devine ceux de l’esprit auquel il a servi d’organe  Deux opérations principales composent le travail de l’intelligence humaine. Placée en face des choses, elle reçoit l’impression plus ou moins exacte, complète et profonde ; ensuite, quittant les choses, elle décompose son impression, et classe, distribue, exprime plus ou moins habilement les idées qu’elle en tire  Dans la seconde de ces opérations, le classique est supérieur. Obligé de s’accommoder à ses auditeurs, c’est-à-dire à des gens du monde qui ne sont point spéciaux et qui sont difficiles, il a dû porter à la perfection l’art de se faire écouter et de se faire entendre, c’est-à-dire l’art de composer et d’écrire  Avec une industrie délicate et des précautions multipliées, il conduit ses lecteurs par un escalier d’idées doux et rectiligne, de degré en degré, sans omettre une seule marche, en commençant par la plus basse et ainsi de suite jusqu’à la plus haute, de façon qu’ils puissent toujours aller d’un pas égal et suivi, avec la sécurité et l’agrément d’une promenade. Jamais d’interruption ni d’écart possible : des deux côtés, tout le long du chemin, on est maintenu par des balustrades, et chaque idée se continue dans la suivante par une transition si insensible, qu’on avance involontairement, sans s’arrêter ni dévier, jusqu’à la vérité finale où l’on doit s’asseoir. Toute la littérature classique porte l’empreinte de ce talent ; il n’y a pas de genre où il ne pénètre et n’introduise les qualités d’un bon discours  Il domine dans les genres qui, par eux-mêmes, ne sont qu’à demi littéraires, mais qui, grâce à lui, le deviennent, et il transforme en belles œuvres d’art des écrits que leur matière semblait reléguer parmi les livres de science, parmi les instruments d’action, parmi les documents d’histoire, traités philosophiques, exposés de doctrine, sermons, polémique, dissertations et démonstrations, dictionnaires mêmes, depuis Descartes jusqu’à Condillac, depuis Bossuet jusqu’à Buffon et Voltaire, depuis Pascal jusqu’à Rousseau et Beaumarchais, bref la prose presque tout entière, même les dépêches officielles et la correspondance diplomatique, même les correspondances intimes, et, depuis Mme de Sévigné jusqu’à Mme du Deffand, tant de lettres parfaites échappées à la plume de femmes qui n’y songeaient pas  Il domine dans les genres qui, par eux-mêmes, sont littéraires, mais qui reçoivent de lui un tour oratoire. Non seulement il impose aux œuvres dramatiques un plan exact, une distribution régulière\footnote{Racine, {\itshape Discours académique} pour la réception de Thomas Corneille : « Dans ce chaos du poème dramatique, votre illustre frère fit voir sur la scène {\itshape la raison}, mais la raison accompagnée de toute la pompe et de tous les ornements dont notre langue est capable. »}, des proportions calculées, des coupures et des liaisons, une suite et un progrès, comme dans un morceau d’éloquence ; mais encore il n’y tolère que des discours parfaits. Point de personnage qui n’y soit un orateur accompli ; chez Corneille, Racine et Molière lui-même, un confident, un roi barbare, un jeune cavalier, une coquette de salon, un valet, se montrent passés maîtres dans l’art de la parole. Jamais on n’a vu d’exordes si adroits, de preuves si bien disposées, de raisonnements si justes, de transitions si fines, de péroraisons si concluantes. Jamais le dialogue n’a si fort ressemblé à une joute oratoire. Tous les récits, tous les portraits, tous les exposés d’affaires pourraient être détachés et proposés en modèle dans les écoles, avec les chefs-d’œuvre de la tribune antique. Le penchant est si grand de ce côté, qu’au moment suprême et dans le plus fort de la dernière angoisse, le personnage, seul et sans témoins, trouve moyen de plaider son délire et de mourir éloquemment.

\section[{II. Sa lacune originelle. — Signes de cette lacune au dix-septième siècle. — Elle s’accroît avec le temps et le succès. — Preuves de cet accroissement au dix-huitième siècle. — Poèmes sérieux, théâtre, histoire, romans. — Conception écourtée de l’homme et de la vie humaine.}]{II. Sa lacune originelle. — Signes de cette lacune au dix-septième siècle. — Elle s’accroît avec le temps et le succès. — Preuves de cet accroissement au dix-huitième siècle. — Poèmes sérieux, théâtre, histoire, romans. — Conception écourtée de l’homme et de la vie humaine.}

\noindent Cet excès indique une lacune. Des deux opérations qui composent la pensée humaine, le classique fait mieux la seconde que la première. En effet, la seconde nuit à la première, et l’obligation de toujours bien dire l’empêche de dire tout ce qu’il faudrait. Chez lui la forme est plus belle que le fonds n’est riche, et l’impression originale, qui est la source vive, perd, dans les canaux réguliers où on l’enferme, sa force, sa profondeur et ses bouillonnements. La poésie proprement dite, celle qui tient du rêve et de la vision, ne saurait naître. Le poème lyrique avorte, et aussi le poème épique\footnote{Voltaire, \href{http://www.voltaire-integral.com/Html/08/21\_Epique.html\#ESSAI}{\dotuline{{\itshape Essai sur le poème épique}}} [\url{http://www.voltaire-integral.com/Html/08/21\_Epique.html\#ESSAI}] [‘polies,’], 290. « Il faut avouer qu’il est plus difficile à un Français qu’à un autre de faire un poème épique… Oserai-je l’avouer ? C’est que, de toutes les nations polies, la nôtre est la moins poétique. Les ouvrages en vers qui sont le plus à la mode en France sont les pièces de théâtre ; ces pièces doivent être écrites dans un style naturel qui approche de la conversation. »}. Rien ne pousse dans ces confins reculés et sublimes par lesquels la parole touche à la musique et à la peinture. Jamais on n’entend le cri involontaire de la sensation vive, la confidence solitaire de l’âme trop pleine\footnote{Sauf dans les {\itshape Pensées} de Pascal, simples notes griffonnées par un chrétien exalté et malade, et qui certainement ne seraient pas restées les mêmes dans le livre imprimé et complet.} qui ne parle que pour se décharger et s’épancher. S’il s’agit, comme dans le poème dramatique, de créer des personnages, le moule classique n’en peut façonner que d’une espèce : ce sont ceux qui, par éducation, naissance ou imitation, parlent toujours bien, en d’autres termes, des gens du monde. Il n’y en a pas d’autres au théâtre ni ailleurs, depuis Corneille et Racine jusqu’à Marivaux et Beaumarchais. Le pli est si fort, qu’il s’impose jusqu’aux animaux de La Fontaine, jusqu’aux servantes et aux valets de Molière, jusqu’aux Persans de Montesquieu, aux Babyloniens, aux Indiens, aux Micromégas de Voltaire. — Encore faut-il ajouter que ces personnages ne sont réels qu’à demi. Dans un caractère vivant, il y a deux sortes de traits : les premiers, peu nombreux, qui lui sont communs avec tous les individus de sa classe et que tout spectateur ou lecteur peut aisément démêler ; les seconds, très nombreux, qui n’appartiennent qu’à lui et qu’on ne saisit pas sans quelque effort. L’art classique ne s’occupe que des premiers ; de parti pris, il efface, néglige ou subordonne les seconds. Il ne fait pas des individus véritables, mais des caractères généraux, le roi, la reine, le jeune prince, la jeune princesse, le confident, le grand prêtre, le capitaine des gardes, avec quelque passion, habitude ou inclination générale, amour, ambition, fidélité ou perfidie, humeur despotique ou pliante, méchanceté ou bonté native. Quant aux circonstances de temps et de lieu, qui de toutes sont les plus puissantes pour façonner et diversifier l’homme, il les indique à peine ; il en fait abstraction. À vrai dire, dans la tragédie, la scène est partout et en tout siècle, et l’on pourrait affirmer aussi justement qu’elle n’est dans aucun siècle ni nulle part. C’est un palais ou un temple quelconque, où, pour effacer toute empreinte historique et personnelle, une convention uniforme importe des façons et des costumes qui ne sont ni français ni étrangers, ni anciens ni modernes\footnote{Voir, au Cabinet des Estampes, les costumes peints des principaux personnages du théâtre au milieu du dix-huitième siècle. — Rien de plus contraire à l’esprit du théâtre classique que de jouer, comme on le fait aujourd’hui, Britannicus, Esther, avec des costumes et un décor exact, tirés des dernières fouilles de Pompéi et de Ninive.}. Dans ce monde abstrait, on se dit toujours « vous », « seigneur » et « madame », et le style noble pose la même draperie sur les caractères les plus opposés. Quand Corneille et Racine, à travers la pompe ou l’élégance de leurs vers, nous font entrevoir des figures contemporaines, c’est à leur insu ; ils ne croyaient peindre que l’homme en soi ; et, si aujourd’hui nous reconnaissons chez eux tantôt les cavaliers, les duellistes, les matamores, les politiques et les héroïnes de la Fronde, tantôt les courtisans, les princes, les évêques, les dames d’atour et les menins de la monarchie régulière, c’est que leur pinceau, trempé involontairement dans leur expérience, laissait par mégarde tomber de la couleur dans le contour idéal et nu que seul ils voulaient tracer. Rien qu’un contour, une esquisse générale que la diction correcte remplit de sa grisaille unie. — Même dans la comédie, qui, de parti pris, peint les mœurs environnantes, même chez Molière si franc et si hardi, le modelé est incomplet, la singularité individuelle est supprimée, le visage devient par instants un masque de théâtre, et le personnage, surtout lorsqu’il parle en vers, cesse quelquefois de vivre, pour n’être plus que le porte-voix d’une tirade ou d’une dissertation\footnote{Les rôles de moraliste et de raisonneur, Cléante ({\itshape Tartufe}), Ariste ({\itshape Femmes savantes}), Chrysale ({\itshape École des femmes}), etc. La discussion entre les deux frères de dona Elvire ({\itshape Festin de Pierre}, III, 5). Le discours d’Ergaste dans {\itshape l’École des maris.} Celui d’Eliante imité de {\itshape Lucrèce} dans {\itshape le Misanthrope} (II, 5). Les portraits par Dorine, dans le {\itshape Tartufe}, I, 1. — Le portrait de l’hypocrite par don Juan ({\itshape Festin de Pierre}, V, 2).}. Parfois on oublie de nous marquer son rang, sa condition, sa fortune, s’il est gentilhomme ou bourgeois, provincial ou parisien\footnote{Rôles d’Harpagon et d’Amolphe.}. Rarement on nous fait sentir, comme Shakespeare, ses dehors physiques, son tempérament, l’état de ses nerfs, son accent brusque ou traînant, son geste saccadé ou compassé, sa maigreur ou sa graisse\footnote{On le voit pour Tartufe, mais par un mot de Dorine et non directement. Cf. dans Shakespeare les rôles de Coriolan, Hotspur, Falstaff, Othello, Cléopâtre, etc.}. Souvent on ne prend pas la peine de lui trouver un nom propre ; il est Chrysale, Orgon, Damis, Dorante, Valère. Son nom ne désigne qu’une qualité pure, celle de père, de jeune homme, de valet, de grondeur, de galant, et, comme un pourpoint banal, s’ajuste indifféremment à toutes les tailles à peu près pareilles en passant de la garde-robe de Molière à celle de Regnard, de Lesage, de Destouches et de Marivaux\footnote{Balzac passait des journées à lire l’{\itshape Almanach des cent mille adresses} et courait en fiacre pendant des après-midi, regardant toutes les enseignes, afin de trouver le nom propre de son personnage. Ce seul petit détail montre la différence des deux conceptions de l’homme.}. Il manque au personnage l’étiquette personnelle, l’appellation authentique et unique qui est la marque première de l’individu. Tous ces détails, toutes ces circonstances, tous ces supports et compléments de l’homme sont en dehors du cadre classique. Pour en insérer quelques-uns, il a fallu le génie de Molière, la plénitude de sa conception, la surabondance de son observation, la liberté extrême de sa plume. Encore est-il vrai que souvent il les omet, et que, dans les autres cas, il n’en introduit qu’un petit nombre, parce qu’il évite de donner à des caractères généraux une richesse et une complexité qui embarrasseraient l’action. Plus le thème est simple, et plus le développement est clair ; or, dans toute cette littérature, la première obligation de l’auteur est de développer clairement le thème qu’il s’est choisi.\par
Il y a donc un défaut originel dans l’esprit classique, défaut qui tient à ses qualités et qui, maintenu d’abord dans une juste mesure, contribue à lui faire produire ses plus purs chefs-d’œuvre, mais qui, selon une règle universelle, va s’aggraver et se tourner en vice par l’effet naturel de l’âge, de l’exercice et du succès. Il était étroit, il va devenir plus étroit. Au dix-huitième siècle, il est impropre à figurer la chose vivante, l’individu réel, tel qu’il existe effectivement dans la nature et dans l’histoire, c’est-à-dire comme un ensemble indéfini, comme un riche réseau, comme un organisme complet de caractères et de particularités superposées, enchevêtrées et coordonnées. La capacité lui manque pour les recevoir et les contenir. Il en écarte le plus qu’il peut, tant qu’enfin il n’en garde qu’un extrait écourté, un résidu évaporé, un nom presque vide, bref ce qu’on appelle une abstraction creuse. Il n’y a de vivant au dix-huitième siècle que les petites esquisses brochées en passant et comme en contrebande par Voltaire, le baron de Thundertentrunck, mylord Whatthen, les figurines de ses contes, et cinq ou six portraits du second plan, Turcaret, Gil Blas, Marianne, Manon Lescaut, le neveu de Rameau, Figaro, deux ou trois pochades de Crébillon fils et de Collé, œuvres où la familiarité a laissé rentrer la sève, que l’on peut comparer à celles des petits-maîtres de la peinture, Watteau, Fragonard, Saint-Aubin, Moreau, Lancret, Pater, Baudouin, et qui, reçues difficilement ou par surprise dans le salon officiel, subsisteront encore, lorsque les grands tableaux sérieux auront moisi sous l’ennui qu’ils exhalaient. Partout ailleurs la sève est tarie, et, au lieu de plantes florissantes, on ne trouve que des fleurs de papier peint. Tant de poèmes sérieux, depuis {\itshape la Henriade} de Voltaire jusqu’aux {\itshape Mois} de Roucher ou à {\itshape l’Imagination de} Delille, que sont-ils sinon des morceaux de rhétorique garnis de rimes ? Parcourez les innombrables tragédies et comédies dont Grimm et Collé nous donnent l’extrait mortuaire, même les bonnes pièces de Voltaire et de Crébillon, plus tard celles des auteurs qui ont la vogue, Belloy, Laharpe, Ducis, Marie Chénier. Éloquence, art, situations, beaux vers, tout y est, excepté des hommes ; les personnages ne sont que des mannequins bien appris, et le plus souvent des trompettes par lesquels l’auteur lance au public ses déclamations. Grecs, Romains, chevaliers du moyen âge, Turcs, Arabes, Péruviens, Guèbres, Byzantins, ils sont tous la même mécanique à tirades. Et le public ne s’en étonne point ; il n’a pas le sentiment historique ; il admet que l’homme est partout le même ; il fait un succès aux {\itshape Incas} de Marmontel, au {\itshape Gonzalve} et aux {\itshape Nouvelles} de Florian, à tous les paysans, manœuvres, nègres, Brésiliens, Parsis, Malabares, qui viennent lui débiter leurs amplifications. On ne voit dans l’homme qu’une raison raisonnante, la même en tout temps, la même en tout lieu ; Bernardin de Saint-Pierre la prête à son Paria, Diderot à ses Otaïtiens. Il est de principe que naturellement tout esprit humain parle et pense comme un livre  Aussi quelle insuffisance dans l’histoire ! À part Charles XII, un contemporain que Voltaire ranime grâce aux récits de témoins oculaires, à part les vifs raccourcis, les lestes croquis d’Anglais, de Français, d’Espagnols, d’Italiens, d’Allemands qu’il sème en courant dans ses contes, ici encore où sont les hommes ? Chez Hume, Gibbon, Robertson qui sont de l’école française et tout de suite adoptés en France, dans les recherches de Dubos et de Mably sur notre moyen âge, dans le Louis XI de Duclos, dans l’Anacharsis de Barthélemy, même dans l’{\itshape Essai sur les mœurs} et dans le {\itshape Siècle de Louis XIV} de Voltaire, même dans la {\itshape Grandeur des Romains}, et l’{\itshape Esprit des Lois} de Montesquieu, quelle étrange lacune ! Érudition, critique, bon sens, exposition presque exacte des dogmes et des institutions, vues philosophiques sur l’enchaînement des faits et sur le cours général des choses, il n’y manque rien, si ce n’est des âmes. Il semble, à les lire, que les climats, les institutions, la civilisation, qui transforment l’esprit humain du tout au tout, soient pour lui de simples dehors, des enveloppes accidentelles qui, bien loin de pénétrer jusqu’à son fond, touchent à peine sa superficie. La différence prodigieuse qui sépare les hommes de deux siècles ou de deux races leur échappe\footnote{« Il n’y a plus aujourd’hui de Français, d’Allemands, d’Espagnols, d’Anglais même, quoique l’on dise ; il n’y a plus que des Européens. Tous ont les mêmes goûts, les mêmes passions, les mêmes mœurs, parce qu’aucun n’a reçu de forme nationale par une institution particulière. » (Rousseau, {\itshape Sur le gouvernement de Pologne}, 170.)}. Le Grec ancien, le chrétien des premiers siècles, le conquérant germain, l’homme féodal, l’Arabe de Mahomet, l’Allemand, l’Anglais de la Renaissance, le puritain apparaissent dans leurs livres à peu près comme dans leurs estampes et leurs frontispices, avec quelques différences de costume, mais avec les mêmes corps, les mêmes visages et la même physionomie, atténués, effacés, décents, accommodés aux bienséances. L’imagination sympathique, par laquelle l’écrivain se transporte dans autrui et reproduit en lui-même un système d’habitudes et de passions contraires aux siennes, est le talent qui manque le plus au dix-huitième siècle. Dans la seconde moitié de son cours, sauf chez Diderot qui l’emploie mal et au hasard, elle tarit tout à fait. Considérez tour à tour, pendant la même période, en France et en Angleterre, le genre où elle a son plus large emploi, le roman, sorte de miroir mobile qu’on peut transporter partout et qui est le plus propre à refléter toutes les faces de la nature et de la vie. Quand j’ai lu la série des romanciers anglais, Defoe, Richardson, Fielding, Smollett, Sterne et Goldsmith, jusqu’à Miss Burney et Miss Austen, je connais l’Angleterre du dix-huitième siècle ; j’ai vu des clergymen, des gentilshommes de campagne, des fermiers, des aubergistes, des marins, des gens de toute condition, haute et basse ; je sais le détail des fortunes et des carrières, ce qu’on gagne, ce qu’on dépense, comment l’on voyage, ce qu’on mange et ce qu’on boit ; j’ai en mains une file de biographies circonstanciées et précises, un tableau complet, à mille scènes, de la société tout entière, le plus ample amas de renseignements pour me guider quand je voudrai faire l’histoire de ce monde évanoui. Si maintenant je lis la file correspondante des romanciers français, Crébillon fils, Rousseau, Marmontel, Laclos, Rétif de la Bretonne, Louvet, Mme de Staël, Mme de Genlis et le reste, y compris Mercier et jusqu’à Mme Cottin, je n’ai presque point de notes à prendre ; les petits faits positifs et instructifs sont omis ; je vois des politesses, des gentillesses, des galanteries, des polissonneries, des dissertations de société, et puis c’est tout. On se garde bien de me parler d’argent, de me donner des chiffres, de me raconter un mariage, un procès, l’administration d’une terre ; j’ignore la situation d’un curé, d’un seigneur rural, d’un prieur résident, d’un régisseur, d’un intendant. Tout ce qui concerne la province et la campagne, la bourgeoisie et la boutique\footnote{Avant 1750, on en trouve quelque chose dans {\itshape Gil Blas} et dans {\itshape Marianne} (Mme Dufour la lingère et sa boutique). — Par malheur le travestissement espagnol empêche les romans de Lesage d’être aussi instructifs qu’il le faudrait.}, l’armée et le soldat, le clergé et les couvents, la justice et la police, le négoce et le ménage, reste vague ou devient faux ; pour y démêler quelque chose, il me faut recourir à ce merveilleux Voltaire qui, lorsqu’il a mis bas le grand habit classique, a ses coudées franches et dit tout. Sur les organes les plus vitaux de la société, sur les règles et les pratiques qui vont provoquer une révolution, sur les droits féodaux et la justice seigneuriale, sur le recrutement et l’intérieur des monastères, sur les douanes de province, les corporations et les maîtrises, sur la dîme et la corvée\footnote{ \noindent On rencontre de bons détails dans les petits contes de Diderot, par exemple dans {\itshape les Deux Amis de Bourbonne.} Mais ailleurs, notamment dans {\itshape la Religieuse}, il est homme de parti et donne une impression fausse.
 }, la littérature ne m’apprend presque rien. Il semble que pour elle il n’y ait que des salons et des gens de lettres. Le reste est non avenu ; au-dessous de la bonne compagnie qui cause, la France paraît vide  Quand viendra la Révolution, le retranchement sera plus grand encore. Parcourez les harangues de tribune et le club, les rapports, les motifs de loi, les pamphlets, tant d’écrits inspirés par des événements présents et poignants ; nulle idée de la créature humaine telle qu’on l’a sous les yeux, dans les champs et dans la rue ; on se la figure toujours comme un automate simple, dont le mécanisme est connu. Chez les écrivains, elle était tout à l’heure une serinette à phrases ; pour les politiques, elle est maintenant une serinette à votes, qu’il suffit de toucher du doigt à l’endroit convenable pour lui faire rendre la réponse qui convient. Jamais de faits ; rien que des abstractions, des enfilades de sentences sur la nature, la raison, le peuple, les tyrans, la liberté, sortes de ballons gonflés et entrechoqués inutilement dans les espaces. Si l’on ne savait pas que tout cela aboutit à des effets pratiques et terribles, on croirait à un jeu de logique, à des exercices d’école, à des parades d’académie, à des combinaisons d’idéologie. En effet, c’est l’idéologie, dernier produit du siècle, qui va donner de l’esprit classique la formule finale et le dernier mot.

\section[{III. Conformité de la méthode philosophique. — L’idéologie. — Abus du procédé mathématique. — Condillac, Rousseau, Mably, Condorcet, Volney, Siéyès, Cabanis, Tracy. — Excès des simplifications et témérité des constructions.}]{III. Conformité de la méthode philosophique. — L’idéologie. — Abus du procédé mathématique. — Condillac, Rousseau, Mably, Condorcet, Volney, Siéyès, Cabanis, Tracy. — Excès des simplifications et témérité des constructions.}

\noindent Suivre en toute recherche, avec toute confiance, sans réserve ni précaution, la méthode des mathématiciens ; extraire, circonscrire, isoler quelques notions très simples et très générales ; puis, abandonnant l’expérience, les comparer, les combiner, et, du composé artificiel ainsi obtenu, déduire par le pur raisonnement toutes les conséquences qu’il enferme : tel est le procédé naturel de l’esprit classique. Il lui est si bien inné, qu’on le rencontre également dans les deux siècles, chez Descartes, Malebranche\footnote{ \noindent « Pour atteindre à la vérité, il suffit de se rendre attentif aux idées claires que chacun trouve en lui-même. » (Malebranche, {\itshape Recherche de la vérité}, liv. I, ch. I.) — « Ces longues chaînes de raisons, toutes simples et faciles, dont les géomètres ont coutume de se servir pour parvenir à leurs plus difficiles démonstrations, m’avaient donné occasion de m’imaginer que {\itshape toutes les choses} qui peuvent tomber sous la connaissance des hommes s’entresuivent de même. » (Descartes, {\itshape Discours de la méthode}, I, 142.) — Au dix-septième siècle, on construit {\itshape a priori} avec des idées, au dix-huitième siècle avec des sensations, mais toujours par le même procédé, qui est celui des mathématiques et qui s’étale tout entier dans l’{\itshape Éthique} de Spinosa.
 } et les partisans des idées pures, comme chez les partisans de la sensation, du besoin physique, de l’instinct primitif, Condillac, Rousseau, Helvétius, plus tard Condorcet, Volney, Siéyès, Cabanis et Destutt de Tracy. Ceux-ci ont beau se dire sectateurs de Bacon et rejeter les idées innées ; avec un autre point de départ que les cartésiens, ils marchent dans la même voie, et, comme les cartésiens, après un léger emprunt, ils laissent là l’expérience. Dans cet énorme monde moral et social, dans cet arbre humain aux racines et aux branches innombrables, ils détachent l’écorce visible, une superficie ; ils ne peuvent pénétrer ni saisir au-delà ; leurs mains ne sauraient contenir davantage. Ils ne soupçonnent pas qu’il y ait rien de plus ; l’esprit classique n’a que des prises courtes, une compréhension bornée. Pour eux, l’écorce est l’arbre entier, et, l’opération faite, ils s’éloignent avec l’épiderme sec et mort, sans plus jamais revenir au tronc. Par insuffisance d’esprit et par amour-propre littéraire, ils omettent le détail caractéristique, le fait vivant, l’exemple circonstancié, le spécimen significatif, probant et complet. Il n’y en a presque aucun dans la {\itshape Logique} et dans le {\itshape Traité des sensations} de Condillac, dans l’{\itshape Idéologie} de Destutt de Tracy, dans les {\itshape Rapports du physique et du moral} de Cabanis\footnote{Voir notamment son mémoire : {\itshape De l’influence du climat sur les habitudes morales}, si vague, si totalement dénué d’exemples, sauf une citation d’Hippocrate.}. Jamais, avec eux, on n’est sur le terrain palpable et solide de l’observation personnelle et racontée, mais toujours en l’air, dans la région vide des généralités pures. Condillac déclare que le procédé de l’arithmétique convient à la psychologie et qu’on peut démêler les éléments de notre pensée par une opération analogue « à la règle de trois ». Siéyès a le plus profond dédain pour l’histoire, et « la politique est pour lui une science qu’il croit avoir achevée\footnote{ \noindent Ce sont là les propres paroles de Siéyès  Ailleurs il ajoute : « Les prétendues vérités historiques n’ont pas plus de réalité que les prétendues vérités religieuses. » ({\itshape Papiers de Siéyès}, année 1772, d’après Sainte-Beuve, {\itshape Causeries du lundi}, V, 194.) — Descartes et Malebranche avaient déjà ce mépris pour l’histoire.
 } » du premier coup, par un effort de tête, à la façon de Descartes, qui trouva ainsi la géométrie analytique. Destutt de Tracy, voulant commenter Montesquieu, découvre que le grand historien s’est tenu trop servilement attaché à l’histoire, et il refait l’ouvrage en construisant la société qui doit être au lieu de regarder la société qui est. — Jamais, avec un aussi mince extrait de la nature humaine, on n’a bâti des édifices si réguliers et si spécieux. Avec la sensation Condillac anime une statue, puis, par une suite de purs raisonnements, poursuivant tour à tour dans l’odorat, dans le goût, dans l’ouïe, dans la vue, dans le toucher, les effets de la sensation qu’il suppose, il construit de toutes pièces une âme humaine. Au moyen d’un contrat, Rousseau fonde l’association politique, et, de cette seule donnée, il déduit la constitution, le gouvernement et les lois de toute société équitable. Dans un livre qui est comme le testament philosophique du siècle\footnote{Condorcet, {\itshape Esquisse d’un tableau historique de l’esprit humain}, neuvième époque.}, Condorcet déclare que cette méthode est « le dernier pas de la philosophie, celui qui a mis en quelque sorte une barrière éternelle entre le genre humain et les vieilles erreurs de son enfance ». — « En l’appliquant à la morale, à la politique, à l’économie politique, on est parvenu à suivre dans les sciences morales une marche presque aussi sûre que dans les sciences naturelles. C’est par elle qu’on a pu découvrir les droits de l’homme. » Comme en mathématiques, on les a déduits d’une seule définition primordiale, et cette définition, pareille aux premières vérités mathématiques, est un fait d’expérience journalière, constaté par tous, évident de soi. — L’école subsistera à travers la Révolution, à travers l’Empire, jusque pendant la Restauration\footnote{ \noindent Voir le {\itshape Tableau historique} présenté à l’Institut par M. J. Chénier en 1808. Son énumération montre que l’esprit classique dominait encore dans toutes les branches de la littérature. — Cabanis n’est mort qu’en 1818, Volney en 1820, Destutt de Tracy et Siéyès en 1836, Daunou en 1840. En 1845, MM. Saphary et Valette professaient encore la philosophie de Condillac dans deux lycées de Paris.
 }, avec la tragédie dont elle est la sœur, avec l’esprit classique qui est leur père commun, puissance primitive et souveraine, aussi dangereuse qu’utile, aussi destructive que créatrice, aussi capable de propager l’erreur que la vérité, aussi étonnante par la rigidité de son code, par l’étroitesse de son joug, par l’uniformité de ses œuvres, que par la durée de son règne et par l’universalité de son ascendant.
\chapterclose


\chapteropen

\chapter[{Chapitre III. Combinaison des deux éléments.}]{Chapitre III. \\
Combinaison des deux éléments.}


\chaptercont

\section[{I. La doctrine, ses prétentions et son caractère. — Autorité nouvelle de la raison dans le gouvernement des choses humaines. — Jusqu’ici ce gouvernement appartenait à la tradition.}]{I. La doctrine, ses prétentions et son caractère. — Autorité nouvelle de la raison dans le gouvernement des choses humaines. — Jusqu’ici ce gouvernement appartenait à la tradition.}

\noindent De l’acquis scientifique que l’on a vu, élaboré par l’esprit que l’on vient de décrire, naquit une doctrine qui parut une révélation et qui, à ce titre, prétendit au gouvernement des choses humaines. Aux approches de 1789, il est admis qu’on vit « dans le siècle des lumières », dans « l’âge de la raison », qu’auparavant le genre humain était dans l’enfance, qu’aujourd’hui il est devenu « majeur ». Enfin la vérité s’est manifestée et, pour la première fois, on va voir son règne sur la terre. Son droit est suprême, puisqu’elle est la vérité. Elle doit commander à tous, car, par nature, elle est universelle. Par ces deux croyances, la philosophie du dix-huitième siècle ressemble à une religion, au puritanisme du dix-septième, au mahométisme du septième. Même élan de foi, d’espérance et d’enthousiasme, même esprit de propagande et de domination, même raideur et même intolérance, même ambition de refondre l’homme et de modeler toute la vie humaine d’après un type préconçu. La doctrine nouvelle aura aussi ses docteurs, ses dogmes, son catéchisme populaire, ses fanatiques, ses inquisiteurs et ses martyrs. Elle parlera aussi haut que les précédentes, en souveraine légitime à qui la dictature appartient de naissance, et contre laquelle toute révolte est un crime ou une folie. Mais elle diffère des précédentes en ce qu’elle s’impose au nom de la {\itshape raison}, au lieu de s’imposer au nom de Dieu.\par
En effet, l’autorité était nouvelle. Jusqu’alors, dans le gouvernement des actions et des opinions humaines, la raison n’avait eu qu’une part subordonnée et petite. Le ressort et la direction venaient d’ailleurs ; la croyance et l’obéissance étaient des héritages ; un homme était chrétien et sujet parce qu’il était né chrétien et sujet  Autour de la philosophie naissante et de la raison qui entreprend son grand examen, il y a des lois observées, un pouvoir reconnu, une religion régnante ; dans cet édifice, toutes les pierres se tiennent, et chaque étage s’appuie sur le précédent. Mais quel est le ciment commun, et où se trouve le fondement premier   Toutes ces règles civiles auxquelles sont assujettis les mariages, les testaments, les successions, les contrats, les propriétés et les personnes, règles bizarres et parfois contradictoires, qui les autorise ? D’abord la coutume immémoriale, différente selon la province, selon le titre de la terre, selon la qualité et la condition de l’individu ; ensuite la volonté du roi qui a fait écrire et qui a sanctionné la coutume  Cette volonté elle-même, cette souveraineté du prince, ce premier des pouvoirs publics, qui l’autorise ? D’abord une possession de huit siècles, un droit héréditaire semblable à celui par lequel chacun jouit de son domaine et de son champ, une propriété fixée dans une famille et transmise d’aîné en aîné, depuis le premier fondateur de l’État jusqu’à son dernier successeur vivant ; ensuite la religion qui ordonne aux hommes de se soumettre aux pouvoirs établis  Cette religion enfin, qui l’autorise ? D’abord une tradition de dix-huit siècles, la série immense des témoignages antérieurs et concordants, la croyance continue des soixante générations précédentes ; ensuite, à l’origine, la présence et les instructions du Christ, puis, au-delà, dès l’origine du monde, le commandement et la parole de Dieu. — Ainsi, dans tout l’ordre social et moral, le passé justifie le présent ; l’antiquité sert de titre, et si, au-dessous de toutes ces assises consolidées par l’âge, on cherche dans les profondeurs souterraines le dernier roc primordial, on le trouve dans la volonté divine. — Pendant tout le dix-septième siècle, cette théorie subsiste encore au fond de toutes les âmes sous forme d’habitude fixe et de respect inné ; on ne la soumet pas à l’examen. On est devant elle comme devant le cœur vivant de l’organisme humain ; au moment d’y porter la main, on recule ; on sent vaguement que, si l’on y touchait, peut-être il cesserait de battre. Les plus indépendants, Descartes en tête, « seraient bien marris » d’être confondus avec ces spéculatifs chimériques qui, au lieu de suivre la grande route frayée par l’usage, se lancent à l’aveugle, en ligne droite, « à travers les montagnes et les précipices ». Non seulement, quand ils livrent leurs croyances au doute méthodique, ils exceptent et mettent à part, comme en un sanctuaire, « les vérités de la foi\footnote{Discours de la méthode.} » ; mais encore le dogme qu’ils pensent avoir écarté demeure en leur esprit, efficace et latent, pour les conduire à leur insu, et faire de leur philosophie une préparation ou une confirmation du christianisme\footnote{Cela est visible chez Descartes et dès son second pas (Théorie de l’esprit pur, idée de Dieu, preuve de son existence, véracité de notre intelligence prouvée par la véracité de Dieu, etc.).}. — En somme, au dix-septième siècle, ce qui fournit les idées mères, c’est la foi, c’est la pratique, c’est l’établissement religieux et politique. Qu’elle l’avoue ou qu’elle l’ignore, la raison n’est qu’un subalterne, un orateur, un metteur en œuvre, que la religion et la monarchie font travailler à leur service. Sauf La Fontaine qui, je crois, est unique en cela comme dans le reste, les plus grands et les plus indépendants, Pascal, Descartes, Bossuet, La Bruyère, empruntent au régime établi leur conception première de la nature, de l’homme, de la société, du droit, du gouvernement\footnote{Pascal, {\itshape Pensées} (sur l’origine de la propriété et des rangs), {\itshape Provinciales} (sur l’homicide et le droit de tuer). — Nicole, {\itshape Deuxième traité de la charité et de l’amour-propre} (sur l’homme naturel et le but de la société). Bossuet ({\itshape Politique tirée de l’Écriture sainte}). La Bruyère ({\itshape des Esprits forts}).}. Tant que la raison se réduit à cet office, son œuvre est celle d’un conseiller d’État, d’un prédicateur extraordinaire que ses supérieurs envoient en tournée et en mission dans le département de la philosophie et de la littérature. Bien loin de détruire, elle consolide ; en effet, jusqu’à la Régence, son principal emploi, consiste à faire de bons chrétiens et de fidèles sujets.\par
Mais voici que les rôles s’intervertissent ; du premier rang, la tradition descend au second, et du second rang, la raison monte au premier. — D’un côté la religion et la monarchie, par leurs excès et leurs méfaits sous Louis XIV, par leur relâchement et leur insuffisance sous Louis XV, démolissent pièce à pièce le fond de vénération héréditaire et d’obéissance filiale qui leur servait de base et qui les soutenait dans une région supérieure, au-dessus de toute contestation et de tout examen ; c’est pourquoi, insensiblement, l’autorité de la tradition décroît et disparaît. De l’autre côté la science, par ses découvertes grandioses et multipliées, construit pièce à pièce le fond de confiance et de déférence universelles qui, de l’état de curiosité intéressante, l’élève au rang de pouvoir public ; ainsi, par degrés, l’autorité de la raison grandit et prend toute la place. — Il arrive un moment où, la seconde autorité ayant dépossédé la première, les idées mères que la tradition se réservait tombent sous les prises de la raison. L’examen pénètre dans le sanctuaire interdit. Au lieu de s’incliner, on vérifie, et la religion, l’État, la loi, la coutume, bref, tous les organes de la vie morale et de la vie pratique, vont être soumis à l’analyse pour être conservés, redressés ou remplacés, selon ce que la nouvelle doctrine aura prescrit.

\section[{II. Origine, nature et valeur du préjugé héréditaire. — En quoi la coutume, la religion et l’État sont légitimes.}]{II. Origine, nature et valeur du préjugé héréditaire. — En quoi la coutume, la religion et l’État sont légitimes.}

\noindent Rien de mieux, si la doctrine eût été complète, et si la raison, instruite par l’histoire, devenue critique, eût été en état de comprendre la rivale qu’elle remplaçait. Car alors, au lieu de voir en elle une usurpatrice qu’il fallait expulser, elle eût reconnu en elle une sœur aînée à qui l’on doit laisser sa part. Le préjugé héréditaire est une sorte de raison qui s’ignore. Il a ses titres aussi bien que la raison elle-même ; mais il ne sait pas les retrouver ; à la place des bons, il en allègue d’apocryphes. Ses archives sont enterrées ; il faut pour les dégager des recherches dont il n’est pas capable ; elles subsistent pourtant, et aujourd’hui l’histoire les remet en lumière  Quand on le considère de près, on trouve que, comme la science, il a pour source une longue accumulation d’expériences : les hommes, après une multitude de tâtonnements et d’essais, ont fini par éprouver que telle façon de vivre ou de penser était la seule accommodée à leur situation, la plus praticable de toutes, la plus bienfaisante, et le régime ou dogme qui aujourd’hui nous semble une convention arbitraire a d’abord été un expédient avéré de salut public. Souvent même il l’est encore ; à tout le moins, dans ses grands traits, il est indispensable, et l’on peut dire avec certitude que, si dans une société les principaux préjugés disparaissaient tout d’un coup, l’homme, privé du legs précieux que lui a transmis la sagesse des siècles, retomberait subitement à l’état sauvage et redeviendrait ce qu’il fut d’abord, je veux dire un loup inquiet, affamé, vagabond et poursuivi. Il fut un temps où cet héritage manquait ; aujourd’hui encore il y a des peuplades où il manque entièrement\footnote{Cf. Sir John Lubbock, {\itshape Origine de la civilisation. —} Giraud-Teulon, {\itshape les Origines de la famille.}}. Ne pas manger de chair humaine, ne pas tuer les vieillards inutiles ou incommodes, ne pas exposer, vendre ou tuer les enfants dont on n’a que faire, être le seul mari d’une seule femme, avoir horreur de l’inceste et des mœurs contre nature, être le propriétaire unique et reconnu d’un champ distinct, écouter les voix supérieures de la pudeur, de l’humanité, de l’honneur, de la conscience, toutes ces pratiques, jadis inconnues et lentement établies, composent la civilisation des âmes. Parce que nous les acceptons de confiance, elles n’en sont pas moins saintes, et elles n’en deviennent que plus saintes lorsque, soumises à l’examen et suivies à travers l’histoire, elles se révèlent à nous comme la force secrète qui, d’un troupeau de brutes, a fait une société d’hommes  En général, plus un usage est universel et ancien, plus il est fondé sur des motifs profonds, motifs de physiologie, d’hygiène, de prévoyance sociale. Tantôt, comme dans la séparation des castes, il fallait conserver pure une race héroïque ou pensante, en prévenant les mélanges par lesquels un sang inférieur lui eût apporté la débilité mentale et les instincts bas\footnote{Principe des castes dans l’Inde ; contraste des Aryens et des aborigènes, Soudras et Parias.}. Tantôt, comme dans l’interdiction des spiritueux ou des viandes, il fallait s’accommoder au climat qui prescrivait un régime végétal ou au tempérament de la race pour qui les boissons fortes étaient funestes\footnote{D’après ce principe, aux îles Hawaï, les habitants ont porté une loi qui défend de vendre des spiritueux aux indigènes et qui permet d’en vendre aux Européens. (Ch. de Varigny, {\itshape Quatorze ans aux îles Sandwich.})}. Tantôt, comme dans l’institution du droit d’aînesse, il fallait former et désigner d’avance le commandant militaire auquel obéirait la bande, ou le chef civil qui conserverait le domaine, conduirait l’exploitation et soutiendrait la famille\footnote{Cf. Le Play, {\itshape de l’Organisation de la famille} (Histoire d’un domaine dans les Pyrénées).}  S’il y a des raisons valables pour légitimer la coutume, il y en a de supérieures pour consacrer la religion. Considérez-la, non pas en général et d’après une notion vague, mais sur le vif, à sa naissance, dans les textes, en prenant pour exemple une de celles qui maintenant règnent sur le monde, christianisme, brahmanisme, loi de Mahomet ou de Bouddha. À certains moments critiques de l’histoire, des hommes, sortant de leur petite vie étroite et routinière, ont saisi par une vue d’ensemble l’univers infini ; la face auguste de la nature éternelle s’est dévoilée tout d’un coup ; dans leur émotion sublime, il leur a semblé qu’ils apercevaient son principe ; du moins ils en ont aperçu quelques traits. Et, par une rencontre admirable, ces traits étaient justement les seuls que leur siècle, leur race, un groupe de races, un fragment de l’humanité fût en état de comprendre. Leur point de vue était le seul auquel les multitudes échelonnées au-dessous d’eux pouvaient se mettre. Pour des millions d’hommes, pour des centaines de générations, il n’y avait accès que par leur voie aux choses divines. Ils ont prononcé la parole unique, héroïque ou tendre, enthousiaste ou assoupissante, la seule qu’autour d’eux et après eux le cœur et l’esprit voulussent entendre, la seule qui fût adaptée à des besoins profonds, à des aspirations accumulées, à des facultés héréditaires, à toute une structure mentale et morale, là-bas à celle de l’Indou ou du Mongol, ici à celle du Sémite ou de l’Européen, dans notre Europe à celle du Germain, du Latin ou du Slave ; en sorte que ses contradictions, au lieu de la condamner, la justifient, puisque sa diversité produit son adaptation, et que son adaptation produit ses bienfaits  Cette parole n’est pas une formule nue. Un sentiment si grandiose, une divination si compréhensive et si pénétrante, une pensée par laquelle l’homme embrassant l’immensité et la profondeur des choses, dépasse de si loin les bornes ordinaires de sa condition mortelle, ressemble à une illumination ; elle se change aisément en vision, elle n’est jamais loin de l’extase, elle ne peut s’exprimer que par des symboles, elle évoque les figures divines\footnote{Voir notamment dans la littérature brahmanique les grands poèmes métaphysiques et les {\itshape Pouranas.}}. La religion est de sa nature un poème métaphysique accompagné de croyance. C’est à ce titre qu’elle est efficace et populaire ; car, sauf pour une élite imperceptible, une pure idée n’est qu’un mot vide, et la vérité, pour devenir sensible, est obligée de revêtir un corps. Il lui faut un culte, une légende, des cérémonies, afin de parler au peuple, aux femmes, aux enfants, aux simples, à tout homme engagé dans la vie pratique, à l’esprit humain lui-même dont les idées, involontairement, se traduisent en images. Grâce à cette forme palpable, elle peut jeter son poids énorme dans la conscience, contrebalancer l’égoïsme naturel, enrayer l’impulsion folle des passions brutales, emporter la volonté vers l’abnégation et le dévouement, arracher l’homme à lui-même pour le mettre tout entier au service de la vérité ou au service d’autrui, faire des ascètes et des martyrs, des sœurs de charité et des missionnaires. Ainsi, dans toute société, la religion est un organe à la fois précieux et naturel. D’une part, les hommes ont besoin d’elle pour penser l’infini et pour bien vivre ; si elle manquait tout d’un coup, il y aurait dans leur âme un grand vide douloureux et ils se feraient plus de mal les uns aux autres. D’autre part, on essayerait en vain de l’arracher ; les mains qui se porteraient sur elle n’atteindraient que son enveloppe ; elle repousserait après une opération sanglante ; son germe est trop profond pour qu’on puisse l’extirper. — Si enfin, après la religion et la coutume, nous envisageons l’État, c’est-à-dire le pouvoir armé qui a la force physique en même temps que l’autorité morale, nous lui trouvons une source presque aussi noble. En Europe du moins, de la Russie au Portugal, et de la Norvège aux Deux-Siciles, il est par origine et par essence un établissement militaire où l’héroïsme s’est fait le champion du droit. Çà et là, dans le chaos des races mélangées et des sociétés croulantes, un homme s’est rencontré qui, par son ascendant, a rallié autour de lui une bande de fidèles, chassé les étrangers, dompté les brigands, rétabli la sécurité, restauré l’agriculture, fondé la patrie et transmis comme une propriété à ses descendants son emploi de justicier héréditaire et de général-né. Par cette délégation permanente, un grand office public est soustrait aux compétitions, fixé dans une famille, séquestré en des mains sûres ; désormais la nation possède un centre vivant, et chaque droit trouve un protecteur visible. Si le prince se renferme dans ses attributions, s’il est retenu sur la pente de l’arbitraire, s’il ne verse pas dans l’égoïsme, il fournit au pays l’un des meilleurs gouvernements que l’on ait vus dans le monde, non seulement le plus stable, le plus capable de suite, le plus propre à maintenir ensemble vingt ou trente millions d’hommes, mais encore l’un des plus beaux, puisque le dévouement y ennoblit le commandement et l’obéissance, et que, par un prolongement de la tradition militaire, la fidélité et l’honneur rattachent de grade en grade le chef à son devoir et le soldat à son chef. — Tels sont les titres très valables du préjugé héréditaire ; on voit qu’il est, comme l’instinct, une forme aveugle de la raison. Et ce qui achève de le légitimer, c’est que, pour devenir efficace, la raison elle-même doit lui emprunter sa forme. Une doctrine ne devient active qu’en devenant aveugle. Pour entrer dans la pratique, pour prendre le gouvernement des âmes, pour se transformer en un ressort d’action, il faut qu’elle se dépose dans les esprits à l’état de croyance faite, d’habitude prise, d’inclination établie, de tradition domestique, et que, des hauteurs agitées de l’intelligence, elle descende et s’incruste dans les bas-fonds immobiles de la volonté ; alors seulement elle fait partie du caractère et devient une force sociale. Mais, du même coup, elle a cessé d’être critique et clairvoyante ; elle ne tolère plus les contradictions ou le doute, elle n’admet plus les restrictions ni les nuances ; elle ne sait plus ou elle apprécie mal ses preuves. Nous croyons aujourd’hui au progrès indéfini à peu près comme on croyait jadis à la chute originelle ; nous recevons encore d’en haut nos opinions toutes faites, et l’Académie des sciences tient à beaucoup d’égards la place des anciens conciles. Toujours, sauf chez quelques savants spéciaux, la croyance et l’obéissance seront irréfléchies, et la raison s’indignerait à tort de ce que le préjugé conduit les choses humaines, puisque, pour les conduire, elle doit elle-même devenir un préjugé.

\section[{III. La raison classique ne peut se mettre à ce point de vue. — Les titres passés et présents de la tradition sont méconnus. — La raison entreprend de la détruire.}]{III. La raison classique ne peut se mettre à ce point de vue. — Les titres passés et présents de la tradition sont méconnus. — La raison entreprend de la détruire.}

\noindent Par malheur, au dix-huitième siècle, la raison était classique, et les aptitudes aussi bien que les documents lui manquaient pour comprendre la tradition. — D’abord on ignorait l’histoire ; l’érudition rebutait parce qu’elle est ennuyeuse et lourde ; on dédaignait les doctes compilations, les grands recueils de textes, le lent travail de la critique. Voltaire raillait les Bénédictins. Pour faire passer son {\itshape Esprit des lois}, Montesquieu faisait de l’esprit sur les lois. Raynal, afin de donner la vogue à son histoire du commerce dans les Indes, avait le soin d’y coudre les déclamations de Diderot. L’abbé Barthélemy devait étaler l’uniformité de son vernis littéraire sur la vérité des mœurs grecques. La science était tenue d’être épigrammatique ou oratoire ; le détail technique ou cru aurait déplu à un public de gens du monde ; le beau style omettait ou faussait les petits faits significatifs qui donnent aux caractères anciens leur tour propre et leur relief original. — Quand même on aurait osé les noter, on n’en aurait pas démêlé le sens et la portée. L’imagination sympathique était absente ; on ne savait pas sortir de soi-même, se transporter en des points de vue distants, se figurer les états étranges et violents de l’esprit humain, les moments décisifs et féconds pendant lesquels il enfante une créature viable, une religion destinée à l’empire, un État qui doit durer. L’homme n’imagine rien qu’avec son expérience, et dans quelle portion de leur expérience les gens de ce monde auraient-ils trouvé des matériaux pour imaginer les convulsions de l’accouchement ? Comment des esprits aussi policés et aussi aimables auraient-ils pu épouser les sentiments d’un apôtre, d’un moine, d’un fondateur barbare ou féodal, les voir dans le milieu qui les explique et les justifie, se représenter la foule environnante, d’abord des âmes désolées, hantées par le rêve mystique, puis des cerveaux bruts et violents, livrés à l’instinct et aux images, qui pensaient par demi-visions, et qui pour volonté avaient des impulsions irrésistibles ? La raison raisonnante ne concevait pas de pareilles figures ; pour les faire rentrer dans son cadre rectiligne, il fallait les réduire et les refaire ; le Macbeth de Shakespeare devenait celui de Ducis, et le Mahomet du Coran, celui de Voltaire. Par suite, faute de voir les âmes, on méconnaissait les institutions ; on ne soupçonnait pas que la vérité n’avait pu s’exprimer que par la légende, que la justice n’avait pu s’établir que par la force, que la religion avait dû revêtir la forme sacerdotale, que l’État avait dû prendre la forme militaire, et que l’édifice gothique avait, aussi bien qu’un autre, son architecture, ses proportions, son équilibre, sa solidité, son utilité et même sa beauté. — Par suite encore, faute de comprendre le passé, on ne comprenait pas le présent. On n’avait aucune idée juste du paysan, de l’ouvrier, du bourgeois provincial ou même du petit noble de campagne ; on ne les apercevait que de loin, demi-effacés, tout transformés par la théorie philosophique et par le brouillard sentimental. « Deux ou trois mille\footnote{ Voltaire, {\itshape Dictionnaire Philosophique}, article {\itshape Supplices.}
 } » gens du monde et lettrés faisaient le cercle des honnêtes gens et ne sortaient pas de leur cercle. Si parfois, de leur château et en voyage, ils avaient entrevu le peuple, c’était en passant, à peu près comme leurs chevaux de poste ou les bestiaux de leurs fermes, avec compassion sans doute, mais sans deviner ses pensées troubles et ses instincts obscurs. On n’imaginait pas la structure de son esprit encore primitif, la rareté et la ténacité de ses idées, l’étroitesse de sa vie routinière, machinale, livrée au travail manuel, absorbée par le souci du pain quotidien, confinée dans les limites de l’horizon visible, son attachement au saint local, aux rites, au prêtre, ses rancunes profondes, sa défiance invétérée, sa crédulité fondée sur l’imagination, son incapacité de concevoir le droit abstrait et les événements publics, le sourd travail par lequel les nouvelles politiques se transformaient dans sa tête en contes de revenant ou de nourrice, ses affolements contagieux pareils à ceux des moutons, ses fureurs aveugles pareilles à celle d’un taureau, et tous ces traits de caractère que la Révolution allait mettre au jour. Vingt millions d’hommes et davantage avaient à peine dépassé l’état mental du moyen âge c’est pourquoi, dans ses grandes lignes, l’édifice social qu’ils pouvaient habiter devait être du moyen âge. Il fallait assainir celui-ci, le nettoyer, y percer des fenêtres, y abattre des clôtures, mais en garder les fondements, le gros œuvre et la distribution générale ; sans quoi, après l’avoir démoli et avoir campé dix ans en plein air, à la façon des sauvages, ses hôtes devaient être forcés de le rebâtir presque sur le même plan. Dans les âmes incultes qui ne sont point arrivées jusqu’à la réflexion, la croyance ne s’attache qu’au symbole corporel et l’obéissance ne se produit que par la contrainte physique ; il n’y a de religion que par le curé et d’État que par le gendarme  Un seul écrivain, Montesquieu, le mieux instruit, le plus sagace et le plus équilibré de tous les esprits du siècle, démêlait ces vérités, parce qu’il était à la fois érudit, observateur, historien et jurisconsulte. Mais il parlait comme un oracle, par sentences et en énigmes ; il courait, comme sur des charbons ardents, toutes les fois qu’il touchait aux choses de son pays et de son temps. C’est pourquoi il demeurait respecté, mais isolé, et sa célébrité n’était point une influence  La raison classique refusait\footnote{ {\itshape Résumé des cahiers}, par Prudhomme, {\itshape Préface}, 1789.
 } d’aller si loin pour étudier si péniblement l’homme ancien et l’homme actuel. Elle trouvait plus court et plus commode de suivre sa pente originelle, de fermer les yeux sur l’homme réel, de rentrer dans son magasin de notions courantes, d’en tirer la notion de l’homme en général, et de bâtir là-dessus dans les espaces  Par cet aveuglement naturel et définitif, elle cesse de voir les racines antiques et vivantes des institutions contemporaines ; ne les voyant plus, elle nie qu’il y en ait. Pour elle, le préjugé héréditaire devient un préjugé pur ; la tradition n’a plus de titres, et sa royauté n’est qu’une usurpation. Voilà désormais la raison armée en guerre contre sa devancière, pour lui arracher le gouvernement des âmes et pour substituer au règne du mensonge le règne de la vérité.

\section[{IV. Deux stades dans cette opération. — Premier stade, Voltaire, Montesquieu, les déistes et les réformateurs. — Ce qu’ils détruisent et ce qu’ils respectent.}]{IV. Deux stades dans cette opération. — Premier stade, Voltaire, Montesquieu, les déistes et les réformateurs. — Ce qu’ils détruisent et ce qu’ils respectent.}

\noindent Dans cette grande expédition, il y a deux étapes. Par bon sens ou par timidité, les uns s’arrêtent à mi-chemin. Par passion ou par logique, les autres vont jusqu’au bout  Une première campagne enlève à l’ennemi ses défenses extérieures et ses forteresses de frontière ; c’est Voltaire qui conduit l’armée philosophique. Pour combattre le préjugé héréditaire, on lui en oppose d’autres dont l’empire est aussi étendu et dont l’autorité n’est pas moins reconnue. Montesquieu regarde la France par les yeux d’un Persan, et Voltaire, revenant d’Angleterre, décrit les Anglais, espèce inconnue. En face du dogme et du culte régnants, on développe, avec une ironie ouverte ou déguisée, ceux des diverses sectes chrétiennes, anglicans, quakers, presbytériens, sociniens, ceux des peuples anciens ou lointains, Grecs, Romains, Égyptiens, Mahométans, Guèbres, adorateurs de Brahma, Chinois, simples idolâtres. En regard de la loi positive et de la pratique établie, on expose, avec des intentions visibles, les autres constitutions et les autres mœurs, despotisme, monarchie limitée, république, ici l’Église soumise à l’État, là-bas l’Église détachée de l’État, en tel pays des castes, dans tel autre la polygamie, et, de contrée à contrée, de siècle à siècle, la diversité, la contradiction, l’antagonisme de coutumes fondamentales qui, chacune chez elle, sont toutes également consacrées par la tradition et forment toutes légitimement le droit public. Dès ce moment, le charme est rompu. Les antiques institutions perdent leur prestige divin ; elles ne sont plus que des œuvres humaines, fruits du lieu et du moment, nées d’une convenance et d’une convention. Le scepticisme entre par toutes les brèches. À l’endroit du christianisme, il se change tout de suite en hostilité pure, en polémique prolongée et acharnée ; car, à titre de religion d’État, celui-ci occupe la place, censure la libre pensée, fait brûler les écrits, exile, emprisonne, ou inquiète les auteurs, et se trouve partout l’adversaire naturel et officiel. En outre, à titre de religion ascétique, il condamne, non seulement les mœurs gaies et relâchées que la nouvelle philosophie tolère, mais encore les penchants naturels qu’elle autorise et les promesses de bonheur terrestre qu’elle fait briller à tous les regards. Ainsi contre lui le cœur et l’esprit sont d’accord  Les textes dans la main, Voltaire le poursuit d’un bout à l’autre de son histoire, depuis les premiers récits bibliques jusqu’aux dernières bulles, avec une animosité et une verve implacables, en critique, en historien, en géographe, en logicien, en moraliste, contrôlant les sources, opposant les témoignages, enfonçant le ridicule, comme un pic, dans tous endroits faibles où l’instinct révolté heurte sa prison mystique, et dans tous les endroits douteux où des placages ultérieurs ont défiguré l’édifice primitif  Mais il en respecte la première assise, et en cela les plus grands écrivains du siècle feront comme lui. Sous les religions positives qui sont fausses, il y a la religion naturelle qui est vraie. Elle est le texte authentique et simple dont les autres sont les traductions altérées et amplifiées. Otez les surcharges ultérieures et divergentes ; il reste l’original, et cet extrait commun, par lequel toutes les copies concordent, est le déisme  Même opération sur les lois civiles et politiques. En France, où tant d’institutions survivent à leur utilité, où les privilèges ne sont plus justifiés par les services, où les droits se sont changés en abus, quelle architecture incohérente que celle de la vieille maison gothique ! Comme elle est mal faite pour un peuple moderne ! À quoi bon, dans un État uni et unique, tous ces compartiments féodaux qui séparent les ordres, les corporations, les provinces ? Un archevêque suzerain d’une demi-province, un chapitre propriétaire de douze mille serfs, un abbé de salon bien renté sur un monastère qu’il n’a jamais vu, un seigneur largement pensionné pour figurer dans les antichambres, un magistrat qui achète le droit de rendre la justice, un colonel qui sort du collège pour venir commander son régiment héréditaire, un négociant de Paris qui, ayant loué pour un an une maison de Franche-Comté, aliène par cela seul la propriété de ses biens et de sa personne, quels paradoxes vivants ! Et, dans toute l’Europe, il y en a de pareils. Ce qu’on peut dire de mieux en faveur « d’une nation policée\footnote{ Voltaire, {\itshape Dialogues, Entretiens entre A, B, C.}
 } », c’est que ses lois, coutumes et pratiques se composent « pour moitié d’abus, et pour « moitié d’usages tolérables »  Mais sous ces législations positives qui toutes se contredisent entre elles et dont chacune se contredit elle-même, il est une loi naturelle sous-entendue dans les codes, appliquée dans les mœurs, écrite dans les cœurs. « Montrez-moi un pays où il soit honnête de me ravir le fruit de mon travail, de violer sa promesse, de mentir pour nuire, de calomnier, d’assassiner, d’empoisonner, d’être ingrat envers son bienfaiteur, de battre son père et sa mère quand ils vous présentent à manger. » — « Ce qui est juste ou injuste paraît tel à l’univers entier », et, dans la pire société, toujours la force se met à quelques égards au service du droit, de même que, dans la pire religion, toujours le dogme extravagant proclame en quelque façon un architecte suprême  Ainsi les religions et les sociétés, dissoutes par l’examen, laissent apercevoir au fond du creuset, les unes un résidu de vérité, les autres un résidu de justice, reliquat petit, mais précieux, sorte de lingot d’or que la tradition conserve, que la raison épure, et qui, peu à peu, dégagé de ses alliages, élaboré, employé à tous les usages, doit fournir seul toute la substance de la religion et tous les fils de la société.

\section[{V. Deuxième stade, le retour à la nature. — Diderot, d’Holbach et les matérialistes. — Théorie de la matière vivante et de l’organisation spontanée. — Morale de l’instinct animal et de l’intérêt bien entendu.}]{V. Deuxième stade, le retour à la nature. — Diderot, d’Holbach et les matérialistes. — Théorie de la matière vivante et de l’organisation spontanée. — Morale de l’instinct animal et de l’intérêt bien entendu.}

\noindent Ici commence la seconde expédition philosophique. Elle se compose de deux armées : la première est celle des Encyclopédistes, les uns sceptiques comme d’Alembert, les autres à demi panthéistes comme Diderot et Lamarck, d’autres francs athées et matérialistes secs comme d’Holbach, La Mettrie, Helvétius, plus tard Condorcet, Lalande et Volney, tous divers et indépendants les uns des autres, mais tous unanimes en ceci, que la tradition est l’ennemi. Tel est l’effet des hostilités prolongées : en durant, la guerre s’exaspère ; on veut tout prendre, pousser l’adversaire à bout, le chasser de tous ses postes. On refuse d’admettre que la raison et la tradition puissent ensemble et d’accord défendre la même citadelle ; dès que l’une entre, il faut que l’autre sorte ; désormais un préjugé s’est établi contre le préjugé  À la vérité, Voltaire « le patriarche » ne veut pas se départir de son Dieu rémunérateur et vengeur\footnote{Voltaire, {\itshape Dictionnaire Philosophique}, article {\itshape Religion.} « Si vous avez une bourgade à gouverner, il faut qu’elle ait une religion. »} ; tolérons en lui ce reste de superstition en souvenir de ses grands services ; mais considérons en hommes le fantôme qu’il regarde avec des yeux d’enfant. Nous le recevons dans notre esprit par la foi, et la foi est toujours suspecte. Il a été forgé par l’ignorance, par la crainte, par l’imagination, toutes puissances trompeuses. Il n’était d’abord que le fétiche d’un sauvage ; vainement nous l’avons épuré et agrandi, il se sent toujours de ses origines ; son histoire est celle d’un songe héréditaire qui, né dans le cerveau affolé et brut, s’est prolongé de générations en générations, et dure encore dans le cerveau cultivé et sain. Voltaire veut que ce rêve soit vrai, parce qu’autrement il ne peut expliquer le bel arrangement du monde et qu’une horloge suppose un horloger ; il faudrait d’abord prouver que le monde est une horloge et chercher si l’arrangement, tel quel, incomplet, qu’on y observe ne s’explique pas mieux par une supposition plus simple et plus conforme à l’expérience, celle d’une matière éternelle en qui le mouvement est éternel. Des particules mobiles et mouvantes dont les diverses sortes ont divers états d’équilibre, voilà les minéraux, la substance inanimée, marbre, chaux, air, eau, charbon\footnote{ \href{http://classiques.uqac.ca/classiques/Diderot\_denis/d\_Alembert/d\_alembert\_2\_reve/reve\_d\_alembert.html}{\dotuline{{\itshape Le rêve de} {\itshape d’} {\itshape Alembert}}} [\url{http://classiques.uqac.ca/classiques/Diderot\_denis/d\_Alembert/d\_alembert\_2\_reve/reve\_d\_alembert.html}], par Diderot, {\itshape passim.}
 }. J’en fais de l’humus, « j’y sème des pois, des fèves, des choux » ; les plantes se nourrissent de l’humus « et je me nourris des plantes ». À chacun de mes repas, en moi, par moi, une matière inanimée devient vivante ; « j’en fais de la chair, je l’animalise, je la rends sensible ». Il y avait en elle une sensibilité latente, incomplète, qui s’achève et devient manifeste. L’organisation est la cause, la vie et la sensation sont les effets ; je n’ai pas besoin d’une monade spirituelle pour expliquer les effets puisque je tiens la cause. « Voyez cet œuf, c’est avec cela qu’on renverse toutes les écoles de théologie et tous les temples de la terre. Qu’est-ce que cet œuf ? Une masse insensible avant que le germe y soit introduit. Et après que le germe y est introduit, qu’est-ce encore ? Une masse insensible, un fluide inerte. » Ajoutez-y de la chaleur, tenez le tout dans un four, laissez l’opération se faire : vous aurez un poulet, c’est-à-dire « de la sensibilité, de la vie, de la mémoire, de la conscience, des passions, de la pensée ». Ce que vous appelez l’âme, c’est le centre nerveux auquel aboutissent tous les filets sensibles. Les vibrations qu’ils lui transmettent font ses sensations ; une sensation réveillée ou renaissante est un souvenir ; des sensations, des souvenirs et des signes, font toutes nos idées. Ainsi, ce n’est pas une intelligence qui arrange la matière, c’est la matière qui en s’arrangeant produit les intelligences. Mettons donc l’intelligence où elle est, dans le corps organisé ; n’allons pas la détacher de son support, pour la jucher dans le ciel, sur un trône imaginaire. Car cet hôte disproportionné, une fois introduit dans notre esprit, finit par déconcerter le jeu naturel de nos sentiments, et, comme un parasite monstrueux, tire à soi toute notre substance\footnote{« Si un misanthrope s’était proposé de faire le malheur du genre humain, qu’aurait-il pu inventer de mieux que la croyance en un être incompréhensible, sur lequel les hommes n’auraient jamais pu s’entendre, et auquel ils auraient attaché plus d’importance qu’à leur propre vie ? » Diderot, \href{http://classiques.uqac.ca/classiques/Diderot\_denis/entretien\_philo\_marechale/entretien\_marechale.html}{\dotuline{{\itshape Entretien d’un philosophe avec la Maréchale de…}}} [\url{http://classiques.uqac.ca/classiques/Diderot\_denis/entretien\_philo\_marechale/entretien\_marechale.html}]}. Le premier intérêt de l’homme sain est de s’en délivrer, d’écarter toute superstition, toute « crainte de puissances invisibles\footnote{Cf. {\itshape Catéchisme universel}, par Saint-Lambert, et la {\itshape Loi naturelle ou} \href{http://gallica.bnf.fr/ark:/12148/bpt6k81624h}{\dotuline{{\itshape Catéchisme du citoyen} {\itshape français}}} [\url{http://gallica.bnf.fr/ark:/12148/bpt6k81624h}], par Volney.} »  Alors seulement il peut fonder une morale, démêler « la loi naturelle ». Puisque le ciel est vide, nous n’avons plus besoin de la chercher dans un commandement d’en-haut. Regardons en bas sur la terre ; considérons l’homme lui-même, tel qu’il est aux yeux du naturaliste, c’est-à-dire le corps organisé, l’animal sensible, avec ses besoins, ses appétits et ses instincts. Non seulement ils sont indestructibles, mais encore ils sont légitimes. Ouvrons la prison où le préjugé les enferme ; donnons-leur l’espace et l’air libre ; qu’ils se déploient dans toute leur force, et tout sera bien. Selon Diderot\footnote{\href{http://classiques.uqac.ca/classiques/Diderot\_denis/voyage\_bougainville/voyage\_bougainville.html}{\dotuline{{\itshape Supplément au voyage de Bougainville}}} [\url{http://classiques.uqac.ca/classiques/Diderot\_denis/voyage\_bougainville/voyage\_bougainville.html}].}, le mariage perpétuel est un abus ; c’est « la tyrannie de l’homme qui a converti en propriété la possession de la femme ». La pudeur, comme le vêtement, est une invention et une convention\footnote{Cf. {\itshape Mémoires de Mme d’Épinay}, conversation avec Duclos et Saint-Lambert chez Mlle Quinault. — Rousseau, \href{http://un2sg4.unige.ch/athena/rousseau/confessions/jjr\_conf\_05.html}{\dotuline{{\itshape Confessions}, première partie, livre V}} [\url{http://un2sg4.unige.ch/athena/rousseau/confessions/jjr\_conf\_05.html}]. — Ce sont là justement les principes enseignés par M. de Tavel à Mme de Warens.}, il n’y a de bonheur et de mœurs que dans les pays où la loi autorise l’instinct, à Otaïti par exemple, où le mariage dure un mois, souvent un jour, parfois un quart d’heure, où l’on se prend et l’on se quitte à volonté, où, par hospitalité, le soir, on offre ses filles et sa femme à son hôte, où le fils épouse la mère par politesse, où l’union des sexes est une fête religieuse que l’on célèbre en public  Et le logicien poussant à bout les conséquences finit par cinq ou six pages « capables de faire dresser les cheveux\footnote{\href{http://classiques.uqac.ca/classiques/Diderot\_denis/d\_Alembert/d\_alembert\_3\_entretien\_fin/entretien\_fin.html}{\dotuline{{\itshape Suite du rêve de d’Alembert}}} [\url{http://classiques.uqac.ca/classiques/Diderot\_denis/d\_Alembert/d\_alembert\_3\_entretien\_fin/entretien\_fin.html}], {\itshape Entretien entre Mlle de Lespinasse et Bordeu. — Mémoires} de Diderot, {\itshape Lettre à Mlle Volant}, III, 66.} », avouant lui-même que sa doctrine « n’est pas bonne à prêcher aux enfants ni aux grandes personnes »  À tout le moins, chez Diderot, ces paradoxes ont des correctifs. Quand il peint les mœurs modernes, c’est en moraliste. Non seulement il connaît toutes les cordes du clavier humain, mais il les classe chacune à son rang. Il aime les sons beaux et purs, il est plein d’enthousiasme pour les harmonies nobles, il a autant de cœur que de génie\footnote{Cf. ses admirables contes, \href{http://classiques.uqac.ca/classiques/Diderot\_denis/entretien\_pere\_enfants/entretien\_pere\_enfants.html}{\dotuline{{\itshape Entretiens d’un père avec ses enfants}}} [\url{http://classiques.uqac.ca/classiques/Diderot\_denis/entretien\_pere\_enfants/entretien\_pere\_enfants.html}] et {\itshape le Neveu de Rameau.}}. Bien mieux, quand il s’agit de démêler les impulsions primitives, il garde, à côté de l’amour-propre, une place indépendante et supérieure pour la pitié, la sympathie, la bienveillance, « la bienfaisance », pour toutes les affections généreuses du cœur qui se donne et se dévoue sans calcul ni retour sur soi  Mais auprès de lui, en voici d’autres, froids et bornés, qui, selon la méthode mathématique des idéologues\footnote{Volney, {\itshape Ibid.} « La loi naturelle… consiste tout entière en faits dont la démonstration peut sans cesse se renouveler aux sens et composer une science aussi précise, aussi exacte que la géométrie et les mathématiques. »}, construisent la morale à la façon de Hobbes. Il ne leur faut qu’un seul mobile, le plus simple et le plus palpable, tout grossier, presque mécanique, tout physiologique, l’inclination naturelle qui porte l’animal à fuir la douleur et à chercher le plaisir. « La douleur et le plaisir, dit Helvétius, sont les seuls ressorts de l’univers moral, et le sentiment de l’amour de soi est la seule base sur laquelle on puisse jeter les fondements d’une morale utile… Quel autre motif que l’intérêt personnel pourrait déterminer un homme à des actions généreuses ? Il lui est aussi impossible d’aimer le bien pour le bien que d’aimer le mal pour le mal\footnote{Helvétius, \href{http://gallica.bnf.fr/ark:/12148/bpt6k88614w}{\dotuline{{\itshape de l’Esprit}}} [\url{http://gallica.bnf.fr/ark:/12148/bpt6k88614w}], passim.}. » — « Les principes de la loi naturelle\footnote{Volney, {\itshape ib.}, ch. III. — Saint-Lambert, {\itshape ib.}, premier dialogue.}, disent les disciples, se réduisent à un principe fondamental et unique, la conservation de soi-même. » « Se conserver, obtenir le bonheur », voilà l’instinct, le droit et le devoir. « Ô vous\footnote{Baron d’Holbach, \href{http://gallica.bnf.fr/ark:/12148/bpt6k88620t}{\dotuline{{\itshape Système de la nature}}} [\url{http://gallica.bnf.fr/ark:/12148/bpt6k88620t}], II, 408, 493.}, dit la nature, qui, par l’impulsion que je vous donne, tendez vers le bonheur à chaque instant de votre durée, ne résistez pas à ma loi souveraine, travaillez à votre félicité, jouissez sans crainte, soyez heureux. » Mais, pour être heureux, contribuez au bonheur des autres ; si vous voulez qu’ils vous soient utiles, soyez-leur utile ; votre intérêt bien entendu vous commande de les servir. « Depuis la naissance jusqu’à la mort, tout homme a besoin des hommes. » — « Vivez donc pour eux, afin qu’ils vivent pour vous. » — « Soyez bons, parce que la bonté enchaîne tous les cœurs ; soyez doux, parce que la douceur attire l’affection ; soyez modestes, parce que l’orgueil révolte des êtres remplis d’eux-mêmes… Soyez citoyens, parce que la patrie est nécessaire à votre sûreté et à votre bien-être. Défendez votre pays, parce que c’est lui qui vous rend heureux et renferme vos biens. » Ainsi la vertu n’est que l’égoïsme muni d’une longue-vue ; l’homme n’a d’autre raison pour bien faire que la crainte de se faire mal, et, quand il se dévoue, c’est à son intérêt. On va vite et loin sur cette pente. Sitôt que pour chacun l’unique règle est d’être heureux, chacun veut l’être à l’instant, à sa guise ; le troupeau des appétits lâchés se rue en avant et renverse d’abord les barrières. D’autant plus qu’on lui a prouvé que toute barrière est nuisible, inventée par des pâtres rusés et malfaisants pour mieux traire et tondre le troupeau. « L’état de société est un état de guerre du souverain contre tous, et de chacun des membres contre les autres\footnote{Baron d’Holbach, {\itshape Système de la nature}, I, 347.}… Nous ne voyons sur la face du globe que des souverains injustes, incapables, amollis par le luxe, corrompus par la flatterie, dépravés par la licence et l’impunité, dépourvus de talents, de mœurs et de vertus… L’homme est méchant, non parce qu’il est méchant, mais parce qu’on la rendu tel. » — « Voulez-vous\footnote{Diderot, \href{http://classiques.uqac.ca/classiques/Diderot\_denis/voyage\_bougainville/voyage\_bougainville.html}{\dotuline{{\itshape Supplément au voyage de Bougainville}}} [\url{http://classiques.uqac.ca/classiques/Diderot\_denis/voyage\_bougainville/voyage\_bougainville.html}].} savoir l’histoire abrégée de presque toute notre misère ? La voici : Il existait un homme naturel, on a introduit au dedans de cet homme un homme artificiel, et il s’est élevé dans la caverne une guerre civile qui dure toute la vie… Si vous vous proposez d’être son tyran…, empoisonnez-le de votre mieux d’une morale contraire à la nature, faites-lui des entraves de toute espèce, embarrassez ses mouvements de mille obstacles ; attachez-lui des fantômes qui l’effrayent… Le voulez-vous heureux et libre, ne vous mêlez pas de ses affaires… Et demeurez à jamais convaincu que ce n’est pas pour vous, mais pour eux que ces sages législateurs vous ont pétri et maniéré comme vous l’êtes. J’en appelle à toutes les institutions politiques, civiles et religieuses ; examinez-les profondément, et je me trompe fort, ou vous verrez l’espèce humaine pliée de siècle en siècle au joug qu’une poignée de fripons se permettait de lui imposer… Méfiez-vous de celui qui veut mettre l’ordre ; ordonner, c’est toujours se rendre maître des autres en les gênant. » Plus de gêne ; les passions sont bonnes, et, si le troupeau veut enfin manger à pleine bouche, son premier soin sera de fouler sous ses sabots les animaux mitrés et couronnés qui le parquent pour l’exploiter\footnote{ Diderot, {\itshape Les Eleuthéromanes.}\par
  \begin{poem}
« Et ses mains, ourdissant les entrailles du prêtre, \\
 En feraient un cordon pour le dernier des rois. » \\-
\end{poem}
 \par
 \noindent Brissot : \emph{« Le besoin étant notre seul titre de propriété, il en résulte que, lorsqu’il est satisfait, l’homme n’est plus propriétaire… Deux besoins essentiels résultent de la constitution de l’animal, la nutrition et l’évacuation… Les hommes peuvent-ils se nourrir de leurs semblables ? Oui, car les êtres ont droit de se nourrir de toute matière propre à satisfaire leurs besoins… Homme de la nature, suis ton vœu, écoute ton besoin, c’est ton seul maître, ton seul guide. Sens-tu s’allumer dans tes veines un feu secret à l’aspect d’un objet charmant ? Il est à toi, tes caresses sont innocentes, tes baisers sont purs. L’amour est le seul titre de la jouissance, comme la faim l’est de la propriété. »} (Essai publié en 1780, reproduit en 1782 dans la {\itshape Bibliothèque du législateur}, cité par Buchez et Roux, \href{http://gallica.bnf.fr/ark:/12148/bpt6k28879s/f447}{\dotuline{{\itshape Histoire parlementaire}, XIII, 431}} [\url{http://gallica.bnf.fr/ark:/12148/bpt6k28879s/f447}]).
 }.

\section[{VI. Rousseau et les spiritualistes. — Bonté originelle de l’homme. — Erreur de la civilisation. — Injustice de la propriété et de la société.}]{VI. Rousseau et les spiritualistes. — Bonté originelle de l’homme. — Erreur de la civilisation. — Injustice de la propriété et de la société.}

\noindent Retour à la nature, c’est-à-dire abolition de la société : tel est le cri de guerre de tout le bataillon encyclopédique. Voici que d’un autre côté le même cri s’élève ; c’est le bataillon de Rousseau et des socialistes qui, à son tour, vient donner l’assaut au régime établi. La sape que celui-ci pratique au pied des murailles semble plus bornée, mais n’en est que plus efficace, et la machine de destruction qu’il emploie est aussi une idée neuve de la nature humaine. Cette idée, Rousseau l’a tirée tout entière du spectacle de son propre cœur\footnote{Ce sont les propres paroles de Rousseau ({\itshape Rousseau juge de Jean-Jacques}, troisième dialogue, 193). « D’où le peintre et l’apologiste de la nature, aujourd’hui si défigurée et si calomniée, a-t-il pu tirer son modèle, si ce n’est de son propre cœur ? »} : homme étrange, original et supérieur, mais qui, dès l’enfance, portait en soi un germe de folie et qui à la fin devint fou tout à fait ; esprit admirable et mal équilibré, en qui les sensations, les émotions et les images étaient trop fortes : à la fois aveugle et perspicace, véritable poète et poète malade, qui, au lieu des choses, voyait ses rêves, vivait dans un roman et mourut sous le cauchemar qu’il s’était forgé ; incapable de se maîtriser et de se conduire, prenant ses résolutions pour des actes, ses velléités pour des résolutions et le rôle qu’il se donnait pour le caractère qu’il croyait avoir ; en tout disproportionné au train courant du monde, s’aheurtant, se blessant, se salissant à toutes les bornes du chemin ; ayant commis des extravagances, des vilenies et des crimes, et néanmoins gardant jusqu’au bout la sensibilité délicate et profonde, l’humanité, l’attendrissement, le don des larmes, la faculté d’aimer, la passion de la justice, le sentiment religieux, l’enthousiasme, comme autant de racines vivaces où fermente toujours la sève généreuse pendant que la tige et les rameaux avortent, se déforment ou se flétrissent sous l’inclémence de l’air. Comment expliquer un tel contraste ? Comment Rousseau l’explique-t-il lui-même ? Un critique, un psychologue ne verrait là qu’un cas singulier, l’effet d’une structure mentale extraordinaire et discordante, analogue à celle d’Hamlet, de Chatterton, de René, de Werther, propre à la poésie, impropre à la vie. Rousseau généralise : préoccupé de soi jusqu’à la manie et ne voyant dans le monde que lui-même, il imagine l’homme d’après lui-même et « le décrit tel qu’il se sent ». À cela d’ailleurs l’amour-propre trouve son compte ; on est bien aise d’être le type de l’homme ; la statue qu’on se dresse en prend plus d’importance ; on se relève à ses propres yeux quand, en se confessant, on croit confesser le genre humain. Rousseau convoque les générations par la trompette du jugement dernier et s’y présente hardiment aux yeux des hommes et du souverain juge : « Qu’un seul te dise, s’il l’ose : Je fus meilleur que cet homme-là\footnote{ \noindent \href{http://un2sg4.unige.ch/athena/rousseau/confessions/jjr\_conf\_00.html}{\dotuline{{\itshape Confessions}}} [\url{http://un2sg4.unige.ch/athena/rousseau/confessions/jjr\_conf\_00.html}], Livre I, I, et fin du V\textsuperscript{e} livre. — Première lettre à M. de Malesherbes. « Je connais mes grands défauts, et je sens vivement tous mes vices. Avec tout cela, je mourrai persuadé que, de tous les hommes que j’ai connus en ma vie, nul ne fut meilleur que moi. » — À Mme B. 16 mars 1770. « Vous m’avez accordé de l’estime sur mes écrits ; vous m’en accorderiez plus encore sur ma vie si elle vous était connue, et davantage encore sur mon cœur s’il était ouvert à vos yeux. Il n’en fut jamais un meilleur, un plus tendre, un plus juste… Tous mes malheurs ne me viennent que de mes vertus. » — À Mme de la Tour. « Celui qui ne s’enthousiasme pas pour moi n’est pas digne de moi. »
 } ! » Toutes les souillures qu’il a contractées lui viennent du dehors ; c’est aux circonstances qu’il faut attribuer ses bassesses et ses vices : « Si j’étais tombé dans les mains d’un meilleur maître…, j’aurais été bon chrétien, bon père de famille, bon ami, bon ouvrier, bon homme en toutes choses. » Ainsi la société seule a tous les torts  Pareillement, dans l’homme en général, la nature est bonne. « Ses premiers mouvements sont toujours droits… Le principe fondamental de toute morale, sur lequel j’ai raisonné dans mes écrits, est que {\itshape l’homme est un être naturellement bon, aimant la justice et l’ordre…} L’{\itshape Émile} en particulier n’est qu’un traité de la bonté originelle de l’homme, destiné à montrer comment le vice et l’erreur, étrangers à sa constitution, s’y introduisent du dehors et l’altèrent insensiblement… La nature a fait l’homme heureux et bon, la société le déprave et le fait misérable\footnote{{\itshape Lettre à M. de Beaumont}, 24. — {\itshape Rousseau juge de Jean-Jacques}, troisième entretien, 193.}. » Dépouillez-le, par la pensée, de ses habitudes factices, de ses besoins surajoutés, de ses préjugés faux ; écartez les systèmes, rentrez dans votre propre cœur, écoutez le sentiment intime, laissez-vous guider par la lumière de l’instinct et de la conscience ; et vous retrouverez cet Adam primitif, semblable à une statue de marbre incorruptible qui, tombée dans un marais, a disparu depuis longtemps sous une croûte de moisissures et de vase, mais qui, délivrée de sa gaine fangeuse, peut remonter sur son piédestal avec toute la perfection de sa forme et toute la pureté de sa blancheur.\par
Autour de cette idée centrale se reforme la doctrine spiritualiste  Un être si noble ne peut pas être un simple assemblage d’organes ; il y a en lui quelque chose de plus que la matière ; les impressions qu’il reçoit par les sens ne le constituent pas tout entier. « Je ne suis pas seulement un être sensitif et passif\footnote{\href{http://classiques.uqac.ca/classiques/Rousseau\_jj/emile/emile.html}{\dotuline{{\itshape Émile}}} [\url{http://classiques.uqac.ca/classiques/Rousseau\_jj/emile/emile.html}], {\itshape Profession de foi du vicaire savoyard}, passim.}, mais un être actif et intelligent, et, quoi qu’en dise la philosophie, j’oserai prétendre à l’honneur de penser. » Bien mieux, ce principe pensant est, en l’homme du moins, d’espèce supérieure. « Qu’on me montre un autre animal sur la terre qui sache faire du feu et qui sache admirer le soleil. Quoi ! je puis observer, connaître les êtres et leurs rapports ; je puis sentir ce qu’est ordre, beauté, vertu ; je puis contempler l’univers, m’élever à la main qui le gouverne ; je puis aimer le bien, le faire, et je me comparerais aux bêtes ! » L’homme est libre, capable de choisir entre deux actions, partant créateur de ses actes ; il est donc une cause originale et première, « une substance immatérielle », distincte du corps, une âme que le corps gêne et qui peut survivre au corps  Cette âme immortelle engagée dans la chair a pour voix la conscience. « Conscience ! instinct divin, immortelle et céleste voix, guide assuré d’un être ignorant et borné, mais intelligent et libre, juge infaillible du bien et du mal qui rends l’homme semblable à Dieu, c’est toi qui fais l’excellence de sa nature. » — À côté de l’amour-propre, par lequel nous subordonnons le tout à nous-mêmes, il y a l’amour de l’ordre, par lequel nous nous subordonnons au tout. À côté de l’égoïsme, par lequel l’homme cherche son bonheur même aux dépens des autres, il y a la sympathie, par laquelle il cherche le bonheur des autres même aux dépens du sien. La jouissance personnelle ne lui suffit pas ; il lui faut encore la paix de la conscience et les effusions du cœur  Voilà l’homme tel que Dieu l’a fait et l’a voulu ; il n’y a point de défaut dans sa structure. Les pièces inférieures y servent comme les supérieures ; toutes sont nécessaires, proportionnées, en place, non seulement le cœur, la conscience, la raison et les facultés par lesquelles nous surpassons les brutes, mais encore les inclinations qui nous sont communes avec l’animal, l’instinct de conservation et de défense, le besoin de mouvement physique, l’appétit du sexe, et le reste des impulsions primitives, telles qu’on les constate dans l’enfant, dans le sauvage, dans l’homme inculte\footnote{ {\itshape Émile}, livre I, et {\itshape Lettre à M. de Beaumont}, passim.
 }. Aucune d’elles, prise en soi, n’est vicieuse ou nuisible. Aucune d’elles n’est trop forte, même l’amour de soi. Aucune n’entre en jeu hors de saison. Si nous n’intervenions pas, si nous ne leur imposions pas de contrainte, si nous laissions toutes ces sources vives couler sur leur pente, si nous ne les emprisonnions pas dans nos conduits artificiels et sales, nous ne les verrions jamais écumer ni se ternir. Nous nous étonnons de leurs souillures et de leurs ravages ; nous oublions qu’à leur origine elles étaient inoffensives et pures. La faute est à nous, aux compartiments sociaux, aux canaux encroûtés et rigides par lesquels nous les dévions, nous les contournons, nous les faisons croupir ou bondir. « Ce sont vos gouvernements mêmes qui font les maux auxquels vous prétendez remédier par eux… Sceptres de fer ! lois insensées ! c’est à vous que nous reprochons de n’avoir pu remplir nos devoirs sur la terre ! » Otez ces digues, œuvres de la tyrannie et de la routine ; la nature délivrée reprendra tout de suite son allure droite et saine, et, sans effort, l’homme se trouvera, non seulement heureux, mais vertueux\footnote{« Article I. Tous les Français seront vertueux  Article II. « Tous les Français seront heureux. » (Projet de Constitution retrouvé dans les papiers de Sismondi, alors écolier.)}.\par
Sur ce principe, l’attaque commence : il n’y en a pas qui pénètre plus avant ni qui soit conduite avec une plus âpre hostilité. Jusqu’ici on ne présentait les institutions régnantes que comme gênantes et déraisonnables ; à présent on les accuse d’être en outre injustes et corruptrices. Il n’y avait de soulevés que la raison et les appétits ; on révolte encore la conscience et l’orgueil. Avec Voltaire et Montesquieu, tout ce que je pouvais espérer, c’étaient des maux un peu moindres. Avec Diderot et d’Holbach, je ne distinguais à l’horizon qu’un Eldorado brillant ou une Cythère commode. Avec Rousseau, je vois à portée de ma main un Eden où du premier coup je retrouverai ma noblesse inséparable de mon bonheur. J’y ai droit ; la nature et la Providence m’y appellent ; il est mon héritage. Seule une institution arbitraire m’en écarte et fait mes vices en même temps que mon malheur. Avec quelle colère et de quel élan vais-je me jeter contre la vieille barrière   On s’en aperçoit au ton véhément, au style amer, à l’éloquence sombre de la doctrine nouvelle. Il ne s’agit plus de plaisanter, de polissonner ; le sérieux est continu ; on s’indigne, et la voix puissante qui s’élève perce au-delà des salons jusqu’à la foule souffrante et grossière, à qui nul ne s’est encore adressé, dont les ressentiments sourds rencontrent pour la première fois un interprète, et dont les instincts destructeurs vont bientôt s’ébranler à l’appel de son héraut. — Rousseau est du peuple et il n’est pas du monde. Dans un salon il se trouve gêné\footnote{\href{http://un2sg4.unige.ch/athena/rousseau/confessions/jjr\_conf\_09.html}{\dotuline{{\itshape Confessions.} Partie II, livre IX, 368}} [\url{http://un2sg4.unige.ch/athena/rousseau/confessions/jjr\_conf\_09.html}]. « Je ne comprends pas comment on ose parler dans un cercle… Je me hâte de balbutier promptement des paroles sans idées, trop heureux quand elles ne signifient rien du tout… J’aimerais la société tout comme un autre, si je n’étais sûr de m’y montrer, non seulement à mon désavantage, mais tout autre que je ne suis. » — Cf. {\itshape Nouvelle Héloïse}, 2\textsuperscript{e} partie, Lettre de Saint-Preux sur Paris, et {\itshape Émile}, fin du livre IV.} ; il ne sait pas causer, être aimable ; il n’a de jolis mots qu’après coup, sur l’escalier ; il se tait d’un air maussade ou dit des balourdises, et ne se sauve de la maladresse que par des boutades de rustre ou des sentences de cuistre. L’élégance lui déplaît, le luxe l’incommode, la politesse lui semble un mensonge, la conversation un bavardage, le bon ton une grimace, la gaieté une convention, l’esprit une parade, la science un charlatanisme, la philosophie une affectation, les mœurs une pourriture. Tout y est factice, faux et malsain\footnote{{\itshape Confessions}, 2\textsuperscript{e} partie, IX, 361. « J’étais si ennuyé des salons, des jets d’eau, des bosquets, des parterres et des plus ennuyeux montreurs de tout cela ; j’étais si excédé de brochures, de clavecin, de tri, de nœuds, de sots bons mots, de fades minauderies, de petits conteurs et de grands soupers, que, quand je lorgnais du coin de l’œil un simple pauvre buisson d’épines, une haie, une grange, un pré, quand je humais, en traversant un hameau, la vapeur d’une bonne omelette au cerfeuil…, je donnais au diable le rouge, les falbalas et l’ambre, et, regrettant le dîner de la ménagère et le vin du cru, j’aurais de bon cœur paumé la gueule à Monsieur le chef et à Monsieur le maître qui me faisaient dîner à l’heure où je soupe et souper à l’heure où je dors, mais surtout à Messieurs les laquais qui dévoraient des yeux mes morceaux, et, sous peine de mourir de soif, me vendaient le vin drogué de leur maître, dix fois plus cher que je n’en aurais payé de meilleur au cabaret. »}, depuis le fard, la toilette et la beauté des femmes jusqu’à l’air des appartements et aux ragoûts des tables, le sentiment comme le plaisir, la littérature comme la musique, le gouvernement comme la religion. Cette civilisation qui s’applaudit de son éclat n’est qu’un trémoussement de singes surexcités et serviles qui s’imitent les uns les autres et se gâtent les uns les autres pour arriver par le raffinement au malaise et à l’ennui. Ainsi, par elle-même, la culture humaine est mauvaise, et les fruits qu’elle fait naître ne sont que des excroissances ou des poisons. — À quoi bon les sciences ? Incertaines, inutiles, elles ne sont qu’une pâture pour les disputeurs et les oisifs\footnote{Discours sur l’influence des sciences et des arts — Lettre à d’Alembert sur les spectacles.}. « Qui voudrait passer sa vie en de stériles contemplations, si chacun, ne consultant que les devoirs de l’homme et les besoins de la nature, n’avait de temps que pour la patrie, pour les malheureux et pour ses amis. » — À quoi bon les beaux-arts ? Ils ne sont qu’une flatterie publique des passions régnantes. « Plus la comédie est agréable et parfaite, plus son effet est funeste », et le théâtre, même chez Molière, est une école de mauvaises mœurs, « puisqu’il excite les âmes perfides à punir, sous le nom de sottise, la candeur des honnêtes gens ». La tragédie, qu’on dit morale, dépense en effusions fausses le peu de vertu qui nous reste encore. « Quand un homme est allé admirer de belles actions dans des fables, qu’a-t-on encore à exiger de lui ? Ne s’est-il pas acquitté de tout ce qu’il doit à la vertu par l’hommage qu’il vient de lui rendre ? Que voudrait-on qu’il fît de plus ? Qu’il la pratiquât lui-même ? Il n’a pas de rôle à jouer, il n’est pas comédien. » — Sciences, beaux-arts, arts de luxe, philosophie, littérature, tout cela n’est bon qu’à efféminer et dissiper l’âme ; tout cela n’est fait que pour le petit troupeau d’insectes brillants ou bruyants qui bourdonnent au sommet de la société et sucent toute la substance publique  En fait de sciences, une seule est nécessaire, celle de nos devoirs, et, sans tant de subtilité ou d’études, le sentiment intime suffit pour nous l’enseigner. — En fait d’arts, il n’y a de tolérables que ceux qui, fournissant à nos premiers besoins, nous donnent du pain pour nous nourrir, un toit pour nous abriter, un vêtement pour nous couvrir, des armes pour nous défendre  En fait de vie, il n’en est qu’une saine, celle que l’on mène aux champs, sans apprêt, sans éclat, en famille, dans les occupations de la culture, sur les provisions que fournit la terre, parmi des voisins qu’on traite en égaux et des serviteurs qu’on traite en amis  En fait de classes, il n’y en a qu’une respectable, celle des hommes qui travaillent, surtout celle des hommes qui travaillent de leurs mains, artisans, laboureurs, les seuls qui soient véritablement utiles, les seuls qui, rapprochés par leur condition de l’état naturel, gardent, sous une enveloppe rude, la chaleur, la bonté et la droiture des instincts primitifs  Appelez donc de leur vrai nom cette élégance, ce luxe, cette urbanité, cette délicatesse littéraire, ce dévergondage philosophique que le préjugé admire comme la fleur de la vie humaine ; ils n’en sont que la moisissure. Pareillement estimez à son juste prix l’essaim qui s’en nourrit, je veux dire l’aristocratie désœuvrée, tout le beau monde, les privilégiés qui commandent et représentent, les oisifs de salon qui causent, jouissent et se croient l’élite de l’humanité ; ils n’en sont que les parasites. Parasites et moisissure, l’un attire l’autre, et l’arbre ne se portera bien que lorsque nous l’aurons débarrassé de tous les deux.\par
Si la civilisation est mauvaise, la société est pire\footnote{« La société est naturelle à l’espèce humaine, comme la décrépitude à l’individu. Il faut des arts, des lois, des gouvernements aux peuples, comme il faut des béquilles aux vieillards. » ({\itshape Lettre à M. Philopolis}, 248.)}. Car elle ne s’établit qu’en détruisant l’égalité primitive, et ses deux institutions principales, la propriété et le gouvernement, sont des usurpations. « Le premier\footnote{\href{http://classiques.uqac.ca/classiques/Rousseau\_jj/discours\_origine\_inegalite/origine\_inegalite.html}{\dotuline{{\itshape Discours sur l’origine de l’inégalité}}} [\url{http://classiques.uqac.ca/classiques/Rousseau\_jj/discours\_origine\_inegalite/origine\_inegalite.html}], passim.} qui, ayant enclos un terrain, s’avisa de dire {\itshape ceci est à moi}, et trouva des gens assez simples pour le croire, fut le vrai fondateur de la société civile. Que de crimes, de guerres, de meurtres, que de misères et d’horreurs n’eût point épargnés au genre humain celui qui, arrachant les pieux et comblant le fossé, eût crié à ses semblables : Gardez-vous d’écouter cet imposteur ; vous êtes perdus si vous oubliez que les fruits sont à tous et que la terre n’est à personne ! » — La première propriété fut un vol par lequel l’individu dérobait à la communauté une partie de la chose publique. Rien ne justifiait son attentat, ni son industrie, ni sa peine, ni la valeur qu’il a pu ajouter au sol. « Il avait beau dire : C’est moi qui ai bâti ce mur, j’ai gagné ce terrain par mon travail. Qui vous a donné les alignements, pouvait-on lui répondre, et en vertu de quoi prétendez-vous être payé d’un travail que nous ne vous avons point imposé ? Ignorez-vous qu’une multitude de vos frères périt ou souffre du besoin de ce que vous avez de trop, et qu’il vous fallait un consentement exprès et unanime du genre humain pour vous approprier, sur la subsistance commune, tout ce qui allait au-delà de la vôtre ? » — On reconnaît, à travers la théorie, l’accent personnel, la rancune du plébéien pauvre, aigri, qui entrant dans le monde, a trouvé les places prises et n’a pas su se faire la sienne, qui marque dans ses confessions le jour à partir duquel il a cessé de sentir la faim, qui, faute de mieux, vit en concubinage avec une servante et met ses cinq enfants à l’hôpital, tour à tour valet, commis, bohême, précepteur, copiste, toujours aux aguets et aux expédients pour maintenir son indépendance, révolté par le contraste de la condition qu’il subit et de l’âme qu’il se sent, n’échappant à l’envie que par le dénigrement, et gardant au fond de son cœur une amertume ancienne « contre les riches et les heureux du monde, comme s’ils l’eussent été à ses dépens et que leur prétendu bonheur eût été usurpé sur le sien\footnote{{\itshape Émile}, livre IV. Récit de Rousseau, 13.} »  Non seulement la propriété est injuste par son origine, mais encore, par une seconde injustice, elle attire à soi la puissance, et sa malfaisance grandit comme un chancre sous la partialité de la loi. « Tous les avantages de la société\footnote{\href{http://classiques.uqac.ca/classiques/Rousseau\_jj/discours\_economie\_politique/discours\_eco\_pol.html}{\dotuline{{\itshape Discours} {\itshape sur} {\itshape l’} {\itshape Économie politique}}} [\url{http://classiques.uqac.ca/classiques/Rousseau\_jj/discours\_economie\_politique/discours\_eco\_pol.html}], 326.} ne sont-ils pas pour les puissants et pour les riches ? Tous les emplois lucratifs ne sont-ils pas remplis par eux seuls ? Et l’autorité publique n’est-elle pas toute en leur faveur ? Qu’un homme de considération vole ses créanciers ou fasse d’autres friponneries, n’est-il pas sûr de l’impunité ? Les coups de bâton qu’il distribue, les violences qu’il commet, les meurtres et les assassinats dont il se rend coupable, ne sont-ce pas des affaires qu’on assoupit et dont au bout de six mois il n’est plus question   Que ce même homme soit volé, toute la police est aussitôt en mouvement, et malheur aux innocents qu’il soupçonne   Passe-t-il dans un lieu dangereux, voilà les escortes en campagne. — L’essieu de sa chaise vient-il à se rompre, tout vole à son secours. — Fait-on du bruit à sa porte, il dit un mot et tout se tait. — La foule l’incommode-t-elle, il fait un signe et tout se range. — Un charretier se trouve-t-il sur son passage, ses gens sont prêts à l’assommer, et cinquante honnêtes piétons seraient plutôt écrasés qu’un faquin retardé dans son équipage. — Tous ces égards ne lui coûtent pas un sol ; ils sont le droit de l’homme riche, et non le prix de la richesse. — Que le tableau du pauvre est différent ! Plus l’humanité lui doit, plus la société lui refuse. Toutes les portes lui sont fermées même quand il a le droit de les faire ouvrir, et, s’il obtient quelquefois justice, c’est avec plus de peine qu’un autre obtiendrait grâce. S’il y a des corvées à faire, une milice à lever, c’est à lui qu’on donne la préférence. Il porte toujours, outre sa charge, celle dont son voisin plus riche a le crédit de se faire exempter. Au moindre accident qui lui arrive, chacun s’éloigne de lui. Que sa pauvre charrette renverse, je le tiens heureux s’il évite en passant les avanies des gens lestes d’un jeune duc. En un mot, toute assistance gratuite le fuit au besoin, précisément parce qu’il n’a pas de quoi la payer. Mais je le tiens pour un homme perdu, s’il a le malheur d’avoir l’âme honnête, une fille aimable et un puissant voisin. — Résumons en quatre mots le pacte social des deux états : {\itshape Vous avez besoin de moi, car je suis riche et vous êtes pauvre : faisons donc un accord entre nous ; je permettrai que vous ayez l’honneur de me servir, à condition que vous me donnerez le peu qui vous reste pour la peine que je prends de vous commander.} »\par
Ceci nous montre l’esprit, le but et l’effet de la société politique. — À l’origine, selon Rousseau, elle fut un contrat inique qui, conclu entre le riche adroit et le faible dupé, « donna de nouvelles entraves au faible, de nouvelles forces au riche », et, sous le nom de propriété légitime, consacra l’usurpation du sol  Aujourd’hui elle est un contrat plus inique, « grâce auquel un enfant commande à un vieillard, un imbécile conduit des hommes sages, une poignée de gens regorge de superfluités, tandis que la multitude affamée manque du nécessaire ». Il est dans la nature de l’égalité de s’accroître ; c’est pourquoi l’autorité des uns a grandi en même temps que la dépendance des autres, tant qu’enfin, les deux conditions étant arrivées à l’extrême, la sujétion héréditaire et perpétuelle du peuple a semblé de droit divin comme le despotisme héréditaire et perpétuel du roi. — Voilà l’état présent, et, s’il change, c’est en pis. « Car\footnote{\href{http://classiques.uqac.ca/classiques/Rousseau\_jj/discours\_origine\_inegalite/origine\_inegalite.html}{\dotuline{{\itshape Discours sur l’origine de l’inégalité}}} [\url{http://classiques.uqac.ca/classiques/Rousseau\_jj/discours\_origine\_inegalite/origine\_inegalite.html}], 178. — \href{http://classiques.uqac.ca/classiques/Rousseau\_jj/contrat\_social/contrat\_social.html}{\dotuline{{\itshape Contrat social}}} [\url{http://classiques.uqac.ca/classiques/Rousseau\_jj/contrat\_social/contrat\_social.html}], I, ch. IV.} toute l’occupation des rois ou de ceux qu’ils chargent de leurs fonctions se rapporte à deux seuls objets, étendre leur domination au dehors, et la rendre plus absolue au dedans. » Quand ils allèguent un autre but, c’est prétexte. « Les mots {\itshape bien public, bonheur des sujets}, gloire de la nation, si lourdement employés dans les édits publics, n’annoncent jamais que des ordres funestes, et le peuple gémit d’avance, quand ses maîtres lui parlent de leurs soins paternels. » — Mais, arrivé à ce terme fatal, « le contrat du gouvernement est dissous ; le despote n’est maître qu’aussi longtemps qu’il est le plus fort, et, sitôt qu’on peut l’expulser, il n’a point à réclamer contre la violence ». Car il n’y a de droit que par consentement, et il n’y a ni consentement ni droit d’esclave à maître. « Soit d’un homme à un homme, soit d’un homme à un peuple, ce discours sera toujours également insensé : {\itshape Je fais avec toi une convention toute à ta charge et toute à mon profit, que j’observerai tant qu’il me plaira et que tu observeras tant qu’il me plaira.} » — Que des fous signent ce traité ; puisqu’ils sont fous, ils sont hors d’état de contracter, et leur signature n’est pas valable. Que des vaincus à terre et l’épée sur la gorge acceptent ces conditions ; puisqu’ils sont contraints, leur promesse est nulle. Que des vaincus ou des fous aient, il y a mille ans, engagé le consentement de toutes les générations suivantes : si l’on contracte pour un mineur, on ne contracte pas pour un adulte, et, quand l’enfant est parvenu à l’âge de raison, il n’appartient plus qu’à lui-même. À la fin nous voici adultes, et nous n’avons qu’à faire acte de raison pour rabattre à leur valeur les prétentions de cette autorité qui se dit légitime. Elle a la puissance, rien de plus. Mais « un pistolet aux mains d’un brigand est aussi une puissance » ; direz-vous qu’en conscience je suis obligé de lui donner ma bourse   Je n’obéis que par force, et je lui reprendrai ma bourse sitôt que je pourrai lui prendre son pistolet.

\section[{VII. Les enfants perdus du parti philosophique. — Naigeon, Sylvain Maréchal, Mably, Morelly. — Discrédit complet de la tradition et des institutions qui en dérivent.}]{VII. Les enfants perdus du parti philosophique. — Naigeon, Sylvain Maréchal, Mably, Morelly. — Discrédit complet de la tradition et des institutions qui en dérivent.}

\noindent Arrêtons-nous ici ; ce n’est pas la peine de suivre les enfants perdus du parti, Naigeon et Sylvain Maréchal, Mably et Morelly, les fanatiques qui érigent l’athéisme en dogme obligatoire et en devoir supérieur, les socialistes qui, pour supprimer l’égoïsme, proposent la communauté des biens et fondent une république où tout homme qui voudra rétablir « la détestable propriété » sera déclaré ennemi de l’humanité, traité « en fou furieux » et pour la vie renfermé dans un cachot. Il suffit d’avoir suivi les corps d’armée et les grands sièges. — Avec des engins différents et des tactiques contraires, les diverses attaques ont abouti au même effet. Toutes les institutions ont été sapées par la base. La philosophie régnante a retiré toute autorité à la coutume, à la religion et à l’État. Il est admis, non seulement qu’en elle-même la tradition est fausse, mais encore que par ses œuvres elle est malfaisante, que sur l’erreur elle bâtit l’injustice et que par l’aveuglement elle conduit l’homme à l’oppression. Désormais la voilà proscrite. « Écrasons l’infâme » et ses fauteurs. Elle est le mal dans l’espèce humaine, et, quand le mal sera supprimé, il ne restera plus que du bien. « Il arrivera donc ce moment\footnote{Condorcet, \href{http://classiques.uqac.ca/classiques/condorcet/esquisse\_tableau\_progres\_hum/esquisse.html}{\dotuline{{\itshape Tableau des progrès de l’esprit humain}}} [\url{http://classiques.uqac.ca/classiques/condorcet/esquisse\_tableau\_progres\_hum/esquisse.html}], Dixième époque.} où le soleil n’éclairera plus sur la terre que des hommes libres, ne reconnaissant pour maîtres que leur raison ; où les tyrans et les esclaves, les prêtres et leurs stupides ou hypocrites instruments n’existeront plus que dans l’histoire et sur les théâtres ; où l’on ne s’en occupera plus que pour plaindre leurs victimes et leurs dupes, pour s’entretenir par l’horreur de leurs excès dans une utile vigilance, pour savoir reconnaître et étouffer sous le poids de la raison les premiers germes de la superstition et de la tyrannie, si jamais ils osaient reparaître. » — Le millénium va s’ouvrir, et c’est encore la raison qui doit le construire. Ainsi nous devrons tout à son autorité salutaire, la fondation de l’ordre nouveau comme la destruction de l’ordre ancien.
\chapterclose


\chapteropen

\chapter[{Chapitre IV. Construction de la société future}]{Chapitre IV. \\
Construction de la société future}


\chaptercont

\section[{I. Méthode mathématique  Définition de l’homme abstrait  Contrat social  Indépendance et égalité des contractants  Tous seront égaux devant la loi, et chacun aura une part dans la souveraineté.}]{I. Méthode mathématique  Définition de l’homme abstrait  Contrat social  Indépendance et égalité des contractants  Tous seront égaux devant la loi, et chacun aura une part dans la souveraineté.}

\noindent Considérez donc la société future telle qu’elle apparaît à cet instant à nos législateurs de cabinet, et songez qu’elle apparaîtra bientôt sous le même aspect aux législateurs d’assemblée. — À leurs yeux le moment décisif est arrivé. Désormais il y aura deux histoires\footnote{ \noindent Barère. {\itshape Point du jour}, nº 1 (15 juin 1789). « Vous êtes appelés à recommencer l’histoire. »
 }, l’une celle du passé, l’autre celle de l’avenir, auparavant l’histoire de l’homme encore dépourvu de raison, maintenant l’histoire de l’homme raisonnable. Enfin le règne du droit va commencer. De tout ce que le passé a fondé et transmis, rien n’est légitime. Par-dessus l’homme naturel, il a créé un homme artificiel, ecclésiastique ou laïque, noble ou roturier, roi ou sujet, propriétaire ou prolétaire, ignorant ou lettré, paysan ou citadin, esclave ou maître, toutes qualités factices dont il ne faut point tenir compte, puisque leur origine est entachée de violence et de dol. Otons ces vêtements surajoutés ; prenons l’homme en soi, le même dans toutes les conditions, dans toutes les situations, dans tous les pays, dans tous les siècles, et cherchons le genre d’association qui lui convient. Le problème ainsi posé, tout le reste suit  Conformément aux habitudes de l’esprit classique et aux préceptes de l’idéologie régnante, on construit la politique sur le modèle des mathématiques\footnote{Condorcet, {\itshape ib.} « Les méthodes des sciences mathématiques, appliquées à de nouveaux objets, ont ouvert des routes nouvelles aux sciences politiques et morales. » — Cf. dans Rousseau, \href{http://classiques.uqac.ca/classiques/Rousseau\_jj/contrat\_social/contrat\_social.html}{\dotuline{{\itshape Contrat social}}} [\url{http://classiques.uqac.ca/classiques/Rousseau\_jj/contrat\_social/contrat\_social.html}], le calcul mathématique de la fraction de souveraineté qui revient à chacun.}. On isole une donnée simple, très générale, très accessible à l’observation, très familière, et que l’écolier le plus inattentif et le plus ignorant peut aisément saisir. Retranchez toutes les différences qui séparent un homme des autres ; ne conservez de lui que la portion commune à lui et aux autres. Ce reliquat est l’homme en général, en d’autres termes « un être sensible et raisonnable, qui en cette qualité évite la douleur, cherche le plaisir », et partant aspire « au bonheur, c’est-à-dire à un état stable dans lequel on éprouve plus de plaisir que de peine\footnote{Saint-Lambert, {\itshape Cathéchisme universel}, premier dialogue, 17.} », ou bien encore « c’est un être sensible, capable de former des raisonnements et d’acquérir des idées morales\footnote{Condorcet. {\itshape Ibid.} Neuvième époque. « De cette seule vérité, les publicistes sont parvenus à déduire les droits de l’homme. »} ». Le premier venu peut trouver cette notion dans son expérience et la vérifier lui-même du premier regard. Telle est l’unité sociale ; réunissons-en plusieurs, mille, cent mille, un million, vingt-six millions, et voilà le peuple français. On suppose des hommes nés à vingt et un ans, sans parents, sans passé, sans tradition, sans obligations, sans patrie, et qui, assemblés pour la première fois, vont pour la première fois traiter entre eux. En cet état, et au moment de contracter ensemble, tous sont égaux ; car, par définition, nous avons écarté les qualités extrinsèques et postiches par lesquelles seules ils différaient. Tous sont libres ; car, par définition, nous avons supprimé les sujétions injustes que la force brutale et le préjugé héréditaire leur imposaient  Mais, tous étant égaux, il n’y a aucune raison pour que, par leur contrat, ils concèdent des avantages particuliers à l’un plutôt qu’à l’autre. Ainsi tous seront égaux devant la loi ; nulle personne, famille ou classe, n’aura de privilège ; nul ne pourra réclamer un droit dont un autre serait privé ; nul ne devra porter une charge dont un autre serait exempt  D’autre part, tous étant libres, chacun entre avec sa volonté propre dans le faisceau de volontés qui constitue la société nouvelle ; il faut que, dans les résolutions communes, il intervienne pour sa part. Il ne s’est engagé qu’à cette condition ; il n’est tenu de respecter les lois que parce qu’il a contribué à les faire, et d’obéir aux magistrats que parce qu’il a contribué à les élire. Au fond de toute autorité légitime, on doit retrouver son consentement ou son vote, et, dans le citoyen le plus humble, les plus hauts pouvoirs publics sont obligés de reconnaître un des membres de leur souverain. Nul ne peut aliéner ni perdre cette part de souveraineté ; elle est inséparable de sa personne, et, quand il en délègue l’usage, il en garde la propriété  Liberté, égalité, souveraineté du peuple, ce sont là les premiers articles du contrat social. On les a déduits rigoureusement d’une définition primordiale ; on déduira d’eux non moins rigoureusement les autres droits du citoyen, les grands traits de la constitution, les, principales lois politiques ou civiles, bref l’ordre, la forme et l’esprit de l’État nouveau.

\section[{II. Premières conséquences  L’application de cette théorie est aisée  Motifs de confiance, persuasion que l’homme est par essence raisonnable et bon.}]{II. Premières conséquences  L’application de cette théorie est aisée  Motifs de confiance, persuasion que l’homme est par essence raisonnable et bon.}

\noindent De là deux conséquences  En premier lieu, la société ainsi construite est la seule juste ; car, à l’inverse de toutes les autres, elle n’est pas l’œuvre d’une tradition aveuglément subie, mais d’un contrat conclu entre égaux, examiné en pleine lumière et consenti en pleine liberté\footnote{Rousseau admirait encore Montesquieu, tout en faisant ses réserves ; mais, depuis, la théorie s’est développée et l’on rejette tout droit historique. « Alors, dit Condorcet ({\itshape Ib.} Neuvième époque), on se vit obligé de renoncer à cette politique astucieuse et fausse qui, oubliant que les hommes tiennent des droits égaux de leur nature même, voulait tantôt mesurer l’étendue de ceux qu’il fallait leur laisser sur la grandeur du territoire, sur la température du climat, sur le caractère national, sur la richesse du peuple, sur le degré de perfection du commerce et de l’industrie, et tantôt partager avec inégalité les mêmes droits entre diverses classes d’hommes, en accorder à la naissance, à la richesse, à la profession, et créer ainsi des intérêts contraires, des pouvoirs opposés, pour établir ensuite entre eux un équilibre que ces institutions seules ont rendu nécessaire et qui n’en corrige même pas les influences dangereuses. »}. Composé de théorèmes prouvés, le contrat social a l’autorité de la géométrie ; c’est pourquoi il vaut comme elle en tous temps, en tous lieux pour tout peuple ; son établissement est de droit. Quiconque y fait obstacle est l’ennemi du genre humain ; gouvernement, aristocratie, clergé, quel qu’il soit, il faut l’abattre. Contre lui la révolte n’est qu’une juste défense ; quand nous nous ôtons de ses mains, nous ne faisons que reprendre ce qu’il détient à tort et ce qui est légitimement à nous  En second lieu, le code social, tel qu’on vient de l’exposer, va, une fois promulgué, s’appliquer sans obscurité ni résistance : car il est une sorte de géométrie morale plus simple que l’autre, réduite aux premiers éléments, fondée sur la notion la plus claire et la plus vulgaire, et conduisant en quatre pas aux vérités capitales. Pour comprendre et appliquer ces vérités, il n’est pas besoin d’étude préalable ou de réflexion profonde : il suffit du bon sens et même du sens commun. Le préjugé et l’intérêt pourraient seuls en ternir l’évidence ; mais jamais cette évidence ne manquera à une tête saine et à un cœur droit. Expliquez à un ouvrier, à un paysan les droits de l’homme, et tout de suite il deviendra un bon politique ; faites réciter aux enfants le catéchisme du citoyen et, au sortir de l’école, ils sauront leurs devoirs et leurs droits aussi bien que les quatre règles  Là-dessus l’espérance ouvre ses ailes toutes grandes ; tous les obstacles semblent levés. Il est admis que, d’elle-même et par sa propre force, la théorie engendre la pratique, et qu’il suffit aux hommes de décréter ou d’accepter le pacte social pour acquérir du même coup la capacité de le comprendre et la volonté de l’accomplir.\par
Confiance merveilleuse, inexplicable au premier abord, et qui suppose à l’endroit de l’homme une idée que nous n’avons plus. En effet, on le croyait raisonnable et même bon par essence  Raisonnable, c’est-à-dire capable de donner son assentiment à un principe clair, de suivre la filière des raisonnements ultérieurs, d’entendre et d’accepter la conclusion finale, pour en tirer soi-même à l’occasion les conséquences variées qu’elle renferme : tel est l’homme ordinaire aux yeux des écrivains du temps : c’est qu’ils le jugent d’après eux-mêmes. Pour eux, l’esprit humain, c’est leur esprit, l’esprit classique. Depuis cent cinquante ans, il règne dans la littérature, dans la philosophie, dans la science, dans l’éducation, dans la conversation, en vertu de la tradition, de l’habitude et du bon goût. On n’en tolère pas d’autre, on n’en imagine pas d’autre, et si, dans ce cercle fermé, un étranger parvient à s’introduire, c’est à la condition d’employer l’idiome oratoire que la raison raisonnante impose à tous ses hôtes, Grecs, Anglais, barbares, paysans et sauvages, si différents qu’ils soient entre eux, et si différents qu’ils soient d’elle-même. Dans Buffon, le premier homme, racontant les premières heures de sa vie, analyse ses sensations, ses émotions, ses motifs aussi finement que ferait Condillac lui-même. Chez Diderot, Otou l’Otaïtien, chez Bernardin de Saint-Pierre, un demi-sauvage de l’Indoustan et un vieux colon de l’Ile-de-France, chez Rousseau, un vicaire de campagne, un jardinier, un joueur de gobelets, sont des discoureurs et des moralistes accomplis. Chez Marmontel, Florian, dans toute la petite littérature qui précède ou accompagne la Révolution, dans tout le théâtre tragique ou comique, le personnage, quel qu’il soit, villageois inculte, barbare tatoué, sauvage nu, a pour premier fond le talent de s’expliquer, de raisonner, de suivre avec intelligence et avec attention un discours abstrait, d’enfiler de lui-même ou sur les pas d’un guide l’allée rectiligne des idées générales. Ainsi, pour les spectateurs du dix-huitième siècle, la raison est partout, et il n’y a qu’elle au monde. Une forme d’esprit si universelle ne peut manquer de leur sembler naturelle ; ils sont comme des gens qui, ne parlant qu’une langue et ayant toujours parlé aisément, ne conçoivent pas qu’on puisse parler une autre langue, ni qu’il y ait auprès d’eux des muets ou des sourds  D’autant plus que la théorie autorise leur préjugé. Selon l’idéologie nouvelle, tout esprit est à la portée de toute vérité. S’il n’y atteint pas, la faute est à nous qui l’avons mal préparé ; il y arrivera, si nous prenons la peine de l’y conduire. Car il a des sens comme nous, et les sensations rappelées, combinées, notées par des signes, suffisent pour former « non seulement toutes nos idées, mais encore toutes nos facultés\footnote{Condillac, \href{http://gallica.bnf.fr/ark:/12148/bpt6k80138k/f4.table}{\dotuline{{\itshape Logique}}} [\url{http://gallica.bnf.fr/ark:/12148/bpt6k80138k/f4.table}].} ». Une filiation exacte et continue rattache à nos perceptions les plus simples les sciences les plus compliquées, et, du plus bas degré au plus élevé, on peut poser une échelle ; quand l’écolier s’arrête en chemin, c’est que nous avons laissé trop d’intervalle entre deux échelons ; n’omettons aucun intermédiaire, et il montera jusqu’au sommet  À cette haute idée des facultés de l’homme s’ajoute une idée non moins haute de son cœur. Rousseau a déclaré qu’il est bon, et le beau monde s’est jeté dans cette croyance avec toutes les exagérations de la mode et toute la sentimentalité des salons. On est convaincu que l’homme, surtout l’homme du peuple, est naturellement sensible, affectueux, que tout de suite il est touché par les bienfaits et disposé à les reconnaître, qu’il s’attendrit à la moindre marque d’intérêt, qu’il est capable de toutes les délicatesses. Les estampes\footnote{{\itshape Histoire de France par Estampes}, 1789 (au Cabinet des Estampes).} représentent dans une chaumière délabrée deux enfants, l’un de cinq ans, l’autre de trois, auprès de leur grand’mère infirme, l’un lui soulevant la tête, l’autre lui donnant à boire ; le père et la mère qui rentrent voient ce spectacle touchant, et « ces bonnes gens se trouvent alors si heureux d’avoir de tels enfants qu’ils oublient qu’ils sont pauvres »  « Ô mon père\footnote{Mme de Genlis, {\itshape Souvenirs de Félicie}, 371-391.}, s’écrie un jeune pâtre des Pyrénées, recevez ce chien fidèle qui m’obéit depuis sept ans ; qu’à l’avenir il vous suive et vous défende ; il ne m’aura jamais plus utilement servi. » — Il serait trop long de suivre dans la littérature de la fin du siècle, depuis Marmontel jusqu’à Bernardin de Saint-Pierre, depuis Florian jusqu’à Berquin et Bitaubé, la répétition interminable de ces douceurs et de ces fadeurs  L’illusion gagne jusqu’aux hommes d’État. « Sire, dit Turgot en présentant au roi un plan d’éducation politique\footnote{Tocqueville, {\itshape L’ancien régime}, 237. — Cf. \href{http://gallica.bnf.fr/ark:/12148/bpt6k890432}{\dotuline{{\itshape L’an 2440}}} [\url{http://gallica.bnf.fr/ark:/12148/bpt6k890432}], par Mercier, 3 vol. On y verra tout le détail d’un de ces beaux rêves. L’ouvrage fut publié d’abord en 1770. « La Révolution, dit un des personnages, s’est opérée {\itshape sans effort}, par l’héroïsme d’un grand homme, d’un roi philosophe digne du pouvoir, parce qu’il le dédaignait, etc. » (\href{http://gallica.bnf.fr/ark:/12148/CadresFenetre?O=NUMM-84457\&M=tdm}{\dotuline{Tome II, 109}} [\url{http://gallica.bnf.fr/ark:/12148/CadresFenetre?O=NUMM-84457\&M=tdm}].)}, j’ose vous répondre que dans dix ans votre nation ne sera plus reconnaissable, et que, par les lumières, les bonnes mœurs, par le zèle éclairé pour votre service et pour celui de la patrie, elle sera au-dessus des autres peuples. Les enfants qui ont actuellement dix ans se trouveront alors des hommes préparés pour l’État, affectionnés à leur pays, soumis, non par crainte, mais par raison, à l’autorité, secourables envers leurs concitoyens, accoutumés à reconnaître et à respecter la justice. » — Au mois de janvier 1789\footnote{{\itshape Mémoires} de M. de Bouillé, 70. — Cf. M. de Barante, {\itshape Tableau de la littérature française au dix-huitième siècle}, 318. « On s’imaginait que la civilisation et les lumières avaient amorti toutes les passions, adouci tous les caractères. Il semblait que la morale était devenue facile à pratiquer et que la balance de l’ordre social était si bien établie que rien ne pourrait la déranger. »}, Necker, à qui M. de Bouillé montrait le danger imminent et les entreprises immanquables du Tiers, « répondait froidement et en levant les yeux au ciel qu’il fallait bien compter sur les vertus morales des hommes »  Au fond, quand on voulait se représenter la fondation d’une société humaine, on imaginait vaguement une scène demi-bucolique, demi-théâtrale, à peu près semblable à celle qu’on voyait sur le frontispice des livres illustrés de morale et de politique. Des hommes demi-nus ou vêtus de peaux de bêtes sont assemblés sous un grand chêne ; au milieu d’eux, un vieillard vénérable se lève, et leur parle « le langage de la nature et de la raison » ; il leur propose de s’unir, et leur explique à quoi ils s’obligent par cet engagement mutuel ; il leur montre l’accord de l’intérêt public et de l’intérêt privé, et finit en leur faisant sentir les beautés de la vertu\footnote{Voir dans Rousseau ({\itshape Lettre à M. de Beaumont}) une scène de ce genre, l’établissement du déisme et de la tolérance, à la suite d’un discours comme celui-ci.}. Tous aussitôt poussent des cris d’allégresse, s’embrassent, s’empressent autour de lui et le choisissent pour magistrat ; de toutes parts on danse sous les ormeaux, et la félicité désormais est établie sur la terre  Je n’exagère pas. Les adresses de l’Assemblée nationale à la nation seront des harangues de ce style. Pendant des années, le gouvernement parlera au peuple comme à un berger de Gessner. On priera les paysans de ne plus brûler les châteaux, parce que cela fait de la peine à leur bon roi. On les exhortera « à l’étonner par leurs vertus, pour qu’il reçoive plus tôt le prix des siennes\footnote{\href{http://gallica.bnf.fr/ark:/12148/bpt6k28870p/f341.table}{\dotuline{Buchez et Roux, {\itshape Histoire parlementaire}, IV, 322, adresse du 11 février 1790}} [\url{http://gallica.bnf.fr/ark:/12148/bpt6k28870p/f341.table}]. « Touchante et sublime adresse », dit un député. Elle fut accueillie de l’assemblée « par des applaudissements sans exemple ». Il faudrait \href{http://gallica.bnf.fr/ark:/12148/bpt6k28870p/f341.table}{\dotuline{pouvoir la citer tout entière}} [\url{http://gallica.bnf.fr/ark:/12148/bpt6k28870p/f341.table}].} ». Au plus fort de la Jacquerie, les sages du temps supposeront toujours qu’ils vivent en pleine églogue, et qu’avec un air de flûte ils vont ramener dans la bergerie la meute hurlante des colères bestiales et des appétits déchaînés.

\section[{III. Insuffisance et fragilité de la raison dans l’humanité  Insuffisance et rareté de la raison dans l’humanité  Rôle subalterne de la raison dans la conduite de l’homme  Les puissances brutes et dangereuses  Nature et utilité du gouvernement  Par la théorie nouvelle le gouvernement devient impossible.}]{III. Insuffisance et fragilité de la raison dans l’humanité  Insuffisance et rareté de la raison dans l’humanité  Rôle subalterne de la raison dans la conduite de l’homme  Les puissances brutes et dangereuses  Nature et utilité du gouvernement  Par la théorie nouvelle le gouvernement devient impossible.}

\noindent Il est triste, quand on s’endort dans une bergerie, de trouver à son réveil les moutons changés en loups ; et cependant, en cas de révolution, on peut s’y attendre. Ce que dans l’homme nous appelons la raison n’est point un don inné, primitif et persistant, mais une acquisition tardive et un composé fragile. Il suffit des moindres notions physiologiques pour savoir qu’elle est un état d’équilibre instable, lequel dépend de l’état non moins instable du cerveau, des nerfs, du sang et de l’estomac. Prenez des femmes qui ont faim et des hommes qui ont bu ; mettez-en mille ensemble, laissez-les s’échauffer par leurs cris, par l’attente, par la contagion mutuelle de leur émotion croissante ; au bout de quelques heures, vous n’aurez plus qu’une cohue de fous dangereux ; dès 1789 on le saura et de reste  Maintenant, interrogez la psychologie : la plus simple opération mentale, une perception des sens, un souvenir, l’application d’un nom, un jugement ordinaire est le jeu d’une mécanique compliquée, l’œuvre commune et finale\footnote{On évalue le nombre des cellules cérébrales (couche corticale), à douze cents millions, et celui des fibres qui les relient à quatre milliards.} de plusieurs millions de rouages qui, pareils à ceux d’une horloge, tirent et poussent à l’aveugle, chacun pour soi, chacun entraîné par sa propre force, chacun maintenu dans son office par des compensations et des contrepoids. Si l’aiguille marque l’heure à peu près juste, c’est par l’effet d’une rencontre qui est une merveille, pour ne pas dire un miracle, et l’hallucination, le délire, la monomanie, qui habitent à notre porte, sont toujours sur le point d’entrer en nous. À proprement parler, l’homme est fou, comme le corps est malade, par nature ; la santé de notre esprit, comme la santé de nos organes, n’est qu’une réussite fréquente et un bel accident. Si telle est la chance pour la trame et le canevas grossier, pour les gros fils à peu près solides de notre intelligence, quels doivent être les hasards pour la broderie ultérieure et superposée, pour le réseau subtil et compliqué qui est la raison proprement dite et se compose d’idées générales ? Formées par un lent et délicat tissage, à travers un long appareil de signes, parmi les tiraillements de l’orgueil, de l’enthousiasme et de l’entêtement dogmatique, combien de chances pour que, dans la meilleure tête, ces idées correspondent mal aux choses ! Là-dessus, dès à présent, il suffit de voir chez nos philosophes, chez nos politiques, l’idylle en vogue  Si tels sont les esprits supérieurs, que dirons-nous de la foule, du peuple, des cerveaux bruts et demi-bruts ? Autant la raison est boiteuse dans l’homme, autant elle est rare dans l’humanité. Les idées générales et le raisonnement suivi ne se rencontrent que chez une petite élite. Pour acquérir l’intelligence des mots abstraits et l’habitude des déductions suivies, il faut au préalable une préparation spéciale, un exercice prolongé, une pratique ancienne, outre cela, s’il s’agit de politique, le sang-froid qui, laissant à la réflexion toutes ses prises, permet à l’homme de se détacher un instant de lui-même pour considérer ses intérêts en spectateur désintéressé. Si l’une de ces conditions manque, la raison, surtout la raison politique, est absente. — Chez le paysan, chez le villageois, chez l’homme appliqué dès son enfance au travail manuel, non seulement le réseau des conceptions supérieures fait défaut, mais encore les instruments internes qui pourraient le tisser ne sont pas formés. Accoutumé au grand air et à l’exercice des membres, s’il reste immobile, au bout d’un quart d’heure son attention défaille ; les phrases générales n’entrent plus en lui que comme un bruit ; les combinaisons mentales qu’elles devraient provoquer ne peuvent se faire. Il s’assoupit, à moins que la voix vibrante ne réveille en lui par contagion les instincts de la chair et du sang, les convoitises personnelles, les sourdes inimitiés qui, contenues par une discipline extérieure, sont toujours prêtes à se débrider. — Chez le demi-lettré, même chez l’homme qui se croit cultivé et lit les journaux, presque toujours les principes sont des hôtes disproportionnés ; ils dépassent sa compréhension ; en vain il récite ses dogmes ; il n’en peut mesurer la portée, il n’en saisit pas les limites, il en oublie les restrictions, il en fausse les applications. Ce sont des composés de laboratoire qui restent inoffensifs dans le cabinet et sous la main du chimiste, mais qui deviennent terribles dans la rue et sous les pieds du passant. — On ne s’en apercevra que trop bien tout à l’heure, quand les explosions iront se propageant sur tous les points du territoire, quand, au nom de la souveraineté du peuple, chaque commune, chaque attroupement se croira la nation et agira en conséquence, quand la raison, aux mains de ses nouveaux interprètes, instituera à demeure l’émeute dans les rues et la jacquerie dans les champs.\par
C’est qu’à son endroit les philosophes du siècle se sont mépris de deux façons. Non seulement la raison n’est point naturelle à l’homme ni universelle dans l’humanité ; mais encore, dans la conduite de l’homme et de l’humanité, son influence est petite. Sauf chez quelques froides et lucides intelligences, un Fontenelle, un Hume, un Gibbon, en qui elle peut régner parce qu’elle ne rencontre pas de rivales, elle est bien loin de jouer le premier rôle ; il appartient à d’autres puissances, nées avec nous, et qui, à titre de premiers occupants, restent en possession du logis. La place que la raison y obtient est toujours étroite ; l’office qu’elle y remplit est le plus souvent secondaire. Ouvertement ou en secret, elle n’est qu’un subalterne commode, un avocat domestique et perpétuellement suborné, que les propriétaires emploient à plaider leurs affaires ; s’ils lui cèdent le pas en public, c’est par bienséance. Ils ont beau la proclamer souveraine légitime, ils ne lui laissent jamais sur eux qu’une autorité passagère, et, sous son gouvernement nominal, ils sont les maîtres de la maison. Ces maîtres de l’homme sont le tempérament physique, les besoins corporels, l’instinct animal, le préjugé héréditaire, l’imagination, en général la passion dominante, plus particulièrement l’intérêt personnel ou l’intérêt de famille, de caste, de parti. Nous nous tromperions gravement si nous pensions qu’ils sont bons par nature, généreux, sympathiques, ou, tout au moins, doux, maniables, prompts à se subordonner à l’intérêt social ou à l’intérêt d’autrui. Il y en a plusieurs, et des plus forts, qui, livrés à eux-mêmes, ne feraient que du ravage. — En premier lieu, s’il n’est pas sûr que l’homme soit par le sang un cousin éloigné du singe, du moins il est certain que, par sa structure, il est un animal très voisin du singe, muni de canines, carnivore et carnassier, jadis cannibale, par suite chasseur et belliqueux. De là en lui un fonds persistant de brutalité, de férocité, d’instincts violents et destructeurs, auxquels s’ajoutent, s’il est Français, la gaieté, le rire, et le plus étrange besoin de gambader, de polissonner au milieu des dégâts qu’il fait ; on le verra à l’œuvre. — En second lieu, dès l’origine, sa condition l’a jeté nu et dépourvu sur une terre ingrate où la subsistance est difficile, où, sous peine de mort, il est tenu de faire des provisions et des épargnes. De là pour lui la préoccupation constante et l’idée fixe d’acquérir, d’amasser et de posséder, la rapacité et l’avarice, notamment dans la classe qui, collée à la glèbe, jeûne depuis soixante générations pour nourrir les autres classes, et dont les mains crochues s’étendent incessamment pour saisir ce sol où elles font pousser les fruits ; on la verra à l’œuvre. — En dernier lieu, son organisation mentale plus fine a fait de lui, dès les premiers jours, un être imaginatif en qui les songes pullulants se développent d’eux-mêmes en chimères monstrueuses, pour amplifier au-delà de toute mesure ses craintes, ses espérances et ses désirs. De là en lui un excès de sensibilité, des afflux soudains d’émotion, de transports contagieux, des courants de passion irrésistible, des épidémies de crédulité et de soupçon, bref l’enthousiasme et la panique, surtout s’il est Français, c’est-à-dire excitable et communicatif, aisément jeté hors de son assiette et prompt à recevoir les impulsions étrangères, dépourvu du lest naturel que le tempérament flegmatique et la concentration de la pensée solitaire entretiennent chez ses voisins Germains ou Latins ; on verra tout cela à l’œuvre. — Voilà quelques-unes des puissances brutes qui gouvernent la vie humaine. En temps ordinaire, nous ne les remarquons pas ; comme elles sont contenues, elles ne sous semblent plus redoutables. Nous supposons qu’elles sont apaisées, amorties ; nous voulons croire que la discipline imposée leur est devenue naturelle, et qu’à force de couler entre des digues elles ont pris l’habitude de rester dans leur lit. La vérité est que, comme toutes les puissances brutes, comme un fleuve ou un torrent, elles n’y restent que par contrainte ; c’est la digue qui, par sa résistance, fait leur modération. Contre leurs débordements et leurs dévastations, il a fallu installer une force égale à leur force, graduée selon leur degré, d’autant plus rigide qu’elles sont plus menaçantes, despotique au besoin contre leur despotisme, en tout cas contraignante et répressive, à l’origine un chef de bande, plus tard un chef d’armée, de toutes façons un gendarme élu ou héréditaire, aux yeux vigilants, aux mains rudes, qui, par des voies de fait, inspire la crainte et, par la crainte, maintienne la paix. Pour diriger et limiter ses coups, on emploie divers mécanismes, constitution préalable, division des pouvoirs, code, tribunaux, formes légales. Au bout de tous ces rouages apparaît toujours le ressort final, l’instrument efficace, je veux dire le gendarme armé contre le sauvage, le brigand et le fou que chacun de nous recèle, endormis ou enchaînés, mais toujours vivants, dans la caverne de son propre cœur.\par
Au contraire, dans la théorie nouvelle, c’est contre le gendarme que tous les principes sont promulgués, toutes les précautions prises, toutes les défiances éveillées. Au nom de la souveraineté du peuple, on retire au gouvernement toute autorité, toute prérogative, toute initiative, toute durée et toute force. Le peuple est souverain, et le gouvernement n’est que son commis, moins que son commis, son domestique. — Entre eux « point de contrat » indéfini ou au moins durable, « et qui ne puisse être annulé que par un consentement mutuel ou par l’infidélité d’une des deux parties ». — « Il est contre la nature du corps politique que le souverain s’impose une loi qu’il ne puisse jamais enfreindre. » — Point de charte consacrée et inviolable « qui enchaîne un peuple aux formes de constitution une fois établies ». — « Le droit de les changer est la première garantie de tous les autres. » — « Il n’y a pas, il ne peut y avoir aucune loi fondamentale obligatoire pour le corps du peuple, pas même le contrat social. » — C’est par usurpation et mensonge qu’un prince, une assemblée, des magistrats se disent les représentants du peuple. « La souveraineté ne peut être représentée, par la même raison qu’elle ne peut être aliénée… À l’instant qu’un peuple se donne des représentants, il n’est plus libre, il n’est plus… Le peuple anglais pense être libre, il se trompe fort ; il ne l’est que durant l’élection des membres du Parlement ; sitôt qu’ils sont élus, il est esclave, il n’est rien… Les députés du peuple ne sont donc ni ne peuvent être ses représentants ; ils ne sont que ses commissaires, ils ne peuvent rien conclure définitivement. Toute loi que le peuple en personne n’a pas ratifiée est nulle, ce n’est pas une loi\footnote{Rousseau, \href{http://classiques.uqac.ca/classiques/Rousseau\_jj/contrat\_social/contrat\_social.html}{\dotuline{{\itshape Contrat social}}} [\url{http://classiques.uqac.ca/classiques/Rousseau\_jj/contrat\_social/contrat\_social.html}], I, ch. 7 ; III, ch. 13, 14, 15, 18, IV, ch. 1. — Condorcet, {\itshape ibid.}, 9\textsuperscript{e} époque.}. » — « Il ne suffit pas que le peuple assemblé ait une fois fixé la constitution de l’État en donnant sa sanction à un corps de lois ; il faut encore qu’il y ait des assemblées fixes et périodiques que rien ne puisse abolir ni proroger, tellement qu’au jour marqué le peuple soit légitimement convoqué par la loi, sans qu’il soit besoin pour cela d’aucune autre convocation formelle… À l’instant que le peuple est ainsi assemblé, toute juridiction du gouvernement cesse, la puissance exécutive est suspendue », la société recommence, et les citoyens, rendus à leur indépendance primitive, refont à leur volonté, pour une période qu’ils fixent, le contrat provisoire qu’ils n’avaient conclu que pour une période fixée. « L’ouverture de ces assemblées qui n’ont pour objet que le maintien du traité social doit toujours se faire par deux propositions qu’on ne puisse jamais supprimer et qui passent séparément par les suffrages : la première, {\itshape s’il plaît au souverain de conserver la présente forme de gouvernement ;} la seconde, {\itshape s’il plaît au peuple d’en laisser l’administration à ceux qui en sont actuellement chargés.} » — Ainsi « l’acte par lequel un peuple se soumet à des chefs n’est absolument qu’une commission, un emploi dans lequel, simples officiers du souverain, ils exercent en son nom le pouvoir dont il les a fait dépositaires et qu’il peut modifier, limiter, reprendre quand il lui plaît\footnote{ Rousseau, {\itshape Contrat social}, III, 1, 18 ; IV, 3…,
 } ». Non seulement il garde toujours pour lui seul « la puissance législative qui lui appartient et ne peut appartenir qu’à lui », mais encore il délègue et retire à son gré la puissance exécutive. Ceux qui l’exercent sont ses employés. « Il peut les établir et les destituer quand il lui plaît. » Vis-à-vis de lui, ils n’ont aucun droit. « Il n’est point question pour eux de contracter, mais d’obéir » ; ils n’ont pas de « conditions » à lui faire ; ils ne peuvent réclamer de lui aucun engagement  Ne dites pas qu’à ce compte aucun homme un peu fier et bien élevé ne voudra de nos charges et que nos chefs devront avoir un caractère de laquais. Nous ne leur laissons pas la liberté de refuser ou d’accepter un office ; nous les en chargeons d’autorité. « Dans toute véritable démocratie, la magistrature n’est pas un avantage, mais une charge onéreuse, qu’on ne peut justement imposer à un particulier plutôt qu’à un autre. » Nous mettons la main sur nos magistrats ; nous les prenons au collet pour les asseoir sur leurs sièges. De gré ou de force, ils sont les corvéables de l’État, plus disgraciés qu’un valet ou un manœuvre, puisque le manœuvre travaille à conditions débattues et que le valet chassé peut réclamer ses huit jours. Sitôt que le gouvernement sort de cette humble attitude, il usurpe, et les constitutions vont proclamer qu’en ce cas l’insurrection est non seulement le plus saint des droits, mais encore le premier des devoirs  Là-dessus la pratique accompagne la théorie, et le dogme de la souveraineté du peuple, interprété par la foule, va produire la parfaite anarchie, jusqu’au moment où, interprété par les chefs, il produira le despotisme parfait.

\section[{IV. Secondes conséquences  Par la théorie nouvelle l’État devient despote  Précédents de cette théorie  La centralisation administrative  L’utopie des économistes  Nul droit antérieur n’est valable  Nulle association collatérale n’est tolérée  Aliénation totale de l’individu à la communauté  Droits de l’État sur la propriété, l’éducation et la religion  L’État couvent spartiate.}]{IV. Secondes conséquences  Par la théorie nouvelle l’État devient despote  Précédents de cette théorie  La centralisation administrative  L’utopie des économistes  Nul droit antérieur n’est valable  Nulle association collatérale n’est tolérée  Aliénation totale de l’individu à la communauté  Droits de l’État sur la propriété, l’éducation et la religion  L’État couvent spartiate.}

\noindent Car la théorie a deux faces, et, tandis que d’un côté elle conduit à la démolition perpétuelle du gouvernement, elle aboutit de l’autre à la dictature illimitée de l’État. Le nouveau contrat n’est point un pacte historique, comme la Déclaration des Droits de 1688 en Angleterre, comme la Fédération de 1579 en Hollande, conclu entre des hommes réels et vivants, admettant des situations acquises, des groupes formés et des institutions établies, rédigé pour reconnaître, préciser, garantir et compléter un droit antérieur. Antérieurement au contrat social, il n’y a pas de droit véritable ; car le droit véritable ne naît que par le contrat social, seul valable, puisqu’il est le seul qui soit dressé entre des êtres parfaitement égaux et parfaitement libres, être abstraits, sortes d’unités mathématiques, toutes de même valeur, toutes ayant le même rôle, et dont nulle inégalité ou contrainte ne vient troubler les conventions. C’est pourquoi, au moment où il se conclut, tous les autres pactes deviennent nuls. Propriété, famille, Église, aucune des institutions anciennes ne peut invoquer de droit contre l’État nouveau. L’emplacement où nous le bâtissons doit être considéré comme vide ; si nous y laissons subsister une partie des vieilles constructions, ce sera en son nom et à son profit, pour les enfermer dans son enceinte et les approprier à son usage ; tout le sol humain est à lui  D’autre part, il n’est pas, selon la doctrine américaine, une compagnie d’assurance mutuelle, une société semblable aux autres, bornée dans son objet, restreinte dans son office, limitée dans ses pouvoirs, et par laquelle les individus, conservant pour eux-mêmes la meilleure part de leurs biens et de leurs personnes, se cotisent afin d’entretenir une armée, une maréchaussée, des tribunaux, des grandes routes, des écoles, bref les plus gros instruments de sûreté et d’utilité publiques, mais réservent le demeurant des services locaux et généraux, spirituels et matériels, à l’initiative privée et aux associations spontanées qui se formeront au fur et à mesure des occasions et des besoins. Notre État n’est point une simple machine utilitaire, un outil commode à la main, dont l’ouvrier se sert sans renoncer à l’emploi indépendant de sa main ou à l’emploi simultané d’autres outils. Premier-né, fils unique et seul représentant de la raison, il doit, pour la faire régner, ne rien laisser hors de ses prises  En ceci l’ancien régime conduit au nouveau, et la pratique établie incline d’avance les esprits vers la théorie naissante. Déjà, depuis longtemps, par la centralisation administrative, l’État a la main partout\footnote{ Tocqueville, \href{http://classiques.uqac.ca/classiques/De\_tocqueville\_alexis/ancien\_regime/ancien\_regime.html}{\dotuline{{\itshape l’Ancien régime}}} [\url{http://classiques.uqac.ca/classiques/De\_tocqueville\_alexis/ancien\_regime/ancien\_regime.html}], livre II tout entier ; et livre III, ch. 3.
 }. « Sachez, disait Law au marquis d’Argenson, que ce royaume de France est gouverné par trente intendants. Vous n’avez ni Parlement, ni États, ni gouverneurs ; ce sont trente maîtres des requêtes, commis aux provinces, de qui dépendent le bonheur ou le malheur de ces provinces, leur abondance ou leur stérilité. » En fait, le roi, souverain, père et tuteur universel, conduit par ses délégués les affaires locales, et intervient par ses lettres de cachet ou par ses grâces jusque dans les affaires privées. Sur cet exemple et dans cette voie, les imaginations s’échauffent depuis un demi-siècle. Rien de plus commode qu’un tel instrument pour faire les réformes en grand et d’un seul coup. C’est pourquoi, bien loin de restreindre le pouvoir central, les économistes ont voulu l’étendre. Au lieu de lui opposer des digues nouvelles, ils ont songé à détruire les vieux restes de digues qui le gênaient encore. « Dans un gouvernement, disent Quesnay et ses disciples, le système des contre-forces est une idée funeste… Les spéculations d’après lesquelles on a imaginé le système des contrepoids sont chimériques… Que l’État comprenne bien ses devoirs, et alors qu’on le laisse libre… Il faut que l’État gouverne selon les règles de l’ordre essentiel, et, quand il en est ainsi, il faut qu’il soit tout-puissant. » — Aux approches de la Révolution, la même doctrine reparaît, sauf un nom remplacé par un autre. À la souveraineté du roi, le {\itshape Contrat social} substitue la souveraineté du peuple. Mais la seconde est encore plus absolue que la première, et, dans le couvent démocratique que Rousseau construit sur le modèle de Sparte et de Rome, l’individu n’est rien, l’État est tout.\par
En effet, « les clauses du contrat social se réduisent toutes à une seule\footnote{Rousseau, {\itshape Contrat social.} I, 6.}, savoir, l’aliénation totale de chaque associé avec tous ses droits à la communauté ». Chacun se donne tout entier, « tel qu’il se trouve actuellement, lui et toutes ses forces, dont les biens qu’il possède font partie ». Nulle exception ni réserve ; rien de ce qu’il était ou de ce qu’il avait auparavant ne lui appartient plus en propre. Ce que désormais il sera et aura ne lui sera dévolu que par la délégation du corps social, propriétaire universel et maître absolu. Il faut que l’État ait tous les droits et que les particuliers n’en aient aucun ; sinon, il y aurait entre eux et lui des litiges, et, « comme il n’y a aucun supérieur commun qui puisse prononcer entre eux et lui », ces litiges ne finiraient pas. Au contraire, par la donation complète que chacun fait de soi, « l’union est aussi parfaite que possible » ; ayant renoncé à tout et à lui-même, « il n’a plus rien à réclamer ».\par
Cela posé, suivons les conséquences. — En premier lieu, je ne suis propriétaire de mon bien que par tolérance et de seconde main ; car, par le contrat social, je l’ai aliéné\footnote{{\itshape Ibidem}, I, 9. « L’État, à l’égard de ses membres, est maître de tous leurs biens par le contrat social… Les possesseurs sont considérés comme dépositaires du bien public. »}, « il fait maintenant partie du bien public » ; si en ce moment j’en conserve l’usage, c’est par une concession de l’État qui m’en fait le « dépositaire ». — Et ne dites pas que cette grâce soit une restitution. « Loin qu’en acceptant les biens des particuliers, la société les en dépouille, elle ne fait que changer l’usurpation en véritable droit, la jouissance en propriété. » Avant le contrat social, j’étais possesseur, non de droit, mais de fait, et même injustement si ma part était large ; car « tout homme a naturellement droit à tout ce qui lui est nécessaire » ; et je volais les autres hommes de tout ce que je possédais au-delà de ma subsistance. C’est pourquoi, bien loin que l’État soit mon obligé, je suis le sien, et ce n’est pas mon bien qu’il me rend, c’est son bien qu’il m’octroie. D’où il suit qu’il peut mettre des conditions à son cadeau, limiter à son gré l’usage que j’en ferai, restreindre et régler ma faculté de donner, de tester. « Par nature\footnote{Rousseau, {\itshape Discours sur l’Économie politique}, 308.}, le droit de propriété ne s’étend pas au-delà de la vie du propriétaire ; à l’instant qu’un homme est mort, son bien ne lui appartient plus. Ainsi, lui prescrire les conditions sous lesquelles il peut disposer, c’est au fond moins altérer son droit en apparence que l’étendre en effet. » En tout cas, comme mon titre est un effet du contrat social, il est précaire comme ce contrat lui-même ; une stipulation nouvelle suffira pour le restreindre ou le détruire. « Le souverain\footnote{Rousseau, \href{http://classiques.uqac.ca/classiques/Rousseau\_jj/emile/emile.html}{\dotuline{{\itshape Émile}}} [\url{http://classiques.uqac.ca/classiques/Rousseau\_jj/emile/emile.html}], livre V, 175.} peut légitimement s’emparer des biens de tous, comme cela se fit à Sparte au temps de Lycurgue. » Dans notre couvent laïque, tout ce que chaque moine possède est un don révocable du couvent.\par
En second lieu, ce couvent est un séminaire. Je n’ai pas le droit d’élever mes enfants chez moi et de la façon qui me semble bonne. « Comme on ne laisse pas la raison\footnote{Rousseau, \href{http://classiques.uqac.ca/classiques/Rousseau\_jj/discours\_economie\_politique/discours\_eco\_pol.html}{\dotuline{{\itshape Discours sur l’Économie politique}}} [\url{http://classiques.uqac.ca/classiques/Rousseau\_jj/discours\_economie\_politique/discours\_eco\_pol.html}], 302.} de chaque homme unique arbitre de ses devoirs, on doit d’autant moins abandonner aux lumières et aux préjugés des pères l’éducation des enfants, qu’elle importe à l’État encore plus qu’aux pères. » — « Si l’autorité publique, en prenant la place des pères et en se chargeant de cette importante fonction, acquiert leurs droits en remplissant leurs devoirs, ils ont d’autant moins de sujet de s’en plaindre qu’à cet égard ils ne font proprement que changer de nom et qu’ils auront en commun, sous le nom de citoyens, la même autorité sur leurs enfants qu’ils exerçaient séparément sous le nom de {\itshape pères.} » En d’autres termes, vous cessez d’être père, mais, en échange, vous devenez inspecteur des écoles ; l’un vaut l’autre ; de quoi vous plaignez-vous ? C’était le cas dans l’armée permanente qu’on appelle Sparte ; là les enfants, vrais enfants de troupe, obéissaient tous également à tous les hommes faits. « Ainsi l’éducation publique, dans des règles prescrites par le gouvernement, et sous des magistrats établis par le souverain, est une des maximes fondamentales du gouvernement populaire ou légitime. » — C’est par elle qu’on forme d’avance le citoyen. « C’est elle\footnote{Rousseau, {\itshape Sur le Gouvernement de Pologne}, 277, 283, 287.} qui doit donner aux âmes la forme nationale. Les peuples sont à la longue ce que le gouvernement les fait être : guerriers, citoyens, hommes quand il le veut, populace, canaille quand il lui plaît », et c’est par l’éducation qu’il les façonne. « Voulez-vous prendre une idée de l’éducation publique, lisez la République de Platon\footnote{Rousseau, {\itshape Emile}, livre I.}… Les bonnes institutions sociales sont celles qui savent le mieux dénaturer l’homme, lui ôter son existence absolue pour lui en donner une relative, et transporter le {\itshape moi} dans l’unité commune, en sorte que chaque particulier ne se croie plus un, mais partie de l’unité, et ne soit plus sensible que dans le tout. Un enfant, en ouvrant les yeux, doit voir la patrie, et, jusqu’à la mort, ne doit voir qu’elle… On doit l’exercer à ne jamais regarder son individu que dans ses relations avec le corps de l’État. » Telle était la pratique de Sparte et l’unique but du « grand Lycurgue »  « Tous étant égaux par la constitution, ils doivent être élevés ensemble et de la même manière. » — « La loi doit régler la matière, l’ordre et la forme de leurs études. » À tout le moins, ils doivent tous prendre part aux exercices publics, aux courses à cheval, aux jeux de force et d’adresse institués « pour les accoutumer à la règle, à l’égalité, à la fraternité, aux concurrences », pour leur apprendre « à vivre sous les yeux de leurs concitoyens et à désirer l’approbation publique ». Par ces jeux, dès la première adolescence, ils sont déjà démocrates, puisque, les prix étant décernés, non par l’arbitraire des maîtres, mais par les acclamations des spectateurs, ils s’habituent à reconnaître pour souveraine la souveraine légitime, qui est la décision du peuple assemblé. Le premier intérêt de l’État sera toujours de former les volontés par lesquelles il dure, de préparer les votes qui le maintiendront, de déraciner dans les âmes les passions qui lui seraient contraires, d’implanter dans les âmes des passions qui lui seront favorables, d’établir à demeure, dans ses citoyens futurs, les sentiments et les préjugés dont il aura besoin\footnote{Morelly, \href{http://classiques.uqac.ca/classiques/Rousseau\_jj/emile/emile.html}{\dotuline{{\itshape Code de la nature}}} [\url{http://classiques.uqac.ca/classiques/Rousseau\_jj/emile/emile.html}]. « À cinq ans, tous les enfants seront enlevés à la famille et élevés en commun aux frais de l’État d’une façon uniforme. » On a trouvé un projet analogue et tout spartiate dans les papiers de Saint-Just.}. S’il ne tient pas les enfants, il n’aura pas les adultes. Dans un couvent, il faut que les novices soient élevés en moines ; sinon, quand ils auront grandi, il n’y aura plus de couvent.\par
En dernier lieu, notre couvent laïque a sa religion, une religion laïque. Si j’en professe une autre, c’est sous son bon plaisir et avec des restrictions. Par nature, il est hostile aux associations autres que lui-même ; elles sont des rivales, elles le gênent, elles accaparent la volonté et faussent le vote de leurs membres. « Il importe, pour bien avoir l’énoncé de la volonté générale, qu’il n’y ait pas de société partielle dans l’État, et que chaque citoyen n’opine que d’après lui\footnote{ Rousseau, {\itshape Contrat social}, II, 3, IV, 8.
 }. » Tout ce qui rompt l’unité sociale ne vaut rien », et il vaudrait mieux pour l’État qu’il n’y eût point d’Église  Non seulement toute Eglise est suspecte, mais, si je suis chrétien, ma croyance est vue d’un mauvais œil. Selon le nouveau législateur, « rien n’est plus contraire que le christianisme à l’esprit social… : une société de vrais chrétiens ne serait plus une société d’hommes. » Car « la patrie du chrétien n’est pas de ce monde ». Il ne peut pas être zélé pour l’État et il est tenu en conscience de supporter les tyrans. Sa loi « ne prêche que servitude et dépendance… il est fait pour être esclave », et d’un esclave on ne fera jamais un citoyen. « {\itshape République chrétienne}, chacun de ces deux mots exclut l’autre. » Partant, si la future république me permet d’être chrétien, c’est à la condition sous-entendue que ma doctrine restera confinée dans mon esprit, sans descendre jusque dans mon cœur  Si je suis catholique, (et sur vingt-six millions de Français, vingt-cinq millions sont dans mon cas), ma condition est pire. Car le pacte social ne tolère pas une religion intolérante ; une secte est l’ennemi public quand elle damne les autres sectes ; « quiconque ose dire {\itshape hors de l’Église point de salut} doit être chassé de l’État »  Si enfin je suis libre-penseur, positiviste ou sceptique, ma situation n’est guère meilleure. « Il y a une religion civile », un catéchisme, « une profession de foi dont il appartient au souverain de fixer les articles, non pas précisément comme dogmes de religion, mais comme sentiments de sociabilité, sans lesquels il est impossible d’être bon citoyen ou sujet fidèle ». Ces articles sont « l’existence de la divinité puissante, intelligente, bienfaisante, prévoyante et pourvoyante, la vie à venir, le bonheur des justes, le châtiment des méchants, la sainteté du contrat social et des lois\footnote{Cf. Mercier, \href{http://gallica.bnf.fr/ark:/12148/bpt6k890432}{\dotuline{{\itshape L’an 2440}}} [\url{http://gallica.bnf.fr/ark:/12148/bpt6k890432}], I, ch. 17 et 18. Dès 1770, il trace le programme d’une religion et d’un culte semblables à ceux des Théophilanthropes, et son chapitre est intitulé : {\itshape Pas si éloigné qu’on le pense.}}. Sans pouvoir obliger personne à les croire, il faut bannir de l’État quiconque ne les croit pas ; il faut le bannir non comme impie, mais comme insociable, comme incapable d’aimer sincèrement les lois, la justice, et d’immoler au besoin sa vie à son devoir »  Prenez garde que cette profession de foi n’est point une cérémonie vaine : une inquisition nouvelle en va surveiller la sincérité. « Si quelqu’un, après avoir reconnu publiquement ces mêmes dogmes, se conduit comme ne les croyant pas, qu’il soit puni de mort ; il a commis le plus grand des crimes : il a menti devant les lois. » — Je le disais bien, nous sommes au couvent.

\section[{V. Triomphe complet et derniers excès de la raison classique  Comment elle devient une monomanie  Pourquoi son œuvre n’est pas viable.}]{V. Triomphe complet et derniers excès de la raison classique  Comment elle devient une monomanie  Pourquoi son œuvre n’est pas viable.}

\noindent Tous ces articles sont des suites forcées du contrat social. Du moment où, entrant dans un corps, je ne réserve rien de moi-même, je renonce par cela seul à mes biens, à mes enfants, à mon Église, à mes opinions. Je cesse d’être propriétaire, père, chrétien, philosophe. C’est l’État qui se substitue à moi dans toutes ces fonctions. À la place de ma volonté, il y a désormais la volonté publique, c’est-à-dire, en théorie, l’arbitraire changeant de la majorité comptée par têtes, en fait, l’arbitraire rigide de l’assemblée, de la faction, de l’individu qui détient le pouvoir public  Sur ce principe, l’infatuation débordera hors de toutes limites. Dès la première année, Grégoire dira à la tribune de l’Assemblée constituante : « Nous pourrions, si nous le voulions, changer la religion, mais nous ne le voulons pas. » Un peu plus tard, on le voudra, on le fera, on établira celle d’Holbach, puis celle de Rousseau, et l’on osera bien davantage. Au nom de la raison que l’État seul représente et interprète, on entreprendra de défaire et de refaire, conformément à la raison et à la seule raison, tous les usages, les fêtes, les cérémonies, les costumes, l’ère, le calendrier, les poids, les mesures, les noms des saisons, des mois, des semaines, des jours, des lieux et des monuments, les noms de famille et de baptême, les titres de politesse, le ton des discours, la manière de saluer, de s’aborder, de parler et d’écrire, de telle façon que le Français, comme jadis le puritain ou le quaker, refondu jusque dans sa substance intime, manifeste par les moindres détails de son action et de ses dehors la domination du tout-puissant principe qui le renouvelle et de la logique inflexible qui le régit. Ce sera là l’œuvre finale et le triomphe complet de la raison classique. Installée dans des cerveaux étroits et qui ne peuvent contenir deux idées ensemble, elle va devenir une monomanie froide ou furieuse, acharnée à l’anéantissement du passé qu’elle maudit et à l’établissement du millénium qu’elle poursuit ; tout cela au nom d’un contrat imaginaire, à la fois anarchique et despotique, qui déchaîne l’insurrection et justifie la dictature ; tout cela pour aboutir à un ordre social contradictoire qui ressemble tantôt à une bacchanale d’énergumènes et tantôt à un couvent spartiate ; tout cela pour substituer à l’homme vivant, durable et formé lentement par l’histoire, un automate improvisé qui s’écroulera de lui-même, sitôt que la force extérieure et mécanique par laquelle il était dressé ne le soutiendra plus.
\chapterclose

\chapterclose


\chapteropen

\part[{Livre quatrième. La propagation de la doctrine.}]{Livre quatrième. \\
La propagation de la doctrine.}
\renewcommand{\leftmark}{Livre quatrième. \\
La propagation de la doctrine.}


\chaptercont

\chapteropen

\chapter[{Chapitre I. Succès de cette philosophie en France. — Insuccès de la même philosophie en Angleterre.}]{Chapitre I. \\
Succès de cette philosophie en France. — Insuccès de la même philosophie en Angleterre.}


\chaptercont
\noindent Des théories analogues ont plusieurs fois traversé l’imagination des hommes, et des théories analogues la traverseront encore plus d’une fois. En tout temps et en tout pays, il suffit qu’un changement considérable s’introduise dans la conception de la nature humaine, pour que, par contre-coup, on voie aussitôt l’utopie et la découverte germer sur les territoires de la politique et de la religion  Mais cela ne suffit pas pour que la doctrine nouvelle se propage, ni surtout pour que, de la spéculation, elle passe à l’application. Née en Angleterre, la philosophie du dix-huitième siècle n’a pu se développer en Angleterre ; la fièvre de démolition et de reconstruction y est restée superficielle et momentanée. Déisme, athéisme, matérialisme, scepticisme, idéologie, théorie du retour à la nature, proclamation des droits de l’homme, toutes les témérités de Bolingbroke, Collins, Toland, Tindal et Mandeville, toutes les hardiesses de Hume, Hartley, James Mill et Bentham, toutes les doctrines révolutionnaires y ont été des plantes de serre, écloses çà et là dans les cabinets isolés de quelques penseurs : à l’air libre, elles ont avorté, après une courte floraison, sous la concurrence trop forte de l’antique végétation à qui déjà le sol appartenait\footnote{« Who born within the last forty years has read a word of Collins, and Toland, and Tindal, and that whole race, who called themselves free thinkers ? » (Burke, {\itshape Reflexions on the French revolution}, 1790.)}  Au contraire, en France, la graine importée d’Angleterre végète et pullule avec une vigueur extraordinaire. Dès la Régence, elle est en fleur\footnote{L’\href{http://www.voltaire-integral.com/Html/02/01OEDIPE.htm}{\dotuline{{\itshape Œdipe}}} [\url{http://www.voltaire-integral.com/Html/02/01OEDIPE.htm}] de Voltaire est de 1718, et ses \href{http://www.voltaire-integral.com/Html/22/11\_Beuchot.html}{\dotuline{{\itshape Lettres} {\itshape sur} {\itshape les Anglais}}} [\url{http://www.voltaire-integral.com/Html/22/11\_Beuchot.html}], de 1728. Les \href{http://un2sg4.unige.ch/athena/montesquieu/mon\_lp\_frame0.html}{\dotuline{{\itshape Lettres persanes}}} [\url{http://un2sg4.unige.ch/athena/montesquieu/mon\_lp\_frame0.html}] de Montesquieu, publiées en 1721, contiennent en germe toutes les idées importantes du siècle.}. Comme une espèce favorisée par le sol et le climat, elle envahit tous les terrains, elle accapare l’air et le jour pour elle seule, et souffre à peine sous son ombre quelques avortons d’une espèce ennemie, un survivant d’une flore ancienne comme Rollin, un spécimen d’une flore excentrique comme Saint-Martin. Par ses grands arbres, par ses taillis serrés, par l’innombrable armée de ses broussailles et de ses basses plantes, par Voltaire, Montesquieu, Rousseau, Diderot, d’Alembert et Buffon, par Duclos, Mably, Condillac, Turgot, Beaumarchais, Bernardin de Saint-Pierre, Barthélemy et Thomas, par la foule de ses journalistes, de ses compilateurs et de ses causeurs, par l’élite et la populace de la philosophie, de la science et de la littérature, elle occupe l’académie, le théâtre, les salons et la conversation. Toutes les hautes têtes du siècle sont ses rejetons, et, parmi celles-ci, quelques-unes sont au nombre des plus hautes qu’ait produites l’espèce humaine  C’est que la nouvelle semence est tombée sur le terrain qui lui convient, je veux dire dans la patrie de l’esprit classique. En ce pays de raison raisonnante, elle ne rencontre plus les rivales qui l’étouffaient de l’autre côté de la Manche, et tout de suite elle acquiert, non seulement la force de sève, mais encore l’organe de propagation qui lui manquait.\par

\section[{I. Causes de cette différence. — L’art d’écrire en France. — À cette époque il est supérieur. — Il sert de véhicule aux idées nouvelles. — Les livres sont écrits pour les gens du monde. — Les philosophes sont gens du monde et par suite écrivains. — C’est pourquoi la philosophie descend dans les salons.}]{I. Causes de cette différence. — L’art d’écrire en France. — À cette époque il est supérieur. — Il sert de véhicule aux idées nouvelles. — Les livres sont écrits pour les gens du monde. — Les philosophes sont gens du monde et par suite écrivains. — C’est pourquoi la philosophie descend dans les salons.}

\noindent Cet organe est « l’art de la parole, l’éloquence appliquée aux sujets les plus sérieux, le talent de tout éclaircir\footnote{Joseph de Maistre, {\itshape Œuvres inédites}, 8, 11.} »  « Les bons écrivains de cette nation, dit leur grand adversaire, expriment les choses mieux que ceux de toute autre nation… » — « Leurs livres apprennent peu de chose aux véritables savants », mais « c’est par l’art de la parole qu’on règne sur les hommes », et « la masse des hommes, continuellement repoussée du sanctuaire des sciences par le style dur et le goût détestable des (autres) ouvrages scientifiques, ne résiste pas aux séductions du style et de la méthode française ». Ainsi l’esprit classique qui fournit les idées fournit aussi leur véhicule, et les théories du dix-huitième siècle sont comme ces semences pourvues d’ailes, qui volent d’elles-mêmes sur tous les terrains. Point de livre alors qui ne soit écrit pour des gens du monde et même pour des femmes du monde. Dans les entretiens de Fontenelle sur la {\itshape Pluralité des mondes}, le personnage central est une marquise. Voltaire compose sa {\itshape Métaphysique} et son {\itshape Essai sur les mœurs} pour Mme du Châtelet, et Rousseau son {\itshape Émile} pour Mme d’Épinay. Condillac écrit le {\itshape Traité des sensations} d’après les idées de Mlle Ferrand, et donne aux jeunes filles des conseils sur la manière de lire sa {\itshape Logique.} Baudeau adresse et explique à une dame son \href{http://gallica.bnf.fr/ark:/12148/bpt6k6457q}{\dotuline{{\itshape Tableau économique}}}\footnote{\href{http://gallica.bnf.fr/ark:/12148/bpt6k6457q}{\url{http://gallica.bnf.fr/ark:/12148/bpt6k6457q}}}. Le plus profond des écrits de Diderot est une conversation de Mlle de l’Espinasse avec d’Alembert et Bordeu\footnote{Ses lettres sur les {\itshape Aveugles} et sur les {\itshape Sourds et Muets} sont en tout ou en partie adressées à des femmes.}. Au milieu de son {\itshape Esprit des lois}, Montesquieu avait placé une invocation aux Muses. Presque tous les ouvrages sortent d’un salon, et c’est toujours un salon qui, avant le public, en a eu les prémices. À cet égard, l’habitude est si forte, qu’elle dure encore à la fin de 1789 ; les harangues qu’on va débiter à l’Assemblée nationale sont aussi des morceaux de bravoure qu’on répète au préalable, en soirée, devant les dames. L’ambassadeur américain\footnote{{\itshape Correspondance} de Gouverneur Morris (en anglais), II, 89. (24 janvier 1790.)}, homme pratique, explique à Washington avec une ironie grave la jolie parade académique et littéraire qui précède le tournoi politique et public. « Les discours sont lus d’avance dans une petite société de jeunes gens et de femmes, au nombre desquelles se trouve ordinairement la belle amie de l’orateur ou la belle dont il désire faire son amie ; et la société accorde très poliment son approbation, à moins que la dame qui donne le ton au petit cercle ne trouve à blâmer quelque chose, ce qui naturellement conduit l’auteur à remanier son œuvre, je ne dis pas l’améliorer. »\par
Rien d’étonnant si, parmi de pareilles mœurs, les philosophes de profession deviennent des hommes du monde. Jamais et nulle part ils ne l’ont été si habituellement et au même degré. « Pour un homme de science et de génie, dit un voyageur anglais, ici le principal plaisir est de régner dans le cercle brillant des gens à la mode\footnote{{\itshape A comparative view}, etc. by John Andrews (1785). — Arthur Young, I, 125. « Je plaindrais volontiers l’homme qui croirait être bien reçu dans un cercle brillant de Londres sans compter sur d’autres raisons que sur son titre de membre de la Société royale. Il n’en serait pas de même à Paris pour un membre de l’Académie des sciences : il est assuré partout d’un excellent accueil. »}. » Tandis qu’en Angleterre ils s’enterrent morosement dans leurs livres, vivent entre eux et ne figurent dans la société qu’à la condition de « faire une corvée politique », celle de journaliste ou de pamphlétaire au service d’un parti, en France, tous les soirs, ils soupent en ville, et sont l’ornement, l’amusement des salons où ils vont causer\footnote{ \noindent « Je rencontrais à Paris les d’Alembert, les Marmontel, les Bailly chez les duchesses ; c’était un immense avantage pour eux et pour elles… Quand un homme chez nous se met à faire des livres, on le considère comme renonçant également à la société des gens qui gouvernent et des gens qui rient… À la vanité littéraire près, la vie de vos d’Alembert et de vos Bailly était aussi gaie que celle de vos seigneurs. » (Stendhal, {\itshape Rome, Naples et Florence}, 377, récit du colonel Forsyth.)
 }. Parmi les maisons où l’on dîne, il n’y en a pas qui n’ait son philosophe en titre, un peu plus tard son économiste, son savant. Dans les correspondances et les mémoires, on les suit à la trace, de salon en salon, de château en château, Voltaire à Cirey chez Mme du Châtelet, puis chez lui à Ferney, où il a un théâtre et reçoit toute l’Europe, Rousseau chez Mme d’Epinay et chez M. de Luxembourg, l’abbé Barthélemy chez la duchesse de Choiseul, Thomas, Marmontel et Gibbon chez Mme Necker, les encyclopédistes aux amples dîners de d’Holbach, aux sages et discrets dîners de Mme Geoffrin, dans le petit salon de Mlle de Lespinasse, tous dans le grand salon officiel et central, je veux dire à l’Académie française, où chaque élu nouveau vient faire parade de style et recevoir de la société polie son brevet de maître dans l’art de discourir  Un tel public impose à un auteur l’obligation d’être écrivain encore plus que philosophe. Le penseur est tenu de se préoccuper de ses phrases au moins autant que de ses idées. Il ne lui est point permis de n’être qu’un homme de cabinet. Il n’est pas un simple érudit, plongé dans ses in-folio à la façon allemande, un métaphysicien enseveli dans ses méditations, ayant pour auditoire des élèves qui prennent des notes, et pour lecteurs des hommes d’étude qui consentent à se donner de la peine, un Kant qui se fait une langue à part, attend que le public l’apprenne, et ne sort de la chambre où il travaille que pour aller dans la salle où il fait ses cours. Ici au contraire, en fait de paroles, tous sont experts et même profès. Le mathématicien d’Alembert publie de petits traités sur l’élocution ; le naturaliste Buffon prononce un discours sur le style ; le légiste Montesquieu compose un essai sur le goût ; le psychologue Condillac écrit un volume sur l’art d’écrire  En ceci consiste leur plus grande gloire ; la philosophie leur doit son entrée dans le monde. Ils l’ont retirée du cabinet, du cénacle et de l’école pour l’introduire dans la société et dans la conversation.

\section[{II. Grâce à la méthode, elle devient populaire.}]{II. Grâce à la méthode, elle devient populaire.}

\noindent « Madame la maréchale, dit un des personnages de Diderot\footnote{\href{http://classiques.uqac.ca/classiques/Diderot\_denis/entretien\_philo\_marechale/entretien\_marechale.html}{\dotuline{Entretien d’un philosophe avec la maréchale de…}} [\url{http://classiques.uqac.ca/classiques/Diderot\_denis/entretien\_philo\_marechale/entretien\_marechale.html}]}, il faudra que je reprenne les choses d’un peu haut  De si haut que vous voudrez, pourvu que je puisse vous entendre  Si vous ne m’entendiez pas, ce serait bien ma faute  Cela est poli, mais il faut que vous sachiez que je n’ai jamais lu que mes {\itshape Heures.} » — Il n’importe, et la jolie femme, bien conduite, va philosopher sans le savoir, trouver sans effort la définition du bien et du mal, comprendre et juger les plus hautes doctrines de la morale et de la religion  Tel est l’art du dix-huitième siècle et l’art d’écrire. On s’adresse à des gens qui savent très bien la vie et qui, le plus souvent, ne savent pas l’orthographe, qui sont curieux de tout et ne sont préparés sur rien ; il s’agit de faire descendre la vérité jusqu’à eux. Point de termes scientifiques ou trop abstraits ; ils ne tolèrent que les mots de leur conversation ordinaire. Et ceci n’est pas un obstacle : il est plus aisé avec cette langue de parler philosophie que préséances et chiffons. Car, dans toute question générale, il y a quelque notion capitale et simple de laquelle le reste dépend, celles d’unité, de mesure, de masse, de mouvement en mathématiques, celles d’organe, de fonction, de vie en physiologie, celles de sensation, de peine, de plaisir, de désir en psychologie, celles d’utilité, de contrat, de loi en politique et en morale, celles d’avances, de produit, de valeur, d’échange en économie politique, et de même dans les autres sciences, toutes notions tirées de l’expérience courante, d’où il suit qu’en faisant appel à l’expérience ordinaire, au moyen de quelques exemples familiers, avec des historiettes, des anecdotes, de petits récits qui peuvent être agréables, on peut reformer ces notions et les préciser. Cela fait, presque tout est fait ; car il n’y a plus qu’à mener l’auditeur pas à pas, de gradin en gradin, jusqu’aux dernières conséquences  « Madame la maréchale aura-t-elle la bonté de se souvenir de sa définition   Je m’en souviendrai : vous appelez cela une définition   Oui  C’est donc de la philosophie   Excellente  Et j’ai fait de la philosophie   Comme on fait de la prose, sans y penser. » — Le reste n’est qu’une affaire de raisonnement, c’est-à-dire de conduite, de bon ordre dans les questions, de progrès dans l’analyse. De la notion ainsi renouvelée et rectifiée, on fait sortir la vérité la plus prochaine, puis, de celle-ci, une seconde vérité contiguë à la première, et ainsi de suite jusqu’au bout, sans autre obligation que le soin d’avancer pied à pied et de n’omettre aucun intermédiaire  Avec cette méthode, on peut tout expliquer, tout faire comprendre, même à des femmes, même à des femmes du monde. C’est elle qui au dix-huitième siècle, fait toute la substance des talents, toute la trame des chefs-d’œuvre, toute la clarté, toute la popularité, toute l’autorité de la philosophie. C’est elle qui a construit les {\itshape Éloges} de Fontenelle, {\itshape le Philosophe ignorant} et {\itshape le Principe d’action} de Voltaire, la {\itshape Lettre à M. de Beaumont} et {\itshape le Vicaire savoyard} de Rousseau, le {\itshape Traité de l’homme} et les {\itshape Epoques de la nature} de Buffon, les {\itshape Dialogues sur les blés} de Galiani, les {\itshape Considérations} de d’Alembert sur les mathématiques, la {\itshape Langue des calculs} et la {\itshape Logique} de Condillac, un peu plus tard l’{\itshape Exposition du système du Monde} de Laplace et les {\itshape Discours généraux} de Bichat et de Cuvier\footnote{Même procédé de nos jours dans les \href{http://gallica.bnf.fr/ark:/12148/bpt6k2023187}{\dotuline{{\itshape Sophismes économiques}}} [\url{http://gallica.bnf.fr/ark:/12148/bpt6k2023187}] de Bastiat, dans les {\itshape Éloges historiques} de Flourens, dans le {\itshape Progrès} d’Edmond About.}. C’est elle enfin que Condillac érige en théorie, qui, sous le nom d’Idéologie, aura bientôt l’ascendant d’un dogme, et qui semble alors résumer toute méthode. À tout le moins, elle résume le procédé par lequel les philosophes du siècle ont gagné leur public, propagé leur doctrine et conquis leur succès.

\section[{III. Grâce au style, elle devient agréable. — Deux assaisonnements particuliers au xviiie siècle, la gravelure et la plaisanterie.}]{III. Grâce au style, elle devient agréable. — Deux assaisonnements particuliers au \textsc{xviii}\textsuperscript{e} siècle, la gravelure et la plaisanterie.}

\noindent Grâce à cette méthode on est compris ; mais, pour être lu, il faut encore autre chose. Je compare le dix-huitième siècle à une société de gens qui sont à table ; il ne suffit pas que l’aliment soit devant eux, préparé, présenté, aisé à saisir et à digérer ; il faut encore qu’il soit un mets, ou mieux une friandise. L’esprit est un gourmet ; servons-lui des plats savoureux, délicats, accommodés à son goût ; il mangera d’autant plus que la sensualité aiguisera l’appétit. Deux condiments particuliers entrent dans la cuisine du siècle, et, selon la main qui les emploie, fournissent à tous les mets littéraires un assaisonnement gros ou fin  Dans une société épicurienne à qui l’on prêche le retour à la nature et les droits de l’instinct, les images et les idées voluptueuses s’offrent d’elles-mêmes ; c’est la boîte aux épices appétissantes et irritantes. Chacun alors en use et en abuse ; plusieurs la vident tout entière sur leur plat. Et je ne parle pas seulement de la littérature secrète, des livres extraordinaires que lit Mme d’Andlau, gouvernante des enfants de France et qui s’égarent aux mains des filles de Louis XV\footnote{Le portier des Chartreux.}, ni d’autres livres plus singuliers encore\footnote{{\itshape Thérèse philosophe.} Il {\itshape y} a toute une littérature de cette espèce.} où le raisonnement philosophique apparaît comme un intermède entre des ordures et des gravelures, et que des dames de la cour ont sur leur toilette avec ce titre : {\itshape Heures de Paris.} Il ne s’agit ici que des grands hommes, des maîtres de l’esprit public. Sauf Buffon, tous mettent dans leur sauce des piments, c’est-à-dire des gravelures ou des crudités. On en rencontrerait jusque dans l’{\itshape Esprit des lois ;} il y en a d’énormes, concertées et compassées, au milieu des {\itshape Lettres persanes.} Dans ses deux grands romans, Diderot les jette à pleines mains, comme en un jour d’orgie. À toutes les pages de Voltaire, ils craquent sous la dent, comme autant de grains de poivre. Vous les retrouvez, non pas piquants, mais âcres et d’une saveur brûlante, dans la {\itshape Nouvelle Héloïse}, en vingt endroits de l’{\itshape Émile}, et d’un bout à l’autre des {\itshape Confessions.} C’était le goût du temps ; M. de Malesherbes, si honnête et si grave, savait par cœur et récitait la {\itshape Pucelle ;} du plus sombre des Montagnards, Saint-Just, on a un poème aussi lubrique que celui de Voltaire, et le plus noble des Girondins, Mme Roland, a laissé des confessions aussi risquées, aussi détaillées que celles de Rousseau\footnote{ Voyez l’édition de M. Dauban, qui a rétabli les morceaux supprimés.
 }  D’autre part, voici une seconde boîte, celle qui contient le vieux sel gaulois, je veux dire la plaisanterie et la raillerie. Elle s’ouvre toute grande aux mains d’une philosophie qui proclame la souveraineté de la raison. Car ce qui est contraire à la raison est absurde, partant ridicule. Sitôt qu’un geste adroit a fait brusquement tomber le masque héréditaire et solennel qui couvrait une sottise, nous éprouvons cette étrange convulsion qui écarte les deux coins de la couche et qui secoue violemment la poitrine, en nous donnant le sentiment d’une détente soudaine, d’une délivrance inattendue, d’une supériorité reconquise, d’une vengeance accomplie et d’une justice faite. Mais, selon la façon dont le masque est ôté, le rire peut être tour à tour léger ou bruyant, contenu ou déboutonné, tantôt aimable et gai, tantôt amer et sardonique. La plaisanterie comporte toutes les nuances, depuis la bouffonnerie jusqu’à l’indignation ; il n’y a point d’assaisonnement littéraire qui fournisse tant de variétés et de mixtures, ni qui se combine si bien avec le précédent  Les deux ensemble ont été, dès le moyen âge, les principaux ingrédients dont la cuisine française a composé ses plus agréables friandises, fabliaux, contes, bons mots, gaudrioles et malices, héritage éternel d’une race grivoise et narquoise, que La Fontaine a conservé à travers la pompe et le sérieux du dix-septième siècle, et qui, au dix-huitième siècle, reparaît partout dans le festin philosophique. Devant cette table si bien servie, l’attrait est vif pour la brillante société dont la grande affaire est le plaisir et l’amusement. Il est d’autant plus vif que, cette fois, la disposition passagère est d’accord avec l’instinct héréditaire, et que le goût de l’époque vient fortifier le goût national. Joignez à cela l’art exquis des cuisiniers, leur talent pour mélanger, proportionner et dissimuler les condiments, pour diversifier et ordonner les mets, leur sûreté de main, leur finesse de palais, leur expérience des procédés, la tradition et la pratique qui, depuis cent ans déjà, font de la prose française le plus délicat aliment de l’esprit. Rien d’étrange si vous les trouvez habiles pour apprêter la parole humaine, pour en exprimer tout le suc et pour en distiller tout l’agrément.

\section[{IV. Art et procédés des maîtres. — Montesquieu. — Voltaire. — Diderot. — Rousseau. — Le Mariage de Figaro.}]{IV. Art et procédés des maîtres. — Montesquieu. — Voltaire. — Diderot. — Rousseau. — {\itshape Le Mariage de Figaro}.}

\noindent À cet égard, quatre d’entre eux sont supérieurs, Montesquieu, Voltaire, Diderot et Rousseau. Il semble qu’il suffise de les nommer ; l’Europe moderne n’a pas d’écrivains plus grands ; et pourtant il faut regarder de près leur talent, si l’on veut bien comprendre leur puissance  Pour le ton et les façons, Montesquieu est le premier. Point d’écrivain qui soit plus maître de soi, plus calme d’extérieur, plus sûr de sa parole. Jamais sa voix n’a d’éclats ; il dit avec mesure les choses les plus fortes. Point de gestes ; les exclamations, l’emportement de la verve, tout ce qui serait contraire aux bienséances répugne à son tact, à sa réserve, à sa fierté. Il semble qu’il parle toujours devant un petit cercle choisi de gens très fins et de façon à leur donner à chaque instant l’occasion de sentir leur finesse. Nulle flatterie plus délicate ; nous lui savons gré de nous rendre contents de notre esprit. Il faut en avoir pour le lire : car, de parti pris, il écourte les développements, il omet les transitions ; à nous de les suppléer, d’entendre ses sous-entendus. L’ordre est rigoureux chez lui, mais il est caché, et ses phrases discontinues défilent, chacune à part, comme autant de cassettes ou d’écrins, tantôt simples et nues d’aspect, tantôt magnifiquement décorées et ciselées, mais toujours pleines. Ouvrez-les ; chacune d’elles est un trésor ; il y a mis, dans un étroit espace, un long amas de réflexions, d’émotions, de découvertes, et notre jouissance est d’autant plus vive que tout cela, saisi en une minute, tient aisément dans le creux de notre main. « Ce qui fait ordinairement une grande pensée, dit-il lui-même, c’est lorsqu’on dit une chose qui en fait voir un grand nombre d’autres, et qu’on nous fait découvrir tout d’un coup ce que nous ne pouvions espérer qu’après une longue lecture. » En effet, telle est sa manière ; il pense par résumés : dans un chapitre de trois lignes, il concentre toute l’essence du despotisme. Souvent même le résumé a un air d’énigme, et l’agrément est double, puisque, avec le plaisir de comprendre, nous avons la satisfaction de deviner. En tout sujet, il garde cette suprême discrétion, cet art d’indiquer sans appuyer, ces réticences, ce sourire qui ne va pas jusqu’au rire. « Dans ma {\itshape Défense de l’Esprit des lois}, disait-il, ce qui me plaît, ce n’est pas de voir les vénérables théologiens mis à terre, c’est de les y voir couler tout doucement. » Il excelle dans l’ironie tranquille, dans le dédain poli\footnote{\href{http://classiques.uqac.ca/classiques/montesquieu/de\_esprit\_des\_lois/partie\_1/de\_esprit\_des\_lois\_1.html}{\dotuline{{\itshape Esprit des lois}, ch. XV, livre 5}} [\url{http://classiques.uqac.ca/classiques/montesquieu/de\_esprit\_des\_lois/partie\_1/de\_esprit\_des\_lois\_1.html}] (raisons en faveur de l’esclavage). Défense de l’{\itshape Esprit des lois.} I, Réponse à la 2\textsuperscript{e} objection. II, Réponse à la 4\textsuperscript{e} objection.}, dans le sarcasme déguisé. Ses Persans jugent la France en Persans, et nous sourions de leurs méprises ; par malheur, ce n’est pas d’eux, mais de nous qu’il faut rire ; car il se trouve que leur erreur est une vérité\footnote{\href{http://un2sg4.unige.ch/athena/montesquieu/mon\_lp\_024.html}{\dotuline{{\itshape Lettre} 24}} [\url{http://un2sg4.unige.ch/athena/montesquieu/mon\_lp\_024.html}] (sur Louis XIV).}. Telle lettre d’un grand sérieux semble une comédie à leurs dépens, sans aucun rapport à nous, toute pleine des préjugés mahométans et d’infatuation orientale\footnote{\href{http://un2sg4.unige.ch/athena/montesquieu/mon\_lp\_018.html}{\dotuline{{\itshape Lettre} 18}} [\url{http://un2sg4.unige.ch/athena/montesquieu/mon\_lp\_018.html}] (sur la pureté et l’impureté des choses). \href{http://un2sg4.unige.ch/athena/montesquieu/mon\_lp\_039.html}{\dotuline{{\itshape Lettre} 39}} [\url{http://un2sg4.unige.ch/athena/montesquieu/mon\_lp\_039.html}] (preuves de la mission de Mahomet).} : réfléchissez : sur le même sujet, notre infatuation n’est pas moindre. Des coups d’une force et d’une portée extraordinaires sont lancés, en passant et comme sans y songer, contre les institutions régnantes, contre le catholicisme altéré qui, « dans l’état présent où est l’Europe, ne peut subsister cinq cents ans », contre la monarchie gâtée qui fait jeûner les citoyens utiles pour engraisser les courtisans parasites\footnote{\href{http://un2sg4.unige.ch/athena/montesquieu/mon\_lp\_075.html}{\dotuline{{\itshape Lettres} 75}} [\url{http://un2sg4.unige.ch/athena/montesquieu/mon\_lp\_075.html}] et \href{http://un2sg4.unige.ch/athena/montesquieu/mon\_lp\_118.html}{\dotuline{118}} [\url{http://un2sg4.unige.ch/athena/montesquieu/mon\_lp\_118.html}].}. Toute la philosophie nouvelle éclôt sous sa main avec un air d’innocence, dans un roman pastoral, dans une prière naïve, dans une lettre ingénue\footnote{\href{http://un2sg4.unige.ch/athena/montesquieu/mon\_lp\_098.html}{\dotuline{{\itshape Lettres} 98}} [\url{http://un2sg4.unige.ch/athena/montesquieu/mon\_lp\_098.html}] (sur les sciences modernes), \href{http://un2sg4.unige.ch/athena/montesquieu/mon\_lp\_046.html}{\dotuline{46}} [\url{http://un2sg4.unige.ch/athena/montesquieu/mon\_lp\_046.html}] (sur le véritable culte), \href{http://un2sg4.unige.ch/athena/montesquieu/mon\_lp\_011.html}{\dotuline{11}} [\url{http://un2sg4.unige.ch/athena/montesquieu/mon\_lp\_011.html}] à \href{http://un2sg4.unige.ch/athena/montesquieu/mon\_lp\_014.html}{\dotuline{14}} [\url{http://un2sg4.unige.ch/athena/montesquieu/mon\_lp\_014.html}] (sur la nature de la justice).}. Aucun des dons par lesquels on peut frapper et retenir l’attention ne manque à ce style, ni l’imagination grandiose, ni le sentiment profond, ni la vivacité du trait, ni la délicatesse des nuances, ni la précision vigoureuse, ni la grâce enjouée, ni le burlesque imprévu, ni la variété de la mise en scène. Mais, parmi tant de tours ingénieux, apologues, contes, portraits, dialogues, dans le sérieux comme dans la mascarade, la tenue demeure irréprochable et le ton parfait. Si l’auteur développe le paradoxe, c’est avec une gravité presque anglaise. S’ils étale toute l’indécence des choses, c’est avec toute la décence des mots. Au plus fort de la bouffonnerie comme au plus fort de la licence, il reste homme de bonne compagnie, né et élevé dans ce cercle aristocratique où la liberté est complète, mais où le savoir-vivre est suprême, où toute pensée est permise, mais où toute parole est pesée, où l’on a le droit de tout dire, mais à condition de ne jamais s’oublier.\par
Un pareil cercle est étroit et ne comprend qu’une élite ; pour être entendu de la foule, il faut parler d’un autre ton. La philosophie a besoin d’un écrivain qui se donne pour premier emploi le soin de la répandre, qui ne puisse la contenir en lui-même, qui l’épanche hors de soi à la façon d’une fontaine regorgeante, qui la verse à tous, tous les jours et sous toutes les formes, à larges flots, en fines gouttelettes, sans jamais tarir ni se ralentir, par tous les orifices et tous les canaux, prose, poésie, grands et petits vers, théâtre, histoire, romans, pamphlets, plaidoyers, traités, brochures, dictionnaire, correspondance, en public, en secret, pour qu’elle pénètre à toute profondeur et dans tous les terrains : c’est Voltaire  « J’ai fait plus en mon temps, dit-il quelque part, que Luther et Calvin », et en cela il se trompe. La vérité est pourtant qu’il a quelque chose de leur esprit. Il veut comme eux changer la religion régnante, il se conduit en fondateur de secte, il recrute et ligue des prosélytes, il écrit des lettres d’exhortation, de prédication et de direction, il fait circuler les mots d’ordre, il donne « aux frères » une devise ; sa passion ressemble au zèle d’un apôtre et d’un prophète  Un pareil esprit n’est pas capable de réserve ; il est par nature militant et emporté ; il apostrophe, il injurie, il improvise, il écrit sous la dictée de son impression, il se permet tous les mots, au besoin les plus crus. Il pense par explosions ; ses émotions sont des sursauts, ses images sont des étincelles ; il se lâche tout entier, il se livre au lecteur, c’est pourquoi il le prend. Impossible de lui résister, la contagion est trop forte. Créature d’air et de flamme, la plus excitable qui fut jamais, composée d’atomes plus éthérés et plus vibrants que ceux des autres hommes, il n’y en a point dont la structure mentale soit plus fine ni dont l’équilibre soit à la fois plus instable et plus juste. On peut le comparer à ces balances de précision qu’un souffle dérange, mais auprès desquelles tous les autres appareils de mesure sont inexacts et grossiers  Dans cette balance délicate, il ne faut mettre que des poids très légers, de petits échantillons ; c’est à cette condition qu’elle pèse rigoureusement toutes les substances ; ainsi fait Voltaire, involontairement, par besoin d’esprit et pour lui-même autant que pour ses lecteurs. Une philosophie complète, une théologie en dix tomes, une science abstraite, une bibliothèque spéciale, une grande branche de l’érudition, de l’expérience ou de l’invention humaine se réduit ainsi sous sa main à une phrase ou à un vers. De l’énorme masse rugueuse et empâtée de scories, il a extrait tout l’essentiel, un grain d’or ou de cuivre, spécimen du reste, et il nous le présente sous la forme la plus maniable et la plus commode, dans une comparaison, dans une métaphore, dans une épigramme qui devient un proverbe. En ceci, nul écrivain ancien ou moderne n’approche de lui ; pour simplifier et vulgariser, il n’a pas son égal au monde. Sans sortir du ton de la conversation ordinaire et comme en se jouant, il met en petites phrases portatives les plus grandes découvertes et les plus grandes hypothèses de l’esprit humain, les théories de Descartes, Malebranche, Leibnitz, Locke et Newton, les diverses religions de l’antiquité et des temps modernes, tous les systèmes connus de physique, de physiologie, de géologie, de morale, de droit naturel, d’économie politique\footnote{Cf. \href{http://www.voltaire-integral.com/Html/21/05Micromegas.html}{\dotuline{{\itshape Micromégas}}} [\url{http://www.voltaire-integral.com/Html/21/05Micromegas.html}], \href{http://www.voltaire-integral.com/Html/21/0840ECU.html}{\dotuline{{\itshape L’homme aux quarante écus}}} [\url{http://www.voltaire-integral.com/Html/21/0840ECU.html}], {\itshape Dialogues entre A, B et C. Dictionnaire philosophique}, passim. — En vers, \href{http://www.voltaire-integral.com/Html/10/33\_Systemes.html}{\dotuline{{\itshape Les systèmes}}} [\url{http://www.voltaire-integral.com/Html/10/33\_Systemes.html}], {\itshape La loi naturelle}, \href{http://www.voltaire-integral.com/Html/09/06PETIT.htm\#AVERTISSEMENT\%20POUR\%20LE\%20POUR\%20ET\%20LE\%20CONTRE.}{\dotuline{{\itshape Le pour et le contre}}} [\url{http://www.voltaire-integral.com/Html/09/06PETIT.htm\#AVERTISSEMENT\%20POUR\%20LE\%20POUR\%20ET\%20LE\%20CONTRE.}], \href{http://www.voltaire-integral.com/Html/09/10\_Discours.html}{\dotuline{{\itshape Discours sur l’homme}}} [\url{http://www.voltaire-integral.com/Html/09/10\_Discours.html}], etc.}, bref, en tout ordre de connaissances, toutes les conceptions d’ensemble que l’espèce humaine au dix-huitième siècle avait atteintes. — Sa pente est si forte de ce côté, qu’elle l’entraîne trop loin ; il rapetisse les grandes choses à force de les rendre accessibles. On ne peut mettre ainsi en menue monnaie courante la religion, la légende, l’antique poésie populaire, les créations spontanées de l’instinct, les demi-visions des âges primitifs ; elles ne sont pas des sujets de conversation amusante et vive. Un mot piquant ne peut pas en être l’expression ; il n’en est que la parodie. Mais quel attrait pour des Français, pour des gens du monde, et quel lecteur s’abstiendra d’un livre où tout le savoir humain est rassemblé en mots piquants   Car c’est bien tout le savoir humain, et je ne vois pas quelle idée importante manquerait à un homme qui aurait pour bréviaire les {\itshape Dialogues}, le {\itshape Dictionnaire} et les {\itshape Romans.} Relisez-les cinq ou six fois, et alors seulement vous vous rendrez compte de tout ce qu’ils contiennent. Non seulement les vues sur le monde et sur l’homme, les idées générales de toute espèce y abondent, mais encore les renseignements positifs et même techniques y fourmillent, petits faits semés par milliers, détails multipliés et précis sur l’astronomie, la physique, la géographie, la physiologie, la statistique, l’histoire de tous les peuples, expériences innombrables et personnelles d’un homme qui par lui-même a lu les textes, manié les instruments, visité les pays, touché les industries, pratiqué les hommes, et qui, par la netteté de sa merveilleuse mémoire, par la vivacité de son imagination toujours flambante, revoit ou voit, comme avec les yeux de la tête, tout ce qu’il dit à mesure qu’il le dit. Talent unique, le plus rare en un siècle classique, le plus précieux de tous, puisqu’il consiste à se représenter les êtres, non pas à travers le voile grisâtre des phrases générales, mais en eux-mêmes, tels qu’ils sont dans la nature et dans l’histoire, avec leur couleur et leur forme sensibles, avec leur saillie et leur relief individuels, avec leurs accessoires et leurs alentours dans le temps et dans l’espace, un paysan à sa charrue, un quaker dans sa congrégation, un baron allemand dans son château, des Hollandais, des Anglais, des Espagnols, des Italiens, des Français chez eux\footnote{\href{http://www.voltaire-integral.com/Html/22/12\_Metaphysique.html\#CHAPITRE\%20I.}{\dotuline{{\itshape Traité de métaphysique}}} [\url{http://www.voltaire-integral.com/Html/22/12\_Metaphysique.html\#CHAPITRE\%20I.}], chap. I. 1 (sur les paysans). — {\itshape Lettres sur les Anglais}, passim. — \href{http://www.voltaire-integral.com/Html/21/06CANDID.html}{\dotuline{{\itshape Candide}}} [\url{http://www.voltaire-integral.com/Html/21/06CANDID.html}], passim. — \href{http://www.voltaire-integral.com/Html/21/09BABYLO.html}{\dotuline{{\itshape La princesse de Babylone}}} [\url{http://www.voltaire-integral.com/Html/21/09BABYLO.html}], ch. VII, VII, IX, X et XI.}, une grande dame, une intrigante, des provinciaux, des soldats, des filles\footnote{\href{http://www.voltaire-integral.com/Html/20/maladie.htm}{\dotuline{{\itshape Dictionnaire} {\itshape philosophique}, article {\itshape Maladie}}} [\url{http://www.voltaire-integral.com/Html/20/maladie.htm}] (Réponses de la princesse). — \href{http://www.voltaire-integral.com/Html/21/06CANDID.html}{\dotuline{{\itshape Candide}}} [\url{http://www.voltaire-integral.com/Html/21/06CANDID.html}] chez Mme de Parolignac. Le matelot dans le naufrage, Récit de Paquette. — \href{http://www.voltaire-integral.com/Html/21/07Ingenu.html}{\dotuline{{\itshape L’Ingénu}}} [\url{http://www.voltaire-integral.com/Html/21/07Ingenu.html}], premiers chapitres.}, et le reste du pêle-mêle humain, à tous les degrés de l’escalier social, chacun en raccourci et dans la lumière fuyante d’un éclair.\par
Car c’est là le trait le plus frappant de ce style, la rapidité prodigieuse, le défilé éblouissant et vertigineux de choses toujours nouvelles, idées, images, événements, paysages, récits, dialogues, petites peintures abréviatives, qui se suivent en courant comme dans une lanterne magique, presque aussitôt retirées que présentées par le magicien impatient qui en un clin d’œil fait le tour du monde, et qui, enchevêtrant coup sur coup l’histoire, la fable, la vérité, la fantaisie, le temps présent, le temps passé, encadre son œuvre tantôt dans une parade aussi saugrenue que celles de la foire, tantôt dans une féerie plus magnifique que toutes celles de l’Opéra. Amuser, s’amuser, « faire passer son âme par tous les modes imaginables », comme un foyer ardent où l’on jette tour à tour les substances les plus diverses pour lui faire rendre toutes les flammes, tous les pétillements et tous les parfums, voilà son premier instinct. « La vie, dit-il encore, est un enfant qu’il faut bercer jusqu’à ce qu’il s’endorme. » Il n’y eut jamais de créature mortelle plus excitée et plus excitante, plus impropre au silence et plus hostile à l’ennui\footnote{\href{http://www.voltaire-integral.com/Html/21/06CANDID.html\#30}{\dotuline{{\itshape Candide}, dernier chapitre }} [\url{http://www.voltaire-integral.com/Html/21/06CANDID.html\#30}]: « Quand on ne disputait pas, l’ennui était si excessif que la vieille osa un jour lui dire : Je voudrais bien savoir lequel est le pire, ou d’être violée cent fois par des pirates nègres, d’avoir une fesse coupée, de passer par les baguettes chez les Bulgares, d’être fouetté et pendu dans un autodafé, d’être disséqué, de ramer aux galères, d’éprouver enfin toutes les misères par lesquelles nous avons passé, ou bien de rester ici à ne rien faire   C’est une grande question, dit Candide. »}, mieux douée pour la conversation, plus visiblement destinée à devenir la reine d’un siècle sociable où, avec six jolis contes, trente bons mots et un peu d’usage, un homme avait son passeport mondain et la certitude d’être bien accueilli partout. Il n’y eut jamais d’écrivain qui ait possédé à un si haut degré et en pareille abondance tous les dons du causeur, l’art d’animer et d’égayer la parole, le talent de plaire aux gens du monde. Du meilleur ton quand il le veut, et s’enfermant sans gêne dans les plus exactes bienséances, d’une politesse achevée, d’une galanterie exquise, respectueux sans bassesse, caressant sans fadeur\footnote{Par exemple, la \href{http://www.voltaire-integral.com/Html/03/08ALZIR1.htm\#…PŒTRE\%20‡\%20LA\%20MARQUISE\%20DU\%20CH¬TELET(2).}{\dotuline{préface d’{\itshape Alzire}}} [\url{http://www.voltaire-integral.com/Html/03/08ALZIR1.htm\#…PŒTRE\%20‡\%20LA\%20MARQUISE\%20DU\%20CH¬TELET(2).}] adressée à Mme du Châtelet, les {\itshape vers à la princesse Ulrique} : « Souvent un peu de vérité, etc. »} et toujours aisé, il lui suffit d’être en public pour prendre naturellement l’accent mesuré, les façons discrètes, le demi-sourire engageant de l’homme bien élevé qui, introduisant les lecteurs dans sa pensée, leur fait les honneurs du logis. Êtes-vous familier avec lui, et du petit cercle intime dans lequel il s’épanche en toute liberté, portes closes, le rire ne vous quittera plus. Brusquement, d’une main sûre et sans avoir l’air d’y toucher, il enlève le voile qui couvre un abus, un préjugé, une sottise, bref quelqu’une des idoles humaines. Sous cette lumière subite, la vraie figure, difforme, odieuse ou plate, apparaît ; nous haussons les épaules. C’est le rire de la raison agile et victorieuse. En voici un autre, celui du tempérament gai, de l’improvisateur bouffon, de l’homme qui reste jeune, enfant et même gamin jusqu’à son dernier jour, et « fait des gambades sur son tombeau ». Il aime les caricatures, il charge les traits des visages, il met en scène des grotesques\footnote{Le bachelier dans le dialogue des {\itshape Mais} ({\itshape Jenny}). — \href{http://www.voltaire-integral.com/Html/27/19\_Cucufin.html}{\dotuline{{\itshape Canonisation de saint Cucufin}}} [\url{http://www.voltaire-integral.com/Html/27/19\_Cucufin.html}]. — {\itshape Conseils à frère Pediculoso}. — {\itshape Diatribe du docteur Akakia}. — {\itshape Conversation de l’empereur de Chine avec frère Rigolo}, etc..}, il les promène en tous sens comme des marionnettes, il n’est jamais las de les reprendre et de les faire danser sous de nouveaux costumes ; au plus fort de sa philosophie, de sa propagande et de sa polémique, il installe en plein vent son théâtre de poche, ses fantoches, un bachelier, un moine, un inquisiteur, Maupertuis, Pompignan, Nonotte, Fréron, le roi David, et tant d’autres qui viennent devant nous pirouetter et gesticuler en habit de scaramouche et d’arlequin. — Quand le talent de la farce s’ajoute ainsi au besoin de la vérité, la plaisanterie devient toute-puissante ; car elle donne satisfaction à des instincts universels et profonds de la nature humaine, à la curiosité maligne, à l’esprit de dénigrement, à l’aversion pour la gêne, à ce fonds de mauvaise humeur que laissent en nous la convention, l’étiquette et l’obligation sociale de porter le lourd manteau de la décence et du respect ; il y a des moments dans la vie où le plus sage n’est pas fâché de le rejeter à demi et même tout à fait. — À chaque page, tantôt avec un mouvement rude de naturaliste hardi, tantôt avec un geste preste de singe polisson, Voltaire écarte la draperie sérieuse ou solennelle, et nous montre l’homme, pauvre bimane, dans quelles attitudes\footnote{\href{http://www.voltaire-integral.com/Html/19/ignorance.htm}{\dotuline{{\itshape Dictionnaire philosophique}, article {\itshape Ignorance}}} [\url{http://www.voltaire-integral.com/Html/19/ignorance.htm}]. — \href{http://www.voltaire-integral.com/Html/21/11CHESTE.html}{\dotuline{{\itshape Les oreilles du comte de Chesterfield}}} [\url{http://www.voltaire-integral.com/Html/21/11CHESTE.html}]. — {\itshape L’homme aux quarante écus.} chap. VII et XI.}. Swift seul a risqué de pareils tableaux. À l’origine ou au terme de tous nos sentiments exaltés, quelles crudités physiologiques ! Quelle disproportion entre notre raison si faible et nos instincts si forts ! Dans quels bas-fonds de garde-robe la politique et la religion vont-elles cacher leur linge sale   De tout cela il faut rire pour ne pas pleurer, et encore, sous ce rire, il y a des larmes ; il finit en ricanement ; il recouvre la tristesse profonde, la pitié douloureuse. À ce degré et en de tels sujets, il n’est plus qu’un effet de l’habitude et du parti pris, une manie de la verve, un état fixe de la machine nerveuse lancée à travers tout, sans frein et à toute vitesse. — Prenons-y garde pourtant : la gaieté est encore un ressort, le dernier en France qui maintienne l’homme debout, le meilleur pour garder à l’âme son ton, sa résistance et sa force, le plus intact dans un siècle où les hommes, les femmes elles-mêmes, se croyaient tenus de mourir en personnes de bonne compagnie, avec un sourire et sur un bon mot\footnote{Bachaumont, III, 194. (Mort du comte de Maugiron.)}.\par
Quand le talent de l’écrivain rencontre ainsi l’inclination du public, peu importe qu’il dévie et glisse, puisque c’est sur la pente universelle. Il a beau s’égarer ou se salir ; il n’en convient que mieux à son auditoire, et ses défauts lui servent autant que ses qualités. — Après une première génération d’esprits sains, voici la seconde, où l’équilibre mental n’est plus exact. Diderot, dit Voltaire, est « un four trop chaud qui brûle tout ce qu’il cuit » ; ou plutôt, c’est un volcan en éruption qui, pendant quarante ans, dégorge les idées de tout ordre et de toute espèce, bouillonnantes et mêlées, métaux précieux, scories grossières, boues fétides ; le torrent continu se déverse à l’aventure, selon les accidents du terrain, mais toujours avec l’éclat rouge et les fumées âcres d’une lave ardente. Il ne possède pas ses idées, mais ses idées le possèdent ; il les subit ; pour en réprimer la fougue et les ravages, il n’a pas ce fond solide de bon sens pratique, cette digue intérieure de prudence sociale qui, chez Montesquieu et même chez Voltaire, barre la voie aux débordements. Tout déborde chez lui, hors du cratère trop plein, sans choix, par la première fissure ou crevasse qui se rencontre, selon les hasards d’une lecture, d’une lettre, d’une conversation, d’une improvisation, non pas en petits jets multipliés comme chez Voltaire, mais en larges coulées qui roulent aveuglément sur le versant le plus escarpé du siècle. Non seulement il descend ainsi jusqu’au fond de la doctrine antireligieuse et antisociale, avec toute la raideur de la logique et du paradoxe, plus impétueusement et plus bruyamment que d’Holbach lui-même ; mais encore il tombe et s’étale dans le bourbier du siècle qui est la gravelure, et dans la grande ornière du siècle qui est la déclamation. Dans ses grands romans, il développe longuement l’équivoque sale ou la scène lubrique. La crudité chez lui n’est point atténuée par la malice ou recouverte par l’élégance. Il n’est ni fin, ni piquant ; il ne sait point, comme Crébillon fils, peindre de jolis polissons. C’est un nouveau venu, un parvenu dans le vrai monde ; vous voyez en lui un plébéien, puissant penseur, infatigable ouvrier et grand artiste, que les mœurs du temps ont introduit dans un souper de viveurs à la mode. Il y prend le dé de la conversation, conduit l’orgie, et par contagion, par gageure, dit à lui seul plus d’ordures et plus de « gueulées » que tous les convives\footnote{« Les romans de Crébillon fils étaient à la mode. Mon père causait avec Mme de Puisieux sur la facilité de composer les ouvrages libres ; il prétendait qu’il ne s’agissait que de trouver une idée plaisante, cheville de tout le reste, où le libertinage de l’esprit remplacerait le goût. Elle le défia d’en produire un de ce genre. Au bout de quinze jours, il lui apporta {\itshape Les bijoux indiscrets} et cinquante louis. » ({\itshape Mémoires sur Diderot} par sa fille.) — {\itshape La Religieuse} a une origine semblable ; il s’agissait de mystifier M. de Croismare.}  Pareillement, dans ses drames, dans ses {\itshape Essais sur Claude et Néron}, dans son {\itshape Commentaire sur Sénèque}, dans ses additions à l’{\itshape Histoire philosophique} de Raynal, il force le ton. Ce ton, qui règne alors en vertu de l’esprit classique et de la mode nouvelle, est celui de la rhétorique sentimentale. Diderot le pousse à bout jusque dans l’emphase larmoyante ou furibonde, par des exclamations, des apostrophes, des attendrissements, des violences, des indignations, des enthousiasmes, des tirades à grand orchestre, où la fougue de sa cervelle trouve une issue et un emploi  En revanche, parmi tant d’écrivains supérieurs, il est le seul qui soit un véritable artiste, un créateur d’âmes, un esprit en qui les objets, les événements et les personnages naissent et s’organisent d’eux-mêmes, par leurs seules forces, en vertu de leurs affinités naturelles, involontairement, sans intervention étrangère, de façon à vivre pour eux-mêmes et par eux-mêmes, à l’abri des calculs et en dehors des combinaisons de l’auteur. L’homme qui a écrit les {\itshape Salons}, les {\itshape Petits Romans}, les {\itshape Entretiens, le Paradoxe du Comédien}, surtout {\itshape le Rêve de d’Alembert} et {\itshape le Neveu de Rameau}, est d’espèce unique en son temps. Si alertes et si brillants que soient les personnages de Voltaire, ce sont toujours des mannequins ; leur mouvement est emprunté ; on entrevoit toujours derrière eux l’auteur qui tire la ficelle. Chez Diderot, ce fil est coupé ; il ne parle point par la bouche de ses personnages, ils ne sont pas pour lui des porte-voix ou des pantins comiques, mais des êtres indépendants et détachés, à qui leur action appartient, dont l’accent est personnel, ayant en propre leur tempérament, leurs passions, leurs idées, leur philosophie, leur style et leur âme parfois, comme {\itshape le Neveu de Rameau}, une âme si originale, si complexe, si complète, si vivante et si difforme, qu’elle devient dans l’histoire naturelle de l’homme un monstre incomparable et un document immortel. Il a dit tout sur la nature\footnote{\href{http://classiques.uqac.ca/classiques/Diderot\_denis/d\_Alembert/d\_alembert\_2\_reve/reve\_d\_alembert.html}{\dotuline{{\itshape Le Rêve de d’Alembert}}} [\url{http://classiques.uqac.ca/classiques/Diderot\_denis/d\_Alembert/d\_alembert\_2\_reve/reve\_d\_alembert.html}].}, sur l’art, la morale et la vie\footnote{Le Neveu de Rameau.}, en deux opuscules dont vingt lectures successives n’usent pas l’attrait et n’épuisent pas le sens : trouvez ailleurs, si vous pouvez, un pareil tour de force et un plus grand chef-d’œuvre ; « rien de plus fou et de plus profond\footnote{Paroles de Diderot lui-même, à propos du {\itshape Rêve de d’Alembert.}} ». Voilà l’avantage de ces génies qui n’ont pas l’empire d’eux-mêmes : le discernement leur manque, mais ils ont l’inspiration ; parmi vingt œuvres fangeuses, informes ou malsaines, ils en font une qui est une création, bien mieux une créature, un être animé, viable par lui-même, auprès duquel les autres, fabriqués par les simples gens d’esprit, ne sont que des mannequins bien habillés  C’est pour cela que Diderot est un si grand conteur, un maître du dialogue, en ceci l’égal de Voltaire, et, par un talent tout opposé, croyant tout ce qu’il dit au moment où il le dit, s’oubliant lui-même, emporté par son propre récit, écoutant des voix intérieures, surpris par des répliques qui lui viennent à l’improviste, conduit comme sur un fleuve inconnu par le cours de l’action, par les sinuosités de l’entretien qui se développe en lui à son insu, soulevé par l’afflux des idées et par le sursaut du moment jusqu’aux images les plus inattendues, les plus burlesques ou les plus magnifiques, tantôt lyrique jusqu’à fournir une strophe presque entière à Musset\footnote{L’une des plus belles strophes de {\itshape Souvenir est} presque transcrite (involontairement, je suppose) du dialogue sur Otaïti.}, tantôt bouffon et saugrenu avec des éclats qu’on n’avait point vus depuis Rabelais, toujours de bonne foi, toujours à la merci de son sujet, de son invention et de son émotion, le plus naturel des écrivains dans cet âge de littérature artificielle, pareil à un arbre étranger qui, transplanté dans un parterre de l’époque, se boursoufle et pourrit par une moitié de sa tige, mais dont cinq ou six branches, élancées en pleine lumière, surpassent tous les taillis du voisinage par la fraîcheur de leur sève et par la vigueur de leur jet.\par
Rousseau aussi est un artisan, un homme du peuple mal adapté au monde élégant et délicat, hors de chez lui dans un salon, de plus mal né, mal élevé, sali par sa vilaine et précoce expérience, d’une sensualité échauffée et déplaisante, malade d’âme et de corps, tourmenté par des facultés supérieures et discordantes, dépourvu de tact, et portant les souillures de son imagination, de son tempérament et de son passé jusque dans sa morale la plus austère et dans ses idylles\footnote{\href{http://gallica.bnf.fr/ark:/12148/bpt6k101488g}{\dotuline{{\itshape Nouvelle Héloïse}}} [\url{http://gallica.bnf.fr/ark:/12148/bpt6k101488g}], passim, et notamment la lettre extraordinaire de Julie, Deuxième Partie, n° 15. — {\itshape Émile}, discours du précepteur à Émile et à Sophie, le lendemain de leur mariage. — Lettre de la comtesse de Boufflers à Gustave III, publiée par Geffroy ({\itshape Gustave III et la cour de France}). « Je charge, quoique avec répugnance, le baron de Cederhielm de vous porter un livre qui vient de paraître : ce sont les infâmes mémoires de Rousseau, intitulés {\itshape Confessions.} Il me paraît que ce peut être celles d’un valet de basse-cour, et même au-dessous de cet état, maussade en tout point, lunatique et vicieux de la manière la plus dégoûtante. Je ne reviens pas du culte que je lui ai rendu (car c’en était un) ; je ne me consolerai pas qu’il en ait coûté la vie à l’illustre David Hume qui, pour me complaire, se chargea de conduire en Angleterre cet animal immonde. »} les plus pures ; sans verve d’ailleurs, et en cela le contraire parfait de Diderot, avouant lui-même « que ses idées s’arrangent dans sa tête avec la plus incroyable difficulté, que telle de ses périodes a été tournée et retournée cinq ou six nuits dans sa tête avant qu’elle fût en état d’être mise sur le papier, qu’une lettre sur les moindres sujets lui coûte des heures de fatigue », qu’il ne peut attraper le ton agréable et léger, ni réussir ailleurs que « dans les ouvrages qui demandent du travail\footnote{ \href{http://un2sg4.unige.ch/athena/rousseau/confessions/jjr\_conf\_03.html}{\dotuline{{\itshape Confessions}, partie I, livre III}} [\url{http://un2sg4.unige.ch/athena/rousseau/confessions/jjr\_conf\_03.html}].
 } »  Par contre, dans ce foyer brûlant, sous les prises de cette méditation prolongée et intense, le style, incessamment forgé et reforgé, prend une densité et une trempe qu’il n’a pas ailleurs. On n’a point vu depuis La Bruyère une phrase si pleine, si mâle, où la colère, l’admiration, l’indignation, la passion, réfléchies et concentrées, fassent saillie avec une précision plus rigoureuse et un relief plus fort. Il est presque l’égal de La Bruyère pour la conduite des effets ménagés, pour l’artifice calculé des développements, pour la brièveté des résumés poignants, pour la raideur assommante des ripostes inattendues, pour la multitude des réussites littéraires, pour l’exécution de tous ces morceaux de bravoure, portraits, descriptions, parallèles, invectives, où, comme dans un crescendo musical, la même idée, diversifiée par une série d’expressions toujours plus vives, atteint ou dépasse dans la note finale tout ce qu’elle comporte d’énergie et d’éclat. Enfin, ce qui manque à La Bruyère, ses morceaux s’enchaînent ; il écrit, non seulement des pages, mais encore des livres ; il n’y a pas de logicien plus serré. Sa démonstration se noue, maille à maille, pendant un, deux, trois volumes, comme un énorme filet sans issue, où, bon gré, mal gré, on reste pris. C’est un systématique qui, replié sur lui-même et les yeux obstinément fixés sur son rêve ou sur son principe, s’y enfonce chaque jour davantage, en dévide une à une les conséquences, et tient toujours sous sa main le réseau entier. N’y touchez pas. Comme une araignée effarouchée et solitaire, il a tout ourdi de sa propre substance, avec les plus chères convictions de son esprit, avec les plus intimes émotions de son cœur. Au moindre choc, il frémit, et, dans la défense, il est terrible\footnote{Lettre à M. de Beaumont.}, hors de lui\footnote{\href{http://gallica.bnf.fr/ark:/12148/bpt6k1014873}{\dotuline{{\itshape Émile}}} [\url{http://gallica.bnf.fr/ark:/12148/bpt6k1014873}], lettre IV, 193. « Il faut bien que les gens du monde se déguisent ; s’ils se montraient tels qu’ils sont, ils feraient horreur, etc. »}, venimeux même, par exaspération contenue, par sensibilité blessée, acharné sur l’adversaire qu’il étouffe dans les fils tenaces et multipliés de sa toile, mais plus redoutable encore à lui-même qu’à ses ennemis, bientôt enlacé dans son propre rets\footnote{Voyez notamment son livre intitulé {\itshape Rousseau juge de Jean-Jacques}, son affaire avec Hume, et les derniers livres des {\itshape Confessions.}}, persuadé que la France et l’univers sont conjurés contre lui, déduisant avec une subtilité prodigieuse toutes les preuves de cette conspiration chimérique, à la fin désespéré par son roman trop plausible, et s’étranglant dans le lacs admirable qu’à force de logique et d’imagination il s’est construit.\par
Avec de telles armes on court risque de se tuer, mais on est bien puissant. Rousseau l’a été, autant que Voltaire, et l’on peut dire que la seconde moitié du siècle lui appartient. Étranger, protestant, original de tempérament, d’éducation, de cœur, d’esprit et de mœurs, à la fois philanthrope et misanthrope, habitant d’un monde idéal qu’il a bâti à l’inverse du monde réel, il se trouve à un point de vue nouveau. Nul n’est si sensible aux vices et aux maux de la société présente. Nul n’est si touché du bonheur et des vertus de la société future. C’est pourquoi il a deux prises sur l’esprit public, l’une par la satire, l’autre par l’idylle  Sans doute aujourd’hui ces deux prises sont moindres ; la substance qu’elles saisissaient s’est dérobée ; nous ne sommes plus les auditeurs auxquels il s’adressait. Les célèbres discours sur l’influence des lettres et sur l’origine de l’inégalité nous semblent des amplifications de collège ; il nous faut un effort de volonté pour lire la {\itshape Nouvelle Héloïse.} L’auteur nous rebute par la continuité de son aigreur ou par l’exagération de son enthousiasme. Il est toujours dans les extrêmes, tantôt maussade et le sourcil froncé, tantôt la larme à l’œil et levant de grands bras au ciel. L’hyperbole, la prosopopée et les autres machines littéraires jouent chez lui trop souvent et de parti pris. Nous sommes tentés de voir en lui tantôt un sophiste qui s’ingénie, tantôt un rhéteur qui s’évertue, tantôt un prédicateur qui s’échauffe, c’est-à-dire, dans tous les cas, un acteur qui soutient une thèse, prend des attitudes et cherche des effets. Enfin, sauf dans les {\itshape Confessions}, son style nous fatigue vite ; il est trop étudié, incessamment tendu. L’auteur est toujours auteur, et communique son défaut à ses personnages ; sa {\itshape Julie} plaide et disserte pendant vingt pages de suite sur le duel, sur l’amour, sur le devoir, avec une logique, un talent et des phrases qui feraient honneur à un académicien moraliste. Partout des lieux communs, des thèmes généraux, des enfilades de sentences et de raisonnements abstraits, c’est-à-dire des vérités plus ou moins vides et des paradoxes plus ou moins creux. Le moindre fait circonstancié, des anecdotes, des traits de mœurs, feraient bien mieux notre affaire ; c’est qu’aujourd’hui nous préférons l’éloquence précise des choses à l’éloquence lâche des mots. Au dix-huitième siècle, il en était autrement, et, pour tout écrivain, ce style oratoire était justement le costume de cérémonie, l’habit habillé qu’il fallait endosser pour être admis dans la compagnie des honnêtes gens. Ce qui nous semble de l’apprêt n’était alors que de la tenue ; en un siècle classique, la période parfaite et le développement soutenu sont des convenances et par suite des obligations. — Notez d’ailleurs que cette draperie littéraire qui nous cache aujourd’hui la vérité ne la cachait pas aux contemporains ; ils voyaient sous elle le trait exact, le détail sensible que nous ne voyons plus. Tous les abus, tous les vices, tous les excès de raffinement et de culture, toute cette maladie sociale et morale que Rousseau flagellait en phrases d’auteur, étaient là sous leurs yeux, dans leurs cœurs, visible et manifestée par des milliers d’exemples quotidiens et domestiques. Pour appliquer la satire, ils n’avaient qu’à regarder ou à se souvenir. Leur expérience complétait le livre, et, par la collaboration de ses lecteurs, l’auteur avait la puissance qui lui manque aujourd’hui. Mettons-nous à leur place, et nous retrouverons leurs impressions. Ses boutades, ses sarcasmes, les duretés de toute espèce qu’il adresse aux grands, aux gens à la mode et aux femmes, son ton raide et tranchant font scandale, mais ne déplaisent pas. Au contraire, après tant de compliments, de fadeurs et de petits vers, tout cela réveille le palais blasé ; c’est la sensation d’un vin fort et rude, après un long régime d’orgeat et de cédrats confits. Aussi son premier discours contre les arts et les lettres « prend tout de suite par-dessus les nues ». Mais son idylle touche les cœurs encore plus fortement que ses satires. Si les hommes écoutent le moraliste qui gronde, ils se précipitent sur les pas du magicien qui les charme ; les femmes surtout, les jeunes gens sont à celui qui leur fait voir la terre promise. Tous les mécontentements accumulés, la fatigue du présent, l’ennui, le dégoût vague, une multitude de désirs enfouis jaillissent, pareils à des eaux souterraines sous le coup de sonde qui pour la première fois les appelle au jour. Ce coup de sonde, Rousseau l’a donné juste et à fond, par rencontre et par génie. Dans une société tout artificielle, où les gens sont des pantins de salon et où la vie consiste à parader avec grâce d’après un modèle convenu, il prêche le retour à la nature, l’indépendance, le sérieux, la passion, les effusions, la vie mâle, active, ardente, heureuse et libre en plein soleil et au grand air. Quel débouché pour les facultés comprimées, pour la riche et large source qui coule toujours au fond de l’homme et à qui ce joli monde ne laisse pas d’issue   Une femme de la cour a vu près d’elle l’amour tel qu’on le pratique alors, simple goût, parfois simple passe-temps, pure galanterie, dont la politesse exquise recouvre mal la faiblesse, la froideur et parfois la méchanceté, bref des aventures, des amusements et des personnages comme en décrit Crébillon fils. Un soir, au moment de partir pour le bal de l’Opéra, elle trouve sur la toilette la {\itshape Nouvelle Héloïse}\footnote{\href{http://un2sg4.unige.ch/athena/rousseau/confessions/jjr\_conf\_11.html}{\dotuline{{\itshape Confessions}, partie II, livre XI}} [\url{http://un2sg4.unige.ch/athena/rousseau/confessions/jjr\_conf\_11.html}]. « Les femmes s’enivrèrent du livre et de l’auteur, au point qu’il y en avait peu, même dans les hauts rangs, dont je n’eusse fait la conquête, si je l’eusse entreprise. J’ai de cela des preuves que je ne veux pas écrire et qui, sans avoir eu besoin de l’expérience, autorisent mon opinion. » Cf. G. Sand, \href{http://gallica.bnf.fr/ark:/12148/bpt6k293789/f77}{\dotuline{{\itshape Histoire de ma vie}, I, 73}} [\url{http://gallica.bnf.fr/ark:/12148/bpt6k293789/f77}].}, je ne m’étonne point si elle fait attendre d’heure en heure ses chevaux et ses gens, si, à quatre heures du matin, elle ordonne de dételer, si elle passe le reste de la nuit à lire, si elle est étouffée par ses larmes ; pour la première fois, elle vient de voir un homme qui aime  Pareillement, si vous voulez comprendre le succès de l’{\itshape Émile}, rappelez-vous les enfants que nous avons décrits, de petits Messieurs brodés, dorés, pomponnés, poudrés à blanc, garnis d’une épée à nœud, le chapeau sous le bras, faisant la révérence, offrant la main, étudiant devant la glace les attitudes charmantes, répétant des compliments appris, jolis mannequins en qui tout est l’œuvre du tailleur, du coiffeur, du précepteur et du maître à danser ; à côté d’eux, de petites Madames de six ans, encore plus factices, serrées dans un corps de baleine, enharnachées d’un lourd panier rempli de crin et cerclé de fer, affublées d’une coiffure haute de deux pieds, véritables poupées auxquelles on met du rouge et dont chaque matin la mère s’amuse un quart d’heure pour les laisser toute la journée aux femmes de chambre\footnote{Estampe de Moreau, {\itshape Les petits parrains. —} Berquin, {\itshape passim}, entre autres {\itshape L’épée. —} Remarquez les phrases toutes faites, le style d’auteur habituel aux enfants, dans Berquin et Mme de Genlis.}. Cette mère vient de lire l’{\itshape Émile} ; rien d’étonnant si tout de suite elle déshabille la pauvrette, et fait le projet de nourrir elle-même son prochain enfant. — C’est par ces contrastes que Rousseau s’est trouvé si fort. Il faisait voir l’aurore à des gens qui ne s’étaient jamais levés qu’à midi, le paysage à des yeux qui ne s’étaient encore arrêtés que sur des salons et des palais, le jardin naturel à des hommes qui ne s’étaient jamais promenés qu’entre des charmilles tondues et des plates-bandes rectilignes, la campagne, la solitude, la famille, le peuple, les plaisirs affectueux et simples à des citadins lassés par la sécheresse du monde, par l’excès et les complications du luxe, par la comédie uniforme que, sous cent bougies, ils jouaient tous les soirs chez eux ou chez autrui\footnote{Description du soleil levant dans {\itshape Émile}, de l’Élysée (un jardin naturel) dans {\itshape la Nouvelle Héloïse. —} Voyez surtout dans \href{http://gallica.bnf.fr/ark:/12148/bpt6k1014873}{\dotuline{{\itshape Émile}}} [\url{http://gallica.bnf.fr/ark:/12148/bpt6k1014873}], fin du livre IV, les plaisirs de Rousseau s’il était riche.}. Des auditeurs ainsi disposés ne distinguent pas nettement entre l’emphase et la sincérité, entre la sensibilité et la sensiblerie. Ils suivent leur auteur, comme un révélateur, comme un prophète, jusqu’au bout de son monde idéal, encore plus pour ses exagérations que pour ses découvertes, aussi loin sur la route de l’erreur que dans la voie de la vérité.\par
Ce sont là les grandes puissances littéraires du siècle. Avec des réussites moindres, et par des combinaisons de toute sorte, les éléments qui ont formé les talents principaux forment aussi les talents secondaires : au-dessous de Rousseau, les écrivains éloquents et sensibles, Bernardin de Saint-Pierre, Raynal, Thomas, Marmontel, Mably, Florian, Dupaty, Mercier, Mme de Staël ; au-dessous de Voltaire, les gens d’esprit vif et piquant, Duclos, Piron, Galiani, le président de Brosses, Rivarol, Chamfort, et, à parler exactement, tout le monde. Chaque fois qu’une veine de talent, si mince qu’elle soit, jaillit de terre, c’est pour propager, porter plus avant la doctrine nouvelle ; on trouverait à peine deux ou trois petits ruisseaux qui coulent en sens contraire, le journal de Fréron, une comédie de Palissot, une satire de Gilbert.\par
La philosophie s’insinue et déborde par tous les canaux publics et secrets, par les manuels d’impiété, les {\itshape Théologies portatives} et les romans lascifs qu’on colporte sous le manteau, par les petits vers malins, les épigrammes et les chansons qui chaque matin sont la nouvelle du jour, par les parades de la foire\footnote{Voyez déjà dans Marivaux (\href{http://gallica.bnf.fr/ark:/12148/bpt6k101465n}{\dotuline{{\itshape La double inconstance}}} [\url{http://gallica.bnf.fr/ark:/12148/bpt6k101465n}]) la satire de la cour, des courtisans et du grand monde gâté, opposé aux petites gens qui ont conservé la bonté primitive, villageois et villageoises.} et les harangues d’académie, par la tragédie et par l’opéra, depuis le commencement jusqu’à la fin du siècle, depuis l’{\itshape Œdipe} de Voltaire jusqu’au {\itshape Tarare} de Beaumarchais. Il semble qu’il n’y ait plus qu’elle au monde ; du moins elle est partout et elle inonde tous les genres littéraires ; on ne s’inquiète pas si elle les déforme, il suffit qu’ils lui servent de conduits. En 1763, dans la tragédie de {\itshape Manco-Capac}\footnote{ Bachaumont, I, 254.
 }, « le principal rôle, écrit un contemporain, est celui d’un sauvage qui débite en vers tout ce que nous avons lu épars dans l’{\itshape Émile} et le {\itshape Contrat social} sur les rois, sur la liberté, sur les droits de l’homme, sur l’inégalité des conditions ». Ce vertueux sauvage sauve le fils du roi sur lequel un grand-prêtre levait le poignard, puis, désignant tour à tour le grand-prêtre et lui-même, il s’écrie : « Voilà l’homme civil ; voici l’homme sauvage. » Sur ce vers, applaudissements, grand succès, tellement que la pièce est demandée à Versailles et jouée devant la cour.\par
Il reste à dire la même chose avec adresse, éclat, gaieté, verve et scandale : ce sera le {\itshape Mariage de Figaro.} Jamais la pensée du siècle ne s’est montrée sous un déguisement qui la rendît plus visible, ni sous une parure qui la rendît plus attrayante. Le titre est la {\itshape Folle journée}, et en effet c’est une soirée de folie, un après-souper comme il y en avait alors dans le beau monde, une mascarade de Français en habits d’Espagnols, avec un défilé de costumes, des décors changeants, des couplets, un ballet, un village qui danse et qui chante, une bigarrure de personnages, gentilshommes, domestiques, duègnes, juges, greffiers, avocats, maîtres de musique, jardiniers, pâtoureaux, bref un spectacle pour les oreilles, pour les yeux, pour tous les sens, le contraire de la comédie régnante, où trois personnages de carton, assis sur des fauteuils classiques, échangent des raisonnements didactiques dans un salon abstrait. Bien mieux, c’est un imbroglio où l’action surabonde, parmi des intrigues qui se croisent, se cassent et se renouent, à travers un pêle-mêle de travestissements, de reconnaissances, de surprises, de méprises, de sauts par la fenêtre, de prises de bec et de soufflets, tout cela dans un style étincelant où chaque phrase scintille par toutes ses facettes, où les répliques semblent taillées par une main de lapidaire, où les yeux s’oublieraient à contempler les brillants multipliés du langage, si l’esprit n’était entraîné par la rapidité du dialogue et par la pétulance de l’action. Mais voici un bien autre attrait, le plus pénétrant de tous pour un monde qui raffole de Parny ; selon le comte d’Artois dont je n’ose citer le mot, c’est l’appel aux sens, l’éveil des sens qui fait toute la verdeur et toute la saveur de la pièce. Le fruit mûrissant, savoureux, suspendu à la branche, n’y tombe pas, mais semble toujours sur le point de tomber ; toutes les mains se tendent pour le cueillir, et la volupté un peu voilée, mais d’autant plus provocante, pointe, de scène en scène, dans la galanterie du comte, dans le trouble de la comtesse, dans la naïveté de Fanchette, dans les gaillardises de Figaro, dans les libertés de Suzanne, pour s’achever dans la précocité de Chérubin. Joignez à cela un double sens perpétuel, l’auteur caché derrière ses personnages, la vérité mise dans la bouche d’un grotesque, des malices enveloppées dans des naïvetés, le maître dupé, mais sauvé du ridicule par ses belles façons, le valet révolté, mais préservé de l’aigreur par sa gaieté, et vous comprendrez comment Beaumarchais a pu jouer l’ancien régime devant les chefs de l’ancien régime, mettre sur la scène la satire politique et sociale, attacher publiquement sous chaque abus un mot qui devient proverbe et qui fait pétard\footnote{« Il fallait un calculateur pour remplir la place, ce fut un danseur qui l’obtint. — C’est un grand abus que de vendre les charges. — Oui, on ferait bien mieux de les donner pour rien. — Il n’y a que les petits hommes qui craignent les petits écrits. — Le hasard fit les distances, l’esprit seul peut tout changer. — Courtisan, on dit que c’est un métier bien difficile. Recevoir, prendre et demander, voilà le secret en trois mots, etc. » — Et tout le monologue de Figaro, toutes les scènes avec Bridoison.}, ramasser en quelques traits toute la polémique des philosophes contre les prisons d’État, contre la censure des écrits, contre la vénalité des charges, contre les privilèges de naissance, contre l’arbitraire des ministres, contre l’incapacité des gens en place, bien mieux, résumer en un seul personnage toutes les réclamations publiques, donner le premier rôle à un plébéien, bâtard, bohème et valet, qui, à force de dextérité, de courage et de bonne humeur, se soutient, surnage, remonte le courant, file en avant sur sa petite barque, esquive le choc des gros vaisseaux, et devance même celui de son maître en lançant à chaque coup de rames une pluie de bons mots sur tous ses rivaux  Après tout, en France du moins, l’esprit est la première puissance. Il suffit toujours que la littérature se mette au service de la philosophie. Devant leur complicité, le public ne fait guère de résistance, et la maîtresse n’a pas de peine à convaincre ceux que la servante a déjà séduits.
\chapterclose


\chapteropen

\chapter[{Chapitre II. Le public en France.}]{Chapitre II. \\
Le public en France.}


\chaptercont

\section[{I. L’aristocratie. — Ordinairement elle répugne aux nouveautés. — Conditions de cette répugnance. — Exemple en Angleterre.}]{I. L’aristocratie. — Ordinairement elle répugne aux nouveautés. — Conditions de cette répugnance. — Exemple en Angleterre.}

\noindent Encore faut-il que ce public veuille bien se laisser convaincre et séduire ; il ne croit que lorsqu’il est disposé à croire, et, dans le succès des livres, sa part est souvent plus grande que celle de l’auteur. Quand vous parlez à des hommes de religion ou de politique, presque toujours leur opinion est faite ; leurs préjugés, leurs intérêts, leur situation les ont engagés d’avance ; ils ne vous écoutent que si vous leur dites tout haut ce qu’ils pensent tout bas. Proposez de démolir le grand édifice social pour le rebâtir à neuf sur un plan tout opposé : ordinairement vous n’aurez pour auditeurs que les gens mal logés ou sans gîte, ceux qui vivent dans les soupentes et les caves, ou qui couchent à la belle étoile, dans les terrains vagues, aux alentours de la maison. Quant au commun des habitants dont le logis est étroit, mais passable, ils craignent les déménagements, ils tiennent à leurs habitudes. La difficulté sera plus grande encore auprès de la haute classe qui occupe tous les beaux appartements ; pour qu’elle accepte votre projet, il faudra que son aveuglement ou son désintéressement soient extrêmes. — En Angleterre, elle s’aperçoit très vite du danger. La philosophie a beau y être précoce et indigène ; elle ne s’y acclimate pas. En 1729, Montesquieu écrivait sur son carnet de voyage : « Point de religion en Angleterre ; quatre ou cinq de la Chambre des Communes vont à la messe ou au sermon de la Chambre… Si quelqu’un parle de religion, tout le monde se met à rire. Un homme ayant dit de mon temps : Je crois cela comme {\itshape article de foi}, tout le monde se mit à rire… Il y a un comité pour considérer l’état de la religion, mais cela est regardé comme ridicule. » Cinquante ans plus tard, l’esprit public s’est retourné ; « tous ceux qui ont sur leur tête un bon toit et sur leur dos un bon habit\footnote{ Mot de Macaulay.
 } » ont vu la portée des nouvelles doctrines. En tout cas, ils sentent que des spéculations de cabinet ne doivent pas devenir des prédications de carrefour. L’impiété leur semble une indiscrétion ; ils considèrent la religion comme le ciment de l’ordre public. C’est qu’ils sont eux-mêmes des hommes publics, engagés dans l’action, ayant part au gouvernement, instruits par l’expérience quotidienne et personnelle. La pratique les a prémunis contre les chimères des théoriciens ; ils ont éprouvé par eux-mêmes combien il est difficile de mener et de contenir les hommes. Ayant manié la machine, ils savent comment elle joue, ce qu’elle vaut, ce qu’elle coûte, et ne sont point tentés de la jeter au rebut, pour en essayer une autre qu’on dit supérieure, mais qui n’existe encore que sur le papier. Le baronnet ou squire, qui est {\itshape justice} sur son domaine, n’a pas de peine à démêler dans le ministre de la paroisse son collaborateur indispensable et son allié naturel. Le duc ou marquis qui siège à la Chambre Haute à côté des évêques a besoin de leurs votes pour faire passer un bill, et de leur assistance pour rallier à son parti les quinze mille curés qui disposent des voix rurales. Ainsi tous ont la main sur quelque rouage social, grand ou petit, principal ou accessoire, ce qui leur donne le sérieux, la prévoyance et le bon sens. Quand on opère sur les choses réelles, on n’est pas tenté de planer dans le monde imaginaire ; par cela seul qu’on est à l’ouvrage sur la terre solide, on répugne aux promenades aériennes dans l’espace vide. Plus on est occupé, moins on rêve, et, pour des hommes d’affaires, la géométrie du {\itshape Contrat social} n’est qu’un pur jeu de l’esprit pur.

\section[{II. Les conditions contraires se rencontrent en France. — Désœuvrement de la haute classe. — La philosophie semble un exercice d’esprit. — De plus, elle est l’aliment de la conversation. — La conversation philosophique au XVIIIe siècle. — Sa supériorité et son charme. — Attrait qu’elle exerce.}]{II. Les conditions contraires se rencontrent en France. — Désœuvrement de la haute classe. — La philosophie semble un exercice d’esprit. — De plus, elle est l’aliment de la conversation. — La conversation philosophique au XVIII\textsuperscript{e} siècle. — Sa supériorité et son charme. — Attrait qu’elle exerce.}

\noindent Tout au rebours en France. « J’y arrivai en 1774\footnote{Stendhal, \href{http://gallica.bnf.fr/ark:/12148/bpt6k6934f}{\dotuline{{\itshape Rome, Naples et Florence}}} [\url{http://gallica.bnf.fr/ark:/12148/bpt6k6934f}], 371.}, dit un gentilhomme anglais, sortant de la maison de mon père qui ne rentrait jamais du Parlement qu’à trois heures du matin, que je voyais occupé toute la matinée à corriger des épreuves de ses discours pour les journaux, et qui, après nous avoir embrassés à la hâte et d’un air distrait, courait à un dîner politique… En France, je trouvai les hommes de la plus haute naissance jouissant du plus beau loisir. Ils voyaient les ministres, mais c’était pour leur adresser des choses aimables et en recevoir les respects ; du reste aussi étrangers aux affaires de la France qu’à celles du Japon », et encore plus aux affaires locales qu’aux affaires générales, ne connaissant leurs paysans que par les comptes de leur régisseur. Si l’un d’eux, avec le titre de gouverneur, allait dans une province, on a vu que c’était pour la montre ; pendant que l’intendant administrait, il représentait avec grâce et magnificence, recevait, donnait à dîner. Recevoir, donner à dîner, entretenir agréablement des hôtes, voilà tout l’emploi d’un grand seigneur ; c’est pourquoi la religion et le gouvernement ne sont pour lui que des sujets d’entretien. D’ailleurs, la conversation est entre lui et ses pareils, et on a le droit de tout dire en bonne compagnie. Ajoutez que la mécanique sociale tourne d’elle-même, comme le soleil, de temps immémorial, par sa propre force ; sera-t-elle dérangée par des paroles de salon ? En tout cas, ce n’est pas lui qui la mène, il n’est pas responsable de son jeu. Ainsi point d’arrière-pensée inquiète, point de préoccupations moroses. Légèrement, hardiment, il marche sur les pas de ses philosophes ; détaché des choses, il peut se livrer aux idées, à peu près comme un jeune homme de famille qui, sortant du collège, saisit un principe, tire les conséquences, et se fait un système, sans s’embarrasser des applications\footnote{Morellet, \href{http://gallica.bnf.fr/ark:/12148/bpt6k64768q/f181}{\dotuline{{\itshape Mémoire}, I, 139}} [\url{http://gallica.bnf.fr/ark:/12148/bpt6k64768q/f181}] (sur les écrits et les entretiens de Diderot, d’Holbach et des athées). « Tout semblait alors innocent dans cette philosophie qui demeurait contenue dans l’enceinte des spéculations, et ne cherchait, dans ses plus grandes hardiesses, qu’un exercice paisible de l’esprit. »}.\par
Rien de plus agréable que cet élan spéculatif. L’esprit plane sur les sommets comme s’il avait des ailes ; d’un regard, il embrasse les plus vastes horizons, toute la vie humaine, toute l’économie du monde, le principe de l’univers, des religions, des sociétés. Aussi bien, comment causer si on s’abstient de philosophie ? Qu’est-ce qu’un cercle où la haute politique et la critique supérieure ne sont point admises ? Et quel motif peut réunir des gens d’esprit, sinon le désir d’agiter ensemble les questions majeures   Depuis deux siècles en France la conversation touche à tout cela ; c’est pourquoi elle a tant d’attraits. Les étrangers n’y résistent pas ; ils n’ont rien de pareil chez eux ; Lord Chesterfield la propose en exemple. « Elle roule toujours, dit-il, sur quelques points d’histoire, de critique ou même de philosophie, qui conviennent mieux à des êtres raisonnables que nos dissertations anglaises sur le temps et sur le whist. » Rousseau, si grognon, avoue « qu’un article de morale ne serait pas mieux discuté dans une société de philosophes que dans celle d’une jolie femme de Paris ». Sans doute, on y babille ; mais, au plus fort des caquets, qu’un homme de poids avance un propos grave ou agite une question sérieuse, l’attention commence à se fixer à ce nouvel objet ; hommes, femmes, vieillards, jeunes gens, tous se prêtent à le considérer sous toutes les faces, et l’on est étonné du bon sens et de la raison qui sortent comme à l’envi de ces têtes folâtres ». — À dire vrai, dans cette fête permanente que cette brillante société se donne à elle-même, la philosophie est la pièce principale. Sans la philosophie, le badinage ordinaire serait fade. Elle est une sorte d’opéra supérieur où défilent et s’entrechoquent, tantôt en costume grave, tantôt sous un déguisement comique, toutes les grandes idées qui peuvent intéresser une tête pensante. La tragédie du temps n’en diffère presque pas, sauf en ceci qu’elle a toujours l’air solennel et ne se joue qu’au théâtre ; l’autre prend toutes les physionomies et se trouve partout, puisque la conversation est partout. Point de dîner ni de souper où elle n’ait sa place. On est à table au milieu d’un luxe délicat, parmi des femmes souriantes et parées, avec des hommes instruits et aimables, dans une société choisie où l’intelligence est prompte et le commerce est sûr. Dès le second service, la verve fait explosion, les saillies éclatent, les esprits flambent ou pétillent. Peut-on s’empêcher au dessert de mettre en bons mots les choses les plus graves ? Vers le café arrive la question de l’immortalité de l’âme et de l’existence de Dieu.\par
Pour nous figurer cette conversation hardie et charmante, il nous faut prendre les correspondances, les petits traités, les dialogues de Diderot et de Voltaire, ce qu’il y a de plus vif, de plus fin, de plus piquant et de plus profond dans la littérature du siècle ; encore n’est-ce là qu’un résidu, un débris mort. Toute cette philosophie écrite a été dite, et elle a été dite avec l’accent, l’entrain, le naturel inimitable de l’improvisation, avec les gestes et l’expression mobile de la malice et de l’enthousiasme. Aujourd’hui, refroidie et sur le papier, elle enlève et séduit encore ; qu’était-ce alors qu’elle sortait vivante et vibrante de la bouche de Voltaire et de Diderot ? Il y avait chaque jour à Paris des soupers comme celui que décrit Voltaire\footnote{\href{http://www.voltaire-integral.com/Html/21/0840ECU.html}{\dotuline{{\itshape L’Homme aux quarante écus}}} [\url{http://www.voltaire-integral.com/Html/21/0840ECU.html}]. — Cf. Voltaire, {\itshape Mémoires}, soupers chez Frédéric II. « Jamais on ne parla en aucun lieu du monde avec tant de liberté de toutes les superstitions des hommes. »} où « deux philosophes, trois dames d’esprit, M. Pinto célèbre juif, le chapelain de la chapelle réformée de l’ambassadeur batave, le secrétaire de M. le prince Galitzin du rite grec, un capitaine suisse calviniste », réunis autour de la même table, échangeaient, pendant quatre heures, leurs anecdotes, leurs traits d’esprit, leurs remarques et leurs jugements « sur tous les objets de curiosité, de science et de goût ». Chez le baron d’Holbach arrivaient tour à tour les étrangers les plus lettrés et les plus marquants, Hume, Wilkes, Sterne, Beccaria, Verri, l’abbé Galiani, Garrick, Franklin, Priestley, Lord Shelburne, le comte de Creutz, le prince de Brunswick, le futur électeur de Mayence. Pour fonds de société le baron avait Diderot, Rousseau, Helvétius, Duclos, Raynal, Suard, Marmontel, Boulanger, le chevalier de Chastellux, La Condamine le voyageur, Barthez le médecin, Rouelle le chimiste. Deux fois par semaine, le dimanche et le jeudi, « sans préjudice des autres jours, on dîne chez lui à deux heures, selon l’usage, usage significatif qui réserve pour l’entretien et la gaieté toute la force de l’homme et les meilleurs moments du jour. En ce temps-là on ne relègue pas la conversation dans les heures tardives et nocturnes ; on n’est pas forcé comme aujourd’hui de la subordonner aux exigences du travail et de l’argent, de la Chambre et de la Bourse : causer est la grande affaire  « Arrivés à deux heures, dit Morellet, nous y étions encore presque tous de sept à huit heures du soir…\footnote{Morellet, \href{http://gallica.bnf.fr/ark:/12148/bpt6k64768q/f175}{\dotuline{{\itshape Mémoires}, I, 133}} [\url{http://gallica.bnf.fr/ark:/12148/bpt6k64768q/f175}].} C’est là qu’il fallait entendre la conversation la plus libre, la plus animée et la plus instructive qui fut jamais… Point de hardiesse politique ou religieuse qui ne fût mise en avant et discutée {\itshape pro et contrà…} Souvent un seul y prenait la parole et proposait sa théorie paisiblement et sans être interrompu. D’autres fois c’était un combat singulier en forme, dont tout le reste de la société était tranquille spectateur. C’est là que j’ai entendu Roux et Darcet exposer leur théorie de la terre, Marmontel les excellents principes qu’il a rassemblés dans les {\itshape Éléments de la Littérature}, Raynal nous dire à livres, sous et deniers, le commerce des Espagnols à la Vera-Cruz et de l’Angleterre dans ses colonies », Diderot improviser sur les arts, la morale, la métaphysique, avec cette fougue incomparable, cette surabondance d’expression, ce débordement d’images et de logique, ces trouvailles de style, cette mimique qui n’appartenaient qu’à lui, et dont trois ou quatre seulement de ses écrits nous ont conservé l’image affaiblie. Au milieu d’eux le secrétaire d’ambassade de Naples, Galiani, un joli nain de génie, sorte de « Platon ou de Machiavel avec la verve et les gestes d’arlequin », inépuisable en contes, admirable bouffon, parfait sceptique, « ne croyant à rien, en rien, sur rien\footnote{Galiani, {\itshape Correspondance}, passim.} », pas même à la philosophie nouvelle, défie les athées du salon, rabat leurs dithyrambes par des calembours, et, sa perruque à la main, les deux jambes croisées sur le fauteuil où il perche, leur prouve par un apologue comique qu’ils « {\itshape raisonnent} ou {\itshape résonnent}, sinon comme des {\itshape cruches}, du moins comme des {\itshape cloches} », en tout cas presque aussi mal que des théologiens. « C’était, dit un assistant, la plus piquante chose du monde ; cela valait le meilleur des spectacles et le meilleur des amusements. »\par
Le moyen, pour des nobles qui passent leur vie à causer, de ne pas rechercher des gens qui causent si bien ! Autant vaudrait prescrire à leurs femmes, qui tous les soirs vont au théâtre et jouent la comédie à domicile, de ne pas attirer chez elles les acteurs et chanteurs en renom, Jelyotte, Sainval, Préville, le jeune Molé qui, malade et ayant besoin de réconfortants, « reçoit en un jour plus de deux mille bouteilles de vins de toute espèce des différentes dames de la cour », Mlle Clairon qui, enfermée par ordre à For l’Évêque, y attire « une affluence prodigieuse de carrosses », et trône, au milieu du plus beau cercle, dans le plus bel appartement de la prison\footnote{Bachaumont, III, 93 (1766), II, 202 (1765).}. Quand on prend la vie de la sorte, un philosophe avec toutes ses idées est aussi nécessaire dans un salon qu’un lustre avec toutes ses lumières. Il fait partie du luxe nouveau ; on l’exporte. Les souverains, au milieu de leur magnificence et au plus fort de leurs succès, l’appellent chez eux pour goûter une fois dans leur vie le plaisir de la conversation libre et parfaite. Lorsque Voltaire arrive en Prusse, Frédéric II veut lui baiser la main, l’adule comme une maîtresse, et plus tard, après tant d’égratignures mutuelles, ne peut se passer de causer par lettres avec lui. Catherine II fait venir Diderot, et, tous les jours, pendant deux ou trois heures, joue avec lui le grand jeu de l’esprit. Gustave III, en France, est intime avec Marmontel, et reçoit comme un honneur insigne une visite de Rousseau\footnote{Geffroy, {\itshape Gustave III}, I, 114.}. On dit avec vérité de Voltaire qu’il a dans la main « son brelan de rois quatrième », Prusse, Suède, Danemark, Russie, sans compter les cartes secondaires, princes et princesses, grands-ducs et margraves qu’il tient dans son jeu. — Visiblement, dans ce monde, le premier rôle est aux écrivains ; on ne s’entretient que de leurs faits et gestes ; on ne se lasse pas de leur rendre hommage. « Ici, écrit Hume à Robertson\footnote{Villemain, \href{http://gallica.bnf.fr/ark:/12148/bpt6k29601j}{\dotuline{{\itshape Tableau de la littérature au dix-huitième siècle}}} [\url{http://gallica.bnf.fr/ark:/12148/bpt6k29601j}]. IV, 409.}, je ne me nourris que d’ambroisie, ne bois que du nectar, ne respire que de l’encens et ne marche que sur des fleurs. Tout homme que je rencontre, et encore plus toute femme, croirait manquer au plus indispensable des devoirs, si elle ne m’adressait un long et ingénieux discours à ma gloire. » Présenté à Versailles, le futur Louis XVI âgé de dix ans, le futur Louis XVIII âgé de huit ans et le futur Charles X âgé de quatre ans, lui récitent chacun un compliment sur son livre  Je n’ai pas besoin de conter le retour de Voltaire, son triomphe, l’Académie en corps venant le recevoir, sa voiture arrêtée par la foule, les rues comblées, les fenêtres, les escaliers et les balcons chargés d’admirateurs, au théâtre une salle enivrée qui ne cesse de l’applaudir, au dehors un peuple entier qui le reconduit avec des vivats, dans ses salons une affluence aussi continue que chez le roi, de grands seigneurs pressés contre la porte et tendant l’oreille pour saisir un de ses mots, de grandes dames debout sur la pointe du pied épiant son moindre geste\footnote{Grimm, \href{http://gallica.bnf.fr/ark:/12148/bpt6k23423v}{\dotuline{{\itshape Correspondance littéraire}, IV}} [\url{http://gallica.bnf.fr/ark:/12148/bpt6k23423v}], 176. — Comte de Ségur, {\itshape Mémoires.} I, 113.}. « Pour concevoir ce que j’éprouvais, dit un des assistants, il faudrait être dans l’atmosphère où je vivais : c’était celle de l’enthousiasme. » — « Je lui ai parlé », ce seul mot faisait alors du premier venu un personnage. En effet, il avait vu le merveilleux chef d’orchestre qui, depuis cinquante ans, menait le bal tourbillonnant des idées graves ou court-vêtues, et qui, toujours en scène, toujours en tête, conducteur reconnu de la conversation universelle, fournissait les motifs, donnait le ton, marquait la mesure, imprimait l’élan et lançait le premier coup d’archet.

\section[{III. Autre effet du désœuvrement. — L’esprit sceptique, libertin et frondeur. — Anciens ressentiments et mécontentements nouveaux contre l’ordre établi. — Sympathies pour les théories qui l’attaquent. — Jusqu’à quel point elles sont adoptées.}]{III. Autre effet du désœuvrement. — L’esprit sceptique, libertin et frondeur. — Anciens ressentiments et mécontentements nouveaux contre l’ordre établi. — Sympathies pour les théories qui l’attaquent. — Jusqu’à quel point elles sont adoptées.}

\noindent Notez les cris qui l’accueillent : « Vive l’auteur de la Henriade, le défenseur des Calas, l’auteur de la Pucelle ! » Personne aujourd’hui ne pousserait le premier ni surtout le dernier bravo. Ceci nous indique la pente du siècle ; on demandait alors aux écrivains non seulement des pensées, mais encore des pensées d’opposition. Désœuvrer une aristocratie, c’est la rendre frondeuse ; l’homme n’accepte volontairement la règle que lorsqu’il contribue à l’appliquer. Voulez-vous le rallier au gouvernement, faites qu’il y ait part. Sinon, devenu spectateur, il n’en verra que les fautes, il n’en sentira que les froissements, il ne sera disposé qu’à critiquer et à siffler. En effet, dans ce cas, il est comme au théâtre ; or au théâtre on veut s’amuser, et d’abord ne pas être gêné. Que de gênes dans l’ordre établi, et même dans tout ordre établi   En premier lieu, la religion. Pour les aimables « oisifs » que décrit Voltaire\footnote{\href{http://www.voltaire-integral.com/Html/21/09BABYLO.html}{\dotuline{{\itshape Princesse de Babylone}}} [\url{http://www.voltaire-integral.com/Html/21/09BABYLO.html}]. — Cf. \href{http://www.voltaire-integral.com/Html/10/23\_Mondain.html}{\dotuline{{\itshape le Mondain}}} [\url{http://www.voltaire-integral.com/Html/10/23\_Mondain.html}].}, pour « les cent mille personnes qui n’ont rien à faire qu’à jouer et à se divertir », elle est le pédagogue le plus déplaisant, toujours grondeur, hostile au plaisir sensible, hostile au raisonnement libre, brûlant les livres qu’on voudrait lire, imposant des dogmes qu’on n’entend plus. À proprement parler, c’est la bête noire ; quiconque lui lance un trait est le bien venu. — Autre chaîne, la morale des sexes. Elle semble bien lourde à des hommes de plaisir, aux compagnons de Richelieu, Lauzun et Tilly, aux héros de Crébillon fils, à tout ce monde galant et libertin pour qui l’irrégularité est devenue la règle. Nos gens de bel air adopteront sans difficulté une théorie qui justifie leur pratique. Ils seront bien aises d’apprendre que le mariage est une convention et un préjugé. Ils applaudiront Saint-Lambert lorsqu’à souper, levant un verre de champagne, il proposera le retour à la nature et aux mœurs d’Otaïti\footnote{Mme d’Épinay, Ed. Boiteau, I, 216, souper chez Mlle Quinault la comédienne, avec Saint-Lambert, le prince de…, Duclos et Mme d’Épinay.}. — Dernière entrave, le gouvernement, la plus gênante de toutes ; car elle applique les autres et comprime l’homme de tout son poids joint à tout leur poids. Celui-ci est absolu, il est centralisé, il procède par faveurs, il est arriéré, il commet des fautes, il a des revers : que de causes de mécontentement en peu de mots ! Il a contre lui les ressentiments vagues et sourds des anciens pouvoirs qu’il a dépossédés, états provinciaux, parlements, grands personnages de province, nobles de la vieille roche qui, comme des Mirabeau, conservent l’esprit féodal, et, comme le père de Chateaubriand, appellent l’abbé Raynal un « maître homme ». Il a contre lui le dépit de tous ceux qui se croient frustrés dans la distribution des emplois et des grâces, non seulement la noblesse de province qui reste à la porte\footnote{Par exemple, le père de Marmont, gentilhomme, militaire, qui, ayant gagné à 28 ans la croix de Saint-Louis, quitte le service, parce que tout l’avancement est pour les gens de cour. — Retiré dans sa terre, il est libéral et enseigne à lire à son fils dans le Compte rendu de Necker. (Maréchal Marmont, \href{http://gallica.bnf.fr/ark:/12148/bpt6k284352}{\dotuline{{\itshape Mémoires}, I}} [\url{http://gallica.bnf.fr/ark:/12148/bpt6k284352}], 9.)} pendant que la noblesse de cour mange le festin royal, mais encore le plus grand nombre des courtisans, réduits à des bribes, tandis que les favoris du petit cercle intime engloutissent tous les gros morceaux. Il a contre lui la mauvaise humeur de ses administrés, qui, lui voyant prendre le rôle de la Providence et se charger de tout, mettent tout à sa charge, la cherté du pain comme le délabrement d’une route. Il a contre lui l’humanité nouvelle, qui, dans les salons les plus élégants, l’accuse de maintenir les restes surannés d’une époque barbare, impôts mal assis, mal répartis et mal perçus, lois sanguinaires, procédures aveugles, supplices atroces, persécution des protestants, lettres de cachet, prisons d’État. — Et j’ai laissé de côté ses excès, ses scandales, ses désastres et ses hontes, Rosbach, le traité de Paris, Mme du Barry, la banqueroute. — Le dégoût vient ; décidément, tout est mal. Les spectateurs de la pièce se disent entre eux, non seulement que la pièce est mauvaise, mais que le théâtre est mal construit, incommode, étouffant, étriqué, à tel point que, pour être à l’aise, il faudra le démolir et le rebâtir depuis les caves jusqu’aux greniers.\par
À ce moment interviennent les architectes nouveaux, avec leurs raisonnements spécieux et leurs plans tout faits, démontrant que tous les grands édifices publics, religions, morales, sociétés, ne peuvent manquer d’être grossiers et malsains, puisque jusqu’ici ils ont été bâtis de pièces et de morceaux, au fur et à mesure, le plus souvent par des fous et par des barbares, en tout cas par des maçons, et toujours au hasard, à tâtons, sans principes. Pour eux, ils sont architectes et ils ont des principes, à savoir la raison, la nature, les droits de l’homme, principes simples et féconds que chacun peut entendre et dont il suffit de tirer les conséquences pour substituer aux informes bâtisses du passé l’édifice admirable de l’avenir. — La tentation est grande pour des mécontents, peu dévots, épicuriens et philanthropes. Ils adoptent aisément des maximes qui semblent conformes à leurs secrets désirs ; du moins ils les adoptent en théorie et en paroles. Les grands mots, liberté, justice, bonheur public, dignité de l’homme, sont si beaux et en outre si vagues ! Quel cœur peut s’empêcher de les aimer, et quelle intelligence peut en prévoir toutes les applications ? D’autant plus que, jusqu’au dernier moment, la théorie ne descend pas des hauteurs, qu’elle reste confinée dans ses abstractions, qu’elle ressemble à une dissertation académique, qu’il s’agit toujours de l’homme en soi, du contrat social, de la cité imaginaire et parfaite. Y a-t-il à Versailles un courtisan qui refuse de décréter l’égalité dans Salente   Entre les deux étages de l’esprit humain, le supérieur où se tissent les raisonnements purs et l’inférieur où siègent les croyances actives, la communication n’est ni complète ni prompte. Nombre de principes ne sortent pas de l’étage supérieur ; ils y demeurent à l’état de curiosités ; ce sont des mécaniques délicates, ingénieuses, dont volontiers on fait parade, mais dont presque jamais on ne fait emploi. Si parfois le propriétaire les transporte à l’étage inférieur, il ne s’en sert qu’à demi ; des habitudes établies, des intérêts ou des instincts antérieurs et plus forts en restreignent l’usage. En cela il n’est pas de mauvaise foi, il est homme ; chacun de nous professe des vérités qu’il ne pratique pas. Un soir, le lourd avocat Target ayant pris du tabac dans la tabatière de la maréchale de Beauvau, celle-ci, dont le salon est un petit club démocratique, reste suffoquée d’une familiarité si monstrueuse. Plus tard, Mirabeau, qui rentre chez lui ayant voté l’abolition des titres de noblesse, saisit son valet de chambre par l’oreille et lui crie en riant de sa voix tonnante : « Ah çà ! drôle, j’espère bien que pour toi je suis toujours monsieur le comte. » — Ceci montre jusqu’à quel point, dans une tête aristocratique, les nouvelles théories sont admises. Elles occupent tout l’étage supérieur, et là elles tissent, avec un bruit joyeux, la trame de la conversation interminable ; leur bourdonnement est continu pendant tout le siècle ; jamais on n’a vu dans les salons un tel déroulement de phrases générales et de beaux mots. Il en tombe quelque chose dans l’étage inférieur, ne serait-ce que la poussière, je veux dire l’espérance, la confiance en l’avenir, la croyance à la raison, le goût de la vérité, la bonne volonté juvénile et généreuse, l’enthousiasme qui passe vite, mais qui peut s’exalter parfois jusqu’à l’abnégation et au dévouement.

\section[{IV. Leur propagation dans la haute classe. — Progrès de l’incrédulité en religion. — Ses origines. — Elle éclate sous la Régence. — Irritation croissante contre le clergé. — Le matérialisme dans les salons. — Vogue des sciences. — Opinion finale sur la religion. — Scepticisme du haut clergé.}]{IV. Leur propagation dans la haute classe. — Progrès de l’incrédulité en religion. — Ses origines. — Elle éclate sous la Régence. — Irritation croissante contre le clergé. — Le matérialisme dans les salons. — Vogue des sciences. — Opinion finale sur la religion. — Scepticisme du haut clergé.}

\noindent Suivons les progrès de la philosophie dans la haute classe. C’est la religion qui reçoit les premiers et les plus grands coups. Le petit groupe de sceptiques qu’on apercevait à peine sous Louis XIV a fait ses recrues dans l’ombre ; en 1698, la Palatine, mère du Régent, écrit déjà « qu’on ne voit presque plus maintenant un seul jeune homme qui ne veuille être athée\footnote{Aubertin, {\itshape l’Esprit public au dix-huitième siècle}, 7.} ». Avec la Régence, « l’incrédulité se produit au grand jour ». « Je ne crois pas, dit encore la Palatine en 1722, qu’il y ait à Paris, tant parmi les ecclésiastiques que parmi les laïques, cent personnes qui aient la véritable foi ou qui croient même en Notre Seigneur. Cela fait frémir… » Déjà, dans le monde, le rôle d’un ecclésiastique est difficile ; il semble qu’il y soit un pantin ou un plastron\footnote{Montesquieu, {\itshape Lettres persanes.} (Lettre 61.) — Cf. Voltaire ({\itshape Dîner du comte de Boulainvilliers}).}. « Dès que nous y paraissons, dit l’un d’eux, on nous fait disputer ; on nous fait entreprendre, par exemple, de prouver l’utilité de la prière à un homme qui ne croit pas en Dieu, la nécessité du jeûne à un homme qui a nié toute sa vie l’immortalité de l’âme ; l’entreprise est laborieuse, et les rieurs ne sont pas pour nous. » — Bientôt le scandale prolongé des billets de confession et l’obstination des évêques à ne point souffrir qu’on taxe les biens ecclésiastiques soulèvent l’opinion contre le clergé et, par suite, contre la religion. « Il est à craindre, dit Barbier en 1751, que cela ne finisse sérieusement ; on pourrait voir un jour dans ce pays-ci une révolution pour embrasser la religion protestante\footnote{Aubertin, 281, 282, 285, 289.}. » — « La haine contre les prêtres, écrit d’Argenson en 1753, va au dernier excès. À peine osent-ils se montrer dans les rues sans être hués… Comme notre nation et notre siècle sont bien autrement éclairés » qu’au temps de Luther, « on ira jusqu’où on doit aller ; on bannira tous prêtres, tout sacerdoce, toute révélation, tout mystère… » — « On n’ose plus parler pour le clergé dans les bonnes compagnies ; on est honni et regardé comme des familiers de l’inquisition… Les prêtres ont remarqué cette année une diminution de plus d’un tiers dans le nombre de leurs communiants. Le collège des jésuites devient désert ; cent vingt pensionnaires ont été retirés à ces moines si tarés… On a observé aussi pendant le carnaval de Paris que jamais on n’avait vu tant de masques au bal contrefaisant les habits ecclésiastiques, en évêques, abbés, moines, religieuses. » — L’antipathie est si grande, que les plus médiocres livres font fureur dès qu’ils sont antichrétiens et condamnés comme tels. En 1748, un ouvrage de Toussaint en faveur de la religion naturelle, {\itshape les Mœurs}, devient tout d’un coup si célèbre, « qu’il n’y a personne dans un certain monde, dit Barbier, homme ou femme se piquant d’esprit, qui ne veuille le voir. On s’aborde aux promenades en se disant : Avez-vous lu {\itshape les Mœurs} ? » — Dix ans plus tard on a dépassé le déisme. « Le matérialisme, dit encore Barbier, c’est le grand grief… » — « Presque tous les gens d’étude et de bel esprit, écrit d’Argenson, se déchaînent contre notre sainte religion… Elle est secouée de toutes parts, et, ce qui anime davantage les incrédules, ce sont les efforts que font les dévots pour obliger à croire. Ils font des livres qu’on ne lit guère ; on ne dispute plus, on se rit de tout, et l’on persiste dans le matérialisme. » Horace Walpole\footnote{Horace Walpole, {\itshape Letters and correspondance}, 27 septembre 1765, 18 et 28 octobre, 19 novembre 1766.} qui en 1765 revient en France et dont le bon sens prévoit le danger, s’étonne de tant d’imprudence : « J’ai dîné aujourd’hui, dit-il, avec une douzaine de savants ; quoique tous les domestiques fussent là pour nous servir, la conversation a été beaucoup plus libre, même sur l’Ancien Testament, que je ne le souffrirais à ma propre table en Angleterre, n’y eût-il pour l’écouter qu’un valet de pied. » On dogmatise partout. « Le rire est aussi démodé que les pantins ou le bilboquet. Nos bonnes gens n’ont plus le temps d’être gais, ils ont trop à faire ; il faut d’abord qu’ils mettent par terre Dieu et le roi ; tous et chacun, hommes et femmes, s’emploient en conscience à la démolition. À leurs yeux je suis un infidèle, parce que j’ai encore quelques croyances debout. » — « Savez-vous ce que sont les philosophes et ce que ce mot signifie ici ? D’abord il comprend presque tout le monde ; ensuite il désigne les gens qui se déclarent ennemis du papisme, mais qui, pour la plupart, ont pour objet le renversement de toute religion. » — Ces savants, je leur demande pardon, ces philosophes sont insupportables, superficiels, arrogants et fanatiques. Ils prêchent incessamment, vous ne sauriez croire avec quelle liberté, et leur doctrine avouée est l’athéisme… Voltaire lui-même ne les satisfait plus ; une de leurs dames prosélytes me disait de lui : il est bigot, c’est un déiste. »\par
Ceci est bien fort, et pourtant nous ne sommes pas au bout : car, jusqu’ici, l’impiété est moins une conviction qu’une mode. Walpole, bon observateur, ne s’y est pas trompé. « D’après ce que je vous ai dit de leurs opinions religieuses ou plutôt irréligieuses, ne concluez pas, écrit-il, que les personnes de qualité, les hommes du moins, soient athées. Heureusement pour eux, pauvres âmes ! Ils ne sont pas capables de pousser le raisonnement si loin, mais ils disent oui à beaucoup d’énormités, parce que c’est la mode et qu’ils ne savent comment contredire. » À présent que « les petits maîtres sont surannés » et que tout le monde « est philosophe », ils sont philosophes ; il faut bien être comme tout le monde. Mais ce qu’ils goûtent dans le matérialisme nouveau, c’est le piquant du paradoxe et la liberté du plaisir. Ce sont des écoliers de bonne maison qui font des niches à leur précepteur ecclésiastique. Ils empruntent aux théories savantes de quoi lui mettre un bonnet d’âne, et leurs fredaines leur plaisent davantage quand elles sont assaisonnées d’impiété. Un seigneur de la cour ayant vu le tableau de Doyen, {\itshape Sainte Geneviève et les pestiférés}, fait le lendemain venir le peintre dans sa petite maison chez sa maîtresse\footnote{{\itshape Journal et mémoires de Collé} publiés par H. Bonhomme, II, 24 (octobre 1755) et III, 165 (octobre 1767).} : « Je voudrais, lui dit-il, que vous peignissiez madame sur une escarpolette qu’un évêque mettrait en branle ; vous me placeriez, moi, de façon que je sois à portée de voir les jambes de cette belle enfant, et même mieux, si vous voulez égayer davantage votre tableau. » La chanson si leste sur {\itshape Marotte} « court avec fureur »   « au bout de quinze jours que je l’ai donnée, dit Collé, je n’ai rencontré personne qui n’en eût une copie ; et c’est le vaudeville, je veux dire l’assemblée du clergé, qui fait toute sa vogue »  Plus un livre licencieux est irréligieux, plus il est goûté ; quand on ne peut l’avoir imprimé, on le copie. Collé compte « peut-être deux mille copies manuscrites de {\itshape la Pucelle} de Voltaire, qui en un mois se sont répandues à Paris ». Les magistrats eux-mêmes ne brûlent que pour la forme. « Ne croyez pas que monsieur l’exécuteur des hautes œuvres ait la permission de jeter au feu les livres dont les titres figurent dans l’arrêt de la Cour. Messieurs seraient très fâchés de priver leurs bibliothèques d’un exemplaire de chacun de ces ouvrages qui leur revient de droit, et le greffier y supplée par quelques malheureux rôles de chicane dont la provision ne lui manque pas\footnote{{\itshape Correspondance littéraire} par Grimm (septembre, octobre 1770).}. »\par
Mais, à mesure que le siècle avance, l’incrédulité, moins bruyante, devient plus ferme. Elle se retrempe aux sources ; les femmes elles-mêmes se prennent d’engouement pour les sciences. En 1782\footnote{Mme de Genlis, \href{http://gallica.bnf.fr/ark:/12148/bpt6k88451j}{\dotuline{{\itshape Adèle et Théodore}}} [\url{http://gallica.bnf.fr/ark:/12148/bpt6k88451j}], I, 312.} un personnage de Mme de Genlis écrit : « Il y a cinq ans je les avais laissées ne songeant qu’à leur parure, à l’arrangement de leurs soupers ; je les retrouve toutes savantes et beaux-esprits. » Dans le cabinet d’une dame à la mode, on trouve, à côté d’un petit autel dédié à la Bienfaisance ou à l’Amitié, un dictionnaire d’histoire naturelle, des traités de physique et de chimie. Une femme ne se fait plus peindre en déesse sur un nuage, mais dans un laboratoire, assise parmi des équerres et des télescopes\footnote{E. et J. de Goncourt, {\itshape la Femme au dix-huitième siècle}, 371-373. — Bachaumont, I, 224 (13 avril 1763).}. La marquise de Nesle, la comtesse de Brancas, la comtesse de Pons, la marquise de Polignac sont chez Rouelle lorsqu’il entreprend de fondre et de volatiliser le diamant. Des sociétés de vingt et vingt-cinq personnes se forment dans les salons, pour suivre un cours de physique ou de chimie appliquée, de minéralogie ou de botanique. À la séance publique de l’Académie des Inscriptions, les femmes du monde applaudissent des dissertations sur le bœuf Apis, sur le rapport des langues égyptienne, phénicienne et grecque. Enfin, en 1786, elles se font ouvrir les portes du Collège de France. Rien ne les rebute. Plusieurs manient la lancette et même le scalpel ; la marquise de Voyer voit disséquer, et la jeune comtesse de Coigny dissèque de ses propres mains. — Sur ce fondement qui est celui de la philosophie régnante, l’incrédulité mondaine prend un nouveau point d’appui. Vers la fin du siècle\footnote{Mme de Genlis, {\itshape Adèle et Théodore}, II, 326.} « on voit de jeunes personnes, qui sont dans le monde depuis six ou sept ans, se piquer ouvertement d’irréligion, croyant que l’impiété tient lieu d’esprit, et qu’être athée, c’est être philosophe ». Sans doute il y a beaucoup de déistes, surtout depuis Rousseau ; mais je ne crois pas que, sur cent personnes du monde, on trouve encore à Paris dix chrétiens ou chrétiennes. « Depuis dix ans\footnote{Mercier, {\itshape Tableau de Paris}, III, 44.}, dit Mercier en 1783, le beau monde ne va plus à la messe ; on n’y va que le dimanche pour ne pas scandaliser les laquais, et les laquais savent qu’on n’y va que pour eux. » Le duc de Coigny\footnote{Metra, {\itshape Correspondance secrète}, XII, 387 (7 mars 1785).} dans ses terres auprès d’Amiens, refuse de laisser prier pour lui, et menace son curé, s’il prend cette licence, de le faire jeter en bas de sa chaire ; son fils tombe malade, il empêche qu’on apporte les sacrements ; ce fils meurt, il interdit les obsèques et fait enterrer le corps dans son jardin ; malade lui-même, il ferme sa porte à l’évêque d’Amiens qui se présente douze fois pour le voir, et meurt comme il a vécu. — Sans doute un tel scandale est noté, c’est-à-dire rare ; presque tous et presque toutes « allient à l’indépendance des idées la convenance des formes\footnote{E. et J. de Goncourt, {\itshape ib.} 456. — Vicomtesse de Noailles, {\itshape Vie de la princesse de Poix}, née de Beauvau.} ». Quand la femme de chambre annonce : « Mme la duchesse, le bon Dieu est là, permettez-vous qu’on le fasse entrer ? Il souhaiterait avoir l’honneur de vous administrer » ; on conserve les apparences. On introduit l’importun, on est poli avec lui. Si on l’esquive, c’est sous un prétexte décent ; mais, si on lui complaît, ce n’est que par bienséance ; « à Surate, quand on meurt, on doit tenir la queue d’une vache dans sa main ». Jamais société n’a été plus détachée du christianisme. À ses yeux une religion positive n’est qu’une superstition populaire, bonne pour les enfants et les simples, non pour « les honnêtes gens » et les grandes personnes. Vous devez un coup de chapeau à la procession qui passe, mais vous ne lui devez qu’un coup de chapeau.\par
Dernier signe et le plus grave de tous  Si les curés qui travaillent et sont du peuple ont la foi du peuple, les prélats qui causent et sont du monde ont les opinions du monde. Et je ne parle pas seulement ici des abbés de salon, courtisans domestiques, colporteurs de nouvelles, faiseurs de petits vers, complaisants de boudoir, qui dans une compagnie servent d’écho, et de salon à salon servent de porte-voix ; un écho, un porte-voix ne fait que répéter la phrase, sceptique ou non, qu’on lui jette\footnote{L’abbé de Lattaignant, chanoine de Reims, auteur de poésies légères et de chansons de soupers, « vient de faire pour le théâtre de Nicolet une parade où l’intrigue est soutenue de beaucoup de saillies polissonnes très à la mode aujourd’hui. Les courtisans qui donnent le ton à ce théâtre trouvent le chanoine de Reims délicieux. » (Bachaumont, IV, 174, novembre 1768.)}. Il s’agit des dignitaires, et, sur ce point, tous les témoignages sont d’accord. Au mois d’août 1767, l’abbé Bassinet, grand vicaire de Cahors, prononçant dans la chapelle du Louvre le panégyrique de Saint Louis\footnote{Bachaumont, III, 253. — Chateaubriand, {\itshape Mémoires}, I, 246.}, « a supprimé jusqu’au signe de la croix. Point de texte, aucune citation de l’Ecriture, pas un mot du bon Dieu ni des saints. Il n’a envisagé Louis IX que du côté des vertus politiques, guerrières et morales. Il a frondé les croisades, il en a fait voir l’absurdité, la cruauté, l’injustice même. Il a heurté de front et sans aucun ménagement la cour de Rome ». D’autres « évitent en chaire le nom de Jésus-Christ et ne parlent plus que du législateur des chrétiens ». Dans le code que l’opinion du monde et la décence sociale imposent au clergé, un observateur délicat\footnote{Chamfort, 279.} précise ainsi les distinctions de rang et les nuances de conduite : « Un simple prêtre, un curé doit croire un peu, sinon on le trouverait hypocrite ; mais il ne doit pas non plus être sûr de son fait, sinon on le trouverait intolérant. Au contraire, le grand vicaire peut sourire à un propos contre la religion, l’évêque en rira tout à fait, le cardinal y joindra son mot. » — « Il y a quelque temps, raconte la chronique, on disait à l’un des plus respectables curés de Paris : Croyez-vous que les évêques, qui mettent toujours la religion en avant, en aient beaucoup   Le bon pasteur, après avoir hésité un moment, répondit : Il peut y en avoir quatre ou cinq qui croient encore. » — Pour qui connaît leur naissance, leurs sociétés, leurs habitudes et leurs goûts, cela n’a rien d’invraisemblable. « Dom Collignon, représentant de l’abbaye de Mettlach, seigneur haut justicier et curé de Valmunster », bel homme, beau diseur, aimable maître de maison, évite le scandale, et ne fait dîner ses deux maîtresses à sa table qu’en petit comité ; du reste aussi peu dévot que possible et bien moins encore que le vicaire savoyard, « ne voyant du mal que dans l’injustice et dans le défaut de charité », ne considérant la religion que comme un établissement politique et un frein moral. J’en citerais nombre d’autres, M. de Grimaldi, le jeune et galant évêque du Mans, qui prend pour grands vicaires ses jeunes et galants camarades de classe, et fait de sa maison de campagne à Coulans un rendez-vous de jolies dames\footnote{Merlin de Thionville, {\itshape Vie et correspondance}, par Jean Reynaud. ({\itshape La chartreuse du Val-Saint-Pierre.} Tout le passage est à lire.) — {\itshape Souvenirs manuscrits} par le chancelier Pasquier.}. Concluez des mœurs aux croyances  En d’autres cas on n’a pas la peine de conclure. Chez le cardinal de Rohan, chez M. de Brienne, archevêque de Sens, chez M. de Talleyrand, évêque d’Autun, chez l’abbé Maury, défenseur du clergé, le scepticisme est notoire. Rivarol\footnote{Rivarol, {\itshape Mémoires}, I, 344.}, sceptique lui-même, déclare qu’aux approches de la Révolution « les lumières du clergé égalaient celles des philosophes ». — « Le corps qui a le moins de préjugés, dit Mercier\footnote{Mercier, IV, 142. — En Auvergne, dit M. de Montlosier, « je me composai une société de prêtres beaux-esprits dont quelques-uns étaient déistes, et d’autres franchement athées, avec lesquels je m’exerçai à lutter contre mon frère ». ({\itshape Mémoires}, I, 37.)}, qui le croirait ? c’est le clergé. » Et l’archevêque de Narbonne expliquant la résistance du haut clergé en 1791\footnote{M. de la Fayette, {\itshape Mémoires}, III, 58.}, l’attribue, non à la foi, mais au point d’honneur. « Nous nous sommes conduits alors en vrais gentilshommes ; car, de la plupart d’entre nous, on ne peut pas dire que ce fût par religion. »

\section[{V. Progrès de l’opposition en politique. — Ses origines. — Les économistes et les parlementaires. — Ils frayent la voie aux philosophes. — Fronde des salons. — Libéralisme des femmes.}]{V. Progrès de l’opposition en politique. — Ses origines. — Les économistes et les parlementaires. — Ils frayent la voie aux philosophes. — Fronde des salons. — Libéralisme des femmes.}

\noindent De l’autel au trône la distance est courte, et pourtant l’opinion met trente ans à la franchir. Pendant la première moitié du siècle, il n’y a point encore de fronde politique ou sociale. L’ironie des {\itshape Lettres persanes} est aussi mesurée que délicate ; l’{\itshape Esprit des Lois} est conservateur. Quant à l’abbé de Saint-Pierre, on sourit de ses rêveries, et l’Académie le raye de sa liste lorsqu’il s’avise de blâmer Louis XIV. À la fin les économistes d’un côté et les parlementaires de l’autre donnent le signal. — « Vers 1750, dit Voltaire\footnote{\href{http://www.voltaire-integral.com/Html/18/ble.htm}{\dotuline{{\itshape Dictionnaire philosophique}, article {\itshape Blé}}} [\url{http://www.voltaire-integral.com/Html/18/ble.htm}]. — L’ouvrage principal de Quesnay ({\itshape Tableau économique}) est de 1758.}, la nation rassasiée de vers, de tragédies, de comédies, de romans, d’opéras, d’histoires romanesques, de réflexions morales plus romanesques encore, et de disputes sur la grâce et les convulsions, se mit à raisonner sur les blés. » D’où vient la cherté du pain ? Pourquoi le laboureur est-il si misérable ? Quelle est la matière et la limite de l’impôt ? Toute terre ne doit-elle pas payer, et une terre peut-elle payer au-delà de son produit net ? Voilà les questions qui entrent dans les salons sous les auspices du roi, par l’organe de Quesnay, son médecin, « son penseur », fondateur d’un système qui agrandit le prince pour soulager le peuple, et qui multiplie les imposés pour alléger l’impôt. — En même temps, par la porte opposée, arrivent d’autres questions non moins neuves. « La France\footnote{Marquis d’Argenson, {\itshape Mémoires}, IV, 141 ; VI, 320, 465 ; VII, 23 ; VIII, 153 (1752, 1753, 1754). — Le {\itshape discours} de Rousseau {\itshape sur l’inégalité} est aussi de 1753. — Sur ce pas décisif de l’opinion, consultez l’excellent livre d’Aubertin, {\itshape l’Esprit public au dix-huitième siècle.}} est-elle une monarchie tempérée et représentative, ou un gouvernement à la Turque ? Vivons-nous sous la loi d’un maître absolu, ou sommes-nous régis par un pouvoir limité et contrôlé ? » — « Les parlementaires exilés… se sont mis à étudier le droit public dans ses sources, et ils en confèrent comme dans des académies. Dans l’esprit public et par leurs études, s’établit l’opinion que la nation est au-dessus du roi, comme l’Église universelle est au-dessus du pape. » — Le changement est frappant, presque subit. Il y a cinquante ans, dit encore d’Argenson, le public n’était nullement curieux des nouvelles d’État. Aujourd’hui chacun lit sa {\itshape Gazette de Paris}, même dans les provinces. On raisonne à tort et à travers de la politique, mais enfin on s’en occupe. » — Une fois que la conversation a saisi cet aliment, elle ne le lâche plus, et les salons s’ouvrent à la philosophie politique, par suite au Contrat social, à l’Encyclopédie, aux prédications de Rousseau, Mably, d’Holbach, Raynal et Diderot. En 1759, d’Argenson, qui s’échauffe, se croit déjà proche du moment final. « Il nous souffle un vent philosophique de gouvernement libre et antimonarchique ; cela passe dans les esprits, et il peut se faire que ce gouvernement soit déjà dans les têtes pour l’exécuter à la première occasion. Peut-être {\itshape la Révolution} se ferait avec moins de contestations qu’on ne pense ; cela se ferait {\itshape par acclamation}\footnote{La nuit du 4 août 1789 semble prédite ici.}. »\par
Non pas encore ; mais la semence lève. Bachaumont, en 1762, note un déluge de pamphlets, brochures et dissertations politiques, « une fureur de raisonner en matière de finance et de gouvernement ». En 1765, Walpole constate que les athées, qui tiennent alors le dé de la conversation, se déchaînent autant contre les rois que contre les prêtres. Un mot redoutable, celui de {\itshape citoyen}, importé par Rousseau, est entré dans le langage ordinaire, et, ce qui est décisif, les femmes s’en parent comme d’une cocarde. « Vous savez combien je suis citoyenne, écrit une jeune fille à son amie. Comme citoyenne et comme amie, pouvais-je recevoir de plus agréables nouvelles que celles de la santé de ma chère petite et de la paix\footnote{{\itshape Correspondance de Laurette de Malboissière}, publiée par la marquise de la Grange (4 septembre 1762, 8 novembre 1762).} ? » — Autre mot non moins significatif, celui d’{\itshape énergie} qui, jadis ridicule, devient à la mode et se place à tout propos\footnote{Lettre de Mme du Deffand à Mme de Choiseul (citée par Geffroy, {\itshape Gustave III et la cour de France}, I, 279).} — Avec le langage, les sentiments sont changés, et les plus grandes dames passent à l’opposition. En 1771, dit le moqueur Besenval après l’exil du Parlement, « les assemblées de société ou de plaisir étaient devenues de petits États Généraux, où les femmes, transformées en législateurs, établissaient des prémisses et débitaient avec assurance des maximes de droit public. » La comtesse d’Egmont, correspondante du roi de Suède, lui envoie un mémoire sur les lois fondamentales de la France, en faveur du Parlement, dernier défenseur des libertés nationales, contre les attentats du chancelier Maupeou. « M. le chancelier, dit-elle\footnote{Geffroy, {\itshape ib..} I, 232, 241, 245.}, a, depuis six mois, fait apprendre l’histoire de France à des gens qui seraient morts sans l’avoir sue. » — « Je n’en doute pas, sire, ajoute-t-elle ; vous n’abuserez pas de ce pouvoir qu’un peuple enivré vous a confié sans limites… Puisse votre règne devenir l’époque du rétablissement du gouvernement libre et indépendant, mais n’être jamais la source d’une autorité absolue. » Nombre d’autres femmes du premier rang, Mmes de la Marck, de Boufflers, de Brienne, de Mesmes, de Luxembourg, de Croy, pensent et écrivent de même. « Le pouvoir absolu, dit l’une d’elles, est une maladie mortelle qui, en corrompant insensiblement les qualités morales, finit par détruire les États… Les actions des souverains sont soumises à la censure de leurs propres sujets comme à celle de l’univers… La France est détruite, si l’administration présente subsiste\footnote{Geffroy, {\itshape ib.}, I, 267, 281. Lettres de Mme de Boufflers (octobre 1772, juillet 1774).}. » — Lorsque, sous Louis XVI, une nouvelle administration avance et retire des velléités de réformes, leur critique demeure aussi ferme. « Enfance, faiblesse, inconséquence continuelle « écrit une autre\footnote{{\itshape Ibid.}, I, 285. Lettres de Mme de la Marck (1776, 1777, 1779).}, nous changeons sans cesse et pour être plus mal que nous n’étions d’abord. Monsieur et M. le comte d’Artois viennent de voyager dans nos provinces, mais comme ces gens-là voyagent, avec une dépense affreuse et la dévastation sur tout leur passage, n’en rapportant d’ailleurs qu’une graisse surprenante : Monsieur est devenu gros comme un tonneau ; pour M. le comte d’Artois, il y met bon ordre par la vie qu’il mène. » — Un souffle d’humanité en même temps que de liberté a pénétré dans les cœurs féminins. Elles s’intéressent aux pauvres, aux petits, au peuple ; Mme d’Egmont recommande à Gustave III de planter la Dalécarlie en pommes de terre. Lorsque paraît l’estampe publiée au profit des Calas, « toute la France, et même toute l’Europe, s’empresse de souscrire, l’impératrice de Russie, pour 5 000 livres\footnote{Bachaumont, III, 14 (28 mars 1766. — Walpole, 6 octobre 1775).}. » — « L’agriculture, l’économie, les réformes, la philosophie, écrit Walpole, sont de {\itshape bon ton}, même à la cour. » — Le président Dupaty ayant fait un mémoire pour trois innocents condamnés à la « roue », on ne parle plus que de cela dans le monde ; « ces conversations de société, dit une correspondante de Gustave III\footnote{Geffroy, {\itshape ib.} Lettre de Mme de Staël (1786).}, ne sont plus oiseuses, puisque c’est par elles que l’opinion publique se forme. {\itshape Les paroles sont devenues des actions}, et tous les cœurs sensibles vantent avec transport un mémoire que l’humanité anime et qui paraît plein de talent, parce qu’il est plein d’âme ». Lorsque Latude sort de Bicêtre, Mme de Luxembourg, Mme de Boufflers et Mme de Staël veulent dîner avec Mme Legros, l’épicière qui « depuis trois années a remué ciel et terre » pour délivrer le prisonnier. C’est grâce aux femmes, à leur attendrissement, à leur zèle, à la conspiration de leurs sympathies, que M. de Lally parvient à faire réhabiliter son père. Quand elles s’éprennent, elles s’engouent : Mme de Lauzun, si timide, va jusqu’à dire des injures en public à un homme qui parle mal de Necker. — Rappelez-vous qu’en ce siècle les femmes étaient reines, faisaient la mode, donnaient le ton, menaient la conversation, par suite les idées, par suite l’opinion\footnote{Collé, {\itshape Journal}, III, 437 (1770) : « Les femmes ont tellement pris le dessus chez les Français, elles les ont tellement subjugués, qu’ils ne pensent et ne sentent plus que d’après elles. »}. Quand on les trouve en avant sur le terrain politique, on peut être sûr que les hommes suivent : chacune d’elles entraîne avec soi tout son salon.

\section[{VI. Espérances infinies et vagues. — Générosité des sentiments et de la conduite. — Douceur et bonnes intentions du gouvernement. — Aveuglement et optimisme.}]{VI. Espérances infinies et vagues. — Générosité des sentiments et de la conduite. — Douceur et bonnes intentions du gouvernement. — Aveuglement et optimisme.}

\noindent Une aristocratie imbue de maximes humanitaires et radicales, des courtisans hostiles à la cour, des privilégiés qui contribuent à saper les privilèges, il faut voir dans les témoignages du temps cet étrange spectacle. « Il est de principe, dit un contemporain, que tout doit être changé et bouleversé\footnote{{\itshape Correspondance}, par Metra, III, 200 ; IV, 131.} ». Au plus haut, au plus bas, dans les assemblées, dans les lieux publics, on ne rencontre parmi les privilégiés que des opposants et des réformateurs. En 1787, presque tout ce qu’il y avait de marquant dans la pairie se déclara dans le Parlement pour la résistance… J’ai vu mettre en avant dans les dîners qui nous réunissaient alors presque toutes les idées qui devaient bientôt se produire avec tant d’éclat\footnote{{\itshape Souvenirs manuscrits} du chancelier Pasquier.}. » Déjà en 1774, M. de Vaublanc, allant à Metz, trouvait dans la diligence un ecclésiastique et un comte colonel de hussards qui ne cessaient de parler économie politique\footnote{Comte de Vaublanc, {\itshape Souvenirs}, I, 117, 377.}. « C’était alors la mode ; tout le monde était économiste ; on ne s’entretenait que de philosophie, d’économie politique, surtout d’humanité, et des moyens de soulager le bon peuple ; ces deux derniers mots étaient dans toutes les bouches. » Ajoutez-y celui d’égalité ; Thomas, dans un éloge du maréchal de Saxe, disait : « Je ne puis le dissimuler, il était du sang des rois » ; et l’on admirait cette phrase. — Seuls quelques chefs de vieilles familles parlementaires ou seigneuriales conservent le vieil esprit nobiliaire et monarchique ; toute la génération nouvelle est gagnée aux nouveautés. « Pour nous, dit l’un d’eux, jeune noblesse française\footnote{Comte de Ségur, {\itshape Mémoires}, I, 17.}, sans regret pour le passé, sans inquiétude pour l’avenir, nous marchions gaiement sur un tapis de fleurs qui nous cachait un abîme. Riants frondeurs des modes anciennes, de l’orgueil féodal de nos pères et de leurs graves étiquettes, tout ce qui était antique nous paraissait gênant et ridicule. La gravité des anciennes doctrines nous pesait. La riante philosophie de Voltaire nous entraînait en nous amusant. Sans approfondir celle des écrivains plus graves, nous l’admirions comme empreinte de courage et de résistance au pouvoir arbitraire… La liberté, quel que fût son langage, nous plaisait par son courage ; l’égalité, par sa commodité. On trouve du plaisir à descendre tant qu’on croit pouvoir remonter dès qu’on veut ; et, sans prévoyance, nous goûtions à la fois les avantages du patriciat et les douceurs d’une philosophie plébéienne. Ainsi, quoique ce fussent nos privilèges, les débris de notre ancienne puissance que l’on minait sous nos pas, cette petite guerre nous plaisait. Nous n’en éprouvions pas les atteintes, nous n’en avions que le spectacle. Ce n’étaient que combats de plume et de paroles qui ne nous paraissaient pouvoir faire aucun dommage à la supériorité d’existence dont nous jouissions et qu’une possession de plusieurs siècles nous faisait croire inébranlable. Les formes de l’édifice restant intactes, nous ne voyions pas qu’on le minait en dedans. Nous riions des graves alarmes de la vieille cour et du clergé qui tonnaient contre cet esprit d’innovation. Nous applaudissions les scènes républicaines de nos théâtres\footnote{ \noindent Ségur, {\itshape ib.}, I, 151. « J’entendis toute la cour, dans la salle de spectacle du château de Versailles, applaudir avec enthousiasme {\itshape Brutus}, tragédie de Voltaire, et particulièrement ces deux vers :\par
  \begin{poem}
Je suis fils de Brutus et je porte en mon cœur \\
 La liberté gravée et les rois en horreur. » \\-
\end{poem}
  }, les discours philosophiques de nos Académies, les ouvrages hardis de nos littérateurs. » — Si l’inégalité durait encore dans la distribution des charges et des places, « l’égalité commençait à régner dans les sociétés. En beaucoup d’occasions, les titres littéraires avaient la préférence sur les titres de noblesse. Les courtisans, serviteurs de la mode, venaient faire la cour à Marmontel, à d’Alembert, à Raynal. On voyait fréquemment dans le monde des hommes de lettres du deuxième et troisième rang être accueillis et traités avec des égards que n’obtenaient pas les nobles de province… Les institutions restaient monarchiques, mais les mœurs devenaient républicaines. Nous préférions un mot d’éloge de d’Alembert, de Diderot, à la faveur la plus signalée d’un prince… Il était impossible de passer la soirée chez d’Alembert, d’aller à l’hôtel de La Rochefoucauld chez les amis de Turgot, d’assister au déjeuner de l’abbé Raynal, d’être admis dans la société et la famille de M. de Malesherbes, enfin d’approcher de la reine la plus aimable et du roi le plus vertueux, sans croire que nous entrions dans une sorte d’âge d’or dont les siècles précédents ne nous donnaient aucune idée… Nous étions éblouis par le prisme des idées et des doctrines nouvelles, rayonnants d’espérance, brûlants d’ardeur pour toutes les gloires, d’enthousiasme pour tous les talents et bercés des rêves séduisants d’une philosophie qui voulait assurer le bonheur du genre humain. Loin de prévoir des malheurs, des excès, des crimes, des renversements de trônes et de principes, nous ne voyions dans l’avenir que tous les biens qui pouvaient être assurés à l’humanité par le règne de la raison. On laissait un libre cours à tous les écrits réformateurs, à tous les projets d’innovation, aux pensées les plus libérales, aux systèmes les plus hardis. Chacun croyait marcher à la perfection, sans s’embarrasser des obstacles et sans les craindre. Nous étions fiers d’être Français et encore plus d’être Français du dix-huitième siècle… Jamais réveil plus terrible ne fut précédé par un sommeil plus doux et par des songes plus séduisants ».\par
Ils ne s’en tiennent pas à des songes, à de purs souhaits, à des espérances passives. Ils agissent, ils sont vraiment généreux ; il suffit qu’une cause soit belle pour que leur dévouement lui soit acquis. À la nouvelle de l’insurrection américaine, le marquis de la Fayette, laissant sa jeune femme enceinte, s’échappe, brave les défenses de la cour, achète une frégate, traverse l’Océan et vient se battre aux côtés de Washington. « Dès que je connus la querelle, dit-il, mon cœur fut enrôlé et je ne songeai plus qu’à rejoindre mes drapeaux. » Quantité de gentilshommes le suivent. Sans doute ils aiment le danger ; « une probabilité d’avoir des coups de fusil est trop précieuse pour qu’on la néglige\footnote{Duc de Lauzun, 80 (à propos de son expédition en Corse).}. » Mais il s’agit en outre d’affranchir des opprimés ; « c’est comme paladins, dit l’un d’eux, que nous nous montrions philosophes\footnote{Ségur, I, 87.} » et l’esprit chevaleresque se met au service de la liberté  D’autres services, plus sédentaires et moins brillants, ne les trouvent pas moins zélés. Aux assemblées provinciales\footnote{Les assemblées du Berry et de la Haute-Guyenne commencent en 1778 et 1779, celles des autres généralités en 1787. Toutes fonctionnent jusqu’en 1789. (Cf. Léonce de Lavergne, {\itshape Les assemblées provinciales.})}, les plus grands personnages de la province, évêques, archevêques, abbés, ducs, comtes, marquis, joints aux notables les plus opulents et les plus instruits du Tiers-état, en tout un millier d’hommes, bref l’élite sociale, toute la haute classe convoquée par le roi, établit le budget, défend le contribuable contre le fisc, dresse le cadastre, égalise la taille, remplace la corvée, pourvoit à la voirie, multiplie les ateliers de charité, instruit les agriculteurs, propose, encourage et dirige toutes les réformes. J’ai lu les vingt volumes de leurs procès-verbaux : on ne peut voir de meilleurs citoyens, des administrateurs plus intègres, plus appliqués, et qui se donnent gratuitement plus de peine, sans autre objet que le bien public. La bonne volonté est complète. Jamais l’aristocratie n’a été si digne du pouvoir qu’au moment où elle allait le perdre ; les privilégiés, tirés de leur désœuvrement, redevenaient des hommes publics, et, rendus à leur fonction, revenaient à leur devoir. En 1778, dans la première assemblée du Berry, l’abbé de Séguiran\footnote{Léonce de Lavergne, {\itshape ib.}, 26, 55, 183. Le bureau des impôts de l’assemblée provinciale de Tours réclame aussi contre les privilèges en fait d’impôts.}, rapporteur, ose dire que « la répartition de l’impôt doit être un partage fraternel des charges publiques ». En 1780, les abbés, prieurs et chapitres de la même province offrent 60 000 livres de leur argent, et quelques gentilshommes, en moins de vingt-quatre heures, 17 000 livres. En 1787, dans l’assemblée d’Alençon, la noblesse et le clergé se cotisent de 30 000 livres pour soulager d’autant les taillables indigents de chaque paroisse\footnote{{\itshape Procès-verbaux de l’assemblée provinciale de Normandie}, généralité d’Alençon, 252. — Cf. {\itshape Archives nationales}, II, 1149 : en 1778, dans la généralité de Moulins, trente-neuf personnes, la plupart nobles, ajoutent de leur argent 18 950 livres aux 60 000 allouées par le roi pour les routes et ateliers de charité.}. Au mois d’avril 1787, le roi, dans l’Assemblée des Notables, parle de « l’empressement avec lequel les archevêques et évêques ont déclaré ne prétendre à aucune exemption pour leur contribution aux charges publiques ». Au mois de mars 1789, dès l’ouverture des assemblées de bailliage, le clergé tout entier, la noblesse presque tout entière, bref le corps des privilégiés, renonce spontanément à ses privilèges en fait d’impôt. Le sacrifice est voté par acclamation ; ils viennent d’eux-mêmes l’offrir au Tiers-état et il faut voir dans les procès-verbaux manuscrits leur accent généreux et sympathique. « L’ordre de la noblesse du bailliage de Tours, dit le marquis de Lusignan\footnote{ \noindent {\itshape Archives nationales}, procès-verbaux et cahiers des États généraux, t. XLIX, 712, 714 (noblesse et clergé de Dijon). T. XVI, 185 (noblesse d’Auxerre). T. XXIX, 352, 455, 458 (clergé et noblesse du Berry). T. CL, 266 (clergé et noblesse de Tours). T. XXIX, clergé et noblesse de Châteauroux (29 janvier 1789), 572 à 582. T. XIII, 765 (noblesse d’Autun). — Pour l’ensemble, voyez {\itshape Résumé des cahiers}, par Prudhomme, 3 vol.
 }, considérant que ses membres sont hommes et citoyens avant que d’être nobles, ne peut se dédommager, d’une manière plus conforme à l’esprit de justice et de patriotisme qui l’anime, du long silence auquel l’abus du pouvoir ministériel l’avait condamné, qu’en déclarant à ses concitoyens qu’elle n’entend plus jouir à l’avenir d’aucun des privilèges pécuniaires que l’usage lui avait conservés, et qu’elle fait par acclamation le vœu solennel de supporter dans une parfaite égalité, et chacun en proportion de sa fortune, les impôts et contributions générales qui seront consenties par la nation. » — « Je vous le répète, dit le comte de Buzançais au Tiers-état du Berry, nous sommes tous frères, nous voulons partager vos charges… Nous désirons ne porter qu’un seul vœu aux états et, par là, montrer l’union et l’harmonie qui doivent y régner. Je suis chargé de vous offrir de vous réunir à nous pour ne faire qu’un seul cahier. » — « Il faut trois qualités à un député, dit le marquis de Barbançon au nom de la noblesse de Châteauroux : probité, fermeté, connaissances ; les deux premières se trouvent également dans les députés des trois ordres ; mais les connaissances se rencontreront plus généralement dans le Tiers-état, dont l’esprit est exercé aux affaires. » — « Un nouvel ordre de choses se déploie à nos yeux, dit l’abbé Legrand au nom du clergé de Châteauroux ; le voile du préjugé est déchiré, la raison en a pris la place. Elle s’empare de tous les cœurs français, sape par le pied tout ce qui n’était fondé que sur les anciennes opinions et tire sa force d’elle-même. » Non seulement les privilégiés font les avances, mais ils les font sans effort ; ils parlent la même langue que les gens du Tiers, ils sont disciples des mêmes philosophes, ils semblent partir des mêmes principes. La noblesse de Clermont en Beauvoisis\footnote{Prudhomme, {\itshape ib.}, II, 39, 51, 59. — Léonce de Lavergne, 384. En 1788, deux cents gentilshommes des premières familles du Dauphiné signent, conjointement avec le clergé et le Tiers-état de la province, une adresse au roi où se trouve la phrase suivante : « Ni le temps, ni les liens ne peuvent légitimer le despotisme ; {\itshape Les droits des hommes dérivent de la nature seule et sont indépendants de leurs conventions.} »} ordonne à ses députés « de demander avant tout qu’il soit fait une déclaration explicite des droits qui appartiennent à tous les hommes ». La noblesse de Mantes et Meulan affirme que « les principes de la politique sont aussi absolus que ceux de la morale, puisque les uns et les autres ont pour base commune la raison ». La noblesse de Reims demande « que le roi soit supplié de vouloir bien ordonner la démolition de la Bastille ». — Maintes fois, après des vœux et des prévenances semblables, les délégués de la noblesse et du clergé sont accueillis dans les assemblées du Tiers par des battements de mains, « des larmes », des transports. Quand on voit ces effusions, comment ne pas croire à la concorde ? Et comment prévoir qu’on va se battre au premier tournant de la route où, fraternellement, l’on entre la main dans la main ?\par
Ils n’ont pas cette triste sagesse. Ils posent en principe que l’homme, surtout l’homme du peuple, est bon ; pourquoi supposer qu’il puisse vouloir du mal à ceux qui lui veulent du bien ? Ils ont conscience à son endroit de leur bienveillance et de leur sympathie. Non seulement ils parlent de leurs sentiments, mais ils les éprouvent. À ce moment, dit un contemporain\footnote{Lacretelle, {\itshape Histoire de France au dix-huitième siècle}, V, 2.}, « la pitié la plus active remplissait les âmes ; ce que craignaient le plus les hommes opulents, c’était de passer pour insensibles ». L’archevêque de Paris, qu’on poursuivra à coups de pierres, a donné cent mille écus pour améliorer l’Hôtel-Dieu. L’intendant Bertier, qu’on massacrera, a cadastré l’Ile-de-France pour égaliser la taille, ce qui lui a permis d’en abaisser le taux d’abord d’un huitième, puis d’un quart\footnote{{\itshape Procès-verbaux de l’assemblée provinciale de l’Ile-de-France} (1787), 127.}. Le financier Beaujon bâtit un hôpital. Necker refuse les appointements de sa place et prête au trésor deux millions pour rétablir le crédit. Le duc de Charost, dès 1770\footnote{Léonce de Lavergne, {\itshape ib.}, 52, 369.}, abolit sur ses terres les corvées seigneuriales et fonde un hôpital dans sa seigneurie de Meillant. Le prince de Bauffremont, les présidents de Vezet, de Chamolles, de Chaillot, nombre d’autres seigneurs en Franche-Comté, suivent l’exemple du roi en affranchissant leurs serfs\footnote{{\itshape Le cri de la raison}, par Clerget, curé d’Ornans (1789), 258.}.\par
L’évêque de Saint-Claude réclame, malgré son chapitre, l’affranchissement de ses mainmortables. Le marquis de Mirabeau établit dans son domaine du Limousin un bureau gratuit de conciliation pour arranger les procès, et chaque jour, à Fleury, fabrique neuf cents livres de pain économique à l’usage « du pauvre peuple qui se bat à qui en aura\footnote{Lucas de Montigny, {\itshape Mémoires de Mirabeau.} I, 290, 368  Théron de Montaugé, {\itshape L’agriculture et les classes rurales dans le pays Toulousain}, 14.} ». M. de Barrai, évêque de Castres, prescrit à tous ses curés de prêcher et propager la culture des pommes de terre. Le marquis de Guerchy monte avec Arthur Young sur les tas de foin pour apprendre à bien faire une meule. Le comte de Lasteyrie importe en France la lithographie. Nombre de grands seigneurs et de prélats figurent dans les sociétés d’agriculture, écrivent ou traduisent des livres utiles, suivent les applications des sciences, étudient l’économie politique, s’informent de l’industrie, s’intéressent en amateurs ou en promoteurs à toutes les améliorations publiques. « Jamais, dit encore Lacretelle, les Français n’avaient été plus ligués pour combattre tous les maux dont la nature nous impose le tribut, et ceux qui pénètrent par mille voies dans les institutions sociales. » Peut-on admettre que tant de bonnes intentions réunies aboutissent à tout détruire ? Tous se rassurent, le gouvernement comme la haute classe, en songeant au bien qu’ils ont fait ou voulu faire. Le roi se rappelle qu’il a rendu l’état civil aux protestants, aboli la question préparatoire, supprimé la corvée en nature, établi la libre circulation des grains, institué les assemblées provinciales, relevé la marine, secouru les Américains, affranchi ses propres serfs, diminué les dépenses de sa maison, employé Malesherbes, Turgot et Necker, lâché la bride à la presse, écouté l’opinion publique\footnote{« La plupart des étrangers ont peine à se faire une idée de l’autorité qu’exerce en France aujourd’hui l’opinion publique, ils comprennent difficilement ce que c’est que cette puissance invisible {\itshape qui commande jusque dans le palais du roi.} Il en est pourtant ainsi. » (Necker (1784), cité par Tocqueville.)}. Aucun gouvernement ne s’est montré plus doux : le 14 juillet 1789, il n’y avait à la Bastille que sept prisonniers, dont un idiot, un détenu sur la demande de sa famille, et quatre accusés de faux\footnote{Granier de Cassagnac, II, 236  Au commencement du règne de Louis XVI, M. de Malesherbes visita, selon l’usage, les maisons qui contenaient des prisonniers d’État. « Il m’a dit à moi-même qu’il n’en avait fait sortir que {\itshape deux.} » (Sénac de Meilhan, \href{http://gallica.bnf.fr/ark:/12148/bpt6k83675w}{\dotuline{{\itshape Du gouvernement, des mœurs et des conditions en France}}} [\url{http://gallica.bnf.fr/ark:/12148/bpt6k83675w}].)}. Aucun prince n’a été plus humain plus charitable, plus préoccupé des malheureux. En 1784, année d’inondations et d’épidémies, il fait distribuer pour trois millions de secours. On s’adresse à lui, même pour les accidents privés ; le 8 juin 1785, il envoie deux cents livres à la femme d’un laboureur breton, qui, ayant déjà deux enfants, vient d’en mettre au monde trois en une seule couche\footnote{{\itshape Archives nationales}, H, 1418, 1149, F, 14, 2073 (Secours à diverses provinces et localités malheureuses).}. Pendant un hiver rigoureux, il laisse chaque jour les pauvres envahir ses cuisines. Très probablement, il est, après Turgot, l’homme de son temps qui a le plus aimé le peuple  Au-dessous de lui, ses délégués se conforment à ses vues ; j’ai lu quantité de lettres d’intendants qui tâchent d’être de petits Turgots. « Tel construit un hôpital, un autre fonde des prix pour les laboureurs ; celui-ci admet des artisans à sa table\footnote{Aubertin, 484 (d’après Bachaumont).} » ; celui-là entreprend le défrichement d’un marais. M. de la Tour, en Provence, a fait tant de bien pendant quarante ans, que, malgré lui, le Tiers-état lui vote une médaille d’or\footnote{Léonce de Lavergne, 472.}. Un gouverneur fait un cours de boulangerie économique  Quel danger de pareils pasteurs peuvent-ils courir au milieu de leur troupeau ? Quand le roi convoque les États Généraux, nul n’est « en défiance », ni ne s’effraye de l’avenir. « On parlait\footnote{\href{http://gallica.bnf.fr/ark:/12148/bpt6k467543/f442.table}{\dotuline{Mathieu Dumas, {\itshape Mémoires}, I, 426}} [\url{http://gallica.bnf.fr/ark:/12148/bpt6k467543/f442.table}]  Sir Samuel Romilly, {\itshape Mémoires}, I, 99  « La sécurité alla jusqu’à l’extravagance. » (Mme de Genlis, {\itshape Mémoires.}) Le 29 juin 1789, Necker disait dans le conseil du roi, à Marly : « Quoi de plus frivole que les craintes conçues à raison de l’organisation des États Généraux ? Rien ne peut y être statué sans l’assentiment du roi. » (M. de Barentin, {\itshape Mémoires}, 187.) — Adresse de l’Assemblée nationale à ses commettants, 2 octobre 1789 : « Une grande révolution, {\itshape dont le projet eût paru chimérique il y a quelques mois}, s’est opérée au milieu de nous. »} de l’établissement d’une nouvelle constitution de l’État comme d’une œuvre facile, comme d’un événement naturel. » — « Les hommes les meilleurs et les plus vertueux y voyaient le commencement d’une nouvelle ère de bonheur pour la France et pour tout le monde civilisé. Les ambitieux se réjouissaient de la large carrière qui allait s’ouvrir à leurs espérances. Mais on n’aurait pas trouvé un individu, le plus morose, le plus timide, le plus enthousiaste, qui prévît un seul des événements extraordinaires vers lesquels les États assemblés allaient être conduits. »
\chapterclose


\chapteropen

\chapter[{Chapitre III}]{Chapitre III}


\chaptercont

\section[{I. La classe moyenne. — Ancien esprit du Tiers. — Les affaires publiques ne regardaient que le roi. — Limites de l’opposition janséniste et parlementaire.}]{I. La classe moyenne. — Ancien esprit du Tiers. — Les affaires publiques ne regardaient que le roi. — Limites de l’opposition janséniste et parlementaire.}

\noindent Pendant longtemps, la philosophie nouvelle, enfermée dans un cercle choisi, n’avait été qu’un luxe de bonne compagnie. Négociants, fabricants et boutiquiers, avocats, procureurs et médecins, comédiens, professeurs ou curés, fonctionnaires, employés et commis, toute la classe moyenne était à sa besogne. L’horizon de chacun était restreint ; c’était celui de la profession ou du métier qu’on exerçait, de la corporation dans laquelle on était compris, de la ville où l’on était né et tout au plus de la province où l’on habitait\footnote{J’ai pu moi-même constater ces sentiments par les récits de vieillards morts il y a vingt ans  Cf. Les {\itshape Mémoires} manuscrits du libraire Hardy (analysés par Aubertin) et les {\itshape Voyages d’Arthur Young.}}. La disette des idées et la modestie du cœur confinaient le bourgeois dans son enclos héréditaire. Ses yeux ne se hasardaient guère au-delà, dans le territoire interdit et dangereux des choses d’État ; à peine s’il y coulait un regard furtif et rare ; les affaires publiques étaient « les affaires du roi ». — Point de fronde alors, sauf dans le barreau, satellite obligé du Parlement et entraîné dans son orbite. En 1718, après un lit de justice, les avocats de Paris s’étant mis en grève, le régent s’écriait avec colère et surprise : « Quoi ! ces drôles-là s’en mêlent aussi\footnote{ Aubertin, {\itshape ib.}, 180, 362.
 } ! » Encore faut-il remarquer que, le plus souvent, beaucoup d’entre eux se tenaient cois. « Mon père et moi, écrit plus tard l’avocat Barbier, nous ne nous sommes pas mêlés dans ces tapages, parmi ces esprits caustiques et turbulents. » — Et il ajoute cette profession de foi significative : « Je crois qu’il faut faire son emploi avec honneur, {\itshape sans se mêler d’affaires d’État sur lesquelles on n’a ni pouvoir ni mission.} » — Dans toute la première moitié du dix-huitième siècle, je ne vois dans le Tiers-état que ce seul foyer d’opposition, le Parlement et, autour de lui, pour attiser le feu, le vieil esprit gallican ou janséniste. « La bonne ville de Paris, écrit Barbier en 1733, est janséniste de la tête aux pieds », non seulement les magistrats, les avocats, les professeurs, toute l’élite de la bourgeoisie, « mais encore tout le gros de Paris, hommes, femmes, petits enfants, qui tiennent pour cette doctrine, sans savoir la matière, sans rien entendre aux distinctions et interprétations, par haine contre Rome et les jésuites. Les femmes, femmelettes et jusqu’aux femmes de chambre s’y feraient hacher… Ce parti s’est grossi des honnêtes gens du royaume qui détestent les persécutions et l’injustice. » — Aussi, quand toutes les chambres de magistrature, jointes aux avocats, donnent leur démission et défilent hors du palais « au milieu d’un monde infini, le public dit : {\itshape Voilà de vrais Romains, les pères de la patrie ;} on bat des mains au passage des deux conseillers Pucelle et Menguy et on leur jette des couronnes ». — Incessamment rallumée, la querelle du Parlement et de la Cour sera l’une des flammèches qui provoqueront la grande explosion finale, et les brandons jansénistes qui couvent sous la cendre trouveront leur emploi en 1791 lorsqu’on attaquera l’édifice ecclésiastique  Mais, dans cet antique foyer, il ne peut y avoir que des cendres chaudes, des tisons enfouis, parfois des pétillements et des feux de paille ; par lui-même et à lui seul, il n’est point incendiaire. Sa structure emprisonne sa flamme et ses aliments limitent sa chaleur. Le janséniste est trop fidèle chrétien pour ne pas respecter les puissances instituées d’en haut. Le parlementaire, conservateur par état, aurait horreur de renverser l’ordre établi. Tous les deux combattent pour la tradition et contre la nouveauté ; c’est pourquoi, après avoir défendu le passé contre le pouvoir arbitraire, ils le défendront contre la violence révolutionnaire et tomberont, l’un dans l’impuissance et l’autre dans l’oubli.

\section[{II. Changement dans la condition du bourgeois. — Il s’enrichit. — Il prête à l’État. — Danger de sa créance. — Il s’intéresse aux affaires publiques.}]{II. Changement dans la condition du bourgeois. — Il s’enrichit. — Il prête à l’État. — Danger de sa créance. — Il s’intéresse aux affaires publiques.}

\noindent Aussi bien, l’embrasement est tardif dans la classe moyenne, et, pour qu’il s’y propage, il faut qu’au préalable, par une transformation graduelle, les matériaux réfractaires soient devenus combustibles  Un grand changement s’opère au dix-huitième siècle dans la condition du Tiers-état. Le bourgeois a travaillé, fabriqué, commercé, gagné, épargné, et tous les jours il s’enrichit davantage\footnote{Voltaire, \href{http://www.voltaire-integral.com/Html/14/09SIEC34.html\#i31}{\dotuline{{\itshape Siècle de Louis XV}, ch. XXXI }} [\url{http://www.voltaire-integral.com/Html/14/09SIEC34.html\#i31}]; \href{http://www.voltaire-integral.com/Html/14/09SIEC34.html\#i30}{\dotuline{{\itshape Siècle de Louis XIV}, ch. XXX}} [\url{http://www.voltaire-integral.com/Html/14/09SIEC34.html\#i30}]. « L’industrie augmente tous les jours ; à voir le luxe des particuliers, ce nombre prodigieux de maisons agréables bâties dans Paris et dans les provinces, cette quantité d’équipages, ces commodités, ces recherches qu’on appelle luxe, on croirait que l’opulence est vingt fois plus grande qu’autrefois. Tout cela est le fruit d’un travail ingénieux encore plus que de la richesse… Le moyen ordre s’est enrichi par l’industrie… Les gains du commerce ont augmenté. Il s’est trouvé moins d’opulence qu’autrefois chez les grands et plus dans le moyen ordre, et cela a mis moins de distance entre les hommes. Il n’y avait autrefois d’autre ressource pour les petits que de servir les grands ; aujourd’hui l’industrie a ouvert mille chemins qu’on ne connaissait pas il y a cent ans. »}. On peut dater de Law ce grand essor des entreprises, du négoce, de la spéculation et des fortunes ; arrêté par la guerre, il reprend plus vif et plus fort à chaque intervalle de paix, après le traité d’Aix-la-Chapelle en 1748, après le traité de Paris en 1763, et surtout à partir du règne de Louis XVI. L’exportation française, qui en 1720 était de 106 millions, en 1735 de 124, en 1748 de 192, est de 257 millions en 1755, de 309 en 1776, de 354 en 1788. En 1786, Saint-Domingue seul envoie à la métropole pour 131 millions de ses produits et en reçoit pour 44 millions de marchandises\footnote{Arthur Young, II, 360, 373.}. Sur ces échanges, on voit, à Nantes, à Bordeaux, se fonder des maisons colossales. « Je tiens Bordeaux, écrit Arthur Young, pour plus riche et plus commerçante qu’aucune ville d’Angleterre, excepté Londres… Dans ces derniers temps, les progrès du commerce maritime ont été plus rapides en France qu’en Angleterre même. » Selon un administrateur du temps, si les taxes de consommation rapportent tous les jours davantage, c’est que depuis 1774 les divers genres d’industrie se développent tous les jours davantage\footnote{Tocqueville, 255.}. Et ce progrès est régulier, soutenu. « On peut compter, dit Necker en 1781, que le produit de tous les droits de consommation augmente de deux millions par an. » — Dans ce grand effort d’invention, de labeur et de génie, Paris, qui grossit sans cesse, est l’atelier central. Bien plus encore qu’aujourd’hui, il a le monopole de tout ce qui est œuvre d’intelligence et de goût, livres, tableaux, estampes, statues, bijoux, parures, toilettes, voitures, ameublements, articles de curiosité et de mode, agréments et décors de la vie élégante et mondaine ; c’est lui qui fournit l’Europe. En 1774, son commerce de librairie était évalué à 45 millions, et celui de Londres au quart seulement\footnote{Aubertin, 482.}. Sur les bénéfices s’élèvent beaucoup de grandes fortunes, encore plus de fortunes moyennes, et les capitaux ainsi formés cherchent un emploi. — Justement, voici que les plus nobles mains du royaume s’étendent pour les recevoir, nobles, princes du sang, états provinciaux, assemblées du clergé, au premier rang le roi, qui, étant le plus besogneux de tous, emprunte à dix pour cent et est toujours en quête de nouveaux prêteurs. Déjà sous Fleury la dette s’est accrue de 18 millions de rente, et, pendant la guerre de Sept Ans, de 34 autres millions de rente. Sous Louis XVI, M. Necker emprunte en capital 530 millions, M. Joly de Fleury 300 millions, M. de Calonne 800 millions, en tout 1 630 millions en dix ans. L’intérêt de la dette, qui n’était que de 45 millions en 1755, s’élève à 106 millions en 1776, et monte à 206 millions en 1789\footnote{Buchez et Roux, {\itshape Histoire parlementaire.} Extrait des états dressés par les contrôleurs généraux, I, 175, \href{http://gallica.bnf.fr/ark:/12148/bpt6k28867s/f209}{\dotuline{205}} [\url{http://gallica.bnf.fr/ark:/12148/bpt6k28867s/f209}]. — {\itshape Rapport} de Necker, I, 376. — Aux 206 millions, il faut ajouter 15 800 000 pour les frais et intérêts des anticipations.}. Que de créanciers indiqués par ce peu de chiffres ! Et remarquez que, le Tiers-état étant le seul corps qui gagne et épargne, presque tous ces créanciers sont du Tiers-état. Ajoutez-en des milliers d’autres : en premier lieu, les financiers qui font au gouvernement des avances de fonds, avances indispensables, puisque, de temps immémorial, il mange son blé en herbe, et que toujours l’année courante ronge d’avance le produit des années suivantes : il y a 80 millions d’anticipations en 1759, et 170 en 1783. En second lieu, tant de fournisseurs, grands et petits, qui, sur tous les points du territoire, sont en compte avec l’État pour leurs travaux et fournitures, véritable armée qui s’accroît tous les jours, depuis que le gouvernement, entraîné par la centralisation, se charge seul de toutes les entreprises, et que, sollicité par l’opinion, il multiplie les entreprises utiles au public : sous Louis XV, l’État fait six mille lieues de routes, et, sous Louis XVI, en 1788, afin de parer à la famine, il achète pour quarante millions de grains.\par
Par cet accroissement de son action et par cet emprunt de capitaux, il devient le débiteur universel ; dès lors les affaires publiques ne sont plus seulement les affaires du roi. Ses créanciers s’inquiètent de ses dépenses, car c’est leur argent qu’il gaspille ; s’il gère mal, ils seront ruinés. Ils voudraient bien connaître son budget, vérifier ses livres : un prêteur a toujours le droit de surveiller son gage. Voilà donc le bourgeois qui relève la tête et qui commence à considérer de près la grande machine dont le jeu, dérobé à tous les regards vulgaires, était jusqu’ici un secret d’État. Il devient politique et, du même coup, il devient mécontent  Car, on ne peut le nier, ces affaires où il est si fort intéressé sont mal conduites. Un fils de famille qui mènerait les siennes de la même façon mériterait d’être interdit. Toujours, dans l’administration de l’État, la dépense a dépassé la recette. D’après les aveux officiels, le déficit annuel était de soixante-dix millions en 1770, de quatre-vingts en 1783\footnote{\href{http://gallica.bnf.fr/ark:/12148/bpt6k28867s/f194}{\dotuline{Buchez et Roux, I, 190}} [\url{http://gallica.bnf.fr/ark:/12148/bpt6k28867s/f194}]. {\itshape Rapport} de M. de Calonne.} : quand on a tenté de le réduire, ç’a été par des banqueroutes, l’une de deux milliards à la fin de Louis XIV, l’autre presque égale au temps de Law, une autre du tiers et de moitié sur toutes les rentes au temps de Terray, sans compter les suppressions de détail, les réductions, les retards indéfinis de payement, et tous les procédés violents ou frauduleux qu’un débiteur puissant emploie impunément contre un créancier faible. « On compte cinquante-six violations de la foi publique depuis Henri IV jusqu’au ministère de M. de Loménie inclusivement\footnote{Chamfort, 105.} » et l’on aperçoit à l’horizon une dernière banqueroute plus effroyable que toutes les autres. Plusieurs, Besenval, Linguet, la conseillent hautement comme une amputation nécessaire et salutaire. Non seulement il y a des précédents, et en cela le gouvernement ne fera que suivre son propre exemple ; mais telle est sa règle quotidienne, puisqu’il ne vit qu’au jour le jour, à force d’expédients et de délais, creusant un trou pour en boucher un autre, et ne se sauvant de la faillite que par la patience forcée qu’il impose à ses créanciers. Avec lui, dit un contemporain, ils n’étaient jamais sûrs de rien, et il fallait toujours attendre\footnote{Tocqueville, 261.}. « Plaçaient-ils leurs capitaux dans ses emprunts, ils ne pouvaient jamais compter sur une époque fixe pour le payement des intérêts. Construisaient-ils ses vaisseaux, réparaient-ils ses routes, vêtaient-ils ses soldats, ils restaient sans garanties de leurs avances, sans échéances pour le remboursement, réduits à calculer les chances d’un contrat avec les ministres comme celles d’un prêt fait à la grosse aventure. » On ne paye que si l’on peut et quand on peut, même les gens de la maison, les fournisseurs de la table, les serviteurs de la personne. En 1753, les domestiques de Louis XV n’avaient rien reçu depuis trois années. On a vu que ses palefreniers allaient mendier pendant la nuit dans les rues de Versailles, que ses pourvoyeurs « se cachaient », que, sous Louis XVI, en 1778, il était dû 792 620 francs au marchand de vin, et 3 467 980 francs au fournisseur de poisson et de viande\footnote{Marquis d’Argenson, 12 avril 1752, 11 février 1753, 24 juillet 1753, 7 décembre 1753. — {\itshape Archives nationales}, O, 738.}. En 1788, la détresse est telle, que le ministre de Loménie prend et dépense les fonds d’une souscription faite par des particuliers pour les hospices ; au moment où il se retire, le Trésor est vide, sauf quatre cent mille francs dont il met la moitié dans sa poche. Quelle administration   Devant ce débiteur qui manifestement devient insolvable, tous les gens qui, de près ou de loin, sont engagés dans ses affaires, se consultent avec alarme, et ils sont innombrables, banquiers, négociants, fabricants, employés, prêteurs de toute espèce et de tout degré : au premier rang les rentiers, qui ont mis chez lui tout leur avoir en viager et qui seront à l’aumône s’il ne leur paye pas chaque année les 44 millions qu’il leur doit, les industriels et marchands, qui lui ont confié leur honneur commercial et auraient horreur de faillir par contre-coup ; derrière ceux-ci, leurs créanciers, leurs commis, leurs ouvriers, leurs proches, bref la plus grande partie de la classe laborieuse et paisible, qui jusqu’ici obéissait sans murmure et ne songeait point à contrôler le régime établi. Désormais elle va le contrôler avec attention, avec défiance, avec colère ; et malheur à ceux qu’elle prendra en faute, car elle sait qu’ils la ruinent en ruinant l’État !

\section[{III. Il monte dans l’échelle sociale. — Le noble se rapproche de lui. — Il se rapproche du noble. — Il se cultive. — Il est du monde. — Il se sent l’égal du noble. — Il est gêné par les privilèges.}]{III. Il monte dans l’échelle sociale. — Le noble se rapproche de lui. — Il se rapproche du noble. — Il se cultive. — Il est du monde. — Il se sent l’égal du noble. — Il est gêné par les privilèges.}

\noindent En même temps elle a monté dans l’échelle sociale, et, par son élite, elle rejoint les plus haut placés. Jadis, entre Dorante et M. Jourdain, entre don Juan et M. Dimanche, entre M. de Sotenville lui-même et George Dandin, l’intervalle était immense : habits, logis, mœurs, caractère, point d’honneur, idées, langage, tout différait. Maintenant la distance est presque insensible. D’une part, les nobles se sont rapprochés du Tiers-état ; d’autre part, le Tiers-état s’est rapproché des nobles, et l’égalité de fait a précédé l’égalité de droit  Aux approches de 1789, on aurait peine à les distinguer dans la rue. À la ville, les gentilshommes ne portent plus l’épée ; ils ont quitté les broderies, les galons, et se promènent en frac uni, ou courent dans un cabriolet qu’ils conduisent eux-mêmes\footnote{Ségur, I, 17.}. « La simplicité des coutumes anglaises » et les usages du Tiers leur ont paru plus commodes pour la vie privée. Leur éclat les gênait, ils étaient las d’être toujours en représentation. Désormais ils acceptent la familiarité pour avoir le sans-gêne, et sont contents « de se mêler sans faste et sans entraves à tous leurs concitoyens »  Certes, l’indice est grave, et les vieilles âmes féodales avaient raison de gronder. Le marquis de Mirabeau, apprenant que son fils veut être son propre avocat, ne se console qu’en voyant d’autres, et de plus grands, faire pis encore\footnote{Lucas de Montigny, {\itshape Lettre} du marquis de Mirabeau du 23 mars 1783.}. « Quoique ayant de la peine à avaler l’idée que le petit-fils de notre grand-père, tel que nous l’avons vu passer sur le Cours, toute la foule, petits et grands, ôtant de loin le chapeau, va maintenant figurer à la barre de l’avant-cour, disputant la pratique aux aboyeurs de chicane, je me suis dit ensuite que Louis XIV serait un peu plus étonné s’il voyait la femme de son arrière-successeur, en habit de paysanne et en tablier, sans suite, sans pages ni personne, courant le palais et les terrasses, demander au premier polisson en frac de lui donner la main que celui-ci lui prête seulement jusqu’au bas de l’escalier. » — En effet, le nivellement des façons et des dehors ne fait que manifester le nivellement des esprits et des âmes. Si l’ancien décor se défait, c’est que les sentiments qu’il annonçait se défont. Il annonçait le sérieux, la dignité, l’habitude de se contraindre et d’être en public, l’autorité, le commandement. C’était la parade fastueuse et rigide d’un état-major social. À présent la parade tombe, parce que l’état-major s’est dissous. Si les nobles s’habillent en bourgeois, c’est qu’ils sont eux-mêmes devenus des bourgeois, je veux dire des oisifs qui, retirés des affaires, causent et s’amusent. — Sans doute ils s’amusent en gens de goût et causent en gens de bonne compagnie. Mais la difficulté ne sera pas grande de les égaler en cela. Depuis que le Tiers s’est enrichi, beaucoup de roturiers sont devenus gens du monde. Les successeurs de Samuel Bernard ne sont plus des Turcaret, mais des Pâris-Duverney, des Saint-James, des Laborde, affinés, cultivés de cœur et d’esprit, ayant du tact, de la littérature, de la philosophie, de la bienfaisance\footnote{Mme Vigée-Lebrun, I, 269, 231 (Intérieur de deux fermiers généraux, M. de Verdun à Colombes, M. de Saint-James à Neuilly). — Le type supérieur du bourgeois, du négociant, a déjà été mis au théâtre par Sedaine (\href{http://gallica.bnf.fr/ark:/12148/bpt6k89671z}{\dotuline{{\itshape le Philosophe sans le savoir}}} [\url{http://gallica.bnf.fr/ark:/12148/bpt6k89671z}]).}, donnant des fêtes, sachant recevoir. À une nuance près, on trouve chez eux la même société que chez un grand seigneur, les mêmes idées, le même ton. Leurs fils, MM. de Villemur, de Francueil, d’Épinay, jettent l’argent par les fenêtres aussi élégamment que les jeunes ducs avec lesquels ils soupent. Avec de l’argent et de l’esprit, un parvenu se dégourdit vite, et son fils, sinon lui, sera initié : quelques années d’exercices à l’académie, un maître de danse, une des quatre mille charges qui confèrent la noblesse lui donneront les dehors qui lui manquent. Or, en ce temps-là, dès qu’on sait observer les bienséances, saluer et causer, on a son brevet d’entrée partout. Un Anglais\footnote{{\itshape A Comparative view}, by John Andrews, 58.} remarque que l’un des premiers mots que l’on emploie pour louer un homme est de dire « qu’il se présente parfaitement bien ». La maréchale de Luxembourg, si fière, choisit toujours Laharpe pour cavalier ; en effet, « il donne si bien le bras ! » — Non seulement le plébéien entre au salon s’il a de l’usage, mais il y trône s’il a du talent. La première place dans la conversation et même dans la considération publique est pour Voltaire, fils d’un notaire, pour Diderot, fils d’un coutelier, pour Rousseau, fils d’un horloger, pour d’Alembert, enfant-trouvé recueilli par un vitrier ; et quand, après la mort des grands hommes, il n’y a plus que des écrivains de second ordre, les premières duchesses sont encore contentes d’avoir à leur table Chamfort, autre enfant-trouvé, Beaumarchais, autre fils d’horloger, Laharpe, nourri et élevé par charité, Marmontel, fils d’un tailleur de village, quantité d’autres moins notables, bref tous les parvenus de l’esprit.\par
Pour s’achever, la noblesse leur emprunte leur plume et aspire à leurs succès. « On est revenu, disait le prince de Hénin, de ces préjugés gothiques et absurdes sur la culture des lettres\footnote{Comte de Tilly, {\itshape Mémoires}, I, 31.}. Quant à moi, j’écrirais demain une comédie si j’en avais le talent, et, si l’on me mettait un peu en colère, je la jouerais. » Et, de fait, « le vicomte de Ségur, fils du ministre de la guerre, joue le rôle d’amant dans {\itshape Nina} sur le théâtre de Mlle Guimard, avec tous les acteurs de la comédie italienne\footnote{Geffroy, {\itshape Gustave III.} Lettre de Mme de Staël (août 1786).} ». Un personnage de Mme de Genlis, revenant à Paris après cinq ans d’absence, dit « qu’il a laissé les hommes uniquement occupés de jeu, de chasse, de leurs petites maisons, et qu’il les retrouve tous auteurs\footnote{ \noindent Mme de Genlis, {\itshape Adèle et Théodore} (1782), I, 312. — Déjà en 1762, Bachaumont cite un grand nombre de pièces écrites par des grands seigneurs : {\itshape Clytemnestre}, par le comte de Lauraguais ; {\itshape Alexandre}, par le chevalier de Fénelon : {\itshape Don Carlos}, par le marquis de Ximénès.
 } ». Ils colportent de salon en salon leurs tragédies, comédies, romans, églogues, dissertations et considérations de toute espèce. Ils tâchent de faire représenter leurs pièces, ils subissent le jugement préalable des comédiens, ils sollicitent un mot d’éloge au {\itshape Mercure}, ils lisent des fables aux séances de l’Académie. Ils s’engagent dans les tracasseries, dans les glorioles, dans les petitesses de la vie littéraire, bien pis, de la vie théâtrale, puisque, sur cent théâtres de société, ils sont acteurs et jouent avec les vrais acteurs. Ajoutez à cela, si vous voulez, leurs autres petits talents d’amateurs : peindre à la gouache, faire des chansons, jouer de la flûte. — Après ce mélange des classes et ce déplacement des rôles, quelle supériorité reste à la noblesse ? Par quel mérite spécial, par quelle capacité reconnue se fera-elle respecter du Tiers ? Hors une fleur de suprême bon ton et quelques raffinements dans le savoir-vivre, en quoi diffère-t-elle de lui ? Quelle éducation supérieure, quelle habitude des affaires, quelle expérience du gouvernement, quelle instruction politique, quel ascendant local, quelle autorité morale peut-elle alléguer pour autoriser ses prétentions à la première place   En fait de pratiques, c’est déjà le Tiers qui fait la besogne et fournit les hommes spéciaux, intendants, premiers commis des ministères, administrateurs laïques et ecclésiastiques, travailleurs effectifs de toute espèce de tout degré. Rappelez-vous ce marquis dont on parlait tout à l’heure, ancien capitaine aux gardes françaises, homme de cœur et loyal, avouant aux élections de 1789 que les connaissances essentielles à un député « se rencontreront plus généralement dans le Tiers-état, dont l’esprit est exercé aux affaires ». — Quant à la théorie, le roturier en sait autant que les nobles, et il croit en savoir davantage ; car, ayant lu les mêmes livres et pénétré des mêmes principes, il ne s’arrête pas comme eux à mi-chemin sur la pente des conséquences, mais plonge en avant, tête baissée, jusqu’au fond de la doctrine, persuadé que sa logique est de la clairvoyance et qu’il a d’autant plus de lumières qu’il a moins de préjugés. — Considérez les jeunes gens qui ont vingt ans aux environs de 1780, nés dans une maison laborieuse, accoutumés à l’effort, capables de travailler douze heures par jour, un Barnave, un Carnot, un Roederer, un Merlin de Thionville, un Robespierre, race énergique qui sent sa force, qui juge ses rivaux, qui sait leur faiblesse, qui compare son application et son instruction à leur légèreté et à leur insuffisance, et qui, au moment où gronde en elle l’ambition de la jeunesse, se voit d’avance exclue de toutes les hautes places, reléguée à perpétuité dans les emplois subalternes, primée en toute carrière par des supérieurs en qui elle reconnaît à peine des égaux. Aux examens d’artillerie, où Chérin, généalogiste, refuse les roturiers, et où l’abbé Bossut, mathématicien, refuse les ignorants, on découvre que la capacité manque aux élèves nobles, et la noblesse aux élèves capables\footnote{Chamfort, 119.} ; gentilhomme et instruit, ces deux qualités semblent s’exclure ; sur cent élèves, quatre ou cinq réunissent les deux conditions. Or, à présent que la société est mêlée, de pareilles épreuves sont fréquentes et faciles. Avocat, médecin, littérateur, l’homme du Tiers, avec lequel un duc s’entretient familièrement, qui voyage en diligence côte à côte avec un comte colonel de hussards\footnote{Comte de Vaublanc, I, 117. — Beugnot, \href{http://gallica.bnf.fr/ark:/12148/bpt6k37337f/f74.table}{\dotuline{{\itshape Mémoires} (premier et deuxième morceau}} [\url{http://gallica.bnf.fr/ark:/12148/bpt6k37337f/f74.table}], la société chez M. de Brienne et chez le duc de Penthièvre).}, peut apprécier son interlocuteur ou son voisin, compter ses idées, vérifier son mérite, l’estimer à sa valeur ; et je suis sûr qu’il ne le surfera pas  Depuis que la noblesse, ayant perdu la capacité spéciale, et que le Tiers, ayant acquis la capacité générale, se trouvent de niveau par l’éducation et par les aptitudes, l’inégalité qui les sépare est devenue blessante en devenant inutile. Instituée par la coutume, elle n’est plus consacrée par la conscience, et le Tiers s’irrite à bon droit contre des privilèges que rien ne justifie, ni la capacité du noble, ni l’incapacité du bourgeois.

\section[{IV. Entrée de la philosophie dans les esprits ainsi préparés. — À ce moment celle de Rousseau est en vogue. — Concordance de cette philosophie et des besoins nouveaux. — Elle est adoptée par le Tiers.}]{IV. Entrée de la philosophie dans les esprits ainsi préparés. — À ce moment celle de Rousseau est en vogue. — Concordance de cette philosophie et des besoins nouveaux. — Elle est adoptée par le Tiers.}

\noindent Défiance et colère à l’endroit du gouvernement qui compromet toutes les fortunes, rancune et hostilité contre la noblesse qui barre tous les chemins, voilà donc les sentiments qui grandissent dans la classe moyenne par le seul progrès de sa richesse et de sa culture  Sur cette matière ainsi disposée, on devine quel sera l’effet de la philosophie nouvelle. Enfermée d’abord dans le réservoir aristocratique, la doctrine a filtré par tous les interstices comme une eau glissante, et se répand insensiblement dans tout l’étage inférieur  Déjà en 1727, Barbier, qui est un bourgeois de l’ancienne roche et ne connaît guère que de nom la philosophie et les philosophes, écrit dans son journal : « On retranche à cent pauvres familles des rentes viagères qui les faisaient subsister, acquises avec des effets dont le roi était débiteur et dont le fonds est éteint ; on donne cinquante-six mille livres de pension à des gens qui ont été dans les grands postes où ils ont amassé des biens considérables, toujours aux dépens du peuple, et cela pour se reposer et ne rien faire\footnote{Barbier, II, 16 : III, 255 (mai 1751). « Le roi est pillé par tous les seigneurs qui l’environnent, surtout dans tous ses voyages à ses différents châteaux, lesquels sont fréquents. » — Et septembre 1750. — Cf. Aubertin, 291, 415 ({\itshape Mémoires} manuscrits de Hardy).} »  Une à une, les idées de réforme pénètrent dans son cabinet d’avocat consultant ; il a suffi de la conversation pour les propager, et le gros sens commun n’a pas besoin de philosophie pour les admettre. « La taxe des impositions sur les biens, dit-il en 1750, doit être proportionnelle et répartie également sur tous les sujets du roi et membres de l’État, à proportion des biens que chacun possède réellement dans le royaume ; en Angleterre, les terres de la noblesse, du clergé et du Tiers-état payent également sans distinction ; rien n’est plus juste. » — Dans les dix années qui suivent, le flot grossit ; on parle en mal du gouvernement dans les cafés, aux promenades, et la police n’ose arrêter les frondeurs, « parce qu’il faudrait arrêter tout le monde ». Jusqu’à la fin du règne, la désaffection va croissant. « En 1744, dit le libraire Hardy, pendant la maladie du roi à Metz, des particuliers font dire et payent à la sacristie de Notre-Dame six mille messes pour sa guérison ; en 1757, après l’attentat de Damiens, le nombre des messes demandées n’est plus que de six cents ; en 1774, pendant la maladie dont il meurt, ce nombre tombe à trois. » — Discrédit complet du gouvernement, succès immense de Rousseau, de ces deux événements simultanés on peut dater la conversion du Tiers à la philosophie\footnote{Traités de Paris et d’Hubersbourg, 1763. — Procès de la Chalotais, 1765. — Banqueroute de Terray, 1770. — Destruction du parlement, 1771. — Premier partage de la Pologne, 1772. — Rousseau, \href{http://classiques.uqac.ca/classiques/Rousseau\_jj/discours\_origine\_inegalite/origine\_inegalite.html}{\dotuline{{\itshape Discours sur l’inégalité}}} [\url{http://classiques.uqac.ca/classiques/Rousseau\_jj/discours\_origine\_inegalite/origine\_inegalite.html}], 1753. — {\itshape La Nouvelle Héloïse}, 1759. — {\itshape Émile et Contrat social}, 1762.}  Au commencement du règne de Louis XVI, un voyageur qui rentrait après quelques années d’absence, et à qui l’on demandait quel changement il remarquait dans la nation, répondit : « {\itshape Rien autre chose, sinon que ce qui se disait dans les salons se répète dans les rues}\footnote{Baron de Barante, \href{http://gallica.bnf.fr/ark:/12148/bpt6k108468q/f315}{\dotuline{{\itshape Tableau de la littérature française au dix-huitième siècle}, 312}} [\url{http://gallica.bnf.fr/ark:/12148/bpt6k108468q/f315}].} »  Et ce qu’on répète dans les rues, c’est la doctrine de Rousseau, le {\itshape Discours sur l’inégalité}, le {\itshape Contrat social} amplifié, vulgarisé et répété par les disciples sur tous les tons et sous toutes les formes. Quoi de plus séduisant pour le Tiers   Non seulement cette théorie a la vogue, et c’est elle qu’il rencontre au moment décisif où ses regards, pour la première fois, se lèvent vers les idées générales ; mais de plus, contre l’inégalité sociale et contre l’arbitraire politique, elle lui fournit des armes, et des armes plus tranchantes qu’il n’en a besoin. Pour des gens qui veulent contrôler le pouvoir et abolir les privilèges, quel maître plus sympathique que l’écrivain de génie, le logicien puissant, l’orateur passionné qui établit le droit naturel, qui nie le droit historique, qui proclame l’égalité des hommes, qui revendique la souveraineté du peuple, qui dénonce à chaque page l’usurpation, les vices, l’inutilité, la malfaisance des grands et des rois   Et j’omets les traits par lesquels il agrée aux fils d’une bourgeoisie laborieuse et sévère, aux hommes nouveaux qui travaillent et s’élèvent, son sérieux continu, son ton âpre et amer, son éloge des mœurs simples, des vertus domestiques, du mérite personnel, de l’énergie virile ; c’est un plébéien qui parle à des plébéiens  Rien d’étonnant s’ils le prennent pour guide, et s’ils acceptent ses doctrines avec cette ferveur de croyance qui est l’enthousiasme et qui toujours accompagne la première idée comme le premier amour.\par
Un juge compétent, témoin oculaire, Mallet du Pan\footnote{{\itshape Mercure britannique}, t. II, 360.}, écrit en 1799 : « Dans les classes mitoyennes et inférieures, Rousseau a eu cent fois plus de lecteurs que Voltaire. C’est lui seul qui a inoculé chez les Français la doctrine de la souveraineté du peuple et de ses conséquences les plus extrêmes. J’aurais peine à citer un seul révolutionnaire qui ne fût transporté de ces théorèmes anarchiques et qui ne brûlât du désir de les réaliser. Ce {\itshape Contrat social}, qui dissout les sociétés, fut le Coran des discoureurs apprêtés de 1789, des jacobins de 1790, des républicains de 1791 et des forcenés les plus atroces… J’ai entendu Marat en 1788 lire et commenter le {\itshape Contrat social} dans les promenades publiques aux applaudissements d’un auditoire enthousiaste. » — La même année, dans la foule immense qui remplit la Grand’Salle du Palais, Lacretelle entend le même livre cité, ses dogmes allégués\footnote{Lacretelle, {\itshape Dix ans d’épreuves}, 21.} « par des clercs de la Basoche, par de jeunes avocats, par tout le petit peuple lettré qui fourmille de publicistes de nouvelle date ». On voit par cent détails qu’il est dans toutes les mains comme un catéchisme. En 1784\footnote{{\itshape Souvenirs manuscrits}, par le chancelier Pasquier.}, des fils de magistrats allant prendre leur première leçon de droit chez un agrégé, M. Sareste, c’est le {\itshape Contrat social} que leur maître leur donne en guise de manuel. Ceux qui trouvent trop ardue la nouvelle géométrie politique en apprennent au moins les axiomes, et, si les axiomes rebutent, on en trouve les conséquences palpables, les équivalents commodes, la menue monnaie courante dans la littérature en vogue, théâtre, histoire et romans\footnote{{\itshape Le Compère Mathieu}, par Dulaurens (1766). « Nous ne devons nos malheurs qu’à la façon dont nous avons été élevés, c’est-à-dire à l’état de société dans lequel nous sommes nés. Or puisque cet état est la source de tous les maux, sa dissolution ne peut être que celle de tous les biens. »}. Par les {\itshape Éloges} de Thomas, par les pastorales de Bernardin de Saint-Pierre, par la compilation de Raynal, par les comédies de Beaumarchais, même par le {\itshape Jeune Anacharsis} et par la vogue nouvelle de l’antiquité grecque et romaine, les dogmes d’égalité et de liberté filtrent et pénètrent dans toute la classe qui sait lire\footnote{Le \href{http://gallica.bnf.fr/ark:/12148/bpt6k89044d}{\dotuline{{\itshape Tableau de Paris}, par Mercier}} [\url{http://gallica.bnf.fr/ark:/12148/bpt6k89044d}] (12 vol.), est la peinture la plus exacte et la plus abondante des idées et des aspirations de la classe moyenne de 1781 à 1788.}. « Ces jours derniers, dit Métra\footnote{{\itshape Correspondance}, par Metra, XVII, 87 (20 août 1784).}, il y avait un dîner de quarante ecclésiastiques de campagne chez le curé d’Orangis, à cinq lieues de Paris. Au dessert et dans la vérité du vin, ils sont tous convenus qu’ils étaient venus à Paris voir le {\itshape Mariage de Figaro…} Il semble que jusqu’ici les auteurs comiques ont toujours eu l’intention de faire rire les grands aux dépens des petits ; ici au contraire ce sont les petits qui rient aux dépens des grands. » De là le succès de la pièce. — Tel régisseur d’un château a trouvé un Raynal dans la bibliothèque, et les déclamations furibondes qu’il y rencontre le ravissent à ce point que, trente ans après, il les récitera encore sans broncher. Tel sergent aux gardes françaises brode la nuit des gilets pour gagner de quoi acheter les nouveaux livres. — Après la peinture galante de boudoir, voici la peinture austère et patriotique : le {\itshape Bélisaire} et les {\itshape Horaces} de David indiquent l’esprit nouveau du public et des ateliers\footnote{Le {\itshape Bélisaire} est de 1780, le {\itshape Serment des Horaces} est de 1783.}. C’est l’esprit de Rousseau, « l’esprit républicain\footnote{Geffroy, {\itshape Gustave III et la cour de France}, « Paris, avec son {\itshape esprit républicain}, applaudit ordinairement ce qui est tombé à Fontainebleau. » (Lettre de Mme de Staël, du 17 septembre 1786.)} » ; il a gagné toute la classe moyenne, artistes, employés, curés, médecins, procureurs, avocats, lettrés, journalistes, et il a pour aliments les pires passions aussi bien que les meilleures, l’ambition, l’envie, le besoin de liberté, le zèle du bien public et la conscience du droit.

\section[{V. Effet qu’elle produit sur lui. — Formation des passions révolutionnaires. — Instincts de nivellement. — Besoin de domination. — Le Tiers décide qu’il est la nation. — Chimères, ignorance, exaltation.}]{V. Effet qu’elle produit sur lui. — Formation des passions révolutionnaires. — Instincts de nivellement. — Besoin de domination. — Le Tiers décide qu’il est la nation. — Chimères, ignorance, exaltation.}

\noindent Toutes ces passions s’exaltent les unes par les autres. Rien n’est tel qu’un passe-droit pour aviver le sentiment de la justice. Rien n’est tel que le sentiment de la justice pour aviver la douleur d’un passe-droit. À présent que le Tiers se juge privé de la place qui lui appartient, il se trouve mal à la place qu’il occupe, et il souffre de mille petits chocs que jadis il n’aurait pas sentis. Quand on se sent citoyen, on s’irrite d’être traité en sujet, et nul n’accepte d’être l’inférieur de celui dont il se croit l’égal. C’est pourquoi, pendant les vingt dernières années, l’ancien régime a beau s’alléger, il semble plus pesant, et ses piqûres exaspèrent comme des blessures. On en citerait vingt cas pour un. — Au théâtre de Grenoble, Barnave enfant\footnote{Sainte-Beuve, \href{http://gallica.bnf.fr/ark:/12148/bpt6k37437q}{\dotuline{{\itshape Causeries du lundi}, II, 24}} [\url{http://gallica.bnf.fr/ark:/12148/bpt6k37437q}]. (Étude sur Barnave.)} était avec sa mère dans une loge que le duc de Tonnerre, gouverneur de la province, destinait à l’un de ses complaisants. Le directeur du théâtre, puis l’officier de garde viennent prier Mme Barnave de se retirer ; elle refuse ; par ordre du gouverneur, quatre fusiliers arrivent pour l’y contraindre. Déjà le parterre prenait parti et l’on pouvait craindre des violences, lorsque M. Barnave, averti de l’affront, vint emmener sa femme et dit tout haut : « Je sors par ordre du gouverneur ». Le public, toute la bourgeoisie indignée s’engagea à ne revenir au spectacle qu’après satisfaction, et en effet le théâtre resta vide pendant plusieurs mois, jusqu’à ce que Mme Barnave eût consenti à y reparaître. Le futur député se souvint plus tard de l’outrage, et dès lors se jura « de relever la caste à laquelle il appartenait de l’humiliation à laquelle elle semblait condamnée ». — Pareillement Lacroix, le futur conventionnel\footnote{Tilly, {\itshape Mémoires}, I, 243.}, poussé, à la sortie du théâtre, par un gentilhomme qui donne le bras à une jolie femme, se plaint tout haut. — « Qui êtes-vous ? » — Lui, encore provincial, a la bonhomie de défiler tout au long ses nom, prénoms et qualités. — « Eh bien, dit l’autre, c’est très bien fait à vous d’être tout cela ; moi, je suis le comte de Chabannes et je suis très pressé. » Sur quoi, « riant démesurément », il remonte en voiture. « Ah ! monsieur, disait Lacroix encore tout chaud de sa mésaventure, l’affreuse distance que l’orgueil et les préjugés mettent entre les hommes ! » — Soyez sûr que chez Marat, chirurgien aux écuries du comte d’Artois, chez Robespierre, protégé de l’évêque d’Arras, chez Danton, petit avocat « chargé de dettes », chez tous les autres, en vingt rencontres, l’amour-propre avait saigné de même. L’amertume concentrée qui pénètre les {\itshape Mémoires} de Mme Roland n’a pas d’autre cause. « Elle ne\footnote{Paroles de Fontanes, qui l’avait connue et l’admirait (Sainte-Beuve, \href{http://gallica.bnf.fr/ark:/12148/CadresFenetre?O=NUMM-201410\&M=tdm}{\dotuline{{\itshape Nouveaux lundis}, VIII}} [\url{http://gallica.bnf.fr/ark:/12148/CadresFenetre?O=NUMM-201410\&M=tdm}], 221).} pardonnait pas à la société la place inférieure qu’elle y avait longtemps occupée\footnote{\href{http://gallica.bnf.fr/ark:/12148/bpt6k46827g}{\dotuline{{\itshape Mémoires de Mme Roland}}} [\url{http://gallica.bnf.fr/ark:/12148/bpt6k46827g}], passim. À quatorze ans, présentée à Mme de Boismorel, elle est blessée d’entendre appeler sa grand’maman « mademoiselle »  « Un peu après, dit-elle, je ne pouvais me dissimuler que je valais mieux que Mlle d’Hannaches dont les soixante ans et la généalogie ne lui donnaient pas la faculté de faire une lettre qui eût le sens commun ou qui fût lisible. » — Vers la même époque, elle passe huit jours à Versailles chez une femme de la Dauphine, et dit à sa mère : « Encore quelques jours et je détesterai si fort ces gens-là, que je ne saurai plus que faire de ma haine  Quel mal te font-ils donc   Sentir l’injustice et contempler à tout moment l’absurdité. » — Au château de Fontenay, invitée à dîner, on la fait manger, elle et sa mère, à l’office, etc  En 1818, dans une petite ville du nord, le comte de…, dînant chez un sous-préfet bourgeois et placé à table à côté de la maîtresse de la maison, lui dit en acceptant du potage « Merci, mon cœur ». Mais la Révolution a donné bec et ongles à la petite bourgeoise et, un instant après, elle lui dit avec son plus beau sourire : « Voulez-vous du poulet, mon cœur ? »}. » Grâce à Rousseau, la vanité, si naturelle à l’homme, si sensible chez un Français, est devenue plus sensible. La moindre nuance, un ton de voix, semble une marque de dédain. « Un jour\footnote{Vaublanc, I, 153.} que l’on parlait devant le ministre de la guerre d’un officier général parvenu à ce grade par son mérite : Ah oui, dit le ministre, officier général de fortune ! Ce mot fut répété, commenté, et fit bien du mal. » Les grands ont beau condescendre, « accueillir avec une égale et douce bonté tous ceux qui leur sont présentés » ; chez le duc de Penthièvre les nobles mangent avec le maître de la maison, les roturiers dînent chez son premier gentilhomme et ne viennent au salon que pour le café. Là ils « trouvent en force et le ton haut » les autres qui ont eu l’honneur de manger avec Son Altesse et « qui ne manquent pas de saluer les arrivants avec une complaisance pleine de protection\footnote{Beugnot, {\itshape Mémoires}, I, 77.} ». Cela suffit ; le duc a beau « pousser les attentions jusqu’à la recherche », Beugnot, si pliant, n’a nulle envie de revenir  On leur garde rancune, non seulement des saluts trop courts qu’ils font, mais encore des révérences trop grandes qu’on leur fait. Chamfort conte avec aigreur que d’Alembert, au plus haut de sa réputation, étant chez Mme du Deffand avec le président Hénault et M. de Pont-de-Veyle, arrive un médecin nommé Fournier, qui en entrant dit à Mme du Deffand : « Madame, j’ai l’honneur de vous présenter mon très humble respect » ; au président Hénault : « Monsieur, j’ai bien l’honneur de vous saluer » ; à M. de Pont-de-Veyle : « Monsieur, je suis votre très humble serviteur », et à d’Alembert : « Bonjour, Monsieur\footnote{Chamfort, 16. — « Qui le croirait ? Ce ne sont ni les impôts, ni les lettres de cachet, ni tous les autres abus de l’autorité, ce ne sont point les vexations des intendants et les longueurs ruineuses de la justice qui ont le plus irrité la nation : c’est le préjugé de la noblesse pour lequel elle a manifesté plus de haine. Ce qui le prouve évidemment, c’est que ce sont les bourgeois, les gens de lettres, les gens de finances, enfin tous ceux qui jalousaient la noblesse, qui ont soulevé contre elle le petit peuple dans les villes et les paysans dans les campagnes. » (Rivarol, {\itshape Mémoires.})} ». Quand le cœur est révolté, tout est pour lui sujet de ressentiment. Le Tiers, à l’exemple de Rousseau, sait aux nobles mauvais gré de tout ce qu’ils font, bien mieux, de tout ce qu’ils sont, de leur luxe, de leur élégance, de leur badinage, de leurs façons fines et brillantes. Chamfort est aigri par les politesses dont ils l’ont accablé. Siéyès leur en veut de l’abbaye qu’on lui a promise et qu’on ne lui a pas donnée. Chacun, outre le grief général, a son grief personnel. Leur froideur comme leur familiarité, leurs attentions comme leurs inattentions, sont des offenses, et sous ces millions de coups d’épingle, réels ou imaginaires, la poche au fiel s’emplit.\par
En 1789, elle est pleine et va crever. « Le titre le plus respectable de la noblesse française, écrit Chamfort, c’est de descendre immédiatement de quelque trente mille hommes casqués, cuirassés, brassardés, cuissardés, qui, sur de grands chevaux bardés de fer, foulaient aux pieds huit ou dix millions d’hommes nus, ancêtres de la nation actuelle. Voilà un droit bien avéré au respect et à l’amour de leurs descendants ! Et, pour achever de rendre cette noblesse respectable, elle se recrute et se régénère par l’adoption de ces hommes qui ont accru leur fortune en dépouillant la cabane du pauvre hors d’état de payer ses impositions\footnote{Chamfort, 335.}. » — « Pourquoi le Tiers, dit Siéyès, ne renverrait-il pas dans les forêts de la Franconie toutes ces familles qui conservent la folle prétention d’être issues de la race des conquérants et de succéder à des droits de conquête\footnote{ Siéyès, \href{http://gallica.bnf.fr/ark:/12148/bpt6k89685n}{\dotuline{{\itshape Qu’est-ce que le Tiers} }} [\url{http://gallica.bnf.fr/ark:/12148/bpt6k89685n}]? 17, 41, 139, 166.
 } ? Je suppose qu’à défaut de police Cartouche se fût rétabli plus solidement sur un grand chemin ; aurait-il acquis un véritable droit de péage ? S’il avait eu le temps de vendre cette sorte de monopole, jadis assez commun, à un successeur de bonne foi, son droit serait-il devenu beaucoup plus respectable entre les mains de l’acquéreur ?… Tout privilège est, de sa nature, injuste, odieux et contraire au pacte social. Le sang bouillonne à la seule idée qu’il fut possible de consacrer légalement à la fin du dix-huitième siècle les abominables fruits de l’abominable féodalité… La caste des nobles est véritablement un peuple à part, mais un faux peuple qui, ne pouvant, faute d’organes utiles, exister par lui-même, s’attache à une nation réelle, comme ces tumeurs végétales qui ne peuvent vivre que de la sève des plantes qu’elles fatiguent et dessèchent. » — Ils sucent tout, il n’y a rien que pour eux. « Toutes les branches du pouvoir exécutif sont tombées dans la caste qui fournit (déjà) l’église, la robe et l’épée. Une sorte de confraternité ou de compérage fait que les nobles se préfèrent entre eux et pour tout au reste de la nation… C’est la cour qui a régné et non le monarque. C’est la cour qui crée et distribue les places. Et qu’est-ce que la cour, sinon la tête de cette immense aristocratie qui couvre toutes les parties de la France, qui, par ses membres, atteint à tout, et exerce partout ce qu’il y a d’essentiel dans toutes les parties de la puissance publique ? » — Mettons fin « à ce crime social, à ce long parricide qu’une classe s’honore de commettre journellement contre les autres… Ne demandez plus quelle place enfin les privilégiés doivent occuper dans l’ordre social ; c’est demander quelle place on veut assigner dans le corps d’un malade à l’humeur maligne qui le mine et le tourmente, … à la maladie affreuse qui dévore sa chair vive ». — La conséquence sort d’elle-même : extirpons l’ulcère, ou tout au moins balayons la vermine. Le Tiers, à lui seul et par lui-même, est « une nation complète », à qui ne manque aucun organe, qui n’a besoin d’aucune aide pour subsister ou se conduire, et qui recouvrera la santé lorsqu’il aura secoué les parasites incrustés dans sa peau.\par
« Qu’est-ce que le Tiers ? Tout. Qu’a-t-il été jusqu’à présent dans l’ordre politique\footnote{« La noblesse, disent les nobles, est un intermédiaire entre le roi et le peuple  Oui, comme le chien de chasse est un intermédiaire entre le chasseur et les lièvres. » (Chamfort.)} ? Rien. Que demande-t-il ? À y devenir quelque chose. » — Non pas quelque chose, mais tout. Son ambition politique est aussi grande que son ambition sociale, et il aspire à l’autorité aussi bien qu’à l’égalité. Si les privilèges sont mauvais, celui du prince est le pire, car il est le plus énorme, et la dignité humaine, blessée par les prérogatives du noble, périt sous l’arbitraire du roi. Peu importe qu’il en use à peine, et que son gouvernement, docile à l’opinion publique, soit celui d’un père indécis et indulgent. Affranchi du despotisme réel, le Tiers s’indigne contre le despotisme possible, et il croirait être esclave s’il consentait à rester sujet. L’orgueil souffrant s’est redressé, s’est raidi, et, pour mieux assurer son droit, va revendiquer tous les droits. Il est si doux, si enivrant, pour l’homme qui, de toute antiquité, a subi des maîtres, de se mettre à leur place, de les mettre à sa place, de se dire qu’ils sont ses mandataires, de se croire membre du souverain, roi de France pour sa quote-part, seul auteur légitime de tout droit et de tout pouvoir   Conformément aux doctrines de Rousseau, les cahiers du Tiers déclarent à l’unanimité qu’il faut donner une constitution à la France ; elle n’en a pas, ou, du moins, celle qu’elle a n’est pas valable. Jusqu’ici « les conditions du pacte social étaient ignorées\footnote{Prudhomme, III, 2 (Tiers-état du Nivernais et {\itshape passim}). Cf. par contre les cahiers de la noblesse du Bugey et de la noblesse d’Alençon.} » ; à présent qu’on les a découvertes, il faut les écrire. Il n’est pas vrai de dire, comme les nobles d’après Montesquieu, que la constitution existe, que ses grands traits ne doivent point être altérés, qu’il s’agit seulement de réformer les abus, que les États Généraux n’ont qu’un pouvoir limité, qu’ils sont incompétents pour substituer à la monarchie un autre régime. Tacitement ou expressément, le Tiers refuse de restreindre son mandat, et n’admet pas qu’on lui oppose des barrières. Par suite, à l’unanimité, il exige que les députés votent, « non par ordre, mais par tête et conjointement »  « Dans le cas où les députés du clergé et de la noblesse refuseraient d’opiner en commun et par tête, les députés du Tiers, qui représentent 24 millions d’hommes, pouvant et devant toujours se dire l’Assemblée nationale malgré la scission des représentants de 400 000 individus, offriront au roi, de concert avec ceux du clergé et de la noblesse qui voudront se joindre à eux, leur secours à l’effet de subvenir aux besoins de l’État, et les impôts ainsi consentis seront répartis entre tous les sujets du roi indistinctement\footnote{{\itshape lb.} Cahiers du Tiers-état de Dijon, de Dax, de Bayonne et de Saint-Séver, de Rennes, etc.}. » — « Le Tiers, disent d’autres cahiers, étant les 99 pour 100 de la nation, n’est pas un ordre. Désormais, avec ou sans les privilégiés, il sera, sous la même dénomination, appelé le peuple ou la nation. » — N’objectez pas qu’un peuple ainsi mutilé devient une foule, que des chefs ne s’improvisent pas, qu’on se passe difficilement de ses conducteurs naturels, qu’à tout prendre ce clergé et cette noblesse sont encore une élite, que les deux cinquièmes du sol sont dans leurs mains, que la moitié des hommes intelligents et instruits sont dans leurs rangs, que leur bonne volonté est grande, et que ces vieux corps historiques ont toujours fourni aux constitutions libres leurs meilleurs soutiens. Selon le principe de Rousseau, il ne faut pas évaluer les hommes, mais les compter ; en politique, le nombre seul est respectable ; ni la naissance, ni la propriété, ni la fonction, ni la capacité, ne sont des titres : grand ou petit, ignorant ou savant, général, soldat ou goujat, dans l’armée sociale chaque individu n’est qu’une unité munie d’un vote ; où vous voyez la majorité, là est le droit. C’est pourquoi le Tiers pose son droit comme incontestable, et, à son tour, dit comme Louis XIV : « L’État, c’est moi ».\par
Une fois le principe admis ou imposé, tout ira bien. « Il semblait, dit un témoin\footnote{Marmontel, {\itshape Mémoires}, II, 247.}, que c’était par des hommes de l’âge d’or qu’on allait être gouverné. Ce peuple libre, juste et sage, toujours d’accord avec lui-même, toujours éclairé dans le choix de ses ministres, modéré dans l’usage de sa force et de sa puissance, ne serait jamais égaré, jamais trompé, jamais dominé, asservi par les autorités qu’il leur aurait confiées. Ses volontés feraient ses lois, et ses lois feraient son bonheur. » La nation va être {\itshape régénérée} : cette phrase est dans tous les écrits et dans toutes les bouches. À Nangis\footnote{Arthur Young, I, 222.}, Arthur Young trouve qu’elle est le fond de la conversation politique. Le chapelain d’un régiment, curé dans le voisinage, ne veut pas en démordre ; quant à savoir ce qu’il entend par là, c’est une autre affaire. Impossible de rien démêler dans ses explications, « sinon une perfection théorique de gouvernement, douteuse à son point de départ, risquée dans ses développements et chimérique quant à ses fins ». Lorsque l’Anglais leur propose en exemple la Constitution anglaise, « ils en font bon marché », ils sourient du peu ; cette Constitution ne donne pas assez à la liberté ; surtout elle n’est pas conforme {\itshape aux principes}  Et notez que nous sommes ici chez un grand seigneur, dans un cercle d’hommes éclairés. À Riom, aux assemblées d’élection\footnote{Malouet, \href{http://gallica.bnf.fr/ark:/12148/CadresFenetre?O=NUMM-46799\&M=tdm}{\dotuline{{\itshape Mémoires}, I}} [\url{http://gallica.bnf.fr/ark:/12148/CadresFenetre?O=NUMM-46799\&M=tdm}], 279.}, Malouet voit « de petits bourgeois, des praticiens, des avocats sans aucune instruction sur les affaires publiques, citant le {\itshape Contrat Social}, déclamant avec véhémence contre la tyrannie, et proposant chacun une Constitution ». La plupart ne savent rien et ne sont que des marchands de chicane ; les plus instruits n’ont en politique que des idées d’écoliers. Dans les collèges de l’Université, on n’enseigne point l’histoire\footnote{Lavalette, I, 7. — {\itshape Souvenirs manuscrits} par le chancelier Pasquier. — Cf. Brissot, {\itshape Mémoires}, I.}. « Le nom de Henri IV, dit Lavalette, ne nous avait pas été prononcé une seule fois pendant mes huit années d’études, et, à dix-sept ans, j’ignorais encore à quelle époque et comment la maison de Bourbon s’est établie sur le trône. » Pour tout bagage, ils emportent, comme Camille Desmoulins, des bribes de latin, et ils entrent dans le monde, la tête farcie « de maximes républicaines », échauffés par les souvenirs de Rome et de Sparte, « pénétrés d’un profond mépris pour les gouvernements monarchiques ». Ensuite, à l’Ecole de Droit, ils ont appris un droit abstrait, ou n’ont rien appris. Aux cours de Paris, point d’auditeurs ; le professeur fait sa leçon devant des copistes qui vendent leurs cahiers. Un élève qui assisterait et rédigerait lui-même serait mal vu ; on l’accuserait d’ôter aux copistes leur gagne-pain. Par suite le diplôme est nul ; à Bourges on l’obtient en six mois ; si le jeune homme finit par savoir la loi, c’est plus tard par l’usage et la pratique  Des lois et institutions étrangères, nulle connaissance, à peine une notion vague ou fausse. Malouet lui-même se figure mal le Parlement anglais, et plusieurs, sur l’étiquette, l’imaginent d’après le Parlement de France  Quant au mécanisme des constitutions libres ou aux conditions de la liberté effective, cela est trop compliqué. Depuis vingt ans, sauf dans les grandes familles de magistrature, Montesquieu est suranné. À quoi bon les études sur l’ancienne France ? « Qu’est-il résulté de tant et de si profondes recherches ? Des conjectures laborieuses et des raisons de douter\footnote{ Prudhomme, \href{http://gallica.bnf.fr/ark:/12148/bpt6k43689r}{\dotuline{{\itshape Résumé des cahiers}}} [\url{http://gallica.bnf.fr/ark:/12148/bpt6k43689r}]. ({\itshape Préface} par Jean Rousseau.)
 }. » Il est bien plus commode de partir des droits de l’homme et d’en déduire les conséquences. À cela la logique de l’Ecole suffit, et la rhétorique du collège fournira les tirades  Dans ce grand vide des intelligences, les mots indéfinis de liberté, d’égalité, de souveraineté du peuple, les phrases ardentes de Rousseau et de ses successeurs, tous les nouveaux axiomes flambent comme des charbons allumés, et dégagent une fumée chaude, une vapeur enivrante. La parole gigantesque et vague s’interpose entre l’esprit et les objets ; tous les contours sont brouillés et le vertige commence. Jamais les hommes n’ont perdu à ce point le sens des choses réelles. Jamais ils n’ont été à la fois plus aveugles et plus chimériques. Jamais leur vue troublée ne les a plus rassurés sur le danger véritable, et plus alarmés sur le danger imaginaire. Les étrangers qui sont de sang-froid et qui assistent à ce spectacle, Mallet du Pan, Dumont de Genève, Arthur Young, Jefferson, Gouverneur Morris, écrivent que les Français ont l’esprit dérangé. Dans ce délire universel, Morris ne peut citer à Washington qu’une seule tête saine, Marmontel, et Marmontel ne parle pas autrement que Morris. Aux clubs préparatoires et aux assemblées d’électeurs, il est le seul qui se lève contre les propositions déraisonnables. Autour de lui, ce ne sont que gens échauffés, exaltés à propos de rien, jusqu’au grotesque\footnote{ Marmontel, II, 245.
 }. Dans tout usage du régime établi, dans toute mesure de l’administration, « dans les règlements de police, dans les édits sur les finances, dans les autorités graduelles sur lesquelles reposaient l’ordre et la tranquillité publiques, il n’y avait rien où l’on ne trouvât un caractère de tyrannie… Il s’agissait du mur d’enceinte et des barrières de Paris qu’on dénonçait comme un enclos de bêtes fauves, trop injurieux pour des hommes ». — « J’ai vu, dit l’un des orateurs, j’ai vu à la barrière Saint-Victor, sur l’un des piliers en sculpture, le croiriez-vous ? j’ai vu l’énorme tête d’un lion, gueule béante, et vomissant des chaînes dont il menace les passants ; peut-on imaginer un emblème plus effrayant de despotisme et de servitude ? » — L’orateur lui-même imitait « le rugissement du lion ; tout l’auditoire était ému, et moi, qui passais si souvent à la barrière Saint-Victor, je m’étonnais que cette image horrible ne m’eût pas frappé. J’y fis ce jour-là même une attention particulière, et, sur le pilastre, je vis pour ornement un bouclier, suspendu à une chaîne mince que le sculpteur avait attachée à un petit mufle de lion, comme on voit à des marteaux de porte ou à des robinets de fontaine ». — Sensations perverties, conceptions délirantes, ce seraient là pour un médecin des symptômes d’aliénation mentale ; et nous ne sommes encore qu’aux premiers mois de 1789   Dans des têtes si excitables et tellement surexcitées, la magie souveraine des mots va créer des fantômes, les uns hideux, l’aristocrate et le tyran, les autres adorables, l’ami du peuple et le patriote incorruptible, figures démesurées et forgées par le rêve, mais qui prendront la place des figures réelles et que l’halluciné va combler de ses hommages ou poursuivre de ses fureurs.

\section[{VI. Résumé.}]{VI. Résumé.}

\noindent Ainsi descend et se propage la philosophie du dix-huitième siècle. — Au premier étage de la maison, dans les beaux appartements dorés, les idées n’ont été que des illuminations de soirée, des pétards de salon, des feux de Bengale amusants ; on a joué avec elles, on les a lancées en riant par les fenêtres  Recueillies à l’entresol et au rez-de-chaussée, portées dans les boutiques, dans les magasins et dans les cabinets d’affaires, elles y ont trouvé des matériaux combustibles, des tas de bois accumulés depuis longtemps, et voici que de grands feux s’allument. Il semble même qu’il y ait un commencement d’incendie ; car les cheminées ronflent rudement, et une clarté rouge jaillit à travers les vitres. — « Non, disent les gens d’en haut, ils n’auraient garde de mettre le feu à la maison, ils y habitent comme nous. Ce sont là des feux de paille, tout au plus des feux de cheminée : mais, avec un seau d’eau froide, on les éteint ; et d’ailleurs ces petits accidents nettoient les cheminées, font tomber la vieille suie. »\par
Prenez garde : dans les caves de la maison, sous les vastes et profondes voûtes qui la portent, il y a un magasin de poudre.
\chapterclose

\chapterclose


\chapteropen

\part[{Livre cinquième. Le peuple.}]{Livre cinquième. \\
Le peuple.}
\renewcommand{\leftmark}{Livre cinquième. \\
Le peuple.}


\chaptercont

\chapteropen

\chapter[{Chapitre I}]{Chapitre I}


\chaptercont

\section[{I. La misère. — Sous Louis XIV. — Sous Louis XV. — Sous Louis XVI.}]{I. La misère. — Sous Louis XIV. — Sous Louis XV. — Sous Louis XVI.}

\noindent La Bruyère écrivait juste un siècle avant 1789\footnote{La Bruyère, édition Destailleurs, II, 97. Addition de la 4\textsuperscript{e} édition (1689).} : « L’on voit certains animaux farouches, des mâles et des femelles, répandus par la campagne, noirs, livides et tout brûlés du soleil, attachés à la terre qu’ils fouillent et remuent avec une opiniâtreté invincible. Ils ont comme une voix articulée, et, quand ils se lèvent sur leurs pieds, ils montrent une face humaine ; et en effet ils sont des hommes. Ils se retirent la nuit dans des tanières où ils vivent de pain noir, d’eau et de racines. Ils épargnent aux autres hommes la peine de semer, de labourer et de recueillir pour vivre, et méritent ainsi de ne pas manquer de ce pain qu’ils ont semé. » — Ils en manquent pendant les vingt-cinq années suivantes, et meurent par troupeaux ; j’estime qu’en 1715 il en avait péri près d’un tiers\footnote{L’oppression et la misère commencent vers 1672  À la fin du dix-septième siècle (1698), les mémoires dressés par les intendants pour le duc de Bourgogne disent que beaucoup de districts et provinces ont perdu le sixième, le cinquième, le quart, le tiers et même la moitié de leur population. (Voir pour les détails la {\itshape Correspondance des contrôleurs généraux de} 1683 {\itshape à} 1698, publiée par M. de Boislisle.) — Or, d’après les mémoires des intendants (Vauban, \href{http://gallica.bnf.fr/ark:/12148/bpt6k54449}{\dotuline{{\itshape Dîme royale}}} [\url{http://gallica.bnf.fr/ark:/12148/bpt6k54449}], chap. VII, § 2), la population de France en 1698 était encore de 19 094 146 habitants  De 1698 à 1715 elle va toujours baissant. Selon Forbonnais, il n’y avait plus en France, sous le Régent, que 16 à 17 millions d’habitants  À partir de cette époque, la population ne baisse plus, mais pendant quarante ans elle croit à peine. En 1753 (Voltaire, \href{http://www.voltaire-integral.com/Html/20/population.htm}{\dotuline{{\itshape Dictionnaire} {\itshape philosophique}, article {\itshape Population}}} [\url{http://www.voltaire-integral.com/Html/20/population.htm}]), le dénombrement des feux donne 3 550 499 feux, outre 700 000 âmes à Paris, ce qui fait de 16 à 17 millions d’habitants si l’on compte par feu 4 personnes 1/2 et de 18 à 19 millions si l’on en compte 5.}, six millions, de misère et de faim. Ainsi, pour le premier quart du siècle qui précède la Révolution, la peinture, bien loin d’être trop forte, est trop faible, et l’on va voir que pendant un demi-siècle et davantage, jusqu’à la mort de Louis XV, elle demeure exacte ; peut-être même, au lieu de l’atténuer, faudrait-il la charger.\par
En 1725, dit Saint-Simon, « au milieu des profusions de Strasbourg et de Chantilly, on vit en Normandie d’herbes des champs. Le premier roi de l’Europe ne peut être un grand roi s’il ne l’est que de gueux de toutes conditions, et si son royaume tourne en un vaste hôpital de mourants à qui on prend tout en pleine paix\footnote{Floquet, \href{http://gallica.bnf.fr/ark:/12148/bpt6k945663/f407}{\dotuline{{\itshape Histoire du parlement de Normandie}, VI, 402}} [\url{http://gallica.bnf.fr/ark:/12148/bpt6k945663/f407}].}. » Au plus beau temps de Fleury et dans la plus belle région de France, le paysan cache « son vin à cause des aides et son pain à cause de la taille », persuadé « qu’il est un homme perdu si l’on peut se douter qu’il ne meurt pas de faim\footnote{Rousseau, \href{http://un2sg4.unige.ch/athena/rousseau/confessions/jjr\_conf\_04.html}{\dotuline{{\itshape Confessions}, 1 \textsuperscript{e} partie, chap. IV}} [\url{http://un2sg4.unige.ch/athena/rousseau/confessions/jjr\_conf\_04.html}] (1732).} ». En 1739, d’Argenson écrit dans son journal\footnote{Marquis d’Argenson, 19 et 24 mai, 4 juillet et 1\textsuperscript{er} août 1739.} : « La disette vient d’occasionner trois soulèvements dans les provinces, à Ruffec, à Caen et à Chinon. On a assassiné sur les chemins des femmes qui portaient du pain… M. le duc d’Orléans porta l’autre jour au conseil un morceau de pain, le mit devant la table du roi et dit : « Sire, voilà de quel pain se nourrissent aujourd’hui vos sujets… » — « Dans mon canton de Touraine, il y a déjà plus d’un an que les hommes mangent de l’herbe »  De toutes parts la misère se rapproche ; « on en parle à Versailles plus que jamais. Le roi interrogeant l’évêque de Chartres sur l’état de ses peuples, celui-ci a répondu que la famine et la mortalité y étaient telles, que les hommes mangeaient l’herbe comme des moutons et crevaient comme des mouches ». En 1740\footnote{{\itshape Résumé de l’histoire d’Auvergne par un Auvergnat} (M. Taillandier), 313.}, Massillon, évêque de Clermont-Ferrand, écrit à Fleury : « Le peuple de nos campagnes vit dans une misère affreuse, sans lits, sans meubles ; la plupart même, la moitié de l’année, manquent du pain d’orge et d’avoine qui fait leur unique nourriture et qu’ils sont obligés d’arracher de leur bouche et de celle de leurs enfants pour payer les impositions. J’ai la douleur, chaque année, de voir ce triste spectacle devant mes yeux, dans mes visites. C’est à ce point que les nègres de nos îles sont infiniment plus heureux ; car, en travaillant, ils sont nourris et habillés, avec leurs femmes et leurs enfants ; au lieu que nos paysans, les plus laborieux du royaume, ne peuvent, avec le travail le plus dur et le plus opiniâtre, avoir du pain pour eux et leur famille, et payer les subsides. » En 1740\footnote{Marquis d’Argenson, 1740, 28 mai, 7 et 21 août, 19 et 24 septembre, 7 novembre.}, à Lille, à propos de la sortie des grains, le peuple se révolte. « Un intendant m’écrit que la misère augmente d’heure en heure ; le moindre risque pour la récolte fait cet effet depuis trois ans… La Flandre est surtout bien embarrassée ; on n’a pas de quoi attendre la récolte, qui ne sera que dans deux mois d’ici. Les meilleures provinces ne sont pas en état d’en fournir aux autres. Dans chaque ville, on oblige chaque bourgeois à nourrir un ou deux pauvres et à lui donner quatorze livres de pain par semaine. Dans la seule petite ville de Châtellerault (qui est de quatre mille habitants), il y avait dix-huit cents pauvres cet hiver sur ce pied-là… La quantité des pauvres surpasse celle des gens qui peuvent vivre sans mendier… et les recouvrements se font avec une rigueur sans exemple ; on enlève les habits des pauvres, leurs derniers boisseaux de froment, les loquets des portes, etc. L’abbesse de Jouarre m’a dit hier que, dans son canton, en Brie, on n’avait pas pu ensemencer la plupart des terres. » — Rien d’étonnant si la famine gagne jusqu’à Paris. « On craint pour mercredi prochain… Il n’y a plus de pain à Paris, sinon des farines gâtées, qui arrivent et qui brûlent (au four). On travaille jour et nuit à Belleville, aux moulins, à remoudre les vieilles farines gâtées. Le peuple est tout prêt à la révolte ; le pain augmente d’un sol par jour ; aucun marchand n’ose ni ne veut apporter ici son blé. La Halle, mercredi, étant presque révoltée, le pain y manqua dès sept heures du matin… On avait retranché les vivres aux pauvres gens qui sont à Bicêtre, au point que, de trois quarterons de mauvais pain, on n’a plus voulu leur donner que demi-livre. Tout s’est révolté et a forcé les gardes ; quantité se sont échappés et vont inonder Paris. On y a appelé tout le guet et la maréchaussée des environs, qui ont été en bataille contre ces pauvres misérables, à grands coups de fusil, baïonnette et sabre. On compte qu’il y en a quarante ou cinquante sur le carreau ; la révolte n’était pas encore finie hier matin. »\par
Dix ans plus tard, le mal est pire\footnote{\href{http://gallica.bnf.fr/ark:/12148/bpt6k276806}{\dotuline{Marquis d’Argenson}} [\url{http://gallica.bnf.fr/ark:/12148/bpt6k276806}], 4 octobre 1749, 20 mai, 12 septembre, 28 octobre, 28 décembre 1750, 16 juin, 22 décembre 1751, etc.}. « De ma campagne, à dix lieues de Paris, je retrouve le spectacle de la misère et des plaintes continuelles bien redoublées ; qu’est-ce donc dans nos misérables provinces de l’intérieur du royaume ?… Mon curé m’a dit que huit familles, qui vivaient de leur travail avant mon départ, mendient aujourd’hui leur pain. On ne trouve point à travailler. Les gens riches se retranchent à proportion comme les pauvres. Avec cela on lève la taille avec une rigueur plus que militaire. Les collecteurs, avec les huissiers, suivis de serruriers, ouvrent les portes, enlèvent les meubles et vendent tout pour le quart de ce qu’il vaut, et les frais surpassent la taille… » — « Je me trouve en ce moment en Touraine, dans mes terres. Je n’y vois qu’une misère effroyable ; ce n’est plus le sentiment triste de la misère, c’est le désespoir qui possède les pauvres habitants : ils ne souhaitent que la mort et évitent de peupler… On compte que par an le quart des journées des journaliers va aux corvées, où il faut qu’ils se nourrissent : et de quoi ?… Je vois les pauvres gens y périr de misère. On leur paye quinze sous ce qui vaut un écu pour leur voiture. On ne voit que villages ruinés ou abattus, et nulles maisons qui se relèvent… Par ce que m’ont dit mes voisins, la diminution des habitants va à plus du tiers… Les journaliers prennent tous le parti d’aller se réfugier dans les petites villes. Il y a quantité de villages où tout le monde abandonne le lieu. J’ai plusieurs de mes paroisses où l’on doit trois années de taille ; mais, ce qui va toujours son train, ce sont les contraintes… Les receveurs des tailles et du fisc font chaque année des frais pour la moitié en sus des impositions… Un élu est venu dans le village où est ma maison de campagne, et a dit que cette paroisse devait être fort augmentée à la taille de cette année, qu’il y avait remarqué les paysans plus gras qu’ailleurs, qu’il avait vu sur le pas des portes des plumages de volaille, qu’on y faisait donc bonne chère, qu’on y était bien, etc  Voilà ce qui décourage le paysan, voilà ce qui cause le malheur du royaume. » — « Dans la campagne où je suis, j’entends dire que le mariage et la peuplade y périssent absolument de tous côtés. Dans ma paroisse, qui a peu de feux, il y a plus de trente garçons ou filles qui sont parvenus à l’âge plus que nubile ; il ne se fait aucuns mariages, et il n’en est pas seulement question entre eux. On les excite, et ils répondent tous la même chose, que ce n’est pas la peine de faire des malheureux comme eux. Moi-même j’ai essayé de marier quelques filles en les assistant et j’y ai trouvé le même raisonnement comme si tous s’étaient donné le mot\footnote{Marquis d’Argenson, 21 juin 1749, 22 mai 1750, 14 février, 19 mars, 15 avril 1751, etc.}. » — « Un de mes curés me mande qu’étant le plus vieux de la province de Touraine, il a vu bien des choses et d’excessives chertés de blé, mais qu’il ne se souvient pas d’une aussi grande misère (même en 1709) que celle de cette année-ci… Des seigneurs de Touraine m’ont dit que voulant occuper les habitants par des travaux à la campagne, à journées, les habitants se trouvent si faibles et en si petit nombre, qu’ils ne peuvent travailler de leurs bras. »\par
Ceux qui peuvent s’en aller s’en vont. « Une personne du Languedoc m’a dit que quantité de paysans désertent cette province et se réfugient en Piémont, Savoie, Espagne, effrayés, tourmentés de la poursuite du dixième en régie… Les maltôtiers vendent tout, emprisonnent tout, comme housards en guerre, et même avec plus d’avidité et de malice, pour gagner eux-mêmes. » — « J’ai vu un intendant d’une des meilleurs provinces du royaume, qui m’a dit qu’on n’y trouvait plus de fermiers, que les pères aimaient mieux envoyer leurs enfants vivre dans les villes, que le séjour de la campagne devenait chaque jour un séjour plus horrible pour les habitants… Un homme instruit dans les finances m’a dit qu’il était sorti cette année plus de deux cents familles de Normandie, craignant la collecte dans leurs villages. » — À Paris, on fourmille de mendiants ; on ne saurait s’arrêter à une porte que dix gueux ne viennent vous relancer de leurs clameurs. On dit que ce sont tous des habitants de la campagne qui, n’y pouvant plus tenir par les vexations qu’ils y essuient, viennent se réfugier dans la ville, … préférant la mendicité au labeur. » — Pourtant le peuple des villes n’est guère plus heureux que celui des campagnes. « Un officier dont la troupe est en garnison à Mézières m’a dit que le peuple est si misérable dans cette ville, que, dès qu’on avait servi le dîner des officiers dans les auberges, le peuple se jetait dessus et le pillait. » — « Il y a plus de douze mille ouvriers mendiants à Rouen, tout autant à Tours, etc. On compte plus de vingt mille de ces ouvriers qui sont sortis du royaume depuis trois mois pour aller aux étrangers, Espagne, Allemagne, etc. À Lyon, il y a plus de vingt mille ouvriers en soie qui sont consignés aux portes ; on les garde à vue, de peur qu’ils ne passent à l’étranger. » À Rouen\footnote{\href{http://gallica.bnf.fr/ark:/12148/bpt6k945663/f415}{\dotuline{Floquet, {\itshape ib.}, VI, 410}} [\url{http://gallica.bnf.fr/ark:/12148/bpt6k945663/f415}] (avril 1752, Adresse du Parlement de Normandie).} et en Normandie, « les plus aisés ont de la peine à avoir du pain pour leur subsistance, le commun du peuple en manque totalement, et il est réduit, pour ne pas mourir de faim, à se former des nourritures qui font horreur à l’humanité »  « À Paris même, écrit d’Argenson\footnote{\href{http://gallica.bnf.fr/ark:/12148/bpt6k276790}{\dotuline{Marquis d’Argenson}} [\url{http://gallica.bnf.fr/ark:/12148/bpt6k276790}], 26 novembre 1751, 15 mars 1753.}, j’apprends que le jour où M. le Dauphin et Mme la Dauphine allèrent à Notre-Dame de Paris, passant au pont de la Tournelle, il y avait plus de deux mille femmes assemblées dans ce quartier-là qui leur crièrent : Donnez-nous du pain, ou nous mourrons de faim. » — « Un des vicaires de la paroisse Sainte-Marguerite assure qu’il a péri plus de huit cents personnes de misère dans le faubourg Saint-Antoine depuis le 20 janvier jusqu’au 20 février, que les pauvres gens expiraient de froid et de faim dans leurs greniers, et que des prêtres, venus trop tard, arrivaient pour les voir mourir sans qu’il y eût du remède. » — Si je comptais les attroupements, les séditions d’affamés, les pillages de magasins, je n’en finirais pas : ce sont les soubresauts convulsifs de la créature surmenée ; elle a jeûné tant qu’elle a pu ; à la fin l’instinct se révolte. En 1747\footnote{Marquis d’Argenson, IV, 124 ; VI, 165 ; VII, 194, etc.}, « il y a des révoltes considérables à Toulouse pour le pain ; en Guyenne, « il y en a à chaque marché ». En 1750, six à sept mille hommes en Béarn s’assemblent derrière une rivière pour résister aux commis ; deux compagnies du régiment d’Artois font feu sur les révoltés et en tuent une douzaine. En 1752, une sédition dure trois jours à Rouen et dans les environs ; en Dauphiné et en Auvergne, les villageois attroupés forcent les greniers et prennent le blé au prix qu’ils veulent ; la même année, à Arles, deux mille paysans armés viennent demander du pain à l’hôtel de ville et sont dispersés par les soldats. Dans la seule province de Normandie, je trouve des séditions en 1725, en 1737, en 1739, en 1752, en 1764, 1765, 1766, 1767, 1768\footnote{\href{http://gallica.bnf.fr/ark:/12148/bpt6k945663/f405}{\dotuline{Floquet, {\itshape ib.}, VI, 400}} [\url{http://gallica.bnf.fr/ark:/12148/bpt6k945663/f405}] à 430.}, et toujours au sujet du pain. « Des hameaux entiers, écrit le Parlement, manquant des choses les plus nécessaires à la vie, étaient obligés, par le besoin, de se réduire aux aliments des bêtes… Encore deux jours et Rouen se trouvait sans provisions, sans grains et sans pain. » Aussi la dernière émeute est terrible, et, cette fois encore, la populace, maîtresse de la ville pendant trois jours, pille tous les greniers publics, tous les magasins des communautés  Jusqu’à la fin et au-delà, en 1770 à Reims, en 1775 à Dijon, Versailles, Saint-Germain, Pontoise et Paris, en 1782 à Poitiers, en 1785 à Aix en Provence, en 1788 et 1789 à Paris et dans toute la France, vous verrez des explosions semblables\footnote{{\itshape Correspondance} par Metra, I, 338, 341. — Hippeau, {\itshape le Gouvernement de Normandie}, IV, 62, 199, 358.}  Sans doute, sous Louis XVI, le gouvernement s’adoucit, les intendants sont humains, l’administration s’améliore, la taille devient moins inégale, la corvée s’allège en se transformant, bref la misère est moindre. Et pourtant elle est encore au-delà de ce que la nature humaine peut porter.\par
Parcourez les correspondances administratives des trente dernières années qui précèdent la Révolution : cent indices vous révéleront une souffrance excessive, même lorsqu’elle ne se tourne pas en fureur. Visiblement, pour l’homme du peuple, paysan, artisan, ouvrier, qui subsiste par le travail de ses bras, la vie est précaire ; il a juste le peu qu’il faut pour ne pas mourir de faim, et plus d’une fois ce peu lui manque\footnote{{\itshape Procès-verbaux de l’assemblée provinciale de Basse-Normandie} (1787), 151.}. Ici, dans quatre élections, « les habitants ne vivent presque que de sarrasin », et depuis cinq ans les pommes ayant manqué, ils n’ont que de l’eau pour boisson. Là, en pays de vignobles\footnote{{\itshape Archives nationales}, G, 319, État de la direction d’Issoudun, et H 1149, H 612, H 1418.}, chaque année « les vignerons sont en grande partie réduits à mendier leur pain dans la saison morte ». Ailleurs, les ouvriers, journaliers et manœuvres ayant été obligés de vendre leurs effets et leurs meubles, plusieurs sont morts de froid ; la nourriture insuffisante et malsaine a répandu des maladies, et dans deux élections on en comte trente-cinq mille à l’aumône\footnote{ \noindent {\itshape Archives nationales.} Lettres de M. de Crosne, intendant de Romeu (17 février 1784) ; de M. de Blossac, intendant de Poitiers (9 mai 1784) ; de M. de Villeneuve, intendant de Bourges (28 mars 1784) ; de M. de Cypierre, intendant d’Orléans (28 mai 1784) de M. de Mazirot, intendant de Moulins (28 juin 1786), de M. de Pont, intendant de Moulins (16 novembre 1779), etc.
 }. Dans un canton reculé, les paysans coupent les blés encore verts et les font sécher au four, parce que leur faim ne peut attendre. L’intendant de Poitiers écrit que, « dès que les ateliers de charité sont ouverts, il s’y précipite un nombre prodigieux de pauvres, quelque soin qu’on ait pris pour réduire les prix et n’admettre à ce travail que les plus nécessiteux ». L’intendant de Bourges marque qu’un grand nombre de métayers ont vendu leurs meubles, que « des familles entières ont passé deux jours sans manger », que, dans plusieurs paroisses, les affamés restent au lit la plus grande partie du jour pour souffrir moins. L’intendant d’Orléans annonce « qu’en Sologne de pauvres veuves ont brûlé leurs bois de lit, d’autres leurs arbres fruitiers », pour se préserver du froid et il ajoute : « Rien n’est exagéré dans ce tableau, le cri du besoin ne peut se rendre, il faut voir de près la misère des campagnes pour s’en faire une idée. » De Riom, de La Rochelle, de Limoges, de Lyon, de Montauban, de Caen, d’Alençon, des Flandres, de Moulins, les autres intendants mandent des nouvelles semblables. On dirait un glas funèbre qui s’interrompt pour reprendre ; même lorsque l’année n’est pas désastreuse, on l’entend de toutes parts. En Bourgogne, près de Châtillon-sur-Seine, « les impôts, les droits seigneuriaux et dîmes, les frais de culture partagent par tiers les productions de la terre et ne laissent rien aux malheureux cultivateurs, qui auraient abandonné leurs champs, si deux entrepreneurs suisses, fabricants de toiles peintes, n’étaient venus jeter par an quarante mille francs d’argent comptant dans le pays\footnote{{\itshape Archives nationales}, H, 200 (Mémoire de M. Amelot, intendant de Dijon  1786).} ». En Auvergne, les campagnes se dépeuplent journellement : plusieurs villages ont perdu, depuis le commencement du siècle, plus d’un tiers de leurs habitants\footnote{Gaultier de Biauzat, {\itshape Doléances sur les surcharges que portent les gens du Tiers-état}, etc. (1789), 188. — {\itshape Procès-verbaux de l’assemblée provinciale d’Auvergne} (1787), 175.}. « Si on ne se hâtait pas d’alléger le fardeau d’un peuple écrasé, dit en 1787 l’assemblée provinciale, l’Auvergne perdrait à jamais sa population et sa culture. » Dans le Comminges, au moment de la Révolution, des communautés menacent de faire abandon de leurs biens si on ne les dégrève pas\footnote{Théron de Montaugé, {\itshape l’Agriculture et les classes rurales dans le Toulousain.} 112.}. « Personne n’ignore, dit l’assemblée de la Haute-Guyenne en 1784, que le sort des communautés les plus imposées est si rigoureux, qu’on a vu plusieurs fois les propriétaires en abandonner le territoire\footnote{{\itshape Procès-verbaux de l’assemblée provinciale de la Haute-Guyenne}, I, 47, 79.}. Qui ne se rappelle que les habitants de Saint-Sernin ont fait jusqu’à dix fois l’abandon de leurs biens et menaçaient encore de revenir à cette résolution affligeante, lorsqu’ils ont eu recours à l’administration ? On a vu il y a quelques années un abandon de la communauté de Boisse combiné entre les habitants, le seigneur et le décimateur de cette communauté » ; et la désertion serait bien plus grande encore, si la loi ne défendait à tous les taillables d’abandonner un fonds surchargé, à moins de renoncer en même temps à tout ce qu’ils possèdent dans la même communauté  Dans le Soissonnais, au rapport de l’assemblée provinciale\footnote{{\itshape Procès-verbaux de l’assemblée provinciale du Soissonnais} (1787), 457 ; {\itshape de l’assemblée provinciale d’Auch}, 24.}, « la misère est excessive ». Dans la Gascogne, « le spectacle est déchirant ». Aux environs de Toul, le cultivateur, après avoir payé l’impôt, la dîme et les redevances, reste les mains vides. « L’agriculture est un état d’angoisses et de privations continuelles où des milliers d’hommes sont obligés de végéter péniblement\footnote{{\itshape Résumé des cahiers}, par Prudhomme, III, 271.}. Dans tel village de Normandie, presque tous les habitants, sans en excepter les fermiers et les propriétaires, mangent du pain d’orge et boivent de l’eau, vivent comme les plus malheureux des hommes, afin de subvenir au payement des impôts dont ils sont surchargés. » Dans la même province, à Forges, « bien des malheureux mangent du pain d’avoine, et d’autres du son mouillé, ce qui a causé la mort de plusieurs enfants\footnote{Hippeau, {\itshape ib.}, VI, 74, 243 ({\itshape Doléances} rédigées par le chevalier de Bertin).} ». Il est clair que le peuple vit au jour le jour ; le pain lui manque sitôt que la récolte est mauvaise. Vienne une gelée, une grêle, une inondation, toute une province ne sait plus comment faire pour subsister jusqu’à l’année suivante ; en beaucoup d’endroits il suffit de l’hiver, même ordinaire, pour amener la détresse. De toutes parts, on voit des bras tendus vers le roi, qui est l’aumônier universel. Le peuple ressemble à un homme qui marcherait dans un étang, ayant de l’eau jusqu’à la bouche ; à la moindre dépression du sol, au moindre flot, il perd pied, enfonce et suffoque. En vain la charité ancienne et l’humanité nouvelle s’ingénient pour lui venir en aide : l’eau est trop haute. Il faudrait que son niveau baissât, et que l’étang pût se dégorger par quelque large issue. Jusque-là le malheureux ne pourra respirer que par intervalles, et à chaque moment il courra risque de se noyer.

\section[{II. Condition du paysan pendant les trente dernières années de l’ancien régime. — Combien sa subsistance est précaire. — État de l’agriculture. — Terres incultes. — Mauvaise culture. — Salaires insuffisants. — Manque de bien-être.}]{II. Condition du paysan pendant les trente dernières années de l’ancien régime. — Combien sa subsistance est précaire. — État de l’agriculture. — Terres incultes. — Mauvaise culture. — Salaires insuffisants. — Manque de bien-être.}

\noindent C’est entre 1750 et 1760\footnote{Article {\itshape Fermiers et Grains} dans l’Encyclopédie, par Quesnay, 1756.} que les oisifs qui soupent commencent à regarder avec compassion et avec alarme les travailleurs qui ne dînent pas. Pourquoi ceux-ci sont-ils si pauvres, et par quel hasard, sur un sol aussi bon que la France, le pain manque-t-il à ceux qui font pousser le grain   D’abord, quantité de terres sont incultes et, ce qui est pis, abandonnées. Selon les meilleurs observateurs, « le quart du sol est absolument en friche… Les landes et les bruyères y sont le plus souvent rassemblées en grands déserts, par centaines et par milliers d’arpents\footnote{Théron de Montaugé, 15  {\itshape Éphémérides du citoyen}, III, 190 (1766) ; IX, 15 (article de M. de Butret, 1767).}. » — « Que l’on parcoure l’Anjou, le Maine, la Bretagne, le Poitou, le Limousin, la Marche, le Berry, le Nivernais, le Bourbonnais, l’Auvergne, on verra qu’il y a la moitié de ces provinces en bruyères qui forment des plaines immenses, qui toutes cependant pourraient être cultivées. » En Touraine, en Poitou, en Berry, ce sont des solitudes de trente mille arpents. Dans un seul canton, près de Preuilly, la bruyère couvre quarante mille arpents de bonne terre. La Société d’Agriculture de Rennes déclare que les deux tiers de la Bretagne sont en friche. — Ce n’est pas stérilité, mais décadence. Le régime inventé par Louis XIV a fait son effet, et depuis un siècle la terre retourne à l’état sauvage. « On ne voit que châteaux abandonnés et en ruine ; tous les chefs-lieux de fiefs, qui autrefois étaient habités par une noblesse aisée, sont aujourd’hui occupés par de pauvres métayers pâtres, dont les faibles travaux produisent à peine leur subsistance et un reste d’impôt prêt à s’anéantir par la ruine des propriétaires et la désertion des colons. » Dans l’élection de Confolens, telle terre affermée 2 956 livres en 1665, n’est plus louée que 900 livres en 1747. Sur les confins de la Marche et du Berry, tel domaine qui en 1660 faisait vivre honorablement deux familles seigneuriales, n’est plus qu’une mince métairie improductive ; « on voit encore la trace des sillons qu’imprimait autrefois le soc de la charrue sur toutes les bruyères des alentours ». La Sologne, jadis florissante\footnote{{\itshape Procès-verbaux de l’assemblée provinciale de l’Orléanais} (1787), mémoire de M. d’Autroche.}, est devenue un marécage et une forêt ; cent ans plus tôt, elle produisait trois fois autant de grains ; les deux tiers de ses moulins ont disparu ; il n’y a plus vestige de ses vignobles ; « les bruyères ont pris la place des raisins ». Ainsi délaissée par la pioche et la charrue, une vaste portion du sol a cessé de nourrir les hommes, et le reste, mal cultivé, ne fournit qu’à peine à leurs premiers besoins\footnote{« On s’étonne qu’un peuple si nombreux soit nourri, lorsque la moitié ou le quart de la terre arable est occupée par des friches stériles. » (Arthur Young, II, 137.)}.\par
En premier lieu, si la récolte manque, ce reste demeure inculte ; car le colon est trop pauvre pour acheter les semences, et maintes fois l’intendant est obligé d’en distribuer ; sans quoi, au désastre de l’année courante s’ajouterait la stérilité de l’année suivante\footnote{{\itshape Archives nationales}, H, 1149. Lettre de la comtesse de Saint-Georges (1772) sur les conséquences de la gelée : « Les terres vont achever cette année de rester incultes, comme il y en a déjà beaucoup, dans notre paroisse surtout. » — Théron de Montaugé, {\itshape ib.}, 45, 80.}. Aussi bien, en ce temps-là, toute calamité pèse sur l’avenir autant que sur le présent ; pendant deux ans, en 1784 et 1785, dans le Toulousain, la sécheresse ayant fait périr les animaux de trait, nombre de cultivateurs sont obligés de laisser leurs champs en friche. — En second lieu, quand on cultive, c’est à la façon du moyen âge. Arthur Young, en 1789, juge qu’en France « l’agriculture en est encore au dixième siècle\footnote{Arthur Young, II, 112, 115. — Théron de Montaugé, 52, 61.} ». Sauf en Flandre et dans la plaine d’Alsace, les champs restent en jachère un an sur trois, et souvent un an sur deux. Mauvais outils ; point de charrues en fer ; en maint endroit, on s’en tient à la charrue de Virgile. L’essieu des charrettes et les cercles des roues sont en bois, et plus d’une fois la herse est une échelle de charrette. Peu de bestiaux, peu de fumures ; le capital appliqué à la culture est trois fois moindre qu’aujourd’hui. Faibles produits : « Nos terres communes, dit un bon observateur, donnent environ, à prendre l’une dans l’autre, six fois la semence\footnote{Le marquis de Mirabeau, \href{http://gallica.bnf.fr/ark:/12148/bpt6k89089c.item}{\dotuline{{\itshape Traité de la population}}} [\url{http://gallica.bnf.fr/ark:/12148/bpt6k89089c.item}], 29.}. » En 1778, dans la riche contrée qui environne Toulouse, le blé ne rend que cinq pour un ; aujourd’hui, c’est huit, et davantage. Arthur Young calcule que, de son temps, l’acre anglaise produit vingt-huit boisseaux de grain, l’acre française dix-huit, que le produit total de la même terre pendant le même laps de temps est de trente-six livres sterling en Angleterre, et seulement de vingt-cinq en France. — Comme les chemins vicinaux sont affreux et que les transports sont souvent impraticables, il est clair que, dans les cantons écartés, dans les mauvais sols qui rendent à peine trois fois la semence, il n’y a pas toujours de quoi manger. Comment vivre jusqu’à la prochaine récolte ? Telle est la préoccupation constante avant et pendant la Révolution. Dans les correspondances manuscrites, je vois les syndics et maires de village estimer la quantité des subsistances locales, tant de boisseaux dans les greniers, tant de gerbes dans les granges, tant de bouches à nourrir, tant de jours jusqu’aux blés d’août, et conclure qu’il s’en faut de deux, trois, quatre mois pour que l’approvisionnement suffise. — Un pareil état des communications et de l’agriculture condamne un pays aux disettes périodiques, et j’ose dire qu’à côté de la petite vérole qui, sur huit morts, en cause une, on trouve alors une maladie endémique aussi régnante, aussi meurtrière, qui est la faim.\par
On se doute bien que c’est le peuple, et surtout le paysan, qui en pâtit. Sitôt que le prix du pain hausse, il n’y peut plus atteindre, et même sans hausse il n’y atteint qu’avec peine. Le pain de froment coûte comme aujourd’hui de trois à quatre sous la livre\footnote{Cf. Galiani, \href{http://gallica.bnf.fr/ark:/12148/bpt6k88409r}{\dotuline{{\itshape Dialogues sur le commerce des blés}}} [\url{http://gallica.bnf.fr/ark:/12148/bpt6k88409r}] (1770), 193. Le pain de froment coûte alors quatre sous la livre.}, mais la moyenne d’une journée d’homme n’est que de dix-neuf sous au lieu de quarante, en sorte qu’avec le même travail, au lieu d’un pain, le journalier ne peut acheter que la moitié d’un pain\footnote{Arthur Young, II, 200, 201, 260 à 265. — Théron de Montaugé, 59, 68, 75, 79, 81, 84.}. Tout calculé, et les salaires étant ramenés au prix du grain, on trouve que le travail annuel exécuté par l’ouvrier rural pouvait alors lui procurer neuf cent cinquante-neuf litres de blé, aujourd’hui dix-huit cent cinquante et un ; ainsi, son bien-être s’est accru de 93 pour 100. Celui d’un maître valet s’est accru de 70 pour 100 ; celui d’un vigneron de 125 pour 100. Cela suffit pour montrer quel était alors leur malaise. — Et ce malaise est propre à la France. Par des observations et des calculs analogues, Arthur Young arrive à montrer qu’en France « ceux qui vivent du travail des champs, et ce sont les plus nombreux, sont de 76 pour 100 moins à leur aise qu’en Angleterre, de 76 pour 100 plus mal nourris, plus mal vêtus, plus mal traités en santé et en maladie ». — Aussi bien, dans les sept huitièmes du royaume, il n’y a pas de fermiers, mais des métayers. Le paysan est trop pauvre pour devenir entrepreneur de culture ; il n’a point de capital agricole\footnote{{\itshape Éphémérides du citoyen}, VI, 81 à 94 (1767), et IX, 99 (1767).}. « Le propriétaire qui veut faire valoir sa terre ne trouve pour la cultiver que des malheureux qui n’ont que leurs bras ; il est obligé de faire à ses frais toutes les avances de la culture, bestiaux, instruments et semences, d’avancer même à ce métayer de quoi le nourrir jusqu’à la première récolte. » — « À Vatan, par exemple, dans le Berry, presque tous les ans les métayers empruntent du pain au propriétaire, afin de pouvoir attendre la moisson. » — « Il est très rare d’en trouver qui ne s’endettent pas envers leur maître d’au moins cent livres par an. » Plusieurs fois, celui-ci leur propose de leur laisser toute la récolte, à condition qu’ils ne lui demanderont rien de toute l’année ; « ces misérables » ont refusé ; livrés à eux seuls, ils ne seraient pas sûrs de vivre  En Limousin et en Angoumois, leur pauvreté est telle\footnote{Turgot ({\itshape Collection des Économistes}), I, 544, 549.}, « qu’ils n’ont pas, déduction faite des charges qu’ils supportent, plus de vingt-cinq à trente livres à dépenser par an et par personne, je ne dis pas en argent, mais en comptant tout ce qu’ils consomment en nature sur ce qu’ils ont récolté. Souvent ils ont moins, et, lorsqu’ils ne peuvent absolument subsister, le maître est obligé d’y suppléer… Le métayer est toujours réduit à ce qu’il faut absolument pour ne pas mourir de faim »  Quant au petit propriétaire, au villageois qui laboure lui-même son propre champ, sa condition n’est guère meilleure. « L’agriculture\footnote{Marquis de Mirabeau, \href{http://gallica.bnf.fr/ark:/12148/bpt6k89089c}{\dotuline{{\itshape Traité de la population}}} [\url{http://gallica.bnf.fr/ark:/12148/bpt6k89089c}], 83.}, telle que l’exercent nos paysans, est une véritable galère ; ils périssent par milliers dès l’enfance, et, dans l’adolescence, ils cherchent à se placer partout ailleurs qu’où ils devraient être. » En 1783, dans toute la plaine du Toulousain, ils ne mangent que du maïs, de la mixture, de menus grains, très peu de blé ; pendant la moitié de l’année, ceux des montagnes vivent de châtaignes ; la pomme de terre est à peine connue, et, selon Arthur Young, sur cent paysans, quatre-vingt-dix-neuf refuseraient d’en manger. D’après les rapports des intendants, le fond de la nourriture en Normandie est l’avoine, dans l’élection de Troyes le sarrasin, dans la Marche et le Limousin le sarrasin avec des châtaignes et des raves, en Auvergne le sarrasin, les châtaignes, le lait caillé et un peu de chèvre salée ; en Beauce, un mélange d’orge et de seigle ; en Berry, un mélange d’orge et d’avoine. Point de pain de froment : le paysan ne consomme que les farines inférieures, parce qu’il ne peut payer son pain que deux sous la livre. Point de viande de boucherie : tout au plus il tue un porc par an. Sa maison est en pisé, couverte de chaume, sans fenêtres, et la terre battue en est le plancher. Même quand le terrain fournit de bons matériaux, pierre, ardoises et tuiles, les fenêtres n’ont point de vitres. Dans une paroisse de Normandie\footnote{Hippeau, VI, 91.}, en 1789, « la plupart sont bâties sur quatre fourches » ; souvent ce sont des étables ou des granges où l’on a élevé une cheminée avec quatre gaules et de la boue ». Pour vêtements, des haillons, et souvent, en hiver, des haillons de toile. Dans le Quercy et ailleurs, point de bas, ni de souliers, ni de sabots. « Impossible, dit Young, pour une imagination anglaise de se figurer les animaux qui nous servirent à Souillac, à l’hôtel du {\itshape Chapeau Rouge ;} des êtres appelés femmes par la courtoisie des habitants, en réalité des tas de fumier ambulants. Mais ce serait en vain que l’on chercherait en France une servante d’hôtel proprement mise. » — Lisez quelques descriptions prises sur place, et vous verrez qu’en France l’aspect de la campagne et des paysans est le même qu’en Irlande, du moins dans les grands traits.

\section[{III. Aspect de la campagne et du paysan.}]{III. Aspect de la campagne et du paysan.}

\noindent Dans les contrées les plus fertiles, en Limagne par exemple, chaumières et visages, tout annonce\footnote{Dulaure, {\itshape Description de l’Auvergne} (1789).} « la misère et la peine »  « La plupart des paysans sont faibles, exténués, de petite stature. » Presque tous récoltent dans leurs héritages du blé et du vin, mais sont forcés de les vendre pour payer leurs rentes et leurs impositions ; ils ne mangent qu’un pain noir fait de seigle et d’orge, et n’ont pour boisson que de l’eau jetée sur le restant des marcs. « Un Anglais\footnote{Arthur Young, I, 235.} qui n’a pas quitté son pays ne peut se figurer l’apparence de la majeure partie des paysannes en France. » Arthur Young, qui cause avec l’une d’entre elles en Champagne, dit que, « même d’assez près, on lui eût donné de soixante à soixante-dix ans, tant elle était courbée, tant sa figure était ridée et durcie par le travail ; elle me dit n’en avoir que vingt-huit ». Cette femme, son mari et son ménage sont un échantillon assez exact de la condition du petit cultivateur propriétaire. Ils ont pour tout bien un coin de terre, une vache et un pauvre petit cheval ; leurs sept enfants consomment tout le lait de la vache. Ils doivent à un seigneur un franchard (42 livres) de froment et trois poulets, à un autre trois franchards d’avoine, un poulet et un sou, à quoi il faut joindre la taille et les autres impôts. « Dieu nous vienne en aide, disait-elle, car les tailles et les droits nous écrasent ! » — Que sera-ce donc dans les contrées où la terre est mauvaise   « Des Ormes (près de Châtellerault) jusqu’à Poitiers, écrit une dame\footnote{ {\itshape Éphémérides du citoyen}, XX, 146 (Lettre de la marquise de… 17 août 1767).
 }, il y a beaucoup de terrain qui ne rapporte rien, et, depuis Poitiers jusque chez moi (en Limousin), il y a vingt-cinq mille arpents de terrain qui ne sont que de la brande et des joncs marins. Les paysans y vivent de seigle dont on n’ôte pas le son, qui est noir et lourd comme du plomb  Dans le Poitou et ici, on ne laboure que l’épiderme de la terre, avec une petite vilaine charrue sans roues… Depuis Poitiers jusqu’à Montmorillon, il y a neuf lieues, qui en valent seize de Paris, et je vous assure que je n’y ai vu que quatre hommes, et trois de Montmorillon chez moi, où il y a quatre lieues ; encore ne les avons-nous aperçus que de loin, car nous n’en avons pas trouvé un seul sur le chemin. Vous n’en serez pas étonné dans un tel pays… On a soin de les marier d’aussi bonne heure que les grands seigneurs », sans doute par crainte de la milice. « Mais le pays n’en est pas plus peuplé, car presque tous les enfants meurent. Les femmes n’ayant presque pas de lait, les enfants d’un an mangent de ce pain dont je vous ai parlé ; aussi une fille de quatre ans a le ventre gros comme une femme enceinte… Les seigles ont été gelés cette année, le jour de Pâques ; il y a peu de froment ; des douze métairies qu’a ma mère, il y en a peut-être dans quatre. Il n’a pas plu depuis Pâques : pas de foin, pas de pâturage, aucun légume, pas de fruits ; voilà l’état du pauvre paysan ; par conséquent, point d’engrais, de bestiaux… Ma mère, qui avait toujours plusieurs de ses greniers pleins, n’y a pas un grain de blé, parce que, depuis deux ans, elle nourrit tous ses métayers et les pauvres. » — « On secourt le paysan, dit un seigneur de la même province\footnote{ Lucas de Montigny, {\itshape Mémoires de Mirabeau}, I, 394.
 }, on le protège, rarement on lui fait tort, mais on le dédaigne. On l’assujettit s’il est bon et facile ; on l’aigrit et l’on l’irrite s’il est méchant… Il est tenu dans la misère, dans l’abjection, par des hommes qui ne sont rien moins qu’inhumains, mais dont le préjugé, surtout dans la noblesse, est qu’il n’est pas de même espèce que nous… Le propriétaire tire tout ce qu’il peut et, dans tous les cas, le regardant lui et ses bœufs comme bêtes domestiques, il les charge de voitures et s’en sert dans tous les temps pour tous voyages, charrois, transports. De son côté, ce métayer ne songe qu’à vivre avec le moins de travail possible, à mettre le plus de terrain qu’il peut en dépaître ou pacages, attendu que le produit provenant du croît du bétail ne lui coûte aucun travail. Le peu qu’il laboure, c’est pour semer des denrées de vil prix, propres à sa nourriture, le blé noir, les raves, etc. Il n’a de jouissance que sa paresse et sa lenteur, d’espérance que dans une bonne année de châtaignes, et d’occupation volontaire que d’engendrer » ; faute de pouvoir louer des valets de ferme, il fait des enfants  Les autres, manœuvres, ont quelques petits fonds, et surtout « vivent sur le spontané et de quelques chèvres qui dévorent tout ». Encore bien souvent, et sur ordre du Parlement, elles sont tuées par les gardes. Une femme avec deux enfants au maillot, « sans lait, sans un pouce de terre », à qui l’on a tué ainsi deux chèvres, son unique ressource, une autre à qui l’on a tué sa chèvre unique et qui est à l’aumône avec son fils, viennent pleurer à la porte du château ; l’une reçoit douze livres, l’autre est admise comme servante, et désormais « ce village donne de grands coups de chapeau, avec une physionomie bien riante »  En effet, ils ne sont pas habitués aux bienfaits ; pâtir et le lot de tout ce pauvre monde. « Ils croient inévitable, comme la pluie et la grêle, la nécessité d’être opprimés par le plus fort, le plus riche, le plus accrédité, et c’est ce qui leur imprime, s’il est permis de parler ainsi, un caractère de souffre-douleur. »\par
En Auvergne, pays féodal, tout couvert de grands domaines ecclésiastiques et seigneuriaux, la misère est égale. À Clermont-Ferrand\footnote{Arthur Young, I, 280, 289, 294.}, « il y a des rues qui, pour la couleur, la saleté et la mauvaise odeur, ne peuvent se comparer qu’à des tranchées dans un tas de fumier. » Dans les auberges des gros bourgs, « étroitesse, misère, saleté, ténèbres ». Celle de Pradelles est « l’une des pires de France ». Celle d’Aubenas, dit Young, « serait le purgatoire d’un de mes pourceaux ». En effet, les sens sont bouchés : l’homme primitif est content dès qu’il peut dormir et se repaître. Il se repaît, mais de quelle nourriture ! Pour supporter cette pâtée indigeste, il faut ici au paysan un estomac plus coriace encore qu’en Limousin ; dans tel village où, dix ans plus tard, on tuera chaque année vingt-cinq porcs, on n’en mange que deux ou trois par an\footnote{La Fayette, {\itshape Mémoires}, V, 533.}  Quand on contemple la rudesse de ce tempérament intact depuis Vercingétorix et, de plus, effarouché par la souffrance, on ne peut se défendre de quelque effroi. Le marquis de Mirabeau décrit « la fête votive du Mont-Dore, les sauvages descendant en torrents de la montagne\footnote{Lucas de Montigny, {\itshape ibid.} (Lettre du 18 août 1777.)}, le curé avec étole et surplis, la justice en perruque, la maréchaussée, le sabre à la main, gardant la place avant de permettre aux musettes de commencer ; la danse interrompue un quart d’heure après par la bataille ; les cris et les sifflements des enfants, des débiles et autres assistants, les agaçant comme fait la canaille quand les chiens se battent ; des hommes affreux, ou plutôt des bêtes fauves, couverts de sayons de grosse laine, avec de larges ceintures de cuir piquées de clous de cuivre, d’une taille gigantesque rehaussée par de hauts sabots, s’élevant encore pour regarder le combat, trépignant avec progression, se frottant les flancs avec les coudes, la figure hâve et couverte de longs cheveux gras, le haut du visage pâlissant et le bas se déchirant pour ébaucher un rire cruel et une sorte d’impatience féroce  Et ces gens-là payent la taille ! et l’on veut encore leur ôter le sel ! Et l’on ne sait pas ce qu’on dépouille, ce qu’on croit gouverner, ce qu’à coups d’une plume nonchalante et lâche on croira, jusqu’à la catastrophe, affamer toujours impunément ! Pauvre Jean-Jacques, me disais-je, qui t’enverrait, toi et ton système, copier de la musique chez ces gens-là aurait bien durement répondu à ton discours. » Avertissement prophétique, prévoyance admirable que l’excès du mal n’aveugle point sur le mal du remède. Éclairé par son instinct féodal et rural, le vieux gentilhomme juge du même coup le gouvernement et les philosophes, l’Ancien Régime et la Révolution.

\section[{IV. Comment le paysan devient propriétaire. — Il n’en est pas plus à l’aise. — Aggravation de ses charges. — Dans l’ancien régime il est le « mulet ».}]{IV. Comment le paysan devient propriétaire. — Il n’en est pas plus à l’aise. — Aggravation de ses charges. — Dans l’ancien régime il est le « mulet ».}

\noindent Quand l’homme est misérable, il s’aigrit ; mais quand il est à la fois propriétaire et misérable, il s’aigrit davantage. Il a pu se résigner à l’indigence, il ne se résigne pas à la spoliation ; et telle était la situation du paysan en 1789 ; car, pendant tout le dix-huitième siècle, il avait acquis de la terre  Comment avait-il fait, dans une telle détresse ? La chose est à peine croyable, quoique certaine ; on ne peut l’expliquer que par le caractère du paysan français, par sa sobriété, sa ténacité, sa dureté pour lui-même, sa dissimulation, sa passion héréditaire pour la propriété et pour la terre. Il avait vécu de privations, épargné sou sur sou. Chaque année, quelques pièces blanches allaient rejoindre son petit tas d’écus enterré au coin le plus secret de sa cave ; certainement, le paysan de Rousseau, qui cachait son vin et son pain dans un silo, avait une cachette plus mystérieuse encore ; un peu d’argent dans un bas de laine ou dans un pot échappe mieux que le reste à l’inquisition des commis. En guenilles, pieds nus, ne mangeant que du pain noir, mais couvant dans son cœur le petit trésor sur lequel il fondait tant d’espérances, il guettait l’occasion, et l’occasion ne manquait pas. « Malgré tous ses privilèges, écrit un gentilhomme en 1755\footnote{ Tocqueville, 117.
 }, la noblesse se ruine et s’anéantit tous les jours, le Tiers-état s’empare des fortunes. » Nombre de domaines passent ainsi, par vente forcée ou volontaire, entre les mains des financiers, des gens de plume, des négociants, des gros bourgeois. Mais il est sûr qu’avant de subir la dépossession totale, le seigneur obéré s’est résigné aux aliénations partielles. Le paysan, qui a graissé la patte du régisseur, se trouve là avec son magot. « Mauvaise terre, Monseigneur, et qui vous coûte plus qu’elle ne vous rapporte. » Il s’agit d’un lopin isolé, d’un bout de champ ou de pré, parfois d’une ferme dont le fermier ne paye plus, plus souvent d’une métairie dont les métayers besogneux et paresseux tombent chaque année à la charge du maître. Celui-ci peut se dire que la parcelle aliénée n’est pas perdue pour lui, puisqu’un jour, par droit de rachat, il pourra la reprendre, et puisqu’en attendant il touchera un cens, des redevances, le profit des lods et ventes. D’ailleurs, il y a chez lui et autour de lui de grandes espaces vides que la décadence de la culture et la dépopulation ont laissés déserts. Pour les remettre en valeur, il faut en céder la propriété ; nul autre moyen de rattacher l’homme à la terre  Et le gouvernement aide à l’opération : ne percevant plus rien sur le sol abandonné, il consent à retirer provisoirement sa main trop pesante. Par l’édit de 1766, une terre défrichée reste affranchie pour quinze ans de la taille d’exploitation, et, là-dessus, dans vingt-huit provinces, quatre cent mille arpents sont défrichés en trois ans\footnote{{\itshape Procès-verbaux de l’assemblée provinciale de Basse Normandie} (1787), 205.}.\par
Voilà comment, par degrés, le domaine seigneurial s’émiette et s’amoindrit. Vers la fin, en quantité d’endroits, sauf le château et la petite ferme attenante qui rapporte deux ou trois mille francs par an\footnote{Léonce de Lavergne, 26 (d’après les tableaux de l’indemnité accordée aux émigrés en 1825). Dans la terre de Blet (voir note 2 p. 300), vingt-deux parcelles sont aliénées en 1760. — Arthur Young, I, 308 (domaine de la Tour-d’Aigues, en Provence), et II, 198, 214. — Doniol, {\itshape Histoire des classes rurales}, 450. — Tocqueville, 36.}, le seigneur n’a plus que ses droits féodaux ; tout le reste du sol est au paysan. Déjà vers 1750, Forbonnais note que beaucoup de nobles et d’anoblis, « réduits à une pauvreté extrême avec des titres de propriété immense », ont vendu au petit cultivateur à bas pris, souvent pour le montant de la taille. Vers 1760, un quart du sol dit-on, avait déjà passé aux mains des travailleurs agricoles. En 1772, à propos du vingtième qui se perçoit sur le revenu net des immeubles, l’intendant de Caen, ayant fait le relevé de ses cotes, estime que, sur cent cinquante mille, « il y en a peut-être cinquante mille dont l’objet n’excède pas cinq sous et peut-être encore autant qui n’excèdent pas vingt sous\footnote{ \noindent {\itshape Archives nationales}, H, 1463 (Lettre de M. de Fontette du 16 novembre 1771). — Cf. Cochut, {\itshape Revue des Deux Mondes}, septembre 1848. La vente des biens nationaux ne paraît pas avoir augmenté sensiblement le nombre des petites propriétés ni diminué sensiblement le nombre des grandes ; ce que la Révolution a développé, c’est la propriété moyenne. En 1848, on compte 183 000 grandes propriétés (23 000 familles payant 500 francs de contributions et au-dessus et possédant 260 hectares en moyenne, 160 000 familles payant de 250 à 500 francs de contributions et possédant 85 hectares en moyenne). Ces 183 000 familles possèdent 18 millions d’hectares.\par
 — En outre, 700 000 propriétés moyennes (payant de 50 à 250 francs d’impôt) et comprenant 15 millions d’hectares. — Enfin 3 900 000 petites, comprenant 15 millions d’hectares (900 000 payant de 25 à 50 francs d’impôt, moyenne 5 hectares et demi, 3 millions payant moins de 25 francs, moyenne 3 hectares 1 neuvième). — D’après les relevés partiels de M. de Tocqueville, le nombre des propriétaires fonciers s’est accru en moyenne de 5 douzièmes ; or la population s’est accrue en même temps de 5 treizièmes (de 26 à 36 millions).
 }. » Des observateurs contemporains constatent cette passion du paysan pour la propriété foncière. « Toutes les épargnes des basses classes, qui ailleurs sont placées sur des particuliers et dans les fonds publics, sont destinées en France à l’achat des terres. » — « Aussi le nombre des petites propriétés rurales va toujours croissant. Necker dit qu’il y en a « une {\itshape immensité} ». Arthur Young, en 1789, s’étonne de leur prodigieuse multitude et « penche à croire qu’elles forment le tiers du royaume ». Ce serait déjà notre chiffre actuel, et l’on trouve encore, à peu de chose près, le chiffre actuel, si l’on cherche le nombre des propriétaires comparé au nombre des habitants.\par
\par
Mais, en acquérant le sol, le petit cultivateur en prend pour lui les charges. Tant qu’il était simple journalier et n’avait que ses bras, l’impôt ne l’atteignait qu’à demi : « où il n’y a rien, le roi perd ses droits ». Maintenant, il a beau être pauvre et se dire encore plus pauvre, le fisc a prise sur lui par toute l’étendue de sa propriété nouvelle. Les collecteurs, paysans comme lui et jaloux à titre de voisins, savent ce que son bien au soleil lui a rapporté ; c’est pourquoi on lui prend tout ce qu’on peut lui prendre. En vain il a travaillé avec une âpreté nouvelle, ses mains restent aussi vides, et, au bout de l’année, il découvre que son champ n’a rien produit pour lui. Plus il acquiert et produit, plus ses charges deviennent lourdes. En 1715, la taille et la capitation, qu’il paye seul ou presque seul, étaient de 66 millions ; elles sont de 93 en 1759, de 110 en 1789\footnote{{\itshape Compte général des revenus et dépenses fixes au} 1\textsuperscript{er} {\itshape mai} 1789 (Imprimerie royale, 1789). — Duc de Luynes, XVI, 49  Buchez et Roux, I, 206, 374. (Il ne s’agit ici que des pays d’élection ; mais, dans les pays d’états, l’augmentation n’est pas moins forte.) — {\itshape Archives nationales}, H\textsuperscript{2}, 1610 (paroisse du Bourget, en Anjou). Extrait des rôles de la taille pour trois métairies à M. de Ruillé : Impôts en 1762, 334 livres 3 sous ; en 1783, 372 livres 15 sous.}. En 1757, l’impôt est de 283 156 000 livres ; en 1789, de 476 294 000. — Sans doute, en théorie, par humanité et bon sens, on veut le soulager, on a pitié de lui. Mais en pratique, par nécessité et routine, on le traite, selon le précepte du cardinal de Richelieu, comme une bête de somme à qui l’on mesure l’avoine, de peur qu’il ne soit trop fort et regimbe, « comme un mulet qui, étant accoutumé à la charge, se gâte plus par un long repos que par le travail ».
\chapterclose


\chapteropen

\chapter[{Chapitre II. Principale cause de la misère : l’impôt.}]{Chapitre II. \\
Principale cause de la misère : l’impôt.}


\chaptercont

\section[{I. Impôts directs. — État de divers domaines à la fin de Louis XV. — Prélèvements du décimateur et du fisc. — Ce qui reste au propriétaire.}]{I. Impôts directs. — État de divers domaines à la fin de Louis XV. — Prélèvements du décimateur et du fisc. — Ce qui reste au propriétaire.}

\noindent Considérons de près les extorsions dont il souffre ; elles sont énormes et au-delà de tout ce que nous pouvons imaginer. Depuis longtemps, les économistes ont dressé le budget d’une terre et prouvé par des chiffres l’excès des charges dont le cultivateur est accablé  Si l’on veut qu’il continue à cultiver, il faut lui faire sa part dans la récolte, part inviolable, qui est d’environ la moitié du produit brut, et de laquelle on ne peut rien distraire sans le ruiner. En effet elle représente juste, et sans un sou de trop : en premier lieu, l’intérêt du capital primitif qu’il a mis dans son exploitation, bestiaux, meubles, outils, instruments aratoires ; en second lieu, l’entretien annuel de ce même capital, qui dépérit par la durée et par l’usage ; en troisième lieu, les avances qu’il a faites dans l’année courante, semences, salaires des ouvriers, nourriture des animaux et des hommes ; en dernier lieu, la compensation qui lui est due pour ses risques et ses pertes. Voilà une créance privilégiée qu’il faut solder au préalable, avant toutes les autres, avant celle du seigneur, avant celle du décimateur, avant celle du roi lui-même ; car elle est la créance de la terre\footnote{{\itshape Collection des Économistes}, II, 832 (Tableau économique par Beaudau).}. C’est seulement après l’avoir remboursée qu’on peut toucher au reste, qui est le bénéfice véritable, le {\itshape produit net.} Or, dans l’état où est l’agriculture, le décimateur et le roi prennent la moitié de ce produit net si la terre est grande, et ils le prennent tout entier si la terre est petite\footnote{{\itshape Éphémérides du citoyen}, IX, 15 (article de M. de Butret, 1767).}. Telle grosse ferme de Picardie, qui vaut 3 600 livres au propriétaire, paye 1 800 livres au roi et 1 311 livres au décimateur ; telle autre, dans le Soissonnais, louée 4 500 livres, paye 2 200 livres d’impôt et plus de 1 000 écus de dîme. Une métairie moyenne près de Nevers donne 138 livres au Trésor, 121 à l’Église, et 114 au propriétaire. Dans une autre, en Poitou, le fisc prend 348 livres, et le propriétaire n’en reçoit que 238. En général, dans les pays de grandes fermes, le propriétaire touche 10 livres par arpent si la culture est très bonne, 3 livres si elle est ordinaire. Dans les pays de petites fermes et de métayage, il touche par arpent 15 sous, 8 sous et même 6 sous  C’est que tout le profit net va au Clergé et au Trésor.\par
Et cependant ses colons ne lui coûtent guère. Dans cette métairie du Poitou qui rapporte 8 sous par arpent, les 36 colons consomment chacun par an pour 26 francs de seigle, pour 2 francs de légumes, huile et laitage, pour 2 francs 10 sous de porc ; en tout, par année et par personne, 16 livres de viande et 36 francs de dépense totale. En effet, ils ne boivent que de l’eau, ils s’éclairent et font la soupe avec de l’huile de navette, ils ne goûtent jamais de beurre, ils s’habillent de la laine de leurs ouailles et du chanvre qu’ils cultivent ; ils n’achètent rien, sauf la main-d’œuvre des toiles et serges dont ils fournissent la matière  Dans une autre métairie sur les confins de la Marche et du Berry, les 46 colons coûtent moins encore, car chacun d’eux ne consomme que pour 25 francs par an. Jugez de la part exorbitante que s’adjugent l’Église et l’État, puisque, avec des frais de culture si minimes, le propriétaire trouve dans sa poche, à la fin de l’année, 6 ou 8 sous par arpent, sur quoi, lorsqu’il est roturier, il doit encore payer les redevances à son seigneur, mettre pour la milice à la bourse commune, acheter son sel de devoir, faire sa corvée, et le reste. Vers la fin du règne de Louis XV, en Limousin, dit Turgot, le roi, à lui seul, tire « à peu près autant de la terre que le propriétaire\footnote{{\itshape Collection des Économistes}, I, 551, 562.} ». Il y a telle élection, celle de Tulle, où il prélève 56 1/2 pour 100 du produit ; il n’en reste à l’autre que 43 1/2 ; par suite « une multitude de domaines y sont abandonnés »  Et ne croyez pas qu’avec le temps la charge devienne moins pesante, ou que dans les autres provinces le cultivateur soit mieux traité. À cet égard les documents sont authentiques et presque de la dernière heure. Il suffit de relever les procès-verbaux des assemblées provinciales tenues en 1787 pour apprendre en chiffres officiels jusqu’à quel point le fisc peut abuser des hommes qui travaillent, et leur ôter de la bouche le pain qu’ils ont gagné à la sueur de leur front.

\section[{II. État de plusieurs provinces au moment de la Révolution. — Taille, accessoires, capitations, vingtièmes, impôt des corvées. — Ce que chacune de ces taxes prélève sur le revenu. — Enormité du prélèvement total.}]{II. État de plusieurs provinces au moment de la Révolution. — Taille, accessoires, capitations, vingtièmes, impôt des corvées. — Ce que chacune de ces taxes prélève sur le revenu. — Enormité du prélèvement total.}

\noindent Il ne s’agit ici que de l’impôt direct, tailles, accessoires, capitation taillable, vingtièmes, taxe pécuniaire substituée à la corvée\footnote{{\itshape Procès-verbaux de l’assemblée provinciale de Champagne} (1787), 24.}. En Champagne, sur 100 livres de revenu, le contribuable paye 54 livres 15 sous à l’ordinaire et 71 livres 13 sous dans plusieurs paroisses\footnote{Cf. {\itshape Notice historique sur la Révolution dans le département de l’Eure}, par Boivin-Champeaux, 37. Cahier de la paroisse d’Epreville : sur 100 francs de rente, le Trésor prend 25 livres pour la taille, 16 pour les accessoires, 15 pour la capitation, 11 pour les vingtièmes, total 67 livres.}. Dans l’Ile-de-France, « soit un habitant taillable de village, propriétaire de vingt arpents de terre qu’il exploite lui-même et qui sont évalués à 10 livres de revenu par arpent ; on le suppose aussi propriétaire de la maison qu’il habite et dont le prix de location est évalué à 40 livres\footnote{{\itshape Procès-verbaux de l’assemblée provinciale de l’Ile-de-France} (1786), 131.} ». Ce taillable paye pour sa taille réelle, personnelle et industrielle 36 livres 14 sous, pour les accessoires de la taille 17 livres 17 sous, pour sa capitation 21 livres 8 sous, pour ses vingtièmes 24 livres 4 sous : en tout 99 livres 3 sous ; à quoi il faut ajouter environ 5 livres pour le remplacement de la corvée : en tout 104 livres pour un bien qu’il louerait 240 livres, plus des cinq douzièmes de son revenu  C’est bien pis si l’on fait le compte pour les généralités pauvres. Dans la Haute-Guyenne\footnote{{\itshape Procès-verbaux de l’assemblée provinciale de la Haute-Guyenne} (1784), tome II, 17, 40, 47.}, « tous les fond de terre sont taxés, pour la taille, les accessoires et les vingtièmes, à plus du quart du revenu, déduction faite seulement des frais de culture, et les maisons au tiers du revenu, déduction faite seulement des frais de réparation et d’entretien ; à quoi il faut ajouter la capitation, qui prend environ un dixième du revenu, la dîme qui en prend un septième, les rentes seigneuriales, qui en prennent un autre septième, l’impôt en remplacement de la corvée, les frais de recouvrement forcé, saisies, séquestres et contraintes, les charges locales ordinaires et extraordinaires. Cela défalqué, on reconnaît que, dans les communautés moyennement imposées, il ne reste pas au propriétaire la jouissance du tiers du revenu, et que, dans les communautés lésées par la répartition, les propriétaires sont réduits à la condition de simples fermiers qui recueillent à peine de quoi récupérer les frais de culture ». En Auvergne\footnote{{\itshape Procès-verbaux de l’assemblée} provinciale {\itshape d’Auvergne} (1787), 253  {\itshape Doléances}, par Gaultier de Biauzat, membre du conseil nommé par l’assemblée provinciale d’Auvergne (1788), 3.}, la taille monte à 4 sous pour livre du produit net ; les accessoires et la capitation emportent 4 autres sous et 3 deniers ; les vingtièmes, 2 sous et 3 deniers ; la contribution pour les chemins royaux, le don gratuit, les charges locales et les frais de perception prennent encore 1 sou 1 denier : total, 11 sous et 7 deniers par livre de revenu, sans compter les droits seigneuriaux et la dîme. « Bien plus, le bureau a reconnu avec douleur que plusieurs collectes payent à raison de 17 sous, de 16 sous, et les plus modérées à raison de 14 sous (par livre). Les preuves en sont sur le bureau ; elles sont consignées dans les registres de la Cour des Aides et des sièges des élections. Elles le sont encore plus dans les rôles des paroisses, où l’on trouve une infinité de cotes faites sur des biens abandonnés que les collecteurs afferment et dont le produit souvent ne suffit pas pour le payement de l’impôt. » — De pareils chiffres sont d’une éloquence terrible, et je crois pouvoir les résumer en un seul. Si l’on met ensemble la Normandie, l’Orléanais, le Soissonnais, la Champagne, l’Ile-de-France, le Berry, le Poitou, l’Auvergne, le Lyonnais, la Gascogne et la Haute-Guyenne, bref les principaux pays d’élections, on trouvera que, sur 100 francs de revenu net, l’impôt direct prenait au taillable\footnote{Voir la note 5 [’p. 308’].} 53 francs, plus de la moitié. C’est à peu près cinq fois autant qu’aujourd’hui.

\section[{III. Quatre impôts directs sur le taillable, qui n’a que ses bras.}]{III. Quatre impôts directs sur le taillable, qui n’a que ses bras.}

\noindent Mais le fisc, en s’abattant sur la propriété taillable, n’a pas lâché le taillable qui est sans propriété. À défaut de la terre, il saisit l’homme. À défaut du revenu, on taxe le salaire. Sauf les vingtièmes, tous les impôts précédents atteignent non seulement celui qui possède, mais encore celui qui ne possède pas. En Toulousain\footnote{Théron de Montaugé, 109 (1763). À cette époque le salaire est de 7 à 12 sous par jour en été.}, à Saint-Pierre de Bajourville, le moindre journalier, n’ayant que ses bras pour vivre et gagnant dix sous par jour, paye huit, neuf, dix livres de capitation. « En Bourgogne\footnote{{\itshape Archives nationales.} Procès-verbaux et cahiers des États Généraux, t. 59, 6. Mémoire à M. Necker par M. d’Orgeux, conseiller honoraire au Parlement de Bourgogne, 25 octobre 1788.}, il est ordinaire de voir un malheureux manœuvre, sans aucune possession, imposé à dix-huit ou vingt livres de capitation et de taille. » En Limousin\footnote{{\itshape Archives nationales}, H, 1418. Lettre de l’intendant de Limoges du 26 février 1784.}, tout l’argent que les maçons rapportent en hiver sert à « payer les impositions de leur famille ». Quant aux journaliers de campagne et aux colons, le propriétaire, même privilégié, qui les emploie, est obligé de prendre à son compte une partie de leur cote ; sinon, n’ayant pas de quoi manger, ils ne travailleraient plus\footnote{Turgot, II, 259.} ; même dans l’intérêt du maître, il faut à l’homme sa ration de pain, comme au bœuf sa ration de foin. « En Bretagne\footnote{{\itshape Archives nationales}, H, 426. (Remontrances du Parlement de Bretagne, février 1783.)}, c’est une vérité notoire que les neuf dixièmes des artisans, quoique mal nourris, mal vêtus, n’ont pas à la fin de l’année un écu libre de dettes » ; la capitation et le reste leur enlèvent cet unique et dernier écu. À Paris\footnote{Mercier, XI, 59 ; X, 262.}, « le cendrier, le marchand de bouteilles cassées, le gratte-ruisseau, le crieur de vieilles ferrailles et de vieux chapeaux », dès qu’ils ont un gîte, payent la capitation, trois livres dix sous par tête. Pour qu’ils n’oublient pas de la payer, le locataire qui leur sous-loue est responsable. De plus, en cas de retard, on leur envoie un « homme bleu », un garnisaire, dont ils payent la journée et qui prend domicile dans leur logis. Mercier cite un ouvrier, nommé Quatre-main, ayant quatre petits enfants, logé au sixième, où il avait arrangé une cheminée en manière d’alcôve pour se coucher lui et sa famille. « Un jour, j’ouvris sa porte, qui n’avait qu’un loquet ; la chambre n’offrait que la muraille et un étau ; cet homme, en sortant de dessous sa cheminée, à moitié malade, me dit : « Je croyais que c’était garnison pour la capitation ». » — Ainsi, quelle que soit la condition du taillable, si dégarni et si dénué qu’il puisse être, la main crochue du fisc est sur son dos. Il n’y a point à s’y méprendre : elle ne se déguise pas, elle vient au jour dit s’appliquer directement et rudement sur les épaules. La mansarde et la chaumine, aussi bien que la métairie, la ferme et la maison, connaissent le collecteur, l’huissier, le garnisaire ; nul taudis n’échappe à la détestable engeance. C’est pour eux qu’on sème, qu’on récolte, qu’on travaille, qu’on se prive ; et, si les liards épargnés péniblement chaque semaine finissent au bout de l’an par faire une pièce blanche, c’est dans leur sac qu’elle va tomber.

\section[{IV. La collecte et les saisies.}]{IV. La collecte et les saisies.}

\noindent Il faut voir le système à l’œuvre. C’est une machine à tondre, grossière et mal agencée, qui fait autant de mal par son jeu que par son objet. Et ce qu’il y a de pis, c’est que, dans son engrenage grinçant, les taillables, employés comme instrument final, doivent eux-mêmes se tondre et s’écorcher. Dans chaque paroisse, il y en a deux, trois, cinq, sept, qui, sous le nom de collecteurs et sous l’autorité de l’élu, sont tenus de répartir et de percevoir l’impôt. « Nulle charge plus onéreuse\footnote{{\itshape Archives nationales}, H, 1423. Lettre de M. d’Aîne, intendant de Limoges (17 février 1782), de l’intendant de Moulins (avril 1779). Procès de la communauté de Mollon (Bordelais) et tableau de ses collecteurs.} » ; chacun, par protection ou privilège, tâche de s’y soustraire. Les communautés plaident sans cesse contre les réfractaires, et, pour que nul ne puisse prétexter son ignorance, elles dressent d’avance, pour dix et quinze ans, le tableau des futurs collecteurs. Dans les paroisses de second ordre, ce sont tous « de petits propriétaires, et chacun d’eux passe à la collecte à peu près tous les six ans ». Dans beaucoup de villages, ce sont des artisans, des journaliers, des métayers, qui pourtant auraient besoin de tout leur temps pour gagner leur vie. En Auvergne, où les hommes valides s’expatrient l’hiver pour chercher du travail, on prend les femmes\footnote{{\itshape Procès-verbaux de l’assemblée provinciale d’Auvergne}, 266.} : dans l’élection de Saint-Flour, il y a tel village où les quatre collecteurs sont en jupon. — Pour tous les recouvrements qui leur sont commis, ils sont responsables sur leurs biens, sur leurs meubles, sur leurs personnes, et, jusqu’à Turgot, chacun est solidaire des autres ; jugez de leur peine et de leurs risques ; en 1785\footnote{Albert Babeau, {\itshape Histoire de Troyes}, I, 72.}, dans une seule élection de Champagne, quatre-vingt-quinze sont mis en prison, et chaque année il y en a deux cent mille en chemin. « Le collecteur, dit l’assemblée provinciale du Berry\footnote{{\itshape Procès-verbaux de l’assemblée provinciale de Berry} (1778), t. I, 72, 80.}, passe ordinairement pendant deux ans la moitié de sa journée à courir de porte en porte chez les contribuables en retard. » Cet emploi, écrit Turgot\footnote{Tocqueville, 187.}, cause le désespoir et presque toujours la ruine de ceux qu’on en charge ; on réduit ainsi successivement à la misère toutes les familles aisées d’un village. » En effet, il n’y a point de collecteur qui ne marche par force et ne reçoive chaque année\footnote{{\itshape Traité de la population}, 2\textsuperscript{e} partie, 26.} « huit ou dix commandements ». Parfois on le met en prison aux frais de la paroisse. Parfois on procède contre lui et contre les contribuables « par établissement de garnisons, saisies, saisies-arrêts, saisies-exécutions, et ventes de meubles »  « Dans la seule élection de Villefranche, dit l’assemblée provinciale de la Haute-Guyenne, on compte cent six porteurs de contraintes et autres recors toujours en chemin. »\par
La chose est passée en usage, et la paroisse a beau pâtir, elle pâtirait davantage si elle faisait autrement. « Près d’Aurillac, dit le marquis de Mirabeau\footnote{{\itshape Archives nationales}, H, 1417. (Lettre de M. de Cypierre, intendant d’Orléans, 17 avril 1765.)}, il y a de l’industrie, du labeur, de l’économie, et, sans cela, rien que misère et pauvreté. Cela fait un peuple mi-parti d’insolvables et de riches honteux qui font les pauvres, crainte de surcharge. La taille une fois assise, tout le monde gémit, se plaint, et personne ne paye. Le terme expiré, à l’heure et à la minute, la contrainte marche, et les collecteurs, quoique aisés, se gardent bien de la renvoyer en la payant, quoique, au fond, cette garnison soit fort chère. Mais ces sortes de frais sont d’habitude, et ils y comptent, au lieu qu’ils craignent, s’ils devenaient plus exacts, d’être plus chargés l’année d’ensuite. » En effet, le receveur, qui paye ses garnisaires un franc par jour, les fait payer deux francs et gagne la différence. C’est pourquoi, « si certaines paroisses s’avisent d’être exactes et de payer sans attendre la contrainte, le receveur, qui se voit ôter le plus clair de son bien, se met de mauvaise humeur, et, au département prochain, entre lui, MM. les élus, le subdélégué et autres barbiers de la sorte, on s’arrange de façon que cette exacte paroisse porte double faix, pour lui apprendre à vivre »  Un peuple de sangsues administratives vit ainsi sur le paysan. « Dernièrement, dit un intendant\footnote{Id., ibid.}, dans l’élection de Romorantin, il n’y eut rien à recevoir par les collecteurs dans une vente de meubles qui se montait à six cents livres, parce qu’elle fut absorbée en frais. Dans l’élection de Châteaudun, il en fut de même d’une autre vente qui se montait à neuf cents livres, et on n’est pas informé de toutes les affaires de cette nature, quelques criantes qu’elles soient. » — Au reste, le fisc lui-même est impitoyable. Le même intendant écrit, en 1784, année de famine\footnote{{\itshape Ibid.}, H, 1418. (Lettre du 28 mai 1784.)} : « On a vu avec effroi, dans les campagnes, le collecteur disputer à des chefs de famille le prix de la vente des meubles qu’ils destinaient à arrêter le cri du besoin de leurs enfants. » — C’est que, si les collecteurs ne saisissent pas, ils seraient saisis eux-mêmes. Pressés par le receveur, on les voit dans les documents solliciter, poursuivre, persécuter les contribuables. Chaque dimanche et chaque jour de fête, ils se tiennent à la sortie de l’église, avertissant les retardataires ; puis, dans la semaine, ils vont de chaumière en chaumière pour obtenir leur dû. « Communément, ils ne savent point écrire et mènent avec eux un scribe. » Sur les six cent six qui courent dans l’élection de Saint-Flour, il n’y en a pas dix qui puissent lire le papier officiel et signer un acquit ; de là des erreurs et des friponneries sans nombre. Outre le scribe, ils ont avec eux les garnisaires, gens de la plus basse classe, mauvais ouvriers sans ouvrage, qui se sentent haïs et qui agissent en conséquence. « Quelques défenses qu’on leur fasse de rien prendre, de se faire nourrir par les habitants ou d’aller dans les cabarets avec les collecteurs », le pli est pris, « l’abus continuera toujours\footnote{{\itshape Archives nationales}, H, 1417. (Lettre de l’intendant de Tours du 15 juin 1765.)} ». Mais, si pesants que soient les garnisaires, on se garde bien de les éviter. À cet égard, écrit un intendant, « l’endurcissement est étrange »  « Aucun particulier, mande un receveur\footnote{{\itshape Ibid.} Mémoire de Randon, receveur des tailles de l’élection de Laon (janvier 1764).}, ne paye le collecteur qu’il ne voie la garnison établie chez lui. » Le paysan ressemble à son âne, qui, pour marcher, a besoin d’être battu, et, en cela, s’il paraît stupide, il est politique. Car le collecteur, étant responsable, « penche naturellement à grossir les cotes des payeurs exacts au profit de celles des payeurs négligents. C’est pourquoi le payeur exact devient négligent à son tour, et laisse instrumenter même lorsqu’il a son argent dans son coffre\footnote{{\itshape Procès-verbaux de l’assemblée provinciale de Berry} (1778), I, 72.} ». Tout compte fait, il a calculé que la procédure, même coûteuse, lui coûtera moins qu’une surtaxe, et, de deux maux, il choisit le moindre. Contre le collecteur et le receveur il n’a qu’une ressource, sa pauvreté simulée ou réelle, involontaire ou volontaire. « Tout taillable, dit encore l’assemblée provinciale du Berry, redoute de montrer ses facultés ; il s’en refuse l’usage dans ses meubles, dans ses vêtements, dans sa nourriture et dans tout ce qui est soumis à la vue d’autrui. » — M. de Choiseul-Gouffier\footnote{Chamfort, 93.} voulant faire à ses frais couvrir de tuiles les maisons de ses paysans exposées à des incendies, ils le remercièrent de sa bonté et le prièrent de laisser leurs maisons comme elles étaient, disant que, si elles étaient couvertes de tuiles au lieu de chaume, les subdélégués augmenteraient leurs tailles. » — « On travaille, mais c’est pour satisfaire les premiers besoins… La crainte de payer un écu de plus fait négliger au commun des hommes un profit qui serait quadruple\footnote{{\itshape Procès-verbaux de l’assemblée provinciale de Berry}, I, 77.} » — «… De là, de pauvres bestiaux, de misérables outils et des fumiers mal tenus, même chez ceux qui en pourraient avoir d’autres\footnote{Arthur Young, II, 205.}. » — « Si je gagnais davantage, disait un paysan, ce serait pour le collecteur. » La spoliation annuelle et illimitée « leur ôte jusqu’au désir de l’aisance ». La plupart, pusillanimes, défiants, engourdis, avilis », « peu différents des anciens serfs\footnote{{\itshape Procès-verbaux de l’assemblée provinciale de la généralité de Rouen} (1787), 271.} », ressemblent aux fellahs d’Égypte, aux laboureurs de l’Indoustan. En effet, par l’arbitraire et l’énormité de sa créance, le fisc rend toute possession précaire, toute acquisition vaine, toute épargne dérisoire ; de fait, ils n’ont à eux que ce qu’ils peuvent lui dérober.

\section[{V. Impôts indirects. — Les gabelles et les aides.}]{V. Impôts indirects. — Les gabelles et les aides.}

\noindent En tout pays, le fisc a deux mains, l’une apparente, qui directement fouille dans le coffre des contribuables, l’autre qui se dissimule et emploie la main d’un intermédiaire, pour ne pas se donner l’odieux d’une nouvelle extorsion. Ici, nulle précaution de ce genre ; la seconde griffe est aussi visible que la première ; d’après sa structure et d’après les plaintes, je serais presque tenté de croire qu’elle est plus blessante  D’abord, la gabelle, les aides et les traites sont affermées, vendues chaque année à des adjudicataires qui, par métier, songent à tirer le plus d’argent possible de leur marché. Vis-à-vis du contribuable, ils ne sont pas des administrateurs, mais des exploitants ; ils l’ont acheté. Il est à eux dans les termes de leur contrat ; ils vont lui faire suer non seulement leurs avances et les intérêts de leurs avances, mais encore tout ce qu’ils pourront de bénéfices. Cela suffit pour indiquer de quelle façon les perceptions indirectes sont conduites  En second lieu, par la gabelle et les aides, l’inquisition entre dans chaque ménage. Dans les pays de grande gabelle, Ile-de-France, Maine, Anjou, Touraine, Orléanais, Berry, Bourbonnais, Bourgogne, Champagne, Perche, Normandie, Picardie, le sel coûte treize sous la livre, quatre fois autant et, si l’on tient compte de la valeur de l’argent, huit fois autant qu’aujourd’hui\footnote{Letrosne (1779). {\itshape De l’administration provinciale et de la réforme de l’impôt}, pages 39 à 262, et 138. — {\itshape Archives nationales.} H, 138 (1782). Cahier du Bugey. « Le sel revient à l’habitant des campagnes, qui le prend chez les revendeurs au détail, depuis 15 jusqu’à 17 sous la livre, par la manière dont le mesurage est fait. »}. Bien mieux, en vertu de l’ordonnance de 1680, chaque personne au-dessus de sept ans est tenue d’en acheter sept livres par an ; à quatre personnes par famille, cela fait chaque année plus de dix-huit francs, dix-neuf journées de travail : nouvel impôt direct, qui, comme la taille, met la main du fisc dans la poche des contribuables et les oblige, comme la taille, à se tourmenter mutuellement. En effet, plusieurs d’entre eux sont nommés d’office pour répartir ce sel de devoir, et, comme les collecteurs de la taille, ils sont « solidairement responsables du prix du sel ». Au-dessous d’eux et toujours à l’exemple de la taille, d’autres sont responsables. « Après que les premiers ont été discutés dans leurs personnes et dans leurs biens, le fermier est autorisé à exercer son action en solidarité contre les principaux habitants de la paroisse. » On a décrit tout à l’heure les effets de ce mécanisme. Aussi bien, « en Normandie, dit le Parlement de Rouen\footnote{\href{http://gallica.bnf.fr/ark:/12148/bpt6k945663/f372}{\dotuline{Floquet, VI, 367}} [\url{http://gallica.bnf.fr/ark:/12148/bpt6k945663/f372}] (10 mai 1760).}, chaque jour on voit saisir, vendre, exécuter, pour n’avoir pas acheté de sel, des malheureux qui n’ont pas de pain ».\par
Mais, si la rigueur est aussi grande qu’en matière de taille, les vexations sont dix fois pires ; car elles sont domestiques, minutieuses et de tous les jours  Défense de détourner une once des sept livres obligatoires pour un autre emploi que pour « pot et salière ». Si un villageois a économisé sur le sel de sa soupe pour saler un porc et manger un peu de viande en hiver, gare aux commis ! Le porc est confisqué et l’amende est de 300 livres. Il faut que l’homme vienne au grenier acheter de l’autre sel, fasse déclaration, rapporte un bulletin et représente ce bulletin à toute visite. Tant pis pour lui s’il n’a pas de quoi payer ce sel supplémentaire ; il n’a qu’à vendre sa bête et s’abstenir de viande à Noël ; c’est le cas le plus fréquent, et j’ose dire que, pour les métayers à vingt-cinq francs par an, c’est le cas ordinaire. — Défense d’employer pour pot et salière un autre sel que celui des sept livres. « Je puis citer, dit Letrosne, deux sœurs qui demeuraient à une lieue d’une ville où le grenier n’ouvre que le samedi. Leur provision de sel était finie. Pour passer trois ou quatre jours jusqu’au samedi, elles firent bouillir un reste de saumure, dont elles tirèrent quelques onces de sel. Visite et procès-verbal des commis. À force d’amis et de protection, il ne leur en a coûté que 48 livres. » — Défense de puiser de l’eau de la mer et des sources salées, à peine de 20 et 40 livres d’amende  Défense de mener les bestiaux dans les marais et autres lieux où il y a du sel, ou de les faire boire aux eaux de la mer, à peine de confiscation et de 300 livres d’amende. — Défense de mettre aucun sel dans le ventre des maquereaux au retour de la pêche, ni entre leurs lits superposés. Ordre de n’employer qu’une livre et demie de sel par baril. Ordre de détruire chaque année le sel naturel qui se forme en certains cantons de la Provence. Défense aux juges de modérer ou réduire les amendes prononcées en matière de sel, à peine d’en répondre et d’être interdits  Je passe quantité d’autres ordres et défenses : il y en a par centaines. Cette législation tombe sur les contribuables comme un rets serré aux mille mailles, et le commis qui le lance est intéressé à les trouver en faute. Là-dessus, vous voyez le pêcheur obligé de défaire son baril, la ménagère cherchant le bulletin de son jambon, le « gabelou » inspectant le buffet, vérifiant la saumure, goûtant la salière, déclarant, si le sel est trop bon, qu’il est de contrebande, parce que celui de la ferme, seul légitime, est ordinairement avarié et mêlé de gravats.\par
Cependant d’autres commis, ceux des aides, descendent dans la cave. Il n’y en a pas de plus redoutables\footnote{Boivin-Champeaux, 44. ({\itshape Cahiers} de Bray et de Gamaches.)}, ni qui saisissent plus âprement tous les prétextes de délit. « Que charitablement un citoyen donne une bouteille de boisson à un pauvre languissant, et le voilà exposé à un procès et à des amendes excessives… Un pauvre malade, qui intéressera son curé à lui aumôner une bouteille de vin, essuiera un procès capable de ruiner non seulement le malheureux qui l’a obtenue, mais encore le bienfaiteur qui la lui aura donnée. Ceci n’est pas une histoire chimérique. » En vertu du droit de gros manquant, les commis peuvent, à toute heure, faire l’inventaire du vin, même chez le vigneron propriétaire, lui marquer ce qu’il peut en boire, le taxer pour le reste et pour le trop-bu : car la ferme est l’associée du vigneron et a sa part dans sa récolte. — Dans un vignoble à Epernay\footnote{Arthur Young, II, 175-178.}, sur quatre pièces de vin, produit moyen d’un arpent et valant 600 francs, elle perçoit d’abord 30 francs, puis, quand les quatre pièces sont vendues, 75 autres francs. Naturellement, « les habitants emploient les ruses les plus fines et les mieux combinées pour se soustraire » à des droits si forts. Mais les commis sont alertes, soupçonneux, avertis, et fondent à l’improviste sur toute maison suspecte ; leurs instructions portent qu’ils doivent multiplier leurs visites et avoir des registres assez exacts « pour voir d’un coup d’œil l’état de la cave de chaque habitant\footnote{{\itshape Archives nationales}, G, 300, G, 319. (Mémoires et instructions de divers directeurs locaux des aides à leurs successeurs.)} ». — À présent que le vigneron a payé, c’est le tour du négociant. Celui-ci, pour envoyer les quatre pièces au consommateur, verse encore à la ferme 75 francs. — Le vin part, et la ferme lui prescrit certaines routes ; s’il s’en écarte, il est confisqué, et, à chaque pas du chemin, il faut qu’il paye. « Un bateau de vin du Languedoc\footnote{Letrosne, {\itshape ibid.}, 523.}, Dauphiné ou Roussillon, qui remonte le Rhône et descend la Loire pour aller à Paris par le canal de Briare, paye en route, sans compter les droits du Rhône, de trente-cinq à quarante sortes de droits, non compris les entrées de Paris. » Il les paye « en quinze ou seize endroits, et ces payements multipliés obligent les voituriers à employer douze ou quinze jours de plus par voyage qu’ils n’en mettraient si tous ces droits étaient réunis en un seul bureau ». — Les chemins par eau sont particulièrement chargés. « De Pontarlier à Lyon, il y a vingt-cinq ou trente péages ; de Lyon à Aigues-Mortes, il y en a davantage, de sorte que ce qui coûte 10 sous en Bourgogne, revient à Lyon à 15 et 18 sous, et à Aigues-Mortes à plus de 25 sous. » — Enfin, le vin arrive aux barrières de la ville où il sera bu. Là il paye l’octroi, qui est de 47 francs par muid à Paris. — Il entre et va dans la cave du cabaretier ou de l’aubergiste ; là il paye encore de 30 à 40 francs pour droit de détail ; à Rethel, c’est de 50 à 60 francs pour un poinçon, jauge de Reims. — Le total est exorbitant. À Rennes\footnote{{\itshape Archives nationales}, H, 426. (Remontrances du parlement de Bretagne, février 1783).}, pour une barrique de vin de Bordeaux, les droits des devoirs et le cinquième en sus l’impôt, le billot, les 8 sous pour livre et les deniers d’octrois montent à plus de 72 livres, non compris le prix d’achat ; à quoi il faut ajouter les frais et droits dont le marchand de Rennes fait l’avance et qu’il reprend sur l’acheteur, sortie de Bordeaux, fret, assurance, droit d’écluse, droit d’entrée pour la ville, droits d’entrée pour les hôpitaux, droits de jaugeage, de courtage, d’inspecteurs aux boissons. Total 200 livres au moins à débourser par le cabaretier pour débiter une seule barrique de vin. » On devine si, à ce prix, le peuple de Rennes peut en boire, et toutes ces charges retombent sur le vigneron, puisque, si les consommateurs n’achètent point, il ne vend pas.\par
Aussi bien, parmi les petits cultivateurs, il est le plus digne de pitié ; au témoignage d’Arthur Young, vigneron et misérable sont alors deux termes équivalents. Sa récolte manque souvent, et « toute récolte hasardeuse ruine l’homme qui n’a pas de capital ». En Bourgogne, en Berry, dans le Soissonnais, dans les Trois-Evêchés, en Champagne\footnote{{\itshape Procès-verbaux de l’assemblée provinciale de Soissonnais} (1787), 45. — {\itshape Archives nationales}, H, 1515. (Remontrances du Parlement de Metz, 1768.) « La classe des indigents forme plus des 12/13 de la totalité des villages de labour et le général de ceux de vignobles. » {\itshape Ibidem, G}, 319. (Tableau des directions de Châteauroux et d’Issoudun.)}, je trouve par tous les rapports qu’il manque de pain et qu’il est à l’aumône. En Champagne, les syndics de Bar-sur-Aube écrivent\footnote{Albert Babeau, I, 21, 89.} que plus d’une fois les habitants de La Ferté, pour échapper aux droits, ont jeté leurs vins à la rivière, et l’assemblée provinciale déclare que « dans la majeure partie de la province, la plus légère augmentation des droits ferait déserter les terres à tous les cultivateurs ». — ; Telle est l’histoire du vin sous l’ancien régime. Depuis le vigneron qui produit jusqu’au cabaretier qui débite, que de gens vexés et quelles extorsions   Quant à la gabelle, de l’aveu d’un contrôleur général\footnote{{\itshape Mémoires} présentés à l’Assemblée des Notables par M. de Calonne (1787), 67.}, elle entraîne chaque année 4 000 saisies domiciliaires, 3 400 emprisonnements, 500 condamnations au fouet, au bannissement, aux galères  Si jamais il y eut deux impôts bien combinés, non seulement pour dépouiller, mais encore pour irriter les paysans, les pauvres et le peuple, ce sont ces deux-là.

\section[{VI. Pourquoi l’impôt est si pesant. — Les exemptions et les privilèges.}]{VI. Pourquoi l’impôt est si pesant. — Les exemptions et les privilèges.}

\noindent Il est donc manifeste que la pesanteur de l’impôt est la principale cause de la misère ; de là des haines accumulées et profondes contre le fisc et ses agents, receveurs, officiers des greniers, gens des aides, gens de l’octroi, douaniers et commis  Mais pourquoi l’impôt est-il si pesant ? La réponse n’est pas douteuse, et tant de communes qui plaident chaque année contre messieurs tels ou tels pour les soumettre à la taille l’écrivent tout au long dans leurs requêtes. Ce qui rend la charge accablante, c’est que les plus forts et les plus capables de la porter sont parvenus à s’y soustraire, et la misère a pour première cause l’étendue des exemptions.\par
Suivons-les d’impôt en impôt  En premier lieu, non seulement les nobles et les ecclésiastiques sont exempts de la taille personnelle, mais encore, ainsi qu’on l’a déjà vu, ils sont exempts de la taille d’exploitation pour les domaines qu’ils exploitent eux-mêmes ou par leurs régisseurs. En Auvergne\footnote{Gaultier de \href{http://gallica.bnf.fr/ark:/12148/bpt6k56882t}{\dotuline{Biauzat, {\itshape Doléances}}} [\url{http://gallica.bnf.fr/ark:/12148/bpt6k56882t}], 193, 225  {\itshape Procès-verbaux de l’assemblée provinciale du Poitou} (1787), 99.}, dans la seule élection de Clermont, on compte cinquante paroisses où, grâce à cet arrangement, toutes les terres des privilégiés sont exemptes, en sorte que toute la taille retombe sur les taillables. Bien mieux, il suffit aux privilégiés de prétendre que leur fermier n’est qu’un régisseur : c’est le cas, en Poitou, dans plusieurs paroisses ; le subdélégué et l’élu n’osent y regarder de trop près. De cette façon, le privilégié s’affranchit de la taille, lui et tout son bien, y compris ses fermes  Or, c’est la taille qui, toujours accrue, fournit par ses délégations spéciales à tant de services nouveaux. Il suffit de repasser l’histoire de ses crues périodiques pour montrer à l’homme du Tiers que, seul ou presque seul, il a payé et paye\footnote{Gaultier de Biauzat. {\itshape Ibid.}} pour la construction des ponts, chaussées, canaux et palais de justice, pour le rachat des offices, pour l’établissement et l’entretien des maisons de refuge, des asiles d’aliénés, des pépinières, des postes aux chevaux, des académies d’escrime et d’équitation, pour l’entreprise des boues et pavés de Paris, pour les appointements des lieutenants généraux, gouverneurs et commandants de province, pour les honoraires des baillis, sénéchaux et vice-baillis, pour les traitements des bureaux de finances, des bureaux d’élection et des commissaires envoyés dans les provinces, pour les salaires de la maréchaussée, des chevaliers du guet, et pour je ne sais combien d’autres choses  Dans les pays d’États, où la taille semble devoir être mieux répartie, l’inégalité est pareille. En Bourgogne\footnote{{\itshape Archives nationales}, procès-verbaux et cahiers des États Généraux, t. 59, 6. (Lettre de M. d’Orgeux à M. Necker.) T. 27, 560 à 574 (Cahiers du Tiers-état d’Arnay-le-Duc).}, toutes les dépenses de la maréchaussée, des haras et des fêtes publiques, toutes les sommes affectées aux cours de chimie, botanique, anatomie et accouchements, à l’encouragement des arts, à l’abonnement des droits du sceau, à l’affranchissement des ports de lettres, aux gratifications des chefs et subalternes du commandement, aux appointements des officiers des états, au secrétariat du ministre, aux frais de perception et même aux aumônes, bref 1 800 000 livres dépensées en services publics, sont à la charge du Tiers ; les deux premiers ordres n’en payent pas un sou.\par
En second lieu, pour la capitation, qui, à l’origine, distribuée en vingt-deux classes, devait peser sur tous à proportion de leurs fortunes, on sait que, dès l’abord, le clergé s’en est affranchi moyennant rachat ; et, quant aux nobles, ils ont si bien manœuvré, que leur taxe s’est réduite à mesure que s’augmentait la charge du Tiers. Tel comte ou marquis, intendant ou maître des requêtes, à 40 000 livres de rente, qui, selon le tarif de 1695\footnote{On a tenu compte dans ces chiffres de l’augmentation du titre de la monnaie, le marc d’argent valant 29 francs en 1695, et 49 francs dans la seconde moitié du XVIII\textsuperscript{e} siècle.}, devrait payer de 1 700 à 2 500 livres, n’en paye que 400, et tel bourgeois à 6 000 livres de revenu, qui, selon le même tarif, ne devrait payer que 70 livres, en paye 720. Ainsi, la capitation du privilégié a diminué des trois quarts ou des cinq sixièmes, et celle des taillables a décuplé. Dans l’Ile-de-France\footnote{{\itshape Procès-verbaux de l’assemblée provinciale de l’Ile-de-France}, 132, 158 ; {\itshape de l’Orléanais}, 96, 387.}, sur 240 livres de revenu, elle prend au taillable 21 livres 8 sous, au noble 3 livres, et l’intendant déclare lui-même qu’il ne taxe les nobles qu’au 80\textsuperscript{me} de leur revenu ; celui de l’Orléanais ne les taxe qu’au 100\textsuperscript{e} ; en revanche le taillable est taxé au 11\textsuperscript{e}  Si l’on ajoute aux nobles les autres privilégiés, officiers de justice, employés des fermes, villes abonnées, on forme un groupe qui contient presque tous les gens aisés ou riches, et dont le revenu dépasse certainement de beaucoup celui de tous les simples taillables. Or nous savons, par les budgets des assemblées provinciales, ce que dans chaque province la capitation prend à chacun de deux groupes : dans le Lyonnais, aux taillables 898 000 livres, aux privilégiés 190 000 ; dans l’Ile-de-France, aux taillables 2 689 000 livres, aux privilégiés 232 000 ; dans la généralité d’Alençon, aux taillables 1 067 000 livres, aux privilégiés 122 000 ; dans la Champagne, aux taillables 1 377 000 livres, aux privilégiés 199 000 ; dans la Haute-Guyenne, aux taillables 1 268 000 livres, aux privilégiés 61 000 ; dans la généralité d’Auch, aux taillables 797 000 livres, aux privilégiés 21 000 ; dans l’Auvergne, aux taillables 1 753 000 livres, aux privilégiés 86 000 ; bref, si l’on fait les totaux pour dix provinces, 11 636 000 livres au groupe pauvre, et 1 450 000 livres au groupe riche : celui-ci paye donc huit fois moins qu’il ne devrait.\par
Pour les vingtièmes, la disproportion est moindre, et nous n’avons pas de chiffres précis ; néanmoins on peut admettre que la cote des privilégiés est environ la moitié de ce qu’elle devrait être. « En 1772\footnote{{\itshape Mémoire} présenté à l’Assemblée des Notables (1787), 1  Voir note 2 [’p. 300’] sur le domaine de Blet.}, dit M. de Calonne, il fut reconnu que les vingtièmes n’étaient pas portés à leur valeur. De fausses déclarations, des baux simulés, des traitements trop favorables accordés à presque tous les riches propriétaires, avaient entraîné des inégalités et des erreurs infinies… La vérification de 4 902 paroisses a démontré que le produit des deux vingtièmes, qui est de 54 millions, devrait monter à 81. » Tel domaine seigneurial qui, d’après son revenu avéré, devrait payer 2 400 livres, n’en paye que 1 216. C’est bien pis pour les princes du sang ; on a vu que leurs domaines sont abonnés et ne payent que 188 000 livres, au lieu de 2 400 000. Sous ce régime qui accable les faibles pour alléger les forts, plus on est capable de contribuer, moins on contribue  C’est l’histoire du quatrième et dernier impôt direct, je veux dire de la taxe en remplacement des corvées. Attachée d’abord aux vingtièmes et par suite répartie sur tous les propriétaires, elle vient, par arrêt du Conseil, d’être rattachée à la taille, et, par suite, mise sur les plus chargés\footnote{{\itshape Procès-verbaux de l’assemblée provinciale d’Alsace} (1787), 116 ; {\itshape de Champagne}, 192. (Par la déclaration du 2 juin 1787, la taxe en remplacement de la corvée peut être portée au 1/6 de la taille, des accessoires et de la capitation réunis.) — {\itshape Ib. de la généralité d’Alençon}, 179 ; {\itshape du Berry}, I, 218.}. Or cette taxe est une surcharge d’un quart ajoutée au principal de la taille, et, pour prendre un exemple, en Champagne, sur 100 livres de revenu, elle prend au taillable 6 livres 5 sous. « Ainsi, dit l’assemblée provinciale, les routes dégradées par le poids d’un commerce actif, par les courses multipliées des riches, ne sont réparées qu’avec la contribution des pauvres. » — À mesure que les chiffres défilent sous les yeux, on voit involontairement se dégager les deux figures de la fable, le cheval et le mulet, compagnons de route : le cheval a droit de piaffer à son aise ; c’est pourquoi on le décharge pour charger l’autre, tant qu’enfin la bête de somme s’abat sous le faix.\par
Non seulement, dans le corps des contribuables, les privilégiés sont dégrevés au détriment des taillables, mais encore, dans le corps des taillables, les riches sont soulagés au détriment des pauvres, en sorte que la plus grosse part du fardeau finit par retomber sur la classe la plus indigente et la plus laborieuse, sur le petit propriétaire qui cultive son propre champ, sur le simple artisan qui n’a que ses outils et ses mains, et, en général, sur le villageois  D’abord, en fait d’impôts, nombre de villes sont abonnées ou franches. Pour la taille et les accessoires, Compiègne, avec 1 671 feux, ne paye que 8 000 francs, pendant que tel village aux environs, Canly, avec 148 feux, paye 4 475 francs\footnote{{\itshape Archives nationales}, G, 322 (Mémoire sur les droits d’aides à Compiègne et aux environs, 1786).}. Pour la capitation, Versailles, Saint-Germain, Beauvais, Étampes, Pontoise, Saint-Denis, Compiègne, Fontainebleau, taxés ensemble à 169 000 livres, sont aux deux tiers exempts et ne versent guère que 1 franc au lieu de 3 francs 10 sous par tête d’habitant ; à Versailles, c’est moins encore, puisque, pour 70 000 habitants, sa capitation n’est que de 51 600 francs\footnote{{\itshape Procès-verbaux de l’assemblée provinciale de l’Ile-de-France}, 104.}. En outre, dans tous les cas, lorsqu’il s’agit de répartir une imposition, le bourgeois de la ville se préfère à ses humbles voisins ruraux. Aussi les habitants des campagnes, qui dépendent de la ville et sont compris dans ses rôles, sont traités avec une rigueur dont il serait difficile de se former une idée… Le crédit des villes repousse sans cesse sur eux le fardeau dont elles cherchent à se soulager, et les citoyens les plus riches de la cité payent moins de taille que le colon le plus malheureux\footnote{{\itshape Procès-verbaux de l’assemblée provinciale du Berry}, I, 85 ; II, 81  {\itshape de l’Orléanais}, 225  « L’arbitraire, l’injustice, l’inégalité, sont inséparables de l’impôt de la taille à chaque changement de collecteur. »}. » C’est pourquoi « l’effroi de la taille dépeuple les campagnes, concentre dans les villes tous les talents et tous les capitaux\footnote{{\itshape Archives nationales}, H, 615. Lettre de M. de Langourda, gentilhomme breton, à M. Necker, 4 décembre 1780 : « Vous mettez toujours les impôts sur la classe des hommes utiles et nécessaires, qui diminue tous les jours : ce sont les laboureurs. Les campagnes sont devenues désertes et personne ne veut plus conduire la charrue. J’atteste à Dieu et à vous, Monseigneur, que nous avons perdu plus d’un tiers de nos blés nains à la dernière récolte, parce que nous n’avions pas d’hommes pour travailler. »} ». Même inégalité hors des villes. Chaque année, les élus et leurs collecteurs, munis d’un pouvoir arbitraire, fixent la taille de la paroisse et la taille de chaque habitant. Entre ces mains ignorantes et partiales, ce n’est pas l’équité qui tient la balance, c’est l’intérêt privé, la haine locale, le désir de vengeance, le besoin de ménager un ami, un parent, un voisin, un protecteur, un patron, un homme puissant, un homme dangereux. L’intendant de Moulins, arrivant dans sa généralité, trouve que « les gens en crédit ne payent rien et que les malheureux sont surchargés ». Celui de Dijon écrit que « les bases de la répartition sont arbitraires à un tel degré, qu’on ne doit pas laisser gémir plus longtemps les peuples de la province\footnote{{\itshape Ib.}, 1149 (lettre de M. de Reverseaux, 16 mars 1781) ; H, 200 (lettre de M. Amelot, 2 novembre 1784).} ». Dans la généralité de Rouen, « quelques paroisses payent plus de 4 sous pour livre et quelques-unes à peine 1 sou\footnote{{\itshape Procès-verbaux de l’assemblée provinciale de la généralité de Rouen}, 91.} ». — « Depuis trois ans que j’habite la campagne, écrit une dame du même pays, j’ai remarqué que la plupart des riches propriétaires sont les moins foulés ; ce sont ceux-là qui sont appelés pour la répartition, et le peuple est toujours vexé\footnote{Hippeau, VI, 22 (1788).}. » — « J’habite une terre à dix lieues de Paris, écrivait d’Argenson, où l’on a voulu établir la taille proportionnelle, mais tout n’a été qu’injustice ; les seigneurs ont prévalu pour alléger leurs fermiers\footnote{Marquis d’Argenson, VI, 37.}. » Outre ceux qui, par faveur, font alléger leur taille, il y a ceux qui, moyennant argent, s’en délivrent tout à fait. Un intendant, visitant la subdélégation de Bar-sur-Seine, remarque « que les riches cultivateurs parviennent à se faire pourvoir de petites charges chez le roi et jouissent des privilèges qui y sont attachés, ce qui fait retomber le poids des impositions sur les autres\footnote{{\itshape Archives nationales}, H, 200 (Mémoire de M. Amelot, 1785).}. » — « Une des principales causes de notre surtaxe prodigieuse, dit l’assemblée provinciale d’Auvergne, c’est le nombre inconcevable des privilégiés qui s’accroît chaque jour par le trafic et la location des charges ; il y en a qui, en moins de vingt ans, ont anobli six familles. » Si cet abus continuait, « il finirait par anoblir en un siècle tous les contribuables le plus en état de porter la charge des contributions\footnote{{\itshape Procès-verbaux de l’assemblée provinciale d’Auvergne}, 253.} ». Notez de plus qu’une infinité de places et de fonctions, sans conférer la noblesse, exemptent leur titulaire de la taille personnelle et réduisent sa capitation au quarantième de son revenu : d’abord toutes les fonctions publiques, administratives ou judiciaires, ensuite tous les emplois dans la gabelle, dans les traites, dans les domaines, dans les postes, dans les aides et dans les régies\footnote{Boivin-Champeaux, {\itshape Doléances de la paroisse de Tilleul-Lambert} (Eure). « Une quantité de sortes de privilégiés, MM. des élections, MM. les maîtres de poste, MM. les présidents et autres attachés au grenier au sel, tous particuliers qui possèdent de grands biens, ne payent que le tiers ou la moitié des impôts qu’ils devraient payer. »}. « Il est peu de paroisses, écrit un intendant, où il n’existe de ces employés, et l’on en voit dans plusieurs jusqu’à deux ou trois\footnote{Tocqueville, 385  {\itshape Procès-verbaux de l’assemblée provinciale du Lyonnais}, 56.}. » Un maître de poste est exempt de taille pour tous ses biens et facultés, et même pour ses fermes jusqu’à concurrence de cent arpents. Les notaires d’Angoulême sont affranchis de la corvée, de la collecte, du logement des gens de guerre, et ni leurs fils, ni leurs premiers clercs ne tirent à la milice. Lorsque dans les correspondances administratives on examine de près le grand filet fiscal, on découvre à chaque instant quelques mailles par lesquelles, avec un peu d’industrie ou d’effort, passent tous les poissons moyens ou gros ; le fretin seul reste au fond de la nasse. Un chirurgien non apothicaire, un fils de famille de quarante-cinq ans, commerçant, mais demeurant chez son père et en pays de droit écrit, échappent à la collecte. Même immunité pour les quêteurs des religieux de la Merci et de l’Étroite Observance. Dans tout l’Est et le Midi, les particuliers aisés achètent cette commission de quêteur moyennant un louis ou dix écus, et mettent trois livres dans un bassin qu’ils font promener dans une paroisse quelconque\footnote{{\itshape Archives nationales}, H, 1422 (Lettres de M. d’Aine, intendant, et du receveur de l’Élection de Tulle, 23 février 1783).} : dix habitants dans une petite ville de la montagne, cinq habitants dans le seul village de Treignac ont de cette façon obtenu leur décharge. Par suite, « la collecte retombe sur les pauvres, toujours impuissants, souvent insolvables », et tous ces privilèges, qui font la ruine du contribuable, font le déficit du Trésor.

\section[{VII. Octrois des villes. — La charge retombe partout sur les plus pauvres.}]{VII. Octrois des villes. — La charge retombe partout sur les plus pauvres.}

\noindent Encore un mot pour achever le tableau. C’est dans les villes qu’on se réfugie, et, en effet, comparées aux campagnes, les villes sont un refuge. Mais la misère y suit les pauvres ; car, d’une part, elles sont obérées, et, d’autre part, la coterie qui les administre assoit l’impôt sur les indigents. Opprimées par le fisc, elles oppriment le peuple, et rejettent sur lui la charge que leur impose le roi. Sept fois en quatre-vingts ans\footnote{Tocqueville, 64, 363.}, il leur a repris et revendu le droit de nommer leurs officiers municipaux, et, pour payer « cette finance énorme », elles ont doublé leurs octrois. À présent, quoique libérées, elles payent encore ; la charge annuelle est devenue perpétuelle ; jamais le fisc ne lâche prise ; ayant sucé une fois, il suce toujours. « C’est pourquoi, en Bretagne, dit un intendant\footnote{{\itshape Archives nationales}, H, 612, 614 (Lettres de M. Caze de la Bove, 11 septembre et 2 décembre 1774, 28 juin 1777).}, il n’y a aucune ville dont la dépense ne dépasse les revenus. » Elles ne peuvent raccommoder leur pavé, elles ne peuvent réparer leurs chemins, « leurs approches sont presque impraticables ». Comment feraient-elles pour s’entretenir, obligées, comme elles le sont, à payer après avoir payé déjà ? Leurs octrois accrus en 1748 devaient fournir en onze ans les 606 000 livres convenues ; mais, les onze ans écoulés, le fisc soldé a maintenu ses exigences, si bien qu’en 1774 elles ont déjà versé 2 071 052 livres et que l’octroi provisoire dure toujours  Or cet octroi exorbitant pèse partout sur les choses les plus indispensables à la vie, et de cette façon l’artisan est plus chargé que le bourgeois. À Paris, ainsi qu’on l’a vu, le vin paye par muid 47 livres d’entrée ; au taux où est l’argent, c’est le double d’aujourd’hui. « Un turbot, sorti de la côte de Honfleur et arrivé en poste, paye d’entrée onze fois sa valeur ; partant, le peuple de la capitale est condamné à ne pas manger de poisson de mer\footnote{Mercier, II, 62.}. » Aux portes de Paris, dans la mince paroisse d’Aubervilliers, je trouve « des droits excessifs sur le foin, la paille, les grains, le suif, la chandelle, les œufs, le sucre, le poisson, les fagots, le bois de chauffage\footnote{{\itshape Doléances} de la paroisse d’Aubervilliers.} ». Compiègne paye toute sa taille au moyen d’un impôt sur les boissons et sur les bestiaux\footnote{{\itshape Archives nationales}, G, 300, G, 322 (Mémoires sur les droits d’aides).}. « Dans Toul et Verdun, les charges sont si pesantes, qu’il n’y a guère que ceux qui y sont retenus par leurs offices et par d’anciennes habitudes, qui consentent à y rester\footnote{{\itshape Procès-verbaux de l’assemblée provinciale des Trois-Évêchés}, 442.}. » À Coulommiers, « le marchand et le peuple sont si surchargés, qu’ils répugnent à faire des entreprises ». Partout, contre les octrois, les barrières et les commis, la haine populaire est profonde. Partout l’oligarchie bourgeoise songe à elle-même avant de songer à ses administrés. À Nevers et à Moulins\footnote{{\itshape Archives nationales}, H, 1422 (Lettre de l’intendant de Moulins, avril 1779).}, « tous les gens riches trouvent moyen de se soustraire à la collecte par différentes commissions ou par le crédit qu’ils ont auprès des élus, de sorte qu’on prendrait pour de vrais mendiants les collecteurs de Nevers de cette année et de l’année précédente ; il n’y a point de petits villages dont les collecteurs ne soient plus solvables, puisqu’on y prend des métayers ». À Angers, indépendamment des jetons et de la bougie qui consomment le fonds annuel de 2 127 livres, les deniers publics se dissipent et s’emploient au gré des officiers municipaux en dépenses clandestines ». En Provence, où les communautés s’imposent librement et devraient, ce semble, ménager le pauvre, « la plupart des villes, notamment Aix, Marseille et Toulon\footnote{{\itshape Archives nationales}, H, 1312 (Lettres de M. d’Antheman, avocat général à la Cour des Comptes d’Aix (19 mai 1783) et de l’archevêque d’Aix (15 juin 1783). — La Provence ne produisait de blé que pour sa consommation pendant sept mois et demi.}, ne payent leurs impositions » locales et générales « que par le droit de piquet ». C’est une taxe « sur toutes les farines qui sont et se consomment sur leur terroir » ; par exemple, sur 254 897 livres que dépense Toulon, le piquet en fournit 233 405. Ainsi, tout l’impôt porte sur le peuple, et l’évêque, le marquis, le président, le gros négociant payent moins pour leur dîner de poisson fin et de becfigues que le calfat ou le porte-faix pour ses deux livres de pain frotté d’ail ! Et le pain dans ce pays stérile est déjà trop cher ! Et il est si mauvais, que Malouet, l’intendant de la marine, le refuse pour ses employés   « Sire, disait en chaire M. de la Fare, évêque de Nancy, le 4 mai 1789, sire, le peuple sur lequel vous régnez a donné des preuves non équivoques de sa patience… C’est un peuple martyr, à qui la vie semble n’avoir été laissée que pour le faire souffrir plus longtemps. »

\section[{VIII. Plaintes des cahiers.}]{VIII. Plaintes des cahiers.}

\noindent « Je suis misérable, parce qu’on me prend trop. On me prend trop, parce qu’on ne prend pas assez aux privilégiés. Non seulement les privilégiés me font payer à leur place, mais encore ils prélèvent sur moi leurs droits ecclésiastiques et féodaux. Quand, sur mon revenu de 100 francs, j’ai donné 53 francs et au-delà au collecteur, il faut encore que j’en donne plus de 14 au seigneur et plus de 14 pour la dîme\footnote{On peut évaluer les droits féodaux au septième du revenu net et la dîme aussi au septième. C’est le chiffre que donne l’Assemblée provinciale de la Haute-Guyenne ({\itshape Procès-verbaux}, 47). — Dans les autres provinces, nombre d’exemples isolés indiquent un chiffre à peu près semblable. — La dîme flotte du dixième au trentième du produit brut, et ordinairement se rapproche plus du dixième que du trentième. La moyenne est, à mon avis, du quatorzième, et, comme il faut défalquer moitié du produit brut pour les frais de culture, elle est du septième. Letrosne dit le cinquième et même le quart.}, et, sur les 18 ou 19 francs qui me restent, je dois en outre satisfaire le rat de cave et le gabelou. À moi seul, pauvre homme, je paye deux gouvernements : l’un ancien, local, qui aujourd’hui est absent, inutile, incommode, humiliant, et n’agit plus que par ses gênes, ses passe-droits et ses taxes ; l’autre, récent, central, partout présent, qui, se chargeant seul de tous les services, a des besoins immenses et retombe sur mes maigres épaules de tout son énorme poids. » — Telles sont, en paroles précises, les idées vagues qui commencent à fermenter dans les têtes populaires, et on les retrouve à chaque page dans les cahiers des États généraux.\par
« Fasse le ciel, dit un village de Normandie\footnote{Boivin-Champeaux, 72.}, que le monarque prenne entre ses mains la défense du misérable citoyen lapidé et tyrannisé par les commis, les seigneurs, la justice et le clergé. » — « Sire, écrit un village de Champagne\footnote{{\itshape Doléances} de la communauté de Culmon (Élection de Langres).}, tout ce qu’on nous envoyait de votre part c’était toujours pour avoir de l’argent. On nous faisait bien espérer que cela finirait, mais tous les ans cela devenait plus fort. Nous ne nous en prenions pas à vous, tant nous vous aimions, mais à ceux que vous employez et qui savent mieux faire leurs affaires que les vôtres. Nous croyions qu’ils vous trompaient, et nous nous disions dans notre chagrin : Si notre bon roi le savait !… Nous sommes accablés d’impôts de toute sorte ; nous vous avons donné jusqu’à présent une partie de notre pain, et il va bientôt nous manquer si cela continue… Si vous voyiez les pauvres chaumières que nous habitons, la pauvre nourriture que nous prenons, vous en seriez touché ; cela vous dirait mieux que nos paroles que nous n’en pouvons plus et qu’il faut nous diminuer… Ce qui nous fait bien de la peine, c’est que ceux qui ont le plus de bien payent le moins. Nous payons les tailles et tout plein d’ustensiles, et les ecclésiastiques et nobles, qui ont les plus beaux biens, ne payent rien de tout cela. Pourquoi donc est-ce que ce sont les riches qui payent le moins et les pauvres qui payent le plus ? Est-ce que chacun ne doit pas payer selon son pouvoir ? Sire, nous vous demandons que cela soit ainsi, parce que cela est juste… Si nous osions, nous entreprendrions de planter quelques vignes sur les coteaux ; mais nous sommes si tourmentés par les commis aux aides, que nous penserions plutôt à arracher celles qui sont plantées ; tout le vin que nous ferions serait pour eux, et il ne nous resterait que la peine. C’est un grand fléau que toute cette maltôte-là, et, pour s’en sauver, on aime mieux laisser les terres en friche… Débarrassez-nous d’abord des maltôtiers et des gabelous ; nous souffrons beaucoup de toutes ces inventions-là ; voici le moment de les changer ; tant que nous les aurons, nous ne serons jamais heureux. Nous vous le demandons, sire, avec tous vos autres sujets, qui sont aussi las que nous… Nous vous demanderions encore bien d’autres choses, mais vous ne pouvez pas tout faire à la fois. » — Les impôts et les privilèges, voilà, dans les cahiers vraiment populaires, les deux ennemis contre lesquels les plaintes ne tarissent pas\footnote{Boivin-Champeaux, 34, 36, 41, 48  Paris ({\itshape Doléances des paroisses rurales de l’Artois}, 301, 308)  {\itshape Archives nationales}, procès-verbaux et cahiers des États généraux, t. XVII, 12 (Lettre des habitants de Dracy-le-Vitreux).}. « Nous sommes écrasés par les demandes de subsides…, nos impositions sont au-delà de nos forces… Nous ne nous sentons pas la force d’en supporter davantage…, nous périssons terrassés par les sacrifices qu’on exige de nous… Le travail est assujetti à un taux et la vie oisive en est exempte… Le plus désastreux des abus est la féodalité, et les maux qu’elle cause surpassent de beaucoup la foudre et la grêle… Impossible de subsister, si l’on continue à enlever les trois quarts des moissons par champart, terrage, etc. Le propriétaire a la quatrième partie, le décimateur en prend la douzième, l’impôt la dixième, sans compter les dégâts d’un gibier innombrable qui dévore la campagne en verdure : il ne reste donc au malheureux cultivateur que la peine et la douleur. » — Pourquoi le Tiers paye-t-il seul pour les routes sur lesquelles la noblesse et le clergé roulent en carrosse ? Pourquoi les pauvres gens sont-ils seuls astreints à la milice ? Pourquoi « le subdélégué ne fait-il tirer que les indéfendus et ceux qui n’ont pas de protections » ? Pourquoi suffit-il d’être le domestique d’un privilégié pour échapper au service   Détruisez ces colombiers qui n’étaient autrefois que des volières et qui maintenant renferment parfois jusqu’à 5 000 paires de pigeons. Abolissez les droits barbares de « motte, quevaise et domaine congéable, sous lesquels plus de cinq cent mille individus gémissent encore en Basse-Bretagne »  « Vous avez dans vos armées, sire, plus de trente mille serfs franc-comtois » ; si l’un d’eux devient officier et quitte le service avec une pension, il faut qu’il aille vivre dans la hutte où il est né ; sinon, lorsqu’il mourra, le seigneur prendra son pécule. Plus de prélats absents, ni d’abbés commendataires. « Ce n’est point à nous à payer le déficit actuel, c’est aux évêques, aux bénéficiers ; retranchez aux princes de l’Église les deux tiers de leurs revenus. » — « Que la féodalité soit abolie. L’homme, le paysan surtout, est tyranniquement asservi sur la terre malheureuse où il languit desséché… Il n’y a point de liberté, de prospérité, de bonheur, là où les terres sont serves… Abolissons les lods et ventes, maltôte bursale et non féodale, taxe mille fois remboursée aux privilégiés. Qu’il suffise à la féodalité de son sceptre de fer, sans qu’elle y joigne encore le poignard du traitant\footnote{Prudhomme, {\itshape Résumé des Cahiers.} III, {\itshape passim}, et notamment de 317 à 340.}. » — Ici, et déjà depuis quelque temps, ce n’est plus le villageois qui parle ; c’est le procureur, l’avocat qui lui prête ses métaphores et ses théories. Mais l’avocat n’a fait que traduire en langage littéraire les sentiments du villageois.
\chapterclose


\chapteropen

\chapter[{Chapitre III}]{Chapitre III}


\chaptercont

\section[{I. État des cerveaux populaires  Incapacité mentale  Comment les idées se transforment en légendes.}]{I. État des cerveaux populaires  Incapacité mentale  Comment les idées se transforment en légendes.}

\noindent À présent, pour comprendre leurs actions, il faudrait voir l’état de leur esprit, le train courant de leurs idées, la façon dont ils pensent. Mais, en vérité, est-il besoin de faire leur portrait, et ne suffit-il pas des détails qu’on vient de donner sur leur condition ? On les connaîtra plus tard et par leurs actions elles-mêmes, quand, en Touraine, ils assommeront à coup de sabots le maire et l’adjoint de leur choix, parce que, pour obéir à l’Assemblée nationale, ces deux pauvres gens ont dressé le tableau des impositions, ou quand, à Troyes, ils traîneront et déchireront dans les rues le magistrat vénérable qui les nourrit en ce moment même et qui vient de dresser son testament en leur faveur  Prenez le cerveau encore si brut d’un de nos paysans contemporains, et retranchez-en toutes les idées qui, depuis quatre-vingts ans, y entrent par tant de voies, par l’école primaire instituée dans chaque village, par le retour des conscrits après sept ans de service, par la multiplication prodigieuse des livres, des journaux, des routes, des chemins de fer, des voyages et des communications de toute espèce\footnote{Théron de Montaugé, 102, 113. Dans le Toulousain, sur cinquante paroisses, dix ont des écoles  Dans la Gascogne, dit l’Assemblée provinciale d’Auch (24), « la plupart des campagnes sont sans maîtres d’école ni presbytères »  En 1778, le courrier de Paris n’arrive à Toulouse que trois fois par semaine ; celui de Toulouse pour Alby, Rodez, etc., deux fois par semaine, pour Beaumont, Saint-Girons, etc., une fois. « À la campagne, dit Théron de Montaugé, on vit pour ainsi dire dans la solitude et dans l’exil. » En 1789, le courrier de Paris n’arrive à Besançon que trois fois par semaine (Arthur Young, I, 257).}. Tâchez de vous figurer le paysan d’alors, clos et parqué de père en fils dans son hameau, sans chemins vicinaux, sans nouvelles, sans autre enseignement que le prône du dimanche, tout entier au souci du pain quotidien et de l’impôt, « avec son aspect misérable et desséché\footnote{Mot du marquis de Mirabeau.} », n’osant réparer sa maison, toujours tourmenté, défiant, l’esprit rétréci et, pour ainsi dire, racorni par la misère. Sa condition est presque celle de son bœuf ou de son âne, et il a les idées de sa condition. Pendant longtemps il est resté engourdi ; il manque même d’instinct\footnote{{\itshape Archives nationales}, G, 300, lettre d’un directeur des aides à Coulommiers (13 août 1781).} ; machinalement et sans lever les yeux, il tire sa charrue héréditaire. En 1751, d’Argenson écrivait sur son journal : « Rien ne les pique aujourd’hui des nouvelles de la cour ; ils ignorent le règne… La distance devient chaque jour plus grande de la capitale à la province… On ignore ici les événements les plus marqués qui nous ont le plus frappés à Paris… Les habitants de la campagne ne sont plus que de pauvres esclaves, des bêtes de trait attachées à un joug, qui marchent comme on les fouette, qui ne se soucient et ne s’embarrassent de rien, pourvu qu’ils mangent et dorment à leurs heures\footnote{Marquis d’Argenson, VI, 425 (16 juin 1751).}. » Ils ne se plaignent pas, « ils ne songent pas même à se plaindre\footnote{Comte de Montlosier, I, 102, 146.} » ; leurs maux leur semblent une chose de nature, comme l’hiver ou la grêle. Leur pensée, comme leur agriculture, est encore du moyen âge. — En Toulousain\footnote{Théron de Montaugé, 102.}, pour découvrir l’auteur d’un vol, pour guérir un homme ou une bête malade, on a recours au sorcier, qui devine au moyen d’un crible. Le campagnard croit de tout son cœur aux revenants, et, la nuit de la Toussaint, il met le couvert pour les morts. — En Auvergne, au commencement de la Révolution, une fièvre contagieuse s’étant déclarée, il est clair que M. de Montlosier, sorcier avéré, en est la cause, et deux cents hommes se mettent en marche pour démolir sa maison. Aussi bien leur religion est de niveau : « Leurs prêtres boivent avec eux et leur vendent l’absolution. Tous les dimanches, aux prônes, il se crie des lieutenances et des sous-lieutenances (de saints) : à tant la lieutenance de saint Pierre   Si le paysan tarde à mettre le prix, vite un éloge de saint Pierre, et mes paysans de monter à l’envi\footnote{{\itshape Tableaux de la Révolution}, par Schmidt, II, 7 (Rapport de l’agent Perrière, qui a habité l’Auvergne).}. » — À ces cerveaux tout primitifs, vides d’idées et peuplés d’images, il faut des idoles sur la terre comme dans le ciel. « Je ne doutais nullement, dit Rétif de la Bretonne\footnote{{\itshape Monsieur Nicolas}, I, 448.}, que le roi ne pût légalement obliger tout homme à me donner sa femme ou sa fille, et tout mon village (Sacy en Bourgogne) pensait comme moi. » Il n’y a pas de place en de pareilles têtes pour les conceptions abstraites, pour la notion de l’ordre social ; ils le subissent, rien de plus. « La grosse masse du peuple, écrit Gouverneur Morris en 1789\footnote{Gouverneur Morris, II, 69 (29 avril 1789).}, n’a pour religion que ses prêtres, pour loi que ses supérieurs, pour morale que son intérêt ; voilà les créatures qui, menées par des curés ivres, sont maintenant sur le grand chemin de la liberté ; et le premier usage qu’elles en font, c’est de s’insurger de toutes parts parce qu’il y a disette. »\par
Comment pourrait-il en être autrement ? Avant de prendre racine dans leur cervelle, toute idée doit devenir une légende, aussi absurde que simple, appropriée à leur expérience, à leurs facultés, à leurs craintes, à leurs espérances. Une fois plantée dans cette terre inculte et féconde, elle y végète, elle s’y transforme, elle se développe en excroissances sauvages, en feuillages sombres, en fruits vénéneux. Plus elle est monstrueuse, plus elle est vivace, accrochée aux plus frêles vraisemblances et tenace contre les plus fortes démonstrations  Sous Louis XV, pendant l’arrestation des vagabonds, quelques enfants ayant été enlevés par abus ou par erreur, le bruit court que le roi prend des bains de sang pour réparer ses organes usés, et la chose paraît si évidente, que les femmes, révoltées par l’instinct maternel, se joignent à l’émeute : un exempt est saisi, assommé, et, comme il demandait un confesseur, une femme du peuple prend un pavé, crie qu’il ne faut pas lui donner le temps d’aller en paradis, et lui casse la tête, persuadée qu’elle fait justice\footnote{Mercier, {\itshape Tableau de Paris}, XII, 83.}  Sous Louis XVI, il est avéré pour le peuple que la disette est factice : en 1789\footnote{Vaublanc, 209.}, un officier, écoutant les discours de ses soldats, les entend répéter « avec une profonde conviction que les princes et les courtisans, pour affamer Paris, font jeter les farines dans la Seine ». Là-dessus, se tournant vers le maréchal-des-logis, il lui demande comment il peut croire à une pareille sottise. « C’est bien vrai, mon lieutenant, répond l’autre ; la preuve, c’est que les sacs de farine étaient attachés avec des {\itshape cordons bleus.} » L’argument leur semblait décisif ; rien ne put les en faire démordre  Il se forge ainsi dans les bas-fonds de la société, à propos du pacte de famine, de la Bastille, des dépenses et des plaisirs de la cour, un roman immonde et horrible, où Louis XVI, la reine Marie-Antoinette, le comte d’Artois, Mme de Lamballe, les Polignac, les traitants, les seigneurs, les grandes dames, sont des vampires et des goules. J’en ai vu plusieurs rédactions dans les pamphlets du temps, dans les gravures secrètes, dans les estampes et dans les enluminures populaires, celles-ci les plus efficaces de toutes, car elles parlent aux yeux. Cela dépasse l’histoire de Mandrin ou de Cartouche, et cela convient justement à des hommes qui pour littérature ont la complainte de Cartouche et de Mandrin.

\section[{II. Incapacité politique  Comment les nouvelles politiques et les actes du gouvernement sont interprétés.}]{II. Incapacité politique  Comment les nouvelles politiques et les actes du gouvernement sont interprétés.}

\noindent Jugez par là de leur intelligence politique. Tous les objets leur apparaissent sous un jour faux ; on dirait des enfants qui, à chaque tournant du chemin, voient dans un arbre, dans un buisson, un spectre épouvantable. Arthur Young, visitant des sources près de Clermont, est arrêté\footnote{Arthur Young, I, 283 (13 août 1789), I, 289 (19 août 1789).} et l’on veut mettre en prison la femme qui lui a servi de guide ; plusieurs sont d’avis qu’il a été « chargé par la reine de faire miner la ville pour la faire sauter, puis d’envoyer aux galères tous les habitants qui en réchapperont ». Six jours plus tard, au-delà du Puy, et malgré son passe-port, la garde bourgeoise vient à onze heures du soir le saisir au lit ; on lui déclare « qu’il est sûrement de la conspiration tramée par la reine, le comte d’Artois et le comte d’Entragues, grand propriétaire du pays ; qu’ils l’ont envoyé comme arpenteur pour mesurer les champs, afin de doubler les taxes »  Ici nous saisissons sur le fait le travail involontaire et redoutable de l’imagination populaire : sur un indice, sur un mot, elle construit en l’air ses châteaux ou ses cachots fantastiques, et sa vision lui semble aussi solide que la réalité. Ils n’ont pas l’instrument intérieur qui divise et discerne ; ils pensent {\itshape par blocs ;} le fait et le rêve leur apparaissent ensemble et conjoints en un seul corps  Au moment où l’on élit les députés, le bruit court en Provence\footnote{{\itshape Archives nationales}, H, 274. Lettres de M. de Caraman (18 mars et 12 avril 1789), de M. d’Eymar de Montmeyran (2 avril), de M. de la Tour (30 mars). « Le plus grand bienfait du souverain a été interprété de la manière la plus bizarre par une populace ignorante. »} « que le meilleur des rois veut que tout soit égal, qu’il n’y ait plus ni évêques, ni seigneurs, ni dîmes, ni droits seigneuriaux, qu’il n’y ait plus de titres ni de distinctions, plus de droits de chasse ni de pêche ; … que le peuple va être déchargé de tout impôt, que les deux premiers ordres supporteront seuls les charges de l’État ». Là-dessus quarante ou cinquante émeutes éclatent presque le même jour. « Plusieurs communautés refusent à leur trésorier de rien payer au-delà des impositions royales. » D’autres font mieux : « lorsqu’on pillait la caisse du receveur du droit sur les cuirs à Brignolles, c’était avec les cris de : Vive le roi ! » — « Le paysan annonce sans cesse que le pillage et la destruction qu’il fait sont conformes à la volonté du roi. » — Un peu plus tard, en Auvergne, les paysans qui brûlent les châteaux montreront « beaucoup de répugnance » à maltraiter ainsi « d’aussi bons seigneurs » ; mais ils allégueront que « l’ordre est impératif, ils ont des avis que « Sa Majesté le veut ainsi\footnote{Doniol, {\itshape Histoire des classes rurales}, 495 (Lettre du 3 août 1789 à M. de Clermont-Tonnerre).} »  À Lyon, quand les cabaretiers de la ville et les paysans des environs passent sur le corps des douaniers, ils sont bien convaincus que le roi a pour trois jours suspendu les droits d’entrée\footnote{{\itshape Archives nationales}, H, 1453 (Lettre d’Imbert-Colomès, prévôt des marchands, du 5 juillet 1789).}  Autant leur imagination est grande, autant leur vue est courte. « Du pain, plus de redevances, ni de taxes », c’est le cri unique, le cri du besoin, et le besoin exaspéré fonce en avant comme un animal affolé. À bas l’accapareur ! Et les magasins sont forcés, les convois de grains arrêtés, les marchés pillés, les boulangers pendus, le pain taxé, en sorte qu’il n’arrive plus ou se cache. À bas l’octroi ! Et les barrières sont brisées, les commis assommés, l’argent manque aux villes pour les dépenses les plus urgentes. Au feu les registres d’impôt, les livres de comptes, les archives des municipalités, les chartriers des seigneurs, les parchemins des couvents, toutes ces écritures maudites qui font partout des débiteurs et des opprimés ! Et le village lui-même ne sait plus comment revendiquer ses communaux  Contre le papier griffonné, contre les agents publics, contre l’homme qui de près ou de loin touche au blé, l’acharnement est aveugle et sourd. La brute lâchée écrase tout en se blessant elle-même, et s’aheurte en mugissant contre l’obstacle qu’il fallait tourner.

\section[{III. Impulsions destructives  À quoi s’acharne la colère aveugle  Méfiance contre les chefs naturels  De suspects ils deviennent haïs  Dispositions du peuple en 1789.}]{III. Impulsions destructives  À quoi s’acharne la colère aveugle  Méfiance contre les chefs naturels  De suspects ils deviennent haïs  Dispositions du peuple en 1789.}

\noindent C’est que les conducteurs lui manquent, et que, faute d’organisation, une multitude n’est qu’un troupeau. Contre tous ses chefs naturels, contre les grands, les riches, les gens en place et revêtus d’autorité, sa défiance est invétérée et incurable. Ils ont beau lui vouloir du bien et lui en faire, elle refuse de croire à leur humanité et à leur désintéressement. Elle a été trop foulée ; elle a des préventions contre toutes les mesures qui viennent d’eux, même les plus salutaires, même les plus libérales. « Au seul nom des nouvelles assemblées, dit une commission provinciale en 1787\footnote{{\itshape Procès-verbaux de l’Assemblée provinciale de l’Orléanais}, 296. « Une défiance toujours tremblante règne encore dans les campagnes… Vos premiers ordres d’assemblées de département n’ont en quelques endroits réveillé que des soupçons. »}, nous avons entendu un pauvre laboureur s’écrier : Hé quoi ! Encore de nouvelles orangeries ! » — Tous leurs supérieurs leur sont suspects, et du soupçon à l’hostilité il n’y a pas loin. En 1788\footnote{{\itshape Tableau de Paris}, XII, 186.}, Mercier déclare que, « depuis quelques années, l’insubordination est visible dans le peuple, et surtout dans les métiers… Jadis, lorsque j’entrais dans une imprimerie, les garçons ôtaient leurs chapeaux. Aujourd’hui ils se contentent de vous regarder et ricanent : à peine êtes-vous sur le seuil, que vous les entendez parler de vous d’une manière plus leste que si vous étiez leur camarade »  Aux environs de Paris, même attitude chez les paysans, et Mme Vigée-Lebrun\footnote{Mme Vigée-Lebrun, I, 158 (1788), I, 183 (1789),} allant à Romainville chez le maréchal de Ségur, en fait la remarque : « Non seulement ils ne nous ôtaient plus leurs chapeaux, mais ils nous regardaient avec insolence ; quelques-uns même nous menaçaient avec leurs gros bâtons. » — Au mois de mars ou d’avril suivant, à un concert qu’elle donne, ses invités arrivent consternés. « Le matin, à la promenade de Longchamps, la populace, rassemblée à la barrière de l’Étoile, a insulté de la façon la plus effrayante les gens qui passaient en voiture ; des misérables montaient sur les marchepieds en criant : L’année prochaine, vous serez derrière vos carrosses et nous serons dedans. » — À la fin de 1788, le fleuve est devenu torrent, et le torrent devient cataracte. Un intendant\footnote{{\itshape Archives nationales}, H, 723 (Lettre de M. de Caumartin, intendant de Besançon, 5 décembre 1788).} écrit que, dans sa province, le gouvernement doit opter, et opter dans le sens populaire, se détacher des privilégiés, abandonner les vieilles formes, donner au Tiers double vote. Clergé et noblesse sont détestés, leur suprématie semble un joug. « Au mois de juillet dernier, dit-il, on eût reçu les (anciens) États avec transport, et leur formation n’eût trouvé que peu d’obstacles. Depuis cinq mois, les esprits se sont éclairés, les intérêts respectifs ont été discutés, les ligues se sont formées. On vous a laissé ignorer que, dans toutes les classes du Tiers-état, la fermentation est au comble, qu’une étincelle suffit pour allumer l’incendie… Si la décision du roi est favorable aux deux premiers ordres, insurrection générale dans toutes les parties de la province, 600 000 hommes en armes et toutes les horreurs de la Jacquerie. » — Le mot est prononcé et l’on aura la chose. Quand une multitude soulevée repousse ses conducteurs naturels, il faut qu’elle en prenne ou subisse d’autres. De même une armée qui, entrant en campagne, casserait tous ses officiers ; les nouveaux grades sont pour les plus hardis, les plus violents, les plus opprimés, pour ceux qui, ayant le plus souffert du régime antérieur, crient « en avant », marchent en tête et font les premières bandes. En 1789, les bandes sont prêtes ; car, sous le peuple qui pâtit, il est un autre peuple qui pâtit encore davantage, dont l’insurrection est permanente, et qui, réprimé, poursuivi, obscur, n’attend qu’une occasion pour sortir de ses cachettes et se déchaîner au grand jour.

\section[{IV. Recrues et chefs d’émeute  Braconniers  Contrebandiers et faux-sauniers  Bandits  Mendiants et vagabonds  Apparition des brigands  Le peuple de Paris.}]{IV. Recrues et chefs d’émeute  Braconniers  Contrebandiers et faux-sauniers  Bandits  Mendiants et vagabonds  Apparition des brigands  Le peuple de Paris.}

\noindent Gens sans aveu, réfractaires de tout genre, gibier de justice ou de police, besaciers, porte-bâtons, rogneux, teigneux, hâves et farouches, ils sont engendrés par les abus du système, et, sur chaque plaie sociale, ils pullulent comme une vermine  Quatre cents lieues de capitaineries gardées et la sécurité du gibier innombrable qui broute les récoltes sous les yeux du propriétaire, provoquent au braconnage des milliers d’hommes d’autant plus dangereux qu’ils bravent des lois terribles et sont armés. Déjà en 1752\footnote{Marquis d’Argenson, 13 mars 1752.}, autour de Paris, on en voit « des rassemblements de cinquante à soixante, tous armés en guerre, se comportant comme à un fourrage bien ordonné, infanterie au centre et cavalerie aux ailes… Ils habitent les forêts, ils y ont fait une enceinte retranchée et gardée, et payent exactement ce qu’ils prennent pour vivre ». En 1777\footnote{{\itshape Correspondance}, par Metra, V, 179 (22 novembre 1777).}, près de Sens en Bourgogne, le procureur général M. Terray, chassant sur sa terre avec deux officiers, rencontre sept braconniers qui tirent sur le gibier à leurs yeux et bientôt tirent sur eux-mêmes : M. Terray est blessé, l’un des officiers a son habit percé. Arrive la maréchaussée, les braconniers font ferme et la repoussent. On fait venir les dragons de Provins, les braconniers en tuent un, abattent trois chevaux, sont sabrés ; quatre d’entre eux restent sur la place et sept sont pris. — On voit par les cahiers des États Généraux que, chaque année, dans chaque grande forêt, tantôt par le fusil d’un braconnier, tantôt et bien plus souvent par le fusil d’un garde, il y a des meurtres d’hommes  C’est la guerre à demeure et à domicile ; tout vaste domaine recèle ainsi ses révoltés qui ont de la poudre, des balles et qui savent s’en servir.\par
Autre recrue d’émeute, les contrebandiers et les faux sauniers\footnote{Beugnot, I, 142. « Pas un seul des habitants de la baronnie de Choiseul ne se mêla à ces bandes, composées des patriotes de Montigny, de contrebandiers ou de mauvais sujets des environs. » — V. sur les braconniers du temps, {\itshape Les deux amis de Bourbonne}, par Diderot.}. Dès qu’une taxe est exorbitante, elle invite à la fraude, et suscite un peuple de délinquants contre son peuple de commis. Jugez ici du nombre des fraudeurs par le nombre des surveillants : douze cents lieues de douanes intérieures sont gardées par 50 000 hommes, dont 23 000 soldats sans uniforme\footnote{Calonne, {\itshape Mémoires présentés à l’Assemblée des Notables}, n\textsuperscript{o} 8. — Necker, \href{http://gallica.bnf.fr/ark:/12148/bpt6k429493}{\dotuline{{\itshape De l’Administra} {\itshape tion des Finances}}} [\url{http://gallica.bnf.fr/ark:/12148/bpt6k429493}], I, 195.}. « Dans les pays de grande gabelle et dans les provinces des cinq grosses fermes, à quatre lieues de part et d’autre de long de la ligne de défense », la culture est abandonnée ; tout le monde est douanier ou fraudeur\footnote{Letrosne, {\itshape De l’Administration des Finances}, 59.}. Plus l’impôt est excessif, plus la prime offerte aux violateurs de la loi devient haute, et, sur tous les confins par lesquels la Bretagne touche à la Normandie, au Maine et à l’Anjou, quatre sous pour livre ajoutés à la gabelle multiplient au-delà de toute croyance le nombre déjà énorme des faux sauniers. « Des bandes nombreuses\footnote{{\itshape Archives nationales} H, 426 (Mémoires des fermiers généraux, 13 janvier 1781, 15 septembre 1782). H, 614 (Lettre de M. de Coëtlosquet, du 25 avril 1777). H, 1431, Rapport par les fermiers généraux, du 9 mars 1787.} d’hommes, armés de {\itshape frettes} ou longs bâtons ferrés et quelquefois de pistolets ou de fusils, tentent par force de s’ouvrir un passage. Une multitude de femmes et d’enfants de l’âge le plus tendre franchissent les lignes des brigades, et, d’un autre côté, des troupeaux de chiens conduits dans le pays libre, après y avoir été enfermés quelque temps sans aucune nourriture, sont chargés de sel, que, pressés par la faim, ils rapportent promptement chez leurs maîtres. » — Vers ce métier si lucratif, les vagabonds, les désespérés, les affamés accourent de loin comme une meute. « Toute la lisière de Bretagne n’est peuplée que d’émigrants, la plupart proscrits de leur patrie, et qui, après un an de domicile, jouissent de tous les privilèges bretons : leur unique occupation se borne à faire des amas de sel pour les revendre aux faux sauniers. » On aperçoit comme dans un éclair d’orage ce long cordon de nomades inquiets, nocturnes et traqués, toute une population mâle et femelle de rôdeurs sauvages, habitués aux coups de main, endurcis aux intempéries, déguenillés, « presque tous attaqués d’une gale opiniâtre », et j’en trouve de pareils aux environs de Morlaix, de Lorient et des autres ports, sur les frontières des autres provinces et sur les frontières du royaume. De 1783 à 1787, dans le Quercy, deux bandes alliées de soixante à quatre-vingts contrebandiers fraudent la ferme de quarante milliers de tabac, tuent deux douaniers et défendent, fusil en main, leur entrepôt de la montagne ; il faudrait pour les réprimer des soldats que les commandants militaires ne donnent pas. En 1789\footnote{{\itshape Archives nationales.} H, 1453 (Lettre du baron de Besenval, du 19 juin 1789).}, une grosse troupe de contrebandiers travaille en permanence sur la frontière du Maine et de l’Anjou ; le commandant militaire écrit que « leur chef est un bandit intelligent et redoutable, qu’il a déjà avec lui cinquante-quatre hommes, qu’il aura bientôt avec lui un corps embarrassant par la disposition des esprits et la misère » ; il serait peut-être à propos de corrompre quelques-uns de ses hommes, et de se le faire livrer puisqu’on ne peut le prendre. Ce sont là les procédés des pays où le brigandage est endémique  Ici en effet, comme dans les Calabres, le peuple est pour les brigands contre les gendarmes. On rappelle les exploits de Mandrin en 1754\footnote{{\itshape Mandrin}, par Paul Simian, {\itshape passim. — Histoire de Beaune}, par Rossignol, 453. — {\itshape Mandrin}, par Ch. Jarrin (1875). Le commandant Fischer, qui attaque et disperse la bande, écrit que la chose était urgente ; car sinon, « en remontant du côté du Forez, ils auraient trouvé deux ou trois cents vauriens n’attendant que le moment de se joindre à eux » (47).}, sa troupe de cent cinquante hommes qui apporte des ballots de contrebande et ne rançonne que les commis, ses quatre expéditions qui durent sept mois à travers la Franche-Comté, le Lyonnais, le Bourbonnais, l’Auvergne et la Bourgogne, les vingt-sept villes où il entre sans résistance, délivre les détenus et vend ses marchandises ; il fallut, pour le vaincre, former un camp devant Valence et envoyer 2 000 hommes ; on ne le prit que par trahison, et encore aujourd’hui des familles du pays s’honorent de sa parenté, disant qu’il fut un libérateur  Nul symptôme plus grave : quand le peuple préfère les ennemis de la loi aux défenseurs de la loi, la société se décompose et les vers s’y mettent  Ajoutez à ceux-ci les vrais brigands, assassins et voleurs. « En 1782, la justice prévôtale de Montargis instruit le procès de Hulin et de plus de 200 de ses complices qui, depuis dix ans, par des entreprises combinées, désolaient une partie du royaume\footnote{Mercier, XI, 116.}. » — Mercier compte en France « une armée de plus de 10 000 brigands et vagabonds », contre lesquels la maréchaussée, composée de 3 756 hommes, est toujours en marche. « Tous les jours on se plaint, dit l’assemblée provinciale de la Haute-Guyenne, qu’il n’y ait aucune police dans la campagne. » Le seigneur absent n’y veille pas ; ses juges et officiers de justice se gardent bien d’instrumenter gratuitement contre un criminel insolvable, et « ses terres deviennent l’asile de tous les scélérats du canton\footnote{Voir ci-dessus, Premier livre.} »  Ainsi chaque abus enfante un danger, la négligence mal placée comme la rigueur excessive, la féodalité relâchée comme la monarchie trop tendue. Toutes les institutions semblent d’accord pour multiplier ou tolérer les fauteurs de désordre, et pour préparer, hors de l’enceinte sociale, les hommes d’exécution qui viendront la forcer.\par
Mais leur effet d’ensemble est plus pernicieux encore ; car, de tant de travailleurs qu’elles ruinent, elles font des mendiants qui ne veulent plus travailler, des fainéants dangereux qui vont quêtant ou extorquant leur pain chez des paysans qui n’en ont pas trop peur pour eux-mêmes. « Les vagabonds, dit Letrosne\footnote{Letrosne, {\itshape ib.} (1779), 539.}, sont pour la campagne le fléau le plus terrible ; ce sont des troupes ennemies qui, répandues sur le territoire, y vivent à discrétion et y lèvent des contributions véritables… Ils rôdent continuellement dans les campagnes, ils examinent les approches des maisons et s’informent des personnes qui les habitent et des facultés du maître  Malheur à ceux qui ont la réputation d’avoir quelque argent !… Combien de vols de grand chemin et de vols avec effraction ! Combien de voyageurs assassinés, de maisons et de portes enfoncées ! Combien d’assassinats de curés, de laboureurs, de veuves qu’ils ont tourmentés pour savoir où était leur argent et qu’ils ont tués ensuite ! » Vingt-cinq ans avant la Révolution, il n’était pas rare d’en voir quinze ou vingt « tomber dans une ferme pour y coucher, intimider les fermiers, et en exiger tout ce qu’il leur plaisait »  En 1764, le gouvernement prend contre eux des mesures qui témoignent de l’excès du mal\footnote{{\itshape Archives nationales}, F\textsuperscript{16},965, et H, 892 (Ordonnance du 4 août 1764, instruction circulaire du 20 juillet 1767. Lettre du lieutenant de la maréchaussée de Toulouse du 21 septembre 1787).} : « Sont réputés vagabonds et gens sans aveu, et condamnés comme tels, ceux qui, depuis six mois révolus, n’auront exercé ni profession ni métier, et qui, n’ayant aucun état ni aucun bien pour subsister, ne pourront être avoués ni faire certifier de leurs bonnes vies et mœurs par personnes dignes de foi… L’intention de Sa Majesté n’est pas seulement qu’on arrête les vagabonds qui courent les campagnes, mais encore tous les mendiants, lesquels, n’ayant point de profession, peuvent être regardés comme suspects de vagabondage. » Pour les valides, trois ans de galères ; en cas de récidive, neuf ans ; à la seconde récidive, les galères à perpétuité. Pour les invalides, trois ans de prison ; en cas de récidive, neuf ans ; à la seconde récidive, la prison perpétuelle. Au-dessous de seize ans, les enfants iront à l’hôpital. « Un mendiant qui s’est exposé à être arrêté par la maréchaussée, dit la circulaire, ne doit être relâché qu’avec la plus grande certitude qu’il ne mendiera plus ; on ne s’y déterminera donc que dans le cas où des personnes dignes de foi et {\itshape solvables} répondraient du mendiant, s’engageraient à lui donner de l’occupation ou à le nourrir, et indiqueraient les moyens qu’elles ont pour l’empêcher de mendier. »\par
Tout cela fourni, il faut encore, par surcroît, l’autorisation spéciale de l’intendant. En vertu de cette loi, 50 000 mendiants, dit-on, furent arrêtés tout d’un coup, et, comme les hôpitaux et prisons ordinaires ne suffisaient pas à les contenir, il fallut construire des maisons de force. Jusqu’à la fin de l’ancien régime, l’opération se poursuit avec des intermittences : dans le Languedoc, en 1768, on en arrêtait encore 433 en six mois, et, en 1787, 205 en quatre mois\footnote{{\itshape Archives nationales}, H, 724, H, 554, F\textsuperscript{4}, 2397, F\textsuperscript{16}, 965. — Lettres des concierges des prisons de Carcassonne (22 juin 1789), de Béziers (19 juillet 1786), de Nîmes (I\textsuperscript{ »}  juillet 1786), de l’intendant, M. d’Aîne (19 mars 1786).}. Vers la même époque, il y en avait 300 au dépôt de Besançon, 500 au dépôt de Rennes, 650 au dépôt de Saint-Denis. Leur entretien coûtait au roi un million par an, et Dieu sait comment ils étaient entretenus ! De l’eau, de la paille, du pain, deux onces de graisse salée, en tout cinq sous par jour ; et, comme depuis vingt ans le prix des denrées avait augmenté d’un tiers, il fallait que le concierge chargé de la nourriture les fit jeûner ou se ruinât  Quant à la façon de remplir les dépôts, la police est turque à l’endroit des gens du peuple ; elle frappe dans le tas, et ses coups de balai brisent autant qu’ils nettoient. Par l’ordonnance de 1778, écrit un intendant\footnote{{\itshape Archives nationales}, H, 554 (Lettre de M. de Bertrand, intendant de Rennes, du 17 août 1785).}, « les cavaliers de la maréchaussée doivent arrêter, non seulement les mendiants et vagabonds qu’ils rencontrent, mais encore ceux qu’on leur dénonce comme tels ou comme personnes suspectes. Le citoyen le plus irréprochable dans sa conduite et le moins suspect de vagabondage ne peut donc se promettre de ne pas être enfermé au dépôt, puisque sa liberté est à la merci d’un cavalier de la maréchaussée constamment susceptible d’être trompé par une fausse dénonciation ou corrompu à prix d’argent. J’ai vu dans le dépôt de Rennes plusieurs maris arrêtés sur la seule dénonciation de leurs femmes, et autant de femmes sur celle de leurs maris ; plusieurs enfants du premier lit à la sollicitation de leur belle-mère ; beaucoup de servantes grosses des œuvres du maître qu’elles servaient, enfermées sur sa dénonciation, et des filles dans le même cas, sur la dénonciation de leur séducteur ; des enfants sur la dénonciation de leur père, et des pères sur la dénonciation de leurs enfants : tous sans la moindre preuve de vagabondage et de mendicité… Il n’existe pas un seul jugement prévôtal qui ait rendu la liberté aux détenus, malgré le nombre infini de ceux qui ont été arrêtés injustement. » — Supposons qu’un intendant humain, comme celui-ci, les élargisse : les voilà sur le pavé, mendiants par la faute de la loi qui poursuit la mendicité et qui ajoute aux misérables qu’elle poursuit les misérables qu’elle fait, aigris de plus, gâtés de corps et d’âme. « Il arrive presque toujours, dit encore l’intendant, que les détenus, arrêtés à vingt-cinq ou trente lieues du dépôt, n’y sont renfermés que trois ou quatre mois après leur arrestation, et quelquefois plus longtemps. En attendant, ils sont transférés de brigade en brigade dans les prisons qui se trouvent sur la route, où ils séjournent jusqu’à ce qu’il en soit arrivé un assez grand nombre pour former un convoi. Les hommes et les femmes sont renfermés dans la même prison, et il en résulte toujours que celles qui n’étaient pas grosses quand elles ont été arrêtées le sont toujours quand elles arrivent au dépôt. Les prisons sont ordinairement malsaines ; souvent la plupart des détenus en sortent malades » ; plusieurs, au contact des scélérats, en sortent scélérats. Contagion morale et contagion physique : l’ulcère grandit ainsi par le remède, et les centres de répression deviennent des foyers de corruption.\par
Et cependant, avec toutes ses rigueurs, la loi n’atteint pas son objet. « Nos villes, dit le parlement de Bretagne\footnote{{\itshape Archives nationales}, H, 426. (Remontrances du 4 février 1783.) — H, 554. (Lettre de M. de Bertrand du 17 août 1785.)}, sont tellement peuplées de mendiants, qu’il semble que tous les projets formés pour bannir la mendicité n’ont fait que l’accroître. » — « Les grands chemins, écrit l’intendant, sont infestés de vagabonds dangereux, de gens sans aveu et de véritables mendiants que la maréchaussée n’arrête pas, soit par négligence, soit parce que son ministère n’est point provoqué par des sollicitations particulières. » Qu’en ferait-on, si elle les arrêtait ? Il y en a trop, on ne saurait où les mettre. Et d’ailleurs comment empêcher des gens à l’aumône de demander l’aumône   Sans doute l’effet en est lamentable, mais il est infaillible. À un certain degré, la misère est une gangrène lente où la partie malade mange la partie saine, et l’homme qui subsiste à peine est rongé vif par l’homme qui n’a pas de quoi subsister. « Le paysan est ruiné, il périt victime de l’oppression de la multitude des pauvres qui désolent les campagnes et se réfugient dans les villes. De là ces attroupements dangereux à la sûreté publique ; de là cette foule de fraudeurs, de vagabonds ; de là cette multitude d’hommes devenus voleurs et assassins uniquement parce qu’ils manquent de pain. Ce n’est là encore qu’une légère idée des désordres que j’ai vus sous mes yeux\footnote{{\itshape Archives nationales}, H, 614. ({\itshape Mémoire} par René de Hauteville, avocat au Parlement, Saint-Brieuc, 25 décembre 1776.)}. » — « Excessive en elle-même, la misère des campagnes l’est encore dans les désordres qu’elle entraîne ; il ne faut point chercher ailleurs la source effrayante de la mendicité et de tous ses vices\footnote{{\itshape Procès-verbaux de l’Assemblée provinciale du Soissonnais} (1787), 457.}. » — À quoi bon des palliatifs ou des opérations violentes contre un mal qui est dans le sang et qui tient à la constitution même du corps social ? Quelle police peut être efficace dans une paroisse où le quart, le tiers des habitants n’ont pour manger que ce qu’ils vont quêter de porte en porte ? À Argentré, en Bretagne\footnote{{\itshape Archives nationales}, H, 616. (Lettre de M. Caze de la Bove, intendant de Rennes, du 23 avril 1774.)}, « sur 2 300 habitants sans industrie ni commerce, plus de la moitié ne sont rien moins qu’à l’aise et plus de 500 sont réduits à la mendicité ». À Dainville, en Artois, « sur 130 maisons, 60 sont sur la table des pauvres\footnote{Périn, {\itshape la Jeunesse de Robespierre}, 301. (Doléances des paroisses rurales en 1789.)} ». En Normandie, d’après les déclarations des curés, « sur 900 paroissiens de Saint-Malo, les trois quarts peuvent vivre, le reste est malheureux »  « Sur 1 500 habitants de Saint-Patrice, 400 sont à l’aumône ; sur 500 habitants de Saint-Laurent, les trois quarts sont à l’aumône. » À Marbœuf, dit le cahier, « sur 500 personnes qui habitent notre paroisse, 100 sont réduites à la mendicité, et en outre nous voyons venir des paroisses voisines 30 ou 40 pauvres par jour\footnote{Hippeau, {\itshape le Gouvernement de Normandie}, VII, 147 à 177 (1789). — Boivin-Champeaux, {\itshape Notice historique sur la Révolution dans le département de l’Eure}, 83 (1789).} ». À Boulbonne\footnote{Théron de Montaugé, 87. (Lettre du prieur du couvent, mars 1789.)}, dans le Languedoc, il y a tous les jours aux portes du couvent « une aumône générale à laquelle assistent 300 ou 400 pauvres, indépendamment de celle qu’on fait aux vieillards et aux malades, qui est la plus abondante ». À Lyon, en 1787, « 30 000 ouvriers attendent leur subsistance de la charité publique » ; à Rennes, en 1788, après une inondation, « les deux tiers des habitants sont dans la misère\footnote{{\itshape Procès-verbaux de l’Assemblée provinciale du Lyonnais}, 57. — {\itshape Archives nationales} F\textsuperscript{4}, 2073. Mémoire du 24 janvier 1788. « Les secours de la charité sont très bornés, et les États de la province ne font aucun fonds pour de tels accidents. »} » ; à Paris, sur 650 000 habitants, le recensement de 1791 comptera 118 784 indigents\footnote{Levasseur, \href{http://gallica.bnf.fr/ark:/12148/bpt6k91852h/f123.table}{\dotuline{{\itshape la France industrielle}, 119}} [\url{http://gallica.bnf.fr/ark:/12148/bpt6k91852h/f123.table}]  En 1862, sur une population presque triple (1 696 {\itshape 000}), il y avait 90 {\itshape 000} indigents.}  Vienne une gelée et une grêle comme en 1788, que la récolte manque, que le pain soit à quatre sous la livre, et qu’aux ateliers de charité l’ouvrier ne gagne que douze sous par jour\footnote{Albert Babeau, {\itshape Histoire de Troyes}, I, 91 (Lettre du maire Huez, 30 juillet 1788).} ; croyez-vous que ces gens-là se résigneront à mourir de faim ? Autour de Rouen, pendant l’hiver de 1788, les forêts sont saccagées en plein jour, le bois de Bagnères est coupé tout entier, les arbres abattus sont vendus publiquement par les maraudeurs\footnote{Hoquet, VII, 506.}. Affamés et maraudeurs, tous marchent ensemble, et le besoin se fait le complice du crime. De province en province, on les suit à la trace : quatre mois plus tard, aux environs d’Étampes, quinze brigands forcent trois fermes avant la nuit, et les fermiers, menacés d’incendie, sont obligés de donner, l’un trois cents francs, l’autre cent cinquante, probablement tout l’argent qu’ils ont en coffre\footnote{{\itshape Archives nationales}, H, 1453. (Lettre de M. de Sainte-Suzanne, du 29 avril 1789.)}. « Voleurs, galériens, mauvais sujets de toute espèce », ce sont eux qui, dans les insurrections, feront l’avant-garde, « et pousseront le paysan aux dernières violences\footnote{Arthur Young, I, 256.} ». Après le sac de la maison Réveillon à Paris, on remarque que, « sur une quarantaine de mutins arrêtés, il n’en est presque point qui n’aient été précédemment des repris de justice, fouettés ou marqués\footnote{{\itshape Correspondance secrète inédite de} 1777 {\itshape à} 1792, publiée par M. de Lescure, II, 351 (8 mai 1789). Cf. C. Desmoulins, {\itshape la Lanterne} : sur 100 émeutiers arrêtés à Lyon, 96 étaient {\itshape marqués.}} ». En toute révolution, la lie d’une société monte à la surface. On ne les avait jamais vus ; comme des blaireaux de forêt ou comme des rats d’égout, ils restaient dans leurs tanières ou dans leurs bouges. Ils en sortent par troupes, et tout d’un coup, dans Paris, quelles figures\footnote{Besenval, II, 344, 350. — Dusaulx, \href{http://gallica.bnf.fr/ark:/12148/bpt6k46758g}{\dotuline{{\itshape la Prise de la Bastille}}} [\url{http://gallica.bnf.fr/ark:/12148/bpt6k46758g}], 352. — Marmontel, II, ch. XIV, 249. — Mme Vigée-Lebrun, I, 177, 188.} « On ne se souvient pas d’en avoir rencontré de pareilles en plein jour… D’où sortent-ils ? Qui les a tirés de leurs réduits ténébreux ?… Etrangers de tous pays, armés de grands bâtons, déguenillés, … les uns presque nus, les autres bizarrement vêtus » de loques disparates, « affreux à voir », voilà les chefs ou comparses d’émeute, à six francs par tête, derrière lesquels le peuple va marcher.\par
« À Paris, dit Mercier\footnote{Mercier, I, 32, VI, 15, X, 179, XI, 59, XII, 83. — Arthur Young, I, 122.}, il est mou, pâle, petit, rabougri, maltraité, et semble un corps séparé des autres ordres de l’État. Les riches et les grands qui ont équipage ont le droit barbare de l’écraser ou de le mutiler dans les rues… Aucune commodité pour les gens de pied, point de trottoirs. Cent victimes expirent par an sous les roues des voitures. » — « Un pauvre enfant, dit Arthur Young, a été écrasé sous nos yeux et plusieurs fois j’ai été couvert de la tête aux pieds par l’eau du ruisseau. Si nos jeunes nobles allaient à Londres, dans les rues sans trottoir, du même train que leurs frères de Paris, ils se verraient bientôt et justement rossés de la bonne manière et traînés dans le ruisseau. » — Mercier s’inquiète en face de ce populaire immense. « Il y a peut-être à Paris deux cent mille individus qui n’ont pas en propriété absolue la valeur intrinsèque de cinquante écus ; et la cité subsiste ! » Aussi bien l’ordre n’est maintenu que par la force et la crainte, grâce aux soldats du guet que la multitude appelle {\itshape tristes-à-patte.} « Ce sobriquet met en fureur cette espèce de milice, qui appesantit alors les coups de bourrade et qui blesse indistinctement tout ce qu’elle rencontre. Le petit peuple est toujours sur le point de lui faire la guerre, parce qu’il n’en a jamais été ménagé. » À la vérité, « une escouade du guet dissipe souvent sans peine des pelotons de cinq à six cents hommes qui paraissent d’abord fort échauffés, mais qui se fondent en un clin-d’œil dès que les soldats ont distribué quelques bourrades et gantelé deux ou trois mutins. » — Néanmoins, « si l’on abandonnait le peuple de Paris à son premier transport, s’il ne sentait plus derrière lui le guet à pied et à cheval, le commissaire et l’exempt, il ne mettrait aucune mesure dans son désordre. La populace, délivrée du frein auquel elle est accoutumée, s’abandonnerait à des violences d’autant plus cruelles qu’elle ne saurait elle-même où s’arrêter… Tant que le pain de Gonesse ne manquera pas, la commotion ne sera pas générale ; il faut que la halle\footnote{{\itshape Dialogues sur le commerce des blés}, par Galiani (1770). « Si les forts de la halle sont contents, il n’arrivera aucun désastre à l’administration. Les grands conspirent et se révoltent ; les bourgeois se plaignent et vivent dans le célibat ; les paysans et les artisans se désespèrent et s’en vont ; les portefaix s’ameutent. »} y soit intéressée, sinon les femmes demeureront calmes… Mais si le pain de Gonesse venait à manquer pendant deux marchés de suite, le soulèvement serait universel, et il est impossible de calculer à quoi se porterait cette grande multitude aux abois, qui voudrait se délivrer de la famine, elle et ses enfants. » — En 1789, le pain manque à Gonesse et dans toute la France.
\chapterclose


\chapteropen

\chapter[{Chapitre IV}]{Chapitre IV}


\chaptercont

\section[{I. La force armée se dissout. — Comment l’armée est recrutée. — Comment le soldat est traité.}]{I. La force armée se dissout. — Comment l’armée est recrutée. — Comment le soldat est traité.}

\noindent Contre la sédition universelle, où est la force   Dans les cent cinquante mille hommes qui maintiennent l’ordre, les dispositions sont les mêmes que dans les vingt-six millions d’hommes qui le subissent, et les abus, la désaffection, toutes les causes qui dissolvent la nation dissolvent aussi l’armée. Sur quatre-vingt-dix millions\footnote{Necker, {\itshape de l’Administration des Finances}, II, 422, 435.} de solde que chaque année elle coûte au Trésor, il y a 46 millions pour les officiers, 44 seulement pour les soldats, et l’on sait qu’une ordonnance nouvelle réserve tous les grades aux nobles vérifiés. Nulle part cette inégalité, contre laquelle l’opinion publique se révolte, n’éclate en traits si forts : d’un côté, pour le petit nombre, l’autorité, les honneurs, l’argent, le loisir, la bonne chère, les plaisirs du monde, les comédies de société ; de l’autre, pour le grand nombre, l’assujettissement, l’abjection, la fatigue, l’enrôlement par contrainte ou surprise, nul espoir d’avancement, six sous par jour\footnote{En 1789, la paye avait été portée à 7 sous 4 deniers, sur lesquels on retenait 2 sous 6 deniers pour le pain. ({\itshape Mercure de France}, 7 mai 1791.)}, un lit étroit pour deux, du pain de chien, et, depuis quelques années, des coups comme à un chien\footnote{Aubertin, 345. Lettre du comte de Saint-Germain (pendant la guerre de Sept Ans). « La misère du soldat est si grande, qu’elle fait saigner le cœur ; il passe ses jours dans un état abject et méprisé, il vit comme un chien enchaîné qu’on destine au combat. »} ; d’un côté est la plus haute noblesse, de l’autre est la dernière populace. On dirait d’un fait exprès pour assembler les contrastes et aigrir l’irritation. « La médiocrité de la solde du soldat, dit un économiste, la manière dont il est habillé, couché et nourri, son entière dépendance, rendraient trop cruel de prendre un autre homme qu’un homme du bas peuple\footnote{Tocqueville, 190, 191.} ». En effet, on ne va le chercher que dans les bas-fonds. Sont exempts du tirage, non seulement tous les nobles et bourgeois, mais encore tous les employés de l’administration des fermes et des ponts et chaussées, « tous les garde-chasse, garde-bois, domestiques et valets à gages des ecclésiastiques, des communautés, des maisons religieuses, des gentilshommes, des nobles\footnote{{\itshape Archives nationales}, H, 1591.} », et même des bourgeois vivant noblement, bien mieux, les fils des cultivateurs aisés, et, en général, tous ceux qui ont un crédit ou un protecteur quelconque. — Il ne reste donc pour la milice que les plus pauvres, et ce n’est pas de bon cœur qu’ils y entrent. Au contraire, le service leur est si odieux, que souvent ils se sauvent dans les bois, où il faut les poursuivre à main armée : dans tel canton qui, trois ans plus tard, fournira en un jour de cinquante à cent volontaires, les garçons se coupent le pouce pour être exempts du tirage\footnote{Maréchal de Rochambeau, {\itshape Mémoires}, I, 427. — Marquis d’Argenson, 24 décembre 1752. « On compte plus de 30 000 hommes suppliciés pour désertion depuis la paix de 1748 ; l’on attribue cette grande désertion au nouvel exercice, qui fatigue et désespère les soldats, surtout les vieux soldats. » — Voltaire, \href{http://www.voltaire-integral.com/Html/20/supplices.htm}{\dotuline{{\itshape Dictionnaire} {\itshape philosophique}, article {\itshape Supplices}}} [\url{http://www.voltaire-integral.com/Html/20/supplices.htm}]. « Je fus effrayé un jour en voyant la liste des déserteurs depuis huit années seulement : on en comptait 60 000. »}. — À cette vase de la société, on ajoute la balayure des dépôts et maisons de force. Parmi les vagabonds qui les remplissent, lorsqu’on a évacué ceux qui peuvent faire connaître leur famille ou trouver des répondants, « il n’y a plus, dit un intendant, que des gens absolument inconnus ou dangereux ; dans ce nombre on prend ceux qu’on regarde comme les moins vicieux, et l’on cherche à les faire passer dans les troupes\footnote{{\itshape Archives nationales}, H, 554. (Lettre de M. de Bertrand, intendant de Rennes, du 17 août 1785.)} ». — Dernier affluent, l’embauchement demi-forcé, demi-volontaire, qui le plus souvent ne verse dans les cadres que l’écume des grandes villes, aventuriers, apprentis renvoyés, fils de famille chassés, gens sans asile et sans aveu. L’embaucheur, payé à tant par homme qu’il recrute et à tant par pouce de taille au-dessus de cinq pieds, « tient ses assises dans un cabaret, régale » et fait l’article : « Mes amis, la soupe, l’entrée, le rôti, la salade, voilà l’ordinaire du régiment » ; rien de plus, je ne vous trompe pas, le pâté et le vin d’Arbois sont l’extraordinaire\footnote{Mercier, XI, 121.}. » Il fait boire, il paye le vin, au besoin il cède sa maîtresse : « après quelques jours de débauche, le jeune libertin qui n’a pas de quoi s’acquitter est obligé de se vendre, et l’ouvrier, transformé en soldat, va faire l’exercice sous le bâton ». — Étranges recrues pour garder une société, toutes choisies dans la classe qui l’attaque, paysans foulés, vagabonds emprisonnés, gens déclassés, ; endettés, désespérés, pauvres diables aisément tentés et de cervelle chaude, qui, selon les circonstances, deviennent tantôt des révoltés et tantôt des soldats.\par
Qui des deux a le meilleur lot ? Le pain du soldat n’est pas plus abondant que celui du détenu, et il est pire ; car on ôte le son pour faire le pain du vagabond enfermé, et on le laisse pour faire le pain du soldat qui l’enferme. — En cet état de choses, il ne faudrait pas que le soldat réfléchît, et voilà justement que ses officiers l’invitent à réfléchir. Eux aussi, ils sont devenus politiques et frondeurs. Quelques années avant la Révolution\footnote{Vaublanc, 149.}, « on parlait déjà » dans l’armée, « on raisonnait, on se plaignait, et, les idées nouvelles fermentant dans les têtes, une correspondance s’établit entre deux régiments. On recevait de Paris des nouvelles écrites à la main ; elles étaient autorisées par le ministre de la guerre, et coûtaient, je crois, douze louis par an. Bientôt elles prirent un ton philosophique, elles dissertèrent, elles parlèrent des ministres, du gouvernement, des changements désirés, et n’en furent que plus répandues ». Certainement, des sergents comme Hoche, des maîtres d’armes comme Augereau, ont lu plus d’une fois ces nouvelles oubliées sur la table, et les ont commentées le soir même dans les chambrées de soldats. Le mécontentement est ancien, et déjà à la fin du dernier règne des mots accablants ont éclaté. Dans un festin donné par un prince du sang\footnote{Ségur, I, 20 (1767).}, la table de cent couverts, dressée sous une tente immense, était servie par les grenadiers, et l’odeur qu’ils répandaient offusqua la délicatesse du prince. « Ces braves gens, dit-il un peu trop haut, sentent diablement le chausson. » Un grenadier répondit brusquement : « C’est parce que nous n’en avons pas », et « un profond silence suivit cette réponse ». — Pendant les vingt ans qui suivent, l’irritation couve et grandit : les soldats de Rochambeau ont combattu côte à côte avec les libres milices de l’Amérique et s’en souviennent. En 1788\footnote{Augeard, {\itshape Mémoires}, 165.}, le maréchal de Vaux, devant le soulèvement du Dauphiné, écrit au ministre « qu’il est impossible de compter sur les troupes », et, quatre mois après l’ouverture des États Généraux, seize mille déserteurs, rôdant autour de Paris, conduiront les émeutes au lieu de les réprimer\footnote{Horace Walpole (5 septembre 1789).}.

\section[{II. L’organisation sociale est dissoute. — Nul centre de ralliement. — Inertie de la province. — Ascendant de Paris.}]{II. L’organisation sociale est dissoute. — Nul centre de ralliement. — Inertie de la province. — Ascendant de Paris.}

\noindent Une fois cette digue emportée, il n’y a plus de digue, et l’inondation roule sur toute la France comme sur une plaine unie  En pareil cas, chez les autres peuples, des obstacles se sont rencontrés : il y avait des lieux élevés, des centres de refuges, quelques vieilles enceintes où, dans l’effarement universel, une partie de la population trouvait des abris  Ici le premier choc achève d’en emporter les derniers restes, et, dans ces vingt-six millions d’hommes dispersés, chacun est seul. Depuis longtemps, et par un travail insensible, l’administration de Richelieu et de Louis XIV a détruit les groupes naturels qui, après un effondrement soudain, se reforment d’eux-mêmes. Sauf en Vendée, je ne vois aucun endroit ni aucune classe où beaucoup d’hommes, ayant confiance en quelques hommes, puissent, à l’heure du danger, se rallier autour d’eux pour faire un corps. Il n’y a plus de patriotisme provincial ou municipal. Le bas clergé est hostile aux prélats, les gentilshommes de province à la noblesse de cour, le vassal au seigneur, le paysan au citadin, la population urbaine à l’oligarchie municipale, la corporation à la corporation, la paroisse à la paroisse, le voisin au voisin. Tous sont séparés par leurs privilèges, par leurs jalousies, par la conscience qu’ils ont d’être chargés ou frustrés au profit d’autrui. L’ouvrier tailleur est aigri contre le maître tailleur qui l’empêche d’aller en journée chez les bourgeois, les garçons perruquiers contre le maître perruquier qui ne leur permet pas de coiffer en ville, le pâtissier contre le boulanger qui l’empêche de cuire les pâtés des ménagères, le villageois fileur contre les filateurs de la ville qui voudraient briser son métier, les vignerons de campagne contre le bourgeois qui, dans un rayon de sept lieues, voudrait faire arracher leurs vignes\footnote{Laboulaye, {\itshape de l’Administration française sous Louis XVI} ({\itshape Revue des Cours littéraires, IV}, 743). — Albert Babeau, I, 111 ({\itshape Doléances et vœux des corporations de Troyes}).}, le village contre le village voisin dont le dégrèvement l’a grevé, le paysan haut taxé contre le paysan taxé bas, la moitié de la paroisse contre ses collecteurs, qui à son détriment ont favorisé l’autre moitié. « La nation, disait tristement Turgot\footnote{Tocqueville, 158.}, est une société composée de différents ordres mal unis, et d’un peuple dont les membres n’ont entre eux que très peu de liens, et où, par conséquent, {\itshape personne n’est occupé que de son intérêt particulier. Nulle part il n’y a d’intérêt commun visible.} Les villes, les villages n’ont pas plus de rapport entre eux que les arrondissements auxquels ils sont attribués ; ils ne peuvent même s’entendre entre eux pour mener les travaux publics qui leur sont nécessaires. » Depuis cent cinquante ans, le pouvoir central a divisé pour régner. Il a tenu les hommes séparés, il les a empêchés de se concerter, il a si bien fait, qu’ils ne se connaissent plus, que chaque classe ignore l’autre classe, que chacune se fait de l’autre un portrait chimérique, chacune teignant l’autre des couleurs de son imagination, l’une composant une idylle, l’autre se forgeant un mélodrame, l’une imaginant les paysans comme des bergers sensibles, l’autre persuadée que les nobles sont d’affreux tyrans. — Par cette méconnaissance mutuelle et par cet isolement séculaire, les Français ont perdu l’habitude, l’art et la faculté d’agir ensemble. Ils ne sont plus capables d’entente spontanée et d’action collective. Au moment du danger, personne n’ose compter sur ses voisins ou sur ses pareils. Personne ne sait où tourner les yeux pour trouver un guide. « On n’aperçoit pas un homme qui puisse répondre pour le plus petit district ; et, bien plus, on n’en voit pas un qui puisse répondre d’un autre homme\footnote{Tocqueville, 304. (Paroles de Burke.)}. » La débandade est complète et sans remède. L’utopie des théoriciens s’est accomplie, l’état sauvage a recommencé. Il n’y a plus que des individus juxtaposés ; chaque homme retombe dans sa faiblesse originelle, et ses biens, sa vie sont à la merci de la première bande qui saura se former. Il ne reste en lui pour le conduire que l’habitude moutonnière d’être conduit, d’attendre l’impulsion, de regarder du côté du centre ordinaire, vers Paris, d’où sont toujours venus les ordres. Arthur Young\footnote{{\itshape Voyages en France}, I, 240, 263.} est frappé de ce geste machinal. Partout l’ignorance et la docilité politiques sont parfaites. C’est lui, un étranger, qui apporte en Bourgogne les nouvelles d’Alsace : l’insurrection y a été terrible ; la populace a saccagé l’hôtel de ville de Strasbourg, et personne n’en sait un mot à Dijon. « Cependant, écrit-il, voilà neuf jours que la chose est arrivée ; mais, quand il y en aurait dix-neuf, je doute qu’on eût été mieux renseigné. » Point de journaux dans les cafés ; nul centre d’information, de résolution, d’action locale. La province subit les événements de la capitale ; « les gens n’osent bouger, ils n’osent pas même se faire une opinion avant que Paris ait prononcé. » — C’est à cela qu’aboutit la centralisation monarchique. Elle a ôté aux groupes leur consistance et à l’individu son ressort. Reste une poussière humaine qui tourbillonne et qui, avec une force irrésistible, roulera tout entière en une seule masse, sous l’effort aveugle du vent.

\section[{III. Direction du courant. — L’homme du peuple conduit par l’avocat. — Les seuls pouvoirs survivants sont la théorie et les piques. — Suicide de l’ancien régime.}]{III. Direction du courant. — L’homme du peuple conduit par l’avocat. — Les seuls pouvoirs survivants sont la théorie et les piques. — Suicide de l’ancien régime.}

\noindent Nous savons déjà de quel côté il souffle, et il suffit, pour en être sûr, de voir comment les cahiers du Tiers ont été faits. C’est l’homme de loi, le petit procureur de campagne, l’avocat envieux et théoricien qui a conduit le paysan. Celui-ci insiste pour que, dans le cahier, on couche par écrit et tout au long ses griefs locaux et personnels, sa réclamation contre les impôts et redevances, sa requête pour délivrer ses chiens du billot, sa volonté d’avoir un fusil contre les loups\footnote{Beugnot, I, 115, 116.}. L’autre, qui suggère et dirige, enveloppe le tout dans les Droits de l’Homme et dans la circulaire de Siéyès. « Depuis deux mois, écrit un commandant du Midi\footnote{{\itshape Archives nationales}, procès-verbaux et cahiers des États Généraux, t. XIII, 405. (Lettre du marquis de Faudoas, commandant de l’Armagnac, à M. Necker, du 29 mai 1789.)}, les juges inférieurs, les avocats dont toutes les villes et campagnes fourmillent, en vue de se faire élire aux États Généraux, se sont mis après les gens du Tiers-état, sous prétexte de les soutenir et d’éclairer leur ignorance… Ils se sont efforcés de leur persuader qu’aux États Généraux ils seraient les maîtres à eux seuls de régler toutes les affaires du royaume, que le Tiers, en choisissant ses députés parmi les gens de robe, aurait le droit et la force de primer, d’abolir la noblesse, de détruire tous ses droits et privilèges, qu’elle ne serait plus héréditaire, que tous les citoyens, en la méritant, auraient le droit d’y prétendre ; que, si le peuple les députait, ils feraient accorder au Tiers-état tout ce qu’il voudrait, parce que les curés, gens du Tiers, étant convenus de se détacher du haut clergé et de s’unir à eux, la noblesse et le clergé, unis ensemble, ne feraient qu’une voix contre deux du Tiers… Si le Tiers avait choisi de sages bourgeois ou négociants, ils se seraient unis sans difficulté aux deux autres ordres. Mais les assemblées de bailliages et de sénéchaussées ont été farcies de gens de robe qui absorbaient les opinions et voulaient primer sur tout le monde, et chacun, de son côté, intriguait et cabalait pour se faire députer. » — « En Touraine, écrit l’intendant\footnote{{\itshape Archives nationales}, tome CL, 174. (Lettre de l’intendant de Tours du 25 mars 1789.)}, l’avis de la plupart des votants a été commandé ou mendié. Les affidés mettaient, au moment du scrutin, des billets tout écrits dans la main des votants, et leur avaient fait trouver, à leur arrivée aux auberges, tous les écrits et avis propres à exalter leurs têtes et à déterminer leur choix pour des gens du palais. » — « Dans la sénéchaussée de Lectoure, une quantité de paroisses et de communautés n’ont point été assignées ni averties pour envoyer leurs cahiers et leurs députés à l’assemblée de la sénéchaussée. Pour celles qui ont été averties, les avocats, procureurs et notaires des petites villes voisines ont fait leurs doléances de leur chef, sans assembler la communauté… Sur un seul brouillon, ils faisaient pour toutes des copies pareilles qu’ils vendaient bien cher, aux conseils de chaque paroisse de campagne. » — Symptôme alarmant et qui marque d’avance la voie que va suivre la Révolution : l’homme du peuple est endoctriné par l’avocat, l’homme à pique se laisse mener par l’homme à phrases.\par
Dès la première année, on peut voir l’effet de leur association. En Franche-Comté\footnote{{\itshape Archives nationales}, H, 784. (Lettres de M. de Langeron, commandant militaire à Besançon, 16 et 18 octobre 1789. — La consultation y est annexée.)}, sur la consultation d’un nommé Rouget, les paysans du marquis de Chaila « se déterminent à ne plus lui rien payer et à se partager le produit des coupes de bois, sans y appeler la maîtrise ». Dans son papier « l’avocat avance que toutes les communautés de la province sont décidées à en faire autant… Sa consultation est tellement répandue dans les campagnes, que beaucoup de communautés sont convaincues qu’elles ne doivent plus rien au roi ni à leurs seigneurs. M. de Marnezia, député à l’Assemblée (nationale), est venu (ici) passer quelques jours chez lui pour sa santé ; il y a été traité de la manière la plus dure et la plus scandaleuse ; l’on a même agité si on ne le conduirait pas à Paris sous escorte. Après son départ, son château a été attaqué, les portes ont été brisées et les murs de son jardin abattus. (Pourtant) aucun gentilhomme n’a autant fait pour les habitants de ses terres que M. le marquis de Marnezia… Les excès en tout genre augmentent ; j’ai des plaintes perpétuelles sur l’abus que les milices nationales font de leurs armes, et je ne puis y remédier. » D’après une phrase prononcée à l’Assemblée nationale, la maréchaussée croit qu’elle va être dissoute et ne veut pas se faire d’ennemis. « Les bailliages sont aussi timides que la maréchaussée ; je leur renvoie sans cesse des affaires, et aucun coupable n’est puni… » — « Aucune nation ne jouit d’une liberté si indéfinie et si funeste aux honnêtes gens ; il est absolument contraire aux droits de l’homme de se voir perpétuellement dans le cas d’être égorgé par des scélérats qui confondent toute la journée la liberté et la licence. » — En d’autres termes, les passions, pour s’autoriser, ont recours à la théorie, et la théorie, pour s’appliquer, a recours aux passions. Par exemple, près de Liancourt, le duc de la Rochefoucauld avait un terrain inculte ; « dès le commencement de la Révolution\footnote{ Arthur Young, I, 344.
 }, les pauvres de la ville déclarent que, puisqu’ils font partie de la nation, les terrains incultes, propriété de la nation, leur appartiennent », et tout de suite, « sans autre formalité », ils entrent en possession, se partagent le sol, plantent des haies et défrichent. « Ceci, dit Arthur Young, montre l’esprit général… Poussées un peu loin, les conséquences ne seraient pas petites pour la propriété dans ce royaume. » Déjà, l’année précédente, auprès de Rouen, les maraudeurs, qui abattaient et vendaient les forêts, disaient que « le peuple a le droit de prendre tout ce qui est nécessaire à ses besoins »  On leur a prêché qu’ils sont souverains, et ils agissent en souverains. Étant donné leur état d’esprit, rien de plus naturel que leur conduite. Plusieurs millions de sauvages sont ainsi lancés par quelques milliers de parleurs, et la politique de café a pour interprète et ministre l’attroupement de la rue. D’une part la force brutale se met au service du dogme radical. D’autre part le dogme radical se met au service de la force brutale. Et voilà, dans la France dissoute, les deux seuls pouvoirs debout sur les débris du reste.
\chapterclose


\chapteropen

\chapter[{Chapitre V. Résumé.}]{Chapitre V. \\
Résumé.}


\chaptercont

\section[{I}]{I}

\noindent Ils sont les successeurs et les exécuteurs de l’ancien régime, et, quand on regarde la façon dont celui-ci les a engendrés, couvés, nourris, intronisés, provoqués, on ne peut s’empêcher de considérer son histoire comme un long suicide : de même un homme qui, monté au sommet d’une immense échelle, couperait sous ses pieds l’échelle qui le soutient  En pareil cas, les bonnes intentions ne suffisent pas ; il ne sert à rien d’être libéral et même généreux, d’ébaucher des demi-réformes. Au contraire, par leurs qualités comme par leurs défauts, par leurs vertus comme par leurs vices, les privilégiés ont travaillé à leur chute, et leurs mérites ont contribué à leur ruine aussi bien que leurs torts  Fondateurs de la société, ayant jadis mérité leurs avantages par leurs services, ils ont gardé leur rang sans continuer leur emploi ; dans le gouvernement local comme dans le gouvernement central, leur place est une sinécure, et leurs privilèges sont devenus des abus. À leur tête, le roi, qui a fait la France en se dévouant à elle comme à sa chose propre, finit par user d’elle comme de sa chose propre ; l’argent public est son argent de poche, et des passions, des vanités, des faiblesses personnelles, des habitudes de luxe, des préoccupations de famille, des intrigues de maîtresse, des caprices d’épouse gouvernent un État de vingt-six millions d’hommes avec un arbitraire, une incurie, une prodigalité, une maladresse, un manque de suite qu’on excuserait à peine dans la conduite d’un domaine privé  Roi et privilégiés, ils n’excellent qu’en un point, le savoir-vivre, le bon goût, le bon ton, le talent de représenter et de recevoir, le don de causer avec grâce, finesse et gaieté, l’art de transformer la vie en une fête ingénieuse et brillante, comme si le monde était un salon d’oisifs délicats où il suffit d’être spirituel et aimable, tandis qu’il est un cirque où il faut être fort pour combattre, et un laboratoire où il faut travailler pour être utile  Par cette habitude, cette perfection et cet ascendant de la conversation polie, ils ont imprimé à l’esprit français la forme classique, qui, combinée avec le nouvel acquis scientifique, produit la philosophie du dix-huitième siècle, le discrédit de la tradition, la prétention de refondre toutes les institutions humaines d’après la raison seule, l’application des méthodes mathématiques à la politique et à la morale, le catéchisme des droits de l’homme, et tous les dogmes anarchiques et despotiques du Contrat social  Une fois que la chimère est née, ils la recueillent chez eux comme un passe-temps de salon ; ils jouent avec le monstre tout petit, encore innocent, enrubanné comme un mouton d’églogue ; ils n’imaginent pas qu’il puisse jamais devenir une bête enragée et formidable ; ils le nourrissent, ils le flattent, puis, de leur hôtel, ils le laissent descendre dans la rue  Là, chez une bourgeoisie que le gouvernement indispose en compromettant sa fortune, que les privilèges heurtent en comprimant ses ambitions, que l’inégalité blesse en froissant son amour-propre, la théorie révolutionnaire prend des accroissements rapides, une âpreté soudaine, et, au bout de quelques années, se trouve la maîtresse incontestée de l’opinion  À ce moment et sur son appel, surgit un autre colosse, un monstre aux millions de têtes, une brute effarouchée et aveugle, tout un peuple pressuré, exaspéré et subitement déchaîné contre le gouvernement dont les exactions le dépouillent, contre les privilégiés dont les droits l’affament, sans que, dans ces campagnes désertées par leurs patrons naturels, il se rencontre une autorité survivante, sans que, dans ces provinces pliées à la centralisation mécanique, il reste un groupe indépendant, sans que, dans cette société désagrégée par le despotisme, il puisse se former des centres d’initiative et de résistance, sans que, dans cette haute classe désarmée par son humanité même, il se trouve un politique exempt d’illusion et capable d’action, sans que tant de bonnes volontés et de belles intelligences puissent se défendre contre les deux ennemis de toute liberté et de tout ordre, contre la contagion du rêve démocratique qui trouble les meilleures têtes et contre les irruptions de la brutalité populacière qui pervertit les meilleures lois. À l’instant où s’ouvrent les États Généraux, le cours des idées et des événements est non seulement déterminé, mais encore visible. D’avance et à son insu, chaque génération porte en elle-même son avenir et son histoire ; à celle-ci, bien avant l’issue, on eût pu annoncer ses destinées, et, si les détails tombaient sous nos prévisions aussi bien que l’ensemble, on pourrait croire à la fiction suivante que Laharpe converti inventa à la fin du Directoire, en arrangeant ses souvenirs.

\section[{II}]{II}

\noindent « Il me semble, dit-il, que c’était hier, et c’était cependant au commencement de 1788. Nous étions à table chez un de nos confrères à l’Académie, grand seigneur et homme d’esprit. La compagnie était nombreuse et de tout état, gens de cour, gens de robe, gens de lettres, académiciens ; on avait fait grand’chère comme de coutume. Au dessert, les vins de Malvoisie et de Constance ajoutaient à la gaieté de bonne compagnie cette sorte de liberté qui n’en gardait pas toujours le ton. On en était alors venu dans le monde au point où tout est permis pour faire rire. Chamfort nous avait lu ses contes impies et libertins, et les grandes dames avaient écouté sans avoir même recours à l’éventail. De là un déluge de plaisanteries sur la religion ; l’un citait une tirade de la {\itshape Pucelle ;} l’autre rapportait certains vers philosophiques de Diderot… Et d’applaudir… La conversation devient plus sérieuse ; on se répand en admiration sur la révolution qu’avait faite Voltaire, et l’on convient que c’était là le premier titre de sa gloire. « Il a donné le ton à son siècle, et s’est fait lire dans l’antichambre comme dans le salon. » Un des convives nous raconta, en pouffant de rire, qu’un coiffeur lui avait dit, tout en le poudrant : « Voyez-vous, monsieur, quoique je ne sois qu’un misérable carabin, je n’ai pas plus de religion qu’un autre »  On conclut que la révolution ne tardera pas à se consommer, qu’il faut absolument que la superstition et le fanatisme fassent place à la philosophie, et l’on en est à calculer la probabilité de l’époque et quels seront ceux de la société qui verront le règne de la raison  Les plus vieux se plaignaient de ne pouvoir s’en flatter ; les jeunes se réjouissaient d’en avoir une espérance très vraisemblable, et l’on félicitait surtout l’Académie d’avoir préparé le grand œuvre et d’avoir été le chef-lieu, le centre, le mobile de la liberté de penser. « Un seul des convives n’avait point pris de part à toute la joie de cette conversation… C’était Cazotte, homme aimable et original, mais malheureusement infatué des rêveries des illuminés. Il prend la parole et, du ton le plus sérieux : « Messieurs, dit-il, soyez satisfaits ; vous verrez tous cette grande révolution que vous désirez tant. Vous savez que je suis un peu prophète, je vous le répète, vous la verrez… Savez-vous ce qui arrivera de cette révolution, ce qui en arrivera pour vous tous tant que vous êtes ici   Ah ! voyons, dit Condorcet avec son air et son rire sournois et niais, un philosophe n’est pas fâché de rencontrer un prophète. — Vous, monsieur de Condorcet, vous expirerez étendu sur le pavé d’un cachot, vous mourrez du poison que vous aurez pris pour vous dérober au bourreau, du poison que le bonheur de ce temps-là vous forcera à porter toujours sur vous ». Grand étonnement d’abord, puis l’on rit de plus belle. Qu’est-ce que tout cela peut avoir de commun avec la philosophie et le règne de la raison ? « C’est précisément ce que je vous dis : c’est au nom de la philosophie, de l’humanité, de la liberté, c’est sous le règne de la raison qu’il vous arrivera de finir ainsi ; et ce sera bien le règne de la raison, car elle aura des temples, et même il n’y aura plus dans toute la France, en ce temps-là, que des temples de la raison… Vous, monsieur de Chamfort, vous vous couperez les veines de vingt-deux coups de rasoir, et pourtant vous n’en mourrez que quelques mois après. Vous, monsieur Vicq-d’Azyr, vous ne vous ouvrirez pas les veines vous-même, mais vous les ferez ouvrir six fois dans un jour, au milieu d’un accès de goutte, pour être plus sûr de votre fait, et vous mourrez dans la nuit. Vous, monsieur de Nicolaï, sur l’échafaud ; vous, monsieur Bailly, sur l’échafaud ; vous, monsieur de Malesherbes, sur l’échafaud ; … vous, monsieur Roucher, aussi sur l’échafaud  Mais nous serons donc subjugués par les Turcs et les Tartares   Point du tout ; je vous l’ai dit, vous serez alors gouvernés par la seule philosophie et par la seule raison. Ceux qui vous traiteront ainsi seront tous des philosophes, auront à tout moment à la bouche les phrases que vous débitez depuis une heure, répéteront toutes vos maximes, citeront comme vous les vers de Diderot et de la {\itshape Pucelle}  Et quand tout cela n’arrivera-t-il   Six ans ne se passeront pas que tout ce que je vous dis ne soit accompli  Voilà bien des miracles, dit Laharpe, et vous ne m’y mettez pour rien  Vous y serez pour un miracle tout au moins aussi extraordinaire ; vous serez alors chrétien  Ah ! reprit Chamfort, je suis rassuré ; si nous ne devons mourir que quand Laharpe sera chrétien, nous sommes immortels  Pour ça, dit alors la duchesse de Gramont, nous sommes bien heureuses, nous autres femmes, de n’être pour rien dans les révolutions. Il est reçu qu’on ne s’en prend pas à nous et notre sexe…  Votre sexe, mesdames, ne vous en défendra pas cette fois… Vous serez traitées tout comme les hommes, sans aucune différence quelconque… Vous, madame la duchesse, vous serez conduite à l’échafaud, vous et beaucoup d’autres dames avec vous, dans la charrette et les mains liées derrière le dos  Ah ! j’espère que dans ce cas-là j’aurai du moins un carrosse drapé de drap noir  Non, madame, de plus grandes dames que vous iront comme vous en charrette et les mains liées comme vous  De plus grandes dames ! Quoi ! les princesses du sang   De plus grandes dames encore…  On commençait à trouver que la plaisanterie était forte. Madame de Gramont, pour dissiper le nuage, n’insista pas sur cette dernière réponse et se contenta de dire de son ton le plus léger : Vous verrez qu’il ne me laissera seulement pas un confesseur  Non, madame, vous n’en aurez pas, ni vous, ni personne ; le dernier supplicié qui en aura un par grâce, sera… » Il s’arrêta un moment : « Eh bien, quel est donc l’heureux mortel qui aura cette prérogative   C’est la seule qui lui restera, et ce sera le roi de France. »
\chapterclose

\chapterclose


\section[{Notes sur l’Ancien-Régime}]{Notes sur l’Ancien-Régime}
\renewcommand{\leftmark}{Notes sur l’Ancien-Régime}


\subsection[{Note 1. livre premier, chapitre II, I. Sur le nombre des ecclésiastiques et des nobles.}]{Note 1\\
livre premier, chapitre II, I\\
Sur le nombre des ecclésiastiques et des nobles.}

\noindent On a obtenu ces chiffres approximatifs par les procédés suivants :\par
1 ° Pour ce qui est de la noblesse, le chiffre était inconnu en 1789. Dans son {\itshape Abrégé chronologique des Édits}, etc. (1789), le généalogiste Chérin déclare qu’il l’ignore. Moheau, à qui Lavoisier s’en réfère dans son rapport de 1791, n’en sait pas davantage ({\itshape Recherches sur la population de la France}, 1778, 105) ; Lavoisier dit 83 000 individus, et le marquis de Bouillé ({\itshape Mémoires}, 50) 80 000 familles, tous deux sans aucune preuve  J’ai relevé, dans le {\itshape Catalogue nominatif des gentilshommes en} 1789, par Laroque et Barthélemy, le nombre des nobles qui ont voté, directement ou par procuration, aux élections de 1789, en Provence, Languedoc, Lyonnais, Forez, Beaujolais, Touraine, Normandie, Ile-de-France ; ce nombre est de 9 167  D’après le recensement de 1790 donné par Arthur Young dans ses {\itshape Voyages en France}, le nombre des habitants de ces provinces est de 7 757 000, ce qui, par proportion, donne un peu plus de 30 000 nobles votants parmi les 26 millions d’habitants de la France  En étudiant la loi, et en dépouillant les listes, on voit que chacun de ces nobles représente un peu moins d’une famille, puisque le fils d’un propriétaire de fief vote s’il a vingt-cinq ans ; je ne crois donc pas qu’on se trompe beaucoup en évaluant à 26 000 ou 28 000 le nombre des familles nobles, ce qui, à raison de 5 personnes par famille, donne 130 000 ou 140 000 nobles  La France en 1789 ayant 27 000 lieues carrées et 26 millions d’habitants, on peut compter une famille noble par lieue carrée et par 1 000 habitants.\par
2° Pour ce qui est du clergé, j’ai relevé, aux {\itshape Archives nationales}, dans les papiers du comité ecclésiastique, l’état nominatif des religieux de 28 ordres : Grands-Augustins 694, Petits-Pères 250, Barnabites 90, Bénédictins anglais 52, Bénédictins de Cluny 298, de Saint-Vanne 612, de Saint-Maur 1 672, Cîteaux 1 806, Récollets 2 238, Prémontrés 399, Prémontrés-Réformés 394, Capucins 3 720, Carmes déchausses 555, Grands-Carmes 853, Hospitaliers de Saint-Jean de Dieu 218, Chartreux 1 144, Cordeliers 2 018, Dominicains 1 172, Feuillants 148, Genovéfains 570, Mathurins 310, Minimes 684, Notre-Dame de la Merci 31, Notre-Sauveur 203, Tiers-ordre de Saint-François 365, Saint-Jean des Vignes de Soissons 31, Théatins 25, abbaye de Saint-Victor 21, Maisons soumises à l’ordinaire 305. Total 20 745 religieux en 2 489 couvents. Il faut y ajouter les Pères de l’Oratoire, de la Mission, de la Doctrine chrétienne et quelques autres ; le total de tous les moines doit osciller autour de 23 000  Quant aux religieuses, j’ai relevé aux {\itshape Archives nationales} leur catalogue dans 12 diocèses comprenant, d’après la {\itshape France ecclésiastique} de 1788, 5 576 paroisses : Diocèses de Perpignan, Tulle, Marseille, Rodez, Saint-Flour, Toulouse, Le Mans, Limoges, Lisieux, Rouen, Reims, Noyon. En tout 5 394 religieuses dans 198 maisons. La proportion donne environ 37 000 religieuses en 1 500 maisons pour les 38 000 paroisses de la France  Ainsi le total du clergé régulier est de 60 000 personnes  Pour le clergé séculier, on peut l’évaluer à 70 000 : curés et vicaires, 60 000 (l’abbé Guettée, {\itshape Histoire de !Église de France}, XII, 142) ; prélats, vicaires généraux, chanoines des chapitres, 2 800 ; chanoines des collégiales, 5 600 ; ecclésiastiques sans bénéfice, 3 000 (Siéyès)  Moheau, très bon esprit et statisticien prudent, écrit en 1778 ({\itshape Recherches}, 100) : « Peut-être n’existe-t-il pas aujourd’hui dans le royaume 130 000 ecclésiastiques »  Le dénombrement de 1866 ({\itshape Statistique de la France, population}) donne maintenant 51 100 membres du clergé séculier, 18 500 religieux, 86 300 religieuses ; total, 155 900 pour une population de 38 millions d’habitants.

\subsection[{Note 2. Livre premier, chapitre II, IV. Sur les droits féodaux et sur l’état d’un domaine féodal en 1783.}]{Note 2. \\
Livre premier, chapitre II, IV. \\
Sur les droits féodaux et sur l’état d’un domaine féodal en 1783.}

\noindent Les renseignements qui suivent sont extraits d’un acte de partage et d’estimation dressé le 6 septembre 1783, et dont je dois la communication à l’obligeance de M. de Boislisle.\par
Il s’agit des terres de Blet et des Brosses  La terre et baronnie de Blet est située dans le Bourbonnais, à deux lieues de Dun-le-Roi  Blet, dit un mémoire de l’administration des aides, est une « bonne paroisse sans être d’objet ; bonnes terres, la plus grande partie en bois, foins et pacages, le surplus en terres labourables de froment, seigle et avoine… Chemins affreux et à périr en hiver. Le commerce en faveur est celui des bêtes à cornes, il s’étend aussi sur les grains ; les bois pourrissent sur pied, par leur éloignement des villes et leur difficile exploitation\footnote{{\itshape Archives nationales}, G, 319 ({\itshape État actuel de la Direction de Bourges au point de vue des aides}, 1774).} ».\par
« Cette terre, dit l’acte estimatif, est dans la mouvance du roi, à cause de son château et forteresse d’Ainay, sous la dénomination de ville de Blet. » La ville était fortifiée autrefois, et son château fort subsiste encore. Elle fut jadis très peuplée, « mais les guerres civiles du seizième siècle et surtout l’émigration des protestants l’ont rendue déserte, au point que, de 3 000 habitants qu’elle renfermait, il s’en trouve actuellement à peine 300\footnote{Aujourd’hui Blet renferme 1629 habitants.} ; c’est le sort de toutes les villes du pays ». La terre de Blet, possédée pendant plusieurs siècles par la maison de Sully, passa par mariage de l’héritière, en 1363, à la maison de Saint-Quentin, où elle fut transmise en ligne directe jusqu’en 1748, date de la mort d’Alexandre II de Saint-Quentin, comte de Blet, gouverneur de Berg-op-Zoom, père de trois filles d’où sont nés les héritiers actuels  Ces héritiers sont le comte de Simiane, le chevalier de Simiane, et les mineurs de Bercy, chacun pour un tiers, qui est de 97 667 livres sur la terre de Blet, et de 20 408 livres sur la terre des Brosses. L’aîné, comte de Simiane, reçoit en outre un préciput (selon la coutume du Bourbonnais) évalué à 15 000 livres, comprenant le château avec la ferme attenante et les droits seigneuriaux, tant honorifiques qu’utiles.\par
\par
Le domaine entier, comprenant les deux terres, est évalué 369 227 livres. La terre de Blet comprend 1 437 arpents, exploités par 7 fermiers, auxquels le propriétaire fournit des bestiaux estimés 13 781 livres. Ils payent ensemble au propriétaire 12 060 livres de fermage (outre quelques redevances en poulets et corvées). Un seul a une grosse ferme et paye 7 800 livres par an, les autres payent 1 300, 740, 640, 240 livres par an  La terre des Brosses comprend 515 arpents exploités par 2 fermiers, auxquels le propriétaire fournit des bestiaux estimés 3 750 livres ; ils payent ensemble au propriétaire 2 240 livres\footnote{En réalité, les fermes de Blet et des Brosses ne rapportent presque rien au propriétaire, puisque les dîmes et le champart (articles 22 et 23) sont compris dans le prix des baux.}  Toutes ces métairies sont pauvres ; une seule comprend deux chambres avec cheminées ; deux ou trois, une chambre avec une cheminée ; toutes les autres consistent en une cuisine, avec four extérieur, étables et granges. Des réparations sont urgentes pour tous les corps de ferme, sauf trois, « l’entretien en ayant été négligé depuis trente ans ». Il faudrait « écurer le bié du moulin et la rivière dont les débordements gâtent la grande prairie, réparer les chaussées des deux étangs, réparer l’église qui est à la charge du seigneur, et dont les couvertures notamment sont dans un état affreux, les eaux pénétrant à travers la voûte », réparer les chemins qui sont aussi à la charge du seigneur, et qui pendant l’hiver sont dans un état déplorable. « Il paraît qu’on ne s’est jamais occupé du rétablissement et réparation de ces chemins. » Le sol de la terre de Blet est excellent, mais il faudrait des saignées et fossés pour l’écoulement des eaux, sans quoi les bas-fonds continueront à ne produire que des mauvaises herbes. La négligence et l’abandon ont laissé leurs marques partout. Le château de Blet n’a pas été habité depuis 1748 ; aussi presque tous les meubles sont pourris et hors d’usage ; ils valaient 7 612 livres en 1748 ; ils ne sont plus estimés qu’à 1 000 livres. « Le moulin à eau occasionne presque autant de dépense qu’il produit de revenu. » — « On ne connaît point l’usage de la chaux pour l’engrais des terres labourables », et pourtant « dans le pays la chaux est à vil prix ». La terre, humide et très bonne, produirait à volonté des haies vives ; pourtant on clôt les champs avec des haies sèches contre les bestiaux « et cette charge, suivant le rapport des fermiers, est évaluée au tiers du produit des fonds »  Ce domaine, tel qu’on vient de le décrire, est évalué comme il suit :\par
1. La terre de Blet, suivant l’usage du pays pour les terres nobles, est évaluée au denier 25, c’est-à-dire 373 060 livres, dont il faut défalquer un capital de 65 056 livres représentant les charges annuelles (portion congrue du cure, réparations, etc.), non comprises les charges personnelles comme les vingtièmes. Elle rapporte net par an 12 300 livres, et vaut net 308 003 livres.\par
2. La terre des Brosses est, suivant l’usage du pays, évaluée au denier 22, car elle cesse d’être noble par le transport des droits de fief et justice à celle de Blet. Sur ce pied elle vaut 73 583 livres, dont il faut défalquer un capital de 12 359 livres pour les charges réelles ; elle rapporte net par an 3 140 livres, et vaut net 61 224 livres.\par
Ces revenus ont les sources suivantes :\par
En premier lieu, les fermages ci-dessus énoncés  En second lieu les droits féodaux que l’on va énumérer.\par
Droits utiles et honorifiques de la terre de Blet :\par
1° Droit de haute, basse et moyenne justice sur toute la terre de Blet et autres villages, les Brosses, Jalay. Le haut, justicier, selon l’acte de notoriété donné au Châtelet, le 29 avril 1702, « connaît de toutes les matières réelles et personnelles, civiles et criminelles, même des actions des nobles et ecclésiastiques, des scellés et inventaires de meubles et effets, des tutelles, curatelles, administration des biens de mineurs, des domaines, droits et revenus usuels de la seigneurie, etc. »\par
2° Droit de gruerie, édit de 1707. Le gruyer du seigneur juge de toutes les affaires concernant les eaux et forêts, usages, délits, pêche et chasse.\par
3° Droit de voirie, ou police des rues, chemins, édifices (sauf les grands chemins). Le seigneur nomme un bailli gruyer et voyer, qui est M. Theurault (à Sagonne), un procureur fiscal, Baujard (à Blet) ; il peut les destituer « attendu qu’ils ne payent point de finance »  « Les droits de greffe étaient ci-devant affermés au profit du seigneur ; mais actuellement qu’il est très difficile de rencontrer des personnes intelligentes dans le pays pour remplir cette charge, le seigneur abandonne ses droits à celui qu’il commet. » (Le seigneur paye 48 livres par an au bailli pour tenir son audience une fois par mois, et 24 livres au procureur fiscal pour y assister.)\par
Il perçoit les amendes et confiscations de bestiaux prononcées par ses officiers. Ce profit, année moyenne, est de 8 livres.\par
Il doit entretenir une prison et un geôlier. (On ne dit pas qu’il y en ait une.) — Il ne se trouve plus dans la seigneurie aucune marque extérieure de fourches patibulaires.\par
Il peut nommer 12 notaires ; de fait il n’y en a qu’un, à Blet ; « encore n’est-il pas occupé », c’est Baujard, procureur fiscal. Cette commission lui est accordée gratuitement, pour maintenir le droit ; « d’ailleurs il serait impossible de rencontrer sur le lieu une personne intelligente pour la remplir ».\par
Il nomme un sergent ; mais depuis longtemps ce sergent ne paye aucun fermage ni loyer.\par
4° Taille personnelle et réelle. En Bourbonnais, jadis la taille était serve et les serfs mainmortables. « Les seigneurs, qui ont encore des droits de bordelage bien établis dans l’étendue de leurs fiefs et justices, sont encore aujourd’hui en possession de succéder à leurs vassaux dans tous les cas, même au préjudice de leurs enfants, si ceux-ci n’étaient résidants avec eux et n’habitaient le même toit. » Mais en 1255, Hodes de Sully, ayant donné une charte, renonça à ce droit de taille réelle et personnelle moyennant un droit de bourgeoisie perçu encore aujourd’hui (voyez plus loin).\par
5° Droit d’épave, sur les bestiaux, meubles, effets, essaims de mouches à miel perdus, trésors trouvés (depuis vingt ans profits nuls sur cet article).\par
6° Droit sur les biens des personnes décédées sans héritiers, des bâtards et aubains décédés, sur les biens des condamnés à mort, aux galères perpétuelles, des bannis, etc. (profits nuls).\par
7° Droit de chasse et de pêche, le second évalué 15 livres par an.\par
8° Droit de bourgeoisie (voy. article 4) d’après la charte de 1255, et le terrier de 1484. — Les plus riches doivent payer par an chacun 12 boisseaux d’avoine de 40 livres et 12 deniers parisis ; les moyens, 9 boisseaux et 9 deniers ; tous les autres, 6 boisseaux et 6 deniers. « Ces droits de bourgeoisie sont bien établis, énoncés dans tous les terriers et aveux rendus au roi et perpétués par une infinité de reconnaissances : on ne peut pénétrer les motifs qui ont engagé les anciens régisseurs ou fermiers de cette terre à en interrompre la perception. Quantité de seigneurs, en Bourbonnais, jouissent et font payer de pareils droits à leurs vassaux en vertu de titres qui pourraient être plus suspectés que ceux qui sont en la disposition des seigneurs de Blet. »\par
9° Droit de guet du château de Blet. Édit du roi de 1497 fixant cette charge, pour les habitants de Blet et tous ceux demeurant dans l’étendue de la justice, pour ceux de Charly, Boismarvier, etc., à 5 sous par feu et par an, ce qui fut exécuté. « Ce n’est que depuis peu qu’on en a cessé la perception, quoique, par les reconnaissances modernes, tous les habitants se soient reconnus sujets auxdits guet et garde du château. »\par
10° Droit de péage pour toutes les marchandises et denrées qui passent par la ville de Blet, sauf les blés, grains, farines et légumes. (Affaire pendante devant le conseil d’État depuis 1727 jusqu’à 1745 et non terminée ; « la perception en a été interrompue dans ce même temps »).\par
11° Droit de potage sur les vins vendus en détail à Blet, attribuant au seigneur 9 pintes de vin par tonneau, affermé en 1782 pour 6 ans, moyennant 60 livres par an.\par
12° Droit de boucherie ou de prendre la langue de toutes les bêtes tuées dans la ville, plus la tête et les pieds de tous les veaux. Pas de boucher à Blet ; cependant, « dans le temps de la moisson et pendant le cours de chaque année, on massacre environ 12 bœufs ». Ce droit est perçu par le régisseur : il est évalué à 3 livres par an.\par
13° Droit sur les foires et marchés, aunage, poids et mesures. Cinq foires par an et un marché par semaine, mais peu fréquentés ; pas de halle. Le droit est évalué à 24 livres par an.\par
14° Corvées de charrois et à bras, par droit de seigneur haut justicier sur 97 personnes à Blet (22 corvées de voitures et 75 corvées à bras), sur 26 personnes aux Brosses (5 corvées de voitures et 21 à bras). Le seigneur paye 6 sous de nourriture pour la corvée à bras et 12 sous de nourriture pour la corvée de voiture à 4 bœufs. « Dans le nombre des corvéables, il s’en trouve la plus grande partie réduite presque à la mendicité et chargée d’une famille nombreuse, ce qui détermine souvent le seigneur à ne point les exiger à la rigueur. » Valeur ainsi réduite des corvées, 49 livres 15 sols.\par
15° Banalité de moulins (sentence de 1736 condamnant Roy, laboureur, à moudre ses grains au moulin de Blet et à l’amende pour avoir cessé d’y moudre depuis trois ans). Le meunier perçoit un seizième de la farine moulue. Le moulin banal, ainsi que celui à vent et 6 arpents adjoints, sont affermés 600 livres par an.\par
16° Banalité de four. Transaction de 1537 entre le seigneur et ses vassaux : il leur accorde d’avoir dans leur maison un petit four de trois carreaux, d’un demi-pied chaque, pour y cuire pâtés, galettes et tourteaux ; d’autre part ils se reconnaissent sujets à la banalité. Il peut percevoir un seizième de la pâte ; ce droit pourrait rapporter 150 livres annuellement ; mais, depuis quelques années, la maison du four est effondrée.\par
17° Droit de colombier ; il y en a un dans le parc du château.\par
18° Droit de bordelage (le seigneur est héritier, sauf lorsque les enfants du mort vivaient avec le mort au moment du décès). Le seigneur de Blet a ce droit sur 48 arpents. Depuis 20 ans, par négligence ou autrement, il n’en a rien tiré.\par
19° Droit sur les terres incultes et désertes et sur les accrues par alluvion.\par
20° Droit purement honorifique de banc et sépulture au chœur, d’encens et de prière nominale, de litre et ceinture funèbre intérieure et extérieure.\par
21° Droit de lods et ventes sur les censitaires, dû par l’acquéreur d’un immeuble censitaire au seigneur, dans les 40 jours. « En Bourbonnais, les lods et ventes se perçoivent au tiers, au quart, au 6\textsuperscript{e}, 8\textsuperscript{e} et 12\textsuperscript{e} denier. » Le seigneur de Blet et Brosses les perçoit au 6\textsuperscript{e} denier. On estime que les ventes se font une fois tous les 80 ans ; ces droits portent sur 1 356 arpents qui valent, les meilleurs 192 livres l’arpent, les moyens 110 livres, les mauvais 75 livres. À ce taux, les 1 350 arpents valent 162 750 livres. — On fait remise aux acquéreurs du quart des lods et ventes. — Rapport annuel de ce droit, 254 livres.\par
22° Droit de dîmes et charnage. Le seigneur a acquis toutes les dîmes, sauf quelques-unes aux chanoines de Dun-le-Roi et au prieur de Chaumont. Les dîmes se lèvent à la 13\textsuperscript{e} gerbe ; elles sont comprises dans les baux.\par
23° Droit de terrage ou champart : c’est le droit de percevoir, après que les dîmes sont levées, une portion des fruits de la terre. « En Bourbonnais, le terrage se perçoit de différentes manières, à la 3\textsuperscript{e} gerbe, à la 5\textsuperscript{e}, 6\textsuperscript{e}, 7\textsuperscript{e}, et communément au quart ; à Blet, c’est à la 12\textsuperscript{e}. » Le seigneur de Blet ne perçoit le terrage que sur un certain nombre de terres de sa seigneurie ; « par rapport aux Brosses, il paraît que tous les domaines possédés par les censitaires sont assujettis à ce droit ». — Ces droits de terrage sont compris dans les baux des fermes de Blet et des Brosses.\par
24° Cens, surcens et rentes dus par des immeubles de diverses sortes, maisons, champs, prés, etc., situés sur le territoire de la seigneurie.\par
Sur la seigneurie de Blet, 810 arpents, divisés en 511 parcelles, aux mains de 120 censitaires, sont dans ce cas, et leur cens total annuel consiste en 137 francs d’argent, 67 boisselées de froment, 3 d’orge, 159 d’avoine, 16 gelines, 130 poules, 6 coqs et chapons ; le total est évalué 575 francs.\par
Sur la terre des Brosses, 85 arpents, divisés en 112 parcelles, aux mains de 20 censitaires, sont dans ce cas, et leur cens total annuel est de 14 francs d’argent, 17 boisselées de froment, 32 d’orge, 26 gelines, 3 poules et 1 chapon. Le total est évalué 126 francs.\par
25° Droits sur les communaux (124 arpents dans la terre de Blet, 164 dans la terre des Brosses).\par
Les vassaux n’ont sur les communaux qu’un droit d’usage. « La presque totalité des fonds sur lesquels ils usent du droit de pâturage appartiennent en propriété aux seigneurs, fors ce droit d’usage dont ils sont grevés ; encore n’est-il accordé qu’à quelques particuliers. »\par
26° Droits sur les fiefs mouvants de la baronnie de Blet.\par
Les uns sont situés dans le Bourbonnais, et il y en a 19 dans ce cas. En Bourbonnais, les fiefs, même possédés par des roturiers, ne doivent au seigneur, à chaque mutation, que la bouche et les mains. Jadis le seigneur de Blet percevait dans cette circonstance le droit de rachapt, mais on l’a laissé tomber en désuétude.\par
Les autres sont situés dans le Berry, où s’exerce le droit de rachapt. Il n’y a qu’un fief dans le Berry, celui de Cormesse, à l’archevêque de Bourges, comprenant 85 arpents, outre une portion de dîmes, et rapportant par an 2 100 livres, ce qui, en admettant une mutation tous les vingt ans, donne annuellement au seigneur de Blet 105 livres.\par
Outre les charges indiquées, il y a les charges suivantes :\par
1° Au curé de Blet, sa portion congrue. D’après la déclaration du roi de 1686, elle devait être de 300 livres. Par transaction en 1692, le curé, voulant s’assurer cette portion congrue, céda au seigneur toutes les dîmes, novales, etc. — L’édit de 1768 ayant fixé la portion congrue à 500 livres, le curé réclama cette somme par exploit. Les chanoines de Dun-le-Roi et le prieur de Chaumont, ayant des dîmes sur le territoire de Blet, devraient en payer une partie. Actuellement elle est toute à la charge du seigneur de Blet.\par
2° Au garde, outre son logement, son chauffage et la jouissance de 3 arpents de friches, 200 livres.\par
3° Au régisseur, pour garder les archives, veiller aux réparations, percevoir les lods et ventes, percevoir les amendes, 432 livres, outre la jouissance de dix arpents de friches.\par
4° Au roi l’impôt des vingtièmes. Précédemment, les terres de Blet et des Brosses payaient 810 livres pour les deux vingtièmes et les deux sous pour livre. Depuis l’établissement du troisième vingtième, elles payent 1 216 livres.

\subsection[{Note 3. Livre premier, chapitre III, III. Différence du revenu réel et du revenu nominal des dignités. et bénéfices ecclésiastiques.}]{Note 3. Livre premier, chapitre III, III. Différence du revenu réel et du revenu nominal des dignités\\
et bénéfices ecclésiastiques.}

\noindent Selon Raudot ({\itshape La France avant la Révolution}, 84), il faut ajouter moitié en sus à l’évaluation officielle ; selon Boiteau ({\itshape État de la France en} 1789, 195), il faut la tripler et même la quadrupler. — Je pense que, pour les sièges épiscopaux, il faut ajouter moitié en sus, et que, pour les abbayes et prieurés, il faut doubler, parfois tripler ou même quadrupler. Voici des faits qui pourront montrer l’écart des chiffres officiels et des chiffres réels :\par
1° Dans l’{\itshape Almanach Royal}, l’évêché de Troyes est évalué 14 000 livres ; dans la {\itshape France ecclésiastique} de 1788, 50 000. D’après Albert Babeau ({\itshape Histoire de la Révolution dans le département de l’Aube}), il rapporte 70 000 livres.\par
Dans {\itshape la France ecclésiastique}, l’évêché de Strasbourg est évalué 400 000 livres. Selon le duc de Lévis ({\itshape Souvenirs}, 156), il rapporte au moins 600 000 livres de rente.\par
2° Dans {\itshape la France ecclésiastique}, l’abbaye de Jumièges est portée pour 23 000 livres. J’ai trouvé dans les papiers du comité ecclésiastique que, selon les moines, elle rapporte à l’abbé 50 000 livres.\par
Dans {\itshape la France ecclésiastique}, l’abbaye de Bèze est évaluée 8 000 livres. Je trouve qu’elle rapporte aux moines seuls 30 000 livres, et la part de l’abbé est toujours au moins égale ({\itshape De l’État religieux}, par les abbés Bonnefoi et Bernard, 1784). Elle rapporte donc à l’abbé 30 000 livres.\par
Bernay (Eure) est porté officiellement à 16 000. Les doléances des cahiers l’estiment à 57 000.\par
Saint-Amand, au cardinal d’York, est marqué 6 000 livres et en rapporte 100 000. (Duc de Luynes, XIII, 215.)\par
Clairvaux est porté dans {\itshape la France ecclésiastique} à 9 000, et dans Waroquier ({\itshape État général de la France en} 1789) à 60 000  D’après Beugnot, qui est du pays et homme d’affaires, l’abbé a de 300 à 400 000 livres de rente.\par
Saint-Faron, dit Boiteau, marqué 18 000 livres, en vaut 120 000.\par
L’abbaye de Saint-Germain des Prés (aux économats) est marquée 100 000 livres. Le comte de Clermont, qui l’avait auparavant, l’affermait 160 000 livres, « sans compter des prés réservés et tout ce que les fermiers fournissaient de paille et d’avoine pour ses chevaux ». (Jules Cousin, {\itshape le Comte de Clermont et sa cour.})\par
Saint-Waast d’Arras, selon {\itshape la France ecclésiastique}, rapporte 40 000 livres. Le cardinal de Rohan en a refusé 1 000 louis par mois que les moines lui offraient pour sa part. (Duc de Lévis, {\itshape Souvenirs}, 156.) Elle vaut donc environ 300 000 livres\par
Remiremont, dont l’abbesse est toujours une princesse du sang royal, l’un des monastères les plus puissants, les plus riches, les plus amplement dotés, est évalué officiellement au chiffre ridicule de 15 000 livres.

\subsection[{Note 4. Livre deuxième, chapitre I, VI. Sur l’éducation des princes et princesses.}]{Note 4. \\
Livre deuxième, chapitre I, VI. \\
Sur l’éducation des princes et princesses.}

\noindent Ce sujet pourrait à lui seul occuper un chapitre à part ; je citerai seulement quelques textes.\par

\begin{quoteblock}
 
\bibl{(Barbier, {\itshape Journal}, octobre 1750.) — La dauphine vient d’accoucher d’une fille.}
 \noindent « La jeune princesse en est à sa quatrième nourrice… J’ai appris à cette occasion que tout se fait par forme à la cour, suivant un protocole de médecin, en sorte que c’est un miracle d’élever un prince et une princesse. La nourrice n’a d’autres fonctions que de donner à téter à l’enfant quand on le lui apporte ; elle ne peut pas lui toucher. Il y a des remueuses et femmes préposées pour cela, mais qui n’ont point d’ordre à recevoir de la nourrice. Il y a des heures pour remuer l’enfant, trois ou quatre fois dans la journée. Si l’enfant dort, on le réveille pour le remuer. Si, après avoir été changé, il fait dans ses langes, il reste ainsi trois ou quatre heures dans son ordure. Si une épingle le pique, la nourrice ne doit pas l’ôter ; il faut chercher et attendre une autre femme ; l’enfant crie dans tous ces cas, il se tourmente et s’échauffe, en sorte que c’est une vraie misère que toutes ces cérémonies. »
\end{quoteblock}


\begin{quoteblock}
 
\bibl{(Mme de Genlis, {\itshape Souvenirs de Félicie}, 74. Conversation avec Madame Louise, fille de Louis XV, devenue carmélite.)}
 \noindent « Je désirai savoir quelle est la chose à laquelle, dans son nouvel état, elle avait eu le plus de peine à s’accoutumer  « Vous ne le devineriez jamais, a-t-elle répondu en souriant ; c’est de descendre seule un petit escalier. Dans les commencements, c’était pour moi le précipice le plus effrayant, j’étais obligée de m’asseoir sur les marches et de me traîner, dans cette attitude, pour descendre. » — En effet, une princesse qui n’avait descendu que le grand escalier de Versailles en s’appuyant sur le bras de son chevalier d’honneur, et entourée de ses pages, a dû frémir en se trouvant livrée à elle-même sur le bord d’un escalier bien raide en colimaçon. (Telle est) l’éducation ridicule à tant d’égards que reçoivent en général les personnes de ce rang ; dès leur enfance, toujours suivies, aidées, escortées, prévenues, (elles) sont ainsi privées de la plus grande partie des facultés que leur a données la nature. »
\end{quoteblock}


\begin{quoteblock}
 
\bibl{(Mme Campan, {\itshape Mémoires}, I, 18, 28.)}
 \noindent « Madame Louise m’a souvent répété qu’à l’âge de douze ans elle n’avait point encore parcouru la totalité de son alphabet…\par
 Il s’agissait de décider irrévocablement si un oiseau d’eau était maigre ou gras. Madame Victoire consulta un évêque… Celui-ci répondit qu’en un semblable doute, après avoir fait cuire l’oiseau, il fallait le piquer sur un plat d’argent très froid, que, si le jus de l’animal se figeait dans l’espace d’un quart d’heure, l’animal était réputé gras  Madame Victoire fit aussitôt faire l’épreuve ; ce jus ne figea point. Ce fut une joie pour la princesse qui aimait beaucoup cette espèce de gibier  Le maigre, qui occupait tant Madame Victoire, l’incommodait ; aussi attendait-elle avec impatience le coup de minuit du samedi saint. On lui servait aussitôt une bonne volaille au riz et plusieurs autres mets succulents. »
\end{quoteblock}


\begin{quoteblock}
 
\bibl{({\itshape Journal} de Dumont d’Urville, commandant du navire sur lequel Charles X quitta la France en 1830. Cité par Vaulabelle, {\itshape Histoire de la Restauration}, VIII, 465.)}
 \noindent « Le roi et le duc d’Angoulême m’interrogèrent sur mes différentes campagnes, mais surtout sur mon voyage de circumnavigation à bord de {\itshape l’Astrolabe.} Mon récit paraissait vivement les intéresser, et, s’ils m’interrompaient, c’était pour m’adresser des questions d’une remarquable naïveté et qui prouvaient que, dépourvus de toute notion, même superficielle, sur les sciences et les voyages, ils étaient aussi ignorants sur ces matières que pouvaient l’être de vieux rentiers au Marais. »
\end{quoteblock}


\subsection[{Note 5. Livre cinquième, chapitre II, III. Sur le chiffre de l’impôt direct.}]{Note 5. \\
Livre cinquième, chapitre II, III. \\
Sur le chiffre de l’impôt direct.}

\noindent Les chiffres suivants sont extraits des procès-verbaux des assemblées provinciales (1778-1787) :\par

\tableopen{}
\begin{tabularx}{\linewidth}
{|l|X|X|X|X|X|}
\hline & \Panel{Taille}{label}{1}{l} & \Panel{Accessoires de la Taille}{label}{1}{l} & \Panel{Capitation taillable}{label}{1}{l} & \Panel{Impôt des routes}{label}{1}{l} & \Panel{Total en multiples de la Taille}{label}{1}{l} \\
\hline
\Panel{Ile-de-France}{label}{1}{l} & 4,296,040 & 2,207,826 & 2,689,287 & 519,989 & 2,23 \\
\hline
\Panel{Lyonnais}{label}{1}{l} & 1,356,954 & 903,653 & 898,089 & 315,869 & 2,61 \\
\hline
\Panel{Généralité de Rouen}{label}{1}{l} & 2,671,939 & 1,595,051 & 1,715,592 & 598,258\footnote{Ce chiffre n’est pas donné par l’assemblée provinciale ; pour suppléer à cette lacune, j’ai pris le dixième de la taille, des accessoires et de la capitation taillable ; c’est le procédé que suit l’assemblée provinciale du Lyonnais. Par la déclaration du 2 juin 1787, l’impôt des routes peut être porté au sixième des trois précédents ; ordinairement il est du dixième ou, par rapport au principal de la taille, du quart.} & 2,46 \\
\hline
\Panel{Génér. de Caen}{label}{1}{l} & 1,939,665 & 1,212,429 & 1,187,823 & 659,034 & 2,56 \\
\hline
\Panel{Berry}{label}{1}{l} & 821,921 & 448,431 & 464,955 & 236,900 & 2,50 \\
\hline
\Panel{Poitou}{label}{1}{l} & 2,309,681 & 1,113,766 & 1,403,402 & 520,000 & 2,30 \\
\hline
\Panel{Soissonnais}{label}{1}{l} & 1,062,392 & 911,883 & 734,899 & 462,883 & 2,94 \\
\hline
\Panel{Orléanais}{label}{1}{l} & 2,353,892 & 1,256,125 & 1,485,720 & 586,385 & 2,34 \\
\hline
\Panel{Champagne}{label}{1}{l} & 1,783,850 & 1,459,780 & 1,377,371 & 807,280 & 3 \\
\hline
\Panel{Génér. d’Alençon}{label}{1}{l} & 1,742,655 & 1,120,041 & 1,067,849 & 435,637 & 2,47 \\
\hline
\Panel{Auvergne}{label}{1}{l} & 1,999,040 & 1,399,678 & 1,753,026 & 310,468 & 2,70 \\
\hline
\Panel{Génér. d’Auch}{label}{1}{l} & 1,440,533 & 931,261 & 797,268 & 316,909\footnote{Même remarque.} & 2,35 \\
\hline
\Panel{Haute-Guyenne}{label}{1}{l} & 2,131,314 & 1,267,619 & 1,268,855 & 308,993\footnote{L’assemblée provinciale porte ce chiffre au onzième de la taille et accessoires réunis.} & 2,47 \\
\hline
\end{tabularx}
\tableclose{}

\noindent La taille en principal étant 1, les chiffres de la dernière colonne représentent, pour chaque province, le total des quatre impositions par rapport à la taille  La moyenne entre tous ces chiffres est 2,53. Or les accessoires de la taille, la capitation et l’impôt des routes sont fixés pour chaque taillable au prorata de sa taille. Il ne reste donc plus qu’à multiplier par 2,53 le chiffre qui représentera la part que la taille prélève sur le revenu net, pour savoir ce que les quatre impôts mis ensemble prélèvent sur ce revenu.\par
Cette part varie de province à province, de paroisse à paroisse, et même d’individu à individu. Néanmoins on peut estimer que la taille prélève en moyenne, surtout quand elle s’attaque au paysan petit propriétaire, dépourvu de protection et de crédit, un sixième du revenu net, soit 16 fr. 66 c. sur 100 fr  Par exemple, d’après les déclarations des assemblées provinciales, en Champagne elle prélève 3 sous et 2/3 de denier par livre, ou 15 fr. 28 c. sur 100 ; dans l’Ile-de-France, 35 livres 14 sous sur 240 livres ou 14 fr. 87 sur 100 ; en Auvergne, 4 sous par livre du revenu net, c’est-à-dire 20 pour 100. Enfin, dans la généralité d’Auch, l’assemblée provinciale estime que la taille et les accessoires prélèvent les trois dixièmes du produit net, d’où l’on peut voir, en prenant les chiffres du budget de la province, que la taille seule prélève 18 fr. 10 c. sur 100 fr. de revenu.\par
Cela posé, si la taille en principal prélève un sixième du revenu net du taillable, c’est-à-dire 16 fr. 66 c. sur 100, le total des quatre impôts ci-dessus prélève 16 fr. 66 c. X 2,53 = 42 fr. 15 c. sur 100 francs de revenu. À quoi il faut ajouter 11 fr. pour les deux vingtièmes et les 4 sous pour livre ajoutés au premier vingtième ; total, 53 fr. 15 c. d’impôt direct sur 100 livres de revenu taillable.\par
La dîme, étant évaluée au septième du revenu net, prélève en outre 14 fr. 28 c. — Les droits féodaux, étant évalués à la même somme, prélèvent aussi 14 fr. 28 c. ; total, 28 fr. 56 c.\par
Total général des prélèvements de l’impôt direct royal, de la dîme ecclésiastique et des droits féodaux, 81 fr. 71 c. sur 100 fr. de revenu net. — Reste au propriétaire taillable 18 fr. 29 c.
 


% at least one empty page at end (for booklet couv)
\ifbooklet
  \pagestyle{empty}
  \clearpage
  % 2 empty pages maybe needed for 4e cover
  \ifnum\modulo{\value{page}}{4}=0 \hbox{}\newpage\hbox{}\newpage\fi
  \ifnum\modulo{\value{page}}{4}=1 \hbox{}\newpage\hbox{}\newpage\fi


  \hbox{}\newpage
  \ifodd\value{page}\hbox{}\newpage\fi
  {\centering\color{rubric}\bfseries\noindent\large
    Hurlus ? Qu’est-ce.\par
    \bigskip
  }
  \noindent Des bouquinistes électroniques, pour du texte libre à participations libres,
  téléchargeable gratuitement sur \href{https://hurlus.fr}{\dotuline{hurlus.fr}}.\par
  \bigskip
  \noindent Cette brochure a été produite par des éditeurs bénévoles.
  Elle n’est pas faite pour être possédée, mais pour être lue, et puis donnée.
  En page de garde, on peut ajouter une date, un lieu, un nom ;
  comme une fiche de bibliothèque en papier,
  pour suivre le voyage du texte. Qui sait, un jour, il vous reviendra ?
  \par

  Ce texte a été choisi parce qu’une personne l’a aimé,
  ou haï, elle a pensé qu’il partipait à la formation de notre présent ;
  sans le souci de plaire, vendre, ou militer pour une cause.
  \par

  L’édition électronique est soigneuse, tant sur la technique
  que sur l’établissement du texte ; mais sans aucune prétention scolaire, au contraire.
  Le but est de s’adresser à tous, sans distinction de science ou de diplôme.
  \par

  Cet exemplaire en papier a été tiré sur une imprimante personnelle
   ou une photocopieuse. Tout le monde peut le faire.
  Il suffit de
  télécharger un fichier sur \href{https://hurlus.fr}{\dotuline{hurlus.fr}},
  d’imprimer, et agrafer (puis lire et donner).\par

  \bigskip

  \noindent PS : Les hurlus furent aussi des rebelles protestants qui cassaient les statues dans les églises catholiques. En 1566 démarra la révolte des gueux dans le pays de Lille. L’insurrection enflamma la région jusqu’à Anvers où les gueux de mer bloquèrent les bateaux espagnols.
  Ce fut une rare guerre de libération dont naquit un pays toujours libre : les Pays-Bas.
  En plat pays francophone, par contre, restèrent des bandes de huguenots, les hurlus, progressivement réprimés par la très catholique Espagne.
  Cette mémoire d’une défaite est éteinte, rallumons-la. Sortons les livres du culte universitaire, débusquons les idoles de l’époque, pour les démonter.
\fi

\end{document}
