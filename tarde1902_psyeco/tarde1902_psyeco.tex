%%%%%%%%%%%%%%%%%%%%%%%%%%%%%%%%%
% LaTeX model https://hurlus.fr %
%%%%%%%%%%%%%%%%%%%%%%%%%%%%%%%%%

% Needed before document class
\RequirePackage{pdftexcmds} % needed for tests expressions
\RequirePackage{fix-cm} % correct units

% Define mode
\def\mode{a4}

\newif\ifaiv % a4
\newif\ifav % a5
\newif\ifbooklet % booklet
\newif\ifcover % cover for booklet

\ifnum \strcmp{\mode}{cover}=0
  \covertrue
\else\ifnum \strcmp{\mode}{booklet}=0
  \booklettrue
\else\ifnum \strcmp{\mode}{a5}=0
  \avtrue
\else
  \aivtrue
\fi\fi\fi

\ifbooklet % do not enclose with {}
  \documentclass[french,twoside]{book} % ,notitlepage
  \usepackage[%
    papersize={105mm, 297mm},
    inner=12mm,
    outer=12mm,
    top=20mm,
    bottom=15mm,
    marginparsep=0pt,
  ]{geometry}
  \usepackage[fontsize=9.5pt]{scrextend} % for Roboto
\else\ifav
  \documentclass[french,twoside]{book} % ,notitlepage
  \usepackage[%
    a5paper,
    inner=25mm,
    outer=15mm,
    top=15mm,
    bottom=15mm,
    marginparsep=0pt,
  ]{geometry}
  \usepackage[fontsize=12pt]{scrextend}
\else% A4 2 cols
  \documentclass[twocolumn]{report}
  \usepackage[%
    a4paper,
    inner=15mm,
    outer=10mm,
    top=25mm,
    bottom=18mm,
    marginparsep=0pt,
  ]{geometry}
  \setlength{\columnsep}{20mm}
  \usepackage[fontsize=9.5pt]{scrextend}
\fi\fi

%%%%%%%%%%%%%%
% Alignments %
%%%%%%%%%%%%%%
% before teinte macros

\setlength{\arrayrulewidth}{0.2pt}
\setlength{\columnseprule}{\arrayrulewidth} % twocol
\setlength{\parskip}{0pt} % classical para with no margin
\setlength{\parindent}{1.5em}

%%%%%%%%%%
% Colors %
%%%%%%%%%%
% before Teinte macros

\usepackage[dvipsnames]{xcolor}
\definecolor{rubric}{HTML}{800000} % the tonic 0c71c3
\def\columnseprulecolor{\color{rubric}}
\colorlet{borderline}{rubric!30!} % definecolor need exact code
\definecolor{shadecolor}{gray}{0.95}
\definecolor{bghi}{gray}{0.5}

%%%%%%%%%%%%%%%%%
% Teinte macros %
%%%%%%%%%%%%%%%%%
%%%%%%%%%%%%%%%%%%%%%%%%%%%%%%%%%%%%%%%%%%%%%%%%%%%
% <TEI> generic (LaTeX names generated by Teinte) %
%%%%%%%%%%%%%%%%%%%%%%%%%%%%%%%%%%%%%%%%%%%%%%%%%%%
% This template is inserted in a specific design
% It is XeLaTeX and otf fonts

\makeatletter % <@@@


\usepackage{blindtext} % generate text for testing
\usepackage[strict]{changepage} % for modulo 4
\usepackage{contour} % rounding words
\usepackage[nodayofweek]{datetime}
% \usepackage{DejaVuSans} % seems buggy for sffont font for symbols
\usepackage{enumitem} % <list>
\usepackage{etoolbox} % patch commands
\usepackage{fancyvrb}
\usepackage{fancyhdr}
\usepackage{float}
\usepackage{fontspec} % XeLaTeX mandatory for fonts
\usepackage{footnote} % used to capture notes in minipage (ex: quote)
\usepackage{framed} % bordering correct with footnote hack
\usepackage{graphicx}
\usepackage{lettrine} % drop caps
\usepackage{lipsum} % generate text for testing
\usepackage[framemethod=tikz,]{mdframed} % maybe used for frame with footnotes inside
\usepackage{pdftexcmds} % needed for tests expressions
\usepackage{polyglossia} % non-break space french punct, bug Warning: "Failed to patch part"
\usepackage[%
  indentfirst=false,
  vskip=1em,
  noorphanfirst=true,
  noorphanafter=true,
  leftmargin=\parindent,
  rightmargin=0pt,
]{quoting}
\usepackage{ragged2e}
\usepackage{setspace} % \setstretch for <quote>
\usepackage{tabularx} % <table>
\usepackage[explicit]{titlesec} % wear titles, !NO implicit
\usepackage{tikz} % ornaments
\usepackage{tocloft} % styling tocs
\usepackage[fit]{truncate} % used im runing titles
\usepackage{unicode-math}
\usepackage[normalem]{ulem} % breakable \uline, normalem is absolutely necessary to keep \emph
\usepackage{verse} % <l>
\usepackage{xcolor} % named colors
\usepackage{xparse} % @ifundefined
\XeTeXdefaultencoding "iso-8859-1" % bad encoding of xstring
\usepackage{xstring} % string tests
\XeTeXdefaultencoding "utf-8"
\PassOptionsToPackage{hyphens}{url} % before hyperref, which load url package

% TOTEST
% \usepackage{hypcap} % links in caption ?
% \usepackage{marginnote}
% TESTED
% \usepackage{background} % doesn’t work with xetek
% \usepackage{bookmark} % prefers the hyperref hack \phantomsection
% \usepackage[color, leftbars]{changebar} % 2 cols doc, impossible to keep bar left
% \usepackage[utf8x]{inputenc} % inputenc package ignored with utf8 based engines
% \usepackage[sfdefault,medium]{inter} % no small caps
% \usepackage{firamath} % choose firasans instead, firamath unavailable in Ubuntu 21-04
% \usepackage{flushend} % bad for last notes, supposed flush end of columns
% \usepackage[stable]{footmisc} % BAD for complex notes https://texfaq.org/FAQ-ftnsect
% \usepackage{helvet} % not for XeLaTeX
% \usepackage{multicol} % not compatible with too much packages (longtable, framed, memoir…)
% \usepackage[default,oldstyle,scale=0.95]{opensans} % no small caps
% \usepackage{sectsty} % \chapterfont OBSOLETE
% \usepackage{soul} % \ul for underline, OBSOLETE with XeTeX
% \usepackage[breakable]{tcolorbox} % text styling gone, footnote hack not kept with breakable


% Metadata inserted by a program, from the TEI source, for title page and runing heads
\title{\textbf{ Psychologie économique }}
\date{1902}
\author{Gabriel Tarde}
\def\elbibl{Gabriel Tarde. 1902. \emph{Psychologie économique}}
\def\elsource{ \href{http://efele.net/ebooks/livres/000286}{\dotuline{http://efele.net/ebooks/livres/000286}}\footnote{\href{http://efele.net/ebooks/livres/000286}{\url{http://efele.net/ebooks/livres/000286}}} }

% Default metas
\newcommand{\colorprovide}[2]{\@ifundefinedcolor{#1}{\colorlet{#1}{#2}}{}}
\colorprovide{rubric}{red}
\colorprovide{silver}{lightgray}
\@ifundefined{syms}{\newfontfamily\syms{DejaVu Sans}}{}
\newif\ifdev
\@ifundefined{elbibl}{% No meta defined, maybe dev mode
  \newcommand{\elbibl}{Titre court ?}
  \newcommand{\elbook}{Titre du livre source ?}
  \newcommand{\elabstract}{Résumé\par}
  \newcommand{\elurl}{http://oeuvres.github.io/elbook/2}
  \author{Éric Lœchien}
  \title{Un titre de test assez long pour vérifier le comportement d’une maquette}
  \date{1566}
  \devtrue
}{}
\let\eltitle\@title
\let\elauthor\@author
\let\eldate\@date


\defaultfontfeatures{
  % Mapping=tex-text, % no effect seen
  Scale=MatchLowercase,
  Ligatures={TeX,Common},
}


% generic typo commands
\newcommand{\astermono}{\medskip\centerline{\color{rubric}\large\selectfont{\syms ✻}}\medskip\par}%
\newcommand{\astertri}{\medskip\par\centerline{\color{rubric}\large\selectfont{\syms ✻\,✻\,✻}}\medskip\par}%
\newcommand{\asterism}{\bigskip\par\noindent\parbox{\linewidth}{\centering\color{rubric}\large{\syms ✻}\\{\syms ✻}\hskip 0.75em{\syms ✻}}\bigskip\par}%

% lists
\newlength{\listmod}
\setlength{\listmod}{\parindent}
\setlist{
  itemindent=!,
  listparindent=\listmod,
  labelsep=0.2\listmod,
  parsep=0pt,
  % topsep=0.2em, % default topsep is best
}
\setlist[itemize]{
  label=—,
  leftmargin=0pt,
  labelindent=1.2em,
  labelwidth=0pt,
}
\setlist[enumerate]{
  label={\bf\color{rubric}\arabic*.},
  labelindent=0.8\listmod,
  leftmargin=\listmod,
  labelwidth=0pt,
}
\newlist{listalpha}{enumerate}{1}
\setlist[listalpha]{
  label={\bf\color{rubric}\alph*.},
  leftmargin=0pt,
  labelindent=0.8\listmod,
  labelwidth=0pt,
}
\newcommand{\listhead}[1]{\hspace{-1\listmod}\emph{#1}}

\renewcommand{\hrulefill}{%
  \leavevmode\leaders\hrule height 0.2pt\hfill\kern\z@}

% General typo
\DeclareTextFontCommand{\textlarge}{\large}
\DeclareTextFontCommand{\textsmall}{\small}

% commands, inlines
\newcommand{\anchor}[1]{\Hy@raisedlink{\hypertarget{#1}{}}} % link to top of an anchor (not baseline)
\newcommand\abbr[1]{#1}
\newcommand{\autour}[1]{\tikz[baseline=(X.base)]\node [draw=rubric,thin,rectangle,inner sep=1.5pt, rounded corners=3pt] (X) {\color{rubric}#1};}
\newcommand\corr[1]{#1}
\newcommand{\ed}[1]{ {\color{silver}\sffamily\footnotesize (#1)} } % <milestone ed="1688"/>
\newcommand\expan[1]{#1}
\newcommand\foreign[1]{\emph{#1}}
\newcommand\gap[1]{#1}
\renewcommand{\LettrineFontHook}{\color{rubric}}
\newcommand{\initial}[2]{\lettrine[lines=2, loversize=0.3, lhang=0.3]{#1}{#2}}
\newcommand{\initialiv}[2]{%
  \let\oldLFH\LettrineFontHook
  % \renewcommand{\LettrineFontHook}{\color{rubric}\ttfamily}
  \IfSubStr{QJ’}{#1}{
    \lettrine[lines=4, lhang=0.2, loversize=-0.1, lraise=0.2]{\smash{#1}}{#2}
  }{\IfSubStr{É}{#1}{
    \lettrine[lines=4, lhang=0.2, loversize=-0, lraise=0]{\smash{#1}}{#2}
  }{\IfSubStr{ÀÂ}{#1}{
    \lettrine[lines=4, lhang=0.2, loversize=-0, lraise=0, slope=0.6em]{\smash{#1}}{#2}
  }{\IfSubStr{A}{#1}{
    \lettrine[lines=4, lhang=0.2, loversize=0.2, slope=0.6em]{\smash{#1}}{#2}
  }{\IfSubStr{V}{#1}{
    \lettrine[lines=4, lhang=0.2, loversize=0.2, slope=-0.5em]{\smash{#1}}{#2}
  }{
    \lettrine[lines=4, lhang=0.2, loversize=0.2]{\smash{#1}}{#2}
  }}}}}
  \let\LettrineFontHook\oldLFH
}
\newcommand{\labelchar}[1]{\textbf{\color{rubric} #1}}
\newcommand{\milestone}[1]{\autour{\footnotesize\color{rubric} #1}} % <milestone n="4"/>
\newcommand\name[1]{#1}
\newcommand\orig[1]{#1}
\newcommand\orgName[1]{#1}
\newcommand\persName[1]{#1}
\newcommand\placeName[1]{#1}
\newcommand{\pn}[1]{\IfSubStr{-—–¶}{#1}% <p n="3"/>
  {\noindent{\bfseries\color{rubric}   ¶  }}
  {{\footnotesize\autour{ #1}  }}}
\newcommand\reg{}
% \newcommand\ref{} % already defined
\newcommand\sic[1]{#1}
\newcommand\surname[1]{\textsc{#1}}
\newcommand\term[1]{\textbf{#1}}

\def\mednobreak{\ifdim\lastskip<\medskipamount
  \removelastskip\nopagebreak\medskip\fi}
\def\bignobreak{\ifdim\lastskip<\bigskipamount
  \removelastskip\nopagebreak\bigskip\fi}

% commands, blocks
\newcommand{\byline}[1]{\bigskip{\RaggedLeft{#1}\par}\bigskip}
\newcommand{\bibl}[1]{{\RaggedLeft{#1}\par\bigskip}}
\newcommand{\biblitem}[1]{{\noindent\hangindent=\parindent   #1\par}}
\newcommand{\dateline}[1]{\medskip{\RaggedLeft{#1}\par}\bigskip}
\newcommand{\labelblock}[1]{\medbreak{\noindent\color{rubric}\bfseries #1}\par\mednobreak}
\newcommand{\salute}[1]{\bigbreak{#1}\par\medbreak}
\newcommand{\signed}[1]{\bigbreak\filbreak{\raggedleft #1\par}\medskip}

% environments for blocks (some may become commands)
\newenvironment{borderbox}{}{} % framing content
\newenvironment{citbibl}{\ifvmode\hfill\fi}{\ifvmode\par\fi }
\newenvironment{docAuthor}{\ifvmode\vskip4pt\fontsize{16pt}{18pt}\selectfont\fi\itshape}{\ifvmode\par\fi }
\newenvironment{docDate}{}{\ifvmode\par\fi }
\newenvironment{docImprint}{\vskip6pt}{\ifvmode\par\fi }
\newenvironment{docTitle}{\vskip6pt\bfseries\fontsize{18pt}{22pt}\selectfont}{\par }
\newenvironment{msHead}{\vskip6pt}{\par}
\newenvironment{msItem}{\vskip6pt}{\par}
\newenvironment{titlePart}{}{\par }


% environments for block containers
\newenvironment{argument}{\itshape\parindent0pt}{\vskip1.5em}
\newenvironment{biblfree}{}{\ifvmode\par\fi }
\newenvironment{bibitemlist}[1]{%
  \list{\@biblabel{\@arabic\c@enumiv}}%
  {%
    \settowidth\labelwidth{\@biblabel{#1}}%
    \leftmargin\labelwidth
    \advance\leftmargin\labelsep
    \@openbib@code
    \usecounter{enumiv}%
    \let\p@enumiv\@empty
    \renewcommand\theenumiv{\@arabic\c@enumiv}%
  }
  \sloppy
  \clubpenalty4000
  \@clubpenalty \clubpenalty
  \widowpenalty4000%
  \sfcode`\.\@m
}%
{\def\@noitemerr
  {\@latex@warning{Empty `bibitemlist' environment}}%
\endlist}
\newenvironment{quoteblock}% may be used for ornaments
  {\begin{quoting}}
  {\end{quoting}}

% table () is preceded and finished by custom command
\newcommand{\tableopen}[1]{%
  \ifnum\strcmp{#1}{wide}=0{%
    \begin{center}
  }
  \else\ifnum\strcmp{#1}{long}=0{%
    \begin{center}
  }
  \else{%
    \begin{center}
  }
  \fi\fi
}
\newcommand{\tableclose}[1]{%
  \ifnum\strcmp{#1}{wide}=0{%
    \end{center}
  }
  \else\ifnum\strcmp{#1}{long}=0{%
    \end{center}
  }
  \else{%
    \end{center}
  }
  \fi\fi
}


% text structure
\newcommand\chapteropen{} % before chapter title
\newcommand\chaptercont{} % after title, argument, epigraph…
\newcommand\chapterclose{} % maybe useful for multicol settings
\setcounter{secnumdepth}{-2} % no counters for hierarchy titles
\setcounter{tocdepth}{5} % deep toc
\markright{\@title} % ???
\markboth{\@title}{\@author} % ???
\renewcommand\tableofcontents{\@starttoc{toc}}
% toclof format
% \renewcommand{\@tocrmarg}{0.1em} % Useless command?
% \renewcommand{\@pnumwidth}{0.5em} % {1.75em}
\renewcommand{\@cftmaketoctitle}{}
\setlength{\cftbeforesecskip}{\z@ \@plus.2\p@}
\renewcommand{\cftchapfont}{}
\renewcommand{\cftchapdotsep}{\cftdotsep}
\renewcommand{\cftchapleader}{\normalfont\cftdotfill{\cftchapdotsep}}
\renewcommand{\cftchappagefont}{\bfseries}
\setlength{\cftbeforechapskip}{0em \@plus\p@}
% \renewcommand{\cftsecfont}{\small\relax}
\renewcommand{\cftsecpagefont}{\normalfont}
% \renewcommand{\cftsubsecfont}{\small\relax}
\renewcommand{\cftsecdotsep}{\cftdotsep}
\renewcommand{\cftsecpagefont}{\normalfont}
\renewcommand{\cftsecleader}{\normalfont\cftdotfill{\cftsecdotsep}}
\setlength{\cftsecindent}{1em}
\setlength{\cftsubsecindent}{2em}
\setlength{\cftsubsubsecindent}{3em}
\setlength{\cftchapnumwidth}{1em}
\setlength{\cftsecnumwidth}{1em}
\setlength{\cftsubsecnumwidth}{1em}
\setlength{\cftsubsubsecnumwidth}{1em}

% footnotes
\newif\ifheading
\newcommand*{\fnmarkscale}{\ifheading 0.70 \else 1 \fi}
\renewcommand\footnoterule{\vspace*{0.3cm}\hrule height \arrayrulewidth width 3cm \vspace*{0.3cm}}
\setlength\footnotesep{1.5\footnotesep} % footnote separator
\renewcommand\@makefntext[1]{\parindent 1.5em \noindent \hb@xt@1.8em{\hss{\normalfont\@thefnmark . }}#1} % no superscipt in foot
\patchcmd{\@footnotetext}{\footnotesize}{\footnotesize\sffamily}{}{} % before scrextend, hyperref


%   see https://tex.stackexchange.com/a/34449/5049
\def\truncdiv#1#2{((#1-(#2-1)/2)/#2)}
\def\moduloop#1#2{(#1-\truncdiv{#1}{#2}*#2)}
\def\modulo#1#2{\number\numexpr\moduloop{#1}{#2}\relax}

% orphans and widows
\clubpenalty=9996
\widowpenalty=9999
\brokenpenalty=4991
\predisplaypenalty=10000
\postdisplaypenalty=1549
\displaywidowpenalty=1602
\hyphenpenalty=400
% Copied from Rahtz but not understood
\def\@pnumwidth{1.55em}
\def\@tocrmarg {2.55em}
\def\@dotsep{4.5}
\emergencystretch 3em
\hbadness=4000
\pretolerance=750
\tolerance=2000
\vbadness=4000
\def\Gin@extensions{.pdf,.png,.jpg,.mps,.tif}
% \renewcommand{\@cite}[1]{#1} % biblio

\usepackage{hyperref} % supposed to be the last one, :o) except for the ones to follow
\urlstyle{same} % after hyperref
\hypersetup{
  % pdftex, % no effect
  pdftitle={\elbibl},
  % pdfauthor={Your name here},
  % pdfsubject={Your subject here},
  % pdfkeywords={keyword1, keyword2},
  bookmarksnumbered=true,
  bookmarksopen=true,
  bookmarksopenlevel=1,
  pdfstartview=Fit,
  breaklinks=true, % avoid long links
  pdfpagemode=UseOutlines,    % pdf toc
  hyperfootnotes=true,
  colorlinks=false,
  pdfborder=0 0 0,
  % pdfpagelayout=TwoPageRight,
  % linktocpage=true, % NO, toc, link only on page no
}

\makeatother % /@@@>
%%%%%%%%%%%%%%
% </TEI> end %
%%%%%%%%%%%%%%


%%%%%%%%%%%%%
% footnotes %
%%%%%%%%%%%%%
\renewcommand{\thefootnote}{\bfseries\textcolor{rubric}{\arabic{footnote}}} % color for footnote marks

%%%%%%%%%
% Fonts %
%%%%%%%%%
\usepackage[]{roboto} % SmallCaps, Regular is a bit bold
% \linespread{0.90} % too compact, keep font natural
\newfontfamily\fontrun[]{Roboto Condensed Light} % condensed runing heads
\ifav
  \setmainfont[
    ItalicFont={Roboto Light Italic},
  ]{Roboto}
\else\ifbooklet
  \setmainfont[
    ItalicFont={Roboto Light Italic},
  ]{Roboto}
\else
\setmainfont[
  ItalicFont={Roboto Italic},
]{Roboto Light}
\fi\fi
\renewcommand{\LettrineFontHook}{\bfseries\color{rubric}}
% \renewenvironment{labelblock}{\begin{center}\bfseries\color{rubric}}{\end{center}}

%%%%%%%%
% MISC %
%%%%%%%%

\setdefaultlanguage[frenchpart=false]{french} % bug on part


\newenvironment{quotebar}{%
    \def\FrameCommand{{\color{rubric!10!}\vrule width 0.5em} \hspace{0.9em}}%
    \def\OuterFrameSep{\itemsep} % séparateur vertical
    \MakeFramed {\advance\hsize-\width \FrameRestore}
  }%
  {%
    \endMakeFramed
  }
\renewenvironment{quoteblock}% may be used for ornaments
  {%
    \savenotes
    \setstretch{0.9}
    \normalfont
    \begin{quotebar}
  }
  {%
    \end{quotebar}
    \spewnotes
  }


\renewcommand{\headrulewidth}{\arrayrulewidth}
\renewcommand{\headrule}{{\color{rubric}\hrule}}

% delicate tuning, image has produce line-height problems in title on 2 lines
\titleformat{name=\chapter} % command
  [display] % shape
  {\vspace{1.5em}\centering} % format
  {} % label
  {0pt} % separator between n
  {}
[{\color{rubric}\huge\textbf{#1}}\bigskip] % after code
% \titlespacing{command}{left spacing}{before spacing}{after spacing}[right]
\titlespacing*{\chapter}{0pt}{-2em}{0pt}[0pt]

\titleformat{name=\section}
  [block]{}{}{}{}
  [\vbox{\color{rubric}\large\raggedleft\textbf{#1}}]
\titlespacing{\section}{0pt}{0pt plus 4pt minus 2pt}{\baselineskip}

\titleformat{name=\subsection}
  [block]
  {}
  {} % \thesection
  {} % separator \arrayrulewidth
  {}
[\vbox{\large\textbf{#1}}]
% \titlespacing{\subsection}{0pt}{0pt plus 4pt minus 2pt}{\baselineskip}

\ifaiv
  \fancypagestyle{main}{%
    \fancyhf{}
    \setlength{\headheight}{1.5em}
    \fancyhead{} % reset head
    \fancyfoot{} % reset foot
    \fancyhead[L]{\truncate{0.45\headwidth}{\fontrun\elbibl}} % book ref
    \fancyhead[R]{\truncate{0.45\headwidth}{ \fontrun\nouppercase\leftmark}} % Chapter title
    \fancyhead[C]{\thepage}
  }
  \fancypagestyle{plain}{% apply to chapter
    \fancyhf{}% clear all header and footer fields
    \setlength{\headheight}{1.5em}
    \fancyhead[L]{\truncate{0.9\headwidth}{\fontrun\elbibl}}
    \fancyhead[R]{\thepage}
  }
\else
  \fancypagestyle{main}{%
    \fancyhf{}
    \setlength{\headheight}{1.5em}
    \fancyhead{} % reset head
    \fancyfoot{} % reset foot
    \fancyhead[RE]{\truncate{0.9\headwidth}{\fontrun\elbibl}} % book ref
    \fancyhead[LO]{\truncate{0.9\headwidth}{\fontrun\nouppercase\leftmark}} % Chapter title, \nouppercase needed
    \fancyhead[RO,LE]{\thepage}
  }
  \fancypagestyle{plain}{% apply to chapter
    \fancyhf{}% clear all header and footer fields
    \setlength{\headheight}{1.5em}
    \fancyhead[L]{\truncate{0.9\headwidth}{\fontrun\elbibl}}
    \fancyhead[R]{\thepage}
  }
\fi

\ifav % a5 only
  \titleclass{\section}{top}
\fi

\newcommand\chapo{{%
  \vspace*{-3em}
  \centering % no vskip ()
  {\Large\addfontfeature{LetterSpace=25}\bfseries{\elauthor}}\par
  \smallskip
  {\large\eldate}\par
  \bigskip
  {\Large\selectfont{\eltitle}}\par
  \bigskip
  {\color{rubric}\hline\par}
  \bigskip
  {\Large TEXTE LIBRE À PARTICPATION LIBRE\par}
  \centerline{\small\color{rubric} {hurlus.fr, tiré le \today}}\par
  \bigskip
}}

\newcommand\cover{{%
  \thispagestyle{empty}
  \centering
  {\LARGE\bfseries{\elauthor}}\par
  \bigskip
  {\Large\eldate}\par
  \bigskip
  \bigskip
  {\LARGE\selectfont{\eltitle}}\par
  \vfill\null
  {\color{rubric}\setlength{\arrayrulewidth}{2pt}\hline\par}
  \vfill\null
  {\Large TEXTE LIBRE À PARTICPATION LIBRE\par}
  \centerline{{\href{https://hurlus.fr}{\dotuline{hurlus.fr}}, tiré le \today}}\par
}}

\begin{document}
\pagestyle{empty}
\ifbooklet{
  \cover\newpage
  \thispagestyle{empty}\hbox{}\newpage
  \cover\newpage\noindent Les voyages de la brochure\par
  \bigskip
  \begin{tabularx}{\textwidth}{l|X|X}
    \textbf{Date} & \textbf{Lieu}& \textbf{Nom/pseudo} \\ \hline
    \rule{0pt}{25cm} &  &   \\
  \end{tabularx}
  \newpage
  \addtocounter{page}{-4}
}\fi

\thispagestyle{empty}
\ifaiv
  \twocolumn[\chapo]
\else
  \chapo
\fi
{\it\elabstract}
\bigskip
\makeatletter\@starttoc{toc}\makeatother % toc without new page
\bigskip

\pagestyle{main} % after style

   \section[{Avant-propos}]{Avant-propos}\phantomsection
\label{ap}\renewcommand{\leftmark}{Avant-propos}

\noindent Ce livre est la substance d’un cours professé au Collège de France en 1900-1901. Il est l’application au domaine économique et, ce me semble, la confirmation d’idées générales que j’ai depuis longtemps exposées. Ce qu’il y a de plus caractéristique dans le point de vue auquel je me place pour explorer à mon tour un champ si souvent parcouru était déjà en germe dans plusieurs études publiées par la \emph{Revue philosophique} dès 1881 et par la \emph{Revue d’économie politique} quelques années plus tard. Envisagée sous son aspect industrieux, laborieux, producteur, comme par son côté criminel, immoral, destructeur, la vie sociale m’a paru relever avant tout de l’\emph{inter-psychologie}, qui étudie ses rapports élémentaires. J’ai pensé cependant qu’il était inutile d’intituler cet ouvrage « Cours d’\emph{inter-psychologie économique} », titre qui eût été peut-être plus exact, mais moins clair et moins simple.\par

\byline{G. T.}

\dateline{Octobre 1901.}
 \phantomsection
\label{v1p1}\section[{A. Partie préliminaire}]{A. Partie préliminaire}\phantomsection
\label{pp}\renewcommand{\leftmark}{A. Partie préliminaire}


\asterism

\subsection[{A.1. Considérations générales et lois sociales}]{A.1. Considérations générales et lois sociales}\phantomsection
\label{ppch1}
\subsubsection[{A.1.a. La société, tissu d’actions inter-spirituelles. Deux grandes espèces du lien social, l’un direct, l’autre indirect.}]{A.1.a. La société, tissu d’actions inter-spirituelles. Deux grandes espèces du lien social, l’un direct, l’autre indirect.}
\noindent Avant d’aborder le sujet spécial de ce traité, il me semble bon de tracer à grands traits la conception générale de la société humaine à laquelle on est conduit quand, tenant fermement en main les principes sur lesquels se fonde à mon avis la science sociale, on les applique non seulement au côté économique mais à tous les autres côtés sociaux de l’humanité.\par
Ces principes, je n’ai pas à les démontrer de nouveau ; je me borne à les répéter en les résumant. La société est un tissu d’actions inter-spirituelles, d’états mentaux agissant les uns sur les autres, mais non pas de n’importe quelle manière. Expliquons-nous clairement. Chaque action interspirituelle consiste dans le rapport de deux êtres animés — d’abord, la mère et l’enfant — dont l’un impressionne l’autre, dont l’un enseigne ou dirige l’autre, dont l’un parle à l’autre qui l’écoute, dont l’un, en un mot, modifie l’autre  \phantomsection
\label{v1p2}mentalement, avec ou sans réciprocité. Je dis d’abord que cette modification, quand elle est de nature à nouer ou à resserrer le lien social entre ces deux personnes est un rapport de modèle à copie. En effet, toutes ces actions d’esprit à esprit ne sont pas, il est vrai, des empreintes du sujet actif sur le sujet passif, des reflets du premier par le second. Souvent, le sujet modifié l’est dans un sens diamétralement contraire à celui du sujet modifiant, il pense et il veut précisément l’opposé de ce qu’il voit penser et vouloir. Parfois même, bien rarement, l’influence qu’il subit a pour effet de lui suggérer un état mental qui n’est ni semblable ni contraire à l’état mental du sujet modifiant, mais qui est quelque chose de tout différent. Or, dans ce dernier cas, le sujet modifiant et le sujet modifié restent étrangers l’un à l’autre, après comme avant la modification. Quand un paysage suggère un sentiment à un paysagiste, quand un animal suggère une idée à un naturaliste, le paysage et le paysagiste, l’animal et le naturaliste, n’entrent pas en rapport social. Pourquoi ? Parce que le lien social n’est pas seulement une action inter-mentale mais encore, et avant tout, un accord inter-mental né de cette action. Accord, cela signifie que l’un désire ce que l’autre désire, ou repousse ce que l’autre repousse, que l’un affirme ce que l’autre affirme ou nie ce que l’autre nie. La société, donc, en son essence intime, doit être définie une communion mentale ; ou mieux, car cette communion n’est jamais parfaite, un groupe de jugements et de desseins qui se contredisent ou se contrarient le moins possible, qui se confirment ou s’entr’aident le plus possible. La société est un système, un système qui diffère d’un système philosophique en ce que les états mentaux dont il se compose sont dispersés entre un grand nombre de cerveaux distincts, au lieu d’être ramassés dans le même cerveau. Mais cette différence, qui constitue toute celle de la logique sociale à la logique individuelle, seule étudiée par les logiciens, n’empêche pas les  \phantomsection
\label{v1p3}règles de celle-ci de s’appliquer, en se transposant, moyennant quelques additions importantes, aux phénomènes de la vie sociale. La vie sociale, avec ses concurrences ou ses coalitions d’efforts, avec ses alliances ou ses luttes de partis et de peuples, est une grande et bruyante dialectique qui tend à résoudre un problème ardu, renaissant à chaque âge de ses solutions mêmes, toujours incomplètes et provisoires, le problème d’un maximum de croyances et de désirs à équilibrer.\par
Partant de là, il est facile de comprendre que c’est seulement dans le cas où l’action du sujet modifiant sur le sujet modifié aboutit à refléter l’état mental du premier dans celui du second, que le lien social se trouve créé ou renforcé entre eux. L’harmonie sociale commence toujours par être un unisson et ne peut commencer que par là. Si l’image de l’état mental modifiant est non pas positive mais négative, il se peut que, indirectement, par suite de la discussion ou de la guerre engagée, un accord social plus profond ou plus vaste finisse par résulter de là, mais, par son effet direct, cette contre-imitation est anti-sociale. On s’explique ainsi la fécondité du phénomène de l’imitation, qui est le rapport social élémentaire, et, de fait, le cas de beaucoup le plus fréquent de l’action inter-spirituelle. La contre-imitation est, en somme, exceptionnelle. Quant à l’hypothèse de deux esprits se suggérant l’un à l’autre des états tout à fait dissemblables sans être opposés, elle ne se réalise jamais, si ce n’est entre esprits déjà très civilisés, ultra-cultivés, au cours, par exemple, d’une conversation où, d’ailleurs, en parlant la même langue et agitant un même fonds de pensées, accumulées par une commune éducation, ils se sont révélé l’un à l’autre infiniment plus de similitudes que de dissemblances, et n’ont pu se différencier que grâce à ces similitudes mêmes.\par
A ce point de vue, nous avons à distinguer deux grandes espèces ou plutôt deux degrés principaux du lien social :  \phantomsection
\label{v1p4}1\textsuperscript{o} le lien le plus fort, celui qui naît, entre le sujet modifiant et le sujet modifié, de la similitude mentale produite par l’action de l’un sur l’autre, — action qui commence par être unilatérale et qui tend toujours à devenir réciproque. C’est le nœud qui lie les parents à l’enfant, le maître au disciple, le meneur au mené, le parleur à l’écouteur, et, en général, tous les gens habitués à vivre, à causer, à travailler ensemble. 2\textsuperscript{o} Le lien, beaucoup plus faible, qui résulte, entre sujets ne se modifiant pas l’un l’autre, entre personnes ne causant pas l’une avec l’autre, ne s’écrivant pas, ne se lisant pas, de la similitude produite en chacun d’eux par l’empreinte qu’ils ont séparément subie d’un même sujet actif. Cette dernière catégorie comprend la majorité des gens qui appartiennent au même milieu social. Sans se connaître entre eux, ils sont vigoureusement liés les uns aux autres par une multitude de fils invisibles, par ces innombrables manières de parler, de penser, de sentir, d’agir, qui leur sont communes parce qu’elles procèdent — si on les analyse chacun à part — des mêmes auteurs premiers, des mêmes inventeurs, découvreurs, initiateurs, anciens ou modernes, connus ou anonymes.\par
Comme on le voit, le rapport proportionnel de ces deux espèces de liens sociaux, du lien primaire et du lien dérivé, va changeant au profit du dernier, à mesure qu’une société se complique et s’étend ; mais le lien primaire n’en reste pas moins le plus important, puisque, sans lui, l’autre ne saurait être. C’est donc à celui-ci qu’il faut nous attacher.
\subsubsection[{A.1.b. Tendance de tout exemple au rayonnement imitatif. Expansivité universelle. Adaptation et répétition, deux problèmes capitaux. Mérite et erreur de Darwin.}]{A.1.b. Tendance de tout exemple au rayonnement imitatif. Expansivité universelle. \emph{Adaptation} et \emph{répétition}, deux problèmes capitaux. Mérite et erreur de Darwin.}
\noindent La première généralisation, bien réelle et bien positive, qui s’offre à nous ici, est la suivante. Un groupe d’esprits  \phantomsection
\label{v1p5}en contact mental étant donné, si l’un d’eux conçoit une idée, une action nouvelle, ou paraissant telle, et que cette idée ou cette action se montre avec une apparence de vérité ou d’utilité supérieure\footnote{ \noindent Cette \emph{apparence d’utilité ou de vérité supérieure} tient à la nature des besoins ou des idées déjà installés dans ces esprits et qui s’y sont répandus et enracinés eux-mêmes de la manière qui vient d’être dite. Entre cent exemples qui s’offrent, chaque individu choisit le plus logiquement ou téléologiquement d’accord avec ses idées ou ses besoins actuels.
 }, elle se communiquera autour de lui, par reflet, à trois, quatre, dix personnes ; et chacune d’elles, à son tour, la répandra autour de soi, et ainsi de suite, jusqu’à ce que les limites du groupe\footnote{ \noindent Pourquoi ce \emph{groupe a-t-il telles limites} et non telles autres ? J’accorde à M. Giddings que c’est parce que les individus qui en font partie ont seuls \emph{conscience d’être de la même espèce sociale}, autrement dit ont un même \emph{esprit de corps.} Mais c’est là une définition plutôt qu’une explication. Car pourquoi l’\emph{esprit de corps} s’est-il formé dans telles limites et non en deçà ni au delà ? N’est-ce pas parce que, en vertu des lois logiques et téléologiques indiquées dans la note précédente, le rayonnement moyen des exemples s’est arrêté là ? — Noter la multiplicité des \emph{esprits de corps} (domestique, professionnel, national, politique, religieux, etc.) et leur entrelacement.
 } soient atteintes. Telle sera, du moins, la tendance de cet exemple, arrêtée souvent par l’obstacle d’autres tendances contradictoires ; et cette tendance générale des inventions, des innovations, des initiatives individuelles, à se propager suivant une sorte de progression géométrique, à se déployer en ce que j’appelle un rayonnement imitatif, — c’est-à-dire en un éventail indéfiniment allongé de rayons rattachant chaque imitateur, par une chaîne d’intermédiaires, au \emph{foyer} initial — me paraît jouer en sociologie un rôle égal à celui que joue, en histoire naturelle, la tendance de chaque animal ou de chaque plante, de chaque variété nouvelle de plante ou d’animal, à se propager suivant une progression géométrique, ou, pour remonter plus haut, la tendance de chaque ovule, de chaque cellule initiale, à une prolifération croissante... J’ajoute que, en physique, à ces deux grandes sortes d’expansions virtuelles ou actuelles correspond la force pareillement expansive qui pousse toute onde, toute vibration, tout mouvement  \phantomsection
\label{v1p6}harmonieusement périodique de la matière pondérable ou de l’éther, à rayonner dans tous les sens où il peut se propager, sous forme de son, de lumière, de chaleur. L’atmosphère est toute vibrante de ces progressions enchevêtrées d’ondes sonores à partir d’une bouche qui a parlé, d’une aile qui a frémi, d’une eau qui tombe ; l’immensité du firmament est toute palpitante de ces épanchements lumineux qui, à partir de chaque étoile et de chaque point d’une étoile, se croisent et se compliquent en chaque point de l’éther. Il n’est pas une force physique qui ne se résolve ainsi en ondulations rayonnantes ou aspirantes au rayonnement. Cela est vrai de la force astronomique elle-même, de la gravitation ; car les mouvements elliptiques des planètes sont eux-mêmes comparables à d’immenses ondes, et, comme tels, on voit chacun d’eux non seulement se répéter sans fin par la gravitation indéfinie de la planète que l’on considère, mais encore se propager au dehors, parmi les astres du même système, sous la forme de ces perturbations périodiques qui sont l’image multipliée des mouvements périodiques de toutes les autres planètes par le mouvement périodique de chacune d’elles. Il n’est pas jusqu’aux forces chimiques qui, vraisemblablement, ne consistent en une circulation de mouvements enchaînés, plus ou moins complexes suivant la complexité de la molécule. Comment s’est formé, à l’origine, l’atome des substances réputées simples ? Nous l’ignorons ; mais, quand nous constatons avec stupeur, par le spectroscope, l’identité de ces atomes, d’hydrogène, d’oxygène, d’azote, etc., dans les astres de notre système et jusque dans les étoiles les plus éloignées, pouvons-nous admettre que ces myriades d’atomes homogènes, disséminées de la sorte dans le ciel infini, sont nés semblables ? ne devons-nous pas conjecturer irrésistiblement que cette similitude, comme toute autre similitude actuelle, a été opérée jadis, en une phase pré-cosmique, pré-chimique, par une propagation expansive, maintenant épuisée,  \phantomsection
\label{v1p7}de types chimiques d’abord créés quelque part — nous n’avons pas à deviner comment — en une petite région limitée, et de là répandus de proche en proche ? A vrai dire, ce besoin d’expansion des types chimiques, qui s’est ainsi satisfait, en des âges insondables, par la formation des corps simples, il n’est pas mort, il n’est qu’endormi ou dissimulé ; et, à chaque combinaison nouvelle qui s’opère, dans les creusets de la vie organique ou de nos laboratoires, il se réveille, il nous émerveille de ses avidités conquérantes. Car, lorsque deux volumes de gaz se combinent grâce au passage d’une étincelle électrique, qu’est-ce autre chose que la mutuelle conquête, la mutuelle assimilation, de chacun d’eux par l’autre, l’échange et l’entrelacement de leurs vibrations intimes qui se servent les uns aux autres de débouchés ? Peut-être, assurément même, je m’exprime mal ; mais, en tout cas, ce qui me paraît clair, c’est que toute combinaison chimique rend témoignage, elle aussi, pour sa part, à ce vœu de propagation de soi, d’expansion rayonnante, qui est au fond de tout être.\par
Si nous songeons aux différences si profondes qui séparent, à tous autres égards, le monde physique, y compris le monde astronomique et le monde chimique, du monde vivant, et celui-ci même du monde social qui en est la surprenante continuation, nous serons frappés de voir tous ces mondes hétérogènes présenter, sous les formes les plus diverses, cette même expansivité essentielle, cette même ambition propagatrice, et n’avoir, pour ainsi dire, rien de commun que ce caractère fondamental. Et, si nous nous demandons ce qu’il y a de commun aussi dans les formes sous lesquelles se manifeste ce même vœu d’expansion universelle, nous devons être étonnés, — à moins que l’habitude ne nous aveugle — de constater qu’elles sont toutes des répétitions. Répétition, la série des ondes lumineuses, électriques, sonores, la gravitation des astres, le tourbillonnement intérieur des molécules. Répétition, le tourbillon  \phantomsection
\label{v1p8}vital, la nutrition, la respiration, la circulation, toutes les fonctions organiques, à commencer par la génération qui les comprend toutes. Répétition, le langage, la religion, le savoir, l’éducation, le travail, toutes les activités sociales, en un seul mot l’imitation.\par
Comment se fait-il que la puissance mystérieuse qui meut l’Univers aime si fort à ressasser et rabâcher, qu’elle se complaise à ces éditions et rééditions infinies de ses propres œuvres, et ne soit jamais lasse de les reproduire ? Nous croirions que c’est par indigence d’esprit et pauvreté d’invention, si, d’autre part, elle ne nous donnait tant de preuves de sa prodigieuse imagination par le spectacle de la vie universelle. Alors, ne serait-ce pas que son imagination s’alimente de ses réminiscences répétées, que ses répétitions sont la condition même de ses créations, ce que peut nous faire comprendre l’observation du monde social où il est visible que, sans une rencontre de rayons imitatifs différents dans un cerveau bien doué, il n’y a point d’invention ni de découverte ? En tout cas, un autre rapprochement que nous apercevons facilement entre les mondes comparés par nous, c’est que les choses qui s’y répètent, mouvements périodiques, types vivants, inventions ou découvertes, sont des \emph{thèmes}, c’est-à-dire des harmonies fécondes en harmonies nouvelles, des adaptations susceptibles de variations. Des adaptations qui se répètent pour se varier, des nouveautés qui se reproduisent pour se renouveler, tel est le spectacle que nous donne l’Univers envisagé sous tous ses aspects.\par
Quel est le rapport de ces trois termes, \emph{adaptation, répétition, variation ?} D’abord, remarquons qu’ils se réduisent à deux, aux deux premiers, la variation n’étant qu’une petite adaptation greffée sur la grande, et révélant la possibilité d’une grande harmonie nouvelle si la voie nouvelle indiquée par elle était poussée à bout. Une variation n’est qu’une \emph{réadaptation.} Par exemple, chaque variation individuelle d’un type humain est elle-même un type nouveau, et  \phantomsection
\label{v1p9}il suffirait d’exagérer, en les équilibrant entre elles, les particularités par lesquelles chaque individu se caractérise pour donner naissance à une nouvelle race humaine. On peut en dire autant d’une hérésie religieuse, d’une innovation linguistique, politique, juridique. Il ne reste donc en présence que deux termes, l’adaptation et la répétition, qui, dans le monde social, s’appellent l’invention et l’imitation\footnote{ \noindent J’ai parlé souvent et je reparlerai encore plus loin d’un troisième terme, l’\emph{opposition}, qui n’est qu’un \emph{moyen terme}, un procédé d’adaptation, et non le seul. Il ne doit pas être mis sur le même rang que les deux autres.
 }.\par
Par suite, il y a, dans chacune des sciences qui ont trait aux divers mondes, aux différents étages de réalités superposées, deux problèmes bien distincts : 1\textsuperscript{o} comment s’opèrent ou se sont opérées les \emph{adaptations} répétées que cette science étudie, à savoir, les types chimiques, les systèmes stellaires, les ondes lumineuses ou sonores initiales, les espèces vivantes, les inventions de tout genre, en comprenant par ce mot toutes les innovations spirituelles ; 2\textsuperscript{o} comment se propagent les répétitions de ces choses harmonieuses, et ce qui résulte de l’interférence de ces propagations.\par
Le premier de ces problèmes est d’un tout autre ordre, et à coup sûr tout autrement ardu, que le second. Et il est aisé de s’assurer que la plupart des sciences, même et surtout réputées les plus parfaites, telles que la chimie, l’ont éludé ou n’ont émis à ce sujet que des conjectures. Le terrain solide des savants, c’est la réponse au second problème. La réponse, si réponse il y a, au premier, c’est la partie aérienne et hypothétique de leurs traités. Il ne faut pas être plus exigeant pour les sociologues qu’on ne l’est pour les chimistes ou les botanistes et déclarer qu’ils n’ont rien appris de scientifique quand ils ont cherché à démêler simplement les lois qui règlent l’expansion des exemples. Mais la vérité est que, comparée aux autres sciences en ce qui touche au premier des deux problèmes dont il s’agit, la  \phantomsection
\label{v1p10}science sociale révèle une supériorité indiscutable. Dans beaucoup de cas, nous sommes en mesure de suivre le mode d’élaboration d’une invention, d’analyser ses éléments, d’éclaircir sa synthèse dans le cerveau de son auteur final, tandis que jamais naturaliste n’a surpris sur le fait la production d’une nouvelle espèce. L’espoir même qui avait lui d’expliquer leur genèse par la sélection naturelle, avec ou sans la sélection sexuelle, s’est dissipé. Le grand mérite de Darwin aura été de montrer la portée méconnue de la tendance des vivants à leur multiplication indéfinie, et de suivre les conséquences qui en découlent, telles que la concurrence vitale et le croisement des espèces. Son erreur, s’il m’est permis d’apprécier ce grand homme en m’autorisant d’autres grands naturalistes, me semble avoir été d’appuyer beaucoup plus sur la concurrence vitale, forme biologique de l’opposition, que sur le croisement et l’hybridité, formes biologiques de l’adaptation et de l’harmonie. Une fonction aussi importante que la production d’une nouvelle espèce ne saurait être une fonction continue et quotidienne, alors que la simple production d’un individu nouveau, la génération, est une fonction intermittente. Un \emph{phénomène exceptionnel}, et non pas un phénomène journalier, doit être à la base de cette nouveauté spécifique. Et, je suis de l’avis de Cournot, une hybridité féconde, par exception, est bien plus propre qu’une accumulation héréditaire de petites variations avantageuses, par concurrence et sélection, à expliquer la formation de nouveaux types vitaux. Encore faut-il reconnaître que l’on indique ainsi les \emph{conditions} seulement du merveilleux phénomène, et non ses \emph{causes}, qui restent le secret des ovules fécondés où il s’opère.\par
Darwin n’a point paru s’apercevoir que les variations individuelles qu’il postule, comme les données et les matériaux avec lesquels la sélection bâtirait de nouvelles espèces, sont elles-mêmes autant de petites créations, de petites réadaptations du type ancien, dont la genèse n’est guère  \phantomsection
\label{v1p11}moins merveilleuse que celle d’un type nouveau, auquel chacune d’elles vise, et est du même ordre, au fond. En sorte que la genèse des innovations vitales est pour ainsi dire, postulée par cela même qui est censé l’expliquer. La croissance d’un embryon individuel n’est pas autre chose que la refonte du type spécifique par un ovule fécondé, — c’est-à-dire par un ovule né d’un croisement, d’un mariage, — qui s’est approprié ce type et en a modifié solidairement toutes les parties en vertu d’une corrélation des plus intimes et des plus profondes, où gît le mystère de la création même des espèces. Si nous nous laissons guider par l’analogie de la tendance à la progression géométrique des exemplaires de chaque espèce avec la tendance sociologique correspondante à la propagation croissante des exemples de chaque sorte, nous observerons que celle-ci elle-même a pour effet de produire d’innombrables croisements, des interférences de rayons imitatifs dans des cerveaux associés, et que c’est là, vraiment, la condition indispensable de l’éclosion d’inventions nouvelles, mais que, au fond, l’opération même d’où naissent celles-ci se cache dans l’intimité du cerveau privilégié que nous nommons génial. N’y aurait-il pas au fond de chaque nouvelle espèce apparue — de même que, à vrai dire, au fond de toute variation individuelle d’une espèce — quelque chose de comparable à un trait de génie ou d’ingéniosité ?\par
Ainsi, la réponse au premier des deux problèmes capitaux que nous avons distingués en toute science n’est pas plus avancée en biologie qu’en sociologie. Elle l’est même bien moins, car nous savons jusqu’à un certain point, par quelques auto-biographies d’inventeurs, comment leur idée s’est formée dans leur esprit, comment, par exemple, Newton a conçu l’attraction universelle, ou Denis Papin la possibilité d’utiliser la force motrice de la vapeur pour la navigation. M. Ribot, dans son remarquable essai sur l’\emph{Imagination créatrice}, ainsi que M. Paulhan, dans son livre sur l’\emph{Invention},  \phantomsection
\label{v1p12}ont éclairci plusieurs points de ce sujet obscur. Mais le comment de la création du moindre type organique nous est profondément inconnu.\par
Quand, dans une exposition universelle rétrospective, nous voyons une suite de moyens de transports successivement apparus au cours des âges, depuis la chaise à porteur et le chariot jusqu’à la voiture suspendue, à la locomotive, à l’automobile, à la bicyclette, nous sommes comme le naturaliste qui, dans un musée, compare la série des vertébrés au cours des âges géologiques, depuis l’amphioxus jusqu’à l’homme. Mais il y a cette différence, que, dans le premier cas, nous pouvons dater exactement l’apparition de certains anneaux de la chaîne et préciser l’invention et l’inventeur d’où chacun d’eux procède, tandis que, dans le second cas, nous sommes réduits à de simples conjectures sur le mode de transformation d’une espèce dans l’autre.\par
Quant au second problème, qui a trait aux répétitions, il n’a pas moins été éclairci par le sociologue dans sa sphère, ou par chacun de ces sociologues partiels qu’on appelle linguistes, mythologues, économistes, juristes, moralistes, qu’il ne l’a été par les physiciens et les biologistes dans leurs domaines particuliers.\par
Ce n’est pas que ces savants, pas plus les sociologues partiels que les autres, aient eu conscience de ne s’occuper que de répétitions et de similitudes. Mais il n’en est pas moins vrai que, au fond de toutes les lois formulées par eux, on ne voit jamais que des rapports établis entre des groupes ou des amas de choses semblables, entre des éditions d’exemplaires naturels, telles ou telles ondes, telles ou telles cellules, telles ou telles formes d’action ou de pensée, qui se disputent la force et la vie, la vie corporelle et la vie mentale. La science idéale, le prototype scientifique universel auquel toutes les sciences concrètes aspirent à se conformer, les mathématiques, qu’est-ce sinon le développement et la combinaison des notions du nombre, de l’étendue  \phantomsection
\label{v1p13}et de la durée, c’est-à-dire de l’unité répétée indéfiniment, répétition qui se nomme groupe ou \emph{total} quand les unités répétées restent distinctes, \emph{quantité} quand elles restent annexées et indistinctes ? La quantité, c’est la possibilité d’une infinité de répétitions d’unités différentes. Et toutes les sciences, quelles qu’elles soient, la sociologie même par la statistique, comme la biologie par ses instruments de laboratoire, s’élèvent en grade à chaque pas qu’elles font vers la formule de leurs lois en termes quantitatifs.
\subsubsection[{A.1.c. L’amplification historique. Exemple de cette loi : le rapport des métropoles aux colonies dans le monde antique et dans le monde moderne.}]{A.1.c. L’amplification historique. Exemple de cette loi : le rapport des métropoles aux colonies dans le monde antique et dans le monde moderne.}
\noindent Maintenant, abandonnons, sans les oublier, ces vues générales, et renfermons-nous dans le domaine social. J’ai tâché de montrer ailleurs, et je n’y reviendrai pas, quelles sont les diverses influences principales qui entravent ou favorisent la tendance à l’expansion imitative des exemples jugés bons, des idées jugées vraies ; pourquoi ces jugements de vérité ou d’utilité sont prononcés dans tel milieu et non dans tel autre, et déterminent ici et non là le triomphe d’un modèle sur ses rivaux. Je ne rappellerai que pour mémoire la loi de la descente des exemples du supérieur à l’inférieur, tantôt des noblesses aux plèbes, tantôt des capitales aux petites villes et aux campagnes ; la loi de l’alternance de l’imitation-coutume, et de l’imitation-mode, etc. Sans entrer dans ce détail, d’ailleurs important, attachons-nous seulement, à présent, à la tendance expansive que nous avons signalée et aux conséquences qui s’ensuivent.\par
Il s’ensuit d’abord un grand fait, assez vague mais très intéressant, qu’il s’agit de mettre en lumière, et qui pourrait être désigné sous le nom d’\emph{amplification historique.} Tous les domaines sociaux vont s’élargissant depuis les débuts d’une histoire jusqu’à son terme : le domaine linguistique, \phantomsection
\label{v1p14} d’abord réduit à une seule famille, puis étendu à une tribu, à une peuplade, à un État municipal, à un Empire ; le domaine religieux, qui, parti d’une petite secte, devient une immense Église ; le domaine politique, le domaine juridique, qui a traversé des phases analogues ; le domaine économique, le \emph{marché}, qui, d’un étroit marché de village, est devenu par degrés international, interocéanique ; le domaine esthétique enfin, et le domaine moral. Si nous comparions, à diverses époques successives, par exemple du {\scshape xii}\textsuperscript{e} siècle à nos jours les changements subis par la carte linguistique de l’Europe, ou aussi bien par sa carte religieuse, par sa carte politique, par sa carte juridique, par sa carte économique, nous constaterions entre toutes ces cartes cette ressemblance que toutes ont été se simplifiant, par la diminution graduelle du nombre des langues, des cultes, des formes politiques, des coutumes, des régimes industriels, juxtaposés et coexistants ; ce qui signifie que les langues survivantes, les cultes, les droits, les industries survivantes, ont été s’amplifiant sans cesse. Ce que les cartes ne peuvent nous dire, c’est le \emph{verso} inaperçu de ce phénomène, c’est-à-dire l’épanouissement des originalités individuelles, grâce à cette diffusion même des banalités triomphantes ; car il est à noter que, par bien des côtés, par la floraison philosophique ou poétique des esprits, l’individualisme progresse à la faveur des progrès du communisme mental exprimés par les cartes en question. Nous laissons cela pour le moment.\par
Cet élargissement des domaines sociaux n’est pas dû essentiellement au progrès de la population, car il se continue même quand la population est stationnaire ou rétrograde, comme dans les derniers siècles de l’empire romain. Il est à remarquer aussi que la marche de ces diverses expansions sociales n’est pas égale ; mais ce serait une erreur de penser qu’il y en a toujours une, toujours la même, — par exemple l’expansion économique — qui devance les autres et les entraîne. C’est tantôt l’une, tantôt l’autre, qui prend l’avance,  \phantomsection
\label{v1p15}et il y a comme une sorte d’émulation de vitesse entre elles\footnote{ \noindent Par exemple, il est visible que, dans les grands empires de l’antique Orient, le groupe politique déborde de beaucoup le groupe social (si l’on entend par \emph{social} tout ce qui est économique, religieux, linguistique, etc., par opposition au côté politique. Il en a été de même sous l’Empire romain, et même dans les temps modernes. L’Empire anglais, l’Empire russe sont des amalgames de nationalités. Mais, dans la Grèce antique, le groupe social dépassait extrêmement le groupe politique. Le monde hellénique s’étendait fort loin, et avançait toujours pendant que, — jusqu’à Alexandre — chaque cité grecque formait un État distinct. Pareillement en Gaule, avant Jules César, quel que fût le morcellement politique, il y avait une nation gauloise internationale pour ainsi dire, dont l’existence est attestée par les monnaies. Au {\scshape iv}\textsuperscript{e} siècle, avant J.-C., date où commence en Gaule le monnayage, les types monétaires sont très variés, mais ils présentent dans leur variété une remarquable unité, comme le remarque M. Alexandre Bertrand dans sa \emph{Religion des Gaulois.} En Italie avant Rome, même phénomène. En Amérique aussi, les tribus iroquoises formaient une \emph{société} beaucoup plus vaste que chacune d’elles. — A notre époque, la \emph{socialisation} de l’Europe marche bien plus vite que son unification politique, et son morcellement politique ne se comprend plus.
 }. La comparaison de leur marche inégale peut donner lieu toutefois à des remarques générales. Quel est celui de ces mouvements progressifs qui est habituellement en tête des autres ? Est-ce le mouvement de la religion ou de la langue, de l’État ou du marché, qui ouvre aux autres la voie triomphale et les conduit à la conquête du monde ? Ou bien y a-t-il, à chaque époque, et en chaque région, des causes particulières qui favorisent ici l’expansion de la langue d’abord, ailleurs celle de la religion, ou celle du marché ou celle des institutions politiques et juridiques ? Et quelles sont ces causes ? Questions dignes d’étude, mais que nous n’agiterons pas encore. Insistons, avant tout, sur la réalité des faits même que nous énonçons. Si l’on veut avoir la preuve manifeste que le \emph{champ social} des anciens, même sous l’Empire romain, était beaucoup moins large et moins profond que le nôtre, il suffira de faire cette remarque bien simple : l’exotisme, le cosmopolitisme littéraire, ce caractère si remarquable de nos littératures contemporaines, et qui va s’accentuant depuis le {\scshape xviii}\textsuperscript{c} siècle, était inconnu des Grecs et des Romains. L’archaïsme, ce repli sur soi, leur était connu, non l’exotisme, je le répète. Jamais la moindre velléité \phantomsection
\label{v1p16} d’importer les littératures persanes, ou indiennes même, à plus forte raison germaniques et scythes, jamais le désir de s’en inspirer, de les imiter, de les utiliser, n’est venu à un ancien. Si les Romains ont romanisé les lettres grecques, on ne saurait voir là rien de comparable à l’importation du roman russe ou de la tragédie anglaise ou du conte norvégien dans notre littérature française. Le rapport du poète latin au poète grec était celui de disciple à maître, non de pair à pair. Mais, à vrai dire, ce n’est là qu’un côté bien superficiel de la question.\par
Il serait éminemment instructif de passer en revue tous les aspects par lesquels les sociétés antiques se montrent à nous comme l’image réduite et anticipée, en quelque sorte, du monde moderne. Non seulement en politique, où cette similitude, au degré près, est apparente, mais par tous les côtés de la vie sociale, il est aisé d’observer dans le monde hellénique ou romain des phénomènes que notre civilisation européenne reproduit avec un vaste agrandissement. Il ne s’agit pas là toujours, ni le plus souvent, d’une reproduction imitative, car, si, par exemple, l’évolution du drame moderne depuis les mystères du moyen âge jusqu’à Racine reproduit en quelque manière, avec plus d’ampleur, l’évolution de la tragédie grecque depuis le char de Thespis jusqu’à Euripide, ce n’est pas le moins du monde parce que la seconde de ces évolutions s’est modelée sur la première, quoique Racine se soit inspiré d’Euripide. Aussi, comme toutes les similitudes sociales qui ne sont pas dues à l’imitation, et que le retour des circonstances analogues, sous l’action d’une logique identique, a provoquées, les similitudes dont il s’agit sont-elles en général extrêmement imprécises. Il ne faut pas non plus confondre avec le caractère amplifiant des vastes et vagues répétitions d’ensemble dont je parle, le caractère expansif des petites et précises répétitions de détail qui sont imitatives. Les choses qui se répètent par imitation (mots d’une langue, rites d’une religion,  \phantomsection
\label{v1p17}actes de travail, etc.), se multiplient sans se grossir ; les choses qui se répètent spontanément d’un âge à un âge postérieur de la même grande élaboration sociale (institutions gouvernementales, juridiques, professionnelles, etc.), se grossissent sans se multiplier toujours. Mais il n’en est pas moins vrai que, si les répétitions imitatives, à chacun des deux âges comparés, n’avaient pas fonctionné, conformément aux lois de logique sociale qui les régissent, les répétitions non imitatives n’auraient pas eu lieu, et que, si celles-ci se présentent comme des amplifications, c’est parce que les autres ont une tendance, souvent satisfaite, à l’expansion.\par
Non seulement entre les âges successifs d’un même cycle de civilisation, mais aussi entre deux cycles sociaux différents, entre deux sociétés hétérogènes, on remarque un grand nombre de similitudes non imitatives, approximatives. Or, de celles-ci pareillement on peut dire que, si chacune de ces sociétés, séparément, ne s’était pas conformée aux lois logiques de l’imitation dans la production et reproduction continue de ses similitudes de détail, précises et multipliées, les ressemblances spontanées entre elles n’auraient pu se produire. — Il est à noter que, entre sociétés hétérogènes, ces répétitions ne peuvent être conçues ni comme des amplifications ni comme des réductions ; tout ce qu’on peut dire c’est que d’une société à une autre, le module des choses spontanément répétées est très différent.\par
Parlons du côté économique de nos sociétés, puisque, aussi bien, c’est le sujet spécial de ce cours\footnote{ \noindent Les banques sont nées en Grèce, ainsi que beaucoup d’autres institutions de crédit qui ont reparu, sans imitation, sur une plus grande échelle, à partir des temps modernes. Les mêmes opérations et spéculations commerciales spécialement maritimes (voy. \emph{Plaidoyers civils} de Démosthène) auxquelles se livrent les Grecs se produisent dans nos Bourses de Londres, de Paris, de New-York. C’est là une répétition non imitative. Mais elle n’aurait pu avoir lieu, si en Grèce, et dans l’Europe moderne, séparément, les lois de l’imitation n’avaient fonctionné ; et, si cette répétition est un agrandissement, c’est parce que les exemples se propagent expansivement...
 }. A propos du capital et du crédit, M. Paul Leroy-Beaulieu est conduit,  \phantomsection
\label{v1p18}incidemment, à confirmer cette vue. De deux passages de Démosthène où il est dit que « il y a deux genres de biens, la fortune et le crédit, ce dernier supérieur à l’autre », et que « si l’on ignore que le crédit est le plus grand capital pour l’acquisition de la richesse, on est ignorant de tout », il déduit, ainsi que de beaucoup d’autres passages d’écrivains grecs, que la situation commerciale d’Athènes, attestée par ces fragments, présente la plus grande analogie avec la situation sociale du marché britannique si bien décrite par Bagehot, à cela près que le marché britannique est infiniment plus étendu.\par
Il dit encore que la tendance du capital à s’essorer et s’universaliser par le crédit n’est pas nouvelle. « Ce n’est pas d’aujourd’hui ni d’hier que le crédit suit cette pente et obéit à cette loi d’expansion et d’universalisation. Les anciens Grecs, les précurseurs du monde moderne, l’avaient déjà éprouvée. Sauf les inventions mécaniques\footnote{ \noindent Exception plus apparente que réelle : l’invention de la trirème, tous les perfectionnements de la métallurgie, des métiers, cela ne compte-t-il pour rien ? Ici aussi, c’est surtout une différence de degré qui sépare les civilisations antiques des nôtres. Nos inventions se sont substituées aux leurs en agrandissant singulièrement leur portée. Malgré tout, notre auteur exagère quand il ajoute que « l’antique Grèce paraît avoir \emph{peu différé} » commercialement de l’Europe actuelle. Elle en différait beaucoup, et c’est seulement à travers bien des singularités et des originalités caractéristiques qu’on parvient à démêler les analogies signalées.
 }, l’antique Grèce paraît avoir peu différé, au point de vue du commerce, de l’Europe contemporaine : \emph{Ce sont les mêmes phénomènes mais actuellement agrandis\footnote{ \noindent C’est moi qui souligne.
 }}. D’après Thucydide, Plutarque, Xénophon, Socrate, et surtout les plaidoyers de Démosthène, il a été possible aux érudits de démontrer qu’Athènes, dans ses beaux jours, fournissait les fonds de roulement du commerce aux habitants de toute la Méditerranée orientale ; de même, la monnaie d’Athènes était prédominante à Cyrène, dans la plus grande partie de la Sicile, en Étrurie\footnote{ \noindent \emph{Traité d’économie polit.}, t. III, p. 391 et suiv.
 }. » Il  \phantomsection
\label{v1p19}ajoute : « Plus tard il en fut de même chez les Romains\footnote{ \noindent Et alors la répétition (vague) de ces phénomènes eut lieu sur une échelle déjà bien plus grande, mais moindre que de nos jours.
 } : tous les âges eurent leur pays neufs où se déversaient non seulement les émigrants, c’est-à-dire les hommes en surabondance, mais aussi les capitaux superflus des vieux pays. La Méditerranée orientale, la mer Noire ou le Pont-Euxin, l’Afrique méditerranéenne, furent les pays neufs des Phéniciens et des Grecs ; l’Espagne, la Sicile, furent les pays neufs des Carthaginois ; puis la Gaule jusqu’au Rhin, la Grande-Bretagne furent des contrées neuves pour les Romains\footnote{ \noindent La Germanie aussi pendant longtemps. — Un historien récent a pu dire (Chélard, \emph{la Civilisation française dans le développement de l’Allemagne}, livre fort intéressant) que, pour les rois français mérovingiens et carolingiens même, l’Allemagne n’avait été « qu’une terre coloniale, une espèce d’\emph{interland} où ils implantaient leur pouvoir... \emph{en procédant exactement comme les puissances modernes dans leurs colonies}, par voie d’expéditions armées, de \emph{missions religieuses} et de \emph{traités de protectorat} avec les chefs indigènes ». Tout cela a été repris par nous, en Asie et en Afrique, en Amérique, sur une plus grande échelle. D’après l’auteur, rien ne ressemble plus « au missionnaire qui aujourd’hui se fait massacrer en Chine ou au Soudan que les premiers apôtres de la Germanie, saint Fridolin, saint Colomban, etc... ».
 }. Ces pays furent respectivement, pour les peuples commerçants d’alors, plus encore au point de vue de l’émigration des capitaux qu’à celui de l’émigration des hommes, ce qu’ont été à l’Europe l’Amérique du Nord, l’Amérique du Sud, l’Australie, en partie les Indes, et l’Afrique bientôt. Le mot de Cicéron, qu’il ne se payait pas un écu en Gaule dont il ne fût tenu écriture au forum, est, à ce sujet, singulièrement significatif. »\par
On voit que la découverte de l’Amérique a eu simplement pour effet d’agrandir beaucoup le champ des \emph{pays neufs}, le domaine de l’exploitation coloniale, des débouchés extérieurs, du rayonnement imitatif et expansif dont les foyers de la civilisation, à toute époque, ont toujours besoin — et qui cependant, un jour, est destiné à leur manquer. — La différence du \emph{module} de civilisation est énorme sans doute entre les Iles Britanniques et les États-Unis, entre le Portugal \phantomsection
\label{v1p20} et le Brésil ; elle le sera bientôt entre la vieille Europe en général et tous les nouveaux mondes colonisés par elle, Amérique, Australie, Asie, Afrique. Mais cette différence n’est pas plus grande que celle qui existait, au {\scshape vii}\textsuperscript{e} siècle avant J.-C., entre les métropoles de la Grèce, de la petite Grèce indigente et minuscule, et leurs riches et puissantes colonies de l’Asie Mineure ou de l’Italie, « de la Grande-Grèce », Sybaris, Crotone, Milet : de même qu’entre Tyr et l’empire Carthaginois. L’activité, la grandeur, la civilisation relatives — au sens luxueux et financier du mot civilisation — de ces rejetons transplantés de l’Hellade, éclipsait autant Argos ou Athènes\footnote{ \noindent Voir à ce sujet le \emph{Parthénon}, par M. Boutmy.
 } que le développement des communications de tout genre, des entreprises industrielles et du commerce aux État-Unis l’emporte sur celui de l’Angleterre même.\par
M. Paul Leroy-Beaulieu fait cette remarque très juste : « Les pays neufs jouissent aujourd’hui de ce privilège d’avoir une nature inoccupée, vierge, et, en même temps, pour la mettre en œuvre, de disposer de capitaux qu’ils n’ont pas formés et qui leur viennent en abondance du vieux monde. » Il n’est pas étonnant que, dans ces conditions, ces pays neufs, États-Unis ou Australie, nous éblouissent de leur luxe de civilisation parvenue, et ce serait naïvement qu’ils en feraient honneur à leur seul mérite. Mais ce n’est pas seulement aujourd’hui, c’est dans le passé aussi, que les pays neufs ont eu le privilège si bien remarqué par l’auteur cité. Et en passant, on peut voir là une explication plausible de cette marche de la civilisation dans le sens habituel du Sud-Est au Nord-Ouest — jusqu’à nos jours du moins, car, à présent, elle se déplace à la fois et déborde dans de multiples directions — qui a tant étonné les philosophes de l’histoire. Si l’on admet — ce qui est loin d’être admis par tout le monde, il est vrai — que la civilisation a commencé  \phantomsection
\label{v1p21}à éclore au Midi et à l’Est, c’est à l’Ouest et au Nord que, de tout temps, se trouvaient les pays neufs, les pays privilégiés sous les deux rapports indiqués : virginité des ressources naturelles et affluence des capitaux formés dans les vieux pays. D’ailleurs, quand, par hasard, ces terres en friches étaient à l’Est et non à l’Ouest, au Sud et non au Nord, il y avait exception à la prétendue loi de rotation géographique. C’est pourquoi les rives orientales, encore plus qu’occidentales, de la Méditerranée, ont été colonisées et civilisées par les Athéniens, comme, à présent, l’Australie et l’Afrique le sont par nous.\par
Quoi qu’il en soit, le rapport des métropoles aux colonies, on le voit, peut être cité comme un exemple bien net — bien net parce qu’ici la similitude est due en partie à l’imitation — de la loi de répétition amplifiante qui nous occupe. Mais ce n’en est pas le meilleur. Car les civilisations coloniales, si elles agrandissent, en la répétant, la civilisation métropolitaine, en sont plutôt la vulgarisation superficielle, peu profonde et souvent peu durable, que la reprise perfectionnée. Sans doute, au point de vue extérieur, monumental et décoratif, les grandes villes commerçantes de la grande Grèce ou des côtes occidentales de l’Asie Mineure, dépassaient beaucoup la vieille Grèce : chez elles se trouvaient les temples les plus grands et les plus riches, mais non les plus vraiment beaux, pas le Parthénon. Et leur prospérité fut courte, pendant que les destins d’Athènes évoluaient toujours. Ceci soit dit sans oublier pourtant que Carthage, en répétant Tyr et l’agrandissant, lui a longtemps survécu. — Mais tout autre est la répétition, moins fidèle et plus profonde, d’une civilisation historique par d’autres civilisations historiques qui, à travers un temps de crise, de barbarie parfois, où se préparent les éléments d’une palingénésie sociale, la ressuscitent transfigurée. Il y a alors amplification à la fois et transformation originale.\par
On peut se demander si l’intervalle de crises, de guerres,  \phantomsection
\label{v1p22}de révolutions sanglantes, qui sépare ces floraisons successives de la sociabilité humaine, est un labour nécessaire à leur épanouissement. Est-ce que, sans la barbarie mérovingienne et la longue série de catastrophes qui ont suivi la chute ou plutôt les couches nocturnes de l’Empire romain, le phénomène inouï de la civilisation moderne aurait pu éclater ? Est-ce que, antérieurement, on aperçoit ou on entrevoit des intervalles analogues entre les éclosions civilisatrices qui ont apparu successivement ?
\subsubsection[{A.1.d. Conséquence nécessaire : l’unification finale du genre humain. Par voie impériale ou fédérative ? Importance ici du facteur géographique, de la forme sphérique de la terre. Hypothèse d’une terre plate, au point de vue économique.}]{A.1.d. Conséquence nécessaire : l’unification finale du genre humain. Par voie impériale ou fédérative ? Importance ici du facteur géographique, de la forme sphérique de la terre. Hypothèse d’une terre plate, au point de vue économique.}
\noindent Ce point d’interrogation soulèverait de longues discussions. Mais quelles que puissent être les solutions données à ce problème, après tout, secondaire, il ne résulte pas moins déjà du point de vue développé dans ce qui précède, que, s’il est justifié par les faits, il fournit une réponse claire et satisfaisante à la grande question de la Destinée humaine. On conçoit alors qu’Auguste Comte ait eu raison de regarder l’évolution de l’\emph{humanité} comme unique en somme. En effet, si les points de départ des évolutions des peuples sont multiples et leurs sentiers divers, on voit que, nécessairement, par toutes les pentes de l’histoire, ces courants convergent vers une même embouchure finale après un certain nombre de confluents intermédiaires, puisque, à force de s’agrandir, le domaine d’une civilisation finalement triomphante doit arriver à couvrir le globe entier.\par
Ainsi se réalisera, inévitablement, en sociologie, le rêve astronomique d’un doux mystique\footnote{ \noindent Bien oublié à présent, le P. Gratry.
 } qui, déplorant le morcellement des astres, des terres habitées éparses dans le firmament, imaginait leur pelotonnement paradisiaque à la fin des temps. S’il n’en sera pas ainsi des astres, il en sera ainsi au  \phantomsection
\label{v1p23}moins des cités humaines qu’une seule et même civilisation fraternelle, aux variantes nombreuses, enveloppera dans une paix féconde.\par
Cette nécessité, d’ailleurs, ne permet pas de prédire d’avance quelle sera la nature de l’état final ; elle n’a trait qu’à la \emph{quantité}, pour ainsi dire, non à la \emph{qualité} de cet état. Il est inévitable, d’après les principes d’où nous sommes partis, que, parmi les diverses formes de civilisation qui aspirent toutes à se propager indéfiniment, il y en ait une qui devienne prépondérante et, en s’appropriant toutes les autres, assouplies et assujetties, s’universalise. Il ne l’est pas que ce soit telle forme de civilisation plutôt que telle autre, — la forme anglaise ou russe, par exemple, plutôt que la forme française ou allemande. La nécessité en question se concilie avec la libre et pittoresque diversité, avec l’indétermination essentielle qui fait l’intérêt poignant de l’histoire. — Dira-t-on que, si un esprit d’une compréhension et d’une pénétration infinies pouvait connaître dans leur dernière intimité les phénomènes actuels, les êtres actuels, il y lirait l’inévitable avènement de tel dénoûment historique et non de tout autre, en sorte que l’imprévu, l’indéterminé, l’accidentel, serait entièrement banni de l’histoire ? Mais ce postulat est arbitraire. Si l’on y réfléchit, on sera stupéfait de la facilité singulière avec laquelle on l’admet, quoiqu’il échappe à tout essai de démonstration et qu’il repose sur une hypothèse qu’on sait irréalisable, inconcevable, impliquant contradiction au fond, celle d’un \emph{cerveau infiniment} intelligent. Sous cette évidence apparente, comme sous toutes les évidences \emph{a priori} en général, il y a une \emph{tautologie} cachée. Cela revient à dire que ce qui sera sera, proposition insignifiante.\par
Cependant, si nous ne pouvons prédire \emph{quelle} sera finalement la civilisation prédominante, nous pouvons conjecturer de quelle manière probablement s’opérera l’unification finale. Il y a à tenir compte d’un facteur très important à  \phantomsection
\label{v1p24}cet égard, quoique, à d’autres égards, son rôle ait été souvent exagéré, le facteur géographique. Et, avant tout, parmi les influences géographiques, il en est une à laquelle on n’a pas pris garde et qui me paraît destinée à jouer un rôle décisif dans le choix des moyens, belliqueux ou pacifiques, de parvenir à l’unité où le monde social aspire. Je veux parler de la forme même de l’habitat humain, de la sphéricité de la terre.\par
Si la terre était une surface plane, comme le croyaient les anciens, et non une surface sphérique, les problèmes que soulève la tendance de l’imitation, en tout ordre des faits, à une progression indéfinie, se poseraient tout autrement et comporteraient d’autres solutions. Il y aurait des États périphériques, dont le rayonnement imitatif, soit en fait d’institutions politiques, soit en fait de besoins, de produits, de mœurs, d’arts, etc., ne pourrait s’étendre que dans un sens, non dans tous les sens, à cause des bornes de la terre, des infranchissables colonnes d’Hercule, tandis que les États du centre jouiraient du privilège de pouvoir rayonner imitativement dans tous les sens, à l’Ouest, à l’Est, au Sud, au Nord. La région centrale de la terre serait donc celle qui, inévitablement, à la longue, servirait de modèle à toutes les autres, et ferait prévaloir sa forme sociale, étendue de proche en proche à tout l’univers. Mais, la terre étant sphérique, aucun avantage naturel de ce genre n’appartient à aucun État, puisque aucun point de la surface d’une sphère ne peut être considéré comme central par rapport aux autres. Il y a d’autres avantages naturels, celui, par exemple, d’appartenir à une zone plus tempérée, moins voisine des pôles ou des tropiques, mais cet avantage est partagé par une foule d’États placés sous la même latitude ; ils forment une longue ligne fort large, et non un point.\par
Dans l’hypothèse de la terre plane, le progrès des voies de transport aurait pour effet, non pas de niveler peu à peu les conditions géographiques de la mise en relation des divers peuples, mais au contraire d’accentuer de plus en plus  \phantomsection
\label{v1p25}leurs inégalités essentielles. Le réseau des chemins de fer, au lieu d’être ou de devenir un tissu qui tend à être à peu près aussi serré partout, ou du moins à présenter des centres multiples, toujours nombreux, de ramifications, qui vont s’anastomosant, serait ou tendrait à devenir de plus en plus une immense toile d’araignée, ayant un centre unique, comme l’est le réseau d’un seul État où tout converge vers la capitale.\par
Cette disposition géographique favoriserait, on le voit, le despotisme ; et, quand l’unité politique du genre humain s’établirait, — car elle est toujours inévitable, à raison de la tendance au rayonnement progressif des exemples et, par suite, des pouvoirs, — cette unité ne pourrait se réaliser que sous la forme \emph{impériale.} Mais la sphéricité de la terre favorise, au contraire, grandement, ou favorisera dans l’avenir (dans un avenir déjà rapproché), l’unification sous forme fédérative. — L’\emph{Empire}, remarquons-le, l’Empire, dont l’Empire romain a été le type le plus parfait, à jamais éblouissant dans la mémoire des hommes, suppose l’illusion de croire que la surface de la terre est plane, — comme elle l’était à peu près, en effet, dans cette faible portion de la planète, qui était « le monde connu des anciens ». L’Empire, c’est, essentiellement, une ville qui se croit et qui est crue le centre du monde, une \emph{urbs} qui projette son image prodigieusement agrandie dans tout un \emph{orbs}, dans un cercle dont \emph{l’urbs} est le centre. Les parties de cet \emph{orbs} constituent une hiérarchie toute naturelle dont les degrés sont mesurés par leur éloignement ou leur voisinage de l’\emph{urbs.}\par
Ce n’est pas que, même sur notre terre sphérique, la domination universelle d’un seul État ne \emph{puisse} s’établir \emph{momentanément ;} et le rêve impérialiste des Anglais, en ce moment, n’a rien d’insensé. Les autres nations auraient tort de ne pas le prendre au sérieux. Mais, supposons cette ambition britannique réalisée, est-ce que cet exemple ne susciterait pas des tentatives d’imitation parmi les peuples  \phantomsection
\label{v1p26}subjugués ? Est-ce qu’il n’y aurait pas quelqu’une des capitales soumises à Londres qui éprouverait à son tour le besoin d’expansion rayonnante d’abord sous forme économique, puis sous forme politique, comme Byzance, Alexandrie, et d’autres villes de l’Asie Mineure ou de la Gaule ont pu rêver en secret sous l’Empire romain de rivaliser avec Rome en richesse d’abord, puis en pouvoirs ? Seulement, ce rêve, chez celles-ci, a dû rester comprimé, par l’impossibilité manifeste de trouver à se développer autrement que \emph{du côté} de Rome même, — car, \emph{de l’autre côté}, du côté non encore romanisé, c’était, sinon la fin de la terre, du moins, ce qui revient à peu près au même, la fin du monde civilisé, la barbarie ou la demi-barbarie. Mais, dans le siècle où nous allons entrer, rien de pareil : les rivales de Londres pourraient donner carrière de tous côtés à leur besoin d’expansion, comme Londres elle-même, et, par suite, acquérir autant de richesse et de pouvoir.\par
On ne voit donc d’unité stable et de paix durable du genre humain dans l’avenir que moyennant une \emph{fédération} de \emph{quelques} nations gigantesques.\par
La terre étant ronde, le trajet de la civilisation dans un sens quelconque, à force d’aller, finit toujours par revenir sur lui-même. Tous les \emph{rayons d’exemples} finissent par s’y \emph{réfléchir}. Si elle était plate, le déplacement de la civilisation serait son éloignement progressif et sans retour, de son point de départ, et rien n’y contraindrait l’imitation à revenir à sa source.\par
— Les conséquences économiques de la platitude terrestre seraient considérables. Par exemple, si, à un moment donné, une région se trouvait présenter des conditions exceptionnelles pour la production du blé à bon marché — comme, de nos jours, le sud de la Russie ou l’ouest des États-Unis ou bientôt la vallée de la Plata, ou plus tard celle du Niger — elle ne pourrait facilement inonder le monde de ses céréales, et provoquer une crise agricole intense qu’à la condition  \phantomsection
\label{v1p27}d’être \emph{centrale et non périphérique.} Périphérique, elle aurait, pour faire parvenir ses produits à la région diamétralement opposée, à supporter des frais de transport doubles de ceux qui incomberaient à une région centrale. — Il en serait de même pour une industrie quelconque.\par
En somme, une terre plate conduirait à l’inégalité croissante entre les États et entre les hommes ; une terre ronde conduit à une égalité croissante. Par égalité, j’entends \emph{réciprocité.} La loi du \emph{passage de l’unilatéral au réciproque} est grandement aidée dans son action par la sphéricité de la terre.\par
On peut concevoir ou imaginer une terre plate qui, sans avoir une surface \emph{infinie} (chose inconcevable), aurait une surface prodigieusement étendue, extrêmement supérieure à la surface ronde de notre globe, c’est-à-dire telle qu’il serait \emph{pratiquement impossible} d’en atteindre les limites. Ce serait comme si elle était infinie. Dans cette hypothèse, l’expansion rayonnante d’une civilisation quelconque ne saurait jamais dépasser un certain rayon, plus ou moins étendu, suivant l’état des moyens de locomotion et de communication, mais toujours fort petit eu égard à la quasi-immensité terrestre. Au delà de ce rayon, d’autres civilisations pourraient rayonner — contiguës parfois, d’autres fois distantes — ou plutôt d’abord très distantes, puis, à mesure que les moyens de communication se perfectionneraient dans chacune d’elles, de moins en moins distantes, et enfin quasi contiguës. De là des conflits fréquents, des conquêtes et des annexions ; mais conquêtes et annexions qui ne sauraient jamais constituer durablement un empire excédant le \emph{rayon-limite} dont je viens de parler... Un État conquérant ne \emph{pourrait donc conquérir toujours qu’à la condition de se déplacer sans cesse}, de perdre d’un côté (par l’impossibilité pratique d’y maintenir sa domination) ce qu’il gagnerait de l’autre... Jamais donc, il n’y aurait de terme assignable à l’ère des guerres, et c’est alors qu’on serait en droit de regarder l’espoir d’une paix  \phantomsection
\label{v1p28}finale, d’une paix romaine généralisée, étendue à la terre entière, comme une utopie. En effet, un État aurait beau s’étendre, il trouverait toujours des voisins, avec qui des causes de conflit ne manqueraient pas de naître... Mais, sur notre terre sphérique, et qui, en outre, n’est pas d’un volume disproportionné à nos moyens de locomotion et de communication mentale, on peut, sans chimère, espérer la fin des batailles, et assigner pour terme à l’ère belliqueuse le moment où une seule et même civilisation, susceptible de variations infinies, et fractionnée en nationalités diverses, mais alliées et solidaires, régnera sur le \emph{globe.}\par
C’est donc, au point de vue soit de la justice, soit de la paix, une chose bienfaisante que la sphéricité de la terre ; et je ne serais pas loin d’y soupçonner une de ces grandioses harmonies naturelles spontanées dont la cause nous échappe, et qui nous émerveillent à la pensée que ces conditions favorables à la justice et à la paix ne sont pas particulières à notre humanité terrestre, mais qu’elles sont communes à toutes les humanités dispersées dans les innombrables planètes des mondes stellaires, à toutes les sociétés lancées comme la nôtre dans le fleuve épars, sanglant ou boueux, d’une évolution historique aux bras multiples, mais convergeant vers une embouchure profonde et paisible\footnote{ \noindent Est-ce que les considérations ci-dessus ne nous conduiraient pas peut-être à penser, — à conjecturer — en songeant aux analogies entre les trois formes de la répétition et de l’expansion universelle — que l’\emph{espace} est un \emph{espace sphérique} et non un espace plan — et que, en cela, les \emph{métagéomètres} ont eu, à travers bien des obscurités, une intuition profonde ?
 }.
\subsubsection[{A.1.e. De là, division tripartite de l’histoire humaine. Les trois grandes phases de la vie de l’humanité.}]{A.1.e. De là, division tripartite de l’histoire humaine. Les trois grandes phases de la vie de l’humanité.}
\noindent Des considérations qui précèdent découlent une conception et une division nouvelles de l’histoire. L’histoire apparaîtra sous un jour tout autre, plus simple et plus intelligible, si  \phantomsection
\label{v1p29}l’on se pénètre de cette idée que tous les efforts de toutes les sociétés embryonnaires et arrêtées dans leur développement jusqu’à ce jour, tendaient inconsciemment au débordement d’une civilisation qui couvrît tout le globe. Ce but ne pouvait être atteint avant les grandes inventions de notre âge, relatives à la locomotion rapide et à la communication instantanée de la pensée. Or, ce but atteint, une nouvelle période de l’histoire humaine s’ouvrira, infiniment plus intéressante et sans doute plus régulière dans son déroulement que la précédente. Il ne s’agira plus seulement, ni surtout, de s’étendre, pour une variété de civilisation nouvellement apparue, pour un nouveau type qui surgira de l’ancien ; car cette extension sera relativement facile et aura vite atteint sa limite infranchissable, le tour complet de notre petit globe. Mais, cela fait, il s’agira pour ce type, et ce sera là sa tâche la plus difficile et la plus chère, de se développer logiquement, avec une pleine harmonie, d’empêcher la naissance des types hostiles et de susciter tous les perfectionnements d’accord avec son principe essentiel. Alors le rêve d’Auguste Comte, l’établissement d’une immense Église, non pas rigide et cristallisée comme il la voulait, mais plastique et extensible à l’infini, pourrait bien devenir réalisable jusqu’à un certain point. Ce qu’on peut dire, c’est qu’à une évolution toute de luttes et d’accidents successifs, de rencontres extérieures, aura succédé un \emph{développement interne.}\par
Quand tout sera colonisé — car le globe n’est pas illimité, et déjà ses bornes sont touchées partout — il faudra bien par force que la fièvre coloniale s’arrête, épuisée, et que les nations civilisées cherchent un autre dérivatif de leur ambition, de leur besoin d’expansion économique ou politique. Qu’adviendra-t-il à ce moment ? Jusqu’ici, — depuis trois siècles au moins, ou plutôt de tout temps, car, avant la découverte de l’Amérique et de l’Océanie, c’était l’ancien continent presque tout entier qui était à découvrir peu à peu  \phantomsection
\label{v1p30} — les colonies, les terres neuves, ont servi d’exutoire et de purgation à nos humeurs malignes, autant que de refuge à nos indépendances, ou de mirage à nos chimères et de proie à nos ambitions. Que deviendrons-nous quand toute la férocité, toute l’avidité, tout le désordre, et aussi bien toute l’activité conquérante, toute la générosité remuante, que nous exportons en Afrique, en Extrême-Orient, dans l’Amérique du Sud, sera refoulée en nous et fermentera dans notre propre sein ? Ce sont de grandes questions qu’il est déjà permis de poser.\par
Comme corollaire de ces vues, une division tripartite de l’évolution humaine s’offre à nous. Trois phases doivent être distinguées dans la vie de l’humanité :\par
En premier lieu, la phase préhistorique, d’une durée prodigieuse et incalculable, où les groupes sociaux étaient si petits et si épars, et, vu l’absence de moyens suffisants de communication, si éloignés les uns des autres, que leur distance, pratiquement infinie, comme celle des systèmes stellaires, équivalait à leur isolement absolu. En second lieu, la phase intermédiaire, où nous entrons dès les premières heures de l’histoire, où nous nous débattons encore douloureusement, et dans laquelle les groupes humains, à force de grandir séparément, se sont touchés, se sont alliés ou heurtés, et, à travers des guerres d’abord de plus en plus fréquentes et meurtrières, puis de moins en moins fréquentes mais de plus en plus formidables, s’acheminent soit vers une immense fédération, soit vers un Empire gigantesque. En troisième lieu, la phase qui suivra le moment où, l’unité de domination politique, sous forme impériale ou fédérative, s’étant établie sur le globe entier, la guerre — du moins la guerre extérieure — sera close à jamais, où il n’y aura plus de terres à explorer ni à coloniser, où, jusqu’au cercle polaire, jusqu’au cœur de l’Afrique, tout sera civilisé, pacifié, régi souverainement.\par
C’est à cette troisième phase que se poseront et se formuleront \phantomsection
\label{v1p31} avec une rigueur et une acuité toutes nouvelles les profonds problèmes sociaux, prématurément agités par les fractions les plus avancées des écoles socialistes. Les poussées successives du socialisme, dans le passé, et surtout dans notre siècle, se sont conformées à la loi de répétition amplifiante\footnote{ \noindent Bien d’autres poussées — celles du néo-catholicisme en fait de religiosité, celles du naturalisme ou du réalisme littéraire, celles de l’esprit d’entreprise en fait d’industrie ou de spéculation financière, etc. — se conforment à la même loi. — Est-ce que cette loi n’embrasse pas et n’explique pas tous les faits qui servent d’appui apparent à une prétendue loi de retour aux formes anciennes, de symétrie des extrêmes, d’évolution en spirale, dont certains sociologues font état ? — Cette loi de répétition amplifiante ne fait point pendant en sociologie à la loi biologique de la répétition de la \emph{phytogenèse} par \emph{ontogenèse}. C’est plutôt l’inverse de cette dernière loi qui a trait à une répétition \emph{abrégée} et non amplifiée. Et c’est pour le moins aussi bien démontré.
 }. L’accès qui nous agite est certainement plus vaste et plus fort que celui de 1848 qui était aussi très supérieur en intensité à celui de 1830. Il est à croire que bien plus majestueuse encore et entraînante sera la reprise du même effort intermittent de rénovation sociale dans l’ère future dont je parle. Aucun souci de politique extérieure ne venant détourner les esprits des questions de politique intérieure et de réorganisation sociale et empêcher de les creuser à fond, de les pousser à bout, — aucun dérivatif colonial ne s’offrant plus aux instincts féroces ou déprédateurs, aux aspirations inquiètes, — la conquête du pouvoir par un parti ou l’harmonie définitive et la fusion absolue des classes, seront le but fixe, le but obsédant et persécuteur des grands ambitieux. Alors aussi le problème de la population se posera dans de tout autres termes : non sous sa forme quantitative, mais sous sa forme qualitative. Il ne s’agira plus d’accroître mais d’améliorer les races humaines.\par
Il faut bien prendre garde, quand on agite la question sociale, et en particulier les questions d’ordre économique, à ne pas oublier qu’elles comportent des réponses très différentes suivant qu’on a égard à la seconde ou à la troisième des trois phases que je viens de distinguer. Beaucoup de  \phantomsection
\label{v1p32}théories n’ont que le tort de se croire applicables à la seconde tandis qu’elles ne le seront qu’à la troisième, dans une certaine mesure, bien entendu.
\subsubsection[{A.1.f. Considérations générales à propos de l’adaptation, de la répétition, de l’opposition. Les trois formes de la répétition universelle, de l’opposition universelle, de l’adaptation universelle.}]{A.1.f. Considérations générales à propos de l’adaptation, de la répétition, de l’opposition. Les trois formes de la répétition universelle, de l’opposition universelle, de l’adaptation universelle.}
\noindent Revenons en arrière pour reprendre quelques-unes des idées générales exposées plus haut et en mieux pénétrer le sens. Insistons un peu sur les rapports généraux de l’adaptation et de la répétition qui, dans le monde vivant et le monde physique même, ainsi que dans le monde social, nous apparaissent comme les deux grands faits capitaux. Partout, avons-nous dit, ce sont des harmonies qui se répètent ; car une onde est vraiment une suite harmonieuse de mouvements, un équilibre mobile revenant sur soi, comme une phrase musicale ; et il en est de même, avec une complexité supérieure, d’un être vivant, onde très compliquée, pourrait-on dire, qui naît, grandit et décroît à l’instar d’une onde sonore ou liquide, et dont l’état adulte correspond à ce que les physiciens appellent le \emph{ventre} de l’onde. Pareillement, un acte d’imitation quelconque, un mot ou une phrase qu’on prononce, une pratique religieuse qu’on accomplit, un travail qu’on exécute, une formalité juridique qu’on remplit, etc., est un \emph{tout} qui a son commencement, son milieu et sa fin, et qui consiste en une série de changements agrégés et solidaires, gravitant autour d’un centre de gravité, la syllabe accentuée du mot ou le mot accentué de la phrase, l’acte essentiel dans la cérémonie rituelle, le point principal du travail ou de la procédure, etc.\par
Comment se sont formées ces harmonies ? Il importe beaucoup de ne pas confondre ce problème avec celui de savoir comment elles se répètent ; confusion que nous trouvons cachée au fond de l’idée darwinienne. On a fait justement observer à Darwin que la lutte pour la vie présuppose  \phantomsection
\label{v1p33}l’association pour la vie, c’est-à-dire l’\emph{organisation} que les répétitions multipliantes des organismes et les luttes qui s’ensuivent n’expliquent pas mais impliquent. C’est comme si on espérait expliquer par des chocs d’ondulations la formation des molécules chimiques qui ondulent, ou par des concurrences d’imitations la genèse des inventions imitées.\par
Ce n’est pas qu’il n’y ait à considérer ces luttes, ces chocs, ces concurrences, ces \emph{oppositions}. Et, de fait, l’\emph{opposition} est un fait général, moins général cependant que l’adaptation et la répétition et qui doit prendre place entre eux, au-dessous d’eux, en sa qualité de simple auxiliaire ou de simple intermédiaire, fréquemment nécessaire. Les harmonies qui se répètent, en effet, peuvent quelquefois, par exception, s’harmoniser entre elles directement, par le seul fait de leur rencontre, grâce à leurs répétitions multipliantes, et former ainsi des adaptations plus hautes ; mais, le plus souvent, elles s’opposent sous quelque rapport et, par là, à force de froissements et de mutuelles corrections, préparent le terrain pour les harmonisations supérieures. Ces heurts et ces froissements ont conditionné et provoqué celles-ci, elles ne les ont pas causées. Partout, dans le monde vivant et social, dans le monde physique même, nous voyons des choses harmonieuses qui, en se multipliant, entrent en lutte les unes contre les autres, des adaptations qui s’opposent : microbes contre cellules, organismes contre organismes, corporations contre corporations, États contre États, molécules contre molécules, conflagration qui précède la combinaison chimique et qui paraît être une multitude innombrable de chocs, etc. Et partout aussi, au résultat de cette crise, nous voyons des oppositions qui s’adaptent : phénomènes de commensalisme et d’acclimatation, fécondations, traités d’alliance, combinaisons chimiques. Le passage du militarisme à l’industrialisme peut être encore cité comme un exemple remarquable de la même transformation.\par
Puisque nous avons distingué trois grands cercles concentriques \phantomsection
\label{v1p34} de réalité — qu’il serait facile, à la vérité, de subdiviser et aussi bien de synthétiser — à savoir le cercle physique, le cercle vivant et le cercle social (ces deux derniers bien plus étroitement liés entre eux qu’avec le premier), nous n’aurons pas de peine à remarquer en chacun d’eux une forme d’adaptation, une forme de répétition et une forme d’opposition, qui la caractérise et y prédomine\footnote{ \noindent On peut commencer par le terme que l’on veut la série des trois termes énumérés. Ici je préfère commencer par l’\emph{adaptation} suivie de la répétition et de l’opposition. C’est plus logique puisque la répétition et l’opposition supposent quelque chose qui puisse se répéter et s’opposer à soi, et ce quelque chose ne peut être qu’un agrégat, un \emph{adaptat} (je demande pardon pour ce néologisme). Ailleurs, j’ai préféré, pour des raisons didactiques, débuter par la \emph{répétition}. En réalité la suite de ces termes forme une chaîne sans fin et dont le commencement peut nous échapper.
 }. Le type de l’adaptation physique, c’est la combinaison chimique, avec son équilibre interne de mouvements enchaînés dont l’équilibre mobile d’un système solaire n’est peut-être que l’amplification grandiose. Car la loi d’agrandissement semble applicable à la nature extérieure. Le type le plus parfait de l’adaptation vivante, n’est-ce pas la fécondation, l’accouplement fécond d’où résulte une nouvelle variété ou une nouvelle race viable ? Et le type propre, élémentaire, de l’adaptation psycho-sociale, n’est-ce pas l’invention, j’entends l’invention viable, imitable, qui commence par lier les idées pour finir par lier les hommes ? Car, à l’origine de toute association entre les hommes, il y a une association entre des idées, qui l’a rendue possible ; et tout ce qui s’opère par collaboration maintenant a été dû d’abord à une conception et à une opération individuelle ; ce qu’on oublie quand on attribue au travail collectif, à la division et à l’association soi-disant spontanées des travaux, les merveilles créées par le génie individuel. Si des centaines d’ouvriers collaborent dans un même atelier, c’est parce que la tâche qu’ils exécutent de concert, tissage, métallurgie, céramique, a été primitivement conçue dans son entier par un inventeur illustre ou obscur et repensée à nouveau par  \phantomsection
\label{v1p35}l’entrepreneur. Et si des centaines d’ateliers collaborent, sans nulle direction d’ensemble, avec une apparence de spontanéité, à une même fabrication, par exemple à la fabrication d’une locomotive ou d’une étoffe de soie, c’est parce que la locomotive ou l’étoffe de soie a été inventée par quelqu’un.\par
La différence marquée entre les trois formes de l’harmonie universelle, que nous venons de rapprocher, c’est que les deux premières restent très mystérieuses pour nous tandis que la nature de la troisième est assez claire. Toutes les fois qu’il y a invention, cela signifie soit que des faits auparavant étrangers les uns aux autres ou paraissant tels (le mouvement de la lune et la chute d’une pomme, l’étincelle électrique et la foudre, etc.), ont été aperçus comme des conséquences d’un même principe, comme des confirmations d’une même proposition, c’est-à-dire comme des affirmations diverses de la même chose au fond, — soit que des procédés et des outils, jusque-là inutiles les uns aux autres (le rail, la roue, la machine à vapeur, — l’aiguille à coudre et la pédale, le courant électrique et l’écriture, etc.), ont été mis dans un rapport tel qu’ils se sont servis réciproquement de moyens en vue d’une même fin, c’est-à-dire qu’ils ont répondu au désir d’une même chose. Il nous apparaît clairement ici que l’accord social, l’harmonie sociale, consiste, au fond, en un faisceau de jugements qui affirment la même idée, ou d’actions qui impliquent la poursuite d’un même but. — Mais qu’est-ce que l’accord harmonieux de mouvements invisibles, produit par la combinaison chimique ? Est-il sûr que ce soit une gravitation commune autour d’un même centre ? Et n’est-ce que cela ? Et l’harmonie des fonctions organiques, née de l’ovule fécondé, savons-nous mieux ce qu’elle est ? Convergence vers une même fin ? ou vers des fins multiples et associées ? ou concordance d’une autre sorte, peut-être, que nous aurions tort de confondre avec la finalité, et qui différerait de celle-ci comme une  \phantomsection
\label{v1p36}théorie diffère d’une machine ? Rien de plus conjectural.\par
Il y a aussi trois formes de la Répétition universelle, que j’ai distinguées depuis longtemps. La forme physique est la plus répandue, c’est l’ondulation qui, de ses rayonnements sphériques et entre-croisés, remplit l’immense éther pendant que de ses rayonnements plus entravés et polarisés ou de ses séries linéaires elle pénètre les corps jusque dans les profondeurs de leurs intimités. La force vivante est la génération, qui, de ses populations grandissantes et concurrentes d’animaux et de plantes, couvre le sol, l’air et les mers. La forme sociale est l’imitation. — Sur les analogies et les différences de ces trois formes typiques, je renvoie aux ouvrages où j’en ai parlé. Mais je me permets de rappeler combien leur comparaison est propre à nous éclairer sur l’impérieux et irrésistible besoin qui pousse la nature à se répéter, à multiplier, pour en exprimer le riche contenu virtuel, les exemplaires de ses œuvres. Les procédés répétiteurs qu’elle a employés nous stupéfieraient d’admiration si nous prenions la peine de regarder ce que nous ne cessons de voir. Que les multiples mouvements périodiques d’une même planète et jusqu’aux moindres perturbations où se réflètent en elle les mouvements périodiques de toutes les autres du même système, se mêlent sans se confondre et se répètent sans se lasser ni s’effacer ; qu’une vibration sonore de l’air, avec ses moindres dentelures particulières, se répète indéfiniment sans se déformer, qu’elle soit portée de la bouche parlante à l’oreille écoutante à travers le fil téléphonique, par une série encore plus extraordinaire de vibrations électriques reproduites exactement avec leurs infinitésimales et innombrables particularités, cela est déjà bien merveilleux. Mais ce n’est rien auprès du miracle de l’hérédité vivante qui, des milliers et des millions de fois, pendant des siècles de siècles, avec une inconcevable fidélité, reproduit dans leurs détails les plus délicats itinéraires si compliqués de l’évolution embryonnaire, les caractères et les fonctions d’une espèce ;  \phantomsection
\label{v1p37}qui cache dans un ovule, dans un atome d’ovule pour ainsi dire, le cliché de ces reproductions, et l’y perpétue endormi à travers plusieurs générations successives jusqu’au jour d’un réveil inopiné. Et tout aussi prodigieux est le phénomène de l’imitation, mémoire sociale, qui, comme la mémoire, imitation interne, dont elle est l’agrandissement énorme, ressuscite et multiplie ce que l’hérédité elle-même est impuissante à reproduire, des états intimes, des idées et des volontés ; l’imitation qui fait la permanence séculaire des coutumes et des mœurs, des langues et des religions, l’identité des racines verbales passées de bouche en bouche, des rites sacramentels, passés de fidèle à fidèle ! — De la première à la seconde, de la seconde à la troisième, de ces trois formes de la répétition, on la voit croître à la fois en compréhension et en pénétration, en exactitude et en liberté. La génération est une ondulation plus compliquée et plus profonde dont les ondes sont détachées ; l’imitation est une génération sans nul contact, une fécondation à distance, qui dissémine les germes d’idées et d’actions bien plus loin encore que les germes vivants, et permet au modèle mort, après les plus longues durées d’enfouissement, de susciter encore des exemplaires de lui-même, agissants, animés, capables de révolutionner le monde.\par
Il y a enfin des formes d’oppositions spéciales à chaque sphère de la réalité. La forme la plus nette de l’opposition physique est le \emph{choc}, la rencontre de deux mouvements diamétralement contraires, sur la même ligne droite. La forme la plus aiguë de l’opposition vivante est le \emph{meurtre}, dans le sens le plus général du mot, qui comprend l’étouffement d’une plante par une autre, ou la manducation d’une plante par un animal aussi bien que le duel mortel de deux bêtes. La forme la plus violente de l’opposition sociale est la \emph{guerre}, qui, en apparence, n’est qu’un duel animal agrandi, mais qui, au fond, en diffère beaucoup par la nature et la conscience précise de sa cause interne : la \emph{contradiction \phantomsection
\label{v1p38}} des jugements ou la \emph{contrariété} des desseins en présence.
\subsubsection[{A.1.g. Lois communes à chacune de ces trois formes dans chacun de ces trois aspects de l’évolution. Principe d’irréversibilité. L’importance de l’accident.}]{A.1.g. Lois communes à chacune de ces trois formes dans chacun de ces trois aspects de l’évolution. Principe d’irréversibilité. L’importance de l’accident.}
\noindent On peut formuler des considérations générales, — appelons-les des lois, si l’on tient à ce vocable un peu abusif, commode d’ailleurs comme tous les monosyllabes — à propos des diverses formes de l’adaptation, de la répétition, de l’opposition. La loi générale des trois formes de la répétition, nous l’avons vu, c’est leur tendance commune, le plus souvent entravée, à la multiplication indéfinie. La même loi d’agrandissement progressif s’applique nécessairement aux trois formes de l’opposition et aux trois formes de l’adaptation puisque les oppositions et les adaptations, se multiplient avec les répétitions dont elles sont les interférences-luttes ou les interférences-alliances. — L’opposition a donné lieu à une autre généralisation bien simple qui a eu un immense succès, bien supérieur, je crois, à son mérite explicatif : toutes les formes de luttes, vivantes, sociales ou même physiques, aboutissent au triomphe du plus fort, qu’on appelle en biologie la survivance du plus apte. Cela est aussi vrai des concurrences de forces physiques ou d’affinités chimiques, que des concurrences d’espèces ou des concurrences de nations. Il suit de là une sélection tour à tour physique, vivante, sociale, qui possède, avant tout, une vertu essentiellement éliminatrice, épuratrice, nullement créatrice.\par
L’adaptation n’aurait-elle pas aussi ses lois générales ? Une tendance à l’\emph{accumulation} croissante, malgré des substitutions et des destructions fréquentes, n’est-elle pas commune à ses trois formes, à la combinaison chimique, qui va se compliquant depuis les corps réputés simples jusqu’aux  \phantomsection
\label{v1p39}substances organiques, — à l’accouplement fécond, qui est un trésor sans cesse grossi de legs héréditaires depuis les monères jusqu’à l’homme, depuis le champignon jusqu’au mammifère le plus élevé, — à l’invention enfin, qui, formée d’un faisceau d’inventions antérieures, y greffe une idée nouvelle, destinée à servir de porte-greffe à son tour, et ainsi de suite, depuis l’invention préhistorique du levier ou de la roue jusqu’à celle des machines les plus perfectionnées de notre temps ? Chaque invention, en effet, ajoute quelque chose aux anciennes qu’elles synthétise encore plus qu’elle ne s’y substitue, de même que chaque espèce ou chaque race nouvelle, créée par une succession de petites innovations vitales dues à des mariages heureux, est une synthèse d’espèces et de races antérieures, — et de même que les corps chimiques nouveaux sont des complications de corps chimiques antérieurs.\par
Non moins que ce principe d’accumulation, le principe d’\emph{irréversibilité}, qui en dérive, me paraît s’appliquer à toutes les formes de l’adaptation. Nous n’apercevons pas de loi qui prédétermine l’apparition de toutes les inventions, ou de la plupart d’entre elles, à telle date plutôt qu’à telle autre, ici plutôt que là ; rien ne nous empêche de supposer que la boussole eût été découverte deux ou trois siècles plus tôt ou plus tard, et aussi bien l’Amérique, ou l’imprimerie, ou l’électricité. Mais, d’une invention quelconque, nous pouvons dire qu’elle ne pouvait naître avant telle autre qui l’a précédée et provoquée ; et, dans une large mesure, nous savons, à n’en pouvoir douter, qu’il y a un enchaînement logique des découvertes et des inventions, c’est-à-dire un ordre irréversible de leur apparition. Or, non moins que la série des idées et des progrès humains, la série paléontologique des faunes et des flores successives, est conçue comme essentiellement irréversible\footnote{ \noindent Sans cela, comment comprendre que la série des phases embryonnaires répète, dans une certaine mesure, l’ordre de succession des espèces antérieures ? On ne peut s’expliquer le mystère de cette répétition, même abréviative, qu’en supposant que la suite paléontologique des espèces d’où procède celle de l’individu considéré est une suite \emph{logique} avant tout, une sorte de déduction biologique, qui, comme telle, doit être recommencée \emph{ab ovo}, en s’abrégeant, à chaque nouvelle génération.
 }, et, aussi bien, la série astronomique  \phantomsection
\label{v1p40}ou géologique des formations chimiques. Il n’est pas jusqu’aux transformations des forces physiques qui, d’après la thermo-dynamique, ne soient lancées sur une pente impossible à remonter, par leur conversion finale en chaleur. Ainsi, par ces trois aspects à la fois, la vie universelle nous apparaît comme \emph{ayant un sens} et une raison d’être.\par
Ce principe d’irréversibilité a une signification si haute qu’on pourrait être tenté d’exagérer sa portée. Il convient de bien le circonscrire pour bien le comprendre. Car, nettement compris, il laisse encore très large la part de l’accident individuel, du génie et de l’initiative personnelle, dans les destinées sociales. L’irréversible est loin d’être une règle sans exception. Il est des séries de découvertes qui peuvent être toujours conçues comme s’étant suivies dans un ordre précisément inverse. Telles sont les découvertes géographiques. Cependant elles ont aussi cela de particulier qu’elles se déduisent nécessairement les unes des autres, qu’on est porté fatalement de l’une d’elles je ne dis pas à \emph{une} autre mais à d’\emph{autres}, dans leur voisinage immédiat. Si donc un sociologue s’avisait de chercher une formule de l’évolution des découvertes géographiques, on peut être sûr qu’il perdrait son temps. En revanche, il est certain que, \emph{par n’importe quelle évolution de découvertes} géographiques, suffisamment prolongées, on devait inévitablement aboutir au tracé définitif d’une carte du monde plus ou moins conforme dans ses principaux traits à celle qu’étudient nos écoliers. Par quelque point qu’on eût abordé en Amérique et commencé à explorer ses côtes, la série, quelle qu’elle eût été, des explorations poussées à bout aurait conduit les navigateurs de n’importe quelle nation ou de n’importe quelle race à la carte d’Amérique qui nous est connue. Tous les chemins,  \phantomsection
\label{v1p41}tous réversibles essentiellement, auraient conduit inévitablement à ce résultat. Et ce que je dis là, on peut le dire aussi bien des découvertes qui ont trait à l\emph{’anatomie}, à la \emph{cristallographie}, et, en général, à toutes les sciences purement \emph{descriptives} d’un tout nettement circonscrit, dont la forme seule importe.\par
Mais en est-il de même des découvertes physiques, chimiques, biologiques, psychologiques ? Est-ce que, par n’importe quelle évolution de découvertes, nous serions arrivés ici à des corps de sciences semblablement constitués, à des théories pareilles ? Plus se multiplient et se diversifient les découvertes géographiques, et plus leur aboutissement commun, la mappemonde, va s’identifiant. Pouvons-nous dire aussi bien que, plus les voies de la recherche scientifique sont variées, et plus les théories se fusionnent et s’unifient ? Conjecturera-t-on que les réalités physico-chimiques, vivantes ou autres, sont, par rapport aux êtres sensibles et plus ou moins intelligents qui cherchent à les connaître et à les pénétrer, de l’amibe au cerveau humain, en passant par toute la gamme psychologique de l’animalité, ce que les continents et les mers sont par rapport aux diverses peuplades ou nations humaines qui se sont efforcées de les explorer ? Dira-t-on que le point de départ des découvertes géographiques, pour chaque peuple qui entre dans la voie des explorations, à savoir le lieu où il se trouve quand ces ailes lui poussent, lui est imposé aussi nécessairement que l’est, pour tout esprit animal ou humain, qui commence à ramasser des connaissances, le point de départ de ses petites ou grandes découvertes astronomiques, chimiques, botaniques, zoologiques ?... A ce point de vue, la série des découvertes géographiques devient irréversible : elle l’est relativement à ce peuple et à son point de départ obligatoire ou réputé tel ; et cette irréversibilité ne diffère en rien d’essentiel de celle des découvertes physico-chimiques, vivantes ou autres, faites par un animal donné.\par
 \phantomsection
\label{v1p42}Chacun des côtés intelligibles de l’univers, en effet, — la réalité astronomique, la réalité chimique, la réalité vitale, la réalité mentale, — peut être considéré comme un continent dont il s’agit pour l’esprit investigateur de faire l’exploration entière. Chaque esprit le touche par un point où il naît placé ; ce point, ce sont les données immédiates de ses sens ; et de ce point il part pour découvrir dans son voisinage le plus rapproché, puis un peu plus loin ; car il y a un voisinage logique et psychologique comme un voisinage géographique, et la série des découvertes quelconques est forcée de parcourir successivement ces degrés de proximité rationnelle ou locale. [{\corr La}] loi suprême de l’invention, comme celle de l’imitation, est d’aller ainsi de proche en proche. A la vérité, il ne suit pas de là que, ce point de départ des investigations scientifiques étant supposé commun à tous les peuples, l’itinéraire de leurs recherches dût être le même ; car, d’un lieu donné, on peut rayonner dans une infinité de sens ; et rien n’empêche d’admettre que, pendant longtemps, les sentiers ou les routes partis de ce même lieu, iront divergeant de plus en plus. Mais, si la comparaison employée est juste, ne faut-il pas concevoir aussi que, passé un certain maximum de divergence, ces chemins iront se confondant de plus en plus dans un même croquis définitif et total du vrai ?\par
Eh bien non, cela n’est pas certain. Car, si l’ordre de succession de nos découvertes géographiques, — ajoutons, si l’on veut, de nos découvertes anatomiques ou cristallographiques, ou d’autres semblables — n’a qu’une importance, après tout, secondaire, c’est que les continents terrestres sont limités ainsi que le nombre des pièces du squelette d’un animal ou le nombre des formes cristallines, et que leurs dimensions n’excèdent pas les forces exploratrices de l’homme. Sans cela, la connaissance, toujours fragmentaire, que l’homme pourrait acquérir des contours terrestres différerait grandement suivant le point d’où seraient partis  \phantomsection
\label{v1p43}ses navigateurs. Or, il est fort possible que ces continents, ces océans, ces forêts vierges des faits chimiques, astronomiques, biologiques, sociaux même, où notre curiosité anxieuse s’est jetée, soient illimités, c’est-à-dire que nulle série de découvertes n’y puisse être \emph{poussée à bout}, faute de bout, Le problème de l’\emph{infini réel} se dresse ainsi devant nous ; et, si on le résout par l’affirmative, cette solution entraîne, on le voit, la \emph{relativité}, je ne dis pas la \emph{subjectivité}, de la science. Mais nous n’avons pas besoin de la résoudre ; car il se peut aussi que, bien que ces étendues aient des limites, l’homme reste à jamais impuissant à les rencontrer, tant il y a de disproportion entre ces domaines grandioses et les pas minuscules de l’animalcule humain. Et, dans un cas comme dans l’autre, la différence du point de départ ou celle du chemin parcouru aura pour effet de changer profondément le résultat final, la science devenue inextensible et réputée définitive.\par
Il découle de là une conséquence digne d’attention. D’une part, rien ne prouve qu’un même point de départ s’imposait aux recherches des premiers chercheurs mathématiciens, astronomes, physiciens, naturalistes, embryonnaires. Et, de fait, dans le très petit nombre de cas où une science, telle que l’astronomie, nous laisse entrevoir des origines multiples, indépendantes les unes des autres, nous remarquons la dissemblance des données premières. La vérité est que, de très bonne heure, ces petites sources distinctes des futures sciences ou n’ont pas tardé à tarir, ou se sont promptement jetées, avec la source principale, chaldéo-égypto-hellénique, dans le grand fleuve du progrès moderne des sciences. Fleuve unique, par malheur, sans comparaison possible avec nulle autre évolution scientifique spontanément née et déroulée en quelque autre monde humain : nous ne sommes donc nullement autorisés à croire que, quelles qu’eussent été les voies et méthodes, les séries enchaînées et ramifiées des recherches scientifiques, elles auraient finalement \phantomsection
\label{v1p44} conduit en astronomie à la loi de Newton, en physique au principe de la conservation de l’énergie, et bien moins encore en biologie à la théorie de la sélection naturelle. On ne peut pas plus dire cela qu’on ne peut dire que toutes les évolutions religieuses, parties d’idées dissemblables suggérées par la diversité des climats, des races, des circonstances, ici du culte des ancêtres surtout, là du culte des astres, devaient aboutir fatalement à l’Évangile du Christ. D’autre part, il n’est pas douteux, quel que soit le caractère obligatoire ou non, du point de départ, qu’à chaque pas dans une direction, choisie entre mille de préférence à telle autre, l’action puissante d’une rencontre accidentelle, d’un génie ou d’un caprice individuel, s’est fait sentir sur le cours du grand fleuve spirituel. Et nous avons lieu de penser, d’après les observations présentées plus haut, que cette influence n’a pas été toujours ni le plus souvent une perturbation passagère, bientôt effacée par la prépondérance d’influences majeures et d’ordre plus rationnel, mais qu’elle a été fréquemment décisive aux époques où le courant scientifique hésitait entre deux versants, sur la ligne de partage de ses eaux. Quand, parmi les mille orientations que la curiosité savante pourrait choisir, il en est une qui a la chance d’être adoptée par un grand chercheur, tel qu’Archimède, Newton, Lavoisier, Pasteur, la foule des disciples se précipite sur ses pas, et toutes les autres routes sont négligées. On avance donc dans le sens indiqué par lui, et plus loin surgit un nouveau pionnier célèbre qui, sans le premier, n’eût pas apparu. Ainsi, les suites d’une initiative individuelle — d’une grande idée, qui souvent peut être logée dans un fort petit homme — sont incalculables. Et nous pouvons conclure hardiment que, si d’autres génies ou d’autres accidents, qui auraient pu naître et qui ne sont point nés, s’étaient produits, la science moderne serait toute différente de son état actuel ; solide aussi mais à d’autres égards, et donnant aussi l’illusion d’une plénitude qui lui ferait défaut.\par
 \phantomsection
\label{v1p45}Mais, telle science, telle puissance militaire et industrielle, telle civilisation (toutes choses égales d’ailleurs). Par suite, on en saurait vraiment exagérer le rôle de l’accidentel ou de l’individuel en histoire, et c’est ce que des historiens très pénétrants commencent à démêler. Ceux mêmes qui le nient ne s’aperçoivent pas du démenti qu’ils se donnent par l’admirable patience de leurs recherches érudites qui portent sur la biographie des grands acteurs historiques et sur les péripéties des grandes guerres. Si l’on interroge ces grands acteurs eux-mêmes, ils sont tous d’accord pour attribuer au « hasard » — lisez à la rencontre non prévue, impossible à prévoir, de lignes régulières de causes et d’effets — ce que des narrateurs phraséologues expliquent par des mots creux. Dans toutes ses conversations sur l’histoire de son temps, Napoléon à Sainte-Hélène fait jouer à l’accident individuel, je ne dis pas même au génie individuel, un rôle énorme, très supérieur à celui que les historiens les plus \emph{accidentalistes} lui prêtent.\par
Ceci ne tend point à démontrer que l’évolution scientifique échappe à toute formule générale, à toute \emph{loi.} Je n’ai point à examiner ici si la \emph{loi} des trois états, d’Auguste Comte, se vérifie, ou si, dans leur ordre de maturité successive, les diverses sciences, des mathématiques à la sociologie, se conforment à la série formulée par lui. Ces deux généralisations abstraites peuvent être vraies sans que la vérité des considérations précédentes soit en rien atteinte. Et l’on peut formuler bien d’autres lois qui, tout en se conciliant pareillement avec la libre diversité des phénomènes, serrent de plus près la réalité des faits\footnote{ \noindent « L’historien, dit en passant Mommsen, n’a pas à suivre dans les détails infinis de la vie individuelle le sillon laissé par les grands faits qu’il relate... » Par cette phrase incidente, le sagace historien a nettement distingué en quoi l’histoire, qui s’occupe, avant tout, des phénomènes sociaux considérés par leur côté individuel, accidentel, pittoresque, diffère de la sociologie, qui, elle, a pour tâche précisément, d’étudier les \emph{sillons} négligés — ou dédaignés à tort — par elle et de rechercher les lois générales auxquelles ils sont soumis.
 }. De ce nombre est  \phantomsection
\label{v1p46}la loi d’irréversibilité relative dont je viens de parler. Elle n’est qu’un corollaire des règles de la logique, qui régissent souverainement le monde mental et inter-mental, comme les principes de la mécanique le monde matériel. Mais, après tout, l’ordre d’apparition des inventions, auquel s’applique cette loi, si important qu’il soit\footnote{ \noindent Il est très important en effet, pour le \emph{succès} d’une invention ou d’une découverte, de venir \emph{avant} plutôt qu’\emph{après} telle autre, ou \emph{vice versa}. Par exemple, telle découverte scientifique, — celle de la terre tournant autour du soleil, — qui, si elle était apparue (par hypothèse impossible) avant les idées religieuses fondées sur l’illusion anthropocentrique, les aurait empêchées de naître, ne suffit pas à les faire mourir parce qu’elle est venue après elles.
 }, n’est pas ce qui importe le plus à considérer. L’essentiel est de savoir, une fois apparues dans un ordre quelconque, comment elles vont s’agencer et s’organiser, pour former un de ces grands agrégats systématiques et rationnels d’idées accidentelles, qui sont la charpente des sociétés : une grammaire, un \emph{Credo}, un corps de lois, un corps de sciences, une constitution, une morale, un art.\par
Il y a ici à distinguer trois périodes : d’abord une période de chaos, où les inventions éparses, clairsemées, réponses fortuites à des besoins différents, ne se heurtent encore ni ne s’associent. Quand tout est à créer en faits de mots, en fait d’idées, en fait de coutumes et d’industries, les premiers inventeurs et initiateurs sont comme les premiers colons d’Amérique, peu exposés à se heurter dans leurs explorations d’une terre vierge. Mais cette période est assez courte, et bientôt commence la seconde période de crise génétique, de laborieuse organisation. Celle-ci s’opère, toujours et partout, par deux procédés opposés qui alternent et collaborent, le duel logique des idées qui impliquent contradiction de croyances, ou des œuvres qui impliquent contrariété de désirs ; et l’accouplement logique des idées qui se confirment ou des œuvres qui s’entr’aident. Ces deux procédés d’élimination et de construction, d’épuration et de  \phantomsection
\label{v1p47}consolidation, d’assainissement et de croissance, sont continuellement opérants dans la vie sociale, envisagée sous tous ses aspects. Enfin, quand cette période d’organisation est parvenue à constituer un système à peu près définitif et arrêté, une troisième période s’ouvre, indéfinie, de développement en richesse et en profondeur : c’est celle où, les grammaires étant à peu près fixées, les dictionnaires vont se grossissant ; où, les principes du Droit étant à peu près assis, les actes législatifs se multiplient ; où, les bases de la constitution étant posées, le pouvoir politique se déploie ; où, un régime industriel, un mode d’organisation du travail, étant établi, la production se développe et remplit le marché\footnote{ \noindent  Voir à ce sujet \emph{Logique sociale}, pp. 192-204.
 }.\par
On remarquera que cette division tripartite des phases traversées par les groupes d’inventions en voie de formation et de transformation, ou plutôt de concentration et d’expansion graduelles, concorde avec la division pareillement tripartite dans laquelle nous avons fait rentrer plus haut l’histoire générale de l’humanité. L’ère \emph{préhistorique} où, avons-nous dit, les diverses sociétés embryonnaires étaient éparses sur le globe, séparées par des distances pratiquement infranchissables, correspond à la phase chaotique des inventions. L’ère \emph{historique} des guerres et des alliances alternantes et entre-croisées qui nous acheminent péniblement vers la grande fédération finale, correspond à la phase des duels et des accouplements logiques entre inventions rapprochées qui vont s’organisant. Et l’ère future, qu’on peut appeler \emph{post-historique}, où, ne pouvant plus s’étendre en surface, la civilisation définitive travaillera à se perfectionner intérieurement, correspondra à la phase dernière des agrégats d’inventions qui, constitués, travaillent à leur enrichissement et à leur perfectionnement interne. Cette analogie n’a rien de surprenant ; elle s’explique  \phantomsection
\label{v1p48}par ce même besoin, universel et fondamental, d’harmonisation logique et téléologique, qui, par des voies uniformes, s’ingénie en nous à accorder nos idées et nos vœux (d’abord sans lien, puis liés en conceptions, en plans, en inventions, enfin développés et perfectionnés) et s’évertue hors de nous, \emph{entre nous}, à systématiser les inventions des divers individus en institutions d’un certain ordre (grammaticales, religieuses, etc.), comme les institutions des divers ordres en nationalités et en États, comme les divers États et les diverses nationalités en une même civilisation.\par
La logique (y compris la téléologie) telle que je l’entends, la logique concrète et vivante, qui n’est pas une entité mais la satisfaction d’un sourd besoin de l’esprit, besoin précisé et fortifié par ses satisfactions mêmes, régit ainsi souverainement le monde mental et le monde social. Non seulement elle préside aux phénomènes psychologiques de l’association des images et des idées et de leur agrégation, mais encore elle domine tous les phénomènes sociologiques, c’est-à-dire les inventions et leurs imitations. Les lois de l’imitation comme les lois de l’invention relèvent d’elle.\par
Je ne voudrais pas quitter ce sujet sans formuler une autre constatation générale, propre à caractériser la nature de ces initiatives individuelles, de ces découvertes, de ces inventions, d’où nous disons que, socialement, tout découle. Ces nouveautés-là consistent toutes, au fond, dans un affranchissement partiel de l’individu qui, échappant pour un moment et sous un certain rapport à la suggestion ambiante de ses semblables, entre en rapport direct avec la Nature, avec l’immensité sauvage, divine, extra-sociale, et rapporte de cette fugitive vision soit une explication nouvelle, imaginaire ou positive, du monde, soit une force nouvelle exploitée et captée, soit un charme nouveau, un beau nouveau. Car il y a des inventions esthétiques comme il en est de théoriques et de pratiques, de morales aussi, de philosophiques et d’industrielles. Chaque homme qui pense  \phantomsection
\label{v1p49}par soi-même ou \emph{qui sent par soi-même} — chose tout aussi rare — puise, en quelque sorte, dans le bassin profond et magique de la Nature une eau féconde dont il arrose autour de lui la société ; et cette eau se répand, multipliée comme le pain de la légende évangélique, par les mille canaux de l’imitation. Artistes, savants, ingénieurs civils ou militaires, ne font ainsi qu’exploiter l’univers au profit de l’homme. Non moins, il est vrai, que le contact direct du monde extérieur, l’exotisme, le contact exceptionnel avec une société étrangère à la nôtre, est une source d’innovations réussies, d’importations utiles, parmi beaucoup d’emprunts malheureux ou désastreux. Mais, si l’on remonte à l’origine première de ces choses importées, on trouve toujours le contact immédiat, le dévisagement hardi de la réalité naturelle, moyennant le soulèvement ou le déchirement momentané du tissu des mutuelles illusions sociales, du voile des influences inter-mentales. Il faut toujours en revenir là.
\subsubsection[{A.1.h. Classification des types sociaux.}]{A.1.h. Classification des types sociaux.}
\noindent Disons maintenant quelques mots d’une question importante, la classification des types sociaux, qui a été soulevée par un sociologue très distingué, M. Steinmetz. Mais d’abord il convient d’interpréter, à la lumière de nos considérations sur la logique sociale, la véritable nature de ces coïncidences si surprenantes et si nombreuses que l’observation révèle aux ethnographes entre des peuples ou des peuplades étrangers les uns aux autres et qui se ressemblent sous tant de rapports sans avoir pu s’imiter. Quoique dépourvues de précision et s’évanouissant souvent si on les regarde de près, ou, plus souvent encore, se laissant expliquer par des emprunts et des imitations cachées si l’on fait des fouilles dans leur passé\footnote{ \noindent M. Alexandre Bertrand, dans sa \emph{Religion du Gaulois}, après avoir exposé la frappante similitude, chez beaucoup de peuples anciens et modernes, des superstitions relatives aux vertus des plantes et aux remèdes bizarres qu’on peut en extraire, fait cette remarque très juste. « Si nous n’avions affaire qu’à des plantes ou à des herbes vraiment salutaires, si la cueillette n’en avait pas été entourée jusqu’au moyen âge des prescriptions les plus bizarres, les plus absurdes, on pourrait croire à la \emph{polygénésie}, pour ainsi dire, de ces remèdes. Les pasteurs des divers pays auraient pu en découvrir isolément et à des dates diverses les propriétés curatives. Mais comment alors expliquer la croyance persistante, en Italie à la fois et en Gaule, à des qualités médicinales imaginaires, à des pratiques aussi folles qui ne peuvent relever que des formules magiques, œuvre des collèges de prêtres qui les auraient fixées à une époque où toute science se concentrait dans la magie. » Remarquons qu’il n’est pas nécessaire le moins du monde de recourir à des collèges de prêtres — plus ou moins problématiques — pour expliquer la propagation imitative des superstitions dont il s’agit d’Italie en Gaule ou de Gaule en Italie.
 }, ces ressemblances sont certaines et tout  \phantomsection
\label{v1p50}à fait dignes d’attention. Mais l’erreur serait de les regarder, avec des yeux de naturaliste égaré en science sociale, comme des manifestations d’un \emph{instinct} humain analogue aux instincts des animaux, ou à ce qu’on désigne ainsi pour en masquer le mystère, alors qu’on a sous la main l’explication toute naturelle de ces faits par l’action d’une même logique qui, sollicitée, en des circonstances semblables, par des difficultés pareilles à surmonter et des ressources pareilles offertes pour en triompher, doit, dans beaucoup de cas, suggérer à l’esprit de différents initiateurs des solutions à peu près les mêmes\footnote{ \noindent Nillson, dans son livre sur les \emph{habitants primitifs de la Scandinavie} (1868), attache beaucoup d’importance à des ressemblances frappantes de formes entre les \emph{pointes de flèches} de la Terre de Feu et celles de la Suède, — et, en général, entre les outils ou armes de silex des peuplades séparées par les plus grands espaces de terre et de mer. Mais ces similitudes toutes spontanées qu’elles soient vraisemblablement, n’ont rien de surprenant. Elles sont \emph{imposées} comme la \emph{solution unique} d’un problème \emph{très simple} dont les données, toutes les mêmes, sont à la fois sous la main des primitifs.\par
 Mais les \emph{combinaisons ethniques de ces éléments} similaires, se ressemblent-elles aussi ? Est-ce que les langues se ressemblent — et les religions — et les mœurs — et est-ce que des mœurs semblables, des mots semblables, des idées semblables, se trouvent semblablement combinées ? Voilà ce qu’il faudrait pour que l’existence d’un \emph{instinct} social, analogue aux instincts des animaux, fût démontrée.
 }. Ce que c’est que l’instinct en vertu duquel toutes les fourmilières ou tous les essaims reproduisent, sans s’être jamais copiés, les mêmes \emph{institutions} quasi politiques et quasi industrielles, nous n’en savons rien. Il se peut — c’est une idée comme une autre — que  \phantomsection
\label{v1p51}l’instinct ne soit, au fond, qu’une logique ou une téléologie vitale fixée par l’hérédité. Tout ce qu’on en sait de clair, c’est que bien rares sont les cas où, comme dans les deux cas cités, l’instinct a suffi, sans l’aide de l’imitation des parents par les jeunes, à la reproduction exacte des phénomènes réputés instinctifs de l’espèce, tels que la nidification ou le chant des oiseaux. Ce qu’on sait aussi, c’est que, à mesure qu’on s’élève sur l’échelle zoologique et psychologique, la part de l’instinct, dans la formation de ces similitudes frappantes d’actions, va se resserrant, tandis que celle de l’exemple va s’élargissant. Chez l’homme, qui est au sommet de la hiérarchie, la première doit être réduite au minimum, et la seconde, par suite, au maximum. De telle sorte que, si, malgré tout, des coïncidences spontanées se produisent entre nations différentes, entre évolutions sociales indépendantes, cela ne peut tenir qu’à une nécessité ou à une suggestion de nature logique. C’est ainsi que les analogies de forme et de mouvement entre l’aile de l’oiseau et l’élytre de l’insecte, entre le poisson et le cétacé, ne pouvant s’expliquer par une commune descendance héréditaire (comme les analogies de beaucoup de religions et de coutumes ne peuvent s’expliquer par une commune descendance imitative), sont attribuées en général à une sorte de nécessité logique qui a forcé la Nature vivante à se répéter spontanément. On ne dira pas ici, je pense, que la vie, au sens abstrait du mot, a obéi, elle aussi, à un \emph{instinct}, encore plus incompréhensible et inimaginable que celui de ses créatures.\par
Comme on le voit, les évolutionnistes qui, généralisant abusivement les coïncidences spontanées des institutions comparées, veulent couler les transformations historiques des différents peuples dans des formules rigides d’évolution, tendent sans le vouloir à faire de l’homme un animal inférieur, impérieusement gouverné par ses instincts.\par
Cela dit, nous sommes d’accord pour reconnaître avec  \phantomsection
\label{v1p52}Steinmetz l’utilité de dresser un tableau, aussi complet que possible, des différents groupes de sociétés classées d’après les similitudes ou les différences qu’on remarque entre elles. Mais, encore ici, nous avons à lutter contre l’obsession si répandue des comparaisons biologiques en sociologie. Steinmetz a beau être opposé à l’idée de la société organisme, il n’en est pas moins fasciné malgré lui par les classifications zoologiques et végétales, et persuadé qu’elles sont le modèle idéal, auquel il rêve, sans l’espérer, de conformer sa classification des types sociaux. Je crois que, si l’on veut se laisser guider par des exemples, il vaut mieux ici prendre pour guide celui des linguistes que celui des naturalistes. Les types sociaux — dont la langue n’est qu’un élément, fondamental il est vrai — sont quelque chose de tout autrement compliqué que les types linguistiques. Mais la classification générale de ceux-là doit reposer sur les mêmes fondements que la classification spéciale de ceux-ci. Or, avant tout, les linguistes ont été conduits à faire, entre les similitudes que présentent les langues différentes, une distinction au fond identique à celle que j’établis entre les similitudes par imitation et les similitudes par nécessité logique. Et, sans méconnaître l’importance de ces dernières dont la signification rationnelle ne leur a pas échappé, c’est sur les premières qu’ils se sont fondés pour diviser les langues en familles. Les langues d’une même famille sont celles qui, par l’identité de leurs racines et de leurs procédés grammaticaux, attestent leur dérivation imitative d’une même souche initiale, c’est-à-dire d’un même modèle commun, répété des milliards de fois de bouche en bouche avec des variantes accumulées, avant de s’incarner à présent dans des copies souvent fort dissemblables. Cette descendance imitative d’un même modèle est, en effet, l’équivalent social de la parenté des individus vivants et des espèces vivantes. On peut aussi bien distinguer des familles de religions, ou de gouvernements, ou de morales, ou de droits, ou  \phantomsection
\label{v1p53}d’arts, et entendre par là les rayons imitatifs émanés d’un même foyer. Par exemple, les divers exemplaires du gouvernement parlementaire, belge, français, italien, allemand, etc., se rattachant imitativement au parlementarisme anglais, forment avec celui-ci une même famille politique. Donc, pour classifier les \emph{types sociaux}, envisagés dans l’ensemble des aspects de la vie sociale, il faut d’abord grouper les sociétés d’après leurs affinités d’origine imitative.\par
Il est à remarquer que ces rapports de similitude par imitation sont liés, en général, au rapport de voisinage géographique. Étudier les groupements géographiques des langues, des religions, des États, c’est étudier leur parenté sociale, qui se combine avec leur parenté physiologique pour former les nations fortes et vivaces, mais qui n’en est pas moins très distincte. La carte politique, — et aussi bien la carte linguistique, religieuse, juridique, etc., — n’offre tant d’intérêt que parce qu’elle implique une sorte de classification sociale. Les classifications sociales les plus claires, bien que les plus vagues à vrai dire, sont celles qui s’expriment en termes géographiques : peuples européens, peuples asiatiques, peuplades africaines, peuplades océaniennes. Mais les plus profondes sont celles qui s’expriment en termes religieux : la \emph{chrétienté}, l’\emph{islam}, le \emph{monde bouddhique}, le \emph{monde brahmanique.} La distinction des civilisations, c’est-à-dire des types sociaux les plus accentués, coïncide avec celle des religions, et persiste longtemps après que la foi religieuse s’est évanouie.\par
Quant aux similitudes non par imitation mais par contrainte logique — ou téléologique — qui existent entre divers peuples indépendants et sans relations connues ou vraisemblables les uns avec les autres, elles doivent aussi donner lieu non pas à une classification naturelle des types sociaux, des familles nationales, mais à une classification toute théorique des diverses sortes de solutions logiques, d’équilibre logique stable que comporte le problème de la vie sociale d’après la  \phantomsection
\label{v1p54}diversité des circonstances extérieures, climat, faune, flore, sol, et des races humaines. Ce serait là quelque chose de comparable à cette \emph{grammaire générale} que rêvaient les idéologues du siècle dernier, et qui, reprise de nos jours avec plus de souci de la réalité des faits, pourrait exercer utilement la sagacité des chercheurs. — Les considérations géographiques, à ce nouveau point de vue, présenteraient encore un réel intérêt, mais tout autre, et, en somme, secondaire. Il s’agirait de marquer, entre peuples séparés d’ailleurs par de grandes distances, et qui, — par hypothèse, hypothèse beaucoup trop facilement accueillie par beaucoup de savants — ne se sont rien emprunté les uns aux autres, les ressemblances géographiques qu’ils offrent à l’observateur, telles qu’une situation analogue dans une vallée, au bas d’une montagne, au bord d’une mer intérieure ou d’un grand lac, dans des plaines fertiles ou des steppes plus ou moins herbues, sous des latitudes tropicales, tempérées, septentrionales. L’école de la « Science sociale », branche nouvelle (et branche gourmande) poussée sur le tronc des idées de Le Play, a rendu à nos études le grand service d’exploiter à fond cette mine de recherches ouverte par Montesquieu et de montrer clairement, par l’insuccès de ses recherches, l’insuffisance du principe qui lui sert de fondement. Que des ressources spontanées du sol habité par un groupe d’hommes dérive le type de famille, le type politique, juridique, moral, qui le distingue, c’est là une exagération évidente, ou plutôt une erreur des plus graves, qui a pour conséquence d’aveugler des esprits très pénétrants sur des vérités palpables. Elle est suffisamment réfutée par le fait que des peuples habitant un sol très dissemblable, mais voisins les uns des autres, se ressemblent beaucoup malgré la dissemblance de leur territoire, de leur faune, de leur flore, de leur climat, et que des peuples situés sur des territoires semblables, dans des conditions physiques et biologiques analogues, mais séparés par de grandes distances,  \phantomsection
\label{v1p55}sont très différents en dépit de cette analogie de leurs habitats. On peut citer comme exemples du premier cas : la Suisse allemande et l’Allemagne, la Norvège et la Suède, les diverses provinces françaises. Visiblement, les ressources extérieures ne sont que l’une des données du problème social, à savoir les \emph{moyens} mis à la disposition de l’homme ; l’autre donnée, ce sont les \emph{buts} que l’homme poursuit. Encore faut-il observer que ces \emph{moyens} ne sont réellement offerts que dans la mesure où, par des découvertes et des inventions successives, individuelles et accidentelles, les ressources du sol, virtualité pure et simple au début, sont mises en lumière et mises en œuvre ; et que ces \emph{buts} diffèrent ou changent, se modifient, se différencient, se compliquent, au gré non de la nature extérieure mais des inventeurs, des initiateurs, des meneurs quelconques qui, dans une large mesure, tracent au désir humain des lits capricieux. Un besoin organique n’est qu’un terrain de culture sur lequel les besoins sociaux, les mobiles économiques les plus divers et les plus changeants, peuvent éclore.\par
Les sociétés humaines doivent encore être classées à un autre point de vue, qu’il convient de ne pas confondre avec les précédents. Autre chose est le degré de dissemblance ou d’hétérogénéité des types sociaux, autre chose est leur hiérarchie. Deux peuples, quoique très dissemblables, quoique appartenant à des types de culture très différents, peuvent être placés au même rang sur l’échelle hiérarchique ; et, inversement, deux peuples placés sur cette échelle à des degrés très inégaux, peuvent appartenir au même type de civilisation. Cette considération ne saurait être oubliée si l’on veut faire une classification vraiment \emph{naturelle} des sociétés. Mais quel sera le fondement du classement hiérarchique dont il s’agit ? Là est la grande difficulté. Rien de plus confus et de plus contradictoire que les vues émises à cet égard. Je me permets de penser qu’on jetterait quelque lueur dans ce chaos en se plaçant au point de vue de la  \phantomsection
\label{v1p56}psychologie inter-cérébrale. On aurait, ce me semble, une excellente pierre de touche du \emph{niveau} des sociétés en se réglant sur l’intensité comparée des actions inter-spirituelles qui s’échangent et s’enchevêtrent dans le sein de chacune d’elles, — sur la proportion relative de ces actions inter-spirituelles et des actions intercorporelles concomitantes, qui décroissent pendant que les autres grandissent quand une société \emph{s’élève}, — et, par suite, sur le degré d’exploitation des forces extérieures, animales, végétales, physico-chimiques, qui, à mesure qu’elle grandit, amoindrissent l’action inter-corporelle et compliquent l’action inter-spirituelle des hommes associés.
\subsubsection[{A.1.i. Nécessité ou non de la mort des sociétés.}]{A.1.i. Nécessité ou non de la mort des sociétés.}
\noindent Terminons ces aperçus généraux par quelques mots sur une question qui nous était naturellement indiquée pour la fin : celle de savoir s’il y a une mort nécessaire des sociétés en vertu d’une loi des âges qui les assujettirait, comme les individus, à passer de la jeunesse à la maturité, puis de la maturité à la vieillesse, et enfin au fatal dénoûment de tout ce qui a vie. Le problème consiste à se demander non pas si toutes les sociétés finiront un jour, mais si, dès leur naissance, et en vertu même des causes qui les poussent à la vie, elles sont condamnées à périr dans un délai plus ou moins déterminé ; si, en un mot, il y a pour elles une \emph{mort naturelle} et non pas seulement une mort violente. La réponse affirmative à cette question a été dictée à la plupart des sociologues soit par la conception biologique du monde social, soit, en même temps, par l’obsession d’un besoin de symétrie qui fait opposer à la nécessité de l’évolution la nécessité d’une dissolution correspondante, opposition manifeste dans les \emph{Premiers principes} de Spencer.\par
 \phantomsection
\label{v1p57}Dans la première moitié de ce siècle, c’est surtout en s’appuyant sur les sciences de la nature inorganique que l’on espérait pouvoir fonder la science des sociétés. Quételet voyait dans le monde social une sorte de système solaire, comme l’indique le titre de son principal ouvrage : « Le système social. » Cette même préoccupation de comparaisons astronomiques se retrouve chez Carey. Chez Comte, les divisions de la sociologie sont empruntées à la mécanique (partie statique et partie dynamique) et il parle de « physique sociale » sauf à se corriger plus tard. Carey va même jusqu’à imaginer une chimie sociale. Il dira que « les combinaisons dans la société sont soumises à la loi des proportions définies ». C’est la physique qui est le plus en faveur auprès des sociologues naissants, et, malgré, çà et là, des tendances marquées à regarder le groupe social comme une sorte d’organisme, le plus souvent il n’est question chez eux que de masse et de mouvement. « Tout acte d’association est un acte de mouvement, » dit Carey. « Les lois générales du mouvement sont celles qui régissent le mouvement sociétaire. Tout progrès a lieu en raison directe de la substitution du mouvement continu au mouvement intermittent. »\par
Mais, encore une fois, la métaphore de la société-organisme allait progressant — au fur et à mesure des progrès de la biologie — et, chez les deux auteurs mêmes que je viens de citer, elle apparaît nettement. Carey en a fait usage, chose étrange — car elle peut servir à toutes fins — pour faire sentir l’unité du genre humain. « De même, dit-il, que l’organisme complexe du corps humain, par l’effet des dépendances et des sympathies de ses diverses parties, forme une unité dans son action, l’humanité entière, dans un sens aussi réel et aussi vrai, devient un seul homme et doit être traitée ainsi. » C’était l’idée courante — née d’une phrase de Pascal — à l’époque où Carey écrivait. Sous la plume de Comte, qui en a fait l’idée capitale \phantomsection
\label{v1p58} de son système, elle devient la déification de l’humanité dont le culte est proposé à tous les hommes comme le seul digne de se substituer au christianisme. Ainsi, ce ne sont pas les nations qui, à cette époque, passaient pour de vrais organismes sociaux, c’était l’humanité considérée dans son ensemble. Remarquons en passant que ces deux manières d’entendre la conception biologique de la société sont inconciliables. Quoi qu’il en soit, Quételet a très bien vu que, si les sociétés sont assimilables à des individus vivants, les nations doivent avoir une \emph{durée moyenne} comme les individus ont une \emph{vie moyenne.} Cette durée moyenne, il l’a cherchée, en bon statisticien, dans son \emph{Système social} (ch. {\scshape iv}). Et il se persuade même l’avoir trouvée. « Si maintenant, conclut-il, on compte 1580 années pour la durée de l’empire des Assyriens, 1663 pour les Égyptiens, 1522 pour les Juifs, 1410 pour les Grecs et 1129 pour les Romains, on trouvera que la durée moyenne de ces cinq empires, qui ont eu le plus de retentissement dans l’histoire, a été de 1461 ans. » Et il fait remarquer, « rapprochement assez singulier », dit-il, que cette durée constitue exactement la période \emph{Sothiaque} ou le cycle caniculaire des Égyptiens. « C’est dans la durée de ce cycle, écrit-il, qu’était renfermée l’existence du Phénix. Cet oiseau en renaissant de ses cendres, formait l’emblème de la coïncidence qui se rétablissait entre les années des Égyptiens et celles des Indiens (?). »\par
Le malheur est pour ces belles imaginations statistiques, que la durée de la nation égyptienne, d’après les nouvelles découvertes des égyptologues, est infiniment supérieure à la durée indiquée plus haut — qu’on ne sait pourquoi Quételet a fixé à telle date plutôt qu’à telle autre la mort des nations dont il parle, leur naissance aussi bien\footnote{ \noindent Il y a bien d’autres objections à présenter contre l’idée de l’organisme social, et je les ai présentées ailleurs. En voici une saillante. Si les nations étaient des organismes, la plupart des États, même les plus civilisés, seraient des monstres doubles ou des monstres triples. L’Autriche monstre triple, avec ses trois nationalités, ses trois corps confondus en une seule tête ; l’Angleterre, monstre double, avec son Irlande soudée de force ; monstre quadruple et quintuple, avec son Empire colonial, si hétérogène) ; la Prusse, monstre double aussi, avec ses provinces polonaises...\par
 Mais la vérité est que les \emph{soudures sociales} sont chose autrement naturelle et facile que les soudures animales ou même végétales ; et cela seul suffit à montrer l’inanité de la métaphore de l’organisme social.
 }, — qu’on se  \phantomsection
\label{v1p59}demande aussi pourquoi il exclut de sa liste et la Chine et le Japon et beaucoup d’autres nations qui, pour avoir eu moins de « retentissement dans l’histoire », n’ont pas eu moins de réalité que les plus notoires. Il dit plus loin, fort sagement, que « l’origine et la fin d’une ville ne sont pas marquées d’une manière précise comme les deux termes extrêmes qui limitent la vie humaine ». Mais cela est encore plus vrai des nations ; car, encore avons-nous quelques exemples de villes disparaissant en un jour et beaucoup d’exemples de villes bâties sur un plan, à une date assignable ; mais l’histoire ne connaît pas de nation née \emph{ex abrupto} et engloutie tout entière dans une catastrophe.\par
A notre avis, il est fort possible que toute société soit destinée \emph{à finir} comme, \emph{peut-être}, tout système stellaire ou toute harmonie physique aussi bien que tout organisme ; mais la preuve que cette fin, si fin il y a, ne doit pas être confondue avec une \emph{mort} et traitée comme telle, c’est qu’elle est matière à discussion et à controverse, tandis que le caractère le plus indiscutable d’un individu vivant est d’être mortel. Il y a quelques raisons, en effet, mais tout à fait étrangères aux métaphores organiques, de penser que toute évolution sociale, en se prolongeant indéfiniment, court non pas dans un délai à peu près fixe, mais dans un délai prodigieusement variable, à son déclin lent, graduel, sans nulle chute brusque comparable au dernier soupir d’un mourant. La première chose, en sociologie, pour savoir si la nécessité de ce déclin continu existe est de rechercher si, vraiment, il existe une tendance naturelle de la population à  \phantomsection
\label{v1p60}croître d’abord, puis à décroître. Est-ce que les mêmes causes, le progrès de la prévoyance, de l’égoïsme, des besoins, qui, pendant une certaine période, au début des peuples, sur une terre neuve à peupler, provoquent la progression ascendante de la population, ne sont pas les mêmes qui, plus tard, une fois certaines limites de densité franchies, déterminent sa diminution par degré ? A priori, c’est vraisemblable ; et l’on peut citer à l’appui de cette vue l’Empire romain, voire même les nations de l’Europe moderne, y compris les États-Unis d’Amérique, où le ralentissement du progrès de la natalité, sinon son déclin, accompagne l’urbanisation et la civilisation grandissantes. Mais, en sens contraire, on peut objecter la Chine où la population continue à surabonder, même dans les provinces où elle est le plus dense.\par
A un point de vue plus proprement social, une société peut être regardée comme déclinante quand, malgré l’état stationnaire ou même progressif de sa population, elle présente un abaissement graduel du niveau de sa civilisation. Cette phase d’abaissement continu est-elle nécessaire ? Si nous nous permettions aussi des arguments d’analogie, la comparaison des trois principales formes de la répétition universelle, sur laquelle nous nous sommes étendus plus haut, serait de nature à nous suggérer une réponse pessimiste. Ne vient-il pas toujours un moment où les sources de l’énergie \emph{ondulatoire}, qui sont les contacts physiques ou chimiques, les chocs et les combustions, — où les sources de l’énergie \emph{générative} ou régénérative (nutritive), qui sont les fécondations et les croisements, — où les sources de l’énergie \emph{imitative}, qui sont les inventions et initiatives individuelles de tout genre, tarissent définitivement ? Or, en vertu d’une nécessité inhérente à leur essence même, est-ce que ces trois formes de l’énergie, quand elles cessent d’être renouvelées, ne doivent pas s’affaiblir par degrés ? Il paraît en être ainsi de l’énergie vibratoire, soit que l’ondulation  \phantomsection
\label{v1p61}propagée reste contenue dans le sein d’une molécule ou se développe en un sens purement linéaire (par exemple, dans un fil électrique), soit qu’elle se déploie librement en tous sens comme la lumière et la chaleur rayonnantes. Si, dans une pile électrique, l’action chimique s’arrête, le courant s’arrête aussitôt, et, tout le long du fil, on constate, à partir de sa source, son affaiblissement progressif, dû au frottement inévitable. Si le soleil cessait de brûler, son rayonnement, pendant quelques siècles encore, continuerait à s’épandre dans l’immensité, mais, toujours pour la même cause, il finirait par s’arrêter, à des distances incommensurables d’ailleurs. — Il en est de même, à coup sûr, de l’énergie vitale, générative ou régénérative, à partir du moment où la fécondation a cessé, soit que cette énergie reste enfermée dans le même organisme (nutrition), soit qu’elle s’extériorise par bourgeonnement ou par enfantement sans fécondation préalable. Enfin, n’en est-il pas de même de l’énergie imitative, de la vitalité sociale des peuples, dès que leur génie inventif s’éteint, soit qu’elle se concentre sur le territoire national, soit qu’elle se dépense au dehors en essais de colonies lointaines ? De là, pourrait-on conclure, la nécessité égale de ces trois grands phénomènes : l’équilibre final de température, ou, en général, de force physique, par le refroidissement et le repos universels, la mort des êtres vivants, et l’épuisement des civilisations.\par
Ce sont là des analogies spécieuses, qui méritent, je crois, d’être discutées. Sans entrer dans cette discussion, pour le moment, je ferai observer que, même dans le monde vivant, la nécessité de la mort, de la mort naturelle et non violente, n’embrasse pas, d’après Weissman, la totalité des individus vivants et qu’elle ne paraît pas s’appliquer du tout aux espèces vivantes. Il est des espèces qui ont traversé, sans périr, plusieurs âges géologiques, et tout semble montrer que, aussi longtemps que persistent les conditions extérieures au milieu desquelles une espèce animale ou  \phantomsection
\label{v1p62}végétale est née, a crû, s’est fortifiée, elle se perpétue indéfiniment sans présenter des traces de dégénérescence spontanée. Rien ne démontre non plus la nécessité d’une dislocation finale du système solaire (je ne dis pas de l’extinction du foyer solaire), et les astronomes d’après Laplace ont longtemps dogmatisé sa stabilité éternelle. On ne voit pas, \emph{a priori} pourquoi une société civilisée ne jouirait pas du même privilège.
 \phantomsection
\label{v1p63}\subsection[{A.2. La valeur et les sciences sociales}]{A.2. La valeur et les sciences sociales}\phantomsection
\label{ppch2}
\subsubsection[{A.2.a. Place de l’économie politique parmi les autres sciences sociales. Théorie des vérités, théorie des utilités, théorie des beautés.}]{A.2.a. Place de l’économie politique parmi les autres sciences sociales. Théorie des vérités, théorie des utilités, théorie des beautés.}
\noindent Avant d’entrer au cœur de notre sujet, nous avons à indiquer la place de l’Économie politique parmi les autres sciences sociales, ce qui va nous conduire à passer en revue les principales notions dont cette science fait usage et à montrer jusqu’à quel point elles lui appartiennent en propre. Il s’agira surtout de la plus fondamentale et de la plus vague de toutes, la Valeur.\par
La Valeur, entendue dans son sens le plus large, embrasse la science sociale tout entière. Elle est une qualité que nous attribuons aux choses, comme la couleur, mais qui, en réalité, comme la couleur, n’existe qu’en nous, d’une vie toute subjective. Elle consiste dans l’accord des jugements collectifs que nous portons sur l’aptitude des objets à être plus ou moins, et par un plus ou moins grand nombre de personnes, crus, désirés ou goûtés. Cette qualité est donc de l’espèce singulière de celles qui, paraissant propres à présenter des degrés nombreux et à monter ou à descendre cette échelle sans changer essentiellement de nature, méritent le nom de \emph{quantités.}\par
Cette quantité abstraite se divise en trois grandes catégories qui sont les notions originales et capitales de la vie en commun : la \emph{valeur-vérité}, la \emph{valeur-utilité} et la \emph{valeur-beauté.} Nous prêtons aux idées, aux informations, aux connaissances scientifiques ou usuelles, et aux signes palpables où elles se matérialisent, tels que les livres, une vérité plus  \phantomsection
\label{v1p64}ou moins grande ; aux biens de tout genre, pouvoirs, droits, richesses, une utilité plus ou moins grande ; aux chefs-d’œuvre de l’art et de la nature, aux choses considérées comme sources de voluptés collectives des sens supérieurs affinés par l’éducation sociale, une beauté plus ou moins grande. Aussi bien que l’Utilité, la Vérité et la Beauté sont filles de l’Opinion, de l’opinion de la masse en lutte ou en accord constant avec la raison d’une élite qui influe sur elle. Et, s’il n’y avait pas d’opinion quelque peu unifiée, il n’y aurait nulle notion vraie, c’est-à-dire sociale, de l’idée de vérité, pas plus que de l’idée de beauté, dont le sens individuel même ne serait pas conçu. Le plus ou moins de vérité d’une idée signifie trois choses diversement combinées : le plus ou moins grand nombre, le plus ou moins grand \emph{poids} social (ce qui veut dire ici \emph{considération, compétence} reconnue) des personnes qui s’accordent à l’admettre, et le plus ou moins d’intensité de leur croyance en elle. Le plus ou moins d’utilité d’un objet, d’un produit ou article quelconque, exprime le plus ou moins grand nombre de gens qui le désirent, dans une société donnée et en un temps donné, le plus ou moins grand poids social (ici poids veut dire pouvoir et droit) de ces personnes, et le plus ou moins d’intensité du désir qu’elles en éprouvent. Le plus ou moins de beauté d’une œuvre artistique ou d’une création naturelle dépend aussi de trois facteurs : le plus ou moins grand nombre d’individus qui se plaisent à la vue ou à l’audition de cette œuvre, de cet être, le plus ou moins grand poids social (c’est-à-dire ici goût et culture du goût) de ces personnes, et le plus ou moins d’intensité ou de finesse de leur plaisir. Ces trois facteurs, ici comme plus haut, peuvent varier séparément, et leurs combinaisons infiniment multiples expliquent la complexité du problème esthétique, aussi bien que celle du problème économique, ou juridique, ou politique, et du problème pédagogique ou religieux.\par
Il est facile de voir que la \emph{vérité} et \emph{l’utilité} mènent le  \phantomsection
\label{v1p65}monde social, comme la \emph{croyance} et le \emph{désir}, auxquels elles correspondent et dont elles sont faites, mènent l’individu. J’ai cru montrer ailleurs\footnote{ \noindent Voir notamment, dans l’\emph{Opposition universelle} le chapitre relatif aux oppositions psychologiques.
 } que la croyance et le désir sont la grande bifurcation psychologique, exprimée souvent en termes moins précis, moins simples et moins élémentaires, comme ceux d’intelligence et de volonté, d’esprit et de conscience ; mais, comme point nécessaire d’application ou de visée de ces deux quantités élémentaires de l’âme, il y a les sensations pures, dépouillées par hypothèse de tout jugement implicite et de tout appétit inconscient. Or, le sentir pur est ce qu’il y a de plus incommunicable de personne à personne, et ce dont la similitude d’une personne à l’autre est le plus indémontrable, tandis que la croyance et le désir se transmettent à autrui avec une facilité toujours plus grande et se reconnaissent comme des états d’âme parfaitement semblables (abstraction faite de leurs points divers d’application ou de visée) dans tout le genre [{\corr humain}]. Mais cette chose impossible, la transmission et l’assimilation des manières de sentir entre les hommes, l’Art l’a tentée, et, passant sur toutes leurs sensibilités mises en vibration son magique archet, les disciplinant, les accordant par l’imposition douce des sensations plus exquises de l’artiste qui se répandent contagieusement dans son public, il a socialisé les sensibilités, comme la religion ou la science les intelligences, comme la politique ou la morale, les volontés. Ainsi, suprême couronnement de la logique et de la téléologie sociales qui vont harmonisant en un système d’idées ou d’actions les esprits et les cœurs des hommes, est née l’esthétique qui travaille à accorder et systématiser leurs \emph{goûts.}\par
Ce sont là les trois maîtresses branches, très inégalement développées, de la science sociale. Si elle était achevée et mûre, elle présenterait à la fois ces trois théories complètes : 1\textsuperscript{o} la théorie des vérités, ou, si l’on aime mieux,  \phantomsection
\label{v1p66}des \emph{lumières}, qui se subdivisent en plusieurs espèces, telles que connaissances linguistiques, fois religieuses, confiances enthousiastes, connaissances scientifiques ; 2\textsuperscript{o} la théorie des utilités, ou, si l’on veut, des \emph{biens}, qui comprennent les \emph{pouvoirs}, les \emph{droits}, les \emph{mérites}, les \emph{richesses ;} 3\textsuperscript{o} la théorie des \emph{beautés}, qui se décomposent en autant d’espèces distinctes qu’il y a de variétés des beaux-arts et de la littérature. Dans un autre ouvrage\footnote{ \noindent \emph{Lois sociales}, Paris, F. Alcan, 1898.
 }, j’ai essayé de faire voir que la sociologie, comme toute science, offre trois aspects différents, suivant qu’on la considère au point de vue de la \emph{répétition} de ses phénomènes, ou de leur \emph{opposition}, ou de leur \emph{adaptation.} Cette division tripartite ne fait pas double emploi avec celle que je viens d’indiquer, elle lui est perpendiculaire pour ainsi dire : chacune des trois \emph{branches} de la sociologie se montre à nous sous les trois \emph{aspects} dont il s’agit, comme nous le verrons bientôt.\par
La caractère quantitatif de tous les termes que je viens d’énumérer est aussi réel que peu apparent ; il est impliqué dans tous les jugements humains\footnote{ \noindent Toute époque, toute civilisation, d’après Nietzche — et c’est là une de ses meilleures considérations — a ce qu’il appelle « \emph{une table des valeurs} ». Par exemple, elle estime que « la vérité est supérieure à l’erreur, ou qu’un acte miséricordieux est préférable à un acte de cruauté ». (V. Lichtenberger, \emph{Philosophie de Nietszche.)} Un groupe des jugements comparatifs de ce genre constitue le caractère propre d’une phase de l’humanité. « La détermination de cette table des valeurs, et en particulier la fixation des plus hautes valeurs, est le fait capital de l’histoire universelle, puisque cette hiérarchie des valeurs détermine les actes conscients ou inconscients de tous les individus et motive tous les jugements que nous portons sur leurs actes. » Et l’on sait que, d’après le fameux philosophe, « la table des valeurs actuellement reconnue par les civilisations européennes est mal faite et demande à être révisée ». On ne saurait contester à Nietzche ni l’existence ni l’importance capitale de cette table des valeurs dont il parle. Mais elle suppose, avant tout, qu’il existe des \emph{quantités} sociales. Car, pour qu’une chose puisse être réputée \emph{plus} ou \emph{moins} qu’une autre, ne faut-il pas qu’elles aient une \emph{commune mesure ?} — Il faut donc admettre des \emph{quantités sociales}.
 }. Il n’est pas d’homme, il n’est pas de peuple qui n’ait poursuivi, pour prix de ses efforts acharnés, un certain \emph{accroissement} ou de richesse, ou de gloire, ou de vérité, ou de puissance, ou de perfection \phantomsection
\label{v1p67} artistique, et qui ne lutte contre le danger d’une \emph{diminution} de tous ces biens. Nous parlons tous et nous écrivons comme s’il existait une échelle de ces diverses grandeurs, sur laquelle nous plaçons plus haut ou plus bas les divers peuples et les divers individus et les faisons monter ou descendre continuellement. Tout le monde est donc implicitement et intimement persuadé que toutes ces choses, et non pas la première seule, sont de vraies quantités, au fond. Méconnaître ce caractère vraiment quantitatif, sinon mesurable en droit et en fait, du pouvoir, de la gloire, de la vérité, de la beauté, c’est donc aller contre le sentiment constant du genre humain et donner pour but à l’effort universel une chimère. Cependant, de toutes ces quantités, une seule, la richesse, a été saisie avec netteté comme telle, et a paru digne, par suite, d’être l’objet d’une science spéciale : l’Économie politique. Mais, quoique cet objet, en effet, à cause de son signe monétaire, se prête à des spéculations d’une précision plus mathématique, parfois même illusoire, les autres termes aussi méritent d’être étudiés chacun par une science à part. Toutes ces sciences, hâtons-nous de le dire, rentrent les unes dans les autres, quoique séparables en théorie. On s’explique par là, et on excuse parfaitement la prétention que l’Économie politique a longtemps affichée d’être la science universelle des sociétés. De fait, il n’est rien, en fait de valeur sociale, ni vérité, ni pouvoir, ni droit, ni beauté quelconque, qui ne puisse être envisagé comme richesse, comme ayant une valeur vénale. Mais l’économiste néglige de voir qu’il n’est pas de richesse non plus, agricole ou industrielle ou autre, qui ne puisse être considérée au point de vue des \emph{connaissances} qu’elle implique, ou des \emph{pouvoirs} qu’elle donne, ou des \emph{droits} dont elle est le fruit, ou de son caractère plus ou moins \emph{esthétique} ou inesthétique. Le savant, le prêtre, l’artiste, le politicien, le juriste, etc., tous ceux qui traitent de l’une des trois branches de la valeur sociale ou de leurs sous-branches,  \phantomsection
\label{v1p68}auraient donc les mêmes titres que l’économiste au caractère d’universalité dont il a revendiqué le monopole. Ajoutons que le côté théorique et le côté esthétique de tous les \emph{biens} vont se développant de plus en plus, non pas aux dépens, mais au-dessus de leur côté utilitaire. Mots, dogmes ou rites, connaissances, droits, pouvoirs, noblesse, gloire, ont pu d’abord n’être recherchés que comme utilités ; mais ils le sont aussi, et toujours davantage, comme objets de connaissance, comme vérités, et comme moyens d’expression artistique, comme beautés.\par
Quoi qu’il en soit, l’Économie politique, ainsi entourée, perdrait, il est vrai, son mystérieux isolement de bloc erratique déposé dans le désert de la sociologie encore à naître par les métaphysiciens ou les logiciens, mais elle y gagnerait d’apparaître à sa vraie place en science sociale, et de voir ses notions usuelles, ses divisions, ses théories, contrôlées par les sciences-sœurs qui s’éclaireraient de sa lumière et l’éclaireraient de la leur. Examinons d’abord quelques-unes des principales notions dont elle fait usage et voyons si elles ne peuvent pas servir, si elles ne servent pas déjà, à tracer les premiers délinéaments d’autres rameaux de la sociologie générale. En tête de ces notions, plaçons, comme nous l’avons dit, la valeur, ajoutons-y celle de monnaie, celles de travail, d’échange, de propriété, de capital, d’association.\par
Mais, avant tout, disons, ce qui ne peut être contesté, que la question de la population n’appartient pas en propre aux économistes. Elle appartient aussi bien aux politiques, aux juristes, aux moralistes, aux esthéticiens mêmes, ainsi qu’aux linguistes et aux mythologues. Chacun de ces groupes de savants se pose à sa manière le problème soulevé par la \emph{tendance} de la population à croître en progression géométrique. Il s’agit, pour l’économiste, d’étudier les rapports de cette \emph{tendance} avec la production des substances et des autres richesses, et de chercher les moyens pratiques  \phantomsection
\label{v1p69}soit de la réfréner dans la mesure où elle semble nous menacer — à tort ou à raison — d’un danger formidable, soit de faire que la production marche aussi vite ou plus vite que la population, par la substitution, notamment, de la grande à la petite industrie, de la grande à la petite culture. Il s’agit, pour la politique, de voir les rapports de cette même tendance avec la puissance des États, et de lever les obstacles qui s’opposent à sa réalisation en présence des nations rivales où elle est moins entravée. Il pourrait aussi se préoccuper de voir grandir le pouvoir de l’opinion, et spécialement celui des célébrités et des popularités de tout genre, à mesure que la population augmente. Le légiste sait aussi, ou doit savoir, que par telles ou telles institutions on peut stimuler ou amortir le progrès de la population, et que ce progrès, d’autre part, l’oblige à des remaniements législatifs et judiciaires. Le moraliste ne peut ignorer que le Devoir se transforme à chaque grand accroissement numérique d’un groupe d’abord étroit, qui, de la morale de clan, passe à la morale de cité puis de grande nation, et de la pratique de la vendetta ou de l’hospitalité traditionnelle au sentiment large de la justice et à la religion de l’humanité. Il devrait donc se réjouir des progrès de la population, s’il ne voyait aussi les populations se heurter en progressant et les guerres d’extermination naître de là. Enfin, le linguiste, le mythologue, l’esthéticien, ont à noter le lien qui existe entre le progrès ou le déclin de la population, c’est-à-dire les stimulants ou les obstacles historiques que rencontre sa tendance à progresser géométriquement, et les transformations de la langue, de la religion, de l’art.\par
Le problème se complique, pour chacun d’eux, si l’on remarque que la tendance en question est ou se révèle très inégale non seulement dans les diverses populations, mais dans les diverses couches de chacune d’elles et dans leurs générations successives. La question de savoir si les populations \phantomsection
\label{v1p70} et si les couches de la population les plus fécondes sont ou ne sont pas les plus inférieures, les plus mal douées, est d’une importance suprême au point de vue de l’avenir social envisagé sous tous ses aspects. Ce grand problème de la population, si multicolore et si changeant, chevauche à la fois, pour ainsi dire, sur la biologie et la sociologie. L’Économie politique n’a donc point le droit de l’accaparer.
\subsubsection[{A.2.b. Valeur-vérité, valeur-gloire, valeur-crédit.}]{A.2.b. Valeur-vérité, valeur-gloire, valeur-crédit.}
\noindent Mais revenons à l’idée de valeur. Je dis qu’elle s’applique aux objets quelconques, hommes ou choses, considérés comme visés par l’attention et la croyance du public, aussi bien qu’aux objets quelconques, hommes ou choses, considérés comme visés par le désir du public. Je dis, en d’autres termes, qu’elle est applicable au plus ou moins de « vérité générale » des connaissances dont une société compose son trésor intellectuel, de même qu’au plus ou moins d’utilité générale des richesses et des autres biens qui forment l’outillage de son activité volontaire. Au nombre de ces connaissances, de ces \emph{lumières}, il faut bien se garder d’omettre, à côté des groupes de connaissances qui constituent sa foi religieuse, sa foi linguistique et sa foi scientifique, sans compter sa foi politique, sa foi juridique et sa foi morale, cette grande constellation de célébrités, de crédits, de gloires, de popularités qui sont allumés par ses actes de foi ou de confiance personnelle en certains hommes à raison de leurs talents ou de leurs vertus supposés.\par
Or, la gloire d’un homme, non moins que son crédit, non moins que sa fortune, est susceptible de grandir ou de diminuer sans changer de nature. Elle est donc une sorte de quantité sociale. Il serait intéressant de mesurer avec une certaine approximation, moyennant des statistiques ingénieuses,  \phantomsection
\label{v1p71}pour chaque espèce de célébrité, cette quantité singulière\footnote{ \noindent J’ai dit ailleurs \emph{(Logique sociale}, chapitre sur l’Esprit social) que, en psychologie sociale, la gloire tenait la place de la conscience, du moi, en psychologie individuelle. Cette manière de voir n’est nullement inconciliable avec celle qui vient d’être indiquée, et d’après laquelle les gloires font partie des « vérités » nationales. La gloire — et j’entendais par là la notoriété \emph{maxima} à chaque moment, notoriété favorable ou défavorable — est la simultanéité et la convergence des attentions, des jugements, portés sur un homme ou sur un fait qui devient dès lors notoire ou glorieux, et cet état vif de l’opinion est en quelque sorte un état de conscience national. Mais chacun de ces faits ou de ces hommes constitue pour le public une \emph{connaissance}, une \emph{vérité.} C’est le côté objectif du phénomène dont la gloire est le côté subjectif.
 }.\par
Le besoin d’un \emph{gloriomètre} se fait sentir d’autant plus que les notoriétés de toutes couleurs sont plus multipliées, plus soudaines et plus fugitives, et que, malgré leur fugacité habituelle, elles ne laissent pas d’être accompagnées d’un pouvoir redoutable, car elles sont un \emph{bien} pour celui qui les possède, mais une \emph{lumière}, une foi, pour la société. Distinction qu’il y a lieu de généraliser. La confiance générale est, pour l’individu qui l’inspire, une grande force, un grand moyen d’action ; mais, pour le public qui l’éprouve, elle est une profonde tranquillité d’âme, une fondamentale condition d’existence. Quand une armée perd la foi en ses chefs, c’est aussi lamentable pour elle que pour eux, quoique diversement. Eux deviennent impuissants ; elle inexistante. — Le problème que je pose est, d’ailleurs, des plus malaisés à résoudre, quoique non insoluble en soi. La \emph{notoriété} est un des éléments de la gloire ; elle peut se mesurer facilement par le nombre d’individus qui ont entendu parler d’un homme ou d’un de ses actes. Mais l’\emph{admiration}, autre élément non moins essentiel, est d’une mesure plus complexe. Il y aurait à la fois à compter le nombre des admirateurs, à chiffrer l’intensité de leurs admirations, et à tenir compte aussi — ce serait là le \emph{hic} — de leur valeur sociale très inégale. Comment ne pas regarder le suffrage de trente ou quarante personnes de l’élite, en chaque genre d’élite, comme bien supérieur à celui de trente ou quarante individus pris au hasard dans une foule ? Mais comment préciser  \phantomsection
\label{v1p72}numériquement cette supériorité-là ? Si ardu que soit ce problème (que certains anthropologistes simplifient fort, en le réduisant à mesurer l’\emph{indice crânien}, le plus ou moins de \emph{dolicho} ou de \emph{brachy-céphalie}), il faut bien qu’il soit susceptible d’une solution, puisqu’en fait il est résolu tous les jours, dans tous les examens universitaires ou administratifs, pour l’appréciation comparée du mérite des candidats.\par
Ce n’est pas la gloire seulement, c’est la noblesse, c’est le crédit, qui donne à un homme « de la valeur ». Le plus ou moins de noblesse était évalué avec beaucoup de finesse et un discernement délicat des nuances dans les salons de l’ancien régime. Cependant, la noblesse semble comporter bien plutôt des types différents que des degrés inégaux. On les classait malgré tout : dans chaque type de noblesse, soit de robe, soit d’épée, ou dans chacun de leurs sous-types, le degré de noblesse d’un homme se mesurait à l’ancienneté et à l’illustration de son nom, c’est-à-dire au nombre et à la valeur sociale des gens qui le savaient noble, et au nombre de générations qui l’avaient réputé tel, avec plus ou moins de respect. La noblesse est une sorte de notoriété héréditaire, de croix d’honneur apportée en naissant\footnote{ \noindent De la noblesse, comme de la gloire, il convient de remarquer qu’elle est une force, un moyen d’action, pour celui qui la possède, mais qu’elle est une foi, une paix, pour le peuple qui l’admet, et qui, en y croyant, la crée.
 }. La richesse est quelque chose de beaucoup plus simple et de beaucoup plus aisément mesurable ; car elle comporte des degrés infinis et fort peu de types différents, dont la différence va s’effaçant. En sorte que la substitution graduelle de la richesse à la noblesse, de la ploutocratie à l’aristocratie, tend à rendre l’état social plus sujet au nombre et à la mesure.\par
Le crédit d’un homme, né de la croyance du public en lui, est pour lui un grand moyen d’action, comme, pour le public, une grande sécurité, apparente ou réelle. Et les économistes ont raison de parler du crédit. Mais le crédit financier d’un  \phantomsection
\label{v1p73}homme, le seul dont ils s’occupent, n’est pas le seul dont il y ait à s’occuper. La confiance qu’un citoyen suscite, comme homme d’État, comme général, comme savant, comme artiste, est un crédit moral, tout autrement important que la confiance de quelques banquiers en sa solvabilité.\par
Comment naît, comment grandit le crédit d’un homme sous toutes ses formes, ou sa célébrité et sa gloire ? Il vaut bien la peine de s’intéresser à ces divers genres de \emph{production}, aussi bien qu’à la production des richesses et de leur valeur vénale. Et peut-être ces sujets plus neufs se prêtent-ils à des considérations qui ne sont pas inférieures en généralité ni en exactitude à celles que les économistes ont décorées du nom de lois. S’il y a des « lois naturelles » qui règlent la fabrication de tels ou tels articles en plus ou moins grande quantité, et la hausse ou la baisse de leur valeur vénale, pourquoi n’y en auraient-ils pas qui régleraient l’apparition, la croissance, la hausse ou la baisse de l’enthousiasme populaire pour tel ou tel homme, du loyalisme monarchique d’un peuple, de sa foi religieuse, de sa confiance en telles ou telles institutions ? Il y en a, assurément, mais non pas celles que les économistes ont formulées. Je prétends que tout ce qu’on a le droit de légiférer ici consiste en imitations rayonnantes de proche en proche à partir d’initiatives individuelles, en rencontres de ces \emph{rayonnements imitatifs} et en conflits ou accords logiques qui résultent de leurs interférences, comme je l’ai abondamment expliqué ailleurs. Par exemple, la célébrité d’un homme naît quand les premiers hommes qui ont découvert ou imaginé un talent parviennent à faire partager leur admiration dans leur entourage qui la propage au dehors, et ainsi de suite, jusqu’à ce que cette expansion admirative se heurte à des âmes déjà remplies d’une admiration rivale et contradictoire qui elle-même est née et a grandi semblablement. Autant de chocs, autant de combats intérieurs où le plus fort des deux sentiments antagonistes, — le plus fort, parce qu’il se sait ou se croit partagé  \phantomsection
\label{v1p74}par le plus grand nombre d’autres hommes ou par des hommes d’un plus grand poids — l’emporte sur l’autre. Le crédit moral ou financier d’un homme naît et croît de même, il s’arrête et rétrograde en vertu des mêmes causes.\par
Quant à la fameuse \emph{loi de l’offre et de la demande}, qui serait le principe suprême de la détermination des valeurs, on a démontré depuis longtemps, depuis les critiques de Cournot à ce sujet, l’insignifiance de sa portée, si on l’entend au sens vague et indéterminé, le seul où elle soit à peu près vraie, et les erreurs où elle conduit si, l’entendant au sens précis, on se risque à la presser un peu. Voilà pour le domaine économique. Mais, si l’on s’avisait de l’exporter dans les domaines voisins, ce serait bien pis. Qu’on essaie d’expliquer par elle les variations de la valeur-foi, de la valeur-confiance, de la valeur-gloire, de la valeur-crédit même. On sait assez bien comment, c’est-à-dire par prédications enflammées et contagieuses d’apôtres, est née en chaque pays d’Europe, s’est propagée et s’est consolidée au moyen âge, la foi chrétienne. On sait encore mieux par quelles causes, — par propagations des idées scientifiques successivement découvertes, et jugées contraires aux dogmes — la foi religieuse, en chaque peuple, a décliné, et par quelles causes opposées, — par prédications nouvelles de nouveaux apôtres apportant de nouveaux arguments — elle s’est ravivée ici ou là\footnote{ \noindent Les prêtres et les religieux ont étudié les facteurs de la production (lisez reproduction) des croyances, des « vérités », avec non moins de soin que les économistes la reproduction des richesses. Ils pourraient nous donner des leçons sur les pratiques propres à ensemencer la foi (retraites, méditations forcées, prédications) et sur les lectures, les conversations, les genres de conduite qui l’affaiblissent. Les journalistes commencent à être très forts aussi dans l’art d’attiser l’enthousiasme et les gloires de leur parti.
 }. Qu’est-ce qui ressemble, en ces contagions et en ces conflits d’exemples, au rapport inverse entre une offre et une demande ? De quelle offre, de quelle demande peut-il être question ici ? On voit donc que cette loi soi-disant fondamentale de la valeur, ne pourrait jamais être, fût-elle  \phantomsection
\label{v1p75}applicable à la valeur vénale, qu’une loi secondaire et spéciale. Mais la valeur vénale elle-même récuse son autorité.\par
Les économistes ont donné le nom de \emph{marché} au domaine géographique et social où est circonscrit le système des valeurs vénales solidaires les unes des autres et où règne l’uniformité de prix. Qu’est-ce qui correspond au « marché » en fait de valeurs morales, de valeurs scientifiques ou artistiques ? Ne serait-ce pas la \emph{société} dans le sens étroit du mot, le « monde » où la conversation roule sur les mêmes sujets, où l’on a reçu une instruction et une éducation communes ? Il est à remarquer que en vertu de la nature \emph{rayonnante} de l’imitation en tout genre d’exemples, les \emph{sociétés} tendent toujours à déborder leurs limites et à s’étendre sans cesse, comme les \emph{marchés}, si, comme ceux-ci, elles n’y parviennent pas toujours. En d’autres termes, [{\corr le}] champ de l’opinion va s’unifiant et s’élargissant\footnote{ \noindent Voir notre ouvrage sur \emph{l’Opinion et la foule.} Paris, F. Alcan, 1901.
 }, et, avec le sien, celui des valeurs sociales de toute sorte dont elle est l’âme. Cette extension de la valeur en surface n’est pas, d’ailleurs, sans compensation ; et, \emph{pendant qu’elle s’accomplit}, elle s’accompagne d’une instabilité plus grande de la valeur, dont les changements sont plus rapides à mesure que son uniformité est plus étendue. L’observation est facile à vérifier par nos cours publics, où l’on voit les Bourses des capitales du monde entier présenter, le même jour, des cotes de moins en moins inégales, mais, d’une époque à l’autre, des variations plus profondes, — quoiqu’une tendance à la consolidation des cours se laisse déjà apercevoir parce que l’œuvre d’uniformisation des cours est bien près d’être achevée. — Mais ce qui est vrai de la valeur vénale l’est aussi bien de la valeur-gloire, de la valeur-popularité, de la valeur-autorité, qui, de plus en plus facilement étendues, sont de moins en moins durables\footnote{ \noindent Il n’y a pas de Bourse de la valeur littéraire des écrits, de la valeur artistique des peintures, etc., mais elle tend à se former par le groupement des critiques littéraires, des critiques d’art, dans les capitales, et par les \emph{académies}, dont la première destination était, ce semble, de remplir cet office de Bourse morale et esthétique, en servant de métronome à l’opinion.
 }.\par
 \phantomsection
\label{v1p76}Sur la hausse ou la baisse de la valeur, dans tous les sens du mot, l’action de la presse, qui est un puissant agent de l’imitation, est indéniable. Elle agit sur les variations de la valeur vénale, par les informations de la Bourse, par les réclames de tout genre, directes ou indirectes, insinuées en entrefilets captieux ; sur les variations de la valeur littéraire ou scientifique des livres, par ses revues de la littérature ou des sciences ; sur les variations de la valeur morale ou esthétique des œuvres quelconques, des produits quelconques, par l’ensemble des idées qu’elle préconise ; sur les variations de la valeur des personnes, et en particulier, de leur réputation et de leur gloire, par ses diffamations ou ses apologies, par ses encensements redoublés ou ses conspirations du silence.\par
Ce qui est manifeste aussi, c’est que le développement de la presse a pour effet de donner aux valeurs morales un caractère de quantité de plus en plus marqué et propre à justifier de mieux en mieux leur comparaison avec la valeur d’échange. Cette dernière, qui devait être bien confuse aussi dans les siècles antérieurs à l’usage courant de la monnaie, s’est précisée à mesure que la monnaie s’est répandue et unifiée. Alors elle a pu donner naissance, pour la première fois, à l’économie politique. De même, avant la Presse quotidienne, les notions de la valeur scientifique ou littéraire des écrits, de la célébrité et de la réputation des personnes, restaient assez vagues, car le sentiment de leurs accroissements et de leurs diminutions graduels pouvait naître à peine ; mais, avec les développements de la presse, ces idées se précisent, s’accentuent, deviennent dignes de servir d’objet à des spéculations philosophiques d’un nouveau genre. La Presse, en effet, en se répandant, tend à rendre plus nombreux et plus semblables les exemplaires \phantomsection
\label{v1p77} des jugements individuels dont l’ensemble s’appelle l’Opinion, et à rendre plus égale ou moins inégale d’un individu aux autres l’intensité de l’adhésion de chacun d’eux à chacune des idées qu’elle leur suggère. Ce sont là les deux \emph{facteurs} principaux dont la gloire d’un homme ou celle d’un livre est le \emph{produit}.
\subsubsection[{A.2.c. Idées de monnaie, de propriété, de travail, d’association, etc.}]{A.2.c. Idées de monnaie, de propriété, de travail, d’association, etc.}
\noindent Pas plus que l’idée de la valeur, l’idée de la \emph{monnaie} n’est du domaine exclusif des économistes. La monnaie sert de mesure aux richesses surtout, mais non pas à elles seules. D’abord, par le fait même qu’elle est le mètre des richesses, qui sont telles en tant que satisfaisant des désirs et jugées propres à les satisfaire, elle est aussi bien le mètre de ces désirs et de ces croyances. Tout aussi bien elle peut servir à mesurer avec une certaine approximation, par le chiffre comparé des legs pieux, des dons faits au clergé ou aux ordres monastiques, à diverses époques, les attiédissements ou les réveils des croyances religieuses et des mystiques désirs ; par les recettes comparées des théâtres où l’on joue les pièces de diverses écoles, la vogue plus ou moins grande de ces formes de l’art ; par une bonne statistique de la librairie (si elle pouvait être faite, \emph{desideratum} énorme), le succès relatif des divers genres d’écrits et des divers écrivains qui les représentent, ainsi que ses variations successives, qui nous renseigneraient sur celles de l’esprit public, etc. La monnaie est donc le mètre universel des quantités sociales, et non pas seulement des richesses\footnote{ \noindent Il y a bien d’autres mètres : chaque espèce de statistique en est un. La hausse ou la baisse de la popularité d’un homme public se mesure assez exactement par la statistique électorale.
 }.\par
On peut faire cette remarque générale : en tant que mesure des richesses, la monnaie n’a trait qu’à des échanges, ventes ou achats ; mais, en tant que mesure des croyances  \phantomsection
\label{v1p78}considérées à part des désirs, elle a trait surtout à des donations ou encore à des vols. C’est par des munificences, par des générosités en faveur d’œuvres scientifiques, ou littéraires, ou philanthropiques, ou patriotiques, par des souscriptions à des statues ou à des manifestations quelconques, que chacun de nous exprime la nature et la force des convictions qui dominent sa vie, des lumières qui constellent son ciel intérieur. C’est par ses dépenses désintéressées aussi qu’il révèle le degré de son admiration esthétique. Quelquefois même c’est par des vols où se montre la perversion d’un esprit sectaire, l’aberration et la profondeur de ses convictions passionnées.\par
Et, de fait, la donation et le vol sont des notions morales, étrangères en soi à l’économie politique, mais l’\emph{échange} est une notion proprement économique. C’est par métaphore ou abus de langage qu’on dit de deux interlocuteurs qu’ils « échangent leurs idées » ou leurs admirations. Échange, en fait de lumières et de beautés, ne veut pas dire sacrifice, il signifie mutuel rayonnement, par réciprocité de don, mais d’un don tout à fait privilégié, qui n’a rien de commun avec celui des richesses. Ici, le donateur se dépouille en donnant ; en fait de vérités, et aussi bien de beautés, il \emph{donne} et \emph{retient} à la fois. En fait de pouvoirs, il fait quelquefois de même : le roi d’ancien régime délègue son pouvoir judiciaire sans jamais s’en dessaisir. — Aussi le libre-échange des idées, des croyances religieuses, des arts et des littératures, des institutions et des mœurs, entre deux peuples, ne saurait-il, en aucun cas, encourir le reproche qu’on a souvent adressé au libre-échange de leurs marchandises, d’être une cause d’appauvrissement pour l’un d’eux. En revanche, et par cela même qu’il est une addition réciproque, non une substitution, il suscite soit des accouplements féconds, soit des chocs meurtriers, entre les choses hétérogènes qu’il fait se confronter. Il peut donc faire beaucoup de mal, quand il ne fait pas beaucoup de bien. Et, comme ce  \phantomsection
\label{v1p79}libre-échange intellectuel et moral sert toujours tôt ou tard d’accompagnement au libre-échange économique, on en peut dire autant de celui-ci, qui, s’il pouvait être séparé de l’autre, serait ordinairement aussi inefficace qu’inoffensif. Mais, je le répète, ils sont inséparables, et, pour être durable indéfiniment, un tarif prohibitif doit se doubler d’un \emph{Index}, ce prohibitionnisme ecclésiastique.\par
Les idées de \emph{perte} et de \emph{gain} sont applicables aux connaissances, comme aux richesses, quoique l’idée d’échange ne convienne proprement qu’à celles-ci. La mémoire humaine a une capacité limitée, la conscience attentive a un champ des plus étroits. Donc, toute idée nouvelle qui entre dans ce champ, apportée par la conversation, le livre ou le journal, en chasse une autre dont elle prend la place. Expulser n’est pas échanger. On a gagné une idée, on en a perdu une autre. A-t-on plus perdu que gagné ? C’est la question du Progrès. Elle est insoluble si l’on n’accorde qu’il y a une commune mesure des deux idées. L’opinion finit, dans les cercles instruits, par prêter aux idées elles-mêmes, aux théories, aux connaissances, un classement hiérarchique par ordre d’importance, où les mieux démontrées (les plus croyables) et les plus fécondes en applications (les plus désirables) passent avant les moins démontrées et les moins fécondes. Quant à savoir si, de deux théories dont l’une est mieux prouvée mais moins applicable et dont l’autre est plus applicable mais moins prouvée, la première ou la seconde \emph{vaut} davantage, a plus de \emph{vérité}, c’est là une difficulté très délicate, non insoluble toutefois, pas plus que ne le sont les problèmes de physique où des quantités hétérogènes, telles que la masse et la vitesse, entrent en combinaison.\par
La notion de \emph{propriété} est-elle applicable à toutes les acceptions de la valeur ? Peut-être, mais pas dans le sens où les économistes l’entendent comme les juristes, celui de \emph{libre disposition.} En ce sens, un homme n’est pas plus propriétaire \phantomsection
\label{v1p80} de sa réputation, de sa gloire, de sa noblesse, de son crédit, qu’il ne l’est de ses membres, dont il ne saurait se dessaisir — comme membres vivants — en faveur d’autrui. Il n’a donc pas à redouter d’expropriation pour ces valeurs-là, les plus importantes de toutes, les plus impossibles à \emph{nationaliser.} Est-il même propriétaire de ses sensations ? Non, car elles sont essentiellement incommunicables à volonté et par la parole. Il l’est un peu plus de ses convictions et de ses passions, qu’il peut communiquer en les exprimant ; mais, comme nous venons de le dire, en les répandant il ne s’en dépouille pas ; il ne les affaiblit pas même, il les fortifierait plutôt dans son propre cœur par cette expansion hors de lui-même. Les idées que vous avez découvertes, vous les possédez d’une tout autre façon que les richesses que vous avez fabriquées, les eussiez-vous inventées et fabriquées le premier. Vos découvertes et vos inventions, vous les possédez d’autant plus, ce semble, que vous les propagez davantage par la conversation et le discours. Quant aux richesses que vous avez créées, si vous les avez transmises par l’échange ou la vente, elles ne vous appartiennent plus. Vous continuez, il est vrai, si vous en êtes l’inventeur, à posséder leur idée même et le mérite de l’avoir trouvée, mais c’est en tant que vérité et célébrité, non en tant qu’utilité. La distinction reste donc justifiée, car elle repose, au fond, je le répète, sur celle des états de l’âme \emph{représentatifs}, qui sont transmissibles par leur expression verbale, et des états de l’âme \emph{affectifs}, qui sont, comme tels, intransmissibles verbalement. La merveille de l’art est de rendre transmissibles, par l’expression vive qu’il en donne, et qui alors est jugée belle, non pas les sensations et les sentiments, il est vrai, mais leur image, et de socialiser de la sorte, d’apprivoiser en quelque sorte, ces états de l’âme individuels et sauvages par nature.\par
L’idée de \emph{travail}, cela est clair, n’appartient pas en propre à l’économiste. Est travail tout effort humain en vue d’un  \phantomsection
\label{v1p81}but, que ce but soit la production ou l’acquisition des richesses, ou du pouvoir, ou du savoir, ou de la célébrité, ou de la beauté. Ce qui appartient en propre à la théorie des richesses, c’est l’emploi abusif de l’idée du travail qu’elle est trop portée à regarder comme la source unique (y compris le capital, travail accumulé suivant elle) des valeurs spéciales étudiées par elle. Nul juriste n’a commis l’erreur de dire que le travail est l’unique source des droits, et le politique sait bien que le pouvoir d’un homme ou sa popularité sont le fruit de la chance plus que de l’effort, qu’ils naissent plutôt d’une convenance accidentelle entre la nature de cet homme et les besoins de son temps et de son pays que d’une persévérance opiniâtre. Le professeur, de son côté, n’aura jamais l’idée de penser que l’acquisition des connaissances ou la production de beaux tableaux et de belles statues sont proportionnelles chez ses élèves à leur application. Il y a une autre erreur que l’Économie politique a toujours commise et qu’elle est seule à commettre, c’est celle qui consiste à ne pas distinguer, à l’égard de l’importance du travail, entre la création d’un nouveau genre de richesses ou d’un nouveau perfectionnement d’une richesse ancienne, et ce qu’elle appelle la \emph{production}, mais ce qui, en réalité, n’est que la \emph{reproduction} de cette richesse à l’exemple contagieux du premier créateur. Le travail \emph{reproducteur} et copiste de l’ouvrier, de l’élève, du disciple en tout genre, est ingrat et pénible, et, dans une certaine mesure, malgré l’inégalité des talents, son résultat se proportionne à son intensité et à sa durée. Mais le travail joyeux du créateur, du producteur véritable, n’en est pas un à vrai dire, et c’est justement, je crois, que j’ai opposé l’inventeur au travailleur\footnote{ \noindent J’ai développé cette opposition, il y a longtemps, en 1885, dans un article intitulé \emph{Darwinisme naturel et Darwinisme social}, publié par la \emph{Revue philosophique}.
 }. L’inventeur peut avoir passé des années et des années à travailler péniblement sans rien trouver, et tout ce travail-là est  \phantomsection
\label{v1p82}perdu ; quand l’idée a lui, toute cette peine est oubliée, et, brève ou longue, ne compte plus, noyée dans sa joie laborieuse. Et c’est sa joie, non sa peine, qu’on lui paie. Et sa joie, bien plus que sa peine, pourrait servir à mesurer la valeur sociale de son invention.\par
L’idée du \emph{capital} a-t-elle un sens en dehors de la théorie des richesses ? Oui, mais si l’on veut la généraliser, il faut l’entendre en un sens différent de ses acceptions scolaires, d’ailleurs confuses et inconciliables. Disons d’abord que le capital doit être conçu comme une condition \emph{sine quâ non} de la reproduction des richesses, et aussi bien des pouvoirs, des droits, des connaissances, des célébrités, des beautés. Mais cette condition quelle est-elle ? Dirons-nous que c’est l’existence d’un outillage, d’un outillage d’école ou d’atelier artistique aussi bien que d’un atelier industriel ou de ferme ? Mais cette condition, qui n’est pas universelle, puisqu’on ne voit rien d’analogue pour la reproduction des pouvoirs, des droits, des popularités, des célébrités, n’est pas non plus absolument indispensable là où elle apparaît. A la rigueur, s’il n’a pas d’outils, l’ouvrier des champs s’en fabriquera avec d’autres outils plus simples, ou même avec ses doigts ; le peintre, s’il n’a ni couleurs ni pinceaux, parviendra à s’en faire aussi ; mais à une condition, nécessaire celle-là et seule nécessaire : c’est que l’un et l’autre auront déjà vu et vu faire des outils pareils qu’ils prendront pour modèle, à moins que, n’en ayant jamais vu, ni vu faire, ils les inventent. D’autre part on aura beau leur mettre en mains des outils tout fabriqués, ils ne pourront s’en servir s’ils ne l’ont déjà appris, c’est-à-dire si, en regardant leur patron ou leur maître, ils n’ont tâché de se modeler sur lui. L’essentiel, ici, est donc qu’on ait ou qu’on ait eu sous les yeux un type d’action qu’on puisse reproduire à plus ou moins d’exemplaires pour arriver à reproduire un type d’œuvre qu’il faut aussi voir ou avoir vu. La distinction du capital et du travail se ramène ainsi, au fond, à celle d’un modèle et  \phantomsection
\label{v1p83}d’une copie, et l’on voit que, à ce point de vue, il est facile de la généraliser. Il n’est pas de pouvoir politique, judiciaire, sacerdotal, qui n’ait été conféré et répandu à l’exemple d’un type traditionnel ou d’un type étranger, et suivant un procédé électoral, ou autoritaire, ou sacramentel, copié de même sur un premier modèle antérieur ou extérieur. Il n’est pas de droit qui, pour se propager, n’exige des modèles analogues. Il n’est pas de beauté littéraire ou artistique qui ne soit \emph{en partie} (car en partie aussi, elle est ou prétend être modèle à son tour) composée d’imitations d’œuvres et de procédés magistraux. Il n’est pas de gloire qui ne soit faite de manifestations rituelles calquées sur un type ancestral ou importé, applaudissements, bouquets offerts, sonnets en Italie, chansons ailleurs, etc., et d’où résulte une illustration d’un certain genre qui rentre dans un type déjà connu. Partout s’impose, comme condition seule indispensable de la \emph{reproduction}, de ce que les économistes appellent la production, l’existence et la connaissance de modèles, qui partout sont fournis par des inventions, par des initiatives individuelles. Le groupe des inventions connues dans un pays constitue son véritable capital, capital qui s’accroît de deux manières bien distinctes : par le grossissement de ce groupe, ou par la diffusion de la connaissance qu’on en a.\par
Que dirons-nous de la \emph{division du travail}, autre notion aisée à exporter hors de l’économie politique, si ce n’est que les économistes lui ont souvent prêté les précieux avantages qui reviennent, en réalité, à l’harmonie des travaux ? La \emph{division des richesses}, leur différenciation toujours croissante au cours de la civilisation, et correspondant à celle des besoins, à la complication de notre psychologie raffinée, a par elle-même une tout autre portée que la \emph{division des travaux} destinés à les reproduire. De même, la division des pouvoirs, leur complexité croissante, a une signification tout autrement importante que la division ou même la coalition des efforts en vue d’arriver au pouvoir. Il y a aussi  \phantomsection
\label{v1p84}une division des gloires, une multiplicité et une différence toujours grandissantes des célébrités et des notoriétés, et elle mérite beaucoup plus d’être signalée à l’attention que la division des efforts pour parvenir à se faire connaître. C’est en se spécialisant dans un travail unique, bien plus que par la diversité des travaux, qu’on arrive à s’illustrer. Non pas que cette diversité soit sans importance, mais, pour atteindre à la gloire, à la popularité, au pouvoir, au savoir, à la perfection esthétique, aussi bien qu’à la fortune, ce qui importe avant tout, c’est l’harmonie des travaux, la convergence des efforts. Il est vrai que l’harmonie des travaux suppose leur division, mais elle est loin de lui être proportionnelle, et elle peut s’accroître sans que celle-ci augmente ou même pendant qu’elle diminue. Il se pourrait fort bien qu’une meilleure « organisation du travail » dans l’avenir fût liée à une moindre division du travail, ce qui ne serait pas un mal ; mais, si par hasard elle s’accompagnait aussi d’une moindre différenciation des richesses, ce serait autrement fâcheux.\par
C’est par l’\emph{association} consciente ou inconsciente, rassemblée ou dispersée des travailleurs que s’opère la solidarité des travaux. Est-il nécessaire d’insister pour montrer que l’idée d’association n’est pas du domaine exclusif des économistes ? C’est évident.\par
Une société ne poursuit pas seulement à chaque époque la plus grande somme possible de richesses, de connaissances, de gloires, de puissances, de beautés ; elle poursuit encore la plus grande somme possible des richesses, des connaissances, etc., jugées les meilleures possibles à l’époque dont il s’agit. Le problème du \emph{maximum}, déjà si ardu, se complique d’un problème d’\emph{optimum} qui, à première vue, semble insoluble, mais qui, chose étrange, est le plus facilement résolu. Il y a, en tout temps et en tout pays, une hiérarchie de gloires diverses unanimement acceptée, et où une société se reflète aussi bien que dans sa hiérarchie de  \phantomsection
\label{v1p85}crimes, non moins admise par tous. Et ceci s’applique pareillement à sa hiérarchie de pouvoirs ou de savoirs. Le crime \emph{majeur} est tantôt le blasphème, tantôt la lèse-majesté, ou le vol, ou l’homicide ; la gloire majeure était hier militaire, avant-hier religieuse, la canonisation ; elle a été, sous la Restauration, poétique ; aujourd’hui elle serait plutôt celle du comédien que du dramaturge, celle du critique littéraire — extraordinairement en vogue de nos jours — que du poète ou du romancier même. A d’autres époques, en Italie, l’homme le plus glorieux a été un grand peintre, un grand sculpteur, un grand architecte. Dans l’Allemagne actuelle, la gloire musicale l’a emporté de beaucoup sur les autres gloires artistiques. Il y aurait à subdiviser : par exemple, chaque période et chaque nation du moyen âge chrétien se peignent dans la nature de ses saints ; là où fleurissent les saint Jérôme et les saint Augustin, les saint François d’Assise ne sauraient germer ni les Catherine de Sienne\footnote{ \noindent Qu’on lise à ce sujet la \emph{Psychologie des saints}, de M. Henri Joly.
 }.\par
Pareillement, toute société a sa hiérarchie de richesses, non moins significative ou symptomatique, et dont la comparaison avec les hiérarchies voisines pourrait être fort utile à l’économiste. Il n’est pas indifférent que la richesse majeure soit en troupeaux, ou en sacs de blé, ou en capitaux, ou que les capitaux les plus recherchés consistent en créances nominatives sur des particuliers ou en rentes d’États au porteur\footnote{ \noindent Comment peut-il se faire que des choses hétérogènes, sans commune mesure apparente, soient classées, soient jugées supérieures ou inférieures les unes aux autres ? Cela n’est possible que parce qu’elles sont l’objet de désirs ou de croyances plus ou moins intenses, plus ou moins généralisés. Si donc l’on refuse d’admettre le caractère quantitatif de ces deux forces psychologiques et sociales à la fois, le classement dont il s’agit devient inexplicable, je ne dis pas simplement injustifiable.
 }. N’y a-t-il pas quelque chose d’irréversible dans cette série, qui va, non seulement du concret à l’abstrait, du particulier au général, mais surtout du personnel à l’impersonnel, de rapports intimes avec un groupe étroit de personnes, parents et serviteurs, pendant l’ère  \phantomsection
\label{v1p86}pastorale, puis agricole, à des rapports anonymes avec d’innombrables et invisibles inconnus ? Et n’y a-t-il rien d’analogue dans les séries voisines ? Le pouvoir et le droit le plus fort, le type idéal du pouvoir et du droit, a commencé par être celui du patriarche sur sa famille, du suzerain sur ses vassaux, lien tout personnel d’homme à homme, formé par l’hérédité du sang ou le serment héréditaire ; il a fini par être le pouvoir, le droit du fonctionnaire sur le public. Le devoir le plus fort, le plus sacré, a commencé par être familial ; il a fini par être patriotique ou humanitaire. La gloire la plus enivrante a commencé par être l’admiration fanatique, passionnée, dévouée, d’un petit groupe de partisans, de thuriféraires serrés dans une chapelle étroite, elle a fini par être l’acclamation relativement froide, et sans dévouement vrai, d’un public dispersé et immense, qui ne connaît pas personnellement son héros\footnote{ \noindent Bien entendu, cette irradiation froide de la renommée d’un homme n’est possible, à l’origine du moins, que par l’existence d’un foyer chaud, d’une coterie enthousiaste.
 }. La vérité la plus crue et la plus vivante, la vérité par excellence, a commencé par être le crédo d’une petite secte, vérité toute personnelle acceptée sur la foi d’autrui ; elle a fini par être la vérité impersonnelle des lois scientifiques. En toutes ces transformations, en tous ces courants multiformes d’une société à l’autre, mais pareillement impossibles à remonter sur leur pente commune, se constate la tendance irrésistible à l’élargissement du champ social, orientation générale de l’histoire.\par
Quant à la multiformité de ces évolutions, indépendamment de leur même sens, elle tient aux caprices de l’opinion, qui dirige tantôt ici, tantôt là, le rayon de son regard visuel, sous l’influence de ses directeurs et des circonstances. Par exemple, de la disposition à admirer ou à respecter telle ou telle chose, ou à admirer en général plus qu’à respecter, naissent les célébrités en tel ou tel genre. Nos nations contemporaines \phantomsection
\label{v1p87} ont de grands talents ou même de grands génies, parce qu’elles ont gardé le sentiment de l’admiration ; elles n’ont plus guère de grands caractères, ou semblent n’en plus avoir, parce qu’elles ont perdu le sentiment du respect. Les grands respects font les grandes vertus renommées, comme les grandes indignations font les grands crimes, comme les grandes convictions font les grandes vérités. J’ajoute : comme les grandes confiances nationales font les grands hommes d’État. Une presse systématiquement diffamatoire tarit leurs [{\corr sources}]. — On voit par là ce qu’il y a d’accidentel dans la gloire. A génie naturel égal, un homme rencontrera ou ne rencontrera pas d’idées géniales, suivant que les éléments de cette idée lui seront ou ne lui seront pas apportés par les courants croisés de l’imitation. Et, à égale génialité d’idées découvertes, elles le rendront illustre ou obscur suivant qu’elles rencontreront ou non un public soucieux d’elles et disposé à les accueillir.
\subsubsection[{A.2.d. Valeur d’un livre, production et concurrence des livres, offre et demande des livres.}]{A.2.d. Valeur d’un livre, production et concurrence des livres, offre et demande des livres.}
\noindent Nous venons de passer en revue quelques-unes des principales notions dont l’économie politique fait usage ; nous avons vu que les plus importantes, celles de valeur, de travail, d’association, ne lui appartiennent pas en propre et gagnent encore en importance à être généralisées ; que d’autres, celles de propriétés, d’échange, de division du travail, se transforment en se transplantant et ne sont pas généralisables sans remaniement. Il en est une dont nous n’avons rien dit encore, et dont nous dirons simplement qu’elle n’a guère d’emploi possible en dehors du domaine économique ; elle est donc, à notre point de vue actuel, la moins importante de toutes, quoique les économistes l’inscrivent en tête de l’une des trois ou quatre grandes divisions de leur science : la \emph{consommation}. On consomme des  \phantomsection
\label{v1p88}richesses, c’est-à-dire que pour les employer on doit les détruire plus ou moins vite ou lentement ; et peut-être peut-on dire aussi que l’on consomme un pouvoir en l’exerçant, ou en en abusant ; mais consomme-t-on sa gloire, son crédit même ? Consomme-t-on ses croyances en y pensant et les chefs-d’œuvre qu’on admire en les regardant\footnote{ \noindent Les richesses tendent â être consommées, mais les gloires ne tendent pas à être détruites. Leur destruction est un accident fréquent, presque universel, mais qui est toujours étranger à leur nature propre. Chacune d’elles, considérée à part des autres, tend à croître indéfiniment jusqu’aux limites de sa \emph{société}, mais non à rétrograder ensuite. Sa rétrogradation ne peut être que l’effet de sa rencontre accidentelle avec d’autres renommées incompatibles avec elle, ou bien de la mort naturelle des admirateurs, quand ils ne sont pas remplacés. C’est la mort, phénomène biologique et non sociologique, qui met le plus sûrement fin aux renommées que des causes sociales ont produites, ou qui donne lieu à leur affaiblissement, à leur refroidissement graduel. A chaque vieux grognard qui mourait, après 1815, la gloire de Napoléon, en ce qu’elle avait de vivant et de contagieux, de brûlant et de rayonnant, était diminuée, comme l’est une illumination par l’extinction d’une bougie. Il est vrai qu’à cette gloire vivante, de première formation, succède, pour les génies privilégiés, une gloire plus froide, fantomatique et crépusculaire, qui peut même aller en s’étendant toujours mais n’en est pas moins à la précédente ce que la vie d’Achille aux champs Elysées était à son existence sous les murs de Troie. — Quand un homme sympathique et bon vient à mourir, il reste de lui quelque chose de vivant encore et qui peut être comparé hardiment à la gloire du plus grand capitaine : le souvenir tendre, affectueux ou reconnaissant, qu’il a laissé au cœur de ceux qui, l’ayant vu et apprécié, le revoient encore, l’entendent parler et reconnaissent pour ainsi dire le son de sa voix. Mais, malheureusement, quand ceux-là meurent à leur tour, ils ne sont pas remplacés ; aussi peut-on dire que, à chacun d’eux qui meurt, le mort aimé perd un peu de sa vie survivante, s’achemine vers sa seconde mort bien plus navrante peut-être que l’autre, car elle n’est remarquée ni pleurée de personne. Toutes les fois que nous apprenons la mort de quelqu’un, si indifférent qu’il nous puisse être, qui a connu le parent ou l’ami dont nous portons le deuil, nous devrions nous attrister profondément, si nous avions le cœur moins léger.\par
 Pour revenir à cette distinction des deux gloires, l’une chaude et vivante, l’autre spectrale et glacée, remarquons qu’elle est susceptible d’autres applications. Est-ce qu’il n’y aurait pas à distinguer aussi entre le pouvoir vivant et le pouvoir survivant, entre la noblesse vivante et la noblesse survivante, « ombre d’elle-même », et même entre la vérité vivante et la vérité pâle, exsangue, froidement répétée de bouche ? — La gloire vivante consiste bien plus en conversations qu’en écrits ; la gloire survivante, en écrits qu’en conversations ; elle finit par ne plus consister qu’en signes commémoratifs, une statue, une plaque de marbre, une inscription sur un rocher...
 } ? En revanche, il est deux idées dont ils font un usage beaucoup trop restreint, celles d’alliance et de luttes, d’\emph{adaptation} et d’\emph{opposition}, et qui sont susceptibles au plus haut degré d’être  \phantomsection
\label{v1p89}généralisées. Ils ne s’occupent de l’opposition qu’à propos de la concurrence des productions ou des consommations, négligeant l’opposition cachée et continuelle des produits qui joue un rôle économique capital, et ils n’ont nul égard non plus aux alliances invisibles des produits, à leurs adaptations fécondes.\par
Mais, pour bien comprendre ce point, prenons un exemple concret qui nous permettra en même temps d’examiner, à la lueur de nos comparaisons, certaines « lois » des économistes. Nous emprunterons cet exemple au \emph{livre.} — On a pu dire de la monnaie qu’elle est une marchandise comme une autre, et c’est déjà une grande erreur. Mais, si l’on disait que les livres sont des marchandises comme d’autres, ce serait une erreur encore bien plus profonde. Ranger les livres parmi les richesses, c’est confondre ce qui a trait à l’intelligence avec ce qui a trait au besoin ou à la volonté. La \emph{valeur} d’un livre est une expression ambiguë, car chacun de ses exemplaires, en tant qu’il est tangible, appropriable, échangeable, consommable, a une valeur vénale qui exprime son degré de \emph{désirabilité}, mais, en lui-même, en tant qu’intelligible, inappropriable, inéchangeable, inconsommable essentiellement, ce qui ne veut pas dire indestructible, il a une valeur scientifique, qui exprime son degré de \emph{crédibilité}, sans compter sa valeur littéraire, qui veut dire son degré de séduction expressive\footnote{ \noindent Il n’y a pas que les livres, nous le savons, qui comportent une valeur autre que leur valeur d’échange ; d’objets fabriqués quelconques on peut dire aussi qu’ils ont, outre leur valeur vénale, leur valeur morale ou esthétique, et aussi leur valeur documentaire (qui rentre dans la valeur scientifique). Cette valeur documentaire est destinée à survivre à toutes les autres. Elle se dégage avec netteté et se renforce quand, dans un tombeau antique, on découvre un vase, une épée, un ustensile quelconque qui ne vaut plus rien pour l’usage auquel il était primitivement destiné mais dont l’intérêt comme document est immense.
 }. Mais, soit considéré comme produit, soit considéré comme enseignement, un livre est susceptible de s’allier à d’autres livres ou de les combattre. Il n’est pas de livre, considéré comme enseignement, \phantomsection
\label{v1p90} qui ne soit fait avec d’autres livres dont il donne souvent la bibliographie, et parmi lesquels il en est dont on peut dire qu’il est fait pour eux, car il les confirme et les complète. Et il n’est pas de livre non plus qui ne soit fait \emph{contre} d’autres livres. De même il n’est pas de produit qui ne soit fait soit \emph{avec} et \emph{pour}, soit \emph{contre} d’autres produits. Les produits se combattent quand, par des procédés différents, ils poursuivent la satisfaction du même besoin, chacun d’eux prétendant être le meilleur et niant la prétention analogue des autres : par exemple, l’éclairage au gaz ou à l’électricité, les tissus de toile ou de coton, la faïence ou la porcelaine, etc. Ils s’allient, ils s’adaptent, quand ils se complètent au point de vue de la satisfaction d’un même besoin ou répondent à des besoins connexes : verres et assiettes, chemises et faux-cols, habit et cravate blanche, etc. En un autre sens, un produit s’allie à tous les produits antérieurs qu’il utilise. Dans l’automobile s’allient la roue, la machine à pétrole, le caoutchouc pneumatique, etc. Il n’est pas de produit qui ne soit ou ne puisse devenir auxiliaire d’un autre, \emph{outil} de cet autre. La distinction de l’outil et du produit n’a donc qu’une vérité relative ou superficielle.\par
Il faut distinguer, pour les marchandises comme pour les livres, deux manières de se combattre : leur \emph{concurrence} et leur \emph{contradiction.} Leur concurrence, c’est-à-dire leur émulation à la poursuite de la meilleure solution d’un même problème, de la satisfaction d’un même besoin, est excellente et louable, même accompagnée d’une contradiction indirecte et implicite, celle de leurs prétentions contraires ; mais leur choc direct, leur contradiction violente, sous la forme de polémiques littéraires ou de guerres commerciales, est détestable.\par
Si nous cherchions les conditions générales de la production des livres, comme les économistes ont cherché celles de la production des marchandises, nous verrions que la distinction célèbre des trois facteurs, Terre, Capital et Travail,  \phantomsection
\label{v1p91}peut à la rigueur s’appliquer ici mais avec de grandes et instructives transformations, notamment en ce qui concerne le capital qui devrait être conçu comme le legs, sans cesse grossi, des bonnes idées du passé, des découvertes et inventions successives. — Mais, s’il est des « lois » de la production (ou reproduction) des marchandises et des livres, pourquoi n’y en aurait-il pas aussi de leur \emph{destruction ?} Et pourquoi n’est-il question de celle-ci, en économie politique, qu’implicitement et à propos de la consommation, chapitre sur lequel on s’étend trop ? Cependant ce ne sont pas seulement les produits consommés qui sont détruits. Sans avoir même servi, combien de mobiliers ont été remplacés et anéantis par d’autres mobiliers, combien de bibliothèques par d’autres bibliothèques ! Que de fois l’humanité a renouvelé son outillage avant qu’il ne fût usé, ses provisions en tout genre encore intactes, comme elle a renouvelé ses sciences et ses arts, ses laboratoires et ses musées ! Car ces choses-là ne sont pas, le plus souvent, mortes de leur belle mort, les unes ont tué les autres. Et ces \emph{duels} incessants, ces duels logiques ou téléologiques des livres et des marchandises, des statues et des meubles, demanderaient à être étudiés de près, non moins que leurs unions fécondes.\par
Comment se fait un livre ? Ce n’est pas moins intéressant que de savoir comment se fabrique une épingle ou un bouton. Dans un cas comme dans l’autre, il y aurait à distinguer ce qui revient à l’association des travaux (lisez la division du travail) et à la concurrence des travailleurs. Il faudrait distinguer aussi diverses sortes d’association qui concourent à la produire, concentrées ou dispersées, plus ou moins intimes, plus ou moins vastes, et, en somme, de plus en plus vastes — et de plus en plus conscientes, malgré leur étendue croissante — à mesure que le grand filet des relations humaines se déploie sur le monde social élargi. Un livre se fait maintenant avec l’aide de collaborateurs, bien plus nombreux, bien plus dispersés et lointains, et en  \phantomsection
\label{v1p92}même temps bien plus connus ou moins inconnus, qu’autrefois. Il en est de même d’un article industriel. — D’ailleurs ces collaborateurs sont bien rarement co-auteurs. La règle, en fait de livres, c’est la production \emph{individuelle}, tandis que leur propriété est essentiellement collective ; car la « propriété littéraire » n’a de sens individuel que si les ouvrages sont considérés comme marchandises, et l’idée du livre n’appartient à l’auteur exclusivement qu’avant d’être publiée, c’est-à-dire quand elle est encore étrangère au monde social. Inversement, la production des marchandises devient de plus en plus collective et leur propriété reste individuelle et le sera toujours, alors même que la terre et les capitaux seraient « nationalisés ». — Il n’est pas douteux que, en fait de livres, la \emph{libre production} s’impose comme meilleur moyen de produire. Une organisation du travail scientifique qui réglementerait législativement la recherche expérimentale ou la méditation philosophique donnerait de lamentables résultats. — En fait de production \emph{livresque} aussi bien qu’industrielle, il y a des \emph{crises.} En quoi consistent-elles ? Dans les crises de la librairie, ne confondons pas ce qui est de nature économique et ce qui est de nature littéraire. Une crise, en théorie, peut provenir aussi bien d’un déficit que d’un excès de fabrication. Mais, en fait, on ne donne ce nom de nos jours, qu’aux phénomènes économiques causés par l’encombrement du marché. En est-il de même des crises littéraires ? Est-ce que ce qui est vivement ressenti et remarqué par le public, ce n’est pas la disette des livres et d’informations quelconques pour répondre à sa vive curiosité ou à son goût du moment, encore plus que leur surabondance ? Et n’est-ce pas pourtant en fait de livres surtout, bien plus qu’en fait de marchandises, que la crainte de la \emph{surproduction} est légitime et justifiée ?\par
La \emph{demande} d’un livre, et aussi bien d’un produit, n’est qu’une des conditions de sa venue au jour. Elle ne suffit  \phantomsection
\label{v1p93}pas. Le besoin public a beau réclamer, à certaines époques, une plus grande abondance de monnaie d’or, tant que de nouvelles mines d’or ne sont pas découvertes, la monétisation de ce métal n’augmente pas. Pendant les ravages du phylloxéra, le vin était demandé instamment, et cette demande contribuait certainement à faire cultiver la vigne tant qu’on pouvait, mais, aussi longtemps que le remède au fléau n’a pas été \emph{trouvé}, il a été impossible d’étancher la soif générale. De même, le goût public a beau appeler à grands cris un grand poète dramatique, ou la curiosité publique un grand rénovateur des études historiques, de telle ou telle science, il faut attendre l’apparition d’un talent naturel servi par les circonstances, sorte de mine naturelle à découvrir et à exploiter pour les besoins intellectuels.\par
La production des livres, comme celle des produits, a des phases dont la série est irréversible. Si nous n’y distinguons rien qui ressemble à la succession des phases chasseresse, pastorale, agricole, industrielle, nous y remarquons sans peine une phase initiale que Le Play n’a pas eu le tort de signaler, celle de la \emph{cueillette.} Il y a eu, certainement, un âge de la cueillette aisée et facile des idées nées d’elles-mêmes, à l’usage des premiers moralistes, des premiers poètes chanteurs, des premiers conteurs... Et, en fait de livres, comme en fait de produits, la fabrication à la main a précédé la \emph{machinofacture}, l’imprimerie, substitution qui n’a pas eu de moins graves conséquences ici que là.\par
La loi des débouchés, de J.-B. Say, suivant laquelle les produits trouvent d’autant plus facilement à s’écouler qu’ils sont plus abondants et plus variés, ne s’applique-t-elle pas beaucoup mieux aux livres, aux connaissances, qu’aux produits ? Ou, pour préciser davantage, ne s’applique-t-elle pas beaucoup mieux aux succès littéraires des livres qu’à leurs succès de librairie, ce qui fait deux ? Cette formule cesse d’être vraie au cas où une marchandise qui s’offre rencontre sur le marché beaucoup d’autres marchandises de même  \phantomsection
\label{v1p94}espèce ou répondant au même besoin : une lampe à pétrole ne se vendra pas d’autant plus facilement dans une localité qu’il y aura déjà dans les magasins plus de lampes à pétrole ou même de lampes électriques ou autres. Au contraire, un nouveau livre a d’autant plus de chances d’être lu et d’intéresser dans une région du public qu’il y a déjà plus de livres traitant des mêmes sujets, répondant aux mêmes problèmes. A l’inverse, une marchandise trouve dans un pays un débit d’autant plus facile qu’il y a déjà plus de marchandises hétérogènes, répondant à d’autres besoins, tandis qu’un livre d’histoire, par exemple, n’aura pas un succès d’autant plus probable dans un pays qu’il y rencontrera plus de livres de sciences naturelles ou mathématiques, ou de théologie. Toute société cultivée présente, intellectuellement, un certain courant général d’opinion et de goût qui lui fait dévorer aujourd’hui telle ou telle espèce de publications et dédaigner toutes les autres, qu’il dévorera demain. Cependant, à mesure que le lien de solidarité intime qui unit toutes les sciences et même les rattache aux diverses branches de l’art, apparaît mieux, ce que je viens de dire est moins applicable, et, dans un milieu \emph{très} cultivé, un livre est d’autant plus sûr d’intéresser que la production livresque, en n’importe quel genre, est plus abondante. Un homme d’une curiosité encyclopédique, et le nombre s’en accroît peu à peu, est d’autant plus avide de nouvelles lectures qu’il en a fait de plus étendues sur de tout autres sujets.\par
Cette exception ou cette élite mise à part, demandons-nous à quoi tient l’inversion signalée plus haut. Dans la mesure où elle est exacte, elle tient à ce qu’un homme instruit, et aussi bien un public instruit, peut s’absorber dans un seul problème qui lui fait longtemps oublier tous les autres, ou se passionner pour un genre littéraire à l’exclusion de tous les autres, tandis que jamais homme, à moins d’être un monomane à enfermer, ne s’est préoccupé passionnément d’un seul besoin, celui de boire ou de se vêtir,  \phantomsection
\label{v1p95}au point d’oublier tous les autres. Le bien-être, poursuivi par l’activité économique, consiste en un chœur et non un solo de besoins, organiques ou artificiels, harmonieusement satisfaits et solidaires les uns des autres, le confort du logement faisant désirer d’autant plus le confort du vêtement et de l’alimentation, et ceux-ci faisant souhaiter l’agrément des voyages, etc. Pourtant il n’est pas douteux que les passions intellectuelles, les problèmes théoriques ou esthétiques, s’enchaînent et s’engendrent comme les besoins du corps. Aussi, quand l’un des grands problèmes de l’esprit, quand l’une des grandes curiosités ou avidités intellectuelles se développe dans un public, on peut être sûr qu’à côté il y a d’autres publics nourrissant les curiosités et les avidités complémentaires. Mais c’est bien rarement dans le même public qu’elles coexistent toutes, tandis que c’est sur le même marché en général, que coexistent les besoins multiples et solidaires auxquels l’industrie répond.\par
Au surplus, l’inversion signalée ne saurait être que passagère, de même que l’application de la loi de Say. En effet, où tend la production des livres ? Elle tend, ou elle court sans le savoir, à reconstituer, sous des formes nouvelles, le bel équilibre mental — toujours temporaire — que le système catholique avait établi au moyen âge dans la chrétienté, ou le polythéisme homérique dans la Grèce antique. Quand la science moderne sera arrivée à son terme, je veux dire à sa période de fixation définitive des principes et des méthodes, il y aura, dans tous les esprits, une même hiérarchie des connaissances, un enchaînement de problèmes systématisés auxquels il aura été répondu par un certain nombre de livres capitaux, et répondu si parfaitement que la plupart des livres nouveaux, s’ils ne sont pas leur réédition avec variantes, seront refoulés par cette suprématie reconnue des anciens. Les livres dissidents seront mis à l’index social. On ne lira plus que les livres orthodoxes, conformes aux principes et les prolongeant, et, à vrai dire,  \phantomsection
\label{v1p96}ceux-là seulement seront, pour un temps plus ou moins long, instructifs et beaux.\par
Pareillement, où tend, où court inconsciemment la production économique ? A la constitution, à la reconstitution d’une morale, sans qu’il y paraisse : je veux dire à l’établissement d’une hiérarchie des besoins réputée par tous juste et normale. Et tout produit nouveau qui sera de nature à ébranler cette hiérarchie en stimulant avec excès certains besoins au détriment de tels autres, sera éliminé. Et les produits jugés conformes à la \emph{téléologie régnante} — car la morale n’est que cela — seront seuls achetés. Il ne sera donc plus vrai de dire alors, — comme il ne l’eût pas été non plus au {\scshape xiii}\textsuperscript{e} siècle — que l’écoulement d’un produit quelconque sera facilité par la présence, sur le marché, de produits plus abondants et plus variés. Quand le produit excommunié s’offrira, il aura beau rencontrer la plus grande abondance et la plus grande variété de produits autorisés par les mœurs, il ne se vendra pas. Au fond, ce n’est pas tant la variété que la solidarité et l’harmonie des produits qui doit importer ici, aux yeux du sociologue. Je ne dis pas peut-être aux yeux de l’artiste, ni même du philosophe, qui voit dans la libre diversité la source et la fin, pour ainsi dire divine, des choses, l’alpha et l’oméga de l’univers.
 \phantomsection
\label{v1p97}\subsection[{A.3. Discussion du plan}]{A.3. Discussion du plan}\phantomsection
\label{ppch3}
\noindent Par les comparaisons multipliées qui précèdent, j’ai cherché à faire sortir l’Économie politique de son isolement majestueux et décevant. Tâchons de montrer maintenant que, si l’on concevait la pensée de constituer à côté d’elle et sur un plan plus ou moins rapproché du sien, une théorie des connaissances (science pédagogique, dans le sens encyclopédique de \emph{logique sociale} pure et appliquée), une théorie des pouvoirs (science politique), une théorie des droits et des devoirs (science juridique et morale), une théorie des beautés (science esthétique), de manière à pouvoir les embrasser toutes ensemble avec elle dans une même théorie générale de la valeur, il faudrait nécessairement faire subir au plan sur lequel la théorie des richesses a été édifiée ou échafaudée jusqu’ici un complet remaniement.\par
Les défauts et les lacunes de ce plan nous sont déjà signalés par les essais embryonnaires de ces théories diverses tels qu’ils se sont déjà produits. A propos des pouvoirs, il est vrai, ou des droits, ou des lumières, on s’est occupé, avant tout, de leur \emph{reproduction}, et de leur \emph{répartition}, de leur diffusion plutôt, comme à propos des richesses, mais on s’est occupé aussi de leurs origines, de leur véritable \emph{production} par insertions successives d’idées nouvelles, ce dont l’économie politique n’a nul souci ; et l’on s’est occupé plus encore de leurs accords ou de leurs désaccords et de leur emploi. L’économie politique ne paraît pas se douter qu’il y ait des accords ou des désaccords (psychologiques) des richesses entre elles.\par
 \phantomsection
\label{v1p98}M. Gide a critiqué avec beaucoup de raison la division en quatre branches, devenue classique : \emph{production} (lisez \emph{reproduction}), \emph{circulation, répartition, consommation.} Il n’a jamais pu comprendre, dit-il, à quoi répond la circulation. « Elle n’est rien de plus qu’une conséquence et un aspect de la division du travail. » Il aurait pu pousser, je crois, sa critique plus avant. D’abord, à quoi bon consacrer toute une partie de la science des richesses à leur \emph{consommation ?} On a pu remarquer combien ce chapitre-là est vide et insignifiant, farci de remplissages hétérogènes, de généralités sur le luxe notamment, dans la plupart des traités. En réalité, la consommation est inséparable de la production qui ne se conçoit pas sans elle, qui ne doit faire qu’un théoriquement avec elle. La reproduction des besoins, chez le consommateur, explique et provoque seule la reproduction des efforts adaptés à leur satisfaction, chez le producteur. La consommation, considérée à part de la production (reproduction) des articles consommés, n’est qu’un fait de jouissance individuelle, qui ne prend un caractère social qu’autant qu’elle sert à leur reproduction directe ou indirecte ou à celle d’autres articles, ou à la création d’articles nouveaux. Dans le premier cas, elle fait partie du titre de la production (reproduction) des richesses, où, en effet, il est nécessairement question à chaque ligne de la consommation des subsistances par les travailleurs. Dans le second cas, elle rentre dans le titre inédit de l\emph{’adaptation} économique, car un nouvel article n’a de succès (sans succès il ne compte pas socialement), que s’il est adapté à des besoins nouveaux ou mieux adapté à des besoins anciens. Reste le cas où la consommation est absolument inféconde : cette destruction pure et simple de richesse, perte sèche pour la société, est un simple accident qui n’intéresse pas plus la science théorique qu’un incendie ou un éboulement de rocher.\par
Des quatre branches de l’Économie politique, deux sont coupées. Conserverons-nous au moins les deux autres, \emph{production \phantomsection
\label{v1p99}} (reproduction) et \emph{répartition ?} Mais ce dernier mot est ambigu. Si, par répartition des richesses, on entend leur diffusion, ce qui suppose la propagation imitative du désir qu’on a de chacune d’elles, de la confiance qu’on a en leur utilité, il n’y a pas lieu de séparer de la reproduction des richesses cette répartition-là qui a avec elle un lien si étroit. Mais, si par répartition des richesses, on vise surtout leur échange, leur appropriation et l’association libre ou forcée qui se crée ainsi entre les co-échangistes, sous l’empire des circonstances dominantes ou des règles imposées, ce n’est pas ici, c’est ailleurs, comme nous allons le voir plus loin, qu’il est à propos d’en parler. Il convient donc d’exposer en bloc, dans un même chapitre, sans solution factice de continuité, tout ce qui a trait à la reproduction et à la diffusion des richesses. Je dis diffusion plutôt que répartition, car elles tendent essentiellement à se répandre de plus en plus, à travers les frontières des pays et les distances des couches sociales : c’est là leur manière naturelle de se répartir.\par
Il ne subsiste qu’une seule des quatre parties de l’économie politique. Lui laisserons-nous son ancien titre : « production des richesses ? » Non, il est doublement défectueux, car, d’une part, il s’agit des manières de reproduire la richesse déjà créée, et non de la produire pour la première fois ; d’autre part, pourquoi ne s’occuper, dans la science économique, que de ces entités, les richesses, et non des hommes qui les demandent ou les fabriquent ? On a reproché avec raison aux criminalistes classiques de n’avoir égard qu’aux \emph{crimes} et non aux \emph{criminels ;} un reproche analogue, celui de s’inquiéter beaucoup plus des \emph{produits} que des \emph{producteurs}, est mérité par nombre d’économistes du passé. Aujourd’hui, il est vrai, en économie politique comme en \emph{criminalistique}, une réaction vive s’opère contre cette obsession de l’abstraction scolastique et ce mépris de la réalité vivante. Donc, il importe de changer l’étiquette du sac dont le contenu n’est plus le même et est destiné à se modifier  \phantomsection
\label{v1p100}encore. Au lieu de « production des richesses », disons \emph{répétition économique ;} et par là nous entendrons les relations que les hommes ont entre eux, au point de vue de la propagation de leurs besoins semblables, de leurs travaux semblables, de leurs jugements semblables portés sur l’utilité plus ou moins grande de ces travaux et de leur résultat, de leurs transactions semblables. Circulation et répartition des richesses ne sont qu’un effet de la répétition imitative des besoins, des travaux, des intérêts et de leur rayonnement réciproque par l’échange. Noter la connexion de cette propagation par imitation avec la propagation des hommes eux-mêmes par génération, car le problème de la population se pose ici comme partout en tête de la science. Il y aurait à marquer la profonde analogie entre les lois qui régissent la répétition des richesses et celle des hommes : la \emph{tendance} à croître en progression géométrique s’applique aux richesses comme aux hommes, en dépit du contraste factice et erroné que Malthus a cru apercevoir entre les deux. Malthus dit que la population \emph{tend} à grandir en progression géométrique, tandis que les subsistances ne seraient susceptibles que d’une progression arithmétique. Mais ces subsistances, qu’est-ce ? Des plantes, des animaux, des produits chimiques ou industriels. Or, les animaux comme les plantes, les bœufs et les moutons comme les céréales, \emph{tendent}, au même titre que l’homme, ni plus ni moins, c’est-à-dire en tant qu’êtres vivants, à une multiplication \emph{rayonnante}, de plus en plus large ; et, si cette \emph{tendance} est contrariée par des obstacles de plus en plus forts en ce qui concerne les plantes et les animaux, elle ne l’est pas moins en ce qui concerne l’espèce humaine. Si, quand l’humanité naissante servait d’aliment favori aux grands fauves des âges préhistoriques, quelque Malthus eût apparu parmi les ours ou les lions des cavernes, il aurait pu faire, en la retournant, la même remarque que le Malthus historique et songer tristement aux faibles progrès de l’espèce humaine, nourriture  \phantomsection
\label{v1p101}des ours ou des lions, pendant que ceux-ci se multipliaient si vite. Quant aux produits industriels, ils \emph{tendent} aussi à se répandre par rayonnement, mais par rayonnement imitatif ; car la seule différence, à cet égard, entre l’art pastoral et l’agriculture d’une part, et l’industrie de l’autre, c’est que le pâtre et l’agriculture font travailler au profit de l’homme la tendance des animaux ou des plantes à se multiplier par répétition — génération (combinée avec la répétition-imitation des divers pâtres ou des divers agriculteurs), tandis que l’industrie proprement dite est une répétition — imitation seulement.\par
Mais poursuivons l’esquisse de notre plan de psychologie économique. Nous arrêterons-nous là, au titre de la Répétition économique ? Non, évidemment. Les besoins et les travaux des hommes ne se répètent pas seulement, ils s’opposent souvent, et plus souvent s’adaptent. C’est à la condition d’être adaptés les uns aux autres qu’ils parviennent à se répéter. On pourrait donc, si l’on voulait, placer en première ligne leurs rapports d’adaptation, mais peu importe. — Sous le titre d’\emph{opposition économique} je me propose de comprendre les rapports des hommes au point de vue de la contradiction psychologique et inaperçue de leurs besoins et de leurs jugements d’utilité, du conflit plus apparent de leurs travaux par la concurrence, par les grèves, par les guerres commerciales, etc. Toute la théorie des \emph{prix}, de la \emph{valeur-coût}, qui suppose des luttes internes et des sacrifices de désirs à d’autres désirs, se rattache à ce même sujet. — Sous le titre d’\emph{adaptation économique}, il sera traité des rapports que les hommes ont entre eux au point de vue de la coopération de leurs inventions anciennes à la satisfaction d’un besoin nouveau ou à la meilleure satisfaction d’un besoin ancien, ou de la coopération de leurs efforts et de leurs travaux en vue de la reproduction des richesses déjà inventées (\emph{association} implicite ou explicite, organisation naturelle ou artificielle du travail). L’invention et l’association \phantomsection
\label{v1p102} seraient présentées sous le même jour, comme il convient. Toutes les « harmonies économiques » seraient discutées là, celles qui existent et aussi celles qui devraient exister et qui existeront.\par
Notre discussion nous conduit donc à substituer aux quatre divisions classiques de la science des richesses, trois divisions fort différentes. Si l’on veut bien se donner la peine d’essayer un refonte de l’économie politique sur ce nouveau type, on verra, je crois, ce qu’elle peut y gagner en élimination de ce qui lui est étranger, en meilleure distribution de ce qui lui appartient et qu’elle possédait déjà, en acquisition de ce qu’elle avait négligé de revendiquer comme sien. Elle deviendra à la fois plus nette et plus dense, mieux délimitée et mieux remplie. Et, en même temps, apparaîtra la fécondité de ce classement tripartite qui peut être appliqué, aussi bien à la théorie des connaissances, à la théorie des pouvoirs, des droits, des devoirs, à l’esthétique. Mais disons quelques mots encore de l’opposition et de l’adaptation économiques. Ce qu’il est essentiel de remarquer, c’est que, en tête de ces deux divisions inédites aussi bien que de la première, se place le problème de la population, posé en termes différents pour chacune d’elles. La répétition économique n’avait à envisager la population qu’au point de vue de son accroissement numérique par génération ; l’opposition économique devrait s’occuper de sa destruction par la guerre, cette anti-génération. Par rapport au progrès lent des subsistances, l’excès possible de la population avait le droit d’effrayer ; par rapport à l’éventualité des conflits, au progrès rapide des armements, le déficit possible de population est, au contraire, un juste sujet d’effroi pour le patriote. Et de quel droit l’économiste se désintéresserait-il de la patrie ?\par
Par un autre côté, plus exclusivement social, cette question capitale de la guerre appartient à l’économiste : par le côté de la fabrication des armes et des munitions, de l’alimentation \phantomsection
\label{v1p103} des troupes, de la construction des vaisseaux de guerre. Cette production de l’outillage belliqueux est une industrie à part, mais qui, protectrice de toutes les autres, ne saurait être passée sous silence ou dédaigneusement traitée en passant. Pourquoi n’aurait-elle pas ses « lois naturelles » aussi, comme les autres ? Si les économistes ne les ont pas cherchées, c’est peut-être parce qu’ils ont senti d’instinct, l’impossibilité de les faire rentrer dans leurs règles trop étroites. Ces singulières \emph{richesses} qui remplissent les arsenaux de l’État ou ses chantiers maritimes ne sont pas destinées, comme les autres, à être échangées ni vendues, mais bien à être \emph{données} ou volées à l’ennemi et \emph{consommées} par cette donation même ou le vol. Que viendrait faire ici l’idée, soi-disant universelle, du libre échange ? Il n’y a nul échange. Peut-on s’en rapporter à l’initiative privée pour construire les grands cuirassés et les fusils perfectionnés ? et doit-on ici appliquer le principe du laissez-faire et du laissez-passer ? La marine de l’État forme avec la marine marchande un profond contraste au point de vue économique, comme la cavalerie avec l’art pastoral, comme l’artillerie avec la carrosserie. L’industrie guerrière est essentiellement collectiviste et prohibitionniste. Elle est et doit rester un monopole de l’État. Autant, pour le progrès des industries pacifiques, la nation est intéressée à la divulgation des secrets de fabrique, des inventions nouvelles, autant elle doit désirer que, pour tout ce qui touche aux industries militaires, les inventions nouvelles restent enveloppées d’un profond mystère, trop souvent, hélas ! décevant. De là l’espionnage et le contre-espionnage, les trahisons, les procès de trahison, et les illusions vaniteuses des diverses armées, jusqu’au jour décisif du combat. Ce caractère mystérieux des procédés de fabrication était commun, dans les temps les plus reculés, à toute industrie, par suite de l’hostilité fréquente des classes ou des cités ; à présent il est restreint en temps de paix à l’industrie guerrière, mais, en temps de  \phantomsection
\label{v1p104}guerre, où toute industrie a la guerre pour âme, le monopole antique tend de nouveau à se généraliser. Pour être monopolisées, les industries guerrières des divers États n’en sont pas moins concurrentes ; au contraire. Mais l’idée de concurrence a ici un sens tout autre que l’acception usuelle, et un sens bien plus pur, où l’idée de conflit destructeur se présente dégagée de tout alliage avec celle de concours fécond.\par
L’étude de la concurrence des industries guerrières, comme celle de la guerre elle-même, comme celle de la concurrence des industries quelconques, conduit à reconnaître l’élargissement progressif de ces diverses oppositions qui deviennent à la fois de moins en moins nombreuses et de plus en plus importantes\footnote{ \noindent Je me permets de renvoyer le lecteur à mes \emph{Lois sociales}, à cet égard.
 }, condition favorable à leur apaisement futur. Par cette tendance à l’amplification indéfinie, l’opposition économique ressemble à la répétition économique et aussi, comme nous allons le voir, à l’adaptation économique.\par
A ce dernier point de vue, comme aux deux précédents, se pose pour l’économiste le problème de la population, dans des termes nouveaux. Il ne suffit pas d’étudier les questions qui se rattachent à l’excès ou au déficit de population, et à la destruction belliqueuse des populations ; il faut s’inquiéter avant tout des causes de l’amélioration ou de la dégénérescence de la race, des conditions hygiéniques et des institutions sociales qui permettent à la race de s’adapter de mieux en mieux à sa destinée, notamment à la reproduction des richesses et à la défense nationale. Il est inouï qu’une science si préoccupée de l’élevage des animaux et de la culture des plantes, en vue des progrès de la richesse, ait si peu de souci de la \emph{viriculture.}\par
Mais, ici comme plus haut, ce problème bio-sociologique de la population n’est que préliminaire, malgré sa gravité hors ligne. Les problèmes exclusivement sociologiques de  \phantomsection
\label{v1p105}l’adaptation économiques sont ceux qui ont trait à l’invention, à l’échange et à l’association, trois choses intimement unies. — Les inventions se contredisent souvent, ou se contrarient, et par ce côté de leurs rapports extérieurs, elles sont la source première de l’opposition économique. Mais chacune d’elles, prise à part, est une adaptation d’inventions anciennes naguère étrangères les unes aux autres et convergeant vers un but nouveau ; et chaque invention nouvelle, en tant qu’elle triomphe, rend plus complète l’adaptation des conditions d’existence aux besoins de l’homme. D’autre part, les inventions accumulées vont s’harmonisant et se systématisant à travers des heurts. L’enchaînement logique des inventions, qui domine toute l’évolution sociale, doit être traitée ici par l’économiste en ce qui le concerne.\par
Par l’échange de plus en plus libre dans un domaine de moins en moins resserré et clos, — toujours clos d’ailleurs et toujours limité, — les diverses richesses se consomment là où elles sont le plus utiles, les hommes s’utilisent de mieux en mieux, chaque homme utilise un plus grand nombre d’hommes plus éloignés et plus étrangers, et réciproquement. L’échange est déjà une espèce d’association vague et inconsciente qui va s’élargissant. — Il n’est pas d’échange sans propriété. La question de la propriété, individuelle et collective, peut sembler pourtant, à des yeux socialistes, se rattacher au chapitre de l’opposition plutôt qu’à celui-ci. « Qui terre a guerre a. » Mais le partage et la délimitation des propriétés n’ont-ils pas été les seuls moyens pratiques de faire juxtaposer sans chocs continuels et meurtriers les avidités rivales, et de substituer, quand elles se heurtent, les procès aux combats, les discussions aux homicides ? Et n’est-ce pas aussi dans l’intérêt de l’harmonie et de la paix sociale que l’héritage a été imaginé ? C’est donc au chapitre de l’adaptation que la propriété doit être discutée et qu’on doit se demander si l’indivision collective réaliserait mieux l’accord social\footnote{ \noindent Tout ce qui a trait à la légitimité ou à l’illégitimité de l’intérêt des capitaux, de la rente des terres, du profit des entreprises, doit être, par suite, traité ici.
 }.  \phantomsection
\label{v1p106}Ce qui est certain c’est que, le principe de la propriété individuelle étant posé, à mesure qu’il s’est propagé et a provoqué les progrès de la législation, les transformations de la jurisprudence, dans le sens de son extension territoriale et de sa cohésion interne, grandissantes à la fois, on a vu le bornage et l’héritage des terres devenir un élément plus fondamental de concorde.\par
Enfin, l’association véritable et consciente, celle qui continue l’échange et le complète par la division croissante des travaux et surtout par la croissante collaboration des diverses catégories de travailleurs à une œuvre commune, constitue le degré supérieur de l’adaptation économique. Toute la science économique aboutit à cette question à laquelle cependant on a souvent accordé moins de développements qu’à celle de la division du travail, qui n’en est qu’un aspect. Toute l’activité économique, en effet, converge vers l’association et ses agrandissements progressifs, depuis la petite culture et les petites industries familiales du plus haut passé jusqu’à la très grande culture et à la très grande industrie que commence à voir le présent, que verra surtout l’avenir, sous la forme d’ateliers ou de fermes gigantesques, dirigés par des sociétés nationales ou internationales supérieures à nos plus grandes compagnies. La loi d’amplification graduelle s’applique là comme plus haut avec une évidence qui doit frapper tous les yeux.\par
Je me persuade que, refondue dans le cadre qui vient d’être esquissé bien imparfaitement, l’économie politique échapperait au reproche d’étroitesse et de sécheresse, et ferait apparaître son vrai visage, tout psychologique et tout logique, tout vivant et tout rationnel. Il s’agit de bien comprendre l’Évolution économique. Mais \emph{évolution} est un mot vague ; on le précise, en l’analysant, par ces trois termes, répétition, opposition, adaptation, qui distinguent ce qu’il confond.
 \phantomsection
\label{v1p107}\subsection[{A.4. Coup d’œil historique}]{A.4. Coup d’œil historique}\phantomsection
\label{ppch4}
\noindent L’insuffisance manifeste des principes de l’économie politique classique a fait sentir depuis longtemps la nécessité de les élargir. A cela ont travaillé en sens différents ou divergents l’école historique d’une part, les écoles socialistes de l’autre. Mais, dans les discussions confuses qui se sont engagées ici et là entre novateurs et conservateurs, il y a eu beaucoup de malentendus. La plupart des coups n’ont pas porté, ou n’ont pas atteint au cœur l’adversaire, qui tout en se défendant mal, a cependant résisté. L’école historique aurait eu raison si elle s’était bornée à dire que les lois économiques véritables, découvertes ou à découvrir, doivent s’appliquer différemment aux diverses phases de la vie pastorale, agricole, industrielle d’une société, ou, dans un autre sens, aux périodes successives de son économie domestique, urbaine, nationale, internationale, comme les lois physiques reçoivent des applications différentes aux corps vivants à mesure qu’ils se transforment. Mais, au lieu de présenter ainsi les résultats de ses savantes recherches comme propres à déployer le riche contenu, jusque-là replié et caché, de l’économie politique véritable, elle a eu l’air de nier au fond l’existence et la possibilité même de celle-ci en attribuant un caractère essentiellement relatif et passager à toutes les prétendues lois qu’elle pourrait jamais formuler. De cela les économistes se sont plaints, et non sans cause : ils ont signalé la contradiction qu’il y a à vouloir réformer  \phantomsection
\label{v1p108}l’économie politique en la détruisant de fond en comble et la déclarant à jamais impossible.\par
Quant aux socialistes, ils n’ont pas contesté le caractère général et permanent des lois économiques, de quelques-unes du moins dont ils se sont emparés, pour combattre l’optimisme économique par ses propres armes. J’ajoute que leur point de vue les a conduits souvent à serrer de plus en plus près les réalités psychologiques méconnues ou négligées par l’économie classique, et à la frapper parfois aux points vraiment vulnérables. Mais, en somme, c’est plutôt sur le domaine de l’\emph{art} que sur celui de la \emph{science} que les socialistes combattent les économistes. Quand ils ont fait de la psychologie, c’est sans le vouloir, comme les économistes eux-mêmes ; aussi en ont-ils fait souvent d’assez mauvaise, les uns et les autres. Et c’est en philanthropes, non en psychologues, que les premiers ont fait la leçon aux seconds, avec la sévérité que l’on sait.\par
D’autres écrivains ont cru remédier assez à l’insuffisance, bien sentie, de l’économie politique, en montrant ses rapports avec la morale ou avec le droit, et leurs travaux les ont conduits à se montrer souvent bien plus psychologues que leurs prédécesseurs. Toutefois c’est surtout à l’art et à la pratique économique qu’ils ont égard.\par
Mais nous, c’est sur le terrain de la science, en ce qu’elle a de plus tranquille et de plus désintéressé, que nous avons à nous placer. A notre avis, — disons-le franchement tout d’abord, avec tout le respect dû à des maîtres de si haut mérite, — l’erreur des premiers architectes de l’économie politique et de leurs successeurs a été de se persuader que, pour constituer en corps de science leurs spéculations, le seul moyen, mais le moyen sûr, était de s’attacher au côté matériel et extérieur des choses, séparé autant que possible de leur côté intime et spirituel, ou, quand c’était impossible, de s’attacher au côté abstrait, et non concret, des choses. Par exemple, il fallait s’occuper des produits plutôt que des  \phantomsection
\label{v1p109}producteurs et des consommateurs ; et, dans le producteur ou dans le consommateur — car enfin on ne pouvait éviter d’en parler — il fallait considérer une dépense de force motrice (travail) ou un réapprovisionnement de force, et non des sensations, des émotions, des idées, des volontés. Être aussi \emph{objectif} et aussi \emph{abstrait} qu’on le pouvait : c’était là la méthode... L’idéal était de dissimuler si bien sous des abstractions, telles que crédit, service, travail, les sensations et les sentiments cachés là-dessous, que personne ne les y aperçût, et de traiter ces abstractions comme des objets, des objets réels et matériels, analogues aux objets traités par le chimiste et le physicien, et, comme eux, tombant sous la loi du nombre et de la mesure. Aussi le chapitre de la monnaie et des finances, où ce double idéal semble se réaliser, où tout semble nombrable et mesurable comme en physique et en chimie, a-t-il été de tout temps le carreau de prédilection du jardin des économistes. Il n’en est pas moins vrai que la valeur, dont la monnaie n’est que le signe, n’est rien, absolument rien, si ce n’est une combinaison de choses toutes subjectives, de croyances et de désirs, d’idées et de volontés, et que les hausses et les baisses des valeurs de Bourse, à la différence des oscillations du baromètre, ne sauraient s’expliquer le moins du monde sans la considération de leurs causes psychologiques, accès d’espérance ou de découragement du public, propagation d’une bonne ou d’une mauvaise nouvelle à sensation dans l’esprit des spéculateurs.\par
Ce n’est point que les économistes aient tout à fait méconnu cet aspect subjectif de leur sujet ; et même, dans ces dernières années, certaines écoles étrangères ont paru le mettre quelque peu en lumière, mais toujours incomplètement à mon gré ; toujours on l’a regardé comme le \emph{verso} et non comme le \emph{recto} de la science économique. Ses maîtres ont cru à tort, je le répète, que la préoccupation dominante, sinon exclusive, du côté extérieur pouvait seule ériger leurs observations à la dignité d’un corps de science. Même quand  \phantomsection
\label{v1p110}ils ont dû envisager directement le côté psychologique des phénomènes étudiés par eux, les mobiles du travailleur et les besoins du consommateur, par exemple, ils ont conçu un cœur humain tellement simplifié, tellement schématique pour ainsi dire, une âme humaine si mutilée, que ce minimum de psychologie indispensable avait l’air d’un simple postulat destiné à soutenir le déroulement géométrique de leurs déductions.\par
Mon intention est de montrer au contraire, que, si l’on veut atteindre en économie politique à des lois véritables, et, par conséquent, vraiment scientifiques, il faut retourner pour ainsi parler, le vêtement toujours utile mais un peu usé des vieilles écoles, faire du verso le recto, mettre en relief ce qu’elles cachent et demander à la chose signifiée l’explication du signe, à l’esprit humain l’explication du matériel social.\par
\subsubsection[{A.4.a. Importance de la psychologie, et surtout de l’inter-psychologie, en économie politique. Le loisir et le travail. Répartition du loisir. Psychologie du travailleur.}]{A.4.a. Importance de la psychologie, et surtout de l’inter-psychologie, en économie politique. Le loisir et le travail. Répartition du loisir. Psychologie du travailleur.}
\noindent La nature éminemment psychologique des sciences sociales, dont l’économie politique n’est qu’une branche, aurait donné lieu à moins de contestations si l’on avait distingué deux psychologies que l’on a l’habitude de confondre en une seule. Si par ce mot on entend l’étude de ce qui se passe dans le cerveau, tel que la conscience du moi nous le révèle, quand le moi est impressionné par les objets du dehors, ou par les images de ces impressions, il convient de remarquer que les objets du moi peuvent être ou bien des choses naturelles, \emph{insondables à fond en leur for intérieur hermétiquement} clos, ou bien d’autres moi, d’autres esprits \emph{où le moi se reflète en s’extériorisant} et apprend à se mieux connaître lui-même en découvrant autrui. Ces derniers objets du moi, qui sont en même temps des sujets comme lui, donnent lieu à un rapport entre eux et lui tout à fait exceptionnel, qui tranche nettement, en haut-relief, parmi les  \phantomsection
\label{v1p111}rapports habituels du moi avec les êtres de la nature, minéraux, plantes, et même animaux inférieurs. D’abord, ce sont là des objets dont le moi ne peut révoquer en doute la réalité, sans infirmer la sienne propre : ils sont l’écueil du [{\corr scepticisme}] d’école. En second lieu, ils sont les seuls objets qui soient saisis par leur dedans, puisque la nature intime est celle-là même dont le sujet qui les regarde a conscience. Mais, quand le moi regarde des minéraux ou des astres, des substances matérielles quelconques, organiques ou inorganiques, les forces qui ont produit ces formes ne peuvent êtres devinées que par hypothèse, et leur signe extérieur est seul perçu.\par
On comprend donc très bien que, lorsqu’il s’agit d’étudier les rapports du moi avec les êtres naturels et de fonder les sciences physiques, y compris même la biologie, le moi s’évertue, en bonne méthode, à s’oublier lui-même le plus possible, à mettre le moins possible de lui-même et des impressions personnelles qu’il reçoit du dehors dans les notions qu’il se fait de la matière, de la force et de la vie, à résoudre, s’il se peut, la nature tout entière en termes d’étendue et de points en mouvement, en notions géométriques et mécaniques, dont l’\emph{origine}, toute psychologique aussi, ne se décèle qu’à des yeux d’analystes très exercés et n’implique d’ailleurs en rien leur \emph{nature} psychologique. Mais est-ce une raison pour que, lorsque le moment est venu d’étudier les rapports réciproques des moi, c’est-à-dire de fonder les sciences sociales, le moi continue à s’efforcer de se fuir lui-même, et prenne pour modèle de ses nouvelles sciences les sciences de la nature ? Par le plus exceptionnel privilège, il se trouve, dans le monde social, voir clair dans le fond même des êtres dont il étudie les relations, tenir en main les ressorts cachés des acteurs, et il se priverait bénévolement de cet avantage, pour se modeler sur le physicien ou le naturaliste qui, ne le possédant pas, sont bien forcés de s’en passer et d’y suppléer comme ils peuvent !\par
 \phantomsection
\label{v1p112}On pourrait donner le nom de psychologie inter-cérébrale, ou d’inter-psychologie (barbarisme commode) à l’étude des phénomènes du moi impressionné par un autre moi, sentant un être sensible, voulant un être volontaire, percevant un être intelligent, \emph{sympathisant} en somme avec son objet. On réserverait le nom de psychologie individuelle à l’étude du moi isolé, impressionné par des objets tout autres que ses semblables. — Or, si, en psychologie individuelle, la méthode qu’on a appelée \emph{introspective}, celle qui consiste à \emph{s’écouter sentir}, à se regarder penser, à se replier sur soi-même et enregistrer ses faits intérieurs, a donné lieu à beaucoup de reproches, souvent fondés, il me semble que ces objections sont sans portée contre l’emploi de cette même méthode en inter-psychologie. L’introspection, quand il s’agit d’observer des phénomènes inter-psychologiques, c’est-à-dire sociaux, est une méthode d’observation subjective et objective en même temps. Et c’est même ici la seule méthode qui atteigne sûrement son objet. Car cet objet, en matière sociale, est toujours, en somme, quelque chose de mental qui se passe dans la conscience ou la subconscience de nos semblables. Et où pouvons-nous mieux étudier cet objet que dans son miroir qui est en nous-mêmes ?\par
Ce n’est pas seulement de l’inter-psychologie, c’est aussi de la psychologie individuelle que relève l’économie politique. On en peut dire autant de toutes les autres sciences sociales. En linguistique, par exemple, beaucoup de choses s’expliquent par l’impression directe des phénomènes naturels sur l’esprit de l’homme : de là les onomatopées, les harmonies imitatives, etc., c’est là la \emph{source}. Mais bien plus de choses encore ne sont explicables que par l’action unilatérale ou réciproque des esprits en contact, qui se sont emprunté les mêmes sons pour exprimer les mêmes idées, et se sont sciemment ou à leur insu imités entre eux au lieu d’imiter les bruits de la nature. En mythologie comparée, l’espoir ou la peur, l’enthousiasme ou l’abattement de l’homme isolé  \phantomsection
\label{v1p113}en présence des grands spectacles de la nature, surtout des animaux et des plantes, jouent un grand rôle, moins grand cependant que l’action exercée sur les foules contagieusement hallucinées par quelque puissant séducteur d’âmes, thaumaturge, ascète, saint, prophète. En esthétique, si l’on veut comprendre la naissance d’un poème, d’un monument, d’une statue, d’un tableau, d’une école de musique, il faut faire une large part aux inspirations du climat, de la flore et de la faune ambiantes, qui ont timbré à leur sceau l’âme de l’artiste ; mais, pour exercer une influence vraiment sociale, il est nécessaire que ces suggestions directes de la nature se combinent intimement avec des suggestions tout autrement profondes et continues du milieu humain, de la tradition ancienne ou de l’engouement momentané.\par
Dans le domaine économique, il en est de même. Il est impossible d’y expliquer les besoins d’alimentation, de vêtement, d’abri, sans avoir égard, avant tout, à l’action directe des agents extérieurs sur la sensibilité de l’individu ; et il n’est pas moins impossible d’y rendre compte des besoins supérieurs d’art, de luxe, de vérité, de justice, ou d’y définir les notions de crédit et de valeur, sans invoquer les actions et les réactions mutuelles des sensibilités, des intelligences, des volontés humaines en échange perpétuel d’impressions. — Toutefois, il n’y a pas lieu d’étudier à part ces deux sortes d’influences, et, à vrai dire, il n’y aurait guère moyen. Car elles s’entre-croisent sans cesse, et c’est seulement en remontant aux premières années de l’enfance ou aux débuts hypothétiques des sociétés qu’on peut atteindre à des phénomènes de psychologie individuelle tout à fait séparés des phénomènes d’inter-psychologie. Ceux-là ne nous apparaissent jamais qu’à travers ceux-ci, verres déformants ou transfigurants qui exercent une réfraction de plus en plus forte au fur et à mesure des progrès de la vie sociale. Même les besoins les plus grossiers de l’organisme, tels que boire et manger, ne sont ressentis que moyennant \phantomsection
\label{v1p114} des communications traditionnelles ou capricieuses d’esprit à esprit : ainsi le besoin de manger se spécifie en désir de manger ici du pain, ailleurs du riz ou des pommes de terre ; le besoin de boire, en désir de boire ici du vin, ailleurs du cidre ou de la bière ; et c’est seulement en se spécifiant de la sorte que ces besoins, estampillés pour ainsi dire par la société, entrent dans la vie économique.\par
Cette importance croissante des considérations tirées de l’inter-psychologie suffit à justifier déjà le reproche que je me permets d’adresser aux économistes de n’avoir pas été assez psychologues ou de l’avoir été mal. Quand ils l’ont été, ils n’ont fait que de la psychologie individuelle, celle précisément dont le rôle est subordonné et sans cesse amoindri. Ce défaut capital se montre avec évidence dans leur conception de ce qu’on a appelé l’\emph{homme économique}, sorte d’être spirituel abstrait, supposé étranger à tout autre sentiment que le mobile de l’intérêt personnel\footnote{ \noindent « L’économie politique, dit Carey, ayant créé à son usage un être auquel elle a donné le nom d’\emph{homme} et de la composition duquel elle a exclu tous les éléments constitutifs de l’homme ordinaire qui lui étaient communs avec l’ange, en conservant soigneusement tous ceux qu’il partageait avec les bêtes fauves des forêts, s’est vue forcée, nécessairement, de retrancher de ses définitions de la richesse tout ce qui appartient aux sentiments, aux affections et à l’intelligence. »
 }. C’est oublier que la conscience du \emph{moi} ne se précise et ne s’accentue, ne se réalise à vrai dire, que par la conscience d’\emph{autrui}, espèce très singulière du genre \emph{non-moi.} S’il en est ainsi, les progrès de l’égoïsme ne sauraient être, dans l’humanité, que parallèles aux progrès de l’altruisme, à la condition d’entendre ce mot, d’ailleurs malsonnant, dans le sens le plus large, de préoccupation d’autrui, bienveillante ou malveillante, sympathique ou antipathique, sentimentale toujours. Le lien qui unit l’égoïsme et l’altruisme ainsi entendu est donc indissoluble, et la prétention d’isoler le premier est chimérique.\par
Cet \emph{homo æconomicus}, qui poursuivrait exclusivement et méthodiquement son intérêt égoïste, abstraction faite de  \phantomsection
\label{v1p115}tout sentiment, de toute foi, de tout parti pris, n’est pas seulement un être incomplet, il implique contradiction. Quel est l’homme dont l’intérêt le plus cher ne soit pas précisément d’éviter toute lésion faite à sa foi et à son orgueil, à son cœur et à son culte ? Dira-t-on que le progrès de la raison, accompagnement présumé du progrès de la civilisation, se charge de réaliser peu à peu l’abstraction imaginée par les économistes et de dépouiller l’homme concret de tous ses mobiles d’action, hormis le mobile de l’intérêt personnel ? Mais rien ne permet cette supposition et il n’est pas un seul des aspects de la vie sociale où l’on ne voie la passion croître et se déployer en même temps que l’intelligence. Dans le langage, est-ce que le style va se décolorant et se refroidissant ? En politique, est-ce que la névrose des partis va s’apaisant ? En religion, est-ce que la part des sentiments et de l’imagination se fait moindre ? Dans le domaine de la science même, est-il certain que la part de l’enthousiasme créateur, fécond en belles hypothèses, en théories larges et spécieuses, ait diminué depuis les Grecs ? Ainsi en est-il dans le monde économique, et nulle part, pas même ici, je n’aperçois trace d’une transformation réfrigérante de l’homme dans un sens de moins en moins passionnel et de plus en plus rationnel. Je n’aperçois pas non plus le contraire, mais il me semble que la passion et la raison, d’âge en âge, progressent ensemble.\par
En concevant l’\emph{homo æconomicus}, les économistes ont fait une double abstraction. C’en est une d’abord, et très abusive, d’avoir conçu un homme sans rien d’humain dans le cœur ; et c’en est une autre, ensuite, de s’être représenté cet individu comme détaché de tout groupe, corporation, secte, parti, patrie, association quelconque. Cette dernière simplification n’est pas moins mutilante que l’autre, d’où elle dérive. Jamais, à aucune époque de l’histoire, un producteur et un consommateur, un vendeur et un acheteur, n’ont été en présence l’un de l’autre, d’abord sans avoir été unis  \phantomsection
\label{v1p116}l’un à l’autre par quelque relation toute sentimentale, voisinage, concitoyenneté, communion religieuse, communauté de civilisation, et, en second lieu, sans avoir été escortés chacun d’un cortège invisible d’associés, d’amis, de coreligionnaires, dont la pensée a pesé sur eux dans la discussion du prix ou du salaire et finalement l’a imposé, au détriment le plus souvent de leur intérêt strictement individuel. Jamais, en effet, même dans la première moitié de {\scshape xix}\textsuperscript{e} siècle — et cependant c’est la seule période de l’histoire du travail où toute corporation ouvrière ait paru anéantie en France — jamais l’ouvrier n’est apparu libre de tout engagement formel ou moral avec des camarades, en présence d’un patron tout à fait dégagé lui-même d’obligations strictes ou de convenances envers ses confrères ou même ses rivaux.\par
Encore pouvait-on croire, depuis la suppression des corporations par la Révolution française jusqu’à 1848 environ, que les associations ouvrières d’autrefois, avec leurs hostilités réciproques, avec leurs puérilités d’orgueil et d’amour-propre collectifs, colorées de mysticisme, étaient chose enfouie pour toujours dans le passé, poussière et cendre irressuscitables. Mais, sans parler du \emph{Compagnonnage} qui n’avait jamais cessé de vivre ou de se survivre, sans parler d’autres \emph{Unions} « compagnonniques » qui attestaient le besoin incompressible d’un \emph{esprit de corps}, les syndicats professionnels, enfin, ont surgi, et, avec eux, des vanités corporatives d’une taille gigantesque, des passions d’une intensité inouïe, des ambitions de conquêtes prodigieuses, une sorte de religion nouvelle, le socialisme, et une ferveur prosélytique inconnue depuis la primitive Église. — Voilà les intérêts, \emph{les intérêts passionnés}, qu’il s’agit d’accorder ensemble et avec les intérêts, tout aussi passionnés, de capitalistes milliardaires coalisés, non moins qu’eux grisés par l’espoir de vaincre, par l’orgueil de la vie, par la soif du pouvoir.\par
 \phantomsection
\label{v1p117}Et c’est ce monde tumultueux de l’activité économique, c’est-à-dire poignante et profonde, souffrante et laborieuse, qui serait régi par une déduction géométrique de froids théorèmes à la Ricardo, applicables à je ne sais quel homme de bois, schématique ou mécanique ! A la psychologie économique il appartient de réintégrer à sa vraie place, la première, tout le côté appelé \emph{sentimental} de la production, de la répartition, de la consommation des richesses ; de l’étudier dans la vie des anciennes corporations, où il se manifeste avec tant de pittoresque originalité, et dans la vie des nouvelles où il éclate avec plus de vigueur encore. C’est en Amérique, c’est dans le pays le plus utilitaire, nous dit-on, le plus avancé dans la voie du progrès économique, que l’on a imaginé les \emph{grèves sympathiques}, les grèves faites par des ouvriers qui n’y ont aucun intérêt et qui en souffrent, simplement pour se solidariser avec des camarades dont le sort les \emph{intéresse.} Et on n’a nulle part vu autant de sacrifices pécuniaires faits à une idée, à une question de principe, à une sympathie, que sur cette terre d’élection de l’intérêt bien entendu.\par
Non seulement les fondateurs de l’économie politique, quand ils se sont montrés quelque peu psychologues, n’ont eu égard qu’à la psychologie individuelle, mais encore ils se sont fait de celle-ci l’idée la plus étroite et la plus mutilée, celle d’ailleurs qu’on s’en faisait de leur temps, au {\scshape xviii}\textsuperscript{e} siècle. Cette psychologie « hédonistique » qui réduit tous les mouvements de l’âme à des peines évitées ou à des plaisirs recherchés, qui ne voit rien au delà du culte du plaisir ou de la fuite de la douleur comme but de l’existence, était si répandue chez les esprits cultivés de ce grand siècle qu’elle les inspirait à leur insu. Voilà pourquoi l’économie politique classique est née à cette époque. Elle n’aurait pu naître plus tôt, au {\scshape xvii}\textsuperscript{e} siècle, où régnait une psychologie individuelle plus haute et plus noble ; et, quand notre siècle, avec Maine de Biran, avec toutes les nouvelles écoles de  \phantomsection
\label{v1p118}psychologie, a montré l’étroitesse du point de vue « hédonistique », on peut dire que l’économie politique, à partir de ce moment, a été mise en demeure de mourir ou de se métamorphoser, de disparaître ou de renaître. Car l’économie politique, telle que nous l’avons connue jusqu’ici, est une sorte de sociologie inconsciente et incomplète qui se fonde sur la seule psychologie de la sensibilité, méconnaissant à peu près celle de l’intelligence et de la volonté, de la foi et du désir. Encore, dans le domaine de la sensibilité, n’a-t-elle trait qu’à l’opposition des états agréables et des états pénibles, et néglige-t-elle le caractère spécifique de ces états. Quelquefois même elle va plus loin, et, chez certains pessimistes, qui broient du noir, elle semble ne faire jouer un rôle important qu’à la douleur, nullement au plaisir. Malthus et tous ses disciples, par exemple, sont persuadés que tout progrès s’opère sous l’unique aiguillon de la souffrance. D’autres, tout en reconnaissant que « à une période avancée de la vie des êtres (individus ou sociétés), le plaisir peut devenir, au moyen de l’imagination qui le fait goûter d’avance, le but et le moteur de l’activité », sont persuadés aussi qu’au début de la vie individuelle ou de la vie sociale la fuite de la souffrance a seule agi. Et je me demande ce qui autorise à exclure des débuts même de l’évolution la recherche de l’agrément positif, directement visé. Elle joue, chez le nouveau-né, un rôle aussi apparent que la répulsion de la souffrance. N’y a-t-il pas de la gourmandise dès ses premiers efforts pour téter ; et de la curiosité dès ses premiers regards ? Est-ce que, chez les peuples les plus primitifs, l’amour de la danse et du chant, de la volupté et du jeu, ne sont pas les fins dominantes et habituelles de l’action ?\par
L’importance véritablement exagérée que les économistes attribuent au \emph{travail}, dont ils s’occupent toujours tandis qu’ils disent à peine, çà et là, quelques mots de son contraire, s’explique par la psychologie mutilée qui les inspire inconsciemment. Le travail, c’est de la douleur ou de l’ennui, en  \phantomsection
\label{v1p119}un mot de la peine. Dire que la valeur des choses consiste dans le travail, soit dans le travail qu’elles ont coûté à leur producteur, soit dans le travail qu’elles épargnent à leur consommateur, c’est définir en termes essentiellement psychologiques, mais d’une psychologie bien insuffisante, la notion économique fondamentale.\par
L’illusion est de croire que notre production agricole, industrielle ou autre, que notre richesse ou notre puissance en tout genre, est le fruit exclusif de notre travail. Notre travail n’y a été que pour une part, il n’a valu que par la collaboration séculaire de tous les ancêtres dont nous sommes les héritiers. Et cela même ne suffit pas, il faut, pour que notre travail ait un effet grand et durable, que nous fassions collaborer à cette œuvre contemporaine notre postérité même, ce qui arrive quand, préoccupés d’elle en vertu d’idées religieuses ou de sentiments domestiques, sa pensée amasse en nous des trésors de dévoûment et d’abnégation qui doublent nos forces et le prix de nos efforts. Nous utilisons nos aïeux, même sans penser à eux ; mais nous ne pouvons utiliser nos enfants et nos petits-enfants, pour ainsi parler, qu’à la condition d’avoir leur pensée présente, de les aimer, et d’être convaincus qu’ils sont notre raison d’être. C’est ainsi que bien des sentiments et bien des croyances, à première vue étrangères à la science économique, se montrent à nous comme des facteurs principaux de la production, dont elle ne peut se dispenser de parler.\par
Disons aussi que la vie économique de l’homme ne se compose pas seulement de travaux, mais de loisirs tout aussi bien ; et le loisir, dont les économistes se désintéressent presque, y est même plus important à considérer, en un sens, que le travail ; car le loisir n’y est pas pour le travail, mais bien le travail pour le loisir. Par leurs travaux les hommes s’entre-servent ; par leurs loisirs, par leurs fêtes et leurs jeux, ils s’unissent en un accord vraiment libre et vraiment social ils s’entre-plaisent. Sur ce point, les religions qui  \phantomsection
\label{v1p120}ont édicté le loisir obligatoire, le repos dominical, ont montré plus de vraie intelligence de la vie sociale que les maîtres de l’économie politique. Le repos dominical est la forme la plus sociale du loisir, car il est le loisir simultané pour tout le monde, le loisir périodique et réglé, regardé comme un devoir des plus sacrés et non comme un simple plaisir.\par
Dans l’emploi de ses loisirs, comme dans l’exercice de son travail, l’homme est imitatif : la flore spontanée du sol obéit aux mêmes lois botaniques que la flore cultivée. Mais, dans le choix et la combinaison des modèles qui remplissent ses loisirs, l’individu exprime bien mieux son originalité intime que dans son obéissance aux coutumes, aux règlements, aux modes, qui lui imposent son genre de travail. L’accroissement progressif de la part des loisirs, dans la distribution des heures de la journée, marque donc et mesure le progrès de l’imitation libre et originale sur l’imitation plate et contrainte.\par
La question de savoir quelle doit être la proportion du loisir et de quelle manière doit s’opérer la répartition du loisir entre les hommes est donc une des premières que l’économie politique devrait traiter ; elle n’a pas moins d’importance que ce qui touche à la répartition du travail et de la richesse. — Partout et toujours, la quantité de travail dont les hommes réunis en société disposent, je ne dis pas celle qu’ils dépensent effectivement, qu’il s’agisse d’une peuplade sauvage ou d’une grande nation moderne, a été plus que suffisante pour produire la quantité de denrées, d’articles ou de services de tout genre nécessaires à la satisfaction des désirs préexistants. Seulement, à mesure que cet excès de forces humaines sans emploi utile apparaît, de nouveaux désirs, de nouveaux besoins surgissent qui le diminuent en suscitant de nouvelles branches de travail. Ainsi, deux progrès parallèles et contraires marchent de front : d’une part, le perfectionnement incessant du travail qui, à durée égale, devient plus productif et tend sans cesse à accroître l’excédent de  \phantomsection
\label{v1p121}forces humaines disponibles ; d’autre part, la complication croissante des besoins qui tend à annihiler ce bénéfice du travail dès qu’il se montre, à faucher cette moisson en herbe. Antinomie prolongée, et qui serait désespérante si elle devait durer toujours. C’est un premier problème de savoir comment cette contradiction doit se résoudre. Et, si on l’examine sans parti pris, on sera porté à juger avec infiniment moins d’admiration la fébrile agitation, l’affairement haletant de certains peuples, destinés sans doute à se calmer à leur tour après cette surexcitation de leur premier âge.\par
C’est un autre problème, non moins grave, de savoir comment se répartira la somme de loisir quelconque existant à un moment donné, et quel est le mode de répartition préférable. De nos jours, la question est d’autant plus importante que, en dépit d’une multiplication et d’une propagation extraordinaire des besoins, la productivité du travail humain a progressé plus vite encore, grâce au merveilleux concours des machines et aux miracles de l’association. On doit se demander si le meilleur parti social à tirer de là est de concentrer sur certaines têtes ou de disséminer sur toutes le loisir ainsi produit. Les deux solutions différentes nous sont offertes par l’histoire de nos sociétés européennes. La solution aristocratique, qui a régné longtemps, a consisté, sinon à dispenser de tout labeur, il s’en faut, du moins à faire profiter exclusivement du loisir disponible, les seules classes supérieures. Dans ce système, l’accroissement de l’excès des forces humaines par l’augmentation de l’effet utile du travail n’avait d’autre résultat que le nombre plus grand des \emph{gens de loisir.} Presque tout le loisir d’un côté, presque tout le travail de l’autre : tel était le régime admis et supporté. S’il avait continué, — et rien ne prouve qu’il n’eût pu se continuer indéfiniment, au cas où le sentiment de \emph{l’intérêt bien entendu} eût dominé sans réserve dans le cœur des gouvernants — le progrès de la civilisation, très réel malgré tout, eût été de rendre de moins en moins  \phantomsection
\label{v1p122}nombreuse la fraction laborieuse des sociétés, et de plus en plus nombreuse la fraction oisive : comme dans l’antiquité.\par
Mais notre siècle a fait triompher la seconde solution, plus conforme aux idées de justice et de fraternité. Au lieu d’être monopolisé par quelques-uns, le loisir se divise et se subdivise entre tous ainsi que le travail ; et, tout le monde travaillant plus ou moins, le progrès s’exprime par la diminution des heures de travail pour tous. — Cette manière de résoudre le problème posé est définitive, nous l’espérons bien. Il faudrait se garder cependant de penser que ses avantages incontestables sont tout à fait sans compensation. Et peut-être, si elle s’imposait entièrement, si elle était poussée à bout, ferait-elle autant de mal que de bien. Il sera utile longtemps encore, peut-être toujours, qu’il y ait çà et là des individus jouissant de leurs pleins loisirs, condition \emph{sine quâ non} de certaines découvertes scientifiques, de certaines beautés poétiques.\par
Mais il reste une troisième solution, qui, si elle ne s’est jamais réalisée, heureusement, aurait dû être formulée par les économistes comme corollaire à leurs idées. S’il fallait croire tout ce qu’ils ont dit sur les vertus du travail et les vices de l’oisiveté, ce qu’il y aurait de mieux à faire serait de supprimer entièrement les heures de loisir et \emph{les gens de loisir.} L’expression de \emph{surpopulation} que je rencontre sous certaines plumes semble suggérée vaguement par cette idée que tout ce qui ne travaille pas sans cesse et n’est pas nécessaire pour le travail à exécuter dans un pays n’est bon à rien et doit disparaître. Mais il y a de louables inconséquences, et il faut louer les apologistes du labeur humain de s’être arrêtés à mi-chemin de leurs déductions.\par
L’abréviation de la durée du travail professionnel produit dans la vie d’un homme le même effet que produit sur un sol jusque-là cultivé le resserrement de la culture. Le loisir accrû, comme la friche plus étendue, se remplit bientôt d’une  \phantomsection
\label{v1p123}végétation libre et folle ; et, par l’emploi des loisirs se révèle le fond de l’âme de l’ouvrier, comme, par la nature de la flore spontanée se décèle la nature chimique du sol. Et, de même que c’est toujours la flore sauvage qui fournit la matière première des plantes cultivées, d’utilité ou d’agrément, ainsi est-ce d’ordinaire la libre fantaisie de l’esprit désoccupé qui fournit le premier germe des idées scientifiques, industrielles, esthétiques, par lesquelles se renouvellent les sciences, les industries, les arts. L’économiste, donc, encore une fois, doit s’occuper autant des problèmes qui se rattachent au loisir que de ceux que soulève le travail. La part du loisir, dans la vie, c’est la part du cœur, de l’imagination, de la famille, de la sociabilité à la fois et de l’individualité originale sous leurs formes les meilleures.\par
Comment remplir les heures de loisir ? Ce nouveau problème devient d’autant plus intéressant pour l’économiste que les loisirs se généralisent et se prolongent. Les solutions qu’il comporte sont innombrables, mais il en est deux qui me paraissent se signaler par la gravité de leurs résultats : la conversation et la lecture. J’ai essayé de montrer ailleurs l’influence de la conversation dans la formation de l’opinion et des mœurs publiques, et, par suite, dans la fixation des valeurs et des prix. Le travail donc, forcé de s’adapter à des usages et à des besoins que la conversation, la communication verbale des esprits, et la lecture des livres ou des journaux, modifient sans cesse, est dirigé dans son cours par le loisir.\par
Là où la population se divise en deux classes, dont l’une, la plus nombreuse, travaille d’arrache-pied sans nul repos, et dont l’autre ne fait presque rien, sauf en temps de guerre, on peut dire, à peu de chose près, que la conversation et la lecture sont monopolisées par cette dernière classe. Par suite, ce qu’on entend alors par l’opinion publique, c’est purement et simplement l’opinion des gens de loisir. Il n’y en a pas d’autre qui compte politiquement, et même économiquement. \phantomsection
\label{v1p124} C’est donc dans l’enceinte étroite de ces gens de loisir lisant, écrivant ou s’écrivant, échangeant entre eux par de fréquents entretiens leurs désirs capricieux, que naissent les nouvelles modes en tout genre de consommation. Ils sont la source des courants de mode qui se répandent ensuite parfois dans un public plus étendu. C’est donc pour eux on par eux que la production industrielle alors se renouvelle et se complique incessamment. Mais, à mesure que la journée de travail s’abrège pour le paysan et pour l’ouvrier, de nouveaux besoins, nés de loisirs nouveaux, prennent naissance dans ces classes et ouvrent un débouché plus large à la production. Car, moins les hommes travaillent, plus ils ont besoin de consommer, si étrange que soit cette anomalie.\par
Il résulte de ce qui vient d’être dit que l’importance de la psychologie du travailleur s’accroît avec l’accroissement de ses loisirs. Autant dire qu’elle grandit avec la civilisation. Et c’est le moment de faire remarquer que chaque changement dans l’industrie, chaque nouveau mode de travail dû à une invention, opère une modification ou une révolution psychologique chez les travailleurs. L\emph{’état d’âme} du pasteur n’est pas celui du chasseur ni du laboureur. L’état d’âme de l’ouvrier américain qui surveille à la fois quatre métiers à tisser dans une usine n’est pas celui du tisserand attardé dans le fond d’un village, qui pousse sa navette en chantant, et pensant à ses récoltes. Or, chaque fois que l’industrie est ainsi révolutionnée par un groupe d’inventions nouvelles, les peuples chez lesquels se rencontrent à un degré éminent les aptitudes mentales réclamées par le nouveau mode de travail sont favorisés aux dépens des nations qui en sont moins pourvues et qui avaient triomphé jusque-là parce qu’elles étaient psychologiquement adaptées à l’ancienne manière de travailler. Le grand succès des « Anglo-Saxons » durant notre siècle s’explique ainsi. L’immense emploi des machines, qui caractérise l’industrie contemporaine, \phantomsection
\label{v1p125} a eu pour conséquence, en effet, de reléguer au second rang, de déprécier considérablement un ensemble de qualités fines qui étaient auparavant cotées en première ligne chez l’ouvrier et que l’ouvrier italien ou français présentait au plus haut degré : l’adresse manuelle, le cachet artistique, choses liées à la souplesse d’esprit, à l’imagination \emph{débrouillarde} et ingénieuse, à la fantaisie un peu vagabonde. Mais, en revanche, la direction des machines réclame à tout prix une qualité bien simple et bien modeste, qui a dès lors acquis une valeur infinie, et qui distingue entre tous l’ouvrier anglais, l’ouvrier américain, l’ouvrier allemand : je veux dire la force et la ténacité d’attention, d’attention fixe ou tournant dans un cercle fixe, sans la moindre distraction. Chez le conducteur de locomotives, d’automobiles surtout, comme chez le bicycliste, il n’est pas de pire défaut qu’une pensée imaginative et distrayante ; ils doivent être beaucoup plus attentifs que le charretier, le cocher, le cavalier même. Il faut plus de force musculaire pour battre le blé au fléau ; mais, pour diriger une batteuse mécanique, il faut ne pas être un seul instant distrait. Pour être un bon menuisier, même en exécutant l’ouvrage le plus simple, il faut de l’adresse et du goût ; pour surveiller une scierie mécanique, il ne faut que de l’attention persévérante.\par
Il ne convient donc pas que les peuples qui l’emportent sur leurs rivaux, à n’importe quelle époque, dans le grand concours industriel, soient trop fiers de leur prospérité. Elle tient le plus souvent à des mérites très humbles auxquels des inventions de génie, — apparues parfois parmi leurs concurrents qu’elles ruinent — ont prêté une valeur de circonstance.\par
Je ferai remarquer, en passant, que la fixité et l’intensité de l’attention, dès qu’elles dépassent le degré moyen et normal, qui est bien au-dessous du degré exigé par la direction et la surveillance des machines modernes, sont une grande cause d’épuisement nerveux, et pourraient bien être  \phantomsection
\label{v1p126}pour quelque chose dans le progrès de la folie et du déséquilibre mental à notre époque. La trop grande stabilité de l’attention doit produire, par une réaction inévitable, l’instabilité de l’attention, qui est la caractéristique des désordres nerveux.\par
Heureusement, si l’obsession mentale, imposée à l’ouvrier contemporain par son mode de travail, est devenue plus intense, elle s’est abrégée. Ici se montre l’utilité des considérations de tout à l’heure sur le loisir. C’est grâce à cet accroissement de loisir qu’un véritable progrès psychologique est, en somme, et malgré tout, lié au progrès économique de notre temps. La substitution de la grande à la petite industrie a pu causer, momentanément, des souffrances à la classe ouvrière, par la disproportion plus grande ou mieux sentie entre ses aspirations et ses ressources ; mais, à coup sûr, elle a eu pour effet d’élargir les idées de l’ouvrier, de l’initier à une vie sociale plus complexe et plus haute, à des généralités de vues et à des générosités de sentiments qui lui étaient auparavant inconnues, à ce degré du moins.\par
L’\emph{esprit d’entreprise} et l’\emph{ardeur au travail} sont les deux conditions psychologiques fondamentales de la prospérité économique d’un peuple. Or, l’esprit d’entreprise est surtout excité ou paralysé par des causes dont les économistes ne tiennent nul compte, telles qu’une série de victoires ou de défaites militaires d’où résulte, par une contagion d’esprit à esprit, de cœur à cœur, une exaltation ou une dépression de l’orgueil collectif. Au fond de toute fièvre productrice, aux États-Unis, en Angleterre, il y a un orgueil national intense. — L’ardeur au travail a des causes plus humbles, mais non moins importantes à signaler : la bonne santé d’abord, le souci de l’avenir, de la famille à élever, etc. Il y entre aussi beaucoup de contagion d’homme à homme.\par
Ajoutons que, si la psychologie du producteur ouvrier ou patron importe toujours davantage, il en est de même de  \phantomsection
\label{v1p127}la psychologie du consommateur. C’est sur ce qui se passe en lui d’intime et de secret, c’est sur ses idées et ses caprices, sur les désirs qui commencent à naître en lui à l’exemple contagieux d’une coterie, à la suite d’opinions adoptées elles-mêmes par mode, — c’est sur ces délicates opérations intérieures, que l’œil du producteur est et doit être sans cesse attaché. Qu’il en ait conscience ou non, le marchand avisé, l’industriel habile, est constamment préoccupé d’inter-psychologie en songeant à ses clients. Par suite, la même préoccupation, sous forme consciente, devrait dominer aussi les spéculations de l’économiste. N’y a-t-il pas des lois qui règlent les courants de mode, en apparence capricieux, la formation des coutumes, la généralisation ou la consolidation des usages, des caprices exceptionnels devenus besoins publics ? Cette question devrait l’intéresser aussi vivement que les lois des courants atmosphériques intéressent le navigateur à voile.\par
L’erreur de croire qu’il était d’une bonne méthode scientifique de considérer la richesse comme l’objet propre de l’économie politique, la richesse abstraite, à part de celui qui la produit ou de celui qui la consomme, a conduit logiquement à regarder le \emph{maximum} et non \emph{l’optimum} de richesses comme le but économique par excellence, et à préconiser non la consommation, mais bien plutôt l’\emph{épargne} de la richesse. La réhabilitation de l’avarice est un des paradoxes auxquels les économistes ont été trop souvent conduits par leur point de vue. Mais ils n’ont pu vaincre la répulsion instinctive et invincible que l’avare, pour des raisons toutes psychologiques et morales, inspire au genre humain, sauvage, barbare ou civilisé. L’épargne, ce mot abstrait à l’usage de nos théoriciens, se réalise, dans la vie pratique, sous bien des formes, les unes louables, les autres blâmables. Il y a la sobriété, épargne d’aliments ; il y a aussi la chasteté, épargne de forces, qui se capitalise parfois en énergie créatrice. Mais il y a aussi le malthusianisme,  \phantomsection
\label{v1p128}épargne de naissances, qui se capitalise en confort croissant chez les époux malthusiens. On peut faire épargne de ses connaissances comme de ses richesses, de même qu’on peut prodiguer les unes et les autres. En économisant ses idées, en les retenant au lieu de les répandre en hâtives improvisations, on les capitalise en systèmes silencieusement élaborés. Les jansénistes, en proscrivant la « fréquente communion », faisaient en quelque sorte épargne des sacrements dont les jésuites étaient prodigues. Ce sont là mille formes différentes de l’épargne des richesses, dans le sens le plus large du mot. Dira-t-on que toutes sont dignes d’éloges ? Et comment distinguera-t-on entre elles, si l’on s’en tient à l’étroite préoccupation de l’économiste ancien ?\par
Tout le problème socialiste, tout le problème social à vrai dire, consiste en problèmes psychologiques. La \emph{loi d’airain} de Ricardo qui a été longtemps (on y a renoncé) le grand cheval de bataille des socialistes, était fondée sur l’idée que les ouvriers bornaient leurs vœux de consommation, en travaillant, à se procurer des subsistances. De là, il suivait que, dès qu’ils avaient de quoi subsister, ils ne travaillaient plus. De là aussi des logiciens féroces — et fort accrédités au {\scshape xviii}\textsuperscript{e} siècle, Arthur Young, par exemple. — avaient déduit que l’élévation du prix des denrées alimentaires était un bien au point de vue de la richesse nationale, puisqu’elle forçait l’ouvrier à travailler davantage. D’autres allaient jusqu’à proposer des lois édictant, non pas comme nos lois ouvrières d’aujourd’hui, un maximum d’heures de travail, mais un minimum, et en même temps un maximum de salaires qui ne pouvait être dépassé\footnote{ \noindent Voir \emph{la Grande industrie}, par Schultze-Gevernitz.
 }.\par
On a fini par ouvrir les yeux à l’évidence et reconnaître qu’on avait pris un faux point de départ. Mais doit-on, pour se corriger, tomber dans l’erreur inverse, et se persuader que \emph{tous} les ouvriers naissent avec des aspirations supérieures à  \phantomsection
\label{v1p129}celle de « vivre et passer », comme disaient mélancoliquement de vieux paysans d’autrefois ? Un psychologue aurait vu qu’ici il convient de ne pas généraliser. Il y a des travailleurs, — c’est le plus grand nombre, je crois — qui, pour but de leur travail, n’ambitionnent d’eux-mêmes guère au delà, je ne dis pas de leur subsistance, mais de leur niveau actuel de vie. Il en est d’autres, et ce sont les plus actifs, les plus énergiques, les plus remuants, qui veulent à tout prix élever ce niveau, améliorer leur sort, se donner un certain luxe à l’exemple de leurs riches voisins, acquérir de l’influence... Or, il suffit qu’il y ait un ouvrier pareil sur 100, sur 1000, pour que ce peu de levain soulève bientôt cette pâte. Voilà la force d’entraînement d’âme à âme avec lequel il faut compter, avant tout, en économie politique. C’est la force qui, petit à petit, a remué tout le peuple des travailleurs et posé avec acuité la question sociale.\par
Si l’on veut voir à quels abus de sophistication, et à quel oubli des faits essentiels les plus profonds esprits peuvent être conduits par la prétention de fonder la science économique sur le côté objectif des choses, ou, pour mieux dire, sur des entités substituées aux réalités vraies, il n’y a qu’à lire tout ce que les économistes ont écrit à propos de la rente foncière. On a entassé à ce sujet subtilités sur subtilités. Après les raffinements d’analyse de Ricardo, Karl Marx est venu, qui a soumis à son alambic les résidus de la distillation de son prédécesseur\footnote{ \noindent Voir sur ce point, notamment, la \emph{Revue d’économie politique}, mars 1899.
 }. Il a découvert une \emph{rente absolue}, distincte de la \emph{rente différentielle} mise en relief par Ricardo, et il a subdivisé celle-ci en divers genres, suivant ses transformations historiques ou ses diversités simultanées. La \emph{rente en travail} de l’esclave diffère de la \emph{rente en nature} du serf et celle-ci de la \emph{rente en argent} du fermier. En outre, dans ce qu’on appelle la rente de nos jours, il y a à distinguer ce qui est rente proprement dite et ce qui est profit des capitaux  \phantomsection
\label{v1p130}engagés dans l’agriculture. Il faut donc avoir égard à l’inégale productivité des capitaux employés sur des terres différentes, ou bien à l’inégale productivité — en général décroissante — de capitaux égaux employés successivement sur la même terre. La première considération a trait à l’agriculture extensive, la seconde à l’agriculture intensive. La rente, dans les deux cas, est le surplus de la productivité du capital employé sur les meilleures terres ou dans les meilleures conditions sur la même terre, par rapport à la productivité du capital employé sur les terres les moins bonnes ou dans de moins bonnes conditions, celles-ci étant les régulatrices du prix.\par
Et l’on peut subtiliser et raffiner encore. Et toutes ces distillations donneront des résultats. Tout cela, au fond, pour prouver quoi ? qu’il y a, en agriculture, des avantages de situation — ajoutons, à l’inverse, des désavantages — comme il y en a dans une industrie quelconque ; et que ces avantages, dont bénéficient les heureux possesseurs momentanés, se produisent sous des formes multiples. Mais cela est évident. Ce qui ne l’est pas, c’est que ce gain de certains agriculteurs (compensé d’ailleurs par les pertes de tant d’autres) soit plus illicite que les bénéfices des autres industriels, quand ils s’élèvent au-dessus de la moyenne. Le phénomène agricole de la rente, à cet égard, rentre, comme l’espèce dans le genre, dans le phénomène humain de la \emph{bonne chance}, qui fait que tel manufacturier s’enrichit là où d’autres se ruinent, que tel commerçant fait sa fortune là où d’autres courent à la faillite. Or, qui oserait proposer de bannir la \emph{bonne chance} du monde des affaires aussi bien que de tout autre monde ? C’est cependant ce qu’on propose, au fond, sans s’en apercevoir, quand, en protestant contre la rente, on suggère sa suppression, qui entraînerait logiquement la suppression aussi bien de tout bénéfice industriel ou commercial un peu exceptionnel, c’est-à-dire la mort de l’espérance humaine. Supprimer le \emph{risque}, dans une certaine mesure,  \phantomsection
\label{v1p131}par des lois sur les accidents du travail, par des assurances contre la maladie, par des pensions de retraites, à la bonne heure. Mais supprimer la \emph{chance}, c’est autre chose. Et, si par [{\corr impossible}], tout risque comme toute chance venait à disparaître de la vie, vaudrait-il encore la peine de vivre ? Aussi n’est-il pas question de la supprimer entièrement. Mais il s’agit de savoir — et c’est là, au fond, la question qui s’agite entre le socialisme et l’individualisme libéral — si le progrès social consiste à augmenter sans cesse la part de la sécurité en diminuant celle de l’espérance, ou à surexciter de plus en plus l’espérance en diminuant de plus en plus la sécurité. C’est en ces termes psychologiques que le problème social doit se poser.
\subsubsection[{A.4.b. Adam Smith psychologue. Son optimisme fondé sur son déisme. Successeurs d’Adam Smith.}]{A.4.b. Adam Smith psychologue. Son optimisme fondé sur son déisme. Successeurs d’Adam Smith.}
\noindent Je crois avoir suffisamment démontré, dans ce qui précède, l’importance majeure de la psychologie — surtout de celle que j’ai appelée l’inter-psychologie — en économie politique. On peut se demander comment une vérité si manifeste a pu être méconnue, et jusqu’à quel point elle l’a été par des esprits fins et pénétrants au plus haut degré. Un rapide coup d’œil historique sur les doctrines des maîtres ne sera pas inutile pour répondre à cette question.\par
C’est dans la société des physiocrates, pendant son séjour à Paris, que Adam Smith, on le sait, — ou plutôt on ne le sait pas assez — a puisé le germe de ses idées économiques. Or, que les physiocrates aient donné à la science qu’ils fondaient une couleur tout \emph{objective}, aussi peu subjective que possible, c’était d’accord avec leurs principes, avec la notion matérialiste qu’ils se faisaient de la richesse. Mais on a le droit d’être surpris qu’Adam Smith les ait suivis dans cette voie. Ce grand philosophe est, en effet, non seulement un économiste, mais un psychologue de premier ordre, et, par  \phantomsection
\label{v1p132}son traité sur la sympathie, il a tracé les premiers linéaments de l’inter-psychologie. On dirait qu’une cloison presque étanche sépare en lui ses deux ordres de recherches.\par
Son traité sur les \emph{sentiments moraux} est une mine d’observations fécondes. « Nous cherchons à examiner notre conduite comme nous pensons qu’un spectateur impartial et juste pourrait l’examiner. Nous nous supposons spectateurs de nos propres actions et nous recherchons quels effets elles produiraient sur nous, envisagées de ce point de vue. » Que cette explication de l’origine des jugements portés sur le bien et le mal soit suffisante, on l’a contesté avec raison. Mais ce qu’il y a de profondément vrai dans l’idée de Smith, c’est que la vie sociale consiste à être toujours \emph{tous en spectacle à tous}, même dans la solitude la plus complète, et qu’il est essentiel à chacun de nous, en ayant conscience de soi, d’avoir conscience d’autrui, de se sentir surveillé, épié, jugé par les yeux environnants de ses semblables.\par
Je glane en passant cette fine remarque : « Nous pouvons avoir lu un poème assez souvent pour y trouver peu d’intérêt, et prendre cependant beaucoup de plaisir à le lire à un autre. S’il a pour cet autre le charme de la nouveauté, nous partageons la curiosité qu’il lui inspire, quoique nous n’en soyons plus capables nous-mêmes... » Quelquefois il lui arrive de juger un peu trop les hommes d’après sa propre nature, douce et bienveillante. Nous sympathisons, dit-il quelque part, avec un homme qui souffre ou qui se plaint, mais non avec un homme en colère. Hélas ! l’expérience montre qu’il est plus facile de suggérer à la foule un sentiment de haine et de colère qu’un élan de pitié. Les journalistes le savent bien.\par
Voici une remarque dont Smith économiste aurait bien pu faire son profit. « Il y a, dit-il, en Angleterre beaucoup de gens qui, comme particuliers, seraient plus troublés de la perte d’une guinée, qu’ils ne le seraient, comme Anglais, de la perte de Minorque, et qui, cependant, s’il eût été en  \phantomsection
\label{v1p133}leur pouvoir de défendre cette forteresse, auraient mille fois sacrifié leur vie plutôt que de la laisser tomber par leur faute au pouvoir de l’ennemi. » Cela veut dire que, individuellement, la conservation de Minorque ne valait pas à cette époque une guinée pour les Anglais, tandis que, socialement et nationalement, elle valait pour eux plus que leur vie. Et la différence entre notre être purement individuel et notre être social, formé par l’honneur, cet écho intérieur de l’opinion traditionnelle, est bien mise en relief par cette observation. Ce qu’il y a de chatoyant dans l’idée de valeur s’y trouve indiqué.\par
Dans un passage de ses \emph{Sentiments moraux}, par hasard, Smith psychologue se souvient qu’il est économiste. Après avoir montré que nous sommes dupes de notre amour de la richesse et de la puissance, car la vue de ce qu’il y a objectivement de grand, de beau, d’harmonieux dans les manifestations extérieures de la fortune et du pouvoir, dans les palais, dans les domaines bien aménagés, dans un État bien centralisé, nous empêche de songer à l’inanité et à l’incohérence de tout cela considéré du côté subjectif, au peu de bonheur qui en résulte, tempéré de tant de tourments ; après avoir détaillé finement cette pensée, il ajoute : « Il est heureux que la nature même nous en impose à cet égard ; l’illusion qu’elle nous donne excite l’industrieuse activité des hommes et les tient dans un mouvement continuel. C’est cette illusion qui leur fait cultiver la terre de tant de manières diverses, bâtir des maisons au lieu de cabanes, fonder des villes immenses, inventer et perfectionner les sciences et les arts. C’est cette illusion qui a changé la face du globe... » Et le traducteur de Smith, Baudrillart, observe en note : « Il est curieux d’entendre Smith fonder sur une illusion ce qui fait le but de l’industrie et la matière de l’économie politique. » Voilà des \emph{curiosités}, malheureusement, que ne nous offrent plus les successeurs économistes de Smith. Mais ce qui me semble surtout \emph{curieux}, c’est de voir un homme  \phantomsection
\label{v1p134}qui vient de reconnaître ici, avec tant de netteté, la supériorité du côté subjectif sur le côté objectif des phénomènes économiques, ou du moins la fécondité du premier, seul explicatif du second, négliger presque complètement le premier, de parti pris, dans son \emph{Traité de la richesse des nations.}\par
Ce n’est pas que, dans ce dernier ouvrage, il oublie tout à fait le fin psychologue qu’il est. Le rôle des sentiments tient quelque place dans ses spéculations. Par exemple, dans sa prédilection pour l’agriculture, on sent bien que l’\emph{état d’âme} de l’agriculteur est surtout ce qu’il préfère. La psychologie du paysan l’intéresse. Dans la comparaison qu’il établit entre les habitants des villes et les habitants des campagnes, et où il met en relief la multitude, la complexité des connaissances que suppose l’art agricole et que possède le moindre cultivateur, il remarque les qualités de \emph{jugement} et de \emph{prudence} qui distinguent celui-ci, et il ajoute : « A la vérité, il est moins accoutumé que l’artisan au commerce de la société... mais son intelligence, habituée à s’exercer sur une plus grande variété d’objets, est en général bien supérieure à celle de l’autre, dont toute l’attention est ordinairement, du matin au soir, bornée à exécuter une ou deux opérations très simples. Tout homme qui, par relation d’affaires ou par curiosité, a un peu vécu avec les dernières classes du peuple de la campagne et de la ville, connaît très bien la supériorité des uns sur les autres\footnote{ \noindent Il est probable que de nos jours, Smith n’eût point noté cette remarque. Peut-être même, entraîné par le courant général, eût-il écrit précisément l’inverse. En tout cas, ce passage est un document propre à révéler que la psychologie de l’ouvrier, depuis Smith, s’est notablement élevée et enrichie.
 }. »\par
Ici et ailleurs, il y a quelque chose d’assez \emph{subjectif} dans les appréciations de Smith, et qui rappelle un peu Sismondi par le ton, non par le fond des idées.\par
Dans son chapitre sur la distinction du travail productif et du travail improductif (qu’il énonce mal, encore plus qu’il ne la comprend mal), il a beau, par la mauvaise formule \phantomsection
\label{v1p135} qu’il donne à sa distinction, révéler l’étroitesse et l’insuffisance du point de vue auquel il prétend réduire l’économiste, et méconnaître l’importance des consommations, dites improductives, qui consistent en jouissances et satisfactions toutes personnelles, en acquisitions intérieures d’idées, de sentiments, de richesses subjectives ; malgré tout, il ne peut s’empêcher de reconnaître ce qu’il y a « de noblesse et de générosité » dans beaucoup de prodigalités, de dépenses purement somptuaires. « Quand un homme riche, dit-il, dépense principalement son revenu à tenir grande table, il se trouve qu’il partage la plus grande partie de son revenu avec ses amis et les personnes de sa société ; mais, quand il l’emploie à acheter de ces choses durables dont nous vous avons parlé, il la dépense alors souvent en entier pour sa propre personne et ne donne rien à qui que ce soit sans recevoir l’équivalent. Par conséquent, cette dernière manière de dépenser, quand elle porte sur des objets de frivolité, est souvent une indication de mesquinerie dans le caractère, et d’égoïsme. »\par
Ce qui est surprenant, malgré tout, c’est le faible rôle que joue la psychologie en ces écrits économiques de Smith, et c’est l’absence complète de la \emph{psychologie collective.} C’est lui, cependant, Smith, qui a le premier étudié la \emph{sympathie}, source et fondement de la psychologie inter-mentale. Comment se fait-il qu’il n’ait jamais senti la nécessité ni l’opportunité de faire usage des fines remarques qu’il a faites sur la mutuelle stimulation des sensibilités les unes par les autres, pour expliquer les rapports économiques des hommes ? Comment se fait-il que, ayant consacré à la \emph{mode} et à la \emph{coutume} un petit chapitre, à propos de leur influence sur la formation des sentiments moraux, il n’ait pas eu l’idée de rechercher leur influence sur la formation des désirs et des besoins, des croyances et des espérances, condition de toute production et de conservation des richesses ?\par
On ne peut s’expliquer cela qu’en songeant à la force des  \phantomsection
\label{v1p136}premières impulsions. C’est dans la société des physiocrates que Smith avait fait son éducation économique, il a conservé toujours le \emph{pli} de leur esprit. Mais eux-mêmes, pourquoi ont-ils envisagé l’objet de la science par le côté le plus matériel ? Sismondi va répondre :\par
« Ce fut, dit-il, de la science des finances que naquit celle de l’économie politique, par un ordre inverse de celui de la marche naturelle des idées. Les philosophes voulaient garantir le peuple des spoliations du pouvoir absolu ; ils sentirent que, pour se faire écouter, il fallait parler aux princes de leur \emph{intérêt} et non de la justice et du devoir ; il cherchèrent à leur faire bien voir quelles étaient la nature et les causes de la richesse des nations, pour leur enseigner à la partager sans la détruire. » Voilà une des raisons pour lesquelles l’économie politique, dès ses débuts, a pris une couleur si positive, et a fait, de parti pris, abstraction de toute considération d’ordre psychologique et moral.\par
Mais si, en économie politique, Adam Smith a cru pouvoir se passer presque entièrement de considérations inter-psychologiques, c’est aussi que ses idées théologiques lui en tenaient lieu. C’est un point qu’il est bon de mettre en lumière.\par
Le théisme de Smith le portait à justifier toutes les passions comme des œuvres divines et à y voir des intentions providentielles, des ruses délicates d’un art caché. A plusieurs reprises, dans sa \emph{théorie des sentiments moraux}, il montre ou croit montrer l’utilité sociale des sentiments bas ou extravagants. Par exemple, après avoir indiqué ce qu’il y a d’illogique et d’absurde dans notre complaisance pour le succès (« Si César eût perdu la bataille de Pharsale, on placerait à présent son caractère un peu au-dessous de celui de Catilina »), il ajoute : « Ce désordre dans nos sentiments moraux n’est cependant pas sans utilité, \emph{et nous pouvons encore ici admirer la sagesse de Dieu dans la faiblesse et dans la folie de l’homme.} Notre admiration pour le succès a le même principe que notre respect pour les richesses et  \phantomsection
\label{v1p137}pour les grandeurs, et elle est également nécessaire pour établir la distinction des rangs et l’ordre de la société. »\par
On comprend qu’un homme si disposé à voir un artiste divin derrière la toile des événements humains et une sagesse divine derrière toute folie humaine, ne devait pas avoir la moindre peine à regarder l’égoïsme lui-même, l’amour de soi, comme investi d’une fonction sacrée, éminemment propre à tisser et consolider l’harmonie sociale. Aussi, quand il fondait toute l’économie politique sur ce principe et qu’il réduisait l’\emph{homo œconomicus} à l’intérêt bien entendu, abstraction faite de toute affection et de toute abnégation, ce n’était point chez lui l’effet d’une conception épicurienne et matérialiste, c’était au contraire une suite naturelle de sa piété et de sa foi en Dieu. Derrière l’homme égoïste il y avait le Dieu bienfaisant, et l’apologie de l’égoïsme du premier n’était, à vrai dire, qu’un hymne en prose à la bonté infinie du second.\par
Mais les successeurs de Smith, dans notre siècle, sont des athées. J’en excepte quelques-uns, Bastiat surtout, dont les \emph{Harmonies économiques} sont fondées sur la même conception de la Providence. Ou du moins, s’ils croient en Dieu, leurs spéculations ne portent nulle trace de cette croyance. C’est pourquoi, en continuant à fonder l’économie politique sur le postulat du pur égoïsme humain et de la lutte des intérêts, après avoir banni l’idée de la Providence, ils ont, sans s’en apercevoir, supprimé la clef de voûte du système, qui a perdu toute sa solidité apparente d’autrefois. Ils ont, si l’on aime mieux, supprimé le ciel de ce paysage, devenu incompréhensible, ou éteint la lumière de cette lanterne, qui n’éclaire et n’explique plus rien.\par
Il faut donc, le théorisme étant écarté des harmonies sociales — économiques ou autres — écarter aussi l’égoïsme comme explication et agent de ces harmonies, et faire appel à d’autres principes, à d’autres mobiles, pour refondre en conséquence l’économie politique élargie.\par
 \phantomsection
\label{v1p138}Malgré tout, si l’on parcourt les œuvres de Smith et celles de ses contemporains du {\scshape xviii}\textsuperscript{e} siècle, il n’est pas difficile d’y glaner çà et là quelques observations psychologiques, mais la plupart bien superficielles... Il n’était pas possible qu’à une époque où tous les écrivains, depuis Rousseau, avaient toujours au bout de leur plume les mots de \emph{sensibilité} et de \emph{vertu}, les économistes eux-mêmes ne donnassent pas parfois une couleur sentimentale à leurs écrits. Par exemple, Melon, à propos de l’utilité du luxe des jardins, dit : « Pourquoi se récrier sur cette folle dépense ? Cet argent gardé dans un coffre serait mort pour la société. Le jardinier le reçoit, il l’a mérité par son travail, excité de nouveau ; ses enfants presque nus en sont habillés, ils mangent du pain abondamment, se portent mieux et \emph{travaillent avec une espérance gaie.} » Melon, précurseur de Fourier en cela, rêve déjà le travail attrayant — et surtout rendu tel par la collaboration des deux sexes. « Lorsque des hommes et des femmes travailleront ensemble à la construction d’un canal ou d’un grand chemin, le travail en sera plus animé et moins dur. » L’\emph{atelier mixte}, en somme, comme il y a ailleurs l’\emph{école mixte}, cela sent déjà le phalanstère.\par
— Mais, à partir du commencement de notre siècle, et chez les successeurs directs d’Adam Smith, l’économie politique a été pendant longtemps se dépouillant de plus en plus du peu de psychologie — sentimentale ou morale — qu’elle contenait en dissolution, pour revêtir un air plus froid, une physionomie plus géométrique. Cette sorte de cristallisation, prise à tort pour une épuration, est frappante, notamment, dans les écrits de J.-B. Say et de Garnier\footnote{ \noindent Noter la dureté de ces économistes, non par inhumanité naturelle, mais par logique de leur système. « A parler rigoureusement, dit J.-B. Say, la société ne doit aucun secours, aucun moyen de subsistance à ses membres. »
 }. Alors s’est installé dans sa chaire scolastique le dogmatisme de l’économie classique qui a pu un moment se croire indiscutable.\par
 \phantomsection
\label{v1p139}Toutefois, elle n’a pu jamais empêcher les murmures des dissidents, des hérésiarques, de s’élever autour de sa chaire ; et je n’ai pas à les énumérer tous. Un mot seulement.\par
Fourier est le premier qui ait fait une large application de la psychologie à la solution des problèmes économiques. C’est à l’étude du cœur, de ses éternels besoins, qu’il a demandé la réponse à ces graves questions. Le malheur est que sa psychologie, assez originale, était des plus incomplètes et retardataire. C’était celle, purement individuelle et voluptueuse, du dernier siècle, attardée dans la nôtre, au moment où Maine de Biran l’avait déjà refondue et renouvelée par sa théorie de l’effort volontaire. Du reste, Fourier est un utopiste avant tout, un rêveur des plus ingénieux et des plus féconds en beaux songes, tantôt puérils tantôt lucides, et non un économiste. Je ne parle de lui ici que pour mémoire.\par
En général, on peut, à certains égards, considérer les doctrines socialistes, éternelles ennemies mais ennemies sœurs des doctrines économiques, comme un effort, plus ou moins inconscient, mais sans cesse renouvelé, pour remédier aux lacunes et aux erreurs de l’économie purement objective en y réintégrant le côté subjectif qui a toujours été trop sacrifié par les maîtres de la science. Avant eux-mêmes, Sismondi, qui est le précurseur de leurs plaintes sinon de leurs idées, avait appelé l’attention sur les souffrances des ouvriers expulsés par les machines et, en philanthrope, il est vrai, plus qu’en philosophe, éloquemment insisté sur le contre-coup des changements du travail dans l’âme du travailleur. Parfois, il rappelle Ruskin par la manière dont il vante, avec amour, le charme propre aux industries demeurées primitives et patriarcales, au travail agricole. Mais le plus souvent il parle en moraliste, préoccupé, avant tout, du « bonheur des hommes ». L’économie politique, d’après lui, ne doit être que « la théorie de la bienfaisance ». Le moindre défaut  \phantomsection
\label{v1p140}de cette définition, — applicable aussi bien à la politique, à la religion, au droit, — est d’être bien vague.\par
Les écoles socialistes, aussi bien les écoles françaises de 1848 que les écoles allemandes de nos jours, ont dégelé et passionné l’économie politique ; et c’est en cela exclusivement qu’elles y ont introduit un élément psychologique nouveau, qui n’a rien changé d’ailleurs aux notions fondamentales. Seulement, la passion inspiratrice de ces doctrines a souvent varié ; et, dans la combinaison de générosité et de haine dont elle se compose, la proportion des deux s’est renversée ; plus généreuse que haineuse en France, elle est devenue plus haineuse que généreuse en Allemagne. Comparez Leroux ou Proudhon même à Karl Marx. Sous l’empire de ces sentiments intenses, les théories économiques se sont colorées et vivifiées, mais, au fond, elles ont gardé et même accentué la prétention ancienne à l’\emph{objectivité}, à la déduction géométrique de formules rigides, ayant un faux air de lois physiques. Il est rare de rencontrer au milieu de ces abstractions d’un caractère individualiste au plus haut degré, des passages tels que celui-ci de Karl Marx qui, cette fois, en passant, a fait de l’inter-psychologie et de la bonne : « De même, dit-il (p. 141) que la force d’attaque d’un escadron de cavalerie ou la force de résistance d’un régiment d’infanterie diffère essentiellement de la somme des forces individuelles déployées isolément par chacun des cavaliers ou fantassins, de même la somme des forces mécaniques d’ouvriers isolés diffère de la force mécanique qui se développe dès qu’ils fonctionnent conjointement et simultanément dans une même opération indivise... A part la nouvelle puissance qui résulte de la fusion de nombreuses forces en une force commune, \emph{le seul contact social produit une émulation et une excitation des esprits animaux qui élèvent la capacité individuelle d’exécution.} » Considération qui a son prix à côté de celle de la spécialisation du travail, dont on a souvent exagéré les effets. Ce  \phantomsection
\label{v1p141}n’est pas seulement parce que le travail y est très différencié que les grands ateliers sont de grands producteurs, c’est aussi parce que les travailleurs y travaillent ensemble.\par
En dehors des écoles socialistes, j’aurais à citer, parmi les écrits de Carey, de Stuart Mill, de Bastiat, de Courcelle-Seneuil et de bien d’autres, des aperçus intéressants au point de vue de la psychologie économique. Stuart Mill a mis en relief l’influence de la coutume dans la fixation des prix, et quelque part il proteste contre la poursuite des dollars donnée pour but unique à l’activité productive.\par
Je voudrais pouvoir citer Cournot au nombre des économistes psychologues, mais je dois reconnaître que son effort a visé au contraire à envisager les faits économiques sous leur aspect mathématique, poussé à bout depuis lors par Léon Walras. Je dis \emph{au contraire}, parce qu’il a cru à tort qu’il ne pouvait \emph{mathématiser} la science économique sans la dépouiller de tout élément subjectif. Mais, s’il avait pris la peine de considérer que, dans les phénomènes de conscience eux-mêmes, il y a un côté justiciable du nombre et de la mesure, que la croyance et le désir ont des degrés, une double échelle de degrés positifs et négatifs, affirmation et négation, désir et répulsion, amour et haine, peut-être aurait-il vu que le seul moyen de faire de la bonne statistique, c’est-à-dire de l’arithmétique sociale, c’est de faire porter les dénombrements du statisticien sur des faits extérieurs, soit, mais qui consistent au fond en croyances et en désirs, en idées et en besoins, en actes de foi et en actes de volonté, en jugements et en décisions. J’ai essayé de le montrer ailleurs\footnote{ \noindent \emph{Lois de l’imitation}, chap. intitulé « L’archéologie et la statistique ».
 }.\par
La tendance à \emph{mathématiser} la science économique et la tendance à la \emph{psychologiser}, loin d’être inconciliables, doivent donc plutôt se prêter à nos yeux un mutuel appui. Dans la statistique réformée et mieux comprise, dans la statistique \phantomsection
\label{v1p142} toule pénétrée d’un esprit inter-psychologique, j’aperçois la conciliation possible et même aisée de ces deux directions, en apparence divergentes.\par
Depuis une quinzaine d’années, ont surgi, en Allemagne et en Autriche, des écoles qui arborent le titre de psychologie économique : Schmoller, Wagner, Menger en sont les chefs. Je regrette que mon ignorance de l’allemand ne m’ait pas permis de suivre leurs savants travaux. Ce que j’en sais, toutefois, par des extraits ou des résumés, me donne à croire que leur manière de comprendre l’application de la psychologie à l’économie politique est loin d’être identique à la mienne. Ils ne tiennent pas compte de l’inter-psychologie, ce me semble, si l’on en excepte Schmoller. Ils font jouer au \emph{désir}, mais non à la \emph{croyance}, un rôle considérable. La nécessité de faire sa part, et sa large part, à la considération des croyances, des idées, des jugements, n’est reconnue explicitement à ma connaissance que par M. Gide qui, dans ses principes — si intéressants, si pénétrés de nouvelles tendances — m’attribue le mérite de l’avoir signalée. Je ne crois pas non plus que les auteurs indiqués aient eu égard au caractère \emph{quantitatif} du désir, et à son identité de nature d’un individu à un autre, qui seul permet de traiter les désirs des masses comme des quantités qui augmentent ou diminuent et que la statistique parvient à mesurer indirectement. D’après M. Bouglé, Wagner estime que la statistique ne saurait s’appliquer aux facteurs psychologiques « impondérables, spirituels\footnote{ \noindent Mais je connais trop peu ces économistes, pour parler plus longtemps de leurs idées. Celles que j’ai à exposer sont le développement de germes posés en substance, pour la première fois, dans la \emph{Revue philosophique}, en septembre et octobre 1881, — c’est-à-dire à une époque antérieure, je crois à l’apparition des écoles étrangères dont je viens de parler mais peu importe — sous le titre de \emph{la Psychologie en économie politique}. S’il y a des coïncidences entre les théories énoncées là et celles d’écoles autrichiennes ou allemandes, je me félicite d’autant plus de cet accord qu’il aura été plus spontané.
 } ».
 \phantomsection
\label{v1p143}\section[{I. La répétition économique}]{I. La répétition économique}\phantomsection
\label{l1}\renewcommand{\leftmark}{I. La répétition économique}

\subsection[{I.1. Division du sujet}]{I.1. Division du sujet}\phantomsection
\label{l1ch1}
\noindent Il s’agit maintenant d’embrasser le vaste champ de la psychologie économique au triple point de vue que nous connaissons, en rangeant sous trois chefs distincts, répétition, opposition, adaptation, toutes les parties de notre sujet. Commençons par la \emph{Répétition économique.}\par
Que faut-il entendre par là ? Est-ce seulement la reproduction des richesses ? Je le veux bien, mais à la condition de faire une analyse complète des causes de cette reproduction. Distinguer la \emph{terre}, le \emph{capital} et le \emph{travail}, ce n’est pas nous éclairer beaucoup. Si l’on va au fond de ces choses on trouve qu’elles se résolvent elles-mêmes en répétitions de diverses natures. La \emph{terre}, qu’est-ce, si ce n’est l’ensemble des forces physico-chimiques et vivantes qui agissent les unes sur les autres, les unes par les autres, et qui consistent, les unes, chaleur, lumière, électricité, combinaisons et substances chimiques, en répétitions rayonnantes de vibrations éthérées ou moléculaires, — les autres, plantes cultivées et animaux domestiques, en répétitions non moins rayonnantes et expansives de générations conformes au  \phantomsection
\label{v1p144}même type organique ou à une nouvelle race créée par l’art des jardiniers et des éleveurs ? — Le \emph{travail}, qu’est-ce, sinon un ensemble d’activités humaines condamnées à répéter indéfiniment une certaine série d’actes appris, enseignés par l’apprentissage, par l’exemple, dont la contagion tend sans cesse à rayonner aussi ? — Et le \emph{capital} lui-même, qu’est-ce, sinon, en ce qu’il a d’essentiel d’après moi, un certain groupe d’inventions données, mais considérées comme connues de leur exploiteur, c’est-à-dire comme s’étant transmises des inventeurs à lui par une répétition intellectuelle de plus en plus généralisée et vulgarisée ? Et si l’on veut que capital signifie, en outre, suivant les notions vulgaires qu’on s’en fait, une certaine partie de la richesse ancienne épargnée et mise à part, qu’est-ce encore, sinon de l’épargne répétée et accumulée ?\par
Mais cela ne suffit pas. La reproduction des richesses suppose, avant tout, la reproduction psychologique des désirs de consommation, et des croyances spéciales attachées à ces désirs, sans lesquels un article matériellement reproduit ne serait point une richesse.\par
En somme, nous voyons que, dans la reproduction des richesses ainsi analysée jusqu’en ses vraies causes profondes, les trois grandes formes de la répétition universelle, ondulation, génération, imitation, sont mises en jeu à la fois. Il importe donc de connaître leurs lois pour prévoir les résultats de leurs combinaisons variées dans le phénomène de l’activité industrielle. Mais ces trois catégories de phénomènes et de lois ont lieu de nous occuper très inégalement ; les deux premières seulement par rapport à la troisième.\par
L’erreur ici est de mêler ces diverses natures de répétitions sans s’en apercevoir et de mal choisir les \emph{unités répétées.} Dans l’école de Le Play, l’importance de la répétition a été comprise, puisque c’est sur elle qu’est fondée implicitement la méthode des monographies. Cette méthode consiste à  \phantomsection
\label{v1p145}penser qu’on se fait une idée complète d’une société en étudiant de près quelques-unes seulement de ses parties, mais de ses parties typiques, reproduites à multiples exemplaires, deux ou trois types de familles, par exemple, si bien que leur connaissance approfondie implique celle du tout. C’est très juste ; mais, si Le Play et ses élèves ont bien vu cela, ils se sont trompés en considérant la famille ou tout autre groupe social, tel que l’atelier même, comme ce qu’il y a de plus régulièrement répété en fait de choses sociales, et en ne descendant pas plus bas, dans le détail des faits, pour y chercher les unités élémentaires, dont ils ont aperçu les répétitions d’ailleurs mais sans leur prêter l’attention qu’elles méritent. Leurs monographies supposent le fonctionnement de beaucoup de répétitions, puisqu’elles partent de l’existence de beaucoup de choses semblables, mais, si elles constatent ces fonctions essentielles, elles ne les expliquent pas. Pourquoi cependant et comment ces similitudes se sont-elles formées ? Pourquoi, à telle époque — pas toujours — la moyenne, non la totalité, des familles de telle classe, dans telle région, a-t-elle quatre enfants au lieu de trois ou de deux ? Pourquoi la proportion des divers chapitres du budget y est-elle à peu près la même, c’est-à-dire pourquoi les besoins divers, de vêtement, de logement, d’ameublement, de divertissements, de livres, etc., y ont-ils atteint une même intensité proportionnelle ? Et pourquoi, d’une époque à l’autre, observe-t-on que la natalité de ces mêmes familles a augmenté, ou diminué ; que le chapitre de leurs budgets consacré à la toilette ou aux plaisirs a diminué, ou plus souvent augmenté ? Il ne faut pas demander à la méthode des monographies une réponse à ces questions : les monographistes constatent ces faits, ils ne les expliquent pas\footnote{ \noindent S’ils les expliquent, comme cela arrive si souvent à M. du Maroussem, qui a singulièrement perfectionné la méthode après M. Cheysson, c’est en greffant sur elle des notions et des préoccupations qui lui étaient primitivement étrangères.
 }, et ils les confondent fréquemment en les constatant.\par
 \phantomsection
\label{v1p146}Ce dernier reproche ne peut pas être adressé à la monographie d’atelier. La monographie de famille implique le fonctionnement combiné de l’hérédité et de l’imitation, des causes naturelles et des causes sociales qui seules peuvent expliquer la similitude des familles qu’on juge toutes sur un échantillon unique ; la monographie d’atelier ne postule que le fonctionnement de l’imitation. En séparant de la sorte deux éléments qui auparavant étaient présentés pêle-mêle, elle réalise un véritable progrès.\par
Mais, si importantes que soient les constatations dues à cette dernière espèce de monographies, elles ne sont pas non plus des explications, et c’est dans le menu détail de la vie d’atelier ou de famille, de la vie sociale en général, qu’il faut descendre pour découvrir les faits vraiment généraux, les répétitions d’actes et d’idées vraiment précises et prodigieusement multipliées où se laissent saisir les lois qui, une fois formulées, permettront de rendre compte des similitudes plus vagues entre des groupes sociaux, familles, ateliers, villes, nations. A travers la variabilité des faits économiques, qui se modifient d’âge en âge, de pays en pays, qui ne sont pas les mêmes à l’époque pastorale, à l’époque agricole, à l’époque industrielle, qui ne sont pas les mêmes en Europe et en Chine, en France même et en Angleterre, les lois de l’imitation ne changent pas, et c’est sur elles que reposent les principes économiques dans la mesure où ils se vérifient. Partout et toujours un nouvel outil, un nouveau procédé, un nouveau produit, jugé plus utile que les outils, les procédés, les produits anciens similaires, se répand ou tend à se répandre par multiplication rayonnante, et sa diffusion même, de plus en plus, lui est une garantie de majeure utilité. Partout et toujours ce jugement d’utilité supérieure, qui, entre cent ou mille exemples concurrents, anciens ou modernes, en fait choisir un de préférence,  \phantomsection
\label{v1p147}est prononcé à raison des idées qui sont déjà installées dans les esprits, des besoins qui sont déjà ancrés dans les mœurs, idées et besoins qui s’y sont formés jadis d’une manière analogue. Et, dans la fixation de ce choix, les considérations intrinsèques concourent avec les influences extrinsèques. Or, en ce qui concerne ces dernières, partout et toujours les exemples de capitale sont plus contagieux dans les provinces, ou les exemples des villes en général dans les campagnes\footnote{ \noindent On a remarqué, par exemple, en Italie (\emph{Rivista italiana di sociologia}, article de M. Coletti) que les émigrants d’origine urbaine ont précédé et entraîné les émigrants d’origine rurale. M. Coletti note que la tendance à émigrer se propage par une véritable \emph{psychose épidémique} dont l’\emph{Inchiesta agraria}, dit-il, décrit des « épisodes vraiment, caractéristiques ». — On a remarqué aussi que l’initiative de l’émigration est partie d’individus isolés et que les familles émigrantes ont suivi...
 }, ou ceux des classes élevées dans les classes supérieures, que \emph{vice versâ.} Partout et toujours, après s’être répandues ainsi, les innovations qui ont eu le plus de succès tendent à s’enraciner en coutumes, qui, à leur tour, seront remplacées plus tard ou fortifiées par des modes nouvelles\footnote{ \noindent Une innovation utile se produit, et au début, c’est à raison de son utilité qu’elle est adoptée par les premiers qui l’adoptent. Mais, de proche en proche, elle se répand, et, de moins en moins, c’est à cause de son utilité ; de plus en plus, c’est par esprit d’imitation. Et il est indéniable que la plupart de ceux qui font bon accueil à cette nouveauté l’auraient repoussée, s’ils avaient ignoré que d’autres l’ont accueillie, et alors même qu’ils auraient parfaitement compris les avantages de son adoption. La force de la routine les eût retenus à la pratique ancienne. Ici donc, comme partout en fait d’imitation-mode, le pouvoir propre de l’imitation consiste à briser l’obstacle de la coutume, Et, dans le cas de l’imitation-coutume, le pouvoir propre de l’imitation consiste à l’emporter sur le pouvoir de la raison, toutes les fois qu’on se soumet à la tradition tout en reconnaissant qu’elle est irrationnelle.\par
 Ainsi, dans l’un et l’autre cas, il y a une force propre inhérente à l’imitation, un certain \emph{désir d’imiter} soit les contemporains, soit les anciens, qui tantôt lutte, tantôt concourt avec la raison et l’intérêt pour régler notre conduite.
 }.\par
A la loi de l’imitation du supérieur par l’inférieur, — du supérieur jugé tel par l’inférieur se jugeant tel, à tort ou à raison, sciemment ou à son insu, — se rattache un fait d’une immense importance économique, le commerce international. Partout et toujours, — même aux âges préhistoriques,  \phantomsection
\label{v1p148}soyons-en persuadés — il y a, parmi les tribus, parmi les cités, parmi les nations voisines, une tribu, une cité, une nation admirée et enviée, qui donne le ton autour d’elle. Le penchant à la copier en tout est si fort que, en dépit des obstacles de la coutume, et, le plus souvent, de la loi, il se fait jour invinciblement par les échanges de peuple à peuple, de peuplade à peuplade\footnote{ \noindent On a retrouvé des traces manifestes du rayonnement de l’antique civilisation pharaonique jusque chez les nègres du Soudan (V. \emph{Tombouctou} du commandant Dubois) et même du Congo. — Partout où l’on remarque dans les classes inférieures des pays arriérés, des coiffures traditionnelles, des costumes locaux, qu’on a l’illusion de croire autochtones, cherchez et vous trouverez que telle coiffure procède d’une mode usitée à telle cour royale il y a quelques siècles, ou que telle coupe de vêtement a une origine non moins princière.
 }. Le législateur a beau les entraver autant qu’il le peut, par des barrières de douane, par des monnaies nationales et locales soigneusement conservées, par des systèmes de poids et mesures multiformes, de telles digues ne servent qu’à révéler la force du courant qu’elles contrarient, c’est-à-dire la tendance des hommes, en tout temps et en tout pays, à prendre modèle sur l’étranger, ou plutôt sur \emph{un} étranger, et à commercer avec lui. Cet engouement pour les produits exotiques, ou plutôt pour l’exotique d’un certain genre, sévit parmi les sociétés les plus barbares ; et il aide à comprendre un fait d’une grande portée économique, à savoir pourquoi les consommations nouvelles, importées du dehors, se propagent plus vite dans un pays que les productions correspondantes. Si, à mesure que la mode des produits étrangers se répand ainsi, les industriels indigènes ne se mettent pas aussitôt à les fabriquer, ce n’est pas toujours faute d’habileté, c’est parce qu’ils savent bien que le caractère exotique de l’article est ce qui le fait rechercher. Plus tard, quand la saveur de l’exotique commence à s’épuiser, il sera temps de faire des contrefaçons nationales de l’article étranger.\par
Mais c’est surtout dans les relations naturelles des concitoyens que la force de l’imitation agit, et, si l’on n’a pas égard  \phantomsection
\label{v1p149}à ce facteur de premier ordre, on ne s’explique pas les faits les plus manifestes, et les plus fondamentaux, tels que l’élévation graduelle du \emph{train de vie} des diverses classes sociales. A la fameuse \emph{loi d’airain}, M. Paul Leroy-Beaulieu oppose, entre autres arguments, celui-ci, que l’étalon de vie de l’ouvrier, l’ensemble de ses besoins jugés impérieux, change constamment et, en fait, constamment s’élève au cours de la civilisation. « Or, dit-il, comment ce niveau de l’existence populaire eût-il pu monter si l’élévation des salaires ne l’eût précédé ? Dans la doctrine du \emph{salaire naturel} et de la \emph{loi d’airain} cette hausse du niveau de la vie est incompréhensible. Il est évident que ce \emph{sont les ressources de l’ouvrier qui déterminent son mode de vivre et non son mode de vivre qui détermine ses ressources.} » La remarque serait parfaitement juste et l’auteur aurait tout à fait raison s’il n’avait l’air de croire ici que le salaire a crû spontanément et que c’est cette hausse spontanée qui explique la hausse de l’étalon de vie. En fait, voici comment les choses se passent. Un ouvrier entre mille gagne un peu plus que les autres, grâce à des circonstances favorables ou à son habileté supérieure, et aussitôt il se paie certains plaisirs, tels que le café, le cigare, etc. En ce qui concerne cet initiateur, l’explication de notre auteur s’applique. Mais les 999 autres ouvriers \emph{veulent}, malgré l’insuffisance de leurs salaires, vivre comme lui, se modeler sur lui, et c’est cette volonté décidée, générale, — devenue générale par imitation égalitaire — qui finit par vaincre les résistances du patron. Bien des grèves, bien des syndicats, sont dus à une action de ce genre. Mais, même sans grève, même sans syndicat, le mécontentement général des ouvriers, par suite des besoins nouveaux qui germent en eux, ensemencés par l’un d’eux, est une force à la longue irrésistible contre laquelle lutte en vain l’intérêt de l’entrepreneur.\par
— Cela dit, il convient de diviser la répétition économique d’après les causes de la reproduction des richesses  \phantomsection
\label{v1p150}qu’elle étudie. Elle a donc trait d’abord : 1\textsuperscript{o} à la reproduction des désirs dont certaines richesses sont l’objet et des jugements portés sur l’aptitude de ces richesses à satisfaire ces désirs ; 2\textsuperscript{o} à la reproduction des travaux dont ces richesses sont le produit. Ce sont là les deux parties principales ; et nous les développerons : la première dans trois chapitres, sur le \emph{rôle économique du désir}, sur le \emph{rôle économique de la croyance}, sur les \emph{besoins}, la seconde dans un chapitre sur \emph{les travaux.} Mais, vu l’importance si grande acquise dans la vie civilisée par les signes monétaires de la richesse, il importe de traiter à part : 3\textsuperscript{o} tout ce qui touche à la transmission, à la circulation des monnaies métalliques ou fiduciaires. De là deux chapitres, l’un sur la \emph{monnaie}, l’autre sur le \emph{capital.}\par
Observation générale et préliminaire. Soit pour les besoins, soit pour les travaux, \emph{reproduction} signifie deux choses bien distinctes, et qui ne doivent jamais être confondues, malgré la liberté que nous prendrons souvent de les exposer pêle-mêle : d’une part, leur \emph{propagation} d’individu à individu, d’où résulte l’extension d’une industrie dans l’espace ; d’autre part, leur \emph{rotation} périodique, par cette imitation de soi-même qu’on nomme habitude chez les individus ou coutume chez les peuples. — En ce qui concerne la monnaie, cette distinction se reflète dans sa \emph{diffusion}, rayonnante dans un domaine de plus en plus vaste, et sa \emph{circulation} proprement dite, qui a lieu par sa rentrée aux mains d’où elle est sortie, rotation monétaire dont Karl Marx s’est fort occupé.
 \phantomsection
\label{v1p151}\subsection[{I.2. Rôle économique du désir}]{I.2. Rôle économique du désir}\phantomsection
\label{l1ch2}
\subsubsection[{I.2.a. Distinction nette des aspects différents sous lesquels la morale, la jurisprudence, la politique (branches diverses de la téléologie sociale) embrassent l’ensemble des désirs humains (et non chacun des désirs spéciaux).}]{I.2.a. Distinction nette des aspects différents sous lesquels la morale, la jurisprudence, la politique (branches diverses de la téléologie sociale) embrassent l’ensemble des désirs humains (et non chacun des désirs spéciaux).}
\noindent L’Économie politique n’est qu’une branche de la \emph{Téléologie sociale}, ou, si l’on aime mieux, de la \emph{Logique de l’action} appliquée aux sociétés. Cette science générale, qui, avec la \emph{Logique sociale} proprement dite, compose à peu près toute la sociologie (l’esthétique seule restant en dehors), étudie le rapport général des moyens sociaux aux fins sociales et, envisagée comme art, conseille les moyens les mieux adaptés aux fins les meilleures suivant les temps et les lieux. Elle doit commencer, en chacune des sciences particulières dont elle se compose, en Politique, en Droit, en Morale, en Économie politique, par définir les caractères des désirs humains dont elle s’occupe, et par étudier la genèse de ces désirs, les causes qui les font se répandre ou se resserrer, grandir ou décroître, les luttes qu’ils soutiennent entre eux, et le concours qu’ils se prêtent. Chacune de ces sciences embrasse bien l’ensemble des désirs humains sous les trois aspects que nous savons ; mais elles diffèrent par les proportions très inégales de chacun d’eux. La morale et la jurisprudence ont plus spécialement trait aux oppositions des désirs, soit, la morale, dans le cœur même de l’individu, soit, la jurisprudence, dans le groupe social, et s’efforcent de les faire coexister, soit au for intérieur, soit sous leurs manifestations extérieures, de manière à supprimer leur lutte, l’une en les circonscrivant sous le nom de \emph{droit}, cette limite des intérêts, l’autre en sacrifiant certains désirs à certains autres sous  \phantomsection
\label{v1p152}le nom de \emph{devoir.} La Politique et l’Économie politique traitent des mêmes désirs humains, mais à un point de vue moins passif, plus actif, au point de vue surtout de leurs adaptations possibles. Le souci de ces sciences n’est pas de régler une question de bornage des désirs, mais bien de les faire s’accorder, l’une, la Politique, en vue d’une action commune (défense du territoire, conquête extérieure, vote d’une loi sociale, etc.), l’autre, l’Économie politique, en une mutuelle assistance d’activités multiples.\par
Prenons pour exemple le désir d’être plus confortablement logé. Envisagé au point de vue moral, ce désir se présente en lutte avec le désir, chez le même individu, d’avoir une famille nombreuse, d’élever et d’instruire à grands frais les enfants, de doter les filles, etc. Au point de vue juridique, ce désir de logement plus confortable, chez le locataire, apparaît en conflit avec le désir du propriétaire de ne pas faire de réparations, avec le désir du tapissier et des autres fournisseurs de hausser le prix de leurs fournitures ; de là tout ce qui a trait au contrat de location, d’achat et de vente. Au point de vue politique, le même désir donne lieu à la perception d’un impôt sur les loyers, et par là, concourt, avec tous les autres désirs semblablement imposés, à accomplir toutes les entreprises administratives ou militaires que le budget de l’État sert à payer. Enfin, au point de vue économique, ce désir donne satisfaction au désir de construction qui dévore les architectes, au désir de faire de beaux meubles qui anime les ébénistes, etc., et, réciproquement, le désir des ébénistes ou des architectes satisfait celui du locataire...\par
On en dirait autant de tout autre désir. La distinction de la Politique et de l’Économie politique, ainsi comprise, est aussi nette que possible. L’une cherche la voie de la plus puissante \emph{collaboration} des désirs d’une nation ou d’un parti dans une même œuvre ; l’autre, celle de leur plus large et de leur plus \emph{réciproque utilisation ;} deux manières très  \phantomsection
\label{v1p153}différentes d’entendre leur adaptation. Et, si leur opposition aussi les inquiète, l’une et l’autre s’efforcent de faire servir cette rivalité même ou cette hostilité des désirs, par la concurrence industrielle ou l’antagonisme des partis, au progrès de leur harmonie.\par
Ces diverses sciences entre lesquelles se divise la Téléologie sociale présentent, on le voit, malgré leur précise distinction, des domaines assez mal délimités à leur frontière, analogues aux \emph{marches} des territoires barbares. Il ne faut pas perdre de vue qu’elles ont toutes pour fondement commun les désirs humains considérés en bloc. Nous avons donc maintenant à examiner ceux-ci dans leur ensemble, mais en nous attachant surtout aux côtés par lesquels ils peuvent s’entr’aider à satisfaire leurs buts différents, et réaliser ou tendre à réaliser un maximum de satisfactions pareilles.\par
A cet égard, il y a d’abord à distinguer le degré d’intensité des désirs, ainsi que leurs similitudes et leurs différences. Passé un certain degré d’intensité, en trop ou en trop peu, les désirs deviennent inutilisables les uns pour les autres. Un désir si faible qu’il est à peine ressenti n’excite personne au travail. D’autre part, des hommes mourant de faim ou de soif se jetteront sur la première boisson ou la première pâture à leur portée, et il n’est rien de tel que la passion d’une femme pour rendre un travailleur paresseux. Un individu ou un peuple laborieux est celui qui est mû, non par un très petit nombre de désirs extrêmement forts, car, dans ce cas, il vit surtout de brigandage, mais par un nombre relativement grand de désirs modérés. C’est en se multipliant que les désirs se modèrent ; et voilà pourquoi la diffusion des nouveaux besoins dans un pays y contribue en même temps à la paix et à l’activité sociales, parce qu’en apaisant les besoins anciens au profit des nouveaux, elle les rend plus facilement utilisables les uns pour les autres.\par
C’est dire que le progrès dans la diversité des désirs s’accompagne \phantomsection
\label{v1p154} du progrès dans leur mutuelle assistance. Mais ce serait une erreur de croire, remarquons-le, que, plus les désirs en contact social deviennent dissemblables, plus leur mutuelle utilisation s’accroît. Groupez des individus qui aient chacun des goûts tout à fait exceptionnels, tirés à un seul exemplaire dans le monde ; vous aurez beau les rapprocher, ils ne pourront, ne se comprenant point, se rendre aucun service. Ils ne parlent point la même langue, la même, c’est-à-dire composée de mots différents, mais répétés semblablement par tous ceux qui la parlent. Ce qui importe, donc, c’est la similitude dans la différence, ou, en d’autres termes, différents ordres de désirs généraux.\par
Or, comment tels ou tels désirs sont-ils devenus généraux, se sont-ils généralisés — dans une région donnée, plus ou moins étroite ou large ? Parce qu’ils se sont propagés de proche en proche, à partir d’un foyer\footnote{ \noindent Quant aux désirs, aux besoins tout à fait primitifs, qui naissent les mêmes chez tous les individus, indépendamment de toute influence reçue des parents, des camarades de milieu, ils sont, en quelque sorte, \emph{pré-économiques}. En réalité, ils se dissimulent sous la livrée des \emph{désirs spécifiés} par la contagion sociale...
 }. Dira-t-on que c’est là sortir des limites de l’Économie politique. Je ne comprendrais pas cette objection. Je la trouve cependant, à mon grand étonnement, sous la plume d’un économiste très distingué, dont j’ai eu déjà à louer les tendances, Courcelle-Seneuil. « Nous n’avons à nous occuper, dit-il, ni des combinaisons de nos désirs, ni de leur règlement. C’est l’objet de la physiologie sociale (?) et de la morale. En définissant le besoin, l’économiste ne peut le considérer que comme un moteur, une force d’intensité variable \emph{dont il ne lui appartient pas de rechercher les lois ;} il lui suffit de savoir qu’elle existe chez tous les individus et dans toutes les sociétés. » Là est l’illusion manifeste. Quel est donc ce besoin qui serait le même partout et toujours, à moins qu’il ne s’agisse de cette tautologie dont nous allons dire un mot, le \emph{désir du bonheur ?} Et comment l’économiste pourra-t-il manier une  \phantomsection
\label{v1p155}force dont il ignore les lois, dont il ne sait ni pourquoi ni de quelle manière elle varie en nature et en degré ?\par
On nous dit que tous les hommes, en tout temps et en tout lieu, s’accordent à désirer le bonheur. Qu’est-ce donc que cela, le bonheur ? C’est le \emph{désiré} purement et simplement. Le désir du désiré... voilà donc la clef de voûte de la science économique ! Mais ce que nous désirons, en somme, ce n’est jamais le « bonheur », c’est une sensation ou une émotion agréable, ou une idée qui nous plaît, ou une action qui nous intéresse. Or, ce n’est pas parce qu’une sensation ou une émotion est agréable, une idée plaisante, une action intéressante, que nous la désirons ; c’est parce que nous la désirons que nous la jugeons agréable, séduisante, intéressante. Dire qu’il en est toujours et partout ainsi, ce n’est pas dire grand’chose ; mais ce qui offrirait de l’intérêt, ce serait de rechercher quelles sont, à diverses époques et dans diverses sociétés, les sensations, les émotions, les idées, les actions, les faits qui sont ou ont été l’objet de désirs semblables chez tous les individus d’une société ou d’une classe. On apprendrait ainsi quels sont ou quels ont été les divers ordres de désirs généraux en des temps et des lieux différents, et que, loin d’être les mêmes universellement, ils ont beaucoup varié. La question de savoir s’il y a une voie ou des voies suivies par ces transformations, et quelles sont leurs causes, s’élèverait alors.
\subsubsection[{I.2.b. Le bonheur, rotation périodique de désirs enchaînés. Courbes fermées et courbes ouvertes de désirs.}]{I.2.b. Le bonheur, rotation périodique de désirs enchaînés. Courbes fermées et courbes ouvertes de désirs.}
\noindent Le bonheur, si nous serrons d’un peu près cette notion si vague pour la préciser, où le verrons-nous réalisé ? Apparemment dans l’état d’un individu ou d’un peuple qui a trouvé son assiette, comme on dit. Et qu’est-ce que cela veut dire ? Cela veut dire que le bonheur est, non pas précisément l’apaisement de nos désirs, mais une rotation en quelque sorte quotidienne de désirs enchaînés, périodiquement renaissants \phantomsection
\label{v1p156} et satisfaits de nouveau pour renaître encore, et ainsi de suite indéfiniment.\par
Je dis \emph{rotation ;} quand, en effet, la série des désirs qui s’enchaînent, entrecoupés de satisfactions alternatives, se présente comme une ligne qui ne revient pas sur elle-même, comme une \emph{courbe ouverte} qui va toujours de l’avant, dans l’inconnu de sensations toujours nouvelles, de desseins toujours inédits, il y a fièvre ambitieuse ou amoureuse, et il peut y avoir ivresse, transport de joie ; il n’y a pas bonheur.\par
Sans doute, tous les besoins de la vie organique sont essentiellement périodiques, le besoin de boire ou de manger, de s’abriter contre le froid, etc. ; ils se répètent dans la journée de l’individu ou dans son année, à intervalles plus ou moins réguliers ; mais les désirs spéciaux, d’origine sociale, qui sont la traduction économique de ces besoins, le désir de tel ou tel plat, de telle ou telle boisson, de tel ou tel vêtement, etc., ne se produisent pas toujours périodiquement ; et même ils commencent toujours par être des fantaisies avant de se consolider en habitudes. On voit même des touristes qui recherchent le changement continuel en fait de menus et de boissons, et qui n’aiment pas à coucher deux fois de suite dans le même lit. Les femmes ultra-élégantes ne portent pas deux fois la même toilette, autrement dit leur désir de s’habiller d’une certaine manière ne se répète pas deux fois. Il y a donc, dans toute vie individuelle, à distinguer les désirs périodiques, périodiquement enchaînés, qui sont les plus nombreux, les plus importants au point de vue de la production industrielle, et les désirs capricieux, non périodiques, qui se suivent sans se répéter régulièrement. C’est surtout sur les habitudes des individus que l’industrie doit compter ; mais leurs passions et leurs caprices, dont la proportion va grandissant à notre époque de crise sociale, sont la pépinière où prennent naissance les nouvelles habitudes de demain.\par
Chacun de nous, et aussi bien chaque peuple, pourrait être  \phantomsection
\label{v1p157}caractérisé — si l’on me permet de reprendre la métaphore de tout à l’heure, — par la nature de la courbe ouverte et de la courbe fermée qui lui sont propres, par la proportion des deux, par la composition des éléments de chacune d’elles, par son degré d’étroitesse ou de largeur. La proportion des deux est très inégale d’un individu à un autre. Tantôt la courbe fermée est très large et la courbe ouverte très petite ; c’est le cas des individus et des peuples qui recherchent beaucoup le confort, mais qui ont peu d’aspirations généreuses et passionnées ; tantôt c’est l’inverse, comme chez les individus et les peuples très idéalistes à la fois et très simples de goûts. — Plus souvent il arrive que la courbe ouverte et la courbe fermée vont s’élargissant, ou se resserrant ensemble, parallèlement. Chaque homme, et aussi bien chaque peuple, porte en soi la virtualité d’une courbe \emph{maxima} de désirs et de satisfactions d’un certain genre, où il déploiera, moyennant l’aide des circonstances, toute l’énergie dont il dispose. Le bonheur de Pierre est un très petit cercle, et son évolution passionnelle et capricieuse est une très petite parabole, aux branches courtes et peu écartées. Le bonheur de Paul est un cercle immense et la série de ses états de passion et de caprice est une parabole d’une prodigieuse envergure, telle que l’ambition d’un Charles XII ou d’un Alexandre, ou les amours d’un don Juan. Si donc à une personne ou à [{\corr une}] race faite pour une orbite très vaste de désirs et de satisfactions, vous imposez une existence resserrée qui serait heureuse pour une autre, vous la rendez très malheureuse. Et l’inverse n’est pas moins vrai. Mais il faut ajouter que le mouvement de la civilisation tend, par une nécessité logique, comme nous venons de le dire plus haut, à élargir sans cesse la roue des besoins, — qui s’accordent de mieux en mieux en se multipliant, — et, par suite, à éliminer les individus ou les peuples nés pour une plus étroite circulation, alors même que plus fine et plus délicate. Tendance non louable de tous  \phantomsection
\label{v1p158}points, et à laquelle ces derniers, quand ils sont des esthéticiens raffinés, ont le droit de résister de toutes leurs forces, désespérément.\par
Il y aurait à examiner comment et pourquoi une courbe de désirs, jusque-là fermée, vient à s’ouvrir, c’est-à-dire pourquoi une habitude ancienne est brisée et, à l’inverse, comment et pourquoi une courbe ouverte vient à se fermer, c’est-à-dire une passion ou un caprice dégénère en habitude, en besoin. Notons qu’à l’origine, aussi haut du moins qu’il nous est possible de remonter dans le passé des groupes humains, toute courbe de désirs nous est présentée comme hermétiquement close et tournant sur place dans la monotonie et l’étroitesse agitées de la vie des tribus sauvages. C’est là, ce semble, la donnée première, quoiqu’il soit permis de supposer que, dans une période antérieure, ces cercles primitifs eux-mêmes ont débuté par être des courbes ouvertes. Le fait est que, si l’on consulte l’histoire sur les véritables rapports qui existent entre les courbes ouvertes et les courbes fermées, ou, pour parler sans métaphore, entre les crises d’anxiété passionnée et révolutionnaire, où domine l’esprit de mode émancipé de l’esprit de coutume, et les périodes de prospérité stationnaire où la coutume élargie reprend son empire, on peut poser la règle qui suit : Toute courbe ouverte tend d’elle-même à se fermer, toute crise de passion révolutionnaire ou rénovatrice à s’apaiser en habitudes nouvelles, en nouvelles coutumes, en nouveaux besoins périodiques. Mais l’inverse n’est pas vrai, une courbe fermée ne tend pas toujours d’elle-même à s’ouvrir, une nation qui tourne sur place dans sa sphère coutumière ne cherche pas toujours à briser ce cercle magique. Il faut le plus souvent, pour le rompre, un choc étranger, une impulsion venue du dehors, l’inoculation du virus européen au Japon, par exemple, ou à certaines peuplades polynésiennes.\par
Cette remarque s’applique aussi bien à l’évolution individuelle qu’à l’évolution sociale, mais avec une différence bonne  \phantomsection
\label{v1p159}à signaler. Pendant la jeunesse, la succession des désirs nouvellement acquis reste longtemps une courbe ouverte, et peu à peu elle va se fermant. Mais il vient fatalement un moment, la vieillesse, où, après s’être fermée, elle va aussi se rétrécissant, sans que la rotation s’accélère à mesure que le cercle se rétrécit. Loin de là, elle se ralentit, et la mort est le terme nécessaire de ce double changement. Rien de pareil à cette nécessité ne semble s’imposer \emph{inévitablement} aux sociétés qui évoluent.\par
Puisqu’un cercle habituel ou coutumier de besoins ne peut s’élargir qu’à la condition d’être d’abord rompu en un point, et qu’il ne peut se rompre, le plus souvent, qu’à la faveur d’un apport étranger, on voit l’importance capitale des rapports internationaux, et de tout ce qui les facilite ou les entrave, au point de vue des progrès de la paix et de l’harmonie sociales que l’élargissement du cercle des besoins, avons-nous dit plus haut, rend seul possibles. Mais en même temps nous ne pouvons méconnaître les dangers, parfois mortels, que la mise en rapport, pour la première fois, de deux nations auparavant sans relations mutuelles, présente pour chacune d’elles, et surtout pour la plus faible des deux, pour l’inférieure, qui emprunte plus qu’elle ne prête et subit presque sans réciprocité la suggestion de l’autre. Un désir, un besoin nouveau, d’origine étrangère, qui nous est apporté, n’est \emph{classé} que lorsqu’il est entré dans la ronde de nos désirs alternatifs et périodiques et qu’il y a pris rang. Mais, d’abord, il commence toujours par la briser un peu et y déranger l’ordre établi, comme ferait une nouvelle planète qui viendrait du dehors prendre place dans notre système solaire. Et la question est de savoir si l’ordre troublé parviendra à se rétablir. Ce caractère momentanément perturbateur de tout nouveau besoin, même inoffensif en apparence, tel que celui de bicyclettes ou de téléphones, explique dans une certaine mesure la résistance des milieux conservateurs à toute importation de ce genre, mais ne la justifie que bien rarement. Car c’est  \phantomsection
\label{v1p160}seulement dans le cas, extrêmement rare, d’une trouée faite au cercle coutumier par une masse de besoins exotiques pénétrant en bloc, que la rupture est complète et que la blessure est trop profonde pour se refermer. Quand un besoin nouveau pénètre isolément, comme il arrive d’ordinaire, il lui suffit, pour parvenir à être \emph{classé}, de trouver plus d’auxiliaires que de rivaux ou d’ennemis parmi les besoins anciens qu’il vient déranger. Or, il ne rencontre de rivalité et d’hostilité possible que dans le très petit groupe des désirs ayant pour objet des articles ou des services à peu près similaires ; mais il doit compter sur la faveur de tous les autres désirs auxquels il procure un débouché de plus à raison de sa dissemblance même et de sa nature hétérogène. Par exemple, le besoin d’éclairage électrique, à son apparition, n’a été repoussé que par les besoins similaires, d’éclairage au gaz, au pétrole, à la bougie ; mais il a été favorablement accueilli par l’ensemble des autres besoins qui voient en lui un nouveau stimulant de l’activité productrice en général. L’enchaînement réciproque qui lie les uns aux autres les désirs différents formant un système économique est bien, il est vrai, un lien téléologique, mais ce n’est point une déduction, un déroulement de fins et de moyens convergeant vers un même but, comme l’organisation militaire ou administrative et la hiérarchie sociale. Cette harmonie de nature politique, par collaboration, pourrait être souvent gravement compromise par l’intrusion d’un élément étranger, de tel désir, de tel mode d’action nouveau qui implique contradiction, au fond, avec la hiérarchie existante ou avec l’ordre régnant, et qui tend à relâcher le faisceau des forces nationales, à gêner leur convergence éventuelle contre l’ennemi. La pénétration des idées exotiques dans une nation y affaiblit le plus souvent l’énergie patriotique, en y dissipant, par exemple, les illusions dont l’orgueil collectif se nourrit ; et de même la pénétration des besoins exotiques dans un pays resté simple et rural jusque-là y affaiblit  \phantomsection
\label{v1p161}l’ardeur belliqueuse, la ténacité indomptable. Mais l’harmonie de nature économique, par mutuelle assistance, comporte, au contraire et appelle ces importations qui la consolident après un passager ébranlement. Aussi est-ce toujours au point de vue politique, et non économique, qu’on se place quand on repousse avec méfiance l’invasion de l’internationalisme et qu’on signale l’affaiblissement national qui en est la suite fréquente. Le point de vue socialiste de l’organisation du travail peut être considéré comme la fusion des deux points de vue politique et économique en un seul, par l’absorption du second dans le premier. C’est l’originalité du socialisme d’avoir ajouté au très petit nombre des buts collectifs que les hommes réunis en nation peuvent poursuivre, gloire patriotique, guerre, conquête, défense du territoire, un grand but nouveau, très digne de leurs efforts, l’organisation consciente et systématique du travail. Seulement, remarquons que, si ce but vient à être atteint, il deviendra bien plus difficile à un nouveau besoin, et, par suite, à une nouvelle industrie, de s’intercaler dans la chaîne des besoin reconnus. Le travail s’ossifiera en s’organisant.\par
Une illusion à craindre, quand on vit comme nous à une époque de progression rapide et fiévreuse des besoins, est de croire qu’elle est l’état normal de l’humanité et pourra se poursuivre sans fin. Il viendra nécessairement un moment où le cœur humain, même américain, ne suffira plus à cette émission continue de nouveaux désirs que les développements de la machinofacture exigent de lui pour qu’il offre des débouchés sans cesse croissants à sa production toujours plus abondante. La nature humaine n’est pas inépuisable en besoins, ni en caprices même, et, tôt ou tard, chaque homme, même le plus ambitieux et le plus imaginatif, se heurte aux limites non seulement de sa force, mais de son désir, devenu inextensible. Quand ce heurt final, quand cet arrêt de croissance se produira pour l’humanité, il est bien certain que le progrès ne pourra plus consister dans un  \phantomsection
\label{v1p162}accroissement continu de la production, idéal de tant d’économistes. Il ne pourra plus viser que l’abréviation croissante du travail humain et l’augmentation du loisir.
\subsubsection[{I.2.c. Naissance des désirs, leur propagation, leurs luttes.}]{I.2.c. Naissance des désirs, leur propagation, leurs luttes.}
\noindent En résumé, chacun de nous tourne ainsi, à chaque instant, dans un cercle plus ou moins grand de désirs périodiques, — aux périodes régulières ou irrégulières — et, à chaque instant, est lancé sur la voie de quelque fantaisie, de quelque passion entraînante, qui tend toujours, souvent parvient, à entrer à son tour dans la ronde des désirs enchaînés, à s’y fixer en habitude. D’autre part, chaque peuple, composé d’un certain nombre d’individus, est l’entrelacement, pour ainsi dire, de ces cercles individuels et aussi bien de ces paraboles individuelles, de ces habitudes et de ces fantaisies qui, considérées en masse, prennent le nom de coutumes et de modes. — Or, si le vœu du bonheur était le désir unique et fondamental, on verrait chaque peuple, comme chaque individu, une fois son cercle d’habitudes ou de coutumes tracé, s’y enfermer, s’y clore à jamais. Mais nous voyons au contraire que, par l’insertion de nouvelles fantaisies et de nouvelles modes, ce cercle tend sans cesse, en général, à s’élargir en se déformant, dans une fièvre de croissance continue, dans une inquiétude constante. Ce n’est donc pas le vœu du bonheur qui explique cet élargissement fiévreux. Dira-t-on que c’est le \emph{vouloir vivre} de Schopenhauer ? Mais qu’est-ce autre chose qu’un nom générique donné à l’enchaînement même des désirs successifs et divers ? Les nommer, ce n’est pas les expliquer.\par
D’où proviennent donc tous ces désirs nouveaux qui viennent s’insérer de temps en temps dans la ronde de nos désirs ? Et d’où proviennent aussi bien les désirs anciens ? On peut répondre, si l’on veut, que la source de tous nos  \phantomsection
\label{v1p163}désirs, même des plus raffinés, est de nature organique et vitale. Il n’est pas jusqu’au désir de voir jouer des tragédies ou de composer des opéras wagnériens qui ne soit le rejeton d’une souche physiologique, le besoin de se divertir, de dépenser ses forces, tout comme le besoin de manger des gâteaux procède du besoin de nutrition, tout comme le désir de monter à bicyclette ou en automobile procède du besoin de locomotion. Les besoins sont le tronc dont les désirs sont les rameaux, et il n’appartiendrait donc, semble-t-il, qu’au naturaliste de résoudre notre problème. La vérité est que tous les désirs possibles sont latents dans les profondeurs de notre organisme ; mais ils y sont cachés comme toutes les statues possibles sont renfermées dans le bloc de marbre. En définitive, cela n’empêche pas le statuaire d’être le véritable auteur de la statue. Le statuaire ici est multiple : c’est l’ensemble des circonstances de la vie. Ces circonstances peuvent être divisées en deux groupes : en premier lieu, la série des rencontres — chacune accidentelle prise à part, toutes nécessaires dans leur ensemble — de l’individu avec les êtres extérieurs qui composent la flore et la faune, le sol et le climat de la région ; en second lieu, la série des rencontres, — non moins accidentelles et non moins nécessaires en même temps — avec les autres hommes qui composent le milieu social.\par
Ces rencontres avec les êtres extérieurs provoquent autant de sensations spéciales, véritables découvertes de la vue, de l’ouïe, de l’odorat, du goût, du tact, qui éveillent certains modes d’action spéciaux, cueillette, poursuite de tel ou de tel gibier, pêche, primitives inventions presque instinctives ; et ce sont ces découvertes et ces inventions élémentaires, plus ou moins spontanées, qui, en se propageant imitativement des premiers qui les ont faites aux individus de leur voisinage et de ceux-ci à d’autres, grâce aux rencontres des hommes entre eux, ont fait naître et enraciner dans tel pays le désir de manger des dattes ou des figues, ailleurs l’appétit de tel  \phantomsection
\label{v1p164}poisson ou de tel gibier, ou bien le goût de tel genre de poterie, de tel genre de tatouage et de décoration. Les rencontres des hommes entre eux n’ont pas seulement servi à cette propagation des découvertes ou inventions spontanément nées chez les individus les mieux doués en face de la nature. Elles ont servi surtout à susciter des découvertes et des inventions d’un degré supérieur qui, en faisant apercevoir la nature à travers les sentiments caractéristiques de la vie sociale, amours et haines, adorations et exécrations, peines et colères, sympathies et antipathies, à travers les verres réfringents des mots et des dogmes, des langues et des religions, des théories philosophiques, des notions scientifiques, donnent au désir une foule d’objets entièrement nouveaux, poursuivis par des voies d’activité tout à fait nouvelles.\par
Prenez un désir quelconque, même des plus anciens, des plus enracinés, le désir de manger du pain en Europe, de boire du vin dans le midi de la France, de se vêtir de drap, etc., vous n’en trouverez pas un qui n’ait commencé par une découverte ou une invention, dont l’auteur le plus souvent reste inconnu. Mais une invention non imitée est comme n’existant pas socialement, économiquement surtout. C’est seulement quand elle se propage, et dans la mesure où elle se propage, qu’elle prend une importance économique, parce que le nouveau désir de consommation — et aussi bien le nouveau désir de production — qu’elle a engendré, s’est répandu à un certain nombre d’exemplaires. Une industrie, née d’une invention, ou plutôt, toujours, d’un groupe d’inventions successives, n’est viable qu’autant que le désir de consommation auquel elle correspond s’est suffisamment répandu d’individu à individu, par une action inter-psychologique curieuse à étudier ; et le développement de cette industrie est entièrement subordonné à la propagation de ce désir. Tant que ce désir, par suite de certains obstacles opposés par la difficulté des communications, les frontières  \phantomsection
\label{v1p165}d’états, la séparation des classes, les mœurs, les idées religieuses, restera renfermé dans une étroite région ou dans une certaine classe peu nombreuse de la nation, cette industrie ne pourra devenir une grande industrie. Elle ne le pourra non plus si le désir se propage, à la vérité, très vite d’individu à individu, de pays à pays, mais, en chaque individu, en chaque pays, est éphémère et ne s’y enracine pas. Ce qu’un industriel, ce qu’un producteur quelconque doit savoir, avant tout, c’est si le désir qu’il satisfait est de ceux qui s’étendent loin mais durent peu, ou de ceux qui, resserrés dans d’étroites limites géographiques, durent fort longtemps. Il est — les éditeurs le savent bien — des livres d’archéologie locale que les archéologues de telle province peuvent seuls désirer lire, mais qui seront lus avec le même intérêt par dix ou vingt générations d’archéologues de cette province ; et on se garde bien d’imprimer et d’éditer ces livres dans les mêmes conditions que les romans en vogue aujourd’hui, dévorés dans le monde entier, et qui demain ne trouveront pas un acheteur. Il y a ainsi, pour toute industrie, pour toute production, à considérer deux sortes de débouchés, \emph{un débouché dans l’espace}, pour ainsi dire, et \emph{un débouché dans le temps}, le premier formé par la répétition-mode, le second par la répétition-coutume du désir spécial que cette industrie doit satisfaire. La proportion de ces deux débouchés varie extrêmement d’une industrie à une autre et, dans chacune d’elles, d’un âge à un autre âge, d’un pays à un autre pays. La difficulté, pour l’industriel avisé, est de se plier à ces conditions si complexes, et, pour l’économiste, de démêler quelques faits généraux parmi cette broussaille de faits particuliers.\par
Le problème se résume, en somme, à ceci : serrer le plus près possible la genèse des inventions, et les lois de leurs imitations. Le progrès économique suppose deux choses : d’une part, un nombre croissant de désirs \emph{différents ;} car, sans différence dans les désirs, point d’échange  \phantomsection
\label{v1p166}possible, et, à chaque nouveau désir différent qui apparaît, la vie de l’échange s’attise. D’autre part, un nombre croissant d’exemplaires \emph{semblables} de chaque désir considéré à part ; car, sans cette similitude, point d’industrie possible, et, plus cette similitude s’étend ou se prolonge, plus la production s’élargit ou s’affermit. — Or, nous venons de le dire, l’apparition successive des désirs différents qui sont venus s’ajouter ou se substituer les uns aux autres, — s’ajouter plus souvent que se substituer — a pour cause la succession des découvertes ou des inventions non pas seulement pastorales, agricoles, industrielles, mais religieuses même, scientifiques, esthétiques ; et la diffusion de chacun de ces désirs, son extension ou son enracinement, a pour cause l’imitation, la contagion mentale d’homme à homme. Il y a donc là, encore une fois, deux problèmes qui s’imposent au seuil de l’économie politique : 1\textsuperscript{o} Y a-t-il un ordre, ou plusieurs ordres, de la succession des inventions et découvertes ; et quel est-il, ou quels sont-ils ? 2\textsuperscript{o} Y a-t-il des faits généraux présentés par la propagation imitative des lois qui les régissent ; et quelles sont ces lois ?\par
Si l’on pouvait répondre à la première de ces deux questions aussi nettement qu’à la seconde, l’économie politique, dans certains cas, appuyée sur la statistique, pourrait se permettre de prédire, presque à coup sûr, quel sera l’état économique d’un pays, de la France, de l’Europe, dans vingt ans, dans un demi-siècle. Malheureusement, quand le statisticien, voyant la courbe graphique de tel ou tel article de commerce, de tel ou tel mode de fabrication ou de locomotion, qui fait des progrès graduels, se hasarde à dire que, dans cinquante ans, telle industrie aura envahi le monde entier ou telle partie de la planète, cette prédiction ne peut jamais être que subordonnée à cette condition expresse : « en admettant que, d’ici là, aucune invention rivale et mieux accueillie ne vienne à surgir ». L’invention future,  \phantomsection
\label{v1p167}c’est là l’écueil de tous les calculs, c’est l’imprévu où se heurtent toutes les prophéties.\par
J’ai répondu ailleurs, autant que j’ai pu, aux deux problèmes ci-dessus posés\footnote{ \noindent Voir, \emph{Logique sociale}, le chapitre intitulé les « lois de l’invention », et les \emph{Lois de l’Imitation.} Paris, F. Alcan.
 }, mais surtout au second qui se prête à des solutions précises. Je me permets d’y renvoyer le lecteur. Ici, je n’ai à dire qu’un mot de la première question, tout simplement pour montrer sa place et la profonde erreur de ceux qui l’oublient. Si dissemblables, si variées que soient les découvertes ou les inventions, elles ont toutes ce trait commun de consister, au fond, en une rencontre mentale de deux idées qui, regardées jusque-là comme étrangères et inutiles l’une à l’autre, viennent, en se croisant dans un esprit bien doué et bien disposé, à se montrer rattachées l’une à l’autre intimement, soit par un lien de principe à conséquence, soit par un lien de moyen à fin, ou d’effet à cause\footnote{ \noindent J’ai cité ailleurs ces deux exemples typiques : 1\textsuperscript{o} l’idée de la rotation de la lune autour de la terre et l’idée de la chute d’une pomme venant à se présenter dans l’esprit de Newton, comme peut-être les effets d’une même cause, indiquée par la seconde ; 2\textsuperscript{o} l’idée de la locomotive à vapeur et l’idée du rail (déjà connu longtemps avant la locomotive) venant à se présenter, dans l’esprit de Stephenson, comme propres à se lier utilement, la seconde pouvant servir d’auxiliaire à la première. — J’aurais pu citer aussi bien d’autres liaisons d’idées, telles que la liaison entre l’idée de l’éclairage et l’idée de l’acétylène, entre l’idée de l’acier et celle du manganèse, entre l’idée de la lumière et celle de l’ondulation ou des courants électriques...
 }. Cette rencontre, cette jonction féconde, voilà l’événement, le plus souvent inaperçu à l’origine, l’événement caché dans la profondeur d’un cerveau, d’où dépend la révolution d’une industrie, la transformation économique de la planète. Le jour où Arstedt a vu l’électricité et le magnétisme par un côté qui les liait l’une à l’autre, le jour où Ampère a repris et développé cette synthèse, le télégraphe électrique était né, destiné à enserrer le globe de son réseau aérien et sous-marin.\par
Ces croisements heureux d’idées dans des cerveaux, peut-on dire qu’ils sont toujours le fruit du travail ? Et osera-t-on  \phantomsection
\label{v1p168}prétendre que l’inventeur est un travailleur comme un autre ? L’inventeur peut être un travailleur, il l’est souvent, il ne l’est pas toujours ; mais ce n’est pas précisément en travaillant, c’est dans ses loisirs qu’il invente, quoique ce puisse être parce qu’il a travaillé ; et son invention n’est jamais un travail. Loin d’être un travail, c’est-à-dire un effort et une peine, elle est une joie intense et profonde, qui dédommage celui qui l’éprouve des fatigues de toute une vie. Quand son invention porte des fruits et lui vaut la gloire ou la fortune — rarement la fortune — c’est sa joie, non sa peine, que l’humanité lui paie ainsi.\par
Ne disons donc plus que le travail est la seule source de la valeur. La source première, c’est l’invention, qui n’est pas un travail ; car le travail, c’est de l’imitation à jet continu, c’est une série périodique d’actes enchaînés, dont chacun a dû être enseigné par l’exemple d’autrui et fortifié par la répétition de soi-même, par l’habitude.\par
Quand on dit que la production des richesses a trois facteurs, le capital, le travail et la terre, on néglige donc le facteur essentiel ; et cette analyse ne pourrait redevenir juste que si l’on entendait par capital un groupe d’inventions, définition que je développerai plus loin. Mais cette acception est étrangère à ceux qui ont écrit sur ce sujet. Aussi, décapitée de la sorte par l’oubli du facteur le plus important, l’analyse des éléments de la production a-t-elle donné lieu aux prétentions les plus justifiées en apparence. Certes, présenté comme la rémunération du capital engagé dans une entreprise, le bénéfice de l’entrepreneur, en qui l’on ne veut voir qu’un capitaliste, peut être souvent jugé excessif et obtenu aux dépens des salaires de l’ouvrier. Les choses changent de couleur si l’on voit dans l’entrepreneur ce qu’il est quelquefois, pas toujours, un inventeur au petit pied, dont l’invention consiste à avoir appliqué d’une certaine manière des inventions connues (ce qui est aussi la jonction utilitaire de deux idées). En tant qu’inventeur, en  \phantomsection
\label{v1p169}effet, l’entrepreneur peut avoir droit à des bénéfices énormes auxquels, comme capitaliste, il ne saurait prétendre en bonne justice. C’est sous ce nouvel aspect, je crois, que l’examen des théories socialistes doit être repris. Je n’en préjuge pas les résultats, mais j’affirme d’avance qu’ils seront très différents des conséquences déduites de prémisses erronées, de données inexactes et incomplètes.\par
On peut voir aussi, incidemment, en se pénétrant du rôle économique de l’invention, à quel point est superficielle la distinction des consommations productives et \emph{improductives.} Si les économistes avaient songé à cette parole de Gœthe, que « nulle sensation n’est passagère » ; s’ils avaient réfléchi à ce que les psychologues nous ont appris sur le pouvoir \emph{dynamogène} de toute sensation nouvelle et sur l’inépuisable fécondité d’une sensation vive en souvenirs, images internes d’elle-même, en suggestions aussi de sensations pareilles chez autrui, par conséquent de désirs pareils ; s’ils avaient eu égard à ce caractère multipliant et prolifère de la sensation, source d’idées neuves, de croisements heureux, qui sont de grandes ou petites inventions, industrielles ou autres ; s’ils avaient pensé à cela, ils auraient évité l’erreur de dire que la destruction d’un produit pour la simple satisfaction d’un désir individuel est une consommation improductive. Les consommations « improductives » de ce genre font tout le charme et tout le prix de la vie, et c’est grâce à leur nombre, à leur succession variée dans une vie de méditation, entrecoupée de travaux professionnels et de loisirs, qu’ont apparu toutes les innovations, grandioses ou minuscules, qui ont enrichi et civilisé le monde.\par
Les consommations qui passent pour seules productives aux yeux de l’économiste, sont celles qui consistent en sensations vulgaires de la faim et de la soif satisfaites, du bien-être matériel, où les forces se retrempent immédiatement par un nouveau travail. Mais, si les sensations d’autre  \phantomsection
\label{v1p170}sorte, par exemple les spectacles successifs d’un voyage d’agrément, en pays étranger, ne servent pas à régénérer tout de suite les forces physiques, elles servent bien davantage à les stimuler ultérieurement chez le touriste, et, à son exemple, chez autrui, en le poussant lui-même et en excitant les autres, par ses récits, à inventer et à travailler pour pouvoir les faire renaître. L’extension du réseau des voies ferrées et la progression de leurs recettes, avec tout le renouvellement industriel qui en est la suite, n’ont pas d’autre explication, au fond, que ces sortes de consommations improductives.\par
En cherchant bien, on verrait que toutes les consommations destinées à devenir des plus productives ont commencé par être improductives. Il n’est pas un objet de première nécessité, la chemise, les souliers, le chapeau, qui n’ait commencé par être un objet de luxe. Les dindons ont été, au moment de leur importation en Europe, des oiseaux d’agrément, et les pommes de terre des curiosités d’agriculteurs grands seigneurs, avant d’entrer dans la nourriture du dernier paysan irlandais. Les bicyclettes, de simple amusement d’oisifs au début, sont en train de devenir le véhicule indispensable du travailleur pressé. L’automobile, cette locomotive émancipée, destinée sans nul doute à un bel avenir, n’est encore, à peu d’exceptions près, qu’un sport d’hommes de loisir, une amusette coûteuse...\par
Et tout cela est conforme, remarquons-le, aux lois de l’imitation, d’après lesquelles un désir nouveau se propage d’autant plus vite — toutes choses égales d’ailleurs — dans un milieu social, que ce milieu est plus \emph{dense socialement}, c’est-à-dire que les individus y sont en contacts spirituels plus fréquents, ce qui est le caractère le plus marqué des classes aristocratiques à certaines époques, et, à d’autres époques, des classes riches dans les capitales. La cascade de l’imitation, en nappes élargies, tombe ainsi toujours de  \phantomsection
\label{v1p171}la noblesse aux classes inférieures, ou des capitales aux villes de second ordre et aux campagnes. On explique par la même loi pourquoi la grande industrie a débuté historiquement, et n’a pas pu ne pas débuter, par la fabrication d’objets de luxe, tels que les glaces de Venise, les draps fins de Florence, les tapisseries françaises, et pourquoi Colbert a eu raison de porter d’abord son attention sur des industries aristocratiques, parce qu’à son époque elles étaient seules viables, correspondant seules à des désirs répandus sur de vastes espaces, par-dessus les étroites frontières des provinces et des localités où restait coutumièrement confinée la vie des classes pauvres.\par
Ces lois de l’imitation, d’ailleurs, sont très complexes, et je ne puis pas entrer dans leur détail. Tout ce que j’en dirai c’est qu’il y faut distinguer deux sortes d’influences : des influences qui tiennent à la nature même des exemples à imiter, et d’autres qui tiennent à la nature des personnes qui donnent ces exemples, aux lieux et aux temps où ils se produisent. J’ai donné aux lois qui régissent les influences de la première sorte le nom de lois logiques ; aux autres, le nom de lois extra-logiques. C’est en combinant ces deux sortes d’influences, et les lois qui les formulent, qu’on peut répondre à cette difficile question : pourquoi, entre plusieurs exemples qui s’offrent à la fois, tel exemple et non tel autre s’est-il propagé dans ce pays, à cette époque, dans cette classe, et non ailleurs ?\par
Il faut tenir compte, ici, et de la nature plus ou moins prestigieuse de l’individu, de la classe, du peuple, d’où vient l’exemple importé, et de l’obstacle ou du secours que cet exemple trouve dans les désirs ou les idées déjà installées dans les esprits. Certains désirs sont entravés ou arrêtés dans leur propagation parce qu’ils se trouvent en lutte avec des besoins ou des idées qui les contredisent : la consommation des vins français se répandrait davantage en Angleterre si l’habitude d’y boire de la bière ne lui faisait  \phantomsection
\label{v1p172}obstacle ; en pays musulman elle est en conflit avec les défenses du Coran. Certains autres désirs sont, au contraire, aidés dans leur expansion par des désirs ou des idées qui tendent ou paraissent tendre au même but.\par
Un exemple frappant de cette dernière classe de désirs est celui de boire des liqueurs alcooliques (désir né, dans ses multiples variétés, d’inventions telles que celle de la distillation de l’alcool, du bitter, du vermouth, de l’absinthe, etc.). Non seulement le désir de cette consommation — qui n’est pas improductive mais bien \emph{destructive}, car il en existe beaucoup de telles — est de ceux qui, une fois éprouvés et satisfaits, se répètent le plus sûrement et s’enracinent le plus vite, mais encore il trouve pour auxiliaire, dans sa propagation au dehors, un désir des plus répandus, et des plus nobles, le désir de causer et de fraterniser avec des camarades. Les deux se satisfont à la fois, l’un stimulant l’autre et réciproquement, par la vie de café. On n’a pas étudié assez l’action des groupements sociaux de tous genres, cafés, salons, sectes, concerts, foules, dans la propagation des désirs économiques, dans leur direction et leur intensité. C’est ce qui rend l’alcoolisme si dangereux, si difficile à extirper. « L’habitude de boire est, sur les bords du Rhin, une des formes de la sociabilité, » dit avec raison M. Schultze-Gavernitz. Un peu partout il en est ainsi, si bien que l’alcoolisme peut être considéré comme une maladie de croissance de la sociabilité en progrès. Le malheur est qu’on meurt très bien d’une maladie de croissance. Boire ensemble, encore plus que manger ensemble, est un besoin social. On trinque, on boit à la santé d’une personne ; on ne mange pas, on ne choque pas les assiettes à sa santé...
\subsubsection[{I.2.d. Distinction entre les désirs qui, en se propageant, se fortifient, et ceux qui, en se propageant, s’affaiblissent. Désirs passifs et désirs actifs.}]{I.2.d. Distinction entre les désirs qui, en se propageant, se fortifient, et ceux qui, en se propageant, s’affaiblissent. Désirs passifs et désirs actifs.}
\noindent Je n’insiste pas sur ces remarques, sur lesquelles j’aurai à revenir. Mais ne négligeons pas de faire observer deux  \phantomsection
\label{v1p173}choses, qui importent beaucoup au point de vue économique : d’abord, que la facilité, la promptitude des divers ordres de désir à se propager, est très inégale ; en second lieu, qu’il est des désirs qui, en se répétant, par l’habitude ou par la propagation, se fortifient, s’intensifient, tandis que d’autres, à mesure qu’ils se répètent, vont s’affaiblissant. Je sais bien qu’on dit quelquefois que tous les désirs sont stimulés par leur satisfaction ; mais, si cette généralisation excessive était exacte, la civilisation, qui a pour conséquence certaine la multiplication de ces désirs soi-disant renforcés, devrait se caractériser par la violence inouïe du cœur humain exacerbé par elle. Il n’en est rien ; et, de fait, on observe que les désirs de sensations vont, d’ordinaire, s’affinant mais s’atténuant. La gourmandise barbare est devenue la friandise civilisée ; le libertinage même s’apprivoise. Il y a des exceptions, nous en avons cité une, l’alcoolisme ; nous devrions, peut-être, y joindre l’habitude de fumer, plus tyrannique d’autant que plus générale. — Les désirs de sentiments et surtout d’idées et d’actes, qui sont en proportion croissante pendant la période ascendante de l’évolution sociale, diffèrent des précédents en ce que leur propagation les renforce en chacun des individus qui les ressentent : désir d’honneur, de gloire, de pouvoir, de sécurité, de liberté, de justice. Il est vrai que la tendance de chacun de ces désirs à s’accroître en se diffusant est combattue par leur multiplicité croissante.\par
Les désirs passifs, les désirs de consommation, dans leur ensemble, tendent à se fortifier moins vite en se propageant que les désirs actifs, les désirs de production qui leur correspondent. Le désir de faire de la musique croît bien plus vite que celui d’en entendre. Il est donc dans la nature des choses que l’activité productrice, en chaque ordre d’action, soit poussée à marcher plus vite que l’avidité consommatrice, et soit obligée de temps en temps de s’arrêter pour l’attendre. D’où la fréquence sinon la périodicité des \emph{crises.}\par
 \phantomsection
\label{v1p174}Cette avance de la production sur la consommation correspondante, ou plutôt du désir de production \emph{à un certain prix} sur le désir de consommation des produits \emph{à ce même prix}, est d’autant plus marqué qu’il s’agit d’une production plus élevée, c’est-à-dire moins matérielle, plus spirituelle et sociale. La surproduction agricole, contrairement aux prédictions sinistres de Malthus, existe quelquefois, en temps de grand progrès ; mais elle est rare et toujours moins forte que la surproduction industrielle, qui excède souvent de beaucoup la demande des objets fabriqués (au prix offert). Quant à la production littéraire et artistique, elle dépasse énormément, et de plus en plus, la consommation de littérature et d’art. Et, cependant, l’appétit de littérature et d’art grandit plus vite par ses propres satisfactions que l’appétit d’articles industriels et surtout que la faim et la soif de subsistances\footnote{ \noindent J’ai dit ailleurs que, dans un pays où de nouveaux articles sont importés de l’étranger, le désir nouveau de consommation qu’ils y éveillent se propage un certain temps avant que les fabricants indigènes conçoivent le désir de les y produire. Cela est vrai, et cette antécédence de l’imitation des consommateurs étrangers sur celle des producteurs étrangers constitue le grand avantage offert par l’ouverture de nouveaux débouchés à l’industrie d’un peuple. Mais cette remarque n’a rien qui contredise l’observation faite plus haut. Quand, dans la colonie envahie par les produits de la métropole, des usines se seront installées à l’instar de celle-ci, les fabricants, comme dans celle-ci, y seront enclins à surproduire.
 }.\par
Parmi les désirs d’ordre supérieur, il en est peu d’aussi importants économiquement que le désir de sécurité pour l’avenir. Ce désir, qui se traduit pécuniairement en actes d’épargne, est un de ceux qui s’avivent et s’enracinent le plus en se répandant, si l’on en juge par l’abondante prolifération des caisses d’épargne et des compagnies d’assurances de tout genre, qui se chargent d’y donner satisfaction. Et la progression des capitaux placés démontre qu’il y est satisfait sur une échelle toujours plus large. Il est fâcheux que ce penchant si louable à la prévoyance soit en conflit avec le désir de la paternité, qui, lui, on le sait, n’est nullement stimulé,  \phantomsection
\label{v1p175}loin de là, par les naissances qui le satisfont : dans les départements français où le nombre des livrets et des polices d’assurances est en progression régulière, la natalité diminue régulièrement. Car il n’est possible à l’ouvrier, au cultivateur, de satisfaire ce désir d’épargne qu’à trois conditions : retranchement des consommations non indispensables à la rigueur, ou restriction malthusienne, ou suffisante élévation des salaires et des gains. Des trois solutions du problème, la dernière serait la meilleure assurément, si, en se généralisant, la hausse des salaires et des gains ne devait finir par une neutralisation réciproque des avantages momentanés ainsi obtenus. Reste la première, qui est douloureuse, ce qui fait que la seconde prévaut enfin. Ainsi s’explique en partie la diminution de la natalité française.\par
Entre la production et la consommation, entre la masse des producteurs et celle des consommateurs, ou bien entre les deux aspects d’une même masse humaine considérée successivement comme productrice et comme consommatrice, il y a, ne l’oublions pas, la même différence qu’entre le côté actif et le côté passif de la vie mentale. Or, dans l’intérêt même du côté passif, c’est au côté actif qu’il faut s’attacher. Le point qu’on \emph{regarde} a plus d’importance que la sphère qu’on \emph{voit ;} ce qu’on \emph{écoute} a plus d’importance que ce qu’on entend, et c’est en \emph{regardant} plus qu’en \emph{voyant} qu’on exerce et fortifie sa vue, c’est en \emph{écoutant} plus qu’en \emph{entendant} qu’on exerce et fortifie son ouïe\footnote{ \noindent Les lecteurs de Maine de Biran se rappelleront ici la haute importance qu’il attachait à la distinction entre \emph{écouter} et entendre, \emph{regarder} et voir, \emph{flairer} et odorer, \emph{déguster} et goûter, \emph{palper} et toucher. Il n’avait point tort de voir dans les impressions actives les déguisements variés de l’\emph{effort volontaire}, au fond identique à lui-même, et de considérer les impressions passives comme secondaires et dérivées.
 }. C’est ce qu’il ne faut pas perdre de vue quand on recherche les effets qu’aurait sur la fortune et la prospérité publique une mesure — protectionniste, par exemple — destinée à faire gagner les producteurs aux dépens des consommateurs, \phantomsection
\label{v1p176} ou une mesure opposée, propre à faire profiter ceux-ci du préjudice de ceux-là. Se préoccuper, avant tout, et en toute occasion, de l’intérêt des consommateurs parce qu’ils sont la majorité, c’est ne pas songer que le consommateur, comme tel — et abstraction faite de la productivité directe ou indirecte de sa consommation — est passif, et qu’un gain fait par lui n’a pas pour effet de le faire sortir de sa passivité. Il dépensera un peu plus ou il économisera un peu moins, voilà tout. Et cela ne sera un avantage général qu’autant que le travail producteur en sera stimulé et encouragé. Mais rien n’encourage et ne stimule la production comme la hausse du prix de ses produits. Aussi cette hausse, à certains moments, quoiqu’elle puisse paraître désavantageuse au public, devient-elle un avantage pour lui-même, par le coup de fouet qu’elle donne à la production et par l’abondance d’articles, bientôt offerts à plus bas prix, qu’elle appelle sur le marché.
\subsubsection[{I.2.e. Couple important de désirs : l’offre et la demande. Au sens objectif, stérilité, et, au sens subjectif, utilité de cette distinction.}]{I.2.e. Couple important de désirs : l’\emph{offre} et la \emph{demande}. Au sens objectif, stérilité, et, au sens subjectif, utilité de cette distinction.}
\noindent Il y a deux classes de désirs — accouplées comme la production et la consommation, et dérivant de ce couple — qui jouent un rôle capital sur la scène économique : le désir de vendre et le désir d’acheter, en d’autres termes l’\emph{offre} et la \emph{demande}. Je dis que l’\emph{offre} et la \emph{demande} consistent en désirs, je devrais ajouter : et en croyances. Par là nous allons empiéter un peu sur le sujet spécial du chapitre suivant. Quoi qu’il en soit, si l’on cherche à donner un sens autre que psychologique à cette grande antithèse économique, aussi encombrante que mal définie, on n’y parviendra pas, à moins d’aboutir à une vérité insignifiante. Quand on dit que l’équation de l’offre et de la demande vient à s’établir, entendrait-on seulement, au sens objectif, que le nombre des objets offerts est précisément égal à celui des  \phantomsection
\label{v1p177}objets demandés, et que l’offre ou la demande dont il s’agit consiste non dans un état d’âme des vendeurs ou acheteurs éventuels, mais dans le fait même de la conclusion de leurs ventes et de leurs achats ? Mais alors, que signifie la fameuse formule, si ce n’est que le nombre des achats est toujours égal à celui des ventes ? Si, par \emph{demande}, on entend le groupe de ceux qui \emph{désirent posséder} un article, ce groupe est indéfini, ce sens vague et abusif ne peut servir à rien déterminer. Si, par \emph{demande}, on entend le groupe de ceux qui \emph{désirent l’acheter au prix actuel}, c’est-à-dire qui préfèrent la perte de cet argent à la non-possession de cet article, il est clair que la demande, entendue en ce dernier sens, seul précis, présuppose le prix, et loin d’en être la cause, en est l’effet. Il importe donc de donner une signification subjective à ces deux termes pour que leur rencontre ici ne donne pas lieu à une contre-vérité ou à une vérité de la Palisse. D’ailleurs, quand on dit que le prix est en raison inverse de l’offre et en raison directe de la demande, on admet que l’offre et la demande peuvent être inégales, que l’une peut croître pendant que l’autre diminue, hypothèse inconcevable s’il s’agit là de faits objectifs, d’opérations réellement conclues sur le marché. Aussi les économistes n’ont-ils pu ne pas faire un peu de psychologie à ce sujet, mais contre leur gré, et le plus souvent à leur insu ; de là l’insuffisance de leurs analyses.\par
Quand on oppose l’offre à la demande, il semble qu’on entende parler seulement du nombre des offreurs ou de la quantité de ce qu’ils offrent, et du nombre des demandeurs ou de la quantité de ce qu’ils demandent. Quand on dit que le prix varie en raison inverse de l’offre, on veut dire que le prix s’élève, ou bien parce que les offreurs deviennent moins nombreux, ou bien, plus souvent, parce que les quantités de marchandises offertes deviennent moins considérables. Quand on dit que le prix varie en raison directe de la demande, on veut dire qu’il s’élève ou s’abaisse suivant que le nombre  \phantomsection
\label{v1p178}des demandeurs ou la quantité de produits qu’ils demandent grandit ou décroît. On a relevé l’inexactitude de ces formules soi-disant mathématiques, car il n’est pas vrai que le prix d’une marchandise s’élève précisément ni approximativement au double, au triple, au quadruple, quand le nombre des demandeurs a doublé, triplé, quadruplé, ou quand la quantité des choses offertes a diminué de moitié, des deux tiers, des trois quarts. L’expérience montre que, s’il s’agit de denrées de première nécessité par exemple, il suffit d’une faible diminution des choses offertes pour que leur prix augmente énormément, tandis que, si l’augmentation ou la diminution porte sur des articles de luxe ou de certains luxes, elle peut être très forte sans que le prix varie beaucoup. La question est de savoir pourquoi il en est ainsi, de remonter aux causes de ces variations que la loi de l’offre et de la demande, ainsi entendue, n’explique pas et qui rendent si incomplète, si indéterminée, la prétendue détermination des prix en vertu de cette loi.\par
D’abord, il est manifeste qu’ici le degré d’intensité des désirs en présence, soit chez les demandeurs, soit même chez les offreurs (le désir de vendre comme celui d’acheter n’étant pas toujours égal à lui-même), doit entrer en ligne de compte aussi bien que le nombre des offreurs ou des demandeurs ou la quantité des choses offertes ou demandées. Si l’article offert est de ceux qui éveillent chez les consommateurs un désir dont la nature est de s’aviver en se propageant, de se fortifier en se répétant, le progrès numérique des demandeurs s’accompagnera d’une intensité croissante de leurs désirs, et le prix tendra, naturellement, à s’élever bien plus que si l’article offert répondait à un de ces désirs que leur répétition affaiblit. Je sais bien qu’il est moins facile de mesurer l’intensité des désirs que la quantité des marchandises ou de compter des marchands et des clients, mais cela n’empêche pas l’observation qui précède d’être théoriquement vraie, et d’ailleurs, en pratique, il y a moyen de  \phantomsection
\label{v1p179}tourner la difficulté de mille manières, par des signes extérieurs du désir que les négociants avisés connaissent bien et sur lesquels ils règlent leur estimation approximative des désirs de leur clientèle, à défaut d’un thermomètre psychologique plus parfait.\par
Mais cela ne suffit pas. Le désir de vendre à tel prix telle quantité de marchandises et le désir d’acheter à ce même prix cette même quantité sont, je le suppose, parvenus, sur un marché, au niveau voulu pour dominer tous les autres désirs concurrents dans le cœur des offreurs et des demandeurs. Mais est-il sûr que l’\emph{intention} effective de vendre ou d’acheter à ce prix naisse de là, et que la totalité ou la plus grande part des marchandises trouve preneur ? Non, si des bruits fâcheux sont mis en circulation, parmi les vendeurs, sur la solvabilité des acheteurs à crédit, ou, parmi les acheteurs, sur la bonne qualité des marchandises en dépit de leur excellente apparence ; et, faute de \emph{confiance} suffisante, le marché sera sans affaires, malgré l’accord des désirs. — Il est vrai que la vente au comptant, de plus en plus répandue pour les menues affaires, simplifie le problème, en ce qui concerne les vendeurs, la dépouille pour eux de l’élément-confiance ; et j’ajoute que souvent la \emph{garantie} donnée par le vendeur à l’acheteur, ou bien, dans les grands magasins, la faculté donnée au client de rendre l’objet acheté s’il n’est pas satisfait, tend à libérer aussi l’acheteur, dans une large mesure, de toute incertitude et de toute méfiance. Mais il ne s’agit là, je le répète, que des menues affaires de la vie. Pour les \emph{grandes affaires}, entreprises, achats et ventes de maisons, de biens, de domaines à la Bourse, il faut toujours se risquer, on se risque de plus en plus, et la grande difficulté est bien plutôt alors de mettre les croyances, les confiances d’accord, que d’accorder les désirs.\par
Pour l’achat des services il en est de même. Voici, à Paris, quelques centaines de médecins qui désirent — de plus en plus fort à mesure que la mode sévit davantage — faire  \phantomsection
\label{v1p180}l’opération de l’appendicite, et il y a en France quelques milliers de malades qui, ayant ou croyant avoir cette maladie, désirent — de plus en plus fort aussi, par obéissance à la mode médicale aussi — se faire opérer. — Remarquons, entre parenthèses, que les médecins désirent faire, même à un prix déterminé, un nombre indéfini d’opérations, tandis que les malades en désirent, à ce prix, un nombre précis. Comment donc, si l’on donne à l’offre des services médicaux un sens purement objectif, un nombre indéterminé de ces services pourrait-il jamais être égal à un nombre précis de demandes ? — Mais revenons. Tous les malades désirent être guéris, mais cela ne veut pas dire que tous, en fait, envoient chercher le médecin, le \emph{demandent ;} car il se peut, en premier lieu, qu’ils soient retenus par le prix excessif de l’opération, étant donnée leur fortune, et que, combattus entre le désir de guérison et la crainte de se ruiner, ils ajournent à plus tard leur résolution en laissant sagement la nature suivre son cours. Quant à ceux qui ont de quoi payer, ou, pour mieux dire, chez lesquels le désir de guérison par l’opération s’est élevé assez haut pour l’emporter décidément sur la crainte de s’appauvrir, la question, en second lieu, la question majeure pour eux, est de savoir s’ils ont ou non confiance dans le chirurgien qui leur propose l’opération. Mais il y a mille degrés de confiance et de crainte, bien des oscillations intérieures de hausse ou de baisse de foi. D’autre part, le chirurgien lui-même est loin d’avoir toujours le même degré d’assurance dans le succès final. Cela dépend de l’habitude plus ou moins grande qu’il a de réussir, de la proportion entre ses réussites et ses échecs ; cela dépend aussi de l’opinion qu’il a de la vigueur du malade. Pour que l’opération soit résolue de part et d’autre, par le chirurgien comme par le malade, et qu’il y ait vraiment équation entre l’offre de l’un et la demande de l’autre, il faut, avant tout, que, chez les deux, la confiance ait atteint un certain niveau, non pas nécessairement égal, mais qui dépend, chez  \phantomsection
\label{v1p181}l’un et chez l’autre, du niveau atteint par le désir d’opérer ou d’être opéré. Le degré de confiance minimum requis pour déterminer la résolution s’abaisse d’autant plus que le désir s’élève davantage, et \emph{vice versa ;} et c’est ainsi qu’on voit de jeunes chirurgiens, passionnément désireux de se signaler, risquer des opérations hasardeuses et presque criminelles.\par
Mais j’empiète là sur la théorie des prix. Je tenais à montrer, dès maintenant, que, dans la notion de l’offre et de la demande, le degré de désir et le degré de confiance sont des éléments non moins importants à considérer que la quantité des choses offertes ou demandées et le nombre des demandeurs ou des offreurs. Non seulement ils sont aussi importants, mais ils sont plus essentiels, plus caractéristiques ; et l’exemple précédent en est une preuve. Ici le nombre des offreurs et des demandeurs se réduit à l’unité, le plus souvent ; il y a un seul malade, à un moment donné, qui n’a en vue qu’un seul chirurgien ; il n’y a, non plus, de part et d’autre, qu’une seule chose offerte ou demandée ; et cependant, il y a, dans ce cas — tout comme dans le cas d’une halle pleine d’acheteurs et de vendeurs de blé, ou d’une Bourse pleine d’agioteurs, — des variations de l’offre et de la demande, des hausses et des baisses de l’une et de l’autre, jusqu’à ce qu’elles s’accordent enfin.\par
Ce cas, où il n’y a qu’un offreur et un demandeur en présence, est devenu exceptionnel au point de civilisation où nous sommes parvenus ; mais il n’en est pas moins le cas élémentaire et fondamental, le point de départ de l’évolution économique et l’explication de tout ce qui a suivi. Dans les marchés primitifs, il y a autant de prix différents du même objet ou du même service qu’il y a de couples d’acheteurs et de vendeurs, et, pour chacun de ces couples, le prix fixé à la suite d’un long marchandage est celui qui établit l’équation entre la demande et l’offre, entre \emph{cette} demande et \emph{cette} offre. Car il y a une offre et une demande  \phantomsection
\label{v1p182}spéciale en train de varier au cours de chacune de ces discussions animées entre deux hommes qui cherchent l’un à vendre le plus cher et l’autre à acheter le moins cher possible. Alors que se passe-t-il ? Les foires de village, surtout dans le Midi, sont instructives à cet égard. Aussi bien qu’à Nancy et à la Salpêtrière, on y pratique l’art de la suggestion par la parole et par le geste. Les maquignons sont de grands hypnotiseurs sans le savoir. Mais c’est un genre de suggestion réciproque en quelque sorte. Le marchand sait bien qu’il doit dissimuler le degré de son désir de vendre ; aussi parle-t-il d’un ton détaché : on lui a offert tel prix, qu’il a refusé ; il connaît dix personnes qui sont follement éprises de ce cheval. L’acheteur éventuel, de son côté, fait le dégoûté : ce cheval est trop petit ou trop faible pour lui, cette robe ne lui plaît pas. Il n’ignore pas, en effet, que, plus le marchand le jugera désireux d’avoir ce cheval, plus il en élèvera le prix. Mais le marchand à l’œil sur lui, il ne perd pas un de ses gestes, un de ses regards, et, d’après ces \emph{signes extérieurs}, jauge approximativement son désir d’acheter ce cheval, sa confiance dans les qualités apparentes de la bête. Pour aviver ce désir et cette confiance il fait trotter le cheval, préalablement muni d’une mesure extraordinaire d’avoine, déguise ses défauts tant qu’il peut, le montre sous son aspect le plus avantageux ; en même temps, il la vante avec des flots de paroles et des gestes charlatanesques, qui ne laissent pas de produire leur effet sur des esprits peu habitués à ces sortes de passes magnétiques. S’il voit, s’il devine que le client se laisse prendre à ses manières, il hausse le prix ; si le client est réfractaire, il cherche à le gagner par des prix plus doux. Mais celui-ci l’épie aussi, le sonde à son tour, et, d’après le besoin et le désir de vendre qu’il lui suppose, tient bon ou faiblit dans le débat du prix. Toutes ces manœuvres témoignent clairement, de part et d’autre, qu’il s’agit ici, avant tout, d’états d’âme à équilibrer ; et c’est parce qu’il s’agit de cela, c’est-à-dire de  \phantomsection
\label{v1p183}quantités en quelque sorte \emph{liquides}, non solides, très facilement oscillantes et changeant de niveau, exerçant l’une sur l’autre une influence réciproque, que l’équilibre cherché est si souvent obtenu. S’il s’agissait de qualités en quelque sorte solides, dures et résistantes, la fameuse équation entre l’offre et la demande serait impossible dans la plupart des cas.\par
Ce qui se passe ainsi dans les transactions primitives ne diffère en rien d’essentiel des phénomènes présentés par l’offre et la demande prodigieusement agrandies dans nos marchés modernes. Quand le marché s’étend, le marchandage n’est plus possible, parce qu’il faut un même prix, un prix momentanément fixe, pour toutes les transactions de même nature. La vie commerciale serait perpétuellement entravée s’il en était autrement. Ce sont les marchands alors qui, spontanément, fixent le prix. Ou plutôt, c’est tantôt la majorité des marchands, s’ils sont syndiqués, tantôt le marchand le plus fort parmi ceux qui se font concurrence. Mais toujours, avant de fixer le prix, ces négociants se renseignent curieusement sur le nombre des acheteurs probables à ce prix, et sur le degré de cette probabilité d’après la vogue de cet article à ce moment. Les statistiques commerciales ne les éclairent pas seules à cet égard ; mille symptômes sont propres à les instruire, s’ils ont de la finesse et de l’expérience. Mais ce n’est qu’après bien des tâtonnements, des essais de prix variés et successifs, que le prix définitif est adopté ; et l’on peut voir dans ces hésitations l’équivalent du marchandage primitif. D’autre part, le grand but visé par les négociants modernes, comme par les petits marchands villageois, c’est toujours d’éveiller, d’attiser le désir et la confiance du client, du public. A cela sert la réclame, équivalent supérieur du verbiage et des gesticulations du maquignon ou du marchand de moutons dans une foire rustique. Mais nous reviendrons sur ce sujet.\par
Observons cependant, avant de finir, que lorsque, au lieu  \phantomsection
\label{v1p184}d’un seul offreur et d’un seul demandeur, on a sur le marché un groupe d’offreurs et un groupe de demandeurs, le phénomène se complique par l’intervention d’un nouveau rapport inter-psychologique, qui s’établit non plus entre demandeurs et offreurs, mais entre chaque demandeur et d’autres demandeurs, entre chaque offreur et d’autres offreurs. Il y a toujours, comme résultante des influences personnelles qui s’exercent ainsi, un \emph{entraînement} en fait d’offre comme en fait de demande, d’où résulte une hausse ou une baisse des prix que rien d’objectif ne paraît motiver.\par
En voilà assez, certainement, pour démontrer, ce qui est l’objet de ce chapitre, le rôle capital du Désir dans les phénomènes économiques, sa présence souvent cachée, et qui devrait ne jamais l’être, dans les notions économiques fondamentales. Occupons-nous maintenant, plus spécialement, du rôle économique des idées et de la croyance.
 \phantomsection
\label{v1p185}\subsection[{I.3. Rôle économique de la croyance}]{I.3. Rôle économique de la croyance}\phantomsection
\label{l1ch3}
\subsubsection[{I.3.a. Action des croyances sur les désirs, et réciproquement. Leurs combinaisons. La réclame et son évolution : la réclame acoustique d’abord, puis visuelle.}]{I.3.a. Action des croyances sur les désirs, et réciproquement. Leurs combinaisons. La réclame et son évolution : la réclame acoustique d’abord, puis visuelle.}
\noindent En économie politique, il importe d’étudier de près l’action des croyances sur les désirs, la puissance qu’ont certaines idées, suscitées dans les esprits, d’éveiller dans les cœurs certains désirs de consommation ou de production, et d’en éteindre d’autres. Il importe aussi d’étudier l’action des désirs sur les croyances, le pouvoir qu’ont certains désirs, surexcités jusqu’à un certain point, par exemple celui de locomotion ou de lecture des journaux, de faire naître et croître la croyance générale en l’innocuité, en la sécurité des satisfactions données à ces désirs, telles que l’extension incessante du réseau des chemins de fer, ou la liberté illimitée de la Presse. Il importe enfin d’étudier les combinaisons fécondes des désirs et des croyances qui, en s’unissant pour former les prémisses du syllogisme de l’action, aboutissent comme conclusion à un devoir d’agir, à un vouloir, à un acte. Toute la conduite dérive de syllogismes pareils, le plus souvent inconscients.\par
Un désir est toujours précédé d’une perception ou d’une idée, d’un jugement sensitif ou intellectuel dont les deux termes sont liés par une persuasion plus ou moins forte. Si nous pouvions remonter jusqu’au premier homme qui a éprouvé le désir d’immortalité posthume, peut-être découvririons-nous que la possibilité, entrevue par lui, de cette vie d’après le trépas, a précédé ce désir. En tout cas, est-il certain que les progrès de la foi religieuse en l’immortalité  \phantomsection
\label{v1p186}de l’âme ont beaucoup contribué à développer le désir de revivre immortellement après la mort. La passion de l’égalité politique, et même économique, qui est le grand moteur du temps présent, a été engendrée et nourrie par l’idée et la conviction du droit à l’égalité. C’est seulement quand la foi à ce dogme de l’égalité humaine s’est répandue à fond parmi les ouvriers que l’ambition leur est venue de participer au confortable de leurs patrons, de s’alimenter, de s’habiller, de s’instruire, de voyager, de s’amuser à leur exemple. La liberté de la Presse, en permettant à toutes les idées, à toutes les opinions de se répandre, et en accélérant leur diffusion, facilite aussi et accélère prodigieusement la généralisation de beaucoup de désirs nouveaux qui donnent naissance à des branches d’industries nouvelles. Ce n’est pas seulement la quatrième page des journaux qui est composée de réclames. Tout le corps du journal est une sorte de grande réclame continuelle et générale. Aussi le \emph{standard of life} et l’activité productrice s’élèvent-ils dans une nation à mesure que les journaux s’y multiplient et que leur lecture s’y répand.\par
S’il pouvait se propager aujourd’hui sur le continent américain ou européen une doctrine telle que le stoïcisme antique ou l’évangélisme primitif, une foi profonde et unanime en la vanité du désir, en la sagesse du non-désir et de la vie réduite à son maximum de simplicité ; si, du moins, l’esthéticisme à la Ruskin faisait des progrès sérieux dans les masses, il est certain que l’industrie moderne serait frappée à mort. Sans souhaiter une telle catastrophe, il est permis d’espérer cependant qu’un jour viendra où, sous des formes nouvelles et meilleures, se fera jour la vérité entrevue par les anciens sages et nos esthéticiens, la nécessité de mettre un frein à la progression des désirs corporels qui nous divisent, aux inutiles complications de l’existence matérielle, pour donner libre essor aux désirs spirituels qui nous rapprochent, qui nous font nous toucher \phantomsection
\label{v1p187} par nos cimes les plus hautes comme les arbres des forêts.\par
Remarque essentielle : les idées, les opinions nouvelles se propagent bien plus vite que les nouveaux besoins dans une société. Aussi une révolution dans les esprits devance-t-elle toujours la révolution dans les coutumes et les mœurs qui en est la conséquence. Il est fort heureux qu’il en soit ainsi, et que, par suite, l’assimilation des pensées soit plus facile à faire que celle des désirs, surtout des désirs inférieurs. Car c’est par la similitude même de leurs objets que nos désirs nous opposent et nous mettent en lutte ; mais la similitude de nos pensées est, au contraire, une des causes les plus puissantes d’harmonie entre nous. Cette différence, soit dit en passant, explique pourquoi la concurrence économique produit en partie de bons effets : c’est qu’elle n’est une lutte qu’en partie, les désirs opposés étant liés à des jugements semblables, à des confiances semblables qui, dans le rival, font voir un concitoyen, un confrère en civilisation. Un \emph{collègue} est le plus souvent un \emph{concurrent.}\par
Quand une invention nouvelle apparaît dans une industrie (lampe électrique, automobile, etc.), s’il s’agissait pour elle d’éveiller et de propager le désir auquel elle correspond, son succès serait d’une lenteur décourageante. Mais, le plus souvent, ce désir est déjà tout propagé depuis longtemps (désir d’y voir le plus clair possible, de voyager le plus rapidement possible, etc.), et c’est seulement la confiance dans le nouveau moyen imaginé de satisfaire ce désir préexistant, qu’il s’agit de répandre et de faire pénétrer dans les esprits.\par
La réclame a été créée précisément dans ce but. Il y aurait à faire l’histoire de la réclame en divers pays, et il est probable que, si on comparait ces diverses lignes d’évolution, on y découvrirait des ressemblances. Ne semble-t-il pas que la \emph{réclame acoustique}, pour ainsi parler, caractérise principalement les pays encore barbares ou arriérés, et que la  \phantomsection
\label{v1p188}\emph{réclame visuelle} va se développant, avec la civilisation, aux dépens de la première ? Les cris des marchands ambulants dans les campagnes, — des chiffonniers, par exemple, qu’on appelait \emph{pillarots} dans le midi de la France, — les annonces faites par les crieurs publics dans les petites villes, — les airs particuliers et traditionnels que criaient ou chantaient les vendeurs des divers métiers en passant dans les rues de Paris au moyen âge et jusqu’à notre siècle, sont la plus archaïque forme de la réclame\footnote{ \noindent « Un poète du {\scshape xiii}\textsuperscript{e} siècle, Guillaume de la Villeneuve, dit Alfred Franklin, nous a décrit l’aspect curieux que présentaient alors les rues de Paris. Son contemporain, Jean de Garlande, a consacré aussi quelques lignes à ces crieurs enragés. » Dès le point du jour, un valet criait l’ouverture des bains « dont les relations avec l’Orient avaient généralisé l’usage » ; venaient ensuite les marchands de poissons, de volailles, de viande fraîche ou salée, d’ail, de miel, d’oignons... « Des femmes criaient de la farine et du lait, des pêches, des poires, etc. » Les raccommodeurs de vêtements, de meubles, de vaisselles, avaient aussi leur ritournelle. Et les marchands de vieux habits... Et les vendeurs d’oublis, etc., etc. — Le \emph{criage} devint un service public, dépendant du roi, qui finit par affermer à un seigneur nommé Simon de Poissy, le produit du criage parisien. — Mercier nous a laissé un tableau pittoresque de Paris au {\scshape xviii}\textsuperscript{e} siècle : « Il n’y a point de ville au monde, dit-il, où les crieurs et les crieuses des rues aient une voix plus aiguë et plus perçante... C’est à qui chantera sa marchandise sur un mode plus haut et plus déchirant. Tous ces cris discordants forment un ensemble dont on n’a point d’idée lorsqu’on ne l’a point entendu... C’est un glapissement perpétuel... » Vraiment, la substitution de l’annonce visuelle à l’annonce acoustique a été un progrès... Dès la deuxième moitié du {\scshape xviii}\textsuperscript{e} siècle, ce changement se produit par la publication de journaux d’annonces, où étaient insérées des réclames dont l’emphase ingénieuse amusait Voltaire. Voir à ce sujet Alfred Franklin, \emph{la Vie privée d’autrefois : l’annonce et la réclame.} Voir aussi à ce sujet la seconde édition de l’\emph{Histoire des classes ouvrières}, de M. Levasseur, t.I, p. 422 et suiv.
 }. Elle a été peu à peu remplacée, — sauf quelques rares vestiges — par les annonces dans les journaux, et par les affiches murales d’une variété infinie, née d’une imagination inépuisable. Les crieurs de journaux de nos boulevards, qui nous assourdissent, sont remplacés à Londres par des porteurs d’affiches en très grosses lettres qui permettent de voir sur leur dos les nouvelles à sensation. Il y a bien aussi une réclame qu’on pourrait appeler olfactive, celle de certains parfums traînant sur nos trottoirs. Mais celle-ci aussi va perdant, avec la civilisation, de son importance relative, et, de plus  \phantomsection
\label{v1p189}en plus, est supplantée par la réclame visuelle correspondante ; celle de toilettes spéciales, tapageuses ou habilement discrètes, qui attirent le regard. La raison de cette évolution, ce remplacement graduel de la réclame acoustique par la réclame visuelle, c’est que cette dernière est bien plus apte que l’autre à se développer en étendue. Sa portée, par les annonces des journaux, par les exemples multipliés des affiches murales, peut s’étendre indéfiniment, tandis qu’il est difficile et coûteux de multiplier beaucoup les crieurs publics. La réclame, en somme, se transforme dans le sens de son rayonnement de plus en plus large, libre et facile. Le nombre des réclames acoustiques ne saurait dépasser un certain chiffre dans les rues d’une ville sans aboutir à un assourdissement général, tandis que le nombre des réclames visuelles peut s’accroître sans que chacune d’elles cesse d’être distincte à la vue, quoiqu’elles puissent se brouiller dans la mémoire.\par
Mais, quel que soit le sens qu’elle affecte et son degré d’expansion, la réclame agit toujours en mettant en branle les conversations. Si l’on cause des nouveautés industrielles, des nouvelles étoffes, des nouveaux articles ou services quelconques, c’est par suite d’un entrefilet lu dans un journal, de la vue d’un homme-enseigne, d’un cri spécial, qui a arrêté un instant l’attention. Arrêter l’attention, la fixer sur la chose offerte, c’est l’effet immédiat et direct de la réclame. De là à capter la \emph{confiance} publique il y a loin, à la vérité. Mais alors intervient la puissance de l’exemple ; et, si la réclame frappe où il faut, aux foyers habituels des rayonnement imitatifs, c’est-à-dire à la Cour sous l’ancien régime, aux capitales dans nos démocraties, en somme, aux parties les plus animées, les plus éclairées de la société, à celles qui ont le plus de \emph{densité sociale} pour ainsi dire, l’innovation introduite là ne tarde pas à s’épancher dans tout le public.\par
Les étalages des magasins, dans les grandes villes, ont l’inconvénient de se neutraliser en partie réciproquement,  \phantomsection
\label{v1p190}comme les tableaux dans un musée. Aussi, quoique excellents pour appeler l’attention sur les marchandises exposées aux vitrines, seraient-ils par eux-mêmes très insuffisants pour décider les passants à l’achat. Le désir vague de posséder tel article, — combattu d’ailleurs par les désirs concurrents nés d’autres étalages — ne se transforme en \emph{résolution de l’acheter} que lorsque, d’une part, le prix s’abaisse jusqu’au niveau de ce désir ou que ce désir s’élève jusqu’au niveau du prix, et que, d’autre part, les marchandises désirées sont aperçues en la possession de quelqu’un qui est par nous jugé digne d’être pris pour modèle, ou avec lequel nous rivalisons, ou sur lequel nous nous conformons par sympathie, parce qu’il est notre égal, notre voisin, notre ami. C’est l’exemple d’autrui qui engendre le degré de confiance voulu en l’utilité de la chose désirée, et transforme par suite ce désir en \emph{volonté d’achat.}
\subsubsection[{I.3.b. Comment naît et grandit la confiance en une nouvauté. Nos droits fondés sur nos attentes.}]{I.3.b. Comment naît et grandit la confiance en une nouvauté. Nos droits fondés sur nos \emph{attentes.}}
\noindent Il est fort intéressant d’étudier comment, au début d’une invention hasardeuse, telle que celle des chemins de fer ou des ponts suspendus, et aussi bien celle du papier-monnaie, du billet de banque, naît et se propage la confiance publique dans cette nouveauté. Cela peut servir à nous apprendre de quelle manière, à une époque beaucoup plus ancienne, est née et s’est propagée cette confiance générale et profonde en l’échangeabilité universelle de l’or et de l’argent, qui constitue essentiellement la monnaie. On est surpris de voir avec quelle facilité ces hardiesses font leur chemin, et, \emph{a priori}, on aurait dû s’attendre à une résistance beaucoup plus prolongée de la méfiance générale au milieu de laquelle elles ont surgi. Mais il a suffi de quelques initiatives heureuses, suivies par une élite, pour donner courage et foi à beaucoup, bientôt à tout le monde. A propos des billets de  \phantomsection
\label{v1p191}banque, Law explique clairement cela, mais beaucoup trop rationnellement, suivant les habitudes de son siècle, dans ses lettres au régent. Il dit que les choses se sont passées de la sorte en Écosse, à Rome, à Naples : « Dans les commencements on a de la peine à accoutumer les peuples au paiement par billets ; mais, voyant qu’il y a une forte caisse pour les convertir en argent à volonté et remarquant la commodité de ces billets dans les payements, peu à peu ils s’introduisent dans le commerce, et, avec le temps, ils sont préférés aux espèces... » C’est juste, mais insuffisant. Si la raison eût autant que cela déterminé le public à accepter la fallacieuse monnaie, on ne comprendrait pas la catastrophe où la folie d’un engouement inouï l’a précipité. Dans une autre lettre de ce financier fameux, je note un passage plus pénétrant où il parle de son système : « Le crédit a porté les actions jusqu’à 2 000 livres à la face de ses adversaires ; et, malgré la crainte et les incertitudes de ceux mêmes qui les ont poussées à ce prix, \emph{le crédit s’est accru pour ainsi dire dans le sein même de la défiance.} » C’est ainsi toujours que les choses ont lieu. Et, si l’on demande pourquoi, au milieu d’une défiance qui règne dans l’esprit des masses \emph{et jusque dans l’esprit même des novateurs}, la foi en l’innovation fait sans cesse des progrès, il faut répondre que c’est en partie grâce à l’opposition même de la défiance externe et aussi bien interne qui, en l’entravant, l’a exaltée. La victoire graduelle de cette confiance, si petite d’abord et si restreinte, sur cette défiance vaste et profonde, a moins lieu de surprendre si l’on considère que la défiance est quelque chose d’assis, de stagnant, — de statique, dirait Comte — et que la foi en l’innovation est un \emph{courant}, quelque chose de dynamique qui se communique et s’échange d’individu à individu. La confiance, dans son mouvement accéléré d’homme à homme, s’accroît en vertu d’une loi semblable à celle de la chute du corps, tandis que la défiance régnante représente le repos et ne peut aller qu’en s’affaiblissant à la  \phantomsection
\label{v1p192}longue. C’est la \emph{vitesse acquise} qui est contagieuse, mais le \emph{repos acquis} ne l’est point.\par
Puis, il ne faut pas oublier que la nature humaine, en somme, est portée à l’optimisme. Pour s’en convaincre, il suffit de faire un rapprochement. Même en admettant les statistiques des chemins de fer les plus favorables aux compagnies, le risque que l’on court d’être victime d’un accident quand on monte en wagon est, d’après le calcul des probabilités, très supérieur à la chance qu’on a de gagner un lot de 100 000 francs quand on achète une obligation du Crédit foncier ou tout autre valeur à lots. Cependant la pensée de l’accident éventuel n’empêche personne ou presque personne de voyager en chemin de fer, tandis que la perspective de gagner le gros lot est la grande attraction qui explique la vogue de certains titres et est payée fort cher par les acquéreurs. D’où j’ai le droit de conclure que l’homme est bien plus enclin à l’espérance qu’à la crainte et que le pessimisme, quand il souffle quelque part, dans une petite région du public, y est importé et factice, en contradiction avec l’optimisme normal et habituel, et destiné à être refoulé par celui-ci — qui a quelque chose de plus dynamique, pour continuer la métaphore de tout à l’heure\footnote{ \noindent Courcelle-Seneuil admire, avec raison, le peu de temps qu’il a fallu pour dissiper les préjugés relatifs au prêt à intérêt. « Il y a lieu d’admirer le progrès de l’opinion sur ce point quand on songe que la première protestation scientifique en forme dirigée contre le prêt à intérêt date de Turgot et que ce prêt est toléré depuis le commencement du siècle seulement pas la théologie catholique. »
 }.\par
— Si, continuellement, nous désirons quelque chose, quelque chose qui se renouvelle sans cesse et renaît de ses propres cendres, continuellement aussi nous \emph{attendons} quelque chose, nous \emph{comptons} sur quelque chose. Et le degré ou la nature de ces \emph{attentes} n’importe pas moins, au point de vue économique, comme au point de vue social en général, que le degré ou la nature de nos désirs. L’espoir que nous avons de gagner à la loterie est une bien faible attente,  \phantomsection
\label{v1p193}mais notre espoir de vivre un mois, sinon un an, est une attente très forte, et nous nous attendons, avec une assurance voisine de la certitude, à toucher bientôt nos coupons de rentes, ou notre traitement. Notre \emph{sécurité}, toujours relative, se compose d’un nombre déterminé d’attentes de ce genre, très inégales et très dissemblables. Il vient un moment où, à force de croître en force, nos attentes donnent naissance à des droits. Le \emph{droit de propriété}, que les économistes postulent sans le justifier suffisamment, est fondé là-dessus, comme Bentham l’a montré à son point de vue avec tant de clarté qu’on en a méconnu la profondeur.\par
Ce qu’il nous faut, ce n’est pas seulement ni surtout la \emph{satisfaction de nos désirs}, c’est, avant tout, la \emph{confirmation de nos attentes}. Rien n’est plus douloureux, pas même l’échec de nos plus chers désirs, que la contradiction apportée à nos espérances, le démenti de nos assurances. Quand nos attentes profondes sont violées de la sorte, nous crions toujours à la violation de nos droits les plus sacrés. L’attente trompée, autrement dit la \emph{croyance contredite}, et le besoin non satisfait, la volonté entravée, autrement dit le \emph{désir contrarié :} voilà le double élément que l’analyse découvre toujours au fond des notions du mal et de l’injustice, comme elle résout toujours en \emph{croyances confirmées} et en \emph{désirs aidés}, les notions de droit et de devoir, de justice et de bien. Empêcher la contradiction des croyances, des attentes, et favoriser leur mutuelle confirmation ; empêcher la contrariété des désirs et favoriser leur aide réciproque ou leur convergence collaboratrice : telle est la fin, consciente ou inconsciente, poursuivie par toutes les législations et toutes les morales. Seulement, le cercle des croyances et des désirs dont il s’agit, s’élargit ou se resserre, se déplace dans un sens ou dans un autre, d’après les temps et les lieux.\par
Le \emph{crédit}, qu’est-ce autre chose qu’un acte de foi ? Et la \emph{monnaie}, qu’est-ce autre chose aussi qu’un acte de foi, la foi de celui qui, possédant une pièce d’or ou un billet de  \phantomsection
\label{v1p194}banque, \emph{s’attend} fermement à pouvoir l’échanger contre telle marchandise que bon lui semblera — attente déçue s’il ne possède qu’une pièce fausse ? Et, dès lors, le rôle toujours grandissant de la monnaie et du crédit n’atteste-t-il pas que le rôle économique de la croyance grandit sans cesse dans les mêmes proportions ?\par
Il n’est pas inutile de faire remarquer en passant, que non seulement la monnaie consiste dans une croyance, dans une confiance d’une très haute intensité — si intense qu’elle est inconsciente, mais encore qu’elle peut servir de mesure approximative, dans certains cas, à la hausse et à la baisse des croyances quelconques presque aussi bien que des désirs quelconques.\par
J’ajoute qu’il y a d’autres \emph{mètres} de la croyance que la monnaie. Je lis dans un ouvrage de Westermrack, professeur de sociologie à Helsingfors, que la proportion des mariages mixtes — entre personnes appartenant à des cultes différents — tend partout à augmenter : « En Bavière, dit-il, de 1835 à 1840, elle s’élevait à 2,8 p. 100 du nombre total des mariages ; de 1850 à 1860, à 3,60 p. 100 ; de 1860 à 1870, à 4,4 p. 100 ; de 1870 à 1875, à 5,6 p. 100 ; de 1876 à 1877, à 6,6 p. 100. » C’est par des statistiques pareilles qu’on tourne la difficulté de mesurer des quantités psychologiques, telles que la foi religieuse.
\subsubsection[{I.3.c. Action delà conversation. Attente des demandes de la clientèle par le producteur : échelle des degrés de probabilité qu’elle monte ou descend au cours du développement économique.}]{I.3.c. Action delà conversation. Attente des demandes de la clientèle par le producteur : échelle des degrés de probabilité qu’elle monte ou descend au cours du développement économique.}
\noindent Si la genèse mentale, si la transmission sociale des attentes et des besoins, des croyances et des désirs, ne jouait dans les rapports économiques des hommes qu’un rôle décroissant, je comprends que cette considération fût négligée par l’économiste. Mais il n’en est rien. Les croyances, notamment, sont un facteur aussi prépondérant, aussi décisif, aussi capricieux en même temps, dans la cote des  \phantomsection
\label{v1p195}valeurs de Bourse que dans les échanges des sauvages. Ceux-ci obéissent surtout à des idées superstitieuses en attachant un grand prix à des fétiches ou à des \emph{porte-bonheur} insignifiants, apportés par des navigateurs ; mais les spéculateurs à la Bourse ne sont pas moins fascinés par des mensonges de Presse, des nouvelles sensationnelles, des conversations de couloirs.\par
La \emph{conversation} est un sujet qui intéresse éminemment l’économiste. Il n’y a pas un rapport économique entre les hommes qui ne s’accompagne d’un échange de paroles d’abord, de paroles verbales ou de paroles écrites, imprimées, télégraphiées, téléphonées. Même quand un voyageur fait des échanges de produits avec des insulaires dont il ignore la langue, ces trocs n’ont lieu que moyennant des signes et gestes qui sont un langage muet. En outre, ces besoins de production et de consommation, de vente et d’achat, qui viennent de se satisfaire mutuellement par l’échange, conclu grâce à des conversations, comment sont-ils nés ? Le plus souvent, grâce à des conversations encore, qui ont propagé d’un interlocuteur à un autre l’idée d’un nouveau produit à acheter ou à produire, et, avec cette idée, la confiance dans les qualités de ce produit ou dans son prochain débit, le désir enfin de le consommer ou de le fabriquer. Si le public ne causait jamais, l’étalage des marchandises serait peine perdue presque toujours, et les cent mille trompettes de la réclame retentiraient en vain. Si, pendant huit jours seulement, les conversations s’arrêtaient à Paris, on s’en apercevrait vite à la diminution singulière du nombre des ventes dans les magasins. Il n’est donc pas de directeur plus puissant de la consommation, ni, par suite, de facteur plus puissant quoique indirect, de la production, que le babil des individus dans leurs heures de loisir.\par
C’est, au fond, ce jeu perpétuel de langues oisives, ce rapport social élémentaire et fondamental de deux esprits  \phantomsection
\label{v1p196}en contact et en voie de contagion mutuelle qui élabore cette grande souveraine de la vie économique ou politique, l’Opinion, régulatrice des usages et des besoins, des goûts et des mœurs, et, par conséquent de l’industrie. Il me semble donc que la connaissance des progrès ou du déclin de la conversation, de ses causes et de ses effets, et sinon des \emph{lois} — on abuse du mot — au moins des pentes générales et habituelles qu’elle suit dans ses transformations historiques, importe au plus haut degré à l’économiste. Ce n’est pas, à mon gré, en parler assez que de compter vaguement le « progrès des communications » parmi les agents du progrès économique. Il faut préciser, il faut détailler ; il faut dire toutes les conséquences économiques qu’entraîne l’accélération ou le ralentissement, la stagnation ou le déplacement, le grossissement ou l’amincissement du courant de la conversation, suivant que la frontière d’une langue dominante avance ou recule, étendant ou restreignant ainsi le nombre des gens qui peuvent causer ensemble, — suivant que la séparation des diverses classes d’une nation devient plus ou moins profonde, ce qui a les mêmes effets, — suivant que les lieux de réunion et de causerie, cercles, cafés, salons, se multiplient ou se raréfient, et que les loisirs des travailleurs augmentent ou diminuent, — suivant que le pouvoir politique devient plus tyrannique ou plus libéral et que la censure ecclésiastique se raffermit ou se relâche, est de plus en plus ou de moins en moins respectée, ce qui grossit ou rétrécit le cercle des sujets d’entretien et ouvre à la conversation de nouveaux débouchés, source de nouveaux besoins, ou lui ferme des débouchés anciens, au détriment de certains besoins et par suite de certaines branches d’industrie. Si l’on réfléchit à cela, on s’aperçoit que la rivalité des langues, leur lutte pour le triomphe et pour la conquête du monde par l’une d’elles, est, non pas une question linguistique seulement, mais l’un des problèmes économiques les plus importants. Les Anglais savent bien que rien ne contribue  \phantomsection
\label{v1p197}plus efficacement à accroître et consolider leur prépondérance industrielle que l’extension de leur langue dans le monde. Imaginez qu’un de nos grands États européens fût resté morcelé en des centaines de dialectes comme au moyen âge, est-ce que la grande industrie y serait possible ? Non, en dépit de toutes les inventions des machines ; car la grande industrie suppose la similitude des mêmes demandes d’articles dans une vaste région, et une telle assimilation n’est possible en général, que par l’extension d’un même idiome qui s’est substitué ou superposé aux autres et les a relégués au rang de patois. Entre gens ne parlant pas la même langue, la communication des idées et des besoins est toujours entravée, et, quoique beaucoup d’idées et de besoins parviennent à franchir cet obstacle, il n’en est pas moins vrai que cette barrière est insurmontable à la plupart d’entre eux, ou n’est surmontable qu’à la longue.\par
Parmi les \emph{attentes}, dont je parlais tout à l’heure, il en est une qui a une influence capitale et directe sur la production des richesses : c’est l’attente des demandes de la clientèle par le producteur. Il produit toujours ce qu’il espère pouvoir vendre, et dans la mesure où il s’attend à trouver des acheteurs. Ce qui caractérise l’entrepreneur et le distingue de l’ouvrier, c’est la nature très dissemblable et le degré très inégal de leur attente. L’ouvrier ne s’attend pas à la vente, dont il n’a point souci ; il ne s’attend qu’à recevoir son salaire, et d’habitude, il y compte avec une entière certitude. Mais l’entrepreneur, lui, ne peut compter que sur la vente, et, en général, il n’a droit d’y compter qu’avec une probabilité plus ou moins forte, souvent assez faible. C’est ce caractère de non-certitude inhérent à l’attente, aux prévisions du directeur d’une entreprise, et c’est le besoin de certitude exigé par l’attente de l’ouvrier, qui rend si ardue la difficulté de supprimer la différence de l’ouvrier et du patron et de réaliser l’organisation socialiste du travail, ou même tout simplement de fonder des sociétés coopératives \phantomsection
\label{v1p198} de production. La question, au fond, est de savoir si on arrivera jamais, par des statistiques commerciales merveilleusement rapides, sûres et parfaites, et par d’autres moyens d’information, à rendre certaines ou presque certaines, les prévisions, toujours plus ou moins conjecturales à présent, des producteurs, de telle sorte qu’il n’y eût plus de risque couru, ni, par conséquent, plus d’injustice, plus d’inconvénient à supprimer le bénéfice du patron, compensation nécessaire de ses risques actuels. Le jour où la nature et l’étendue des demandes des consommateurs seraient susceptibles d’être ainsi prédites à coup sûr par les producteurs, c’est alors, et seulement alors, que l’État pourrait songer sérieusement à se mettre à leur place, à diriger de haut le travail national centralisé et organisé, ou que, du moins, les ouvriers pourraient revendiquer leur participation aux bénéfices du patron, devenu leur camarade, un camarade plus intelligent et mieux doué, mieux payé comme tel, et comme créateur de l’entreprise, mais non à raison de risques qui n’existeraient plus.\par
Or, est-il impossible que les \emph{attentes} et les prévoyances des producteurs, s’élevant peu à peu sur l’échelle des probabilités, atteignent jamais le sommet ? On ne saurait, à priori, nier la possibilité de cette transformation. En effet, le degré moyen de la probabilité attachée aux prévisions des producteurs a beaucoup varié au cours de l’histoire économique, et en divers sens. Tant que le stade de la petite industrie, de l’industrie locale, n’est pas franchi, l’artisan sait toujours d’avance ce qu’il doit fabriquer en quantité et en qualité, puisqu’il travaille sur commande la plupart du temps ; et, quand il travaille avant commande, sa clientèle lui est personnellement si connue qu’il prévoit avec une quasi-certitude, en qualité et en quantité, ce qui lui convient. Quand le rayon de la clientèle s’étend au delà du cercle des voisins et que le marché s’élargit, l’industriel produit pour le \emph{public}, clientèle anonyme et inconnue ; et, à mesure que  \phantomsection
\label{v1p199}s’opère cet agrandissement de son champ d’action, le degré moyen de la probabilité de ses conjectures va en s’abaissant. Mais jusqu’à un certain point seulement. Et ne semble-t-il pas que déjà, en se prolongeant, cette extension des débouchés ouverts à l’industrie tende, inversement, à rendre de moins en moins incertaine la prévoyance du chef d’industrie, mieux renseigné par des statistiques spéciales, par des journaux quotidiens, par des lettres, par des télégrammes ? Il est donc fort possible que, les moyens d’information devenant plus rapides et plus sûrs pendant que le marché continue à s’étendre, la très grande industrie en arrive un jour à retrouver, par l’excès même de ses audaces, la sécurité complète des humbles et timides artisans primitifs.\par
A priori, donc, cela n’a rien d’inconcevable. Mais, je dois le dire, si je consulte l’expérience, je n’en vois pas moins fort peu de fondement au rêve d’une organisation générale et centrale du travail par l’État. Jamais, sans nul doute, les besoins de l’ensemble des citoyens ne pourront être prédits avec autant de rigueur et de certitude que ceux d’un corps d’armée en marche ; cependant nous savons à quel point est défectueux en temps de campagne le service de l’intendance militaire même la plus parfaite. Il n’est pas de jour où ne se fasse douloureusement sentir tantôt l’excès tantôt le déficit des approvisionnements requis. A fortiori, sous le régime collectiviste, aurait-on journellement à se plaindre de l’intendance civile, dont la tâche serait tout autrement compliquée.\par
Toutefois, on s’est mépris, je crois, quand, à ce propos, on s’est émerveillé sur les effets produits par le régime du laisser-faire et du laisser-passer, grâce auquel une grande capitale comme Paris, une grande nation comme la France, se trouve chaque jour, approvisionnée à peu près comme il convient, pourvue d’une si grande diversité de choses utiles, en quantité à peu près proportionnée à ses besoins divers.  \phantomsection
\label{v1p200}On a semblé voir là une vertu mystérieuse inhérente à la « liberté », c’est-à-dire à l’absence de toute réglementation, de toute coordination autoritaire, consciente et prévoyante. Et on a été porté à supposer parfois que l’approvisionnement s’adapte de lui-même aux besoins, précisément parce que personne ne pense à cette adaptation. La vérité est — et elle est, je l’avoue évidente — qu’on doit faire honneur du résultat signalé à la prévoyance consciente, à l’intelligence coordinatrice des particuliers, des marchands, qui, chacun dans sa sphère étroite, voient assez clair, dirigent et règlent leurs opérations et proportionnent les marchandises offertes aux besoins prévus de leurs clients habituels. Ceci n’a pas échappé à l’observation de Courcelle-Seneuil qui dit quelque part : « Sous l’empire de la liberté, le champ d’action de chaque individu est bien plus restreint que celui du gouvernement dans la région de l’autorité, mais il est exactement déterminé et limité ; chacun peut obtenir une connaissance, sinon complète, du moins satisfaisante, de tout ce qui l’intéresse dans sa petite sphère. » Ajoutons que cette connaissance est, il est vrai, incomplète pour chaque commerçant, mais que, comme leurs erreurs ne sauraient être commises toujours dans le même sens, soit en plus, soit en moins, elles se compensent dans l’ensemble, et cette compensation est d’autant plus exacte que leur nombre est plus grand.\par
Au fond, cela veut dire que ce qu’on appelle liberté n’est que de l’autorité morcelée et disséminée partout. La liberté, en tant qu’elle est une source d’harmonies économiques ou autres, consiste en autorités directrices. Il n’y a pas de direction d’ensemble, il est vrai ; mais là où cette direction d’ensemble, centrale et gouvernementale, ne saurait être que vague et aveugle, fondée sur des calculs trop compliqués pour ne pas donner lieu à de désastreuses erreurs, elle est remplacée avec avantage par une foule de petites directions de détail, séparément clairvoyantes et appuyées  \phantomsection
\label{v1p201}sur des informations justes ou d’une inexactitude compensée. Il en est du talisman magique, dissimulé sous le nom de « lois naturelles », qu’on prête à la liberté économique, comme des vertus non moins mystiques et non moins merveilleuses prêtées à l’\emph{inconscience} par certains sociologues. J’ai montré ailleurs\footnote{ \noindent \emph{Logique sociale}, p. 200 et suiv.
 } que ce qu’on a pris pour de l’inconscience dans le travail harmonisateur des idées et des tendances d’une société, n’est, en dernière analyse, que de la \emph{multi-conscience}, aboutissant par degrés à de la \emph{pluri-conscience} qui est une étape sur le chemin de l’\emph{uni-conscience.} Mais cette phase finale n’est pas toujours atteinte, ni ne peut toujours l’être. Voilà, pourquoi, malgré la tendance manifeste de la grande production industrielle d’à présent à se concentrer en un petit nombre de mains, par l’alliance de monopoleurs gigantesques, il n’est pas sûr, ni probable, que ce petit nombre finisse par se réduire à l’unité de l’État.
 \phantomsection
\label{v1p202}\subsection[{I.4. Les besoins}]{I.4. Les besoins}\phantomsection
\label{l1ch4}
\noindent Les deux chapitres qui précèdent nous ont fait connaître les éléments, désirs et croyances, dont les \emph{besoins} sont la combinaison. Nous savons déjà comment ceux-ci se propagent et s’enracinent. Mais l’importance du sujet nous commande de revenir sur ces deux points.\par
\subsubsection[{I.4.a. Besoins, combinaison de croyances et de désirs. Leur propagation intra nationale et internationale. Lois somptuaires de deux espèces opposées. Besoins virtuels.}]{I.4.a. Besoins, combinaison de croyances et de désirs. Leur propagation intra nationale et internationale. Lois somptuaires de deux espèces opposées. Besoins virtuels.}
\noindent Quelques mots suffisent, relativement à la propagation du besoin. Il y en a deux variétés importantes, et qu’il ne faut pas confondre. Partout et toujours, nous voyons les besoins de consommation, en dépit de tous les obstacles, descendre plus ou moins vite, dans une même nation, des classes supérieures aux classes inférieures, des grandes villes aux petites villes et de celles-ci aux campagnes ; et cette assimilation graduelle de toutes les couches de la population, si elle est la source de bien des révolutions ou des agitations sociales, a pour effet de fortifier l’unité et l’originalité nationale. Mais, partout et toujours, en même temps, les besoins d’un peuple, surtout d’un peuple fort, riche, glorieux, se communiquent ou tendent à se communiquer aux peuples voisins ; et ces importations étrangères, si elles se multiplient, peuvent aller jusqu’à dissoudre la nationalité, que, d’ailleurs, jamais, elles ne renforcent. En revanche, elles développent le commerce international et contribuent à la  \phantomsection
\label{v1p203}paix du monde, ainsi qu’à l’élévation du niveau général de la civilisation.\par
Les causes de ces deux sortes de propagation, l’une \emph{intra}-nationale, l’autre \emph{inter}-nationale, sont aussi différentes que leurs effets. — Les besoins des classes supérieures sont reflétés par les classes inférieures, sous l’empire de mobiles tels que le snobisme ou le sentiment du droit à l’égalité ; les besoins de l’étranger sont copiés par les nationaux, et d’abord par les classes supérieures de la nation, par l’élite de la capitale, en vertu du \emph{philonéisme} qui leur est propre, de cette rage de nouveauté qui les tourmente et qui est stimulée en elles par les satisfactions mêmes qu’elle a déjà reçues. Le plus souvent, cette communication de besoins, d’un peuple à un autre, est mutuelle, alors même que la supériorité d’une nation est reconnue ; tandis que, bien rarement, les capitales accueillent l’exemple des provinces, ou les noblesses, dans une monarchie, l’exemple des classes populaires. Cet échange de besoins, entre nations, précède toujours l’échange de leurs produits.\par
Il est à remarquer que les lois somptuaires, ces digues presque toujours à demi impuissantes par lesquelles les gouvernements passés ont essayé d’arrêter les courants imitatifs, étaient destinées à lutter non contre le penchant à l’introduction des besoins étrangers, peut-être que ce penchant était assez faible jadis, mais contre la tendance à singer les classes supérieures. Il y a deux sortes de lois somptuaires, comme le remarque justement J.-B. Say. Les unes, inspirées par l’esprit démocratique, ont pour but, dans les républiques anciennes, d’empêcher le luxe trop grand des riches, de peur que l’éclat de leur rayonnement n’offusque les yeux du pauvre et ne lui fasse trop cruellement sentir le chagrin de son humilité. Celles-ci sont exceptionnelles et de peu d’intérêt pour nous. Les autres, aristocratiques, ont pour but d’empêcher les plébéiens d’imiter les patriciens, les roturiers d’imiter les gentilshommes. Ces  \phantomsection
\label{v1p204}dernières cherchent à s’opposer à la contagion imitative du luxe des grands pour que la distance ne vienne pas à s’amoindrir entre eux et les petits. Mais tout ce qu’elles ont pu faire a été de retarder le mal qu’elles combattaient et de révéler la force irrésistible, torrentielle, du penchant qu’elles prétendaient endiguer. Aussi les gouvernements, même les plus aristocratiques, ont-ils renoncé à ces mesures de prohibitionnisme intérieur ; et, quant aux gouvernements populaires, ils sont presque partout en train d’agir en un sens précisément opposé à celui qui vient d’être indiqué. Sinon par des lois, du moins par des suggestions de tout genre, par des facilités de transport, notamment, offertes aux ruraux que l’on convie à venir admirer les expositions des villes, les étalages des magasins, à venir convoiter toutes sortes de raffinements de vie qui leur manquent, — les États modernes poussent les classes pauvres à accroître leur confort, à vivre comme les classes plus fortunées ; et, dans leurs colonies, ils s’efforcent de décider les indigènes à consommer des articles européens, ils cherchent à répandre chez des Cochinchinois, chez des Arabes, chez des Hindous, chez des nègres africains, le goût d’étoffes, d’ameublements, de boissons, que les industriels de la métropole leur envoient. En arrivera-t-on à édicter des lois qui seraient précisément le contraire des lois somptuaires, c’est-à-dire qui imposeraient de force aux peuplades de nos colonies l’imitation de notre luxe étranger ? C’est possible. N’est-ce pas un peu ce qu’on a voulu faire en Chine ?\par
La conduite des États modernes à l’égard de leurs colonies est-elle en cela moins tyrannique que l’était, à l’égard des classes inférieures, la conduite inverse des États d’autrefois ? Elle l’est davantage, car il est assurément moins despotique d’interdire certaines consommations que de les rendre obligatoires ou quasi obligatoires. Et qu’est-ce qui est le plus déraisonnable, en soi, d’élever des digues impuissantes contre le torrent de l’imitation niveleuse, ou  \phantomsection
\label{v1p205}de rendre ce torrent plus rapide encore et plus débordant, sans songer aux ravages que peut causer sa précipitation aveugle ? Ce qu’on peut dire en faveur des gouvernements nouveaux, c’est qu’ils semblent agir ainsi dans le sens de l’effort universel, de la grande tendance historique au nivellement démocratique.\par
Le penchant à copier le supérieur ou l’étranger est une force qui existe à l’état latent longtemps avant de se révéler par des actes, et c’est une grande erreur de la juger inexistante aussi longtemps qu’elle reste inexprimée. Mieux que nulle loi somptuaire, la haine qui sépare certaines classes ou certains peuples, lutte contre cette tendance virtuelle. Mieux encore que la haine, l’inintelligence et l’incuriosité sont des préservatifs excellents contre la contagion de l’exemple. On s’explique ainsi la persistance de certaines peuplades sauvages dans leurs vieilles mœurs malgré le contact des nations civilisées qui les touchent sans les attirer. Tels les Australiens à côté des colons anglais, les Peaux-Rouges aux États-Unis, les Arabes nomades à côté des Arabes sédentaires de l’Yémen. « Les Arabes errants, dit J.-B. Say à ce sujet, ont sans cesse sous les yeux le spectacle des Arabes de l’Yémen qui jouissent de plusieurs des agréments de la vie ; ils trouveraient dans l’Arabie de vastes régions où ils pourraient se fixer comme eux, cultiver la terre, trafiquer, amasser des provisions. Il ne leur faudrait pas plus de peine, ils n’auraient pas besoin de plus de courage pour les défendre qu’ils n’en déploient pour attaquer des caravanes. Néanmoins, il ne paraît pas qu’aucune tribu errante se soit jamais fixée. » Je ne sais si J.-B. Say ne généralise pas trop. En tout cas, d’autres écrivains qui ont généralisé encore davantage, et d’après lesquels jamais, en aucun pays, une tribu nomade n’a été séduite par l’attrait de la vie agricole, se sont manifestement trompés. Il est de fait, par exemple, que, chez les Lapons, la vie sédentaire gagne sans cesse du terrain aux dépens de la vie errante.\par
 \phantomsection
\label{v1p206}Même intelligent, même curieux de ce qui le domine ou l’entoure, un groupe d’hommes peut paraître, à tort, exempt de toute velléité d’imitation, aussi longtemps que l’état de ses ressources le met dans l’impossibilité d’imiter ce qu’il admire. L’admiration alors a un faux air d’amour platonique. C’est ainsi que, dans les pays de caste où la prohibition du mariage entre un homme et une femme de castes différentes est absolue, on ne voit jamais l’amour essayer de franchir cette barrière que tout le monde sait insurmontable. Ce n’est pas à dire qu’il n’y ait des affinités naturelles entre bien des jeunes gens séparés de la sorte, qui s’aimeraient certainement si cette barrière venait à s’abaisser. Autant dire qu’il existe en eux un amour virtuel, qui n’attend, pour se montrer, que la levée d’un obstacle. Pareillement, l’admiration du paysan français de Louis XIV pour les beaux vêtements brochés, pour les superbes carrosses, des gentilhommes de la cour qu’il voyait passer sur les grandes routes en bêchant sa terre, pouvait paraître un sentiment purement désintéressé, sans nulle convoitise. Mais, chez le roturier enrichi, qui assistait aux exhibitions du même luxe seigneurial, l’admiration se mêlait déjà d’envie ; et, à mesure que, au cours du {\scshape xviii}\textsuperscript{e} siècle, ce roturier est devenu plus riche, que sa fortune grandissante lui a fait entrevoir la possibilité de se régler sur ce brillant modèle, l’envie s’est, chez lui, substituée à l’admiration, qui n’avait jamais été, d’ailleurs, au fond, que de l’envie en puissance.\par
Il faut avoir égard à ces vérités psychologiques toutes simples si l’on veut comprendre pourquoi il suffit d’un perfectionnement industriel qui abaisse le prix d’une marchandise pour que cet article, jusque-là réservé à un cercle étroit de gens riches, et, en apparence, nullement convoité par les autres classes, se répande avec la plus grande rapidité dans des couches nouvelles et plus vastes de la population. C’est que la demande virtuelle de cet article, à l’insu même de  \phantomsection
\label{v1p207}ceux qui l’achètent maintenant, préexistait en eux à sa demande actuelle.\par
En fait, les économistes, sans s’en apercevoir, postulent à chaque instant cette propagation de désirs latents. Voici, par exemple, une remarque fort juste de M. Hector Denis. « Supposez, dit-il, qu’à un moment donné le coût du transport d’un produit soit de 10 pour une distance de 100 kilomètres. Si l’amélioration des moyens de transport permet de transporter cette marchandise à la même distance avec des dépenses moindres de moitié, le marché s’étendra dans une mesure beaucoup plus considérable que l’économie réalisée. En effet, représentez par un cercle le marché d’un produit qui, au prix de 10, peut parcourir en tout sens une distance de 100 kilomètres, égale au rayon de ce cercle. Si la dépense se réduit de moitié, la marchandise pourra parcourir 200 kilomètres avec la dépense primitive ; le cercle qui représentera le marché de ce produit sera non pas le double, mais le quadruple de la surface du cercle primitif. Voilà ce qui explique l’unification rapide des marchés et la constitution d’une économie internationale... » Fort bien. Encore faut-il que les besoins auxquels le produit répond, se soient propagés dans le cercle plus vaste dont il s’agit ; ce qui a lieu souvent, mais pas toujours. Il s’agit donc, avant tout, pour un industriel avisé, de s’assurer si, en fait, le besoin de l’article en question s’est propagé au delà des limites anciennes de sa consommation ; et, pour le théoricien, il s’agit de se rendre compte des causes qui déterminent la propagation de tel ou tel besoin dans tel ou tel milieu. Problème d’ordre psychologique et d’ordre logique essentiellement.
\subsubsection[{I.4.b. Cycle individuel des besoins, l’habitude, et cycle collectif, la coutume. Comparaison avec le cycle des travaux.}]{I.4.b. Cycle individuel des besoins, l’habitude, et cycle collectif, la coutume. Comparaison avec le cycle des travaux.}
\noindent La propagation de nouveaux besoins d’individu à individu peut n’être qu’une mode passagère, et, dans ce cas, son action  \phantomsection
\label{v1p208}est assez superficielle. Elle n’exerce une influence importante, elle ne détermine l’établissement d’une nouvelle industrie viable et durable que lorsqu’elle fait entrer et pénétrer la nouveauté qu’elle apporte dans le cycle périodique des désirs habituels ou coutumiers. En élargissant ce cycle et le compliquant, elle produit son effet le plus naturel et le plus profond. La rotation de ce cycle est donc l’objet qui doit le plus arrêter notre attention.\par
Le cycle des besoins, comme le cycle des travaux, peut être considéré sous deux formes : sa forme individuelle, l’habitude, et sa forme collective, la coutume. Au début, les deux se confondent presque dans la routine familiale, tous les membres d’une même maison ayant périodiquement à peu près les mêmes besoins. Mais, quand les tribus se sont juxtaposées, puis fusionnées en cités, le cycle individuel se sépare du cycle urbain, consistant en fêtes et rites périodiques, en fonctionnements, réguliers et périodiques aussi, d’institutions religieuses ou politiques ; et la distinction des deux s’approfondit quand le cycle national s’est ajouté en cycle urbain. Or, le cycle des besoins individuels, comme le cycle des besoins collectifs, ne cesse de se compliquer et de s’agrandir au cours d’une civilisation en progrès, et la raison en est que la propagation des besoins mutuellement échangés ne cesse de s’étendre indéfiniment. Cette évolution est très ancienne, elle date des premiers pas de l’humanité ; car c’est en vain que Le Play cherche des tribus closes en soi, garanties contre toute irradiation des exemples extérieurs par la palissade des préjugés traditionnels. Il n’est pas une coutume des peuples les plus routiniers qui n’ait commencé par être une mode importée du dehors. Cette tasse de café que l’Arabe sous sa tente offre à tout visiteur, elle a beau être une tradition héréditaire à présent, elle a été d’abord une innovation. Et le thé a beau être une liqueur nationale en Russie, les Russes boiraient-ils le \emph{samovar} s’ils n’avaient jamais été en contact avec les Tartares et les Chinois ?\par
 \phantomsection
\label{v1p209}Il y aurait à étudier ici les rapports mutuels du cycle des travaux et du cycle des besoins ; l’effet produit sur le premier par la complication, la régularisation et l’abréviation du second ; leur dépendance commune à l’égard du cycle monétaire, dont nous parlerons plus loin ; les causes qui compliquent ou simplifient, abrègent ou ralentissent, chacun de ces cycles. Effleurons seulement ce trop vaste sujet. — Le cycle des besoins, collectifs ou individuels, va se compliquant de plus en plus, s’abrégeant jusqu’à un certain point, et se régularisant de plus en plus. Tout besoin nouveau, en effet, même d’alimentation — par exemple, le besoin de manger de la viande ou du pain blanc, de boire du café, du thé, de l’eau-de-vie — commence, dans les classes pauvres, par être exceptionnel, intermittent, reproduit à intervalles très longs et mal réglés, puis à des intervalles moins longs et déjà réglés, à certains jours de foire ou de fête, le dimanche, enfin tous les jours et deux fois par jour. Le besoin de fumer parcourt les mêmes phases. Le besoin de renouveler ses vêtements commence aussi, chez les primitifs, par être des plus rares ; il ne revient qu’après l’usure complète de l’étoffe ; plus tard, c’est après une période de temps plus ou moins mal définie, une douzaine d’années, deux ou trois ans, enfin tous les ans et à chaque saison. Le besoin de renouveler son mobilier est en train, dans les classes riches, d’évoluer de la sorte. Le besoin de voyager, jadis exception rarissime, se reproduit chez les civilisés d’aujourd’hui à des époques fixes, aux grandes vacances, à Pâques.\par
Le cycle des travaux évolue-t-il de la même manière ? Non. On remarque entre les deux des différences importantes. Le cycle des travaux individuels, élémentaires, où tourne l’ouvrier de plus en plus spécialisé, va se simplifiant pendant que celui de ses besoins va se compliquant. C’est là une anomalie d’où résultent un dégoût croissant du travail et un désir impérieux de le resserrer dans des limites toujours \phantomsection
\label{v1p210} plus étroites. Quant au travail productif total, c’est-à-dire collectif, il va se compliquant et se régularisant, sinon s’abrégeant.\par
Le cycle des travaux d’où résulte un produit donné a bien rarement une durée égale à la période de reproduction du besoin que ce produit satisfait. Et les besoins individuels les plus impérieux, tels que la faim et la soif, ont une rotation bien plus rapide que celle des travaux qui leur correspondent. A des besoins qui se reproduisent deux ou trois fois par jour, il est répondu, en agriculture, et même en industrie, par des travaux dont la rotation est annuelle. De ce défaut de synchronisme est né le salaire. Si le besoin de manger ne se répétait que tous les ans, il n’aurait jamais paru nécessaire de payer tant par jour les ouvriers agricoles ; chaque travailleur aurait pu attendre la récolte et la vente de la récolte pour être payé de ses peines moyennant sa quote-part du prix de vente. — Y a-t-il lieu de penser que, avec le progrès de la civilisation, l’écart entre le cycle des besoins individuels et celui des travaux ira en s’abrégeant, dans l’industrie du moins, et rendra de moins en moins nécessaire la rémunération du travailleur sous la forme de salaire ? Contrairement à cette espérance, il semble que, le cycle des travaux devenant de plus en plus collectif, sa durée s’allonge pendant que celle de la reproduction des besoins individuels reste toujours aussi courte. L’ère du salariat n’est donc pas près de se clore ; les progrès incessants, en tout pays, du fonctionnarisme, ce salariat bourgeois, en sont la preuve. Et l’extension du socialisme d’État n’a-t-il pas pour effet de transformer de plus en plus les ouvriers eux-mêmes en fonctionnaires, en salariés du gouvernement ?\par
Nous avons dit que nous entendions par \emph{besoin} une combinaison de désir et de jugement. Nous avons besoin d’un article, quand nous \emph{désirons} l’exemption d’un certain mal ou l’acquisition d’un certain bien et que nous \emph{croyons} cet article propre à nous faire atteindre ce but. Observons à ce sujet  \phantomsection
\label{v1p211}que tous nos désirs sont intermittents et périodiques, mais que nos croyances ne le sont pas. Elles sont continues, quoiqu’elles ne soient pas toujours conscientes ; en tout cas, si elles se réveillent après s’être endormies, ce n’est pas à intervalles réglés. Il n’y a pas une périodicité quotidienne, hebdomadaire ou annuelle des idées et des jugements. Cette différence a une importance économique. En effet, la faculté de prévoyance, d’emmagasinement, de capitalisation, dérive de là. Si, une fois un désir satisfait, momentanément disparu, le jugement porté sur l’efficacité de la marchandise naguère désirée disparaissait aussi, s’éclipsait comme ce désir lui-même et ne renaissait qu’avec lui, nous ne ferions aucun cas des choses utiles dans l’intervalle de leur consommation. C’est peut-être parce que, chez le sauvage, la croyance est serve du désir, et n’en est qu’un accessoire et une dépendance, que, le fruit mangé, il coupe l’arbre. L’indépendance croissante de l’idée, du jugement, relativement au désir, est la condition indispensable des progrès économiques, aussi bien que de tout autre progrès social. Et la persistance des croyances, non moins que l’intermittence périodique des désirs, est le postulat de l’Économie politique.
\subsubsection[{I.4.c. Budgets des familles et des États, où cette périodicité s’exprime. Régularité croissante des revenus et des dépenses. Dépenses accidentelles qui deviennent régulières. Pourquoi les budgets sont annuels. Pourquoi ils vont grossissant. Budgets d’ouvriers.}]{I.4.c. Budgets des familles et des États, où cette périodicité s’exprime. Régularité croissante des revenus et des dépenses. Dépenses accidentelles qui deviennent régulières. Pourquoi les budgets sont annuels. Pourquoi ils vont grossissant. Budgets d’ouvriers.}
\noindent La périodicité des travaux et des besoins, soit individuels, soit collectifs, se peint avec une précision mathématique dans les budgets des familles, des sociétés, des États. Aussi n’est-ce pas à tort que Le Play a fondé sur l’analyse des budgets domestiques son étude des divers états sociaux. Il y a à distinguer dans un budget, privé ou public, les \emph{revenus} et les \emph{dépenses.} Les revenus (expression très heureuse donnée à des ressources qui reviennent périodiquement) proviennent de travaux exécutés soit par celui qui perçoit les revenus, soit par autrui ; les dépenses ont trait à la satisfaction \phantomsection
\label{v1p212} des besoins. (Par là, nous sommes conduits à nous occuper un peu, par anticipation, du \emph{cycle monétaire} sauf à y revenir plus tard.)\par
Il en est des revenus comme des sources. Rien de plus irrégulier ni de plus capricieux, en apparence, que les pluies, mais rien n’est plus régulier que le débit des sources, qu’elles alimentent. Elles grossissent au printemps, diminuent en été avec une régularité assez grande. De même, la caisse d’un négociant est alimentée par beaucoup de petits bénéfices variables, mais, somme toute, s’élève ou s’abaisse d’après les saisons, avec des alternatives prévues d’avance ; et, d’une année à l’autre, le total de ses recettes ne varie guère ou ne change que suivant une même courbe générale d’ascension ou de déclin. Dans toutes les carrières, l’aspiration générale est d’avoir, autant que possible, des revenus assurés, réguliers dans leur débit. La grande différence, à cet égard, entre la barbarie et la civilisation, c’est que les barbares n’ont que des revenus très incertains, très inégaux d’une année à l’autre. Il semble que, d’un individu à l’autre, alors — et surtout en remontant plus haut, à l’époque sauvage — l’inégalité des revenus soit moindre qu’aux âges de civilisation riche et prospère, mais qu’elle soit plus grande d’un temps à l’autre. La civilisation a pour effet d’inégaliser et de différencier, mais de régulariser et d’assurer, en moyenne, les revenus, aussi bien privés que publics. A quelques millions près, c’est-à-dire à quelques \emph{millièmes} près, le budget des recettes de nos États contemporains peut être prévu plusieurs années d’avance. Voilà ce qui eût grandement surpris un roi mérovingien, et même un Louis XIV !\par
A l’époque chasseresse, il n’y a pas, à proprement parler, de revenus. On vit au jour le jour, sans monnaie, sans provisions. A l’époque pastorale, le croît et le lait des troupeaux constituent déjà un revenu, que, malheureusement, les sécheresses et les épizooties fréquentes rendent des plus  \phantomsection
\label{v1p213}mal assurés et des plus variables. L’âge agricole est sujet aussi à de redoutables incertitudes, mais moindres, à cause de la variété plus grande des sources de revenus, qui souvent se compensent. Enfin, l’âge industriel a eu pour effet, dans son ensemble, de rendre les revenus plus sûrs et plus réguliers, malgré le caractère aléatoire de la vente des produits. Déjà la petite industrie, qui commence par être un accessoire et un complément de l’agriculture, tend à régulariser les recettes du paysan-artisan ; et, quant à la grande industrie, elle vise à obtenir, et elle obtient de plus en plus, par l’ampleur croissante de ses débouchés, la régularité des bénéfices que le petit artisan doit à l’étroitesse nettement circonscrite de sa clientèle personnellement connue. L’industrialisme a pour accompagnement nécessaire un accroissement général de la richesse publique, qui multiplie les placements de capitaux à intérêts fixes, sources de revenus fixes, et qui se traduit par un rendement plus régulier ou plus régulièrement progressif des impôts, par des fonctions publiques plus nombreuses, plus régulièrement rétribuées. En outre, les compagnies d’assurance de tout genre, les caisses de retraite, les caisses d’épargne, tout ce qui assure et précise les revenus futurs, n’est possible que par le développement de l’industrie.\par
Beaucoup de carrières, libérales ou autres, commencent par donner des revenus très indéterminés et très variables (la médecine, le barreau, les entreprises de maçonnerie, etc.). Et il est beaucoup de jeunes gens que cet aléa séduit et enfièvre. Mais cette effervescence se calme vite, et bientôt ils aspirent à une fixité de gains périodiques. Ils y parviennent en avançant dans l’exercice de leur profession. Car les médecins, les avocats, les architectes, etc., à mesure que leur clientèle se grossit, voient leurs revenus se préciser ; et la plupart ont soin d’économiser, dans leurs années de prospérité, pour s’assurer plus tard des ressources fixes, équivalent de la retraite des fonctionnaires.\par
 \phantomsection
\label{v1p214}Les budgets des familles ont toujours été annuels. Ceux des États ne l’ont pas toujours été. Aux Pays-Bas, de 1815 à 1830, le budget a été décennal. Dans quelques États secondaires de l’Allemagne, il est encore triennal ou biennal. M. de Bismarck aurait bien voulu que le budget impérial fût biennal, et il a obtenu que le chapitre militaire du budget de l’Allemagne fût septennal. Mais, de plus en plus, la périodicité annuelle prévaut pour les budgets publics comme pour les budgets privés. Il est remarquable, que les plus petits comme les plus gros budgets, les plus différents à tous autres égards, se ressemblent sur ce point. Il y a cependant certains articles des budgets publics qui reviennent à plus longue échéance. La périodicité des Expositions universelles et des frais qu’elles occasionnent tend à être vaguement décennale, comme celle des crises commerciales. Mais, à ces exceptions près, l’annualité est la règle générale des budgets, comme des statistiques de tout genre. Si l’on en cherche la raison, on verra que cette période des budgets leur est imposée moins par le cycle des travaux que par la période générale des besoins, et que celle-ci est sous la dépendance de la périodicité astronomique. Si l’homme habitait une planète dont la rotation sur elle-même s’accomplît précisément dans le même temps que sa rotation autour du soleil — comme fait la lune qui tourne sur elle-même et autour de la terre dans le même espace de temps — la lumière solaire éclairerait constamment, sans discontinuité, la surface lumineuse de cette planète, la seule partie de sa surface où la vie pût se développer. Il n’y aurait point de nuit, point de différence des saisons, point d’année, ni de jours à vrai dire. Dans cette hypothèse, l’idée d’un budget pourrait-elle naître ? Non. Car cette idée implique essentiellement, d’après la notion que nous en avons, une périodicité non artificielle de recettes et de dépenses, c’est-à-dire de besoins à satisfaire. Or, dans cette hypothèse, il n’y aurait rien de naturellement périodique dans les besoins. Chacun  \phantomsection
\label{v1p215}d’eux, pris à part, le besoin de vêtement, par exemple, pourrait se répéter à des intervalles de temps quelconques. On n’aurait pas besoin d’un costume par saison, puisqu’il n’y aurait pas de saisons, mais on porterait son vêtement un certain temps, variable d’après les caprices de la mode, qui ne seraient contenus dans aucune limite d’origine naturelle, comme ils le sont sur notre planète par le cycle de l’année.\par
Quant à la périodicité des travaux, c’est seulement en agriculture qu’elle est naturellement condamnée à tourner dans le cercle de l’année. Pour le paysan, il y a une chaîne sans fin d’occupations variées qui se reproduisent les mêmes tous les ans. Pour l’industriel, il n’existe qu’une production plus ou moins abondante du même produit, d’après les saisons. Ce n’est qu’une périodicité \emph{quantitative}, et non à la fois \emph{qualitative} et \emph{quantitative} comme celle de l’agriculture. Encore faut-il remarquer que, plus une industrie progresse, moins cette périodicité quantitative est apparente ; et, dans les usines les plus importantes, on travaille avec une activité à peu près toujours égale. Si une certaine oscillation annuelle subsiste néanmoins dans la hausse ou la baisse alternatives de la production industrielle, c’est à contre-cœur que les fabricants la subissent, parce qu’ils y sont contraints soit quelquefois par le lien de leurs travaux avec les travaux agricoles\footnote{ \noindent Les suspensions périodiques des travaux \emph{industriels}, dans les industries naissantes, dérivent souvent de la périodicité des travaux agricoles. Par exemple, jusqu’au commencement de ce siècle dans le Norfolk, « le filage se pratiquait chez les paysans, ce qui obligeait les tisserands des villes à arrêter leurs métiers pendant la moisson, afin de permettre aux fileurs de vaquer aux travaux des champs. Ces interruptions périodiques du travail furent même consacrées en 1662 par une loi dont le texte disait : « Attendu qu’on a maintenu l’usage depuis un temps immémorial et trouvé convenable de cesser le tissage chaque année en temps de moisson eu égard aux fileurs dont les tisserands emploient le fil et qui, à cette époque, sont généralement occupés à la moisson, aucun tisserand ne travaillera à son métier du 15 août au 15 sept. sous peine d’une amende de 40 sch. par métier. » (Laurent Duchesne, \emph{Industrie de la laine en Angleterre}, 1900.) En se développant, l’industrie de la laine a supprimé ces entraves périodiques.
 }, soit, plus souvent, par le cycle annuel des besoins. Ils auraient intérêt à ce que la période des besoins  \phantomsection
\label{v1p216}ne fût point réglée, dans son ensemble, par celle des mouvements astronomiques, mais qu’elle se raccourcît sans cesse, ce qui rendrait la demande des mêmes articles plus fréquente. Rien ne leur serait plus facile que de répondre, par une rotation plus rapide des travaux, à cette rotation accélérée des besoins de la consommation.\par
Ainsi, par ses besoins, bien plus que par ses travaux, l’homme persiste dans sa subordination à l’égard de la nature extérieure, à l’égard du soleil notamment dont les rayons le fouettent périodiquement comme s’il était une toupie qu’ils font tourner. Comme travailleur, comme producteur, il s’affranchit peu à peu, jusqu’à un certain point, de cette servitude ; comme consommateur, il y demeure davantage assujetti. Mais cela ne veut pas dire que son assujettissement soit toujours de même nature, ni de même degré. La servitude du primitif à l’égard du cycle des saisons pendant l’année, à l’égard du cycle des heures pendant la journée, est comparable à l’esclavage antique ; elle est sans réciprocité aucune. Il ne peut jamais faire de la nuit le jour, ni de l’hiver l’été. Mais le civilisé, s’il reste, dans une certaine mesure, dépendant des variations astronomiques et atmosphériques, diurnes ou annuelles, et de leurs effets sur la vie des animaux ou des végétaux, les utilise de plus en plus, les fait servir de plus en plus à ses fins. En sorte que la conformité de la période de ses besoins à la période des changements physiques dont il s’agit est, chez le civilisé, une harmonisation plutôt qu’une subordination. En combattant avec succès le froid de l’hiver et l’obscurité de la nuit, l’homme riche et cultivé de nos jours s’affranchit vraiment du joug de la saison glacée et des ténèbres nocturnes, et l’idée de l’hiver ou de la nuit ne lui rappelle plus que les plaisirs spéciaux qu’ils lui ramènent (patinage, théâtres, soirées), comme l’idée de l’été est liée pour lui à celles d’autres agréments caractéristiques (voyages, villégiatures). Le soleil est son conseiller plutôt que son maître.\par
 \phantomsection
\label{v1p217}En tout budget, soit de dépenses, soit de recettes, il y a la part de l’imprévu, c’est-à-dire du non-périodique\footnote{ \noindent Ici se reflète, sous forme budgétaire, la distinction, faite dans un chapitre antérieur, entre les courbes fermées et les \emph{courbes ouvertes} de nos désirs...
 }. Vous gagnez un lot à la loterie, une forte somme au jeu ; un bœuf meurt dans votre étable, un mur de votre maison s’écroule, vous faites un voyage exceptionnel. C’est là le \emph{casuel} du passif et de l’actif. Il en est de même pour le budget de l’État. Mais cette part du non-périodique budgétaire va-t-elle en augmentant ? Non. L’accidentel même, en se multipliant, se régularise et laisse apercevoir l’action de causes périodiques. Dans le budget de l’État, la part afférente aux frais de justice criminelle est à peu près la même d’une année à l’autre, pour la même saison, et différente d’une saison à l’autre, quoique chaque crime pris à part semble étranger à toute influence, même indirecte, des changements atmosphériques. Un particulier très riche inscrit, pour les cas imprévus, une somme totale qu’il peut prévoir, et qui, de fait, ne varie guère.\par
Si l’on compare les vieux budgets aux budgets nouveaux, dans une même classe sociale, on sera frappé de cette remarque, que nulle dépense inscrite jadis comme périodique n’y devient imprévue et accidentelle, mais que beaucoup de dépenses, regardées jadis comme exceptionnelles, y prennent rang parmi les dépenses annuelles, hebdomadaires ou quotidiennes. Les seules dépenses périodiquement réglées ont été d’abord les dépenses alimentaires, qui ne l’étaient pas, du reste, à l’époque chasseresse, où la régularité des repas était inconnu. Les dépenses relatives aux vêtements ont été longtemps, très longtemps, non périodiques comme il résulte de ce qui a été dit plus haut à cet égard. Dans les classes pauvres, parmi beaucoup de familles de paysans encore, un vêtement est fait pour durer indéfiniment ; autrefois, il en était de même dans les classes les plus riches,  \phantomsection
\label{v1p218}où le vêtement était considéré comme quelque chose d’acheté une fois pour toutes. Un vestige de ces anciennes idées subsiste dans l’usage d’offrir la \emph{corbeille} du mariage et dans l’apport du \emph{trousseau.} La corbeille et le trousseau contenaient primitivement tout ce que devait posséder la jeune mariée, en fait de linge et en fait de robes, jusqu’à la fin de ses jours. — A présent, ce n’est plus le vêtement seulement, c’est le mobilier à son tour qui tend à être renouvelé périodiquement, à des intervalles réglés, de plus en plus abrégés ; on voit venir, pour cette dépense aussi, dans les familles les plus riches, la périodicité annuelle.\par
Il y aurait ici à considérer en détail, soit pour les individus et les familles, soit pour les associations, soit pour les États, bien des côtés différents de la rotation budgétaire : en premier lieu, la grandeur du cercle tournant, c’est-à-dire le nombre et l’intensité des éléments dont il se compose, sources de revenus et besoins à satisfaire ; en second lieu, la nature des éléments de ce cercle ; en troisième lieu, la vitesse de rotation de ce cercle, vitesse très inégale et très changeante malgré la similitude et la persistance de la vitesse moyenne, annuelle ou diurne, des différents cercles, comme nous venons de le voir ; en quatrième lieu, la ressemblance plus ou moins accusée, en grandeur et en nature, des cercles où tournent les diverses familles, les diverses associations, les divers États, dans une région donnée, et l’inégale contagion assimilatrice exercée par tels ou tels types de budgets, privés ou publics, qui reflètent ces divers cercles de travaux et de besoins. Dans une région donnée, à une époque donnée, il y a toujours un type de budget dominant, qui se diffuse et se répand de plus en plus loin, soit parmi les familles, soit parmi les États. Et nous voyons très bien, quand il s’agit du budget des États, à quelles causes, à quelles influences, est due cette contagion internationale ; pourquoi, par exemple, le chapitre des travaux publics ou de l’instruction publique, ou de l’armée, a prodigieusement \phantomsection
\label{v1p219} enflé dans les budgets de nos États contemporains.\par
Cette étude nous entraînerait fort loin : il y aurait, à propos du budget des recettes de l’État, à tracer, dans ses grandes lignes historiques, l’évolution comparée de l’impôt, de ses diverses formes si multiples, pour en dégager certaines constatations générales ; et il y aurait aussi à esquisser de même l’évolution comparée des dépenses publiques. Un travail pareil s’imposerait à nous en ce qui concerne les budgets des particuliers ; les sources différentes et changeantes de leurs revenus, les natures différentes et changeantes de leurs dépenses nous feraient apercevoir les causes générales de ces variations. Le rapprochement entre les budgets des États et les budgets des particuliers pourrait être encore plus instructif pour la bonne intelligence de ceux-ci que pour celle de ceux-là, où, à la lumière des discussions parlementaires, nous voyons clairement quelles sont les causes psychologiques, inter-psychologiques (fermentation contagieuse de certaines idées lancées par certains publicistes, diffusion de certaines habitudes, devenus des besoins impérieux, etc.), qui font inscrire un nouvel article au budget de l’État ou grossissent la dotation de tel autre chapitre. Nous y voyons aussi que nul besoin nouveau ne prend place au budget sans avoir lutté contre des besoins rivaux qu’il refoule et qui le repoussent ; et les débats des Chambres nous révèlent de la sorte ce qui se passe, au sein de chaque famille, sans qu’on s’en aperçoive, chaque fois qu’une nouvelle habitude contractée oblige à une nouvelle dépense au préjudice des anciennes.\par
Sans entreprendre, pour le moment, une recherche si considérable, bornons-nous à cette généralité, qui saute aux yeux, que tous les cycles de besoin et de budgets, privés ou publics, — publics parce que privés — ont une \emph{tendance} commune à s’élargir sans cesse. Ils ne la réalisent pas toujours, parfois ils sont forcés de se rétrécir ; mais quand, à la suite d’une \emph{inflation} prolongée du budget, public ou  \phantomsection
\label{v1p220}privé, un krach se produit, une chute brusque, tout le monde voit dans ce phénomène une maladie, une anomalie, tandis que le grossissement paisible et régulier qui l’avait précédé passait pour un développement sain et normal. Il y a quelque chose d’irrésistible, en effet, dans l’entraînement mutuel qui pousse les particuliers, les villes, les États, à dépenser de plus en plus, et à se créer, pour répondre à leurs besoins croissants, des ressources toujours plus élevées. Dans le court intervalle de 1891 à 1899 (que serait-ce si nous prenions notre point de comparaison plus haut ?), le budget des recettes ordinaires de la Ville de Paris a passé de 264 millions à 304 ; et celui des autres villes de France s’est accrû de 410 millions à 459. L’accroissement parallèle des dépenses a été au moins aussi rapide. Or, comme le remarque à ce sujet M. Paul Leroy-Beaulieu, on peut s’expliquer, en ce qui concerne Paris, ce gonflement des budgets par la courbe ascendante de sa population ; mais cette excuse ne saurait être alléguée pour l’ensemble des autres communes de France, dont la population moyenne n’a pas grandi. C’est donc seulement la contagion de l’exemple, soit de l’exemple des villes qui se copient les unes les autres, rivalisent les unes avec les autres, soit de l’exemple des particuliers qui, dans chacune d’elles, se communiquent les unes aux autres leurs formes nouvelles de besoin et leurs modes nouveaux d’activité, qui peut expliquer cette rapide progression de leurs dépenses et de leurs recettes.\par
La comparaison des budgets d’ouvriers est très malaisée, car elle donne généralement lieu à peu d’écritures. Ce qu’on peut assurer, c’est que la proportion pour laquelle figurent, dans ces budgets, les dépenses d’alimentation, toujours les plus fortes, va en diminuant : pas aussi rapidement cependant qu’on pourrait le souhaiter. Par exemple, je lis, dans la \emph{Théorie du salaire} de M. Beauregard, un curieux document, d’où il résulte que, en 1764, à Abbeville, un ouvrier  \phantomsection
\label{v1p221}tisseur dépensait environ 70 p. 100 de ses dépenses totales pour sa nourriture et celle de sa famille ; et je vois que, en 1880, d’après une enquête faite à la Société industrielle de Mulhouse, cette proportion n’est plus que de 61 p. 100, chiffre qui coïncide presque avec un résultat (63 p. 100) déduit d’autres recherches contemporaines relatives à d’autres familles ouvrières. Cette différence de quelques unités n’est pas énorme ; elle montre toutefois que l’ouvrier consacre une partie de moins en moins minime de son gain à s’instruire ou se récréer. Il faut s’en réjouir, car c’est là une des plus sûres conditions de paix sociale. — Plus le travail devient, non pas fatigant, mais fastidieux par sa monotonie — réelle ou jugée telle, à raison de la culture de l’ouvrier, — et plus il importe d’accroître le chapitre des récréations et des plaisirs dans le budget des familles ouvrières. Ce chapitre des récréations nous apparaît singulièrement écourté dans les familles demi-barbares dont Le Play, dans son livre sur les \emph{Ouvriers européens}, nous vante l’organisation patriarcale ; peut-être en est-il ainsi, parce que le travail de ces artisans est par lui-même assez récréatif, ou semble tel à leur simplicité. Et, si ce chapitre est court, il est drôlement rempli : eau-de-vie, visites au cimetière, etc. Chez les Arabes de Turquie, par exception, je vois figurer, au nombre des récréations, les contes débités par une sorte de trouvère rustique : c’est un des rares côtés où l’art apparaisse dans ces existences primitives. — Le mérite de Le Play a été de voir que la \emph{question sociale} s’était posée aux peuples les plus barbares comme à nous-mêmes, et qu’elle y avait été résolue depuis longtemps ; mais son erreur a été de penser que ces solutions, appropriées à ces phases inférieures de la vie de société, peuvent suffire aux peuples civilisés de nos jours. Les \emph{récréations} qu’il indique sont devenues manifestement insuffisantes pour rendre heureux l’ouvrier moderne, et notre paix sociale, à nous, réclame des réponses plus compliquées au problème social.
 \phantomsection
\label{v1p222}\subsection[{I.5. Les travaux}]{I.5. Les travaux}\phantomsection
\label{l1ch5}
\subsubsection[{I.5.a. Définition du travail. Le travail et l’invention.}]{I.5.a. Définition du travail. Le travail et l’invention.}
\noindent Qu’est-ce que le travail ? C’est étendre un peu abusivement le sens de ce mot que d’y comprendre toute dépense de force humaine en vue d’un but. Précisons. Tout travail, d’abord, implique un but, des moyens et des obstacles. Mais une partie de jeu, de jeu de cartes ou de jeu de paume, suppose aussi un but, des moyens et des obstacles. Il en est de même d’une série d’opérations militaires. Confondrons-nous cependant sous le même terme de \emph{travaux}, ces formes si différentes d’activité ? Non, ce qui caractérise le travail, au sens économique du mot, et le différencie du jeu, c’est que le but qu’il poursuit est la production d’une richesse propre à satisfaire un désir soit d’autrui, soit du travailleur lui-même, désir autre que le désir même d’accomplir ce qu’il fait. Quand je joue à la balle, je ne produis rien qui satisfasse un autre désir que celui même de jouer à la balle. Autre différence, qui dérive de la première : les \emph{obstacles} que le joueur tâche de vaincre par les moyens qu’il emploie sont des obstacles qu’il se crée lui-même pour avoir le plaisir de les surmonter. Mais les obstacles qui résistent au travailleur lui sont opposés contre son gré, de même que son but lui est imposé, soit par l’ordre ou le désir d’autrui, soit par son ordre et son désir personnel.\par
Ce n’est pas pour les mêmes motifs, il s’en faut, que les opérations militaires diffèrent des travaux industriels, des travaux proprement dits. Certes, les \emph{obstacles} que les soldats \phantomsection
\label{v1p223} ont à vaincre sont bien sérieux, et le but qu’ils poursuivent n’a rien de capricieux ni d’arbitraire. Mais ce but consiste à détruire encore plus qu’à produire. La destruction des forces de l’ennemi est la fin directe des efforts de l’armée en campagne ; et, quant à la production de gloire ou d’influence ou même d’activité entreprenante et enrichissante qui s’ensuivra, elle a une valeur très haute à la vérité, elle ouvre souvent au peuple victorieux des débouchés immenses et le dédommage largement des sacrifices qu’il a faits sur le champ de bataille. Aussi, n’est-ce pas du tout parce que je considère ces efforts belliqueux comme improductifs que je leur conteste le nom de travaux industriels ; car il n’en est pas de plus productifs, même en fait de richesses, quand ils sont couronnés de succès. Loin d’être en raison inverse du militarisme, comme le veut Spencer, l’industrialisme lui est en général proportionnel. Mais la production de richesses par la guerre, outre qu’elle n’est qu’indirecte et obtenue moyennant une destruction nécessaire et constante de richesses, a deux caractères qui la distinguent de la production de richesses par le travail : elle est aléatoire au plus haut degré, tandis que celle-ci est relativement assurée ; et elle est sans proportion aucune avec l’intensité de l’effort belliqueux, tantôt très abondante quand il a été très court et très faible, tantôt très petite quand il a été énorme, tandis que la production par le travail est toujours dans un certain rapport approximatif avec l’effort laborieux. — Encore y a-t-il lieu de mettre à part ici les opérations militaires en temps de paix, les manœuvres qui tendent à faire l’éducation et l’apprentissage du soldat. Ce sont là de vrais travaux qui atteignent toujours leur but, sans nul aléa, et qui produisent un résultat mesuré à leur étendue, à leur énergie et à leur durée.\par
Le prétendu travail que coûte une victoire, ne pourrait-il pas être assimilé au travail, non moins improprement nommé ainsi, que coûte une invention ? Le travail proprement \phantomsection
\label{v1p224} dit suppose la certitude de la production, mais l’efficacité des opérations militaires, comme celle des recherches du savant ou de l’ingénieur acharné à la solution d’un problème de mécanique est essentiellement incertain. Quand, au moment décisif, sur un champ de bataille, un coup d’œil juste du général fait pencher d’un côté la victoire hésitante, c’est à cette idée subite qu’elle est due, non à l’accumulation des efforts antérieurs. Et quand, sur mille chercheurs, un seul, par une intuition soudaine, découvre le mot de l’énigme posée à tous, ce n’est pas aux longs et stériles efforts des autres, ce n’est pas même à la longueur et à l’intensité des siens propres — souvent moindres que les leurs — qu’il convient d’attribuer le mérite de la découverte.\par
\emph{Reproduire}, et non produire, au fond, est l’effet propre du travail. Le travail n’atteint sûrement son but, il n’est sûr de faire ce qu’il a en vue, que parce que ce qu’il a en vue est un service ou un article d’un type déjà connu, un modèle qu’il cherche à imiter, et aussi parce qu’il emploie pour réaliser cette copie des procédés déjà connus, qu’il s’agit de rééditer pour vaincre des obstacles connus eux-mêmes et déjà vaincus par ces moyens. Tout est imitation et reproduction dans le travail économique.\par
Ce n’est pas qu’un \emph{effort de recherches} cesse d’être un travail. Tout travail, à vrai dire, — sauf peut-être le travail de surveillance et de direction des machines dans certains cas, — est une suite de petits problèmes tour à tour posés à lui-même et résolus par le travailleur. A chaque instant, le maçon qui bâtit un mur de moellons se demande comment il comblera le vide qui s’offre à lui et cherche une pierre propre à le remplir ; à chaque instant, un écolier qui fait un thème ou une version cherche une expression latine ou française adaptée à sa pensée. Mais ici ce n’est pas du nouveau qui est cherché, c’est un but mille fois visé qui est poursuivi par des procédés employés des milliers de fois.  \phantomsection
\label{v1p225}Tout autre est l’effort qui tend à l’inconnu. Rechercher du nouveau n’est pas travailler, si ce n’est en tant que la recherche consiste en actes connus dont l’enchaînement seul et l’orientation sont chose nouvelle.\par
La finalité est essentielle au travail. Quand, à force de répéter un même acte, le travailleur l’opère mécaniquement, presque en dormant, quand il n’a plus conscience du but ni du moyen, ni de l’adaptation du second au premier, et que la reproduction des actes devient automatique, il y a fonction vitale, travail vital si l’on veut, mais il n’y a plus travail au sens psychologique et social du mot. A l’inverse, quand, à tâtons, on poursuit très consciemment un but sans savoir par quels moyens, il n’y a pas, non plus, de travail. Le travail est donc une forme d’activité intermédiaire entre la routine de l’automate et l’innovation du génie.\par
La prière est-elle un travail ? Oui, \emph{pour le croyant.} Il y a là un but précis, plaire à la divinité, produire des richesses spirituelles toujours proportionnées à la durée et à la ferveur de l’oraison ; un obstacle à vaincre, les distractions, les tentations de la chair ; un moyen connu, la formule sacramentelle qu’on récite. L’accomplissement de tout rite religieux compte, aux yeux des fidèles, parmi les travaux les plus féconds et les plus indispensables au salut privé ou public. Aussi n’y a-t-il rien de plus parfaitement imitatif, de plus exactement conformiste que les pratiques de la piété, en toute religion. Quand il en est autrement, ce qu’on fait n’est plus considéré comme un travail, mais comme une récompense divine. Saint François d’Assise travaillait quand il récitait son rosaire, quand il disait sa messe ; mais quand, devant un beau paysage, un élan de son lyrisme mystique l’emportait et qu’il épanchait la joie de son âme en cantiques inouïs et improvisés, il ne travaillait pas ; au contraire, il se délassait par ces ravissements de ses labeurs quotidiens.\par
Je viens de séparer avec toute la netteté possible le travail \phantomsection
\label{v1p226} et l’invention. Je dois ajouter cependant que, dans la réalité des faits, ils sont intimement mêlés, à des doses, il est vrai, prodigieusement inégales. Il y a deux parts à faire dans l’activité de l’artisan le plus routinier : une part, de beaucoup la plus considérable, de reproduction imitative, et une toute petite part d’ingéniosité, qui sert de ferment et de condiment à la première et lui donne sa saveur spéciale. C’est par où la main-d’œuvre mérite toujours d’être payée plus cher que la fabrication mécanique. — Inversement, il n’y a pas d’œuvre géniale qui n’ait sa part d’imitation, et dans la facture du plus grand artiste on reconnaît un mélange de ce qui lui appartient en propre avec ce qui lui a été appris par ses maîtres ou ses rivaux.\par
Appellerons-nous travail le fonctionnement normal des organes d’un être vivant, et d’abord des membres d’un animal ? La seule espèce humaine, avec ses outils variés, sortes de membres d’emprunt, exécute toutes sortes de travaux différents qui rappellent les besognes séparément accomplies par d’innombrables espèces animales, chacune avec ses membres spéciaux, comme le remarque Louis Bourdeau. Aussi bien et mieux que la dent des rongeurs, sa scie et son ciseau coupent le bois ; mieux que le bec du pic, sa tarière, sa vrille, son vilebrequin percent et trouent un arbre. Mieux que la queue du castor, sa truelle applique le mortier. Mieux que le ver à soie et l’araignée, il file et tisse avec ses métiers à filer et à tisser. — Encore une fois, peut-on donner le nom de travail à la série d’actions similaires par lesquelles un oiseau parvient à faire son nid ou une araignée sa toile ? Il y a ici un but, des moyens et des obstacles, tout comme lorsqu’un maçon fait une maison ou qu’un tisserand fabrique une pièce de drap ; et ce but, comme ici, est un but sérieux, imposé à l’animal, non choisi par jeu sans autre motif que de s’amuser ; il est conscient, non automatique ; il vise non la production d’une œuvre nouvelle, mais la reproduction d’une œuvre ancienne,  \phantomsection
\label{v1p227}d’un type consacré. Et ces obstacles sont sérieux aussi et connus. Et ces moyens sont des actes conformes à une chaîne infinie d’actes tout semblables accomplis soit par l’individu lui-même dans son passé, soit par les générations antérieures. — La seule différence entre ces travaux animaux et les travaux humains, c’est que les premiers consistent en actes et en buts qui se répètent par hérédité principalement, par une impulsion instinctive, innée, secondairement par imitation, tandis que les autres consistent en actes et en buts qui se répètent par imitation avant tout. Mais, si l’on se rappelle, avec nous, que l’hérédité est l’équivalent vital de l’imitation, on reconnaîtra que cette différence confirme plutôt qu’elle n’infirme les analogies précédentes. Parler des travaux de la vie est donc très légitime. — En poussant plus loin la comparaison, on peut appliquer au monde vivant la distinction du travail et de l’invention. Il y a certainement, à l’origine de tout nouveau travail vital, c’est-à-dire de toute nouvelle fonction organique, et aussi bien de tout individu nouveau, une invention ou une accumulation d’inventions vitales, si l’on peut qualifier ainsi, par métaphore, le fait mystérieux, inexpliqué, qui suscite toute variation importante ou insignifiante d’un type spécifique et prépare ou décide une refonte organique. Ce fait mystérieux est la fécondation, la rencontre de deux lignées qui se croisent et suscitent en se combinant \emph{autre chose} qu’elles-mêmes, une variante d’une mélodie ancienne ou une nouvelle mélodie. — L’erreur darwinienne est d’avoir cru expliquer l’origine des espèces sans s’appuyer avant tout sur ce fait, et en prenant pour fondement principal de ses explications le simple travail vital, prolongé pendant des siècles, moyennant le crible éliminatoire de la sélection. Cette erreur est tout à fait comparable à celle des économistes — ces inspirateurs si fréquents de Darwin par leurs idées sur la population et la concurrence, prototype de la concurrence vitale, — qui ont, pour ainsi dire, décapité leur  \phantomsection
\label{v1p228}science ou plutôt l’ont créée acéphale en confondant pêle-mêle avec tous les genres de travaux vulgaires l’effort de la recherche inventive, ingénieuse ou géniale.\par
Voici un exemple, entre mille, de cette décapitation. Les avantages de la division du travail, d’après Adam Smith, sont de trois sortes : 1\textsuperscript{o} une plus grande adresse acquise par l’ouvrier ; 2\textsuperscript{o} l’économie du temps perdu à passer d’une occupation à une autre ; 3\textsuperscript{o} l’invention d’un grand nombre de machines et d’outils qui facilitent et abrègent le travail\footnote{ \noindent Il y a d’autres avantages que Smith n’a pas vus : meilleur classement des ouvriers d’après leurs aptitudes différentes ; plus grande utilité tirée des outils, dont les trois quarts resteraient à chaque instant inutilisés si un seul ouvrier s’en servait, etc.
 }. — On voit à quel humble rang, tout à fait accessoire, l’invention est reléguée. C’est la division du travail qui est la grande cause, la source fécondante ; l’invention — sans laquelle ni le travail indivis ni le travail divisé ne serait — n’est qu’un petit effet, un flot dérivé. — Stuart Mill a fait, mais sans y insister, de timides réserves à cet égard. Il a contesté que la spécialisation des besognes fût la \emph{seule} cause des inventions. A vrai dire, elle n’en est pas le moins du monde la cause ; elle n’est qu’une des conditions qui permettent parfois, bien rarement, à l’esprit inventif de réaliser quelques légers perfectionnements. Quant aux grandes et capitales idées vraiment rénovatrices, elles sont nées du loisir et de la liberté d’esprit, non du travail et de la contrainte d’un esprit assujetti à une seule et unique occupation. En second lieu, comme le fait observer Mill, « quelle que puisse être la cause des inventions, dès qu’elles sont réalisées, l’accroissement de la puissance du travail est dû non pas à la division du travail mais aux inventions elles-mêmes... ».\par
La distinction du travail et de l’invention ne correspond pas à celle du travail manuel et du travail mental\footnote{ \noindent Voir sur le \emph{travail mental} la belle étude de M. Fouillée dans la \emph{Revue des Deux-Mondes}, d’août 1900.
 }. Le travail \phantomsection
\label{v1p229} mental d’un écolier qui apprend ses leçons, d’un acteur qui étudie ses rôles, d’un notaire qui rédige un acte conformément à ses formulaires, d’un caissier de banque ou de magasin qui tient ses écritures en ordre, n’a rien d’inventif ; et même les longs calculs à l’aide desquels Leverrier est parvenu à découvrir l’existence d’une nouvelle planète dans une région précise du ciel, n’avaient en eux-mêmes absolument rien de génial. Mais, s’il y a un travail mental distinct de l’invention mentale, il n’y a pas d’invention manuelle distincte du travail manuel. Toute invention est mentale essentiellement. \emph{Mens agitat molem.} De là, tout procède et tout rayonne.\par
Pour compléter la distinction de l’invention et du travail et l’accentuer en antithèse, je pourrais ajouter que tout travail est plus ou moins pénible et que toute invention est plus ou moins agréable. Toutefois on pourrait me contester la vérité de ce contraste et me faire remarquer qu’il est des travaux accompagnés de plaisir et non de peine. Mais, si l’on examine de près ces travaux plaisants ou intéressants, on s’apercevra que tout l’intérêt ou tout le charme qu’on y trouve tient au caractère en partie nouveau des petits problèmes qu’ils posent et qu’ils résolvent, nouveauté bien faible, variété bien insignifiante, mais qui suffit à donner un petit air d’invention minuscule aux petites solutions successivement réalisées. Ainsi, un travail intéresse et plaît dans la mesure où il est difficile et ingénieux, où il est inventif, comme celui de l’artisan primitif, du maçon des vieilles cathédrales ; et, quand cet élément de diversité artistique y fait absolument défaut, il est rebutant et fastidieux au plus haut degré.
\subsubsection[{I.5.b. La fatigue et l’ennui, fatigue musculaire et fatigue nerveuse. Le remède à l’ennui. L’hérédité des professions ennuyeuses. La fatigue régie par une loi opposée à la loi de Weber.}]{I.5.b. La fatigue et l’ennui, fatigue musculaire et fatigue nerveuse. Le remède à l’ennui. L’hérédité des professions ennuyeuses. La fatigue régie par une loi opposée à la loi de Weber.}
\noindent C’est ici le lieu de parler de deux phénomènes psychologiques ou physio-psychologiques qui sont provoqués par le  \phantomsection
\label{v1p230}travail : la fatigue et l’ennui. Les économistes n’ont eu égard qu’au premier, et encore avec une attention bien insuffisante. Ils ne l’ont traité, comme les socialistes, qu’au point de vue de la durée du travail. Quant à l’ennui, ils n’en ont tenu aucun compte, non plus que les socialistes, par l’habitude de négliger tout le côté psychologique des faits économiques. Et c’est une lacune importante.\par
Il y a deux sortes de fatigue, la fatigue musculaire et la fatigue nerveuse. Dans tous les débats relatifs aux effets des machines sur le travail humain, on semble n’avoir eu égard qu’à la première. On a constaté à tort que les machines, sous ce rapport, avaient allégé le labeur humain. Ce qui est vrai, c’est que le besoin de travail s’est développé, par la multiplication des besoins de jouissance, plus vite encore que la productivité du travail, et que, par suite, la quantité du travail n’a pas diminué ; mais il est certain qu’elle est devenue moins fatigante pour les muscles. Il n’en est pas de même pour les nerfs. Le travail exigé pour la surveillance des machines produit une fatigue nerveuse bien plus dangereuse pour l’homme que la fatigue musculaire qui lui est épargnée par elles. L’attention fixe et continue qui est requise par leur emploi est contraire aux tendances naturelles du cerveau. « En étudiant les phénomènes cérébraux, dit Mosso, nous voyons avec quelle rapidité s’épuise l’énergie des éléments nerveux ; quelques secondes de travail (d’attention) suffisent pour amener cet épuisement dans les cellules cérébrales ; et l’activité prolongée du cerveau, malgré cette rapide lassitude de ses éléments, ne s’explique que par la présence, dans les circonvolutions, de milliards de cellules qui se suppléent réciproquement. » On ne doit pas être surpris de voir survenir tant d’accidents de chemins de fer causés par l’excessive tension nerveuse d’employés arrachés d’hier à la vie des champs, où leur esprit se reposait, au milieu des plus rudes travaux, en une délicieuse inattention, pour être brusquement condamnés à une fixité d’attention  \phantomsection
\label{v1p231}contre nature. — « La fatigue, dit Mosso, exerce une grande influence sur le temps de réaction : s’il faut 134 millièmes de seconde avant qu’un sujet touché au pied fasse un signe avec la main, il faut, lorsque l’attention s’épuise, 200 à 250 millièmes de seconde. » Qui sait si cette observation n’explique pas, en partie, pourquoi certains signaux n’ont pas été perçus à temps ou ne l’ont pas été du tout ? — La \emph{fatigue de l’attention} est un supplice nouveau et plus subtil, inconnu à tous les grossiers enfers d’autrefois, et que la machinofacture a introduite dans le monde moderne. Le progrès des maladies mentales, le progrès du suicide, le progrès de l’alcoolisme, dérivent partiellement de là.\par
La fatigue intellectuelle, produite aussi par le développement de la bureaucratie, cette grande machine administrative, par l’extension des professions libérales, grandit sans cesse. Et quelle fatigue ! Est-ce que la sueur du cultivateur qui bêche peut se comparer à l’épuisement cérébral d’un examinateur au baccalauréat après un mois d’examens, ou même de l’élève qu’il examine ? Je parle de l’examinateur, non du professeur. Le travail du professeur qui fait sa leçon rentre dans la catégorie de ces travaux attachants, quoique fatigants, dont je parlais tout à l’heure. Comme le maçon artiste, comme l’antique artisan, le professeur résout à chaque instant de petites difficultés neuves et renaissantes. N’importe, il s’épuise aussi. Et Mosso fait à ce sujet une remarque assez curieuse, qu’il prétend avoir vérifiée sur lui-même : le professeur, dit-il, se fatigue d’autant plus vite que son auditoire est plus nombreux. Si cette observation est exacte, elle est de nature à prouver la réalité de l’action inter-spirituelle inconsciente. Car, assurément, nul professeur n’a conscience de cette influence exercée sur lui par la seule présence de chacun de ses auditeurs.\par
Ainsi, la révolution opérée dans la psychologie du travailleur par la machinofacture consiste en ce qu’elle a diminué la fatigue des muscles et augmenté celle des nerfs. A-t-elle accru  \phantomsection
\label{v1p232}ou diminué l’ennui ? Toute autre question non moins importante à résoudre. Car, s’il était prouvé que, en rendant le travail moins fatigant, au moins musculairement, l’invention des machines l’a rendu plus ennuyeux, où serait le gain définitif pour l’humanité ? Entre un procédé nouveau qui fatigue moins mais qui ennuie plus, et un procédé ancien qui fatigue plus mais qui ennuie moins, l’hésitation est permise, et l’obstination de beaucoup de paysans, de beaucoup d’ouvriers dits arriérés, à employer le dernier de préférence au premier, n’a rien qui justifie la compassion méprisante des hommes « de progrès ».\par
Seulement empressons-nous d’ajouter que, au point de vue de l’ennui, comme à celui de la fatigue, l’apologiste des machines n’est pas à court d’arguments. La machinofacture a pour effet de concentrer les travailleurs dans les usines, ce qu’on lui a assez reproché ; mais, précisément, en substituant de la sorte au travail dispersé et solitaire, le travail rassemblé et fait en commun, elle tend, d’une part, à rendre la même dépense de force moins fatigante qu’elle ne le serait, d’autre part la même tâche moins ennuyeuse. On n’a étudié la fatigue que chez l’individu isolé ; mais ce phénomène ne relève pas de la seule psychologie individuelle, il appartient à la psychologie collective. Plus lente à venir, moins rapide à croître, est la lassitude de l’effort quand on s’efforce ensemble ; n’est-ce pas pour accroître leurs forces par le stimulant réciproque de leur rassemblement que les oiseaux migrateurs ont soin de s’attrouper avant d’entreprendre la périlleuse traversée des mers ? Dès que l’un d’eux s’est écarté du groupe, il est perdu. On a pu remarquer aussi avec quelle énergie relativement infatigable s’opèrent les travaux des champs qui, par exception, rassemblent les travailleurs, la moisson, par exemple, cette longue corvée de quatorze ou quinze heures qui se fait en chantant à pleine voix, et le dépiquage au fléau. La même remarque s’applique à l’ennui de certains travaux qui, accomplis dans les veillées  \phantomsection
\label{v1p233}des villages, par plusieurs familles groupées autour d’un même foyer, paraissent presque amusants : égrener du maïs, des pois, etc.\par
On peut répondre, il est vrai, que la fatigue et l’ennui sont contagieux, et que, si les meneurs d’un groupe de travailleurs affectent ou témoignent d’être fatigués ou ennuyés, leurs camarades ne tardent pas à leur tour à se plaindre d’une fatigue ou d’un ennui qu’ils n’auraient pas ressenti sans cela. En somme, la psychologie collective est caractérisée surtout par une exagération des phénomènes de la psychologie individuelle dans les sens les plus opposés.\par
Mais, laissant de côté la question des machines, cherchons, d’une manière plus générale, les conditions qui rendent le travail ennuyeux et pénible. Il faut tenir compte, avant tout, d’un phénomène qui, non moins que les deux précédents, caractérise l’être vivant : l’habitude. Une machine ne s’habitue pas plus qu’elle ne se fatigue, mais il est, pour les organismes, et spécialement pour les organismes animés, des lois de l’habitude qu’il faut étudier dans leurs rapports avec celles de la fatigue et de l’ennui, si l’on veut traiter sérieusement le sujet du travail. On s’habitue à la fatigue même, à l’ennui même. On s’habitue en voyant les autres s’habituer ; c’est contagieux aussi. On s’habitue plus rapidement et plus complètement à une gêne, à une privation, à un excès, quand on est ensemble que lorsqu’on est isolés. L’habitude a son contraire, qui est assez fréquent : n’y a-t-il pas, en effet, des maux, des incommodités, d’abord tolérables, qui, en se répétant chez le même individu ou en se propageant d’un individu à d’autres, deviennent de plus en plus pénibles et, à la fin, tout à fait intolérables ?\par
Maine de Biran a démontré, dans son fameux mémoire, cette loi fondamentale de l’habitude psychologique, que, par leur répétition, les sensations s’émoussent, mais les états actifs se fortifient, l’attention, la volition. Par états actifs,  \phantomsection
\label{v1p234}nous entendrions plutôt les croyances et les désirs qui, de fait, s’intensifient en se répétant et en se propageant. — Nous dirons donc que la répétition d’un acte de travail diminue, en général, son caractère affectif, mais augmente le désir de la production spéciale accompli par lui, et la foi en son utilité. L’attachement persévérant à un même travail tend, par suite, à reculer — jusqu’à un certain point — le moment où ce qu’il y a de pénible est senti, et où la fatigue commence. De là l’avantage de se spécialiser.\par
S’habitue-t-on à l’ennui ? Est-ce qu’il ne semble pas, au contraire, que l’ennui de certaines occupations, quand elles se répètent, va en augmentant ? Il importe donc d’éviter que le travail, même peu fatigant, ne soit fastidieux ; et il est à remarquer que la répétition d’un même travail sans variation commence à ennuyer assez longtemps avant de commencer à fatiguer. Ce qu’il y a de plus à redouter pratiquement, c’est beaucoup moins l’excès de fatigue, qui est très rare, que l’excès d’ennui causé par la nature du travail, mal appropriée à la nature du travailleur. On aura beau abréger la journée du travail, la réduire à huit heures, à six heures, ce sera encore trop pour celui qui, pendant ce laps de temps, aura à faire une besogne jugée par lui, à tort ou à raison, fastidieuse au plus haut degré. N’oublions pas d’ajouter que, à dépense de force égale, le travailleur se fatigue d’autant plus vite que sa tâche l’ennuie davantage. Le calcul du nombre d’heures de travail doit donc varier suivant qu’il s’agit d’une tâche plus ou moins intéressante ou rebutante.\par
Il faut, par conséquent, réduire au minimum l’ennui du travail si l’on veut obtenir, sans fatigue, le maximum du travail. — Mais comment y parvenir ? Est-ce par les procédés de Fourier, par ce papillotage d’occupations multiformes qui devait rendre, d’après lui, le travail attrayant ? Ce morcellement du temps, cette mosaïque de travaux divers, ne conviennent qu’à des enfants et, de fait, ne se  \phantomsection
\label{v1p235}voient que dans les collèges. Pour des adultes, c’est la \emph{variation du même travail}, bien plutôt que la \emph{variété des travaux}, qui est le remède contre l’ennui. Ce n’est pas au point de vue de l’ennui, mais c’est, disons-le en passant, à un point de vue plus élevé, que la question de la diversité ou de l’uniformité des travaux soulève un grand et difficile problème. Le développement intellectuel et moral de l’individu exige, cela est certain, l’alternance et la variété des occupations, dans une large mesure. Or, le développement économique de la société exige la division et la spécialisation du travail. Entre ces deux exigences contradictoires, laquelle choisir ? Dirons-nous qu’elles doivent, pour le plus grand bien du monde, se relayer, et qu’il convient que chacune d’elles l’emporte à son tour ? En tout cas, le problème social posé par cette antinomie comporte un certain nombre de solutions approximatives et divergentes qui sont autant de voies différentes ouvertes à l’évolution historique et font comprendre l’allure zigzaguante de l’histoire.\par
Mais revenons à la question de l’ennui. Le remède principal contre cette maladie psychologique du travailleur, ce n’est pas surtout la variété, c’est le \emph{risque} et la \emph{chance} qui donnent de l’intérêt au travail. \emph{Intéressant} est le contraire d’\emph{ennuyeux.} Supprimez tout risque et toute chance, et vous rendez toute besogne aussi ennuyeuse que l’est aujourd’hui un travail administratif. — Il faut aussi, pour qu’un travail n’ennuie pas, qu’il ait été soit imposé dès l’enfance par la famille, contrainte non sentie qui ne soulève aucune révolte, soit adopté librement par l’individu adulte. Le caractère héréditaire de beaucoup de professions, surtout des plus pénibles — ou de celles qui sont jugées les plus ennuyeuses, à tort souvent, par exemple les professions agricoles — n’a donc rien qui mérite d’être réprouvé. Plus une profession nécessaire est rebutante ou jugée telle, et plus il est à désirer qu’elle se recrute par l’hérédité. La réciproque est vraie : aussi est-il à désirer que les professions libérales,  \phantomsection
\label{v1p236}jugées les plus agréables, soient celles où la proportion de la transmission héréditaire soit la moindre. Effectivement, il en est ainsi.\par
L’instruction supérieure n’aurait-elle pas pour effet inévitable de rendre ennuyeux des genres de travaux qui sans elle ne le seraient pas et qui sont indispensables ? Il est certain que jamais, muni de son diplôme de licencié ès lettres ou même de bachelier, un jeune homme ne pourra labourer ou bêcher sans un mortel dégoût. Tailler la vigne, greffer des arbres, n’est pas fatigant ; mais, au bout d’une heure, la plupart des « intellectuels », j’en ai peur pour eux, seront excédés d’ennui à se rendre utiles de la sorte. Il est fâcheux, donc, d’avoir à reconnaître que l’idéal de l’instruction \emph{intégrale} ne saurait se réaliser sans profond dommage pour la civilisation, qui suppose à la fois certaines besognes très fastidieuses et certaines cultures très délicates.\par
Nous voulons tous que tous les travaux, même les plus ennuyeux et les plus dangereux, qui sont nécessaires pour l’entretien de notre vie civilisée, soient régulièrement accomplis. Mais il est beaucoup de ces travaux que nous ne voulons pas exécuter nous-mêmes. Si \emph{personne} ne voulait exécuter ces travaux rebutants ou périlleux, il faudrait employer la force pour contraindre quelques parias, tels que nègres, fellahs, asiatiques jaunes, à ces corvées. La question est de savoir si, \emph{au fond}, ce n’est pas toujours \emph{par force} que ces travaux-là sont accomplis. L’action de la force, de la tyrannie asservissante, se dissimulera tant qu’on voudra, il est à craindre qu’elle ne subsiste toujours, — amoindrie, espérons-le !\par
On a proposé l’alternance des travaux intellectuels et des travaux manuels, comme hygiène et comme remède à l’injustice dont il s’agit. Mais Mosso fait une remarque, appuyée sur des expériences, qui peut servir à justifier la séparation des travaux intellectuels et des travaux manuels, et à dissiper \phantomsection
\label{v1p237} l’espoir de les voir alterner les uns avec les autres dans la journée d’un même travailleur. Tout au plus, l’exercice intermittent d’un métier mécanique pourra-t-il être conseillé aux travailleurs du cerveau, à titre de sport. Et encore, la physiologie a-t-elle ses réserves à faire. « La sensation de malaise et la prostration qui caractérisent la fatigue intellectuelle, dit Mosso, viennent de ce que le cerveau épuisé doit envoyer des excitations plus fortes à des muscles plus faibles pour les faire contracter. C’est donc une erreur physiologique d’interrompre des leçons pour faire faire aux écoliers des exercices gymnastiques, dans l’espoir qu’on diminuera ainsi la fatigue du cerveau. » Plus le surmenage cérébral, donc, des professions libérales ira en augmentant, et plus on s’éloignera du rêve de Tolstoï.\par
Une dernière observation à propos de la fatigue. Elle croît, dès qu’elle a commencé, beaucoup plus vite que sa cause. Et on peut faire la même remarque sur l’ennui. Si, après un certain temps de travail, la fatigue naît et, pendant une demi-heure, devient égale à 1, il ne faut pas croire que, après la seconde demi-heure, elle égalera 2, après la troisième demi-heure 3, etc. Non, elle aura triplé ou quadruplé, par exemple, pendant que la durée du même travail aura simplement doublé. — La fatigue est donc régie par une loi qui est justement l’opposé de la loi de Weber. La sensation, on se le rappelle, croît moins vite que son excitation. Ainsi, la fatigue, toute sentie qu’elle est, ne se comporte pas comme une sensation ordinaire. C’est qu’elle est, avant tout, une douleur, et l’on pourrait dire de la douleur en général ce que je viens de dire de la douleur-lassitude : elle croît plus vite que sa cause, jusqu’à un certain point du moins, au delà duquel elle croît moins vite, puis reste stationnaire et finalement s’anéantit dans la syncope\footnote{ \noindent Un pessimiste dira-t-il que, en cela, le plaisir diffère de la douleur, et ne croit jamais que beaucoup moins vite que sa cause ? L’observation manquerait de justesse. Je sais bien que, si un pauvre vient à gagner 100 000 francs à la loterie après en avoir déjà gagné 100 000, son plaisir n’est pas doublé ; mais c’est qu’il a atteint son maximum de joie à son premier gain. Si, peu à peu, par petites sommes accumulées, il est parvenu à gagner 100 000 francs, sa joie pendant quelque temps, a progressé plus rapidement que son avoir. L’espoir et le ravissement d’un amoureux à chaque nouvelle petite faveur qu’il reçoit, s’avivent bien plus vite que ne grandit le nombre ou l’importance de ces privautés.
 }. La  \phantomsection
\label{v1p238}même réserve s’applique à la fatigue. Cela ne tient-il pas à l’élément-désir qui se trouve combiné avec l’élément-sensation dans l’état complexe appelé douleur ?
\subsubsection[{I.5.c. Le travail, de moins en moins fatigant, devient-il de moins en moins ennuyeux ? Chants du travail primitif.}]{I.5.c. Le travail, de moins en moins fatigant, devient-il de moins en moins ennuyeux ? Chants du travail primitif.}
\noindent Au cours de la civilisation, le travail humain, primitivement très improductif quand il était exécuté par des esclaves, est devenu de plus en plus productif, et de moins en moins pénible ; mais est-il devenu de moins en moins ennuyeux et insipide ? Il se peut fort bien que le minimum général d’ennui ou le maximum de bonheur dans le travail, corresponde, non aux plus hauts degrés mais aux états moyens du progrès social, à la phase où l’agriculture et la petite industrie domestique donnent le ton. On a un bon signe de la gaieté du travailleur, quand il a l’habitude de chanter en travaillant. Et ne semble-t-il pas que, depuis l’invasion de la grande industrie, les métiers où l’on chante soient de plus en plus remplacés par les professions où l’on fume et où l’on boit, pour secouer la torpeur qui vous y envahit ?\par
D’après le compte rendu que j’en lis dans la dernière \emph{Revue philosophique}, Wundt, dans son dernier ouvrage (\emph{la Psychol. ethnique}), adopte l’avis de Bûcher, qui a montré « que les chants accompagnant le travail, sont, suivant toute probabilité, la forme la plus primitive de la poésie et de l’expression musicale ». Il est certain que « dans la plupart des travaux simples, des mouvement identiques se succèdent et provoquent des répétitions rythmiques ».\par
Le travail a donc commencé par être une page de vers  \phantomsection
\label{v1p239}avant d’être une page de prose. — Cette origine commune de la poésie, de la musique et du travail (ordinairement on dit plutôt : \emph{et de la danse}) montre ce que le point de vue utilitaire des économistes a d’insuffisant. Le travail a longtemps conservé ce caractère rythmique et musical qu’il semble avoir eu à l’origine. « Dans les petits ateliers de la Grèce, nous dit M. Guiraud, le travail était plus attrayant que dans nos grandes usines. Dans certains corps de métiers, on avait coutume d’égayer sa tâche par des chants\footnote{ \noindent Ne faut-il pas voir un reste ou une suite de ces chants du travail dans les ritournelles spéciales par lesquelles les marchands du moyen âge annonçaient le passage de leurs marchandises, et qui étaient l’équivalent acoustique de nos annonces visuelles ?
 }. C’était l’usage notamment des meuniers, des broyeuses de grain, des baigneurs, des fileuses et des tisseuses. Parmi ces chansons, les unes remontaient à une origine très ancienne et étaient anonymes, les autres étaient attribuées à des poètes connus. Au Pirée, on se servait de flûtes, de fifres et de sifflets pour donner de l’entrain aux ouvriers de l’arsenal maritime et régler leurs mouvements. »\par
Dans ses \emph{Ouvriers européens}, Le Play consacre un chapitre spécial, à propos de chacune de ses monographies de familles ouvrières, aux \emph{récréations.} Or, le plus souvent, il nous apprend que la \emph{fête} la plus recherchée, chez les ouvriers ruraux et primitifs qu’il nous décrit, consiste en un travail bénévole et gratuit, mais exécuté en commun, par esprit de mutuelle assistance, et suivi d’un grand repas. Les vestiges de ces vieilles coutumes se retrouveraient encore facilement. Au fond de quelques-unes de nos provinces les moins modernisées, il était encore d’usage, il y a quelques années, que, pour le transport du bois de chauffage d’un propriétaire rural à sa maison, ses voisins s’offrissent spontanément et gratuitement pour ce labeur fatigant, mais toujours très gai, exécuté de très grand matin et suivi d’un repas plantureux. C’est que le \emph{travail en commun}, à la campagne,  \phantomsection
\label{v1p240}étant une exception, est une satisfaction rare donnée au besoin de sociabilité. Aussi est-il joyeux, accompagné de chants, et diffère-t-il profondément du travail rassemblé dans les ateliers. Non seulement le charroi du bois, mais la moisson, les vendanges, le fauchage, l’épluchage dans les veillées, ont ce caractère de joie toute sociale. Quand, au bord d’une de nos rivières navigables, un chaland neuf va être lancé à l’eau, les bateliers se rassemblent, et, avec de longs \emph{hourrahs}, qui tiennent à la fois du cri de guerre et du cri de joie, le poussent ensemble vers le courant. Mais ces joies primitives, que le progrès économique a peu à peu supprimées, et, je me hâte de le dire, souvent remplacées, elles ne sont plus guère qu’un souvenir parmi nous. Sous des noms divers, Le Play les a retrouvées en pleine floraison parmi les populations de l’Oural, et aussi, de son temps, dans le Béarn, dans la Basse-Bretagne. L’un des plus vifs plaisirs que puissent goûter des hommes habituellement épars et solitaires est de se trouver réunis et de se sentir coopérer à une action commune, soit guerre, soit travail. Il n’est pas de stimulation plus tonifiante que cette mutuelle excitation qui naît de leur simple contact. Cette considération peut servir à nous faire comprendre comment les primitifs, naturellement si paresseux, ont pu contracter l’habitude du travail : il est à remarquer, en effet, que les premiers travaux, de chasse, de pêche, de défrichement, ont dû être exécutés en commun, c’est-à-dire gaîment. L’utopie du \emph{travail attrayant}, servant d’amorce au travail aride et ingrat, s’est donc réalisée de la sorte au début de la civilisation, et sous des formes bien meilleures que le rêve de Fourier. Car l’attrait qui a rendu le travail supportable, et même agréable, a été de nature sociale et non égoïste.\par
Le degré d’agrément ou d’intérêt du travail tient, en grande partie, à la nature des collaborateurs qu’il suppose. Car tout travail est une collaboration : avec la nature à la fois et avec les autres hommes. Avec la nature : depuis l’agriculteur \phantomsection
\label{v1p241} ou le pâtre qui dirige les forces végétales et les forces animales, jusqu’à l’industriel qui met en œuvre les forces physiques et chimiques, il n’est pas de travailleur qui n’agisse de concert avec quelque agent naturel sans lequel toute sa dépense d’activité personnelle serait perdue. Avec les autres hommes : il n’est pas un ouvrage accompli dans une société, même par le plus solitaire des artisans, qui ne soit un fragment d’un tout, une maille d’un tissu, une besogne partielle se rattachant à une élaboration générale dont la conception le domine. Or, sous ce double rapport, le travail agricole, qui consiste à surveiller et à diriger le travail organique des plantes ou des animaux, se distingue avantageusement du travail de l’ouvrier moderne qui dirige et surveille le fonctionnement d’une machine. Dans le premier cas, l’homme collabore avec un génie merveilleux, celui de la vie, profondément caché mais bienfaisant, devant lequel le sien propre est contraint de s’humilier, même en l’asservissant, tandis que, dans le second cas, il commande à des énergies aveugles et sans but qui doivent à l’ingéniosité du constructeur toute la finalité qu’elles simulent. Aussi le travail des champs est-il infiniment plus intéressant que celui des ateliers ; on y goûte une joie qui a quelque chose de confraternel, le charme d’une association avec d’autres âmes, paisibles et fécondes. Le jardinier avec ses arbres, pas plus que le pâtre avec ses brebis ou ses bœufs, ne se sent isolé. Mais la machine ne tient pas compagnie à l’ouvrier, pas même la machine-outil. Aucun bicycliste ne s’attache à sa bicyclette même, à plus forte raison aucun mécanicien ou aucun chauffeur ne s’attache à sa locomotive comme un cavalier à son cheval, comme un cornac à son éléphant, comme un chamelier à son chameau, comme une dompteuse même à ses fauves. On ne sait jamais au juste ce qui sortira d’un germe planté, d’un noyau de pêche ou d’un cep de vigne. Aussi l’attente de ce qu’on récoltera est-elle pleine d’un intérêt toujours nouveau et inépuisable,  \phantomsection
\label{v1p242}que le laboureur octogénaire ressent encore presque autant que ses petits-enfants. Mais l’on sait toujours exactement ce qu’on obtiendra à l’aide d’une machine ; grand avantage au point de vue objectif, grande infériorité au point de vue subjectif, car il n’y a aucun plaisir à la longue à la voir fonctionner. — Le travail agricole ressemble en cela aux expériences de laboratoire d’un jeune savant, chercheur et entreprenant, qui essaie des combinaisons nouvelles et en attend le résultat. Le travail de l’agriculture est si bien une association avec cette étrange et divine personne voilée qu’on nomme la vie, que la longue intimité établie entre elle et le paysan a imprimé sur lui comme un reflet d’elle. Il y a quelque chose du génie de la nature vivante dans l’ingéniosité rusée et tenace du paysan, en quelque pays du monde qu’on l’observe.\par
Parmi les professions libérales, il en est qui, comme la profession agricole, consistent dans la direction et la surveillance de la vie : telles sont les fonctions du médecin et du chirurgien. Et je dirais que ce sont les plus intéressantes s’il n’en existait d’autres qui, plus attachantes encore, sont le maniement des vies les plus hautes, des âmes humaines, soit des âmes d’enfants (professeurs, instituteurs), soit des âmes adultes (journalistes, avocats, magistrats). Quant à celles où l’on a pour tâche de copier ou de rédiger des écrits qui sont destinés à mettre en mouvement non des hommes concrets pour ainsi dire, mais des hommes abstraits, des rouages administratifs, il n’est rien qui égale leur insipidité en fait de corvée humaine. Les galériens, quelquefois, sur leurs bancs de rameurs, par une mer lumineuse, chantaient en battant sur un rythme uniforme « les flots harmonieux » ; je ne crois pas que jamais clerc de notaire ou rédacteur de ministère ait chanté en faisant sa copie.
 \phantomsection
\label{v1p243}\subsubsection[{I.5.d. Degrés inégaux de la considération attachée aux divers travaux. Considération des travaux, non toujours proportionnelle à leur utilité. Hausse et baisse alternatives, en Grèce, de l’estime des travaux manuels. Causes de ces variations en tout pays.}]{I.5.d. Degrés inégaux de la \emph{considération} attachée aux divers travaux. Considération des travaux, non toujours proportionnelle à leur utilité. Hausse et baisse alternatives, en Grèce, de l’estime des travaux manuels. Causes de ces variations en tout pays.}
\noindent Non seulement, au point de vue de la psychologie individuelle surtout, il importe beaucoup de considérer le degré de fatigue et le degré d’ennui qui s’attache aux divers genres de travaux, — mais il importe aussi, et bien plus encore, au point de vue de la psychologie inter-mentale, d’avoir égard au degré de considération ou de déconsidération dont les divers genres de travaux sont l’objet.\par
Une première remarque à faire ici, c’est qu’on se tromperait étrangement si l’on pensait que les métiers sont plus ou moins considérés suivant qu’ils sont réputés plus ou moins utiles ou nécessaires. Il n’y avait pas, parmi les \emph{arts} de Florence, de métier plus avilissant que celui de boulanger, tandis que celui de drapier, assurément moins indispensable, jouissait de la plus haute estime publique. — Dans l’antique Égypte, où l’on sait l’importance majeure qu’était censé avoir l’embaumement des cadavres, l’embaumeur était souverainement méprisé. En Russie, les \emph{popes} ne sont nullement considérés, malgré le haut prix que l’on attache, par suite de la foi religieuse intense de ce pays, à l’administration des sacrements. — Dans l’Inde, — dans l’Inde ancienne du moins (d’après les \emph{Lettres édifiantes}) — tout cordonnier, tout « ouvrier travaillant le cuir » et, en plusieurs endroits, « les pêcheurs et les gardiens de troupeau » sont réputés parias. — Il est rare qu’une profession amusante, une profession qui procure du plaisir, voire même un plaisir spirituel, noble, élevé, par exemple celle des comédiens, soit cotée haut dans la considération générale. — Rien de plus contraire, on le voit, aux explications utilitaires des faits sociaux.\par
Si l’on cherche les causes d’où dépend l’inégale estime des divers genres de travaux, la meilleure méthode, à mon avis,  \phantomsection
\label{v1p244}est, non pas de se placer à un moment donné de l’histoire et de se demander pourquoi, à ce moment, les divers pays estimaient inégalement les mêmes professions, — mais bien de prendre un certain nombre de professions, de les suivre, dans un même pays, à travers une période de temps plus ou moins longue, pendant laquelle leur considération a subi des variations en plus ou en moins, et de voir à quelles variations concomitantes celles-ci paraissent se rattacher.\par
Nous voyons cependant assez bien, pourquoi, à Sparte, tout travail industriel est méprisé, tandis qu’à Athènes l’oisiveté seule passait pour un déshonneur. Nous constatons que cette différence se rattache à la nature aristocratique ou démocratique de la constitution, quoiqu’il y ait des exceptions à la règle, par exemple Corinthe, cité aristocratique où le travail industriel était honoré. D’ailleurs, « à Thespies, c’était une honte d’apprendre un métier ou de s’occuper même d’agriculture. Dans plusieurs républiques, la qualité de citoyen était incompatible avec l’exercice d’une profession mécanique. A Thèbes, les boutiquiers et les détaillants n’avaient accès aux magistratures que dix ans après qu’ils étaient retirés des affaires. » (Guiraud, \emph{La main-d’œuvre industrielle dans l’ancienne Grèce}).\par
Mais il s’agit là du mépris ou de l’honneur attaché à un travail quelconque ; la question que nous agitons est tout autre, c’est de savoir à quoi tient l’inégale considération des divers genres de travaux. Or, à cet égard, pourquoi, en Grèce, pays peu propre à l’agriculture, mais éminemment propre au commerce, les occupations agricoles ont-elles toujours paru plus nobles que les autres ? On ne le voit pas bien clairement.\par
Si nous remontons à l’époque homérique, ce n’est pas seulement l’agriculture, c’est le travail industriel aussi qui nous paraît hautement considéré. La preuve manifeste en est que les dieux et les rois mêmes travaillent. Vulcain est un forgeron qui forge des boucliers, des casques, des lances  \phantomsection
\label{v1p245}pour tout le personnel de l’Olympe ; il construit les portes de l’appartement de Junon, il fabrique les armes d’Achille. Il fabrique aussi la toile de fer où il capte Mars et Vénus. « Athéna tisse le péplum d’Héra et le sien. La nymphe Calypso promène une navette d’or sur son métier », et Circé fait des toiles merveilleuses. « On citait même des dieux qui avaient consenti, comme Apollon et Poseidon, à garder les troupeaux du roi Loamédon et à édifier les remparts de Troie. » Voilà pour les dieux et les déesses, voici pour les rois et les reines. Nausicaa fait la lessive ; Ulysse est un charpentier de première force, et un bon menuisier aussi. Il a fait lui-même son lit nuptial, et il s’en vante. Pâris a aidé à bâtir sa propre maison.\par
A l’époque d’Homère encore, on citait les noms d’artisans devenus célèbres et glorieux. Homère a daigné incruster dans ses vers les noms d’Epeios, de Phéréclos, de Tychios, de Laerkès, d’Icmalios (Guiraud). Pour qu’un artisan pût atteindre à la gloire, à une célébrité presque égale à celle des héros, ne fallait-il pas qu’alors le travail industriel, — métallurgique notamment, — fût généralement en honneur ? Mais, quelques siècles après, dans presque toutes les cités grecques, le travail industriel devient méprisable : pourquoi ? C’est, manifestement, parce que l’esclavage, déjà existant à l’époque homérique, a été se développant beaucoup. Peu à peu, l’homme libre s’est déchargé sur l’esclave de toute besogne mécanique quand il l’a pu, et n’a retenu qu’une partie du travail agricole, la plus facile, la surveillance. Aussi, même l’artisan libre a-t-il participé à la déconsidération graduelle d’un labeur habituellement servile.\par
En règle générale, on peut dire qu’un métier gagne en considération quand il se recrute dans des couches sociales de plus en plus élevées, et inversement. A cela tient, par exemple, du moyen âge aux temps modernes, la considération grandissante du métier de versificateur. Le \emph{jongleur}, misérable rimailleur ambulant et affamé, amuseur des châteaux \phantomsection
\label{v1p246} et des chaumières, est devenu peu à peu comédien ordinaire du roi comme Molière, poète de cour, enfin poète national auréolé de la plus grande gloire du monde. — J’en dirai autant des miniaturistes, des enlumineurs de missels du moyen âge qui, à la Renaissance italienne, nous apparaissent transformés en peintres illustres. Il n’est pas douteux que le recrutement des littérateurs et des artistes, du {\scshape xiii}\textsuperscript{e} siècle à la Renaissance, s’est opéré dans des classes sociales de plus en plus hautes. — Du commencement de l’Empire romain à la fin de l’Empire, nous voyons baisser le prestige du métier militaire (quoique, assurément, il devint de plus en plus utile et nécessaire pour lutter contre les barbares) parce que, après avoir attiré jusqu’à la fin de la République toute l’élite de la jeunesse romaine, il n’exerçait plus d’attrait à la fin que sur des malheureux sans ressources, sur des vagabonds et des déclassés. — Dans notre siècle, le métier militaire a toujours été coté très haut, mais il a présenté néanmoins, de 1815 à 1855 environ, et de 1870 à nos jours, une baisse puis une hausse très sensible de son prestige ; et la cause que j’indique n’est pas étrangère à ces oscillations.\par
Ce n’est pas seulement à raison du rang des personnes qui l’exercent qu’un métier s’élève ou s’abaisse dans l’estime publique ; c’est encore à raison du rang des personnes au profit desquelles il s’exerce. Voilà pourquoi les plus vils offices honorent un homme, dans une vieille monarchie, quand ils sont remplis pour le monarque. Le valet de chambre du roi, sous Louis XIV, était très fier de ses fonctions. Les pages des grands vassaux, sous la féodalité, ne l’étaient guère moins, quoique, après tout, leur rôle fût celui de petits domestiques.\par
Une remarque générale se présente ici : il n’y a rien d’humiliant, en tout temps et en tout lieu, à se servir soi-même — comme Ulysse fabriquant lui-même son lit nuptial — ni à servir les personnes de sa famille et de sa maison —  \phantomsection
\label{v1p247}comme Nausicaa lavant le linge des siens. Et, à l’extrême opposé de l’évolution économique, il n’y a rien d’humiliant non plus à travailler \emph{pour le public}, pour un très grand nombre de personnes qu’on ne connaît pas, qui ne vous touchent en rien, pour une foule dispersée et impersonnelle. Mais, entre ces deux phases extrêmes, le préjugé humain, très tenace, attache un caractère plus ou moins servile au fait de travailler pour une personne — ou même pour un petit groupe de personnes individuellement connues et étrangères à sa famille, sauf des exceptions passagères, comme celles des pages féodaux et des valets de chambre monarchiques. Quelle est l’explication historique de ce préjugé, qui est si lent à disparaître ? En est-il ainsi simplement à cause d’une antique association d’idées provenant de ce que les esclaves seuls, durant presque toute l’antiquité, travaillaient pour des personnes non parentes ?\par
Indépendamment du caractère des personnes qui exercent une profession ou des personnes qu’elle sert, cette profession est plus ou moins honorablement cotée d’après le but qu’elle poursuit et les moyens qu’elle emploie pour l’atteindre. Il est remarquable que, en général, les professions qui protègent contre des dangers de pertes, de perte de la vie ou de perte des biens, sont réputées plus honorables que celles qui ouvrent des perspectives de gain. La profession protectrice, par excellence, contre les périls d’invasion et de spoliation étrangère, c’est la profession militaire. La profession sacerdotale doit être mise à côté, car elle est réputée, à l’origine surtout, défendre l’homme contre des légions d’invisibles ennemis, de puissances occultes et néfastes. Aussi ces deux professions sont-elles partout et toujours très haut cotées, tandis que la spéculation commerciale, et la fabrication industrielle, même en grand, qui donnent tant de chances de fortune rapide sont estimées bien plus bas.\par
J’ajoute, et avec bien plus de généralité encore, que les métiers qui nous préservent de la douleur physique l’emportent \phantomsection
\label{v1p248} en considération sur les métiers qui nous procurent du plaisir physique. Les médecins et les chirurgiens sont bien plus estimés que les cuisiniers, les parfumeurs, les danseuses. On peut même dire que les métiers consacrés à nous garantir contre des peines de nature spirituelle, c’est-à-dire sociale, par exemple, le métier de juge qui nous épargne la souffrance de l’injustice subie, des intérêts lésés, inspirent plus de respect, sinon d’admiration toute personnelle, que les arts voués au divertissement spirituel, tels que l’art dramatique, le roman, la musique.\par
Cette importance plus grande attachée longtemps par les hommes à l’\emph{exemption de la douleur} et non à l’\emph{acquisition de la joie} semble donner raison à Schopenhauer : mais le progrès épicurien de la civilisation tend à faire disparaître cette singularité et à renverser même l’ordre indiqué.\par
Si la nature des buts poursuivis par les divers métiers influe sur le degré d’honorabilité qui leur est inhérent, la nature des moyens employés n’a pas moins d’influence. Les métiers où l’on n’atteint son but que par une action matérielle sur les choses, par l’emploi d’agents physico-chimiques, de moteurs mécaniques, sont classés inférieurs à ceux où l’on agit par la direction des forces végétales et animales, et ceux-ci, à leur tour, sont estimés bien au-dessous de ceux où l’on parvient à ses fins en exerçant une action inter-mentale autour de soi. Aussi les professions où l’on exerce l’action inter-mentale la plus étendue ou la plus profonde, ou les deux à la fois, sur les volontés, sur les intelligences, sur les sensibilités même, sont-elles les plus considérées, et d’autant plus que leur action s’étend ou s’approfondit davantage. De là le prestige : 1\textsuperscript{o} des métiers où l’on \emph{commande}, où l’on communique son vouloir au vouloir docile de ses semblables (armée, magistrature, administration publique) ; 2\textsuperscript{o} des métiers où l’on \emph{enseigne}, où l’on communique sa pensée par une sorte d’électrisation spirituelle à l’esprit d’autrui (clergé, professeurs, orateurs, grands publicistes) \phantomsection
\label{v1p249} ; 3\textsuperscript{o} des métiers, je veux dire des arts où l’on \emph{émeut}, où l’on fait vibrer les sensibilités étrangères à l’unisson de sa propre sensibilité (poètes, littérateurs, artistes).\par
Le corps des journalistes, depuis le commencement du siècle, s’est recruté, en moyenne, dans des milieux de moins en moins recommandables ; ce qui aurait dû le faire baisser beaucoup dans l’estime générale ; mais, comme, d’autre part, l’action contagieuse qu’il exerce n’a cessé de s’étendre par la facilité des communications et la diffusion graduelle du journal, le métier de journaliste, somme toute, a grandi dans l’opinion. Et, quant aux journalistes de marque, ils comptent, de plus en plus, parmi les étoiles de première grandeur des constellations nationales. — Dans certains pays, notamment aux États-Unis, où l’industrie offre de si grands avantages à tout individu entreprenant, le rebut seul de la jeunesse instruite s’adonne aux carrières administratives, qui sont, par suite, moins honorées qu’ailleurs ; mais, comme, à mesure que grandit et se centralise la République des États-Unis, le pouvoir des fonctionnaires s’y étend, on commence déjà à voir croître et monter, là comme ailleurs, le prestige des fonctions publiques, bien qu’il reste toujours fort au-dessous de celui de maître d’hôtel, profession pour laquelle les Américains expriment une admiration profonde.\par
Aussi, quand un métier, qui par lui-même n’exerce aucune action inter-mentale, un métier manuel quelconque, veut s’élever en grade dans l’estime de tous, n’a-t-il rien de mieux à faire que de s’organiser en ghilde, en corporation, comme au moyen âge, en syndicat, comme de nos jours ; parce qu’alors, outre l’action inter-mentale qui s’exerce réciproquement entre ses membres, le groupe ainsi formé possède et déploie un certain pouvoir dans l’État, il impressionne et suggestionne le public. Au moyen âge, les commerçants ont été longtemps plus considérés que les artisans ; n’est-ce pas, par hasard, parce que les ghildes commerciales, ligue hanséatique ou autres, ont précédé d’une centaine  \phantomsection
\label{v1p250}d’années les ghildes industrielles ? De notre temps, au contraire, la grande industrie est plus honorée que le grand commerce, et les syndicats industriels ou agricoles ont précédé les syndicats commerciaux, qui se traînent à leur suite.\par
Ce n’est pas l’union seulement, c’est aussi le rapprochement physique, qui fait la force et le pouvoir. Aussi les anciennes corporations avaient-elles soin de se grouper dans la même rue, ou le même quartier. Toutes les villes de France avaient leur \emph{rue des bouchers} (elle existe encore à Limoges), leur rue des boulangers, etc. En Angleterre, il en était de même. « Les membres de chaque métier, nous dit Ashley dans son histoire des doctrines économiques anglaises, vivaient généralement dans la même rue ou dans le voisinage. Ainsi à Londres, les selliers habitaient autour de Saint-Martin le Grand, et en étaient les paroissiens ; les lormiers vivaient dans Cripplegate, les tisserands dans Carmon-Street, les forgerons dans Smithfield. A Bristol, il y avait la rue des foulons, la rue du blé, la rue des couteliers, l’allée des bouchers, l’allée des cuisiniers, et ainsi de suite. » Un groupement semblable devait fortifier considérablement le sentiment de la vie corporative, car il permettait aux professionnels de s’influencer continuellement et réciproquement. — Soyons certains que les anciennes ghildes et corporations, de même que les syndicats actuels, ont été suscités, consciemment ou inconsciemment, beaucoup moins par la pensée d’un accroissement de bénéfice ou de salaire, que par le désir d’un rehaussement de considération. Et, de fait, ces associations ont toujours mieux réussi à grandir en honneur et en pouvoir qu’en richesse. Économiquement, du reste, la chose importe : on supporte plus aisément, \emph{on sent moins} la fatigue ou l’ennui d’une profession quand elle devient plus considérée. Il y a des travaux — administratifs et ministériels, par exemple, — qui ne seraient pas supportables, tant ils sont fastidieux et fatigants, s’ils n’étaient pas l’objet d’un certain respect.\par
 \phantomsection
\label{v1p251}Donc, toutes choses égales d’ailleurs, un métier où l’on vit groupés, d’une vie sociale intense, est supérieur en considération à un métier où l’on travaille isolés. Ajoutons qu’un métier urbain, même où l’on n’est pas groupés, mais qui s’exerce dans un milieu où l’action inter-mentale est à son plus haut point d’intensité, est plus considéré, aux yeux même d’un campagnard, qu’un métier rural. Si, par une exception, heureusement très rare encore, un paysan se pique d’avoir des cartes de visite à l’exemple de la ville, il s’y qualifiera \emph{buraliste} ou \emph{épicier.} A sa place je serais bien plus fier d’être un fin laboureur, un cultivateur riche et indépendant ; mais non, ce qui le relève à ses yeux, c’est d’avoir une petite boutique, à l’instar de la ville voisine, et d’y vendre quelquefois, le dimanche, une livre de sucre ou un paquet de bougies.\par
Conformément à ce qui vient d’être dit plus haut, toutes les révolutions politiques qui changent la répartition du pouvoir entre les diverses classes, entre les diverses professions, ont leur contre-coup sur le degré de considération qui s’y attache. Quand l’ancienne aristocratie des cités grecques antiques a été abattue, au {\scshape vi}\textsuperscript{e} siècle avant J.-C., par les tyrans, ces sortes de petits Césars précurseurs des démocraties, nous ne sommes pas surpris d’apprendre par M. Guiraud que « l’établissement de la tyrannie a eu pour effet de rehausser dans tout le monde grec la condition des travailleurs ». Dans les républiques démocratiques, ce changement s’accentua, et il eut pour conséquence de rehausser bien plus encore la condition des ouvriers urbains que celle des agriculteurs. A Athènes, la majorité des assemblées populaires se composait de foulons, de cordonniers, de charpentiers, d’artisans quelconques, « et il résulte d’un texte d’Aristophane que les campagnards s’y trouvaient généralement en minorité ».\par
Jusqu’à la première moitié environ du {\scshape xix}\textsuperscript{e} siècle, en dépit de la Révolution française, l’inégalité de considération qui  \phantomsection
\label{v1p252}séparait les métiers manuels des professions libérales était restée immense, et même, entre les divers métiers manuels, la différence des rangs était très fortement sentie. C’était le temps où, sans encourir de ridicule, le compagnonnage pouvait se refuser à admettre dans son sein les cordonniers, par exemple, comme indignes et infâmes. Mais à partir du moment où tous les métiers ont participé également au pouvoir politique par la pratique du suffrage universel, on a vu s’amoindrir, de même qu’à Athènes, la distance entre eux au point de vue de la considération ; en effet, celle des professions qui auparavant avaient le monopole du pouvoir s’est abaissée relativement, et celle des professions qui, après avoir été exclues du pouvoir, y ont accédé, s’est accrue d’autant.\par
Cette égalisation démocratique des rangs entre les professions a une grande importance économique, car elle rend seules possibles les grandes fédérations corporatives, qui groupent ensemble, aux États-Unis et sur le continent européen, de nombreux corps de métiers, longtemps séparés autrefois par l’esprit \emph{inégalitaire} qui régnait entre eux.\par
Enfin, il est à remarquer qu’une profession s’élève dans l’esprit public quand les moyens dont elle dispose pour atteindre ses fins viennent à s’accroître par suite de certaines inventions. Le génie inventif, pareil à l’Esprit « qui souffle où il veut », féconde aujourd’hui telle profession, demain telle autre ; et, suivant ses caprices, qui ne sont pas cependant sans être dominés par des lois logiques, les métiers qu’il a favorisés se trouvent rehaussés dans l’opinion. C’est la cause évidente pour laquelle la profession d’ingénieur a si fort grandi en considération pendant le {\scshape xix}\textsuperscript{e} siècle ; les inventions multipliées de l’industrie ont eu cet effet, de même que les éclairs répétés du génie militaire pendant le premier Empire avaient contribué à rendre si prestigieux le métier des armes.\par
La profession médicale était des plus humbles, sauf quelques \phantomsection
\label{v1p253} individualités brillantes, jusqu’à la fin du {\scshape xviii}\textsuperscript{e} siècle. Il a fallu les découvertes nombreuses qui ont renouvelé la biologie pour l’élever au rang des professions les plus hautement honorées.\par
C’est probablement la cause pour laquelle une carrière, comme je l’ai dit plus haut, se recrute parfois dans des couches de plus en plus élevées de la population. N’est-ce pas parce que des jongleurs avaient eu d’heureuses inspirations et doté leur art de quelques beautés nouvellement découvertes que des individus \emph{mieux nés} y ont été attirés ?\par
Je n’ose prétendre avoir résolu, par ce qui précède, tous les problèmes soulevés par le degré inégal et le degré changeant d’honorabilité qui s’attache aux divers genres de travaux. C’est un sujet presque inexploré et qui appelle de longues recherches. Je crois cependant qu’en combinant ensemble les remarques générales ci-dessus indiquées, on parvient à résoudre certaines difficultés. Mais, avant tout, il faut avoir égard à l’évolution des croyances religieuses, car c’est elle, encore plus que l’évolution des intérêts économiques, qui rend infâmes tels métiers pratiquement recherchés, ou entoure de respects profonds l’accomplissement de besognes sans utilité pratique. Dans une certaine mesure, ces deux évolutions, celle des dogmes ou des principes et celle des intérêts, s’influencent réciproquement et tendent à se mettre d’accord, mais, dans une certaine mesure aussi, elles sont indépendantes ; et c’est leur indépendance, très importante à noter dans tous les domaines de la science sociale, qui explique les bizarreries présentées par l’inégalité des jugements portés, dans divers pays ou dans diverses époques, sur l’honorabilité ou l’indignité des professions. Si l’on recherchait, pour chaque pays et pour chaque époque, quelle a été la profession la plus respectée et quelle a été la profession la plus méprisée, on constaterait presque toujours que l’origine de cette vénération ou de cette infamie est toute religieuse.
 \phantomsection
\label{v1p254}\subsubsection[{I.5.e. Classification générale des travaux.}]{I.5.e. Classification générale des travaux.}
\noindent Les esprits nés classificateurs, espèce nombreuse, ne me pardonneraient pas d’avoir abandonné le sujet qui nous occupe sans avoir au moins esquissé une classification des travaux humains. Tout travail, avons-nous dit, suppose un but, des moyens et des obstacles. On doit donc classer les travaux d’après leurs buts, d’abord, c’est-à-dire d’après les besoins auxquels ils répondent, et ensuite d’après les moyens qu’ils emploient ou les obstacles qu’ils ont à vaincre pour atteindre leurs fins. — Les besoins sont organiques ou sociaux. Mais, à vrai dire, il n’est pas un besoin si organique qu’il ne revête une forme sociale ; ni si social qu’il n’ait son fondement dans l’organisme. D’une de ces deux catégories à l’autre, on passe par des degrés insensibles. Les besoins où le caractère organique domine, mais domine de moins en moins, — s’alimenter, se vêtir, s’abriter, se chauffer, s’éclairer, s’asseoir, se coucher, se déplacer, se bien porter, s’amuser, se reproduire, — se traduisent en désirs précis d’être nourri, vêtu, logé, chauffé, éclairé, meublé, transporté, soigné, amusé, marié, conformément à la mode ou à la coutume régnante dans son cercle ou dans sa classe\footnote{ \noindent Entre les deux catégories de besoins s’interpose le besoin d’amour physique, qui est leur trait d’union, ainsi que les besoins de maternité et de paternité. Les industries qui les concernent ont lieu d’occuper le moraliste beaucoup plus que l’économiste. Aussi est-il inutile de s’en occuper ici.
 }. Puis viennent les besoins où le caractère social est prédominant et prédomine de plus en plus, mais où s’exprime avec une grande richesse de couleurs un même instinct de sympathie naturelle, toute physiologique, du semblable pour le semblable, de l’assimilé pour l’assimilé : étendre, compliquer, raffiner, ses communications mentales avec les autres hommes (en parlant mieux sa langue, en apprenant les langues étrangères, en s’instruisant le  \phantomsection
\label{v1p255}plus possible) ; se faire respecter et considérer des autres hommes (en défendant ses droits contre eux ou appuyant une autorité publique pour la défense des droits de tous, en acquérant de nouveaux droits, en s’enrichissant, en parvenant au pouvoir) ; se faire aimer des autres hommes (en pratiquant la bienfaisance, en se dévouant) ; charmer les autres hommes, les élever et s’en faire applaudir (en propageant le culte d’un art ancien ou le goût d’un beau nouveau).\par
Classés d’après les moyens qu’ils emploient pour réaliser ces fins, les travaux se divisent d’abord en deux grandes branches : ceux où domine la dépense de force musculaire, et ceux où domine la dépense de force nerveuse. En parcourant l’énumération précédente des besoins humains, on verra que le travail employé à les satisfaire commence par être presque entièrement musculaire et finit par être à peu près exclusivement nerveux. On passe par degrés du premier genre de travail, où la fatigue (musculaire) est plus à craindre que l’ennui, au second où l’ennui est plus à éviter que la fatigue. Dans les travaux artistiques le pire des défauts est que le travailleur se soit ennuyé en les exécutant : son œuvre en reste glacée et guindée, insipide au spectateur comme à lui-même. Ruskin en fait la remarque à propos des ornements qui décorent les édifices. « Je crois\footnote{ \noindent \emph{Les sept lampes de l’architecture} (trad. franç.).
 }, dit-il, que la véritable question à poser touchant tout ornement est simplement celle-ci. A-t-il été fait avec joie, l’artisan était-il heureux en y travaillant ? Ce peut être le travail le plus pénible possible, mais il faut que l’ouvrier ait été heureux, ou l’œuvre ne sera pas vivante. On a récemment édifié une église gothique près de Rouen ; elle est, à dire vrai, assez vile dans sa composition générale, mais excessivement riche en détails. La plupart de ceux-ci sont dessinés avec goût et, de toute évidence, sont l’œuvre d’un homme ayant  \phantomsection
\label{v1p256}de près étudié les travaux d’autrefois. Mais c’est tout aussi mort que les feuilles en décembre : il n’y a pas sur toute la façade une seule touche tendre, une seule touche ardente. Les hommes qui l’ont faite l’avaient en haine et furent contents d’en avoir fini. Tant qu’on travaillera de la sorte, on ne fera que surcharger vos murailles de formes d’argile. Les guirlandes de lierre du Père-Lachaise sont une décoration plus gaie. » Si cette condition subjective est requise absolument pour le mérite objectif, pour la bonne exécution d’ouvrages de maçonnerie, à plus forte raison l’urgence s’en fait sentir dans les œuvres du sculpteur, du peintre, de l’écrivain, même là où il ne s’agit que du travail courant, en quelque sorte, de leur métier. Écrivez avec effort, soit, mais jamais avec ennui.\par
Continuons. Le travail principalement musculaire se subdivise en travaux qui atteignent leurs fins (répondant à des besoins principalement organiques) par le moyen de la captation et de la direction de forces physiques empruntées soit à des hommes, soit à des animaux, soit à des plantes, soit à des matières inorganiques qu’il s’agit, dans ce cas, d’extraire, de transporter et de transformer : autant de besognes différentes. Le travail principalement nerveux (répondant à des besoins principalement sociaux) se subdivise en autant de travaux distincts qu’il y a de genres d’action inter-spirituelle. Comme il y a trois sortes d’éléments psychologiques, combinés à doses inégales dans tous les états intimes — la sensation, la croyance et le désir, — il y a trois grandes classes d’actions inter-spirituelles qui tendent à susciter chez autrui les états intimes où l’un d’eux est dominant, c’est-à-dire à \emph{impressionner}, à \emph{convaincre} ou à \emph{décider.} Les beaux-arts ont pour caractère d’être expressifs et impressionnants ; le professeur dans sa chaire, le publiciste dans son livre ou son journal, veulent être convaincants, persuasifs ; l’orateur parlementaire, le tribun, le prédicateur, cherchent avant tout à provoquer des décisions, des résolutions. Le ministre  \phantomsection
\label{v1p257}décide et commande, le savant affirme et démontre, le peintre ou le poète frappe la rétine ou l’imagination de couleurs neuves.\par
A un point de vue plus général, on a l’habitude de distinguer, dans toutes les professions, deux sortes de moyens mis en œuvre : les \emph{outils} et la \emph{matière première.} Mais observons que c’est une erreur de croire qu’il y a toujours, dans une industrie, une matière première. Y en a-t-il dans l’industrie pastorale ou agricole ? Non. La couvée ou la semence dont le pâtre ou le laboureur surveille la croissance, n’est nullement l’équivalent du bois que le menuisier varlope, du fer que lime le serrurier : il s’agit, pour le pâtre et le laboureur, non de transformer artificiellement quelque chose mais de laisser s’opérer un développement naturel. La terre où la semence est plantée n’est pas une matière première non plus ; rien ne correspond ici à l’extraction, au transport, à la transformation du minerai de fer, de la pierre à bâtir, de l’argile à faire des tuiles. S’il y a nécessairement, dans l’industrie proprement dite, et surtout dans l’industrie machinofacturière, une matière première, c’est qu’il s’agit d’obtenir ici par la matière ce qu’on obtenait auparavant par la vie, humaine, animale ou végétale. Car le travail organique des tissus vivants, que l’éleveur de bestiaux, l’agriculteur, et aussi bien le médecin ou le chirurgien, surveillent et dirigent, consiste aussi, précisément, à transporter et à transformer des matières premières, inorganiques, par des procédés infiniment ingénieux. Mais on peut voir, si l’on veut, l’équivalent d’une matière première dans le roc dur des préjugés, des institutions antiques, des usages enracinés, que le savant et le philosophe travaillent à réformer, ou que le politicien exploite, ou que le juriste consacre. Quant aux habitudes de l’œil, de l’ouïe, du goût, que l’artiste respecte, sur lesquelles il s’appuie pour faire accepter ses innovations, elles rappellent bien moins la pierre que taille le maçon que le sol sur lequel la maison est bâtie. Elles sont bien moins  \phantomsection
\label{v1p258}une matière première qu’un emplacement. L’artiste se fonde là-dessus, mais il ne fabrique rien avec cela.\par
L’outillage des professions qui répondent à des besoins organiques, individuels, à des rapports corporels de l’individu avec l’ensemble de la nature, diffère beaucoup de l’outillage des professions qui répondent à des besoins sociaux, à des rapports inter-spirituels. Le premier peut être considéré comme un prolongement et un grossissement de nos mains, une spécialisation et une mobilisation de leurs modes d’action singulièrement fortifiés. Tels sont le marteau, les tenailles, la scie, etc. Le second présente bien aussi ces caractères (la plume, le pinceau, l’ébauchoir, etc.), mais ce n’est vrai que d’une faible partie des outils maniés par le travailleur intellectuel, à titre auxiliaire ; les grands, les véritables outils dont il fait usage, machine à imprimer, lithographie, photographie, photogravure, télégraphe, etc. sont le prolongement et le grossissement graduel, devenu immense : 1\textsuperscript{o} de ses organes d’expression, de sa physionomie, de ses gestes, et, avant tout, de sa voix articulée, où s’expriment ses pensées, ses volontés, ses émotions ; 2\textsuperscript{o} de ses sens, qui sont, en effet, prodigieusement développés par le microscope et le télescope, par le thermomètre, par bien d’autres instruments scientifiques.\par
Tous les outils, soit pour les travaux manuels, soit pour les travaux intellectuels eux-mêmes, sont, remarquons-le, des substances à l’état solide, et non à l’état liquide ou gazeux. Je n’aperçois pas une seule exception à cette règle. L’eau qui fait tourner le moulin, la vapeur qui fait mouvoir le cylindre de la locomotive, sont des forces mises en œuvre par des outils, mais ce ne sont pas des outils. L’encre dont se sert la plume de l’écrivain, la couleur à l’huile où se trempe le pinceau du peintre, sont liquides ou plastiques, mais la plume et le pinceau sont solides ; la lumière de la lampe qui éclaire le savant, l’électricité du télégraphe qui expédie les informations du journaliste, \phantomsection
\label{v1p259} sont des mouvements de molécules d’éther, mais la lampe et le télégraphe sont des corps durs. Pourquoi en est-il ainsi ? Parce qu’on ne s’appuie que sur ce qui résiste : la solidité, c’est la résistance et l’appui aussi bien. Outillage et solidité sont deux idées si intimement unies que, même dans le travail de la vie animale ou végétale, d’un bout à l’autre de l’échelle zoologique, nous constatons cette liaison indissoluble. Les outils de l’être vivant sont, pour chaque cellule, ses appendices ou ses expansions plus ou moins mobiles et toujours d’un tissu plus ou moins résistant, et, pour l’organisme tout entier, ses membres, toujours d’une certaine dureté relativement au reste du corps. Claude Bernard et toute son école ont insisté avec raison sur la distinction des parties liquides et des parties solides de l’organisme et fait résider dans les dernières seules tout l’exercice des fonctions proprement vitales\footnote{ \noindent Il ne saurait donc y avoir de travail organique, encore moins social, sur un astre en ignition, tel que le soleil.
 }. Bien mieux, il serait aisé de montrer que plus l’on s’élève sur l’échelle zoologique ou botanique, plus l’outillage animal ou végétal acquiert de dureté ; le contraste est grand, à cet égard, entre les champignons et les chênes, ou, encore plus, entre les méduses et les mammifères supérieurs. La plus grave difficulté de la vie, à ses débuts, a dû être de franchir le passage des êtres mous et gélatineux, par où elle a commencé, aux êtres à squelette extérieur, puis intérieur, qui sont venus ensuite. La sécrétion d’une substance calcaire ou siliceuse a été pour elle l’équivalent de ce qu’a été pour l’humanité primitive l’art de tailler le silex, et plus tard, la métallurgie. On comprend, en effet, d’après les considérations qui précèdent, l’importance singulière de ces découvertes. C’est avec raison qu’on fonde sur elles la distinction des grandes phases de l’humanité et qu’on dit : l’âge de la pierre éclatée, l’âge de la pierre polie, l’âge des métaux.\par
On peut classer les travaux d’après la nature des obstacles  \phantomsection
\label{v1p260}qu’ils ont à vaincre, non moins que d’après la nature des moyens qu’ils emploient pour les surmonter. Dans les industries pastorales ou agricoles, ou aussi bien médicales, qui font travailler pour nous les forces vivantes, l’obstacle est tantôt l’excès, tantôt le défaut de vitalité, en somme la difficulté qu’opposent les instincts et les besoins héréditaires, les impétuosités de la jeunesse, les ténacités de l’âge mûr, les inerties de la vieillesse, la mort enfin, à la direction qu’on veut imprimer. Dans l’industrie proprement dite, et d’abord dans l’industrie du transport, l’obstacle est soit le poids, soit le volume, soit l’adhérence, soit la fragilité de l’objet à transporter. Dans l’industrie de transformation, l’obstacle provient soit des affinités chimiques, soit des cohésions physiques, soit des forces motrices ou autres. Dans les travaux d’action inter-spirituelle, l’obstacle résulte d’abord de l’inattention de ceux à qui l’on s’adresse et qu’il s’agit de frapper ; puis, des idées ou des désirs, des sentiments quelconques à déraciner en eux, comme contraires au but qu’on poursuit, c’est-à-dire des actions inter-spirituelles opposées qui ont été exercées par d’autres sous la forme de coutumes anciennes et de modes nouvelles ou de caprice individuels.\par
Est-ce que la nature de l’outil ne doit par varier d’après la nature de l’obstacle ? Et la meilleure définition de l’outil ne serait-elle pas qu’il est un \emph{contre-obstacle ?} Si l’on se place à ce point de vue, ce qui vient d’être dit plus haut sur le caractère de solidité inhérent à l’outillage doit être entendu dans un sens tout spirituel en ce qui concerne l’outillage des professions supérieures. L’outil véritable, dans l’action inter-spirituelle, ne serait ni la plume ni la machine à imprimer, mais la rigidité des convictions, la dureté des passions, par lesquelles on bat en brèche le granit des convictions et des passions contraires. Un publiciste peut n’être lui-même ni convaincu ni passionné à la rigueur, mais, en tout cas, et c’est là l’essentiel, il se sert toujours de convictions ou de passions extérieures, de préjugés et de partis  \phantomsection
\label{v1p261}pris condensés en institutions solides et tenaces, il se sert d’abord de la plus ancienne et de la plus résistante des institutions nationales, la langue, cet admirable outil, pour parvenir à en détruire d’autres. Une révolution sociale n’est possible qu’à raison de la stabilité, de la solidité des institutions sur lesquelles on s’est appuyé — langue, dogmes politiques, dogmes religieux, coutumes — pour pulvériser d’autres principes, d’autres préjugés, d’autres mœurs.\par
On peut classer les travaux non seulement d’après leurs fins, leurs moyens et leurs obstacles, mais encore à d’autres points de vue, par exemple, d’après la manière dont ils sont utilisés. Tous les travaux procurent des services, mais il est des services qui sont utilisés directement, d’autres qui ne le sont qu’indirectement, par les produits où ils se matérialisent et s’incarnent. Distinguons donc les \emph{travaux-services} et les \emph{travaux-produits ;} et voyons si la proportion de ces deux grandes classes de travaux est restée la même. N’est-il pas manifeste que la proportion relative des travaux-produits va grandissant et celle des travaux-services diminuant au cours du progrès économique ? Un patricien de l’antiquité, un seigneur du moyen âge ne parvenait à satisfaire ses besoins et ses caprices que moyennant les services rendus par des légions d’esclaves ou des nuées de valets. L’homme riche de nos jours achète dans divers magasins, sous forme d’articles variés, la satisfaction de besoins et de fantaisies encore plus multiples, et un ou deux domestiques lui suffisent.\par
On peut distinguer encore les \emph{travaux-produits} d’après la nature du produit. Le travail artistique se distingue des autres en ce que, moyennant une série d’actes similaires et répétés, il aboutit à une œuvre qui, elle, n’est pas destinée à être reproduite par lui, quoiqu’elle le soit parfois avec des variantes profondes, car elle prétend être une création originale et unique, et elle mérite d’autant mieux le nom d’œuvre d’art qu’elle justifie davantage cette prétention. \phantomsection
\label{v1p262} Mais les travaux industriels, moyennant des séries d’actes similaires et répétés, reproduisent à l’infini des séries d’ouvrages non moins semblables. Le passage de la première catégorie à la seconde est d’ailleurs graduel et comporte d’innombrables degrés de transition. Une distinction assez familière aux ouvriers a quelque rapport avec la précédente : ils distinguent le \emph{travail courant} et le travail non courant. Par le premier, ils entendent une tâche toujours la même qu’il est facile de soumettre à un tarif, tandis que le second se prête bien plus malaisément à une évaluation faite d’avance : pour le premier, on fera travailler l’ouvrier \emph{à prix fait ;} pour le second, à la journée ou à l’heure. Or, il est à remarquer que la proportion de ces deux genres de travaux va changeant, toujours dans le même sens, au cours du progrès industriel. Le travail courant empiète sans cesse sur le travail non courant, c’est-à-dire sur le travail artistique ou du moins sur le travail imprévu, pittoresque, qui fait un appel constant à l’ingéniosité du travailleur. A mesure que s’assimilent les besoins, tout en se multipliant, les produits qui les satisfont peuvent être exécutés non plus sur commande individuelle, mais sur commande générale, et leur fabrication devient \emph{courante.} Le développement des magasins de confection, où tout le travail est courant, refoule les anciennes échoppes où tout se faisait sur mesure. Par là, on pourrait dire que la proportion du travail ennuyeux grandit sans cesse, avec la civilisation, aux dépens du travail attrayant, si l’on oubliait que les machines se chargent de plus en plus de toute cette besogne qui serait fastidieuse pour des hommes.\par
Cette classification des travaux, tout incomplète qu’elle est, suffit à nous montrer leur prodigieuse hétérogénéité, et, par suite, l’extrême difficulté de résoudre le problème de leur juste rémunération. Il s’agit, pour satisfaire le besoin de justice, qui devient toujours plus urgent, de trouver une \emph{mesure commune} à ces travaux si hétérogènes : véritable  \phantomsection
\label{v1p263}quadrature du cercle qui s’impose à nous. On ne peut résoudre ce problème qu’en le tournant. Ce n’est ni la durée, ni l’intensité du travail qui peut suffire à fournir le mètre commun que l’on cherche. Est-ce que le degré d’\emph{insipidité}, d’ennui, ne devrait pas aussi entrer en ligne de compte ? Oui, assurément, et aussi le degré de considération et d’honorabilité. Le prix d’un genre de travail croît souvent en raison inverse de la considération qui lui est inhérente ; c’est en partie pour cela que, dans notre siècle, de tous les salaires féminins, celui des servantes est le plus rétribué et s’est élevé bien plus que celui des ouvrières. Le degré d’\emph{insalubrité}, de \emph{périllosité}, si l’on me permet ce néologisme, mérite aussi d’arrêter l’attention. Mais, avant tout, c’est au prix courant du produit ou du service, résultant du travail, que se mesure et se mesurera toujours la valeur du travail producteur ou servant. Et les variations de prix du produit ou du service régleront toujours, en l’absence d’autres règles plus rationnelles, les variations du salaire. Mais ne parlons pas encore des salaires dont le sujet se rattache à [{\corr la}] théorie générale des prix.
\subsubsection[{I.5.f. Transformations historiques du travail.}]{I.5.f. Transformations historiques du travail.}
\noindent Demandons-nous maintenant s’il y a \emph{une} ou \emph{plusieurs} évolutions historiques du travail, et quelle est leur explication générale.\par
D’abord, commençons par remarquer une singularité que nous offre l’homme primitif. La plupart des sauvages sont paresseux, et l’on explique, en général, leur paresse par leur absence de mémoire et de prévoyance. Ils oublient leurs besoins passés, leur faim et leur soif d’hier, quand ils n’en souffrent plus momentanément, et ils ne prévoient pas leurs besoins futurs. Mais, ce qui est bien étrange, ces êtres si oublieux et si imprévoyants sont en même temps les  \phantomsection
\label{v1p264}êtres les plus vindicatifs du monde ; ce qui signifie qu’ils gardent le souvenir opiniâtre des injures remontant au plus lointain passé et qu’ils songent, en se vengeant, à se garantir contre la menace d’injures nouvelles à l’avenir. Ainsi, oublieux des maux les plus douloureux, de la faim, de la soif, du froid durement subis, ils se souviennent des moindres piqûres d’amour-propre ; et, imprévoyants pour tout le reste, ils prévoient très nettement les moindres périls que leur honneur peut avoir à courir. Ce contraste serait inexplicable si l’on n’avait présent à l’esprit le trait peut-être le plus caractéristique du sauvage, son extraordinaire amour-propre, sa prodigieuse et extravagante préoccupation de l’opinion des siens.\par
A première vue, on peut s’étonner que la puissance de l’opinion publique se montre ainsi plus forte et plus tyrannique là où le public est le moins nombreux et même le plus dispersé physiquement. Mais sa dispersion physique ne l’empêche pas d’être socialement très dense par les similitudes et les solidarités qui unissent les membres d’un même clan ou d’une même tribu, par l’intensité de leurs actions inter-mentales, par les liens du sang qui s’ajoutent à ceux d’une éducation commune et qui compensent bien au delà leur petit nombre. A quel point le sauvage dépend de l’opinion d’autrui, modèle sur l’opinion d’autrui l’idée qu’il a de lui-même, et prend pour son être réel son être fictif quand cette fiction s’est accréditée dans son entourage, on le voit par ce qui se passe quand un prisonnier de guerre, au lieu d’être scalpé, est, exceptionnellement, adopté par une famille ennemie, accueilli par les femmes, couché par elles sur la natte d’un mort dont il prend le nom et est censé être la réincarnation. Dès lors, il s’attache sincèrement à sa nouvelle famille, à sa nouvelle nationalité, et combat, avec ses nouveaux concitoyens, ses anciens compatriotes qui, puisqu’il s’est laissé capturer, le tiennent pour mort. Cette double fiction, de sa mort civique dans sa patrie d’origine, et de  \phantomsection
\label{v1p265}sa métempsycose dans sa nouvelle patrie, devient sa règle de conduite, profondément écrite au fond de son cœur.\par
Le sauvage est donc tout amour-propre, et n’est prévoyant et mémoratif qu’en ce qui touche à l’honneur. Mais, chez l’homme qui se civilise, le champ de la mémoire et de la prévoyance s’étend, et le progrès de sa mémoire générale s’accompagne du déclin d’une mémoire particulière, celle des injures. A mesure qu’il devient plus prévoyant en fait de besoins, il devient moins vindicatif. Par suite, il devient plus laborieux et plus paisible, plus paisible parce que plus laborieux, et plus laborieux parce que plus paisible.\par
Mais quelle est la nature du travail qui s’impose d’abord à l’homme, et quel est l’ordre dans lequel se succèdent les divers genres de travaux qui prédominent aux époques successives du développement d’un peuple ? Le point de départ est-il partout le même, et la route suivie est-elle la même pour tous ? Nullement. Le point de départ diffère beaucoup d’après la nature du sol et du climat où une race commence à entrer dans les voies du travail, et aussi d’après les aptitudes et les tendances de cette race. Ici le premier travail sera la cueillette des fruits spontanés d’un sol privilégié, comme dans certaines îles de l’Océanie ou dans certaines vallées de l’ancien continent. Là ce sera la pêche, et tel ou tel genre de pêche ; ailleurs la chasse, et telle ou telle chasse très différente d’après la faune de la contrée ; ou bien, et le plus souvent simultanément, quelques rudiments d’art pastoral et même d’agriculture. L’école de Le Play n’a donc point tort d’insister sur l’importance de cet élément géographique et climatérique, au début de l’évolution économique du moins. Son erreur est de vouloir faire découler de là, comme si c’était le facteur constamment dominant de la vie sociale, le caractère domestique, politique, moral, esthétique, de la nation, à jamais condamnée à telle constitution, à telles institutions de tout genre, parce que son habitat est propre au pâturage ou à la culture des céréales. La  \phantomsection
\label{v1p266}nature du travail, et les changements survenus dans la nature du travail, à partir de ses débuts, dépendent, avant tout, de l’invention. Il n’est pas une des formes primitives du travail qui n’ait été profondément transformée d’âge en âge par des inventions successives et qui n’ait été rendue par elles applicable à des territoires et à des climats nouveaux. La \emph{cueillette} n’a pas disparu. La \emph{cueillette}, réduite d’abord, dans les forêts, au fait de manger des baies ou de ramasser des branches tombées, est devenue peu à peu l’exploitation des grands arbres séculaires en vue des constructions navales, par les scieries mécaniques ; après avoir consisté à ramasser des éclats de silex à fleur de terre, elle est devenue l’extraction des minéraux, et de minéraux plus variés, à des profondeurs de plus en plus grandes. Combien de pays, à richesses minérales profondes, à présent exploitées par une population minière dense et prospère, auraient été des déserts incultes chez nos aïeux des âges paléolithiques ou néolithiques ! La \emph{pêche} a subi moins de transformations ; pourtant, l’invention de la navigation à voiles, la découverte de nouvelles îles et de nouvelles mers (de Terre-Neuve par exemple) l’ont singulièrement élargie et diversifiée. Notons aussi les progrès de la pisciculture, dus aux découvertes de nouvelles espèces de poissons à répandre dans les eaux des lacs et des rivières, et de nouveaux procédés pour les répandre. La \emph{chasse} a été métamorphosée, on peut le dire, par la domestication de certains animaux, le chien, le faucon, par chaque arme nouvellement inventée, arc et arbalète, fusil, qui ont tant contribué à la destruction des bêtes fauves, condition première de la civilisation dans une contrée. Enfin, le \emph{pâturage} a été renouvelé, soit par de nouvelles espèces d’animaux domestiques, soit par de nouvelles plantes propres à les nourrir : aux prairies naturelles se sont ajoutées les prairies artificielles qui ont étendu le pâturage et l’élevage aux régions primitivement les moins propres à ce travail. — Et je n’ai  \phantomsection
\label{v1p267}parlé que des formes de la production réputées les plus stationnaires, les plus réfractaires au progrès ; je n’ai rien dit ni de l’agriculture ni de l’industrie proprement dite, parce qu’il est trop clair que le génie inventif est la grande cause de toutes leurs révolutions et de toutes leurs évolutions mêmes.\par
Par évolution ou révolution du travail, il faut entendre, non seulement l’apparition de nouvelles formes ou la disparition d’anciennes formes du travail, mais encore les changements survenus dans la proportion numérique des diverses catégories déjà existantes de travaux et de travailleurs. Pourquoi telle catégorie de travaux, les métiers commerciaux ou les métiers industriels ou les professions libérales, va-t-elle se développant à telle époque et en tel pays, ou s’amoindrissant à telle autre ? Et, plus spécialement, pourquoi telle branche du commerce ou de l’industrie, telle profession libérale, grandit-elle plus vite que les autres, ou décroît-elle, à telle époque et en tel pays ? Pourquoi, par exemple, le nombre proportionnel des fonctionnaires a-t-il augmenté en France au {\scshape xix}\textsuperscript{e} siècle et va-t-il en augmentant toujours ? La réponse à une telle question ne peut être que très complexe, les causes qui font se précipiter l’élite d’une nation tantôt vers telles carrières tantôt vers telles autres sont multiples et variées ; et cette complexité, cette variété, a déjà une signification importante, car elle montre l’impossibilité d’assujettir à une formule uniforme et réglée d’évolution la série si pittoresque, si capricieuse de ses changements. On peut seulement relever l’uniformité habituelle, non contestée, d’un certain ordre de succession dans l’importance tour à tour dominante des grandes classes de travaux considérés \emph{in abstracto.} Cet ordre historique bien connu se reproduit encore, comme par une image en raccourci et en abrégé, aux États-Unis, quand un territoire nouveau est colonisé. « Dans l’ordre de l’assiette et du développement de la contrée, dit M. Paul Leroy-Beaulieu, d’après un document \phantomsection
\label{v1p268} américain, les industries se succèdent ainsi : le chasseur ; après le chasseur, le \emph{trapper} (preneur d’animaux au piège) ; le berger ou le gardien de troupeaux suit, et l’élevage du troupeau est pour un temps l’industrie dominante, puis l’agriculture et les manufactures... » Mais, si intéressante que soit cette constatation faite dans une région déterminée, elle ne donne lieu qu’à une formule bien vague, non sans exception, et ne fournit au problème posé ci-dessus qu’une réponse bien insuffisante. Plus on s’élève sur l’échelle des professions, jusqu’aux plus vraiment sociales, qui consistent en actions inter-spirituelles, et plus on les voit échapper à toute règle préconçue de succession, à toute loi d’évolution un peu précise.\par
Mais, à défaut d’une loi d’évolution qui permette de prédire l’ordre dans lequel se succéderont, en détail, les diverses natures de travail dans un pays nouveau, il nous est permis de formuler des \emph{lois de cotisation} qui s’appliquent aux itinéraires les plus variables pour expliquer chacun de leurs tracés, comme la loi de la pesanteur s’applique toujours la même à toutes les chutes de pierre, de canons, d’obus et de corps quelconques, si bizarre que puisse être leur chemin aérien. Nous avons déjà indiqué quelques-unes de ces lois de causation à propos des causes qui font varier le degré de considération attaché aux diverses professions. Disons, avant tout, que la proportion numérique des individus adonnés aux diverses professions varie en raison de l’accroissement ou décroissement relatif du désir général satisfait par chacune d’elles, ou de la considération attachée à chacune d’elles, et que ce désir relatif s’accroît en raison des facilités qu’il trouve à se satisfaire par suite d’inventions nouvelles qui ont créé de nouveaux produits propres à lui procurer une meilleure satisfaction, ou qui ont abaissé le prix des produits anciens. Il n’y a pas d’autre raison, par exemple, du prodigieux développement qu’a reçu l’industrie des transports depuis l’invention des chemins de fer.
 \phantomsection
\label{v1p269}\subsubsection[{I.5.g. Autres aperçus sur la périodicité des travaux.}]{I.5.g. Autres aperçus sur la périodicité des travaux.}
\noindent On le voit, le sujet que nous venons d’indiquer, les transformations du travail, se rattache plutôt à la troisième partie de ce cours, à l’adaptation économique, où il sera traité du rôle de l’invention. Nous l’abandonnons donc pour le moment, pour nous occuper d’un autre sujet, qui appartient intimement à notre première partie, et qui a déjà été effleuré plus haut, celui de la \emph{périodicité} du travail.\par
Le cycle des travaux, chez les primitifs, consiste, pour chaque travailleur, à achever son œuvre jusqu’au bout, puis à la recommencer. Un vannier fait un tour de roue de son cycle productif chaque fois qu’il a fini une corbeille ; un tonnelier, chaque fois qu’il achève un fût ; un pêcheur, chaque fois qu’il vient de jeter un coup d’épervier et de ramener son filet ; un serrurier, chaque fois qu’il vient de faire une clé, ou une serrure, etc. Cette période est, en général, très courte, le nombre des actes différents dont la succession constitue le cycle productif étant très petit. Il y a des exceptions : l’ouvrier d’Orient qui fait à la main un châle ou un tapis y consacre souvent bien des mois ; après quoi il recommence. On sait le temps qu’a mis Pénélope à tisser sa toile. Mais, de même qu’en astronomie la rotation de la terre autour d’elle-même se complique de la rotation de la terre autour du soleil, ellipse bien plus majestueuse dont ces rotations sont des fractions élémentaires, de même ici, outre ces recommencements d’un même travail plusieurs fois par jour ou par semaine, nous observons, en agriculture et même en industrie, un recommencement d’une même série de travaux différents au bout d’une période plus ample, d’une \emph{période annuelle} en général.\par
C’est là d’abord un cycle tout individuel ; mais quand la division du travail partage entre plusieurs ouvriers, entre plusieurs groupes d’ouvriers, souvent fort distants les uns  \phantomsection
\label{v1p270}des autres, l’accomplissement total d’une même œuvre, le cycle productif total devient collectif, et, en général, la durée de sa période de rotation s’allonge, mais en même temps la durée de la fraction d’œuvre que chaque ouvrier fragmentaire accomplit, va diminuant, et il tourne de plus en plus vite dans un cercle de plus en plus étroit d’opérations dont la série se répète incessamment, pendant que le cercle total, formé par un enchaînement régulier de ces petits cercles partiels, va s’agrandissant, sinon se ralentissant toujours.\par
Il y a une autre différence à noter entre la production primitive et la production civilisée : au début, tout travail étant manuel et supposant un certain degré d’ingéniosité, une œuvre faite (panier, bas, chapeau, vêtement, etc.) n’était jamais recommencée exactement pareille par le même ouvrier, et, à plus forte raison, par des ouvriers différents. Chaque fois, il variait un peu, comme un demi-artiste qu’il était. Mais, plus la production s’est développée, et plus la répétition du produit est devenue exacte ; la fabrication par les machines a poussé à bout cette évolution. Par les machines, la séparation a été tranchée entre l’élément-répétition, proprement industriel, et l’élément-variation, qui a quelque chose d’artistique. Les degrés de cette séparation, commencée dès l’âge de la pierre polie, ou peut-être même éclatée, sont incessants à suivre. L’artisan du moyen âge n’est plus qu’à moitié chemin de cet itinéraire. L’artiste actuel et la machine actuelle sont les termes extrêmes de cette bifurcation. Quant à l’ouvrier actuel, dernier vestige de l’artisan, il est destiné à être absorbé de plus en plus soit par l’artiste, soit par la machine, — sauf, bien entendu, l’ouvrier mécanicien. L’ouvrier mécanicien, c’est la machine en acte, comme le cavalier c’est le cheval utilisé.\par
Chaque phase économique peut être caractérisée par l’ampleur, la durée, la complexité du cycle productif total qui lui correspond. Il devient de plus en plus ample,  \phantomsection
\label{v1p271}de plus en plus complexe, de plus en plus régulièrement périodique. Mais, au point de vue de la durée, il est une limite vite atteinte, rarement dépassée, qui tend à dominer et donner le ton, même en dehors du domaine agricole où elle s’impose nécessairement : c’est le cycle annuel. La production agricole a un cycle enfermé dans le laps de temps exigé par la floraison et la fructification végétales. Ce laps de temps, c’est l’année. Le croît du bétail est assujetti à cette même période. La rotation productive, en fait d’évolution végétale ou animale, ne saurait guère être abrégée. En fait de fabrication industrielle, elle peut l’être à la vérité, mais les besoins auxquels chaque industrie correspond, celle des vêtements chauds ou frais par exemple, se reproduisent à des époques fixes de l’année et condamnent, par suite, le fabricant à des alternatives annuelles de morte saison et de saison d’activité. C’est ainsi que le cours des astres tient sous sa dépendance le cours du travail humain, aussi bien que le cours du travail animal ou végétal, et sert de métronome à ce labeur rythmique.\par
Au point de vue théorique, il y a grand avantage à confondre dans la même expression de \emph{travail}, pour un instant, toutes les activités vivantes, soit végétales, soit animales, soit humaines, qui concourent à produire, ou produisent même isolément, une richesse à l’usage de l’homme. Quant aux activités simplement physiques, telles que la force des vents ou des marées, ou des rayons solaires, ou des substances chimiques, il y a entre elles et les activités vivantes — précisément parce que celles-ci sont faites de celles-là, mais de celles-là dirigées et systématisées — une différence profonde, théoriquement des plus importantes. Que la production soit végétale, ou animale, ou humaine, le produit est toujours reconnaissable à une saveur spéciale qui caractérise toutes les œuvres de la vie. Qu’on se serve pour écrire du papyrus, ou du parchemin, ou même du grossier papier de nos aïeux fabriqué à la main, il y aura  \phantomsection
\label{v1p272}toujours quelque chose d’intéressant pour un œil d’artiste dans cette substance, parce que chaque feuille de ce papier aura sa nuance distincte, sa particularité propre, comme chaque parchemin ou chaque papyrus. Il n’en sera pas de même du papier sorti des grandes papeteries à vapeur, quoique celui-ci puisse, à d’autres égards, être jugé pratiquement préférable. Le produit fabriqué par les machines, surtout si la matière première en est de nature inorganique, a toujours quelque chose de froid, dans son identité trop parfaite, dans sa régularité trop impeccable.\par
Il n’en est pas moins vrai qu’on passe par degrés et insensiblement de la fabrication animale ou végétale, sous la direction de l’homme, à la fabrication machinale, et que toutes ces manières différentes de travailler s’équivalent presque pour la majorité des consommateurs, au point de vue utilitaire, quand le produit répond à peu près au même besoin ; en sorte que l’abaissement de prix résultant du passage de l’un de ces genres de production à l’autre, s’étend peu à peu à tous les produits similaires quelle que soit leur origine. Mais il n’en importe pas moins de distinguer cette origine au point de vue théorique aussi bien qu’esthétique.\par
— Dans les formules relatives au cycle de production, et aux transformations de ce cycle, il convient donc de ne pas mettre tout à fait à part, hors de tout contact avec le travail de l’animal ou de la plante, le travail manuel ou intellectuel de l’homme, comme si celui-ci seul devait compter dans la valeur. Je comprendrais qu’on établit une démarcation profonde entre le \emph{travail vivant} d’une part, et, d’autre part, le \emph{travail} ou plutôt l’\emph{opération mécanique ou physico-chimique ;} mais creuser ce fossé entre le travail de l’ouvrier humain, d’une part, et, d’autre part, les travaux de la plante, de l’animal ou les nouveautés de la machine massés ensemble, c’est peu conforme à la nature des choses qui établit une coupure nette entre l’organique et l’inorganique, entre le monde vivant et le monde non-vivant, mais  \phantomsection
\label{v1p273}non entre l’homme et le reste de la nature. Or, sous le nom de capital, on entend l’ensemble des collaborateurs végétaux et animaux de l’homme, pêle-mêle avec quelques-unes des forces physico-chimiques que ces \emph{travailleurs} emploient, et aussi avec les machines et les outils, les substances inorganiques fabriquées par l’homme, ou les denrées et les aliments quelconques fabriqués par les plantes ou les animaux domestiques. Cette notion du capital peut être pratiquement très utile et même nécessaire. Mais quelle fécondité théorique peut-on attendre d’une notion aussi confuse et aussi bâtarde, qu’on oppose — comme s’il pouvait y avoir là opposition — à la notion, singulièrement rétrécie et mutilée, du travail ?\par
Avant tout, l’homme est un être vivant. Et c’est ce qu’il oublie, c’est ce qu’on s’efforce de lui faire oublier, quand, sous ce nom de capital ou de cheptel, on confond pêle-mêle une paire de bœufs, des semences, une charrue, une charrette, des bâtiments, des meubles, lui faisant ainsi perdre de vue la collaboration de ses co-associés végétaux et animaux dans la grande œuvre de la production du blé ou du lait, ou de toute autre chose nécessaire à son existence. Le paysan, lui, sent bien cette collaboration et cette association avec nos frères inférieurs, les vivants dont nous vivons. S’il lui arrive encore, par exception, d’atteler sa femme à la charrue pour remplacer le bœuf absent, il atteste par là aussi bien l’estime où il tient le bœuf que son mépris pour la femme. Les anciens comprenaient dans le cheptel vivant le groupe de leurs esclaves. On a émancipé les esclaves ; il nous reste, non pas à émanciper, mais à considérer théoriquement d’un œil moins dédaigneux les animaux domestiques et les plantes cultivées elles-mêmes. A mesure que l’homme se civilise, il doit se sentir à la fois le maître et le protecteur de la nature vivante tout entière ; et, de plus en plus pénétré de ses devoirs envers les autres vivants, il doit, sinon rémunérer leurs services, du moins les traiter avec  \phantomsection
\label{v1p274}douceur, avec la conscience toujours plus vive de leur parenté, et ne pas méconnaître cette parenté dans ses théories économiques. Soyons naturalistes autant que psychologues en économie politique. A ce point de vue, il y a deux éléments seulement de la production (et non pas trois) : le \emph{travail} et la \emph{nature inorganique.} Cette dualité, c’est, au fond, la dualité de la \emph{vie} et du \emph{milieu}, sur laquelle se fonde l’opération vitale essentielle, qui consiste à élaborer constamment le milieu, à s’y adapter et l’adapter à soi. Il faut faire rentrer le travail économique dans le grand travail universel de la vie. Si l’on doutait de cette identité fondamentale de l’activité humaine et de l’activité vivante en général, il suffirait, pour en donner la preuve, de rappeler que le rythme général de l’activité industrielle, dans ses hausses et ses basses alternatives, de même que l’activité vivante, dans ses excitations ou ses affaissements périodiques, est réglé par la rotation de la terre autour d’elle-même ou autour du soleil, c’est-à-dire est diurne ou annuel. Aussi annuels que les passages des bancs de poissons et les voyages des oiseaux migrateurs sont les arrivées et les départs de ces bandes d’ouvriers industriels ou agricoles qui, dans les pays les plus divers, émigrent périodiquement. Le Play, dans ses \emph{Ouvriers européens} a beaucoup étudié le régime des émigrations périodiques dans le bassin de l’Oka, en Russie. A Saint-Pétersbourg, des porte-faix émigrent de la sorte pendant l’hiver. « L’essence de ce régime, dit Le Play, est de faire exécuter dans certaines villes ou certaines régions, par des ouvriers étrangers, les travaux, intermittents pour la plupart (ou, pour mieux dire annuels), auxquels la population locale ne peut suffire. Les émigrants chargés de ce service appartiennent toujours à des districts agricoles vers lesquels ils se trouvent constamment rappelés par l’intermittence de leurs travaux, ainsi que par le désir de revoir leurs parents, leur femme et leurs enfants. » Il est à noter, comme trait caractéristique de mœurs russes, que les  \phantomsection
\label{v1p275}ouvriers russes qui émigrent ainsi, habitués à la communauté de la vie de la famille et du \emph{mir}, ne peuvent s’en passer même en voyage et la remplacent, pendant leur vie d’émigration, par des associations volontaires appelées \emph{artèles.} Telle est l’origine, tout agricole et rurale, des corporations ouvrières de la Russie.\par
Ce n’est pas seulement en Russie, c’est un peu partout, que ces émigrations périodiques ont lieu. En Angleterre, le renchérissement de la main-d’œuvre agricole a amené, vers le milieu du {\scshape xix}\textsuperscript{e} siècle, la formation de bandes agricoles sur lesquelles une enquête a été faite en 1865 par le gouvernement anglais, à la suite de plaintes, souvent fondées, au sujet des dangers que présentait la contagion de l’immoralité qui y régnait. Ce sont là aussi des foules intermittentes, dont l’intermittence est réglée par les cours des saisons, quoiqu’elles diffèrent profondément des précédentes. Ici rien qui ressemble à une \emph{artèle}, à une grande famille. Ces bandes, composées en majorité d’adolescents des deux sexes, extrêmement émancipés, sont conduites par un chef qui exerce sur elles une grande autorité. Ces chefs de bandes, d’après l’enquête, sont, en général, « des hommes grossiers, de mauvaises mœurs, dissolus, ivrognes ». Ils sont des entrepreneurs qui travaillent à forfait et obtiennent des ouvriers et ouvrières à leur solde une somme de travail très supérieure à celle que ceux-ci pourraient fournir s’ils étaient isolés.\par
— On le voit, quelle que soit le degré de complication où parvienne l’organisation industrielle, tout est essentiellement périodique, d’une périodicité en général annuelle, dans les manifestations du travail humain comme dans celles du travail animal ou du travail végétal. Je ne parle pas des périodicités élémentaires, qui se meuvent dans les limites de temps les plus variées. Car tout est essentiellement périodique dans le travail végétal ou animal comme dans le travail humain : circulation du sang ou de la sève, respiration,  \phantomsection
\label{v1p276}sécrétion des glandes, etc. Il est vrai que ce caractère de périodicité est propre à l’action des agents inorganiques eux-mêmes, astres qui gravitent, molécules qui vibrent, flux et reflux, courants de l’atmosphère et de la mer, évolution des eaux qui montent et redescendent de la mer à la montagne et de la montagne à la mer. Aussi n’est-ce point par ce caractère de périodicité que les êtres vivants se séparent des agents physiques, et j’ai voulu dire simplement que c’est par là que les êtres vivants, y compris l’homme, se rapprochent, non qu’ils se séparent du reste de la nature. Mais il y a des caractères plus profonds qui les en séparent.\par
Et je sais bien aussi que chaque espèce vivante cherche à se faire servir, non seulement par toutes les autres espèces vivantes à sa portée, autant que possible, mais encore par tous les agents physiques et chimiques quelconques, et que cette distinction paraît assez indifférente à son égoïsme utilitaire. Cependant cette règle n’est pas sans exception, et il est des végétaux dont les vrilles ne s’enroulent pas de la même manière autour d’une tige d’arbuste ou d’une barre de fer. Et je ne suis pas sûr que tous les oiseaux fassent leur nid indifféremment dans le creux d’un arbre ou le trou d’un mur. Puis, ne semble-t-il pas que l’homme doive affirmer sa supériorité à l’égard des autres vivants, précisément en distinguant ce que ceux-ci confondent ? Et, de fait, est-ce que le progrès de la civilisation n’a pas pour effet de faire sentir chaque jour plus nettement et plus profondément à quel point la manière dont l’homme est servi par les plantes ou les animaux domestiques diffère de la manière dont il est servi par la chaleur, par la lumière, par l’électricité, par les minéraux, et par les machines qu’il a composées avec ces forces ou ces substances inanimées ? Est-ce que l’agriculture et l’élevage ne deviennent pas chaque jour plus différents de l’industrie proprement dite, si bien qu’un traité d’économie rurale semble n’avoir plus rien de commun avec  \phantomsection
\label{v1p277}un traité d’économie politique ordinaire où l’industrie est surtout visée ?\par
Par les plantes et les animaux l’homme est servi comme il l’est, au fond, par les autres individus de son espèce. Le plus humble des êtres vivants porte en lui-même un vouloir propre, conscient ou inconscient, mais toujours ingénieux, qui est le principe même de vie, et où la volonté humaine se mire comme dans sa vivante image. Aussi ne pouvons-nous l’utiliser que comme nous utilisons le vouloir de nos semblables, en faisant en sorte que son but propre converge vers l’accomplissement même du nôtre. Au contraire, les choses inorganiques n’ont pas de fin à elles, elles n’ont donc, strictement, que l’ingéniosité, tout apparente et tout artificielle, que nous leur prêtons par notre génie inventif. Elles n’ont pas de but, ce qui veut dire qu’elles ne sauraient travailler, d’après notre définition du travail. Il n’y a en elles, en fait d’adaptation, que ce que nous y mettons, sous forme de machines ; tandis que, lorsque, par l’élevage, par la sélection méthodique, par des moyens indirects, qui n’ont rien de commun avec l’invention d’une machine nouvelle, nous \emph{suscitons} une nouvelle plante ou un nouvel animal, mieux adapté à nos fins, ces variétés jaillies du sein fécond de l’imagination vivante, sous notre provocation, soit, mais sans que nous sachions comment, nous étonnent toujours par l’inattendu des propriétés, des vertus, des charmes, que nous découvrons en elles. Nous \emph{découvrons} les innovations végétales ou animales, nous ne les \emph{inventons} pas.\par
Nous provoquons de nouvelles \emph{habitudes} chez les êtres vivants ; l’impulsion qui leur est donnée par nous, à tâtons, l’habitude la conserve et l’enracine. Rien d’analogue à l’habitude pour les choses inorganiques. Si nous les manions bien plus complètement, \emph{tanquam cadavera}, elles n’arrivent jamais à se diriger elles-mêmes, il faut toujours les surveiller avec le même soin. Elles ne nous obéissent pas, elles vont où nous les poussons.\par
 \phantomsection
\label{v1p278}Enfin, les services que nous rendent les autres êtres vivants sont réciproques, alors même que nous les engraissons pour les manger. Ils seraient mangés tout aussi bien, et feraient plus maigre chère, à l’état de liberté. Il n’est pas d’animal domestique qui ne se persuade vaguement que l’homme est son serviteur. Mais les services que nous rendent les forces physiques et les substances chimiques sont unilatéraux ; qu’est-ce que nous leur donnons en retour ?\par
On ne saurait donc confondre dans des formules théoriques ce que la pratique différencie de plus en plus. On comprend qu’au début de l’évolution industrielle, quand elle se présentait comme noyée dans l’évolution \emph{agricole} et pastorale, beaucoup plus ancienne et plus développée, on ait embrassé les deux dans les mêmes notions ; mais, à mesure que s’accentue une distinction fondée sur la nature des choses, il importe d’y avoir égard théoriquement aussi bien que pratiquement. Le terme idéal où court l’humanité, sans en avoir encore une conscience précise, c’est, d’une part, de composer avec l’élite de toutes les faunes et de toutes les flores de la planète un harmonieux concert d’êtres vivants conspirant, dans un même \emph{système de fins}, aux fins mêmes de l’homme, librement poursuivies ; et, d’autre part, de capter toutes les forces, toutes les substances inorganiques, pour les asservir ensemble, comme de simples moyens, aux fins désormais convergentes et consonnantes de la vie. C’est au point de vue de ce terme éloigné qu’il faut se placer pour comprendre à quel point les conceptions fondamentales de l’économie politique demandent à être révisées.
\subsubsection[{I.5.h. Le loisir périodique, les vacances. Leur origine et leur extension.}]{I.5.h. Le loisir périodique, les vacances. Leur origine et leur extension.}
\noindent Je ne puis quitter le sujet du travail sans dire un mot de son contraire, le loisir, et sans faire remarquer que le loisir, comme le travail, a un caractère de plus en plus périodique.  \phantomsection
\label{v1p279}Partout et toujours, il y a eu des \emph{jours fériés}, revenant annuellement ou mensuellement ou hebdomadairement aux mêmes dates. Mais, depuis les temps modernes, nous voyons s’étendre un usage qui, d’abord limité, ce semble, à une profession, a fait tache d’huile au dehors et tend à envahir toutes les carrières : c’est ce repos prolongé, d’un ou deux mois, au milieu de l’été ou à l’automne, qu’on nomme les \emph{vacances.} L’institution des vacances (ainsi, d’ailleurs, que celle du repos dominical ou sabbatique, hebdomadaire) est très distincte de l’institution des fêtes : les fêtes ont une origine religieuse ou patriotique, elles sont une commémoration joyeuse, un rassemblement en l’honneur d’un saint, d’un héros, d’un événement heureux ou glorieux ; les vacances (comme le dimanche) ne commémorent rien, ne célèbrent rien, elles dispersent ou isolent les individus plutôt qu’elles les rassemblent, elles sont réputées un repos salutaire plutôt qu’une joie fortifiante et tonique.\par
Les fonctionnaires égyptiens, athéniens, romains, connaissaient-ils les vacances ? Il ne le semble pas. On ne saurait voir un embryon de vacances dans les saturnales des esclaves. Il y a des genres de travaux qui n’en comportent pas, en aucun temps et en aucun pays : l’agriculture, les travaux des cuisiniers et des domestiques en général, la boulangerie, la boucherie, la plupart des industries ; enfin, la guerre, qui a l’\emph{hivernage}, précisément l’opposé du repos \emph{estival.} Il n’y a, en somme, que les professions libérales, qui, sans trop d’inconvénient, comportent des vacances, sauf la médecine et quelques autres exceptions. Et, parmi les professions libérales, celle qui, la première a goûté les douceurs de cette coutume, c’est, paraît-il, la magistrature. Il faut lui en savoir un gré infini. Je me demande si ce n’est pas à la propriété agricole aussi que nous en sommes redevables ; car, bien que les travaux des champs soient ininterrompus, n’est-ce pas cependant l’amour de la terre qui a fait sentir le caractère distinctif, dans nos climats, des mois d’août, de septembre,  \phantomsection
\label{v1p280}d’octobre, qui est d’être la saison des récoltes ? Et n’est-ce pas parce que les magistrats d’ancien régime étaient des propriétaires ruraux qu’ils interrompaient leurs labeurs judiciaires pour aller surveiller leurs vendanges, sinon leurs moissons ? Quoi qu’il en soit, une fois cet exemple donné par les magistrats, toutes les professions industrielles même et commerciales, se sont efforcées de le suivre, et, dans la mesure du possible, y ont réussi. Seulement, en se propageant, cette vieille institution change un peu de caractère, s’empreint d’un esprit nouveau. Après avoir eu pour première et principale raison d’être la nécessité, pour le magistrat, ou le fonctionnaire quelconque, de récolter les fruits de son domaine rural, les vacances deviennent, pour les vagabonds distingués et internationaux de nos jours, la satisfaction donnée au besoin impérieux et périodique, et de plus en plus universel, de voyager. Ce n’est plus le temps des vendanges, c’est le temps des voyages aux villes d’eaux, aux bains de mer, à l’étranger. C’est un repos devenu très agité, mais très instructif.\par
Autre remarque. Jamais on n’a tant chanté que de nos jours les louanges du travail, ni tant exorcisé tous les \emph{revenus sans travail.} Il ne s’agit de rien moins que de les détruire. Jamais cependant le besoin de s’enrichir sans peine ne s’est autant répandu ni aussi souvent satisfait. La spéculation financière en haut, en bas les valeurs à lots, les jeux, les loteries, les paris aux courses, etc., sont les mille formes sous lesquelles se fait jour cette passion publique pour la bonne chance, pour ce favoritisme du sort que le socialisme maudit et qui se répand avec le socialisme.
 \phantomsection
\label{v1p281}\subsection[{I.6. La monnaie}]{I.6. La monnaie}\phantomsection
\label{l1ch6}
\subsubsection[{I.6.a. Définition. Comment a pu se produire ce caractère de désirabilité constante, universelle et indéfinie, qui est propre à la monnaie, et qui explique son échangeabilité universelle.}]{I.6.a. Définition. Comment a pu se produire ce caractère de désirabilité constante, universelle et indéfinie, qui est propre à la monnaie, et qui explique son échangeabilité universelle.}
\noindent Quand la répétition a fonctionné un certain temps et produit un certain nombre de choses semblables dans un ordre quelconque de réalités, elle y donne lieu à des quantités spéciales, qui sont la synthèse de ces similitudes. Toutes les quantités physiques, chaleur, lumière, électricité, résultent du fonctionnement de l’ondulation. La force musculaire d’un homme, mesurable au dynamomètre, sa force nerveuse, son énergie circulatoire ou respiratoire, sont des quantités aussi, et l’ensemble de ces quantités organiques, la vitalité, est une quantité assez imprécise, il est vrai, réelle cependant, qui résulte du [{\corr fonctionnement}] de toutes les régénérations cellulaires incessantes appelées la nutrition, ainsi que de la génération, d’où elles procèdent. Eh bien, la quantité proprement économique née du fonctionnement imitatif de toutes les consommations et de toutes les reproductions industrielles, c’est la \emph{valeur-coût} incarnée dans la monnaie. Nous allons étudier sa nature et son rôle.\par
Avant tout, il est intéressant de se demander comment une monnaie a pu s’établir. Il n’y a pas d’article de première nécessité même, le pain, la viande, qui soit de consommation universelle. Beaucoup de peuples préfèrent le riz au pain ; les végétariens ne mangent pas de viande. D’ailleurs, les articles les plus nécessaires et les plus répandus ne sont désirés qu’à certains moments, irrégulièrement ou périodiquement renaissants. Et, au delà d’une certaine quantité, \phantomsection
\label{v1p282} on ne les désire plus. Un homme serait fâché d’avoir plus de deux ou trois costumes complets, plus de cinq ou six paires de bottines, etc. Il n’y a qu’une chose que tout le monde désire, et désire à tout moment, et désire en quantité illimitée. C’est l’argent. — Pourquoi ce caractère, qui lui est exclusivement propre, de \emph{désirabilité constante, universelle et indéfinie ?} Parce que chacun sait, parce que chacun est convaincu\footnote{ \noindent Rappelons-nous que la croyance échappe à la loi de périodicité du désir. Le caractère de continuité attaché à la valeur de la monnaie lui vient de ce qu’elle est fondée sur la foi encore plus que sur le désir.
 } que, la monnaie étant désirée de la sorte par tout le monde, il se procurera aisément, moyennant sa monnaie, tout ce dont il aura besoin. — Ainsi, il y a cercle vicieux : la monnaie est désirée constamment et indéfiniment par chacun, parce que chacun sait que tous les autres la désirent constamment et indéfiniment.\par
Mais alors comment a-t-il pu se faire que ce désir constant et indéfini soit né chez un individu et se soit propagé chez les autres ? N’est-il pas contradictoire de le supposer puisque nous venons d’admettre que le fait même de cette propagation et du jugement qui la constate était la cause de ce désir chez un individu quelconque ? — Observons que cette apparence d’insolubilité avec laquelle le problème de l’origine de la monnaie se présente à nous a son pendant en linguistique, quand on essaie d’y éclaircir l’origine du langage : le problème de savoir comment on a pu adopter tel mot pour signifier telle chose a un air pareillement insoluble et pour des raisons analogues. Mais la vie est habituée à dénouer le plus aisément du monde tous ces nœuds gordiens. Il suffit de se rappeler que, au début, les sociétés se réduisent à des groupes très étroits et très clos. C’est seulement dans cette hypothèse que l’établissement initial de l’institution monétaire se conçoit. On comprend sans peine que, dans les limites resserrées d’un clan primitif, où tous les besoins et tous les goûts sont à peu près semblables, le  \phantomsection
\label{v1p283}désir enfantin de posséder certains objets déterminés, pour des raisons utilitaires et encore mieux pour des motifs esthétiques, se soit vite généralisé\footnote{ \noindent Je lis sous la plume d’un des économistes les plus distingués : « Comme tous les organes généraux nécessaires à la vie et au progrès des sociétés, comme le langage, comme l’échange, comme le droit, la monnaie est née de la collectivité \emph{agissant instinctivement}, non de l’invention d’un homme de génie. » C’est là l’opinion générale. — Mais s’il en était ainsi, si la monnaie était due à \emph{un instinct}, nous ne verrions pas des civilisations ou des demi-civilisations, telles que celles des Aztèques ou des Incas, dépourvues de monnaie véritable, se contenter de monnaies tellement grossières ou embryonnaires qu’elles ne méritent pas ce nom. Nous ne verrions pas l’évolution de la monnaie, comme l’évolution de l’écriture, se poursuivre dans un seul groupe de peuples classiques, hautement privilégiés à cet égard, dont l’exemple a rayonné partout. La \emph{monnaie frappée}, c’est-à-dire la monnaie véritable et complète, est née chez les Lydiens, d’après les recherches les plus récentes, et il n’est pas douteux qu’elle y naquit de la volonté d’un monarque, expression de l’idée d’un inventeur, d’un initiateur de génie. De là la monnaie frappée s’est répandue partout, comme l’écriture alphabétique, c’est-à-dire l’écriture par excellence, s’est répandue de Phénicie chez tous les peuples du monde.\par
 L’explication de la monnaie par l’instinct est à mettre sur le même rang que l’explication du langage par un \emph{don divin}, conséquence que M. de Bonald déduisait de l’insolubilité apparente du problème posé par l’origine du langage.
 }, par la stimulation réciproque, se soit élevé à une hauteur pratiquement infinie, en sorte que l’idée de se servir de ces choses infiniment et continuellement appréciées par tous comme mesure générale de valeur, comme moyen d’échange et comme moyen d’accumulation des richesses, se soit offerte d’elle-même. Puis, les barrières des clans tombant par degrés, ces monnaies diverses sont entrées en concurrence les unes avec les autres et l’une d’elles a triomphé par sélection imitative.\par
Le difficile étant donc, à l’origine, de s’accorder, dans une région donnée, sur un objet qui y soit désiré par tous et échangeable contre tout, la monnaie est née dès qu’un tel objet a existé. Cet objet a été très différent d’après les régions. Tantôt c’était un article répondant à un besoin déterminé, périodique même, mais un article durable et mobilisable à la fois, « le blé et le tabac dans les premières colonies des États-Unis, le thé dans la Tartarie chinoise ; les fourrures en Sibérie » (J.-B. Say), le bétail à l’époque pastorale ; \phantomsection
\label{v1p284} tantôt, et plus souvent c’était un objet servant de parure, un objet brillant et inutile, attirant les regards en tout temps. Finalement, c’est sur deux ou trois espèces de ce dernier genre, l’or, l’argent, le cuivre, que s’est fixée et arrêtée la série des métamorphoses monétaires. — On a expliqué et justifié après coup ce choix de certains métaux par les qualités physiques ou chimiques qui les distinguent. Or, sans nul doute, il importe que l’objet quelconque adopté pour monnaie soit rare, en quantité limitée, non extensible à volonté, puisque, sans cela, il n’aurait jamais été nécessaire de recourir à l’échange pour l’obtenir. Mais ce n’est pas sa rareté, encore moins son utilité, pas même sa beauté, qui suffit à expliquer son privilège d’\emph{échangeabilité universelle}. Ces caractères n’ont été que la cause occasionnelle qui a concentré sur l’or, l’argent et d’autres métaux, la conviction universelle et constante qu’ils sont échangeables en tout temps contre n’importe quel article ou service dans le commerce. Et c’est parce que cette concentration n’a pu s’opérer que graduellement, à la longue, à la suite d’une longue évolution monétaire, qu’il est infiniment difficile de porter atteinte à ce privilège des métaux précieux. Ce n’est qu’à la longue aussi, sinon par la traversée d’étapes plus ou moins semblables, qu’on peut espérer de concentrer un jour sur un autre objet, sur un morceau de papier, par exemple, imprimé d’une certaine façon, une pareille unanimité et une pareille constance de foi. Si jamais, parmi les billets de banque qui servent de papier-monnaie, en droit ou en fait, dans des régions circonscrites, il en est un, le billet de la Banque de France ou celui de la Banque d’Angleterre, qui parvient à s’universaliser, ce n’aura été qu’après une série de débordements successifs par-dessus ses frontières premières. — En attendant, la découverte d’une mine d’or suscite toujours une \emph{immense espérance}, comparable en quelque sorte à l’espoir du ciel chrétien ou plutôt musulman importé dans une peuplade barbare, et, par là, elle est  \phantomsection
\label{v1p285}une source de richesses inouïe, une surexcitation extraordinaire du désir et de l’effort producteur.
\subsubsection[{I.6.b. Sa nature. Elle est économiquement ce que les mathématiques sont intellectuellement.}]{I.6.b. Sa nature. Elle est économiquement ce que les mathématiques sont intellectuellement.}
\noindent Cela dit sur l’origine de la monnaie, tâchons de préciser sa nature et son rôle. — Sa nature est non seulement d’être le seul objet universellement et constamment échangeable, mais encore de devenir de plus en plus le seul objet échangeable en fait. — La monnaie, en naissant, accapare peu à peu et monopolise l’\emph{échangeabilité} dont elle dépouille toutes les autres marchandises. L’échange d’une marchandise contre une autre marchandise n’est dès lors que tout à fait exceptionnel, de plus en plus exceptionnel. Le fait normal, habituel, constant, c’est l’échange de la monnaie contre une marchandise, ou d’une marchandise contre de la monnaie. D’après Macleod, l’échangeabilité est le caractère \emph{essentiel} de la richesse. C’est aussi l’une des définitions qu’en donne Stuart Mill. A ce compte, il n’y aurait, dans un pays hautement civilisé, d’autre richesse que la monnaie (c’est un peu l’idée générale et vulgaire), car, peu à peu, elle s’approprie le pouvoir de s’échanger contre toutes les autres marchandises, et celles-ci ne le sont \emph{que contre elle.} Quant à l’échange d’une espèce de monnaie contre une autre espèce de monnaie, de dollars contre des thallers, de louis contre des roubles ou contre des billets de banque, d’un écu de 5 francs contre des pièces divisionnaires, il a un caractère tout à fait à part, qui n’a de commun que le nom avec l’échange d’une marchandise contre de la monnaie, ou même d’une marchandise contre une autre marchandise. L’échange d’une monnaie contre une autre monnaie (d’égale valeur, bien entendu) est un passage du même au même, puisque les deux choses sont identiques en ce qu’elles ont d’essentiel, leur valeur. Mais l’échange d’un tableau contre un piano, ou d’un bœuf contre  \phantomsection
\label{v1p286}une armure, est le remplacement d’un genre d’utilité par une utilité tout autre, absolument et essentiellement dissemblable. L’échange d’une monnaie contre une autre monnaie peut être comparé à une \emph{proposition analytique}, à une tautologie, tandis que l’échange d’une monnaie contre une marchandise, et même celui de deux marchandises l’une contre l’autre, est comparable à une \emph{proposition synthétique}, pour continuer à employer la terminologie de Kant\footnote{ \noindent C’est seulement à la Bourse, quand des titres financiers s’échangent contre d’autres titres financiers, que l’échange des signes monétaires les uns contre les autres prend un sens réellement important. C’est qu’ici, il ne s’agit pas seulement, comme chez un changeur, de troquer une monnaie, une \emph{certitude} de richesses, contre une autre monnaie, contre une autre certitude, mais aussi et surtout de troquer une certaine probabilité de gain ou de perte contre une autre probabilité d’un degré souvent très différent.
 }.\par
— Mais précisons encore mieux la nature de la monnaie. — La monnaie n’est-elle pas dans le monde de l’action économique ce que sont les mathématiques dans le monde de la pensée ? N’est-ce pas pour répondre à des nécessités au fond toutes semblables que nous soumettons au nombre et à la mesure, à l’empire des mathématiques, toutes nos connaissances, toutes nos observations, toutes nos expériences, en dépit de leurs diversités qualitatives, — et que nous évaluons en monnaie toutes nos joies et toutes nos douleurs, tous nos désirs, et tous leurs moyens de se satisfaire, en dépit de l’hétérogénéité manifeste de ces choses ?\par
Nous exprimons les qualités universelles en quantités, en formules numériques, proprement scientifiques, pour rendre nos idées, nos perceptions, comparables et co-échangeables entre elles, démontrables et communicables d’homme à homme, et \emph{socialisables ;} et nous évaluons les biens de tout genre, si hétérogènes qu’ils puissent être, en monnaie, pour permettre leur échange et leur communication d’homme à homme, leur socialisation aussi.\par
Un sujet d’études est d’autant plus près d’être embrassé par une vraie science qu’il s’en dégage des lois plus mathématiques. \phantomsection
\label{v1p287} On commence toujours par formuler des lois \emph{qualitatives}, — puis des lois \emph{demi-qualitatives}, comme par exemple celle de l’offre et de la demande où il s’agit non d’équation = mais de < ou de > : « la valeur \emph{diminue} quand l’offre \emph{augmente} », sans qu’on prétende que la diminution de l’une est \emph{égale} à l’augmentation de l’autre, ce qui rend la formule très vague ou fausse si on la précise. — Enfin, des lois d’\emph{équation.}\par
De même, à mesure qu’un marché s’élargit, s’élève, devient plus vraiment social et civilisé, l’échange des marchandises et des services cesse de s’y faire par le troc, par le troc d’abord capricieux et sans nulle règle puis un peu plus réglé, — ensuite s’opère par achat et vente mais à des prix très variables, objet de marchandages incessants, à des prix tout individuels, — enfin par achat et vente à des prix fixes, uniformes sur tout un grand territoire.\par
En se \emph{mathématisant}, les lois d’une science en progrès deviennent plus claires, plus commodes, propres à s’appliquer à un plus grand nombre de problèmes, et à se répandre en un plus grand nombre d’esprits. Car la \emph{science} est, avant tout, la \emph{connaissance socialisée et indéfiniment socialisable ;} ce devrait être là sa définition essentielle. — Et, en se \emph{monétisant}, les choses échangeables s’échangent plus facilement, plus rapidement et beaucoup plus loin. La monétisation de l’échange est la condition \emph{sine qua non} du \emph{commerce.} Le commerce est l’action économique socialisée de plus en plus, comme la science est la pensée socialisée de plus en plus. (\emph{La science en train de se faire} répond à l’\emph{industrie ;} la \emph{science faite} et, par suite, se vulgarisant, répond au \emph{commerce.)}\par
Sans doute, un Robinson de génie, né dans une île déserte et s’y développant intellectuellement tout seul, autodidacte original, pourrait remarquer les similitudes et les répétitions des phénomènes, leurs rapports de \emph{plus} et de \emph{moins}, et même d\emph{’égalité}, mais ce \emph{côté numérique} des  \phantomsection
\label{v1p288}choses ne le frapperait que faiblement, infiniment moins que leur côté ondoyant et divers et leurs variations incessantes. Si l’instinct du progrès intellectuel, par hasard, le tourmentait, c’est à diversifier de plus en plus ses sensations et ses perceptions, à les accumuler en lui avec leurs diversités propres, qu’il s’attacherait, non à les uniformiser en les réduisant à des idées générales. Si l’esprit humain a tourné son besoin investigateur dans la voie des généralisations, des similitudes et répétitions phénoménales exprimées en \emph{signes}, en \emph{mots}, c’est qu’il y a été forcé pour entrer en communication avec ses semblables. Mais, \emph{pour se comprendre lui-même de mieux en mieux}, il n’aurait jamais eu besoin de langage. Son développement intellectuel, s’il fût resté exclusivement individuel, aurait pu, à la rigueur, aller fort loin, mais à la condition de s’attacher avant tout aux variations et aux diversités qualitatives des phénomènes, au \emph{côté poétique des choses.} Livré à lui-même, sans les excitations de la société ambiante, le cerveau de l’individu est capable de se développer poétiquement ; mais, scientifiquement, jamais. Il se peut qu’il y ait des poètes cachés chez certains animaux de génie. Il n’y a, à coup sûr, aucun savant.\par
Isolé, l’individu non plus n’aurait jamais inventé rien de pareil à la monnaie. C’est trop clair. Mais il n’aurait pas même eu l’idée, probablement, de comparer ses divers désirs, — dont la diversité seule l’eût frappé — pour reconnaître leur réelle comparabilité, leurs degrés de \emph{plus} ou de \emph{moins.} Ce n’est pas qu’il n’y ait un sens tout individuel de l’idée de \emph{valeur.} Mais ce sens ne se dégage qu’après que celui de la \emph{valeur sociale}, de la valeur proprement dite, a été conçu.\par
L’empire des mathématiques s’étend sans cesse plus loin dans le monde de la pensée, comme celui de la monnaie dans le monde de l’action. Après avoir envahi toute l’astronomie, toute la physique, toute la chimie, le \emph{point de vue mathématique} s’empare de la biologie, où les instruments des mesures jouent un rôle toujours croissant, cherche à conquérir \phantomsection
\label{v1p289} la psychologie et commence à s’annexer la sociologie par la statistique démographique, commerciale, judiciaire, etc.\par
Le \emph{point de vue pécuniaire}, après avoir régi toute l’activité industrielle, s’impose en politique extérieure, où l’argent est le nerf de la guerre, où la nation la plus riche est la plus respectée, et, en politique intérieure, devient souveraine aussi, par la corruption de la Presse, par les marchandages des partis. Il n’est presque rien, en fait de \emph{biens} de tout genre, même esthétiques, même religieux, qui ne s’achète et ne se vende : messes, rédemption des péchés, exemptions de jeûnes, leçons d’artistes, j’allais dire talent.\par
L’évolution mathématique passe de l’arithmétique à l’algèbre, de la théorie des nombres à celle des fonctions. L’évolution monétaire passe de la monnaie métallique à la monnaie de papier (signe \emph{algébrique} en quelque sorte de la monnaie), et du commerce des marchandises (où une quantité de monnaie est troquée contre un article ou un service) au commerce des valeurs de Bourse (où les titres financiers s’échangent les uns contre les autres). A la Bourse, les \emph{valeurs}, rapports entre une somme d’argent et un objet, sont elles-mêmes \emph{évaluées} les unes par rapport aux autres. C’est un rapport du second degré. Par la cote, elles se présentent comme fonctions les unes des autres, haussant ou baissant ensemble suivant certaines lois.\par
Il n’en est pas moins vrai que tout n’est pas vénal, et que, \emph{pour la même raison au fond}, tout n’est pas mesurable et nombrable. Il y a des choses uniques en soi, incomparables essentiellement ; et il y a des biens tout personnels, incommunicables, inappréciables.\par
Soit par la suggestion autoritaire, soit par la démonstration, nous ne pouvons communiquer à autrui nos pensées (ce qui est l’équivalent du \emph{don} des biens, début unilatéral de l’\emph{échange} des biens) qu’à la condition de les présenter par leur côté mesurable et quantitatif. S’il s’agit de faire entrer  \phantomsection
\label{v1p290}de force, par démonstration, notre jugement dans la tête d’autrui, il faut un syllogisme plus ou moins explicite, c’est-à-dire un rapport d’espèce à genre ou de genre à espèce établi entre deux idées, ce qui signifie que l’une est \emph{incluse} dans l’autre, est \emph{du nombre} (indéterminé ou déterminé mais réel) des choses \emph{similaires}, et perçues en tant que similaires, que l’autre, la proposition \emph{générale}, embrasse et contient.\par
— Par suggestion autoritaire même, on ne peut transvaser son idée dans la tête d’autrui, qu’autant qu’elle est faite d’éléments semblables aux éléments d’idées contenus dans ce cerveau étranger, et cette similitude ne peut apparaître que par le \emph{langage}, qui est un composé d’\emph{idées générales}, de choses vues par leur côté similaire et nombrable. Pour être communicable, une pensée doit être, sinon démontrable, du moins exprimable ; dans les deux cas, composée d’éléments comparables et nombrables. La monnaie, c’est-à-dire une commune mesure de biens hétérogènes, est, de même, la condition devenue nécessaire du commode échange de ces biens, sinon de leur don ou de leur vol. On évalue d’ailleurs, involontairement, irrésistiblement, les objets donnés, quand on les reçoit, et les objets qu’on vole ou qu’on nous vole.
\subsubsection[{I.6.c. Pourquoi elle ne sert pas de moyen d’échange aux valeurs-vérités.}]{I.6.c. Pourquoi elle ne sert pas de moyen d’échange aux valeurs-vérités.}
\noindent La monnaie, étant la commune mesure des valeurs-utilités, sert à mesurer des croyances aussi bien que des désirs, puisque l’utilité est une combinaison de désir et de croyance incarnée dans un objet. Demandons-nous, maintenant, pourquoi elle ne sert pas de moyens d’échange aux valeurs-vérités, de même qu’aux valeurs proprement dites, aux valeurs-utilités ? Le problème vaut la peine d’être posé. D’abord, remarquons que la \emph{vérité} n’est pas une combinaison de désir et de croyance, elle n’est que la \emph{crédibilité}  \phantomsection
\label{v1p291}d’une idée. En effet, quoique la connaissance puisse être regardée comme la satisfaction du désir de connaître, et que, à ce point de vue, elle soit aussi une utilité, et même une utilité d’autant plus élevée qu’elle satisfait une curiosité plus forte, il n’en est pas moins vrai que la connaissance se présente aussi sous un autre aspect, indépendamment de la curiosité à laquelle elle correspond, et par rapport seulement \emph{à la force d’adhésion mentale qu’elle suscite et au nombre des individus chez lesquels elle la suscite.} Cette abstraction du désir, qui est possible pour les connaissances, ne l’est pas pour les marchandises ou les services. Puis, en ce qui concerne ces derniers objets, leur consommation destructive, qui suppose l’échange et l’appropriation exclusive, est la condition même de la satisfaction du désir auquel ils correspondent. Mais les connaissances n’ont pas besoin d’être la propriété exclusive de quelqu’un pour satisfaire son désir de savoir. Il ne leur est donc pas essentiel d’être un objet d’échange pour se communiquer, quoiqu’elles puissent l’être, par exception, dans le cas de secrets jalousement gardés et qu’on ne communique que moyennant la communication d’autres secrets. On peut donc, quand il s’agit de connaissances diverses, contradictoires ou d’accord entre elles, en train de se répandre ensemble dans un milieu social, considérer le résultat du choc ou de l’alliance des croyances qui leur sont inhérentes sans considérer en même temps le choc ou l’alliance des désirs de connaître qu’elles satisfont. Il n’y a jamais à sacrifier un désir de connaître à un autre désir de connaître, il n’y a à sacrifier, dans certains cas, qu’une croyance à une autre croyance qui implique contradiction avec la première. Mais, quand il s’agit de marchandises diverses en train de se répandre dans le public, il n’en est pas de même. Le succès des unes suppose le triomphe du désir satisfait par elles sur le désir sacrifié auquel les autres donnent satisfaction, en même temps que la confiance en la bonne qualité et la supériorité des premières est un démenti  \phantomsection
\label{v1p292}implicite opposé à la confiance en la supériorité des secondes.\par
Supposons une société où chaque connaissance serait attachée essentiellement à la possession d’un talisman, d’un livre miraculeux qui la contiendrait, en sorte qu’on ne pourrait la posséder sans acquérir ce livre, et qu’en transmettant ce livre à autrui on lui transmettrait aussi cette connaissance, dont on se dépouillerait à l’instant même. Dans cette hypothèse, on ne pourrait acquérir une nouvelle connaissance qu’en sacrifiant une ancienne, alors même que les deux n’auraient rien de contradictoire. Les connaissances seraient donc objet de commerce, au même titre que des denrées quelconques. Il y aurait une valeur vénale, exprimable en monnaie, des diverses connaissances, une Bourse des valeurs-vérités.\par
Pourquoi cette hypothèse n’est-elle pas réalisable, en toute rigueur du moins ? Au fond, c’est parce qu’elle implique la non-existence d’une fonction essentielle de notre esprit, la mémoire. Toute pensée, toute connaissance, consiste en sensations remémorées, une sensation n’étant qu’un cliché dont la vie intellectuelle est le perpétuel tirage. Que nos sensations, nos perceptions, pour se produire, exigent l’appropriation et la destruction, lente ou rapide, d’une substance ou d’une force extérieure, cela se comprend, puisqu’elles sont un rapport de notre esprit avec les réalités du dehors. Mais nos pensées, emploi cérébral des multiples exemplaires intérieurs de nos sensations anciennes, ne sauraient être sous une dépendance analogue ou aussi étroite. Quand un monarque oriental, un jour de grande fête, répand des aumônes autour de lui, fait pleuvoir des largesses de vêtements, d’aliments, de pièces d’argent, il ne peut donner sans se dépouiller, si riche qu’il soit. Mais, quand un Claude Bernard ou un Pasteur propage parmi ses auditeurs des vérités qu’il a découvertes ou qu’il a lui-même reçues de ses maîtres, il ne s’en dépouille pas, ses dons intellectuels ne l’appauvrissent en rien. Pour qu’il en fût autrement, il faudrait \phantomsection
\label{v1p293} qu’il oubliât ses idées au fur et à mesure qu’il les exprime. Qu’est-il donc arrivé ? La suite de ce qui s’était produit quand ses vérités lui ont apparu. De même qu’elles sont nées de la rencontre de souvenirs, d’images, de sensations anciennes reproduites des millions de fois au dedans de lui-même, pareillement, grâce à sa parole, elles se répètent au dehors en exemplaires nombreux dans le cerveau de ses élèves, et cette répétition extérieure peut être considérée comme la continuation sociale des reproductions internes qui l’ont précédée. L’imitation, mémoire sociale, est ainsi toujours la continuation extérieure de la mémoire, imitation interne.\par
Dans une certaine mesure, dans une mesure variable — et dont les variations mériteraient examen — les conditions de l’hypothèse de tout à l’heure semblent se réaliser. D’abord, les connaissances n’étant que des répétitions combinées de sensations, de perceptions, il est certain qu’il faut, avant tout, avoir été à même d’éprouver les sensations élémentaires pour que les connaissances dérivées de leur répétition et de leur combinaison interne soient possibles. Les observations directes, les expériences de laboratoire ou autres, fondement de la science, doivent être achetées, comme les satisfactions quelconques des sens produites par l’industrie. Elles supposent des consommations et des destructions de substances ou de forces, des voyages, des achats d’appareils ou de matériaux. En second lieu, la combinaison des souvenirs de ces sensations élémentaires, une fois qu’elle a eu lieu dans l’esprit de l’inventeur, se matérialise en un livre, une conférence, afin de se répandre, et ceux-là seuls qui en ont pu acheter le livre ou assister à la conférence acquièrent ces nouvelles vérités. Si le conférencier ne se dépouille de rien en donnant ces vérités à son auditoire, ses auditeurs, quand ils ont payé le droit de l’entendre, se sont privés d’autres agréments pour obtenir celui-là.\par
Tout enseignement, surtout dans le passé, commence par  \phantomsection
\label{v1p294}être plus ou moins hermétique, \emph{ésotérique}, et plus il l’est, plus on se rapproche des conditions de l’hypothèse ci-dessus. Dans les siècles mystiques du moyen âge, la science paraissait attachée à la possession d’un manuscrit. L’étude de la jurisprudence, comme celle de la théologie, était quelque chose de cabalistique, une aimantation par le contact, restreinte à un petit groupe d’élus. Entre la communication des \emph{connaissances} d’homme à homme, et la transmission des \emph{richesses} d’homme à homme, il y avait beaucoup moins de différence au moyen âge qu’à présent, et dans les temps primitifs de la Grèce qu’au siècle de Xénophon\footnote{ \noindent Les initiés aux idées de Pythagore formaient un groupe aussi fermé que les initiés aux mystères d’Eleusis.
 }. L’évolution sociale semble avoir marché dans le sens d’une accentuation toujours plus marquée de cette différence.\par
A vrai dire, cependant, la communication des pensées a toujours différé essentiellement de celle des richesses. Même quand un sauvage ne révèle sa recette secrète de poison ou d’engin militaire à un autre sauvage que moyennant la révélation par celui-ci d’un autre secret, cet échange ne ressemble en rien à l’échange qu’ils font de leurs armes ou de leurs outils. Car chacun d’eux se dépouille de son arme ou de son outil en l’échangeant contre un autre, tandis qu’il n’oublie pas son secret en apprenant celui d’autrui.\par
En somme, cela revient à dire que les conditions de la transmission sociale des \emph{sensations} ne sont pas les mêmes que celles de la transmission sociale des \emph{pensées.} On ne peut, par la parole et par l’exemple, — sauf le cas de suggestion hypnotique — susciter chez autrui une sensation de goût sucré ou de douce tiédeur qu’on éprouve en mangeant tel fruit ou revêtant telle fourrure. Pour communiquer à un de ses semblables cette jouissance-là il faut lui procurer un peu du sucre qu’on mange ou un vêtement pareil à celui qu’on porte. Mais, par la simple parole, on suscite en lui des pensées toutes semblables à celles qu’on possède. Si donc  \phantomsection
\label{v1p295}ces pensées répondent en lui à une curiosité très vive, aussi vive que la gourmandise ou la sensualité de l’homme inculte, une simple parole lui aura valu des jouissances gratuites, aussi intenses que les plaisirs matériels les plus coûteux.\par
Par suite, les désirs spirituels s’offrent à nous comme le grand, l’immense débouché de l’activité humaine dans l’avenir, et le progrès tend à la supériorité grandissante de leur développement sur celui des désirs physiques. Cherchons les désirs dont les satisfactions soient aussi aisées à répandre, à généraliser, que ces désirs eux-mêmes. Nous ne trouverons que le désir de connaître. Il est aussi facile, plus facile même, on s’en aperçoit souvent, de répandre des connaissances parmi des écoliers, ou même parmi des adultes, que de faire naître parmi eux le goût de les posséder. Au contraire, le désir d’exercer tel ou tel métier se répand facilement par l’exemple d’autrui, mais il n’est pas aussi facile de fournir du travail de ce genre à tous ceux qui en demandent. Le désir de consommer se propage encore plus aisément que celui de produire, mais la difficulté est de multiplier les moyens de consommation avec autant de rapidité. Cette inégalité entre la vitesse de diffusion de presque tous les désirs et celle des objets qui leur correspondent crée des problèmes anxieux et en apparence insolubles. Par bonheur, du choc des appétits et des désirs qui bataillent entre eux dans la lutte économique se dégage peu à peu, stimulé incessamment par ses satisfactions mêmes, le désir profond, caché sous tous les désirs, — sous les désirs les plus matériels même, tous réductibles, au fond, à des soifs d’expériences nouvelles, de vérification, de sécurité, de certitude, — la curiosité, passion finale, confluent et réconciliation de toutes les autres, qu’elle résume en les accordant. Là est le salut, l’au-delà non pas posthume mais ultérieur, où toutes les antinomies économiques se résoudront, autant qu’elles puissent être résolues, dans le domaine immensément \phantomsection
\label{v1p296} élargi des félicités de l’art, de la pensée et de l’amour, aussi gratuites et aussi indivises qu’infinies, collectivisme idéal qui dispensera du collectivisme brutal, vraisemblablement impossible. Et il était bon, m’a-t-il semblé, d’indiquer cette considération ici, pour bien marquer la subordination de l’activité économique à l’activité intellectuelle, seule capable, non seulement de donner la solution théorique, mais de résoudre pratiquement les graves problèmes qu’elle pose.
\subsubsection[{I.6.d. Caractère tout subjectif de la monnaie.}]{I.6.d. Caractère tout subjectif de la monnaie.}
\noindent Revenons à la monnaie. En essayant de la définir plus haut, nous avons défini aussi bien l’économie politique. La monnaie est la notion économique par excellence. L’économiste est, avant tout, un financier. Jusqu’où va le domaine des finances, de l’achat et de la vente, jusque-là, mais pas au delà, va la science économique. Elle s’arrête au seuil du monde intellectuel. Les unes après les autres, toutes les richesses, même les plus incomparables entre elles en apparence, deviennent de plus en plus évaluables en monnaie, cette commune mesure des choses les plus hétérogènes ; mais, de moins en moins, les connaissances, les vérités, se prêtent à ce genre d’évaluation. Tout ce qui a trait à la production et à la consommation sur place, qui s’opèrent dans les familles et les tribus closes des premières sociétés, échappe à l’économiste ; les seules relations d’échange entre familles, entre tribus, où la monnaie apparaît, commencent à le regarder. La civilisation a pour effet de faire entrer successivement dans le commerce, c’est-à-dire dans le champ de l’économiste, une foule de choses qui auparavant étaient sans prix, des droits et des pouvoirs mêmes ; aussi la théorie des richesses a-t-elle empiété sans cesse sur la théorie des droits et sur la théorie des pouvoirs, sur la jurisprudence et la politique. Mais, au contraire, par  \phantomsection
\label{v1p297}la gratuité toujours croissante des connaissances libéralement répandues, la frontière se creuse entre la théorie des richesses et ce qu’on pourrait appeler la théorie des lumières. Science encore innomée qui consisterait à formuler les lois de la découverte et de la vulgarisation des vérités (réelles ou imaginaires), de la formation et de la transformation des langues, des religions et des sciences, et de leur propagation de classe en classe, de peuple en peuple, ainsi que de leurs luttes et de leurs accords, à peu près comme l’économie politique recherche les lois de la création et de la vulgarisation des richesses, de leurs concurrences et de leurs concours. Appelons \emph{logique sociale} cette théorie des lumières, et nous la verrons s’opposer très nettement à la \emph{téléologie sociale} dont l’économie politique est l’expression la plus étendue et la plus parfaite jusqu’ici.\par
Elle doit cette perfection relative précisément à la possession d’une quantité spéciale, la monnaie, qui lui donne un cachet singulier de précision. Mais, on le voit, elle ne doit cet avantage de forme qu’à l’infériorité de son contenu. C’est parce que les richesses ne participent pas au privilège éminent des connaissances, de pouvoir s’acquérir sans nul sacrifice, qu’il existe une monnaie ; et c’est parce que les connaissances ont ce privilège que leur libre diffusion ne donne lieu à rien d’équivalent. Il n’en est pas moins certain que toutes les connaissances, si dissemblables qu’elles soient, sont comparables sous un certain rapport, comme toutes les richesses ; à savoir par le degré de croyance qui s’attache à elles. Et si, tenant compte à la fois de l’intensité moyenne de cette croyance et du nombre d’individus qui la partagent, on mesurait par le \emph{produit} de ces deux données la \emph{quantité de vérité} qui est propre à chaque idée répandue dans le public, on pourrait dresser un \emph{inventaire des lumières} d’une nation, aussi bien — mais beaucoup plus mal aisément — qu’un inventaire de ses richesses. On pourrait aussi, par cette manière toute simple de mesurer la \emph{vérité  \phantomsection
\label{v1p298}sociale} des idées, suivre les variations historiques de la vérité des diverses connaissances et rechercher le sens général de ces variations. Seulement, faute d’un mètre objectif de cette valeur-vérité, les notions ainsi obtenues n’auraient rien de frappant pour l’esprit, malgré leur importance capitale.\par
La monnaie est le mètre objectif des valeurs, mais plus objectif en apparence qu’en réalité. Si on y regarde de près, on voit sans peine que toute la \emph{substance monétaire}, pour ainsi dire, des métaux précieux, est subjective, qu’elle consiste, avant tout, dans une croyance générale, dans un acte de foi universel en eux, et que la théorie classique de la monnaie-marchandise est erronée. Si elle était vraie, l’or et l’argent devraient perdre de leur valeur monétaire dans un pays dès qu’ils commencent à y surabonder, et, par suite de leur dépréciation, les prix de toutes choses devraient se mettre à augmenter dans la proportion même de l’augmentation de leur quantité. Par exemple, dans tous les pays à circulation d’argent, où l’argent, démonétisé ailleurs, afflue de toutes parts, tous les prix devraient s’élever très vite de nos jours. Mais, en fait, rien de pareil encore. « La lumière s’est faite, dit M. Mongin\footnote{ \noindent \emph{Revue d’économie polit.}, février 1897.
 }, les enquêtes se sont multipliées, les agents consulaires ont fourni leurs renseignements, et tous les témoignages concordent sur ce point que la masse des prix est restée stationnaire. L’habitant de l’Extrême-Orient continue à recevoir le même poids d’argent pour sa journée de travail, pour ses porcelaines, ses incrustations et ses produits agricoles. » Autre observation plus instructive encore. Depuis un quart de siècle, un très grand nombre d’articles fabriqués ont baissé de prix par degré, et « cette baisse des prix a coïncidé avec une production des métaux précieux d’une extraordinaire puissance, avec une économie notable des frais d’exploitation, avec l’emploi le plus large  \phantomsection
\label{v1p299}des moyens de crédit, avec les éléments les plus propres à déprécier la monnaie métallique. Toutes les conditions monétaires semblent s’étre réunies pour entraîner la hausse générale des prix et c’est la baisse qui l’emporte ! » Autre remarque encore, très pénétrante. Dans la théorie classique, il est question « d’un rapport entre les quantités de monnaie et les quantités de marchandises. Mais quelles sont ces quantités de monnaie ? Celles qui existent dans le monde entier ou celles qui existent dans chaque nation considérée isolément ? Il faudrait opter, et cependant les raisonnements oscillent de l’un à l’autre, suivant les besoins de la thèse particulière qu’on veut établir\footnote{ \noindent Voir la suite p. 150.
 } ».\par
De tout cela, l’auteur cité conclut qu’il convient de rectifier la théorie classique en y faisant jouer un rôle prépondérant à l’empire de la coutume, lequel s’expliquerait, d’après lui, par une sorte d’effort universel afin d’obtenir la fixité de l’unité de compte, et, par suite, de maintenir la valeur nominale des monnaies où s’incarne cette mesure générale des valeurs, — jusqu’au moment du moins où soit l’afflux, soit la raréfaction des métaux précieux atteignent un point où il est impossible de soutenir plus longtemps cette fiction. — L’explication de M. Mongin me semble vraie mais insuffisante. Il y a ici autre chose que la coutume. Si les variations du stock monétaire, même considérables, même connues de tous, n’influent pas — pour un temps du moins — sur les prix, c’est que chacun, en recevant une pièce de monnaie, sait qu’il ne la gardera pas assez longtemps en poche pour que sa dépréciation ait sensiblement augmenté, pour que son pouvoir d’achat ait sensiblement diminué, jusqu’au moment où il la remettra en circulation. Cette diminution du pouvoir d’achat de la monnaie dans l’intervalle de son passage dans la poche de chaque consommateur, est une quantité infinitésimale dont nul ne tient  \phantomsection
\label{v1p300}compte et qui, par suite, ne parvient jamais à \emph{s’intégrer ;} jusqu’au moment où la disproportion éclate soudain aux yeux entre la valeur nominale et la valeur réelle de la monnaie, et où un krack se produit.\par
Et, de fait, au bout d’un temps, la surproduction d’or ou d’argent se fait sentir sur les prix, qu’elle élève, mais elle les élève toujours beaucoup moins que la quantité d’or ou d’argent ne s’est accrue. De 1492 à 1600, en 108 années, après la découverte de l’Amérique, la quantité d’or et d’argent, en circulation sur le continent européen, a \emph{décuplé} d’après les estimations les plus sérieuses. Or, les prix n’ont pas tout à fait \emph{quintuplé} (ils n’ont augmenté que de 470 p. 100 environ), ce qui est beaucoup au point de vue des bouleversements sociaux qui en ont été la conséquence, ruine des uns, enrichissement soudain des autres, mais ce qui est très inférieur à ce que l’application de la loi de l’offre et de la demande aurait exigé. Autre exemple. De 1851 à 1870, il y a eu un énorme afflux de métaux précieux sur le marché européen, la quantité d’or et d’argent précédemment existante s’est trouvée doublée. Quel effet ce prodigieux accroissement a-t-il eu sur les prix ? C’est un problème très débattu. M. Paul Leroy-Beaulieu, après l’avoir longuement discuté, est amené à conclure que, quoique augmentée de moitié dans cet intervalle, la masse d’or et d’argent circulante n’a été dépréciée que dans la mesure de 15 à 20 p. 100\footnote{ \noindent \emph{Traité d’économie polit.}, t III, p. 208.
 }.\par
Tous ces faits sont inexplicables si l’on n’a égard à ce qu’il y a de purement subjectif, psychologique, dans la nature de la monnaie. — Aussi l’effet direct de la découverte d’une mine d’or ou d’argent, et le plus important à considérer, ce n’est point la dépréciation monétaire qui s’ensuit, et, par conséquent, la hausse des prix, mais c’est la surexcitation de l’activité par la surexcitation des espérances ; phénomène qui, conformément à la loi générale  \phantomsection
\label{v1p301}de la propagation des exemples, se produit d’abord dans les parties les plus denses, les plus affairées, les mieux informées de la population, dans les milieux urbains, et de là se répand peu à peu jusque dans les campagnes. — Les historiens n’hésitent pas à compter parmi les causes principales de la grandeur d’Athènes l’exploitation des mines d’argent du Laurium, qui étaient d’une richesse exceptionnelle à l’époque classique. On a pu dire, avec quelque exagération il est vrai : « Sans Laurium, point de marine athénienne ; sans marine athénienne, point de bataille de Salamine ; sans bataille de Salamine, point de siècle de Périclès. » (Théodore Reinach.) S’il en est ainsi, voilà une découverte, peut-être accidentelle, la découverte de la mine dont il s’agit, qui a \emph{aiguillé} toute l’histoire ancienne et moderne dans la voie où elle se déroule encore.
\subsubsection[{I.6.e. Le pouvoir, le droit, l’argent.}]{I.6.e. Le pouvoir, le droit, l’argent.}
\noindent Arrêtons-nous à considérer encore les caractères distinctifs de la monnaie et les transformations économiques opérées par son avènement. Par elle l’économie politique revêt un air de \emph{physique sociale} qui a séduit et trompé, non sans excuse, les premiers sociologues. La monnaie a ce caractère commun avec la \emph{force}, notion essentielle de la physique, d’être une possibilité, une virtualité infinie. La force est la possibilité d’une certaine quantité de mouvement dans une infinité de directions ; la monnaie est la possibilité d’une certaine quantité de valeur obtenue par une infinité d’achats. Il est remarquable que l’évolution économique conduit inévitablement de l’échange en nature à l’achat et à la vente, des biens concrets aux valeurs monétaires, c’est-à-dire des réalités pures et simples aux virtualités réalisables, de l’énergie actuelle à l’énergie potentielle\footnote{ \noindent Avec un peu d’imagination métaphysique, ne serait-on pas en droit de se demander, à ce propos, si cette nécessité évolutive ne jetterait pas quelque jour sur la genèse de la \emph{force} physique ? Est-ce que, de même qu’il y a eu, nous le savons, une phase économique \emph{pré-monétaire}, il n’y a pas eu, en un passé insondable au physicien, une phase physique \emph{pré-dynamique} où ce n’étaient pas encore des forces, des virtualités réalisables de mille manières différentes, qui étaient les grands agents naturels, mais simplement des chocs et des impulsions, des mouvements reçus ou transmis ? Nous ne pouvons vraiment nous faire aucune idée un peu nette de l’énergie potentielle, si ce n’est en supposant qu’elle capitalise et représente de l’énergie cinétique, comme l’argent est le signe et la concentration de la richesse.
 }.\par
 \phantomsection
\label{v1p302}Il y a quelque analogie entre certains problèmes fondamentaux que soulève la monnaie et certaines difficultés non moins fondamentales que la mécanique soulève. Toute hausse des prix, par exemple, peut s’expliquer soit par une hausse de la valeur des objets dont le prix est coté plus haut en monnaie, soit par un abaissement de la valeur de la monnaie. Même alternative en mécanique relativement au mouvement apparent d’un corps : il peut s’expliquer soit par son mouvement réel, les corps environnants étant immobiles, soit par le mouvement de ceux-ci en sens inverse. — Une question semblable ne saurait se poser à d’autres égards, en biologie et en sociologie : on ne demandera pas si le vieillissement d’un individu tient au rajeunissement des individus voisins, ou si le développement d’un art, d’une science, d’une institution, tient à l’atrophie d’autres arts, d’autres sciences, d’autres institutions. Le seul fait que, à l’occasion de ces \emph{changements} et de beaucoup d’autres, on ne peut pas se poser le dilemme que comportent les \emph{mouvements} mécaniques et aussi bien les variations des valeurs, suffit à montrer que ces changements ont une réalité plus certaine, à coup sûr moins subjective, que ces mouvements et ces variations.\par
Suivant les physiciens, les phénomènes physiques seraient une conversion continuelle d’énergie virtuelle en énergie actuelle et \emph{vice versa.} Pareillement, la vie économique est un perpétuel échange de la monnaie contre de la richesse concrète, et de la richesse concrète contre de la monnaie. L’argent a cela de particulier qu’il est une virtualité qu’on  \phantomsection
\label{v1p303}échange, sans travail aucun, pour n’importe quel genre de réalité vénale. Multiplier les virtualités de cet ordre, en accroître la quantité absolue et la quantité proportionnelle, c’est le signe d’une société qui s’enrichit. Il est trois grandes catégories de virtualités qui sont les forces sociales par excellence, et dont l’accroissement simultané sert de mesure au progrès social : le Pouvoir, le Droit, l’Argent. Ce sont les trois principaux moyens d’action sur les hommes. Le droit et le pouvoir agissent sur autrui par la crainte, car ils ont à leur disposition la force matérielle incarnée dans l’armée et la police, et dont la menace seule suffit, ou l’idée vaguement apparue, pour lui valoir l’obéissance et le respect ; la richesse agit par l’\emph{espérance}, par le désir et la confiance qu’elle inspire. Mais cela n’est vrai de la richesse en nature qu’à l’égard d’un très petit nombre d’individus déterminés qui convoitent précisément les objets possédés par nous. La monnaie seule donne barre sur tout le monde, et dans un rayon bien plus étendu que le droit et le pouvoir. Le droit et le pouvoir ont une action circonscrite dans les limites d’un territoire plus ou moins étroit, commune, province, nation. La monnaie étend son champ d’action bien au delà des frontières d’un État, sur toute la terre civilisée. De là ce caractère international des sociétés fondées sur la richesse monétaire.\par
Une autre différence entre la monnaie d’une part, et le droit ou le pouvoir de l’autre, est à première vue beaucoup moins à l’avantage de la première. On dépense sa richesse monétaire ou autre, en s’en servant ; tandis que, en faisant usage de son pouvoir ou de son droit, on n’épuise ni l’un ni l’autre. Et même, si cela est quelquefois moins vrai du pouvoir que du droit, si l’on voit l’autorité d’un homme politique s’épuiser en s’exerçant, on voit, en revanche, bien souvent cette autorité se fortifier et s’accroître par son exercice même, comme un droit s’enracine par sa seule durée. — Mais cette différence n’a trait qu’aux petits emplois inférieurs de la monnaie, aux achats ; elle est inapplicable  \phantomsection
\label{v1p304}aux placements supérieurs, et toujours plus nombreux, de la richesse monétaire, par le prêt et le crédit. Le crédit a permis à la richesse monétaire, non seulement de s’utiliser sans se dépenser, mais de s’augmenter même, indéfiniment, grâce à l’intérêt de l’argent, par l’usage qu’on en fait. Nous reparlerons de l’intérêt de l’argent plus loin.
\subsubsection[{I.6.f. La terre et l’argent, ressemblances et différences entre leurs rôles économiques, entre leurs modes d’accroissement, et de répartition.}]{I.6.f. La terre et l’argent, ressemblances et différences entre leurs rôles économiques, entre leurs modes d’accroissement, et de répartition.}
\noindent Dans les sociétés primitives, il y a peu d’argent. A défaut de monnaie, on est obligé de payer en nature les services, ce qui est une satisfaction momentanée et passagère donnée à ceux qu’on désire s’attacher ; excepté quand on leur donne des \emph{terres.} Les terres sont des virtualités aussi, mais des virtualités qui ne sont échangeables, en l’absence de monnaie, que contre d’autres virtualités semblables, et qui ne sont convertibles que moyennant travail en un nombre limité de réalités définies : les récoltes. La terre est une possibilité de jouissances, qui, à la différence de l’argent, ne s’actualise jamais sans labeur, ne s’actualise qu’en un certain lieu et non partout, et est enfermée dans un cercle très étroit de réalisations très précises. De ces différences découlent certaines institutions qui s’imposent d’elles-mêmes aux époques où la rémunération de grands services ne peut s’effectuer qu’en terres\footnote{ \noindent Heeren, dans son \emph{Essai sur l’influence des croisades}, 1808, explique ainsi l’origine du \emph{système féodal}, incidemment. « Ce système au fond, et pris dans toute sa rigueur, est purement militaire. Il s’établira toujours quelque chose d’analogue parmi les peuples guerriers et qui manquent d’argent pour payer leurs soldats. Au lieu de solde, on donne à ceux-ci des terres : et réciproquement ils s’engagent, moyennant la possession de ces fonds, au service militaire. » En note : « Telle est encore à peu près aujourd’hui (1808) l’organisation de l’Égypte sous les Mamelouks. »
 }.\par
D’abord, la rémunération ne serait qu’illusoire si le donataire des terres n’avait pas le moyen de les faire cultiver, s’il devait les travailler lui-même. Donc, ce genre de payement  \phantomsection
\label{v1p305}implique l’institution de l’esclavage ou du servage. De plus, il suppose une grande simplicité de besoins et de goûts et l’amour d’une vie sédentaire. Enfin, lorsqu’un roi barbare n’a d’autre trésor de guerre, d’autres ressources quasi financières pour payer ses capitaines ou ses officiers que les terres conquises, ce trésor s’épuise vite. Pour le renouveler de temps en temps, il faut entreprendre de nouvelles guerres, le droit des gens d’alors admettant — et ne pouvant pas ne pas admettre — l’expropriation des vaincus. Les exproprier, c’est comme, à présent, exiger d’eux une indemnité en argent. Et, même en temps de paix, il convient que, par divers procédés, le trésor royal, c’est-à-dire le domaine royal, se grossisse de recettes intermittentes : de là la \emph{sécularisation} des biens monastiques, ou bien la \emph{confiscation} des terres appartenant aux condamnés et à leurs familles. Cette odieuse forme de pénalité s’explique de la sorte, si elle ne s’excuse pas. Ne nous vantons pas trop de l’avoir abolie : car la nécessité qui l’avait fait établir a cessé d’exister. Une expropriation en masse, qui pouvait se justifier dans une certaine mesure aux époques de barbarie, serait monstrueuse à notre époque où, s’il n’y a plus de terre libre, on n’a plus besoin de terre libre pour rémunérer les services, puisqu’on a l’argent, la monnaie métallique ou fiduciaire, aux sources inépuisables.\par
La comparaison de la terre avec l’argent, au point de vue de leur rôle économique, est digne d’attention. Nous verrons plus loin que le prêt à intérêt s’explique, avant tout, par le fermage ou par le bail à cheptel. Il est arrivé pour l’argent ce qui est arrivé pour toute invention nouvelle qui, par le fait même qu’elle se substitue à une autre, en revêt la livrée ; les premières haches de bronze rappellent les haches en silex ; les premières maisons de pierre ont pris la forme des maisons de bois qu’elles ont remplacées. De même, l’argent qui, peu à peu, a pris la place de la terre dans les convoitises et les ambitions humaines, s’est modelé sur lui, a voulu être  \phantomsection
\label{v1p306}frugifère comme lui. Mais la vérité est qu’il diffère profondément de la terre et que l’ère où l’argent donne le ton, en succédant à celle où régnait la terre, l’a ensevelie sans bruit et à jamais.\par
Ces différences sont importantes, au point de vue soit de la manière dont de nouvelles terres ou de nouvelles quantités d’or et d’argent ou de monnaie de papier, viennent à s’ajouter à celles qui existent déjà, soit de la manière dont ces terres ou ces monnaies se distribuent entre les membres d’une population donnée.\par
En premier lieu, c’est très rarement par l’occupation d’une terre vierge, c’est le plus souvent par la conquête violente d’une terre déjà occupée, que le territoire d’un peuple s’agrandit, son territoire continental ou son territoire colonial. Même quand, à l’origine d’un de ces agrandissements, il y a une découverte, la découverte d’une île ou d’un continent, il n’a presque jamais suffi de découvrir, il a fallu conquérir aussi. — Mais, quand la quantité d’or ou d’argent à distribuer entre les hommes en rapports mutuels de commerce vient à s’accroître véritablement, c’est toujours par suite d’une découverte, la découverte d’une mine d’or ou d’argent, qui était \emph{res nullius} auparavant. Les découvertes de ce genre se singularisent d’ailleurs parmi toutes les autres et méritent bien d’arrêter les regards de l’économiste. D’abord, elles sont purement fortuites ; les découvertes géographiques mêmes sont bien loin de l’être au même degré, car le calcul et le raisonnement y ont leur grande part. Mais, malgré leur caractère remarquablement accidentel, les découvertes des métaux précieux (y compris celles d’anciens trésors cachés) ont des effets profonds et prolongés, comme nous l’avons vu pour celle des mines d’Athènes. Quand on découvre des filons de kaolin ou des carrières de ciment, ou même des mines de fer, ces matières premières sont destinées à se détruire plus ou moins vite par l’usage qu’on en fait. Les porcelaines se brisent, le  \phantomsection
\label{v1p307}ciment ou la chaux ne servent pas deux fois, le fer se rouille, ou bien, une fois employé, il ne sert à d’autres usages que moyennant une perte de substance. Au contraire, l’or et l’argent découverts se conservent presque inaltérablement, à peine diminués par le frai. Toute richesse autre que la monnaie n’est échangeable qu’accessoirement, elle est consommable essentiellement. Mais la monnaie est inconsommable essentiellement, et, essentiellement, elle est échangeable. Comment peut-on dire d’une richesse qui s’échange toujours sans se consommer jamais qu’elle est une marchandise comme une autre ? — Les parties des métaux précieux utilisées pour la bijouterie sont refondues sans perte et monnayées quand on veut. En somme, les métaux précieux ont ce privilège presque complet, et presque unique, de pouvoir s’accumuler indéfiniment comme les théorèmes mathématiques qui, une fois découverts, ne cessent jamais d’être vrais. Et à la vérité éternelle de ceux-ci je comparerais l’utilité quasi éternelle de ceux-là, si la valeur de l’or et de l’argent ne s’altérait à la longue, bien lentement, il est vrai, comparée à celle des autres produits.\par
Mais, si les découvertes des métaux précieux portent sur des objets presque impérissables, elles s’épuisent vite, et leurs effets s’atténuent de plus en plus. Leur maximum d’efficacité correspond au moment où elles se révèlent au public ; tandis que la découverte de la locomotive ou du télégraphe électrique a des conséquences toujours grandissantes. En cela la découverte d’une mine d’or ne souffre aucune comparaison avec celle d’une île nouvelle. Si la découverte de l’Amérique n’avait consisté qu’à y découvrir des mines d’or ou d’argent, et que, sur ce nouveau continent, ni végétal, ni animal n’eût pu vivre, les résultats du merveilleux voyage de Colomb seraient depuis longtemps effacés : il n’en subsisterait qu’une notable élévation de tous les prix, chose de peu d’importance au bout d’une génération ou deux. — D’ailleurs, pour les découvreurs d’une mine d’or, ou pour le  \phantomsection
\label{v1p308}monarque qui bénéficie de cette trouvaille, ou pour les actionnaires de la société qui l’exploite, cette découverte est assimilable à celle d’une île nouvelle, et même elle est plus avantageuse encore. C’est comme s’ils s’étaient emparés d’un Eldorado inoccupé et paradisiaque, qu’ils n’ont besoin ni de conquérir ni de défricher même pour jouir de ses fruits spontanés. Mais, pour le reste de l’humanité, ce bienfait qui leur tombe du ciel se réduit à un stimulant de la production, à un aiguillon de l’espérance et du travail ; tandis qu’une nouvelle terre ouverte aux explorateurs accroît la population humaine, enrichit la faune et la flore, offre à l’imagination de nouveaux spectacles naturels qui, en la diversifiant, la réjouissent et ajoutent à la lyre du cœur de nouvelles cordes, de nouvelles fibres patriotiques, destinées à vibrer dans le grand concert d’une civilisation élargie.\par
Quant à la monnaie de papier, ce n’est pas par des découvertes, c’est par des entreprises gouvernementales et financières, par des émissions d’assignats ou de billets de banque, que sa quantité s’augmente.\par
En second lieu, ce n’est pas de la même manière qu’une terre nouvellement conquise et qu’une masse d’or nouvellement extraite, se distribuent entre les individus d’une nation. C’est pour rémunérer des services militaires ou politiques que les terres conquises en tout pays, sont, à l’origine, distribuées en grands domaines entre les officiers du conquérant, entre les premiers concessionnaires de l’État, s’il s’agit d’un gouvernement moderne. Ces premiers \emph{latifundia}, dont un grand nombre, dans la Gaule romaine, avaient l’étendue de nos communes actuelles, héritières souvent de leur nom\footnote{ \noindent Les grands domaines concédés par l’État français en Tunisie peuvent soutenir la comparaison comme étendue avec ces \emph{villœ} antiques.
 }, ont été divisés et subdivisés par chaque propriétaire entre ses vassaux, entre ses serfs, entre ses fermiers, ou bien, — si déjà des sources d’argent ont jailli, par exemple dans des villes industrielles, — entre des acquéreurs \phantomsection
\label{v1p309} qui ont eux-mêmes affermé ou vendu des parcelles de leur fragment de bien. Peu à peu, on aboutit au morcellement actuel du sol entre des propriétaires petits, moyens ou grands. Et, certes, nul ne peut dire que la répartition du sol conquis national (ou colonial) ainsi opérée, soit la meilleure qui puisse être imaginée, au point de vue du \emph{maximum} et de l’\emph{optimum} du rendement. Mais il n’y a pas plus de raisons de penser que la répartition de l’or et de l’argent extraits soit la meilleure concevable, au point de vue de l’équité ou de l’utilité générale. Ici, à la vérité, on ne voit pas de distribution arbitraire et imposée de force. Les premiers découvreurs ou frappeurs des métaux précieux les ont répartis entre leurs concitoyens par l’échange librement consenti. Toutefois, les formes violentes de la spoliation et les formes visibles du privilège ne sont pas les plus redoutables ; le pouvoir que confère la possession de l’or, s’il est moins manifeste et moins envié que le pouvoir attaché à la possession du sol, est d’une nature infiniment plus subtile et plus efficace, il va beaucoup plus loin et beaucoup plus vite, et agit invisiblement. A combien d’abus de ce pouvoir, à combien d’exactions et de rapines impunies, impunissables à vrai dire, a donné lieu le monopole d’avoir de l’or, chez ses premiers détenteurs ! Ce n’est pas sans motif que l’Église et la conscience universelle ont flétri l’usure.\par
Du reste la répartition du sol est liée à celle de l’or, et, dès que le pouvoir de l’or s’est accentué, la première est devenue une dépendance de la seconde. A partir du moment où le capital mobilier fit son apparition historique, l’achat ou la vente des biens fonciers deviennent les modes habituels de la répartition des terres, et la terre va là où va l’argent.\par
Si donc on se demande : Quels sont les inconvénients du morcellement actuel du sol, et quels seraient ses remèdes ? on doit se demander d’abord : quels sont les inconvénients et quels seraient les remèdes de la distribution actuelle du capital mobilier ? Les deux questions sont connexes et ne  \phantomsection
\label{v1p310}peuvent être divisées. — Comme nous nous réservons de traiter du collectivisme plus tard, nous n’avons pas à les étudier dès maintenant. Disons seulement qu’elles intéressent la politique et la morale à un plus haut point encore que la science économique, et que la première injustice et la plus criante qui s’offre à nos yeux, en fait de répartition du sol ou de l’or, est d’une nature telle qu’elle semble presque irrémédiable. L’inégalité des apportionnements territoriaux ou monétaires, en effet, est grande entre les individus ; mais elle est plus grande encore et plus monstrueuse entre les États. Tel peuple occupe un immense territoire, fertile et à demi inexploité, où se dilate à volonté son insuffisante population ; tel autre étouffe dans d’étroites frontières, sous un ciel rigoureux, sur un sol ingrat. Tel peuple déborde de capitaux ; tel autre en est totalement dépourvu. Et il paraît encore plus difficile de remédier à cette grande et fondamentale iniquité, à cette souveraine injustice internationale, en ce qui concerne le sol et le climat. Car, par la guerre, une nation brave et mal lotie en terres, peut rétablir la justice à son profit en expropriant en partie quelque nation voisine et plus privilégiée. Mais le peuple pauvre aura beau piller son voisin riche en capitaux, l’or trouvera mille canaux souterrains pour rentrer dans les bourses d’où il est sorti. Impossible, ce semble, d’empêcher entre les nations cette inique et prodigieusement inégale distribution des richesses. Et n’est-ce pas là, de beaucoup, l’inégalité la plus injustifiable ? Elle l’est bien plus, à coup sûr, que toutes les inégalités qu’on prétend détruire entre les individus d’une même nation.
\subsubsection[{I.6.g. Effets psychologiques du règne de l’argent ; ses bienfaits, substitution des payements en argent aux payements en nature, facilités des voyages, etc.}]{I.6.g. Effets psychologiques du règne de l’argent ; ses bienfaits, substitution des payements en argent aux payements en nature, facilités des voyages, etc.}
\noindent Mais laissons là ce problème et disons un mot des effets psychologiques ainsi que des conséquences économiques et sociales de la monnaie. De ses effets psychologiques d’abord.  \phantomsection
\label{v1p311}L’avènement de la monnaie a enrichi le cœur humain de sentiments nouveaux et de vices nouveaux. Nous lui devons l’orgueil financier, la béatitude spéciale du milliardaire appuyé sur son portefeuille, comme l’orgueil d’un capitaine se fonde sur son armée. Ce que le guerrier antique dit à sa lance et à son bouclier, dans une épigramme grecque : « Grâce à vous, je suis libre, j’ai des loisirs sans fin, je me fais servir par des esclaves », le riche moderne peut le dire a son coffre-fort. Le culte de l’or, cette passion qui a quelque chose de religieux par le caractère vaste et vague, indéterminé et illimité, des perspectives de bonheur que son objet lui fait entrevoir, est une fibre importante de l’âme humaine. Le plaisir d’économiser, de gagner de l’argent, est un enivrement tout spécial qui n’a rien de commun avec le simple avantage de recevoir un bien déterminé, un bijou, un meuble, un livre. Autre chose est le plaisir de manger un bon fruit, autre chose la satisfaction intime et profonde de sentir sa santé se fortifier. Il y a de l’un à l’autre la différence de l’actuel au virtuel, j’allais dire du \emph{fini} à l’\emph{infini.} De même, le chagrin d’apprendre que votre banquier a fait faillite, que votre notaire a pris la fuite vous emportant vos économies, — douleur ressentie, il y a quelques années, par un nombre considérable de paysans français — est quelque chose de tout à fait à part parmi les afflictions humaines, et qui ne se compare à rien. Dans la plénitude de joie sourde et constante qui remplit le cœur d’un avare en train de s’enrichir, l’analyse découvrirait une combinaison unique d’éléments empruntés à la joie de l’amoureux qui sent son espérance grandir, à celle de l’ambitieux qui monte vers le pouvoir, à celle du croyant qui se croit sûr du ciel. Ce n’est pas sans raison que les anciens confiaient leurs trésors aux temples, qui ont été les premières banques de dépôt. L’or est une religion malheureusement éternelle.\par
Et, par contre-coup, nous sommes redevables aussi à la monnaie de cet amour mystique de la pauvreté qui exaltait  \phantomsection
\label{v1p312}l’âme d’un saint François d’Assise. Cet amour paradoxal, ce défi jeté au mépris général de la pauvreté, autre sentiment spécial et malheureusement si répandu, né de la monnaie — n’a pu naître que dans une société capitaliste, comme celle des cités italiennes du moyen âge. — Sous une autre forme, plus récente et plus contagieuse, se produit la réaction contre le culte de l’or : la haine du capitaliste, cette inspiration violente de Karl Marx et de son école, passion qui, en ce moment, remue le monde.\par
Mais est-il nécessaire d’insister pour montrer à quelle profondeur l’âme humaine a été labourée par les métaux précieux ? Il n’est point de drame ni de comédie qui épuisera jamais ce sujet. Parlons plutôt des [{\corr conséquences}] sociales de la monnaie. — Le transport de la force par l’électricité n’est rien en comparaison des services qu’a rendus aux hommes, et qu’est destiné à leur rendre encore, le transport de la valeur, par la monnaie métallique d’abord, fiduciaire ensuite. La monnaie a fait conservables indéfiniment des valeurs essentiellement passagères ; elle a fait mobilisables à des distances de plus en plus grandes, et avec une facilité croissante, des valeurs jusque-là localisées ; elle a fait, donc, susceptibles d’accumulation et de concentration indéfinies des utilités successives et éparses. Avant tout, elle a fait comparables des choses hétérogènes, elle a fait évaluables et nombrables des choses sans autre commune mesure. Par cette comparabilité, par cette mesurabilité de plus en plus généralisée, universalisée, elle soumet au calcul et au raisonnement ce qui était avant elle « affaire de goût », elle fournit une base en apparence rationnelle aux décisions volontaires par lesquelles certains biens sont sacrifiés à certains autres, et semble justifier de la sorte les empiétements successifs de la raison calculatrice sur le cœur coutumier et conservateur.\par
Elle seule a permis de voyager avec facilité, sans danger, sans escorte. Avant elle, qui disait voyageur disait pèlerin  \phantomsection
\label{v1p313}ou banni, et le pèlerinage était réputé une pénitence, comme le bannissement était regardé, avec raison, comme le plus cruel des châtiments. Toute la largeur d’esprit, toute l’ouverture d’imagination et d’âme, que donnent les voyages, c’est à la monnaie que nous le devons. — On peut même ajouter qu’elle a été un grand agent d’émancipation. Chaque pas du serf vers l’indépendance est marqué par une conversion de ses redevances en somme fixe, par le recul du métayage devant le développement du fermage. L’évaluation des redevances diverses en argent fait apercevoir entre elles des inégalités, des injustices qui, auparavant, ne pouvaient apparaître. Elle rend ces redevances, si hétérogènes qu’elles soient, comparables entre elles dans une vaste région, dans toute la région où a cours la monnaie en question, et rend, par suite, solidaires entre eux, étroitement unis d’un lien désormais senti, les débiteurs de ces rentes. Elle transforme aussi, sans en avoir l’air, la nature du rapport établi par ces redevances entre le seigneur et le tenancier, qui deviennent simplement l’un créancier, l’autre débiteur. Un payement en nature diffère presque autant d’un payement en argent, qu’un cadeau en nature d’un cadeau en argent. Le payement en nature est un lien personnel, d’homme à homme ; il symbolise et complète l’hommage ; de même qu’un cadeau en nature est un hommage affectueux, un tribut cordial. Mais un payement en argent, c’est quelque chose d’impersonnel, de sec et de froid.\par
Notons, en passant, que cette substitution des payements en argent aux payements en nature s’est accomplie par degrés, de proche en proche, et \emph{de haut en bas}, conformément aux lois générales de la descente des exemples. Elle s’est produite en Angleterre, d’après Ashley, dans le domaine royal bien longtemps avant de s’étendre aux domaines seigneuriaux, puis à tout le royaume. Dès le {\scshape xii}\textsuperscript{e} siècle, le roi, dans son domaine, se faisait payer en argent la plupart des redevances, à cause de la difficulté de charroyer jusqu’à lui les  \phantomsection
\label{v1p314}redevances en nature, et aussi pour entretenir et solder ses troupes. Ce n’est qu’au {\scshape xv}\textsuperscript{e} siècle qu’on voit les seigneurs anglais affermer leurs terres. C’est seulement quand le goût du luxe, c’est-à-dire des produits étrangers, non indigènes, à l’exemple des cours royales, est venu aux gentilshommes, que le besoin d’être payés en monnaie s’est fait vivement sentir à eux. Aussi une classe de commerçants n’a-t-elle pu commencer à se former que pour l’usage des hautes classes d’abord et la satisfaction de leurs fantaisies luxueuses\footnote{ \noindent « Les denrées que, dans un vieux dialogue anglais, le commerçant, suivant sa propre description, emporte avec lui, semblent être tous des articles de luxe, dont le besoin se fait sentir seulement dans les hautes classes... du drap de pourpre, des pierres précieuses... etc. » (Ashley.)
 }.\par
Ce passage des payements en nature aux payements en argent est irréversible, parce qu’il est un corollaire de la loi fondamentale du monde social, le rayonnement des exemples. Il est inévitable, à mesure que les hommes s’imitent et s’assimilent, que les choses humaines se comparent et s’évaluent. — L’évolution économique des diverses formes de la monnaie n’est pas moins orientée dans un même sens général. De la monnaie en nature, — c’est-à-dire de l’ivoire ou du tabac, ou de tout autre article choisi comme moyen d’échange, — on passe à la monnaie en métal, puis à la lettre de change ou au billet à ordre, puis au billet au porteur, — autrement dit, au billet de banque, à cours libre ou à cours forcé, — par la même raison logique qui contraint l’évolution psychologique à passer de la sensation à l’image, de l’image à l’idée et à l’idée de plus en plus abstraite et générale, concentrée et mobilisable. Cela ne veut pas dire d’ailleurs que cette série de phases soit unique et unilinéaire. Il y a bien des variantes et non sans importance. Le point de départ est multiple, en tout cas. Dans le monde classique, la monnaie a procédé du \emph{bétail} (pecus, pecunia). Dans l’Extrême-Orient, en Perse aussi, elle provient des armes, couteaux ou cimeterres. La sapèque, par exemple, a commencé \phantomsection
\label{v1p315} par être une monnaie de bronze en forme de couteau muni d’un anneau pour l’enfiler. Peu à peu l’anneau s’est épaissi et la lame a disparu. Ces couteaux monétaires avaient certainement été, à l’origine, des couteaux ou des poignards bien réels. Les primitives monnaies persanes avaient la forme de cimeterres. En Afrique et ailleurs, cet emploi des armes comme moyen d’échange est répandu. Cela se conçoit conformément à ce que nous avons dit plus haut sur la nécessité d’avoir pour monnaie une chose qui soit l’objet d’un désir constant et universel. Cette condition est réalisée soit par les bijoux, soit par les armes, soit par le bétail, suivant les époques ou les régions\footnote{ \noindent J’ai dit plus haut que les bijoux, chez les sauvages, très vaniteux comme on sait, avaient pu être la première monnaie. Et cette hypothèse semble confirmée par l’emploi industriel de nos métaux précieux, cet emploi étant principalement, presque exclusivement, décoratif, à l’usage des bijoutiers. Mais ne pourrait-on pas dire inversement et complémentairement, que le rôle décoratif de l’or et de l’argent leur vient peut-être moins de leurs caractères chimiques et de leur éclat que du prestige qui leur est attaché par leur rôle monétaire ? Si le fer était encore monnaie, comme il l’était au temps d’Homère, des incrustations de fer seraient encore un ornement recherché. La vue du fer, dans une décoration d’appartement, éveillerait l’idée de richesse. Les objets en nickel ne sont pas des objets de luxe, quoique ce métal soit aussi brillant que l’argent. Quand l’argent aura cessé d’être frappé, ce qui ne tardera guère, il perdra peu à peu tous les charmes qu’on lui reconnaît encore comme moyen de décoration.
 }.\par
On a prétendu que cette évolution économique de l’échange nous ramène, poussée à bout, au troc primitif, qui en serait à la fois, de la sorte, l’alpha et l’oméga. Ce qui a donné lieu à cette vue erronée, ce sont les \emph{Chambres de compensation} (les Clearing-house), grâce auxquelles d’immenses opérations financières sont soldées avec un minimum de déplacement monétaire par le simple échange de liasses de papiers commerciaux. Je regrette de voir M. Gide, d’ordinaire si pénétrant, accueillir ici la métaphore illusoire de la \emph{spirale} dont les sociologues ont tant abusé. En réalité, ce troc de valeurs de papier contre d’autres valeurs de papier, c’est-à-dire de possibilités indéfinies de richesses contre d’autres possibilités indéfinies de richesses, diffère autant du troc primitif, du troc  \phantomsection
\label{v1p316}d’un gibier contre un fruit ou [{\corr d’un}] vase d’airain contre une captive, que les propositions verbales d’un métaphysicien diffèrent des jugements de localisation instinctivement formés par un animal ou même par un homme qui perçoit un arbre ou un rocher. Peu importe, en effet, que le progrès du commerce nous conduise à remuer d’immenses sommes en déplaçant quelques pièces d’or : ce n’est pas moins de monnaie toujours qu’il s’agit dans ces chèques et ces billets troqués les uns contre les autres ; et, par eux, la monnaie consomme son [{\corr triomphe}] final, son œuvre à la fois gigantesque et terrible, merveilleuse et désastreuse, qui consiste à rendre tout évaluable pour rendre tout vénal, à tout niveler sous sa règle pour tout soumettre à son joug.
\subsubsection[{I.6.h. Ses méfaits. La liberté terrienne et la liberté monétaire ; le droit terrien et le droit monétaire.}]{I.6.h. Ses méfaits. La liberté terrienne et la liberté monétaire ; le droit terrien et le droit monétaire.}
\noindent J’ai parlé de ses bienfaits ; que n’aurais-je pu dire de ses méfaits ? Elle a fait sortir de terre des armées permanentes, elle a créé le despotisme des temps nouveaux, administratif et centralisateur, insidieux et envahissant. Avant elle, avant son règne, le monde a vu des démocraties que j’appellerai \emph{terriennes}, telle que Rome primitive. Depuis elles, les démocraties sont devenues presque fatalement ploutocratiques. Je disais qu’elle a été émancipatrice. Oui, à certains égards ; mais il est plus vrai de dire qu’elle a substitué au sens ancien, profond, profondément humain, de l’idée de liberté, un sens tout nouveau, plus superficiel peut-être et plus artificiel, plus aisément généralisable par suite. La liberté véritable, dans la première acception du mot, c’est l’indépendance, fondée, sinon sur la suppression de tout désir, chose impossible, admirable absurdité du stoïcisme, du moins sur la réduction du désir à un faisceau étroit de besoins simples et forts que satisfait seule, mais que satisfait pleinement, la propriété d’un coin de terre. C’est la \emph{liberté terrienne}, dont  \phantomsection
\label{v1p317}le patriarche antique a fourni le type, reproduit depuis par les premiers colons d’Amérique. Étendu au village, au bourg, ce type s’est réalisé dans le fief français, dans l’ancien manoir anglais, dont le caractère propre était de se suffire à soi-même, de n’avoir rien à acheter ni à vendre au dehors.\par
L’argent a brisé tous ces murs de clôture du désir, il pourchasse partout les derniers vestiges de cette farouche indépendance rustique et quasi stoïque qui est le démenti de ses prétentions, l’écueil de ses ambitions, et qui cherche sans cesse à se reformer, jusqu’en plein âge moderne, dans l’âme d’\emph{un Rousseau} ou d’\emph{un Tolstoï.} Qui sera vaincu dans ce duel ? Je l’ignore. Pour le moment, il est clair que l’idéal économique d’affranchissement propre aux sociétés modernes — et sur ce point économistes de toute école, y compris socialistes, sont d’accord — est précisément le contraire de l’idéal stoïcien. Au lieu de viser, comme Zénon, à l’émancipation par l’anéantissement du désir, les civilisés d’aujourd’hui tendent à s’émanciper, croient-ils, par la complication infinie des désirs et la solidarité de plus en plus intime des individus qui ne peuvent plus se passer les uns des autres, ou plutôt dont les uns, riches de monnaie, se font servir par les autres, — dans un rayon de plus en plus étendu, prolongé jusqu’aux limites du globe. Quand les hommes sont détachés de l’idéal de la liberté terrienne, il ne leur reste plus qu’à courir à l’extrême opposé, à l’idéal de la \emph{liberté monétaire}, celle du financier cosmopolite qui, véhiculé par les wagons-lits, est domicilié tour à tour dans tous les grands hôtels et toutes les belles villas à louer de l’univers entier. Et, fascinés par l’exemple de ces brillants nomades, se modelant sur eux dans la mesure de leur bourse, toutes les classes du peuple, l’une après l’autre, rompent la chaîne du bonheur ancien, de la simplicité sédentaire au champ natal, pour entrer à leur tour dans la voie de cette frénésie de locomotion, de cette fureur de navigation vers un port imaginaire.\par
 \phantomsection
\label{v1p318}Est-ce que cette fièvre montera toujours ? Est-ce que l’on prendra toujours, de bonne foi, comme le meilleur indice du progrès, le degré même de cette instabilité, de cette inquiétude qui devient morbide ? J’espère bien que non. Mais à quoi bon anticiper sur l’avenir ? Constatons plutôt la transformation lente et profonde que l’avènement de l’or a fait subir non seulement à la notion de la liberté, mais à celle du droit. Cournot a très bien montré, dans un chapitre de son \emph{Traité sur l’enchaînement des idées fondamentales}, la métamorphose, ou pour mieux dire la métempsycose de la notion de droit quand elle passe de la bouche du juriste, surtout du juriste ancien, dans celle de l’économiste. L’ère proprement économique s’ouvre pour les sociétés quand le sentiment du droit, tel qu’il remplissait l’âme d’un \emph{paterfamilias} ou d’un féodal, est atteint dans ses sources et tend à être remplacé par une conception toute rationnelle, qui a infiniment moins de prise sur le cœur. Or, ce qui caractérise le plus nettement le droit tel que l’entend le juriste, c’est d’être avant tout un \emph{droit terrien :} et je ne veux pas parler uniquement du régime juridique des biens, mais du régime des personnes et même du droit pénal. Ce qui donne le ton à toutes ces parties du droit ancien, c’est le rapport de l’homme à la terre, à une terre qui lui appartient, à lui individuellement, ou à lui et aux siens collectivement, à une terre qui lui est chère par-dessus tout, où dorment tous ses souvenirs, où toutes ses espérances germent et croissent. Tous les droits anciens reposent sur une coutume, qui suppose l’attachement héréditaire de plusieurs générations à un même sol. Tous les droits anciens ont trait à la dispute d’un même sol par les hommes d’une même race, ou au travail de ce sol, ou à l’héritage de ce sol, ou à la hiérarchie des personnes que paraît exiger la culture de ce sol ou sa défense, ou à l’expiation des crimes commis contre les dieux protecteurs de ce sol ou contre les détenteurs des fruits de ce sol, le vol et le sacrilège étant réputés plus criminels  \phantomsection
\label{v1p319}que l’assassinat. Mais, autant le juriste ancien, et même moderne encore, est préoccupé des relations, des adaptations de l’homme à sa terre, autant l’économiste est dominé par la préoccupation majeure des rapports de l’individu avec l’argent : prix, salaires, profits, rente, capital, voilà les mots autour desquels pivotent toutes ses spéculations. Si le droit ancien, le droit juridique pour ainsi dire, est un droit terrien, le droit nouveau est un \emph{droit monétaire.}\par
L’avantage, l’immense avantage de cette transformation, est d’avoir contribué à battre en brèche l’exclusivisme antique de la cité, et plus tard du fief, ce protectionnisme féroce qui limitait aux remparts de la ville ou du bourg l’étroit jardin moral réservé aux obligations senties comme telles, à l’échange des devoirs et des affections. Cet élargissement graduel du domaine moral, qui peu à peu tend à comprendre le globe entier, est le bénéfice le plus net de la civilisation, et c’est au règne de la monnaie, sous l’impulsion de la sympathie imitative sans cesse agissante, que ce progrès inappréciable s’est accompli. Mais, à notre avis, il serait acheté trop cher s’il devait avoir pour effet de détacher tout à fait l’homme de la terre, de \emph{sa} terre, et de reléguer aux catacombes du cœur humain les vieux sentiments enracinés dont il a vécu jusqu’ici. Car l’union de l’homme à la terre, de même que l’union de l’homme à la femme, à laquelle on peut la comparer sans manquer de respect à l’amour, est, de toutes les harmonies naturelles, la plus pleine et la plus profonde, et l’on ne peut lui reprocher, comme à l’amour, que d’être close en soi, égoïsme à deux, sociabilité incomplète. Toutefois, ne peut-on pas l’\emph{ouvrir} sans la briser ? Et n’est-ce pas à résoudre ce problème que travaillera l’avenir ? La ploutocratie régnante n’aura peut-être servi qu’à préparer et amener l’ère d’une vaste démocratie rurale où, par une meilleure et plus égale répartition des terres, non par la nationalisation ou l’internationalisation chimérique du sol, les vœux les plus chers de l’homme seront comblés ; où le paysan ne  \phantomsection
\label{v1p320}sera plus le seul amant de la terre, mais encore le lettré, le savant, l’intellectuel, qui trouveront plus de charme à tailler leur vigne qu’à mener la vie de bureau ; où l’on verra se vider les grandes villes, et les campagnes se repeupler, parce qu’on s’apercevra que la concentration urbaine de la population, due à la poursuite fiévreuse de l’or, a perdu sa raison d’être quand l’insécurité ambiante ne force plus les hommes à quitter leurs champs... Déjà quelque anticipation du tableau de cette félicité ne nous est-elle pas donnée par le spectacle des petits pays, la Suisse, la Norvège, où la paix sociale serait complète, si le voisinage de nos troubles ne l’altérait souvent ? On utilisera toujours alors la monnaie, on reconnaîtra ses services, on les étendra encore ; mais on ne sera plus dupe de sa magie propre, de ses illusions spéciales dont notre âge est abusé. L’or a un faux air libérateur, car il substitue aux servitudes visibles et précises que crée la terre des servitudes inaperçues, invisibles, insaisissables, infinies, dissimulées par leur complication même et leur apparence de réciprocité, si souvent trompeuse. [{\corr L’or}] a un faux air égalitaire ; il n’est que niveleur, et les inégalités qu’il creuse, inaperçues aussi, sont tout autrement profondes et injustifiables que les inégalités de nature terrienne, qui frappent les yeux.\par
La terre et l’or sont les deux grandes attractions, à la fois opposées et complémentaires, du désir humain. Il oscille entre les deux, sollicité par les deux ; et ce dilemme, parfois douloureux, parfois sanglant, qui s’impose à lui, consiste, au fond, à opter entre les deux faces des choses, le côté des diversités, des originalités sans mesure et sans prix, du pittoresque local et national, du charme et du génie propre à une institution, à un peuple, à un pays, et le côté des similitudes, des uniformités, de l’utile banal, vénal, international, sans caractère. Il y a des diversités nécessaires comme il y a des libertés nécessaires, auxquelles elles sont liées ; et l’or n’en tient nul compte et tout âge monétaire les  \phantomsection
\label{v1p321}méconnaît. L’or est l’ennemi sourd de l’idée de patrie. S’il a paru longtemps l’aider et la favoriser, si, en permettant aux diverses provinces d’un même empire de commercer ensemble, il a dilaté le patriotisme étroit des premières cités jusqu’au patriotisme des grands États modernes, il n’a élargi ce sentiment que pour l’ébrécher, et n’est-il pas manifeste, à présent, qu’il tend partout à le démolir, comme un vieux rempart croulant et incommode pour la circulation générale ? L’amour de la patrie, de la \emph{tellus patria}, qui a sa magie ensorcelante, sa valeur transcendante, sacrée et sans prix, avec tout ce qui émane d’elle, est le grand obstacle aux suprêmes entreprises de l’or, qui osent enfin s’avouer. La lutte est ouverte ; il s’agit de savoir si le règne de l’or, père du cosmopolitisme, remportera ce triomphe final, de déraciner les hommes de leur patrie.
\subsubsection[{I.6.i. Loi des transformations monétaires : diminution du nombre des monnaies en cours et extension du domaine des survivantes. Attachement des modernes mêmes aux monnaies nationales.}]{I.6.i. Loi des transformations monétaires : diminution du nombre des monnaies en cours et extension du domaine des survivantes. Attachement des modernes mêmes aux monnaies nationales.}
\noindent Il ne réussira, je l’espère bien, qu’à désarmer les patriotismes, à les élargir encore, à les adoucir sans les affaiblir. Aussi n’ai-je nul regret à voir la monnaie poursuivre le cours de son évolution dont il me reste à indiquer la loi principale. Cette loi, qui est conforme à celle de la langue, de la religion, de toutes les institutions sociales, c’est le passage continuel d’une ère de monnaies multiples, ayant chacune une sphère rétrécie de circulation, à une sphère de monnaies moins nombreuses, mais séparément plus répandues ; en d’autres termes, c’est la diminution graduelle du nombre des monnaies en cours, mais l’accroissement graduel du domaine propre aux monnaies survivantes. C’est ainsi que les langues deviennent de moins en moins nombreuses, mais de plus en plus répandues.\par
Sans avoir la même importance que l’unification des langues dans une région donnée, l’unification des monnaies  \phantomsection
\label{v1p322} — ainsi que celle des poids et mesures — y concourt notablement à aider l’action [{\corr inter-spirituelle}]. Et, quand elle se borne à supprimer ainsi des entraves qui s’opposaient à la formation ou à la renaissance d’un patriotisme national, — comme c’est le cas pour l’Allemagne au cours de ce siècle — les hommes s’y prêtent assez aisément. Mais, quand elle se heurte aux frontières des nations qu’elle entreprend de franchir, ce n’est pas sans de vives résistances qu’elle parvient parfois à s’opérer. L’attachement des hommes, même les plus novateurs, à leurs vieilles monnaies, à leurs vieilles mesures, si arbitraires que soient ces unités, est très digne de remarque. Car il nous révèle, par une analogie \emph{a fortiori}, quelles ténacités humaines il y aurait à vaincre si l’on entreprenait de combattre, décidément, l’attachement des individus à des habitudes tout autrement chères, tout autrement enracinées. Aux États-Unis même, — le croirait-on ? — on s’est engoué, il y a quelques années, du « dollar des ancêtres » et on y est revenu\footnote{ \noindent \emph{La Monnaie}, par Arnauné (Alcan, 4898).
 }. Voilà qui est plus fort peut-être : malgré notre entichement du système décimal, nous avons été contraints de frapper, dans l’Indo-Chine, après plusieurs tentatives malheureuses pour y introduire notre système monétaire, des piastres et même des sapèques\footnote{ \noindent Les Athéniens avaient beau être novateurs et même révolutionnaires, par tempérament, ils ont senti la nécessité, sous bien des rapports essentiels, et notamment en tout ce qui touchait au culte, d’être conservateurs à outrance. Ils l’ont été même en ce qui concerne leurs monnaies. Non seulement le type n’en a jamais été changé, la chouette y figurant toujours, mais encore, depuis Pisistrate au moins, qui ajouta au côté droit la tête de Minerve, « le style même et la manière de représenter ce double type s’immobilisent ; jusqu’au temps d’Alexandre, l’art athénien (monétaire) reste stationnaire ». Il est curieux de voir les Athéniens, sur ce point et sur bien d’autres, aussi traditionnalistes que les Égyptiens, aussi respectueux des formes hiératiques. Quand une ville antique, par suite d’une révolution, d’une conquête, se voit amenée a changer le type de ses monnaies, l’\emph{ancien type persiste dans le nouveau sous forme réduite}, à titre d’organe rudimentaire en quelque sorte (Lenormant, p. 108). Toutes les monnaies antiques témoignent, par leurs figures et leurs emblèmes, du caractère profondément religieux de la vie antique. Ce que dit Lenormant à propos des \emph{médaillons contorniates} (p. 189), à savoir que l’effigie d’Alexandre le Grand figurant sur ces pièces destinées aux cochers du cirque, était regardée comme un \emph{porte-bonheur}, pourrait être généralisé. Les anciens, si superstitieux, attachaient à toute chose un sens de bon ou de mauvais augure ; et il est infiniment probable que leurs monnaies mêmes étaient à leurs yeux des espèces de \emph{talismans} quand elles représentaient leurs divinités tutélaires. Toute pièce de monnaie était \emph{médaille} pour eux, dans l’acception pieuse du mot \emph{médaille}, qui n’est pas pour rien synonyme d’amulette pour les femmes et les enfants.
 }.\par
 \phantomsection
\label{v1p323}La pièce de monnaie tient à la fois du bijou, de l’amulette et de la médaille. Elle atteste ce besoin esthétique qui se mêle aux préoccupations les plus utilitaires. On aurait pu se contenter partout, comme en Indo-Chine, de marquer sur les pièces le poids du métal. Cela ne s’est fait que par exception depuis l’invention de la monnaie frappée, et là où la circulation est peu active. Dès qu’une nation s’enrichit et que son commerce se développe, ses pièces de monnaie deviennent des objets d’art. Mais ce caractère artistique n’est bien senti que lorsque les pièces vieillissent, se démodent, se raréfient. Les vieilles monnaies ont pour les collectionneurs un indicible charme, inexplicable. Ce charme agit même sur les profanes. L’un des inconvénients les plus regrettables — inévitable d’ailleurs — de la refonte des monnaies et de leur unification, est la disparition graduelle des vieilles pièces, la perte de l’attrait attaché à leur vétusté et à leur diversité caractéristique. Je dirais que cette perte est compensée par le progrès de la frappe, si, par malheur, la frappe ne devenait pas plus froide, plus régulièrement et mécaniquement uniforme. De vieilles pièces, gauchement frappées, avec des bavures, avaient un air d’œuvres faites à la main. — Qui sait si le plaisir de collectionner des pièces anciennes, différentes de type, n’entrait pas pour quelque chose — pour bien peu, je le crains, — dans la passion acharnée du thésauriseur d’autrefois ? Harpagon, n’en doutons pas, était un numismate sans le savoir. Il collectionnait des médailles en croyant n’empiler que des livres tournois ou des livres parisis. Assurément, de nos jours, il aurait un plaisir moindre à entasser des pièces de  \phantomsection
\label{v1p324}20 francs toutes à l’effigie de la République française ou de Napoléon III.\par
Mais, si belles et si chères qu’elles soient, les monnaies archaïques, sont irrévocablement condamnées à périr comme les vieux patois si savoureux que gazouillent les dernières paysannes au fond des bois, au cœur des monts, dans les pays délicieusement arriérés. Si les monnaies nationales résistent encore à l’invasion du numéraire étranger, résisteront-elles toujours ? Les langues nationales sont bien plus résistantes et plus tenaces, et cependant elles cèdent quelquefois à l’inondation d’un idiome conquérant. Une monnaie nationale, comme une langue nationale, quand elle a un domaine suffisamment étendu pour répondre à tous les besoins d’échange économique ou d’échange mental, est propre en même temps à faciliter les relations entre les nationaux et à les entraver avec l’étranger. En tant qu’elle facilite les premières, elle obéit à cette grande loi d’amplification croissante, qui, avons-nous dit, domine le monde social ; mais, en tant qu’elle entrave la seconde, n’est-elle pas contraire à cette loi, et comme telle, ne doit-elle pas être fatalement emportée un jour ou l’autre par le torrent fatal ? N’est-il pas essentiel à l’idée de monnaie et à l’idée de langue, que la langue rende tout exprimable, et que la monnaie rende tout échangeable ; et cela ne suppose-t-il pas, finalement, une monnaie, une langue, sinon unique du moins universelle ?\par
Toujours l’anxieuse question de savoir si les nations seront ou ne seront pas ! Elles seront ou elles ne seront pas, cela dépend de la manière dont s’opérera la pacification finale, car il est inévitable qu’elle s’opère un jour. Si elle s’accomplit par la voie fédérative, les patries subsisteront, et l’international ne triomphera qu’à la condition de respecter toutes choses nationales, même les monnaies, sauf à leur superposer une monnaie internationale. Si elle s’accomplit par la voie impériale, par la voie du passé, toute nationalité \phantomsection
\label{v1p325} périra, y compris celle du vainqueur, « enseveli dans son triomphe ». Et le même niveau linguistique, politique, scientifique, monétaire, passera sur le globe aplati. Quant à prédire lequel de ces deux dénouements de l’histoire a le plus de chance de se réaliser, je ne m’y aviserai pas. L’évolution de l’humanité est comme un fleuve à delta, elle peut avoir plusieurs embouchures.
\subsubsection[{I.6.j. Petits problèmes relalifs à la monnaie.}]{I.6.j. Petits problèmes relalifs à la monnaie.}
\noindent — Au sujet de la monnaie se rattachent plusieurs problèmes secondaires que les économistes ont agités, des lois qu’ils ont formulées. Indiquons quelques-unes de ces questions. On connaît la loi de Gresham, d’après laquelle « la mauvaise monnaie chasse la bonne ». Elle n’est pas sans exception, même de nos jours ; aux États-Unis, la monnaie d’argent, quoique dépréciée, n’a nullement chassé l’or, dont la masse a été en augmentant dans la circulation du pays. Dans le passé, il ne semble pas que cette loi jette un grand jour sur les phénomènes relatifs au concours des monnaies royales et des monnaies seigneuriales, toutes pareillement altérées. Apparemment on ne croira pas que, si les monnaies des rois ont fini par se substituer à celles des seigneurs, c’était parce que les monarques étaient plus faux-monnayeurs que leurs grands vassaux. Il en est de ces fausses monnaies monarchiques dont les peuples se sont contentés pendant des siècles, comme, en général, de tous les mensonges conventionnels dont le monde vit, et qui ont toujours eu beau jeu contre la vérité quand ils ont été mis en circulation par des autorités respectées. Ce chapitre de l’altération des monnaies serait une curieuse page à joindre à l’histoire des transformations du mensonge. Les procédés changent, le fond reste. On n’altère plus le poids des monnaies ni leur titre, mais le papier à cours forcé les a remplacés \phantomsection
\label{v1p326} avec avantage. « Le papier à cours forcé, dit M. Arnauné, est la fausse monnaie des gouvernements modernes. »\par
On s’est demandé quelles sont les causes qui, dans une société donnée, à un moment donné, font varier la quantité de numéraire nécessaire pour tous les payements. Cette question a été étudiée par un éminent géomètre, Joseph Bertrand, avec une précision remarquable\footnote{ \noindent \emph{Revue des Deux-Mondes}, 1\textsuperscript{er} sept. 1881.
 }. Supposant une nation insulaire et fermée, où l’habitude s’est établie d’effectuer tous les payements le premier de chaque mois, il montre aisément que, puisque chaque caisse devra s’approvisionner en conséquence, la quantité de numéraire indispensable sera au moins égale au douzième du chiffre total des payements annuels. Mais, si toutes les caisses, quoique continuant à payer un jour par mois, ont des jours de paye différents, la quantité de numéraire exigée sera beaucoup moindre ; elle serait considérablement amoindrie si l’usage s’établissait de payer tous les jours ; autrement dit, l’accélération du mouvement de la monnaie, le nombre croissant de ses changements de main, équivaut à l’accroissement de sa quantité. — Tout dépend donc des usages, lesquels dépendent de certaines initiatives imitées, dont l’imitation dépend du crédit, c’est-à-dire de la confiance plus ou moins grande que les hommes en rapport d’affaires ont les uns dans les autres.\par
Un autre problème a été étudié par le savant déjà cité : celui de savoir, s’il est vrai que la quantité de numéraire venant tout à coup à doubler dans une nation, la conséquence serait une baisse de moitié sur tous les prix. Bertrand montre très bien que ce résultat n’est ni certain, ni probable, et que, plus vraisemblablement, cette infusion monétaire agira surtout en surexcitant la production. « La prévoyance, accrue par le bien-être, augmentera la réserve de chacun. » Beaucoup d’économistes sont du même avis, et  \phantomsection
\label{v1p327}nous avons vu plus haut que les phénomènes économiques observés à la suite des découvertes de gisements aurifères ou argentifères leur donnent pleinement raison. L’expérience et le calcul à priori sont ici d’accord. C’est une chose remarquable que cette surexcitation de l’activité productrice par le simple fait de la découverte d’un filon aurifère. Si, au lieu de découvrir une mine d’or, on avait découvert un gisement de guano d’égale valeur, suffisant pour fumer toutes les terres et augmenter prodigieusement toutes les récoltes, est-il bien sûr que la nation eût été, en fin de compte, plus enrichie qu’elle ne l’est par l’extraction des lingots ?\par
Par cette considération l’on répond implicitement à un problème plus général que les économistes ont agité, celui de savoir si la monnaie, en tant que monnaie, est une richesse par elle-même, ou n’est, comme telle, qu’un substitut de richesses. Alors même qu’elle consisterait en métaux impropres à tout autre usage que celui d’accomplir leur fonction monétaire, elle mériterait d’être \emph{ajoutée} au nombre des richesses véritables, c’est-à-dire des objets \emph{jugés} de nature à satisfaire les \emph{désirs.} Comment une simple entité pourrait-elle jouer le rôle fécondant que tout le monde est forcé de lui reconnaître ? N’est-elle pas, elle aussi, l’objet de désirs distincts, et n’inspire-t-elle pas une foi profonde en son efficacité à satisfaire une infinité d’autres désirs ? — Mais ce que nous disons là de la monnaie métallique, il faut le dire aussi bien, et pour la même raison, de la monnaie fiduciaire, voire même de simples billets de commerce qui circulent un temps de main en main, monnaie passagère et restreinte. « S’il était vrai, dit très bien Macleod, qu’un billet à ordre fût dépourvu de valeur jusqu’à son payement, il s’ensuivrait que l’argent non plus n’a pas de valeur jusqu’à ce qu’il ait servi à acheter quelque chose, et qu’il n’a pas de valeur distincte de celle des marchandises. » A plus forte raison, la remarque est-elle applicable aux billets de Banque.\par
 \phantomsection
\label{v1p328}Il en est des billets de banque comme des mots. Les mots commencent par avoir l’air d’être un simple signe et un simple substitut d’images, qui, elles-mêmes, passent pour des équivalents des sensations qu’elles ressuscitent vaguement et illusoirement en nous. Mais il n’en est pas moins vrai qu’on se tromperait grandement si l’on ne voyait de réel, au fond des idées que les images et au fond des images que les sensations, méconnaissant ainsi le fait évident que ce sont là autant d’éléments psychologiques distincts dont l’esprit s’enrichit à chaque pas qu’il fait. Le mot, l’idée, apparaît bientôt comme une chose autonome que l’on ne songe plus à échanger contre les images dont elle est la combinaison. Le billet de banque commence aussi par paraître un simple substitut d’espèces métalliques, contre lesquelles à chaque instant on sait ou on croit qu’il peut être échangé ; mais il ne tarde pas à révéler qu’il a sa valeur propre, indépendante de la leur, de même que les espèces métalliques sont si souvent et si universellement désirées, en partie, pour elles-mêmes. Le même travail de logique mentale et d’action inter-mentale qui a provoqué l’évolution intellectuelle d’où le mot est sorti, le mot de plus en plus abstrait, incarnant une idée de plus en plus générale, a produit l’évolution économique d’où est issue la monnaie de papier.\par
C’est donc bien à tort que les économistes préoccupés de bâtir leur science sur des fondements tout objectifs ont une prédilection pour le sujet de la monnaie et des finances, où leur rêve, de premier abord, semble se réaliser. En fait, il n’est rien de tel que les phénomènes financiers pour mettre en relief ce qu’il y a d’essentiellement subjectif dans les choses économiques. Par exemple, si les causes de la variation des valeurs de Bourse étaient objectives, comment expliquer cet effondrement ou cette dépression de tous les cours qui a lieu quand une catastrophe ou un événement fâcheux quelconque vient à atteindre un seul de ces  \phantomsection
\label{v1p329}titres, mais un de ceux qui donnent le ton aux autres, une rente d’État notamment ? Il semble que la baisse devrait être strictement limitée à ce titre, et que les autres, loin de la suivre dans sa chute, devraient, au contraire, se trouver rehaussés par comparaison. Mais il n’en est rien. Pourquoi ? Parce qu’il s’agit là d’un phénomène de psychologie et surtout de psychologie inter-spirituelle. Quand un événement heureux nous arrive, dans une circonstance particulière, nous voyons tout en rose aussitôt, nous sommes enclins à espérer, à avoir confiance en tout ; quand un malheur particulier nous frappe, un découragement général tend à nous accabler. Et cette disposition optimiste ou pessimiste, nous la communiquons sans le vouloir à notre entourage, elle émane de nous pour impressionner nos voisins, nos collègues, fussent-ils nos rivaux, qui, sans avoir aucunement les mêmes raisons que nous de se décourager ou d’avoir confiance, reflètent notre couleur psychologique à leur insu. Les palais de la Bourse sont ainsi, sans qu’il y paraisse, des laboratoires, continuellement actifs, de psychologie collective.\par
Il me resterait, enfin, à parler de la \emph{circulation} de la monnaie et des rapports entre ce \emph{cycle monétaire} et le \emph{cycle des besoins} ou celui \emph{des travaux}, dont il a été question ci-dessus. Mais cette question sera plus utilement discutée à propos du \emph{capital}, sujet qui se rattache intimement à celui de la monnaie, et dont nous allons nous occuper.
 \phantomsection
\label{v1p330}\subsection[{I.7. Le capital}]{I.7. Le capital}\phantomsection
\label{l1ch7}
\subsubsection[{I.7.a. Définitions diverses du capital. Son origine. Son caractère essentiel.}]{I.7.a. Définitions diverses du capital. Son origine. Son caractère essentiel.}
\noindent L’idée du capital, très distincte de l’idée de monnaie à l’origine, — car il a pu exister des cheptels avant que nulle monnaie n’existât — tend à se confondre avec elle en se développant. Le moment est donc venu d’étudier cette notion complexe et confuse, qui appartient essentiellement à la répétition économique, puisqu’il est un seul point sur lequel on s’accorde dans les définitions multiples du capital, c’est qu’il sert à la reproduction des richesses.\par
Si nous prenons ce mot dans son acception vulgaire, celle où capital s’oppose à revenu grâce au prêt à intérêt, nous n’aurons pas de peine, ce me semble, à en découvrir l’origine vraisemblable. La propriété mobilière, si l’on entend par là la possession de grands trésors en monnaie ou en créances, je ne dis pas en troupeaux nomades, n’a pu naître ou grandir qu’après la propriété territoriale. Or, celle-ci suppose une distinction universelle et profonde entre le sol et ses fruits, entre la possession du sol et la possession de ses fruits. Naturellement, donc, quand les fortunes en trésors ont commencé à prendre rang à côté des fortunes en terres, cette propriété nouvelle qui surgissait a été conçue comme devant, à l’image de l’autre, impliquer une dualité toute pareille. De là la distinction du capital et du revenu, calquée sur celle du sol et des fruits.\par
M’objectera-t-on l’étymologie de \emph{capital}, la même que celle de \emph{cheptel}, d’où il semblerait résulter que, le premier \phantomsection
\label{v1p331} capital ayant dû être un troupeau, l’idée de capital serait antérieure et non postérieure au développement de la phase agricole et terrienne ? Je suis aussi de cet avis, et j’en dirai plus loin les raisons ; mais il s’agit de l’idée vulgaire et actuelle du capital, et, sans le moindre doute possible, jamais la possession de grands troupeaux, si elle n’avait été accompagnée ou suivie de la propriété de terres cultivées, n’aurait pu suggérer cette idée à laquelle il semble essentiel, vulgairement, que l’idée de revenu s’oppose. Jamais troupeau ne s’est distingué de son croît de cette manière, de la même manière que le sol cultivé se distingue de ses récoltes. Même considéré dans son ensemble, et à part des têtes qui le composaient, le troupeau, à l’époque pastorale, ne pouvait être regardé comme quelque chose d’impérissable, d’inaltérable en soi : combien de troupeaux périssaient souvent tout entiers au cours d’une épizootie !\par
C’est donc la propriété immobilière, encore une fois, qui, en frappant la propriété mobilière à son effigie, l’a fait imaginer comme fructifère aussi, et, par là, transformée en capital au sens ordinaire du mot ! Ce sens n’a pas satisfait les économistes, je le conçois ; et il faut les louer d’avoir cherché à l’idée qui nous occupe un fondement plus rationnel. Le malheur est que tous leurs efforts pour préciser cette notion et la justifier en évitant de recourir à l’analogie qui l’a suggérée — car la puissance de l’analogie n’est pas moins créatrice au point de vue économique que linguistique — ont été bien mal récompensés jusqu’ici. Le capital, pour l’un, c’est tout l’outillage humain ; pour un autre, c’est l’accumulation des produits épargnés et destinés à une production ultérieure ; pour Stuart Mill, c’est la somme soi-disant mise à part pour payer les ouvriers, le fameux « fonds des salaires » ; pour M. Bohm-Bowerck, c’est un moyen détourné de produire ce qu’on pourrait produire directement, mais avec bien plus de temps et de difficultés ;  \phantomsection
\label{v1p332}c’est surtout du temps gagné. On a distingué les capitaux fixes et les capitaux circulants, les capitaux à revenus et les capitaux de jouissance.\par
A titre de curiosité, je citerai l’idée que Macleod se fait du capital. D’abord, il étend fort loin le sens de ce mot. Il désigne par là \emph{toute chose qui peut procurer un profit ou un revenu, qui peut faire gagner de l’argent.} Les marchandises d’un négociant sont pour lui un capital ; des \emph{terrains} qu’on achète avec l’intention de les revendre, sont des capitaux. « Le crédit commercial est un capital commercial. » Ceci posé, il divise les capitaux en deux grandes catégories, qu’il croit devoir désigner comme deux quantités, l’une \emph{positive}, l’autre \emph{négative}, à la manière des géomètres. Il considère comme \emph{positifs} les capitaux qui consistent en profits passés (terres, maisons, etc.) et comme \emph{négatifs} ceux qui se fondent sur l’espérance de profits futurs (crédit). Je ne dirai rien de cet essai d’application des mathématiques à l’économie politique, si ce n’est que, comme beaucoup de tentatives pareilles, il est malheureux et infécond. Entre le \emph{passé} et l’\emph{avenir}, l’auteur établit une opposition factice, dont l’état zéro, intermédiaire, est le \emph{présent.} Je sais bien que les géomètres ont recours souvent à cette convention, mais ici elle est arbitraire et inexacte ; car ce n’est jamais en tant que passés, c’est toujours en tant qu’\emph{actuels} que les profits antérieurs ou présents ont une valeur. Il n’y a de véritable opposition économique, exprimable légitimement par les signes mathématiques + et -, que dans le cas où l’on oppose des créances à des dettes, des prêts à des emprunts, des productions à des destructions. Mais l’antithèse de Macleod ne repose sur rien de solide.\par
A travers toutes les divergences des économistes relativement à la notion du capital, l’idée générale qu’ils s’en font peut se résumer en cette définition : le capital est cette partie des \emph{produits} anciens qui est nécessaire ou utile aux  \phantomsection
\label{v1p333}\emph{services} nouveaux (au travail) pour créer de nouveaux produits, soit\footnote{ \noindent Semblables, comme lorsque, grâce à la semence du blé de l’an dernier, le cultivateur, par son travail actuel, fait lever une nouvelle moisson. Différents, comme lorsque, avec l’aide de sa varlope, de ses planches, de son établi, un ébéniste fabrique une commode ou un secrétaire, ou que, avec une lime, un serrurier fabrique une clef.
 } semblables à ces produits anciens, soit différents de ces produits, mais toujours semblables à d’autres produits anciens. Par là, on voit que quelque chose de l’idée vulgaire du capital est retenu dans ses notions les plus subtiles, à savoir qu’il est essentiellement reproducteur. Mais, par cette définition, on peut se rendre compte de ce qu’il y a d’indéterminé et d’indistinct dans l’idée du capital ainsi compris. Car, d’une part, le cas où le produit ancien — la semence — sert à créer un produit nouveau qui lui ressemble, méritait d’être mis en relief, — ce cas est celui de la production agricole — et de n’être pas confondu avec celui où le produit nouveau, comme il arrive toujours dans la production industrielle, est sans similitude aucune avec le produit ancien (la machine ou l’outil) qui sert à le fabriquer. D’autre part, où s’arrête la portion des produits anciens qui est \emph{utile} à la création des produits nouveaux, semblables où différents ? Il y a mille degrés possibles d’utilités. La partie des produits anciens qui est \emph{absolument nécessaire} à la production des produits nouveaux peut être seule déterminée avec rigueur, et c’est celle qu’il importe de spécifier.\par
Si l’on y réfléchit, on verra que cette partie indispensable consiste uniquement dans l’existence et la connaissance des secrets de métier, des méthodes de culture, des procédés employés pour l’extraction des matières premières et pour la fabrication des outils ou machines propres à fabriquer les produits nouveaux. Sans doute, il faut bien aussi, pour que cette reproduction soit possible, qu’il existe au dehors des matières premières, des minerais de fer, ou, s’il s’agit de reproduction agricole, des plantes et des animaux.  \phantomsection
\label{v1p334}Car on aurait beau connaître à merveille les méthodes de culture du maïs, s’il n’y avait plus de maïs dans le monde, il serait impossible d’en reproduire. Mais il n’est pas nécessaire, pour qu’une récolte de maïs puisse être plus tard faite par nous, que nous soyons nous-mêmes propriétaires de graines de maïs ; il suffit qu’il en existe quelque part, en la possession de quelqu’un qui pourra nous la transmettre. En somme, la seule chose indispensable en toute rigueur à la production d’une locomotive nouvelle, c’est la connaissance détaillée des pièces d’une locomotive, de la manière de les fabriquer et d’abord d’extraire les matériaux dont elles sont faites. Ce faisceau d’idées, dont chacune est une invention grande ou petite, due à un inventeur connu ou inconnu, ce faisceau d’inventions rassemblé dans un cerveau : voilà la seule portion des produits anciens — car c’est bien là un produit mental, le fruit d’un enseignement scolaire — qui soit requise de toute nécessité pour la construction d’une locomotive. On en dirait autant de la fabrication d’un article quelconque.\par
Certes, l’individu qui, réduit à ce legs intellectuel du passé, n’aurait ni semences, ni approvisionnements, ni outils, serait dans de déplorables conditions pour faire œuvre agricole ou industrielle. Mais il ne serait pas dans l’impossibilité de produire un peu plus tôt ou un peu plus tard, — tandis que, si, pourvu des semences ou des matériaux les plus abondants, amassés et accumulés par l’épargne, et de l’outillage le plus perfectionné, il est en même temps ignorant des secrets de l’industrie qu’il prétend diriger, ou des méthodes de la culture à laquelle il se livre, il sera frappé d’impuissance productrice en dépit de tout son prétendu capital. Supposez que tous les ingénieurs américains, par un phénomène d’amnésie épidémique, viennent à perdre subitement la mémoire de toutes leurs connaissances techniques, les États-Unis auront beau être réputés la nation la plus riche et la plus capitaliste du monde, toute production  \phantomsection
\label{v1p335}s’y arrêtera. Mais les ravages d’une guerre, fût-elle dix fois plus désastreuse encore que la guerre de Sécession et eût-elle détruit tout l’outillage et tous les approvisionnements, n’empêcheraient pas l’Amérique, si elle restait éclairée et instruite, de reconquérir sa prospérité. En quelques années, comme l’ont remarqué avec raison Stuart Mill et Henry Georges, les maux économiques de ces grandes catastrophes sont réparés, avec une merveilleuse facilité, dont il n’y a pas lieu de s’étonner néanmoins, d’après la remarque qui précède.\par
L’histoire de l’industrie aux États-Unis fournit une remarquable illustration de cette vérité. M. Wright, dans son livre intéressant à ce sujet\footnote{ \noindent \emph{L’évolution industrielle aux États-Unis}, trad. fr., 1901.
 }, nous raconte comment la grande fabrication fut inaugurée dans sa patrie. Ce fut en 1790, par Samuel Slater, que le président Jackson appelait le père de l’industrie américaine. Slater était Anglais, il était familier depuis son enfance avec les machines anglaises, et, ayant eu connaissance des efforts, jusqu’alors impuissants, que faisaient les États-Unis pour obtenir des machines à filer le coton, il alla en Amérique. Mais il n’y apportait ni machine, ni dessin quelconque figurant des modèles ou des plans de machines : les lois anglaises à cette époque interdisaient avec la dernière sévérité le transport à l’étranger de toute indication relative à des secrets dont la Grande-Bretagne prétendait conserver le monopole (c’était sa manière d’entendre à cette époque le laissez-faire et le \emph{laissez-passer}). Donc, tout ce qu’apportait Slater avec lui en Amérique, quand il s’embarqua le 13 septembre 1789, consistait, dit M. Wright, « dans une cargaison, il est vrai, infiniment précieuse, mais une cargaison contenue tout entière dans son cerveau ». Cette \emph{cargaison} invisible et impondérable a été tout le capital d’où la grande industrie américaine est sortie...
\subsubsection[{I.7.b. Le capital essentiel et le capital auxiliaire, le germe et le cotylédon. Leurs manières très distinctes de s’accroître ou de périr.}]{I.7.b. Le capital essentiel et le capital auxiliaire, le \emph{germe} et le \emph{cotylédon}. Leurs manières très distinctes de s’accroître ou de périr.}
 \phantomsection
\label{v1p336}\noindent Dans la notion du capital, à mon avis, il y a donc deux choses à distinguer : 1\textsuperscript{o} le capital essentiel, nécessaire : c’est l’ensemble des inventions régnantes, sources premières de toute richesse actuelle ; 2\textsuperscript{o} le capital auxiliaire, plus ou moins utile : c’est la part des produits, nés de ces inventions, qui sert, moyennant des services nouveaux, à créer d’autres produits.\par
Ces deux éléments diffèrent à peu près comme, dans la graine d’une plante, le germe diffère de ces petites provisions d’aliment qui l’enveloppent et qu’on appelle des \emph{cotylédons}. Les cotylédons ne sont pas indispensables ; il y a des plantes qui se reproduisent sans cela ; ils sont très utiles seulement. La difficulté n’est pas de les remarquer, en ouvrant la graine, car ils sont relativement volumineux. Le germe, tout petit, se dissimule entre eux. Les économistes qui ont fait consister le capital uniquement dans l’épargne et l’accumulation de produits antérieurs, ressemblent à des botanistes qui regarderaient la graine comme constituée tout entière par les cotylédons.\par
Remarquons-le cependant. Si ce que je viens de dire est vrai de l’agriculture aussi bien que de l’industrie, c’est surtout vrai de l’industrie, ou c’est autrement vrai. La connaissance des secrets de métier, c’est-à-dire des \emph{inventions} industrielles, qui est le capital-germe de l’ingénieur, est l’équivalent de la connaissance des méthodes de culture, c’est-à-dire des \emph{découvertes} relatives aux propriétés des plantes et des animaux, à leur mode de croissance, qui est le capital-germe de l’agriculteur. Mais, à vrai dire, l’agriculteur ignore profondément le mode d’action des espèces végétales et animales qu’il fait travailler à son profit, dont il dirige simplement l’art mystérieux et héréditaire, tandis que l’ingénieur sait comment opèrent les forces physicochimiques \phantomsection
\label{v1p337} qu’il met en œuvre, et ne peut les diriger qu’en le sachant. Il convient donc de ne pas confondre les différences si profondes qui séparent l’agriculture de l’industrie, au point de vue non seulement du capital \emph{essentiel}, mais du capital \emph{auxiliaire}, dualité qui, du reste, est applicable à la première comme à la seconde.\par
Ces deux fractions du capital, si prodigieusement inégales par leur importance réelle, qui est en raison inverse de leur importance apparente, ont deux manières bien distinctes de s’accroître. La première s’accroît par une dépense de génie et d’ingéniosité ; la seconde, par le travail et l’épargne. L’une et l’autre peuvent être détruites, mais pas de la même façon. Le capital-germe, pour ainsi parler, ne peut être détruit que par la substitution à une invention ancienne d’une invention nouvelle jugée plus propre à satisfaire le même besoin, ou bien par la substitution à un besoin ancien d’un besoin nouveau. Il y a bien une autre cause d’anéantissement, l’\emph{oubli total}, mais il est inutile d’en parler, car elle ne se produit que comme conséquence de l’une ou de l’autre des deux précédentes. Quand, par exemple, le fusil s’est substitué à l’arc pour répondre au désir de chasser, l’invention de l’arc a été détruite, non comme idée, car elle a subsisté comme telle, mais en tant qu’adaptation de cette idée à un besoin régnant, ce qui est l’essentiel. Et, quand la conversion de l’empire romain au christianisme y a fait disparaître la passion des jeux du cirque, ou des cérémonies païennes hors des temples, pour les remplacer par l’attraction puissante de la vie monastique ou le désir de la prière en commun dans l’intérieur de temples nouveaux autour d’une chaire, on peut dire que le type du cirque et celui du temple antique ont cessé de vivre socialement. Quant au capital-cotylédon, — c’est-à-dire à l’\emph{outillage}, aux \emph{matériaux}, aux \emph{approvisionnements}, plus ou moins utiles, mais non absolument nécessaires à la réalisation de l’idée \emph{germinative}, il suffit, pour la détruire, d’un débordement  \phantomsection
\label{v1p338}de fleuve ou de la mer, d’une catastrophe physique quelconque, ou d’un pillage belliqueux, causes de destruction qui n’atteignent pas le capital-germe. En revanche, tout ce qui porte atteinte à celui-ci altère par contre-coup celui-là. Après que le monde romain eut été christianisé, non seulement le changement des mœurs, effet du changement des croyances, vint frapper à mort les belles inventions architecturales, désormais inanimées, du cirque, du temple grec, ajoutons des thermes, des aqueducs, etc., mais encore il eut pour effet d’anéantir presque entièrement la valeur des innombrables exemplaires de ces types dont les architectes avaient couvert le sol de l’Empire, et c’est comme si un vaste tremblement de terre eût, en abattant ces beaux édifices, englouti cet outillage du passé.\par
Le cotylédon, le capital-matériel, suit donc le sort du germe, du capital intellectuel, tandis que le germe ne suit pas le sort du cotylédon et survit fort souvent à celui-ci. Ce n’est pas seulement l’invention ancienne qui est tuée par une invention nouvelle et jugée préférable, c’est aussi bien tout l’outillage adapté à l’invention ancienne et l’installation \emph{ad hoc.} Si, par hypothèse, la direction des ballons était découverte pratiquement, c’est-à-dire par des procédés à la fois plus économiques et plus commodes que les autres moyens connus de locomotion, tous les travaux de terrassement des chemins de fer, tout leur matériel, gares, locomotives, rails, perdraient leur utilité, seraient des capitaux morts. — Au contraire, on aurait beau détruire toutes les voies ferrées de l’univers, si nul moyen de locomotion plus utile n’était inventé, ou si le besoin de voyage n’était pas extirpé du cœur humain, la connaissance détaillée de la locomotive par les ingénieurs conserverait toute sa valeur de capital, de \emph{richesse reproductrice.} — Cette remarque suffit pour montrer que des deux fractions du capital, la plus importante, et de beaucoup, est la plus minime, la partie subjective.\par
 \phantomsection
\label{v1p339}Observons à ce propos que la destruction d’un capital-invention est bien rarement complète et en tout cas n’est jamais que graduelle. Ce qu’il y a de graduel, de progressif, dans sa destruction partielle, mérite d’abord d’être remarqué. Quand le procédé Bessemer a été découvert, il n’a fait reculer que peu à peu les anciens procédés d’acération, et, si on regarde les tableaux statistiques qui peignent d’une part la progression régulière des kilogrammes d’acier produits par le procédé nouveau, d’autre part la décroissance non moins régulière de l’acier fabriqué à l’ancienne mode, on sera frappé de cette expansion lente et graduelle d’une innovation dont la supériorité a cependant été reconnue dès son apparition. Il me paraît difficile de ne pas voir dans cette continuité de progression croissante ou décroissante, révélée par tant de courbes statistiques, l’effet d’une contagion de proche en proche et la preuve, entre mille, de l’efficacité de l’exemple d’autrui, de l’exemple \emph{du prochain}, qui, mieux que tous les raisonnements, nous détermine. La démonstration la plus opérante sur les intéressés, c’est la vue de ceux qui ont déjà adopté avec succès la nouvelle invention. La navigation à vapeur, comme nombre et comme tonnage de navires, est loin de s’être développée aussi rapidement qu’on aurait pu le prédire d’après les avantages évidents qu’elle présentait, dès son origine au commencement du siècle, sur la navigation à voile, qui a reculé avec une extrême lenteur.\par
Quand, par exception, d’impérieuses nécessités interdisent d’attendre l’exemple des voisins pour se décider à son tour, parce que, précisément, on a un intérêt urgent à inventer mieux que ses voisins plutôt qu’à les imiter, on ne voit plus ces lentes progressions. C’est le spectacle que nous donnent les États européens dans cette étrange paix armée où les armements se renouvellent avant d’avoir servi, où, tous les ans, éclôt dans chaque état-major, quelque perfectionnement de canon, de fusil, de cuirassé, de tactique, qui, du soir au  \phantomsection
\label{v1p340}lendemain, oblige à refondre des batteries, à reconstruire des flottes, à refaire des plans de mobilisation. Dans cette effroyable guerre que, en pleine paix, les inventions militaires se font entre elles au sein d’un même État, et où il périt autant de capitaux que dans les plus grandes batailles rangées, le but visé, ou l’illusion nourrie, est d’avoir quelque secret ignoré de l’ennemi de demain, du voisin, que l’on copie sans doute, comme pis aller, si on le juge en possession de secrets plus infaillibles, mais que, avant tout, on cherche à terrifier par la menace d’armes invincibles et inimitables. Aussi, dès que le nouveau modèle de fusil, de canon, de poudre, a été expérimenté et jugé supérieur, ce n’est pas peu à peu, c’est brusquement qu’on l’applique partout, sans regarder à la dépense.\par
Mais, en fait d’inventions militaires mêmes, et à plus forte raison en fait d’inventions industrielles, il n’arrive presque jamais, comme je le disais plus haut, que la nouvelle venue détruise complètement son aînée. Après l’avoir chassée, soit peu à peu et par degré, soit tout à coup, de presque tout son domaine, elle ne parvient pas à la déloger de quelque retranchement suprême où elle se blottit. La navigation à voile subsiste et subsistera probablement toujours, mais réduite à un petit cabotage, à côté de la navigation à vapeur. La navigation à rames elle-même se survit dans le canotage et la pêche fluviale. Les inventions agricoles ont rétréci le champ de l’art pastoral mais ne l’ont pas exterminé. La chasse vit toujours, bien que la période chasseresse soit passée. Quand l’invention des maisons, en bois d’abord, puis en pierre, s’est substituée à l’invention antique de la tente, la tente a néanmoins continué à servir aux nomades, de moins en moins nombreux. Il existe encore, çà et là, dans les petites villes, des réverbères que les becs de gaz ou les lampes électriques n’ont point éteints. Et les derniers sauvages survivants se servent encore d’arcs et de flèches, en dépit de l’invention du fusil. — Par exception, l’art de  \phantomsection
\label{v1p341}faire éclater le silex et de le polir a disparu tout à fait devant l’invasion du bronze ; mais l’invasion du fer, plus tard, n’a refoulé celui-ci que jusqu’à un certain point. L’invention des allumettes chimiques a restreint l’emploi du briquet, elle ne l’a pas supprimé. Les chemins de fer n’ont pas supprimé non plus toutes les voitures publiques et il est peu probable qu’ils y parviennent un jour. Il en est des inventions comme des populations qui, expulsées par des races conquérantes des vastes plaines qu’elles occupaient, se réfugient dans le cœur des montagnes où elles se défendent indéfiniment.\par
Ce phénomène important a des causes évidentes : grâce à la diversité infinie des conditions d’existence, il n’est pas d’invention si simple et si grossière qui ne se trouve mieux adaptée à un certain genre d’emploi que l’invention la plus parfaite répondant au même besoin général. Ou bien, si tout emploi lui est enlevé, elle s’en crée le plus souvent un nouveau, et l’organe change de fonction. Le génie humain, comme le génie de la vie, tient beaucoup à ses idées et les utilise le plus qu’il peut.\par
Notons aussi que notre industrie, comme le monde vivant, nous fournit des exemples d’adaptation aussi parfaite que possible, non susceptibles de progrès. De même que l’aile de l’hirondelle de mer est merveilleusement adaptée aux grandes traversées aériennes, l’invention de la fourchette et de la cuillère, qui n’a pas progressé depuis un ou deux siècles, répond parfaitement au besoin de manger, — l’invention du fil et de l’aiguille, depuis les temps préhistoriques, au désir de coudre, — l’invention de la plume au désir d’écrire, — l’invention de la bêche au désir d’ameublir le sol à force de bras. Il est vrai que, lorsque ces désirs viennent à être modifiés dans une partie du public par des inventions nouvelles, on voit la charrue à vapeur faire reculer la bêche, la machine à écrire faire tomber quelques plumes, la machine à coudre empiéter sur le domaine de l’aiguille  \phantomsection
\label{v1p342}simple et du dé, et la sonde œsophagique remplacer passagèrement la cuillère et la fourchette... Mais cela n’empêche pas les \emph{outils} ainsi délogés de quelques-unes de leurs anciennes positions par des \emph{machines}, de conserver toute leur perfection d’ajustement qui les rend inexpugnables là où ils se sont retranchés.\par
Les conséquences qui découlent de ce fait sont bonnes à signaler encore plus que ses causes : la plus remarquable est \emph{la loi d’accumulation des inventions}, car, si la suivante détruisait entièrement la précédente, elles se succéderaient sans s’accumuler. Il pourrait y avoir progrès encore, dans le sens de perfectionnement, mais non d’enrichissement général. Une autre conséquence, intéressante au point de vue de la théorie des prix, c’est que, derrière l’invention nouvelle et meilleure, veille l’invention ancienne, toujours prête à reconquérir son ancien domaine au cas où l’exploiteur de la nouvelle abuserait de sa supériorité et voudrait la faire payer trop cher. Si la compagnie qui exploite le monopole des allumettes chimiques prétendait nous les vendre 1 franc la boîte, tout le monde reviendrait à l’usage habituel du briquet. M. Paul Leroy-Beaulieu a fort bien mis en évidence cette remarque sous le nom de \emph{loi de substitution} qui reçoit les applications les plus variées. Il y a là un frein assuré contre les abus de toute exploitation privilégiée, une limite infranchissable opposée aux prétentions de tout monopole.
\subsubsection[{I.7.c. Deux causes de destruction du capital-invention.}]{I.7.c. Deux causes de destruction du capital-invention.}
\noindent Malgré tout, le capital-invention subit fréquemment des pertes partielles, sinon complètes, et, quoique destinées à être comblées avec avantage, elles ne laissent pas d’être momentanément ressenties avec douleur. Pour juger de la gravité de ces pertes, distinguons entre les deux causes de destruction des inventions, ou plutôt de leur utilité, de leur  \phantomsection
\label{v1p343}adaptation sociale. Ces deux causes, nous le savons, sont : 1\textsuperscript{o} leur remplacement par d’autres mieux adaptées aux mêmes besoins, aux mêmes goûts, aux mêmes caprices du public ; 2\textsuperscript{o} le changement des besoins, des goûts, des caprices du public, par suite d’une nouvelle foi religieuse ou politique, d’une nouvelle esthétique, d’un nouveau vent de mode qui vient à souffler. La première cause est toujours avantageuse, en somme, à l’humanité, — sinon à la génération qui voit s’opérer la crise. \emph{Si l’on ne vise qu’au bonheur de la génération actuelle}, si l’on ne tient compte que des vivants, il faut accorder à Sismondi que, bien souvent, le plus souvent même, les grandes inventions industrielles, destinées au plus merveilleux avenir, sont détestables, puisqu’elles commencent par infliger plus de douleurs vives et durables aux travailleurs réduits par elles à l’inaction et plongés dans la misère, qu’elles ne procurent d’agréments légers aux consommateurs et de grands bénéfices aux premiers exploiteurs. Et je ne sais pas de quel droit les socialistes ou même les individualistes quelconques (car les socialistes ne sont que des individualistes outranciers et sans le savoir) contrediraient en ceci la thèse de Sismondi et l’accuseraient d’être un rétrograde. Qui dit individualiste dit \emph{actualiste} (qu’on me pardonne ce néologisme) ; se préoccuper de l’individu seul, c’est se préoccuper exclusivement du moment actuel ou des vivants actuels. Par la même raison que les individualistes refusent de justifier l’immolation du bien-être de quelques-uns au but national quand ce but exige ce sacrifice, ils doivent trouver injuste et déraisonnable que la génération présente se sacrifie aux générations futures.\par
Mais la question est de savoir si ce double sacrifice au but national et aux générations futures n’est pas le fondement même du devoir en ce qu’il a de plus sacré, et si le devoir ainsi entendu n’est pas l’expression des aspirations les plus essentielles de l’individu, projeté par sa propre nature hors de lui-même, au delà de lui-même. Si la \emph{raison} devait être  \phantomsection
\label{v1p344}entendue au sens étroit où elle signifierait \emph{égoïsme} et \emph{actualisme}, il faudrait dire que la raison, fleur terminale de la vie, est la négation de la vie, qu’elle s’oppose — inutilement — à l’évolution vivante, au cours irrésistible des générations précipitées les unes vers les autres.\par
L’individualisme des économistes classiques est en contradiction avec leur optimisme. Par exemple, quand ils voient dans le \emph{laissez-faire} la solution la meilleure aux problèmes économiques, quand ils essaient de prouver que les maux produits par la liberté, par le libre rayonnement des inventions et des machines, se guériront par la liberté même, que les ouvriers quitteront l’industrie ancienne pour se porter vers l’industrie nouvelle, en se détournant d’une branche de la production devenue branche gourmande pour affluer vers une autre branche d’un développement insuffisant, en sorte que, finalement, la production se rééquilibrera avec la consommation et les choses iront pour le mieux ; quand ils raisonnent ainsi, en optimistes, ils se placent implicitement et inconsciemment au point de vue précisément opposé au point de vue individualiste. Ils supposent, en effet, que des biens \emph{futurs}, des biens respectés par d’autres hommes, par d’autres individus que les individus actuels le plus souvent, doivent entrer en compensation des maux \emph{présents} et que l’individu présent, en se sacrifiant de la sorte, doit être heureux de son immolation, en vertu de la solidarité qui le lie à ses semblables, présents ou futurs. L’individualisme économique ne saurait donc être optimiste qu’en se démontant. — Il s’ensuit aussi, comme corollaire, que l’optimisme économique devient soutenable dès que, renonçant à l’individualisme, on professe la doctrine plus haute de l’abnégation et du dévouement.\par
Mais, si la première cause de destruction partielle du capital-invention, l’invention nouvelle et meilleure, est toujours, finalement, un avantage social, en est-il de même de la seconde cause, de celle qui consiste dans le changement  \phantomsection
\label{v1p345}des besoins, des goûts, des caprices du public ? Non, pas toujours, ni le plus souvent. A la vérité, quand l’avènement d’une religion supérieure, abolissant les sacrifices sanglants et les jeux sacrés du cirque, apporte au cœur des sentiments de pitié et de commisération fraternelle, et à l’esprit des sujets plus élevés de spéculation, on peut se consoler des temples et des amphithéâtres délaissés et tombant en ruines ; quand une révolution politique, aussi bien, ce qui est rare, remplace avec avantage ce qu’elle a détruit, il n’y a pas à regretter beaucoup la suppression de toutes les industries du luxe spécial et démodé qu’entretenait une cour brillante. Mais, tous les ans la mode fantasque, en fait de vêtements, et, tous les dix ans, en fait d’ameublements, change inutilement les goûts du public et réduit à néant les prodiges d’ingéniosité exécutés par les artistes, les tailleurs, les ébénistes, pour répondre à la demande antérieure. A quoi bon, sinon à faire gagner quelques commerçants ?\par
La simple \emph{richesse}, fruit du travail, est faite, normalement, pour être consommée et détruite ; sa conservation même a sa consommation et sa destruction pour raison d’être. Mais le capital, le capital-invention, fruit de l’ingéniosité ou du génie, est fait, normalement, pour durer toujours et s’accroître sans fin ; et les changements de mœurs qui le détruisent — sinon comme idée, du moins comme valeur et réalité sociale — devraient être une exception. Si cette exception, en notre âge de crise, semble être devenue la règle, cette anomalie ne peut être que passagère. Tôt ou tard, la même nécessité logique qui a forcé les sociétés errantes sur la terre à se fixer, à convertir leurs tentes en maisons, forcera les sociétés moralement nomades à devenir moralement sédentaires, à s’asseoir en un idéal définitif, religieux, militaire, voluptueux, dont la stabilité favorise l’accumulation des inventions et s’oppose à leur substitution. C’est l’âge des perfectionnements progressifs et indéfinis. Une société essentiellement révolutionnaire, qui, par des cataclysmes \phantomsection
\label{v1p346} périodiques, détruirait toutes ses espèces sociales pour créer à leur place d’autres types de toutes pièces, ressemblerait à ce monde catastrophique de Cuvier dont la science ne veut plus.
\subsubsection[{I.7.d. Réfutation d’erreurs à ce sujet.}]{I.7.d. Réfutation d’erreurs à ce sujet.}
\noindent Si, comme nous venons de le montrer, la connaissance des inventions utiles est le capital essentiel, germe et source du capital-matériel, il nous sera facile de juger certaines propositions accréditées parmi les économistes au sujet du capital. L’idée de Lasalle, suivant laquelle l’esclavage aurait été la première origine du capital, doit évidemment être rejetée. Le capital est né de la première invention, de celle du feu d’abord, moyennant un procédé de frottement dont la connaissance permettait de \emph{reproduire} à volonté les services rendus par la chaleur solaire : cuisson des aliments, poterie, défense contre le froid, etc. Cette connaissance a été la première richesse \emph{indéfiniment reproductrice} des primitifs. Aussi n’ont-ils pu tarir d’hymnes à sa louange. — La cueillette des fruits spontanés du sol, des graines, des baies, a dû donner nécessairement l’idée à quelque sauvage, un peu moins imprévoyant que ses pareils, de réserver pour la faim à venir quelques-uns de ces fruits. Puis, leur conservation a suggéré leur ensemencement, d’où leur \emph{reproduction}, enfin leur culture et leur sélection. Parallèlement à cette évolution s’en produisait une autre. La chasse ou la pêche des animaux, surabondante tel jour, insuffisante tel autre, a donné l’idée à un chasseur ou à un pêcheur primitif de conserver le gibier mort ou le poisson sorti de l’eau pour la faim future, puis de capter des animaux vivants, qui pouvaient non seulement se conserver plus longtemps mais se reproduire et se multiplier. La capture des jeunes surtout a été un progrès sensible qui a rendu facile à réaliser \phantomsection
\label{v1p347} l’idée de l’apprivoisement. Ainsi s’est développé le capital-humain, c’est-à-dire à la suite de tâtonnements successifs qui ont fait découvrir les conditions d’ensemencement propres à rendre fécondes les graines conservées, et les conditions d’élevage propres à rendre domestiques et féconds en captivité les animaux capturés vivants\footnote{ \noindent Ainsi, entre la chasse, la pêche, la cueillette, d’une part, et, d’autre part la domestication des plantes ou des animaux, il y a une étape intermédiaire qui explique sans peine le passage de la première phase à la seconde : c’est la conservation prévoyante des plantes ou des animaux capturés.
 }. Jusque-là la conservation des fruits et du gibier n’avait servi qu’à empêcher la prompte destruction de la richesse produite par la cueillette, la chasse ou la pêche, sans assurer la reproduction de cette richesse. Cette certitude de reproduction, il n’y a que le capital qui la donne, et tout ce qui la donne est capital. Quand on conserve les graines et les animaux pour les manger, ce sont des produits comme d’autres ; quand on les conserve pour les semer ou les faire reproduire — ou pour les faire travailler, et servir ainsi indirectement à d’autres reproductions — ce sont des capitaux.\par
Or, il est certain, d’après cela, que l’homme a dû commencer à capitaliser bien avant le régime esclavagiste, qui attend, pour se développer, la constitution du régime pastoral. L’esclavage a été sans doute institué à l’image de la domestication des animaux et non \emph{vice versa.} Chez beaucoup de peuples chasseurs, tels que les Peaux-Rouges, qui ont déjà des animaux domestiques, au moins le chien, l’esclavage est inconnu.\par
La progression, l’accumulation indéfinie des capitaux est-elle réellement, comme Karl Marx l’affirme, une loi constante de notre organisation économiste ? Et, si elle l’est vraiment, a-t-elle la cause que Karl Marx indique, c’est-à-dire la spoliation partielle du travailleur par le capitaliste, le non-paiement par celui-ci d’une partie du travail de l’ouvrier, de son \emph{surtravail ?}\par
 \phantomsection
\label{v1p348}Notre distinction du capital-germe et du capital-cotylédon, répond à ces deux questions. Ce qui s’accumule vraiment, nous le savons, en vertu d’une nécessité non pas historique et restreinte à notre société moderne, mais logique et universelle, c’est le capital-germe, le legs des idées indestructibles du génie humain. A ce point de vue, parler de \emph{régime capitaliste}, comme si le \emph{capitalisme} était une phase transitoire du développement social, c’est employer l’expression la plus impropre, la plus capable d’égarer l’esprit. — Quant au capital-matériel, né de ce capital intellectuel, il se détruit et se reproduit à chaque instant, et c’est à lui seul que s’applique la remarque de Stuart Mill sur la rapidité avec laquelle le capital se régénère après les ravages d’une guerre ou d’une révolution. Mais il ne se régénère pas toujours, on l’a vu s’anéantir pour ne plus renaître ; et le spectacle des nations en décadence, qui vont s’appauvrissant, est bien fait pour nous convaincre que nulle nécessité interne ne le contraint à se grossir toujours.\par
C’est aussi à lui seul que peut s’appliquer, dans une certaine mesure, l’explication de la capitalisation par l’exploitation du travailleur. Quant à la formation progressive du capital essentiel, du capital spirituel, elle a pour cause non le vol, ni même l’épargne égoïste, mais, au contraire, le don et la dépense prodigue de soi. Songez à tout ce qu’il y a eu de tribulations et de fatigues dans la vie tourmentée des hommes de génie qui ont fondé la science et l’industrie modernes, et vous conviendrez que, comme chacun de nous, le plus humble des prolétaires, s’il est exploité par son patron, exploite aussi sans le savoir ces grands bienfaiteurs. C’est au \emph{surtravail non payé} ou plutôt à la \emph{surdouleur} mal récompensée de ces hommes qu’il doit d’être nourri, d’être logé, d’être vêtu, d’être gratuitement instruit comme il l’est, soigné dans ses maladies, transporté si rapidement et si économiquement d’un bout du monde à l’autre. — Ce n’est pas sans raison et par un jeu puéril de mots qu’Auguste  \phantomsection
\label{v1p349}Comte rapproche si souvent, comme liés ensemble, les idées générales et les sentiments généraux. L’inventeur peut être égoïste dans sa vie ordinaire, mais il ne l’est pas quand il invente, quand, descendant dans l’abstraction et l’analyse plus profondément que ses prédécesseurs, il met la main sur des secrets de la nature qui sont des puissances nouvelles pour l’humanité\footnote{ \noindent Pour l’humanité, et pour sa patrie d’abord, J.-B. Say cite ces paroles d’un homme d’État anglais, prononcées dans un discours public en 1824 : « Je ne puis m’empêcher de penser, en jetant un regard sur la lutte où nous avons été engagés pendant un quart de siècle, de déclarer que, si nous l’avons terminée glorieusement, nous en sommes entièrement redevables aux nouvelles ressources que nous a créées le génie de Watt. J’ajouterai que, sans les améliorations mécaniques et scientifiques qui ont donné à l’industrie et à la richesse de ce pays un développement graduel mais toujours certain, nous aurions été contraints de souscrire une paix humiliante avant l’époque où la victoire a favorisé nos armes. »\par
 C’est donc Watt, bien plus que Wellington, qui a triomphé de Napoléon. Le génie industriel a battu le génie militaire.
 }. Par une suite, par un enchaînement de recherches désintéressées, d’enthousiasmes féconds, non par un entassement de calculs et d’efforts personnels, s’enrichit le vrai capital-humain. Que la source de cette curiosité amoureuse, de cette passion du vrai pour le vrai, vienne à tarir, que les conditions nécessaires de cette libre activité scientifique viennent à être supprimées, ce capital cessera de s’accroître, et le progrès industriel s’arrêtera avec le progrès scientifique dont il est l’effet. Ces conditions, quelles ont-elles été jusqu’ici ? En général, un certain degré d’aisance et de loisir, d’aisance et de loisir héréditaires même, qui, dès les temps préhistoriques, a dû être le privilège de quelques-uns ; et, en second lieu, la connaissance non moins privilégiée des vérités déjà acquises. La réunion de ces deux privilèges dans les cités commerçantes de l’Asie Mineure ou de la Grande-Grèce au {\scshape v}\textsuperscript{e} ou {\scshape vi}\textsuperscript{e} siècle avant J.-C. a donné naissance aux mathématiques et à la physique. Supposez qu’une démocratie égalitaire ait de tout temps nivelé les fortunes dans la Grèce antique, l’aristocrate Archimède, parent de Hiéron, Archimède, de qui dérivent la géométrie et la mécanique, ne serait pas né ou aurait été un mort-né  \phantomsection
\label{v1p350}social. Tous les découvreurs de vérités, tous les inventeurs d’utilités même, ou peu s’en faut, ont été des hommes libres dans l’antiquité, des « bourgeois » dans les temps modernes\footnote{ \noindent Si l’inventeur, — à la différence du travailleur, du « mercenaire », [{\corr souverainement}] méprisé — était admiré et parfois divinisé par les anciens, cela tenait à ce que les esclaves n’inventaient rien. Si les esclaves eussent été inventifs, on peut être assuré que les anciens, tout en utilisant leurs inventions comme leurs travaux, auraient affecté de mépriser celles-là comme ceux-ci.
 }. Qu’ils doivent cette inventivité supérieure à des avantages de situation sociale, nullement à une supériorité de race, j’en suis persuadé. Mais, si injustes qu’ils paraissent, ces avantages ont été nécessaires ; et, sans ces injustices, il faut avouer qu’une injustice autrement profonde, quoique célébrée et admirée partout, à savoir notre supériorité sur nos ayeux, eût été impossible. Ce n’est donc pas, à vrai dire, le travail non payé de l’esclave ou le sur-travail non payé de l’ouvrier qui a permis au capital de naître et de grandir, c’est, encore une fois, le loisir de l’homme libre antique et du « bourgeois » moderne, le loisir, père du plaisir qu’ils ont eu à découvrir et à inventer, après le tourment douloureux de la recherche, et qu’ils se sont fait payer quelquefois, mais jamais trop cher. — Cette considération, à la vérité, est la seule justification possible des inégalités et des iniquités du passé et du présent ; mais elle suffit souvent. En sera-t-il de même des inégalités et des iniquités de l’avenir, qui ne seront peut-être pas moindres ? C’est probable, mais nous n’avons pas ici à prophétiser.\par
Ce que nous pouvons affirmer, c’est que le nivellement des fortunes, s’il entraînait en même temps le nivellement des connaissances par l’abaissement de l’enseignement supérieur, mal compensé par l’exhaussement de l’enseignement primaire, serait désastreux pour le progrès de l’industrie. Les socialistes les plus éclairés le savent bien, et ce qu’ils demandent au fond, c’est, non la suppression de l’inégalité, mais la substitution d’une inégalité résultant d’un \emph{concours}  \phantomsection
\label{v1p351}exclusivement à une inégalité résultant en partie de l’héritage. Plus les problèmes à résoudre deviennent ardus par l’épuisement des inventions relativement faciles et à fleur de sujet, plus il importe de surcultiver l’élite intellectuelle appelée à découvrir péniblement leur solution pour le bien commun.
\subsubsection[{I.7.e. L’appropriation collective du capital. Machine et talent. Partage, entre capitaliste, ouvrier et public, des avantages de la machine.}]{I.7.e. L’appropriation collective du capital. Machine et talent. Partage, entre capitaliste, ouvrier et public, des avantages de la machine.}
\noindent Ainsi, en résumé, telles sont les conditions de la genèse, de la conservation et de l’accroissement du capital essentiel : le génie d’abord, ou l’ingéniosité, condition fondamentale ; un certain degré de stabilité sociale et en même temps d’inégalité sociale ; un degré suffisant aussi de liberté, d’aisance, de loisir ; et une instruction supérieure pour un élite, et non pas médiocre pour tous. — Ajouterons-nous à ces conditions cette autre, réputée indispensable parmi les économistes libéraux, la propriété individuelle du capital, — ou la condition inverse, préconisée par les écoles socialistes : l’appropriation collective du capital ?\par
En ce qui concerne le capital-invention, la réponse est facile : toutes les inventions, naissant de l’individu, commencent par être la propriété exclusive de leur auteur ; mais toutes aussi, au bout d’un temps plus ou moins long, finissent par tomber dans le domaine commun, dans le patrimoine collectif. Et nous voyons clairement ici la justice à la fois et l’utilité de cette transformation : l’appropriation d’abord industrielle de l’invention nouvelle a été sans nul doute une condition de sa genèse, condition sans laquelle le plus souvent elle n’aurait pas eu lieu. Mais, d’autre part, au bout d’un temps, l’appropriation collective de cette invention a été une condition presque nécessaire de sa conservation. Les recettes transmises de père en fils comme un secret de famille ne tardent pas à se perdre. Ainsi, pour le capital-invention, la propriété individuelle est une cause  \phantomsection
\label{v1p352}d’accroissement, et la propriété collective une cause de conservation. Les deux concourent au même but, loin de se combattre ; et l’on va toujours ici de la propriété individuelle à la propriété collective, à l’inverse de la formule d’évolution, devenue banale, qui veut que la propriété collective ait précédé et engendré la propriété individuelle.\par
Ce qui est vrai du capital-invention l’est-il aussi du capital-outillage ? Pas au même degré. L’État aussi bien que l’individu peut faire bâtir de nouvelles gares et de nouvelles usines, et accroître ainsi le capital-matériel ; et l’individu aussi bien que l’État peut l’entretenir et le conserver. Mais, en fait, c’est toujours l’initiative privée qui a précédé l’initiative publique pour la première mise en œuvre des inventions, comme c’est toujours un individu qui les a conçues ; et c’est toujours par la diffusion de cet exemple individuel, propagé de proche en proche, que l’exploitation d’une invention s’est généralisée et vulgarisée, comme la connaissance de cette invention. Ici comme là, c’est en se répétant et se multipliant, que la propriété individuelle devient générale, et parfois, mais jusqu’ici exceptionnellement, tend à devenir même collective, à se nationaliser après s’être généralisée. Nous ne pouvons prétendre en quelques mots trancher la question capitale du socialisme. Tout ce que nous sommes en droit d’affirmer, à propos de la question spéciale que nous nous sommes posée, c’est que, au point de vue de la conservation, comme à celui de l’accroissement du capital-matériel, la propriété individuelle a fait ses preuves et que la propriété collective n’a pas encore fait les siennes.\par
Le but à poursuivre est que, d’une part, les connaissances théoriques et techniques, d’autre part les moyens pratiques de les mettre en œuvre, ne soient pas refusés au hommes les plus capables de les comprendre et de les réaliser, de les perfectionner et de les développer. Pour cela, que faut-il ? L’expropriation violente des moins capables ? Non, mais leur expropriation lente et sûre : c’est-à-dire l’accès ouvert à  \phantomsection
\label{v1p353}tous de l’enseignement théorique et technique, et la diffusion du crédit combiné avec l’association.\par
Il y a cette différence entre le capital intellectuel et le capital-matériel, qu’il est bien plus aisé de vulgariser le premier que le second. L’instruction intégrale, si elle était offerte à tous par l’État, mettrait tout le monde — une faible élite, en fait, il est vrai — à même de posséder mentalement toutes les découvertes et toutes les inventions, comme idées de l’esprit ; mais cela ne ferait que rendre plus douloureuse la difficulté, l’impossibilité de permettre à tous l’emploi pratique de ces connaissances. Toutefois il n’est pas nécessaire d’être capitaliste, à notre époque, pour fonder une entreprise avec chance de succès : il suffit de s’associer et d’emprunter les capitaux d’autrui. Dix prolétaires en s’associant trouvent un crédit qu’aucun d’eux séparément n’aurait rencontré.\par
Pour bien comprendre ce qu’est l’outillage social, et ce qu’il doit être, il est bon de remonter à son origine psychologique. Une découverte ou une invention, qui accroît la science ou la puissance de l’homme, ou à la fois les deux, s’incarne toujours, soit, au dedans de nous, dans notre mémoire nerveuse ou musculaire, sous la forme d’un cliché mental ou d’une habitude acquise, d’une notion ou d’un talent, — soit, au dehors, dans un livre ou une machine. Un livre n’est qu’une rallonge et un appendice de notre cerveau ; une machine n’est qu’un membre additionnel. On peut dire indifféremment qu’un livre est un souvenir extérieur, ou qu’un souvenir est un livre interne, qu’une sorte de bibliothécaire invisible, caché dans notre sous-moi, nous met sous les yeux au moment voulu. De même, une machine est un talent extérieur et un talent est une machine interne. Coudre et tricoter constituent pour une jeune fille deux talents qui peuvent être suppléés par une machine à coudre ou une machine à faire des bas. Chanter un air est un talent qui, lorsque le chanteur chante sans âme et sans art, peut être  \phantomsection
\label{v1p354}remplacé assez bien par une boîte à musique. Bien mieux, le rouleau d’un phonographe n’est-il pas souvent la voix même du meilleur chanteur matérialisée et mobilisée ? C’est ainsi que les habiletés diverses et multiples des anciens artisans, leurs longs apprentissages, leurs emmagasinements lents d’habitudes spéciales, ont été rendus inutiles en grande partie par la fabrication des machines successives. Celles-ci ne sont que la projection extérieure et aussi l’amplification souvent prodigieuse de ces talents et des organes par lesquels ces talents s’exerçaient ; et l’on peut dire aussi bien que, si la destruction de ces machines forçait ces talents à renaître, si, par exemple, la suppression des imprimeries faisait ressusciter les calligraphes et les enlumineurs de manuscrits, ou la suppression des filatures les anciennes fileuses, les talents ainsi renaissants seraient la réincorporation simplifiée et rapetissée des machines détruites\footnote{ \noindent Les ouvriers, dit Gide, peuvent être remplacés par des \emph{machines, non les patrons.} — Est-ce vrai ? Est-ce que l’invention de la locomotive n’a pas supprimé tous les entrepreneurs de messageries aussi bien que leurs postillons, conducteurs et valets d’écurie ? Est-ce que la machine à coudre n’a pas supprimé beaucoup de petits ateliers de couture, laissant à la fois sans emploi la patronne et ses deux ou trois ouvrières ? Est-ce que les machines à tisser à la vapeur n’ont pas condamné à l’inaction aussi bien les maîtres tisserands que les ouvriers ? En réalité, c’est la \emph{direction} du travail, aussi bien que le \emph{travail} lui-même, qui est produite mécaniquement par les machines perfectionnées. \emph{Dans une certaine mesure}, les presses à vapeur règlent le travail des ouvriers, le leur distribuent.
 }.\par
Ce rapprochement est assez propre à éclairer la question de savoir jusqu’à quel point est fondée la doctrine suivant laquelle les ouvriers ont droit à se partager tout le produit de leur travail combiné avec le travail des machines qu’ils font marcher mais qu’ils n’ont ni inventées ni fabriquées même. La logique exige, à ce point de vue, d’après le résultat de notre analyse, que l’on aille jusqu’à dire que le travail des divers ouvriers doit leur être payé en raison de sa seule durée ou de sa seule intensité, sans nul égard à leur habileté inégale et dissemblable. Et, de fait, Karl Marx semble  \phantomsection
\label{v1p355}avoir cédé à une impérieuse nécessité logique de ce genre en formulant, dans son premier volume, sa théorie de la valeur (si simpliste, fondée sur l’heure de travail), qu’il a raturée plus tard. En effet, l’ouvrier qui a acquis un talent par l’apprentissage, fait jouer, nous le savons, une machine intérieure et, dans cet acte complexe qu’on appelle confusément son travail, il y a à distinguer la dépense d’effort musculaire et d’attention requise pour mettre en jeu cette machine interne, et le fonctionnement même de celle-ci, conformément à un système de rouages imaginé par un inventeur, exécuté en chaque ouvrier par les patrons chez lesquels il a fait son apprentissage, par les camarades sur l’exemple desquels il s’est réglé. Cette machine interne, l’ouvrier la possède, comme l’entrepreneur possède les machines extérieures qu’il a achetées ou qu’il a construites ; et si celui-ci n’a droit de bénéficier en rien du fonctionnement de la machine extérieure, on ne voit pas pourquoi l’ouvrier bénéficierait seul du fonctionnement de sa machine interne, qui existe en lui en vertu d’un véritable don social, par l’apprentissage et l’exemple. Il n’y a pas moins de raison, au fond, de rendre gratuit le jeu de cette machine interne, que le jeu d’une machine extérieure qui rendrait le même service. Et il n’y a pas moins de raison, à l’inverse, de faire payer le jeu de la machine externe à celui qui l’a fabriquée ou qui l’a achetée au fabricant, que l’exercice d’un talent à celui qui a pris la peine de l’acquérir.\par
De là, vais-je conclure, poussant à bout une comparaison qui, après tout, n’est pas raison, que la désappropriation des machines, des usines, des fermes, des chantiers, impliquerait aussi bien la désappropriation des talents, et que l’impossibilité de celle-ci frappe d’impuissance celle-là ? Non, celle-là est possible dans une certaine mesure, et, en ce qu’elle a de pratique, s’accomplit sous nos yeux. Il y a longtemps déjà que M. Fouillée nous a montré l’étendue et l’extension croissante de la propriété collective. Mais ce qui  \phantomsection
\label{v1p356}vaut mieux que l’expropriation universelle, c’est autrement dit la socialisation des propriétés individuelles, la multiplication et la mobilisation de ces propriétés, et c’est ce bienfait surtout qui est assuré par la substitution des machines extérieures, échangeables et mobilisables, qu’on peut multiplier à volonté, aux machines internes et individuelles. Grand progrès, mais qui entraînait comme conséquence inévitable la séparation entre la possession de la machine et la possession de la force de travail nécessaire pour faire marcher la machine. Et de là tous les conflits entre le capitaliste et les travailleurs. La question est de savoir si, pour les apaiser, le meilleur moyen est de mettre fin à cette séparation en rendant les ouvriers futurs propriétaires collectivement des machines, comme les ouvriers passés étaient propriétaires individuellement de leurs outils. Cela est loin d’être démontré. Ne perdons pas de vue que l’appropriation des machines par tous n’importe qu’autant qu’elle est exigée par leur adaptation à tous. L’utilité collective, non la propriété collective, est le but final.\par
Le seul problème pratique à résoudre est de savoir dans quelles proportions le propriétaire de la machine, l’ouvrier et le consommateur, participeront aux avantages que le travail de la machine apporte à la société. Or, quand une invention nouvelle apparaît, — par exemple un perfectionnement notable apporté à l’éclairage électrique — la divulgation graduelle de ce secret, qui ne reste jamais un secret longtemps, a trois effets distincts : 1\textsuperscript{o} d’accroître le bénéfice des entrepreneurs qui se l’approprient successivement, à cause de la productivité plus grande ou meilleure du travail des ouvriers qu’ils emploient ; 2\textsuperscript{o} de faire hausser les salaires de l’ouvrier, dont le prix tend toujours à s’élever, comme le dit très bien M. Levasseur, quand son travail devient plus productif ; 3\textsuperscript{o} enfin, de procurer au consommateur un avantage évident, soit que, le produit (la lampe électrique, par exemple) restant au même prix, son utilité (sa lumière), ait  \phantomsection
\label{v1p357}augmenté, soit que, son utilité restant la même, son prix ait diminué.\par
Rien de plus inégal d’un pays à un autre, rien de plus variable d’un temps à un autre que la proportion de ces trois éléments, de ces trois gains que se partagent le capitaliste-entrepreneur, l’ouvrier et le public. Il est inutile de chercher une loi empirique de ces variations et de ces inégalités dont l’explication, dans chaque cas, doit être demandée à la théorie générale de la valeur qui suffit à en rendre compte et que nous développerons plus tard. Mais il n’est pas douteux que, à mesure que l’exploitation de l’invention se prolonge et se vulgarise, si quelque perfectionnement nouveau n’est pas inventé, le bénéfice de l’entrepreneur va diminuant — de là résulte la baisse du taux de l’intérêt — pendant que le salaire de l’ouvrier se maintient ou progresse en général et que l’avantage du consommateur augmente. Il résulte de ce mouvement comparé des trois sortes de gains entre lesquels se répartit le bienfait total apporté par une invention à l’humanité, que le capitaliste qui l’exploite tend à être de plus en plus dépossédé des bénéfices, d’abord excessifs, qu’elle lui a valus. Si, par hypothèse, — et l’hypothèse n’a rien d’impossible — la source des inventions venait à tarir, on peut être assuré que l’exploitation progressive des inventions anciennes, de plus en plus vulgarisées, réaliserait cette tendance générale à une moindre inégalité des traitements, des honoraires, des salaires et des prix, où M. Paul Leroy-Beaulieu n’a pas eu tort de voir une des lois fondamentales de l’économie politique. Mais ce n’est vrai, remarquons-le, que d’une vérité \emph{conditionnelle ;} et ce n’est pas à notre époque, d’une inventivité si tumultueuse, que la \emph{condition} indiquée s’est réalisée.
\subsubsection[{I.7.f. Intérêt des capitaux. Le revenu sans travail.}]{I.7.f. Intérêt des capitaux. Le revenu sans travail.}
\noindent Dans ce qui précède, nous avons effleuré primitivement beaucoup de questions qui ont trait à l’\emph{opposition} et à l’\emph{adaptation \phantomsection
\label{v1p358}} économique. Rentrons dans le cercle de la \emph{répétition} économique, en disant quelques mots d’une question, à laquelle nous venons de toucher d’ailleurs, à propos des bénéfices de l’entrepreneur-capitaliste : la question, si importante, de l’intérêt des capitaux, qui est l’une des principales sources des revenus privés.\par
J’ai dit plus haut que la distinction du capital et du revenu s’était modelée sur celle de la terre et des fruits ou même, surtout quand les troupeaux ont été considérés comme un immeuble par distinction, sur la distinction du bétail et de son croît. Dans le bel ouvrage de M. Kovalesky sur la \emph{Coutume des Ossètes}, je trouve une confirmation instructive de cette dernière dérivation. Il y montre que, les prêts de bétail ayant précédé les prêts d’argent, ceux-ci ont pris modèle sur ceux-là. Et c’est ainsi qu’on s’explique un phénomène historique dont il serait bien malaisé, je crois, de rendre compte par la loi de l’offre et de la demande, je veux dire le taux énorme de l’intérêt de l’argent dans les sociétés primitives, où l’argent, il est vrai, était fort rare, mais où le besoin d’argent était fort restreint. M. Kovalesky fait observer que, lorsqu’on prêtait une vache, on stipulait qu’à la fin de l’année elle serait rendue avec un veau, avec son croît annuel ; et, si elle n’était rendue qu’au bout de deux ans, ce n’était pas un veau seulement mais une autre vache qui était due en sus. C’était donc un intérêt annuel de 50 p. 100. Or, plus tard, quand on prêta de l’argent, on se crut autorisé par ce précédent à demander un intérêt de 50 p. 100 pareillement. Il en était de même chez les Romains et les Grecs primitifs\footnote{ \noindent Autre exemple de cette formation imitative. « L’institution du cautionnement consenti par des étrangers, dit Kovalesky, se forma sur le type de la caution familiale. » Voir le développement de cette idée, p. 152 de son ouvrage plus haut cité.
 }.\par
Les esclaves ont été, après ou en même temps que les bestiaux, un capital qu’on plaçait à intérêt. Chez les Romains de l’Empire encore, ils étaient la principale forme du capital,  \phantomsection
\label{v1p359}puisqu’ils étaient l’équivalent de nos machines. A Athènes il y avait beaucoup de gens qui louaient leurs esclaves à des chefs d’industrie, comme nous plaçons nos fonds dans une affaire pour toucher des dividendes. — Ce capital servile ne pouvait aussi facilement ni aussi rapidement que le nôtre s’accroître en temps de paix. C’est en temps de guerre, par le pillage et la réduction des vaincus en esclavage, qu’il s’augmentait ; tandis que, à présent, notre capital pécuniaire s’accroît surtout dans les périodes paisibles et se détruit en temps de guerre, perspective qui contribue assurément à rendre les guerres plus rares.\par
Il s’agit là, bien entendu, du \emph{capital-cotylédon}, du capital-matériel, non du capital essentiel, spirituel. C’est grâce à la monnaie surtout que ce capital secondaire, d’une haute importance néanmoins, a pu s’accroître avec la rapidité dont nous sommes témoins. Aussi est-ce sous sa forme monétaire qu’il convient d’envisager le capital-matériel pour comprendre la hauteur de son rôle économique. Le capital en devenant monnaie, et la monnaie en devenant capital, ont déployé une envergure immense qu’on ne pouvait prévoir avant cette alliance ou cette confusion féconde. — La monnaie métallique, — et, dans une certaine mesure, toujours grandissante, la monnaie de papier\footnote{ \noindent Je ne dis pas la monnaie \emph{fiduciaire}, car toute monnaie d’or, d’argent ou de papier, est fiduciaire avant tout.
 }, — a deux caractères. Le premier est, comme nous l’avons dit, d’être un moyen d’échange tellement supérieur à tout autre, comme rapidité, commodité et universalité, qu’elle supplante toutes les marchandises à cet égard et finit par les dépouiller entièrement de cette qualité, — d’où les crises, quand elle se raréfie. Le second caractère de la monnaie, beaucoup moins mis en lumière, est d’être seule un moyen — fictif, mais réputé réel — de \emph{capitalisation indéfinie et universelle}, c’est-à-dire d’emmagasinement et de conservation imaginaire des produits passés en vue de la production  \phantomsection
\label{v1p360}ou de la consommation futures. Si la monnaie n’était que le \emph{signe} du capital (matériel), c’est-à-dire des produits épargnés et mis en réserve, outils ou machines, aliments ou autres denrées, elle devrait se détruire plus ou moins rapidement comme le font les blés en grenier ou les vins en chai ou les marchandises en magasin. S’il en était ainsi, si la monnaie était une substance qui se détruisît avec la rapidité moyenne de destruction propre aux articles emmagasinés, par le seul effet du temps, la monnaie, dans cette hypothèse, pourrait être vraiment regardée comme le simple signe représentatif de produits concrets réellement conservés et accumulés ; et, dans ce cas, on ne verrait point l’épargne annuelle de nouveaux capitaux élever la pyramide du capital à des hauteurs toujours plus formidables.\par
Cela vaudrait-il mieux ? Il y a certainement quelque chose de factice dans cette expression monétaire que nous donnons à nos épargnes réelles. Nous ne pouvons, en fait, épargner, « mettre de côté », que des biens essentiellement passagers ; mais, en les exprimant par des chiffres de francs, de dollars ou de livres sterlings, qui se conservent éternellement dans l’inventaire de nos fortunes, nous nous donnons l’illusion de soustraire au temps des richesses mortelles. De là la fiction partielle inhérente à la notion du capital monétaire, qui est conçu implicitement comme une richesse immortalisée, devenue indestructible. Mais, en réalité, la masse énorme de capitaux que nous sommes censés posséder ainsi\footnote{ \noindent « \emph{On estime} à 290 milliards (\emph{en capital}) la richesse de la France, et la monnaie (métallique) n’atteint pas les 5 p. 100 de cette somme. De même ce que l’on appelle le \emph{revenu} des Français est évalué à 25 milliards de francs qui donnent lieu, sans doute, à un chiffre de transaction triple ou quadruple, et la monnaie métallique en France n’est pas évaluée à plus de 8 milliards ou 8 milliards et demi. » (P. Leroy-B.)\par
 Je voudrais bien savoir s’il n’y a pas beaucoup de doubles emplois dans cette évaluation de \emph{200 milliards} donnée au \emph{capital} français.
 }, que nous chiffrons en francs, en dollars, en livres sterlings, en roubles, en thallers, inscrits sur des morceaux de papier, billets de banque, titres d’actions ou d’obligations,  \phantomsection
\label{v1p361}billets à ordre, etc., est bien supérieure, de plus en plus supérieure, à la masse des métaux précieux existants, qui est infime à côté, et aussi à la quantité réellement existante des épargnes en nature. Il y a là une double fiction. C’est une des nombreuses illusions d’immortalité par lesquelles l’homme cherche à oublier l’arrêt fatal : \emph{debemur morti nos nostraque.}\par
Dans une certaine mesure, il est vrai, l’hypothèse que je viens d’émettre tout à l’heure, celle d’une monnaie qui irait s’évanouissant pour correspondre aux évanouissements des capitaux concrets, se réalise par la dépréciation graduelle des monnaies, dont la puissance d’achat diminue sensiblement de génération en génération. Mais cette lente usure est toujours bien moins rapide que ne l’est, en moyenne, l’oxydation, l’altération, la destruction multiforme des produits épargnés et de l’outillage. Le contraste entre les deux subsiste donc, et il en résulte que la monnaie, métallique ou de papier, est une expression menteuse du capital.\par
Mais est-ce que le droit \emph{au revenu sans travail}, à l’intérêt de ce capital monétaire, est fondé sur ce mensonge ? Ne semble-t-il pas, en effet, que c’est en se représentant le capital monétaire comme une sorte de terre indéfiniment stable et permanente, quoique indéfiniment extensible, et, de fait, sans cesse accrue par des capitalisations nouvelles, par des Amériques et des Océanies sans cesse \emph{découvertes}, qu’on se croit le droit de la faire fructifier tous les ans comme une bonne métairie et de moissonner sa récolte sous forme d’intérêt ? Il faut cependant remarquer d’abord que cette extraordinaire inflation du capital réel par son expression monétaire a eu cet excellent effet, de faire baisser dans des proportions considérables le taux de l’intérêt. Nous avons vu que, à l’époque pastorale-agricole, il était de 50 p. 100. Or, non seulement à notre époque, il est descendu par degré à 3 p. 100 ou même au-dessous, mais, à toute époque où la circulation rapide de la monnaie métallique \phantomsection
\label{v1p362} et des instruments de crédit a fait miroiter aux yeux des chiffres de richesse en partie fantasmagoriques, l’intérêt est descendu aussi bas ou presque aussi bas que de nos jours. « A Venise, en 1624, un grand armateur, Jean Thierry, place 10 millions au taux de 3 p. 100 à la banque d’État. En Hollande, au temps de Louis XIV, on prête de l’argent à 2 p. 100\footnote{ \noindent Article de M. Paul Bureau, dans la \emph{Science sociale}, mars 1893.
 }. » En Espagne de même, au {\scshape xviii}\textsuperscript{e} siècle.\par
Puis, à quoi se réduit ce mensonge, si mensonge il y a ? à quoi se réduit même l’illusion ? Illusion féconde, d’ailleurs, qui consiste à multiplier les actes de foi et de confiance, source de toute activité. Il y aurait erreur criante, si la richesse concrète épargnée et conservée, dont la monnaie capitalisée est réputée être l’expression, se détruisait sans se renouveler, mais elle se renouvelle en s’augmentant d’ordinaire à chaque renouvellement, grâce à cette capitalisation ; elle se renouvelle au moins une fois par an, quand il s’agit d’approvisionnements agricoles ; plusieurs fois par an quand il s’agit d’approvisionnements industriels ; en sorte qu’un intérêt mensuel ou trimestriel de l’argent prêté se comprend très bien industriellement, et la prépondérance ou la survivance des habitudes agricoles a seule pu faire prévaloir et maintenir l’usage de l’intérêt annuel même dans le commerce et l’industrie. Quand on vous dit que la France ou toute autre nation épargne, capitalise, un milliard de francs par an, allez-vous croire que les tas de blés en grenier, de vins en chai, de souliers ou de vêtements en magasin, de machines construites, etc., représentés par ce chiffre formidable, sont destinés à rester ainsi inutilisés indéfiniment, jusqu’à ce qu’un nouveau stock pareil s’y ajoute l’année d’après ? Vous savez bien ce que cela signifie : c’est que les consommateurs français, au lieu de dépenser tous leurs revenus pour leurs consommations personnelles, en ont \emph{prêté} une partie à des producteurs français ou étrangers,  \phantomsection
\label{v1p363}qui ont employé ces sommes prêtées, — s’élevant à un milliard, par hypothèse — à faire consommer par des ouvriers, en vue d’en reproduire la valeur et bien au delà, le blé ou le vin épargnés, les vêtements ou les souliers emmagasinés, etc., sans compter la réparation de l’usure des machines. C’est grâce à ces prêts de choses épargnées, ou de leurs équivalents monétaires, que ce cycle reproductif a eu lieu, ainsi que l’excédent de richesses qui s’y ajoute presque toujours. Il est donc naturel que, entre le prêteur et l’emprunteur, qui ont tous deux collaboré à ce résultat, le bénéfice se partage suivant des conventions préalables.\par
Le capital monétaire a un faux air de chose continue et continuellement en train de grossir, comme un fleuve qui déborde parfois et ne remonte jamais vers sa source. Mais, sous cette continuité illusoire, dont l’illusion même a son prix, se cachent bien des discontinuités réelles ; sous cet écoulement incessant, bien des rotations et des tourbillonnements sur place. Ce capital consiste toujours en placements distincts, en prêts multiples, qui sont autant d’impulsions décisives données à des reproductions périodiques de richesses par des cycles de travaux végétaux, animaux, humains, aux périodes définies et inégales, infiniment variées.
\subsubsection[{I.7.g. Circulation du capital monétaire. Le cycle monétaire dans ses rapports avec le cycle des travaux et celui des besoins. Tendance à l’agrandissement de ces cycles et à l’accélération de leur rotation.}]{I.7.g. Circulation du capital monétaire. Le cycle monétaire dans ses rapports avec le cycle des travaux et celui des besoins. Tendance à l’agrandissement de ces cycles et à l’accélération de leur rotation.}
\noindent Pendant que les épargnes réelles et concrètes, représentées par la capitalisation monétaire, se renouvellent incessamment et, grâce à elle, la monnaie, elle, ne se renouvelle pas, elle dure seulement ; mais en même temps, elle \emph{circule}, elle revient, à chaque cycle reproductif, à chaque renouvellement augmentatif des capitaux réels, dans les mains d’où elle est d’abord sortie ; et l’on peut donner le nom de rotation monétaire à ce trajet circulaire qu’elle accomplit ainsi. Occupons-nous maintenant de ce cycle et de ses rapports  \phantomsection
\label{v1p364}avec le cycle des travaux et celui des besoins, dont il a été question ci-dessus.\par
Dirons-nous que la circulation monétaire se divise en deux cycles : le \emph{cycle producteur} (accompli depuis la sortie de la monnaie des mains du fabricant jusqu’à sa rentrée par la vente des produits) et le \emph{cycle consommateur} (accompli depuis la sortie de la monnaie des mains d’un consommateur quelconque jusqu’à sa rentrée par de nouvelles recettes) ? Cette distinction aurait pour inconvénient d’établir entre les deux cycles mis en regard une opposition ou une inversion purement apparentes. Le plus souvent le consommateur est en même temps producteur, et c’est à raison de sa production qu’il encaisse de nouveau après avoir décaissé. S’il n’est pas, à proprement parler, producteur, il remplit une fonction sociale, ne fût-ce que celle de rentier prêteur d’argent, laquelle, aussi longtemps qu’elle rend des services, lui donne droit à percevoir de nouveaux revenus après avoir fait de nouvelles dépenses. En somme, pour tout homme, fabricant ou rentier, il y a un même cycle monétaire, qui consiste dans le passage alternatif et continu de ses dépenses à ses recettes, de ses recettes à ses dépenses, c’est-à-dire, de la satisfaction de ses besoins à l’exécution de ses travaux ou à l’accomplissement de ses devoirs, et de l’accomplissement de ses devoirs à la satisfaction de ses besoins. On voit que la rotation monétaire chevauche à la fois sur la rotation des besoins et sur celle des travaux et qu’elle est bien distincte des deux.\par
Les \emph{besoins} et les \emph{travaux} n’ont pas besoin d’être mis en rapport par la \emph{monnaie} quand l’industrie reste renfermée dans le vase clos de la famille primitive où l’on travaille pour soi. Mais, dès que l’on se met à travailler pour les besoins étrangers à ceux de la famille et de la cité, la \emph{monnaie} seule permet de faire correspondre les travaux aux besoins, c’est-à-dire que, grâce à elle seulement, les travailleurs trouvent à acquérir ce dont ils ont besoin et peuvent \phantomsection
\label{v1p365} produire ce dont leurs clients éloignés et inconnus ont besoin aussi.\par
L’\emph{ère capitaliste} commence, d’après Karl Marx, à l’époque où le travailleur est séparé de ses moyens de production : alors il dépend du seul possesseur du capital \emph{monétaire} de réunir ces deux éléments nécessaires et séparés de la production, le travail et les moyens de production (outillage et matières premières). Or, avant que cette séparation ait lieu, c’est-à-dire dans la famille primitive, on commence déjà à travailler parfois en vue d’un débouché extérieur ; et la marchandise produite, non consommée dans le sein de la famille, doit être transformée en argent par la vente. Cette transformation monétaire du produit n’est que le prélude à la séparation du travail et des moyens de production, c’est-à-dire à la nécessité de transformer l’argent en moyens de production et services d’ouvriers pareillement achetés, pour parvenir à la fabrication des produits... Je dis que la première transformation rend peu à peu la seconde nécessaire, ce qui veut dire que le seul fait de travailler de plus en plus pour un marché extérieur conduit l’ouvrier à n’avoir plus sous la main tous les moyens de production qu’il lui faut, puis à n’en avoir aucun ou presque aucun, pas même ses outils remplacés par des machines. Cela résulte de la \emph{complication du cycle de production totale} par suite de l’\emph{association des travaux} (division du travail) qui fait qu’il n’y a qu’à choisir, pour les travailleurs, entre deux choses : la propriété collective de ces moyens de production (expérience difficile) ou leur propriété exclusive au profit d’un seul, qui prend le nom d’entrepreneur, de patron, de capitaliste. En effet, il n’y a pas de milieu : il ne se peut que chacun des ouvriers ait sa part individuelle de ces moyens de production qui sont impartageables dans un atelier tant soit peu important.\par
Donc, à partir de ce moment, la monnaie (le capital monétaire) permet seule aux travaux de répondre aux besoins.  \phantomsection
\label{v1p366}Et elle remplit cette fonction indispensable en passant de main en main jusqu’à ce qu’elle revienne (non pas la \emph{même} monnaie, mais une monnaie d’égale valeur, ce qui seul importe) aux mains d’où elle est partie. Voilà le cycle monétaire qui s’accomplit pendant que le cycle des travaux opère sa rotation.\par
La monnaie est devenue le rouage principal de ce mécanisme très compliqué que nous appelons la répétition économique, sorte d’horloge dont nous avons cherché à démonter les ressorts et à faire voir les moteurs psychologiques. C’est aussi le rouage le plus rapidement tournant. « La rotation du capital fixe, dit Karl Marx avec raison, englobe plusieurs rotations du capital circulant : pendant que le capital fixe accomplit une rotation (c’est-à-dire s’use entièrement et force le fabricant à le refaire), le capital circulant fait plusieurs tours de roue. »\par
Il y a ici à remarquer deux tendances contraires du progrès économique : l’une au ralentissement, l’autre à l’accélération de la période de rotation monétaire, et il s’agit de savoir laquelle des deux a finalement l’avantage. D’une part la proportion des choses faites sur commande va diminuant sans cesse, comme nous l’avons déjà fait observer, et celle des choses faites à confection va croissant. Ce ne sont plus seulement les vêtements, les chaussures, les chapeaux, les meubles, qui, après avoir été longtemps fabriqués tout exprès pour des personnes déterminées, sont confectionnés pour le public ; ce sont aussi les maisons elles-mêmes. Car on ne bâtit guère plus sur commande qu’à la campagne ou dans les petites villes ; dans les grandes villes, on bâtit de plus en plus par spéculation, à confection en quelque sorte, pour revendre à qui se présentera. Or, ce changement a pour effet de retarder, pour l’entrepreneur, pour le capitaliste qui met ses fonds dans une industrie, l’époque où, par la vente des articles confectionnés, il sera remboursé de ses débours et touchera son bénéfice. Si l’on donne pour  \phantomsection
\label{v1p367}point de départ et pour point d’arrivée au cycle de reproduction capitaliste, les débours de fonds et leur remboursement augmenté d’un gain, il s’ensuit que ce cycle va à la fois s’agrandissant et se ralentissant ; s’agrandissant, car ce n’est plus, par exemple, \emph{une} maison, mais \emph{un groupe} de maisons qui forment le total de l’œuvre entreprise ; se ralentissant, car la rentrée des fonds est d’autant différée.\par
Mais, d’autre part, suivant une remarque de Karl Marx, « les circonstances qui rendent la journée de travail plus productive, comme la coopération, la division du travail, l’emploi des machines, diminuent la période de travail. Par exemple, les machines raccourcissent le temps de la construction des maisons... Le plus souvent, ces améliorations qui raccourcissent la période de travail et par conséquent le temps pendant lequel le capital circulant reste engagé, nécessitent une dépense plus grande de capital fixe\footnote{ \noindent Il y a d’ailleurs, ici, bien des distinctions à faire. L’usure des diverses parties du capital fixe est fort inégale. « Dans un chemin de fer, les rails, les billes, les terrassements, les ponts, les tunnels, les locomotives et les wagons diffèrent au point de vue de la durée de rotation (c’est-à-dire de renouvellement). La rotation des rails et du matériel roulant est la plus rapide : une locomotive doit être renouvelée tous les dix ou douze ans. » (Karl Marx.)
 }... ». Grâce au crédit, l’industrie peut ne pas attendre d’avoir vendu ses produits pour recommencer à produire. Le crédit favorise donc l’accélération du cycle monétaire de l’industrie. En outre, plus l’industrie des transports se développe et progresse, par la multiplication des trains ou des bateaux, par la régularité et la précision croissantes des heures d’arrivée et de départ, et plus s’amoindrit la nécéssité de grands approvisionnements de matières premières, qui immobilisent des capitaux, dans toutes les autres industries. On peut se contenter de provisions d’autant moins considérables qu’on est plus sûr de pouvoir les renouveler au fur et à mesure des besoins de la fabrication. Et, moins il y a de capital immobilisé de la sorte, plus il y a de capital circulant. Toute invention, toute innovation qui augmente la  \phantomsection
\label{v1p368}vitesse des trains ou des navires à vapeur, abrège donc la durée de rotation du travail industriel aussi bien que du travail commercial. De même, toute invention qui substitue au travail manuel le travail machinal ou qui perfectionne les machines, tend, en général, à abréger la production industrielle. Quand, par des procédés de chauffage artificiel, on sèche le linge plus vite, la rotation de l’industrie des blanchisseuses se raccourcit extrêmement. Autre exemple (que j’emprunte également à Marx) : le procédé Bessemer a « raccourci énormément » le temps de fabrication de l’acier.\par
Ajoutons que, plus, pour accomplir un travail donné, on dispose de capitaux monétaires, qui permettent de faire travailler à la fois un plus grand nombre d’ouvriers ou des machines plus puissantes, et moins il faut de temps pour accomplir cette tâche. Il y a donc rapport inverse entre la quantité d’argent mise à la disposition de l’entrepreneur et la durée de son cycle reproducteur. Or, par suite de l’abondance croissante des capitaux et de l’extension générale du crédit, les entrepreneurs ont à leur portée des capitaux toujours plus grands, qui leur donnent les moyens de travailler de plus en plus vite. Karl Marx a bien vu tout cela.\par
Pour toutes ces raisons réunies, il n’est pas douteux que, en s’agrandissant, le cycle reproducteur, et, par suite, le cycle monétaire, s’accélèrent. Et l’on en a la preuve manifeste par l’incroyable rapidité avec laquelle la dernière Exposition Universelle, y compris les deux Palais, est sortie de terre. Si la rotation des besoins, qui est retenue dans des limites naturelles par les lois de la vie et des saisons, ne servait de frein à la rotation des travaux, celle-ci irait se précipitant avec une vitesse extravagante. Il s’agit ici surtout, presque exclusivement, du cycle de la reproduction industrielle. Le cycle de la reproduction agricole, de même que celui des besoins organiques, n’est guère susceptible d’abréviation : tout ce que peut faire la sélection des éleveurs \phantomsection
\label{v1p369} et des jardiniers pour obtenir des variétés de plantes ou d’animaux dont la croissance et la maturité soient plus rapides, est bien peu de chose. Et Karl Marx n’a point tort de signaler le contraste entre l’agriculture et l’industrie, à cet égard, comme propre à expliquer la situation précaire et dépendante des agriculteurs. « Ils ne peuvent, dit Hodgskin (cité par lui), apporter leurs marchandises sur le marché qu’au bout d’un an. Pendant tout ce temps, ils doivent faire des emprunts au cordonnier, au tailleur, au forgeron, au charron et autres producteurs dont ils emploient les produits, achevés en quelques jours ou en quelques semaines. »
\subsubsection[{I.7.h. Karl Marx et la rotation économique.}]{I.7.h. Karl Marx et la rotation économique.}
\noindent — Puisqu’il vient d’être question de Karl Marx, je dois dire un mot de sa manière de comprendre la « rotation économique ». Cela me servira à mieux expliquer ma propre pensée. Sous trois noms différents, il étudie, en réalité, un seul et même cycle. Il s’occupe de la production capitaliste, et, à ses yeux, le capital est un protée qui, au cours de cette production, sans cesse achevée et recommencée, se transforme de monnaie en marchandise et de marchandise en monnaie. Cette production se compose de cinq termes algébriques : A, l’argent sorti de la bourse du capitaliste ; M, les marchandises achetées avec cet argent, c’est-à-dire les moyens de production et la force de travail des ouvriers ; P, la production elle-même, la série des travaux productifs ; M′, la nouvelle marchandise, le produit fabriqué, qui vaut plus, par hypothèse, que M et enfin A′, le prix de vente de ce produit, prix supérieur aux débours. Il appelle la série de cinq termes, le \emph{cycle du capital-argent.} Puis, prenant pour point de départ, cette fois, M au lieu de A et chevauchant sur la série suivante, il aboutit à un nouveau M, c’est-à-dire à de nouveaux achats de moyens de production et de force  \phantomsection
\label{v1p370}de travail, et il appelle cela le \emph{cycle du capital-marchandises.} Enfin, commençant par P, et finissant par un nouveau P, il appelle cette dernière série le \emph{cycle du capital-production.}\par
Ce sont là trois cycles en un seul, n’en faisant qu’un seul, et, par suite, il n’y a pas lieu d’espérer grand’chose de leur rapprochement. Toute l’ingéniosité tenace et subtile de Karl Marx n’est pas de trop pour donner un air de fécondité théorique aux relations de ces trois séries circulaires. Il est évident, que sa seule préoccupation en posant ses équations quasi algébriques est de justifier sa conception du \emph{surtravail} non payé au travailleur, surtravail qui serait la source unique de la \emph{plus-value} du produit par rapport au prix des moyens de production et du travail, de M par rapport à M, et qui constituerait le gain illicite du capitaliste. Tous ses raisonnements à ce sujet sont d’ailleurs d’une logique parfaite si l’on admet sa notion de la valeur, d’après laquelle la valeur de tout produit serait proportionnelle à la quantité de travail humain dépensée à le produire. Malheureusement, ce principe a été reconnu faux, plus tard, par l’auteur lui-même, et il l’est manifestement. Il est permis de regretter que tant de force et de profondeur d’esprit, tant de puissance dans les analyses, ait été mise au service d’un point de vue erroné.\par
Karl Marx a fort bien montré ce qu’il y a de vraiment \emph{cyclique} dans les transformations de l’argent du capitaliste. Cet argent, après avoir été consacré à acheter des produits et des services, d’où résulte une série d’opérations, se retrouve, avec augmentation, par la vente des produits ou des services ainsi opérés. Le premier stade est un achat, le dernier est une vente : aller et retour. Mais Karl Marx semble croire que, en cela, le cas du capitaliste est une singularité unique, qu’il n’y a de cycle monétaire que pour lui, et que, en tout cas, pour lui seul, la monnaie a une tendance à grossir en tournant. Cependant il y a un cycle  \phantomsection
\label{v1p371}monétaire pour l’ouvrier lui-même. De même que le capitaliste, avec l’argent qu’il a déjà gagné et qu’il débourse, achète des matières premières, des machines, des services, et à la fin est remboursé pour \emph{redébourser} encore, pareillement l’ouvrier, après avoir reçu sa paie, la dépense et de nouveau reçoit son salaire qu’il dépense de nouveau. Il en est ainsi pour tout le monde, puisque chacun de nous, riche ou pauvre, à moins qu’il ne soit hospitalisé — aux frais d’un établissement qui lui-même a ses recettes et ses dépenses alternantes — tourne dans un cercle de recettes et de dépenses périodiquement renouvelées. Et, si le cycle monétaire du capitaliste tend à se grossir à chaque tour de roue — tendance, qui, du reste, est loin de se réaliser toujours, comme le prouve le nombre des faillites — on peut dire aussi bien de l’ensemble des revenus, y compris ceux des ouvriers manuels, qu’ils vont progressant en moyenne, comme on en a la preuve notamment par la progression générale et constante des contributions indirectes ; que cette progression est la règle et l’état normal, et que, lorsqu’elle s’interrompt ou rétrograde, tout le monde se plaint à grands cris de cette anomalie, toujours passagère à moins qu’elle ne soit le symptôme d’une irrémédiable et mortelle décadence.\par
— Les trois cycles que je compare, le cycle des besoins, le cycle des travaux (individuels ou collectifs) et le cycle monétaire, ne sauraient se confondre. Ils sont parfaitement distincts, sans nul synchronisme, ils sont non des entités abstraites mais des réalités tangibles. On le voit bien quand, faute de monnaie suffisante, le cycle monétaire se ralentit par force ; le cycle des travaux doit se ralentir aussi, ainsi que le cycle des besoins, non sans des souffrances vivement ressenties. C’est une des causes principales des crises économiques ; nous y reviendrons. Quand, au contraire, par suite d’une découverte de mines d’or ou d’argent, ou d’une période d’heureux trafic international, il y a surabondance de monnaie jetée sur le marché d’un pays, la rotation des  \phantomsection
\label{v1p372}travaux est stimulée, et, par contre-coup, celle des besoins : on ménage moins ses vêtements, ses meubles, on les renouvelle plus souvent, on consomme plus souvent des aliments ou des boissons de luxe. Le danger est que la roue des travaux se mette à tourner si vite que la roue des besoins ne puisse la suivre : d’où une autre sorte de crises. Mais quelquefois, à l’inverse, c’est l’accélération spontanée de la consommation qui stimule la production et appelle de nouveaux capitaux ou permet à la monnaie engorgée de circuler. — Ce cas est rare ; à la vérité, comme je l’ai dit, une infinité de désirs virtuels et endormis préexistent en nous à nos désirs actuels ; en ce sens, les besoins précèdent les travaux qui les satisfont. Mais, au point de vue économique, un besoin ne prend rang qu’à partir du moment où \emph{il s’éveille ;} et nous savons que son réveil est dû, le plus souvent, à l’abaissement du prix des objets propres à le satisfaire, abaissement causé par des inventions nouvelles ou l’application récente et plus étendue d’inventions anciennes. En ce sens, c’est l’agrandissement du cycle de la production, par suite de cet abaissement de prix, qui provoque l’agrandissement du cycle de la consommation.\par
— J’ai distingué de la propagation indéfinie des besoins et des travaux, par une série d’imitations d’individu à individu, série nullement circulaire, la rotation des besoins et des travaux, par l’effet de l’habitude, cette imitation de soi-même. Cette distinction s’applique-t-elle à la monnaie ? Voici un passage de Karl Marx qui semble, à première vue, être une réponse implicite, et affirmative, à cette question. « Le cycle de la monnaie, dit-il (p. 379), c’est-à-dire son reflux vers son point de départ, considéré comme un moment de la rotation du capital, est un phénomène \emph{absolument différent}, et même inverse (?) de la circulation de la monnaie, qui exprime son \emph{éloignement} de son point de départ et son passage par une série de mains. » — Notons, en passant, que le terme de \emph{circulation} employé ici par le traducteur pour désigner \phantomsection
\label{v1p373} un phénomène qui, précisément, n’a rien de circulaire, rien de rotatoire, est des plus impropres ; mais peu importe. Ce qu’il est essentiel de faire observer, c’est qu’on se tromperait grandement si l’on admettait, comme Marx \emph{semble} le supposer ici (peut-être m’abusé-je sur le fond, assez obscur, de sa pensée), que l’argent \emph{tournant} et l’argent \emph{fuyant}, pour ainsi dire, font deux. C’est le même argent, considéré soit en tant qu’il tourne, soit en tant que son cercle se dilate de plus en plus.\par
D’abord, remarquons qu’il ne saurait être question, en tout ceci, des pièces de monnaie ou des billets de banque individuellement considérés. Si l’on considère un écu ou un louis, individuellement, dans ses pérégrinations à travers le monde, il doit être infiniment rare de le voir revenir dans les mains d’où il est sorti ; rien ne doit être plus irrégulier que les zigzags de ses promenades dans le pays où son vagabondage est circonscrit, et d’où il ne sort guère. Mais ces itinéraires sont aussi insignifiants que bizarres ; car une pièce de monnaie n’a pas d’individualité à vrai dire. Toute son essence consiste dans sa possibilité d’échange jusqu’à un certain taux de valeur, et toute autre pièce de monnaie équivalente lui est \emph{monétairement identique.} Il faut partir de là pour comprendre la rotation monétaire.\par
Ceci admis, il est certain que, pour chaque homme ou pour chaque famille, pour chaque société commerciale, pour chaque corporation, pour chaque état, il y a un aller et retour plus ou moins périodique d’une certaine quantité de monnaie, qui entre comme revenu (bénéfice, salaire, rente, honoraires, recettes quelconques) et qui sort comme dépense. Et c’est là le mouvement essentiel et unique de la monnaie, c’est là le cycle monétaire nécessaire et fondamental, dans lequel il faut distinguer d’une part la rotation de ce cycle, d’autre part son élargissement graduel, qui correspond à la propagation imitative des besoins et des travaux, comme nous allons le voir. Chaque fois, en effet, que les bénéfices  \phantomsection
\label{v1p374}d’un négociant se sont grossis par rapport à ceux de l’année précédente, cela signifie ou que la fabrication a eu lieu à meilleur marché ou que le produit a été vendu plus cher, ou que, si les frais de production ou les prix de vente n’ont pas varié, la clientèle s’est étendue. Eh bien, dans ces trois cas, et non pas seulement dans le troisième, l’accroissement de bénéfice — qui rend disponible un nouveau capital pour agrandir la production, — signifie que le besoin satisfait par le produit s’est propagé ou va se propager. Dans le troisième cas, c’est clair. Dans le second, ce n’est pas moins certain, car pourquoi le prix du produit se serait-il élevé (les frais de production, par hypothèse, étant restés les mêmes), si ce n’est parce que le désir auquel le produit répond et le jugement favorable porté sur le produit se sont répandus ? Et, dans le premier cas, est-ce que l’abaissement des frais de production ne va pas avoir pour effet, finalement, en abaissant le prix du produit, d’en étendre le débouché, d’en \emph{éveiller} la demande actuelle\footnote{ \noindent Sans compter que l’augmentation des revenus du négociant s’accompagne aussi, habituellement, d’une augmentation de ses dépenses personnelles, d’une extension de ses besoins personnels.
 } ?\par
Voilà pour les négociants. Il en est de même pour un particulier quelconque, ouvrier, rentier, propriétaire, domestique même. Lorsque le revenu de chacun d’eux s’accroît, c’est, en général, sous l’impulsion de ses besoins qui se sont accrus de quelque besoin nouveau, par suite d’exemples contagieux du dehors, qui ont créé de toutes pièces des désirs nouveaux ou tiré de leur sommeil des désirs inconscients. — Pareillement, quand les revenus d’un État s’augmentent, c’est que les consommations de tout genre, frappées par l’impôt, s’y sont développées, propagées, entre-croisant leurs irradiations imitatives indéfiniment multipliées, et c’est ainsi que les productions de tout genre, imposées aussi, s’y sont agrandies pour correspondre à la demande grandissante des consommateurs.\par
 \phantomsection
\label{v1p375}Par suite, s’il n’est pas vrai qu’il y ait deux masses de monnaie distinctes dont l’une serait en train de se propager de plus en plus loin, et l’autre en voie d’aller et de retour perpétuel, il y a à distinguer de la rotation monétaire l’élargissement même de cette rotation, toujours causé par une propagation de besoins ou d’activités laborieuses. Et cet élargissement implique l’expansion de la monnaie qui va de plus en plus loin avant de revenir à son point de départ. Les choses que nous consommons, à mesure que notre consommation se diversifie et se complique, ne nous viennent-elles pas de producteurs de plus en plus éloignés, et, inversement, nos revenus, à mesure qu’ils s’accroissent, ne nous parviennent-ils pas de mains de plus en plus lointaines, sous forme de coupons ou de dividendes versés par des sociétés anonymes ou autres, dispersées sur le globe entier ? Le domaine géographique où se répand l’ensemble d’une monnaie à l’usage d’un même marché, s’étend ainsi, comme ce marché lui-même, pendant que les fractions innombrables de cette monnaie tournent en cycles inégaux, de plus en plus larges en moyenne, malgré des resserrements individuels et accidentels.\par
De tout ce qui précède, il résulte donc que tous les phénomènes financiers, tous les mouvements et les fonctions de la monnaie, soit comme capital, comme moyen de reproduction, soit comme moyen de consommation et d’échange, procèdent, ainsi que tous les phénomènes sociaux en général, de l’action intermentale grâce à laquelle les exemples se répandent d’abord et s’enracinent ensuite en habitudes ou en coutumes.
\subsubsection[{I.7.i. Le crédit, ses origines. Le prêt, antérieur peut-être à l’échange. Formes successives du prêt.}]{I.7.i. Le crédit, ses origines. Le prêt, antérieur peut-être à l’échange. Formes successives du prêt.}
\noindent — Je pourrais m’arrêter là ; mais j’ai encore à faire remarquer, à propos du capital (\emph{du capital-cotylédon} et non du capital-germe, pour employer notre expression habituelle),  \phantomsection
\label{v1p376}que sa véritable source, dès les plus anciens débuts de l’évolution économique, est un acte de foi et de confiance, premier embryon du crédit, qui est, manifestement, l’âme de la vie productrice des sociétés civilisées. On n’a dit que la moitié de la vérité quand on a vu dans le contrat d’échange le fait économique essentiel et initial. L’échange, à vrai dire, ne favorise et ne développe directement que la consommation. L’agent direct de la production est un autre contrat, non moins initial, non moins fondamental, le contrat de prêt. Par l’échange, on se rend service l’un à l’autre, mais en se défiant l’un de l’autre : donnant donnant ; par le prêt, on se confie.\par
Les prêts d’outils, de denrées, d’objets quelconques, sont usités en tout village arriéré, de même que parmi les enfants. On prête beaucoup plus encore qu’on n’échange parmi les primitifs. Il y a, dans un groupe de familles villageoises, une maison relativement bien outillée, et dix ou quinze maisons pauvres qui l’entourent. Quand les membres de ces maisons dépourvues des ustensiles agricoles ou domestiques indispensables à certains travaux exceptionnels, veulent exécuter ces travaux, ils sont forcés d’emprunter à leur voisin riche ces outils. Celui-ci joue à leur égard le rôle de créancier prêteur de capitaux. Prêt gratuit en apparence, mais, en réalité, reconnu et rémunéré par de fréquents petits cadeaux. Il n’est gratuit en réalité que lorsqu’il est habituellement réciproque ; mais quand, suivant notre hypothèse qui exprime le cas ordinaire, c’est toujours la même personne qui prête ses meubles et ses ustensiles, sans que ses emprunteurs aient jamais rien à lui prêter à leur tour, la coutume exige que ceux-ci, à certaines époques réglées, lui apportent des primeurs de leur verger, des œufs de leur poulailler, etc. C’est là le premier rudiment de l’intérêt des capitaux. Sans ces prêts en nature, une bonne partie du travail producteur n’aurait pas lieu dans les populations non civilisées. C’est donc le prêt, bien plus que l’échange, qui, chez elles, sert à  \phantomsection
\label{v1p377}activer la production. En effet, quelle apparence y a-t-il que, entre deux voisins d’égale et pareille rusticité et simplicité de mœurs, ayant les mêmes besoins et les mêmes occupations, l’un se trouve posséder précisément les outils nécessaires au travail de l’autre, et \emph{vice versa}, de telle sorte qu’il leur soit avantageux de les échanger ? Loin d’avoir un excédent d’outils superflus, aucun d’eux, le plus souvent, n’a tous ceux qu’il lui faudrait. Un seul, le plus riche, les a tous, et celui-là se gardera bien d’en aliéner une partie, indispensable, contre une autre, ni plus ni moins nécessaire. A quoi servirait l’échange en cas pareil ? A ajourner simplement l’emprunt mutuel, le prêt alternatif, qui deviendrait urgent tôt ou tard.\par
Un second stade est atteint quand ce ne sont plus des ustensiles seulement, mais des bestiaux qui sont prêtés. Le prêt de la vache, le prêt de la brebis, est un acte de foi important qui, d’après la coutume des ossètes, nous le savons, donne droit à une partie du croît. — Enfin, quand, au lieu de bétail, c’est une somme d’argent qu’on prête, le pas décisif est fait vers « l’ère capitaliste ». Tout ce qui suit s’en déduit. Si, au lieu de prêter de l’argent à un emprunteur pour qu’il achète des moyens de production et produise, on achète soi-même, avec son propre argent, des machines et des journées d’ouvriers, et qu’on fabrique ce que l’emprunteur aurait fabriqué, est-ce qu’il n’y a pas là un acte de foi aussi, un risque volontairement couru, une espérance, et même un quasi-prêt ? Est-ce que l’argent qu’on engage ainsi dans une entreprise, on ne le prête pas en quelque sorte avec l’espoir qu’il vous sera restitué sinon par ceux auxquels on l’a remis (vendeurs de machines, ouvriers), du moins par les acheteurs futurs de l’article fabriqué ? Toute industrie, donc, de même que tout commerce et tout crédit, est un prêt ou un quasi-prêt ; et c’est là, bien plus que l’échange, le rapport proprement économique entre deux hommes. — Ajoutons que \emph{rendre} est précisément l’inverse et le complément  \phantomsection
\label{v1p378}d’\emph{emprunter}, que ce sont là les deux faces opposées du prêt, et qu’un prêt suivi d’une restitution constitue précisément le cycle monétaire, sur lequel nous nous sommes si longtemps étendu.\par
Ne pourrait-on pas conjecturer, avec une grande vraisemblance, que, à l’origine, l’institution de la monnaie a exigé, avant de s’établir, une dépense considérable d’actes de foi analogues à ceux que nous venons de voir à la base de toute production avec ou sans capital monétaire ? Quand, dans une région d’abord très étroite, dans le cercle d’une ou deux tribus, un article, tel que certains coquillages ou de l’ivoire, commence à jouer le rôle de monnaie, on n’est pas bien sûr, en l’acceptant, de trouver facilement à s’en défaire. Un barbare alors, un sauvage qui se dessaisit d’un bien réel, d’un sac de blé ou d’une pièce d’étoffe pour recevoir, non pas un autre bien propre à satisfaire un de ses besoins actuels, mais cette monnaie embryonnaire, donne une \emph{certitude} de satisfaction immédiate contre une simple \emph{probabilité}, au début assez faible, de satisfaction ultérieure. Il a fallu sans doute autant de force de foi pour accepter ces premières monnaies qu’il en faut aujourd’hui à un banquier pour recevoir le papier d’un négociant, de solvabilité incertaine, à qui il remet son argent\footnote{ \noindent Reste à savoir si, à l’origine, l’acheteur, en donnant sa monnaie en paiement, n’est point resté personnellement garant de l’échangeabilité future de cette monnaie. Ne reste-t-il aucun vertige de cela dans les coutumes des peuples les plus arriérés ?
 }. C’est peu à peu, bien lentement, à coup sûr, que la probabilité de pouvoir, au moment voulu, échanger la monnaie contre une satisfaction quelconque des besoins connus s’est élevée en degré et convertie en certitude ; transformation qui, seule, a fait disparaître entièrement le caractère de \emph{crédit} inhérent au contrat de vente primitif, même au comptant. Mais, sous une nouvelle forme et plus hardie, à partir de ce moment, le crédit tend à renaître, tel que nous le connaissons. Ne nous vantons pas, hommes  \phantomsection
\label{v1p379}modernes, ni nous, ni les hommes civilisés quelconques qui nous ont précédés, d’avoir inventé le crédit. Il est contemporain des plus anciens âges de l’humanité la plus grossière, de la première aube économique. L’échange lui-même, à vrai dire, tel qu’il est donné parfois aux voyageurs de l’observer dans les rapports de tribus ou de peuplades barbares, implique une sorte de crédit. Les marchands déposent leurs marchandises (comme le faisaient les Phéniciens sur les rivages de la Méditerranée) en un terrain neutre, puis ils se retirent. Les acquéreurs s’approchent et, si la marchandise leur plaît, l’emportent en laissant à la place ce qu’ils ont apporté. N’y a-t-il pas ici acte de confiance mutuelle ? La vente au comptant est chose extrêmement rare encore dans nos petites villes, dans nos bourgs ; elle se développe — pour les petits achats au détail, — aux dépens de la vente à crédit, dans tout pays qui se civilise. Ainsi, l’âme de toute industrie, comme de toute religion, comme de tout pouvoir et de tout droit, comme de tout art, c’est la foi. Elle suscite les passions et les besoins encore plus peut-être qu’ils ne la stimulent.\par
Tout ce qui vient d’être dit au sujet du prêt concerne, ne l’oublions pas, le \emph{capital-cotylédon} seulement. Quant au \emph{capital-germe} qui consiste en connaissances d’inventions et de découvertes, il ne saurait être, en toute rigueur, ni prêté ni échangé, puisque celui qui le possède ne s’en dessaisit pas en le communiquant à autrui. Il y a là \emph{émanation}, non aliénation. Il ne saurait être donné non plus, ni volé, pour la même raison. Cependant sa possession exclusive est un bien d’une grande importance souvent, et la perte de ce monopole par le partage de cette possession devenue indivise avec autrui, par une communication à titre onéreux ou à titre gratuit, équivaut à un don, à un prêt ou à une vente. En fait, dans les familles primitives, tout ce capital-germe, à savoir les secrets de fabrication, de remèdes, de poisons, etc., est jalousement gardé, héréditairement transmis, ce qui  \phantomsection
\label{v1p380}montre le sentiment profond qu’on a de son importance majeure. Et, tandis qu’on se prête complaisamment le capital-cotylédon, les outils, le mobilier du ménage, on se garde bien, dans chaque famille, de rien laisser percer en dehors d’elle des inventions séculaires qui lui sont propres et qui s’y perpétuent. Quand ils finissent, pourtant, par se répandre dans les maisons voisines, ce n’est point par vente ni échange, c’est par voie de confidence amoureuse, d’indiscrétion échappée à un Samson dompté par quelque Dalila, ou bien c’est par la violence, par la torture qui arrache ces secrets comme ceux de trésors cachés.
 \phantomsection
\label{v2p1}\section[{II. L’opposition économique}]{II. L’opposition économique}\phantomsection
\label{l2}\renewcommand{\leftmark}{II. L’opposition économique}

\subsection[{II.1. Division du sujet}]{II.1. Division du sujet}\phantomsection
\label{l2ch1}
\noindent Après avoir envisagé les phénomènes de la vie économique sous l’aspect des répétitions qu’ils impliquent, il s’agit de les étudier au point de vue de leurs oppositions. C’est un vaste sujet, confusément traité par les économistes à propos de la concurrence, — notion ambiguë où l’idée de lutte se combine avec celles de concours et de collaboration — et à propos des crises et des grèves, où l’idée de lutte se montre à nu. Il importe, avant tout, de soumettre ce nouveau domaine à une analyse exacte et complète.\par
Ce qui s’oppose, ce ne sont jamais des états, ce sont des forces et des actions ; ou du moins les états ne sont opposables qu’à raison des forces et des actions qui les ont produits\footnote{ \noindent Voir notre livre sur l’\emph{Opposition universelle.}
 }. Quand il s’agit d’oppositions sociales, dont les oppositions économiques font partie, cela signifie — puisque le social n’est que du psychologique multiplié et mutualisé — que les termes opposés économiquement sont toujours, au fond, non des états d’âme, mais des actes d’âme, c’est-à-dire  \phantomsection
\label{v2p2}des affirmations affrontées à des négations ou des désirs affrontés à des aversions, des volitions à des nolitions. Ajoutons que l’idée d’opposition est plus compréhensive que celle de lutte. La lutte suppose que les termes opposés sont coexistants ; mais ils peuvent être successifs, et, dans ce cas, leur opposition constitue un rythme.\par
Cela dit, remarquons que les \emph{reproductions} de richesses, qui sont des actions procédant à la fois de désirs et de jugements, peuvent s’opposer entre elles soit par les propositions soit surtout par les desseins qu’elles impliquent ; que les \emph{consommations}, actions d’un autre genre, impliquant aussi des jugements et des buts, peuvent s’opposer entre elles de ces deux manières ; et enfin que les reproductions peuvent s’opposer aux consommations sous ces deux mêmes points de vue, qu’il conviendra toutefois de confondre le plus souvent, pour ne pas compliquer les questions. Il y a aussi à traiter des oppositions monétaires, financières, où se reflètent et se combinent les précédentes, parvenues à un degré supérieur d’élaboration dans les conflits de la spéculation et des jeux de Bourse. Nous voyons ainsi réapparaître, sous un autre angle, les distinctions qui nous ont déjà servi à embrasser tout le champ de la répétition économique. Celle-ci nous a paru ne pouvoir être étudiée en entier que si l’on y distingue les activités reproductives (le \emph{travail}), les activités consommatrices (les \emph{besoins} exprimés par les budgets), et les signes représentatifs du rapport des produits aux besoins (la \emph{monnaie).} Pareillement, nous sommes conduits maintenant à distinguer les oppositions nées de la reproduction, puis de la consommation, puis des deux, et enfin des finances. — Soit pour la reproduction, soit pour la consommation, nous avons distingué, avant tout, deux sortes de répétitions : 1\textsuperscript{o} une répétition interne, née de l’habitude, de l’imitation de soi-même, ce que nous avons appelé la \emph{rotation} économique ; 2\textsuperscript{o} une répétition extérieure, l’imitation d’homme à homme, par laquelle s’opère la \emph{propagation \phantomsection
\label{v2p3}} des exemples en fait de besoins comme en fait de travaux. Et, dans la sphère monétaire, un reflet de cette distinction s’est montré à nous.\par
De même, à présent, nous avons à distinguer les oppositions internes et les oppositions extérieures. Les premières, source et explication des secondes, sont des combats de désirs ou d’idées dont le champ de bataille est l’esprit individuel, mais qui, tout en étant psychologiques essentiellement, offrent le plus grand intérêt économique, car la valeur des choses en résulte, dans le sens où valeur signifie prix. Toute la \emph{théorie des prix}, autrement dit des \emph{valeurs-coûts}, se rattache donc à l’opposition économique et en est le premier chapitre.\par
Puis viennent les conflits de jugements et de désirs, non plus dans l’esprit d’un même individu, mais entre individus ou entre groupes d’individus différents. Ce sont là les \emph{luttes} proprement économiques qui prennent le nom de \emph{crises} quand elles se produisent sous une forme aiguë.\par
Les oppositions, avons-nous dit, peuvent être des \emph{luttes} ou des \emph{rythmes.} Les luttes présentent un intérêt plus poignant, c’est elles surtout qui arrêteront notre attention.\par
Les reproductions de richesses luttent entre elles, soit parce que des reproductions \emph{similaires} se disputent le même besoin que chacune affirme mieux satisfaire (ce que l’autre nie), soit parce que chaque reproduction tend à développer le besoin qu’elle satisfait, aux dépens des autres besoins que les reproductions \emph{différentes} satisfont. Les affiches commerciales se combattent de la même manière, au fond, et non moins fortement et bien plus continuellement, que les affiches électorales. Il y a aussi à noter les luttes des co-producteurs entre eux pour ou contre la continuation de la reproduction dans des conditions semblables (grèves). Une espèce tout à fait singulière de ces luttes internes de la production, est celle des industries militaires, des fabrications d’armes et d’engins guerriers quelconques, navires de guerre, torpilles, \phantomsection
\label{v2p4} etc. Les consommations luttent entre elles, soit quand deux ou plusieurs besoins différents du même individu se disputent la même somme d’argent à dépenser ou la même force de travail à employer, soit quand des besoins similaires de deux ou plusieurs individus se disputent le même produit propre à les satisfaire. De ces deux formes de la lutte économique en fait de consommation, la première se produit chaque fois qu’une variation s’opère dans un budget de dépenses, privé ou public : si peu que s’élève ou s’abaisse la perte afférente à un article budgétaire, on peut être sûr que des besoins différents se sont combattus. La seconde apparaît sinistrement dans les famines et les disettes, et, normalement, chaque fois que le prix d’un objet augmente par suite de sa rareté.\par
La consommation n’est pas seulement besoin, elle est jugement, remarquons-le. Nous ne consommons aucun produit, aucun service, sans porter une appréciation explicite ou implicite sur le plus ou moins d’utilité, d’équité, d’élégance, de beauté, de cette dépense. Et, en même temps que toute manière de consommer signifie une approbation donnée à ce genre de consommation, elle implique une désapprobation infligée à une autre ou à d’autres manières de dépenser son argent. Il y a toute une éthique et une esthétique enveloppées au fond d’un budget de famille ou d’État, surtout une éthique ; et les révolutions petites ou grandes qui les transforment dénotent toujours des crises morales.\par
Non seulement les reproductions combattent d’autres reproductions, et les consommations d’autres consommations, mais encore la reproduction s’oppose souvent à la consommation et réciproquement. Il y a lutte entre la reproduction et la consommation : 1\textsuperscript{o} quand il y a excès de la première sur la seconde (crises proprement dites) ; 2\textsuperscript{o} quand il y a excès de la seconde sur la première (famines, déficit de n’importe quel article) ; 3\textsuperscript{o} quand la reproduction est en quantité suffisante mais de qualité inférieure aux exigences du  \phantomsection
\label{v2p5}consommateur ; 4\textsuperscript{o} quand, à l’inverse, c’est le consommateur qui est insuffisant, qui ne sait pas se servir du produit, excellent et abondant. Ces deux dernières formes du conflit entre reproducteurs et consommateurs sont bien moins frappantes et bruyantes que les autres, et l’on ne s’en occupe guère. Elles méritent cependant d’être notées. Les journaux, échos des conversations, sont souvent remplis de doléances et de récriminations contre les fabricants de toutes sortes de produits auxquels on reproche non pas d’être trop chers ou trop rares, mais de ne pas répondre exactement aux goûts et aux caprices du public. Si le public veut des fiacres fermés et qu’on lui en offre d’ouverts, qui l’enrhument, s’il veut des chaussures élégantes et qu’on lui en étale de lourdes, s’il juge frelaté du vin qu’on lui affirme naturel, et qui peut-être est naturel ; dans tous ces cas, qui sont innombrables, il y a une opposition qualitative aussi grave que peut l’être une opposition qualitative qui, sous le nom de crise ou de grève, défraie les discussions des savants.\par
On est logiquement conduit, en effet, par l’obsession du côté objectif des phénomènes économiques, sous prétexte de les traiter scientifiquement, à n’entendre l’équation des produits aux besoins — ou, si l’on veut, de l’offre à la demande — que dans un sens quantitatif, et à méconnaître l’importance de leur équation qualitative, c’est-à-dire de leur adaptation, dont les anciennes corporations étaient avant tout préoccupées. Leur grand souci était, à juste titre, que la marchandise fût de qualité meilleure, correspondit mieux aux vœux et aux goûts des acheteurs, tandis que notre préoccupation principale et presque exclusive paraît être que la fabrication augmente sans cesse en abaissant ses prix pour se répandre sur des acheteurs plus nombreux. Je sais bien que les producteurs anciens, par la puérilité fréquente de leurs règlements, visaient plus souvent la perfection qu’ils ne l’atteignaient ; et que nous, au contraire, en poursuivant  \phantomsection
\label{v2p6}la quantité seule des produits, nous obtenons fréquemment leur qualité. Mais nous avons, — et c’est un peu la faute de nos théoriciens, — trop d’indulgence pour ces peccadilles que nos codes dénomment « tromperies sur la qualité de la marchandise vendue » et sur lesquelles nos agents de police ferment les yeux. L’histoire de la probité commerciale, si l’on parvenait à l’écrire, pourrait bien se réduire à l’évolution du mensonge commercial : beau sujet pour un livre à faire. Quoi qu’il en soit, il est certain que, lorsque, à une certaine époque, sur certaines catégories d’articles, souffle un vent de friponnerie courante alimenté par une réclame éhontée, il y a conflit entre le producteur fripon et le consommateur dupé, qui trop tard ouvre les yeux ; et c’est là une \emph{crise qualitative} aussi digne d’attention, malgré son caractère en général moins retentissant et moins alarmant, que les \emph{crises quantitatives} dont on a si patiemment étudié les causes et cherché les lois. Ces crises sourdes, elles aussi, sont-elles périodiques ? Sont-elles aussi \emph{lêgiférables ?} C’est à examiner.\par
Assez fréquent, pareillement, quoique inverse du précédent, est le cas où le défaut d’adaptation entre le produit et le besoin consiste en ce que le consommateur consomme mal, emploie gauchement le produit, le détourne vers des fins autres que sa véritable distinction, ou tout simplement le gaspille. Ce gaspillage va croissant dans les sociétés en train de se civiliser à la moderne, et c’est un des plus graves abus, et des plus inaperçus, de notre état social\footnote{ \noindent Voir le livre spirituel et pénétrant de Novicow sur \emph{les Gaspillages modernes.}
 }. Il arrive aussi fort souvent que des peuples barbares, copistes de nos nations civilisées, ne savent pas porter les vêtements, les chapeaux, les bijoux qu’ils nous achètent, nos oripeaux dont ils s’affublent. Bien plus souvent encore, combien de livres sont mal lus et compris tout à l’envers de la pensée de l’auteur ! Combien de connaissances ingurgitées \phantomsection
\label{v2p7} et mal digérées ! Tout cela rentre dans le 4\textsuperscript{o} des oppositions économiques, ci-dessus distinguées, entre la production et la consommation.\par
Les oppositions quantitatives n’en restent pas moins très dignes de l’attention qu’on leur a prêtées. On peut même dire qu’elles sont beaucoup plus fréquentes qu’il ne semble. Si l’on entend par \emph{quantité de désir} la \emph{somme} des désirs individuels relatifs à un objet ou à un acte, multipliée par l’\emph{intensité} moyenne de ces désirs, on a lieu de penser que la quantité d’un désir de consommation quelconque se trouve bien rarement égale à la quantité du désir de production correspondant. Mais cette équation exclusivement psychologique n’est pas ce qui importe ici ; ce qui importe, c’est que la quantité de choses que le consommateur désire acquérir soit égale à celle des choses que les producteurs désirent vendre. Or, le total des sommes d’argent que le public, à un moment donné, destine à l’achat d’un article est bien rarement égal au total des sommes d’argent que l’ensemble des reproducteurs de cet article souhaite et compte retirer de sa reproduction. Entre ces deux quantités, entre ces deux totaux, il y a toujours un certain écart, sensible ou insensible. S’il est insensible, on dit qu’il y a équation, c’est l’état normal. S’il est senti, il y a crise ou commencement de crise. Mais c’est seulement quand la production est sentie supérieure à la consommation qu’on a l’habitude de dire qu’il y a crise, car c’est une anomalie toujours difficile à faire disparaître et qui ne disparaît jamais sans souffrances. Quant à l’anomalie inverse, qui consiste dans l’infériorité sentie de la production relativement aux besoins de la consommation, elle s’efface d’elle-même et sans douleur s’il s’agit d’articles industriels d’une fabrication extensible indéfiniment au gré de l’homme. Mais, s’il s’agit de produits agricoles, de subsistances, c’est la plus grave des crises, c’est la famine ou la disette. La famine, la disette, c’est la seule crise connue des peuples primitifs. \phantomsection
\label{v2p8} Elle est précisément l’opposé de nos crises industrielles.\par
Outre les oppositions des produits avec les produits, des besoins avec les besoins, et les oppositions des uns avec les autres, il y a les oppositions monétaires, financières. Celles-ci consistent aussi soit en conflits de la monnaie avec elle-même, soit en désaccord de la monnaie avec les besoins et les marchandises. Le premier cas a lieu quand, par suite de l’altération des monnaies ou de quelques-unes d’entre elles, leur valeur nominale est en lutte avec leur valeur réelle, qui la contredit, — ou bien quand deux métaux, l’or et l’argent, liés par un rapport officiel et fixe de valeur, ont entre eux un rapport réel différent, d’où résultent des conflits semblables nés de semblables contradictions. C’est la grande question du bimétallisme. La monnaie est en désaccord avec les marchandises et les besoins, quand la quantité de monnaie en circulation est en excès ou en déficit relativement à la quantité de marchandises produites ou de services offerts et au besoin général d’échange. L’argent est alors plus offert que demandé ou plus demandé qu’offert. D’autres crises financières, plus circonscrites, ont pour cause une épidémie momentanée de faillites, de désastres commerciaux, qui ne sont autre chose qu’un manque de monnaie dans certaines mains d’où l’on attendait une continuation de versements, attente trompée, contradiction douloureuse des espérances par les faits, qui est due soit à des spéculations aventureuses, à des erreurs d’appréciation et d’évaluation de chances, soit à des catastrophes telles que le phylloxera ou le Panama.\par
Nous n’avons parlé jusqu’ici que des oppositions-luttes. Les oppositions-rythmes, très importantes aussi à considérer, consistent dans les alternatives contraires de hausse et de baisse, d’augmentation et de diminution, d’\emph{inflation} et de \emph{dépression} comme disent les Anglais, que présente la production ou la consommation de chaque article, ou de  \phantomsection
\label{v2p9}l’ensemble des articles, y compris la production et la circulation des monnaies, et d’où résultent les oscillations des prix.\par
Après ce coup d’œil jeté sur l’ensemble des oppositions économiques, envisageons-les séparément et commençons, conformément à ce qui vient dit d’être plus haut, par la théorie des prix.
 \phantomsection
\label{v2p10}\subsection[{II.2. Les prix (Théorie de la valeur-coût.)}]{II.2. Les prix \emph{(Théorie de la valeur-coût.)}}\phantomsection
\label{l2ch2}
\noindent Entendue dans son acception la plus strictement économique, l’idée de valeur comporte un sens individuel et un sens social, distincts l’un de l’autre ; et, dans chacun de ces sens, il faut distinguer deux aspects opposés et juxtaposés de cette idée, suivant qu’elle suppose une lutte ou une alliance de désirs et de jugements. La théorie des prix, c’est la théorie de la valeur comprise comme lutte de désirs et sacrifice des désirs moindres à un désir plus fort. Dans ce qui va suivre, nous nous placerons au seul point de vue de la valeur-prix, qui est, d’ailleurs, le sens usuel du mot valeur.\par
Même isolé de ses semblables, l’individu conçoit vaguement une sorte d’idée du prix des choses quand, hésitant entre deux biens qu’il ne peut poursuivre à la fois, il sacrifie l’un à l’autre : il sait alors ce que ce dernier lui \emph{coûte}. S’il risque un danger, celui de périr en mer, pour satisfaire un désir, celui de tuer un phoque, ce phoque lui coûte ce risque, et cela signifie simplement qu’il désire encore plus tuer cet animal qu’il ne redoute la perspective d’être noyé. Un amateur d’art, dans un incendie, voit qu’il n’a pas le temps de sauver à la fois deux tableaux de maître, il emporte l’un et voue l’autre aux flammes. Le premier lui \emph{coûte} le second, ce qui veut dire que le premier \emph{vaut} plus que le second à ses yeux. Supposez maintenant que les deux tableaux sont possédés par deux amateurs différents, et que chacun d’eux désire avoir les deux : si chacun d’eux désire  \phantomsection
\label{v2p11}encore plus avoir celui de l’autre qu’il ne tient à garder le sien, il sacrifiera le sien, et, \emph{à ce prix}, acquerra l’autre. On passe ainsi, le plus naturellement du monde, du sens individuel au sens social du mot prix, du \emph{prix psychologique} au \emph{prix économique}, qui n’en est que la projection extérieure. Le prix économique n’est encore qu’en germe dans l’échange dont je parle ; mais, quand les échanges se seront multipliés et que, par les mutuelles relations des co-échangistes, l’idée d’une monnaie aura pris naissance, la notion de prix, sans en avoir en rien changé de nature, se sera accentuée. Elle continuera à signifier toujours, au fond, concurrence et sacrifice de désirs, dans le cœur de chaque individu.\par
Mais soyons plus précis. L’hésitation de l’individu, précédant son sacrifice et sa décision, a lieu entre deux \emph{syllogismes pratiques} qui s’affrontent en lui. Sans nous en apercevoir nous passons notre vie à faire des syllogismes non classés par les logiciens, dans le genre de celui-ci : « j’ai soif, or, je crois qu’il y a une source dans ce bois ; donc, je dois aller vers ce bois » ; « je rêve la députation, or, je crois que ce journal peut m’empêcher d’être élu, donc, je dois ménager ce journaliste ». Ces syllogismes s’abrègent extrêmement, ils sont le plus souvent implicites : tantôt, c’est la majeure, — exprimant toujours un désir, un but, — qui est sous-entendue, parce qu’elle est habituelle et dominante, telle que le désir du paradis chez le bon musulman ; tantôt on sous-entend la mineure, — qui est une croyance — relative à un moyen d’atteindre le but. Une prémisse ne devient ainsi inconsciente ou subconsciente que si le désir ou la croyance dont il s’agit est très intense : l’avare ne songe jamais à se dire qu’il aime l’argent, ni l’ambitieux le pouvoir ; et, à la vue d’une source, un homme qui a soif ne se dit pas qu’elle est le moyen de se désaltérer. Cela est évident. Mais toujours, la conclusion qui résulte des prémisses, expresses ou implicites, est un devoir qui s’impose à la volonté avec une force proportionnée aux forces combinées  \phantomsection
\label{v2p12}de cette croyance et de ce désir, de ce jugement et de ce but\footnote{ \noindent De là, comme je l’ai dit ailleurs, la singulière puissance inhérente au \emph{devoir moral} proprement dit. Il y a devoir moral quand la majeure du syllogisme pratique étant un but constant et souverain de la conduite et la mineure étant une conviction profonde, on obéit aux deux sans y penser, ce qui donne à la conclusion un air d’\emph{impératif catégorique}, un air absolu et non motivé.
 }.\par
Les conclusions des syllogismes pratiques qui se présentent à la fois étant, en général, de force très inégale, l’hésitation n’est pas longue, d’ordinaire, entre les deux, et la plus puissante l’emporte. La plus puissante, soit parce qu’elle résulte du jugement le plus arrêté quoique du besoin le plus faible, soit parce qu’elle résulte du besoin le plus impérieux quoique du jugement le moins assuré. Entre deux marchandises différentes du même prix, que nous ne pouvons acheter à la fois, et qui se disputent notre volonté d’achat par deux conclusions contradictoires de syllogismes rapides : « tu dois acheter cet article, tu dois acheter cet autre », l’article que nous achetons n’est pas toujours celui qui répond au désir le plus vif, c’est quelquefois celui qui, répondant au désir le moins prononcé ou le moins urgent, nous paraît être de meilleure qualité.\par
Parmi ces innombrables clients qui encombrent le \emph{Louvre} ou le \emph{Bon Marché}, il n’en est pas un, sauf peut-être çà et là quelque milliardaire, qui, avant d’acheter, n’ait eu le cerveau confusément rempli par des chocs de syllogismes, par des duels logiques, analogues à ceux dont je viens de parler. L’achat, pour chacun d’eux, a été le dénoûment d’un de ces duels. — Quand un prix est établi, — nous dirons tout à l’heure comment — le public se partage en deux catégories relativement à une marchandise offerte : ceux qui, au résultat de beaucoup de duels pareils, concluent au devoir de garder leur argent plutôt que de le donner en échange de cette marchandise, et ceux qui concluent à un devoir inverse.\par
 \phantomsection
\label{v2p13}Ces combats de désirs et de croyances en nous sont ce qu’il y a vraiment d’essentiel au fond de l’idée de valeur entendue au sens qui nous occupe. C’est à tort qu’on a voulu la fonder exclusivement sur le fait objectif de l’échange. On dit « valeur d’échange », mais est-ce que, en dehors même de tout échange, il n’y a pas lieu à l’idée de valeur-prix ? Est-ce que le barbare qui suppute les dangers à courir, les hommes à sacrifier probablement, pour conquérir tel château ou tel domaine, n’attribue pas à ce château, à ce domaine, une \emph{valeur de spoliation} en quelque sorte, mesurée à ce que sa conquête lui \emph{coûtera ?} Et, si le régime communiste parvenait à s’installer parmi nous, est-ce que les bons de pain, les bons de viande, les bons de concert, etc., n’auraient pas une \emph{valeur de répartition, appréciée} très inégalement ? Ces bons auraient beau ne pouvoir s’échanger, par hypothèse, on n’en dirait pas moins, ou on n’en penserait pas moins, que tel bon de théâtre vaut deux, trois, quatre bons de trajet en chemin de fer sur telle ligne. — A travers toutes ces transformations sociales, de la spoliation à l’échange, de l’échange à la répartition communiste, l’idée de valeur, l’idée d’\emph{équivalence} aurait conservé quelque chose de commun et de constant : c’est qu’elle signifierait toujours une comparaison entre des \emph{pesées} intérieures de désirs et de croyances.\par
Notons que le fait de l’\emph{équivalence}, de l’\emph{égalité de valeur}, est rarement observé dans la conscience individuelle, où l’inégalité des valeurs est le fait habituel : sans cela on passerait sa vie à hésiter, comme font les aliénés atteints de la folie du doute. Dans le monde social, au contraire, sur un marché, l’équivalence est la règle, toute chose est réputée valoir son prix, le poids d’argent contre lequel on l’échange, ni plus ni moins. Mais, en réalité, tous les acheteurs à ce prix ont acheté parce que l’article à leurs yeux valait plus que ce prix, et tous les vendeurs à ce même prix ont vendu parce que ce prix à leurs yeux valait plus que cet article.  \phantomsection
\label{v2p14}L’équivalence de l’article et du prix n’est que la compensation en quelque sorte de ces inégalités contradictoires, grâce auxquelles chacun des deux contractants, et non pas seulement un seul, ont gagné en faisant l’affaire.\par
Notons aussi que la notion d’équivalence s’étend bien au delà du domaine économique. L’idéal de la justice primitive était le talion, équivalence des préjudices, et c’est une équivalence de préjudices encore que poursuit, au fond, inconsciemment, toute pénalité en tant qu’elle conserve quelque chose de vindicatif. Si l’on songe à la durée séculaire du régime des vendettas et à leur généralité dans le passé, si l’on songe aux \emph{représailles} militaires ou révolutionnaires de tous les temps, peut-être s’apercevra-t-on que ces équivalences de préjudices ont eu une importance sociale plus grande encore que les équivalences de services.\par
\subsubsection[{II.2.a. Deux sens du mot valeur, l’un relatif aux luttes internes, subjectives, d’inléréts, d’où résultent les prix. Problème double : pourquoi prix égal pour tous ? Pourquoi, par le prix, équation entre offre et demande ? Hypothèse d’un marché où la proportion des divers besoins, dans le cœur de chaque candidat à Tâchât, serait la même, et où leurs fortunes seraient égales. Double échelle des consommateurs suivant l’intensité de leur désir ou le degré de leur fortune.}]{II.2.a. Deux sens du mot valeur, l’un relatif aux luttes internes, subjectives, d’inléréts, d’où résultent les prix. Problème double : pourquoi prix égal pour tous ? Pourquoi, par le prix, équation entre offre et demande ? Hypothèse d’un marché où la proportion des divers besoins, dans le cœur de chaque candidat à Tâchât, serait la même, et où leurs fortunes seraient égales. Double échelle des consommateurs suivant l’intensité de leur désir ou le degré de leur fortune.}
\noindent Après ces préliminaires indispensables, abordons le problème capital de la théorie des prix. Ce problème est double : 1\textsuperscript{o} un article ou un service étant donné, pourquoi, sur un même marché, a-t-il un prix égal pour tous ? 2\textsuperscript{o} pourquoi ce prix, s’il est stable, a-t-il pour effet de rendre le nombre des exemplaires offerts de cet article ou de ce service, égal au nombre des preneurs ? La première question, en général, est négligée, comme si rien n’était plus évident que la justice et la nécessité d’un même prix pour tout le monde. Cela est cependant si peu manifeste que, en fait, dans mille occasions, le prix des mêmes choses est inégal pour les diverses classes de consommateurs. Rien de plus juste que cette inégalité, quand on voit les chirurgiens, par exemple, faire payer la même opération plus ou moins cher, suivant la fortune du malade. L’État procède de même, au fond, quand, par la proportionnalité de l’impôt à la fortune  \phantomsection
\label{v2p15}présumée du contribuable, il nous vend si inégalement son même service de protection générale\footnote{ \noindent M. René de Kérallain, critiquant la formule démocratique de l’impôt « chacun doit payer à proportion de ses facultés » remarque non sans une apparence de raison : « On ne comprendrait pas que le boucher, le boulanger taxassent pour nous la viande ou le pain à proportion de notre fortune. A quel titre l’État s’arroge-t-il ici le droit de se conduire autrement ? » En réalité, le boucher, le boulanger, \emph{quand ils le peuvent}, quand ils ont affaire à des étrangers, par exemple, ne se gênent pas pour imiter l’État.
 }.\par
Partout où subsiste l’habitude de marchander, le prix est inégal pour les divers acheteurs, mais les inconvénients du marchandage vont croissant à mesure que les affaires se multiplient et exigent plus de célérité. On en arrive donc inévitablement, passé un certain niveau commercial, à l’usage du prix fixe, publiquement affiché par le vendeur. Concevrait-on que le vendeur publiât plusieurs prix fixes pour le même article, suivant les diverses classes de clients ? Non, et pourquoi ? Parce que la vie urbaine, où le prix fixe a pris naissance, est un perpétuel échange d’exemples, une comparaison incessante de nous avec autrui, qui nourrit et développe en chacun de nous le besoin d’être traité comme les autres, de ne pas payer plus cher que les autres. Nous avons fini ainsi par regarder comme la définition même de la justice cette égalité de traitement, quoique, au fond, il n’y ait rien de moins juste. Et nous sommes tout étonnés quand nous apprenons par M. de Foville que, en plein Paris, de nos jours, « pour les menus articles qu’on n’achète que de loin en loin, les exigences du vendeur arrivent à se proportionner uniquement à l’état de fortune de sa clientèle. Les mêmes balles d’enfants, les mêmes bâtons de cerceaux, qui coûtent un sou aux Buttes-Chaumont, en coûtent deux au Luxembourg et trois aux Champs-Elysées ». Cet auteur nous donne immédiatement la raison de cette étrangeté : c’est que, de la part du marchand de ces petits articles, l’ardeur à les vendre le plus cher possible l’emporte de plus en plus sur le désir contraire de l’acheteur. Et,  \phantomsection
\label{v2p16}plus cet acheteur est riche, moins il résiste à la perte inutile d’un ou deux sous. Il n’en est pas moins vrai que M. de Foville a raison de dire : « Le commerce ainsi pratiqué n’est plus qu’une variété de l’impôt sur le revenu. »\par
L’évolution du prix, à cet égard, est fort nette. A l’origine, il est très divers d’un lieu à l’autre, mais, en chaque lieu, il est très stable d’un temps à un autre. A mesure qu’une société progresse, le prix va s’\emph{uniformisant} mais il est loin d’aller se stabilisant ; il est de plus en plus uniforme et de moins en moins stable. « A New-York, dit M. Zolla, l’indicateur des chemins de fer fait bien connaître l’heure des départs des trains, mais ne dit pas le prix des parcours. Les Compagnies américaines restent maîtresses de leurs tarifs et les changent de temps en temps. »\par
Voilà pour la première question. Arrivons à la seconde. Il y a toujours une certaine marge, quelquefois énorme, entre le prix minimum auquel le marchand pourrait livrer sa marchandise sans désavantage et le prix maximum auquel les consommateurs se décideraient à la payer s’il le fallait. C’est cette marge qui sert de champ aux oscillations du prix. Pourquoi cette marge ? Pourquoi ces oscillations ? Quelle est la loi qui les régit ?\par
Ce qui complique ce problème, c’est, d’une part, la diversité et l’inégalité proportionnelle des goûts et des opinions des individus, d’autre part l’inégalité de leurs fortunes. Mais supposons, pour simplifier, que « sur un de nos marchés, toutes les personnes désireuses d’acquérir un produit aient une fortune égale et que leur désir d’acquérir ce produit ait chez chacune d’elles la même intensité proportionnelle, relativement à celle de leurs autres désirs. Supposons aussi que leur confiance en la qualité de ce produit soit égale. Il en résultera que chacune d’elles sera également disposée à payer, s’il le faut, la même fraction \emph{maxima} de son revenu, soit, par exemple, 100 francs au plus, pour satisfaire cette envie. Pourquoi ? Parce que, entre tous les  \phantomsection
\label{v2p17}désirs divers dont la satisfaction pourrait être obtenue au moyen de cette somme, le plus intense, mais de bien peu, est encore le désir satisfait par le produit en question. Mais, si le prix était sensiblement plus fort, 110 francs par exemple, des désirs plus intenses entreraient en ligne et l’emporteraient sur celui-ci. Ce prix \emph{maximum} s’impose donc, par suite de la \emph{lutte interne} qui a lieu ainsi dans le cœur de chaque acheteur intentionnel. Dès lors, au point de vue de la fixation du prix, qu’importera le nombre des personnes concurrentes, si du moins chacune d’elles ne consulte que son intérêt, suivant le point de vue cher aux économistes ? Qu’il y en ait cent ou un million, le fabricant — à le supposer en possession d’un monopole, pour plus de simplicité — sera averti, s’il connait cette situation morale et pécuniaire, d’avoir à ne pas dépasser 100 francs dans la fixation de son prix. Et il cotera 100 francs à coup sûr. Il produira plus ou moins suivant le nombre des acheteurs prévus, mais le prix ne variera pas\footnote{ \noindent Les parties entre guillemets de cet alinéa sont empruntés à ma \emph{Logique sociale.}
 }. »\par
La concurrence des acheteurs n’aurait donc, dans cette hypothèse, aucune influence sur le prix. Je me trompe, elle en aurait une, et très forte, dans le cas, si habituel dans la vie civilisée, où chacun des acheteurs intentionnels serait renseigné plus ou moins sur le nombre des compétiteurs à l’achat. Il se produirait alors, en chacun d’eux, un avivement de son désir par la contagion du désir d’autrui, phénomène d’autant plus intense que le nombre des compétiteurs serait plus grand et mieux connu et qu’ils seraient plus rapprochés les uns des autres, en contact physique ou social plus étroit. Quand les compétiteurs appartiennent à la même classe, qu’ils se connaissent bien et sont réunis dans la même salle d’un hôtel des ventes, on voit à quelle frénésie d’exaspération mutuelle leurs désirs peuvent monter dans \emph{le feu des enchères}. C’est là une considération \emph{inter- \phantomsection
\label{v2p18}psychologique} qui vient s’ajouter aux considérations de tout à l’heure, et il ne faut pas dire qu’elle s’y ajoute comme un élément perturbateur et accidentel, car cette mutuelle surexcitation des désirs et des goûts semblables est un élément habituel et normal de la détermination des prix. Mais on voit qu’il importe de distinguer soigneusement ici ce qui relève de la psychologie individuelle et de la psychologie collective, deux éléments combinés suivant des proportions extrêmement variables. Quand le second, celui de la contagion réciproque des cerveaux, est presque nul, on peut dire, encore une fois, que le nombre des compétiteurs à l’achat — dans l’hypothèse d’états d’âme semblables et de fortunes égales, d’où nous sommes partis, — n’influe presque en rien sur le prix. Il résulte alors, non d’une concurrence des acheteurs, mais d’une concurrence des goûts, des besoins, des désirs, des jugements, en chacun d’eux.\par
Mais sortons de nos deux hypothèses — quoique cependant on puisse utilement examiner la question de savoir si elles ne tendent pas à se réaliser de mieux en mieux par le nivellement des esprits et peut-être même, somme toute, des fortunes\footnote{ \noindent Il y aurait à distinguer entre les époques et les pays d’\emph{éruption inventive}, où les immenses fortunes jaillissent, comme en Amérique, accentuant les inégalités, et les époques ou les pays de prospérité purement laborieuse, adonnés au développement imitatif de grandes inventions du passé tombées dans le domaine commun. Ici, il y a nivellement croissant des fortunes, et aussi des âmes comme en Chine ; là l’inverse. (Voir à ce sujet \emph{Logique sociale}, p. 361-363.) Ce contraste rappelle celui des formations plutoniques et des stratifications sédimentaires en géologie. Finalement, les forces niveleuses semblent devoir l’emporter.
 }, — et rentrons dans la réalité des faits. Voici un marché composé d’un public où se mêlent les besoins et les caprices les plus divers, où se coudoient tous les degrés de la pauvreté et de la richesse. Le problème alors se compliquera, mais la cause psychologique n’en continuera pas moins à agir et restera la clef de la véritable solution. Admettons qu’il s’agit d’une marchandise dont la production n’est pas extensible à volonté, celle du vin, par exemple, à  \phantomsection
\label{v2p19}fortiori celle d’œuvres d’art ancien, de vieux ameublements, de vieilles reliures, et que le marchand (monopoleur, par hypothèse) se demande quel est le prix maximum auquel il peut coter cet article pour en écouler tous les exemplaires. Ce prix doit être tel que, s’il a 1000 exemplaires à vendre, par exemple, il se trouve tout juste, dans le public de son marché, 1000 personnes tenant plus à acquérir cet objet qu’à garder cette somme. Le prix devra varier considérablement suivant que les personnes du public les plus désireuses de posséder cet article (et les plus confiantes en sa qualité) sont en même temps les moins riches, ou que les plus riches sont les plus désireuses aussi. Il est évident que dans la seconde hypothèse, le prix coté sera beaucoup plus élevé que dans la première.\par
Supposons que le nombre des consommateurs qui désirent posséder cet article soit de 1 million. Ce million se divise en deux échelles de population qu’on peut graduer : 1\textsuperscript{o} suivant le degré d’intensité de leur désir (je laisse de côté, pour plus de simplicité, le degré de leur confiance dans la qualité de l’article) ; 2\textsuperscript{o} suivant le chiffre de leur fortune. On peut figurer, si l’on veut, ces deux échelles, à la manière des statisticiens, sous la forme de deux pyramides de la population — l’une composée de couches numériques superposées, dont la plus basse, la plus nombreuse, 4 ou 500 000 par exemple, est formée de ceux qui ont le moins de fortune, et la plus haute, la moins nombreuse, réduite à l’unité le plus souvent, comprend les gens les plus riches, — l’autre représentant ce public spécial divisé en couches analogues dont les deux extrêmes exprimeraient le nombre des acheteurs les plus désireux ou les moins désireux (avec tous les degrés intermédiaires, bien entendu, comme pour l’autre pyramide). Cela posé, il peut arriver trois cas, et non pas seulement les deux que je viens de prévoir tout à l’heure : 1\textsuperscript{o} le degré du désir des consommateurs éventuels peut être en raison directe du chiffre de leur fortune ; 2\textsuperscript{o} il peut être  \phantomsection
\label{v2p20}en raison inverse ; 3\textsuperscript{o} il peut n’y avoir aucune corrélation, aucune relation ni inverse, ni directe, entre ces deux quantités. Ces trois cas mériteraient d’arrêter l’attention séparément, au point de vue d’une théorie complète des prix.\par
Le second est assez exceptionnel, et le deviendra d’autant plus que les sociétés se mélangeront davantage en se démocratisant : c’est seulement dans une nation dont les diverses classes sont séparées par des cloisons imperméables, qu’on voit le désir de certains articles circonscrit dans un cercle étroit de gens riches ou titrés, les classes inférieures n’osant pas prétendre à de telles consommations. Il n’en est pas moins vrai que, de tout temps, même au nôtre, il est des mets, des vêtements, des meubles à l’usage des classes les plus pauvres, et qui, par l’effet de l’habitude, sont beaucoup plus recherchés là où la pauvreté est la plus grande. On est surpris, dans les enchères publiques des villages, de l’ardeur des personnes indigentes à se disputer certains meubles que leur goût préfère à tous autres et dont les personnes riches ne voudraient à aucun prix. Inutile d’ajouter que ces objets ont beau être utiles et même rares parfois, la combinaison de leur rareté avec leur utilité ne parvient jamais à en élever le prix bien haut, pas plus que l’inutilité de certains articles de luxe, pour des causes précisément opposées, comme nous allons le voir, ne les empêche de se vendre à des prix exorbitants.\par
Le premier cas, celui où le degré du désir dont un article est l’objet se proportionne au chiffre de fortune de ceux qui en ont envie, se réalise à propos de toutes les consommations de luxe, et, par conséquent, tend à se multiplier dans les sociétés en train de s’enrichir, alors même que, en s’enrichissant, elles se démocratisent. Le désir de se distinguer des autres classes par la nature des consommations brillantes et coûteuses attend, naturellement, pour s’aviver, que les besoins d’un ordre plus impérieux soient satisfaits, et il est d’autant plus vif que ces besoins sont satisfaits plus complètement, \phantomsection
\label{v2p21} c’est-à-dire qu’on a une plus large aisance. Or, par le fait même qu’il en est ainsi, ce désir des objets coûteux et à la mode a pour effet de les rendre plus coûteux encore. On s’explique de la sorte les prix extraordinaires de certains tableaux de maître, des tapisseries anciennes, des meubles anciens, des automobiles, des toilettes sorties de certains ateliers, les honoraires prodigieux de certains spécialistes qu’il est de mode de consulter, de certains acteurs ou chanteurs en renom qu’il est de mode d’entendre à chacune de leurs \emph{créations}, etc. Le haut prix du diamant est justiciable de la même explication : c’est certainement dans la classe la plus riche que le besoin d’acquérir cette pierre précieuse est le plus intense, par suite, non plus d’une mode, mais d’une coutume héréditaire, remarquablement tenace.\par
Quant au troisième cas prévu plus haut, — celui où il n’y a aucun rapport, ni direct ni inverse, entre le degré du désir d’un article et le chiffre de fortune des candidats à l’achat — est-il très fréquent ? Non, en toute rigueur, — pas plus, du reste, que ne le sont, au pied de la lettre, les deux autres cas. Il est certain que chaque élévation de notre fortune nous apporte des goûts nouveaux, des besoins nouveaux que nous n’éprouvions guère auparavant, et que, aux divers étages de la fortune correspondent des modes de consommation, en général, assez différents. Cependant le nombre croît des articles d’alimentation, ou de vestiaire, ou d’ameublement même, ou de plaisir, qui deviennent communs à toutes les classes, ou tendent à le devenir. Il y a donc une tendance de notre troisième cas à se généraliser, d’où résultera à la longue une atténuation de l’extravagance ou de l’avilissement de certains prix et aussi bien de certains salaires.\par
Le prix ne s’élève pas, remarquons-le, en raison du nombre des gens qui « ont envie » de posséder l’article, mais bien en raison de l’intensité de leur désir combiné avec le niveau de leurs ressources pécuniaires. Leur nombre ne joue  \phantomsection
\label{v2p22}ici qu’un rôle indirect et secondaire, quand il a pour effet, assez habituellement, pas toujours, d’élargir l’écart entre les fortunes les plus hautes et les plus basses, ou entre les désirs les plus intenses et les plus faibles. Si petit que soit un marché, quand l’inégalité des fortunes y est très grande, les objets de grand luxe y sont cotés très haut ; et, si vaste que soit un marché, dans un pays de démocratie rurale où les conditions se nivellent, l’extravagance des prix exceptionnels s’y atténue. Si de vieux timbres-poste se payent 2 000, 3 000, 4 000 francs, cela prouve non pas qu’il y a une foule de philathélistes qui se les disputent, mais bien qu’il y a quelques philathélistes richissimes — peut-être deux seulement — qui auraient été embarrassés pour satisfaire avec la même somme, s’ils ne l’avaient pas dépensée ainsi, une fantaisie aussi chère ou plus chère à leur cœur que celle-ci, toute légère et frivole qu’elle est. Avant de se décider, l’amateur a vu défiler rapidement devant ses yeux l’image de ses diverses fantaisies à satisfaire, il les a pesées à son insu dans les balances ténues et invisibles de sa conscience, et il a trouvé que la velléité de posséder cette œuvre insignifiante, complément de sa collection, était encore la plus pesante\footnote{ \noindent Une partie de cet alinéa est emprunté à ma \emph{Logique sociale}.
 }...
\subsubsection[{II.2.b Problème qui se pose au marchand monopoleur,et solution de ce problème. Comparaison avec la théorie de l’utilité finale. Cas de la concurrence.}]{II.2.b Problème qui se pose au marchand monopoleur,et solution de ce problème. Comparaison avec la théorie de l’utilité finale. Cas de la concurrence.}
\noindent Mais serrons de plus près le problème que doit résoudre le négociant monopoleur dont nous avons parlé, quand il cherche un prix qui lui permette de vendre le plus cher possible tous les exemplaires, en nombre limité et momentanément inextensible, de sa marchandise. Il se fait ou il doit se faire le raisonnement suivant (en tout cas, les choses s’opèrent à la longue comme s’il se faisait le raisonnement suivant) : J’ai 1 000 exemplaires à vendre ; mon marché se  \phantomsection
\label{v2p23}compose de cent mille personnes dont le revenu oscille entre deux mille francs et un million. Mais mon prix doit être unique, le même pour tous, riches ou pauvres, injustice flagrante imposée par l’état de nos mœurs et la publicité de nos affaires. Que ne puis-je prendre chaque client riche à part, le tenir dans l’ignorance complète du prix auquel son voisin moins riche aurait obtenu ce même article, et lui imposer un prix spécial ! Le prix le plus élevé qu’il consentirait à y mettre serait celui où cesserait presque pour lui la supériorité du désir d’avoir mon article sur les désirs différents qu’il pourrait satisfaire avec cette somme et qu’il doit sacrifier. Il s’établirait en lui, au moment de l’achat, un concours de caprices ou de besoins parmi lesquels le caprice ou le besoin satisfait par mon article l’emporterait de bien peu. — Par malheur, mon prix doit être le même, et il doit être tel que, dans 1 000 cœurs sur 100 000, le concours dont je parle soit à l’avantage du désir particulier auquel ma marchandise répond. Cette lutte ne sera réellement vive que chez les plus pauvres ou les moins désireux de mon article, ou à la fois les moins désireux et les plus pauvres, parmi ces mille. Quant aux autres, plus ils seront riches et désireux (et confiants dans la qualité de l’article), moins les désirs (et les jugements) avec lesquels le désir (et le jugement) en question entrera en conflit lui opposeront de résistance. Pour eux, pas de difficulté ; que je cote l’article quelques francs de plus ou de moins, ils ne m’échapperont pas. Aussi n’ai-je à me préoccuper que des plus pauvres, des moins désireux (et des moins confiants) parmi les mille dont j’ai besoin. Ceux-ci sont sur la limite embarrassante. Chez eux, le désir de mon article l’emportera-t-il encore ou non en intensité sur ceux qu’ils pourraient satisfaire avec le prix, si j’augmente le prix de quelques francs ou de quelques centimes\footnote{ \noindent Tout ce paragraphe est emprunté à ma \emph{Logique sociale}.
 } ? Voilà la question... Mais la réponse, on le voit,  \phantomsection
\label{v2p24}devient très facile, car l’hésitation se trouve circonscrite entre un maximum et un minimum très rapprochés : et les tâtonnements du vendeur ne sauraient être bien longs.\par
Quoique cette manière de voir diffère profondément de celle mise à la mode en ces dernières années\footnote{ \noindent Je dois faire remarquer que la théorie de la valeur exposée ci-dessus se trouve énoncée en ce qu’elle a d’essentiel, dans deux articles que j’ai publiés en \emph{septembre} et \emph{octobre} 1881, dans la \emph{Revue philosophique}, sous le titre significatif de : \emph{La psychologie en économie politique.}
 }, qui explique le prix par ce qu’on a appelé l’\emph{utilité finale}, il y a cependant un point de commun entre les deux, c’est qu’elles s’attachent l’une et l’autre à la considération d’une \emph{limite.} Cette théorie de l’utilité finale, qui a pour objet de faire naître le prix d’une combinaison originale des idées d’utilité et de rareté, consiste en ceci. Je laisse à M. Gide le soin de l’expliquer, avec sa lucidité habituelle. « Prenons l’exemple classique de l’eau. Imaginons la quantité d’eau dont je puis disposer journellement distribuée en 5, 6, 10, 20 seaux rangés sur mon étagère. Le seau n\textsuperscript{o} 1 a pour moi une utilité maxima, car il doit servir à me désaltérer ; le seau n\textsuperscript{o} 2 en a une, grande aussi, quoique moindre, car il doit servir à mon pot-au-feu ; le seau n\textsuperscript{o} 3, moindre, car il doit servir à ma toilette..., etc. ; le seau n\textsuperscript{o} 6, à arroser le pavé de ma cuisine. Supposons que le 6\textsuperscript{o} seau soit le dernier et que, \emph{mon puits ne pouvant en fournir davantage, je ne puisse m’en procurer d’autre ;} je dis qu’en ces cas chacun des seaux aura une certaine valeur, mais que cette valeur ne pourra être plus grande que celle du dernier. Pourquoi ? Parce que c’est celui-là surtout dont la privation peut me toucher. Si, en effet, le premier seau, celui qui devait servir à ma boisson, vient à être renversé par accident, vais-je crier miséricorde en disant que je suis condamné à mourir de soif ?... Il est clair que je ne me priverai pas de boire pour cela ; seulement je suis obligé de sacrifier pour le remplacer un autre seau. Lequel ? Naturellement, \emph{celui qui m’est le moins utile}, le seau n\textsuperscript{o} 6. Voilà pourquoi celui-là détermine la valeur de  \phantomsection
\label{v2p25}tous les autres. » Il ne s’agit ici que de la valeur individuelle ; mais supposons que tous les individus en relation d’échange soient dans le même cas, qu’ils ne disposent aussi que de six seaux, le prix véritable de l’eau correspondra, dans ce cas, à l’\emph{utilité du sixième seau}, c’est-à-dire à l’\emph{intensité du moindre besoin satisfait pour l’eau par chacun d’eux.} — Mais cela ne nous dit pas \emph{quel sera le prix} résultant de cette \emph{utilité finale}. Le problème du prix reste irrésolu.\par
Tout le mérite de cette théorie, comme le reconnaît M. Gide, est d’affirmer que la valeur a son fondement dans le désir. C’est un pas vers la théorie psychique des prix, mais un pas bien insuffisant. — Cette analyse des divers degrés d’utilité afférents aux diverses parties d’un même produit nous indique, à la vérité, l’une des causes qui font varier le désir, mais ce n’est pas la seule, ni la plus importante, ni la plus constante. Elle n’agit pas toujours. Toutes les marchandises, toutes les choses utiles, ne sont pas, comme l’eau, décomposables en parties distinctes applicables à des besoins différents et inégaux, et, ajoutons, inégalement répandus dans le public. Essayez donc d’appliquer cette analyse aux articles répondant à un seul besoin, à telle machine, à tel outil très spécial ; il faudra ici, pour comprendre le taux du prix et les variations du prix, avoir égard non à l’\emph{utilité finale}, dont il n’y a pas lieu de s’occuper ici, mais au plus ou moins de diffusion, dans le public, du besoin dont il s’agit, et au degré de fortune de ceux qui éprouvent ce besoin. Mais, même à l’égard des articles répondant à des besoins divers et inégaux, les considérations dont je parle s’imposeront toujours. — En deux mots, ce qu’il y a de juste, dans la théorie de l’utilité finale, c’est ce qu’elle emprunte à la psychologie ; et ce qu’elle a d’erroné ou d’incomplet provient de ce qu’elle cherche à s’appuyer surtout sur un fondement objectif, et aussi de ce qu’elle ne tient compte que de la psychologie individuelle, nullement de l’inter-psychologie.\par
— Notons que le producteur, le vendeur, hésitant à fixer  \phantomsection
\label{v2p26}son prix, ne connaît jamais avec certitude, au début du moins, l’inégalité des désirs de son public, non plus que l’inégalité de fortune de ce public, et le rapport de ces deux inégalités. En fait, le négociant n’a que des données plus ou moins probables à ce sujet ; il \emph{croit} plus ou moins aux idées qu’il se fait là-dessus. Et son prix est d’autant plus fixe qu’il y croit davantage. Mais l’expérience rectifie bientôt ses erreurs premières, et, par tâtonnements, il parvient à découvrir le prix définitif qui lui est le plus avantageux.\par
— Nous avons supposé que l’article mis en vente est en quantité limitée, momentanément inextensible. Supposons maintenant qu’il s’agit d’un article qui se prête à une fabrication illimitée, comme à peu près tous les articles mis en vente dans les magasins de nouveautés. Le problème qui se pose ici, au commerçant, monopoleur toujours, se complique encore, parce qu’il renferme deux inconnues, \emph{fonctions} d’ailleurs l’une de l’autre : quelle quantité dois-je produire et quel prix dois-je adopter pour réaliser le plus grand bénéfice ? Il est évident — d’après la pyramide des fortunes dont nous avons parlé plus haut, — que, à chaque accroissement de la quantité fabriquée, devra correspondre, si l’on veut que toute cette quantité soit vendue, un abaissement du prix, et que, pour pouvoir hausser le prix, il faudra restreindre la quantité mise en vente. Reste à savoir si le prix devra s’abaisser plus vite ou moins vite, d’après les principes précédemment exposés, que ne s’accroîtra la quantité de marchandises à vendre. S’il doit s’abaisser plus vite, il y aura avantage à restreindre la fabrication ; dans le cas contraire, à l’étendre. Or, pourquoi serait-il nécessaire, par exemple, de diminuer le prix des deux tiers pour pouvoir vendre une quantité double ? N’est-ce pas parce que les couches moins fortunées du public auxquelles il faudrait descendre pour trouver les mille acheteurs nouveaux (en sus des mille déjà obtenus par un prix supérieur) se composent d’individus moins désireux de l’article (ou moins confiants),  \phantomsection
\label{v2p27}en même temps qu’ils sont moins riches ? On voit qu’on revient toujours à une pesée approximative et à une concurrence de désirs ou de jugements, à des chocs intérieurs de syllogismes implicites ou explicites. Il faut se représenter le cœur de chacun de nous, avec les désirs rivaux et tumultueux qui le remplissent, comme une chambre de députés dont chacun tire de son côté le budget de l’État, c’est-à-dire ici la fortune disponible de l’individu. Ce sont les plus forts, \emph{les plus persuasifs}, qui emportent la plus grosse part. Mais ils se font toujours obstacle les uns aux autres, chacun d’eux trouve opposition à sa propre demande de fonds dans les demandes concurrentes de ses collègues.\par
Dans ce qui précède, j’ai supposé un négociant qui exerce un monopole et fixe le prix sans avoir à s’inquiéter du prix fixé par des rivaux ; j’ai donc écarté l’hypothèse vraiment invraisemblable, quoique si chère aux économistes, où, tous les privilèges de situation et autres étant supprimés, la concurrence, entièrement libre, entre les producteurs, entre les commerçants, agirait sur les prix pour les abaisser. Mais, dans cette hypothèse même, la théorie de la valeur ci-dessus indiquée serait-elle inapplicable en ce qu’elle a d’essentiel, et faut-il penser que l’abaissement du prix descendrait, comme l’a supposé Stuart Mill, jusqu’à une limite extrême marquée par le coût de fabrication ou à peu près ? Non. D’abord, puisqu’on est en goût de suppositions, pourquoi n’en pas faire une dernière, et non la moins admissible, à savoir que les producteurs s’entendront dans leur intérêt commun pour se retenir tous ensemble sur la pente d’un abaissement aussi désastreux\footnote{ \noindent Ce qui précède et une partie de ce qui suit est emprunté à ma \emph{Logique sociale.}
 }. Des accords pareils, sous le nom de \emph{trusts} ou sous d’autres appellations, se produisent sur une échelle gigantesque en Amérique, et, ailleurs, sur une échelle moindre mais très vaste encore, pèsent d’un grand poids sur les prix de vente et les salaires. Alors  \phantomsection
\label{v2p28}s’établit un monopole collectif et nous retombons dans le cas précité, celui où le producteur unique fixe lui-même le prix de son produit conformément à la loi de son bénéfice maximum. Au lieu d’un seul producteur, il y en aura 10, 100, 1000, qui, associés, se demanderont ensemble jusqu’à quel point, vu le plus ou moins d’intensité et d’étendue des goûts du public et de ses manières de voir, vu aussi le niveau moyen des fortunes, l’accroissement du nombre des achats, obtenu par la diminution du prix de vente, procurerait une augmentation du bénéfice net. Et, de concert, ils arrêteront le prix à cette limite. Bien entendu, puisqu’il s’agit pour eux, avant tout, de lire dans le fond des cœurs et des esprits, ils se garderont bien de heurter de front les habitudes et les opinions courantes du public sur le « juste prix », en élevant trop haut leurs prétentions, au risque de soulever des protestations et de provoquer peut-être des mesures législatives. Aussi ne suis-je pas surpris de voir, dans le livre de M. Paul de Rouziers sur les Trusts américains, que, loin d’abuser de leur liberté d’action, les directeurs de ces grandes industries monopolisées tâchent de plaire au public par la modération de leurs tarifs.\par
— Cette théorie des prix frappe par son évidence, si on essaie de l’appliquer à l’outillage militaire. Le prix maximum auquel un stock de fusils, de cartouches, un vaisseau cuirassé, peuvent être vendus à un gouvernement, dépend : du besoin relatif, comparé à ses autres besoins simultanés, que ce gouvernement a de s’armer ; de la confiance plus ou moins grande qu’il a dans l’efficacité des armes qu’on lui offre ; enfin des ressources de son budget ou de son crédit. Sans doute, s’il y a plusieurs États à la fois qui ont envie d’acheter ces armes, le vendeur aura chance de les vendre plus cher, mais seulement si, parmi les nouveaux États compétiteurs, il en est un qui a un plus grand besoin de ces armes, une plus grande confiance en elles et plus de ressources financières. D’un seul État, s’il est très riche et aux  \phantomsection
\label{v2p29}prises avec un besoin très urgent, on obtiendra un prix plus avantageux que de trois ou quatre État concurrents mais moins riches et moins pressés par la nécessité d’armements immédiats\footnote{ \noindent Faisons remarquer aussi combien, dans l’exemple choisi, il est clair que la valeur a pour source première l’invention. Toute la valeur d’une arme ne provient-elle pas du cerveau de son inventeur ? Et ne suffit-il pas d’une invention jugée préférable, d’un simple perfectionnement, pour déprécier presque entièrement tous les modèles anciens, tous les anciens types de navire ou de fusil par exemple ? Et si je dis \emph{presque}, c’est parce que, après tout, l’invention récente ne fait jamais qu’amoindrir l’utilité des précédents et supprimer leur excellence ; et le bénéfice de celle-ci reste acquis au public qui, si les fabricants des allumettes chimiques voulaient les vendre trop cher, se remettrait au briquet. C’est là l’explication de la loi de \emph{substitution des besoins} très heureusement formulée par M. Paul Leroy-Beaulieu.
 }.\par
Un mot seulement du cas, exceptionnel, je crois, où la concurrence entièrement libre des producteurs se combattant à armes parfaitement égales, se réalise et a pour effet d’abaisser le prix. Jusqu’où descendra le prix ? Non pas jusqu’au coût de production, mais jusqu’au point où le bénéfice des producteurs serait moindre que les bénéfices qu’ils \emph{désireraient} et \emph{espéreraient} faire en se livrant à d’autres genres de fabrication, moyennant la vente de leurs marchandises actuelles. Il se peut, dans ce cas, qu’ils aient intérêt à fixer bien au-dessous même du coût de production le prix de leur article actuel. Cela arrive tous les ans pour les grandes maisons qui veulent se débarrasser de leurs fonds de magasins, ou pour les commerçants malheureux qui liquident. Ceux-ci, forcés de vendre dans un délai déterminé et très bref, abaissent le prix peu à peu jusqu’à ce que la clientèle amorcée s’approche, le désir d’avoir l’article cessant d’être combattu en elle par des désirs plus forts.\par
C’est, en vérité, extrêmement simple, mais il me semble que, si l’on perd de vue ces idées si tangibles, on ne voit plus rien de clair dans ce sujet si complexe, tandis que, à leur lumière, tout s’explique et se débrouille sans peine.
 \phantomsection
\label{v2p30}\subsubsection[{II.2.c. Dole grandissant du prestige personnel, de la contagion psychologique, dans les marches. Influence de la conversation.}]{II.2.c. Dole grandissant du prestige personnel, de la contagion psychologique, dans les marches. Influence de la conversation.}
\noindent Est-ce tout ? Non. Il y a encore bien des côtés importants de la question dont nous n’avons pas tenu compte. D’abord nous n’avons encore presque rien dit des phénomènes de contagion psychologique qui, intervenant dans les opérations de psychologie individuelle ci-dessus étudiées, ont pour effet de les aiguiller dans une voie qu’elles n’auraient pas prise d’elles-mêmes. Ces désirs et ces croyances que nous avons dit exister dans l’esprit des producteurs ou des consommateurs, et dont nous avons montré le rôle capital comme facteurs du prix, ne restent jamais en contact social inerte, sans agir sur les autres. Il s’opère, de vendeur à client et de client à vendeur, de consommateur à consommateur et de producteur à producteur, concurrents ou non, un continuel et invisible passage d’états d’âme, un échange de persuasions et d’excitations, par la conversation, par les journaux, par l’exemple, qui précède les échanges commerciaux, souvent les rend seuls possibles, et contribue toujours à régler leurs conditions.\par
Le prix étant déterminé par des comparaisons de désirs et des jugements, tout ce qui influe sur ces états psychologiques des acheteurs ou des vendeurs éventuels doit être regardé comme facteur du prix. Or, qu’est-ce qui influe ainsi sur les désirs et les jugements, et, par suite, sur les décisions relatives à la conclusion des marchés ? D’abord, nous le savons, la vue des objets, la lecture de certaines réclames, de certains prospectus, la connaissance de certains faits extérieurs, — qui agissent, non parce que extérieurs, mais parce que connus, toujours plus ou moins inexactement. — Et il y a aussi, ce qu’on oublie trop, des influences d’un autre ordre, plus subtiles et plus profondes, plus décisives souvent, des suggestions de personne à  \phantomsection
\label{v2p31}personne, soit, avant le marché, dans les entretiens familiers, soit, au moment du marché, entre les deux contractants. Ce dernier facteur du prix, essentiellement personnel, joue un rôle manifeste dans les transactions primitives ; et, même de nos jours, il n’a pas cessé d’être visible autant qu’important toutes les fois que le prix ne naît pas fixe, c’est-à-dire ne sort pas tout fait de la volonté en apparence libre, — en réalité soumise aussi à bien des influences personnelles ou autres — du commerçant. Voyez marchander ensemble un paysan et un marchand de moutons, un maquignon et un amateur de chevaux, ou bien un éditeur et un romancier célèbre en discussion sur la vente d’un manuscrit, le patron d’une usine et le chef d’un syndicat d’ouvriers en grève cherchant à s’accorder sur un nouveau tarif d’heures de travail, etc. Ici, il n’est pas douteux que, toutes choses objectives restant les mêmes, le résultat sera différent, parfois très différent, suivant la nature des personnes en présence. C’est par la même raison qu’un traité de paix ou un traité de commerce entre deux nations sera plus ou moins avantageux à chacune d’elles suivant qu’elle aura choisi pour la représenter tel ou tel diplomate. Un propriétaire sait bien que tel de ses domestiques ou de ses métayers excelle à faire de bons marchés, tandis que un autre, non moins intelligent cependant, n’en fait que de mauvais. Il est certain que le don d’être persuasif et convaincant, privilège très mystérieux dans ses sources, se fait sentir avec une force singulière dans les marchés de tout genre, non seulement sur les champs de foire mais dans nos grands magasins même où l’on sait récompenser comme il convient l’aptitude très inégale des employés à enjôler le client\footnote{ \noindent Cet action du commis sur les clients des grands magasins semble n’être pas un facteur du prix, puisqu’ici le prix est prédéterminé. Mais elle peut contribuer à son maintien, à sa stabilité, elle peut contribuer même à le faire hausser plus tard.
 }.\par
C’est par une série de marchandages\footnote{ \noindent Dans ce marchandage, le désir le plus intense, toutes choses égales d’ailleurs, a le plus de chance de l’emporter. De là le préjudice causé aux consommateurs par la prétendue « liberté » de la boulangerie ou de la boucherie. Notons à ce sujet une très juste remarque de Foville. Entre le boucher, et sa clientèle, le marchandage est possible, mais l’ardeur est inégale des deux parts. « Songez que c’est son budget tout entier qui est en jeu, et que cela n’est qu’une partie du budget de l’autre. L’attaque, dans chaque cas particulier, sera donc plus ardente que la défense et nous finirons par capituler ».
 }, de luttes entre des  \phantomsection
\label{v2p32}suggestions personnelles et inégales exercées entre acheteurs et vendeurs, que se forme en partie et s’établit le prix, là où il ne naît pas fixe (à la Bourse, par exemple) ; et j’ajoute que, là où il naît fixe, il n’est en partie que la consécration d’habitudes prises sous l’empire antérieur d’influences analogues, multiples et accumulées. Le taux du prix, le niveau où il se maintient, dépend donc beaucoup du fait accidentel que la supériorité suggestive aura appartenu, dans l’ensemble des transactions, aux acheteurs plutôt qu’aux vendeurs ou à ceux-ci plutôt qu’à ceux-là. Accidentel ? pas toujours. N’y a-t-il pas souvent une raison, aperçue ou inaperçue, tirée du rang social, du degré de culture, de la race, qui prédispose toute une catégorie de contractants, dans ses rapports avec une autre, à posséder une persuasivité supérieure ? Et, dans ce cas, le prix qui s’établira ne sera-t-il pas nécessairement, inévitablement, préjudiciable à l’autre classe ? — A regarder les choses « de haut », d’une manière superficielle et ontologique, on pourrait voir dans le prix des articles une autorité extérieure et impersonnelle, qui, supérieure aux individus, s’impose à eux et les contraint. Mais, en réalité, quand on entre dans le détail précis et explicatif, on voit qu’il n’est point de prix qui n’ait été fixé par quelques volontés dominantes qui se sont emparées du marché. Il suffit d’un groupe de marchands de bestiaux entreprenants et roués qui traversent un pays pour y faire subitement hausser ou baisser sans cause extérieure le prix des bœufs ou des moutons ; il suffit, à la Bourse, d’une élite de haussiers ou de baissiers pour décider du sort d’une valeur. Les prix du blé, cotés à la Bourse de  \phantomsection
\label{v2p33}Londres ou de New-York au résultat du conflit entre deux armées de spéculateurs à la hausse ou à la baisse, commandées par des chefs connus et inégalement influents, font la loi au monde entier. Et dès le début de l’évolution commerciale, il en a été ainsi. Les acheteurs et les vendeurs se sont toujours réglés — dans une région d’abord très circonscrite, puis de plus en plus étendue — sur un prix fixé par la pression prépondérante de quelques gros marchands du voisinage.\par
Dira-t-on que celle influence d’un prestige personnel, d’une fascination spéciale et perturbatrice, sur la détermination des prix, va en diminuant ? Il semble, au contraire, qu’elle grandit avec les moyens d’action, presse, télégraphe, téléphone, que le progrès de la civilisation prête aux individus influents. Le boniment des charlatans de foire ne s’étend pas plus loin que le rayon de leur voix ; celui des charlatans de l’annonce et de la réclame va chercher ses dupes dans une immense région. A la vérité, [{\corr le}] véhicule le plus puissant des influences personnelles, la conversation, exerçait jadis, sous la forme des marchandages entre acheteur et vendeur et des causeries entre clients du même magasin, une action directe sur le prix, qui a été en s’affaiblissant, du moins dans les transactions courantes. Dans un canton rural, c’est après discussions que le prix est fixé ; et, comme là tout le monde se connaît, les acheteurs ne tardent pas à s’informer les uns les autres du prix consenti par les divers marchands de la localité\footnote{ \noindent S’il est vrai que la conversation joue un rôle prépondérant dans la fixation des prix, on en aura la preuve en constatant que le domaine d’un même prix reste en général renfermé dans le domaine d’un même idiome, et que, passée la frontière de deux langues, le prix change plus ou moins. Toutes les conversations qui ont lieu en même temps dans la même langue, peuvent être considérées comme les mailles d’un même tissu, liées les unes aux autres solidement. Les divers \emph{tissus} de conversation qui se juxtaposent ainsi en Europe ont bien aussi un lien entre eux, mais infiniment plus lâche : quelques points de couture seulement. Sans les journaux, qui, échos internationaux les uns des autres, ensemencent des mêmes sujets de conversation, pour une bonne part, tout un continent ou le monde entier en dépit de la diversité des langues, les \emph{tissus} dont je parle seraient absolument détachés les uns des autres et n’auraient rien de commun. Mais cette similitude partielle que la presse établit entre les conversations des diverses langues ne va pas jusqu’aux conversations d’ordre privé, d’où procèdent la plupart des \emph{prix.} Ici la différence des idiomes crée un hiatus difficile à franchir.
 } ; ce prix tend de la sorte à  \phantomsection
\label{v2p34}s’uniformiser dans cette petite région. Mais, plus tard, quand le commerçant s’adresse \emph{au public} et non aux clients \emph{ut singulis}, son prix, publié par les annonces, naît fixe et naît uniforme. S’il ne naissait pas uniforme, d’ailleurs, ce n’est pas par la conversation qu’il le deviendrait, les clients d’un même magasin ne se connaissant presque jamais. La conversation, donc, devient ainsi de plus en plus étrangère, par son action directe, à la fixation et à l’uniformité du prix. Mais il n’en est ainsi que dans les menus achats quotidiens ; dans les grandes affaires, d’où découlent les autres, l’influence personnelle reprend ses droits. Et, même en ce qui concerne les prix courants, la conversation a une action indirecte qui va se développant. Si les hommes civilisés ne causaient pas entre eux, sur des sujets de plus en plus variés, et si le cercle des interlocuteurs possibles, de plus en plus divers, ne s’élargissait pas peu à peu aux dépens de toutes les barrières des langues, des races, des classes, des États, on ne verrait pas les courants de mode se répandre, les mêmes goûts, les mêmes désirs d’achat, se propager sur tout un continent et permettre à la grande industrie et au grand commerce de déployer leurs ailes. Le négociant ne pourrait s’adresser par la Presse à \emph{son public}, c’est-à-dire à une clientèle étendue ayant des besoins semblables, si ce public n’existait pas, et il n’existe que parce que certains caprices, d’abord individuels, sont parvenus à se communiquer d’individu à individu par l’exemple muet ou l’exemple verbal, par ce dernier surtout.
\subsubsection[{II.2.d. L’idée du juste prix. Son action sur le prix réel. Le juste prix d’une invention.}]{II.2.d. L’idée du juste prix. Son action sur le prix réel. Le juste prix d’une invention.}
\noindent C’est surtout en contribuant à former, à préciser, à généraliser l’idéal du \emph{juste prix} que la conversation agit indirectement \phantomsection
\label{v2p35} sur le \emph{prix réel} des choses. Alimentée incessamment et attisée par les journaux, par les livres, par les théâtres, par la vie, elle donne naissance à l’opinion publique, qui oppose aux prix, aux salaires existants, les prix, les salaires qui \emph{devraient exister}, conformément à certains principes en vogue. Or, l’idée du juste prix, qui a plané de tout temps au-dessus des marchés, a toujours exercé sur le prix réel une puissance attractive qui tend à amoindrir l’écart des deux. Cette attraction peut se comparer à celle du \emph{droit naturel} sur le \emph{droit civil} à Rome ; à cela près qu’elle s’est exercée aussi souvent dans un sens conservateur que dans un sens progressiste. Là où règne le respect filial ou superstitieux de la « sagesse antique » le prix ancien tend à paraître plus juste que le prix nouveau. De là l’action, signalée par Stuart Mill, de la coutume sur le prix. Loin d’être \emph{perturbatrice}, comme elle l’est aux yeux des économistes persuadés que le prix soi-disant déterminé par l’offre et la demande est seul normal, elle doit passer pour l’influence la plus légitime parmi des populations régies par un idéal archaïque et coutumier. Au moyen âge et jusqu’au {\scshape xvi}\textsuperscript{e} siècle, tout ce qui, en fait de prix ou de salaires, s’écartait des habitudes séculaires, était réputé arbitraire et injuste, et l’idéal consistait à revenir le plus possible au passé. Au contraire, à une époque comme la nôtre, éprise de nouveauté, le juste prix, c’est le prix de demain, celui où semble tendre la progression de certains tarifs ou la diminution de certains autres. Dans le premier cas, l’idée du juste prix agit donc comme un frein qui s’oppose aux changements ; dans le second cas, comme un aiguillon qui les accélère\footnote{ \noindent Dans la \emph{Revue d’Économie politique} d’avril 1899, je lis le résumé d’un article allemand sur le \emph{mouvement des prix} au Japon. L’auteur est surpris de constater que « malgré l’accroissement de la richesse et l’état stationnaire du stock monétaire, les \emph{prix haussent} », tandis que, d’après les principes classiques, ils devraient baisser. — Le critique de la \emph{Revue d’économie politique}, répond, il est vrai que cela peut tenir à ce que « le Japon s’incorpore de plus en plus à l’économie mondiale... de sorte que la hausse des prix s’expliquerait par le fait que le stock monétaire des autres nations vient s’ajouter au stock monétaire national offert en échange des marchandises de son marché » (p. 765). Mais cette hypothèse est toute gratuite et contredite par les faits. La monnaie européenne ne concourt pas avec la monnaie japonaise pour l’achat des céréales japonaises : or, les céréales japonaises ont haussé de prix au Japon quoique la « culture se développe dans une proportion importante ». Une seule explication est possible : c’est que les Japonais, \emph{pour les prix} comme pour tout le reste, se sont mis à la mode européenne autant qu’ils ont pu. — Autre exemple. Je lis dans une publication de l’\emph{Office du travail} que la suppression du droit d’entrée de 7 francs prélevée sur les blés étrangers importés en France n’a eu pour conséquence une diminution du prix du pain que dans quelques villes et au bout de quelques mois. Comment expliquer autrement que par le pouvoir de l’habitude, de la conformité à l’usage, \emph{aux précédents}, cette extraordinaire continuation de prix exagérés portant sur une denrée de première nécessité ?\par
 L’ouvrier d’autrefois ne désirait habituellement que le salaire réglé par la coutume, le salaire uniforme et toujours le même. Ou du moins, il croyait que celui-ci était le juste prix, même en en désirant un autre. Il en était de même de son petit patron, dont le bénéfice n’était pas moins stationnaire et traditionnel. — A présent l’ouvrier \emph{désire le salaire le plus fort possible} et considère que le juste prix est ce \emph{prix maximum.} — Et de qui a-t-il appris à concevoir ce désir et cette croyance ? Il l’a appris en voyant son patron, son grand patron, \emph{désirer les bénéfices les plus grands possibles} et se persuader qu’il y a droit.
 }.\par
 \phantomsection
\label{v2p36}C’est surtout sur le prix des services, sur les salaires et honoraires de tout genre, que l’influence du juste prix sur le prix réel se fait sentir. Elle a eu certainement pour effet, dans notre société égalitaire, de faire baisser les gros traitements et hausser les petits. M. Paul Leroy-Beaulieu remarque très justement, que le salaire est plus variable d’un lieu à l’autre et plus fixe d’une époque à l’autre que le prix de n’importe quel article. Remarquons, en passant, la singulière importance de ce fait. Le \emph{marché des services}, disons-nous, est toujours moins étendu et va s’élargissant moins vite que le \emph{marché des produits ;} et il en est ainsi, d’abord, parce qu’il est bien plus facile aux marchandises de se déplacer qu’aux hommes de changer de domicile. Il y aura donc toujours plus d’inégalité entre les salaires des ouvriers des divers pays qu’entre les prix des marchandises. Par suite, les pays où les rémunérations quelconques sont le plus élevées, sont exposés à être inondés de produits étrangers, \emph{à moins que} le plus haut prix du travail ne soit compensé par sa productivité plus grande ou que les barrières \phantomsection
\label{v2p37} de douanes n’arrêtent les marchandises extérieures aux frontières. Mais la première de ces deux exceptions ne saurait être que passagère : comment croire que la productivité supérieure du travail ira toujours croissant, en Angleterre ou en Amérique, autant que le prix de la journée d’ouvrier ? Il ne restera donc d’autre moyen de combattre l’invasion étrangère que la protection douanière — qui elle-même est de plus en plus repoussée par l’assimilation internationale. De là un problème des plus anxieux.\par
Mais demandons-nous quelle est l’explication du fait signalé par M. Leroy-Beaulieu. La raison en est, dit-il, que le travail « cette marchandise d’un caractère particulier », est influencé « non seulement par des motifs d’ordre économique » mais encore « par des motifs extra-économiques ». C’est reconnaître l’importance des notions et des conceptions idéales dans la formation des prix et la nécessité d’y avoir égard. — C’est reconnaître aussi la prédominance de l’esprit de coutume, même à notre époque de crise. Le même auteur dit ailleurs : « Pour le travail et sa rémunération il faut souvent tenir compte d’un élément particulier important \emph{qui ne joue aucun rôle pour les autres marchandises :} l’élément moral ou éthique. » Il serait plus exact, je crois, de dire que cet élément joue un rôle plus grand dans la fixation des salaires que dans celle du prix des marchandises. D’ailleurs, comment nier l’action de l’idée que chaque époque ou chaque pays se fait sur la justice en fait de prix ? A quel genre de consommation la morale est-elle donc tout à fait étrangère, si l’on entend par morale la règle supérieure et profonde de la conduite en vertu des convictions et des passions majeures qui mènent la vie ? Et, si l’on fait abstraction de ces convictions et de ces passions dominantes, qui, sourdes ou conscientes, sont les forces sociales et individuelles par excellence, qu’explique-t-on en économie politique ?\par
Le \emph{trust} qui a monopolisé l’industrie du pétrole en Amérique, \phantomsection
\label{v2p38} \emph{Standard Oil Company}, a fait baisser le prix des pétroles, au lieu de le faire hausser. Pourquoi ? Est-ce la \emph{concurrence} ici, — ce que les économistes appellent ainsi, c’est-à-dire la rivalité d’établissements différents — qui a eu ce résultat, si excellent pour le consommateur ? Non, puisqu’il n’y avait plus de rivaux à redouter. Mais il y avait à craindre l’\emph{opinion publique}, animée contre les trusts, et qui aurait jugé révoltante leur oppression ; et la menace \emph{de lois} contre ces associations a été l’obstacle intérieur, l’\emph{obstacle psychologique}, qui, mieux que la concurrence extérieure, a eu cet heureux effet. — Ainsi, c’est le pouvoir législatif, ici, qui tient en échec les monopoleurs. Il n’est donc pas vrai que l’intervention des règlements législatifs dans la fixation du prix ou dans la production industrielle soit inutile ou illégitime. Il n’est pas vrai que de soi-disant « lois naturelles » agissant providentiellement dans l’intérêt du public dispensent de faire des lois civiles et soient toujours préférables à celles-ci.\par
On a beau dire, avec J.-B. Say, que « la valeur \emph{naturelle} d’un service est le prix auquel on pourrait l’obtenir s’il était livré à la plus entière concurrence », on ne peut méconnaître que le prix déterminé par les enchères publiques n’est nullement le prix le plus juste ni même toujours le plus avantageux pour le public. Si toutes les fonctions publiques étaient données aux enchères, beaucoup d’entre elles seraient gratuites ou ridiculement rémunérées, ce qui en exclurait les gens sans fortune, souvent les plus dignes, et exposerait les titulaires à de dangereuses tentations.\par
Les économistes, en considérant comme le prix \emph{naturel} ou \emph{normal} le prix auquel aboutit la concurrence la plus libre, la plus effrénée, ont cru éliminer de la sorte l’idée gênante du \emph{juste} prix. Mais, en réalité, ils n’ont fait que \emph{justifier} ainsi les prix réels précisément, les plus abusifs souvent, formés sous l’empire tyrannique du plus fort. Et le malheur est que cette manière de voir, qui est elle-même, au fond, [{\corr une}]  \phantomsection
\label{v2p39}manière inconsciente de concevoir le juste prix tout en le niant, exerce, par là, une certaine action, non des moins regrettables, sur le prix réel. Quand tout le monde est persuadé, sur la foi d’anciens économistes, que le prix automatiquement déterminé par « le libre jeu de l’offre et de la demande » est la justice même, il n’est pas douteux que cette croyance générale contribue à laisser s’établir sans protestation, voire même avec l’assentiment général, des prix exorbitants ou des prix infimes que la conscience publique eût repoussés en d’autres temps. Les honoraires de certains praticiens, les bénéfices de certains négociants, n’ont pu s’élever comme ils l’ont fait qu’à la faveur des théories économiques régnantes ; et, inversement, les salaires de certaines industries, — à une époque déjà éloignée de nous, en Angleterre notamment, entre 1830 et 1860, — se sont abaissés, par suite de la même cause, plus qu’ils ne seraient descendus sans la vulgarisation de ces idées. — De même, si tout le monde était convaincu, avec Karl Marx (première manière) que le juste prix des marchandises se mesure au nombre d’heures de travail employées à les fabriquer, on verrait les prix tendre à se rapprocher de cet idéal, tout contraire qu’il est aux tendances générales\footnote{ \noindent A tort on m’objecterait que l’idée du juste prix est incompatible avec la notion toute subjective de la valeur des choses. Dans les pays, il est vrai, et aux époques où les prix ne changent presque pas, le prix de chaque chose, sa valeur devient une qualité qui lui est inhérente aux yeux de tous comme son volume ou son poids, quelque chose d’\emph{objectif} essentiellement. Aussi Ashley remarque-t-il que la valeur, aux yeux de saint Thomas d’Aquin, avait ce caractère : elle était chose « attachée à l’objet, existante par soi-même, qu’on le voulût ou non, dont on devait reconnaître la réalité intrinsèque. » La doctrine du \emph{juste prix} était toute pénétrée, en droit canon, de cette illusion réaliste. Mais est-ce à dire que, au point de vue, seul vrai, de la subjectivité de la valeur, la notion du juste prix perde tout fondement ? Non. C’est ainsi que l’idée d’une harmonie des couleurs ne perd rien de sa vérité parce qu’on a découvert que la couleur est chose purement subjective.
 }.\par
Il importe donc grandement d’avoir égard à la formation de l’idée du juste prix, puisqu’elle est un des facteurs  \phantomsection
\label{v2p40}essentiels du prix réel. Pourquoi cette idée est-elle si obsédante ? Pourquoi est-il impossible, quoi qu’on fasse, de l’éliminer, soit en la proscrivant comme une chimère métaphysique, soit en la confondant avec la réalité du prix stable toujours, au fond, imposé par le fort et subi par le faible ? N’est-ce pas parce que le sentiment de la sympathie de l’homme pour l’homme naît des contacts mêmes qui mettent en lutte l’homme avec l’homme, et s’alimente de tous les rapports de la vie sociale ? Si les amours-propres et les orgueils s’affrontent, si les intérêts s’opposent, les sensibilités s’harmonisent, en se rapprochant ; et c’est par la conversation surtout, par la transfusion mutuelle des états d’âme, que ce rapprochement a lieu. De là cette habitude constante de se comparer entre eux, de se mirer mentalement les uns dans les autres, qui fait que les hommes sont à la longue possédés par l’idée que les avantages d’une affaire doivent être partagés également entre les parties contractantes. Ce partage égal des avantages, ou aussi bien des désavantages, est un optatif majeur que suggère nécessairement la vie de sympathie imitative. Le négociant ou l’industriel le plus égoïste, au moment où il va imposer un prix ou un salaire inique à quelqu’un qui sera forcé d’y consentir, ne peut s’empêcher de songer à l’appréciation de sa conduite par un spectateur impartial. Or, le prix est regardé comme juste par un spectateur impartial lorsqu’à ses yeux les deux parties contractantes trouvent dans l’affaire, en échange d’un service égal ou d’une peine égale, un égal avantage, c’est-à-dire la satisfaction de désirs également intenses, quoique dissemblables, ou l’assurance inégale de satisfaire des désirs inégaux mais de telle manière qu’il y ait compensation entre l’assurance plus grande d’un désir moindre et l’assurance moindre d’un désir plus fort. Je ne prétends pas que le spectateur impartial ait conscience du calcul psychologique que je lui prête, mais il le fait sans s’en douter.\par
 \phantomsection
\label{v2p41}Comme exemple de la compensation dont il vient d’être question, citons le salaire de l’ouvrier comparé au bénéfice de son patron. L’ouvrier est certain de toucher son salaire hebdomadaire ou mensuel, le patron n’est pas sûr de toucher son bénéfice, simplement probable. Aussi serait-il injuste que le bénéfice espéré de celui-ci ne fût pas supérieur au salaire assuré de celui-là ; mais la distance entre les deux, pour que la justice soit respectée, doit diminuer à mesure que l’espérance du patron se rapproche davantage de la certitude de l’ouvrier, et, quand le patron est à peu près aussi sûr de son bénéfice que l’ouvrier de son salaire, ce qui est le cas de beaucoup de petites industries rurales, la justice exige que ce salaire et ce bénéfice soient à peu près égaux, comme il arrive d’ordinaire. Inversement, à mesure que le bénéfice de l’entrepreneur devient moins certain, c’est-à-dire à mesure qu’on s’éloigne des conditions de la fabrication primitive et qu’on avance dans la voie de la grande, de la plus grande industrie, travaillant pour une clientèle de plus en plus vaste, de moins en moins personnellement connue et fidèlement attachée, il est juste que l’écart aille en grandissant entre le bénéfice aléatoire visé par le patron et le salaire régulièrement touché par l’ouvrier. — Est-ce à dire que cette progression doive se poursuivre toujours ? Non, car il est à remarquer que, passé une certaine limite, l’agrandissement du marché, au lieu d’augmenter l’incertitude du grand industriel, diminue son aléa, par le progrès des informations de tout genre, des statistiques, des nouvelles télégraphiques, des signes extérieurs qui finissent par ne plus laisser de doute, dans beaucoup de cas, sur l’étendue et l’intensité de la demande dans un marché donné, et, par suite, sur le chiffre du gain futur. S’il est vrai qu’il en soit ainsi et que le prolongement de notre évolution économique doive ramener les chefs des grandes industries monopolisées, des trusts américains ou autres, à cette quasi certitude du bénéfice attendu qui caractérisaït \phantomsection
\label{v2p42} l’artisan du moyen âge, la justice exigera que ces géants de la machinofacture cessent d’engloutir des gains exorbitants. Aussi devront-ils, pour ne pas révolter la conscience publique, mettre un frein volontaire et prudent à leurs appétits et se contenter — comme ils le font déjà en Amérique — d’un bénéfice \emph{raisonnable}, c’est-à-dire pas trop cyniquement léonin.\par
Bien entendu, il ne saurait être question d’abaisser jamais, sous prétexte de justice, le bénéfice du chef d’industrie de l’avenir au niveau du salaire de ses ouvriers, comme s’il s’agissait d’un forgeron de campagne qui travaille avec un apprenti. Nous savons, en effet, que le chef d’industrie moderne est souvent, en même temps qu’un surveillant de travaux et un capitaliste, une sorte d’inventeur. Comme surveillant, il peut n’avoir droit qu’à un salaire égal au salaire moyen de ses ouvriers ; mais, comme capitaliste, il a droit à un prélèvement d’intérêt ; et, comme inventeur — si du moins il peut être réputé tel à quelques égards — il peut légitimement prétendre à une rémunération tout à fait à part. Mais il faut reconnaître que, le plus souvent, il exploite les inventions d’autrui, et que la seule idée qui lui appartienne en propre, celle de les avoir importées ici ou là, dans telle ou telle condition, est assez simple, sinon assez facile à réaliser.\par
Quant à l’inventeur proprement dit, il est fort mal aisé de dire quel est le prix de son invention. C’est là un des points les plus délicats et les plus importants de la théorie de la valeur. — N’oublions pas qu’une invention n’est, après tout, qu’un entre-croisement d’imitations différentes qui se sont fécondées mutuellement dans un cerveau ; un inventeur a donc eu pour collaborateurs tous les auteurs des inventions élémentaires qu’il combine en une conception nouvelle, et, à vrai dire, tout le genre humain, immense taillis sans lequel ces baliveaux épars ne seraient pas sortis de terre. Par suite, l’homme de génie serait mal venu à prétendre  \phantomsection
\label{v2p43}échanger le secret de sa découverte contre un prix égal au montant total des richesses que l’exploitation de son idée vaudra à l’humanité, sa mère et sa collaboratrice.\par
A vrai dire, il est impossible de trouver un fondement objectif quelconque au prix d’une invention qui vient de naître. Que valait l’invention de Watt au moment où il l’a conçue ? Quel était son juste prix et quel était le prix maximum qu’il en pouvait demander ?\par
On ne peut répondre, évidemment, en se fondant sur le nombre d’heures de travail, — inconnu de Watt lui-même, d’ailleurs — que ce grand homme aurait employées à découvrir le principe de sa machine. On ne peut répondre davantage en se fondant sur l’étendue — encore conjecturale — des services que rendra cette idée, ni sur la durée — non moins problématique — de ces services. On ne peut trouver les éléments d’une réponse — car il en faut une — que dans l’état d’âme de l’inventeur et celui des compétiteurs à l’achat de son invention. Le prix maximum qu’il pourra obtenir d’eux dépendra du degré de la foi qu’ils ont dans les bénéfices à tirer de l’invention offerte, combiné avec l’intensité du désir d’acquisition qu’ils éprouvent, et avec la fortune dont ils disposent. Mais ce prix maximum sera supérieur au juste prix, la demande de l’inventeur sera excessive et injuste si, par hasard, en son for intérieur, il a une foi beaucoup moindre dans le succès et l’utilité de son idée et un désir beaucoup moindre de la retenir en sa possession exclusive pour l’exploiter soi-même. Un spectateur impartial, pouvant, par hypothèse, lire dans les âmes des contractants, dira alors que l’inventeur a abusé de la situation. Le juste prix, ici manifestement, comme partout en réalité, n’est définissable qu’en termes psychologiques. Et il en est de même du prix réel, que l’idée du juste prix sert en partie à déterminer.\par
Il s’agit toujours, dans un cas comme dans l’autre, de mettre en balance des poids de désir et de croyance semblables \phantomsection
\label{v2p44} ou différents, égaux ou inégaux. Seulement, il s’agit tantôt d’une comparaison à établir entre les désirs ou les croyances dissemblables d’un même individu, tantôt de comparer des désirs ou des croyances semblables d’individus différents. Quand un vendeur privilégié, ou un syndicat de vendeurs, gouverne le marché, et ne se laisse guider que par l’égoïsme pur, abstraction faite de tout sentiment sympathique, dans la fixation du prix, il se décide en vertu d’une pesée approximative des doses de croyance et de désir qu’il suppose exister à la fois dans le cœur de l’acheteur qui, vu sa fortune, se trouve \emph{sur la limite} dont nous avons parlé plus haut, et il demande le prix le plus fort que ce dernier sera disposé à donner, c’est-à-dire un prix tel que le désir de retenir cette somme d’argent soit presque égal en lui au désir de posséder l’objet en vente. Mais, quand le spectateur impartial intervient, il juge cette demande abusive parce que, comparant, lui, les désirs et les croyances du vendeur avec les désirs et les croyances de l’acheteur, il regarde comme juste le prix qui donnerait une satisfaction égale aux deux.
\subsubsection[{II.2.e. Impossibilité de rendre compte des prix par des conMiléi ations simplement objectives.}]{II.2.e. Impossibilité de rendre compte des prix par des conMiléi ations simplement objectives.}
\noindent C’est assez parler du juste prix, dont les variations multicolores et l’influence si étrangement inégale sur le prix réel nous entraîneraient trop loin. Il suffit d’avoir indiqué la clef d’explication qui s’applique à tous ces changements. Mais je ne puis terminer cette exposition de la théorie psychologique des prix sans avoir fait toucher du doigt, par un exemple, encore une fois, l’impossibilité de rendre compte des prix réels par des considérations purement objectives. Nul ne peut nier l’influence des ventes à terme, à découvert, faites à la Bourse par des spéculateurs financiers, sur les prix de vente du commerce. C’est là un des grands facteurs du prix des céréales, aussi bien que du coton, des lainages, des  \phantomsection
\label{v2p45}matières premières en général. Or, ici, que signifie la loi de l’offre et de la demande, dans la mesure vague où elle s’applique ? Elle signifie que le prix des marchandises est déterminé non pas par les quantités réelles des marchandises offertes ou demandées, mais par leurs quantités \emph{supposées}. Ce n’est point l’insuffisance ou la surabondance réelle de la récolte du blé, dans une année et dans une région données, qui fait hausser ou baisser le prix du blé, c’est l’opinion répandue dans le public, le plus souvent par des mensonges de Presse financière, et surtout par des manœuvres frauduleuses de Bourse, relativement à cet excès ou à ce déficit. Il se vend dans les Bourses de commerce, souvent, en quelques semaines, sous forme de marchés à terme, des quantités de blé imaginaires six ou sept fois plus fortes que la quantité de blé réellement emmagasinée. Et cette quantité réelle est, pour ainsi dire, noyée dans le flot du blé chimérique qui contribue à la détermination de son prix. Le prix du blé réel est à la merci du blé imaginaire dont le prix est fixé, avant la venue même du blé réel, escompté d’avance suivant le miroitement de ses mille degrés de probabilité spécieuse ou fictive, par les spéculateurs des principales Bourses du monde. En général, s’il faut en croire des spécialistes\footnote{ \noindent Voir dans la \emph{Revue d’économie politique} (1898) plusieurs articles intéressants de Charles-Williams Smith sur \emph{la spéculation internationale sur les marchandises sur les fonds publics}, où l’auteur tend à prouver (et prouve par une surabondance de documents) que les \emph{marchés à terme} sur des \emph{marchandises imaginaires}, par exemple sur le blé, ont pour effet de faire baisser le prix au détriment des producteurs. En sens contraire, Paul de Rouziers, sur les Trusts américains (\emph{Les industries monopolisées}, 1898).
 } qui apportent force documents à l’appui de leur thèse (combattue fortement, il est vrai), la lutte des spéculateurs à la baisse et des spéculateurs à la hausse du prix, c’est-à-dire des vendeurs et des acheteurs à terme, tournerait à l’avantage des premiers, qui sont de grands capitalistes internationaux se solidarisant et se concertant facilement. Aussi, disent-ils, le prix du blé, du coton, de la laine, etc., de toutes les choses les plus nécessaires à la vie, est-il en  \phantomsection
\label{v2p46}réalité déterminé, non par une prétendue loi naturelle qui fonctionnerait automatiquement comme les lois de l’équilibre des liquides, mais par la volonté prépotente d’une centaine de grands financiers qui imposent à des millions de producteurs de blé, de coton, de laine, etc., des prix souvent désastreux.\par
Les économistes, qui signalent ces dangereux effets de la spéculation de Bourse, ont l’habitude de la condamner (quand ils la condamnent) pour une assez mauvaise raison ; parce que, disent-ils, ces marchés à terme relatifs à des marchandises fictives ont pour effet de fausser ou d’empêcher la détermination « normale » du prix des marchandises par le fonctionnement de la loi de l’offre et de la demande. Ils ne voient pas que, dans les cas de ventes à terme, l’influence de l’offre et de la demande intervient encore, et que dans ce cas, comme dans tous les autres, même réputés normaux, il s’agit toujours de l’offre ou de la demande qu’on \emph{croit exister}, et non de celle qui existe réellement.\par
Avant l’extension des marchés et l’institution des Bourses, il n’y avait pas de ventes à terme pour fixer tyranniquement le prix du blé. Mais est-ce que le prix du blé, sous l’ancien régime par exemple, était déterminé par l’insuffisance ou la surabondance \emph{réelle} du blé dans une région donnée, ou à une époque donnée ? Non. A cette époque, où l’on était fort mal renseigné, où l’on ne connaissait que les moissons de son village, on jugeait de l’abondance ou de la disette d’après la quantité de blé apportée sous la halle de la petite ville voisine. Il suffisait alors à quelques accapareurs (car il y avait alors des accapareurs, comme il y a aujourd’hui de grands banquiers joueurs de Bourse), de drainer les moissons d’une ou deux communes, ou d’emmagasiner leurs propres récoltes (c’est là le cas des grands propriétaires) pour créer l’apparence d’une disette toute artificielle, d’où n’en résultait pas moins, comme si elle eût été réelle, une hausse prodigieuse du prix du blé.\par
 \phantomsection
\label{v2p47}La différence entre ce passé et notre présent, est qu’alors la spéculation le plus souvent triomphante était à la \emph{hausse}, ruineuse pour le consommateur, tandis qu’à présent ce sont, nous dit-on, les spéculateurs à la \emph{baisse} qui ruinent les producteurs.
\subsubsection[{II.2.f. La loi de l’offre et de la demande.}]{II.2.f. La loi de l’offre et de la demande.}
\noindent Notre explication du prix étant exposée, il nous sera facile de critiquer à sa lumière la fameuse loi de \emph{l’offre et de la demande} qui a été regardée si longtemps comme la clef d’or de la théorie de la valeur. Cette loi est une formule à la fois vague et commode, — commode parce qu’elle est vague, et de là son immense succès — de la manière dont s’opèrent les variations des prix ; mais elle ne donne nullement la cause de ces variations. Par exemple, à la Bourse, on voit toujours que, quand une valeur est offerte par un plus grand nombre de vendeurs, elle baisse, que, quand elle est demandée par un plus grand nombre d’acheteurs, elle hausse ; (sans qu’il y ait, d’ailleurs, la moindre proportionnalité entre l’augmentation de la demande ou de l’offre et le degré de la hausse ou de la baisse). Mais la question est de savoir pourquoi, aujourd’hui plutôt qu’hier, à 4 heures du soir plutôt qu’à midi, le nombre des offreurs s’est trouvé accru ou celui des demandeurs ; et pourquoi, ce qui ne revient pas au même, le désir des uns ou des autres s’est augmenté jusqu’à tel ou tel point. C’est précisément à cette question que j’ai tâché de répondre.\par
Pourquoi, par exemple, pendant que l’ensemble des produits et d’abord le blé baissaient de prix sans interruption, de 1874 à 1887, le prix des bestiaux, des viandes importées même, du beurre, du lait, s’élevait-il ou se maintenait-il ? Cependant le nombre des bêtes à cornes augmentait bien plus rapidement, certes, que la population. Si  \phantomsection
\label{v2p48}donc le \emph{besoin de manger de la viande} eût été le même qu’autrefois, la viande, beaucoup plus offerte, aurait dû s’avilir comme le blé. Mais l’accroissement du besoin de manger de la viande a fait contrepoids à l’augmentation de l’offre. Et vous croyez m’avoir expliqué la hausse du prix de la viande en me disant que la demande de viande a augmenté ? Mais pourquoi ce besoin de manger de la viande s’est-il développé ? Voilà ce qu’il s’agit de savoir, car c’est là la vraie cause de cette exception à la baisse générale des prix de 1874 à 1887. En faisant cette recherche, on verrait facilement que les vrais « facteurs » de cette propagation des habitudes carnivores et, par suite, du prix de la viande, sont des idées et des diffusions d’idées : l’idée, démontrée par des savants, que la viande contribue plus qu’une nourriture exclusivement végétale à fortifier les bras de l’ouvrier ; l’idée que tous les hommes sont égaux, d’où le penchant redoublé à copier ses supérieurs, en mangeant de la viande comme eux, etc. ; l’idée que l’ouvrier a \emph{droit} à un certain confort, dont la nourriture animale, y compris le lait et le beurre, fait partie, etc. En d’autres termes, c’est parce que l’\emph{étalon de vie} s’est élevé que le \emph{prix de la viande} a haussé.\par
L’offre, dans les deux sens indiqués, (nombre des offreurs ou intensité de leurs désirs), aura beau augmenter, si l’on ignore cette augmentation dans le public du marché (je ne dis pas à la Bourse, où cette ignorance est plus malaisée\footnote{ \noindent Encore y a-t-il des cas où, dans la même séance de Bourse, \emph{au même instant}, la même valeur se vend à des prix très différents. Cela s’est vu à la Bourse de New-York, le 10 mai 1901, au moment d’une crise provoquée par le duel de deux trusts. Les demandes d’achat se succédaient avec une telle rapidité que les agents de change n’avaient plus la possibilité d’établir une cote ; « et des valeurs se vendaient à 700 dollars dans un coin du hall, que dans un autre coin, à la même minute précise, on pouvait avoir pour 400 dollars... » 
 }), le prix n’en sera point diminué. Elle aura beau diminuer, si l’on ignore cette diminution, le prix n’en sera point accru. La demande, dans les deux sens en question, aura beau augmenter ou diminuer, si l’on ignore cette augmentation \phantomsection
\label{v2p49} ou cette diminution, le prix ne variera point. C’est donc seulement \emph{en tant que connues} que les variations de l’offre et de la demande agissent sur les variations du prix. Mais elles sont le plus souvent \emph{très mal connues}, assez souvent même imaginaires. N’importe, c’est leur augmentation ou leur diminution \emph{crue}, imaginée, et non leur augmentation ou leur diminution réelle, qui est efficace. Et en quel sens est-elle efficace ? Est-ce que le vendeur qui croit — sur la foi d’une nouvelle de Bourse ou de journal — que la demande d’un article ou sa possession a doublé va, immédiatement, être amené pour ainsi dire par force à exiger un prix deux fois plus fort de cet article ? Pas le moins du monde. D’abord, remarquons que, s’il est seul à avoir connaissance de cette augmentation de la demande, si les autres possesseurs de l’article et les acheteurs présumés l’ignorent \emph{encore}, personne ne consentira à lui payer ses marchandises un centime de plus. Il faut donc que son exigence trouve à s’appuyer sur la croyance, pareille à la sienne, des autres vendeurs et des acheteurs, croyance qui est très inégale, très faible chez les uns, très forte chez les autres, et qui porte sur une augmentation très exagérée chez les uns, très inférieure à la réalité chez les autres.\par
Dira-t-on que, du pêle-mêle de ces \emph{estimations} si diverses, il se dégage \emph{une moyenne} qui, fatalement, détermine le prix, comme un abaissement de la température détermine le niveau du mercure dans le thermomètre ? C’est l’hypothèse la plus gratuite et la plus invraisemblable. L’observation, dans la mesure où l’on peut observer ces choses, montre que, dès la première information parvenue à l’un des vendeurs ou à l’un des acheteurs, il y a diffusion imitative de cette nouvelle et de la foi en elle parmi le public du marché (parmi le monde de la Bourse, par exemple) et qu’avant peu il s’établit un même niveau de croyance à peu près unanime et à peu près égale momentanément, qui est, non pas la moyenne d’évaluations spontanément formées, mais la généralisation \phantomsection
\label{v2p50} d’une évaluation individuelle imposée par le prestige du nom, de la fortune, de l’autorité personnelle, et d’ordinaire beaucoup plus optimiste ou beaucoup plus pessimiste que ne le serait la moyenne si la plupart des hommes étaient abandonnés aux inspirations de leur jugement individuel. Les entraînements de l’exemple les font dévier de ce point et s’arrêter bien au-dessus ou bien au-dessous.\par
Maintenant, l’\emph{opinion}\footnote{ \noindent Est-il utile de dire que là où il n’y a pas d’\emph{opinion} publique, il n’y a point de \emph{valeur} possible ? Quand la conversation fonctionne peu, quand il n’y a ni littérature ni art pour l’alimenter, il n’y a pas d’opinion publique à vrai dire. Donc, il y a des états sociaux dont la vie économique se passe de l’idée de valeur. L’idée de valeur va s’accentuant, se précisant, s’élargissant, avec le progrès de la civilisation.
 } s’étant fixée à un certain niveau relativement aux variations de l’offre ou de la demande virtuelles \emph{au prix actuel}, tout le monde sent que ce prix doit varier aussi. Pourquoi ? Parce que la fonction du prix est précisément d’établir l’équation de l’offre et de la demande réelles entendues au sens du nombre des choses effectivement offertes et du nombre des demandes effectives, c’est-à-dire des ventes et des achats ; car c’est \emph{le prix qui règle la hauteur des deux.} Dès lors, le problème qui se pose à tous, c’est, — puisqu’on croit que l’équilibre entre l’offre et la demande au prix actuel vient de se rompre — de le rétablir au plus tôt. Et comment le rétablit-on ? En tâtonnant. Un vendeur prend l’initiative de proposer un prix un peu différent du précédent, et ce prix a pour effet de faire varier l’offre et la demande ; mais, il est bien rare que, du premier coup, l’équilibre soit atteint ; il faut, auparavant, traverser une petite crise d’oscillations du prix, comme il s’en voit à toutes les séances de Bourse un peu agitées.\par
Or, quels sont les caractères propres du prix qui établit l’équation dont il s’agit ? Là est le nœud du problème de la valeur-prix. J’ai essayé de le résoudre, en supposant, d’abord, pour le simplifier, que toutes les fortunes sont égales : il devient clair alors qu’on n’a plus à tenir compte que de l’inégalité \phantomsection
\label{v2p51} des désirs et des croyances entre lesquels doit opter chaque acheteur ou chaque vendeur qui hésite à sacrifier certains désirs à certains autres, certains jugements à certains autres, dans l’impossibilité où il est, par les limites de sa fortune, de les satisfaire tous à la fois, de les formuler tous à la fois, et au même degré. Puis, revenant à la vérité des faits, à l’inégalité des fortunes, on complique de ce nouvel élément les données du problème à résoudre. La conclusion est celle-ci : le prix établissant l’équation voulue est celui qui correspond à la somme la plus forte que peuvent donner de l’article ou du service en question les fortunes les plus faibles parmi le groupe de tous ceux qui, à ce prix, préfèrent l’acquisition de cet article ou de ce service à la perte de l’argent qu’il coûte. Ce groupe est limité par le groupe de ceux qui préfèrent garder leur argent. C’est \emph{sur la limite} de ces deux groupes que se rencontre le prix dont il s’agit, le prix qui est propre à établir l’équation demandée. Sur cette frontière flottent ceux qui désirent presque autant garder leur argent que d’en faire cet emploi. Pour ceux-ci, et ils sont souvent très nombreux, il y a conflit mutuel, \emph{duel logique} ou \emph{duel téléologique}, entre deux jugements contradictoires, entre deux contraires désirs. C’est d’un grand nombre de ces combats intérieurs, de ces crises sourdes et cachées, que sort le prix, leur dénoûment.
\subsubsection[{II.2.g. Influence d’ordre logique.}]{II.2.g. Influence d’ordre logique.}
\noindent Il y a une considération dont je n’ai rien dit jusqu’ici et qui mérite d’être notée : elle a trait à l’influence \emph{toute logique} qu’exerce sur le prix d’un article, dont les conditions d’offre et de demande sont restées les mêmes, le changement survenu dans le prix d’un autre article, sans nul rapport avec le premier, par suite d’une variation de l’offre ou de la demande résultant de causes psychologiques. Sur un même marché,  \phantomsection
\label{v2p52}l’ensemble des prix forme système, et leur solidarité est parfois d’ordre exclusivement logique, quand elle n’est pas l’effet d’un entraînement irrationnel. Si l’on doutait de la nécessité de distinguer l’action économique des croyances et des jugements de celle des désirs, il suffirait de songer à l’extrême \emph{sensibilité} des cotes de la Bourse, où, dès qu’une valeur est atteinte par un événement inattendu, un grand nombre d’autres, sans nul lien de dépendance causale avec elle, baissent aussi. Mais voici un exemple plus tangible, qui nous fait remonter aux premières origines de l’échange. Quand le capitaine Wallis en 1767, après quelques escarmouches, fut parvenu à entrer en rapports d’échange avec les naturels d’O-tahiti, une sorte de prix ne tarda pas à s’établir. Pour un cochon de grosseur moyenne, qu’amenaient les O-tahitiens, on leur donnait un nombre précis de petits clous de fer : c’était le tarif. Mais, quelques jours après, les matelots ayant répondu aux avances des jeunes o-tahitiennes, se trouvèrent conduits, par leur goût devenu très vif pour ces belles insulaires, à s’emparer de tout le fer qu’ils pouvaient dérober du navire. « Les clous qu’on avait apportés pour commercer n’étant pas toujours sous leurs mains, ils en arrachèrent de différentes parties du vaisseau, particulièrement ceux qui attachent les taquets d’amures », si bien que leur galanterie devenait un véritable danger pour la conservation du navire. Or, ces clous étaient plus grands que les clous habituels ; d’où il résulta que, à partir de ce moment les indigènes refusèrent de donner leurs porcs pour les petits clous dont ils s’étaient parfaitement contentés jusque-là. Ils se disaient paraît-il que, puisqu’un \emph{article} de bien moindre valeur à leurs yeux, était coté un gros clou, à fortiori, et logiquement, un gros animal bon à manger ne pouvait être vendu moins cher. — Ici, qu’on le remarque, le nombre des offreurs et le nombre des demandeurs sont restés les mêmes, et il n’y a pas de raison de croire que le désir de vendre, chez les uns, ou le désir d’acheter chez les autres, aient sensiblement \phantomsection
\label{v2p53} varié, tout de suite après la mise en circulation des gros clous. Mais la croyance que la valeur d’un porc est égale à quelques petits clous a été remplacée, dans l’esprit des O-tahitiens, par la croyance qu’elle est égale à quelques gros clous ; c’est toute la différence ; et cette substitution d’un jugement à un autre, qui s’explique elle-même par une opération logique, par un syllogisme implicite, a suffi à faire hausser le prix.
\subsubsection[{II.2.h. Les frais de production.}]{II.2.h. Les frais de production.}
\noindent De plus en plus, nous le savons, le prix est fixé par le vendeur. Faut-il accorder à certains économistes que, en le fixant, le vendeur est déterminé ou doit être déterminé avant tout par le montant des frais de production ? Nullement. Le vendeur se règle, ne craignons pas de le répéter, sur l’intensité du désir d’acquisition qu’il suppose, à certains signes, exister chez un nombre suffisant d’acheteurs éventuels, et sur le degré de fortune que ceux-ci lui paraissent posséder. Aussi, quand un article nouveau est mis à la mode dans un monde riche, le prix est-il extrêmement supérieur aux frais de production. Plus tard, quand ce qui était objet de luxe sera devenu, par cascades ou cascatelles d’imitation, article de première ou de seconde nécessité, le prix peu à peu s’abaissera. Mais s’abaissera-t-il, d’une manière permanente du moins, jusqu’à la limite du coût des frais ? Non. Rien de plus inégal que ce coût pour des producteurs différents, et cette prétendue limite est l’illimité même ; à moins qu’on ne prétende, par hasard, qu’il s’agit du \emph{taux moyen} des frais. Mais pourquoi serait-ce le taux moyen, variable lui-même, plutôt que le taux le plus bas ou le taux le plus élevé ? Il semble que le taux le plus bas serait la seule limite vraiment infranchissable, et que, en effet, dans les moments de concurrence effrénée entre producteurs \phantomsection
\label{v2p54} de même article, l’abaissement du prix doit descendre jusque-là, mais temporairement, jusqu’à ce que tous les concurrents, sauf un seul, ayant succombé dans cette lutte meurtrière, le survivant relève bientôt ses prix. Mais quelle raison y a-t-il de réserver l’épithète de \emph{normal} et de \emph{légitime}, parmi ces prix changeants, au seul prix temporaire et passager, qui est égal au taux minimum des frais ? Il n’y en a pas. Il y en a d’autant moins que la concurrence des producteurs n’a rien de plus normal en soi, il s’en faut, que leur association. Or, quand ils s’associent, prennent-ils pour fondement du prix qu’ils imposent au public le coût de production ? Non, ils se préoccupent, avant tout, de scruter le \emph{cœur} et la \emph{bourse} du public, de la partie du public à laquelle ils s’adressent, et de rechercher, — pour chaque prix proposé entre eux, en discussion — quel est le nombre d’acheteurs qui, vu leur désir d’acquisition et leurs idées, combinés avec leur fortune, seront vraisemblablement disposés à accepter ce prix ? Il faut toujours en revenir là. Et l’État collectiviste lui-même serait obligé de faire des calculs pareils.\par
Bien que la théorie de la valeur des produits fondée sur la quantité de travail dépensée à les fabriquer soit surabondamment réfutée par tout ce qui précède, il n’est pas inutile de présenter encore quelques considérations à ce sujet. D’après un économiste italien, l’idée de valeur aurait subi l’évolution historique suivante : basée primitivement sur l’utilité, elle irait se fondant de plus en plus sur la quantité de travail. Nous savons qu’il n’en est rien, que la valeur a toujours été causée par des désirs et des croyances qui ont changé d’âge en âge ; que les mutations de désirs et de croyances, de besoins, sont fonction d’inventions et de découvertes successives ; et que jamais le degré d’\emph{utililé crue} d’un produit obtenu grâce à celles-ci, j’entends d’utilité relative et comparée à celle des autres produits à sacrifier pour l’acquérir, n’a cessé d’être le fondement du prix. On ne  \phantomsection
\label{v2p55}tient compte, et on n’a jamais tenu compte, de la dose de travail fournie par l’ouvrier, dans l’estimation de la valeur des choses, que dans la mesure où la croyance au droit de l’ouvrier d’exiger cela, par suite de principes nouveaux émis par des penseurs, s’est généralisée et enracinée. Mais pourquoi serait-ce aux époques les plus avancées de l’évolution économique que cette croyance se répandrait et se fortifierait, justifiée en apparence ? C’est plutôt aux époques de barbarie. En effet, c’est seulement dans l’hypothèse d’une industrie et d’une consommation routinières, immuables, non progressistes, que cette estimation de la valeur d’un produit d’après la quantité de travail employée à le faire peut avoir chance de s’accréditer. Si le prix du produit, dans cette hypothèse, devenait insuffisant pour payer le travail nécessaire à sa reproduction ultérieure — travail qui serait alors nécessairement égal au travail dépensé antérieurement pour le produire — on renoncerait à cette fabrication ; ou bien, si l’on voulait fabriquer de nouveau il faudrait bien relever le prix au niveau précédent. — Supposez, au contraire, une époque inventive et rénovatrice, où, dans l’intervalle de quelques années, toute fabrication ait reçu des améliorations notables, c’est sur la quantité de travail \emph{ultérieure}, et non sur la quantité de travail \emph{antérieure}, que se réglerait le prix du produit, si du moins le besoin de ce \emph{mètre} objectif se faisait sentir\footnote{ \noindent Carey est à citer ici, parce que, tout en fondant comme Bastiat et comme Marx sa théorie de la valeur sur la quantité de travail, il a entrevu les [{\corr difficultés}] auxquelles elle se heurte. « Pour que la quantité de travail, dit-il, puisse devenir une mesure de la valeur, il faut qu’il existe un pouvoir égal de disposer des services de la nature », c’est-à-dire qu’il n’y ait pas facilité plus grande pour l’un que pour l’autre des travailleurs d’utiliser ces forces, ou qu’aucune invention nouvelle n’ait apparu, propre à donner à celui qui en dispose un avantage sur autrui pour l’exploitation de ces forces. Autrement dit, Carey est forcé de convenir que la valeur des choses dépend, \emph{en partie au moins}, de l’accident du génie, de l’invention nouvelle et aussi bien des circonstances qui permettent de monopoliser une invention ancienne ou nouvelle. « Une hache fabriquée il y a cinquante ans, d’une qualité égale à la meilleure fabriquée de nos jours, dit-il, et qui serait restée sans emploi, ne s’échangerait pas aujourd’hui contre une somme équivalente à la moitié de celle qu’on l’eût payée au moment de sa fabrication... » Voilà une objection grave, certes, contre la théorie du travail mesure de la valeur. Carey aboutit à cette formule : la valeur d’une chose, dit-il, est déterminée \emph{par le prix que coûterait sa reproduction} (au moment où elle a été fabriquée). « On peut aujourd’hui se procurer, en échange du travail d’un ouvrier habile pendant un seul jour, un exemplaire de la Bible mieux fabriqué que celui qu’on eût obtenu il y a cinquante ans en échange du travail d’une semaine. La conséquence nécessaire de ce fait a été une diminution dans la valeur de tous les exemplaires de la Bible existant... » — Cependant il se fait lui-même une objection : Quand il s’agit d’un article \emph{qui ne saurait être reproduit} — un chef-d’œuvre du maître mort, une œuvre artistique de premier ordre — comment sera déterminée la valeur ? Il répond — il est forcé de répondre et sa réponse obligatoire est un acheminement vers notre théorie — : « Sa valeur n’a d’autre limite que les caprices de ceux qui \emph{désiraient} posséder cet article ou qui possèdent le moyen de le payer... » Il aurait dû dire : « sa valeur n’a d’autre limite que le \emph{désir le plus fort}, combiné avec le \emph{jugement le plus farorable}, de ceux qui offrent de l’acheter, le tout combiné \emph{avec le taux de leur fortune...} »
 }.\par
 \phantomsection
\label{v2p56}Mais la qualité du travail, à ce point de vue, importe bien plus que sa quantité. Ou plutôt sa quantité n’a de sens que dans les limites d’une qualité déterminée. A propos de chaque espèce de travail considérée à part, on peut dire s’il y a augmentation ou diminution de travail. A cet égard, la notion de quantité de travail est fort nette. Mais elle s’obscurcit étrangement quand, rapprochant deux ou plusieurs travaux d’espèces différentes, on leur cherche une commune mesure. Là est l’écueil de la théorie que je combats. Il y a cependant une commune mesure, mais une seule, et elle est toute psychologique : c’est la quantité du désir satisfait par le produit du travail. Car nous savons que le désir reste essentiellement le même, quelle que soit l’hétérogénéité des objets désirés.
 \phantomsection
\label{v2p57}\subsection[{II.3. Les luttes}]{II.3. Les luttes}\phantomsection
\label{l2ch3}
\subsubsection[{II.3.a. Luttes de la production avec elle-même : d’abord, entre coproducteurs d’un même atelier.}]{II.3.a. Luttes de la production avec elle-même : d’abord, entre coproducteurs d’un même atelier.}
\noindent Dans le chapitre précédent, nous avons étudié une lutte de désirs et de jugements tout interne, une concurrence purement psychologique, mais d’où, par la détermination des prix, découlaient les conséquences économiques les plus importantes. Cette lutte psycho-économique a embrassé à la fois le champ de la production, de la consommation et de la monnaie.\par
Arrivons maintenant aux luttes économiques externes, c’est-à-dire inter-psychologiques. Pour être sûr de n’omettre aucune partie de ce grand sujet, nous allons traiter successivement des conflits ou des désaccords : 1\textsuperscript{o} de la production avec elle-même ; 2\textsuperscript{o} de la consommation avec elle-même ; 3\textsuperscript{o} de la production avec la consommation ; 4\textsuperscript{o} de la monnaie avec elle-même ; 5\textsuperscript{o} de la monnaie avec la production et la consommation. Et chacune de ces divisions demande à être subdivisée.\par
Les luttes de la production avec elle-même peuvent être divisées ainsi : 1\textsuperscript{o} Entre co-producteurs d’un même atelier, d’une même usine ; entre patrons et ouvriers notamment. L’arme habituelle de ces luttes, de la part des patrons, est la fermeture des ateliers, le chômage forcé ; de la part des ouvriers, la \emph{grève}.\par
2\textsuperscript{o} Entre producteurs différents d’un même article, d’une même marchandise, d’abord dans une même nation. C’est le cas de la concurrence proprement dite. Les armes  \phantomsection
\label{v2p58}employées dans cette lutte réputée féconde sont des plus variées, et souvent empoisonnées. L’abaissement du prix du produit n’est, en effet, qu’un des moyens, — et non le plus sûr ni le plus fréquent, — que les producteurs mettent en œuvre pour triompher de leurs rivaux. Ils ont surtout confiance dans la réclame mensongère, dans la calomnie répandue contre les produits concurrents, dans l’art de faire souffler vers soi le vent de la mode, dans l’achat des journalistes et parfois des pouvoirs publics.\par
3\textsuperscript{o} Entre producteurs nationaux et étrangers d’un même produit. On lutte ici à coups de tarifs de douane, jusqu’à ce que de cette guerre douanière on passe de temps en temps, à la guerre véritable. C’est une grande erreur de ne s’occuper du protectionnisme qu’à propos de la concurrence, comme si celle-ci était la règle dont celui-là n’est que l’exception, ou comme si le concurrence était le phénomène positif dont le protectionnisme ne serait que le côté négatif, le revers de la médaille. En réalité, la libre concurrence entre producteurs appartenant à des nations différentes n’a jamais été qu’un fait exceptionnel, et toujours précédé par la protection ou la prohibition, à laquelle elle s’oppose et qu’elle nie. La protection, la prohibition est, historiquement, le fait positif et primitif, dont le libre échange n’est que la négation subséquente. Toute évolution industrielle commence par le protectionnisme ou le prohibitionnisme familial, auquel succède le libre échange restreint et partiel, suivi d’un protectionnisme élargi, le protectionnisme municipal, d’où l’on passe, à travers des traités de commerce de ville à ville, à un protectionnisme provincial, puis plus tard à un protectionnisme national, à un protectionnisme fédéral. Libre échange et protectionnisme alternent ainsi, en s’élargissant sans cesse l’un et l’autre, sans qu’on puisse prédire avec certitude quel sera le terme final, puisque le libre échange, idéal international sans cesse souriant, ne pourra se réaliser pleinement que si l’unité politique du monde  \phantomsection
\label{v2p59}entier, sous forme fédérative ou impériale, parvient à s’établir.\par
4\textsuperscript{o} Entre producteurs (nationaux ou étrangers) d’articles \emph{différents}, il existe aussi une lutte constante et profonde, mais inaperçue, tandis que tout le monde est frappé du fait inverse, à savoir les services mutuels que se rendent les producteurs d’articles différents en se servant de débouché les uns aux autres. Oui, dans beaucoup de cas ; mais, dans beaucoup de cas aussi, ils se font tort les uns aux autres. Car les divers produits industriels se disputent entre eux le \emph{désir} de chaque individu, comme les diverses idées circulant dans l’air se disputent sa \emph{croyance.} Ni la quantité de croyance, ni la quantité de désir ne sont indéfiniment extensibles.\par
Les luttes de la consommation avec elle-même sont de plusieurs sortes. Elles ont lieu : 1\textsuperscript{o} Entre consommateurs nationaux dont les uns veulent se réserver la jouissance exclusive de certains articles, soit par égoïsme, soit par vanité : car il y a des monopoles de consommation, comme des monopoles de production. Et les premiers, en tout temps, même aux âges démocratiques, ne sont pas moins avidement recherchés que les seconds. L’arme de ces luttes est tantôt la \emph{loi somptuaire} par laquelle une classe aristocratique interdit aux classes inférieures certains aliments, certains vêtements, certaines parures, et qui se survit, en temps démocratique, par l’interdiction de porter certains insignes, certaines décorations, sans y avoir droit ; tantôt la cherté artificielle de certains produits, comme, par exemple, celle des automobiles, grâce à laquelle pendant quelques années, certains groupes riches ont cherché à se réserver le privilège de ce nouveau mode de locomotion. Quand il s’agit d’articles de première nécessité, du blé par exemple en temps de famine, la guerre intestine des consommateurs devient sauvage, et l’on sait à quels excès monstrueux elle peut conduire.\par
 \phantomsection
\label{v2p60}2\textsuperscript{o} Entre consommateurs nationaux et consommateurs étrangers. Bien longtemps avant de désirer l’exportation de ses produits et encore moins l’importation des produits exotiques, une nation commence par être fière de ce qu’elle consomme, comme en général de tout ce qui lui appartient, et par se persuader que nul article du dehors ne vaut les articles indigènes. Il lui semble que son blé, son vin, ses fruits, ses étoffes, à plus forte raison ses armes traditionnelles, sont autant de gloires nationales et qu’on ne saurait sans manquer au premier devoir du patriotisme y faire participer l’étranger non naturalisé. De là des lois qui défendent la sortie d’abord de toutes sortes de marchandises, puis de quelques-unes seulement, des blés notamment et des armes.\par
3\textsuperscript{o} Non seulement les consommateurs (nationaux ou étrangers) d’un même produit, sont en conflit, latent ou manifeste, quand ce produit est en quantité limitée et inextensible, mais encore les consommateurs d’articles différents, même des articles les plus hétérogènes, se combattent souvent, et ont des intérêts contraires, par la même raison que les producteurs de ces articles, comme il a été indiqué plus haut, se nuisent fréquemment. Il est fâcheux, pour les consommateurs de maïs, de voir se développer la consommation du tabac, plante dont la culture se répand aux dépens de celle du maïs. Le développement des dépenses de luxe, en détournant les bras et les capitaux vers les industries correspondantes, peut faire renchérir les objets de première nécessité et nuire gravement à leurs consommateurs. Il y a une foule de consommations en apparence sans nul lien entre elles, dont les unes impliquent contradiction de ce que les autres affirment ou poursuivent implicitement, ou leur apportent une confirmation implicite, en sorte que le développement des unes contrarie ou favorise le développement des autres.\par
Les luttes de la production avec la consommation sont plus intéressantes encore. Elles comprennent : 1\textsuperscript{o} Les efforts  \phantomsection
\label{v2p61}contraires et contradictoires des producteurs et des consommateurs (soit nationaux soit étrangers) qui cherchent à vendre le plus cher ou à acheter le meilleur marché possible. Dans ce grand conflit constant et universel, les producteurs ont pour arme l’accord clandestin, les trusts ; les consommateurs, quand ils ont conscience de leur force numérique, les lois de maximum, les taxes municipales, les tarifications et les réglementations légales.\par
2\textsuperscript{o} Les désaccords entre la quantité (ou la qualité) des produits, et la quantité ou la nature des besoins. Cette discordance ou plutôt cette rupture d’accord, peut consister, en ce qui concerne la quantité, dans un \emph{excès} ou dans un \emph{déficit} de production. Dans un excès : cela peut tenir soit à une surproduction suscitée par une fièvre industrielle qui s’est propagée, soit à un resserrement de la consommation. Dans un déficit : soit parce que la consommation a diminué, soit parce que la consommation s’est accrue. Et, parmi les causes qui peuvent faire augmenter la consommation, il convient de mettre à part l’accroissement rapide de la population.\par
En ce qui concerne la qualité, les désaccords peuvent consister soit dans l’abus des contrefaçons et des fabrications de pacotille qui trompent et dégoûtent un public soucieux d’aliments sains, de vêtements et d’ameublements solides, soit, à l’inverse, dans une recherche exagérée de l’excellence et de la solidité des produits, devenus (sous l’empire des règlements de corporation démodés) trop soignés et trop coûteux pour les goûts d’un public vulgaire ou pressé, qui préfère des articles de qualité médiocre mais fréquemment renouvelés à des articles de premier ordre mais inusables. C’est à ce double point de vue qu’il serait bon d’apprécier les réglementations corporatives du moyen âge et les législations modernes sur le travail industriel.\par
Ces désaccords entre produits et besoins sont en même temps, on le voit sans peine, des conflits entre producteurs et consommateurs. Les premiers, quand ils s’aperçoivent de  \phantomsection
\label{v2p62}la surabondance de leur fabrication, emploient toutes sortes de mensonges pour la dissimuler, de même que la mauvaise qualité de leurs produits, et pour les écouler sans en abaisser le prix ; mais, bon gré malgré, ils doivent les offrir à des prix de plus en plus bas, pendant que les consommateurs, découvrant la fraude, se font valoir. Ou bien, si l’offre est inférieure à la demande, les produits aux besoins, les fabricants chercheront à exagérer la rareté des produits pour en élever le prix encore davantage. Au cours de ces combats de Bourses, il y a bien des morts et des blessés, bien des faillites et des ruines du côté des producteurs, et, du côté des consommateurs, bien des privations et des gênes, non moins douloureuses quoiqu’elles fassent moins de bruit.\par
Il y a enfin à considérer les luttes de nature monétaire. D’abord, la monnaie peut être en lutte avec elle-même. 1\textsuperscript{o} Dans chaque nation, l’unification monétaire, qui d’ailleurs est bien rarement complète, a toujours été précédée d’une longue période où des monnaies diverses, royales, seigneuriales, se disputaient le marché, et dont quelques-unes, à la longue, à raison de leur pureté plus grande, ou du prestige supérieur de l’autorité dont elles émanaient, ont fini par refouler les autres, par envahir leur domaine, jusqu’à ce qu’une refonte générale des monnaies ait consacré leur triomphe. Rien de plus important, au point de vue de la prospérité industrielle et commerciale, que les péripéties et le dénouement de cette concurrence inter-monétaire. — A ce conflit séculaire se rattache la grave question du bi-métallisme, qui en est la reprise agrandie et simplifiée. La monnaie d’or et la monnaie d’argent poursuivent leur duel avec un acharnement marqué longtemps après que la \emph{livre tournois} et la \emph{livre parisis} ont cessé de se combattre. Tantôt l’argent l’emporte comme au Mexique ; tantôt, et de plus en plus, l’or prévaut. Les coups décisifs, dans cette lutte, sont portés par les découvertes de mines qui font varier la  \phantomsection
\label{v2p63}valeur relative des deux métaux, légalement fixée. Cette fixité est un mensonge conventionnel constamment contredit, et cette contradiction, qui est un état de conflit psychologique chronique, a pour effet, comme on sait, l’émigration de la bonne monnaie. L’altération des monnaies, pratiquée de temps à autre par les rois faux-monnayeurs de jadis, créait un mensonge conventionnel analogue, cause de perturbations encore plus redoutables.\par
2\textsuperscript{o} Les luttes des diverses monnaies pour la domination du marché, après avoir pris fin dans chaque État, s’avivent d’État à État. Au point de civilisation où nous sommes parvenus, le franc, la livre sterling, le mark, le florin, etc., se gênent réciproquement dans leurs mouvements circulatoires, et cherchent à gagner du terrain les uns sur les autres. La nécessité de changer de monnaie à chaque passage de frontière, et de se tenir au courant des variations continuelles du change, est devenue une entrave de plus en plus sentie, dont les peuples européens s’affranchissent par degrés moyennant des conventions internationales.\par
En second lieu, la monnaie peut être en désaccord avec les besoins de la production et de la consommation, auxquels elle doit répondre ; et ces troubles fonctionnels, appelés, quand ils deviennent aigus, crises financières, engendrent les conflits parfois les plus douloureux. Ils sont de deux sortes :\par
1\textsuperscript{o} Quand, par la découverte d’abondantes mines d’or ou d’argent, comme au {\scshape xvi}\textsuperscript{e} et au {\scshape xvii}\textsuperscript{e} siècles, la masse des métaux précieux en circulation sur tout un continent vient à s’accroître brusquement, ou quand, sur une moindre échelle, dans une nation victorieuse, le payement d’une énorme indemnité de guerre y produit les mêmes effets, le système de tous les prix est profondément bouleversé, et une lutte de tous les jours s’engage, sourde et tenace, entre ceux qui ont intérêt à enrayer le mouvement ascendant des prix, parce que les prix anciens leur sont avantageux, \phantomsection
\label{v2p64} et ceux qui ont intérêt à le favoriser. Combien ce marchandage a dû être général, prolongé et âpre, nous en pouvons juger par ce fait que, la masse des métaux précieux ayant décuplé en Europe pendant [{\corr le}] siècle qui a suivi la découverte de l’Amérique, la moyenne des prix a seulement quintuplé.\par
2\textsuperscript{o} Plus courtes, mais beaucoup plus violentes sont les crises financières dues à une cause précisément contraire, au déficit brusque de monnaie sur un marché. Ce sont des anomalies presque périodiques dont nous nous occuperons plus loin.\par
3\textsuperscript{o} A ces discordances \emph{quantitatives} entre la monnaie et les besoins de l’échange, entre l’organe et sa fonction, doivent s’ajouter les discordances \emph{qualitatives} qui résultent de ce que l’unité ou les unités monétaires sont mal choisies, — le \emph{franc} par exemple est une unité trop basse, et surtout le \emph{centime}, ce qui complique continuellement et inutilement les calculs — ou bien de ce que les pièces ont été mal frappées, mal composées.\par
Nous allons examiner, dans l’ordre indiqué ci-dessus, les principales formes des luttes économiques que nous venons d’énumérer. Nous réservons pour un chapitre final l’ensemble des conflits aigus nés de ces luttes. Nous allons d’abord étudier celles-ci en ce qu’elles ont d’habituel et pour ainsi dire de normal. Occupons-nous donc, en premier lieu, des luttes de la production avec elle-même, qui se subdivisent en : 1\textsuperscript{o} luttes entre co-producteurs d’un même atelier, d’une même fabrique, c’est-à-dire entre patrons et ouvriers ; 2\textsuperscript{o} luttes entre producteurs d’un même article dans des ateliers différents, mais nationaux ; 3\textsuperscript{o} entre producteurs similaires mais internationaux ; 4\textsuperscript{o} entre producteurs d’articles hétérogènes.\par
La première de ces subdivisions n’a d’intérêt qu’en ce qui concerne les \emph{grèves}, qui, étant un conflit des plus aigus, seront examinées plus loin. Passons donc à la deuxième subdivision.
 \phantomsection
\label{v2p65}\subsubsection[{II.3.b. Puis, entre producteurs nationaux d’un même article (concurrence). Limites, causes, effets de la concurrence. Monopoles, trusts.}]{II.3.b. Puis, entre producteurs nationaux d’un même article (concurrence). Limites, causes, effets de la concurrence. Monopoles, trusts.}
\noindent Non moins intéressants que les conflits du patron et de ses ouvriers, co-producteurs d’un même produit dans un même atelier ou une même usine, sont les conflits des producteurs d’un même article dans des ateliers ou des usines différents. C’est à cette lutte surtout qu’ont eu égard les économistes qui ont parlé de la concurrence et vanté ses vertus magiques. On peut s’étonner, disons-le tout d’abord, d’entendre les mêmes écrivains, assez souvent, célébrer à la fois les louanges de la concurrence et celles de l’échange. L’échange est l’harmonie des producteurs d’articles dissemblables ; la concurrence est l’antagonisme des producteurs d’articles similaires. Louer en même temps l’association et la guerre, — la guerre que l’on maudit d’ailleurs sous son esprit militaire, mais que l’on surfait, en revanche, sous ses formes industrielles, — n’est-ce pas quelque peu contradictoire ? Stuart Mill nous explique très bien pourquoi les économistes ont de tout temps fermé les yeux à certaines vérités et outré l’importance de leurs principes. A propos de la \emph{concurrence} et de la \emph{coutume} et de leur action sur la répartition des produits, il écrit : « Les économistes se sont accoutumés à donner une importance presque exclusive au premier de ces mobiles, à exagérer l’effet de la concurrence et à tenir peu de compte de l’autre principe qui le combat... C’est ce qu’on peut jusqu’à un certain point concevoir si l’on considère que \emph{c’est seulement grâce au principe de la concurrence que l’économie politique a quelque prétention au  \phantomsection
\label{v2p66}caractère scientifique.} En tant que les rentes, les profits, les salaires, les prix, sont déterminés par la concurrence, on peut leur assigner des lois. Supposez que la concurrence soit leur unique régulateur et l’on pourra poser des principes d’une généralité étendue et d’une exactitude scientifique qui les régiront. \emph{C’est avec raison que l’économiste pense que c’est là son domaine propre}. » Ainsi, Stuart Mill partage l’erreur qu’il signale et qu’il prend pour une vérité évidente. Il n’a pas vu, non plus que les autres écrivains, que l’action de la coutume — et aussi des autres formes de l’imitation — peut donner lieu, aussi bien et beaucoup mieux que l’action de la concurrence, à des « principes d’une généralité étendue et d’un exactitude scientifique. » Et c’est pour n’avoir pas vu cela que, comme tous ses collègues, il s’est opiniâtré à poursuivre un ancien sillon où il a fleuri bien plus d’herbes folles que de blé...\par
Les économistes ont cru avoir suffisamment traité des effets de la coutume, quand ils en ont dit un mot, en passant, comme s’il s’agissait d’une simple \emph{perturbation} des courbes soi-disant régulières tracées sous l’influence de la concurrence, c’est-à-dire de l’offre et de la demande. Mais ils ont oublié que les astronomes regardent les perturbations comme soumises elles-mêmes à des lois régulières, aussi régulières que les lois d’après lesquelles les ellipses planétaires sont produites, et que perturbations et ellipses sont des conséquences d’un même principe. L’économie politique ne sera constituée que le jour où on aura reconnu quelque chose d’analogue à ce qu’ont vu les astronomes.\par
Si la concurrence est la loi suprême des prix, les prix déterminés par la concurrence sont toujours les seuls justes, et, quand la concurrence tend à ruiner une industrie par des prix décourageants, comme ceux de la soie en 1885, tout syndicat qui s’établit — comme il s’en est établi à cette époque entre une maison lyonnaise et des maisons italiennes\footnote{ \noindent Voir Claudio Jannet, \emph{le Capital}.
 }  \phantomsection
\label{v2p67}— doit être interdit. Cependant, ce sont les prix imposés par ce syndicat d’accaparement qui ont sauvé l’industrie de la soie. Cet exemple n’est pas unique ; mais, le serait-il, il suffirait à montrer le caractère tout empirique des vertus prêtées à la concurrence. En réalité, tout syndicat qui se forme tend à imposer un prix ou un salaire autre que celui que la libre concurrence établirait, et, dans une certaine mesure, il y parvient le plus souvent, on peut dire toujours. Donc, ou il faut proscrire tous les syndicats, ou il faut admettre que leur visée peut être légitime, que le prix obtenu par leur influence peut être plus juste et plus utile au bien général que le prix résultant de la concurrence ; et dès lors, c’est en dehors de celle-ci, on doit en convenir, qu’il faut chercher la pierre de touche du juste prix, du prix naturel et normal.\par
Disons franchement que les prix fixés par la concurrence sont toujours plus ou moins injustes, quelquefois d’une injustice criante, soit par excès, comme les honoraires de certains spécialistes ou de certains artistes même, soit par défaut, comme le salaire de beaucoup d’ouvriers et surtout de beaucoup d’ouvrières, de certains fonctionnaires aussi, comme le prix de vente de certains produits agricoles, et qu’il est nécessaire de rectifier ces erreurs, de réfréner ces exagérations dans un sens ou dans l’autre\footnote{ \noindent Carey a montré une différence importante entre la concurrence des produits et la \emph{concurrence des services.} Si l’homme qui n’a que ses bras pour vivre et qui les offre est en concurrence avec un autre qui les offre aussi, il peut se trouver conduit à les offrir pour rien, c’est-à-dire à accepter la vie d’esclave, qui travaille gratis. En ce sens, Carey a raison de dire que « toute la question de la liberté et de l’esclavage pour l’homme est contenue dans celle de la concurrence... »
 }. La difficulté est de trouver un bon frein, qui ne fasse pas plus de mal que le mal combattu par lui. Reconnaissons que l’établissement des syndicats n’est qu’une solution provisoire. Tout syndicat de vendeurs tendant à susciter un syndicat d’acheteurs correspondant, tout syndicat de patrons ou d’ouvriers tendant à susciter un syndicat contraire d’ouvriers ou de patrons, la  \phantomsection
\label{v2p68}lutte d’influences entre syndicats fait renaître la concurrence, sous une forme seulement accentuée et amplifiée. Dira-t-on que les injustices produites par la concurrence primitive, que les syndicats ont étouffée, risquent ainsi de s’amplifier elles-mêmes par l’effet de cette transformation ? Et ajoutera-t-on qu’il n’y a nul moyen de sortir de là si ce n’est de recourir à l’intervention d’une autorité supérieure, celle de l’État, c’est-à-dire la force d’un parti politique triomphant ? Nous verrons pourquoi cette conclusion n’est pas inévitable.\par
Les producteurs du même article ne sont pas tous rivaux entre eux ; il n’y a de rivalité et d’antagonisme qu’entre ceux dont chacun pourrait sans trop de difficulté servir la clientèle des autres si ceux-ci venaient à cesser leur production. Or, ce groupe est toujours une simple fraction, en général minime — même aux époques de civilisation avancée — du chiffre total des producteurs similaires répandus sur tout le territoire social. En effet, 1\textsuperscript{o} il est beaucoup d’industries qui, par leur nature, — maçonnerie, charpenterie, menuiserie, cordonnerie sur mesure, etc., — ne comportent qu’une production restreinte pour une clientèle circonscrite ; 2\textsuperscript{o} même dans les industries à champ plus ample, où l’emploi des machines est toujours possible, il ne l’est que dans une certaine mesure au delà de laquelle la surveillance de trop gigantesques ateliers excéderait la portée du regard d’un seul et même directeur ; ou bien il est des limites à la somme des capitaux qu’on peut effectivement réunir sur une place donnée pour étendre la production ; ou enfin, étant donné le réseau des voies ferrées et des voies maritimes, l’état des moyens de locomotion et de communication, de trafic international ou intra-national, il y a un rayon passé lequel le transport des marchandises fabriquées serait pratiquement impossible, soit à cause du prix, soit à cause du temps toujours limité dans lequel la consommation de ces articles est contenue. Par toutes ces raisons, on n’a jamais vu encore  \phantomsection
\label{v2p69}et, — \emph{malgré l’extension croissante du rayon dont il s’agit} — on ne verra que dans un avenir assez lointain, un producteur quelconque, à lui seul, à moins d’une coalition avec ses confrères, pouvoir prétendre avec chances sérieuses de succès à accaparer la fabrication de son produit sur toute la surface du globe terrestre.\par
Jusqu’ici, en somme, les producteurs d’un même article qui se font réellement obstacle sont en petit nombre, quoique en nombre grandissant ; ce sont ceux qui sont trop voisins les uns des autres pour ne pas se disputer une même clientèle. — Quant à ceux qui ne sont pas compris dans ce cercle de voisinage — cercle qui va s’élargissant toujours — et qui sont à une distance suffisante pour ne pas se gêner, distance variable d’après les idées, les mœurs, les inventions relatives à la locomotion, non seulement entre ceux-là il n’y a pas rivalité mutuelle, mais il y a une réelle et importante solidarité. Car, chacun d’eux est intéressé à ce que, hors de sa sphère propre de clientèle, il se crée d’autres ateliers, d’autres usines semblables, qui développent ou entretiennent le besoin et l’habitude de consommer l’article qu’il produit. Si, par malheur pour lui, ces collègues éloignés venaient à cesser leur fabrication, il arriverait fatalement ou probablement, que par suite de la contagion imitative, la demande de cet article diminuerait, s’affaiblirait, dans la région même où rayonnent ces produits. Par exemple, supposez que toutes les fabriques de soieries, sauf une seule, viennent à s’arrêter. La survivante triomphera d’abord, doublera, triplera sa production, soit ; mais bientôt, quoi qu’elle puisse faire, il lui sera impossible de combler le vide laissé par ses concurrentes, par ses collaboratrices aussi bien. Et enfin, le besoin d’étoffes de soie s’étant perdu faute d’aliments dans beaucoup de pays et cette désuétude s’étant étendue de proche en proche, il pourra très bien se faire que, au bout d’un certain nombre d’années, la fabrique unique, en dépit de son monopole, à raison même de son  \phantomsection
\label{v2p70}monopole, produise moins de soieries qu’elle n’en produisait auparavant.\par
Ce que je dis des producteurs d’un même article peut s’appliquer aux consommateurs de ce même article. Eux aussi sont solidaires et mutuellement auxiliaires, plutôt que rivaux, quand c’est à des magasins différents qu’ils vont s’adresser. Chacun d’eux est intéressé à ce que le besoin de l’article qu’il veut acheter reste répandu dans le public, ne s’y affaiblisse pas, ne s’y restreigne pas, afin que la satisfaction de son propre besoin reste toujours possible et ne devienne pas plus difficile ni plus coûteux.\par
Mais revenons aux co-producteurs. Il résulte malheureusement de ce qui précède une conséquence inquiétante : c’est que les progrès de la civilisation, en étendant constamment pour chaque fabrique sa sphère d’action, a pour effet inévitable de faire croître la proportion des co-producteurs antagonistes aux dépens de la proportion des co-producteurs auxiliaires. Cependant, cet effet est en partie compensé par une autre conséquence, inévitable aussi : à mesure que les industries voisines voient s’étendre leur clientèle lointaine, elles se disputent moins vivement leur clientèle rapprochée, et finissent souvent par s’entendre, en dépit de leur rivalité à cet égard, de moins en moins sentie, pour exploiter de mieux en mieux le client extérieur. Quand une industrie ne comporte qu’un débouché local et restreint, — boulangerie, boucherie, etc., — les industriels qui l’exercent dans une même localité sont, d’ordinaire, à l’état d’âpre concurrence. Il n’en est pas de même quand elle a un grand débouché, national ou international. On voit alors se rassembler, se pelotonner dans une même région, des fabriques, des usines similaires, par exemple les papeteries dans la région d’Angoulême, les fabriques de coton aux environs de Manchester, ailleurs les forges, etc., et ces établissements sont, pourrait-on dire, encore plus associés que rivaux, alors même qu’ils ne sont pas syndiqués. Ils collaborent et s’entr’aident dans  \phantomsection
\label{v2p71}une large mesure, ils se procurent tour à tour des ouvriers et travaillent ensemble à la conquête de nouveaux pays par l’industrie nationale. Il ne faut donc pas s’étonner si, dans la grande industrie, les semblables se rapprochent, et si, dans la petite industrie, ils se fuient. Les vrais rivaux des grands fabricants sont bien moins leurs voisins et confrères que les grands fabricants similaires des autres pays.\par
Mais c’est là qu’est le danger. Car, précisément parce qu’ils sont étrangers les uns aux autres, séparés par la nationalité, la religion, la race, les mœurs, les concurrents ici deviennent facilement des ennemis et se traitent comme tels ; tandis que, chez les rivaux de la petite industrie locale, l’âpreté de la rivalité est adoucie par les rapports habituels de bon voisinage et de concitoyenneté. De là le fait redoutable que la concurrence des grands industriels, d’une nation à l’autre, est devenue une nouvelle source de guerres, comme on l’a vu notamment par la guerre des États-Unis contre l’Espagne, où une presse payée par des industriels intéressés à l’exploitation de Cuba a mis le feu aux poudres. Et tout semble présager que cette source grossira de plus en plus, pendant que diminueront peut-être, sans jamais tarir, les sources anciennes, jaillies des orgueils et des amours-propres nationaux et de causes politiques. On peut craindre qu’il n’y ait point compensation toujours.\par
Est-ce que le progrès de la civilisation tend à multiplier aussi entre les consommateurs d’un même article les causes d’antagonisme, et à restreindre le sentiment de la solidarité ? Il faut distinguer. Plus les consommateurs sont nombreux, plus le prix s’abaisse ; et chacun d’eux est par suite, intéressé à ce que ses désirs soient partagés, à ce que ses besoins se répandent. Mais il n’en est ainsi que pour les produits dont la fabrication est illimitée et ne devient pas plus mal aisée en s’étendant. Quant aux marchandises dont la production est renfermée dans des limites qui peuvent, il est vrai, se dilater, mais moins vite et plus mal aisément que ne progresse la  \phantomsection
\label{v2p72}consommation correspondante, la hausse de leur prix est une suite évidente du nombre croissant de leurs consommateurs ; et chacun d’eux a intérêt à ce que les autres soient moins nombreux. Or, les produits qui sont dans ce dernier cas sont d’un ordre particulièrement important : ce sont les objets d’alimentation d’une part, et, d’autre part, les objets de grand luxe. A l’égard de ces deux catégories si différentes d’articles, la civilisation n’a cessé d’étendre et d’aviver le sentiment de la rivalité, de l’hostilité presque, entre leurs co-consommateurs. Jadis, le consommateur de blé ou de viande, en Normandie, ne pouvait voir un rival dans le consommateur de blé ou de viande en Provence, ou même en Bourgogne. Mais, à présent que le sac de blé ou la bête à cornes peuvent être si facilement, et à si faible prix, transportés de Marseille à Rouen, les Normands, en temps de famine, verraient dans les Marseillais, aussi bien que dans les Parisiens ou les Picards, des obstacles à la satisfaction de leur faim. Heureusement, ce ne sont pas seulement les blés et les bestiaux d’une province éloignée, mais ceux encore d’Amérique ou d’une autre terre lointaine, qui peuvent être importés chez nous, et, comme il est bien improbable qu’un déficit de récoltes se produise à la fois sur tous les continents, cette extension du rayon de la concurrence entre consommateurs aura eu pour effet, finalement, de la rendre beaucoup moins âpre et d’écarter tout risque sérieux de famine. Mais il n’en est pas moins vrai que, un jour ou l’autre, cette chose invraisemblable peut arriver. Et, dans ce cas, les mangeurs de pain de chaque pays se sentiront pour adversaires, pour ennemis mortels, les mangeurs de pain du monde entier.\par
De là la gravité, la majesté des questions sociales à notre époque. Le rayon de l’envie et de la haine possibles, à partir de chaque cœur, s’est prodigieusement étendu, le nombre des personnes à envier et à détester éventuellement s’est augmenté dans des proportions fabuleuses. C’est d’autant plus redoutable que, à la différence de l’amour, amorti par  \phantomsection
\label{v2p73}l’absence, la haine s’avive par l’absence même, par la distance et l’ignorance de son objet. Moins on se connaît, plus on peut se haïr, au moindre obstacle qu’on s’oppose. On a beau être gêné dans l’expansion de ses désirs par la rivalité d’un compatriote, d’un concitoyen, d’un collègue, on ne le haït pas, on ne sent naître l’animosité contre lui qu’à la longue, à force de froissements et de gênes réciproques ; et il faut des griefs accumulés pour qu’on arrive à se haïr jusqu’à s’armer les uns contre les autres ; mais, s’il s’agit d’un étranger lointain, d’un Chinois, d’un Turc, d’un Africain, il suffit d’une contrariété légère d’intérêts pour que la guerre éclate. — On doit bien prendre garde à cette considération si l’on ne veut pas se faire d’illusion, cruellement déçue, sur la durée, sur la stabilité de l’équilibre de la Paix entre les nations modernes.\par
Dans un trop court passage sur « l’histoire de la concurrence »\footnote{ \noindent \emph{Nouveau dictionnaire d’économ. pol.} V\textsuperscript{o} \emph{Concurrence}.
 }, M. Beauregard dit que la concurrence a commencé par exister entre les groupes et s’est développée en se produisant entre les individus. Cette vue ne me semble pas juste : le clan primitif, qui est, avec la famille, le plus ancien groupe connu, ne rivalisait en rien, au point de vue économique, avec les clans voisins ; et, dans chaque clan, chaque maisonnée formait un phalanstère, se suffisant à lui-même. Quand la concurrence est née, c’est entre les individus qui, par leur initiative vagabonde, se détachaient de ces groupes ; et cet individualisme n’a été qu’une étape intermédiaire entre le collectivisme familial, d’où il procède, et le collectivisme corporatif, puis civique, où il aboutit. Alors les corporations entrent en conflit, mais la rivalité entre elles a été bien faible d’abord, car, à l’époque où elles sont nées, le rayon de la concurrence possible entre les producteurs d’un même article était très borné, à cause de la difficulté des communications et des transports. C’était entre  \phantomsection
\label{v2p74}les marchands voyageurs, entre les aventuriers du négoce, qui franchissaient au prix de leur vie les fleuves et les monts hérissés de châteaux pillards, que la concurrence se faisait sentir avec force ; c’est à ces premiers pionniers du commerce, puis de l’industrie individuelle, qu’est due, par degrés, la brillante floraison de notre industrie moderne. Mais n’est-il pas visible que ce nouvel accès d’individualisme nous mène tout droit à une nouvelle et plus ample phase d’association où la concurrence entre individus renaîtra, renaît déjà, sous la forme de trusts gigantesques qui guerroient à coups de tarifs, de syndicats unis et rangés en ordre de bataille qui entament des campagnes émouvantes, sur une ligne immense, où il y a des morts souvent, et surtout des blessés sans nombre... Ainsi, à l’inverse de l’évolution indiquée par la formule de M. Beauregard, je dirais plutôt que la concurrence va des individus aux groupes, et à des groupes de plus en plus vastes.\par
Soit individuelle, soit collective, d’ailleurs, la concurrence met toujours en présence deux volontés sciemment et délibérément antagonistes, la volonté de deux industriels rivaux ou la volonté de deux chefs de sociétés rivales qui travaillent par tous les moyens, licites ou même illicites, à s’exterminer commercialement. Il y a là, dans cette guerre commerciale, dont les obus et les boulets s’entre-croisent continuellement au sein de la Paix la plus profonde en apparence, une extraordinaire dépense d’inventions qu’on ne remarque pas assez. Dans son bel \emph{Essai sur l’Imagination créatrice}, M. Ribot développe cette idée fort juste que l’imagination commerciale, dans ses procédés, ressemble à l’imagination militaire. Chez le grand négociant qui a l’idée d’une opération, qui la médite longtemps, qui, après l’avoir conçue et méditée, la modifie à chaque instant, d’après les informations télégraphiques et épistolaires reçues de ses nombreux correspondants, le commerce est « une forme de la guerre ». Outre l’intuition initiale qui révèle l’affaire et  \phantomsection
\label{v2p75}le moment opportuns, l’imagination commerciale suppose un plan de campagne bien étudié dans les détails, pour l’attaque et pour la défense, un coup d’œil rapide et sûr à tous les moments de l’exécution pour modifier ce plan incessamment. »\par
Mais gardons-nous de nous méprendre sur la nature de ces inventions quasi militaires que la lutte industrielle ou commerciale fait jaillir de part et d’autre, comme le combat fait inventer les ruses de guerre. On aurait tort d’en conclure, avec l’école classique, que la concurrence est la mère du progrès industriel et commercial, la source abondante de la richesse. Ces inventions dont il s’agit, ces inventions nées de la bataille, ne doivent pas être confondues avec les inventions nées de la paix, du loisir, de la recherche désintéressée et amoureuse. Par celles-ci les sciences avancent, et, à la suite des sciences, les industries ; par celles-là certains industriels s’enrichissent et d’autres sont ruinés, ce qui est bien différent\footnote{ \noindent Ajoutons que cette sorte d’imagination se répète fort. D’après Claudio Jannet, les procédés plus ou moins abusifs employés par les groupements modernes de négociants, par les trusts ou kartels, sont semblables à ceux que pratiquaient les \emph{guildes} de marchands, au moyen âge, en Allemagne, en Angleterre, en France (p. 307).
 }. Les grands progrès de l’armement et de la technique militaire sont-ils dus à la guerre ? Non. Pendant vingt ans de guerre sous la Révolution et l’Empire, la balistique n’a pas fait un pas ; et c’est pendant la longue paix qui a suivi 1815 que s’est lentement préparée la transformation profonde des armes à feu, comme un simple corollaire des progrès de la chimie et de la métallurgie, dues à des recherches de laboratoire. Ce sont des savants et des ingénieurs qui, paisiblement, par le réseau des chemins de fer, par les télégraphes, les téléphones, les bicyclettes, l’aérostation, ont révolutionné tout l’art de la guerre, de la guerre terrestre et de la guerre maritime. Dans une paix profonde ont été inventés les torpilleurs ; dans une paix profonde se sont accumulées depuis trente ans les inventions  \phantomsection
\label{v2p76}militaires les plus transformatrices que le monde ait vues, depuis l’invention de la poudre à canon, née dans le creuset de tranquilles alchimistes. Il est vrai que les guerres de l’Empire ont provoqué l’éruption du génie de Napoléon ; mais les idées géniales qui lui ont fait gagner des batailles sont, comme toutes les inventions de circonstance imaginées par les généraux ou les hommes d’État, d’une nature toute spéciale. Elles sont utiles à la condition de n’être pas imitées par les adversaires, tandis que les inventions fondamentales, qui constituent le trésor héréditaire et sans cesse grossi de l’humanité, sont destinées à être imitées et doivent leur fécondité à leur imitation même. Plus elles débordent, comme le Nil et le Niger, plus elles fertilisent.\par
Ce qui est vrai des luttes militaires l’est aussi bien des luttes industrielles. Ce n’est pas elles qui ont fait découvrir la puissance de la vapeur, la conversion des forces physiques, le transport de l’électricité à distance, toutes les connaissances scientifiques dont les machines modernes ne sont qu’une application et un emploi. La plupart même de ces machines n’ont point été suscitées par la concurrence. La concurrence, par son action directe, n’a stimulé vraiment d’autre inventivité que celle de la réclame, ce Protée aux mille formes ; et là, par exemple, il faut reconnaître qu’elle a eu une prodigieuse efficacité. Mais ces ingéniosités de la réclame, comme celles des politiques et des grands guerriers, ont ce caractère de perdre leur utilité à mesure qu’on les imite, parce qu’elles se neutralisent en s’imitant\footnote{ \noindent Les \emph{fausses statistiques} sont une variété perfectionnée, réservée à notre âge, du mensonge et du faux. — On fait de fausses statistiques de nos jours pour la même raison qu’au moyen âge on faisait de \emph{fausses décrétales.} Quand on ne croit qu’aux chiffres, il faut tromper à l’aide des chiffres, comme, lorsqu’on croyait à l’infaillibilité du pape, il fallait tromper à l’aide d’un document pseudo-pontifical.
 }.\par
Ce n’est point seulement l’art de la réclame, c’est aussi, par malheur, l’art de la contrefaçon, de la falsification, du mensonge lucratif sous les déguisements les plus variés,  \phantomsection
\label{v2p77}qui se déploie par l’action de la concurrence. La camelote, la pacotille, les aliments frelatés, sont ses enfants naturels et légitimes. Quand un monopole s’est établi, — par l’effet même de la concurrence, car le monopole naît de la concurrence aussi inévitablement que la conquête résulte de la guerre, — on est frappé de ses abus ; mais on oublie les abus antérieurs dont les siens ne sont que la reproduction, le plus souvent atténuée et partielle. Je prends l’exemple le plus défavorable à ma thèse : les monnaies. Le monopole de battre monnaie a paru la source des altérations monétaires si justement reprochées à la monarchie française, surtout sous les Valois. Mais, quand tous les seigneurs battaient monnaie, il y avait autant de faux-monnayeurs que de seigneurs, et c’est comme moins impure et plus répandue que la monnaie royale a fini par s’établir au-dessus de toutes les autres, qu’elle a englouties. Il est vrai que, lorsque le droit de monétiser a été monopolisé par le roi, il n’a pas tardé à en abuser ; mais, en somme, ce monopole a été finalement utile, même au point de vue de la pureté des métaux, et l’unité qu’il nous a value est incontestablement préférable à l’incohérente diversité d’autrefois. Soyons certains que, si la monnaie était soumise, comme une marchandise quelconque, au régime de la libre concurrence, ce serait — comme le remarque, je crois, Fourier quelque part — le règne absolu de la fausse monnaie. Et, à ce propos, puisque, d’après les économistes classiques, la monnaie est une marchandise comme une autre, on ne voit pas pourquoi ils ne préconisent pas le \emph{laisser-faire} en ce qui la concerne aussi. Ou bien, s’ils conviennent que la \emph{libre frappe} aurait ses inconvénients et que le monopole de l’État se justifie à cet égard, on ne voit pas pourquoi ils élèvent des objections radicales contre tout autre monopole industriel de l’État.\par
Nous venons de dire en passant que le fruit naturel de la lutte économique est le triomphe économique, le monopole,  \phantomsection
\label{v2p78}où elle aboutit fatalement\footnote{ \noindent Je dis fatalement, à raison même des progrès de l’outillage et de la fabrication, et de l’énormité des capitaux employés. Un grand industriel, de nos jours — ce qui eût semblé paradoxal autrefois — a parfois intérêt à produire à un prix qui est inférieur au prix de revient ; car il perd moins en fabriquant à perte qu’il ne perdrait en arrêtant sa production ou même en la ralentissant. L’un d’eux, américain, M. Carneggie, dit avoir connu des manufacturiers qui ont produit de la sorte pendant des mois ou des années — jusqu’au moment où un \emph{trust} s’est formé entre eux, \emph{comme la seule solution possible} de cette situation critique.
 }, après des péripéties plus ou moins prolongées. La distribution actuelle de la richesse aux États-Unis en est une preuve saisissante. Là on a vu la concurrence se donner libre carrière ; et, pendant longtemps, les résultats étaient faits pour confirmer en apparence les prévisions des économistes. Une population nombreuse vivait dans une médiocrité universelle, sans indigence ni opulence. Mais, à présent, depuis la guerre de Sécession, les choses ont bien changé\footnote{ \noindent Voir, à ce sujet, les \emph{Grandes fortunes aux États-Unis}, par M. de Varigny (1888) et les \emph{Trusts} américains, par M. de Rouziers (1899).
 }. Les États-Unis donnent le spectacle, dit M. de Varigny, « d’une accumulation énorme de capitaux dans un petit nombre de mains, d’immenses fortunes à côté de grandes misères, conséquences inéluctables de la grande industrie, de la grande propriété, se substituant, par la force des choses, à une production restreinte, à une aisance moyenne mais générale ». La démocratie américaine voit ainsi surgir de son sein des rois variés, aux royaumes entrelacés mais gigantesques, le roi du pétrole, le roi du fer, le roi du cuivre, etc. Elle voit des industries vitales, nécessaires, monopolisées par quelques hommes ; et, chose plus étrange, elle n’a pas trop à se plaindre jusqu’ici d’être exploitée par eux plutôt que par la foule des petits concurrents qu’ils ont écrasés. De même que les grands magasins, les grandes usines vendent à meilleur marché.\par
La concurrence entre les compagnies de chemins de fer, quand elle a lieu, ne peut se terminer que par leur accord, tant leur intérêt à s’accorder est manifeste. (V. Colson,  \phantomsection
\label{v2p79}\emph{Transports et tarifs}, à ce sujet.) Cette question de la concurrence des chemins de fer s’est posée partout. Elle a reçu des solutions diverses. En France, « on peut dire que c’est avec l’approbation et même les encouragements de l’administration que l’accord s’est établi pour le partage amiable du trafic entre les divers réseaux. Aussi y a t-il eu, dans la création de notre réseau, fort peu de doubles emplois entraînant un gaspillage inutile de capitaux, et, si son développement a été parfois trop rapide, du moins est-il resté logique et rationnel. »\par
Nous n’avons rien à envier ici aux Américains ni aux Anglais. En Angleterre, après une période d’engouement, de 1845 à 1848, où des multitudes de concessions ont été accordées à des Compagnies, des fusions s’opérèrent entre celles-ci, qui se trouvèrent réduites à 8 ou 10, nombre à peu près égal à celui des 6 grandes compagnies françaises. Ces compagnies anglaises avaient des réseaux enchevêtrés. Après une guerre acharnée, pendant laquelle le public souffrait de l’inégalité et de la variabilité la plus intolérable des tarifs, elles ont fini par s’accorder. Mais, comme il a fallu avoir des prix basés sur les capitaux incalculables engloutis dans tous ces tâtonnements, « les chemins de fer anglais sont les plus chers de l’Europe », quoique les compagnies ne réalisent pas beaucoup de bénéfices.\par
« Aux États-Unis, une législation semblable à celle de l’Angleterre a amené une situation analogue, avec cette différence, qu’au lieu d’être arrivées à la période de la coalition, les Compagnies en sont encore aux tentatives d’entente » (c’est-à-dire qu’elles sont \emph{en retard} sur les nôtres) (Colson écrivait en 1892 ; je crois qu’à présent la phase de \emph{coalition} est plus avancée en Amérique). « \emph{Malgré la concurrence}, les prix du transport \emph{des voyageurs} sont restés plus élevés aux États-Unis qu’en France. » Pour les marchandises, c’est différent.\par
Les voies navigables font aux chemins de fer une concurrence \phantomsection
\label{v2p80} intéressante. Les fleuves et canaux sont le plus souvent battus dans cette lutte, parce que, contrairement à ce qu’on croit, le transport par eau (par eau douce) est plus cher, somme toute, que par les voies ferrées. Mais, en revanche, les chemins de fer sont partout battus par les bateaux marins. — Là aussi, il y a finalement coalition ou défaite complète, et monopole.\par
La libre concurrence des chemins de fer américains donne lieu à de grands abus qui sont un argument très fort en faveur soit des chemins de fer d’État soit de la surveillance par l’État des grandes compagnies de chemins de fer. Les Compagnies américaines se prêtent à des manœuvres des \emph{trusts} pour écraser les concurrents : elles abaissent leurs tarifs en faveur des industriels qu’elles protègent, qu’elles ont intérêt à protéger, et les élèvent contre les industriels rivaux. Sans ce concours des Compagnies de chemins de fer, les trusts seraient infiniment moins puissants.\par
Or, nous sommes choqués de voir ces inégalités de tarif. A vrai dire, pourquoi ces inégalités-là nous paraissent-elles injustes et insupportables, alors que le fait même de la construction d’une ligne ferrée ici plutôt que là, plus près ou plus loin de telle ou telle usine, constitue une inégalité non moins criante, non moins préjudiciable aux unes et utile aux autres ?\par
Jusqu’ici, il est vrai, tous ces monopoles, aboutissements de la concurrence, ont été partiels et passagers. Aucun d’eux n’a pu s’exercer sur toute la surface du globe, aucun d’eux n’a pu se maintenir indéfiniment. Mais cela tient à deux causes : d’une part, le progrès des moyens de transport et de communication n’est pas achevé, il se continue toujours, ne cessant d’abaisser les tarifs et de faire surgir, contre tout monopole momentané, des concurrents inattendus, venus de plus loin ; d’autre part, l’ère des nationalités escarpées et closes n’est pas encore terminée, et nous n’avançons que  \phantomsection
\label{v2p81}lentement dans la voie, soit de l’impérialisme, soit de la fédération, deux embouchures entre lesquelles auront à opter les courants multiples de nos évolutions nationales dans leur marche progressive vers l’union ou l’unité. Mais, quand cette union ou cette unité se dessinera par-dessus les nationalités estompées et adoucies, et quand le progrès des communications et des transports aura atteint son terme (en attendant quelque grande invention ultérieure peut-être), des monopoles à la fois universels et durables seront possibles.\par
Voilà ce que l’école socialiste n’a point tort de mettre en lumière, mais ce qui, à vrai dire, ne me paraît point lui donner le droit de conclure que l’État, finalement, l’État « mondial », mettra la main sur tous ces monopoles dont il sera la concentration et la consolidation supérieures. On peut concevoir une autre issue de l’évolution économique, surtout si l’évolution politique opte pour la forme fédérative de préférence à la forme impérialiste.\par
La logique n’autorise pas non plus la même école à déduire de la transformation des petites industries morcelées en grandes industries centralisées, la nécessité d’une transformation analogue qui ferait passer l’agriculture du régime de la petite propriété à celui des \emph{latifundia}. D’abord, n’envisager la question de la propriété immobilière agricole, qu’au point de vue économique, disons que c’est en méconnaître la vraie importance, qui est, avant tout, d’ordre politique et d’ordre moral. Mais, même au point de vue restreint où l’on se place, où voit-on se manifester, en fait, la tendance qu’on imagine ? M. Vandervelde a fait de vains efforts, en combinant ingénieusement les chiffres des statistiques belges, pour y apercevoir une diminution de la propriété paysanne. Il n’y a pas réussi. M. Karl Kautsky, dans son livre si touffu et si documenté sur la \emph{Question agraire} (1900), où il met en si haut relief la supériorité de la grande exploitation sur la petite, conclut, non pas que celle-ci se  \phantomsection
\label{v2p82}transforme en celle-là, mais que le morcellement de la propriété se continue à côté d’une certaine concentration de la propriété qui s’opère, et qu’en somme on ne doit guère « compter sur une rapide absorption des petites propriétés par les grandes ». Y a-t-il lieu de s’en étonner ? Nullement. Que la concurrence industrielle mène au monopole industriel, je le veux ; mais où est la concurrence agricole qui nous mènera au monopole agricole ? Les agriculteurs voisins se font-ils vraiment concurrence ? Leur production est limitée par sa nature même, et il ne dépend pas de leur volonté de la doubler, de la tripler, pour la rendre très supérieure aux besoins de la consommation. Les cultivateurs d’une même vallée, d’un même plateau, sont des collaborateurs : s’ils étaient moins nombreux, ou s’ils travaillaient moins, quelques-uns des consommateurs habituels de leurs produits mourraient de faim ou souffriraient de grandes privations. — A la vérité, les agriculteurs américains ou russes font, en ce moment, une concurrence véritable et terrible à nos laboureurs français ou anglais ; mais c’est là un fait exceptionnel, anormal, que personne n’aura l’idée de comparer à la rivalité de deux usines ; c’est une vraie crise qui prendra fin soit par un changement radical et complet de culture soit de toute autre façon. En tout cas, pas plus que les agriculteurs français ou anglais entre eux, les agriculteurs américains entre eux ne se font concurrence ; ils sont bien d’accord pour produire le plus de blé possible et en inonder le plus possible les vieux pays. Ce n’est pas du tout par la concurrence que se sont formés les \emph{latifundia} du \emph{far-West}, c’est à raison de la non-valeur d’un sol vierge qui s’offrait à peu près pour rien aux premiers défricheurs. Quant à la concurrence que se font des groupes d’agriculteurs séparés par l’Atlantique, ce n’est pas celle-là qui peut favoriser l’extension des grandes propriétés. Les domaines qui résistent le mieux en France aux chocs venus d’Outre-mer, ce sont les petits champs des paysans à qui le  \phantomsection
\label{v2p83}prix du blé importe assez peu puisqu’ils consomment celui qu’ils produisent. Quand l’agriculture met en lutte, par exception, les agriculteurs voisins les uns contre les autres, c’est qu’elle s’\emph{industrialise ;} et, à mesure qu’elle s’industrialise, en effet, par la culture des betteraves, par exemple, en vue de l’industrie sucrière, elle fait sentir aux propriétaires d’un même pays une certaine rivalité. Mais des questions si graves ne sauraient être traitées en courant.\par
Revenons aux effets de la concurrence. Si nous avons refusé de lui reconnaître cette vertu créatrice de l’Invention, qui lui est attribuée à tort, accorderons-nous au moins qu’elle a une influence souveraine et toujours bienfaisante sur l’abaissement des prix ? Nous ne l’accorderons pas sans beaucoup de réserves. C’est seulement dans les marchés \emph{en gros}, d’après Stuart Mill, que la concurrence exerce une action prépondérante sur la détermination des prix. « Mais, dit-il, le prix du détail, le prix payé par le consommateur réel, semble ne ressentir que très lentement et très imparfaitement l’effet de la concurrence ; et, lorsque la concurrence existe, \emph{le plus souvent, au lieu de faire baisser les prix, elle ne fait que partager les profits résultant de l’élévation des prix entre un plus grand nombre de marchands}. » On s’explique ainsi pourquoi les boulangers se multiplient sans que le prix du pain s’abaisse, ou presque pas, pendant que l’offre du blé devient deux ou trois fois plus forte.\par
On voit que l’effet de la concurrence est très ambigu à cet égard. Or, pourquoi, dans certains cas, fait-elle réellement baisser les prix, et, dans d’autres cas, fait-elle simplement augmenter le nombre des bénéficiaires du prix existant ? Pourquoi, quand il s’agit de professions libérales (médecins, avocats, etc.) où les honoraires sont réglés par l’usage, la survenance de nouveaux médecins ou de nouveaux avocats fait-elle diminuer les bénéfices des anciens médecins ou avocats, mais ne diminue-t-elle en rien leurs tarifs habituels ? Et pourquoi, dans la plupart des professions  \phantomsection
\label{v2p84}manuelles, en est-il de même ? L’action souveraine des jugements sur les prix est ici manifeste : on juge tel prix seul juste, seul honorable, et l’on s’y tient ferme.\par
Mais, même là où un abaissement du prix usuel n’est point jugé une injustice ou une inconvenance, le simple sentiment de l’intérêt commun suffit souvent pour empêcher la diminution du prix. Les nouveaux venus dans la profession sentent qu’ils ont plus d’intérêt à maintenir le prix qu’à l’abaisser pour faire pièce à leurs rivaux et prédécesseurs. L’abaissement du prix leur procurerait un avantage momentané mais au détriment de leur avantage futur. S’ils ont lieu d’espérer que, même sans changer les prix, ils se feront leur place au soleil, ils se garderont bien d’y toucher. Ajoutons que les concurrents sont toujours et partout des collègues, et, comme tels, liés par un certain sentiment de confraternité. Si l’un d’eux est tenté d’abaisser le prix commun, il est retenu par la crainte d’être mal vu et tenu à l’écart par les autres. Il a un double intérêt, sympathique et pécuniaire, à se conformer aux usages. Enfin les nouveaux arrivants d’un métier n’ont pas toujours, ni le plus souvent, besoin d’abaisser les prix pour offrir à une partie des consommateurs un réel avantage. En effet, il leur suffit d’être plus à portée de certains consommateurs, plus près d’eux ou en communication plus facile avec eux. La plupart des boulangers, des bouchers, des cordonniers, etc., qui s’établissent dans une localité, dans un quartier, y ont été en quelque sorte appelés par les besoins d’une clientèle toute trouvée, composée de tous ceux qui, au même prix, trouveront plus commode d’aller chez eux. C’est seulement, cas exceptionnel, lorsque le nouveau venu ne présente aucun avantage de ce genre à un nombre suffisant de consommateurs, qu’il lui arrive d’abaisser les prix pour amorcer le client, sauf à les relever ensuite. Et c’est ce que savent la plupart des consommateurs. Aussi se méfient-ils de ces appâts grossiers et ne se décident-ils jamais sans hésitation à  \phantomsection
\label{v2p85}quitter des fournisseurs dont ils sont contents pour s’attacher à un nouveau commerçant qu’ils ne connaissent pas encore.\par
Ne peut-il pas arriver, n’arrive-t-il pas quelquefois, que, loin d’avoir pour effet un abaissement des prix, la concurrence des offreurs, des vendeurs, provoque leur élévation ? Ce cas peut se produire toutes les fois que les vendeurs sont syndiqués ou associés par des liens assez intimes pour leur permettre de sentir clairement leur véritable intérêt et de se le faire sentir les uns aux autres. Or, si un nouveau venu vient s’ajouter à leur groupe — un nouveau boulanger, par exemple, ajouté au nombre des boulangers d’une ville — et que le prix habituel soit maintenu, le résultat sera que les anciens boulangers, à ce prix, gagneront moins. La concurrence aura produit, même sans abaissement de leurs prix, un amoindrissement de leurs bénéfices. Comment faire pour éviter cet inconvénient ? Il n’y en a qu’un, c’est de s’entendre, et, d’un commun accord, d’élever un peu le prix de manière à ce que la même clientèle, répartie sur un plus grand nombre de commerçants, donne à chacun des anciens un bénéfice égal à celui d’autrefois. N’est-ce pas pour cela, au fond, — en vertu d’une sorte d’accord inconscient et spontané — que les honoraires des médecins, des avocats, et d’autres professions, se sont élevés à mesure que le nombre des médecins, des avocats, etc., augmentait très vite ? Et ne semble-t-il pas qu’ici les prix, les honoraires aient été non pas en raison inverse mais plutôt en raison directe de l’offre croissante, la demande étant d’ailleurs demeurée à peu près la même ?\par
Est-ce que — phénomène corrélatif et complémentaire du précédent — il n’arrive pas aussi bien que l’augmentation survenue du nombre des demandeurs, des acheteurs, ait pour effet non pas d’élever mais au contraire d’abaisser le prix ? Quand la clientèle d’un négociant est en voie d’accroissement, il a souvent intérêt, pour accélérer la vitesse de cette  \phantomsection
\label{v2p86}progression, à diminuer ses prix, ce que cette progression même d’ailleurs, lui rend possible et de plus en plus aisé. C’est l’explication du fait, signalé avec étonnement par certains observateurs, que, dans les industries monopolisées d’Amérique, les prix se sont abaissés au lieu de s’élever.\par
— En résumé, la concurrence est loin d’avoir mérité les hymnes enthousiastes entonnés en son honneur par des générations d’économistes. Les vertus magiques qu’on lui a prêtées sont celles de deux grandes facultés mentales qui ont souvent travaillé pour elle, dans l’ombre, mais plus souvent encore sans elle : l’Imagination créatrice, source des inventions, et l’Intelligence calculatrice, qui répand, exploite, utilise les inventions et en tire le meilleur parti. Par son action indirecte, en tant qu’elle a stimulé ces deux forces, la lutte économique a servi au développement de l’industrie ; elle a puissamment aidé l’initiative individuelle — que les économistes ont eu le grand mérite, le mérite immense et inoubliable, d’avoir préconisée ; — mais, par son action directe, immédiate, elle n’a été féconde qu’en réclames et en falsifications, en déploiement multiforme du mensonge. — Je ne voudrais pas qu’on se méprit sur ma pensée : je ne conteste pas, somme toute, son utilité. Elle a le mérite, comme toute lutte, politique, religieuse, esthétique, de tendre tous les ressorts de l’être individuel vers un but clair et précis qui élève au plus haut point de finalité l’organisme vivant ; qui apaise momentanément tous ses troubles, toutes ses oppositions intérieures, et, grâce à cet état d’harmonie parfaite, le rend créateur. L’effort qu’elle suscite est précisément le contraire d’elle-même, c’est-à-dire un acte d’association organique, de collaboration corporelle et cérébrale intense, sans nulle discordance intérieure. Et c’est cet effort qui est fécond et rénovateur. Et c’est une erreur profonde, injurieuse pour l’humanité, pour l’ordre universel, de penser que la lutte, la contradiction, l’obstacle réciproque, la mutuelle destruction partielle ou totale, soit l’unique ou le principal stimulant \phantomsection
\label{v2p87} de l’effort producteur. Cette erreur, sans doute, n’est pas propre aux économistes, ils l’ont empruntée aux naturalistes qui ont été séduits longtemps, magistralement il est vrai, par l’idée paradoxale de voir dans la bataille continuelle des vivants la cause fondamentale des progrès de la vie, dans le meurtre généralisé des individus la création même des espèces. Et, certes, il est bon que ce paradoxe ait été poussé à bout par le génie d’un Darwin, puisque, à présent, il reste établi que la sélection naturelle, excellent agent d’élimination épuratrice, ne crée rien et postule ce qu’elle prétend expliquer, les rénovations vivantes, sous la forme des variations individuelles, et que le secret de ces créations de la vie se cache à nos yeux dans la profondeur de l’ovule fécondé au lieu de consister dans le choc extérieur d’organismes qui se combattent. Pareillement, il est bon que la même apologie de la guerre et des combats, la même tentative d’expliquer par la concurrence les merveilles du génie humain, se soit produite en économie politique ; car sa stérilité définitive a permis de mieux voir ce que le prestige belliqueux de l’idée d’opposition nous masquait : le rôle capital de l’invention, née de recherches prolongées dans la solitude, le recueillement et la paix. Le plus grand effort, en effet, et le plus fructueux, n’est pas celui que provoque la lutte, c’est celui que suscite une curiosité vive, une foi ardente, une grande idée entrevue et poursuivie à travers les ténèbres. Le vrai professeur d’Énergie, ce n’est point un soldat ni un capitaine, c’est un Archimède, un Newton, un Pasteur, un Lavoisier, un Kant, un Auguste Comte.\par
On me pardonnera de revenir avec insistance à la charge de l’erreur que je combats. Elle n’est point seulement propre à fausser l’esprit, mais à pervertir le cœur. Elle consiste à croire, au fond, que, derrière la toile où se tissent les événements humains, il y a une sorte d’ironie méphistophélique, déconcertante, qui s’amuse à faire naître le bien du mal et le mal du bien, à douer de fécondité salutaire la haine meurtrière, \phantomsection
\label{v2p88} l’exaspération et le conflit belliqueux des égoïsmes et des rapacités, et à rendre stériles ou nuisibles l’amour, la foi, le désintéressement, l’abnégation. Désolante doctrine dont il y aurait à déplorer la vérité, tout en l’enseignant, si elle était vraie, mais qui, démontrée fausse, doit être extirpée radicalement, parce qu’elle est un encouragement au mal vanté par elle, et qu’elle paralyse les élans généreux frappés par elle d’impuissance. Ne voit-on pas ce que la propagation graduelle de la théorie de la concurrence vitale et de la sélection a déchaîné de convoitises féroces entre les nations et entre les classes ? Il a fallu une société saturée du droit de la force, bien ou mal déduit de ces hypothèses, pour rendre possible cette somme énorme d’attentats contre le faible ou le vaincu que, sous le nom de \emph{politique coloniale.} ou de \emph{lutte des classes}, nos hommes d’État européens déjà pratiquent ou nos théoriciens justifient d’avance.\par
Tout ce qu’il y a de mieux à dire en faveur de cette relation anarchique et irrationnelle qu’on appelle la guerre, non pas pour la vanter, mais pour l’excuser, c’est qu’elle est une triste nécessité, une voie douloureuse qu’il faut suivre avant d’atteindre, par la victoire et la conquête, à son terme inévitable. Ce qu’il y a d’utile et d’avantageux, ce n’est pas la lutte, c’est, grâce au traité de paix où elle aboutit un jour ou l’autre, une pacification passagère, puis l’agrandissement de la lutte ultérieure. Car, plus la lutte s’agrandit, plus elle se rapproche de son dénoûment final : l’accord durable et universel.
 \phantomsection
\label{v2p89}\subsubsection[{II.3.c. Puis, entre producteurs nationaux et producteurs étrangers du même article. Libre échange et protection.}]{II.3.c. Puis, entre producteurs nationaux et producteurs étrangers du même article. Libre échange et protection.}
\noindent Il a été à peu près exclusivement question, dans ce qui précède, de la concurrence entre producteurs de la même nation, lutte où la loi ne saurait intervenir en faveur des uns ou des autres que par la plus criante injustice, trop souvent réalisée. Il en est autrement quand il s’agit de la concurrence entre les producteurs nationaux et les producteurs étrangers d’un même article ou d’articles similaires. Ici il est naturel que l’intérêt commun des fabricants du pays pousse le législateur, dans certains cas, à les défendre contre l’invasion des produits exotiques. Dans quel cas ? Dans celui où cet intérêt des producteurs est d’accord, au fond, avec celui des consommateurs, c’est-à-dire avec l’intérêt général de la nation, pour diverses raisons : soit parce que, à défaut de cette protection, serait mise en péril une industrie de première nécessité, telle que la production du blé, qui, lorsque l’impérieux besoin de la ressusciter sur le territoire se ferait sentir, au cours d’une guerre par exemple\footnote{ \noindent « Qu’arriverait-il, demande Roscher, à l’Angleterre, si jamais un second blocus continental avait lieu pendant qu’elle serait engagée dans une guerre avec l’Amérique ? » Le danger pour une nation d’être affamée, quand sa subsistance dépend d’arrivages étrangers, n’a rien d’imaginaire. Il y a eu plusieurs blocus continentaux, comme le remarque l’auteur cité. « Un blocus continental fut dirigé par l’empereur Otton II contre Venise en 983. Dans les Pays-Bas, sous Charles-Quint, la famine sévissait chaque fois qu’un démêlé avec le Danemark entraînait la fermeture de la Belgique. Un fait semblable se produisit en 1807 et dans les années suivantes en Norvège quand l’importation ordinaire des grains du Danemark fut interrompue par l’Angleterre. » L’agriculture américaine a cela de particulier que, disposant encore d’une immense étendue de terres neuves, à peu près sans valeur, comme en disposaient les peuples primitifs, elle applique à leur exploitation des procédés perfectionnés de culture que, seuls, des peuples très avancés en civilisation peuvent posséder. Cette combinaison des avantages de la barbarie la plus grossière avec ceux de la civilisation la plus haute est un phénomène unique, qui ne s’était plus vu, qui ne se reverra probablement jamais, ou du moins à ce degré. On comprend donc que pour se défendre contre un débordement des produits dû à cette merveilleuse rencontre, les peuples de l’ancien continent aient recours à des digues momentanées, à des infractions exceptionnelles aux lois économiques qu’ils proclament. Si exceptionnelles qu’elles soient, ces infractions ne sauraient l’être plus que le danger auquel on les oppose.
 }, ne serait  \phantomsection
\label{v2p90}de nature à renaître qu’avec une excessive lenteur et de grandes difficultés ; soit parce que, dans le cas d’une industrie moins nécessaire ou même de luxe, il suffit d’un droit protecteur assez léger pour la conserver au pays, sacrifice minime pour la bourse des acheteurs en comparaison des avantages que trouve l’État à voir grandir ou à ne pas voir décroître des fortunes privées qui se répandent autour d’elles en dépenses de tout genre dont d’autres industries s’alimentent, et aussi en accroissements d’impôts dont bénéficie le Trésor public ; soit enfin parce que, moyennant des droits de douane assez élevés mais jugés temporaires, on espère acclimater de nouvelles industries qui, une fois enracinées et vivaces, s’ajouteront aux sources anciennes de la richesse nationale.\par
Voilà trois cas, nettement distincts, très fréquents, où la protection légale est due aux producteurs nationaux, dans les États les plus éclairés et les plus civilisés. Mais, si l’on remonte aux origines de la civilisation, on s’aperçoit qu’il n’est pas même besoin de la loi, pour protéger les produits du cru. Chaque groupe familial et social, alors, est pieusement ou fièrement attaché, par coutume héréditaire, à ses manières propres de se nourrir, de se vêtir, de se parer, de s’abriter, de s’armer, que chacun répute très supérieures aux articles \emph{similaires} en usage chez les groupes voisins. Ce n’est point par misonéisme, par horreur de la nouveauté, que le primitif répugne ainsi à acquérir les aliments ou les vêtements exotiques ; et la preuve en est que, lorsque des articles répondant à des besoins \emph{nouveaux} apparaissent aux mains de l’étranger, les indigènes se jettent dessus avec toutes les  \phantomsection
\label{v2p91}marques d’une curiosité et d’une avidité extrêmes. Les femmes et les jeunes gens surtout témoignent d’un désir extraordinaire d’obtenir par vol ou par échange ces objets surprenants, à peu près comme les européens, à une certaine époque, se sont engoués des potiches chinoises ou japonaises. Il n’est pas un récit de voyage où un navigateur qui vient de découvrir une île nouvelle ne nous montre les insulaires empressés à lui offrir tout ce qu’ils ont de plus précieux pour obtenir des verroteries ou de simples clous. Il est probable que le commerce international ou \emph{inter-tribal}, d’abord restreint à quelques objets, a précédé le commerce intérieur, auquel il a servi peut-être de modèle. Autrement dit, il y a lieu de penser qu’on a commercé de peuple à peuple avant de songer à commercer de famille à famille. La mésestime générale du négoce, même intérieur, tient peut-être à cette dérivation.\par
A toute époque, donc, même aux plus primitives, même dans l’ère des marchés les plus clos et les plus étroits, le libre échange a été spontanément pratiqué à l’égard de certains articles, en même temps que le protectionnisme coutumier au profit des industries anciennes du clan, de la tribu, du canton. Mais le libre échange est contagieux comme l’imitation d’où il dérive : d’abord étroitement limité, il ne peut tarder à s’étendre. Ce qui n’était qu’une fantaisie passagère et individuelle devient une habitude et une coutume ; après avoir accueilli le tabac ou le café à titre de curiosité, on arrive à ne pouvoir s’en passer, à ranger cette consommation parmi les dépenses de première nécessité. Il n’y a plus de raison pour ne pas échanger avec l’étranger le blé et le vin quand on lui achète du café, du thé, du chocolat. Et c’est ainsi que, peu à peu, toute une petite ou grande région se trouve avoir les mêmes modes d’alimentation, d’habillement, d’ameublement, de divertissement, etc., ce qui la prédestine fatalement à former une seule et même petite ou grande nation, bien unie et bien forte. Qu’en résulte-t-il ? Simplement, \phantomsection
\label{v2p92} que les murs de clôture, maintenant abaissés, dont s’étaient autrefois hérissés les familles, les clans, les castes de cette région, se sont maintenant relevés pour former autour d’elle un vaste rempart définitif. Mais ce rempart aussi aura des brèches où passeront librement les marchandises curieuses, merveilleuses, importées par des trafiquants venus de loin, Phéniciens, Vénitiens, Florentins Anglais. Et ainsi de suite. Les barrières défensives ne s’abaissent donc jamais entre les peuples que pour se reformer plus loin et enfermer une enceinte plus vaste ; et jamais il n’arrive que l’interdiction de commercer avec l’étranger soit complète, c’est-à-dire étendue à toutes les marchandises et à toutes les nations, ni que la liberté du commerce international soit pareillement absolue.\par
Les peuples parvenus à une certaine phase de leur développement, à la phase des traités de commerce, n’ont jamais considéré le libre échange, toujours partiel et relatif, que comme étant, \emph{au même titre que la protection douanière} ou la prohibition même, une arme utile à employer dans la lutte des nations pour la prééminence industrielle et commerciale. Tantôt les droits protecteurs, tantôt la liberté de l’échange, paraissent le moyen le plus propre à atteindre ce but. Cela dépend des articles dont il s’agit ou des nations avec lesquelles on traite. Parfois une nation avisée — l’Angleterre, — a l’air de croire et a l’art de persuader partiellement et momentanément à ses voisins que le libre échange est toujours favorable à tous les contractants au même degré, et des théoriciens s’empressent de dogmatiser et monétiser cette erreur, sciemment ou inconsciemment intéressée. Ce qu’il y a d’abusif dans cette généralisation séduisante ne se révèle à tous les yeux que lorsque, grâce à une application, même partielle et temporaire, de ce principe vulgarisé, la nation d’où il émane a fortement établi la prépondérance de ses industries d’exportation. Car ce mot de \emph{libre échange} est aisément illusoire et trompeur si l’on oublie, comme on  \phantomsection
\label{v2p93}y est trop porté, qu’il est synonyme de \emph{libre achat} et de \emph{libre vente.} S’il était sûr qu’une nation échangeât toujours des marchandises contre des marchandises, des denrées contre des denrées, les produits qu’elle a en excès contre ceux qui lui font défaut, il n’y aurait pas l’ombre d’un doute que cette réciprocité de services serait avantageuse aux deux parties, sans qu’il fût possible de dire, en général, si l’avantage est plus ou moins grand pour l’une que pour l’autre puisqu’il s’agit de choses hétérogènes. Mais il peut fort bien arriver, et il arrive le plus souvent, qu’une nation vende à une autre, pendant un certain temps du moins, beaucoup plus que celle-ci ne lui achète, et qu’elle dépouille ainsi peu à peu celle-ci de ses épargnes antérieures, sacrifiées à des entraînement divers, comme ceux qui ruinent un prodigue en train de « manger son capital » en enrichissant ses fournisseurs ; après quoi la nation enrichie ou bien se détournera de sa victime épuisée pour chercher de nouveaux débouchés, ou bien emploiera ses bénéfices accumulés en capitaux, à bâtir des usines, à créer des chemins de fer, à acheter des terres, un peu partout et notamment chez le peuple dépouillé, dont elle exploitera le sol et le sous-sol, qu’elle repeuplera de ses émigrants et peu à peu colonisera. C’est ainsi que, de tous les moyens de conquête connus, le libre échange, au profit de certaines races, peut être le plus sûr, sinon le plus prompt.\par
Par le libre échange, les nations se battent à visage découvert ; par le protectionnisme, elles se battent avec une armure. Voilà toute la différence, mais elles ne s’en portent pas moins, dans les deux cas, des coups terribles.\par
Ce n’est pas que je conteste la vérité des arguments, devenus lieux-communs, par lesquels l’économie politique démontre l’utilité éminente du libre échange au point de vue de la prospérité générale, comme stimulant de la production, comme impulsion au progrès de la division du travail \phantomsection
\label{v2p94} et de la solidarité économique. Le seul vice de cette argumentation, mais il est grave, est de ne pas tenir compte de la distinction des nationalités et de leur hostilité possible. Cet oubli du morcellement de l’humanité en groupes ethniques est en effet commun aux économistes comme aux socialistes, et les premiers en ont donné l’exemple aux seconds. Dans l’étendue d’un même État, le libre échange, la libre concurrence, établie entre les cantons, entre les provinces dont il se compose, ne saurait tourner, finalement, malgré l’écrasement de bien des industriels sous leurs rivaux triomphants, qu’à l’accroissement de la fortune publique, et à l’organisation la plus progressive, sinon la meilleure, du travail, par la répartition et la localisation intelligentes des industries sur le territoire national. Et, s’il convient, malgré tout, que l’État intervienne souvent, au point de vue de la justice, de la protection des faibles, de diverses considérations politiques et militaires, dans cette mêlée des intérêts concurrents, il importe surtout de ne pas trop l’entraver. Car, si une province se trouve ainsi sacrifiée à d’autres, si la supériorité d’industrie et de richesse se concentre dans telle partie du territoire, la nation dans son ensemble, compensation faite des défaites et des victoires, des gains et des pertes, trouve un avantage manifeste à cette sélection des plus forts et des plus industrieux par la libre bataille. Mais, si ces provinces d’un même État redevenaient des États distincts, est-ce que les provinces qui ont le plus perdu (ou le moins gagné, ce qui revient politiquement au même) à la libre concurrence, n’auraient pas intérêt à se claquemurer derrière des lignes de douane, et pourrait-on alléguer contre cette politique douanière la compensation dont il vient d’être question ? Ce serait se placer à un point de vue supra-national qui se présente toujours à l’état de rêve humanitaire, de plus en plus obsédant, je le sais, et je ne m’en plaindrai pas si, par cette obsession, les nations civilisées sont conduites un jour à cette union fédérative — sinon à cette unité  \phantomsection
\label{v2p95}impérialiste, — qui rendra seule possible le libre échange général et universel.\par
Ainsi, au point de vue national, le libre échange, je le répète, est une arme comme une autre, comme le protectionnisme, comme, aux époques primitives, les secrets de fabrication mystérieuse, dans la guerre économique des peuples\footnote{ \noindent Ajoutons que l’impossibilité de plus en plus grande de maintenir longtemps les secrets de fabrication nationale à mesure que la publicité des informations se répand partout, rend plus nécessaire le recours aux droits protecteurs pour défendre l’industrie nationale contre l’invasion du dehors. De nos jours, on a si bien reconnu l’inanité des efforts contre la divulgation des inventions, même de celles qui ont trait à la défense militaire, que, par les Expositions Universelles, les nations se chargent de publier elles-mêmes leurs découvertes et leurs perfectionnements et d’en faire part à leurs rivaux.
 } ; mais, au point de vue supra-national, le libre échange est un idéal d’harmonie et de paix futures, de désarmement final, qui mériterait d’être poursuivi alors même qu’il y aurait lieu de le regarder — et ce n’est pas le cas — comme à jamais irréalisable.\par
Sans doute le spectacle de la réalité actuelle, de la férocité et de la perfidie vraiment sauvages, dont les peuples réputés civilisés font preuve dans leurs rapports mutuels sous de vains mensonges de politesse diplomatique — et dont ils donneraient des témoignages bien plus manifestes encore sans le dérivatif de leurs colonies où leur soif de sang et de rapines s’assouvit sur les races dites inférieures, — ce spectacle, je le reconnais, est peu propre à nourrir l’espoir de la pacification finale, si ce n’est sous la forme de l’oppression de tous par un seul, le plus fort, le plus rapace et le pire de tous. Imaginez une société \emph{amicale} d’anthropophages blancs qui, tout en dépeçant des étrangers noirs ou jaunes, se regardent les uns les autres de travers, laissant voir clairement le désir ardent qu’ils ont de se manger les uns les autres, ou tout au moins de dévorer une jambe, un bras, une oreille ou un œil à leur voisin, à leur meilleur ami, dès qu’il n’y aura plus de dépeçage extérieur à faire, c’est-à-dire  \phantomsection
\label{v2p96}de « colonisation » : n’est-ce pas là tout à fait l’image du \emph{concert européen} d’aujourd’hui ? Oui, c’est là une chose lamentable. Et cependant, en dépit de ces retours offensifs de barbarie, qui s’expliquent en partie par l’avènement de races plus brutales, sottement admirées de leurs victimes, en partie par la vulgarisation d’une morale adaptée aux instincts de ces races, de la morale du combat pour la vie et du droit de la force, — en dépit de tout, la civilisation continue son œuvre bienfaisante. Contre l’esprit de corps, cent fois détruit et cent fois renaissant sous des formes toujours plus hautes et plus larges, contre l’âpreté de l’orgueil collectif, de l’égoïsme et de l’exclusivisme national qui usurpe le nom de patriotisme, l’esprit de sympathie et d’imitation combat sans cesse, assimilant les peuples, multipliant entre eux les relations, ébréchant les frontières. Et cette lente action, souterraine et continue, prévaudra contre les explosions intermittentes des forces contraires.\par
Il est donc permis à l’imagination philosophique de se transporter en rêve à l’époque, lointaine mais certaine, où, par le prolongement séculaire de cette action assimilatrice et pacificatrice, le globe entier (sous forme fédérative ou sous forme impériale ? là est le point douteux), ne sera plus, politiquement, qu’un seul grand État, composé d’un certain nombre de nations, ou de nationalités, unies ensemble. Alors, c’est incontestable, tous les excellents effets que les économistes ont attribués au libre échange international théoriquement considéré, ou plutôt supra-nationalement, seront appelés à se produire sans le moindre inconvénient. Chaque nation aura sa spécialité industrielle, conforme à sa vocation naturelle ou acquise, d’après son sol, son climat, sa race. Les mieux douées verront leur population s’accroître et s’enrichir ; les plus déshéritées se dépeupleront et s’appauvriront ; elles en seront quittes pour émigrer. Le degré de densité et de prospérité des populations, dans les diverses parties de chaque continent, sera déterminé de plus en plus  \phantomsection
\label{v2p97}par des causes naturelles et rationnelles, de moins en moins par des causes historiques et factices, comme à présent. Il n’y aura plus aucune raison de prohiber ni d’entraver en rien le commerce d’une nation avec une autre.\par
Mais on peut se demander, si, en revanche, le besoin de réglementer la concurrence que les individus se feront entre eux, le libre échange individuel et non plus international, ne se fera pas sentir alors encore plus que de nos jours. Le champ ne sera-t-il pas ouvert à des \emph{trusts}, à des monopoles, beaucoup plus gigantesques que ceux dont on commence déjà à s’alarmer ? Il est vrai qu’à ces alarmes les théoriciens de la libre concurrence ont répondu ingénieusement en rendant le protectionnisme même responsable de ces grands accaparements. C’est, disent-ils, parce que l’Amérique, où se produisent les trusts les plus remarqués, s’est entourée de barrières protectionnistes, que ces grandes centralisations de l’industrie du pétrole, du fer, ou de tout autre article, y sont devenus possibles. Supprimez ces barrières, et des concurrents redoutables, inexpugnables, surgiront en Europe, en Asie, en Australie, contre ces accapareurs fameux. Mais, en vérité, ces penseurs semblent avoir oublié que la terre est ronde, c’est-à-dire qu’elle est une surface limitée et non indéfinie. Ce qu’ils disent sur l’impossibilité pour la libre concurrence d’aboutir à se détruire elle-même en suscitant un roi universel du pétrole, ou, si l’on veut, un empereur du pétrole, du fer, de la laine, du coton, serait vrai si la terre était plate et sans limites ; car, dans ce cas, jamais un industriel ne pourrait accaparer toute la production terrestre en fait de métallurgie, par exemple, ou de n’importe quelle branche d’industrie. Toujours, en dehors des limites de la région accaparée, si vaste qu’elle fût, la concurrence trouverait un asile inviolable. Mais la terre est ronde, et, avec cela, assez petite en somme ; qui plus est, elle se rapetisse chaque jour par la facilité croissante d’en faire le tour et la connaissance plus précise, vue de plus  \phantomsection
\label{v2p98}près, que nous en avons. Non seulement il n’y a plus lieu de regarder comme irréalisable, mais on ne peut même qualifier très difficile l’accaparement de tous les minerais du globe, en ce qui concerne certains métaux. Or, s’il en est ainsi, ne voyons-nous pas que, plus la limitation de la libre concurrence internationale perdra de sa raison d’être par l’agrandissement des nations et leur union finale, plus la libre concurrence inter-individuelle devra être limitée et réglementée ? Et cette perspective n’est pas faite pour rassurer ceux qui, comme moi, font découler d’initiatives individuelles tous les progrès sociaux. Comment conciliera-t-on avec la nécessité de stimuler au lieu de l’assoupir la libre activité des individus initiateurs, si l’on veut la continuation du progrès social, la nécessité non moins impérieuse de mettre un frein aux abus de la force, même de la force du génie et de la volonté ? A nous déjà se pose ce problème, et nous savons avec quelle gravité ; consolons-nous des anxiétés qu’il nous cause en songeant qu’il torturera bien plus douloureusement encore le cerveau de nos petits-neveux.\par
— Encore un mot au sujet du protectionnisme. La question de la protection des industries nationales menacées par la concurrence étrangère n’est pas sans analogie avec la question de l’assistance publique des indigents. L’objection théorique contre l’assistance publique, c’est que, donnant au paresseux la certitude d’être secouru, nourri et logé en dépit de sa paresse, elle le dispense de tout effort sauveur, en sorte que l’assistance, par le fait même qu’elle est établie, tend non pas à diminuer le nombre des pauvres mais à l’augmenter. Elle est un énervement de l’énergie des faibles. Or, cela est peut-être vrai, mais, malgré tout, chacun sent que, lorsque, pour une raison ou pour une autre, un homme est tombé dans la misère et qu’il lui est devenu (ou qu’il lui a toujours été) impossible de vivre de son travail, c’est un devoir moral de l’assister, car lui refuser l’assistance, c’est le condamner à mort. On aura beau être  \phantomsection
\label{v2p99}impitoyable pour les indigents, il y aura toujours des imprévoyants, des imprévoyants par nature, par nécessité de tempérament, qui tomberont dans l’indigence. — N’en est-il pas un peu de même de certains groupes d’ouvriers et de patrons qui, si on ne les assiste par des droits protecteurs, seront condamnés à la mort économique, c’est-à-dire à l’inaction ? On a beau dire, théoriquement, que, arrachés à un travail d’habitude mais peu fructueux, incapables de résister à la concurrence étrangère, ils n’auront qu’à porter leurs bras et leurs efforts vers d’autres formes de travail, plus viables, plus fécondes. \emph{Cela leur est impossible}, à raison de leur nature et des circonstances qui les retiennent à leur sol, à leur famille, à leur village, ou qui les empêchent d’être accueillis dans les ateliers où ils voudraient s’offrir. C’est à cette impossibilité de nature qu’il faut avoir égard pour décider s’il y a lieu de les protéger par des tarifs douaniers — comme c’est à l’impossibilité d’effort présentée par beaucoup d’indigents qu’il faut avoir égard pour décider s’il y a lieu de leur assurer d’avance l’assistance publique, — au risque de hâter leur chute dans la misère en leur donnant cette assurance...\par
Ce problème de la protection, comme celui de l’assistance, doit donc être résolu diversement suivant la diversité des conditions sociales (facilité plus ou moins grande de déplacements, de changements de carrière, d’expatriation, etc.), qui rendent plus ou moins complète, plus ou moins étendue, l’impossibilité de changer de métier quand le métier qu’on a est devenu impropre à lutter contre la concurrence étrangère.
 \phantomsection
\label{v2p100}\subsubsection[{II.3.d. Spécialement, entre production nationale et production étrangère des armes. L’industrie militaire, ses caractères et ses lois propres. Comparaison avec l’industrie religieuse.}]{II.3.d. Spécialement, entre production nationale et production étrangère des armes. L’industrie militaire, ses caractères et ses lois propres. Comparaison avec l’industrie religieuse.}
\noindent Parmi les luttes des producteurs nationaux contre les producteurs étrangers d’articles similaires, il faut ranger, mais mettre à part et en un relief singulier, le cas des industries consacrées à la production d’armes, d’armures, d’engins militaires quelconques. C’est là un sujet très vaste, qu’il nous suffira d’effleurer.\par
Les produits de ces industries militaires ont ceci de spécial de répondre à des besoins essentiellement contradictoires entre eux, puisque leur objet est la mutuelle destruction. Ces produits ne sont destinés, le plus souvent, ni à être vendus ni à être échangés, car, en général, ils sont fabriqués par l’État même qui en a besoin, et, quoique cette fabrication soit une des formes les plus hautes de la grande industrie, elle a cela de commun avec l’industrie domestique, de ne point travailler pour le dehors, de n’être l’objet d’aucun commerce véritable. Ces produits sont destinés à être consommés, comme les autres, soit lentement, par l’usure des armes proprement dites, arcs, fusils, canons, forteresses, navires, bateaux-torpilleurs, etc., analogue à celle des outils de l’industrie civile ; soit rapidement, par l’explosion des poudres et l’émission des projectiles, par l’envoi d’une torpille à un bateau ennemi, etc., poudres, projectiles, torpilles qui sont aux armes à feu et aux torpilleurs ce que l’encre est à l’encrier, la planche à la varlope, le morceau de viande à la fourchette, le morceau de fer à la lime. La distinction entre l’arme et l’outil est confuse à l’origine, et ne se précise qu’au cours de l’évolution industrielle. La hache,  \phantomsection
\label{v2p101}le couteau, la faulx, le marteau, etc., sont à la fois outils et armes ; l’arc sert à chasser aussi bien qu’à guerroyer, comme le fusil aussi bien, à cela près que la chasse a cessé depuis longtemps d’être un travail industriel, elle est devenue un sport, tandis qu’elle a été au début la grande industrie alimentaire. Les archéologues sont embarrassés, en présence des débris de l’âge paléolithique et même néolithique, pour distinguer si ces fragments de silex éclatés ou polis ont servi pour la nourriture et le vêtement ou pour la défense. Même de nos jours, il y a force ambiguïtés ; la dynamite sert aux ingénieurs autant et plus qu’aux artilleurs.\par
Si différentes qu’elles soient des industries civiles, les industries militaires offrent d’importantes similitudes avec celles-ci et exercent sur elles une importance considérable. Les forces principales successivement employées par l’industrie militaire se déroulent dans le même ordre que pour l’industrie civile : forces humaines en tout temps, mais d’abord exclusivement humaines (pugilat), puis animales (chiens de guerre, lions apprivoisés, éléphants, chevaux)\footnote{ \noindent La cavalerie avait jadis un emploi tout autre que maintenant. A l’époque pharaonique, grecque, romaine, la force musculaire des chevaux traînant des chars à faulx rendait précisément les mêmes services que nous rendent aujourd’hui les substances explosibles de nos armes à feu. La vitesse seule, et non la force, de ces quadrupèdes, est à présent appréciée dans les combats.
 }, puis végétales (arcs, flèches), enfin physico-chimiques (substances explosibles). — Comme l’industrie civile, l’industrie militaire a dû chacune de ses transformations à des inventions d’abord individuelles et locales qui se sont propagées par émulation imitative. Il faut seulement remarquer que l’imitation de clan à clan, de cité à cité jadis, aujourd’hui de nation à nation, en fait d’armes et d’engins militaires quelconques, a toujours fonctionné bien plus vite et bien plus en grand qu’en fait d’industries privées. Il s’agit là d’imitation brusque, délibérée, à distance, à toute distance : l’exemple du Japon contemporain en est la preuve. Combien  \phantomsection
\label{v2p102}de sauvages africains ne connaissent de nous que nos fusils ! Parfois, bien rarement cependant, le côté militaire est le seul qu’un peuple emprunte à un autre peuple qu’il déteste et jalouse. L’émulation nationale en jeu dans ce genre d’imitation est de tout autre nature que la concurrence des productions civiles.\par
On s’explique ainsi ce fait remarquable — digne, à coup sûr, d’être signalé dans un traité sur la science économique — que l’industrie militaire, dans toutes ses branches, alimentation, vêtement, chaussure, voirie, ponts, etc., a toujours marché en tête de l’industrie civile et devancé ses progrès. Elle a donné, dans les arsenaux terrestres et maritimes, les premiers exemples de grands magasins de confections. Les premières provisions de bouche ont été des \emph{munitions}, des greniers d’intendance. Je lis dans l’\emph{Histoire de la civilisation russe}, par Milioukov, que la grande industrie a été importée artificiellement dans son pays par le gouvernement des tzars pour « répondre au besoin de draps pour l’armée ». Il en a été de même chez d’autres peuples. D’après Levasseur, les ateliers de fabrication pour le service des légions, sous l’Empire romain, avaient une grande importance, et l’organisation de ces ateliers a pu suggérer les \emph{collegia}. La voirie militaire a été aussi l’initiatrice et l’instigatrice de la voirie civile. Les routes stratégiques des Romains ont été les premières grandes routes. Ce sont aussi des routes stratégiques que le Premier Consul a fait tracer tout d’abord sur le sol français, comme premier délinéament du réseau de notre carte routière. Les premières voitures ont dû être des chars de guerre. Les chemins de fer sont peut-être le seul grand progrès de la locomotion qui n’ait pas été militaire à l’origine ; et cela seul suffit à singulariser le {\scshape xix}\textsuperscript{e} siècle. Les ballons n’ont, jusqu’ici, servi qu’aux armées. Les premiers navires romains ont été construits tout exprès pour la seconde guerre punique. Partout la marine marchande a été précédée et non pas seulement  \phantomsection
\label{v2p103}protégée par la marine de l’État. Curtius nous apprend, dans son \emph{Histoire de la Grèce}, que, chez les Athéniens, les constructions navales furent grandement perfectionnées par « d’heureuses inventions » provoquées par Périclès en vue de la guerre. Le besoin d’avoir des trirèmes plus larges, plus spacieuses, afin de laisser plus de place aux hoplites, a suscité les progrès des chantiers maritimes\footnote{ \noindent On voit que je suis loin de nier d’une manière absolue l’efficacité de \emph{la lutte.} Il n’en est pas moins vrai que c’est \emph{en pleine paix}, que les inventions, même guerrières, ont été produites.
 }, comme le besoin de grands rassemblements de fidèles dans les premières basiliques chrétiennes a fait imaginer la voûte à arêtes. Le service postal a été d’abord pratiqué par les besoins administratifs et militaires, en France, en Angleterre, en Russie, partout.\par
Depuis Archimède au moins, et probablement bien avant lui, le génie militaire a été pour le génie civil l’occasion de ses plus grands triomphes. En est-il de même pour l’architecture ? Non, ce semble, à première vue. Ici, ce n’est point l’architecture militaire qui semble avoir le pas sur l’architecture civile, c’est, plus souvent, l’architecture religieuse : temples, pyramides, mosquées, cathédrales, bien plus que les châteaux forts, paraissent avoir été l’effort suprême des architectes jusqu’à nous. Toutefois n’y a-t-il pas lieu de penser que les premiers murs, destinés à boucher sans doute les cavernes des troglodytes, ont dû être des remparts, de même que, vraisemblablement, le bronze et le fer leur ont servi d’arme avant de leur servir d’outil, ainsi que le silex ?\par
Notons encore que les premières épargnes ont été des trésors de guerre, que les spéculations des fournisseurs d’armées ont précédé partout les agiotages de Bourse, et que la science des finances est née de la stratégie.\par
Si les industries [{\corr militaires}] des diverses nations s’opposent entre elles, ce n’est point à la manière des autres industries nationales, par la vente simultanée de leurs produits sur un même marché : nous savons qu’elles n’ont pas, en général,  \phantomsection
\label{v2p104}de prix vénal, de valeur-coût, mais seulement une \emph{valeur-emploi} des plus appréciées, et qu’il n’est pas question pour elles de libre échange. Elles ne s’en font pas moins une concurrence continuelle de nation à nation ; et, pour mettre fin aux ruineuses prodigalités que cette concurrence entraîne en pleine paix, on n’a que la ressource de ces trusts spéciaux qu’on appelle les alliances politiques. La triple alliance, l’alliance franco-russe sont des trusts de ce genre. Il est manifestement chimérique de penser que l’opposition des industries militaires puisse être résolue par le collectivisme. Ne sont-elles pas déjà nationalisées ? Mais elles ont beau être le résultat d’une collaboration collective, et leurs produits ont beau être une propriété collective, leur hostilité essentielle n’en est pas amoindrie. Les \emph{trusts alliances}, dont la fédération n’est que la forme la plus haute, sont ici l’unique solution du problème.\par
Les industries militaires, [{\corr à la}] différence des autres industries, s’opposent encore et surtout par la consommation qui est faite de leurs produits sur un champ de bataille. Cette consommation est d’une espèce à part : éminemment destructive, il serait faux, cependant, et injuste au plus haut degré, de la qualifier improductive. Loin de là ; elle produit dans le camp vainqueur les biens réputés suprêmes : la puissance, la richesse, la gloire ; aux vaincus même elle procure une paix honorable. On peut donc dire qu’une théorie de la consommation économique n’est pas complète, qu’elle omet une de ses parties malheureusement les plus considérables, si elle néglige de s’appliquer aux services et aux produits spéciaux, manœuvres et projectiles, qui se consomment durant un combat. Ce sont bien là, en effet, des consommations de forces et de substances, car le jet des flèches ou des balles, les coups d’épée ou de baïonnette, les fatigues d’un assaut, répondent à un besoin qui est ainsi satisfait, et à l’un des besoins les plus enracinés, les plus universels, celui d’attaque ou de défense, de  \phantomsection
\label{v2p105}domination ou de salut. Et, non moins que la satisfaction de la soif ou de la faim après un long jeûne, la satisfaction de ce besoin capital s’accompagne, pendant ou après la bataille, d’une joie intense, qui compte parmi les toniques principaux du cœur humain.\par
Mais, observons-le en passant, si marqué que soit le caractère de consommation dans les dépenses faites sur le champ de bataille, il faut, pour le reconnaître, définir la consommation en termes psychologiques, comme nous venons de le faire. Définie en termes objectifs, elle devient inapplicable à ces étranges manières de donner satisfaction à un besoin. Quand je consomme du pain, de l’eau, des chaussures, de la chaleur, etc., j’ajoute à ma substance ou à mon énergie propre, des portions de substances ou d’énergie extérieure que je m’intériorise et m’approprie. Cette intégration est commune à toute consommation de la vie civile ; et, si l’on veut que ce soit là le trait distinctif de toute consommation, il est évident qu’un coup de baliste ou de canon n’en est pas une. Quand je dépense ainsi de la force musculaire et des cailloux, de la poudre et des boulets, je m’en dépouille, je m’en dépossède, j’en accrois l’avoir d’autrui par une sorte de donation violente et de présent fatal. — Certains peuples barbares ont connu la procédure bizarre qu’on appelle « jeûner contre quelqu’un ». En jeûnant à la porte de son débiteur, on le contraignait à payer, grâce à des idées [{\corr superstitieuses}]. C’était là une non consommation destructive et productive en même temps, et, en cela, comparable aux consommations dont je parle.\par
Remarquons aussi que les dépenses de ce genre peuvent se compliquer beaucoup sans cesser d’être des consommations \emph{nécessaires}. Ce n’est pas par luxe, mais par nécessité, qu’on a passé de l’arc et des flèches aux canons Krupp et aux balles coniques. Le taux du nécessaire en fait de consommations pareilles, comme pour toutes les autres, mais bien plus visiblement encore que pour celles-ci, s’élève toujours. \phantomsection
\label{v2p106} Quand le bateau-torpilleur sous-marin sera reconnu pratique, il faudra nécessairement que toutes les marines civilisées l’adoptent, si cher qu’il puisse être. Ce sera là une dépense de première nécessité, comme l’était pour un Iroquois la confection d’un arc et de flèches empoisonnées. Il n’y a, en effet, aucun caractère objectif qui permette de reconnaître où commence le luxe en fait de dépenses militaires. La différence consiste uniquement dans le mobile qui les inspire : quand elles tendent à la défense, au salut national, elles sont indispensables, si ruineuses qu’elles puissent paraître ; quand elles sont provoquées par une simple mégalomanie nationale, par un prurit de conquête ou de gloire, elles peuvent être jugées de luxe. Il en est de même pour toutes les autres dépenses : leur caractère luxueux apparaît quand il ne s’agit plus, pour elles, de maintenir l’être physique \emph{ou social} de l’individu dans son intégrité, mais de l’étendre, de le gonfler par la gloriole qu’on attend d’elles.\par
Les produits de l’industrie militaire ont encore cela de particulier que leur emploi importe beaucoup plus que leur possession, et qu’il est plus difficile de s’en bien servir que de les fabriquer ou de les acquérir. Le \emph{travail}, tout belliqueux, nécessité par la consommation intelligente de ces produits, est, chez le général comme chez le soldat, une industrie militaire tout autre que celle dont il vient d’être parlé : on la nomme l’art de la guerre. Cette production militaire d’un ordre plus élevé est comparable elle-même aux productions civiles. Par exemple, de même que, après l’invention de la machine à vapeur, la substitution de la grande industrie à la petite a eu pour effet de rendre l’ouvrier non propriétaire de son outillage, possédé par des capitalistes ; pareillement, par suite de l’invention de la poudre, la grande guerre, succédant aux petites guerres, a eu pour conséquence que le soldat ou le capitaine moderne, à la différence des routiers du moyen âge et de leurs chefs,  \phantomsection
\label{v2p107}a le maniement d’armes perfectionnées et coûteuses, de machines de guerre formidables, de cuirassés géants, qui ne lui appartiennent pas. L’équivalent du capitalisme industriel nous est présenté de la sorte par le militarisme moderne. Et je ne sais s’il se trouve un ennemi du capital assez acharné pour déplorer qu’il en soit ainsi, ou si, au contraire, un collectiviste y verra un exemple, bon à suivre, de la nationalisation de l’outillage. Mais il serait facile de répondre au premier que la facilité donnée aux condottièri et aux soldats du moyen âge d’avoir un outillage militaire à eux était l’un des plus graves dangers d’alors ; et au second que, le caractère essentiel de la production militaire étant, par définition, d’être nationale, la nationalisation du capital qu’elle emploie se justifie par des raisons tout à fait spéciales. Ajoutons que, dans le nouveau sens dont il s’agit, comme dans le sens précédent, l’industrie militaire a donné le ton à l’industrie civile. La \emph{grande guerre}, avec son enrégimentation disciplinée et centralisée, a été de beaucoup antérieure à la \emph{grande industrie} avec ses vastes ateliers bien organisés.\par

\asterism

\noindent On serait grandement embarrassé si l’on essayait d’appliquer les théories soit des économistes soit des socialistes à la répartition des fruits du travail militaire entre les travailleurs, c’est-à-dire entre les soldats ou officiers de tous grades, y compris le général en chef. Je ne vois pas la lumière que jetterait sur cette question la loi de l’offre et de la demande ou le principe du droit de l’ouvrier au produit intégral de son travail. La difficulté réellement insoluble provient d’abord de ce que le produit principal de ce travail collectif d’une armée est lui-même indivis et indivisible : c’est un accroissement général de foi patriotique, d’influence et de puissance nationale ; et il en est de ce résultat comme de beaucoup d’œuvres industrielles, telles que la construction d’un pont ou d’un monument, qu’il est impossible de  \phantomsection
\label{v2p108}répartir — dont il est non moins impossible de répartir la valeur en argent, toujours incertaine et arbitraire — entre les ouvriers et les entrepreneurs qui ont concouru à son exécution. Mais, alors même que le seul fruit d’une bataille gagnée serait le butin en espèces, il ne serait guère moins mal aisé de le distribuer entre les officiers et les soldats suivant la proportion exacte de leurs mérites respectifs. La rémunération du général, en gloire ou autrement, est souvent un bénéfice exagéré, dont on peut dire aussi qu’il consiste dans le travail non payé de ses soldats. Mais quels risques d’humiliation il a courus ! Et dans combien de cas est-il juste de lui attribuer l’honneur du succès ! Il suit de là que la solde, ce salariat militaire, s’impose nécessairement comme le seul mode pratique de paiement du travail militaire\footnote{ \noindent L’expression « industrie militaire » comporte un autre sens tout différent, quoique intéressant aussi. Il a trait aux époques où la guerre et l’échange ne se sont pas encore nettement différenciés. Le \emph{commerce guerrier} n’a pas été pratiqué seulement dans les colonies naissantes, aux Indes anglaises du {\scshape xviii}\textsuperscript{e} siècle, par exemple. Sous les Mérovingiens, sous les Carolingiens, les négociants qui se risquaient à de longs voyages devaient voyager en caravanes, « prêtes au combat, dit Levasseur, aussi bien qu’au commerce. Le franc Samo qui, vers la fin du règne de Clotaire II, régna sur les Vénètes, était un négociant des environs de Sens. Il était parti à la tête d’une troupe nombreuse de trafiquants pour se rendre chez les Slaves ; mais, à son arrivée, il les avait trouvés en pleine rébellion contre les Huns. Laissant alors de côté ses affaires, il avait pris les armes avec ses compagnons et avait assuré la victoire au parti des révoltés qui l’élurent roi par reconnaissance ».
 }.\par

\asterism

\noindent La comparaison des industries religieuses avec les industries militaires ne serait pas sans intérêt. J’entends par industries religieuses, soit la fabrication des articles de tout genre qui a lieu dans les monastères pour le service exclusif des moines, soit, encore mieux, la fabrication, même en dehors des monastères, des objets qui servent au culte, qui ont un caractère rituel et sacramentel, et promettent des biens d’ordre surnaturel. Or, pas plus aux industries religieuses entendues dans ces deux sens qu’aux industries  \phantomsection
\label{v2p109}militaires, le principe de la concurrence et du laisser-faire n’est applicable. Chaque fabricant de ce genre est censé pénétré de la gravité de ses fonctions et affecte une foi vive ; il est délégué par l’autorité supérieure pour remplir une mission, service auxiliaire du sacerdoce. Les divisions de l’industrie religieuse correspondent à celles de l’industrie militaire : il y a ici à distinguer une industrie de l’abri (architecture religieuse) — du vêtement (costumes ecclésiastiques, ornements, insignes tels que crosse et mitre) — de l’outillage (ostensoirs, calices, missels, crucifix, châsses, etc.) — de l’alimentation (pain azyme, nourriture spéciale de certains ordres monastiques). Il est à noter que, dans chacune de ces branches, l’industrie religieuse a été beaucoup moins transformée que l’industrie militaire correspondante. Cela est surtout vrai de l’outillage : autant les armes ont changé du moyen âge à nos jours, autant les objets du culte sont restés immuables. La même remarque, un peu atténuée, s’étend au costume et aux ornements du clergé. L’architecture religieuse, quoiqu’elle soit la partie la plus changeante de l’industrie religieuse, a bien moins varié que l’architecture militaire. D’une église gothique à une église de nos jours, d’un monastère du {\scshape xiii}\textsuperscript{e} siècle à un couvent d’aujourd’hui, il y a bien moins de différence que d’un château du moyen âge à l’un de nos forts actuels, des remparts d’alors à notre système de fortification, d’un corps de garde ancien à une de nos casernes. La religion devant son prestige en grande partie à son air d’immutabilité, quoi qu’elle doive toujours sa naissance à une innovation, c’est en résistant à toute invention nouvelle, et non, comme l’industrie militaire, en attisant le génie des inventeurs, quelle lutte contre les religions rivales. Aussi évolue-t-elle très peu ; mais, dans la mesure où elle évolue, chacune de ses transformations ou de ses modifications est causée par une nouveauté individuelle. Encore faut-il ajouter que, lorsqu’un changement volontaire et notable est introduit dans le costume \phantomsection
\label{v2p110} ou les ustensiles du clergé, dans les monuments sacrés, dans les prières, etc., c’est avec l’intention ou la prétention de revenir à un usage du passé. Rien de pareil à cet archaïsme pieux dans les changements apportés aux industries militaires.\par
Ainsi, on ne peut dire des industries religieuses ce que j’ai dit des industries militaires, qu’elles marchent en tête des industries civiles et stimulent leurs progrès. Mais, si elles restent en arrière, bien souvent c’est elles pourtant qui, par une action indirecte, les ont aiguillonnées. Pour répondre au besoin mystique de conserver éternellement le corps ou l’image du corps jusqu’au retour de l’âme, les croyances égyptiennes ont forcé les esprits ingénieux à découvrir l’art de l’embaumement, les artistes à représenter la forme humaine avec une fidélité remarquable, les tisseurs à tisser des étoffes d’une finesse et d’une solidité merveilleuse. L’enveloppement rituel des momies a certainement beaucoup contribué au progrès du tissage. La décoration des châsses, la mosaïque des temples, les miniatures des manuscrits sacrés, ont été les premiers chefs-d’œuvre de l’art. Il n’est pas jusqu’aux règles relatives au jeûne et au maigre qui n’aient contraint les cuisiniers et les fournisseurs à imaginer de nouveaux mets. Donc, tout en repoussant l’invention, la religion la suscite à son profit d’abord, et au profit surtout de la vie civile. En outre, les industries religieuses attirent autour d’elles et font prospérer à leurs côtés les industries civiles qui progressent à l’ombre de leur immobilité même. Et l’on peut fort bien concevoir une société où, sans nulle industrie militaire, les besoins religieux auraient suffi à déployer l’imagination des industriels et, encore mieux, celle des artistes.
 \phantomsection
\label{v2p111}\subsubsection[{II.3.e. Enfin, entre producteurs d’articles hétérogènes. Loi des débouchés, question de luxe.}]{II.3.e. Enfin, entre producteurs d’articles hétérogènes. Loi des débouchés, question de luxe.}
\noindent Après avoir parlé des conflits entre producteurs (soit nationaux, soit étrangers) d’un même article ou d’articles similaires, il convient de dire quelques mots des oppositions d’intérêt, moins évidentes, moins retentissantes, réelles cependant, entre producteurs d’articles hétérogènes, répondant à des besoins différents.\par
En premier lieu, cas assez rare, deux industries, quoique très hétérogènes et répondant à des besoins très différents, peuvent être antagonistes parce qu’elles se disputent la même matière première. Sous l’ancien régime, on a souvent refusé d’autoriser la création de nouvelles forges ou de nouvelles verreries parce qu’on a craint que les boulangeries ne vinssent à manquer de combustibles nécessaires pour cuire le pain. « Le fait est arrivé, dit M. Germain Martin, à Beaucaire, et les consuls de Toulouse réclamaient la suppression des forges au nom de l’humanité. Mieux valait manquer de fer que de pain. »\par
En second lieu, dans un pays, à un moment donné, il n’y a qu’une quantité à peu près fixe de bras et de capitaux qui s’offrent à l’ensemble des industries. Si donc une industrie nouvelle survient, elle ne pourra s’établir qu’en prélevant sa part de capitaux et de bras, au détriment des industries déjà installées. Mais ce préjudice n’est le plus souvent que momentané et bientôt plus que compensé. En général, la création d’une industrie nouvelle, en surexcitant les espérances de gain, attire de nouveaux bras, jusque-là inoccupés, ou redouble l’activité des travailleurs anciens, et  \phantomsection
\label{v2p112}pousse à la formation de nouveaux capitaux par l’élévation de l’intérêt offert. De plus, s’il s’agit d’une industrie d’exportation, qui s’empare du marché extérieur, c’est, manifestement une source de richesses qui vient de s’ouvrir, ajoutée aux autres. Et, même s’il s’agit d’une industrie bornée au marché intérieur, il se peut que le tort fait par l’industrie nouvelle aux industries plus ou moins similaires soit inférieur à l’avantage qu’elle procure aux industries tout à fait hétérogènes en leur offrant un moyen nouveau d’échange, d’après la \emph{loi des débouchés} de J.-B. Say.\par
Toutefois, ce serait une erreur de croire, conformément à cette loi, que toute nouvelle industrie ne peut qu’être favorable aux intérêts de \emph{toutes} les industries hétérogènes. Sans doute, elle est toujours un avantage, plus ou moins grand, pour tous les consommateurs, en ce sens qu’elle leur permet un choix plus étendu entre tous les emplois possibles de leur argent qui, à somme égale, a un pouvoir d’achat plus varié et plus étendu. Mais il ne s’ensuit pas qu’elle soit utile à tous les producteurs. Elle n’est utile qu’à ceux dont la nouvelle production, pour des raisons que nous allons dire, tend à rendre les produits plus recherchés ; elle est nuisible à ceux qu’elle tend à supplanter ou à refouler ; chaque marchand de vin qui s’établit dans un quartier ouvrier y favorise les débits de tabac (car plus on boit, plus on fume), mais nuit aux intérêts des épiciers ou des bouchers auxquels s’approvisionnent les familles ouvrières. — Un théâtre qui vient à s’établir dans une petite ville y favorise l’industrie des tapissiers, des décorateurs, et aussi, par son action indirecte sur les mœurs, qu’il rend moins austères, certaines autres industries de luxe ou de plaisir ; mais, en procurant aux oisifs un plus agréable emploi de leurs soirées que les éternelles parties de jeu, il diminue les profits des gérants de cercle ou de café. — Toute industrie nouvelle inaugurant un nouveau progrès de l’éclairage favorise la production des livres, des journaux, des travaux de couture  \phantomsection
\label{v2p113}exécutés à la lampe le soir, etc., mais entrave ou refoule les industries criminelles des escarpes nocturnes, dans les grandes villes. Toujours ou presque toujours, un article si nouveau qu’il soit, jeté sur le marché, insinue et accrédite dans les esprits, en se répandant, quelque jugement implicite, quelque idée, quelque manière de voir ou d’apprécier la valeur des choses, qui est une confirmation ou une négation muette, inconsciente, des jugements, des idées, des appréciations, implicitement propagés par les autres articles. Rarement, très rarement, elle ne les confirme ni ne les dément. — Quand, par exemple, l’industrie des chemins de fer est importée dans une contrée à demi barbare, le besoin tout nouveau qu’elle y suscite et y satisfait, le besoin des longs voyages confortables et peu coûteux, y affaiblit et y fait s’évanouir par degrés une foule de besoins anciens : chacun de ces produits ou de ces services nouveaux, chaque voyage à l’étranger, est une suite de démentis infligés à l’orgueil naïf de l’indigène sédentaire qui juge son petit canton supérieur en tout, en beautés pittoresques, en agréments de vie, au reste de l’univers ; et les fabricants de produits locaux, de costumes traditionnels, de meubles archaïques, où s’incarnent ces préjugés héréditaires, ne tardent pas à se ressentir des coups terribles qui leur sont portés ainsi, à moins que, au contraire, la mode ne s’empare de ces produits coutumiers et ne leur ouvre, momentanément, un large débouché extérieur, dont ils n’avaient jamais pu rêver la perspective. Plus spécialement, ce goût nouveau des voyages lointains, importé par les chemins de fer, a pour effet de dégoûter des petits déplacements, des excursions de voisinage, et la vogue des pèlerinages éloignés se substitue à celle des petits pèlerinages locaux, maintenant méprisés, au grand regret de beaucoup de maîtres d’hôtel et de boutiquiers. C’est ainsi qu’un produit peut en combattre un autre, alors même qu’ils n’ont rien de similaire et répondent à des besoins tout à fait différents. — Autre exemple. L’établissement \phantomsection
\label{v2p114} d’une imprimerie dans un pays qui en était privé jusque-là, non seulement, ce qui est clair, y tue la profession des copistes et des enlumineurs de manuscrits, mais encore, suivant la nature des livres imprimés, — aujourd’hui la Bible ou saint Jérôme, demain Plutarque ou Virgile — y répand des idées propres à fortifier ou à affaiblir les croyances régnantes et y dévie le courant des mœurs et des usages, ce qui favorise certaines industries anciennes et en fait péricliter d’autres.\par
Même si un produit nouveau n’impliquait aucun jugement contradictoire aux idées latentes dans d’autres produits hétérogènes, il ne laisserait pas d’en combattre quelques-uns ou quelqu’un par le seul fait qu’il suscite, qu’il appelle à l’existence économique, un désir nouveau, d’abord fantaisie et caprice, bientôt habitude. Rappelons-nous la rotation quotidienne ou annuelle des désirs enchaînés dans la vie de l’individu, dont il a été si longuement question dans la première partie de cet ouvrage. Cette ronde ne saurait s’élargir indéfiniment. Le cœur humain n’est pas d’une élasticité sans fin, et, au delà d’un certain nombre de désirs, il se heurte à sa limite infranchissable ; un besoin nouveau ne peut entrer dans le cercle magique qu’en resserrant tous les autres anneaux de la chaîne ou en expulsant l’un d’entre eux. Puis, si, comme nous l’avons dit plus haut, en premier lieu, les capitaux disponibles à chaque instant pour l’alimentation des diverses industries sont limités, nous devons ajouter, en second lieu, que la somme des revenus dont les consommateurs disposent pour l’achat de ces divers produits ne l’est pas moins. Il est vrai que, à mesure que les besoins se multiplient, les revenus tendent à s’accroître, les salaires, les honoraires, les appointements, toutes les formes de la rémunération des services, vont en augmentant. Mais, en tout cas, il est une chose absolument inextensible et strictement mesurée à chacun de nous : c’est le temps. Il faut toujours un certain temps pour satisfaire un désir quelconque, et la journée est trop  \phantomsection
\label{v2p115}courte, la vie trop brève, pour permettre au plus riche et au mieux portant, de se donner successivement toutes les satisfactions qui s’offrent à lui. Il doit élire quelques besoins et en éliminer beaucoup.\par
Ainsi, pour toutes les raisons, psychologiques avant tout\footnote{ \noindent C’est tantôt, me dira-t-on, à cause des limites de son budget (s’il s’agit d’une bourse petite ou moyenne), tantôt (s’il s’agit de gens très riches), à cause des limites de son temps, qu’un consommateur sacrifie une consommation à une autre. Il est évident que, si un milliardaire passe ses vacances à voyager, il ne peut, quoiqu’il en ait les moyens, les passer à organiser des fêtes dans son château... Mais, au fond, c’est par des motifs tout psychologiques et logiques que cette exclusion d’une dépense par une autre a lieu ici même. En effet, pour donner la préférence à telle dépense sur telle autre, l’homme simplement aisé ou le milliardaire a dû avoir une raison, et cette raison, c’est qu’il a \emph{affirmé} que tel emploi de son argent lui procurerait plus d’avantages que tel autre, et \emph{nié} que cet autre serait aussi avantageux. — Ce serait, du reste, une erreur de croire que cette raison est toujours la meilleure possible, et que le choix, pour être raisonné, est judicieux. Dans ces luttes économiques, comme dans les luttes militaires, le vaincu vaut souvent mieux que le vainqueur ; et les arrêts du destin sont toujours à réviser.
 }, qui viennent d’être indiquées, les besoins tantôt s’ajoutent, tantôt se substituent les uns aux autres ; et c’est par cette double loi de substitution et d’addition, d’élimination et d’accumulation — dont la seconde l’emporte de plus en plus sur la première, mais non indéfiniment — que s’explique l’évolution des industries, correspondant à l’évolution des budgets. Voilà pourquoi, d’âge en âge, se modifie et se transforme la proportion relative des industries alimentaires, des industries du vêtement, des industries du logement, des industries de l’instruction, du livre, du journal, des industries d’art et de plaisir, et aussi bien la proportion relative des dépenses budgétaires affectées par chaque classe de la société à satisfaire les besoins auxquels répondent ces diverses catégories de production. La série de ces transformations est ce qu’il y a de plus capricieux et de plus différent d’un peuple à un autre. Mais, à travers ces variations, un ordre général s’observe. Il est à remarquer que les besoins impérieux, et qui, comme tels, doivent être satisfaits les premiers, à peu près exclusivement d’abord, sont  \phantomsection
\label{v2p116}en même temps les moins extensibles, les moins susceptibles de développement. Et de ce rapport inverse entre l’\emph{urgence} et l’\emph{élasticité} des besoins, il résulte que les moins impérieux, bien qu’ils commencent à être satisfaits longtemps après les autres, sont appelés tôt ou tard à les dépasser en importance industrielle et budgétaire, puisque c’est, naturellement, dans le sens des désirs les plus élastiques que la loi d’accumulation pourra se donner carrière et prévaloir plus complètement sur la loi de substitution. L’appétit et la soif ont des limites étroites, le besoin de se vêtir aussi, mais la curiosité, l’ambition, la vanité, semblent presque illimitées ou n’ont que des bornes qui reculent sans cesse. Elles offrent au génie inventif, à l’activité productrice, un débouché indéfini.\par
La question du luxe se présente ici, et nous n’en dirons qu’un mot, pour dissiper une équivoque. Toute consommation nouvelle, même quand elle se rapporte aux besoins les plus fondamentaux de l’organisme, la consommation du froment par exemple, dans un pays habitué au pain de seigle, la consommation des chemises au temps d’Isabeau de Bavière, commence par \emph{paraître} luxueuse. Mais elle n’a cette apparence que passagèrement, tandis que les satisfactions données aux besoins d’ostentation et de divertissement, soit nouvelles soit anciennes, gardent toujours ce caractère d’être de luxe. Or, le danger du luxe, apparent ou réel, est d’autant plus grand qu’il a trait à des besoins moins urgents et plus élastiques. En effet, l’extension indéfinie dont les besoins précisément les plus artificiels sont susceptibles peut attirer à eux une proportion toujours croissante de l’activité [{\corr productrice}] à tel point que la production des choses qui répondent aux besoins les plus impérieux en soit amoindrie ou altérée. Pour peu que la passion des tulipes ou des orchidées se développe dans certaines classes très riches, la culture des fleurs fait reculer les champs de blé. — Ajoutons que le progrès du luxe véritable est inconciliable avec celui  \phantomsection
\label{v2p117}de la population. A chaque nouveau besoin qui entre dans une famille, il s’y procrée un enfant de moins. C’est ainsi, et pour des raisons analogues au fond, que, à chaque progrès de l’ameublement et du confortable dans les châteaux féodaux, à partir de la Renaissance, le nombre des \emph{domestiques} diminuait. Si donc un pays, dans sa lutte pour l’existence nationale, a besoin de voir croître sa population, il doit s’opposer au développement du luxe sous toutes ses formes. Il doit désirer surtout que le luxe décline sans que les fortunes cessent de grandir, c’est-à-dire qu’il décline par la volonté même des riches sous l’empire de l’opinion, et que l’épargne accrue se capitalise en placements dans des industries vraiement utiles à l’intérieur ou à l’étranger. — « Prenez garde de supprimer le luxe, dit quelque part un économiste ; comme le cœur humain n’est pas parfait, vous verriez aussitôt grandir l’ambition, la soif du pouvoir. » C’est possible. Mais pourquoi ne pas supposer aussi bien qu’on verrait grandir la soif du savoir, les aspirations esthétiques, et, à défaut de luxe matériel, le besoin de \emph{luxe intérieur} ou de \emph{beau intérieur}, de haute et de délicate moralité, d’art exquis et complexe ? On confond dans la notion vague de luxe trois choses bien distinctes : le \emph{confort}, qui consiste en satisfactions délicates et raffinées données aux besoins nécessaires et fondamentaux de l’organisme ; le \emph{luxe} proprement dit qui répond, avant tout, aux besoins d’amour-propre et de vanité, de plaisir coûteux, et l’\emph{art} qui satisfait l’amour du beau. Par le développement social, ces trois choses accentuent leur différence et leur divergence, d’abord dissimulée. Or, la moins extensible, la moins élastique, des trois catégories de besoins satisfaits de la sorte, c’est la première : le besoin de confort ne comporte que des limites assez étroites. Bien plus susceptible de s’étendre est le besoin de \emph{paraître} et de briller, source des désirs d’amour-propre et de vanité. Mais l’amour du beau, l’avidité de l’exquis, est de tous les besoins celui qui comporte le plus d’extension et présente le plus \phantomsection
\label{v2p118} d’élasticité. C’est donc dans ce dernier sens que, finalement, la production doit trouver son débouché le plus illimité, et où elle doit se précipiter torrentiellement. — Il est à remarquer que, en s’unissant au premier de ces besoins, le troisième donne naissance à une combinaison des plus fécondes, \emph{l’art industriel, l’art décoratif}, qui pourrait bien être destiné au plus glorieux avenir et, sous son ombre, étouffer le luxe proprement dit, appelé à décroître ou à disparaître.\par
Mais n’oublions pas que le plus extensible encore, le plus élastique des besoins humains, c’est la \emph{curiosité}, et que l’élargissement du savoir, de la critique, de la pensée est encore la voie où la sève humaine continuera à couler quand toutes les autres branches du désir auront cessé de croître.
 \phantomsection
\label{v2p119}\subsubsection[{II.3.f. Conflits de la consommation avec elle-même. Lois somptuaires.}]{II.3.f. Conflits de la consommation avec elle-même. Lois somptuaires.}
\noindent Après les luttes de la production avec elle-même, les luttes de la consommation avec elle-même ont lieu de nous occuper un instant.\par
Les motifs pour lesquels les consommateurs nationaux d’un même produit peuvent être en conflit sont de plusieurs sortes : tantôt, s’il s’agit surtout du blé, du riz ou autres denrées alimentaires, l’insuffisance de la production met aux prises les appétits rivaux et hostiles ; tantôt, s’il s’agit d’objets moins utiles ou de luxe, un caractère honorifique, une présomption de supériorité sociale s’attache à la consommation de quelques-uns d’entre eux. Je n’insiste pas sur le premier cas : on sait assez à quelles horribles extrémités la faim déchaînée contre la faim, la soif contre la soif dans certains sièges célèbres, ont de tout temps porté les hommes, redevenus cannibales en dépit du plus haut degré de civilisation. Les famines périodiques de l’ancien régime, les grandes famines de l’Inde anglaise\footnote{ \noindent Les famines du moyen âge ont été souvent horribles. « Dans celle de 1030 à 1032, en France, on supplicia un homme qui en avait égorgé et dévoré 48 autres, d’après Sismondi... Même en Toscane, les villes, à l’exception de Florence, expulsaient ordinairement leurs pauvres en temps de disette, d’après Villani. » (Roscher.)
 }, même de nos jours, sont des illustrations merveilleusement atroces de cette concurrence aiguë des consommateurs affamés. En tout temps, sous des formes infiniment plus douces, et à propos d’articles quelconques dont la production est inférieure aux besoins de ceux qui désirent et peuvent les acheter au prix actuel, les acheteurs éventuels se combattent, et leur combat fait hausser le prix. Quand, ce qui  \phantomsection
\label{v2p120}est rare, cette limitation étroite des produits est due à un calcul des producteurs qui croient avoir plus d’intérêt à les limiter pour en élever le prix qu’à les multiplier en l’abaissant, le conflit existe plutôt entre les producteurs d’une part et les consommateurs d’une autre, genre de lutte qui sera étudié plus loin, qu’entre les divers consommateurs. Mais le plus souvent ce sont des causes naturelles, des circonstances étrangères à la volonté réfléchie des industriels, qui rendent la production insuffisante ; et alors c’est seulement entre les divers consommateurs que la lutte s’engage.\par
Le second cas, relatif aux consommations vaniteuses, est plus intéressant. Dans une société divisée en castes, ou en classes superposées que séparent des diaphragmes étanches, chaque groupe de la population a ses consommations spéciales et caractéristiques qui désignent et rappellent à chaque instant son rang dans la hiérarchie sociale. Il y a une façon noble et une façon roturière de se nourrir, de s’habiller, de se coiffer, de s’amuser, etc., et chacun des étages de la noblesse, chacune des subdivisions de la roture, se différencie des autres par ses particularités distinctives sous ces divers rapports. Or, aussi longtemps que la séparation entre ces compartiments sociaux paraît à peu près infranchissable, le désir de les franchir, en réalité ou en apparence, par l’imitation des modes d’alimentation, de vêtement, d’amusement, propres aux compartiments supérieurs, est peu vif et peu répandu ; et il est inutile de dresser des barrières légales pour s’opposer aux progrès de cette sorte d’exosmose insignifiante. Il suffit de la peur du ridicule ou du scandale pour empêcher l’inférieur d’imiter les usages du supérieur, ce qui aurait l’air d’une usurpation ou d’une mascarade grotesque. A présent encore, dans des coins de provinces « arriérées », une fille du peuple n’ose pas « porter chapeau », de crainte que ses amies ne se moquent [{\corr d’elle}]. Le besoin des \emph{lois somptuaires} ne commence à se faire sentir que lorsque la distinction des classes s’est atténuée au point d’éveiller chez le  \phantomsection
\label{v2p121}plébéien l’\emph{envie} de ce que jusque-là il \emph{admirait}, et le désir général, intense, de copier, pour se l’approprier, ce qui fut longtemps le monopole exclusif du patricien. Les classes inférieures croient s’égaler ainsi aux classes supérieures et proclamer bien haut cette égalité, quoique, par là, elles avouent implicitement la supériorité de celles-ci : contradiction, vaguement sentie de tous, à la faveur de laquelle se maintient et se prolonge longtemps, en pleine démocratie, le prestige des puissances abattues.\par
Par les lois somptuaires, les aristocraties, quand elles sont encore très puissantes, mais déjà ébranlées et enviées, essaient de lutter contre le flux d’imitation jalouse qui monte lentement pour les submerger. Et l’on a trop dit que cet effort a toujours été impuissant. Certainement il n’a jamais fait que retarder l’égalisation des classes par leur assimilation ; jamais une digue, si haute fût-elle, n’a arrêté pour toujours le moindre torrent ; et le fleuve des exemples, dans sa continuelle descente des montagnes aux vallées sociales, n’est pas un des courants les moins invincibles. Mais un retard d’un demi-siècle ou même de quelques lustres suffit à combler les vœux des hommes politiques, dont la prévoyance est toujours, et nécessairement, à courte portée. Ne soyons donc pas trop enclins à tourner en ridicule les édits du passé qui prohibaient le port des vêtements de soie chez les roturiers ou telle autre dépense de luxe. Si mal exécutés qu’aient pu être ces règlements — et ils ne l’étaient peut-être pas plus mal que ceux de nos jours sur le port des décorations nationales ou étrangères — ils étaient une continuelle menace de poursuites possibles et entravaient sérieusement les progrès des consommations prohibées. Ils protégeaient le monopole de consommation des classes supérieures aussi efficacement que nos droits de douane, si souvent éludés pourtant, protègent nos industries nationales.\par
Mais la phase des lois somptuaires n’a qu’un temps, et il vient un moment où, les frontières des classes continuant à  \phantomsection
\label{v2p122}s’effacer ou leurs remparts à s’abaisser, les lois ne sauraient plus sans contradiction les reconnaître inégales, car leur égalité de droit y est écrite en lettres d’or, bien longtemps avant que leur inégalité de fait ait disparu. Est-ce à dire que, à défaut des lois somptuaires, en cette nouvelle ère qui va s’ouvrir, les anciennes classes dirigeantes aient renoncé à toute arme défensive contre les empiètements imitatifs du vulgaire ? Il est curieux de voir, en temps de démocratie, les déguisements multiples sous lesquels se dissimule, pour se perpétuer, le besoin de consommations distinctives, et aussi les procédés changeants par lesquels ce besoin parvient souvent à se satisfaire, à défendre au moins les débris de son monopole d’autrefois. Pour dérouter le parvenu snob qui cherche à copier en tout les usages d’un monde où il n’est pas né, le mondain a soin de changer souvent de modes et ses signes quasi maçonniques de reconnaissance, en sorte que son copiste est toujours en retard d’un an ou deux dans ses velléités de mimétisme. Ou bien, si le groupe qui veut se \emph{distinguer} a le privilège de la très grande fortune, il se protège par l’élévation extravagante des prix qu’il offre à ses fournisseurs spéciaux pour se réserver à lui seul l’honneur de ces fournitures. Puis ce n’est plus par des différences bien visibles, bruyantes et criardes, que ses consommations spéciales se distinguent des consommations similaires à l’usage des autres classes, c’est par des nuances de qualité de plus en plus difficiles à apercevoir, et que ne remarquent plus les yeux grossiers de la foule égalitaire. Mêmes coupes, mêmes couleurs d’étoffes, mêmes formes de voitures, ou peu s’en faut ; à peine une différence plus manifeste dans les coiffures, l’odieux chapeau à haute forme ici, là le chapeau mou ; nul uniforme de classe d’ailleurs, rien que des costumes professionnels, et encore réduits à leur minimum. Mais la finesse du regard connaisseur qui cherche autour de soi les nuances différentielles et révélatrices s’aiguise et s’avive à mesure qu’elles deviennent plus imperceptibles.\par
 \phantomsection
\label{v2p123}Il y a une catégorie de lois somptuaires dont nous n’avons rien dit, ce sont celles qui, dans les républiques antiques, à Athènes, à Rome, étaient édictées dans l’intérêt des classes populaires offusquées du luxe des grands. Ces restrictions législatives imposées aux dépenses funéraires, à la magnificence des festins ou des fêtes privées, avaient pour but de contraindre les riches à un train de vie qui ne s’éloignât pas trop de celui des pauvres. De nos jours des mesures pareilles, loin d’avoir pour conséquence une diminution de l’inégalité entre les classes, auraient cet effet anti-démocratique d’accroître la supériorité des grandes fortunes en forçant les riches à épargner toujours davantage. C’est qu’aujourd’hui le riche n’a que l’embarras du choix entre les placements faciles et fructueux, sinon toujours sûrs, de son épargne, dans le pays même ou au dehors ; tandis que, dans l’antiquité, il ne pouvait faire d’autre usage de ses revenus, du revenu de ses terres à peu près exclusivement, que de les dépenser pour lui \emph{ou pour le public}. Moins il dépensait, donc, pour sa consommation personnelle, et plus il était obligé de dépenser pour ses concitoyens, en fêtes publiques, en jeux, en libéralités aux temples, en achats de navires de guerre. Les lois somptuaires des républiques anciennes étaient, par suite, beaucoup moins absurdes et inefficaces, mais, en revanche, elles étaient bien plus oppressives encore, que les économistes ne l’ont pensé ; elles avaient pour le peuple un double avantage : l’avantage moral d’humilier l’orgueil des patriciens, et l’avantage matériel de jouir de leurs revenus à leur place. A présent, nos démocraties, il est vrai, ne votent plus de lois pareilles ; mais, par d’autres mesures, telles que l’impôt progressif, elles poursuivent et atteindront le même but. D’une part, elles entraveront ainsi les consommations de luxe des riches ; d’autre part, elles confisqueront l’épargne qu’ils auraient pu faire moyennant cette privation si elle avait été volontaire.\par
Entre nationaux et étrangers, et, aussi bien, jadis, entre  \phantomsection
\label{v2p124}nationaux de diverses provinces claquemurées dans leurs douanes particulières, la lutte des consommateurs d’un même produit se présente sous d’autres aspects. Ici, ce n’est pas la vanité nationale ou provinciale qui est en jeu, dans nos temps modernes du moins, mais c’est la peur de manquer du nécessaire à un moment donné. De là les lois prohibitives de la vente des grains, ou de la laine, ou du fer, etc., hors du territoire de la nation ou de la province, et les peines terribles qui sanctionnent ces interdictions\footnote{ \noindent Il a été rendu en France, de 1304 à 1689, 31 ordonnances ou lettres patentes permettant ou interdisant l’exportation du blé de province à province. Le commerce avec l’étranger n’a commencé à être autorisé, et encore à titre exceptionnel et provisoire, qu’en 1502.
 }. Quelquefois, si ce n’est la vanité, c’est au moins une fierté ou une piété nationale assez louable qui a dicté ces mesures légales ; comme, par exemple, lorsque la Grèce ou l’Égypte actuelles interdisent l’exportation des œuvres d’art exhumées par les fouilles des archéologues. Le monopole collectif de la possession de ces débris au profit des musées grecs ou égyptiens, se justifie sans peine, et le libre échange appliqué à ces poteries ou à ces marbres aurait aux yeux des générations nouvelles, héritières d’un glorieux passé, un air d’impiété filiale.\par
Je ne m’occuperai pas des luttes entre consommateurs de produits différents. Il a déjà été question de ce sujet à propos du conflit entre producteurs de produits différents : ces deux aspects d’un même phénomène sont inséparables.
 \phantomsection
\label{v2p125}\subsubsection[{II.3.g. Luttes de la production avec la consommation.}]{II.3.g. Luttes de la production avec la consommation.}
\noindent Arrivons aux luttes les plus intéressantes peut-être, celles des producteurs avec les consommateurs. Sous le nom de \emph{crises}, elles ont, quand elles sont arrivées à l’état aigu, arrêté l’attention des économistes et donné matière à de beaux travaux. Mais, présentées ainsi à part, comme des anomalies passagères, détachées des luttes sourdes, constantes et universelles, auxquelles elles se rattachent, et qui sont l’état normal, ou paraissent telles, les crises ne sauraient être pleinement comprises ni interprétées sainement. Les physiologistes savent bien que, pour avoir une idée exacte et complète des maladies, il faut les rattacher par un lien étroit à l’état de santé. Il n’en est pas moins vrai que la pathologie a droit à des traités spéciaux ; aussi consacrerons-nous plus loin un chapitre spécial aux crises, en confondant sous ce titre aussi bien les crises monétaires que les crises commerciales, indissolublement liées les unes aux autres. Pour le moment, nous ne voulons qu’indiquer brièvement le caractère général et permanent des oppositions économiques dont les crises sont des poussées intermittentes à l’état aigu. Le désaccord d’intérêts qui existe toujours et partout entre producteurs et consommateurs d’un même article, les uns désirant vendre le plus cher et les autres acheter le moins cher possible, est un conflit d’abord tout individuel, exprimé en marchandages souvent très bruyants qui font toute l’animation des foires et des marchés. Ce désaccord est irrémédiable ou semble tel ; entre co-producteurs du même article, patrons et ouvriers, industriels et industriels, l’on finit toujours \phantomsection
\label{v2p126} par s’accorder, mais sur le dos du consommateur qui reçoit seul les coups des adversaires réconciliés. Par cet accord des co-producteurs, la lutte entre eux et le consommateur tend à revêtir une forme collective et à s’élever alors au paroxysme d’intensité, tout à fait anormale, quoique souvent périodique, qui a fait l’objet de tant d’études savantes. Pour que la lutte d’individu à individu se transforme en une lutte de groupe à groupe, il est nécessaire que les intéressés prennent conscience du désaccord de leurs intérêts, et qu’ils aient les moyens de se concerter facilement. Cette facilité de communication et de coalition existe pour les producteurs bien longtemps avant d’exister au même degré pour les consommateurs. Aussi voyons-nous de tout temps sous la forme de corporations, de syndicats, de trusts, etc., se produire des entreprises hardies tentées par les industriels d’une région contre l’ensemble de leurs clients, et c’est pour réagir contre ces tentatives souvent réussies que les gouvernements ou les municipalités, se faisant spontanément les défenseurs des intérêts publics, édictent des lois telles que les lois de maximum ou les taxes municipales de la boulangerie, de la boucherie, etc.\par
Quand elles s’appliquent à un marché restreint, circonscrit, fermé, ces mesures de protection des acheteurs contre les vendeurs atteignent d’ordinaire leur but. Les taxes municipales ont eu une efficacité difficile à contester. Il n’en est pas de même des lois de maximum, du moins en ce qui concerne les temps modernes. Il se peut que dans les États quasi municipaux de l’antiquité, à Athènes, et peut-être même à Rome, elles aient eu une action puissante. Mais ni l’expérience qu’en a faite Philippe le Bel, ni celle, plus désastreuse encore, qu’en a tentée la Convention en 1793, ne sont encourageantes. Les peines les plus fortes, l’amende de 300 à 1000 francs pour ceux qui vendraient au-dessus du prix maximum, \emph{la mort} pour quiconque aurait enfoui des farines ou des grains, avaient beau être prononcées, les  \phantomsection
\label{v2p127}marchés ne s’approvisionnaient plus, le blé se cachait, la viande aussi, et les denrées quelconques ; car, par un enchaînement logique, le maximum s’était peu à peu étendu à tous les objets de première nécessité. Si bien que Legendre proposa à la tribune de décréter « un carême civique ». On avait même été conduit à imposer un maximum aux objets fabriqués, et enfin, même, aux salaires. A présent, c’est plutôt un minimum des salaires qu’une assemblée démocratique serait disposée à voter.\par
Cependant ce serait une erreur d’induire de ces tentatives avortées que tout essai législatif d’imposer dans un grand pays des prix obligatoires aux industriels et aux commerçants, doit être à jamais frappé d’impuissance. On en disait autant il y a quelques années encore, à l’inverse, des tentatives d’accaparement par lesquelles les producteurs s’efforcent de temps en temps de rançonner à leur gré les consommateurs. Mais le succès prodigieux de certains trusts a opposé un démenti complet à ces pronostics\footnote{ \noindent Et ces succès ne sont pas une singularité de notre temps. Les \emph{trusts,} les ligues entre grands industriels, apparaissent dès le {\scshape xviii}\textsuperscript{e} siècle. En 1740, les fabricants de drap du Languedoc se coalisent pour obtenir la fourniture exclusive des draps de la troupe. — Vers la même époque, on constate que « tous les différents entrepreneurs et propriétaires des mines du Languedoc ont des conventions entre eux, suivant lesquelles ils vendent tour à tour le charbon à un prix qu’ils ont fixé. » En 1765, \emph{Trust} à Lyon... etc.
 }, avec cette atténuation importante toutefois que les accapareurs, loin de hausser abusivement les prix, se sont trouvés intéressés souvent à les abaisser plutôt, de peur de soulever l’opinion et de provoquer quelqu’une de ces armes législatives qu’on peut diriger contre eux et qu’ils savent bien, eux, n’être pas toutes vaines. — Mais c’est assez parler des trusts, dont il a déjà été question plus haut, à propos de la lutte entre producteurs, à laquelle ils se rattachent aussi bien qu’à celle entre producteurs et consommateurs.\par
La question du prix des produits est toujours la grande pomme de discorde entre producteurs et consommateurs. Mais c’est surtout quand il y a un désaccord criant, brusquement \phantomsection
\label{v2p128} senti, entre la quantité des produits et la somme des besoins à satisfaire, que ce conflit devient aigu, par suite de la baisse ou de la hausse énorme des prix qui en résulte. Ce désaccord, ou plutôt cette rupture soudaine d’accord, étant la principale cause des crises, surtout dans le cas de \emph{surproduction}, nous nous réservons d’en parler plus loin. Mais il importe de noter ici l’origine toute psychologique des surproductions.\par
Celles-ci sont dues à la simple exagération d’une contradiction de croyances qui existe toujours, mais atténuée, dans la vie économique la plus normale. On ne s’aperçoit pas, en effet, que les espérances de vente et de gain qui animent l’activité des divers industriels d’une même catégorie, ou même de catégories différentes, et qui constituent, au fond, toute la valeur des capitaux de toutes sortes qu’ils emploient (ateliers, usines, fermes, outillage agricole, industriel, etc.), sont toujours plus ou moins contradictoires entre elles. Cela peut se dire, il est vrai, de la valeur d’un produit aussi bien que de celle d’un capital ; mais surtout de cette dernière. La valeur d’un produit consiste dans l’espoir de satisfaire par lui certains désirs, espoir qu’il suggère à ses acheteurs ; et cet espoir est démenti implicitement par l’espoir pareil que suggèrent à d’autres fractions du public les produits rivaux. Mais, s’il y a ici contradiction de jugements, il n’y a pas contradiction de désirs, et les diverses parties du public qui s’adressent à des fournisseurs concurrents se juxtaposent sans se combattre. En outre, l’\emph{espérance} qui fait acheter le produit, quand elle se fonde sur le souvenir de satisfactions déjà obtenues par un produit identique, approche de la certitude. Mais la valeur d’une fabrique qu’on vient de fonder, d’une propriété rurale qu’on vient d’acquérir\footnote{ \noindent Mac-Leod, un des rares économistes psychologues, avait compris cette âme psychologique de l’idée de valeur. « Quand un homme, dit-il, a fondé quelque part un commerce et, par sa réputation, créé une demande pour ses produits ou pour ses services, l’espoir de la continuation de la demande et le droit de recevoir les profits à réaliser constituent une propriété indépendante et distincte, susceptible de vente et de transport. Ce droit est reconnu par les tribunaux. » Ce n’est pas dans ce cas seulement, c’est partout et toujours que valeur et espoir ne font qu’un.
 }, d’un capital  \phantomsection
\label{v2p129}quelconque, consiste dans l’espoir de reproduire certains produits et de les vendre avec avantage ; et cet espoir, qui n’est jamais certain, qui devient de moins en moins certain (jusqu’a une certaine phase de la civilisation), à mesure que l’industrie se développe, se heurte de deux manières à l’espoir à la fois contradictoire et contraire des producteurs rivaux. Ce ne sont pas seulement leurs jugements, ce sont leurs désirs qui s’opposent. L’avoué, par exemple, qui achète un \emph{titre nu}, paie fort cher l’espérance de se tailler une clientèle aux dépens de celle de ses collègues ; et ceux-ci ont payé encore plus cher l’espérance de conserver ou d’accroître la leur ; ce qui signifie que, par leurs prévisions, d’une part, et, de l’autre, par leurs intérêts et leurs efforts, ils sont en lutte. Ce qui est vrai des offices d’avoués l’est aussi bien des usines, des ateliers, des fermes d’une région, en tant que ces sources de produits se disputent une clientèle qui n’est pas susceptible de s’étendre à leur gré, ou qui ne l’est que moyennant un abaissement du prix de vente, parfois au-dessous du prix de revient.\par
C’est donc à la valeur des capitaux, bien plus et tout autrement qu’à la valeur des produits, que la contradiction des espérances est inhérente. Et j’indiquerai, en passant, incidemment, la difficulté délicate qui naît de là quand on cherche à faire, comme l’ont essayé tant d’écrivains, un inventaire général de la fortune publique. Qu’on y fasse figurer côte à côte, en les additionnant ensemble, tous les produits et services annuels, concurrents ou non, rien de plus légitime. Mais convient-il d’additionner les capitaux à la suite les uns des autres avec la valeur respective que leur prêtent des espérances qu’on sait doublement contradictoires entre elles et irréalisables à la fois ? Il est clair, d’après ce qui précède, que l’on fera beaucoup de doubles emplois, si l’on fait figurer avec le chiffre de leur valeur vénale, où  \phantomsection
\label{v2p130}s’incarnent ces contradictions profondes de jugements et de volontés, les offices d’avoués, de notaires, d’huissiers, les ateliers, les fabriques, etc. C’est de toute évidence pareillement, si l’on s’avise d’inscrire à l’actif de la richesse nationale le total des valeurs de Bourse possédées par les nationaux : n’y a-t-il pas une contradiction et une contrariété manifestes dans l’ascension simultanée, à la Bourse, des fonds de certains États, tels que la Russie et l’Angleterre, qu’on sait antagonistes, sur le marché économique comme dans l’arène politique\footnote{ \noindent Il est encore plus clair qu’on ferait un double emploi manifeste, en additionnant, dans un inventaire général de la fortune de l’Europe, la valeur de l’outillage militaire des diverses nations, dont chacune fonde sur le sien des espérances dont l’espérance de chaque autre est le démenti le plus complet.
 } ? Pourtant, ce serait se tromper que de soustraire en partie les unes des autres certaines valeurs, au lieu de les additionner, ou de traiter les unes comme des valeurs positives, les autres comme des valeurs nulles ou plutôt négatives. Comme nulles, c’est impossible ; toutes contredites qu’elles sont, les espérances les moins fondées existent réellement. Comme négatives, c’est encore plus faux ; car elles sont des forces de premier ordre, qui concourent, avec les espérances les plus justifiées, et non moins efficacement parfois que celles-ci, ne serait-ce que par leur action stimulante sur celles-ci, à la reproduction des richesses. La somme des espérances nationales de reproduction et de bénéfice, exprimée par la valeur totale des capitaux nationaux, ne saurait donc, sans erreur, être diminuée de la somme des espérances de ce genre destinées à être déçues. Il est heureux, au point de vue de l’accroissement général des richesses, que ces espérances se contredisent, et la suppression de ces contradictions n’est concevable que par la renonciation à ce progrès économique sous un régime de production universelle par l’État.\par
Ce qui est fâcheux, sans contestation possible, c’est de voir, de temps à autre, sous l’empire de stimulants nouveaux, \phantomsection
\label{v2p131} tels qu’une invention récente, les espérances contradictoires dont il s’agit s’aviver, s’exalter jusqu’au paroxysme, leur contradiction atteindre à la fureur, et l’intensité de la production sous toutes les formes s’exagérer au point d’aboutir tout à coup à l’avilissement des produits par l’encombrement des marchés. C’est qu’alors à la contradiction mutuelle des producteurs s’est ajoutée celle, bien plus grave, de l’offre des produits disproportionnée à leur demande. Je dis que cette disproportion est aussi une contradiction, car les offreurs croient pouvoir écouler toute leur marchandise à un certain prix et s’y efforcent, tandis que les demandeurs croient ou plutôt savent que ceux-ci ne pourront s’en défaire à ce prix et ils sont résolus à ne l’acheter qu’à un prix inférieur.
 \phantomsection
\label{v2p132}\subsubsection[{II.3.h. Conflits monétaires.}]{II.3.h. Conflits monétaires.}
\noindent Les conflits de nature monétaire ne nous retiendront pas longtemps. Ils sont de deux sortes, nous le savons, ceux de la monnaie avec elle-même, et ceux de la monnaie avec les besoins de la production et de la consommation, de l’échange en un mot, auxquels elle a cessé d’être adaptée. Parlons d’abord des premiers.\par
Les marchés ont toujours été disputés par des monnaies différentes de titre, de substance, d’origine, qui s’y sont fait une concurrence à la fois sourde et acharnée, comparable à celle que se font des langues différentes pour les besoins du commerce ou de la diplomatie. Plus haut nous remontons dans le passé des peuples classiques, du moins jusqu’à l’époque où l’invention de la monnaie a été importée chez chacun d’entre eux, et plus les monnaies en circulation nous y apparaissent nombreuses et variées, comme les idiomes, en même temps que leur domaine, comme celui des idiomes, y est plus étroit et plus circonscrit. Mais cette diversité des monnaies locales, quoique bien plus grande dans l’antiquité que de nos jours, ainsi que l’étroitesse de leur champ d’action, était loin alors, sans doute, de gêner et d’impatienter autant les divers peuples dans leurs relations commerciales ou privées, que la diversité des monnaies actuelles, bien moins nombreuses cependant et plus largement répandues. Car l’expansion du commerce et le développement des communications et des voyages ont grandi bien plus vite encore que l’expansion des monnaies survivantes, au cours de la concurrence et de la sélection monétaires. \phantomsection
\label{v2p133} Il en est de même des langues européennes dont la pluralité, malgré l’élimination progressive d’un grand nombre d’entre elles, est sentie comme une gêne croissante, qui fait tenter des efforts réitérés, parfois chimériques, pour l’établissement d’une langue internationale.\par
Il n’en est pas moins vrai que, dans la Grèce antique si morcelée, les innombrables monnaies de ses minuscules républiques devaient lutter constamment et péniblement les unes contre les autres. Quand un numismate, à présent, regarde dans ses médailles si bien rangées ces petits disques d’or ou d’argent, il a sous les yeux, sans s’en douter, les débris d’un champ de bataille séculaire. Tout le moyen âge aussi, avec son morcellement féodal, a souffert de la rivalité des monnaies seigneuriales entre elles, et de celles-ci avec les monnaies royales\footnote{ \noindent En 1262, par exemple, édit de saint Louis, ordonnant que sa monnaie aurait cours dans tout le royaume, tandis que celle des seigneurs ne pourrait circuler hors du domaine de chacun d’eux.
 }. A cette souffrance s’ajoutait celle que causaient les altérations fréquentes des monnaies. Le droit de battre monnaie était considéré par les rois, et à leur exemple, par les seigneurs, comme l’exploitation d’une sorte de mine d’or ou d’argent. Mais le despotisme qui imposait le cours des monnaies décriées n’empêchait pas leur dépréciation, et on recherchait d’autant plus la bonne monnaie. Des phénomènes analogues ont dû se produire à Athènes, quand Solon, dans l’intérêt des débiteurs, mit en circulation une monnaie affaiblie ; et à Rome, quand, à plusieurs reprises, le titre et le poids des monnaies y ont été altérés pour permettre aux plébéiens de se libérer plus facilement envers les patriciens.\par
Il est aisé de constater, dans cette longue et complexe histoire des monnaies, pleine de péripéties, l’application des lois générales qui président à la propagation des exemples. D’abord, la diffusion si rapide de l’invention des monnaies frappées, à partir du {\scshape vii}\textsuperscript{e} ou du {\scshape viii}\textsuperscript{e} siècle avant notre ère,  \phantomsection
\label{v2p134}en Lydie, atteste la force du courant de mode qui est parvenu à emporter si vite les obstacles opposés par la multiplicité et la clôture des États. Mais ce n’est pas seulement l’idée même qui a été importée, c’est le mode d’exécution. A travers des variantes infinies, un très petit nombre de types se font jour. Cela est surtout frappant au moyen âge. La numismatique féodale, d’après Lenormant, est à la fois remarquable par la diversité locale des monnaies et par leur uniformité dans leur aspect général. Et cette similitude, ici comme dans l’antiquité, s’explique toujours par le prestige contagieux de quelque grand nom, de quelque grande gloire, encore plus peut-être que par un calcul utilitaire. Le motif d’utilité existait, sans nul doute. Sur les marchés grecs, lorsqu’une monnaie avait trouvé meilleur accueil que d’autres, parce qu’elle était celle « de quelque grande cité commerciale dont les opérations s’étendaient au loin » (Lenormant), on copiait, on contrefaisait partout cette monnaie en vogue. « Le public trompé par la parfaite ressemblance des types acceptait ce numéraire d’imitation, et le tour était joué. La monnaie inconnue ou mauvaise circulait sous le couvert de la bonne et lui faisait une concurrence ruineuse, au grand profit des contrefacteurs. » Ainsi se répandirent et jouèrent le rôle de monnaie internationale dans tout le monde grec, « du {\scshape v}\textsuperscript{e} au milieu du {\scshape vi}\textsuperscript{e} siècle, comme espèces d’argent, les \emph{tétradrachmes} d’Athènes, et, après eux, les \emph{statères} de Rhodes ; comme espèces d’or, les statères de Cyzique\footnote{ \noindent Au moyen âge, où le type \emph{chartrain}, le type du \emph{temple}, le type du \emph{châtel}, etc., se disputaient les marchés, la monnaie de \emph{Tours} (livres tournois), par l’exactitude de son poids et la finesse de son titre, acquit une grande vogue et peu à peu fut copiée partout, jusqu’en Palestine. Mais on peut croire que la vogue du type \emph{parisis} n’était pas due à ces simples considérations d’utilité.
 } ». Mais, certainement, le renom glorieux d’Athènes était aussi pour quelque chose dans l’engouement pour sa monnaie, ainsi que la beauté de ses empreintes. On ne peut douter non plus de l’influence prestigieuse exercée par Philippe \phantomsection
\label{v2p135} de Macédoine et par Alexandre le Grand sur le triomphe des monnaies à leur effigie qui régnèrent durant deux siècles sur un immense territoire\footnote{ \noindent « Quelquefois, c’est dans une région nettement déterminée qu’une espèce de monnaie jouit d’une situation de faveur. Par exemple, en Thrace et dans tout le bassin du Danube, de 280 à 80 avant Jésus-Christ, les tétradachmes d’argent de Thasos au type d’Hercule... » C’est ainsi que, de nos jours, les piastres règnent au Mexique. En Epire, en Illyrie, sur toute la côte de l’Adriatique, au {\scshape iv}\textsuperscript{e} siècle avant Jésus-Christ, circulaient les statères d’argent corinthiennes « avec leur belle tête d’Aphrodite armée et leur Pégase ». Lenormant est d’avis que l’admiration de leur beauté a souvent suffi à les faire imiter à l’étranger.
 }.\par
Ce qui est plus certain encore, c’est que, dans toute la numismatique mérovingienne, gauchement copiste des monnaies impériales, se fait sentir l’admiration profonde du génie romain. Et, plus tard, après que Charlemagne eut apparu, c’est l’action contagieuse d’un si grand modèle qui explique l’aspect uniforme des monnaies féodales, leur monotonie dans leur variété, dont il vient d’être question. Partout ces monnaies procèdent, dit Lenormant, « d’une imitation dégénérée des monnaies carlovingiennes ». Quelques siècles après, le sequin de Venise, le florin de Florence, durent à la fois au grand commerce et à la grande célébrité de ces deux républiques la faveur dont ils jouirent. — Mais, incomparablement supérieure à toute autre, en extension et en persistance, ici comme ailleurs, comme en fait de droit, de langage, d’art, est l’imitation de Rome. Nous retrouvons jusqu’à nos jours, et plus persistant encore dans le monde anglo-saxon que dans le monde latin, un dernier vestige de la monnaie romaine. « Les trois mots \emph{libra}, \emph{solidus}, \emph{denarius}, dit \emph{Cournot}, devenus dans notre patois \emph{livre}, \emph{sou}, \emph{denier}, se sont maintenus comme l’ineffaçable empreinte des origines romaines de notre civilisation. La même échelle monétaire a été portée par les Normands en Angleterre où elle s’est perpétuée, mais associée aux noms saxons de \emph{pound}, de \emph{shilling} et de \emph{penny...} Aujourd’hui, sur tous les points du globe, le marchand ou le planteur de race  \phantomsection
\label{v2p136}anglo-normande, en supputant ses pounds, ses shillings et ses pences, inscrit à côté de ses chiffres de petits signes, pour lui hiéroglyphiques, qui ne sont que les initiales des trois mots latins : trace bien légère, marque bien bizarre de choses qui se sont passées si loin de lui, à tant de siècles de distance !... »\par
Si l’influence de l’imitation, soit utilitaire, soit moutonnière, soit esthétique, a fortement agi sur les gouvernants qui ont frappé la monnaie et l’ont mise en circulation, elle ne s’est pas exercée avec moins de force sur le public où la monnaie a circulé. On a foi dans une monnaie, parce qu’on voit les autres y avoir foi : c’est évident pour la monnaie de papier. Si les motifs esthétiques sont étrangers à la préférence qu’on accorde à telle monnaie sur telle autre, ce choix n’en est guère plus rationnel en général, et la moutonnerie y prédomine, sauf parmi les escompteurs et les banquiers qui vérifient seuls les poids et les titres. — Quels que soient du reste les motifs de l’imitation d’autrui en fait de circulation monétaire comme en fait de propagation linguistique ou religieuse, elle est incontestable, et la loi de répétition amplifiante s’applique à ce rayonnement imitatif comme à tout autre.\par
Et le terme où tend, visiblement, cette lutte des monnaies pour l’imitation, est l’unité monétaire, idéal de mieux en mieux réalisé, bien qu’il soit destiné sans doute à ne l’être jamais qu’en partie. Par là s’offre à nous comme trop facile à être résolue pour être longuement discutée, la question du \emph{bimétallisme} et du \emph{monométallisme} qui est, de nos jours, la plus anxieuse des luttes de la monnaie avec elle-même. La solution qui s’impose, dans un avenir plus ou moins rapproché, comme conséquence indirecte de toute l’évolution historique de la monnaie, est le monométallisme or. Par la même raison que dans sa lutte antique avec la monnaie d’argent, la monnaie de fer a été battue comme incommode par sa lourdeur, la monnaie d’argent ne saurait tarder longtemps \phantomsection
\label{v2p137} à être éliminée par la monnaie d’or. Depuis deux siècles déjà, en Angleterre, l’or est en train de chasser l’argent, de le réduire au rôle d’appoint. Des considérations d’ordre pratique peuvent déterminer certains gouvernements à retarder l’avènement du règne de l’or, mais la discussion de ces raisons d’opportunité n’est pas de notre ressort, et, théoriquement, ne présente aucun intérêt.\par
Les luttes de la monnaie avec la production et la consommation, c’est-à-dire ses désaccords avec les besoins de l’échange, peuvent être de nature quantitative ou qualitative. Quantitative, par rareté excessive de la monnaie, qui trouble et entrave les affaires, ou, à l’inverse, par surabondance \emph{subite} de monnaie, qui bouleverse les prix habituels. Qualitative, par suite de la dépréciation des monnaies altérées, ou de leur substance devenue trop lourde et trop peu précieuse, mal adaptée à ses fonctions, comme le fer ou l’airain jadis, comme l’argent bientôt (on voit que le problème du bimétallisme se rattache aussi, par un côté, à ce nouvel aspect de notre sujet), enfin par le mauvais choix de l’unité monétaire (le \emph{franc}, par exemple, et surtout le \emph{centime}, qui n’est d’aucun usage par lui-même et complique tous les calculs). Les désaccords quantitatifs nous occuperont plus loin à propos des crises financières. Quant aux désaccords qualitatifs, il suffira d’en dire un mot, pour montrer leur gravité quand ils ont pour cause l’altération des monnaies, dont il a été déjà question plus haut à un autre point de vue. Un des plus grands faux monnayeurs royaux a été Guillaume III d’Angleterre. Pour se tirer d’embarras, après 1688, il imagina de frapper, avec toutes sortes de métaux vils et quelque peu d’argent, une monnaie qui valait à peine le sixième de sa valeur nominale. Il faut lire dans Macaulay les perturbations et les souffrances qui furent la conséquence de cette mesure insensée, accomplie d’ailleurs avec l’approbation du Parlement. Il s’ensuivait, entre acheteurs et vendeurs, des querelles sans nombre. « L’ouvrier et  \phantomsection
\label{v2p138}le patron se prenaient de mots régulièrement tous les samedis, jour de paye. Le jour de foire et le jour de marché, on n’entendait que clameurs, reproches, injures, malédictions, et c’était un bonheur quand il n’y avait ni baraque renversée ni tête cassée... Le prix des denrées, des chaussures, de l’ale, de la farine d’avoine, s’éleva rapidement. Le paysan ignorant et isolé était misérablement écrasé entre deux classes de gens, dont l’une ne lui donnait de l’argent qu’à sa valeur nominale et dont l’autre ne voulait le prendre qu’au poids. » Les mêmes calamités se sont reproduites durant la Révolution française par le cours forcé des assignats.
 \phantomsection
\label{v2p139}\subsection[{II.4. Les crises (luttes aiguës)}]{II.4. Les crises \emph{(luttes aiguës)}}\phantomsection
\label{l2ch4}
\subsubsection[{II.4.a. Les luttes aiguës entre coproducteurs : grèves. Grèves anciennes et nouvelles.}]{II.4.a. Les luttes aiguës entre coproducteurs : grèves. Grèves anciennes et nouvelles.}
\noindent A mesure que nous avançons dans l’étude de l’opposition économique, les désaccords dont nous avons à nous occuper présentent une progression marquée d’intensité et d’importance. Après avoir parlé des débats tout intérieurs, \emph{intra}-psychologiques, d’où résultent les \emph{prix}, nous avons étudié le phénomène \emph{inter}-psychologique des luttes habituelles et normales, qui mettent aux prises, modérément, les individus avec les individus. Enfin nous allons traiter des conflits aigus, \emph{grèves} et \emph{crises}, anomalies exceptionnelles qui se produisent sous la forme de combats entre des masses d’individus, sciemment ou inconsciemment solidarisées d’intérêts les unes contre les autres. Commençons par les grèves.\par
Les grèves sont des phénomènes de marchandage collectif. Dans tout marchandage qui dure un peu, il y a des péripéties : l’une des parties fait semblant de ne plus vouloir acheter, ou de ne plus vouloir vendre, ce qui est une grève en miniature. Ce qui fait pendant aux grèves, dans le domaine de la consommation, n’est-ce pas, chez les consommateurs, le refus de consommer à certaines conditions, par exemple le boycottage de certains magasins ou le désabonnement à certains journaux dont le changement de politique est réprouvé par une partie de leurs lecteurs ? Ces faits ont une importance presque égale aux chômages volontaires des ouvriers. Le protectionnisme poussé jusqu’au prohibitionnisme par l’exagération des tarifs douaniers, c’est la grève  \phantomsection
\label{v2p140}obligatoire des consommateurs d’un pays à l’égard des produits d’un autre pays.\par
En réalité, ce n’est pas entre patrons et ouvriers qu’existe l’antagonisme le plus irréductible d’intérêts. Car si, pendant que les ouvriers s’entendent pour hausser le salaire, les patrons s’entendaient pour hausser le prix du produit, les exigences des premiers n’auraient rien dont les seconds eussent à souffrir. Il n’y a que les consommateurs qui en pâtiraient. C’est entre les producteurs (patrons et ouvriers compris) et les consommateurs qu’est la discordance profonde d’intérêts et de vœux. Sans doute, elle est un peu tempérée par ce fait que les producteurs sont en même temps consommateurs. Mais les producteurs d’un article ne le consomment, en général, que faiblement, et sont surtout consommateurs des autres articles. Aussi, dans leur préoccupation de voir s’élever le prix de vente du produit qu’ils fabriquent, ont-ils infiniment peu à tenir compte du préjudice qui en résultera pour eux-mêmes quand ils le consommeront. Et ils ne songent pas, ils n’ont pas trop à songer, que, si tous les ouvriers et patrons de toutes les industries agissaient de même, ils auraient tous perdu d’une main ce qu’ils auraient gagné de l’autre. Un tel accord est bien peu probable. D’ailleurs, s’il semblait vouloir s’établir, il s’agirait de l’empêcher. Chaque catégorie de producteurs — ou chaque fédération de catégories coalisées — a intérêt à ce que les autres catégories, les autres fédérations ne l’imitent pas quand elle élève ses prix, c’est-à-dire ne forcent pas à leur tour les consommateurs de leurs articles à les payer plus cher.\par
Cela revient à dire que la lutte entre producteurs et consommateurs en suppose ou en amène forcément une autre, celle des catégories de producteurs les unes contre les autres, sans distinguer dans chacune d’elles les patrons et les ouvriers. Par exemple, la guerre existe entre les producteurs agricoles, usant de tous leurs pouvoirs électoraux pour faire hausser par des droits protecteurs le prix des denrées nationales, \phantomsection
\label{v2p141} et les producteurs manufacturiers, désireux de lever les barrières douanières.\par
Et cette guerre-là ne peut se terminer que par le triomphe de l’un des deux intérêts et le sacrifice total de l’autre, ou bien par le consentement des deux à leur sacrifice partiel ; jamais par l’immolation d’un tiers aux deux combattants demeurés intacts, comme \emph{peut} se terminer — je ne dis pas comme se termine le plus souvent en fait — le conflit des patrons et des ouvriers dans la même branche de la production.\par
On voit cependant, à notre époque, des ouvriers et même des patrons s’associer avec des ouvriers et des patrons appartenant à des industries différentes et concurrentes. Aux États-Unis, par exemple, la \emph{fédération américaine du Travail} comprend un grand nombre de métiers distincts dont les ouvriers sont parvenus à s’unir en vue d’une action commune contre leurs patrons par la \emph{grève sympathique.} Mais rien ne prouve mieux que ce fait le caractère politique et social, et non pas seulement économique, de ces opérations quasi militaires. D’ailleurs, le groupement \emph{professionnel} est, en Amérique comme en Angleterre, le fondement des \emph{Trades-Unions}. M. Wright, dans son \emph{Évolution industrielle aux États-Unis}, montre que « c’est là une base solide, si l’on en juge par l’expérience faite dans notre pays » (en Amérique). Si les ouvriers des professions les plus diverses se coalisent, ce n’est qu’en vue de la fameuse « lutte des classes ». Quand une coalition pareille se produit, c’est toujours par l’initiative et la propagande entraînante des ouvriers d’un corps de métier en relief et particulièrement désigné pour cet apostolat, tel que celui des typographes, et c’est après beaucoup de résistances, vaincues une à une par beaucoup d’influences et de suggestions personnelles, que des congrès répétés aboutissent à cette alliance offensive et défensive.\par
On est surpris, soit dit en passant, de voir le rôle prépondérant, \phantomsection
\label{v2p142} presque égal à celui des typographes, qui a été joué aux États-Unis, dans ces efforts tentés pour la concentration des forces ouvrières, par les ouvriers \emph{cigariers.} On comprend aisément la puissance des typographes, en songeant à celle de la Presse, dont ils disposent. Mais celle des cigariers ? L’explication du fait est bien simple et montre que je n’ai pas exagéré l’importance de la conversation, comme agent des transformations sociales. Les cigariers, en effet, ont un travail qui leur permet de causer beaucoup entre eux. « Dans les ateliers, dit M. Vigouroux\footnote{ \noindent \emph{Concentration des forces ouvrières.}
 }, ils se font vis-à-vis ; leurs doigts sont occupés, mais leur langue fonctionne d’autant plus volontiers que leur attention n’est pas absorbée comme celle d’un tisseur, d’un lamineur, d’un scieur à la mécanique. » Aussi fournissent-ils aux assemblées populaires beaucoup d’orateurs.\par
La lutte entre ouvriers et patrons, quand elle atteint ce paroxysme, se complique d’une autre lutte, non moins vive, entre les ouvriers syndiqués et les ouvriers non syndiqués. Dans celle-ci, l’arme employée est le boycottage des ouvriers dissidents, complété aux États-Unis par le \emph{label}, marque de fabrique par laquelle les ouvriers syndiqués certifient que le produit ainsi estampillé sort de leurs mains et le recommandent par là au public qui leur est favorable, à l’exclusion des produits similaires non revêtus de ce timbre d’origine. Les ouvriers cigariers, en particulier, sont parvenus à expulser de la circulation tous les cigares \emph{scab} (c’est-à-dire \emph{tabou}) qui ne portent pas leur label. Une Presse à leur service répand la persuasion que ces cigares, fabriqués par des immigrants chinois ou russes, sont dangereux pour la santé publique et propagateurs de maladies contagieuses.\par
Il y a, dans les grèves, à distinguer nettement deux sortes d’oppositions psychologiques d’où elles procèdent. D’abord, elles opposent les désirs des patrons qui veulent le maintien des salaires ou des heures de travail aux désirs  \phantomsection
\label{v2p143}des ouvriers qui veulent l’augmentation des salaires et la diminution de la durée du travail. Elles mettent aux prises, ensuite, la volonté des ouvriers qui ne veulent pas qu’on travaille avec celle des ouvriers qui veulent travailler. Mais ces contrariétés de désirs sont attisées par des contradictions de jugements qui les compliquent : les uns jugent juste, les autres injuste le salaire actuel. Si ces contrariétés et ces contradictions étaient toujours également intenses, la lutte serait sans fin. Mais, heureusement, il n’en est pas ainsi. Souvent, le désaccord des jugements, en dépit des professions de foi intransigeantes qu’on arbore, est moins profond que celui des volontés ; et quand, par des négociations, on est parvenu à se faire \emph{entendre raison} les uns aux autres, cela signifie que le conflit des désirs est résolu par l’accord des jugements : ouvriers et patrons finissent, malgré leurs vœux contraires, par reconnaître qu’un certain salaire, fixé après débats, est légitime. — Il peut arriver aussi, à l’inverse, que le conflit persistant des jugements soit tranché par l’accord survenu dans les désirs : patrons et ouvriers, malgré leur foi opposée en des droits contradictoires, s’aperçoivent qu’ils ont pareillement intérêt à reprendre le travail. — Il importe de bien distinguer ces deux dénoûments différents du conflit : ce qui serait grave, ce serait que le second devînt prédominant, comme il semble que ce soit le cas de nos jours. La paix sociale est bien mal assurée quand elle ne repose que sur une harmonie passagère des intérêts, qui dissimule mais n’efface pas l’antagonisme durable des principes.\par
Si l’intérêt était le seul grand moteur des actions humaines, on verrait les divers groupes d’ouvriers de la même industrie se faire la guerre au moins aussi souvent qu’ils la font à leurs patrons. Ce n’est pas seulement en temps de grève, c’est en tout temps qu’ouvriers et ouvriers similaires ont des intérêts divergents, quand des usines rivales se disputent une même clientèle. Quand les ouvriers d’une verrerie ou  \phantomsection
\label{v2p144}d’une manufacture quelconque sont en même temps propriétaires, enracinés à leur pays par des liens anciens, ils ont un intérêt majeur au développement de l’établissement où ils travaillent et au déclin des établissements rivaux : par ce côté, comme nous l’avons dit plus haut, ils ont le même intérêt que leurs patrons. Pourquoi cependant ne voit-on jamais, de nos jours du moins, ces divergences d’intérêt provoquer des conflits entre fractions de la population ouvrière, sauf le cas où elles appartiennent à des nationalités différentes, comme, par exemple, les luttes sanglantes entre ouvriers français et ouvriers italiens dans les départements du midi de la France, ou entre ouvriers américains et coolies chinois aux États-Unis\footnote{ \noindent Ou entre ouvriers syndiqués et non syndiqués, ou entre ouvriers grévistes et ouvriers qui veulent continuer ou reprendre le travail.
 } ? Pourquoi, au contraire, les grèves d’une usine ont-elles une tendance, de plus en plus accusée, à se généraliser, à se communiquer, par contagion à distance, aux usines rivales elles-mêmes, si bien qu’en Amérique les forces ouvrières sont, à chaque instant, susceptibles d’être mobilisées comme des armées sur toute l’étendue d’un vaste territoire ? N’est-ce pas parce que l’esprit de classe, ou l’esprit de nationalité, l’emporte sur l’esprit de groupe d’atelier, ce dernier étant un agrégat superficiel et factice, tandis que la classe, ou la nation, est un agrégat naturel et solide, aux profondes attaches dans le cœur ? Et n’est-ce pas aussi parce que les ouvriers sont unanimes à affirmer et revendiquer les mêmes droits, alors même qu’ils ont des intérêts partiellement ou passagèrement opposés ? Ce qui distingue éminemment les grèves de notre époque de toutes les grèves antérieures, c’est précisément cette prépondérance des \emph{principes} sur les \emph{intérêts}, des croyances sur les désirs, parmi les causes qui les provoquent. — Les grèves, d’ailleurs, ne sont rien de particulier à notre siècle. Le passé les a connues, et même le très haut passé. Dans leur ouvrage sur le \emph{Trade-unionisme}, Sydney et Béatrice Webb rappellent \phantomsection
\label{v2p145} que, en 1490 avant Jésus-Christ, l’Égypte a vu une sorte de grève étrangement semblable à une de celles de nos jours. Les briquetiers juifs s’étaient révoltés, à cette date reculée, contre les autorités pharaoniques qui leur avaient ordonné de fabriquer des briques sans paille, à peu près comme, de nos jours, en 1892, les fileurs de coton de Stalybridge se sont mis en grève parce qu’on leur donnait une matière défectueuse à travailler. — Chez les typographes lyonnais, au {\scshape xvi}\textsuperscript{e} siècle, « ces chômages concertés n’étaient pas chose rare, car on avait créé un mot pour les désigner\footnote{ \noindent M. Hauser, \emph{Ouvriers du temps passé}, 1899 (Paris, F. Alcan). Lire tout le chapitre très intéressant, intitulé : \emph{Histoire d’une grève au {\scshape xvi}\textsuperscript{e} siècle}.
 } », le mot \emph{tric.} Ces trics ont tous les caractères de nos grèves : les meneurs, dont parlent les documents, y échauffent les têtes ; les \emph{triqueurs} « menacent les compagnons et apprentis qui ne veulent pas quitter l’ouvrage, de les expulser de la confrérie, et ils exécutent leurs menaces ». Ils les battent et les blessent. Ils se donnent « une véritable organisation militaire », prennent des bannières et enseignes comme signes de ralliement, désignent un capitaine, des lieutenants et chefs de bandes. A la force publique qui essaie vainement de réprimer leurs désordres, ils résistent ouvertement, ils rossent le guet. « Le procureur du roi les accuse d’avoir battu le prévôt et les sergents jusqu’à mutilation et effusion de sang. » — Les grévistes se plaignent de l’insuffisance des salaires, mais, avant tout, ils protestent contre la violation de leurs droits, et l’énoncé de leurs griefs, chose exceptionnelle sous l’ancien régime, est un exposé de principes. (Cette exception s’explique par l’influence de la Réforme, qui prit, à ses débuts, la couleur d’une révolution sociale.) « C’est (déjà) un acte d’accusation contre le capitalisme ; les patrons sont dénoncés comme des exploiteurs, s’engraissant de la \emph{sueur} — le mot y est — de ceux qui les font vivre de leur travail. »\par
Des émeutes de ce genre ont troublé Lyon en 1519, en  \phantomsection
\label{v2p146}1529, en 1530. L’une d’elles, en 1539, se prolongea pendant trois ou quatre mois et ne se termina que par un édit royal. Cette grève, où l’on sent l’action des passions religieuses et révolutionnaires de l’époque, présente une liaison remarquable avec une grève parisienne de la même date. Le règlement de la grève parisienne a servi de modèle, nous dit M. Hauser, au tric lyonnais. — Dans l’ouvrage de M. Germain Martin sur la \emph{Grande Industrie} en France sous Louis XV, je lis que des associations ouvrières ayant la grève pour objet ou pour instrument de combat contre les patrons s’étaient déjà formées au {\scshape xviii}\textsuperscript{e} siècle. « Un arrêt de 1777 constate que les ouvriers des manufactures de papier \emph{du royaume} s’étaient liés par une \emph{association générale} au moyen de laquelle ils arrêtaient ou favorisaient à leur gré l’exploitation des papeteries et par là se rendaient maîtres du succès ou de la ruine des entrepreneurs. Lesdits ouvriers se sont fait entre eux des règlements dont ils maintiennent l’observation par des amendes prononcées \emph{tant contre les maîtres} que contre les ouvriers. » Le \emph{boycottage} des ouvriers est déjà pratiqué. Les ouvriers « prétendent avoir le droit d’introduire dans les maisons de leurs maîtres qui bon leur semble, sans sa permission et malgré sa défense... Les maîtres fabricants qui ont passé outre et pris le parti de former des ouvriers qui fussent plus dociles ont vu ceux de l’association qui sont employés dans les autres fabriques \emph{s’attrouper, menacer, injurier et attaquer les élèves qu’ils formaient.} » Ce fait arriva en Dauphiné, à Ambert, à Thiers, etc. — A Lyon, en 1737, émeutes successives, chansons homicides contre \emph{Vaucanson} et autres. A Thiers, une sorte de \emph{syndicat patronal} s’opposa à l’\emph{Association ouvrière}.\par
Mais une telle extension d’un mouvement ouvrier était chose tout à fait exceptionnelle alors. Si l’on cherche, en effet, entre les grèves d’autrefois et celles d’aujourd’hui une différence \emph{objective} bien caractérisée, on n’en trouvera qu’une : la facilité, l’aptitude plus grande, de plus en plus  \phantomsection
\label{v2p147}marquée à présent, de ces phénomènes jadis locaux, à se généraliser, à s’internationaliser même. Pour la première fois depuis que le monde est monde, la possibilité de la \emph{grève générale} nous apparaît, et inquiète de temps en temps les hommes d’État. C’est conforme à cette loi de répétition amplifiante que nous avons tant de fois signalée.\par
Cette différence entre les grèves d’aujourd’hui et les grèves d’autrefois est déjà énorme. Mais, si nous les considérons par leur côté subjectif, par toutes les tempêtes de passions et de revendications qu’elles soulèvent sous les crânes, la différence est bien plus profonde encore. La psychologie de l’ancien gréviste est celle d’un écolier tapageur et révolté qui casse les meubles de son dortoir, conspue le pion, le patron, mais reconnaît l’autorité du recteur, c’est-à-dire du roi, et courbe la tête, soumis ou résigné, sous la sentence royale. Après tout, entre son patron et lui, la distance sociale est faible, et la passion de l’égalité n’est pas née en lui. Du chef d’atelier à ses ouvriers il y a infiniment moins loin que du seigneur à ses paysans aux mêmes époques. Et il en a été ainsi jusqu’à la Révolution française. « On a remarqué avec raison, dit Hanotaux\footnote{ \noindent \emph{La Jeunesse de Richelieu.}
 }, que, dans les cahiers des États généraux de 1789, les plaintes des ouvriers sont moins nombreuses et moins pressantes que celles des paysans. » A présent, c’est l’inverse. Aussi les grèves d’aujourd’hui sont-elles bien moins comparables aux grèves d’ancien régime, qu’aux jacqueries du moyen âge, par la fureur des haines qui s’y déploient, par l’amoncellement de griefs qui trouvent occasion d’y éclater, par l’audace des programmes affichés, par les ravages matériels parfois, incendie de chantiers ou destruction d’usines. Mais les grévistes actuels diffèrent profondément des jacques anciens par les mobiles qui les poussent, par la foi socialiste qui les anime et qui prend toutes les allures d’une religion naissante. De là l’intensité du prosélytisme qui s’y développe, des passions qui y naissent \phantomsection
\label{v2p148} des convictions ; de là l’action puissante des apôtres qui fanatisent, l’exaltation réciproque qui va jusqu’au délire et à l’hallucination collective, jusqu’à l’héroïsme du dévouement ou au paroxysme du crime. Ces fausses jacqueries ont ainsi un faux air de croisade. Les femmes, plus fanatisées encore que les hommes, quand elles y participent, y deviennent facilement féroces et meurtrières. Aussi haï, plus haï encore que le patron ou l’ingénieur de la mine, est le camarade non gréviste, comme l’est un renégat par un croyant. On brise les outils de ce travailleur impie, on menace de l’assassiner, comme l’ingénieur qu’on écharpe ; et ces crimes collectifs sont réputés de légitimes expiations.\par
L’action intermentale se donne là carrière ; et, comme partout où elle est surexcitée au delà d’un certain degré, elle se traduit par une sorte de fièvre collective qui a ses phases successives, qui grandit comme une onde, atteint son apogée et puis décroît. Dans les milieux politiques, l’obsession générale d’une affaire produit les mêmes effets morbides, traverse les mêmes phases. Parfois cette maladie déborde des milieux où elle a pris naissance, exerce sa contagion au dehors. Dans la grande grève de Chicago, en 1894, qui a été, peut-être, le plus grave et le plus important des conflits ouvriers jusqu’à ce jour, le pays tout entier manifesta sa sympathie ou son hostilité. Elle l’avait partagé en deux, à peu près comme chez nous une \emph{affaire} qui fut un cauchemar national.\par
Cette grève vraiment historique mérite une mention spéciale. Elle est née, en mai 1894, à l’occasion d’une demande formulée par certains employés de la \emph{Pulman’s Company}, tendant au rétablissement des salaires de l’année précédente\footnote{ \noindent \emph{L’évolution industrielle aux États-Unis}, par Wright, trad. fr. (1901).
 }. D’après M. Wright, le refus opposé par cette compagnie de chemins de fer à cette prétention était assez justifié. Quoi qu’il en soit, une vaste association d’employés de chemins de fer, comprenant 150 000 membres, prit fait et  \phantomsection
\label{v2p149}cause pour ce petit groupe d’ouvriers. Mais, par contre, la cause du directeur de la Compagnie fut chaudement embrassée par l’Association générale des représentants des principales compagnies de chemins de fer américains, la \emph{General manager’s Association.} Les Compagnies représentées dans cette société employaient plus du quart de tous les employés de chemins de fer aux États-Unis. La lutte, qui dura plus de deux mois, fut terrible. Mais elle révéla un grand fait qui se dégage aussi des autres grèves dites historiques et qui a servi de leçon aux États-Unis : c’est que, si les adversaires se sont frappés de grands coups, s’ils ont détruit et gaspillé des sommes folles dans ce long échange de préjudices par représailles, ils ont fait ensemble plus de mal encore au public, spectateur à la fois et victime de leur bataille. « La suspension des transports à Chicago paralysa les affaires et fut une cause de ruine pour un grand nombre de gens dont l’industrie, le commerce, les voyages et les approvisionnements indispensables dépendent d’un service régulier des transports à l’arrivée, au départ ou dans la traversée de Chicago. » Ajoutons que « tout ce qui accompagne inévitablement une grève importante fut mis en jeu à Chicago : émeutes, menaces, violences, meurtres, incendies, vols, crimes de toutes sortes furent commis. L’armée dut intervenir. Le total des forces réquisitionnées pendant la durée de la grève fut de 14 186 soldats. » Ce qui frappa tout le monde, ce fut l’absolue insouciance des deux parties relativement aux intérêts du public, aux droits du public, sacrifiés avec le plus grand sans-gêne pendant la mêlée. Là est l’écueil de la grève générale : elle donnera au public conscience de l’opposition de ses intérêts et de ses droits avec les intérêts des coproducteurs qui se battent sur son dos, et par là le public sera conduit un jour à prendre les mesures nécessaires pour se faire respecter. D’autant mieux que d’autres grèves, plus récentes, donnent lieu de faire une remarque du même ordre, mais plus grave encore : à savoir,  \phantomsection
\label{v2p150}la parfaite indifférence des grévistes, et aussi bien de leurs adversaires, à l’égard des intérêts nationaux que leur lutte compromet au plus haut point.\par
La multiplication des grèves nous achemine donc inévitablement à une réglementation et à une restriction législative du droit de grève. A mesure que l’ouvrier sera plus généralement considéré comme remplissant une \emph{fonction sociale}, ainsi qu’il y prétend, il devra s’attendre à être traité comme un \emph{fonctionnaire}, et son patron aussi bien. Or, on est d’accord pour reconnaître qu’il ne saurait appartenir à tous les fonctionnaires et \emph{en tout temps} de se mettre en grève et de rester en grève indéfiniment. Dire à la fois que le travail est une fonction publique et que le droit à la grève est absolu et sans limites, n’est-ce pas tout à fait contradictoire et incohérent ? De ce que le travailleur est un fonctionnaire on déduit que l’État — comme jadis le roi — a le droit et le devoir de régler les questions des salaires et les conditions du travailleur. Mais du même principe ne doit-on pas déduire aussi que l’État a le droit et le devoir de réglementer les chômages et les grèves ?
\subsubsection[{II.4.b. Crises proprement dites ; crises économiques comparées à crises politiques ou autres. Deux sortes de crises : crises-guerres et crises-chutes. Leur cause.}]{II.4.b. Crises proprement dites ; crises économiques comparées à crises politiques ou autres. Deux sortes de crises : crises-guerres et crises-chutes. Leur cause.}
\noindent Parlons maintenant des crises, mot compréhensif qui embrasse à la fois les crises commerciales, les crises agricoles, les crises monétaires, et en général tous les troubles aigus et passagers, d’apparence morbide, dont la production et la consommation peuvent souffrir. Qu’est-ce que ces crises ? Quelles sont leurs causes ? Ces anomalies obéissent-elles à des lois ? Leur reproduction fréquente au cours de notre siècle est-elle un phénomène transitoire propre à caractériser l’âge anarchique que nous traversons, ou y a-t-il lieu de croire à leur répétition indéfinie ? Nous allons essayer de répondre à ces questions.\par
On dit une crise commerciale, comme on dit une crise  \phantomsection
\label{v2p151}ministérielle, une crise religieuse, une crise morale, une crise linguistique même. Qu’est-ce donc qu’il y a de commun entre ces perturbations d’ordre si différent pour qu’un même mot serve à les désigner ? Je n’y vois rien de semblable, si ce n’est qu’au fond de ces troubles il y a toujours des attentes trompées, par suite soit d’une opposition de volontés ou de jugements en présence et en lutte, soit d’un obstacle opposé à des volontés ou d’un démenti infligé à des jugements par des événements extérieurs. — Une mauvaise récolte, une épizootie, un tremblement de terre qui ruine tout un pays, et y détermine une crise agricole ou commerciale, est un démenti opposé par les faits physiques aux espérances, aux prévisions des agriculteurs ou des commerçants, ainsi qu’un obstacle opposé par les forces naturelles aux volontés humaines. — Un ministère tombe, d’une manière inattendue, sous les coups d’une coalition parlementaire : on dit qu’il y a crise, parce que l’attente de tout un parti, sinon du pays, qui croyait à la durée plus prolongée du ministère, a été contredite par un vote de méfiance, et parce que la volonté tenace des ministres de rester au pouvoir a été combattue et vaincue par la volonté contraire de leurs ennemis. Si tout le monde prévoyait la chute des ministres, l’on ne dit plus qu’il y a crise, ou, si on le dit par habitude, cela tient à ce que, le plus souvent, le renversement d’un cabinet est imprévu, au moment du moins où il a lieu.\par
Une usine fabrique énormément, puis tout à coup s’arrête : il y a pléthore, elle ne peut plus écouler ses produits. Crise industrielle. Ici encore une attente, celle des fabricants, a été déçue, démentie par les faits. Et cette déception a pour effet de rendre très vive la concurrence des fabricants de produits similaires, la contrariété des désirs de ces producteurs rivaux, qui offrent à des prix de plus en plus bas leur stock de marchandises. Toute surproduction amène ainsi une crise ; et tout déficit de la production, à l’inverse, toute disette notamment, provoque une crise aussi,  \phantomsection
\label{v2p152}car il y a ici une attente trompée du consommateur, ce qui est encore plus grave.\par
Il peut y avoir contradiction de volontés ou de jugements sans qu’il y ait attente trompée. Dans ce cas, il y a crise chronique si l’on veut, mais non pas crise aiguë, crise proprement dite. Par exemple, on compte en ce moment, d’après M. Jules Domergue\footnote{ \noindent \emph{Révolut. économique} (1890). M. Domergue, étant bimétalliste, est un peu porté à exagérer le chiffre des populations attachées au métal blanc.
 }, 300 millions d’hommes qui s’obstinent à ne vouloir payer ni être payés autrement qu’en or, et 826 millions d’hommes qui ne veulent ni payer ni être payés qu’en argent. C’est là un mal chronique, un dissentiment des plus fâcheux pour le développement du commerce international, et l’un des côtés les plus graves de la question du bimétallisme. Mais est-ce là une crise monétaire ? Non, la vraie crise monétaire est une souffrance momentanée due à la raréfaction de la monnaie métallique, or ou argent, qui a été drainée à l’étranger par une période d’enfièvrement réciproque, d’entreprises chimériques, de surproduction et de surélévation inconsidérée des prix : alors, à l’écueil redoutable des échéances, des naufrages commerciaux se produisent, signal d’autres désastres, et des milliers de déceptions douloureuses succèdent aux visions d’Eldorado.\par
Appellerons-nous \emph{crises juridiques} les conflits entre un droit ancien qui subsiste et un droit nouveau qui surgit, entre une ancienne et une nouvelle jurisprudence ? Oui, aussi longtemps que ce conflit n’est pas encore résolu et que l’ambiguïté des textes contradictoires fait naître de nombreux procès, dont la solution détruira les espérances, les confiances des plaideurs vaincus. Chaque fois qu’une innovation législative, ordonnance royale, édit impérial, décret, déchire le tissu d’un code ou d’une coutume établie, il s’écoule un certain temps avant que la déchirure soit recousue, et c’est là vraiment un temps de crise. Combien de fois ces perturbations ont dû être subies à la fin du moyen âge, en France  \phantomsection
\label{v2p153}et en Allemagne, quand les légistes superposaient peu à peu le droit romain exhumé au droit coutumier ! — Mais, quand il s’agit d’une lutte sourde et séculaire qui constitue presque toute l’évolution d’un droit, par exemple de la lutte entre le \emph{Jus quiritium} et le droit prétorien, qui s’est prolongée pendant toute la République romaine, on peut hésiter à nommer crise un débat si prolongé. A vrai dire, c’est une série de petites crises séparées par des intervalles de jurisprudence momentanément assise et rassurante. De nos jours, le code civil ou le code pénal de Napoléon est traité par nos chambres de députés comme le \emph{Jus quiritium} par le préteur de Rome, et les tribunaux civils ou correctionnels sont embarrassés pour accorder avec l’esprit autoritaire de ces codes l’inspiration démocratique des dispositions nouvellement votées. Ils ne s’accordent pas plus entre eux pour l’interprétation de ces textes récents, par exemple pour l’application de la loi sur les accidents du travail ; et, dans la mesure où les décisions qu’ils rendent trompent des attentes qu’une jurisprudence mieux fixée eût empêché de naître, on peut dire qu’il y a crise juridique, crise de croissance d’ailleurs et de rénovation nécessaire. — Observons que les crises juridiques sont bien plus profondes à la fois et moins aiguës, plus longues et moins vives, que les crises économiques.\par
Les crises politiques, dont les crises ministérielles ne sont qu’un incident, consistent en déplacements plus ou moins brusques du Pouvoir qui, passant d’un parti à un autre, d’une coterie à une autre coterie, transforme en cruels mécomptes une foule d’espoirs très chers, en apparence fondés. Les crises religieuses, tourment infernal de toute une génération partagée entre deux croyances qui se la disputent, conversion douloureuse d’un peuple à une nouvelle foi ou à une nouvelle incrédulité, ont aussi pour effet d’arracher à un grand nombre d’âmes les espérances dont elles vivaient, les illusions qui faisaient leur force ; et, si nous pouvions faire revivre les polythéistes pieux du temps de Constantin,  \phantomsection
\label{v2p154}par exemple, ou les sectateurs d’Odin évangélisés par des moines irlandais, peut-être découvririons-nous que leur état d’âme lamentable à la disparition de leurs rêves héréditaires, quand il leur a fallu renoncer aux voluptés promises en cette vie ou en l’autre, à Vénus ou au Walhalla, sans parvenir encore à les remplacer pleinement par la foi au Christ, n’était pas sans analogie avec l’état d’âme des milliers de familles modernes qu’une série de désastres financiers, au cours d’une grande crise commerciale, plonge dans la ruine et la détresse. Vues par leur côté psychologique, par cette angoisse pessimiste et désespérée qui leur est commune, toutes les crises, religieuses, politiques, économiques, sociales se confondent. — On appelle plus spécialement \emph{crises sociales} les périodes révolutionnaires d’expropriation en masse, comme les cités antiques de la Grèce et de la Grande-Grèce en ont tant traversées dans leurs convulsions historiques, et comme il n’est pas impossible que les sociétés modernes en traversent encore... Là aussi, là surtout, ce sont des milliers et des millions d’attentes cruellement trompées qui s’exhalent en plaintes et malédictions. Songez aux douleurs des bannis de Sybaris et de Crotone quand ils s’en allaient chassés de leurs belles demeures, de leurs champs fertiles, où ils étaient nés, où dormaient leurs aïeux.\par
Si les uns se lamentent et se désespèrent, je sais bien que d’autres se réjouissent ; je sais bien qu’il n’est pas de crise commerciale d’où ne jaillissent des fortunes subites, qu’il n’est pas de crise religieuse d’où n’éclosent des espérances et des illusions nouvelles, plus fortes souvent et plus fortifiantes que celles qu’elles remplacent ; qu’il n’est pas de crise ministérielle dont un parti ne s’applaudisse ; qu’il n’est pas de crise sociale où n’éclatent des transports de joie. Là où il y a beaucoup d’attentes trompées, il y a aussi beaucoup d’\emph{attentes surpassées}, beaucoup d’heureuses surprises. Et, de fait, il est beaucoup de crises salutaires par leurs conséquences finales, par le renouvellement qu’elles ont  \phantomsection
\label{v2p155}pour dénouement. Mais, prises à part, considérées au moment où elles ont lieu, sans égard à leurs suites — tantôt funestes, tantôt bienfaisantes — elles sont toujours accompagnées de plus de douleurs qu’elles n’apportent de joies, elles se soldent momentanément par un excédent de souffrance et de tristesse. On peut en donner pour raison cette pensée de Schopenhauer : « Demandez-vous, quand une bête en mange une autre toute vive, si le plaisir de la première égale en intensité la torture de la seconde. » Assurément non. Jamais, en moyenne, la satisfaction d’une classe, d’une secte, d’une fraction du pays, qui en dépouille et en ruine une autre, ne saurait être égale à la dure peine du parti dépouillé et réduit à la misère, à ce qui est la misère pour lui, étant données ses habitudes de vie antérieure. Il en est ainsi en vertu d’une considération de Laplace souvent citée et qui est, au fond, une application anticipée faite par ce grand géomètre du principe de Weber sur le logarithme des sensations. Il en est du parti qui exproprie le parti contraire comme du joueur qui gagne au jeu, quand l’enjeu est toute la fortune des deux joueurs. Celui qui gagne voit doubler sa fortune, ce qui augmente un peu, mais ne double pas son bonheur, tandis que l’autre voit s’anéantir la sienne, ce qui, s’il était déjà gêné, fait bien plus que doubler sa gêne. — Ce n’est donc pas sans raison que les crises sont redoutées, même si elles paraissent devoir être finalement utiles et fécondes.\par
Il faut bien distinguer, d’après ce que nous avons dit plus haut, deux sortes de crises : les unes où l’obstacle opposé aux volontés, le démenti opposé aux jugements, cause de l’attente trompée, provient de volontés ou de jugements contradictoires ; les autres où il provient de la résistance des choses, de circonstances que nul n’a voulues expressément quoiqu’elles puissent être la résultante de faits volontaires. Dans le premier cas, les crises sont l’effet de vraies guerres, de luttes intenses, de concurrences exaspérées ; dans  \phantomsection
\label{v2p156}le second cas, ce sont des chutes générales dans une sorte de fossé où tout le monde vient tomber. Cette distinction des crises-guerres et des crises-chutes est générale. Elle s’applique à la politique, à la religion, à la morale, comme au domaine économique. Il y a des cabinets, des gouvernements, qui succombent sous les coups d’un adversaire ; il en est d’autres qui s’engloutissent comme d’eux-mêmes dans quelque catastrophe imprévue, dans un désastre national. Il est des religions qui meurent de mort violente, tuées et piétinées par une religion prosélytique qui se répand ; il en est d’autres qui sombrent dans quelque révolution des mœurs ou des idées provoquées par quelque grand commerce extérieur, par l’importation de découvertes scientifiques, par la révélation d’un monde étranger, sans nulle polémique anti-religieuse, sans nul conflit avec une foi rivale. De même, il est des industries qui dépérissent ou qui périssent, des ateliers qui se ferment, des chantiers abandonnés, parce que des industries rivales et antagonistes ont apparu, que des ateliers nouveaux se sont ouverts, que des chantiers nouveaux sont pleins de vie et d’animation grandissante ; et il est des industries qui tout à coup se mettent à languir, des fabriques qui congédient leurs ouvriers, sans nulle concurrence directe et meurtrière, sans que rien de nouveau soit venu leur porter ombrage, tout simplement parce qu’on s’aperçoit que les débouchés sur lesquels on comptait sont engorgés, et que l’or manque pour payer les traites.\par
La différence entre ces deux sortes de crises, c’est que les premières, les crises-guerres, peuvent être, et sont assez souvent fortifiantes, car elles se terminent par un triomphe, immédiatement accompagné d’un élan en avant, d’un accès d’entreprise et de fièvre ; mais les secondes, les crises-chutes, n’aboutissent immédiatement qu’à une dépression de forces. A la guerre de deux cultes ou de deux systèmes de gouvernement, succède une recrudescence de foi religieuse ou d’ardeur politique ; à la lutte ardente de deux procédés \phantomsection
\label{v2p157} industriels succède la prospérité du procédé décidément vainqueur ; mais, quand la crise religieuse a pour cause le démenti des faits ou la résistance des choses, quand la crise politique a pour cause une catastrophe, un écueil historique où la barque gouvernementale est venue se heurter, quand la crise industrielle a pour cause l’impossibilité de vendre aux prix actuels et de trouver des espèces métalliques pour les payements à l’étranger ; alors, ce n’est pas la surexcitation, c’est le découragement, le scepticisme, l’atonie, qui suit la fermentation critique, et c’est là un mal incontestable.\par
Si, par-dessous la cause immédiate de ces deux sortes de crises, on descend à leur cause profonde, on trouvera qu’elle diffère essentiellement. La cause profonde des crises-guerres, c’est une invention, un perfectionnement, une innovation qui vient d’éclore et qui, pour grandir, doit refouler quelque industrie fondée sur une invention ancienne. La cause profonde des crises-chutes, c’est, non pas une éruption inventive précisément, mais bien la fièvre d’imitation aiguë qui a sévi quelque temps ou même très longtemps après l’apparition d’inventions, par exemple, en 1843 ou 1857, la crise produite par l’\emph{emballement} public pour les entreprises hâtives et précipitées des chemins de fer. Dans les deux cas, c’est bien quelque inventeur, quelque initiateur qui est l’auteur premier du mal et du bien consécutif ; mais, dans un cas, son initiative a agi comme telle, par son caractère de nouveauté, et quoiqu’elle commence seulement à être suivie ; dans le second cas, elle doit l’action qu’elle exerce à la passion entraînante, à la fureur contagieuse avec laquelle elle a été suivie et imitée, et non à sa singularité neuve et distinctive. Dans un cas, la crise s’explique par un phénomène de psychologie individuelle, l’invention, dans l’autre cas, par un phénomène de psychologie collective.\par
Si l’on avait distingué ces deux sortes de crises, l’idée n’aurait jamais pu venir de professer que les crises en général \phantomsection
\label{v2p158} sont périodiques, qu’elles sont assujetties à se reproduire tous les dix ans par exemple, ce qui est d’ailleurs manifestement faux. Que les crises-chutes présentent une certaine périodicité, passe encore ; nous en reparlerons plus loin ; mais, quant aux crises-guerres, elles ne se reproduisent pas plus à des intervalles réglés que les découvertes du génie. L’évolution d’une industrie quelconque consiste en une série d’inventions et d’innovations capitales séparées par les intervalles les plus inégaux, les plus capricieusement irréguliers, et dont chacune, à son apparition, a déterminé une fermentation belliqueuse et perturbatrice, puis rénovatrice, de cette industrie. Comptez ainsi toutes les idées de génie qui, depuis la torche de résine des temps primitifs, jusqu’au globe électrique de nos jours, ont révolutionné successivement l’industrie de l’éclairage, — ou toutes celles qui, depuis la domestication de l’âne ou du cheval jusqu’à la locomotive, ont révolutionné l’industrie des transports. Entre l’invention de la lampe grecque ou romaine à huile et l’invention de la chandelle qui est venue lui faire concurrence, une vingtaine de siècles au moins se sont écoulés, tandis que de nos jours la bougie, la lampe Carcel, la lampe modérateur, l’éclairage au gaz, au pétrole, à l’électricité, se sont succédé avec une rapidité prodigieuse, très inégale d’ailleurs. L’industrie des transports donne lieu à une observation analogue : qu’est-ce qui aurait pu faire prévoir, avant son avènement, l’apparition et le succès de la vélocipédie ? Et même, avant le perfectionnement du vélocipède primitif par la bicyclette à caoutchouc pneumatique, quel prophète eût été assez clairvoyant pour la prédire ? Il n’y a pas de loi qui règle la marche des créations de l’imagination humaine, et, par suite, il n’y a point de loi qui permette d’annoncer d’avance les conflits et les troubles suscités par ces créations.\par
Cette innovation, qui provoque une crise-guerre, n’est pas toujours une invention proprement dite ; ce peut être,  \phantomsection
\label{v2p159}ce qui revient au même, la découverte d’un nouveau débouché, ou quelque initiative d’ingénieur, telle que le percement d’un isthme ou d’une montagne, ou quelque décret d’homme d’État, tel que le traité de Napoléon III avec l’Angleterre en 1860. La levée d’une digue protectionniste a pour effet de mettre en état de guerre l’industrie nationale naguère protégée avec les industries étrangères dont elle était garantie jusque-là par cette muraille de Chine. La métallurgie française, après le traité de commerce avec la Grande-Bretagne, a essayé un moment de lutter contre les fers anglais, et plusieurs de nos provinces sont encore couvertes des ruines causées par cette grande bataille qui a été une prompte défaite. Il a suffi d’un perfectionnement de la navigation à vapeur, d’un abaissement des frais de transport maritime, pour faire batailler à distance, à la distance de milliers de lieues, à travers l’Atlantique, les agriculteurs du Far-West avec les producteurs de blé du continent ; crise agricole, la plus terrible, la plus alarmante de toutes les guerres économiques dont l’histoire fasse mention.\par
Les luttes industrielles, autrefois, étaient reléguées au dernier plan des préoccupations publiques ; les luttes militaires seules s’offraient en haut relief au regard de tous. Mais à présent, l’intérêt et l’émotion du public s’attachent aux unes autant qu’aux autres, et plus nous allons, plus s’avive l’attention passionnée que l’on prête aux conflits économiques, pendant que l’importance des batailles semble plutôt s’amoindrir. En tout cas, les guerres économiques vont se multipliant et s’étendant à la fois pendant que les guerres militaires vont se raréfiant et qu’on s’efforce de plus en plus de les localiser, si l’on n’y parvient pas toujours.\par
Nous avons dit que la guerre industrielle, ou aussi bien commerciale, résulte d’une invention, d’une initiative nouvelle ; nous devons ajouter que, au cours de la campagne engagée de la sorte, le génie inventif est stimulé à se déployer, mais, en général, d’une tout autre manière. Quelquefois, \phantomsection
\label{v2p160} l’effort de la lutte fait découvrir de nouvelles inventions, dignes de se répandre elles-mêmes pour le plus grand profit de l’humanité. Mais, le plus souvent, ce sont des ruses de guerre, des habiletés de tacticiens, d’heureux mouvements stratégiques, inventions dont toute l’utilité consiste à n’être pas imitées au dehors, qui sortent du choc des esprits mutuellement fécondés.
\subsubsection[{II.4.c. Crises-guerres de la consommation.}]{II.4.c. Crises-guerres de la consommation.}
\noindent Je viens de parler des crises qui éclosent des luttes internes de la production. Mais il y a encore, quoiqu’on n’en parle point, d’autres crises graves, perturbatrices obscures du fond des âmes, qui sont suscitées par les luttes internes de la consommation. Celles-ci se produisent chaque fois que, à la voix d’un tribun ou d’un apôtre, doué d’\emph{imagination morale}, pour employer l’expression de Ribot, une nouvelle poussée de désirs se répand dans le public et qu’un conflit s’engage entre des besoins anciens et des besoins nouveaux qui se traduisent par des formes différentes de la consommation. Pendant combien d’années le besoin nouveau de s’agenouiller dans une église, d’y prier sur le tombeau des martyrs, d’y entendre la messe ou un sermon, a-t-il lutté dans le cœur des premiers chrétiens avec le besoin ancien d’aller au cirque ou d’écouter des vers récités dans une salle publique ? Chacun de nous souffre quand, par suite d’une exigence nouvelle de son goût et de sa santé, un besoin nouveau vient rompre l’équilibre de son budget et le force à opter entre cette dépense et quelque autre de ses dépenses habituelles. Et, quand cette souffrance individuelle se généralise, quand toute une classe, celle des ouvriers, par exemple, sollicitée par des novateurs tantôt bienfaisants, tantôt funestes, éprouve le besoin d’être mieux logée, mieux nourrie, mieux vêtue, de lire des livres ou des journaux, d’aller au théâtre, et ne peut satisfaire ces nouveaux désirs  \phantomsection
\label{v2p161}qu’en renonçant à l’alcool, au café, à d’anciennes habitudes, on peut bien appeler crise, et crise douloureuse, le conflit qui met tant de cœurs aux prises avec eux-mêmes, tragiquement parfois. Ces guerres intérieures de besoins concurrents qui s’entre-disputent le budget familial ou le budget personnel ont toujours pour cause, quand elles ont lieu à la fois dans un grand nombre d’âmes, un apostolat, un prosélytisme politique ou religieux, une innovation morale suggérée par une nouvelle conception de l’univers et de la vie, par des découvertes de la science ou des visions d’illuminé. Et il n’est rien qui doive autant intéresser l’économiste que ces crises psychologiques, car de leur issue dépend, avec la transformation des usages et des mœurs, le cours que la consommation va prendre et qui imposera à la production sa forme, sa direction et son niveau. La production, en effet, se règle sur la consommation, et la consommation se transforme au gré des principes de morale ambiante où s’exprime toute modification dans la manière de concevoir la nature des choses et le but de l’existence.\par
Mais ce ne sont pas seulement des inventions morales qui suscitent ces crises-guerres de la consommation ; d’autres fois, plus souvent même, ce sont des inventions mécaniques, des inventions quelconques, qui, en abaissant le prix de certains articles réputés de luxe jusque-là, font luire pour de nouvelles couches de la population la possibilité de les acquérir et peu à peu en répandent le besoin. Ces besoins nouveaux, se glissant dans le budget du ménage, y jettent le trouble ; d’où un état de gêne et de fermentation qui se communique de proche en proche et atteint vite la proportion d’un véritable problème social. Il s’agit de savoir comment ce problème sera résolu, comment les besoins nouvellement importés, ces intrus du budget de l’ouvrier ou du paysan ou des classes lettrées, y prendront place définitive, y seront adoptés et accueillis par les besoins établis, soit moyennant un resserrement de ceux-ci, soit plutôt grâce à  \phantomsection
\label{v2p162}l’extension des revenus, à l’élévation des salaires, des honoraires, des bénéfices de tout genre.\par
La solution est différente suivant qu’il s’agit d’une crise produite par des inventions morales ou par des inventions mécaniques, par un renouvellement des principes ou par une transformation de l’outillage : dans le premier cas, la crise se termine par l’expulsion de quelque ancien besoin remplacé par un nouveau : par exemple, le luxe des grands repas de noces, des grands festins funéraires, d’une fastueuse hospitalité, propre aux primitifs, est remplacé par le confort habituel, ou la recherche des toilettes somptueuses par le goût des ameublements délicats, ou bien les sacrifices d’hécatombes aux dieux par l’érection de splendides cathédrales. Dans le second cas, il n’y a guère substitution, il y a plutôt addition de besoins, et l’\emph{étalon de vie}, le train de maison, s’élève plutôt qu’il ne change, tandis que, après les conversions religieuses ou politiques, il change plus qu’il ne s’élève.\par
Mais, dans les deux cas, on le voit, les crises-guerres de la consommation, comme celles de la production, sont filles de l’invention.
\subsubsection[{II.4.d. Crises-chutes.}]{II.4.d. Crises-chutes.}
\noindent Parlons maintenant des crises-chutes, les seules, ou à peu près, dont les économistes se préoccupent. Celles-ci, avons-nous dit, sont filles de l’imitation, ou plutôt d’une des maladies de l’imitation. Ceci cependant ne peut pas être admis sans explication ou sans réserve. Ces crises éclatent toutes les fois qu’une grande inégalité, un écart considérable, apparaît entre les produits et les besoins. Cet écart, nous le savons, peut tenir soit à l’insuffisance des produits relativement aux besoins, soit, au contraire, à leur excès. Mais les crises de surproduction, celles que notre âge, par bonheur pour lui, connaît le mieux, sont aussi les seules, ou à peu près, dont l’économie politique ait à s’occuper.  \phantomsection
\label{v2p163}Car le phénomène inverse, dont nos pères ont eu surtout à souffrir, et dont la famine, épouvante du passé, est le type, a des causes qui n’ont rien d’économique. L’excès des besoins sur les produits, dans les années de famine ou de disette, provient d’une calamité physique, d’une sécheresse prolongée, d’un tremblement de terre qui a détruit toutes les habitations d’une région, etc., catastrophes d’ordre physique, non d’ordre social, comme un naufrage ou un incendie. Il serait puéril ici d’incriminer l’imitation qui a contribué à répandre le besoin de manger du pain plutôt que des racines, ou de s’abriter dans des maisons plutôt que dans des grottes. Il n’en est pas de même si la production d’un article ou d’un groupe d’articles est devenue insuffisante parce que le besoin des consommations correspondantes s’est répandu encore plus vite, par engouement passionné, que la fabrication n’a pu s’étendre. Ici la contagion imitative est vraiment la cause du mal, mais le cas est si rare, et le mal si passager, qu’il ne vaut pas la peine de s’en occuper. Ne parlons donc que des crises de surproduction.\par
Ces crises de surproduction sont dues à des entraînements imitatifs qui sévissent, non pas parmi les consommateurs principalement, mais surtout parmi les producteurs. Voyons, par exemple, comment Émile de Laveleye rend compte de la crise anglaise de 1810 qui prit naissance en Angleterre et passa la Manche ensuite. « L’affranchissement des colonies espagnoles et portugaises, dit-il, à la suite de l’invasion de l’Espagne par les armées françaises, semblait devoir ouvrir un marché illimité au commerce anglais. Celui-ci aussitôt inonda l’Amérique du Sud de produits de tout genre. On alla jusqu’à envoyer une cargaison de patins à des pays qui ignoraient ce que c’est que la neige et la glace, et la colonie de Sydney reçut assez de sel d’Epsom pour faire purger tous les habitants pendant cinquante ans une fois par semaine. » C’était du délire. Et, entre parenthèses, \phantomsection
\label{v2p164} on voit les exceptions que comporte cette merveilleuse harmonie providentielle, chantée par Bastiat, qui proportionnerait spontanément, par le laissez-passer, et ajusterait exactement toujours, les produits aux besoins. C’est aussi un bel exemple du mutuel échauffement des cervelles commerciales. Cette folie collective des fabricants anglais, dont chacun produisait plus parce qu’il voyait les autres produire davantage, a été la vraie cause de la crise de 1810\footnote{ \noindent Remarquons, à ce propos, à quel point la solidarité économique des nations modernes s’affirme, dès le commencement de ce siècle (et même bien avant), en dépit de leur hostilité militaire et politique. En 1810, la France et l’Angleterre étaient en guerre acharnée, et il semblait que tout mal de l’une dût être un bien pour l’autre. Cependant la crise de cette année, éclatée en Angleterre, se propagea aussitôt en France.
 }, et l’entreprise napoléonienne de cette année contre l’Espagne n’en a été que l’occasion ; car, sans cet emballement des producteurs britanniques, l’invasion du territoire espagnol n’aurait jamais eu cette conséquence.\par
En 1825, éclata chez nos voisins encore, une autre crise. « Le souvenir\footnote{ \noindent \emph{Le marché monétaire.}
 } de cette grande convulsion économique, dit Laveleye, s’est conservé en Angleterre comme celui des tremblements de terre de Lisbonne au Portugal et des éruptions du Vésuve à Naples. » Les romanciers en ont tiré parti. Elle fut précédée d’engouements fiévreux et épidémiques, comparables à celui de 1810. « Les entreprises les plus inconsidérées trouvaient des actionnaires confiants. On vit s’établir ainsi une société pour percer l’isthme de Panama (déjà !) \emph{dont on ne connaissait pas encore la configuration}, une autre pour pêcher des perles sur les côtes de la Colombie, une autre pour convertir en beurre le lait des vaches des pampas de Buenos-Ayres. La confiance était sans bornes parce que tout le monde gagnait et que toutes les valeurs faisaient prime, précisément parce que la confiance montait... » Nous avons vu cela en France aux beaux jours de l’\emph{Union générale}. Quelques mois après, à la fin de 1825, « l’inquiétude et la méfiance dégénéraient en panique ; l’on  \phantomsection
\label{v2p165}se rua sur les banques, il y eut ce que les Anglais appellent énergiquement un \emph{run}, un assaut général. » En décembre de cette même année, 70 banques anglaises suspendirent leurs payements.\par
La crise de 1847, qui troubla aussi la France, a été provoquée de même par une fièvre contagieuse de l’esprit public. On s’\emph{emballa} pour les constructions de chemins de fer, et ce fut là la véritable cause des désastres financiers de cette époque, où la mauvaise récolte de 1846 ne joue qu’un rôle secondaire. Ce n’est pas l’invention des chemins de fer qu’il faut accuser de cet \emph{emballement}, car elle était déjà ancienne à cette date ; mais c’est l’état psychologique où, à l’égard de cette invention, quelques années de prospérité avaient mis le public. Les temps prospères surexcitent les visionnaires, les pseudo-inventeurs, les gens à projets fumeux, les utopistes. Tout retour de prospérité générale est lié à une recrudescence de la crédulité publique. Alors, de tous les dons psychologiques qui rendent un individu entraînant, le plus efficace est la puissance d’imagination, encore plus que l’orgueil et l’énergie du caractère. Les hommes qui émergent sont caractérisés par la disproportion énorme entre leur foi et les motifs de leur foi. L’intensité de leur vision leur tient lieu de preuve, de même que, lorsque leur vision se sera répandue dans des esprits incapables de la concevoir d’eux-mêmes, son succès lui vaudra démonstration aux yeux de ceux-ci. Les visionnaires \emph{croient} tout ce qu’ils s’imaginent, parce qu’ils l’imaginent très fort, parce qu’ils le \emph{voient.} Ce qui fait le grand musicien, c’est l’intensité, la précision, la complexité de son audition intérieure, de sa mémoire acoustique, d’où le degré exceptionnel de son imagination musicale. Ce qui fait le grand peintre, c’est la vigueur, la fidélité, la richesse de sa mémoire et de son imagination visuelles. Et, par le fait qu’on a cette puissance d’imagination optique ou auditive, on possède le talent inné de la musique ou de la peinture. Pareillement, ce qui fait le grand  \phantomsection
\label{v2p166}révolutionnaire, c’est la netteté, la force, la compréhension de sa conception idéale, d’où procède son talent d’entraînement populaire ou même d’organisation. Appliquée aux affaires, cette aptitude chimérique suscite des sociétés anonymes, pullulantes, qui aboutissent à des catastrophes. Le succès d’affaires ainsi lancées est souvent comparable à la propagation d’une secte religieuse qui conduit ses adeptes au martyr.\par
M. Clément Juglar a consacré aux crises qui nous occupent de consciencieuses et subtiles études. Écoutons-le à son tour. « Les symptômes, dit-il, qui précèdent les crises sont les signes d’une grande prospérité. Nous signalerons : les entreprises et spéculations de tout genre ; la hausse des prix de tous les produits, des terres, des maisons ; la demande des ouvriers, la hausse des salaires ; la baisse de l’intérêt ; la crédulité du public qui, à la vue d’un premier succès, ne met plus rien en doute. Le goût du jeu, en présence d’une hausse continue, s’empare des imaginations avec le désir de devenir riche en peu de temps, comme dans une loterie. » N’est-ce pas là la peinture d’une maladie morale autant qu’économique ?\par
Le même auteur remarque que les guerres, ni les révolutions, n’ont le plus souvent pour effet de provoquer une crise commerciale. « Ainsi, dit-il, au milieu des guerres de l’Empire, avec le blocus continental, la marche des affaires n’est pas interrompue ; elle se développe quand même, en France et en Angleterre, malgré la lutte de leurs armées. » En 1855, il en a été de même. — Cela n’a rien de [{\corr surprenant}] à notre point de vue psychologique. Car une guerre et une révolution s’accompagnent d’un long accès d’enthousiasme, de confiance et d’ardeur, qui retentit jusque dans le monde des affaires. Il est vrai que la dépression des forces suit souvent la fièvre.\par
L’excès des produits sur les besoins peut résulter aussi bien d’un resserrement de la consommation que d’une surproduction. \phantomsection
\label{v2p167} Mais, quand la consommation se resserre par la diminution des revenus, ce phénomène, source de tant de privations et de douleurs cuisantes, frappe moins les yeux que la surabondance des marchandises étalées. En réalité, la diminution des revenus cause peut-être autant de faillites, de désastres industriels et commerciaux, que l’accroissement excessif de fabrication. — Il doit arriver bien rarement, s’il arrive jamais, que la surproduction coïncide avec la restriction volontaire de la consommation et que les deux causes opposées des crises dont il s’agit s’ajoutent l’une à l’autre. Car les dispositions psychologiques, les « états d’âme » sont éminemment contagieux ; et, à la même époque, dans une même région, le même vent d’optimisme et d’illusion souriante qui fait exagérer l’activité des fabriques pousse à l’exagération des dépenses. Mais la crise survient précisément parce que cette confiance téméraire, soit chez les fabricants, soit chez les consommateurs, a été suivie d’un épuisement des ressources chez ceux-ci, et d’une saturation des débouchés pour ceux-là.\par
La surproduction est donc, en général, contemporaine de la \emph{sur-consommation}, et à ces deux phénomènes succèdent deux phénomènes contraires, contemporains aussi, le ralentissement de la production et celui de la consommation. Rien de plus tranché que le contraste psychologique de ces périodes successives. La première est une poussée d’espérance et de joie, de foi et de force ; la seconde est une dépression du cœur, une accumulation de privations et de gênes, de découragements et de tristes prévisions, de détresses et d’angoisses.\par
Ajoutons vite que, si la vie du consommateur se passe ainsi en dilatations et contractions alternatives de ses dépenses, la tendance à les dilater, en somme, finit par l’emporter, la joie triomphe de la tristesse, l’espérance du désespoir, la confiance de la crainte. Aussi le \emph{train de vie} va-t-il s’élevant sans cesse, peu à peu, à travers ces oscillations,  \phantomsection
\label{v2p168}dans toutes les couches de la population, même les plus inférieures. On est bien plus prompt, en effet, à dépenser davantage quand le revenu s’accroît qu’à dépenser moins quand il s’abaisse. De là ce fait qui a paru surprenant et qu’on a taxé de paradoxal, que la somme des épargnes augmente assez souvent aux époques de bas prix et de bas salaires et diminue dans les années où les revenus grandissent. Ce n’est pas dans les périodes de prospérité qu’on économise le plus. Comme la hausse des prix et des salaires et l’excès de la production sont dues à un même vent d’optimisme aventureux, qui souffle sur tous les cœurs, les dépenses vont s’exagérant et l’on économise peu, ou relativement peu. Dans les périodes de baisse et d’inquiétude, au contraire, on devient économe parce qu’on devient craintif, et, avec des revenus amoindris, on épargne davantage. « Il peut arriver, dit M. Juglar, que, dans les périodes de hausse, les bénéfices augmentent, par suite des économies qui l’emporteront sur la somme des profits de la période de baisse. »\par
\emph{Prospérité, crise :} que de joies variées, inouïes, que de désespoirs lancinants, contiennent ces simples mots, lus d’un œil si sec par un statisticien entre des colonnes de chiffres ! Si l’on ne pense qu’aux marchandises gâtées en magasin, non vendues dans des docks, c’est peu de chose qu’une crise, mais, vue du côté subjectif, quel spectacle de champ de bataille couvert de morts et de blessés ! Il faut rapprocher des statistiques commerciales ou industrielles qui nous peignent aux yeux les hausses ou les baisses des prix, de la fabrication, des salaires, les graphiques du suicide et de la criminalité, et aussi de l’aliénation mentale.\par
M. Juglar\footnote{ \noindent Voir le \emph{Journal de la Société de statistique de Paris}, juillet et septembre 1806.
 } a montré le rapport entre les crises et les dépressions de la natalité légitime, dans tous les pays. En France, par exemple, après l’augmentation du nombre des mariages qui a suivi immédiatement la guerre de 1870-71,  \phantomsection
\label{v2p169}survient une crise commerciale qui, de 1873 à 1877, s’accompagne d’un ralentissement des affaires, et, en même temps, d’une diminution numérique des mariages : de 21 000 ils descendent à 18 000. « Une nouvelle période de prospérité s’ouvre de nouveau et le chiffre des mariages se relève de 18 000 à 21 400 en 1882 : le chiffre maximum de 1872 est dépassé ! » Puis éclate la crise de 1882, et le nombre des mariages diminue. Il y a une similitude frappante, en somme, entre la courbe graphique des mariages et celle du bilan des grandes banques qui servent de thermomètre à la prospérité ou à la détresse générales. En Angleterre, en Allemagne, en Italie, en Autriche, tout comme en France, nous retrouvons cette coïncidence remarquable. La grande nuptialité et la grande natalité de l’Allemagne depuis ses victoires de 1870 s’expliquent en grande partie par sa grande prospérité industrielle et commerciale à partir de cette époque.
\subsubsection[{II.4.e. Crises monétaires et commerciales. Leurs trois phases, d’après Juglar. Les découvertes de gisements aurifères et argentifères. Concurrence de deux métaux.}]{II.4.e. Crises monétaires et commerciales. Leurs trois phases, d’après Juglar. Les découvertes de gisements aurifères et argentifères. Concurrence de deux métaux.}
\noindent Avant d’aller plus loin, récapitulons brièvement ce que nous venons de dire. Une crise, avons-nous dit, est toujours une attente trompée, compensée, mais mal compensée pour le moment où elle règne, par une attente surpassée. Cette déception générale et profonde qui caractérise les crises économiques, comme toutes les autres, est due soit à une lutte de volontés et de jugements en conflit, soit à une résistance des faits. De là la distinction des crises-luttes et des crises-chutes, qui s’applique soit aux crises de la production, soit aux crises de la consommation, soit aux crises monétaires dont nous n’avons encore rien dit. La production a ses crises-luttes, ses crises-guerres, sous la forme de concurrences aiguës entre producteurs similaires, entre corporations pareilles, entre trusts rivaux et gigantesques ; elle a ses crises-chutes, dans les cas de surproduction fiévreuse \phantomsection
\label{v2p170} qui tombe au précipice d’une consommation manifestement insuffisante, quoique celle-ci n’ait point diminué. La consommation a ses crises-luttes, toutes psychologiques, déchirements du cœur — et du budget — entre des besoins antagonistes ; et elle a ses crises-chutes quand, abusivement surexcitée, elle se heurte — ce qui est rare, sauf le cas de disette ou de famine — à une production insuffisante. J’aurais pu ajouter qu’il y a des crises-guerres causées par la lutte des producteurs et des consommateurs les uns contre les autres, quand, par des lois contre l’accaparement, par des lois de maximum, par des tarifs municipaux de la boulangerie ou de la boucherie, les consommateurs se coalisent en face des producteurs eux-mêmes coalisés. Et les agitations nées de ces luttes se distinguent profondément des crises-chutes occasionnées, comme il vient d’être dit, par un désaccord non voulu entre la consommation et la production. Nous avons montré que les crises-guerres ont pour cause première une éruption inventive plutôt qu’un engouement imitatif ; tandis que les crises-luttes s’expliquaient par un grand courant d’imitation plutôt que par un jaillissement d’inventions.\par
Mais il nous reste à dire, pour achever ce qui concerne les crises, que la monnaie, comme la production et la consommation dont elle est le trait d’union, a ses crises-guerres et ses crises-chutes aussi, inséparables à vrai dire de toutes les autres, distinctes pourtant et dignes d’un examen spécial. D’une part, les guerres des haussiers et des baissiers, à la Bourse, à coup de nouvelles fausses ou vraies, de faits révélés ou inventés de toutes pièces, sont l’équivalent, sur ce marché des valeurs, des guerres que se font, sur le marché des produits, deux industries rivales. Le plus souvent, et ce phénomène est caractéristique de notre époque, une crise de Bourse, un krack, est déterminé par la simple lutte de deux grands banquiers, de deux grands rois de la finance, de deux syndicats d’accaparement. Telle a  \phantomsection
\label{v2p171}été la crise de l’\emph{Union générale}, à la suite du combat de Bourse qui s’est livré entre elle et la banque Rotschild. Telle a été aussi la crise du 10 mai 1901 et des jours suivants à New-York. L’\emph{Evening News} nous l’explique ainsi : « La crise tout entière provient de ce qu’un duel gigantesque s’est engagé entre le groupe Gould et Rockefeller contre le groupe Vanderbilt et Morgan. Les Gould avaient résolu, coûte que coûte, d’acquérir le contrôle des chemins de fer du Pacifique Nord qui est actuellement entre les mains des Vanderbilt. Ils voulaient prendre leur revanche sur les Vanderbilt qui avaient réussi à accaparer l’Union Pacific. Les deux syndicats ennemis s’étaient mis en conséquence à acheter des actions du Pacifique Nord à n’importe quel prix. Ils avaient fini par acquérir plus que le stock disponible de titres, et, lorsqu’ils se mirent à exiger la livraison des titres, les cotes montèrent alors aux hauteurs fantastiques ou descendirent dans les bas-fonds effroyables qu’on a vus hier (le 10 mai). » Il y eut là des ruines lamentables.\par
D’autre part, quand, par suite d’un afflux subit des métaux précieux, ils se déprécient, ou quand, à l’inverse, le drainage des espèces métalliques les rend insuffisantes pour répondre aux besoins de l’échange — c’est-à-dire aux besoins abstraits d’adapter des produits à des besoins concrets, — dans ces deux cas, dans le dernier surtout, auquel est spécialement affecté le nom de \emph{crise monétaire}, on peut voir l’équivalent de ce qui a lieu, quand il y a excès ou déficit de production. — Je n’insiste pas sur le fait évident que, dans toutes ces différentes sortes de crises spéciales à la monnaie, le caractère essentiel est l’attente trompée, source féconde et abondante de douleurs.\par
Occupons-nous d’abord des crises du dernier genre, par déficit de monnaie ; elles se rattachent intimement au sujet que nous venons de traiter tout à l’heure. L’excès de la production sur la consommation, en effet, ne saurait jamais  \phantomsection
\label{v2p172}suffire à expliquer les phénomènes des crises commerciales telles que notre siècle les voit si souvent éclater. Émile de Laveleye et M. Juglar l’ont bien montré. Cet excès ne peut exister que dans un petit nombre d’industries à la fois. Or, un des traits habituels des crises, et celui qui s’accentue de plus en plus chez elles, est d’être une maladie qui s’étend à un grand nombre d’industries à la fois. « Tout pays, dit Laveleye, qui fera de grandes affaires avec peu d’argent et qui aura un vaste mouvement d’importation et d’exportation, sera exposé à ces perturbations économiques. C’est pourquoi nul n’en a plus souffert que l’Angleterre d’abord, l’Amérique ensuite. » « L’encombrement des produits soit dans les fabriques, soit dans les entrepôts, dit M. Juglar, n’est pas la seule cause des crises ; ce sont surtout les différences de prix qui, même en l’absence de tout excès (de production), peuvent produire les mêmes effets. »\par
Les économistes ne s’accordent guère sur la cause des crises commerciales. Mais « il y a trois accidents qui sont reconnus comme formant le cortège de toutes les crises : 1\textsuperscript{o} la hausse des prix qui les précède ; 2\textsuperscript{o} le drainage des espèces métalliques ; 3\textsuperscript{o} la baisse des prix qui permet et facilite la liquidation\footnote{ \noindent Clément Juglar, ouvrage cité.
 } ». Tant que la hausse des prix se prolonge, c’est la prospérité, c’est le vent de foi et de joie générale qui souffle toujours, et nous savons par quelles mutuelles stimulations d’entraînement imitatif, par quelle foule d’actions intermentales, s’entretient cette poussée unanime d’espérances destinées à être déçues, à la suite d’inventions qui les ont suscitées. Mais, un beau jour, les hauts prix cessent de pouvoir se soutenir. Pourquoi ? Parce que de hauts prix supposent abondance de monnaie, et la monnaie manque. Elle manque, pourquoi ? Parce que les espèces métalliques ont été drainées à l’étranger. Là est l’écueil à redouter. Sous l’impulsion de la confiance ambiante, on a trop acheté de matières premières à l’étranger, \phantomsection
\label{v2p173} qui a fait crédit, et, à l’échéance, il faut payer, et payer non pas avec du papier de commerce ou même des billets de banque, mais avec des espèces, seul gage international de la monnaie fiduciaire.\par
L’excès de confiance a fait à la fois l’excès de production et l’abus du crédit, qui n’est que la confiance financière. Et, avant même qu’on ait constaté l’insuffisance de la consommation aux prix élevés qui règnent, on s’est heurté à l’insuffisance de la circulation métallique. Pas de crise monétaire possible, évidemment, dans les pays où il n’y a que des affaires au comptant. C’est pour la même raison qu’il n’y a pas de crise commerciale possible dans les pays et aux époques où l’on fabrique sur commande. « La richesse des nations, dit un économiste, peut se mesurer à la violence des crises qu’elles éprouvent. » Il n’y en a pas en Russie, ni dans le midi de l’Espagne. Le remplacement de la fabrication sur commande, pour une clientèle connue et \emph{assurée}, par la fabrication à confection pour une clientèle incertaine, plus ou moins probable, de moins en moins sûre, a seul ouvert la porte aux crises de surproduction industrielle. Cela tient à ce que le rayon de la sphère des débouchés a été s’étendant sans cesse, et que, au fur et à mesure, a été diminuant sans cesse la certitude de l’écoulement total des produits, comme on est de moins en moins sûr, en empruntant, de pouvoir rendre à l’échéance à mesure qu’on est forcé d’emprunter davantage pour acheter une plus grande quantité de matières premières. Cela signifie que le progrès de l’industrie et du commerce oblige à se risquer de plus en plus, à descendre toujours plus bas l’échelle des degrés de probabilités dont on doit se contenter en affaires, jusqu’au jour où, par l’association, les syndicats, les trusts, on s’efforce de la remonter.\par
C’est l’abus du crédit, du prêt, qui explique la hausse factice des prix. « Les prix s’élèvent, dit M. Juglar, en proportion du crédit, sans augmentation des métaux précieux sous  \phantomsection
\label{v2p174}forme de monnaie. La puissance de la monnaie et l’efficacité de ses services se manifestent non pas par sa quantité mais par la facilité et la rapidité de sa circulation. » Ce n’est pas l’émission soi-disant exagérée des billets de banque qui provoque les crises, comme on l’a prétendu à tort\footnote{ \noindent Rappelons à ce propos, que, dans un livre sur le \emph{Crédit et les banques}, Coquelin a donné pour cause aux crises commerciales et financières le monopole des banques d’émission. On ne parle plus de cette application, d’ailleurs rigoureusement logique, du libéralisme économique au sujet qui nous occupe. Le malheur est pour cette idée de Coquelin que, — comme le fait remarquer Laveleye — des trois grands pays commerciaux (il dirait quatre aujourd’hui, y compris l’Allemagne) celui qui a été le plus ravagé par les crises, où elles ont été les plus violentes, est celui ou règne la liberté d’émission des banques, et où les banques sont plus nombreuses que partout ailleurs, l’Angleterre : tandis que le pays où règne le monopole le plus absolu de la banque d’État, la France, est celui où les crises sont les plus douces.
 }. « Cet excès d’émission seul n’aurait pu déterminer les crises commerciales, si le public n’avait précédé la Banque dans cette voie par le nombre de ses engagements sous forme d’effets de commerce, de lettres de change, pour les transformer par l’escompte en un instrument d’une circulation plus facile, plus étendue, à l’aide des virements, des billets de banque et des espèces même, s’il en était besoin. »\par
La hausse des prix provient donc, encore une fois, des progrès du crédit. « Cette puissance d’achat contre une simple promesse augmente la demande des produits et, par conséquent, la hausse des prix, qui, portant d’abord sur quelques produits, ne tarde pas à s’étendre et à se généraliser ; la prospérité règne dans tout le pays, tout le monde est plus riche. L’espoir de réaliser un profit par les achats à crédit, en précipitant de nouvelles couches d’acheteurs dans la même voie, accroît encore la rapidité de la hausse, d’autant plus que le crédit augmente avec l’augmentation des prix. » (Noter cette causalité réciproque.) Alors l’élévation des prix atteint un point tel que, dans le commerce en gros, « les échanges sont plus difficiles ; de là des offres en assez grand nombre pour renverser l’équilibre instable du crédit. Les marchandises ne circulent plus, le papier ne trouve  \phantomsection
\label{v2p175}plus de compensation facile ; il s’accumule dans les portefeuilles des banques, n’est pas payé à l’échéance, a recours au ré-escompte » et l’encaisse métallique des banques diminue.\par
A ces signes on reconnaît la crise imminente. Les affaires s’arrêtent, on est obligé de livrer en baisse les produits achetés en hausse. « Tout crédit, toute confiance disparaît : c’est un sauve-qui-peut général. Il ne s’agit plus d’affaires à terme, c’est du comptant que l’on réclame pour liquider. » Mais « cet état aigu ne saurait persister plus de douze à quinze jours ». Puis, « le calme se rétablit, le taux de l’escompte redescend presque aussi vite qu’il était monté, la période de liquidation s’ouvre et dure plusieurs années\footnote{ \noindent Les citations qui précèdent sont empruntées à l’ouvrage de M. Juglar, sur les \emph{Crises commerciales.}
 } ».\par
Il y a donc trois phases successives dans le phénomène que nous étudions : 1\textsuperscript{o} l’exaltation progressive et rapide de la confiance générale, de la crédulité publique devenue bientôt, par le mutuel exemple, une furie d’affaires ; 2\textsuperscript{o} la contraction brusque de la confiance, la panique ; 3\textsuperscript{o} la renaissance lente de la confiance. Et nous venons de voir que le soudain passage de la furie à la panique — comme dans une bataille — a été occasionné par la raréfaction de la monnaie métallique, parce qu’elle est, jusqu’ici (ne préjugeons point l’avenir) la seule monnaie internationale. Remarquons en passant combien cette constatation de fait, si souvent reproduite, démontre l’erreur de penser que la monnaie est une marchandise comme une autre\footnote{ \noindent Autant vaudrait dire que les mots sont des sons comme d’autres : un mot est, avant tout, un signe, ce qui n’est point un son comme tel ; s’il est significatif, il est mot par ce côté, et d’autres sensations que les sons peuvent être signes.
 }. Puisque la cause immédiate des crises est l’exportation du numéraire, c’est qu’évidemment la monnaie n’est pas une marchandise comme une autre, ou plutôt n’est pas une marchandise du tout ; car il n’est pas de marchandise dont l’exportation ait  \phantomsection
\label{v2p176}été préjudiciable au pays exportateur et y ait déterminé une crise. Aussi Michel Chevalier, qui sentait sans doute la force d’une telle objection, a-t-il toujours protesté, malgré l’évidence, contre l’explication des crises par l’émigration de l’or et de l’argent.\par
La vérité est que toutes les marchandises, avant l’invention de la monnaie proprement dite, ont eu quelque chose de monétaire, chacune d’elles étant signe, jusqu’à un certain point, de toutes les autres. Mais les métaux dits précieux ont peu à peu monopolisé, comme je l’ai dit, cette fonction monétaire et, par suite, démonétisé toutes les marchandises. Si cette fonction n’était point concentrée sur eux, elle serait, bien incommodément il est vrai, exercée par toutes les denrées ou tous les produits quelconques. Chaque article, par l’échange, servirait à s’en procurer n’importe quel autre. Il arriverait souvent que, lorsqu’on se trouverait posséder un article désiré par autrui, on le lui céderait et l’on accepterait en retour un autre article sans en avoir nul besoin actuel ou prévu mais avec la pensée de s’en servir pour l’échanger à son tour quand on aurait quelque désir à satisfaire. Chaque produit, sans la monnaie, pourrait donc tenir lieu de monnaie à la rigueur, car il serait regardé comme une possibilité de jouissances indéterminée, indéfinie, en même temps qu’il continuerait à être un moyen direct de jouissance spéciale et précise. Mais de ces deux caractères, l’un d’être indirectement utile à tout, l’autre d’être directement utile à quelque chose de particulier, le premier a été accaparé, pour l’avantage et la commodité de tous, par l’or ou l’argent, et tous les autres produits en ont été dépouillés. Cette \emph{démonétisation de tous les produits}, sauf les espèces métalliques, ne doit pas être perdue de vue si l’on veut comprendre les crises. C’est parce que l’habitude de faire fonctionner monétairement tous les autres produits s’est tout à fait perdue, qu’on voit une nation, après une période de surproduction, regorger de marchandises en magasins, en  \phantomsection
\label{v2p177}docks, toutes demandées par des besoins auxquels elles répondent, par des besoins même plus nombreux qu’elles, et, malgré cela, donner le spectacle d’une crise désastreuse pendant laquelle des maisons de concurrence tombent les unes sur les autres et des quantités de choses utiles s’avarient dans des caves, le tout à défaut d’une circulation métallique suffisante. Ces métaux, ou, dans une région plus restreinte, leurs équivalents de papier, sont devenus les seuls entremetteurs reconnus entre les possesseurs des produits et les consommateurs qui en ont besoin, et ces produits ne sauraient jamais aller d’eux-mêmes vers ces besoins ou ces besoins vers ces produits. Quoi qu’il y ait autant ou plus de besoins existants que de produits existants, c’est comme s’il y en avait beaucoup moins, et l’équation réelle entre la production et la consommation, sinon entre l’offre et la demande, n’empêche pas l’équilibre économique d’être rompu.\par
Cette importance capitale de la monnaie, ce rôle tout à fait à part qui la caractérise, se révèle clairement quand des gisements d’or ou d’argent sont découverts sur n’importe quel point du globe. L’émotion en retentit jusqu’aux antipodes. Peut-être la Révolution française de 1848 a-t-elle échauffé moins de cervelles que la découverte des mines d’or de Californie qui a eu lieu cette même année\footnote{ \noindent Il faut lire dans l’ouvrage de M. de Varigny sur l’\emph{Océan pacifique}, au chapitre sur San Francisco, l’histoire, palpitante d’intérêt — et même d’actualité, car elle rappellera à nos contemporains les mines d’or du Transvaal — de cette découverte et de ses suites.
 }. Quand la baguette de Moïse fit jaillir une source dans le désert, ce dut être une grande joie pour les Hébreux altérés, mais une joie courte, car la soif de l’eau s’étanche vite. La soif de l’or est insatiable et elle grandit à mesure qu’elle se satisfait ; aussi, quand une fontaine d’or se met à couler quelque part, on a toujours vu croître et s’étendre la frénésie des hommes avides d’y boire jusqu’à ce qu’elle fût épuisée, de plus en plus assoiffés à mesure qu’ils y boivent davantage et qu’ils  \phantomsection
\label{v2p178}sont plus nombreux à y accourir, cherchant vainement à s’y désaltérer. Après les grandes espérances religieuses d’outre-tombe, après la foi vive au paradis de Mahomet ou au ciel chrétien, il n’est rien de tel pour surexciter les imaginations, pour enflammer les âmes, pour les lancer dans l’inconnu des aventures légendaires au risque du martyre, que ces perspectives soudaines, éblouissantes, d’enrichissement. Rien de pareil n’éclate quand toute autre découverte ou invention se produit, même propre à révolutionner l’industrie et à enrichir beaucoup de gens. Jamais l’invention d’une machine, alors même qu’on prévoit qu’elle finira par susciter des fortunes colossales, bien plus colossales que n’en a jamais fait surgir une mine californienne ou transvaalienne, jamais l’invention même de la machine à vapeur ou du transport de la force par l’électricité, ou du télégraphe, ou du téléphone, n’a frappé aussi fort les esprits, n’a ébranlé aussi dangereusement les cerveaux les plus solides que la nouvelle d’un gisement aurifère, ou même argentifère, découvert au fond d’une solitude américaine ou africaine. Psychologiquement, l’effet produit est beaucoup plus intense et de toute autre nature. La fièvre de production que détermine une machine nouvellement inventée n’a rien de commun avec le délire spécial, délire héroïque ou criminel des grandeurs, suivant les tempéraments, qui s’empare des aventuriers du monde entier sitôt que la pioche d’un mineur a heurté contre une pépite. Ceux d’entre eux qui émergent de la foule, qui subissent au plus haut degré l’action de cette fièvre, se reconnaissent à des traits communs et fortement marqués. Il y a chez tous du sang des Argonautes à la recherche de la toison d’or, du sang des Pizarre et des Cortez, un héroïsme de rapacité, un lyrisme de cupidité mystique, d’intrépidité féroce et farouche, qui fait mépriser les risques mortels, en vue de perspectives infinies. Conquistadores du {\scshape xvi}\textsuperscript{e} siècle, émigrants européens de 1850 en Californie, de nos jours au Transvaal, se ressemblent par  \phantomsection
\label{v2p179}là\footnote{ \noindent Qu’on lise, dans le livre de M. de Varigny, les aventures du comte de Raousset-Boulbon (p. 318-326). C’est, sur une moindre échelle, l’audace inouïe d’un Cortez.
 }. C’est qu’il s’agit de conquérir un talisman merveilleux, grâce auquel on passera brusquement de la pauvreté à l’opulence. Un pauvre enrichi du soir au lendemain, c’est comme un pâtre devenu roi. L’or n’est pas seulement la clé de tous les plaisirs, il est un grand pouvoir sur les hommes, le plus sûr, le plus incontesté, le plus international de tous les pouvoirs ; il confère, accumulé en monceaux, une véritable royauté. Et cette puissance, comme toute autre, grise, et sa seule espérance est enivrante, et, comme l’ambition de régner, démoralisante. Pour l’or, comme pour la couronne, tout est permis. Devant une telle vision, tous les préjugés anciens sont oubliés, toutes les racines sont rompues avec le passé, avec la terre natale, parfois avec l’honnêteté native aussi. L’avidité sublimée s’exalte en ambition, et des rêves de conquête territoriale, d’\emph{impérialisme}, éclosent tout naturellement de cette exaltation. L’égoïsme incandescent devient despotisme.\par
« A la fin de l’année 1848, dit M. de Varigny, 6 000 mineurs (en Californie) fouillaient avec acharnement les cours d’eau, les rivières, les sables, trouvant de l’or toujours et partout. La fièvre gagnait les États de l’Est ; les récits les plus étranges, les nouvelles les plus fabuleuses enflammaient les imaginations ; d’interminables caravanes d’émigrants quittaient le Missouri pour envahir la terre promise... On partait, sans hésitation, droit vers l’Ouest, s’orientant sans boussole, abandonnant sans regrets champs et vieux parents, femmes et enfants en larmes, oubliant tout dans le prestigieux mirage d’une fortune qui dépassait tous les rêves. Combien de ces hardis émigrants sont morts de faim dans la rude traversée des Montagnes Rocheuses ! Combien ont succombé à la soif dans l’atroce désert du Colorado !... » Bientôt les États de l’ancien monde envoyèrent leurs cohues  \phantomsection
\label{v2p180}d’émigrants. San-Francisco offrait le spectacle de la Tour de Babel ; mais, au milieu de ce polyglottisme, régnait une étrange unité, celle d’une passion identique, d’un but commun vers lequel tous les cœurs étaient tendus. Si contagieuse était cette folie que ceux qui prêchaient ou écrivaient contre elle en étaient saisis. Les journalistes font des articles véhéments contre l’\emph{auri sacra fames ;} puis, un beau jour, ils jettent leur plume et vont là-bas, eux aussi, « le pic d’une main, la carabine de l’autre », faire cette pêche miraculeuse à l’or, cette industrie guerrière, mi-exploration, mi-conquête, irrésistible attrait. « Les ministres de l’Évangile, nouveaux Cassandres, font retentir les églises de leurs anathèmes contre la soif de l’or ; puis ils s’embarquent — comme missionnaires — pour la Californie. »\par
Et quelle émigration ! En dix-huit mois, la population de la Californie s’élève de 1 500 à plus de 100 000 âmes. Population extraordinaire, entièrement composée d’hommes, et d’hommes très jeunes. Pas de police, pas de lois. Les rixes et les meurtres se multiplient, jusqu’au jour où, spontanément, les honnêtes gens se coalisent pour établir un minimum d’ordre sous la menace du lynchage. On devrait être surpris de voir ainsi, peu à peu, de ce chaos monstrueux et tumultueux, sortir une grande et florissante cité, si l’on ne se rappelait que toutes ces âmes bouillonnantes étaient animées d’une même passion, et que là où existe l’unanimité sur un point important, les éléments les plus disparates d’ailleurs ne tardent pas à s’organiser.\par
Avec moins d’intensité, les découvertes de mines d’argent provoquent les mêmes phénomènes. En Californie encore, de 1861 à 1878, on en a découvert trois filons extraordinairement riches.\par
— Au point de vue de notre classification particulière des crises, comment qualifierons-nous les crises déterminées par les découvertes de gisements aurifères et argentifères ? Dirons-nous que ce sont des crises-guerres ou des crises- \phantomsection
\label{v2p181}chutes ? Nous dirons d’abord que ce n’est point du tout une crise, mais, au contraire, une ère de prospérité, qui est ouverte par des afflux de métal neuf. En haussant les salaires et les prix, elles sont un tonique du travail. Elles sont à l’industrie ce que l’enrichissement brusque du dictionnaire d’une langue par l’exhumation de vieux mots ou la mise en circulation de nouvelles formes est à la littérature. Cet afflux d’inventions verbales a une action plus large et plus vive, sinon aussi profonde, qu’une poussée d’idées nouvelles, d’imagination dramatique ou romancière, pour féconder l’ensemble du terrain littéraire. Et, de même, la découverte des mines d’or du Transvaal a stimulé récemment la production industrielle d’une tout autre manière que l’invention des automobiles.\par
Seulement, par réaction et aussi par le trouble qu’elles apportent dans la valeur relative des deux métaux précieux, ou dans l’estimation des fortunes anciennes, les découvertes métalliques ne tardent pas à faire naître un état critique ou de graves embarras. On peut considérer chaque découverte d’un métal comme faite au profit de l’autre et propre à favoriser cet autre dans la guerre qu’ils se font continuellement et qui a nom bi-métallisme. Au moment où l’or californien débordait en Europe, les économistes considéraient si bien la découverte des placers d’outre-atlantique comme une défaite de l’or qu’ils prédisaient comme imminent et certain le triomphe de l’argent. Le monométallisme-argent leur paraissait s’imposer absolument. Peu de temps après les découvertes de mines d’argent donnaient un démenti complet à ces prophéties, si bien qu’à présent, en sens inverse, les prophètes annoncent la démonétisation complète du métal blanc. Cette concurrence des deux métaux, on le voit, se comporte comme celle de deux procédés rivaux dans une industrie quelconque, de deux procédés d’éclairage, par exemple, ou de chauffage ; et cette concurrence, comme toute autre, paraît devoir aboutir fatalement à un monopole,  \phantomsection
\label{v2p182}ce qui ne veut pas dire qu’on ne doive pas faire tout ce qu’on peut pour l’empêcher ou pour le retarder, quand il présente plus d’inconvénients que d’avantages pour la nation dont on a les intérêts à cœur.\par
Ainsi, dans les rapports des deux métaux précieux, la découverte d’une mine de l’un deux s’accompagne, peu de temps après, d’une crise-lutte qui peut avoir des suites désastreuses. Quant aux rapports de l’ensemble de la circulation monétaire, les deux métaux réunis, avec l’ensemble de la production et de la consommation, un gisement découvert d’or ou d’argent est loin, avons-nous dit, de les troubler ; il ne fait, tout d’abord, que les surexciter. Les 4 milliards de francs d’or qui, de 1848 à 1856, ont été extraits des mines californiennes, et qui sont venus inonder l’Europe, ont été la pluie fécondante d’une moisson de prospérités. Mais la crise de 1857 est sortie de là. Elle est née d’une rencontre de cette découverte de mines avec l’extraordinaire engouement suscité à la même époque par l’extension du réseau des chemins de fer. Cette découverte vint à point pour renforcer cet engouement et le pousser à outrance. Aussi est-ce surtout en Amérique, où elle eut lieu, que se déployèrent d’abord les conséquences, momentanément fâcheuses, bientôt merveilleuses, de cette union féconde. En 1856, les États-Unis, dit Laveleye « avaient déjà construit 24 000 milles de chemins de fer et 50 000 de télégraphes, trois fois autant que l’Angleterre et six fois autant que la France. » Notons que, sans la guerre de Crimée vraisemblablement, la France et l’Angleterre auraient suivi de plus près ou de moins loin la grande République de l’Amérique du Nord dans cette voie de progrès. — Après avoir fait explosion en Amérique, la crise de 1857 passa l’Atlantique « comme un cyclone » et ravagea successivement l’Angleterre, la France, l’Allemagne. Ce fut la première à s’étendre dans le monde entier. Le contre-coup s’en fit ressentir à Java et au Brésil.\par
 \phantomsection
\label{v2p183}C’est que la \emph{surproduction} de l’or ou de l’argent a pour effet leur \emph{surconsommation} plus rapide encore, en ce sens qu’elle multiplie les échanges, seule manière de consommer la monnaie. Consommation singulière, du reste, qui est leur conservation indéfinie et non leur destruction, sauf leur très lente usure. En cela les métaux précieux différent de tous autres produits. Pour reprendre ma comparaison de tout à l’heure, un apport de mots nouveaux ou une exhumation de mots anciens est un gain durable de la langue, car les mots ne s’usent qu’à la longue, par le frottement, comme les monnaies, et sont destinés par nature à servir toujours, tandis que des idées dramatiques ou littéraires sont assez promptement détruites par l’usage qu’on en fait et le succès qu’elles ont. Quoi qu’il en soit, les pièces neuves de monnaie ont beau s’entasser sur les anciennes et grossir leur masse, ce grossissement ne laisse pas d’être insuffisant bientôt pour répondre aux besoins d’une activité commerciale encore plus vite accrue ; et il équivaut tout à coup, momentanément, à une véritable raréfaction de l’or. Car tout est relatif. Ainsi se produisent les \emph{crises-chutes} commerciales qui ont la cause monétaire que nous indiquons.
\subsubsection[{II.4.f. Où tendent les crises.}]{II.4.f. Où tendent les crises.}
\noindent On voit, par la longue énumération des diverses sortes de crises que nous venons de parcourir, à combien d’accidents variés est sujet le cours de la vie économique, et à quel point serait chimérique la prétention d’assurer tout le monde contre tous les risques industriels ou commerciaux. Imagine-t-on des sociétés d’assurances contre l’insolvabilité des banquiers ou contre l’insolvabilité de leurs débiteurs ? Il faudrait, pour cela, commencer par assurer les industriels contre le succès de leurs concurrents, autant dire contre l’éventualité d’inventions nouvelles préjudiciables aux anciennes.\par
 \phantomsection
\label{v2p184}On doit cependant se préoccuper de chercher des remèdes aux crises, et j’ajouterai même aux concurrences qui sont des crises lentes et chroniques, salutaires souvent. Sans insister sur ce sujet, qui se rapporte à la troisième partie de ce cours, indiquons déjà en quelques mots, de quelle manière on remédie à ces perturbations. Cette manière diffère suivant qu’il est question d’une crise-lutte ou d’une crise-chute. Le seul procédé pratique, découvert un jour par hasard\footnote{ \noindent D’après M. Juglar, c’est vers le milieu de ce siècle que cette découverte fortuite a eu lieu à Londres. Une crise commençait quand la Banque de Londres, par hasard, se trouva élever son escompte, et l’on s’aperçut que, à cette nouvelle, un vaisseau chargé d’or à destination du continent fit décharger sa marchandise métallique et renonça à ce transport, devenu sans intérêt. On comprit la signification de ce fait.
 }, contre les crises-chutes, contre les crises commerciales, compliquées de crises monétaires, est l’élévation du taux de l’escompte, qui, employée à temps, les enraye ou les arrête. Par ce frein du crédit, on empêche la surproduction de se précipiter dans l’effondrement des prix et des salaires, et le crédit de s’y engouffrer.\par
Quant aux crises-guerres, on les prévient ou on les termine soit en interposant entre les combattants industriels, s’ils sont de nationalité différente, la barrière infranchissable de droits protecteurs qui sont des cuirasses merveilleuses, soit, s’ils sont compatriotes, en supprimant au contraire l’intervalle qui les sépare et les associant ensemble. Le protectionnisme dans un cas, l’association dans l’autre, aboutissent par deux voies opposées au même but, la fin de la guerre. Souvent même ces deux moyens de pacification se présentent comme étroitement liés, en ce sens que le premier favorise le second et lui permet de se réaliser. C’est du jour où une grande nation, telle que les États-Unis, a hérissé de murailles protectrices son marché contre la concurrence extérieure, que les syndicats et les trusts deviennent possibles entre les concurrents nationaux, de rivaux devenus solidaires et coalisés, ce qui permet aux producteurs des articles protégés \phantomsection
\label{v2p185} de bénéficier de la totalité des droits au préjudice fréquent des consommateurs.\par
En fait, c’est habituellement pour mettre fin à des conflits et à des crises, que les syndicats, les fédérations, les associations sous mille formes, se nouent. Telle a été, par exemple, en 1885, la fédération des brodeurs suisses\footnote{ \noindent Voir Génart, \emph{Les syndicats ouvriers.}
 }. Son but, d’après ses statuts, est de « porter remède à la surproduction, d’obtenir des prix rémunérateurs, de relever l’industrie de la broderie ». D’ailleurs, si elle a mis fin à l’âpre concurrence des brodeurs fédérés, ç’a été pour la remplacer par une lutte plus âpre encore entre les brodeurs fédérés d’une part, et, d’autre part, les brodeurs non fédérés. Quand on entre dans cette voie, on est forcé d’aller jusqu’au bout, à l’association de \emph{tous} les rivaux.\par
Les chemins de fer illustrent brillamment cette vérité, que la lutte économique est un simple moyen d’arriver à l’accord final, qu’elle y tend toujours et y aboutit le plus souvent, qu’elle n’a d’autre utilité incontestable que ce terme dernier, et qu’il vaudrait mieux, dans tous les cas où la chose est possible, que l’unité de direction, au lieu d’être laborieusement et douloureusement produite par une longue guerre, eût été constituée dès le début. Partout où les voies ferrées sont nées sous un régime de libre concurrence, aux États-Unis par exemple, cette rivalité désastreuse n’a pas tardé à se terminer par une fusion des compagnies rivales ou l’écrasement de la plupart d’entre elles par l’une d’elles triomphante ; et ce triomphe ou cette fédération, obtenus par un prodigieux gaspillage de capitaux, ont donné finalement, pour l’intérêt général, des résultats bien inférieurs à ceux des chemins de fer construits par l’État ou par des compagnies concessionnaires de l’État pour l’exploitation de quasi monopoles régionaux. Le réseau des chemins de fer exécuté sur un plan d’ensemble a fini par coûter beaucoup moins et par valoir beaucoup mieux, à tous égards. « Souvent \phantomsection
\label{v2p186} (dit Génart, ouvrage cité), les lignes établies en concurrence commencent par se faire une guerre acharnée, et le public en bénéficie un instant, mais la baisse (des tarifs) n’est jamais durable, et la lutte se termine toujours par un accord. La constatation en a été faite partout : en Angleterre, en Amérique surtout : ici l’alternance de la concurrence et du syndicat a toujours amené de violentes perturbations : sous cette influence le prix du transport par tonne a varié, entre New-York et Chicago, \emph{de 5 à 57 dollars}, — entre New-York et Saint-Louis, \emph{de 7 à 46 dollars}, — et le Erié Railway a réclamé \emph{tantôt 2, tantôt 37 dollars.} »\par
Un autre remède à la guerre industrielle, remède souvent pire que le mal, est l’\emph{accaparement}, qui n’est pas toujours une variété de la coalition, puisqu’il peut être le fait d’un seul. On avait espéré à tort que, par l’extension graduelle des marchés, l’accaparement deviendrait à la longue impossible. On n’avait pas réfléchi que les causes mêmes qui font grandir les marchés, à savoir la facilité des communications, l’étendue et la rapidité des informations, donnent de plus grandes facilités et de plus grands moyens d’actions aux accapareurs. A l’époque où les marchés étaient infiniment nombreux, extrêmement étroits et très clos, on pouvait sans beaucoup de peine, dans l’étendue restreinte de chacun d’eux, s’emparer de tous les produits d’un certain genre, le blé, le vin, la laine. Mais, en revanche, on n’eût pu alors concevoir même la pensée d’étendre à toute une province, à tout un État, \emph{a fortiori} au monde entier, cette mainmise générale. A présent, au contraire, l’accaparement dans une petite région, dans un des marchés fermés et exigus d’autrefois, dans une petite ville et sa banlieue, n’est plus possible, parce que de toutes parts, si on l’essayait, afflueraient les produits rivaux ; on ne peut accaparer qu’à la condition d’accaparer en grand, et il devient de moins en moins difficile d’accaparer en grand et en très grand\footnote{ \noindent D’après des économistes très compétents, la \emph{Société des métaux}, qui finit si tragiquement en 1882 par le suicide de Denfert-Rochereau, aurait réussi probablement, si ce syndicat n’avait pas voulu porter trop haut le prix des cuivres accaparés par lui. Mais, comme le dit très bien CI. Jannet, « la sagesse, chez des pareils spéculateurs, est la chose qu’on peut humainement le moins attendre de leur part. Au vertige des millions s’ajoutent les entraînements de la vie privée surexcitée par ces succès d’argent, et les enivrements d’orgueil entretenus par les parasites et les flatteurs... élément psychologique qu’il ne faut jamais perdre de vue dans les affaires... »
 }\par
 \phantomsection
\label{v2p187}Certes, l’accaparement \emph{mondial} n’est pas chose encore très pratique, quoique nous approchions de cet étrange idéal, mais, dans les limites d’un État, d’un immense État même, rien de plus aisé aujourd’hui, dans certain cas, que d’être accapareur, si cet État, par des tarifs de douane, met obstacle à l’importation des produits du dehors offerts à des prix moins élevés. Dans l’étendue de l’État protégé de la sorte par des droits douaniers, il devient de plus en plus malaisé, à mesure que l’accaparement devient de plus en plus vaste, de faire surgir une maison rivale ; celle-ci aurait besoin, pour lutter efficacement, d’être organisée sur un aussi grand pied et de disposer de capitaux aussi considérables. Peu prudents seraient les capitalistes qui se risqueraient dans de telles entreprises.\par
— Il nous reste, en finissant, à indiquer, la question capitale qui s’élève relativement aux crises : celle de savoir si, par l’élargissement progressif du marché, elles vont et elles iront s’atténuant de plus en plus tout en s’agrandissant\footnote{ \noindent Aux {\scshape xiii}\textsuperscript{e} et {\scshape xiv}\textsuperscript{e} siècles, si l’on en juge par le produit (en impôts royaux) des \emph{foires de Champagne}, qui a été enregistré de 1275 à 1320, « il se produisait en hausse et en baisse des variations beaucoup plus considérables que celles qu’[{\corr accuse}] en 45 années du {\scshape xix}\textsuperscript{e} siècle le produit total des douanes françaises, malgré nos révolutions, nos guerres et nos changements de tarif. » Il devait donc y avoir alors des crises aiguës, encore plus douloureuses que les nôtres. Le produit en question a varié de 1300 (en 1275) à 790 (en 1288) ; et il a baissé de 1375 (en 1296) à 760, à 300, à 250 (en 1298, 1310 et 1320). Seulement, ces crises de jadis n’atteignent qu’un cercle restreint de gens, les négociants en gros, fort rares alors. L’immense majorité des industriels, travaillant sur commande, ne risquaient jamais de se trouver pris dans l’engrenage d’une crise.
 }, ou si, au contraire, comme le prédit Karl Marx, contredit en cela, il est vrai, par quelques-uns de ses disciples  \phantomsection
\label{v2p188}même, notamment par Bernstein, elles deviendront à la fois de plus en plus vastes et de plus en plus aiguës. Si insoluble en toute rigueur que puisse paraître ce problème, il y a des raisons d’incliner vers la solution optimiste, et de penser que, au fur et à mesure de l’agrandissement des marchés, les crises iront s’atténuant par leur amplification même.
 \phantomsection
\label{v2p189}\subsection[{II.5. Les rythmes (opposition de termes successifs).}]{II.5. Les rythmes \emph{(opposition de termes successifs).}}\phantomsection
\label{l2ch5}
\subsubsection[{II.5.a. Critique de la loi de Spencer sur le rythme du mouvement. Idée de M. Poincaré. Rythmes oscillations (oppos.) et rythmes circulations (répétit.).}]{II.5.a. Critique de la loi de Spencer sur le rythme du mouvement. Idée de M. Poincaré. Rythmes oscillations (oppos.) et rythmes circulations (répétit.).}
\noindent Nous venons de traiter des oppositions économiques qui se produisent sous forme de lutte entre des termes contraires et simultanés. Parlons maintenant — [{\corr ou}] plutôt parlons plus à loisir, car nous en avons déjà dit un mot à propos des crises — du cas fréquent où les termes contraires sont successifs et se produisent alternativement : tels que la hausse et la baisse des valeurs de Bourse.\par
On se rappelle la loi de Spencer : « tout mouvement est rythmique », tout mouvement, ou pour mieux dire tout changement. Ce n’est pas le lieu de discuter cette formule, entendue par son auteur dans un sens tellement large qu’il perd toute signification et que les rythmes les plus réguliers, vraiment dignes de ce nom, y sont confondus avec les alternances les plus irrégulières et les plus vagues. Pour expliquer l’irrégularité et l’imprécision de ces dernières, Spencer croit qu’il suffit d’avoir égard à la complexité des phénomènes alternants, des phénomènes sociaux notamment. Il est bien naturel, d’après lui, que des phénomènes se répètent inexactement, soit dans le même sens, soit en sens inverse, quand ils sont compliqués. Qui dit régulier dit simple ; complication et irrégularité sont synonymes. — Rien de plus vraisemblable, à première vue, que cette explication, et rien de plus faux. Il arrive fort souvent que, dans un ordre de phénomènes, un degré nouveau de complexité ramène la régularité perdue. Chaque feuille d’une forêt  \phantomsection
\label{v2p190}tremble capricieusement et le murmure lointain d’une forêt est un son musical. Il n’est pas dans la nature de répétition plus régulière, plus précise, en somme, que la reproduction héréditaire des caractères d’une espèce dans la série des générations successives, et y a-t-il rien de plus prodigieusement compliqué que les phénomènes de la vie ?\par
Les mathématiciens expliquent par la \emph{loi des grands nombres} cette régularité relative qui sort souvent de la complication même des phénomènes. A l’appui des considérations qui précèdent, je pourrais invoquer la haute autorité de M. Poincaré qui, au Congrès international de physique en 1900, émettait cette conjecture que la simplicité de la loi de Newton elle-même était peut-être apparente et illusoire. « Qui sait, disait-il, si cette loi n’est pas due à quelque mécanisme compliqué, au choc de quelque matière subtile animée de mouvements irréguliers, et si elle n’est devenue simple que par le jeu des moyennes et des grands nombres ? Sans doute, si nos moyens d’investigation devenaient de plus en plus pénétrants, \emph{nous découvririons le simple sous le complexe, puis le complexe sous le simple, puis de nouveau le simple sous le complexe, et ainsi de suite, sans que nous puissions prévoir quel sera le dernier terme}... » Cette alternance supposée de simplicité et de complexité, de régularité et d’irrégularité, dans les couches superposées des phénomènes depuis l’infinitésimal jusqu’à l’infini, est une vue des plus profondes, à laquelle on est conduit, non seulement par la considération des faits physiques, mais par celle des faits sociaux. Elle constitue elle-même un rythme, un rythme vague, mais non moins important pour cela.\par
— Avant d’aller plus loin, prévenons une confusion d’idées par une distinction nécessaire. Tous les rythmes réguliers ne sont point des suites alternantes de termes \emph{opposés ;} le plus souvent, la régularité rythmique est le privilège des séries de termes \emph{semblables} qui se répètent. Et certainement l’idée de rythme éveille l’idée d’harmonie plutôt que celle  \phantomsection
\label{v2p191}de contrariété. — Les adaptations rythmiques, c’est-à-dire régulièrement répétées, ne doivent pas être confondues avec les oppositions rythmiques. Les pulsations du cœur, les actes respiratoires, les opérations fonctionnelles d’une glande quelconque, de même que les actions d’un ouvrier qui travaille dans une usine, sont des répétitions régulières et périodiques. Mais cette périodicité n’implique nulle opposition ; il y a recommencement intermittent sans qu’il y ait jamais inversion ni destruction de ce qui a d’abord été produit. Cette périodicité implique adaptation, harmonie, accord. La répétition des ellipses planétaires, la répétition des fonctions vitales, la répétition des actes laborieux, sont des moyens adaptés à l’accomplissement de l’œuvre, connue ou inconnue, que poursuivent l’évolution astronomique, l’évolution vivante, l’évolution économique. Par ces périodes réglées, chaque être s’achemine harmonieusement à sa fin. Mais, quand une comète perturbatrice traverse et retraverse un système solaire, quand un accès de fièvre se produit et se reproduit au cours d’une maladie mortelle, quand des troubles sanglants du travail et du crédit, grèves, crises, kraks, se répètent régulièrement à certaines époques, verrons-nous là aussi des voies normales, quoique zigzagantes, du progrès astronomique, vivant ou social ?\par
La distinction du normal et de l’anormal, du sain et du morbide, est difficile, je le sais, mais elle s’impose ici. Il ne nous est pas permis de l’éluder. Remarquons que, si l’opposition de termes \emph{simultanés}, sur un champ de bataille où l’on s’entre-tue, dans une Bourse où les uns ruinent les autres, est un spectacle douloureux et anxieux, le problème soulevé par l’opposition de termes \emph{successifs}, dont l’un défait ce que l’autre vient de faire, est encore plus insoluble, à moins qu’on n’aperçoive une marche en avant poursuivie moyennant ces va-et-vient. Hors de là, le déploiement alternatif et indéfini du \emph{pour} et du \emph{contre} ne saurait être regardé que comme l’organisation de l’absurdité. Cette alternance,  \phantomsection
\label{v2p192}certes, n’est pas de l’incohérence lorsque les actes opposés et alternatifs tendent à un même but, quand, par exemple, les mouvements alternatifs et inverses du piston\footnote{ \noindent J’aurais pu choisir bien d’autres exemples. Le rythme de l’élévation aérienne et de la chute alternative des eaux est nécessaire à l’irrigation des continents et à leur nivellement. C’est une fonction planétaire, comme les pulsations du cœur ou les mouvements d’inspiration ou d’expiration des poumons sont des fonctions organiques.
 } font marcher la locomotive dans le même sens, ou, aussi bien, quand les trajets alternatifs de la locomotive elle-même de la gare B à la gare C et de la gare C à la gare B, concourent également à accélérer le mouvement commercial dans une même direction ascendante, ou l’évolution sociale vers un même terme idéal ; — ajouterai-je : ou aussi bien quand les périodes alternatives et inverses de l’évolution et de la dissolution sociales semblent concourir pareillement aux fins de l’évolution universelle ? Non, rien n’autorise ces conjectures. — Mais, quand les deux changements inverses poursuivent des fins opposées, quand, par exemple, la santé tend à la conservation et à la reproduction de la vie, tandis que la maladie tend à la mort ou à la stérilité, il est clair que leur alternance n’est utile à rien de commun aux deux. Il vaudrait mieux, pour la fin de la maladie, que la santé ne réapparût pas, et, pour la fin que poursuit la santé, que la maladie ne fît pas de nouveau invasion dans son domaine.\par
On voit, par ce qui précède, d’une part la différence qui existe entre la répétition de phases enchaînées et recommencées normalement et la répétition de phases alternantes et inverses, — d’autre part, le passage fréquent de l’une à l’autre, c’est-à-dire de l’opposition à l’adaptation. Les adaptations rythmiques peuvent être symbolisées par une roue qui tourne : chaque rotation a lieu dans le même sens que la précédente, du 1\textsuperscript{er} au 360\textsuperscript{e} degré de la circonférence ; c’est un même mouvement qui se continue en se répétant. Les oppositions rythmiques seraient plutôt symbolisées par un  \phantomsection
\label{v2p193}pendule qui oscille : à chaque chute du pendule, se détruit l’effet de son ascension antérieure ; et c’est en une suite de productions et de destructions de cet effet que consiste son oscillation. Toute oscillation est ainsi la rupture d’un équilibre, un trouble qui cherche à s’apaiser et n’y parvient pas, tout comme les accès périodiques d’une fièvre intermittente et incurable, ou les retours réguliers des crises financières.\par
C’est une erreur de regarder le phénomènes constitutifs de la vie saine, l’oxygénation et la désoxygénation des globules du sang, par exemple, comme comparables à l’oscillation d’un pendule, à moins que ce ne soit à l’oscillation du pendule d’une horloge, grâce à laquelle le mouvement des aiguilles \emph{tourne} toujours. Cette oscillation qui se convertit en rotation pourrait être choisie comme symbole de la conversion universelle des contrastes en accords, des conflits en harmonies. La désoxygénation du globule, c’est la dépense et l’emploi, par ce globule, de l’oxygène qu’il a acquis au contact de l’air ; elle n’est donc pas l’inverse mais la suite normale de l’oxygénation. Peut-être est-elle l’inverse chimique mais non l’inverse physiologique. De même, l’expiration n’est le contraire de l’inspiration qu’au point de vue mécanique ; nullement au point de vue vital. La vie est une rotation, une gravitation elliptique si l’on veut ; elle n’est pas une oscillation. Il en est de même de l’activité sociale. Dans une administration, on a la sensation de tourner une manivelle, non de faire aller et venir une balançoire. — Nous venons de voir l’\emph{oscillation} de l’inspiration et de l’expiration servir de condition à la \emph{circulation} du sang et au fonctionnement \emph{rotatoire} de tous les appareils vitaux. De même, on peut se demander si l’oscillation de la hausse et de la baisse à la Bourse, des périodes d’inflation et de dépression industrielles, n’est pas la condition de la \emph{rotation} économique et fondamentale, c’est-à-dire de l’enchaînement alternatif de la production et de la consommation, circulation toujours grandissante. La consommation a beau être une destruction de  \phantomsection
\label{v2p194}richesses, elle ne s’oppose point, socialement, à la production dont elle est l’emploi et la conversion. Elle ne s’y oppose que physiquement. Au point de vue social, la consommation est la \emph{suite} et non l’inverse de la production des richesses qu’elle tend à reproduire.\par
On peut se demander même si l’oscillation alternative du plaisir et de la peine, de l’espérance et de la crainte, que nous retrouvons sous tant d’oppositions économiques, n’est pas une condition du développement de l’esprit ?\par
Les répétitions des harmonies — par exemple des fonctions vitales ou sociales — deviennent de plus en plus régulières, rythmiques ; il est à remarquer, au contraire, que les répétitions des inversions, des contrastes, des désaccords, deviennent de plus en plus irrégulières, à moins que ces oppositions ne soient, comme nous venons d’en montrer plusieurs exemples, des procédés normaux d’harmonisation. Une série de troubles, de troubles organiques ou de troubles sociaux, présente de plus en plus d’irrégularité à mesure qu’ils s’apaisent ou que la mort s’ensuit ; une série d’actes de travail devient de plus en plus réglée à mesure qu’elle s’enracine en habitude féconde et se propage en coutume vivace.\par
En tant que les accès de fièvre sont périodiques et réguliers, ne peuvent-ils pas être considérés comme un \emph{travail} de la vie qui tend à s’en délivrer, à rentrer dans l’état normal, par une reprise intermittente de son effort salutaire ? Et de même, en tant que les crises économiques sont périodiques, ne peut-on pas les considérer comme un travail de la logique sociale en vue d’expulser les causes de ces perturbations et d’aboutir à une réorganisation de la production industrielle ? — Je suis frappé, à ce propos, de cette remarque incidente de M. Juglar : « plus on remonte haut dans le passé, plus on constate l’irrégularité de la durée des périodes... » D’après lui, la périodicité des crises serait devenue de plus en plus marquée depuis 1690, et cela tiendrait \phantomsection
\label{v2p195} à la création des banques d’escompte. Si ce phénomène continuait à se vérifier, ne devrait-on pas y voir une raison de soupçonner que les crises jouent un rôle de plus en plus utile et nous acheminent vers un terme qui serait très proche ?
\subsubsection[{II.5.b. Périodique alternance des époques de prospérité et de crise depuis le xixe siècle.}]{II.5.b. Périodique alternance des époques de prospérité et de crise depuis le xix\textsuperscript{e} siècle.}
\noindent Quoi qu’il en soit de cette explication, le problème que soulève la périodicité des crises au cours du {\scshape xix}\textsuperscript{e} siècle mérite d’être agité, et d’arrêter notre attention. — Mais d’abord, établissons nettement les phénomènes qu’il s’agit d’interpréter. Il est clair que, aussi longtemps que la production des richesses reste exclusivement ou principalement agricole et rurale, il ne saurait être question d’une périodicité régulière des crises. Rien de plus irrégulier que la succession alternante des bonnes et des mauvaises récoltes ; quand sept vaches grasses suivent précisément sept vaches maigres, c’est un fait tout à fait exceptionnel. « Les cinquante années de 1715 à 1765, dit Laveleye, furent caractérisées par une remarquable exemption de saisons de disette, comparées aux cinquante années précédentes. » De 1793 à 1814, au contraire, les mauvaises récoltes furent très fréquentes. Cela nous explique pourquoi, dans les pays de petite industrie et de petit commerce, où, comme dans l’antique Égypte, la richesse ou la détresse générales, la félicité ou la tristesse publiques, dépendent, avant tout, de l’abondance ou de l’insuffisance des fruits de la terre, rien de pareil au retour régulier de nos états de malaise économique tous les neuf, dix ou onze ans, n’est jamais observé. Si donc Émile de Laveleye, dans son livre sur les crises, se bornait à nier, relativement à ces peuples encore peu avancés, le caractère périodique des crises commerciales ou industrielles, il aurait raison sans nul doute. Mais il nie à tort cette périodicité en ce qui concerne les perturbations \phantomsection
\label{v2p196} économiques du {\scshape xix}\textsuperscript{e} siècle, et sa seule excuse est d’avoir écrit à une époque où cette périodicité n’avait pas encore eu le temps de s’affirmer par le prolongement de la série déjà commencée. L’avènement de la grande industrie capitaliste a rendu le mouvement de la richesse publique, et du crédit public, en grande partie indépendant des péripéties de l’agriculture, et l’a subordonné aux péripéties de la finance, à la marée ascendante ou descendante des capitaux sur le grand marché des valeurs ; de telle sorte que la complication même des faits, des produits, des intérêts, a eu pour effet, ici comme ailleurs, de rendre régulier ce qui était désordonné auparavant.\par
Il y a, d’abord, des périodicités annuelles. Celles-ci ont existé de tout temps, mais sont plus marquées que jamais. De même que le thermomètre a des variations annuelles assez régulières, l’encaisse des banques, de la Banque de France notamment, monte ou baisse suivant les saisons avec une certaine régularité. Elle baisse, par exemple, au moment des moissons, quand le besoin d’espèces se fait sentir davantage, puis se relève. Mais cette ondulation annuelle ne fait que denteler une ondulation quasi-décennale, beaucoup plus grandiose, que les statistiques des banques ont mise en pleine lumière. Un tableau graphique joint par M. Clément Juglar à sa brochure sur les \emph{Crises commerciales et financières} (1900) ne laisse aucun doute à cet égard. On y voit la ligne graphique des escomptes, c’est-à-dire du \emph{portefeuille} des banques, toujours inverse de la ligne graphique de leur \emph{encaisse} métallique. Quand l’encaisse augmente, le portefeuille se vide ; et \emph{vice versa.} Mais ce qu’il y a de surprenant, c’est la régularité avec laquelle la série de ces inversions se poursuit. Ce qui ne l’est pas moins, c’est que cette répétition régulière est en même temps un agrandissement régulier. La loi d’amplification historique reçoit ici une confirmation manifeste.\par
Notons ce qu’il y a d’étrange dans cette régularité, mais  \phantomsection
\label{v2p197}d’abord dans celle des oscillations de la Bourse, qui peut en être rapprochée utilement. Les cotes de la Bourse, au commencement de ce siècle, ne contenaient qu’un très petit nombre de valeurs. La liste s’est allongée sans cesse, sollicitant les capitalistes par les genres d’attraction les plus variés. C’est comme une ville où, quand elle s’agrandit, les vitrines des magasins se multiplient, les étalages se diversifient, suscitant des velléités toujours plus nombreuses et plus diverses d’achats. Or, on aurait pu croire que, à mesure qu’augmenterait ainsi la diversité des valeurs cotées, l’inégalité des effets produits sur elles par les causes qui font varier les cours irait en s’accroissant, c’est-à-dire que, lorsque quelques-unes de ces valeurs monteraient, d’autres baisseraient, par compensation, ou du moins que leur vitesse de hausse ou de baisse, si elles montaient ou baissaient simultanément, deviendrait de plus en plus inégale. Mais non ; cette liste de valeurs est comme la série des anneaux d’un reptile, que le premier anneau, la tête, dirige toujours, et dirige de mieux en mieux à mesure qu’on s’élève sur l’échelle de cette classe d’animaux. Si l’un des grands fonds d’État est touché par un événement, aussitôt toutes les autres valeurs, même celles qui, d’après le calcul des probabilités, devraient bénéficier plutôt qu’avoir à souffrir de la catastrophe prévue, se mettent à baisser ensemble, moins vite, il est vrai, mais d’une vitesse qui tend de plus en plus à se rapprocher de la sienne. Pourquoi ? Parce que le marché est, avant tout, dominé par des influences psychologiques, par des actions intermentales dont le courant traverse en même temps tous les cerveaux, et, à partir de quelques esprits qui ont des raisons sérieuses d’être découragés ou remplis d’espoir, propage leur découragement ou leur confiance bien au-delà de leur groupe, dans tous les groupes de la Bourse, affairés, fiévreux, éminemment aptes à exercer et à subir les contagions de ce genre.\par
Ce qui se passe à la Bourse, se passe aussi bien à la  \phantomsection
\label{v2p198}Banque. On aurait pu penser, à priori, que, avec la multiplicité et la diversité croissantes des entreprises industrielles, des affaires de tout genre, qui donnent lieu à des négociations de lettres de change ou de billets à ordre, une sorte de compensation s’opèrerait de plus en plus entre la prospérité de certaines affaires et l’insuccès de certaines autres, de manière à éviter par degrés l’encombrement ou la raréfaction du portefeuille des banques. C’est l’inverse qui est arrivé et pour la même raison psychologique.\par
Je défie qui que ce soit de justifier par la raison seule, par le calcul froid et judicieux des vraisemblances, à l’usage des esprits sensés, abandonnés à eux-mêmes, \emph{sans influence d’autrui}, les oscillations vaguement rythmiques d’une valeur quelconque, par exemple du 3 p. 100 anglais au cours des deux derniers siècles. Dans son livre sur les \emph{Crises}, M. Juglar en donne le tableau graphique de 1731 à 1883. Plusieurs de ces abaissements ou de ces relèvements successifs s’expliquent par les péripéties de la défaite ou de la victoire, par des catastrophes ou des succès ; mais la plupart sont irrationnels. Car, aux yeux d’un homme de bon sens, il est clair que, depuis 1815, la prospérité anglaise, secondée ou non par la politique anglaise, quelle qu’elle fût, devait inspirer pleine confiance, une confiance grandissant continuellement, aux acheteurs de la rente anglaise. Il n’y avait donc jamais lieu de descendre aussi bas que 77 en 1831, ou même 81 en 1847. En sens inverse, il semble que, aux yeux d’une raison saine et soustraite, encore une fois, aux suggestions ambiantes, les cours extraordinaires de 107 en 1737 et de 106 en 1752 étaient fort exagérés à ces deux époques où la fortune de l’Angleterre était loin d’être aussi solidement assise qu’elle l’a été plus tard.\par
En somme, les oscillations de la cote des valeurs de Bourse reflètent vaguement les oscillations, plus régulières, de l’ensemble des prix des marchandises, par suite de l’alternance des époques de prospérité et des époques de crise.  \phantomsection
\label{v2p199}Cependant, une crise financière et même commerciale, si elle permet de révoquer en doute la solvabilité d’un certain nombre de banquiers et de commerçants non encore en faillite, ne saurait en rien faire douter de la solvabilité d’un État tel que l’Angleterre ou la France. Si donc la baisse générale des prix se communique à la cote des valeurs d’État, c’est que le même grand courant de \emph{pessimisme}, succédant à un grand courant d’optimisme, domine le marché des valeurs aussi bien que le marché des produits.\par
Notons, à ce sujet, que les salaires n’oscillent pas comme les prix, ou oscillent bien moins ; les salaires haussent pendant les époques de prospérité bien plus qu’à la suite des crises ils ne s’abaissent, si tant est qu’ils s’abaissent. C’est que les salaires trouvent dans le cœur humain, dans la conscience humaine, pour résister à la baisse générale, des points d’appui qui manquent aux autres prix. D’une part, l’ouvrier se fait un point d’honneur de ne pas travailler au rabais ; d’autre part, aux yeux de celui même qui l’emploie, le taux de son salaire habituel est considéré comme un droit, non comme un simple fait.\par
Il est à noter aussi que le taux de l’intérêt, pour les placements de capitaux dans la vie civile, en obligations hypothécaires notamment, n’est pas soumis aux mêmes fluctuations que le taux de l’escompte pour le papier de commerce. A travers ces hauts et ces bas alternatifs du taux de l’escompte depuis cent ans, le taux de l’intérêt civil a été s’abaissant, de même que les salaires vont s’élevant. Les périodes d’oscillation du taux de l’intérêt civil, si périodes il y a, sont, en tout cas, beaucoup plus amples et beaucoup plus irrégulières que celles des prix et du taux de l’escompte.\par
Encore une remarque. Si tout changement était nécessairement rythmique, comme le veut la loi d’Herbert Spencer ; si, par suite, l’abaissement industriel d’un pays pendant un laps de temps était une raison suffisante de son relèvement consécutif, il faudrait conclure que toute guerre, suspension  \phantomsection
\label{v2p200}forcée de la production d’une nation, doit être suivie fatalement d’un essor de l’industrie. Or, cela ne se vérifie pas toujours. Il est certain que la guerre de 1870 semble avoir été un coup de fouet vigoureux donné aux affaires françaises. Les bilans de la Banque de France, immédiatement après l’année terrible, révèlent un brusque et grandiose accroissement de l’activité productive ; progression bien plus forte encore et plus accentuée que celle qui se produit, à la même époque, dans les bilans de la Banque d’Angleterre\footnote{ \noindent « Le plus fort accroissement (des chiffres) en Angleterre, porte sur l’encaisse, qui s’élève de 24 à 35 ou 36 millions de livres sterlings de 1867 à 1879, (soit de 282 millions de francs), pendant que, au même moment, en France, de 1867 à 1877, elle s’élève de 748 à 2 281 millions de francs, soit de 1 533 millions. » (Juglar).
 }. Mais, s’il en a été ainsi après 1870, il en a été tout autrement après 1815. Comment se fait-il, si tout est rythmique dans le monde social comme dans le monde naturel, que la paix de 1815 ait été suivie jusqu’en 1822, non pas d’une fièvre d’industrie renaissante, exubérante, mais, au contraire, d’une dépression persistante des prix et d’une stagnation commerciale sur laquelle se lamente Sismondi ? « Un cri de détresse, dit-il, s’élève de toutes les crises manufacturières du vieux monde, et toutes les campagnes du nouveau lui répondent ; partout le commerce est frappé de la même langueur. » Ne semble-t-il pas que, après vingt ans de guerres continuelles, le retour de la paix aurait dû ramener aussitôt le travail et l’abondance, et d’autant plus vite et d’autant plus fort que les forces productives avaient été plus longtemps comprimées, comme celles d’un ressort qui se détend ? Cette différence entre les deux époques que je compare s’explique pourtant sans peine, si l’on tient compte des effets psychologiques de l’habitude, et de cette habitude collective, la coutume. Vingt ans de guerre avaient brisé les habitudes de travail et enraciné les habitudes de déprédation militaire ; la courte guerre de 1870-1871 n’a pas eu le temps de détourner de la sorte le courant général des idées  \phantomsection
\label{v2p201}et des volontés. Elle n’a fait qu’interrompre une période de prospérité qui a repris de plus belle après cette interruption.
\subsubsection[{II.5.c. Limitation et interprétation de cette périodicité.}]{II.5.c. Limitation et interprétation de cette périodicité.}
\noindent Tout cela dit, il n’en reste pas moins certain que les bilans comparés des diverses banques nationales, et notamment ceux de la Banque de France, où se peint le mouvement commercial du pays, ou plutôt de tous les pays de civilisation européenne, révèlent, depuis plus d’un demi-siècle au moins, une périodicité assez régulière, et de plus en plus marquée, qu’il s’agit d’interpréter.\par
D’abord, disons quelle est la nature de ces périodes. Chacune d’elles consiste en une phase d’espoir grandissant, de hauts prix, d’entreprises aventureuses, à laquelle succéda, à travers une courte crise, une phase inverse de découragement, de bas prix de stagnation des affaires. Si l’on compte la crise, malgré sa brièveté habituelle, pour une phase intermédiaire, cela en fait trois, que M. Juglar désigne ainsi : prospérité, crise, liquidation. Cette dernière, quoiqu’elle s’exprime par la baisse continue, n’en est pas moins une phase de convalescence, une guérison graduelle des maux causés par la débâcle, le rétablissement plus ou moins lent de l’harmonie brisée. A ce point de vue, on peut dire que ce cycle tripartite, qui semble tourner indéfiniment, est une petite évolution économique complète, qui, comme toutes les évolutions, d’après nous, comprend les trois moments successifs de la répétition (toute prospérité est une reproduction multipliante de richesses), de l’opposition (la crise n’en est qu’une forme), et de l’adaptation (la liquidation n’est que le moyen de s’adapter peu à peu aux nouvelles conditions du marché).\par
Ainsi présentée, la périodicité du phénomène qui nous occupe n’aurait rien que de naturel. Il n’y aurait pas plus à  \phantomsection
\label{v2p202}s’étonner de ces flots de la Banque ou de la Bourse que des vagues de la mer, pas plus même que des cycles de la vie végétale ou animale qui se répètent de génération en génération. Et l’on s’explique que beaucoup de bons esprits, se fondant sur le tableau statistique qui vient d’être présenté, prédisent la poursuite indéfinie des périodes de crises financières et commerciales.\par
Cependant, n’est-on pas exposé ici à prendre pour le rythme d’une \emph{rotation} normale et constante — s’il m’est permis de reprendre une métaphore de tout à l’heure — le rythme d’une simple \emph{oscillation} morbide et temporaire ? Le retour régulier et le régulier accroissement des accès de fièvre d’un malade signalent simplement l’approche de l’état aigu où se résoudra la maladie par la guérison ou par la mort. Et cette maladie sociale de notre âge, dont les crises commerciales ne seraient que l’un des symptômes, n’en pourrions-nous indiquer les causes ou les caractères psychologiques : la surexcitation mutuelle des convoitises, des avidités, des audaces, en même temps que la rupture de tout frein intérieur ou extérieur capable de la modérer, la projection hors des cadres anciens avant tout reclassement nouveau, l’affolement réciproque d’individus qui isolément seraient sages, mais à qui nulle raison collective, de tradition ou de législation, ne tient lieu encore de leur raison personnelle abdiquée ou affaiblie ? Ce sont là des traits moraux de notre temps qu’il est bien difficile de nier, quand on les voit se traduire par d’autres signes bien plus éloquents que la fréquence des sinistres commerciaux, à savoir par la progression numérique des suicides, des névroses, des folies déclarées.\par
La succession rythmique indéfinie de ces deux phases opposées et alternantes, long accès collectif d’illusion folle et longue dépression collective de timidité imbécile, serait quelque chose de bien plus anormal, sous son apparence de régularité, quelque chose de bien plus troublant et déconcertant \phantomsection
\label{v2p203} que les guerres les plus violentes. Car une guerre est une opposition qui aboutit à son terme, un problème qui se résout toujours ; tandis qu’une suite sans fin d’oppositions successives serait un problème toujours posé et jamais résolu. Empressons-nous d’ajouter que, quelle que puisse être l’interprétation du phénomène des crises périodiques, une simple considération suffit à nous assurer qu’il ne se reproduira pas indéfiniment. Supposons que cette périodicité se prolonge quelque temps encore, elle sera formulée en loi désormais acceptée, indiscutée, bientôt répandue partout et connue de tous les intéressés. Qu’arrivera-t-il fatalement ? Le fait seul que, au cours d’une phase ascendante, les industriels et les commerçants croiront pouvoir prédire à coup sûr son terme prochain, empêchera ce terme, c’est-à-dire la crise, de se produire, en tempérant l’excès de leur confiance ; et \emph{vice versa}, au cours de la phase descendante, la prévision générale du relèvement des affaires à date fixe les fera se relever bien avant cette date. En deux mots, les crises ne sauraient être longtemps périodiques sans être généralement prévues, ni être prévues sans être prévenues, c’est-à-dire sans cesser d’être périodiques. Ainsi se vérifie en grand, dans le monde économique, ce principe psychologique, mis en lumière par Guyau, que, en prenant conscience d’une habitude ou d’un instinct, en le formulant, nous échappons à sa loi.\par
Mais, encore une fois, que signifie la périodicité en question, même temporaire, comme je le crois ? Si nous consultons à ce sujet la littérature socialiste, nous y lisons que c’est là un caractère propre à l’industrie bourgeoise, une conséquence du capitalisme. « De même, dit Kautsky, que \emph{l’industrie capitaliste} passe tour à tour par des périodes de prospérité et de crise, de même, en politique, nous trouvons des époques de grand combat et de progrès rapides... » etc. Mais adressons-nous au grand maître. Il y a chez Karl Marx parfois une tournure ontologique, j’allais dire mythologique,  \phantomsection
\label{v2p204}d’esprit, qui lui vient de Hégel. Le capital, la valeur, sont pour lui des êtres qu’il anime de sa passion et de sa vie. Dans un curieux passage\footnote{ \noindent Voir, le \emph{Capital}, livre II, \emph{trad. franç.} (Giard et Brière, 1900).
 }, où il est question des révolutions de la valeur, c’est-à-dire des crises, il repousse avec désinvolture l’objection qu’on aurait pu tirer de ces dépressions brusques des prix contre sa théorie de la valeur ; car, si celle-ci se mesure à la force de travail dépensée, comment se peut-il faire que, sans que cette force de travail ait en rien diminué, son produit perde tout à coup la moitié ou les trois quarts de sa valeur antérieure ? Voici ce qu’il écrit à ce sujet : « Ceux qui considèrent la valeur comme une abstraction oublient que le mouvement du capital industriel est cette abstraction \emph{in actu}. La valeur prend ici différentes formes, effectue différents mouvements dans lesquels elle se conserve et s’accroît. Plus les révolutions de valeur deviennent aiguës et fréquentes, plus le mouvement de \emph{la valeur, devenue autonome et agissant automatiquement avec la puissance d’un phénomène naturel élémentaire}, se fait sentir à l’encontre de la prévoyance et du calcul du capitaliste isolé. Les révolutions périodiques de la valeur confirment donc ce que l’opinion ordinaire veut qu’elles réfutent : le fait que la valeur, comme capital, devient indépendante et qu’elle conserve et accentue son indépendance par son mouvement. » La périodicité des crises est donc ici, aux yeux de Marx, une raison de croire que ces oscillations rythmiques des prix attestent l’autonomie du mouvement évolutif de la valeur, indépendamment des désirs et des options de l’homme, des calculs du producteur et des besoins du consommateur.\par
Ailleurs, cependant, Karl Marx prend la peine de rechercher les vraies causes des crises industrielles, et dans le passage suivant il semble rattacher en grande partie à l’inventivité continue de notre âge la périodicité de ces phénomènes\footnote{ \noindent Voir le \emph{Capital}, livre II, p. 187 de la trad. fr.
 }. « L’existence du capital industriel (matériel des  \phantomsection
\label{v2p205}usines, constructions, etc.) est raccourcie, dit-il, par la révolution incessante des procédés de fabrication, qui est activée par l’expansion du régime capitaliste, et qui nécessite le renouvellement des moyens de production longtemps avant qu’ils ne soient arrivés à leur limite d’usure. On peut admettre que, dans les branches les plus importantes de la grande industrie, ce cycle de vie comprend aujourd’hui dix ans en moyenne, chiffre qui n’a, du reste, aucune importance pour nos conclusions. Ce cycle de rotations reliées entre elles, d’une durée de plusieurs années pendant lesquelles le capital est captif de son élément fixe, fournit une base matérielle aux crises périodiques, pendant lesquelles les affaires parcourent des périodes de stagnation, de vivacité moyenne, de précipitation et de trouble. » Cela est un peu obscur, mais l’idée générale qui s’en dégage semble, être en somme, que les périodes de précipitation, de surproduction fiévreuse qui précèdent les crises et les préparent, sont dues, en majeure partie, aux excès de confiance suscités par les nouvelles inventions qui révolutionnent les industries. Il ne faudrait donc voir, d’après cela, dans la série périodique des crises depuis cent ans, qu’un signe manifeste de notre inventivité moderne ; et, si celle-ci est « activée par l’expansion du régime capitaliste », il n’y aurait pas lieu de tant maudire le \emph{capitalisme}. A ce point de vue, la périodicité presque régulière des crises pendant le {\scshape xix}\textsuperscript{e} siècle s’expliquerait par la fécondité même du génie industriel de notre temps, qui répondrait toujours au moment voulu, par de nouvelles poussées d’inventions à l’appel d’une nouvelle vague d’optimisme général prête à se soulever après une dépression de pessimisme... D’où on pourrait conclure que, inévitablement, cette périodicité doit aller s’effaçant quand l’invention se raréfiera et s’épuisera, ce qui arrivera un jour ou l’autre.\par
Déjà la régularité de la série diminue. L’avant-dernière crise, celle qui a commencé en 1873, a duré une quinzaine d’années, au lieu de dix ou onze. Avant 1815, sous l’ancien  \phantomsection
\label{v2p206}régime, rien n’était plus irrégulier que les intervalles d’apparition des crises. Il est à remarquer qu’elles ne se sont suivies d’une manière à peu près régulière que durant la rapide et décisive progression économique du monde. Ce moment de régularité relative des périodes a coïncidé avec l’ère du grand développement industriel et commercial ininterrompu, sans autre obstacle que celui qu’il s’opposait à lui-même par sa trop grande précipitation... On dirait d’une carafe qui s’engorge et dont les dégorgements convulsifs deviennent dès lors momentanément périodiques. Ce qui est certain, c’est que, inévitablement, la périodicité des crises s’arrêtera avec les crises elles-mêmes, quand le progrès industriel et commercial aura atteint son terme, qui ne saurait être indéfini. Mais, avant même cette époque éloignée, il n’est pas douteux qu’elles seront de plus en plus facilement arrêtées dans leur germe. Une crise, en effet, débute toujours par un encombrement simplement particulier, spécial à une industrie donnée, dans une localité ou une région ; et pour empêcher que ce malaise se propage, s’étende de plus en plus loin par suite des progrès de la division du travail et de l’échange, que faut-il ? Bernstein expose très bien comment les crises sont prévenues, par les cartels notamment. Il reconnaît que « à des perturbations locales, particulières, il peut être aisément remédié grâce à la masse des capitaux, au crédit et à la rapidité des moyens de communications ». Mais, s’il en est ainsi, quelle est la crise qui ne peut être arrêtée à ses débuts, puisqu’il n’en est pas une qui ne commence par être « une perturbation locale ou particulière » avant de se généraliser ? Bernstein regarde la spéculation — à laquelle se rattache sans aucun doute la périodicité des crises — comme une \emph{maladie infantile} du régime capitaliste, maladie qui se guérira d’elle-même, à mesure que les progrès de l’information rapide, instantanée, diminueront la marge de l’inconnu en fait de données du problème économique.\par
 \phantomsection
\label{v2p207}Quant à l’alternance de l’optimisme et du pessimisme, de l’enthousiasme et de la prudence timorée, dans le public, elle a ses causes bien connues, où se reflète la loi universelle de l’action alternant avec la réaction. C’est surtout en politique et dans la gestion des finances d’un État que se vérifie la maxime : « A père avare fils prodigue », et qu’on en touche du doigt l’explication psychologique. Après que, de 1875 à 1888, la sagesse des financiers italiens eut relevé les finances de leur pays, quel fut le premier effet de cette habileté économe sur l’état d’esprit de leurs successeurs ? Ceux-ci conçurent aussitôt les plus vastes projets, enhardis dans leurs entreprises ruineuses précisément par la grandeur des résultats dus à l’économie antérieure. Et toujours il en a été de même. Il est sans exemple peut-être qu’à un gouvernement économe n’ait point succédé un gouvernement \emph{mégalomane}, dont la mégalomanie est le contre-coup de la modestie prudente qui l’a précédé. Ce qui est vrai des finances publiques ne l’est pas moins des finances privées. Quand, pendant une période de liquidation, les usines, les banques, les industries quelconques ont été gérées un certain temps par une majorité d’industriels prudents, un peu craintifs, une nouvelle génération, ou pour mieux dire une nouvelle équipe d’esprits entreprenants s’efforce d’entrer en scène, et y parvient toujours, apportant sa force de foi neuve, inaugurant une nouvelle ère de surproduction qui aboutira à une nouvelle crise, et ainsi de suite, non pas indéfiniment, mais jusqu’à ce que ce rythme soit assez généralement connu et prévu pour être prévenu.\par
La surproduction aboutissant à la crise, c’est-à-dire à l’avilissement des produits, est un fait qui a lieu pareillement en littérature, dans les beaux-arts, en politique. Il apparaît de temps en temps, en poésie ou en peinture, au sein des vieilles écoles épuisées, une école nouvelle qui se dépense en promesses, en œuvres précipitées, use et abuse de l’enthousiasme suscité par ses débuts, puis s’affaisse dans le discrédit. \phantomsection
\label{v2p208} N’est-ce pas de la même manière qu’un système politique, après avoir apparu et pris faveur, fait faillite et laisse la place à un autre ? Reste à savoir s’il n’y aurait pas quelque vague synchronisme approximatif entre ces périodes alternantes de hausse et de baisse, d’exaltation et de dépression, d’engouement et de dégoût, en littérature, en politique, en art, dans le domaine économique. Il semble qu’à certaines époques, on remarque plutôt le contraire, c’est-à-dire l’enchevêtrement de ces différentes périodes, comme si le génie inventif se donnait, par exemple, d’autant plus facilement carrière du côté littéraire ou artistique qu’il trouve la voie industrielle ou la voie politique plus obstruée. Mais, si l’on ne s’attache pas trop aux exceptions de détail, si l’on se borne à jeter un coup d’œil d’ensemble sur les grandes lignes de l’histoire littéraire comparées à celles de l’évolution économique, on sera assez disposé à accorder à M. Renard\footnote{ \noindent Voir son ouvrage sur l’\emph{Histoire littéraire}, où il a consacré un long et intéressant chapitre à l’influence des conditions économiques sur la littérature.
 } que les époques de prospérité, par exemple les premières années du règne personnel de Louis XIV, de 1661 à 1672, se signalent à la fois par le mouvement ascendant de la fortune publique et l’éclat du génie littéraire, tandis que la fin du même règne est marquée par la dépression et l’épuisement des talents autant que par la misère et la ruine générales. Même observation en ce qui concerne le moyen âge, où la belle période poétique coïncide avec la richesse industrielle du {\scshape xiii}\textsuperscript{e} siècle, et où la détresse lamentable, durant la guerre de Cent ans, s’accompagne d’une pauvreté et d’une indigence d’imagination non moins remarquable. De nos jours, depuis une vingtaine d’années, n’a-t-on pas vu aussi l’optimisme et le pessimisme se succéder en littérature, comme note dominante, et y refléter en quelque sorte la hausse et la baisse de l’espérance et du crédit ? Mais il ne faudrait pas trop presser ces formules.
 \phantomsection
\label{v2p209}\section[{III. L’adaptation économique}]{III. L’adaptation économique}\phantomsection
\label{l3}\renewcommand{\leftmark}{III. L’adaptation économique}

\subsection[{III.1. Division du sujet}]{III.1. Division du sujet}\phantomsection
\label{l3ch1}
\subsubsection[{I}]{I}
\noindent Nous sommes arrivés à la partie la plus ardue de notre travail, c’est-à-dire à celle où doivent être résolues les difficultés de la vie économique. Dans les deux premières parties, nous avons exposé les \emph{données} d’abord, puis les \emph{problèmes}, dont nous avons, dans la troisième, à chercher les \emph{solutions}. Solutions toujours provisoires, disons-le tout d’abord : quelles qu’elles soient, elles ont toujours pour effet, en supprimant certaines oppositions, d’en susciter d’autres, plus larges, plus amples, ce qui n’en constitue pas moins un progrès quand, par cette transformation et cette amplification même, la lutte économique s’adoucit ou se raréfie. N’oublions pas que nos trois termes, répétition, opposition, adaptation, forment un cercle en train de tourner sans cesse, jusqu’à épuisement de vie sociale. Une fois formée, une adaptation nouvelle se développe en se répétant, et, par ses répétitions, s’oppose à d’autres qui se sont répétées aussi, puis, par cette opposition même, ou directement, s’adapte à d’autres, harmonie d’un degré supérieur où se résolvent les contradictions précédentes. C’est là une dialectique sociale qui peut rappeler les triades de Hegel, à cela près qu’elle n’exige nullement  \phantomsection
\label{v2p210}la violation des lois de la logique ordinaire et suppose seulement la distinction de la logique individuelle et de la logique sociale. C’est, si l’on aime mieux, une suite de drames en trois actes, dont chacun consiste en une \emph{exposition}, un \emph{nœud} et un \emph{dénoûment.}\par
Jetons un premier coup d’œil général sur notre sujet. Il y a à distinguer, nous le savons, l’adaptation quantitative et l’adaptation qualitative de la production à la consommation. Les deux sont faciles au début de l’évolution économique. Alors le petit cordonnier, le petit menuisier, le petit tailleur ambulant, travaillent pour une clientèle qui leur est personnellement connue et savent sans le moindre doute ce qu’ils doivent produire, en quantité et en qualité. Une fois donnée l’invention des procédés que ces artisans mettent en œuvre, — car il faut toujours partir de là — rien de plus simple quand on travaille sur commande, que d’adapter le nombre et la diversité des procédés au nombre et à la diversité des besoins et des goûts. Le problème ne commence à se compliquer que lorsque le marché s’agrandit. Le producteur doit se décider d’après des inductions et des probabilités et deviner le nombre, les goûts, les besoins, de ses futurs acheteurs inconnus. Pour répondre à son embarras d’autres inventeurs ont dû imaginer des agences de publicité et d’information, des statistiques, et la nécessité des commerçants intermédiaires se fait sentir. Mais l’adaptation des produits aux besoins n’est pas la seule forme de l’adaptation économique. D’abord, elle en suppose une autre, très différente quoiqu’intimement liée à la première : à savoir, l’adaptation quantitative et qualitative du \emph{producteur} à la production, des services aux produits, en nombre et en qualité ; il s’agit de trouver le nombre voulu d’ouvriers suffisamment propres au travail qu’il y a à faire ; ce qui soulève, par ce côté, le problème de la population. Par l’hérédité des professions, par la réglementation de l’apprentissage, les anciennes corporations avaient essayé une solution, qui n’a rien perdu de  \phantomsection
\label{v2p211}son importance, mais à laquelle il a fallu ajouter les bureaux de placement, les Bourses de commerce, et autres moyens ingénieux d’opérer la rencontre entre le travailleur et le travail auquel il est propre. — Enfin, pour que la production s’ajuste le mieux possible à la consommation, ne faut-il pas que chacun de ces termes s’harmonise le mieux possible avec lui-même, c’est-à-dire que les diverses espèces de production s’entravent le moins possible, s’entr’aident le plus possible, convergent le mieux possible vers les mêmes fins nationales ; qu’il y ait, en un mot, la meilleure organisation de travail, spontanée ou consciente ; et que les diverses espèces de besoins et de consommations se conforment, dans leur hiérarchie spontanée ou consciente, à une sorte de programme collectif de la conduite, du plan de vie générale aussi logique qu’il se peut ? Deux grands problèmes dont les sociétés, de tout temps, ont été tourmentées et qui ont reçu des solutions successives. En ce qui concerne le premier, nous avons eu la solution \emph{esclavagiste} dans l’antiquité, la solution \emph{monastique} et \emph{corporative} au moyen âge, la solution \emph{libérale} à l’époque contemporaine, en attendant la solution \emph{socialiste} ou tout autre, dont la formule est cherchée. En ce qui concerne le second, c’est-à-dire l’organisation des besoins pour ainsi parler, nous avons eu les formes successives de la morale, qui consiste toujours à harmoniser les désirs les plus divers d’un même individu, ou, aussi bien, des individus différents, en les orientant vers la poursuite commune et constante d’un même idéal qui change d’âge en âge et de peuple à peuple : la domination, l’indépendance, la gloire, le plaisir, la \emph{richesse.} Quand, dans ce dernier cas, tout est \emph{économique}, le but ainsi que les moyens, quand la richesse se sert de fin à elle-même, c’est l’absurdité même, un cercle vicieux. C’est cependant cette hypothèse qui est la plus chère aux fondateurs de l’économie politique ; c’est sur elle qu’ils ont fondé la conception de leur \emph{homo œconomicus.} En réalité, la domination despotique, l’indépendance stoïcienne, le salut chrétien, \phantomsection
\label{v2p212} le plaisir et le confort épicuriens, ont été des mobiles tout autrement efficaces de l’activité. Ajoutons que le but commun des désirs intra-individuels ou inter-individuels, leur point de convergence, n’est jamais plus souverainement harmonisateur que lorsqu’il est situé dans le lointain, sinon dans l’imaginaire, comme la béatitude mystique ou l’idylle utopique. La force de l’idéal socialiste est d’apparaître dans l’éloignement du futur. Il en est ainsi, parce que les croyances dont l’objet est inaccessible sont les seules qui ne puissent pas être démenties. Peu importe que, pour la même cause, elles ne puissent pas être démontrées : leur propagation leur tient lieu de démonstration. Avant tout, en effet, pour obtenir la paix sociale, il est essentiel qu’il existe une foi unanime et que l’accord des idées sur une même conception du vrai et du juste se superpose au désaccord des désirs, puisque toujours, quoi qu’on fasse, il subsistera entre ceux-ci des dissonances.\par
On peut dire, d’une manière très générale, que les conditions psychologiques de la paix sociale, de l’harmonisation des intérêts, changent profondément quand une société passe du règne exclusif de la \emph{coutume} à l’influence dominante de la \emph{mode} (sauf à revenir à la coutume élargie). Aux temps de coutume exclusive, la paix sociale se fonde sur le respect et la résignation, le mépris de l’étranger, le culte des aïeux ; aux temps de mode dominante la paix sociale, sorte d’équilibre mobile, de stabilité dynamique, se fonde sur l’espérance et la joie, sur l’ambition, sur la fiction de l’égalité, sur l’avidité des exemples exotiques. L’erreur de Le Play est de n’avoir pas fait cette distinction.
\subsubsection[{II}]{II}
\noindent — Mais serrons de plus près et abordons par d’autres côtés le sujet qui nous occupe. Peut-être, pour l’embrasser dans toute sa complexité, conviendrait-il de prendre un à un  \phantomsection
\label{v2p213}tous les problèmes posés par chacune des espèces d’opposition économique et de rechercher comment elles ont été résolues historiquement, ou théoriquement pourraient l’être. Mais il est inutile de nous assujettir à l’ordre même de cette énumération. Essayons plutôt une classification générale des principales solutions de ces problèmes, des institutions bien connues où s’incarne l’harmonie économique sous ses divers aspects.\par
Une première division qui s’impose ici, comme dans la partie précédente, c’est celle des adaptations \emph{intra}-individuelles et \emph{inter}-individuelles. Dans la théorie des prix, nous avons montré la bataille interne des désirs de divers articles et des jugements sur leur valeur comparée, et l’issue de ce conflit psychologique par le sacrifice des désirs non satisfaits et des jugements non écoutés au désir et au jugement triomphants. Il reste à voir le phénomène inverse, c’est-à-dire non plus la lutte mais l’alliance des divers désirs dans un même cœur, des diverses idées dans un même esprit, au point de vue économique ; d’où résulte l’accord des travaux et des besoins. Si nous considérons l’individu isolé, pré-social, nous dirons qu’il s’harmonise avec lui-même lorsque la série de ses occupations concourt à satisfaire exactement et complètement la série de ses besoins, tous physiques, manger, boire, se garantir du froid, s’accoupler, qui concourent ensemble sans qu’il en ait conscience, à la conservation temporaire de son être et à la conservation indéfinie de son espèce. Son harmonie interne suppose donc une adaptation à la fois qualitative et quantitative de sa production individuelle à sa consommation individuelle. Tel aurait dû être conçu l’idéal du sage stoïcien, sorte de Robinson métaphysique.\par
Mais laissons là cette abstraction et revenons à l’homme vrai. Chez l’individu social, il suffit, pour qu’il y ait en lui harmonie économique, qu’il produise des choses quelconques, propres ou non à satisfaire ses désirs, mais ayant une valeur  \phantomsection
\label{v2p214}vénale qui lui permette de satisfaire par l’échange tous les besoins qu’il a, et aussi ceux qu’il aura, ce qui suppose un excédent de revenu destiné à être épargné en vue de la maladie et de la vieillesse. Est-ce à dire que la série de ses actes puisse être aussi incohérente que possible, et de même la série de ses besoins, pourvu que la condition exigée soit remplie ? Non, car il faut aussi que ses actes productifs s’adaptent entre eux, et que ses désirs de consommation s’harmonisent aussi. Or, ses actions successives ne forment un accord, et pareillement ses besoins successifs, que si les premières convergent vers la réalisation d’une œuvre qui donnera satisfaction directe et indirecte à la passion-maîtresse exprimée par les seconds. Mais, la suggestion ambiante pouvant seule soulever l’homme au-dessus de lui-même, cette passion-maîtresse ne saurait être que la poursuite d’un but collectif : agir sur les autres hommes ou les faire agir, les servir ou se faire servir par eux, le tout conformément à un plan qu’on a conçu (\emph{invention}) en vertu de certaines \emph{découvertes} faites par l’observation prolongée de la nature. (Par ses découvertes, l’individu adapte sa croyance à la nature ; par ses inventions, il adapte la nature, y compris la nature humaine, à son désir.)\par
L’harmonie interne de l’individu avec lui-même et l’harmonie externe des individus entre eux sont donc intimement liées, puisque la première ne peut s’accomplir, dans l’individu social, que moyennant un plan, petit ou grand, d’organisation sociale, plus ou moins partielle ou générale. Il n’en est pas moins vrai que l’adaptation individuelle et l’adaptation sociale font deux, et que l’une peut progresser un certain temps pendant que l’autre décline. Pendant que les individus deviennent de plus en plus incohérents, accueillant pêle-mêle, au cours d’un âge prodigieusement inventif, les besoins les plus hétérogènes, et se livrant à tous les désordres, l’échange peut se développer beaucoup, ainsi que la division du travail. Il est à remarquer que, dans une civilisation \phantomsection
\label{v2p215} en progrès, la mutuelle assistance ou la convergence finale des besoins différents d’individus différents, l’harmonie sociale, précède l’harmonie individuelle, la mutuelle assistance\footnote{ \noindent Cette \emph{mutuelle assistance} de désirs enchaînés et périodiquement renaissants dans la journée d’un épicurien n’est elle-même, à y regarder de près, qu’une \emph{convergence finale ;} seulement, la \emph{fin} ici est inconsciente, elle est \emph{la conservation de la vie} et de la personnalité. Il en est de même de la \emph{mutuelle assistance}, des désirs inter-individuels par l’échange : elle dissimule leur convergence vers la fin nationale, vers la durée de la nation.
 } ou la convergence finale des besoins différents d’un même individu. Mais ce contraste ne peut se poursuivre indéfiniment. D’abord, à l’origine de tout travail maintenant divisé, collectif, nous trouvons un travail indivis, individuel, accompli par un même travailleur, dont les actes successifs formaient ainsi une harmonie individuelle avant de donner lieu, plus tard, à une harmonie sociale. Puis, le progrès durable de la division du travail suppose des travailleurs qui ont trouvé leur assiette mentale et morale, et se sont fixés en une spécialité. Et le développement prolongé de l’échange suppose des consommateurs sortis en majorité de la phase des caprices instables, pour goûter l’équilibre mobile d’un enchaînement périodique de désirs alternativement renaissants, ce qui peut être regardé comme l’équivalent individuel de l’échange, de l’aide mutuelle des besoins dans une société.\par
On contestera peut-être ce qui vient d’être avancé plus haut, que toutes les collaborations, spontanées et inconscientes, appelées \emph{divisions du travail}, si complexes et si étendues qu’elles soient, procèdent d’un seul et même travail initial, suscité par une invention. Cette proposition n’en est pas moins certaine. C’est toujours dans le cerveau d’un individu que se présente tout d’abord, sous la forme d’une invention, l’adaptation à un but commun d’actes regardés jusque-là comme étrangers ou même contraires les uns aux autres ; et ce n’est qu’ensuite que cette association d’idées s’extériorise et se déploie en une association d’hommes par  \phantomsection
\label{v2p216}la division du travail. La solidarité des innombrables ouvriers métallurgistes, mécaniciens, ébénistes, terrassiers, etc. qui collaborent à la construction d’un chemin de fer n’est que la projection au dehors et le déploiement de la liaison étroite établie entre les idées de ces différents modes d’action par la conception première de la locomotive circulant sur rails. J’en dirai autant, bien entendu, de la solidarité des ouvriers qui construisent un télégraphe ou un téléphone, ou des fabricants et opérateurs quelconques qui concourent, sciemment ou à leur insu, à la production photographique. Ici comme partout, l’invention initiale a été considérablement grossie par des perfectionnements successivement greffés sur elle, autant d’inventions minuscules et auxiliaires, mais cela ne change rien à la vérité de notre proposition. Si ce n’est pas du seul cerveau de Guttemberg, c’est de lui et des quelques esprits d’inventeurs fécondés par lui, et aussi de ceux qui l’ont éveillé lui-même, qu’émane l’harmonie intime d’intérêts entre les travailleurs sans nombre, papetiers, fondeurs de caractères, imprimeurs, relieurs, libraires, unis par la collaboration au livre ou au journal, comme dans la gigantesque union américaine des typographes. Avant de s’opérer objectivement par la division du travail, la co-adaptation des divers travaux nécessaires pour la fabrication d’un vêtement, d’un chapeau, d’un meuble, d’une maison, a dû s’opérer subjectivement dans l’esprit du premier qui a conçu l’idée de fabriquer des produits de ce genre.\par
L’échange est l’harmonie externe des besoins divers, comme la division du travail est l’harmonie externe des divers travaux\footnote{ \noindent Ce sont-là deux procédés bien dissemblables et bien inégaux d’adaptation économique, L’échange, j’entends l’échange indirect, par l’intermédiaire de la monnaie, co-adapte tous les besoins les uns aux autres dans toute l’étendue d’un même marché, tandis que la division du travail ne solidarise qu’une très minime fraction des travaux, et ces groupes disséminés restent étrangers les uns aux autres au point de vue de la production totale, aussi bien que les travaux demeurés indivis et indifférenciés. Le nombre des travaux, non pas différenciés, mais nés différents, peut s’accroître dans un pays, par suite d’inventions créées, ou importées, sans que la division du travail s’y développe. Si on importe dans ce pays cent espèces nouvelles de plantes domestiques et de fleurs cultivées, c’est-a-dire cent nouveaux modes de culture, différents les uns des autres, le travail des horticulteurs et des agriculteurs y deviendra plus complexe mais non plus cohérent. Seulement, ces nouveaux produits s’ajouteront aux autres articles du marché, et l’échange y nouera des liens à la fois plus compliqués et plus solides. — Si l’une de ces cultures, celle du prunier d’Agen par exemple, prend une telle extension qu’il vaille la peine d’y affecter des ouvriers agricoles spécialisés, les uns adonnés au greffage, d’autres à la récolte des fruits, d’autres au séchage au four, etc., il y aura alors une nouvelle division du travail ; et il se pourra que cela serve au progrès de l’échange, comme lorsqu’un paysan remet à un boulanger de son voisinage, son collaborateur, un sac de blé contre un certain nombre de livres de pain ; mais, en somme, les co-producteurs ne seront pas plus ni autrement liés entre eux par la solidarité de leurs besoins qu’ils ne le sont avec les étrangers.
 }. Dirai-je de l’échange ce que je viens de dire  \phantomsection
\label{v2p217}de la division du travail, c’est-à-dire qu’avant de s’harmoniser au dehors, dans la société, les besoins différents ont dû s’harmoniser au-dedans de l’individu ? Oui, bien longtemps avant qu’une société, une petite cité d’abord, présente des courants réguliers d’échanges, à des prix acceptés de tous et fixés par la coutume ou par la mode, les individus qui la composent, \emph{patres familias} indépendants, clos dans leurs maisons et sans commerce entre eux, ont commencé par donner le spectacle d’un train de vie réglé, digne et noble, dont la dignité et la noblesse consistent précisément dans cette régularité d’habitudes quasi-religieuse, où tout conspire vers un même idéal traditionnel ou tout au moins vers une même fin naturelle instinctivement poursuivie. C’est parce qu’on pouvait compter sur la renaissance périodique et la persistance durable de ces désirs enchaînés, aussi bien que sur leur similitude et leur uniformité d’un individu à un autre, que l’idée de travailler pour satisfaire quelqu’un de ces désirs chez autrui, avec l’espérance d’acquérir d’autrui la satisfaction personnelle de quelqu’autre désir, a pu naître chez un indigène entreprenant (peut-être à l’exemple d’un étranger) et donner lieu à un rudiment de commerce. Jamais parmi des hommes primitifs aussi capricieux, par hypothèse, aussi changeants de goûts et de besoins, aussi extravagants  \phantomsection
\label{v2p218}que le sont parfois les ultra-civilisés, le commerce n’aurait pris naissance ou n’aurait pu vivre.\par
Il est vrai que, plus tard, chaque idée nouvelle et chaque besoin nouveau qu’apporte une invention est une cause de trouble ou de fermentation pour le système mental et moral de l’individu, tandis que tout produit nouveau jeté dans la circulation y consolide, nous l’avons vu, les liens de l’échange. Mais cette anticipation de l’harmonie externe sur l’harmonie interne n’est qu’apparente : et la première ne peut être vraie, solide et profonde qu’à partir du moment où l’idée nouvelle est entrée dans le système des croyances par l’élimination de celles qui la contredisent, et où le besoin nouveau s’est classé et enraciné dans les habitudes.
\subsubsection[{III}]{III}
\noindent Ainsi, qu’il s’agisse des travaux ou des besoins, ou aussi bien des deux, — car l’échange est l’adaptation des travaux aux besoins aussi bien que l’adaptation réciproque des besoins les uns aux autres — l’harmonie subjective, individuelle, est la cause et l’explication de l’harmonie objective, sociale. Et c’est, nous allons le voir, par un double procédé d’\emph{invention} et de \emph{critique}, double face de l’élaboration logique, — par l’invention créatrice dans un cas, par la critique épuratrice dans l’autre cas — que l’harmonie interne est opérée. Ajoutons que la division du travail, suscitée par l’invention, et l’échange, issu de l’habitude formée par un classement critique des besoins et des idées, sont simplement les formes spontanées et libres de l’association humaine soit pour la production soit pour la consommation. Or, l’association ainsi ébauchée est toujours imparfaite et défectueuse, et, pour s’élever à ses formes accomplies, c’est-à-dire à l’association proprement dite, réfléchie, consciente, disciplinée, il faut une conception individuelle encore, un plan, un programme  \phantomsection
\label{v2p219}tantôt plus génial et inventif que critique, tantôt plus critique et judicieux qu’inventif, auquel se conformera le groupement nouveau d’intérêts, d’esprits et de volontés. Cela est d’autant plus vrai qu’il s’agit d’un groupement plus vaste. Si donc la grandiose association rêvée par les collectivistes se réalisait jamais, elle aussi, elle surtout, serait sortie du cerveau d’un homme.\par
Ainsi, il n’est point d’harmonie sociale, et spécialement économique, qui n’ait été précédée et préparée par une harmonie psychologique, et à l’origine de toute association entre hommes nous trouvons une association entre les idées d’un homme. Arrêtons-nous un moment pour indiquer la signification philosophique de ce fait constaté. Il s’ensuit, évidemment, que la société n’est pas un organisme ; mais s’ensuit-il qu’elle ne soit pas une réalité distincte de ses membres ? Voilà une question qui réclame une réponse nette. Si l’idée de l’organisme social peut être défendue, ce n’est qu’en tant qu’elle est une expression, malheureuse il est vrai, du \emph{réalisme social}, c’est-à-dire de la société conçue comme \emph{un} être réel et non pas seulement comme un \emph{certain nombre} d’êtres réels. Or le meilleur appui de cette conception, ne serait-ce point la découverte des « lois naturelles » qui, indépendamment de toute volonté individuelle, conduiraient les individus, par des voies toutes tracées d’avance, à une organisation politique, morale, économique, de plus en plus parfaite ? La doctrine du \emph{laisser-faire} a donc les plus grandes affinités avec celle de la société-organisme, et les coups dirigés contre celle-ci atteignent l’autre par contre-coup. Si l’on avait des raisons de croire à l’harmonisation spontanée des sociétés, on en aurait par cela même de tenir une société pour \emph{un} être réel, au même titre qu’une plante ou un animal. Mais vraiment l’illusion de cette prédestination providentielle ne se dissipe-t-elle pas de plus en plus, même au point de vue économique ? Quant au point de vue politique, il suffit d’ouvrir les yeux pour voir les nations monter ou descendre,  \phantomsection
\label{v2p220}se fortifier ou s’affaiblir, suivant qu’elles ont trouvé ou non, au moment voulu, la main forte d’un homme d’État ; et il n’est plus permis de croire à un \emph{sens inné de la direction} qui piloterait les peuples sans nul conducteur apparent.\par
Cependant, la renonciation à cette erreur longtemps séduisante doit-elle nous conduire à nier toute réalité propre du tout social, à le considérer comme un simple total, expression numérique des individus rassemblés ? Non. Si nous nous refusons à admettre des \emph{lois naturelles} dans le sens indiqué, et aussi bien des \emph{formules d’évolution} qui en sont la forme la plus récente, nous admettons en tout individu un besoin plus ou moins vif de coordination logique des idées, de coordination finale des actes, besoin qui s’avive par le rapprochement des individus, qui devient une tendance générale à une logique et à une finalité croissantes, en toute catégorie de faits sociaux, et finit par y faire partout de l’ordre avec du désordre, à y carder le chaos en monde. Cette manière de voir diffère de celle des harmonies providentielles ou des évolutions unilinéaires en ce que, au lieu d’assujettir le train social à suivre une seule voie, toujours la même, elle lui laisse bien plus libre jeu. Et par là, on est conduit non à nier la réalité sociale mais à la concevoir comme tout autrement vivante et vraie, tout autrement riche en manifestations et en itinéraires imprévus. Autre chose est une formule algébrique qui fournit des solutions à une foule de problèmes différents, autre chose est une équation arithmétique qui ne s’applique qu’à un problème et ne comporte qu’une solution. Je suis réaliste aussi, en ce sens que la société réalise seule à mes yeux, comme aux yeux de mes adversaires, des virtualités contenues dans les individus et qui ne sauraient être réalisées par chacun d’eux isolément ; mais je dis que ces virtualités sont des idées et des volontés individuelles, je les place dans des cerveaux au lieu de ne les situer nulle part, si ce n’est dans des nuages ontologiques ; et je dis que ces virtualités sont innombrables, inépuisables \phantomsection
\label{v2p221} comme leur source spirituelle, au lieu de les limiter à un nombre strictement déterminé ou plutôt prédéterminé.
\subsubsection[{IV}]{IV}
\noindent Cette explication donnée une fois pour toutes, revenons à notre sujet. Il y a une grande distinction qui s’offre à nous, plus générale encore que celle de l’adaptation quantitative et de l’adaptation qualitative dont il a été question plus haut : c’est celle de l’adaptation \emph{négative} et de l’adaptation \emph{positive.} La première n’est que la suppression d’une opposition, et ses degrés sont marqués par l’adoucissement graduel de la lutte entre les termes opposés, comme lorsque, par l’abaissement du prix d’un article, la résistance qu’il rencontre de la part des autres besoins qui se disputent notre revenu va s’affaiblissant. Quand l’utilité \emph{gratuite}, pour employer les formules de Bastiat, s’est substituée finalement à l’\emph{utilité onéreuse}, l’adaptation négative est consommée à cet égard. L’opposition peut d’ailleurs être supprimée par deux moyens différents : 1\textsuperscript{o} par la séparation complète des termes opposés ; c’est ce que fait le Droit en délimitant nettement le domaine des activités individuelles ; c’est aussi ce qu’on obtient dans les rapports internationaux par des barrières protectionnistes qui mettent fin à la concurrence de deux groupes d’industriels nationaux et étrangers ; 2\textsuperscript{o} par la victoire définitive de l’un des termes sur l’autre ou sur les autres ; c’est le cas d’un industriel triomphant qui monopolise une industrie\footnote{ \noindent Le problème posé par les oppositions \emph{rythmiques} n’est susceptible que d’une solution : \emph{une adaptation négative unilatérale.} Mais celle-ci comporte deux formes. L’alternative périodique de la hausse et de la baisse des fonds publics par exemple, peut se transformer en : 1\textsuperscript{o} la hausse seulement, mais de plus en plus ralentie et aboutissant à un \emph{plateau} indéfini (quant à la hausse continue et indéfinie, elle est impossible) ; 2\textsuperscript{o} ou bien la baisse seulement, jusqu’à un certain \emph{bas-niveau} au-dessous duquel elle ne descend pas et où elle se maintient. En fait, dans une foule de cas, l’une ou l’autre de ces deux solutions est réalisée. Toutes les nouvautés sociales commencent par des alternatives de succès et de déclin, par des oscillations de hausse et de baisse de leur valeur, jusqu’au moment où décidément le penchant ascendant ou le penchant déclinant l’emporte et ne tarde pas à se fixer en une opinion \emph{définitive.} Bien entendu, ce \emph{définitif} ne dure jamais qu’un temps, il vient toujours un moment où une \emph{nouvauté nouvelle} apparaît qui expulse l’ancienne ou lui prête une valeur différente et modifie l’opinion sur son compte. Mais, \emph{tant que} cette innovation imprévue n’apparaît pas, le \emph{prix reste fixe, l’opinion reste immuable.}
 }.  \phantomsection
\label{v2p222}Mais de ces deux moyens d’empêcher deux termes d’être opposés, un seul consiste à les adapter négativement l’un à l’autre : c’est le premier. Quant au second, il consiste à supprimer l’un des deux termes et non à adapter l’un à l’autre.\par
Le droit est le grand bornage des appétits et des intérêts rivaux, le grand procédé humain de l’harmonie négative, parce qu’il circonscrit le champ des activités, la proie des avidités, et dresse entre elles des barrières jugées presque unanimement inviolables. — On sait qu’il commence par être conçu comme un \emph{privilège}, — sorte de droit unilatéral par lequel un homme est défendu contre tous les autres sans que les autres le soient contre lui — avant d’être conçu comme une règle de \emph{justice}, sorte de privilège réciproque. — Qui dit droit dit \emph{propriété}, propriété de terres ou de récoltes, d’outillage ou de produits, de capitaux ou de revenus ; qui dit droit individuel dit propriété individuelle de ces choses ou de quelques-unes de ces choses. Le premier qui, rompant l’indivision du groupe primitif, famille ou village, a enclos un terrain d’une haie et se l’est approprié, a pu être un usurpateur, mais a été certainement un grand pacificateur sans le savoir. La garantie, d’abord unilatérale, dont il s’est entouré, et qui a tari déjà bien des sources de querelles antérieures entre lui et ses proches, a été imitée autour de lui, et, en se propageant, s’est mutualisée. En se morcelant, en effet, la propriété des terres, et aussi des capitaux, devient une harmonie négative de moins en moins unilatérale, de plus en plus réciproque, des intérêts rivaux. Il en est de la propriété comme du  \phantomsection
\label{v2p223}protectionnisme, cet autre procédé d’adaptation négative des intérêts nationaux et non plus individuels. Quand une nation se hérisse de douanes, c’est comme lorsqu’un chef de tribu s’enclôt d’une haie ou d’un mur. A mesure que son exemple est suivi, et il l’est toujours, la protection ainsi produite d’unilatérale devient réciproque. Supprimer la propriété privée, ce serait donc rendre la contradiction, l’hostilité des intérêts privés plus violente que jamais ; et il est surprenant de voir cette suppression préconisée par des ennemis de la concurrence. On se précipiterait, faute de barrières protectrices, vers la porte des jouissances indivises, c’est-à-dire \emph{à diviser}. Aussi ne peut-il être sérieusement question de détruire toutes les formes de la propriété privée, mais seulement quelques-unes d’entre elles, celle des biens fonciers, par exemple, ou des mines, ou des machines. Cela mérite d’être discuté ; d’autant mieux que ce qu’il y a d’harmonisant dans la propriété, c’est non pas le fait même de son établissement, mais la croyance générale en sa légitimité, d’où résulte le respect et la résignation qu’elle impose aux déshérités. Dès que, chez le propriétaire ou le capitaliste, la foi en son droit vient à s’affaiblir, ou, chez le prolétaire, la reconnaissance de ce droit, le pouvoir pacifiant de la propriété diminue, et les désirs contraires que cette foi unanime tenait enchaînés s’apprêtent à se jeter les uns sur les autres comme des dogues affamés. Cela signifie que les temps sont proches d’une transformation du principe de la propriété, mais nullement de sa destruction.
\subsubsection[{V}]{V}
\noindent Parlons maintenant de l’\emph{adaptation positive.} Il ne s’agit plus ici seulement d’empêcher deux croyances de se contre-dire, deux désirs de se contrarier ; il s’agit encore de faire en sorte qu’ils se confirment ou qu’ils s’aident. Disons d’abord  \phantomsection
\label{v2p224}qu’il y a entre les conditions de l’adaptation négative et celles de l’adaptation positive, entre la \emph{paix sociale} et la \emph{force sociale}, une opposition qu’il s’agit de résoudre. L’équilibre des intérêts et des droits ne saurait être stable et prolongé sans faire obstacle au progrès de la collaboration des activités, des ambitions, des espérances. Cet \emph{équilibre} et ce \emph{maximum} des énergies sociales ne sauraient être poursuivis qu’alternativement. On ne peut combiner les énergies pour les féconder qu’en les mobilisant ; on ne peut les mobiliser qu’en ébranlant la stabilité de leur équilibre antérieur.\par
Cela dit, remarquons que la loi du passage de l’unilatéral au réciproque s’applique à l’adaptation positive comme à l’autre, et d’une manière plus saisissable. L’esclavage, par exemple, est la première forme, unilatérale, de l’harmonie positive entre les travaux et les besoins, entre les travaux de l’esclave et les besoins du maître, sans réciprocité ; et c’est par degrés qu’on passe de cette forme, à travers des phases intermédiaires telles que le servage, à la mutuelle assistance des travailleurs modernes. — La mutuelle assistance n’est d’ailleurs qu’une \emph{collaboration} inconsciente, embryonnaire, imparfaite, de laquelle on s’élève, par degrés aussi, à la collaboration proprement dite, voulue, achevée et consciente. Le boulanger assiste le cordonnier et réciproquement ; chacun d’eux a son besoin distinct, que l’autre satisfait ; et, s’ils collaborent ensemble à la vie municipale ou à la vie nationale, ou à la société internationale même, qui résulte de cet échange, ils ne s’en doutent guère ou ne s’en inquiètent pas. Mais, dans un monastère bénédictin du haut moyen âge, les moines boulangers et les moines cordonniers, et pareillement les moines tisserands, forgerons, menuisiers, maçons, avaient conscience d’être des collaborateurs à la même œuvre de défrichement, de conquête spirituelle et matérielle d’une contrée, de salut posthume. C’est la vertu de l’association proprement dite de transformer partout en collaboration manifeste l’assistance mutuelle ;  \phantomsection
\label{v2p225}et, dans une nation bien unie, le patriotisme, même en temps de paix, et non pas seulement aux heures du péril commun, fait aussi ce miracle d’orienter en haut, vers une même fin supérieure, toutes les actions d’un grand pays, si dissemblables qu’elles puissent être.\par
Comment s’opère l’harmonisation soit négative, soit positive, dans le monde économique ? Le grand agent de cette opération séculaire, nous l’avons dit, c’est l’Invention. Il ne lui suffit pas d’être l’unique source des adaptations positives, elle contribue aussi, pour une large part, aux adaptations négatives. Par exemple, en même temps qu’elle fait s’harmoniser positivement, en les co-adaptant à un même produit, des travaux auparavant étrangers les uns aux autres, elle établit un accord négatif, par l’abaissement du prix de ce produit, entre le besoin auquel il répond et d’autres besoins qui étaient auparavant inconciliables avec celui-ci pour un grand nombre de petites fortunes. D’autre part, elle fait surgir de nouvelles fortunes, qui accroissent le nombre des participants au bienfait de la propriété. — Cependant, à côté de l’Invention, — à moins d’étendre le sens de ce mot jusqu’à y comprendre toute initiative, toute innovation individuelle — il y a place pour un autre agent, secondaire mais complémentaire, dans l’élaboration logique et téléologique de la société. On peut l’appeler, si l’on veut, la \emph{critique}. L’invention est une synthèse ; la critique est une analyse de la raison raisonnante qui démêle les contradictions cachées des éléments de l’état social à chaque époque, les met en relief, les fait sentir de proche en proche et propage ainsi un mécontentement, une irritation, une indignation, parfois une fureur révolutionnaire. Une révolution, quand elle éclate, est toujours la résultante d’une coalition d’esprits critiques qui ont exercé leur sagacité à démêler, à étaler, tout ce que les institutions de leur temps recèlent d’absurde, de contradictoire et de mensonger, de contraire à l’idée que leur temps lui-même se fait de la vérité et de la  \phantomsection
\label{v2p226}justice. Ils travaillent ainsi constamment à ébranler, pour l’épurer, la société où ils vivent et dont ils vivent, pendant que d’autres esprits plus recueillis et moins bruyants, des inventeurs, aperçoivent le secours que peuvent se prêter les intérêts les plus contraires, la confirmation que peuvent se fournir les idées les plus contradictoires, et s’efforcent de consolider, de rapprocher du vrai et du juste, l’état social au milieu duquel le plus souvent ils s’isolent. Si dissemblables qu’ils soient, les inventeurs et les révolutionnaires n’en ont pas moins une passion commune pour la logique sociale, dont ils appellent le règne, et dont ils préparent, avec une très inégale efficacité, l’avènement.\par
\emph{Invention} et \emph{Révolution :} voilà donc en deux mots les deux sources de ces harmonies économiques que chantait Bastiat. Il serait curieux de comparer leur rôle dans l’œuvre du progrès humain accompli jusqu’à ce jour. Si les conséquences prolongées, indéfiniment croissantes, des grandes inventions, telles que l’écriture et l’imprimerie, la navigation à voiles et la navigation à vapeur, la boussole, la locomotive, le télégraphe, étaient mises en parallèle avec les effets, rapidement ou lentement décroissants, des plus grandes révolutions, voire même de la Révolution française, on serait stupéfait de l’ignorance et de l’ingratitude des hommes qui célèbrent par tant de fêtes commémoratives les grandes journées révolutionnaires, et n’ont jamais l’idée de fêter l’anniversaire des découvertes qui ont régénéré l’humanité. La subordination des révolutions aux inventions apparaîtra mieux si l’on observe que, sans celles-ci, celles-là n’auraient pas lieu ; qu’en effet les luttes de classes, dont les révolutions sont le dénouement, ont pour cause l’enrichissement et l’élévation d’une classe par l’exploitation de nouvelles inventions industrielles qu’elle s’est appropriées et qui l’ont mise en conflit soit avec une classe supérieure, aristocratique, forte du monopole ancien d’inventions militaires et politiques, soit avec une classe inférieure, plébéienne, \phantomsection
\label{v2p227} où le rayonnement imitatif du confort ou du luxe de la bourgeoisie a suscité le besoin de participer à l’appropriation des inventions qui l’ont élevée et enrichie. Par le fait même qu’elle est une nouvelle harmonie d’intérêts qu’elle groupe ensemble, une invention crée des antagonismes nouveaux entre ce groupe et quelques-uns des groupes déjà formés, comme toute affirmation nouvelle apporte avec elle de nouvelles négations ; et l’on dit parfois qu’elle tend ainsi à \emph{révolutionner} l’état social. Mais cela n’est pas tout à fait exact ; car les oppositions nouvelles nées des inventions appellent d’autres inventions, plus complexes, qui les résoudront souvent en adaptations positives et dispenseront des crises révolutionnaires, harmonisations simplement négatives. On ne saurait voir, en général, dans les transformations législatives, où les révolutions viennent aboutir et se faire consacrer, que des opérations purement critiques de la dialectique sociale, des plantations de nouvelles bornes entre des activités naguère opposées, maintenant juxtaposées, mais nullement alliées. Toute révolution, en somme, tend et aboutit à une transformation et le plus souvent à un simple déplacement de la propriété, à une expropriation légale. J’aurais donc pu écrire plus haut : \emph{invention} et \emph{législation}, aussi bien que : invention et révolution ; et cette antithèse aurait eu à peu près le même sens. Les législateurs, même ceux dont on vante le plus le génie, n’inventent guère rien de durable. La plupart de leurs idées les plus géniales, comparables à celles d’un général pendant la bataille, sont des manœuvres de tactique parlementaire, des inventions auxiliaires et passagères, destinées à ne pas être imitées, — à la différence des inventions véritables dont la destination essentielle est l’imitation qui les déploie et les perpétue. Une révolution qui serait vraiment créatrice d’un ordre nouveau, et non pas seulement éliminatrice d’un désordre ancien, une révolution et une législation qui feraient l’harmonie profonde et générale des  \phantomsection
\label{v2p228}intérêts, non l’immolation des uns aux autres, seraient dans le monde une extraordinaire nouveauté.\par
Remontons à la source psychologique de la distinction développée ci-dessus. Il est, parmi les individus isolés, sauvages, qui vivent à part de la société, deux sortes d’esprits bien distincts : les esprits avisés, ingénieux, habiles à tirer parti, en toute occasion, de ce qu’ils ont sous la main pour obtenir le maximum d’effet avec le minimum d’effort ; et les esprits judicieux, classificateurs, rangeurs, qui bouleversent tout parfois pour tout réarranger. Ces deux natures d’esprit prédominent tour à tour, côte à côte, dans une société, et, suivant que les initiateurs inventifs ou les initiateurs critiques y donnent le ton, on y remarque une exaltation ou une équilibration des forces sociales, un élan vers la prospérité et le progrès ou un effort vers la justice. Progrès social et justice sociale s’entravent souvent dans leur marche alternative, de même qu’on voit, dans un même individu, l’ingéniosité rarement unie à l’esprit d’ordre et la fécondité artistique à la moralité sévère. L’idée de moralité ne diffère de l’idée de justice qu’en ce que la première présente de préférence sous l’aspect individuel, et la seconde sous l’aspect social, la réalisation d’un accord négatif entre des désirs concurrents et limitrophes. Et, pareillement, l’ingéniosité ne diffère du génie qu’en ce que les inventions minuscules nées de la première meurent le plus souvent dans l’individu qui est leur berceau, tandis que la seconde donne naissance d’ordinaire à des inventions qui se répandent au dehors et, en se généralisant, se socialisent.\par
Il résulte de ce qui précède que l’invention est l’agent initial et demeure toujours l’agent principal de l’adaptation économique, c’est-à-dire de ce qu’on appelle en un terme mal défini l’évolution économique. Nous allons donc commencer par examiner ce grand et obscur sujet. Nous parlerons ensuite de l’adaptation négative, où l’invention ne joue qu’un rôle indirect (le premier rôle appartenant ici aux  \phantomsection
\label{v2p229}révolutions), et qui se résume dans la \emph{propriété}, aux formes et aux transformations sans nombre. Après quoi, nous passerons aux divers degrés et aux formes variées de l’association, qui est, avons-nous dit, l’image agrandie et extériorisée de l’invention, et nous examinerons successivement : d’abord les formes unilatérales de l’association des travaux (esclavage, servage) et des besoins (usage primitif des présents ou des pillages) ; puis les formes réciproques, qu’il est bon de subdiviser : 1\textsuperscript{o} associations spontanées de production (division du travail) et association spontanée de consommation (échange) ; 2\textsuperscript{o} associations réfléchies, proprement dites\footnote{ \noindent Les associations proprement dites dérivent directement de l’invention, car elles se conforment à un plan individuel, qui a dû être imaginé par quelqu’un ; les associations spontanées, n’en dérivent qu’indirectement par le fractionnement entre plusieurs hommes de diverses opérations exécutées d’abord ou conçues par un seul.
 }, soit de production (grande industrie, trusts), soit de consommation (mutualités, coopératives). Les associations monétaires, financières, auront lieu aussi de nous arrêter.\par
Un dernier mot. Toute l’économie politique d’Adam Smith et de son école est fondée sur le postulat de l’accord spontané des égoïsmes : de là les \emph{harmonies économiques} de Bastiat. La question est de savoir si les égoïsmes s’harmonisent d’eux-mêmes ou artificiellement. Cette question est tranchée dans un sens opposé à celui de Smith pour quiconque a embrassé dans son ensemble l’opposition économique, qui nous a montré l’hostilité si fréquente, et si souvent essentielle, radicale, des intérêts. Il s’ensuit que l’harmonisation des intérêts ne peut être obtenue que moyennant des \emph{artifices}. Ces artifices sont les inventions.
 \phantomsection
\label{v2p230}\subsection[{III.2. L’imagination économique}]{III.2. L’imagination économique}\phantomsection
\label{l3ch2}
\subsubsection[{III.2.a. Préliminaires : Second sens du mot valeur (valeur-emploi) : valeur de production et de consommation el valeur d’échange.}]{III.2.a. Préliminaires : Second sens du mot \emph{valeur} (valeur-emploi) : valeur de production et de consommation el valeur d’échange.}
\noindent Dans notre théorie des prix, c’est-à-dire de la valeur-coût, nous avons dit qu’il y avait un autre sens du mot valeur ; et c’est le moment d’en parler, puisque sous cet aspect, qui est le meilleur et l’inverse psychologique du premier, la valeur des choses augmente avec le nombre et la variété des inventions. Si les économistes n’entendent guère la valeur suivant cette acception, parce qu’ils préfèrent le mot utilité, le langage courant lui fait sa large part. On dit couramment, et justement, que la valeur d’une idée s’accroît quand ses applications se multiplient ; on dit de même que la valeur d’une machine augmente quand elle trouve de nouveaux emplois. Un député dira : « Cette profession de foi m’a \emph{valu} toutes les voix de telle commune. » Ici \emph{valu} ne veut pas dire \emph{coûté}, mais précisément le contraire. On dira d’une jeune fille qui s’est mariée à la suite d’un bal : « ce quadrille lui a valu son mariage. » Si ce bal, au contraire, avait eu pour effet de faire rompre son projet d’union, on dirait : « cette danse lui a \emph{coûté} son mariage. » En somme, une chose vaut : 1\textsuperscript{o} ce que coûte son acquisition, le sacrifice que son acquisition exige ; 2\textsuperscript{o} ce qu’elle permet d’acquérir par l’usage qu’on en fait ou qu’on peut en faire, le service qu’on obtient d’elle ou qu’on en attend. Toute chose a ainsi une \emph{valeur-coût} et une \emph{valeur-emploi}, un prix et une utilité, et les deux, quoiqu’intimement liées, progressent d’un pas très inégal, souvent même en raison inverse l’une de  \phantomsection
\label{v2p231}l’autre. C’est ce que Bastiat entendait dire quand il opposait l’utilité gratuite à l’utilité onéreuse. Il y a beaucoup de vérité dans ses aperçus à ce sujet. Si son optimisme en cela ne l’a pas abusé, c’est qu’ici le résultat qu’il proclame providentiel, à savoir le développement incessant de l’utilité gratuite aux dépens de l’utilité onéreuse — ou de la valeur-emploi aux dépens de la valeur-coût — est l’effet nécessaire de la Logique, individuelle ou collective, qui travaille sans cesse à arranger et réarranger les multitudes d’idées et de besoins, de jugements et de desseins, de croyances et de désirs, juxtaposés pêle-mêle dans la société ou dans chaque esprit, de manière à ce qu’ils se contredisent ou se contrarient de moins en moins et se confirment ou s’entr’aident de plus en plus. A chaque degré nouveau de cette élaboration harmonisatrice, c’est-à-dire à chaque invention qui permet d’utiliser d’une nouvelle manière, soit en produisant, soit en consommant, un produit antérieur, l’\emph{emploi} de ce produit augmente et son \emph{coût} diminue ; et cela signifie que sa \emph{valeur de production} et sa \emph{valeur de consommation} (deux aspects complémentaires de son emploi) varient pour ainsi dire en raison inverse de sa \emph{valeur d’échange.} Ces trois sens de la valeur doivent être nettement distingués\footnote{ \noindent Je me permets de renvoyer le lecteur pour compléter ce sujet à ma \emph{Logique sociale}, p. 378-382.
 }.\par
Les principaux procédés employés par la Logique sociale pour avancer dans la voie de l’harmonisation positive des intérêts sont, avons-nous dit, l’\emph{échange} et l’\emph{association}, mais, avant tout, l’\emph{invention}, qui est à l’origine de tout nouvel échange comme de toute nouvelle association.\par
L’échange, il est vrai (dont la vente et l’achat ne sont qu’une espèce) suppose la concurrence des désirs et des jugements, le duel logique, dont la valeur-coût est le règlement. Mais l’échange n’en a pas moins pour effet de conclure l’alliance de deux désirs et de deux jugements, leur mutuelle  \phantomsection
\label{v2p232}assistance et leur confirmation réciproque. L’échange ainsi a deux faces opposées et juxtaposées. Quand un homme achète une paire de bottines dans un magasin, son achat a été précédé d’un conflit intérieur entre le désir de posséder ces chaussures et le regret de se déposséder du prix ; et le marchand de son côté, a eu à opter, soit au moment même de la vente, soit au moment de la fixation du prix, entre le désir de posséder le prix et le regret de se déposséder des bottines. Mais, quand l’échange a eu lieu, c’est-à-dire quand, chez l’une des deux parties, le désir de posséder les bottines l’a emporté, et, chez l’autre, le désir de posséder le prix, ces deux désirs ont formé un accord logique des plus étroits, puisque chacun d’eux a trouvé dans l’autre un moyen de se satisfaire. D’autre part, en ce qui concerne les jugements impliqués dans cette opération, le client, avant d’acheter, a hésité entre ces deux propositions contradictoires : « ces bottines sont trop chères à ce prix, — ces bottines ne sont pas trop chères, » et la dernière a prévalu ; et le marchand, avant de fixer le prix, est resté quelque temps indécis entre ces deux propositions non moins contradictoires : « à ce prix, ces bottines sont trop bon marché — elles ne sont pas trop bon marché », et la dernière également a eu la préférence. On voit que les deux jugements qui l’ont emporté se confirment entre eux.\par
Un produit, après l’échange, ne vaut pas plus qu’avant l’échange, au sens de valeur-coût. Loin de là, il vaut, en général, beaucoup moins, et, par le fait seul qu’une voiture, par exemple, a été vendue, même sans avoir servi, elle a perdu la moitié de sa valeur vénale. Mais, à l’inverse, un produit, après l’échange, vaut plus qu’avant, au sens de valeur-emploi ; il a passé, en se déplaçant, en des mains qui en feront un meilleur usage\footnote{ \noindent On comprend très bien la peine que se donne Karl Marx pour tâcher de démontrer que l’échange n’ajoute aucune plus-value aux marchandises échangées. Il suffit, en effet, de ce fait pour réduire à néant sa théorie de la valeur. Comment, si un produit vaut seulement en raison de la quantité de travail qu’il incarne, le déplacement de ce produit, son passage d’une main qui la désire moins à une main qui la désire plus, pourrait-il accroître sa valeur ?
 }.\par
 \phantomsection
\label{v2p233}Le bénéfice social de l’échange, remarquons-le, au point de vue de l’harmonisation des intérêts et des pensées, est d’autant plus grand que l’échange est plus équitable, que le prix réel s’est plus rapproché du juste prix. Si un fournisseur exploite le besoin personnel ou momentané que j’ai de sa marchandise pour me la faire payer deux ou trois fois plus qu’elle ne vaut pour la moyenne des gens, mon regret de l’argent dont je me déposséderai sera bien plus fort, et mon désir final d’acheter en sera amoindri. Il est vrai que le désir de vendre sera augmenté chez le marchand, mais y aura-t-il compensation ? Non, et voici pourquoi. Il résulte, comme on sait, d’un théorème de Laplace que, à fortune égale, le plaisir d’un gain ne saurait égaler le chagrin d’une perte équivalente. Cela est clair, s’il s’agit d’une grosse somme, de 20 000 francs par exemple gagnés ou perdus sur une fortune de 40 000 francs. Celui qui perd voit sa fortune réduite de \emph{moitié ;} celui qui gagne voit la sienne grossie du \emph{tiers} seulement. Or, au degré près, si petite que soit la somme, le même phénomène se produit. Puis, toute diminution de ressources, en revenu ou en capital, est sentie comme la privation de jouissances connues, habituelles et chères, et toute augmentation, comme la perspective de plaisirs nouveaux, indéterminés et relativement indifférents puisqu’on s’en passait jusque-là. Par suite, il est faux de dire que, dans une catastrophe financière, l’enrichissement des uns fait compensation à la ruine des autres, et que l’intérêt général n’a pas à s’en inquiéter. Il n’est pas plus vrai de penser qu’il n’a rien à voir dans la répartition équitable ou léonine des avantages de l’échange dans les affaires de tout genre. La justice, en somme, est la grande voie séculaire de l’utilité sociale en progrès\footnote{ \noindent Le passage qui précède est emprunté à ma \emph{Logique sociale.}
 }.\par
 \phantomsection
\label{v2p234}Il est désirable, en outre, au point de vue de l’harmonisation croissante des désirs et des jugements économiques, du plus grand bien de tous, que le juste prix devienne un prix de plus en plus abaissé\footnote{ \noindent M. Hector Denis, dans son \emph{Histoire des prix}, si fortement pensée et documentée, préconise la hausse des prix ; et il faut lui accorder qu’elle présente des avantages temporaires, notamment comme condition de la hausse des salaires. Mais celle-ci même ne saurait se poursuivre indéfiniment, et, si elle coïncide avec la hausse de tous les produits, à quoi sert-elle ? Il est désirable que, à salaire égal, le travailleur se procure une quantité croissante de produits, grâce à leur dépréciation croissante.
 }. Il en est ainsi pour deux raisons : premièrement, parce que cet abaissement rend l’article coté plus bas accessible à un plus grand nombre de bourses et multiplie d’autant les échanges, dont chacun est un bien social nouveau ; secondement, parce que dans le cœur de chacun de ceux qui, au prix ancien, auraient encore acheté, le prix nouveau a pour effet de rendre plus inégal encore le duel intérieur des désirs et des jugements et plus facile le sacrifice du désir ou du jugement vaincu. Sans doute, chez le vendeur, à l’inverse, le plaisir de vendre, à ce prix amoindri, s’est affaibli ; mais il s’est multiplié. Somme toute, il y a eu gain pour la société.
\subsubsection[{III.2.b. Conditions physiologiques, psychologiques et sociales de l’invention. Ribot, Faulhan. Son caractère déductif.}]{III.2.b. Conditions physiologiques, psychologiques et sociales de l’invention. Ribot, Faulhan. Son caractère déductif.}
\noindent Après ces quelques mots sur la \emph{valeur-emploi}, essayons de formuler quelques considérations générales, si périlleux qu’il soit de généraliser ici, sur les conditions psychologiques ou extérieures de l’invention, sur ses différentes classes, sur ses principales conséquences sociales, et enfin sur la direction apparente de ses déroulements à travers des méandres infinis.\par
Sur les conditions physiologiques de l’invention, nous ne savons pas grand’chose de net, si ce n’est que ce travail de création cérébrale congestionne plus ou moins le cerveau. « Pouls petit, contracté, la peau pâle, froide, la tête bouillante, \phantomsection
\label{v2p235} les yeux brillants, injectés, égarés : telle est la description classique » de cet état, d’après Ribot. Il semble qu’il existe quelques rapports entre la fonction génératrice et l’imagination créatrice. Si l’on admet, comme je suis tenté de le faire, que la génération est une sorte d’invention vitale, cette corrélation n’aura rien de surprenant. Il est certain que, d’après les confidences de certains inventeurs, l’apogée de l’inventivité correspond chez eux, à l’âge du plus haut point de force sexuelle. De vingt-cinq à trente-cinq ans surgissent en eux les conceptions que le reste de leur vie développera, rectifiera, enluminera\footnote{ \noindent Un inventeur, correspondant de Ribot, lui écrit : « Mon imagination a été la plus intense vers l’âge de vingt-cinq à trente-cinq ans environ (j’ai actuellement quarante-trois ans). Après cette période, il semble que le reste de l’existence ne sert qu’à produire les conceptions moins importantes, formant la suite naturelle des conceptions principales, nées pendant la période de jeunesse. »
 }. Mais cette règle, qui n’est jamais sans exception, s’applique surtout aux inventions esthétiques, et moins aux inventions industrielles qu’à toutes les autres. En ce qui concerne ces dernières, la précocité de leurs auteurs est remarquable ainsi que le prolongement de leur fécondité jusqu’à la vieillesse. Dès l’adolescence, dès l’enfance même, se révèlent leurs aptitudes spéciales par la construction de petites machines, d’une ingéniosité souvent frappante. Il n’est pas de cour de collège qui ne possède son petit mécanicien en herbe, son ingénieur-né, constructeur d’instinct comme le castor.\par
La psychologie de l’homme d’imagination, en tout ordre d’idées, a été très bien tracée par Ribot\footnote{ \noindent \emph{Essai sur l’Imagination créatrice} (1900). — Paulhan, \emph{Psychologie de l’Invention} (1901). — Voir aussi : Séailles, \emph{Essai sur le génie dans l’art ;} Souriau, \emph{Théorie de l’invention.}
 }, et aussi, à un point de vue moins étendu, par Paulhan, qui insiste avant tout sur la logique et la téléologie propres de l’inventeur. Cela s’accorde, au fond, avec l’idée de Ribot, que l’imaginatif n’est pas seulement un rêveur, mais un passionné, dont l’idée fixe s’alimente d’un sentiment fixe. Il n’y a pas,  \phantomsection
\label{v2p236}d’après lui, d’invention qui n’implique un élément émotionnel, un désir, et il n’est pas d’émotion — peur ou colère, tristesse ou joie, haine ou amour, etc., — qui ne puisse être un ferment d’invention. Cette préoccupation tenace nous explique la distraction profonde du génie, sa « finalité interne » qui, lorsqu’elle se rencontre avec un « hasard extérieur » qui la favorise, suivant une bonne définition de M. Souriau, devient féconde. Voilà pourquoi, dans sa rêverie voisine de l’hallucination, les images tendent à devenir les \emph{états forts} et les sensations les \emph{états faibles}, à l’inverse de l’état normal. Cette tendance contre nature peut se réaliser plus ou moins ; son terme dernier, c’est la folie, qui se trouve être ainsi, non pas la condition du génie, comme on l’a conjecturé à la légère, non pas même l’accompagnement habituel du génie, mais sa maladie accidentelle et sa chute dans l’excès qui l’anéantit. Il ne faut pas confondre non plus avec la mégalomanie de l’aliéné la foi en sa mission qui est nécessaire à l’inventeur pour le soutenir dans ses luttes, dans son dévouement à son idée.\par
Quand le moi s’absorbe longtemps en un but, il est rare que le \emph{sous-moi}, ce qu’on appelle abusivement l’inconscient, ne participe pas à cette obsession, ne conspire pas avec notre conscience, ne collabore pas à notre travail mental. Cette conspiration, cette collaboration d’une domesticité fidèle et cachée que nous portons en nous, c’est l’\emph{inspiration}, qui, toute mythologie écartée, ne demeure pas moins mystérieuse. Sans l’activité continuelle de cette ruche intérieure, de cette foule de petites consciences (peut-être) auxiliaires et servantes de la nôtre, il n’est pas, à vrai dire, un seul phénomène de la vie intellectuelle la plus ordinaire, une seule association d’idées renaissantes à propos, remémorées au moment voulu comme par un bibliothécaire invisible et infiniment diligent, qui puisse être expliquée. Mais il est des cas où ce secours que le moi reçoit du sous-moi nous saisit d’étonnement par son abondance ou par l’importance \phantomsection
\label{v2p237} de ses résultats, ou par la solution soudaine qu’il apporte à des problèmes agités en vain pendant des jours et des mois, et les poètes, en cas pareil, se disent \emph{inspirés.} Les savants et les ingénieurs auraient aussi bien le droit de le dire. — Et c’est là une des raisons pour lesquelles le travail de l’inventeur ne saurait être confondu avec le travail proprement dit, avec le travail du travailleur, où il n’entre aucune collaboration notable de l’inconscient.\par
Dans son énumération des diverses formes ou des diverses voies de l’imagination poétique, artistique, mythologique, scientifique, morale, politique, économique, etc., qu’il étudie successivement, M. Ribot subdivise avec raison l’\emph{imagination économique} en deux branches : l’imagination industrielle et l’imagination commerciale. Nous y reviendrons. Je ne lui reprocherai que d’avoir omis dans son casier, d’ailleurs si substantiellement rempli, une case importante, l’\emph{imagination linguistique}, source de l’invention verbale. N’a-t-elle pas dû être la première grande manifestation du génie humain ? Elle est née, comme l’invention mythique, de la logique de l’analogie. Le besoin de personnification s’est fait jour, dans la parole, par la \emph{phrase} où tout sujet est conçu comme un moi, et, dans la pensée, par l’\emph{animisme.} Dans les littératures, les métaphores, les « bonheurs d’expression », procèdent de cette même logique analogique, et doivent compter parmi les inventions (à la fois esthétiques et pratiques) les plus réussies, les plus imitées qu’il y ait. Quelle prodigieuse dépense d’invention verbale suppose la création d’une langue, ou simplement d’un dialecte ! Et, si l’on songe à la diversité infinie, à la multiplicité innombrable des idiomes primitifs, dont le nombre n’a fait que décroître, on ne doutera pas que la parole ait été le premier exercice imaginatif de l’homme préhistorique, en réponse au plus urgent de ses besoins, celui de l’action inter-mentale. Dans une certaine mesure, en effet, les diverses espèces linguistique, mythologique, poétique, scientifique, industrielle, \phantomsection
\label{v2p238} etc., de l’imagination, semblent avoir atteint l’une après l’autre, le point de leur floraison la plus éclatante, quoiqu’elles aient de tout temps coexisté, et cette série à certains égards est irréversible. Il en est ainsi au cours de la vie de l’individu lui-même, quand, ce qui est assez fréquent, il change d’aptitude en passant d’un âge à l’autre. Ribot cite comme une exception extraordinaire l’exemple suivant. « Je peux citer, dit-il, le cas d’un savant connu qui a débuté par le goût des arts (surtout plastiques), a traversé rapidement la littérature, a consacré sa vie aux sciences biologiques où il s’est fait une réputation méritée, puis, sur le retour, s’est dégoûté totalement des recherches scientifiques pour revenir à la littérature et finalement aux arts qui l’ont repris tout entier. »\par
— Les conditions sociales de l’invention sont bien plus faciles à préciser que ses conditions industrielles. Et nous les avons souvent mises en lumière. C’est à ce point de vue, notamment, que bien des inégalités de classe et des injustices du sort reçoivent une certaine justification. Car il est des classes, comme des nations, bien plus inventives que d’autres, et dont la supériorité s’explique ainsi. L’invention étant fille du loisir et de l’étude, c’est dans les classes aisées, dans les professions libérales, qu’éclosent en général les idées destinées à révolutionner les métiers manuels, et à élever le niveau des classes populaires. C’est une erreur d’attribuer aux ouvriers une grande part dans l’invention des machines, ou du moins dans leur idée-mère. Notre Académie des sciences a beaucoup contribué à leur conception à la fois et à leur vulgarisation. Au {\scshape xviii}\textsuperscript{e} siècle, c’est elle que l’on consulte à propos de toute invention, et ses plus grands savants, tels que Réaumur, sont des inventeurs. Réaumur fit faire des progrès à la métallurgie, à la faïencerie, etc. Vaucauson, autre académicien, n’est connu que par son canard automatique ; il n’en a pas moins inventé les machines les plus utiles, qui ont fait progresser le tissage de  \phantomsection
\label{v2p239}la soie. — La part des nations dans l’inventivité n’est pas moins inégale que celle des classes, et s’explique aussi par des considérations plutôt sociales que biologiques, par des influences d’éducation plutôt que de race. Et les nations les plus inventives ne sont pas toujours celles qui profitent le plus de leurs bonnes idées. On le voit par les Expositions universelles, qui ont cela de particulier, qu’elles établissent entre les États une comparaison au point de vue de l’invention industrielle beaucoup plus qu’au point de vue de la prospérité industrielle, c’est-à-dire de la propagation imitative des inventions. Aussi le classement des nations, dans ces grands concours internationaux, est-il bien différent de celui que suggère la lecture des statistiques de l’industrie et du commerce.\par
L’influence du milieu social a été souvent exagérée et entendue en ce sens abusif que l’homme de génie est une simple résultante des masses, un être purement représentatif de son milieu. S’il en était ainsi, comme le remarque Ribot, il ne commencerait pas toujours par susciter dans son milieu tant d’oppositions. Mais c’est son milieu qui lui apporte tous les éléments dont son invention n’est que le croisement original. Prenez une machine quelconque, vous n’y trouverez que des machines plus simples, des ressorts et des forces qui, séparément, étaient connus et répandus avant elle. De ces inventions antérieures, propagées par l’exemple, et se rencontrant dans un cerveau spécial, la conception nouvelle est éclose. Toute invention, quelle qu’elle soit, théorique ou pratique, n’est qu’une combinaison d’imitations. Mais c’est la nature de cette combinaison qui reste à expliquer. D’abord, cette synthèse suppose une analyse préalable, une abstraction qui a dissocié les éléments des inventions anciennes et y a entrevu la possibilité de nouvelles associations. Et cette dissociation, comme cette association, s’opère par un besoin intense de finalité ou de logique, suivant qu’il s’agit d’une invention pratique ou  \phantomsection
\label{v2p240}théorique. Dans l’idée d’un nouveau type architectural, le type ogival par exemple, il n’y a que des figures déjà vulgarisées, mais leur rapprochement pour la première fois n’est pas une simple rencontre cérébrale ; cette rencontre a suscité l’idée de leur adaptation commune à un but qui, pour la première fois, s’est trouvé réalisé de la sorte : ce but, en général préexistant, souvent très ancien, c’est, par exemple, dans le cas du style ogival, d’exprimer avec force certains élans ou certains frissons du cœur chrétien, des espérances et des terreurs caractéristiques. Tout concourt à cette fin dans l’architecture gothique, en même temps qu’elle répond au besoin de rassembler des foules plus vastes autour de la chaire ou de l’autel. — L’invention du moulin à vent, ne contenait, elle aussi, que des éléments usités depuis des siècles : meules, force motrice des vents, tour, toiture mobile... L’ingéniosité a été de faire concourir ces choses à la réalisation meilleure d’un vœu bien antique : moudre la farine avec le moins de fatigue possible. Vues sous cet aspect, rassemblées ainsi, ces choses, qui paraissaient auparavant étrangères les unes aux autres, se présentent comme collaboratrices. — De même, quand Huyghens a eu l’idée de la théorie ondulatoire de la lumière, les phénomènes du rayonnement lumineux et ceux de la propagation des ondes sonores ou des rides formées par le vent à la surface d’un lac, ont été aperçus par lui comme des conséquences d’un même principe, des applications d’une même formule, après avoir passé jusque-là pour être sans rapports entre eux.\par
La conception inventive peut être, on le voit, utilitaire ou désintéressée, c’est-à-dire avoir pour objet un rapport de moyen à fin ou de conséquence à principe (d’espèce à genre) ; ce qui ne l’empêche pas, dans les deux cas, dans le premier aussi bien que dans le second, d’être un fait purement intellectuel, un raisonnement, j’ajoute un raisonnement déductif, provoqué, il est vrai, par un ferment de passion, de désir spécial, que cette conception satisfait, mais  \phantomsection
\label{v2p241}qui n’a rien de commun avec les désirs tout autres que cette conception, quand elle est de nature utilitaire, est destinée à satisfaire. — Autre chose est la conception de l’invention industrielle, autre chose la volonté de la réaliser. Je remarque à ce propos la comparaison établie par M. Ribot, et aussi par M. Paulhan, entre l’\emph{invention} et la \emph{volition}. « L’imagination, dit le premier, est dans l’ordre intellectuel ce que la volonté est dans l’ordre des mouvements. » L’imagination créatrice a ses avortements comparables aux impuissances du vouloir. Les rêveries sont l’équivalent des aboulies. L’imprévu est le caractère commun de l’invention et de la décision volontaire. A l’une comme à l’autre il est essentiel de ne pouvoir être prédites d’avance. Peut-être ces analogies, d’ailleurs frappantes, vont-elles prendre une autre signification si nous faisons observer que dans tout acte de volonté il y a, au fond, une invention petite ou grande, un plan plus ou moins nouveau (car, sans cela, il y aurait habitude et automatisme) dont il n’est que l’exécution. Tout acte de volonté est précédé d’un syllogisme téléologique qui a conclu à un \emph{devoir d’action}. Or, est-ce qu’un syllogisme, un raisonnement quelconque, n’implique pas toujours une invention, et est-ce qu’une invention n’implique pas un syllogisme, implicite ou formulé ? De deux prémisses rapprochées, qui sont deux propositions, affirmatives ou négatives, je déduis une troisième proposition qui exprime un nouveau \emph{devoir d’affirmation :} voilà le syllogisme logique, scolastique, ordinaire. De deux prémisses rapprochées, dont l’une exprime une volition, un but, et dont l’autre exprime un moyen adapté à ce but, je déduis un \emph{devoir d’action} qui consiste à mettre en œuvre ce moyen : voilà le syllogisme de finalité.\par
Je sais bien qu’il est de tradition dans les écoles que le syllogisme n’est pas un procédé de découverte, qu’il n’est qu’une méthode de vérification, tandis que les fameux \emph{canons} de l’induction seraient les seules et les véritables  \phantomsection
\label{v2p242}voies de l’invention. Mais c’est précisément le contraire qui me paraît vrai\footnote{ \noindent M. Paulhan a bien vu qu’il existe des rapports entre l’invention et le raisonnement, mais il ne les précise pas et les croit plus superficiels peut-être qu’ils ne sont. « Le développement de l’invention rappelle ici — dit-il à propos des procédés de composition propres à Sardou et à Zola — l’enchaînement des termes du syllogisme. Aussi voyons-nous le raisonnement signalé comme procédé d’invention par M. Sardou, et la déduction être également invoqué dans le cas de M. Zola. » Et il ajoute que, d’après Binet, chez Sardou « le procédé de travail conserve toujours la même nature psychologique, c’est le raisonnement. C’est avec du raisonnement que Sardou conduit une pièce depuis le point de départ jusqu’à l’œuvre complète. » A plus forte raison, ou avec plus de raison apparente, pouvons-nous dire que toute invention industrielle, mécanique, est syllogistique.
 }. Avant toutes recherches expérimentales, il faut d’abord une hypothèse, fondée sur une association de jugements d’où l’on a déduit, à titre de simple possibilité plus ou moins probable, une conclusion conjecturale. C’est ensuite pour faire hausser cette conjecture sur l’échelle des probabilités jusqu’à la certitude, ou, au contraire, pour la dépouiller de toute apparence de vérité, que les méthodes empiriques sont utiles. Elles servent à le vérifier. En syllogisant, à son insu le plus souvent, un Pasteur, un Helmholtz, un Lavoisier — qui commence par « soupçonner » l’existence de l’oxygène avant de la \emph{démontrer} expérimentalement — font acte d’invention. En expérimentant, ils font des \emph{constats} qui démontrent le caractère réel ou imaginaire de l’hypothèse inventée.
\subsubsection[{III.2.c. Côté accidentel des inventions, complément de leur côté logique. Son importance historique.}]{III.2.c. Côté accidentel des inventions, complément de leur côté logique. Son importance historique.}
\noindent Ce caractère déductif qui est essentiel à l’invention explique l’enchaînement des théorèmes ou des lois successivement formulés par une science et des outillages successifs d’une industrie. Mais, pour avoir été déduite, chacune de ces innovations n’en a pas moins été inattendue, impossible à prévoir, et en partie fortuite. Quand, dans une exposition rétrospective, on regarde la série des moyens de locomotion depuis le palanquin jusqu’aux locomotives les plus récentes, comme quand on regarde dans un muséum la  \phantomsection
\label{v2p243}série paléontologique des vertébrés, depuis les premiers rudiments de la marche, de la natation ou du vol jusqu’à leurs perfectionnements supérieurs, on est saisi d’une double impression de logique et de bizarrerie qui est tout le mystère de l’individualité vivante, toujours originale quoique toujours \emph{conclue} de ces deux prémisses rapprochées qui sont ses parents.\par
Même au cours de l’argumentation d’un logicien impeccable, l’éclosion de chaque argument nouveau dans son esprit et sous sa plume est une trouvaille qui cause une surprise. Elle est due à l’insertion spontanée d’un souvenir sur un autre souvenir ; en sorte que le déroulement de la dissertation la plus constamment orientée vers son idée-maîtresse n’est jamais rectiligne et suit un tracé pittoresque. Mais il est très rare qu’il n’y ait ainsi qu’une seule logique en œuvre dans un cerveau, et, à plus forte raison, dans une société, grand cerveau collectif et immensément compréhensif. Les déductions les plus diverses, les plus contradictoires entre elles, s’y avancent par zig-zags, se croisant, se déviant, se confondant parfois puis se séparant de nouveau. Les hypnotiseurs ont remarqué que, entre l’hypnose profonde et le réveil complet de quelques-uns de leurs sujets, il existe un état intermédiaire, une \emph{veille somnambulique\footnote{ \noindent C’est l’expression de Delbœuf. V. à ce sujet la \emph{Revue philosophique} de février 1887.
 }}, pendant laquelle ceux-ci restent suggestibles et accueillent pêle-mêle des suggestions contradictoires superposées en eux et accumulées. Il pourrait bien en être de la raison et de la liberté des gens plongés dans un milieu social intense où ils agissent profondément et à leur insu les uns sur les autres, comme des hypnotisés dont il s’agit. Même au {\scshape xiii}\textsuperscript{e} siècle, à l’époque où le système catholique avait atteint sa perfection et sa pureté relatives, on est surpris de voir tout ce qu’il entrait d’éléments antichrétiens, je ne dis pas seulement non-chrétiens, dans les mœurs, les lois, les institutions. La  \phantomsection
\label{v2p244}facilité avec laquelle une idée en désaccord avec l’orthodoxie se glissait dans les esprits est assez remarquable. L’empereur Frédéric II, — le bras droit de l’Église, après tout — recevait à sa table, ensemble, des évêques et des émirs, et souvent des ecclésiastiques en faisaient autant. Spencer a bien mis en lumière, dans son \emph{Introduction à la Science sociale}, la contradiction entre cette « religion de la haine » et cette « religion de l’amour » qui coexistent depuis la naissance du christianisme en tout pays chrétien. Quand la logique sociale est chose si complexe et passe son temps à se combattre elle-même à coups d’inventions différentes, la mêlée des inventeurs est, naturellement, pleine des péripéties les plus surprenantes.\par
Comme exemple de déductivité capricieuse et logique à la fois, et les deux au plus haut degré, on peut citer la série des modes féminines, qui passe pour le domaine de la fantaisie pure. Comment sont inventés les nouveaux modèles qui, chaque année, sont mis à la mode par une élite d’élégantes et se propagent ensuite de Paris aux grandes villes, de celles-ci aux petites villes et aux campagnes, dans toutes les couches sociales depuis les plus riches jusqu’aux plus pauvres, conformément à la loi générale des exemples ? M. du Maroussem va nous l’apprendre\footnote{ \noindent \emph{Le vêtement à Paris}, t. II.
 }. Cet auteur n’admet pas les peintures faites par certains journaux de « scènes d’inspiration » d’où jaillirait l’invention d’un nouveau type de vêtement, ou plutôt, remarquons-le, d’une nouvelle variante d’un type régnant. Car le but poursuivi, le problème auquel il s’agit de trouver une réponse, est toujours de satisfaire « cette double exigence, en apparence contradictoire, des élégantes : être mise comme personne et être mise comme tout le monde. » Il n’y a là nulle contradiction à vrai dire, ce n’est qu’une difficulté qui consiste « à découvrir une idée originale dans le ton général de la mode (déjà régnante) ». Ce but étant donné, l’on voit que le problème posé est susceptible  \phantomsection
\label{v2p245}de multiples solutions, qu’il s’agit de formuler, de déduire en quelque sorte, et entre lesquelles il faudra choisir. D’abord, grandes tailleuses et grands couturiers font appel aux artistes. « Des dessinateurs et des dessinatrices appartenant aux classes libérales soumettent leurs fines aquarelles : ils ont fait descendre de leurs cadres les dogaresses du {\scshape xv}\textsuperscript{e} siècle, les ligueuses du {\scshape xv}\textsuperscript{e} siècle, les marquises et les incroyables du {\scshape xviii}\textsuperscript{e} siècle, afin de donner par elles à leurs arrière-petites-filles une savante leçon de coquetterie. Puis les \emph{premières d’atelier} ont dû créer la partie technique. » Autrement dit, les inventions sont nées non des tailleurs et tailleuses qui en profitent, mais d’artistes qui ont dû, pour les concevoir, combiner, en vue du but que nous savons, des types de vêtements du passé avec le type actuellement admis. « Toujours, d’ailleurs, les créations se multiplient par l’imitation des créations voisines. Les plagiats constituent la coutume courante des ateliers rivaux. » Il y a, en effet, concours de modèles en projets, qui se disputent le cœur des élégantes. Ce concours public a lieu dans certaines réunions, telles qu’un grand mariage, le Grand-Prix de Longchamps. Le jury, c’est-à-dire un aréopage des plus frivoles, décide ce qu’il lui plaît, et sa décision est docilement acceptée par le monde entier.\par
— L’importance du côté accidentel des inventions a été si souvent méconnue, parce qu’on a cru à tort y voir la négation de leur caractère logique\footnote{ \noindent Nous trouvons dans un passage de Bacon, — où il exagère le caractère accidentel des inventions, — la confusion d’idées signalée ici : « ... Vous n’oseriez dire que Prométhée dut à ses méditations la connaissance de la manière d’allumer du feu et qu’au moment où il frappait un caillou pour la première fois, il s’attendait à voir jaillir du feu ; mais vous avouerez bien qu’il ne doit cette invention qu’au hasard..., et que c’est à la chèvre sauvage que nous devons celle des emplâtres, au rossignol celle des modulations de la musique, à la cigogne celle des lavements, à ce couvercle de marmite qui sauté en l’air celle de la poudre à canon ; \emph{en un mot c’est au hasard et à tout autre chose que la dialectique} que nous avons obligation de toutes ces découvertes. »
 }, qu’il importe d’y insister encore. Ce qu’il y a de fortuit dans l’invention nous est  \phantomsection
\label{v2p246}démontré d’abord par la dissemblance des industries observées chez les insulaires de l’Océanie, c’est-à-dire chez les sauvages qui ont eu le moins de relations mutuelles et ont le plus vécu sur un fonds d’inventions autochtones. Les lacunes de celles-ci sont parfois extraordinaires. Les Otahitiens que visita Wallis en 1767 étaient déjà en possession de certaines industries et même d’arts grossiers, comme sculpteurs et musiciens ; ils savaient faire le feu\footnote{ \noindent Certaines peuplades, quoique continentales, et ayant par suite plus de facilités pour communiquer avec d’autres, ignoraient le feu au {\scshape xviii}\textsuperscript{e} siècle encore ; par exemple les Indiens à \emph{longues oreilles} de la Guyane, comme nous l’apprend une lettre (\emph{Lettres édifiantes}) de 1730. Ces sauvages étaient encore à l’âge de la pierre polie ou plutôt des \emph{cailloux aiguisés les uns contre les autres.}
 } et avaient des recettes culinaires que le capitaine anglais jugeait exquises. Cependant ils ignoraient la poterie et ne connaissaient que des plats d’écorces ou des coupes de noix de coco. Bien mieux, eux qui faisaient si bien le feu, ils n’avaient aucune idée de l’eau chaude, et la vue d’une marmite où bouillait de l’eau les comblait de stupéfaction. Ils n’avaient pour couteaux que des coquilles, pas même des silex. C’est que, livrée à elle-même, une tribu insulaire, si ingénieuse soit-elle, découvre fort peu de choses, un peu au hasard, et ne pousse jamais loin les conséquences, les plus naturelles en apparence, de ses bonnes idées. Aussi ce qu’une peuplade a découvert est rarement ce qu’une autre a trouvé dans l’île voisine\footnote{ \noindent La même remarque est applicable à des peuples déjà avancés en civilisation, quand, bien qu’assez rapprochés parfois, ils sont sans relations mutuelles. Par exemple, nous sommes surpris d’apprendre que les Péruviens des Incas connaissaient la balance et que cependant les Aztèques l’ignoraient.
 }. Si, au lieu d’être agglomérée en quelques grands continents, la terre émergée des mers était morcelée en myriades de petites îles, séparées par des mers de navigation dangereuse, il est infiniment probable : 1\textsuperscript{o} que, nulle part, dans cette Micronésie couvrant la mappemonde, la civilisation ne se serait développée ; 2\textsuperscript{o} que ces innombrables embryons de culture humaine feraient un bariolage  \phantomsection
\label{v2p247}incohérent. Seul le mutuel stimulant, seul le mutuel échange de ces découvertes spontanées a permis à quelques-uns de ces essais d’humanité de se civiliser, et à leur ensemble de s’harmoniser en se fusionnant.\par
Ce qu’il y a de capricieux et de fortuit dans la direction du génie inventif cessera d’étonner si l’on songe qu’il commence presque toujours par être au service d’un jeu ou sous la dépendance d’une idée religieuse ou superstitieuse. Beaucoup d’institutions économiques ont une origine religieuse. Grant Allen\footnote{ \noindent Voir \emph{Rivista di sociologia}, sept.-oct., 1899, article de \emph{Salvioli} sur \emph{gli mardi dell’ agricultura}.
 } a cru, et il n’est pas le seul, découvrir dans le totémisme l’explication des premières pratiques agricoles. Partout on voit se confondre, au début des évolutions économiques, « les marchés avec les fêtes de l’Église, les pèlerinages avec les voyages des marchands, les missions avec les travaux de colonisation, les trésors des temples avec l’office des banques, chez les païens et les musulmans comme chez les chrétiens », et, en remontant plus haut, le lien des deux ordres de phénomènes est encore bien plus intime. Par suite, de la nature des idées mythologiques, qui sont si bizarrement variables d’une peuplade à l’autre, dépend la nature des inventions accueillies. Elle dépend aussi de la manière dont on s’amuse. Galton explique la domestication des animaux comme une suite de l’habitude de l’homme de jouer avec eux. Et, à l’appui de cette idée, on peut remarquer avec Roscher que les Peaux-Rouges, qui auraient négligé, d’après lui, de domestiquer le bison et le renne, animaux utiles, savent fort bien apprivoiser des perroquets et des singes, aussi nombreux que les hommes dans beaucoup de leurs huttes. L’observation de Galton pourrait être généralisée dans une large mesure. L’évolution sociale commence et finit par des jeux et des fêtes. C’est en se jouant que l’homme a appris peu à peu tous ses modes de travail ; et le développement des industries les plus pénibles, les plus  \phantomsection
\label{v2p248}ingrates, tend à rendre la vie plus joyeuse, à y remplir de plaisirs plus variés les loisirs plus longs. Le travail est une phase à traverser entre l’insouciance paresseuse des primitifs et la gaîté vive des civilisés futurs.\par
Si les sauvages étaient aussi misonéistes qu’on le croit, ils ne se communiqueraient pas les secrets industriels découverts dans chacune de leurs tribus, et alors la dissemblance de ces secrets, la bizarrerie de leur juxtaposition, frapperaient les observateurs. Mais, heureusement, l’horreur que les primitifs auraient pour les nouveautés est une légende apocryphe. Chez eux, la masse est, comme chez nous et plus que chez nous, à la fois prompte à s’engouer de l’exotique et cependant routinière ; mais l’élite échappe souvent au joug de la coutume. Il y a toujours un novateur parmi les insulaires mêmes. Le plus souvent, c’est un chef, un aristocrate du pays, ou bien une reine, comme cette « princesse otahitienne » de quarante-cinq ans qui s’était éprise de Wallis. « Un des principaux de sa suite, ajoute ce grand navigateur, nous sembla plus disposé que le reste des Otahitiens à imiter nos manières : nos gens, dont il devint bientôt l’ami, lui donnèrent le nom de Jonathan. On le revêtit d’un habit à l’anglaise qui lui allait très bien. Il voulait se servir d’un couteau et d’une fourchette, mais, quand il avait pris un morceau, entraîné par la force de l’habitude, il portait la main à sa bouche, et sa fourchette allait vers son oreille. » A plus forte raison les sauvages s’imitent-ils entre eux. De là cette similitude relative de leur état social, qui abuse les voyageurs et quelques sociologues sur ce qu’il y aurait d’\emph{instinctif} et de \emph{nécessaire} dans leur développement.\par
Quand même il serait vrai, ce qui n’est pas, que d’une invention donnée, relativement simple, une seule autre, relativement complexe, peut se déduire, c’est-à-dire que les inventions se suivent comme les anneaux d’une chaîne unique, il n’en serait pas moins certain que ces anneaux se succèdent à des intervalles très variables, tantôt de quelques  \phantomsection
\label{v2p249}jours ou de quelques mois, tantôt de plusieurs siècles ; et cette différence, qui tient au hasard des circonstances, a des conséquences incalculables. Il était certainement plus facile de découvrir que l’aimant attire le fer, comme l’ont fait déjà les anciens, que de remarquer qu’il se dirige vers le pôle ; et l’on comprend dès lors très bien que la première de ces deux découvertes ait été antérieure à la seconde ; mais il n’était nullement nécessaire pour cela qu’elles fussent séparées par plus d’un millier d’années. Or, supposons que cet intervalle eût été sensiblement plus long ou plus court, et imaginez les déviations que le cours de l’histoire aurait subies. Si les Romains ou les Carthaginois avaient connu la boussole, il est probable qu’ils eussent découvert l’Amérique. Qu’on songe à l’effet produit par ce miraculeux voyage transatlantique opéré sous les Antonins ou sous les Sévères : quel réveil du monde endormi ! quel coup de fouet donné à l’activité languissante de ce temps-là, à l’esprit de conquête et d’entreprises de tout genre, au prosélytisme chrétien !\par
Et, puisqu’il s’agit de la plus grande découverte des temps modernes, achevons de montrer par cet exemple les effets durables, incessamment grossis, que peuvent avoir les contre-coups fortuits des événements. Eût-on pu deviner que la prise de Grenade par les troupes de Ferdinand et d’Isabelle aurait une influence directe et décisive sur la découverte du Nouveau-Monde ? Pourtant, rien de plus manifeste. En effet, Colomb, avant la prise de la ville des Maures, venait d’essuyer, de la part de ces deux souverains, un second et plus humiliant refus, qui l’avait profondément découragé. Rebuté par ses compatriotes gênois, puis par le roi du Portugal, puis par le roi d’Angleterre, enfin, et deux fois de suite, par les monarques espagnols, il ne lui restait plus qu’à subir sa destinée. Mais Grenade est prise, et aussitôt les dispositions intimes de Ferdinand et d’Isabelle, d’Isabelle surtout, sont changées à son égard. Cette conquête les a mis en appétit d’expansion plus vaste, les a rendus plus confiants et plus  \phantomsection
\label{v2p250}entreprenants. Ils accueillent les nouvelles ouvertures des protecteurs de Colomb, et, quelques mois après, l’Amérique était abordée par ses caravelles.\par
Toutes choses égales d’ailleurs, une invention devenue possible a d’autant plus de chance de se réaliser que le genre de problème dont elle est une des solutions éventuelles préoccupe davantage les esprits, et des esprits plus nombreux et plus éclairés. Mais, précisément, un besoin est stimulé, jusqu’à un certain degré du moins, par ce qui le satisfait, et le génie inventif se tourne d’autant plus vers un ordre de recherches, sous l’empire d’un besoin plus général et plus intense, que dans cette direction, il a été déjà fait plus d’inventions heureuses. Si donc celles-ci ont été en partie fortuites, la part du hasard ne peut qu’aller croissant dans les découvertes postérieures. Au {\scshape xiv}\textsuperscript{e}, au {\scshape xv}\textsuperscript{e} siècles, on voit naître, croître et se répandre l’avidité des découvertes géographiques et des annexions coloniales, secondée par la passion, fortifiée aussi et vulgarisée, du prosélytisme religieux au service de l’ambition monarchique ou en lutte contre cette ambition. Cette avidité naît dès le moment où la découverte de la boussole (1302) rend possibles ses satisfactions. Elle se révèle d’abord chez des princes désireux de se signaler, des rois de Portugal d’abord. Jean I\textsuperscript{er} commence à faire explorer la côte occidentale d’Afrique, bien timidement, jusqu’au cap Boyador ; puis un coup de vent fit découvrir à un vaisseau portugais l’île de Madère. Si petite que fût l’île, ce fut un encouragement donné à l’esprit d’aventure. On pousse alors plus loin l’exploration de la côte africaine, on découvre les îles du Cap-Vert et les Açores, et, à chaque découverte, la soif des voyages redouble et se répand. Chez Jean II, cette passion devient une obsession. Sous son règne et par ses ordres eut lieu le long voyage de Barthélemy Diaz, qui atteignit le cap de Bonne-Espérance. C’est à l’exemple des princes portugais que les souverains espagnols, anglais et français, favorisent à leur tour les longs voyages maritimes. En  \phantomsection
\label{v2p251}vérité, cette suite d’explorations réussies est si méthodique que la part de l’accidentel semble s’y réduire à presque rien, mais, si l’on songe que, sans la découverte de la boussole, qui est toute fortuite, cette série ne se fût pas déroulée, parce que la fin visée par ce déroulement ne se fût pas imposée aux âmes, on est forcé de convenir que le hasard a droit de revendiquer en grande partie la paternité de ces merveilleux progrès de la géographie.\par
Puis, ce qu’on trouve est si rarement ce qu’on cherchait ! Et il y a si peu de ressemblance et de proportion entre le résultat net, objectif, du labeur d’une époque, et les mobiles profonds de ses acteurs ! Combien d’âmes humaines enchaînées bout à bout, se transmettant leurs désirs et leurs espérances, leurs passions caractéristiques, plus tard indéchiffrables, il a toujours fallu pour réaliser un grand événement humain ! Qui devinerait, d’après les résultats de la colonisation espagnole au {\scshape xvi}\textsuperscript{e} siècle, l’état d’âme incomparable de ses grands promoteurs ? En premier lieu, l’habitude de découvrir des choses merveilleuses, des plantes et des animaux étranges, des contrées inouïes, avait développé, chez les Espagnols de ce temps, une crédulité dont on peut donner la mesure par ce trait\footnote{ \noindent La même cause n’a-t-elle pas produit, sous de tout autres formes, un effet analogue de notre temps ? Est-ce que la prodigieuse inventivité du {\scshape xix}\textsuperscript{e} siècle n’a pas développé aussi chez nos contemporains, sous de faux airs de septicisme, une crédulité non moins extraordinaire dont témoignent, non seulement les progrès du spiritisme et la renaissance du mysticisme, mais encore, et plus généralement, l’accueil avide fait par les classes éclairées, par les masses même, aux nouveautés quelconques, en fait de remèdes, de réformes législatives, d’expériences de vivisection sociale à tenter sur la foi de quelques utopistes ?
 }. Comme ils avaient ouï dire par des insulaires de Porto-Rico que, dans l’île de Bimini, l’une des Lucayes, il y avait une fontaine de Jouvence, rendant la jeunesse aux vieillards, ils étaient persuadés du fait, et c’est leur foi en ce conte bleu qui les décida à conquérir tout ce groupe d’îles, à les parcourir minutieusement avec beaucoup de peine, toujours à la recherche de la fameuse source. Colomb lui-même ne croyait-il pas avoir découvert le siège  \phantomsection
\label{v2p252}du paradis terrestre ? C’est absurde, soit, mais c’est parce que toutes les imaginations espagnoles, plus ou moins, étaient montées à ce ton que toutes les audaces espagnoles ont pu se déployer et ont conquis le monde.\par
L’enthousiasme qui s’est emparé alors de ce grand peuple et qui est devenu une des grandes forces historiques, immense torrent à présent tari, est une exaltation, unique dans l’histoire de passions combinées, dont la rencontre et la combinaison ne se reverront plus. L’impérialisme anglais d’à présent, cette fureur de coloniser le globe pour l’anglicaniser, il est vrai, mais surtout pour l’inonder de marchandises britanniques, n’a rien de commun avec cette ambition espagnole, tout autrement complexe et généreuse, où se mêlaient à hautes doses prosélytisme, soif d’aventures, mirage de l’or, crédulité imaginative, curiosité passionnée, héroïsme intempérant. Un concours de circonstances, un accident de l’histoire, est à la source de cette grandeur passagère, comme un autre concours d’accidents heureux a fait la grandeur plus durable de l’Angleterre.\par
Car il est des accidents qui ont des conséquences indéfinies, peut-être éternelles. Et comment peut-on supposer que l’accidentel exclut le rationnel quand on voit qu’à l’origine de toute habitude il y a un caprice, et à la base de toute loi naturelle une « collocation arbitraire des causes » comme dit Stuart Mill ? L’air, comme la mer, est sillonné d’invisibles routes que suivent dans leurs migrations périodiques les oiseaux et les vaisseaux. Et beaucoup de ces voies ont un air de nécessité, de loi fondée sur la nature des choses, tant elles sont invariables malgré leurs tortuosités. Quand les oiseaux migrateurs traversent les continents, ils suivent les vallées des grands fleuves, tels que le Rhin ou le Rhône, au lieu d’aller tout droit. Et c’est aussi le long des vallées qu’ont passé les migrations humaines, que les premiers chemins, puis les routes royales, enfin les chemins de fer même, ont été tracés. Ainsi, les accidents géologiques qui jadis ont  \phantomsection
\label{v2p253}tracé le cours des eaux, qui ont si capricieusement dessiné, et si arbitrairement, dans une grande mesure au moins, leur ligne de partage, ont imposé pour loi le même arbitraire éternel, le même désordre fixe, aux voyages des animaux et même des hommes. Les sociologues de certaines écoles devraient songer à cela quand ils font consister leur science uniquement à rechercher et à formuler de prétendues lois d’évolution qui assujettiraient les transformations sociales de tous les peuples, indépendamment de leurs emprunts mutuels, à suivre le même itinéraire, en quelque sorte réglé d’avance. En admettant même qu’ils mettent la main sur quelque formule de ce genre qui ne soit pas trop démentie par les faits, n’en serait-il pas de la régularité approximative de ces changements sociaux comme de celle des voyages dont je viens de parler ; c’est-à-dire ne tiendrait-elle pas à de simples accidents qui auraient eu des conséquences permanentes ? Et, dans ce cas, pourrait-on voir là autre chose que des documents intéressants à interpréter, des faits complexes à expliquer, par l’application des lois véritables de la science des sociétés, lois qu’il convient de chercher ailleurs ?
\subsubsection[{III.2.d. Imagination industrielle. Ses subdivisions. Ses caractères. Comment elle adapte la nature à l’homme et les hommes les uns aux autres.}]{III.2.d. \emph{Imagination industrielle.} Ses subdivisions. Ses caractères. Comment elle adapte la nature à l’homme et les hommes les uns aux autres.}
\noindent — Après ces considérations sur les conditions internes et extérieures de l’invention en général, et sur son caractère à la fois rationnel et accidentel, déductif et imprévu, occupons-nous spécialement de l’invention économique. — Elle se divise vaguement en deux grandes classes, qui se mélangent sur leurs confins : 1\textsuperscript{o} l’invention industrielle, qui consiste en \emph{transformations} de matières premières par des travaux humains, ou des fonctionnements de machines de mieux en mieux adaptées à la fabrication de certains produits ; 2\textsuperscript{o} l’invention commerciale, qui consiste en \emph{déplacements} des produits ainsi fabriqués, pour rendre de plus en plus facile leur rencontre \phantomsection
\label{v2p254} avec les besoins individuels auxquels ils sont le mieux adaptés.\par
L’invention industrielle se subdivise : tantôt elle crée un produit nouveau ; tantôt elle améliore simplement la production d’un ancien produit qu’elle généralise en en abaissant le prix, ce qui équivaut à la création d’un nouveau produit à l’égard des classes où le produit ancien s’introduit pour la première fois. Il n’est pas nécessaire de distinguer entre la création de nouveaux produits et celle de nouveaux besoins ; car chaque produit nouveau, alors même qu’il semble n’être que la satisfaction d’un besoin ancien (une étoffe nouvelle, un nouveau mode d’éclairage, un nouveau mets, un nouveau jeu) enrichit le cœur humain d’un plaisir nouveau, objet d’une nouvelle direction du désir, c’est-à-dire d’un nouveau besoin. Et, s’il est surtout donné à l’artiste et au poète d’inaugurer des sensations supérieures, des combinaisons neuves et délicates de sentiments complexes, qui font de chacun de leurs chefs-d’œuvre une vraie « révélation », un sens ajouté à la sensibilité humaine, l’industriel participe à ce privilège, bien qu’à un moindre degré et à un étage inférieur de l’âme.\par
Le progrès économique, l’harmonisation économique, due à l’invention industrielle, consiste en ce que la satisfaction des besoins les plus urgents, les plus naturels, est rendue par elle de moins en moins inconciliable, pour un nombre croissant de personnes, avec celle de besoins de moins en moins urgents, de plus en plus artificiels, c’est-à-dire sociaux et vraiment humains. Cet accord se réalise, grâce à la succession des inventions industrielles — et surtout de celles qui ont trait aux machines, — non seulement, comme il vient d’être dit, par l’abaissement graduel des prix, mais encore par la diminution graduelle de la durée moyenne des travaux et l’accroissement des loisirs. La productivité croissante du travail, aidé par les forces animales, végétales, mécaniques, laisse au travailleur plus de temps pour satisfaire ses  \phantomsection
\label{v2p255}besoins de luxe, en même temps qu’il a une proportion plus grande de son revenu ou de son salaire à leur consacrer. Ce n’est pas uniquement à obtenir un maximum d’effet avec un maximum d’effort que tend le progrès humain dans toutes ses voies — dans la voie linguistique, juridique, religieuse, politique, aussi bien qu’économique, — c’est aussi, et avant tout, à obtenir un maximum d’effet avec une dépense de temps minima. Le temps, encore plus que la force, est l’étoffe dont la vie est faite. On refait la force perdue, on ne ressaisit pas le temps perdu. C’est donc une accumulation de petites économies de temps, en fait de travaux et de locomotion, comme en fait de communication verbale, de rites, de formalités, de fonctions administratives, qui constitue le plus net du progrès humain.\par
En tant qu’elle abaisse le prix des objets de désir déjà existants, l’invention diminue momentanément l’intensité de la lutte entre les désirs de l’individu, sauf à l’aviver ensuite. Mais, en tant qu’elle crée un objet nouveau de désir, elle complique cette lutte interne des désirs et la rend plus vive entre les désirs déjà en conflit. Cela est surtout vrai des innovations doctrinales qui, en répandant de nouvelles idées, suscitent des sentiments et des besoins nouveaux, des envies à demi impuissantes de copier les classes plus riches. — Mais, dans l’un et l’autre cas, et mieux encore dans le second que dans le premier, l’invention favorise l’harmonisation des désirs entre individus différents. Dans le second cas, un nouvel article d’échange, facilitant l’utilisation des anciens produits, est jeté dans la circulation. Et le premier cas, à cet égard, se ramène au second : par l’abaissement des prix de ses consommations antérieures, le consommateur dispose d’un excédent de revenus avec lequel il achète des articles dont il se passait jusque-là.\par
Ce n’est donc point par l’invention économique que s’harmonisent les désirs de chaque individu pris à part, ils se compliquent au contraire et se combattent davantage, jusqu’à  \phantomsection
\label{v2p256}ce que l’\emph{invention morale}, au fur et à mesure qu’ils s’accroissent, apprenne à les enchaîner et à les hiérarchiser. Plus les habitudes d’un individu sont régulières et orientées vers un but supérieur, et plus un objet quelconque par lui consommé \emph{vaut}, dans le sens de valeur-emploi, si infime que soit sa valeur-coût. Et c’est ainsi que la morale, téléologie sociale, et esthétique individuelle à la fois de la conduite, complète l’activité économique. Quant à l’harmonisation des désirs entre individus, c’est l’invention économique qui l’opère, puisqu’elle est la source à la fois de la division du travail et de l’échange, de l’association productrice et de l’association consommatrice.\par
Cependant, en faisant servir de plus en plus l’homme par les êtres naturels ou les forces naturelles, et de moins en moins l’homme par l’homme, est-ce qu’elle ne tend pas, finalement, à isoler l’individu, ou du moins à le dissocier d’autrui, à lui rendre de moins en moins nécessaire le secours de ses semblables ? On peut imaginer, en effet, que, poussé à bout, le perfectionnement du machinisme conduise le machiniste, habillé, nourri, chauffé, éclairé, servi de toutes manières par des machines exclusivement, à se passer du monde entier, comme le sage idéal rêvé par les stoïciens, mais pour de tout autres raisons. Toutefois ce n’est là qu’un rêve ; et, fût-il réalisable, la dissociation de l’individu ne s’ensuivrait nullement. Si la socialisation des individus cessait de se fonder sur l’échange des produits, ce ne serait qu’à la condition de se fonder chaque jour davantage sur l’échange des exemples ; et c’est là l’essentiel, même au point de vue de la formation et du déploiement de l’originalité individuelle, qui s’alimente d’exemples triés et croisés. D’ailleurs, la nature ne pourra jamais être asservie et adaptée à l’homme que moyennant la construction et la direction de machines par des spécialistes. Le travail musculaire de l’homme finira, il est vrai, même pour les constructions des machines, par être presque annihilé ; mais le travail nerveux \phantomsection
\label{v2p257} et cérébral n’en deviendra que plus intense, plus compliqué et plus différencié. Il subsistera toujours deux grandes catégories de travaux : 1\textsuperscript{o} la direction humaine des forces animales, végétales, physico-chimiques ; 2\textsuperscript{o} la direction humaine des directeurs humains de ces forces. La dualité des métiers manuels et des carrières libérales subsistera donc, mais avec cette différence que, non moins que la direction des hommes, la direction des forces naturelles sera une dépense continue d’intelligence, un exercice spirituel. La force manuelle, dans les métiers dits manuels, ne jouera plus qu’un rôle tout à fait secondaire, analogue à l’effort musculaire nécessaire pour écrire dans les professions libérales.\par
Ce que l’invention industrielle amoindrit sans cesse, c’est l’adaptation \emph{unilatérale} de l’homme à l’homme par l’asservissement de l’un à l’autre. Jamais pourtant on ne parviendra à dompter chez l’homme fort le besoin de domination et d’inégalité à son profit. Mais il peut, grâce à l’invention, trouver plus d’avantages à exercer ce besoin d’être servi en se faisant servir par des bêtes, des plantes, des forces physiques domestiquées, qu’en asservissant d’autres hommes. Et, à mesure que, par les progrès de la science, adaptation unilatérale de l’esprit humain à la nature, progresse l’industrie, adaptation unilatérale de la nature à la volonté de l’homme, on voit s’étendre et grandir, par l’échange des exemples et des produits à la fois, par l’association sous mille formes, l’adaptation réciproque des hommes entre eux, des intelligences, des sensibilités, des volontés humaines entre elles.\par
Ce n’est point par le développement de l’industrie proprement dite, c’est bien plutôt par celui de l’agriculture, que l’industrie tend à s’affranchir de la solidarité sociale. On conçoit à la rigueur l’isolement farouche du cultivateur rural, entouré de ses animaux domestiques, de ses arbres fruitiers, de ses plantes cultivées, qui sont autant de machines à son service, répondant à tous ses besoins. Ces machines, il n’a  \phantomsection
\label{v2p258}pas eu besoin de les construire, et elles se dirigent presque toutes seules. C’est qu’en effet il est quelque peu abusif de confondre dans la même expression d’invention industrielle les progrès de l’agriculture et ceux de l’industrie. L’importance des machines en agriculture sera toujours très limitée ; le rôle principal y appartient à la vie des animaux ou des plantes, qui sont les agents mystérieux de la production agricole. Quant aux créations de nouvelles variétés de plantes ou d’animaux par l’élevage et la culture, elles diffèrent profondément des inventions proprement dites. Elles consistent à avoir stimulé l’\emph{imagination de la vie} plutôt qu’à avoir exercé l’imagination humaine. Pour bien comprendre ceci, remarquons que la terre est, non pas, comme le croyaient les physiocrates, la seule source, mais seulement le grand dépôt, avec l’\emph{eau} et l’\emph{air}, des forces physico-chimiques, véritables agents de la production des richesses, et que ces forces sont mises en œuvre pour être transformées en richesses, par : 1\textsuperscript{o} ces inventions obscures de la vie appelées espèces vivantes, végétales ou animales (y compris l’homme) ; 2\textsuperscript{o} les inventions de l’industrie, par la vertu desquelles se reproduisent en exemplaires innombrables, inépuisables, les divers types de vêtements, de meubles, etc., de même que les types vivants se répètent en générations sans fin. On s’explique donc sans peine que l’agriculture se soit développée avant l’industrie : il est naturel que l’homme ait su s’emparer de machines toutes faites avant de savoir en faire à son tour.\par
Ce n’est pas seulement par leur fécondité en reproductions indéfinies que les machines ressemblent aux espèces vivantes. C’est encore par le mode de leur fonctionnement, par cette série périodique de mouvements qui reviennent à leur point de départ pour recommencer de nouveau à tourner dans ce cycle. Par ce caractère rythmique, qui est si frappant quand on entre dans une fabrique, les machines, ces « contrefaçons d’êtres animés », dit très bien Louis Bourdeau, rappellent les  \phantomsection
\label{v2p259}pulsations, les halètements, les spasmes des animaux, seulement avec une précision et une accentuation extraordinaires qui nous transportent en un « règne nouveau », en une faune fantastique affranchie de tout besoin de repos et de sommeil, insensible à la douleur comme à la pitié, infatigable. On a remarqué certains rapports entre l’évolution d’un art et l’évolution d’un être vivant ; mais, à cet égard aussi, l’imagination industrielle ressemble à la vie et à l’art. Une sorte de \emph{loi des âges} lui est applicable. « L’invention mécanique et industrielle, dit Ribot, a, comme l’invention esthétique, ses périodes de préparation, d’apogée, de stagnation, celle des précurseurs, des grands inventeurs, des simples perfectionnements. » Il cite comme exemples les applications de la vapeur, depuis l’éolipyle de Hiéron jusqu’à l’époque héroïque de Newcomen et de Watt, suivie des perfectionnements de leurs successeurs. C’est ainsi qu’un art grandit lentement jusqu’à son âge classique, auquel succède l’ère prolongée des raffinements, comme l’être vivant va de l’enfance à la jeunesse et à la maturité, puis à la vieillesse.\par
Malgré la similitude des exemplaires de chaque produit édités par les machines, malgré la similitude des besoins qu’elle suscite ainsi dans des individus différents, ce n’est pas l’assimilation des individus, c’est au contraire leur différenciation que le progrès industriel tend à opérer. Il ne pourrait vivre sans elle. Si tous les individus se ressemblaient parfaitement, s’ils avaient tous les mêmes besoins et les mêmes goûts en proportions égales, et ne différaient que par la nature de leurs occupations, par la spécialisation de leurs travaux, le nombre des métiers ne pourrait dépasser celui des besoins et des goûts d’un homme quelconque, pris au hasard. Or, si multiples que puissent être les désirs de consommation qui se reproduisent journellement ou annuellement chez un seul homme, il n’y aurait pas là de quoi alimenter beaucoup d’usines, de quoi garnir les vitrines de la moindre Exposition. Donc, les progrès de l’industrie et du  \phantomsection
\label{v2p260}régime économique exigent non seulement la complication des désirs de chaque consommateur, mais encore la différenciation des consommateurs. Et il est manifeste que cette diversité individuelle et le progrès industriel sont en rapport de causalité réciproque. Comptez les besoins qui, à un moment donné, sont communs à tous les individus en relations économiques ; vous n’en trouverez que fort peu, ou pas. Car le besoin de manger est lui-même interrompu chez certains malades, ainsi que le besoin d’être vêtu, d’être chauffé, d’être abrité, l’est chez certains excentriques. Mais comptez les besoins qui ne sont partagés que par une fraction minime de la nation, vous trouverez que leur nombre, leur nombre absolu et leur nombre proportionnel, va toujours croissant.
\subsubsection[{III.2.e. Réponse à une objection de Sismondi. Invention industrielle toujours favorable aux consommateurs, pas toujours aux producteurs, du moins à ses débuts. Diminue la durée du travail, mais non, finalement, le salaire, etc.}]{III.2.e. Réponse à une objection de Sismondi. Invention industrielle toujours favorable aux consommateurs, pas toujours aux producteurs, du moins à ses débuts. Diminue la durée du travail, mais non, finalement, le salaire, etc.}
\noindent Sismondi reproche aux inventions industrielles ce qui est précisément, au point de vue de l’extension du champ social, leur plus louable efficacité. Il montre, en effet, que, tant qu’un marché reste clos en soi, toute augmentation de l’effet utile du travail par suite d’une invention équivaut à l’expulsion d’un certain nombre de travailleurs, qui, ne gagnant plus rien, appauvrissent d’autant la petite société en question. « Aussi, dit-il, le bénéfice qu’on attend de la découverte d’un procédé économique se rapporte-t-il presque toujours au commerce étranger. » C’est certain. Le premier effet d’une invention industrielle, née quelque part, dans une région fermée, est de la forcer à s’ouvrir pour épancher une partie de ses produits au dehors, et, par suite, pour accueillir des produits extérieurs. Ce n’est qu’à cette condition que les bras devenus inutiles immédiatement après l’avènement d’une invention peuvent se réemployer ensuite. « Les fabricants de bas en Angleterre, avant l’invention du métier à bas, n’avaient pour consommateurs que des  \phantomsection
\label{v2p261}Anglais. Depuis cette invention jusqu’au moment où elle a été imitée hors de leur île, ils ont eu pour consommateurs tout le continent. Toute la souffrance est tombée alors sur les producteurs continentaux, toute la jouissance est demeuréè aux Anglais : le nombre de leurs ouvriers, au lieu de diminuer, s’est augmenté, leurs gages se sont élevés, les profits des fabricants se sont accrus aussi. ». Sismondi ajoute, il est vrai : « et la découverte a paru avoir pour résultat une aisance universelle, puisque tous ceux qui en souffraient étaient étrangers et vivaient à de grandes distances, tandis que tous ceux qu’elle enrichissait étaient rassemblés sous les yeux de l’inventeur. Chaque perfectionnement qu’on a apporté aux procédés de l’industrie a eu presque toujours ce résultat : il a tué, à de grandes distances, d’anciens producteurs qu’on ne voyait pas et qui sont morts ignorés ; il a enrichi, autour de l’inventeur, des producteurs nouveaux\footnote{ \noindent Quand il n’en est pas ainsi, quand une invention nouvelle a pour effet de \emph{tuer} non pas des concurrents étrangers et lointains, mais des rivaux compatriotes et rapprochés, elle a infiniment plus de peine à s’établir. Melon écrivait au {\scshape xviii}\textsuperscript{e} siècle : « Il a été proposé de procurer à une capitale de l’eau abondamment par des machines faciles et peu coûteuses. Croirait-on que la principale objection, qui peut-être en a empêché l’exécution, a été la demande : que deviendront les porteurs d’eau ? »
 } ». Il ne faut par nier la vérité de cette cruelle remarque. L’effet des inventions industrielles, en cela, n’est pas sans analogie avec celui des inventions militaires. Plus nous progressons, plus s’étend le rayon des distances auxquelles sont tués, à partir du producteur nouveau comme centre, les producteurs anciens qu’il rend superflus. Mais, tôt ou tard, ces morts ressuscitent par l’imitation de l’invention à l’étranger, et, compensation faite du mal que se sont faits ainsi les producteurs rivaux, il reste, comme bénéfice net, l’accroissement de la consommation qui est dû à l’abaissement du prix, et qui, à la fois, suit et stimule le développement de la production.\par
Ainsi, toute invention est favorable — dès son apparition — aux intérêts des consommateurs, et, si elle commence par  \phantomsection
\label{v2p262}mettre en lutte les producteurs entre eux, soit les patrons avec leurs ouvriers expulsés par les machines nouvelles, soit les patrons avec d’autres patrons, cette opposition, qui s’adoucit toujours, se change souvent en adaptation. Schultze-Gavernitz, dans son livre sur la \emph{Grande Industrie}, montre à quel point les ouvriers du Lancashire se sont adaptés, par un assouplissement devenu héréditaire, à leurs machines à tisser.\par
Quand une invention industrielle vient d’être lancée, elle est monopolisée en droit ou en fait, par l’inventeur ou ses ayants-droit, j’entends les propriétaires des machines où elle s’incarne, les actionnaires des sociétés anonymes entre lesquels cette propriété se divise. Et il est juste, il est utile qu’il en soit ainsi ; sans cela le génie inventif perdrait l’un de ses stimulants les plus actifs, et personne, au début, ne se risquerait à prendre sa cause en mains. Mais il importe que ce monopole ne s’éternise pas, et, de fait, plus tôt ou plus tard, toute invention finit par tomber dans le domaine commun. Dans une société bien réglée, il faudrait, d’une part, limiter la durée du monopole dont il s’agit, mais, d’autre part, en assurer, pendant tout ce temps, la jouissance aux intéressés. Il ne convient pas que ceux qui n’ont en rien participé ni à la conception de la découverte ni aux dépenses et aux risques de sa protection initiale, se croient le droit de s’en partager, dès l’origine, tout le bénéfice avec l’inventeur et ses protecteurs. Voilà, par exemple, une usine américaine, qui, pour une production égale aux besoins de sa clientèle, emploie 100 ouvriers. Survient une invention qui permet à l’usinier, ayant-droit de l’inventeur, de produire avec 80 ouvriers travaillant le même nombre d’heures, soit dix heures par jour, la même quantité de marchandises. Il renvoie donc 10 ouvriers. Est-ce là une solution qui s’impose ? Non, il serait plus juste et plus humain de conserver 100 ouvriers en réduisant à huit heures par jour la durée de travail, ce qui reviendrait au même si  \phantomsection
\label{v2p263}le salaire était réduit en même temps de deux dixièmes. Mais c’est contre cette réduction du salaire, en cas pareil, — et le cas est fréquent aux États-Unis, de nos jours — que les ouvriers américains protestent. Cette protestation est-elle fondée ? Elle n’a que le tort d’être prématurée. Pour le moment, si elle était accueillie, elle aurait pour résultat d’accaparer au profit exclusif des ouvriers tout l’avantage de l’invention, à l’exclusion du patron dont le bénéfice resterait le même, et des acheteurs de l’article dont le prix ne pourrait s’abaisser. Mais, finalement, une fois écoulée la période de temps nécessaire pour qu’une invention tombe dans le domaine commun, — et cette période de temps s’abrège de plus en plus en Amérique, en Europe même, par suite de l’imitativité croissante — le vœu formé par les ouvriers d’Outre-Mer ne peut pas ne pas s’accomplir. Au point de vue même de l’intérêt des fabricants, il est désirable que la diminution de la durée du travail n’entraîne pas la diminution de la moyenne des salaires. Supposons, en effet, que, dans toutes les usines à la fois, la production ait été améliorée par des inventeurs, et que, dans chacune d’elles, le salaire des ouvriers restés aussi nombreux mais ne travaillant plus que huit heures au lieu de dix, ait été rogné des deux dixièmes. Qu’en résultera-t-il ? Que toute la masse des travailleurs du pays, c’est-à-dire la population à peu près tout entière, se trouvera avoir ses revenus diminués et n’aura qu’une somme moindre à consacrer aux achats de toute nature. Il faudra donc, de toute nécessité : ou que les usines restreignent leur production dans la proportion de deux dixièmes ; ou bien que, continuant à produire le même nombre d’articles, elles les vendent à un prix inférieur de deux dixièmes au prix antérieur. Le résultat final sera donc le même pour les fabricants que si toutes ces usines, dès le moment où elles avaient diminué la durée du travail, avaient maintenu le taux des salaires. Cela signifie que, en fin de compte, les inventions doivent  \phantomsection
\label{v2p264}fatalement se \emph{socialiser} d’elles-mêmes ; ce qui ne veut pas dire, d’ailleurs, qu’il soit \emph{toujours} inutile, ni \emph{toujours} injuste, de hâter par des moyens légaux leur socialisation. C’est une question d’opportunité, qui ne se prête à aucune généralisation téméraire.\par
Mais nous avons raisonné comme si les ouvriers dont le travail a été réduit de dix à huit heures ne faisaient rien des deux heures de loisir qui leur sont laissées. Ayons égard à l’emploi qu’ils feront de ces deux heures, et nous apercevrons mieux toute la fécondité sociale de l’invention. L’œuvre magique de la civilisation n’apparaît qu’à partir du moment où du travail plus productif naît le loisir, et, avec le loisir, le besoin croissant de vie sociale, de conversation, de discussion, de plaisirs et de fêtes. Que l’alcoolisme soit souvent, çà et là, la maladie de croissance de cette sociabilité en progrès, je l’accorde ; mais ce mal est passager. Ce qui triomphera, à la fin, ce ne peut être que la vie de l’esprit. Toute cette fièvre intense de production qui suscite, de l’autre côté de l’Atlantique, des milliers d’inventions chaque année parmi des millions de travailleurs acharnés, élabore, à son insu, une Amérique future, idéaliste, où après une crise impérialiste et militariste peut-être, les délices sociales de l’art et de l’activité intellectuelle désintéressée seront goûtées par-dessus tout. Cette élaboration est inévitable ; il ne se peut que tant d’inventions accumulées n’aboutissent pas à un prodigieux élargissement des loisirs humains, et que ces loisirs ne trouvent pas leur principal emploi dans les contacts et les échanges spirituels, dans le plaisir de s’instruire et de s’impressionner réciproquement, dans la culture intensive d’une sociabilité à la fois raffinée et saine.\par
Cet exemple des États-Unis montre bien, soit dit à ce propos, à quel point s’abusent ceux qui, méconnaissant la part capitale et prépondérante du génie inventif dans la prospérité des nations, font honneur aux guerres, aux troubles civils, aux agitations et oppositions de toutes sortes, des  \phantomsection
\label{v2p265}bienfaits dus à l’invention. Parce que l’imagination humaine fleurit sur les ruines mêmes entassées par nos discordes, on est parfois porté à croire que ce sont elles qui l’ont fait éclore et que, loin du bruit des armes, cette fleur du recueillement et de la méditation resterait stérile. Tout ce qu’il y a de vrai au fond de cette erreur profonde, c’est que, par une erreur précisément inverse, Spencer a eu tort d’établir entre le militarisme et l’industrialisme une antithèse aussi factice que prolongée. Il suffit, pour la réfuter, de se rappeler Athènes et Florence, Venise aussi bien, trois États des plus batailleurs et en même temps des plus prospères industriellement et commercialement que le monde ait vus. Mais, si les perturbations sociales n’empêchent pas les inventions d’éclore pour contribuer à les apaiser, c’est une paradoxale banalité, un préjugé littéraire, d’attribuer aux perturbateurs le mérite des inventeurs. Je prends pour exemple la république florentine\footnote{ \noindent J’aurais pu prendre pour exemple aussi bien l’État athénien. Car, longtemps après avoir cessé d’être belliqueuse en perdant sa liberté, Athènes, sous les Romains, eut un retour de prospérité commerciale \emph{merveilleuse}, dit Lenormant \emph{(Les Monnaies)} à partir du moment où ses vainqueurs lui abandonnèrent la souveraineté de l’île de Délos (167 ans av. J.-C.). Les Athéniens firent de Délos un port franc qui devint le centre principal de leur commerce, et, surtout pendant quatre-vingts ans, de la destruction de Corinthe à la guerre de Mithridate, fut comme le dit Festus, \emph{le plus grand marché du monde.} »
 }. Perrens constate que la période où Florence a été agitée par des dissensions intestines et des guerres extérieures a été aussi l’ère de sa grande prospérité, et il est enclin à penser qu’elles en ont été la cause. Mais la vérité est que Florence a été riche et prospère malgré ses troubles, malgré son instabilité et sa prétendue liberté politique, aussi longtemps qu’elle a monopolisé, en fait, certaines inventions industrielles et certaines idées commerciales ou financières, imaginées ou exploitées par elle avant toute autre cité. Le même historien nous apprend que la France achetait en grande quantité aux Florentins du moyen âge « ces draps perfectionnés \emph{dont ils  \phantomsection
\label{v2p266}avaient seuls le secret} » ; qu’un citoyen de cette ville avait découvert la teinture en pourpre par l’orseille ; que l’industrie du drap et son trafic « formaient un monopole concentré en un petit nombre de mains. » Il nous apprend encore que la découverte de la boussole est venue favoriser l’essor de ce grand commerce. Chance inouïe, en effet, pour un peuple négociant, que de s’approprier pour un temps, en l’utilisant l’un des premiers, ce merveilleux \emph{sens de la direction} donné aux navires. Plus tard, quand cette invention, ainsi que bien d’autres, longtemps accaparées par les Florentins, se sont vulgarisées, il ne faut pas s’étonner si la prospérité de Florence décline, et on doit se garder d’imputer ce déclin à la pacification monarchique des {\scshape xv}\textsuperscript{e} et {\scshape xvi}\textsuperscript{e} siècles, — quelque jugement politique qu’on porte d’ailleurs sur les Médicis — surtout si l’on observe qu’à cette époque une découverte, capitale aussi, est venue retirer aux cités et aux régions méditerranéennes pour le transporter aux pays occidentaux, riverains de l’Océan, le privilège de la grande navigation, du grand négoce, des grandes richesses.\par
Notons aussi que, d’après Perrens, les progrès de Florence dans \emph{l’art de la laine}\footnote{ \noindent Autres inventions que les Florentins s’approprièrent : je veux parler de celles, très antiques, qui faisaient le prix et l’originalité de l’industrie orientale. Les Florentins, dans leurs voyages annuels à Pékin, avaient appris, notamment, « à imiter et même surpasser les orientaux dans la fabrication des étoffes de brocart d’or et d’argent ».
 }, à sa plus belle phase, « sont dus principalement à l’ordre religieux des \emph{Umiliati} », que « stimulée par l’exemple » l’industrie laïque prospérait pareillement et que les religieux en question restaient étrangers à toutes les discordes civiles. Comment la prospérité de Florence serait-elle une suite de ses luttes de partis s’il est reconnu qu’un monastère paisible, en dehors de ses troubles, tient la tête de la plus importante de ses industries et lui donne l’impulsion stimulante ? Nous avons des raisons de penser, au contraire, que ses agitations, soi-disant fécondes, ont beaucoup contribué à ralentir le mouvement ascendant  \phantomsection
\label{v2p267}de sa richesse. Le même écrivain nous en fournit une preuve, sans y penser. A propos des banques dont les Florentins ont eu les premiers l’idée, il écrit : « Cette nouvelle source de luxe, où ils défiaient toute concurrence, leur parut d’autant plus précieuse que la concurrence s’établissait dans la fabrication. N’en ayant pu longtemps conserver secrets les procédés, ils voyaient \emph{leurs compatriotes bannis les porter au loin}. »
\subsubsection[{III.2.f. Imagination commerciale. Ses diverses espèces. Inventions locomotrices. Le besoin de locomotion. Chemins de fer, agents d’harmonisation économique ; sont-ils aussi des agents d’égalisation démocratique ? Substitution momentanée des aristocraties nationales aux aristocraties individuelles.}]{III.2.f. \emph{Imagination commerciale.} Ses diverses espèces. Inventions locomotrices. Le besoin de locomotion. Chemins de fer, agents d’harmonisation économique ; sont-ils aussi des agents d’égalisation démocratique ? Substitution momentanée des aristocraties nationales aux aristocraties individuelles.}
\noindent Ce qui précède a trait à l’imagination industrielle mais s’applique presque aussi bien à l’imagination commerciale. Aussi serons-nous plus bref en ce qui concerne cette dernière. Elle embrasse cependant un champ très étendu, où sont comprises : les inventions relatives aux moyens d’échange, monnaies, lettres de change, assignats, aux diverses institutions de commerce, aux spéculations commerciales ; les inventions de moyens de transport ; les découvertes géographiques des explorateurs négociants, etc. Occupons-nous seulement des progrès de la locomotion. D’abord, ils vont nous servir d’exemple excellent pour montrer les services mutuels que peuvent se rendre différentes inventions adaptées les uns aux autres. Il est certain que, si les grandes inventions qui ont révolutionné la production industrielle au {\scshape xix}\textsuperscript{e} siècle n’avaient pas eu pour auxiliaires les grandes inventions locomotrices leurs contemporaines, qui ont ouvert à la production agrandie des débouchés proportionnés à son agrandissement, la révolution de l’industrie eût avorté infailliblement. Il n’est pas moins certain que les ailes prêtées au commerce par la facilité et le bon marché des transports auraient servi à transporter peu de choses si les transformations internes de l’industrie n’étaient venues à point pour les employer. — Autre coïncidence. C’est par une rencontre vraiment remarquable, et qu’on ne remarque pas,  \phantomsection
\label{v2p268}tant elle paraît naturelle, que les inventions relatives à la locomotion marine et terrestre ont coïncidé avec celles qui facilitent la communication des idées entre les hommes, télégraphie électrique terrestre et sous-marine, téléphone. Supposez que les premières eussent apparu sans les secondes : non seulement le service des chemins de fer, sans échange de renseignements télégraphiques entre les gares, eût été des plus défectueux et des plus périlleux, mais, pour une autre cause encore, le stimulant donné au besoin de voyage eût été bien moins efficace. Le lien social entre membres d’une même famille, d’une même profession, sinon d’une même patrie, eût risqué de s’affaiblir au point de se rompre. Pendant des périodes assez longues, ils auraient perdu contact social entre eux. Mais, fort heureusement, les télégraphes ont servi en cela de correctif ou de complément aux chemins de fer, et, pendant que des distances de plus en plus grandes séparent les parents, les amis, les concitoyens, ils se sentent de plus en plus rapprochés moralement et cordialement. D’un bout à l’autre de l’Europe, deux frères se communiquent plus vite à présent les nouvelles de leur famille qu’ils ne l’auraient pu, il y a soixante-dix ans, à la distance de deux chefs-lieux de canton du même arrondissement. Rien n’a plus contribué à développer la fièvre de locomotion que cette facilité de communication idéale et instantanée, comme rien n’a tant contribué que les voyages à multiplier les communications télégraphiques et téléphoniques\footnote{ \noindent Jadis, — surtout si nous remontons à l’époque où le service des postes était lent, intermittent, peu sûr, — le voyageur qui partait était perdu pour les siens jusqu’à son retour ; il entrait à leurs yeux dans la nuit noire de l’absence sans nouvelles rapides, de l’inconnu rempli d’angoisses. Nous avons peine à nous figurer l’état de cœur d’une mère et de son fils s’embrassant à l’heure des adieux dans des conditions pareilles. L’absent alors, au bout de quelques mois, devenait facilement un mort, qui ressuscitait quelquefois. Le titre du Code civil, sur l’\emph{Absence}, si complètement tombé en désuétude, nous peint sans le vouloir cette psychologie de nos pères ou de nos grands-pères.
 }.\par
Admirons aussi, une fois de plus, par l’exemple des chemins \phantomsection
\label{v2p269} de fer, à quel point un besoin peut être stimulé par ses satisfactions mêmes. C’est une erreur de croire que cette frénésie de voyage qui des classes les plus riches descend aux plus pauvres, des capitales aux villes de second ordre et aux campagnes, du Nord relativement civilisé au Midi relativement « arriéré », soit un besoin inné de l’homme. Avant d’être touché par la contagion de la fièvre locomotrice propre à notre temps, l’individu ne voyage que par force, il ne sait pas ce que c’est que voyager par plaisir ; et, d’ailleurs, parmi nos populations les plus civilisées, le prurit de déplacement est, en grande partie, un des symptômes de l’universel déclassement, du mécontentement grandissant de la vie contemporaine. Même de nos jours, en effet, c’est par force qu’on se déplace, poussé par un aiguillon plus ou moins douloureux, tel que la soif de l’avancement, la cupidité, l’inquiétude, l’absence de toute affection profonde, le déracinement du sol natal, des amitiés héréditaires, le vagabondage moral. Aujourd’hui chacun ou chaque chose doit aller chercher de plus en plus loin ce qui lui est adapté : chaque marchandise son acheteur, chaque voyageur son emploi. Pourtant, il y a une part notable à faire aux voyages d’agrément, ou qualifiés tels, quoique beaucoup soient moins des plaisirs que des nécessités subies ou des obligations imposées, nécessités d’information, obligations de mode. Et cette part va toujours grandissant. Et cela doit paraître extraordinaire quand on sait le peu de goût que le primitif, l’homme naturel, quand il est heureux, a pour les voyages. M. Kovalesky\footnote{ \noindent Voir son \emph{Régime économique de la Russie.}
 } a très bien observé cela, dans « les profondes couches populaires ». « Voyez, dit-il, le paysan français. Il reste sa vie durant sur le lopin de terre dont il est propriétaire. Je connais dans la \emph{rivière} de Nice des gens du peuple qui, vivant dans le voisinage direct de cette ville, ne se sont jamais déplacés pour la voir. Il n’en est guère de même en Russie. Je puis citer l’exemple de provinces  \phantomsection
\label{v2p270}entières où, régulièrement, d’année en année, le tiers du village va chercher du travail à la distance de quelques centaines de kilomètres, et cela aux prix de sacrifices infinis et uniquement afin de pouvoir vivre et entretenir leurs familles. C’est pour moi le signe le plus manifeste de la grande misère qui commence à envahir nos campagnes. » C’est par force que les peuples pasteurs sont nomades ; c’est par force que les peuples agriculteurs, tels que les paysans russes, le deviennent accidentellement ; et, si ce n’est pas uniquement par force, si c’est aussi par plaisir, que les peuples civilisés se font voyageurs et instables, on peut dire que, même chez ceux-ci, pour un voyage d’agrément, il y a cent voyages d’obligations, d’obligations réelles ou imaginaires, que l’habitude de voyager a elle-même suscitées ou fortifiées en se répandant.\par
Aussi ne puis-je croire que le besoin de locomotion — quoiqu’il compte parmi les besoins les plus élastiques, les plus extensibles, après celui d’instruction — soit destiné à progresser indéfiniment. Les besoins plus profonds et plus anciens du cœur, qu’il refoule en se développant, ceux de vie de famille, de maternité et de paternité, l’amour du sol natal, ne se laisseront pas toujours dominer par lui. La locomotion continuera à se transformer ; car elle a évolué dans un sens assez précis et intéressant à noter : on a voyagé primitivement pour des motifs religieux (pèlerinages) puis militaires, puis commerciaux, puis industriels, et enfin par hygiène ou par curiosité. Surtout il importe de noter la différence entre les voyages collectifs d’autrefois, sous forme religieuse, militaire, ou commerciale — par exemple dans le \emph{commerce guerrier} pratiqué par les Anglais dans l’Inde au {\scshape xviii}\textsuperscript{e} siècle — et les voyages individuels d’à présent qui se multiplient. La locomotion s’individualise. Mais, si elle doit continuer longtemps encore à changer de nature en ce double sens, il est peu vraisemblable qu’elle croisse beaucoup en fréquence et en longueur de trajet  \phantomsection
\label{v2p271}moyen, ou plutôt qu’elle ne rétrograde pas un jour sous ce rapport, quand les individus déclassés peu à peu se reclasseront et, en redevenant contents de leur sort, redeviendront fidèles à leur sol. Il faut distinguer à cet égard entre le transport des voyageurs et le transport des marchandises : ce dernier ne peut que se développer sans cesse, pendant de longues années après que le premier se sera arrêté ou sera revenu en arrière. Et il se peut même qu’il se développe à la faveur de cet arrêt ou de cette rétrogradation : car, en se déplaçant moins (bien entendu, dans une certaine mesure), les individus des divers peuples se différencieront davantage par leurs productions et leurs consommations, industrielles ou intellectuelles, et donneront lieu de la sorte à un échange plus compliqué, plus varié, plus étendu, d’idées, d’œuvres d’art, de marchandises.\par
Mais c’est à d’autres considérations que nous devons nous attacher, car les progrès de la locomotion soulèvent de grands problèmes. A quelle action finale tendent ces progrès ? A l’harmonie plus large et plus profonde des intérêts, ou à leur lutte sur une plus grande échelle, ou à celle-là à travers celle-ci ? Est-ce qu’ils tendent à égaliser ou hiérarchiser les hommes et les peuples, et est-il certain que la démocratie ait à s’applaudir de leurs effets futurs, qu’il serait déjà possible d’entrevoir ?\par
Par l’échange qu’ils étendent au point de fondre par degrés en un seul marché « mondial » tous les marchés régionaux et continentaux qu’ils agrandissent — par la division du travail qu’ils étendent aussi et compliquent entre les individus, et à laquelle ils superposent une division du travail, plus haute et infiniment plus large, entre les nations, — par l’unité de mesure et de monnaie que peu à peu ils établissent — par l’association qu’ils rendent possible et facile entre des individus séparés par les plus grandes distances, à travers les frontières des États — il n’est pas douteux que ces grands progrès de la locomotion font œuvre  \phantomsection
\label{v2p272}d’harmonisation économique dans des dimensions inconnues du passé. Mais il n’est pas douteux non plus qu’ils mettent en lutte les concurrents commerciaux et industriels, agricoles mêmes, les personnes et les nations rivales, à des distances de plus en plus inouïes ; que, par eux, les agriculteurs des États-Unis et les paysans français se battent à mort, que partout à cause d’eux, pour défendre les industries nationales contre les traits meurtriers qui les assaillent de toutes parts, du fond des continents lointains, les États sont forcés de se hérisser de douanes et les patriotismes de se redresser contre l’invasion des mœurs, des idées, des produits exotiques. Et l’on peut se demander s’ils n’ont pas autant contribué à exaspérer ce nationalisme politique qu’à tisser au-dessus de lui un internationalisme social et, au-dessous de lui, un internationalisme financier, qui, par tout pays, sont en conflit aigu avec lui, troublant l’univers de leurs discordes. Comment finiront ces oppositions grandioses, aussi gigantesques que les adaptations qui les ont enfantées ? Ce ne sont pas seulement les chemins de fer dits stratégiques — réseau guerrier sans cesse resserré — qui méritent ce nom ; à vrai dire, tout est stratégique et belliqueux dans ces voies ferrées par lesquelles les nations se font la guerre, commerciale, industrielle, ou militaire, quand ce ne sont pas des compagnies rivales. Est-ce que, en présence de ce spectacle, il est permis de garder l’espoir que la fédération pacifique des nations sera grandement aidée, comme on l’avait cru, par cette cotte de mailles de rails qui revêt la planète ? — Et, par suite, est-ce que, des immenses chocs qu’on peut prévoir entre adversaires géants, on ne doit pas s’attendre à voir jaillir quelque suprématie conquérante et oppressive, aussi fatale à la liberté qu’à l’égalité démocratique ?\par
Pour répondre, faisons observer que le caractère le plus net, le plus important, de tous les progrès de la locomotion, et spécialement des derniers, si supérieurs à tous les autres réunis, consiste à avoir ébréché ou rompu tour à tour toutes  \phantomsection
\label{v2p273}les digues qui contenaient dans le bassin de la famille, de la province ou de la nation, la tendance constante à l’expansion des exemples. Les chemins de fer en particulier, et la navigation à vapeur, ont déchaîné des courants et des torrents d’imitation d’une puissance incomparable, et l’humanité est devenue un concours ou un combat général, très passionné, des grands types de civilisation pour la domination du globe. Que, ravagés par ces déchaînements, en proie à ces perturbations, les États aient cherché à relever les digues rompues, et à ralentir aussi le mouvement d’assimilation pour le rendre salutaire de dévastateur qu’il commence par être, cela s’explique sans peine et n’a nul besoin de s’excuser. Mais ralentir n’est pas refouler, et quand, à l’action d’une force continue, qui s’amasse et s’accumule sans fin, on n’oppose qu’une résistance intermittente et par accès, le résultat final ne peut laisser l’ombre d’un doute : les barrières protectionnistes seront submergées tôt ou tard par le déluge de l’exotisme envahissant, ou renversées à coups de canons par quelque armée non moins diluvienne au service de cette invasion dite pacifique.\par
On dit que les chemins de fer sont un agent de transformation des sociétés dans un sens égalitaire et démocratique. Pour bien juger de la vérité de cette idée courante, distinguons leurs effet sur les rapports des individus et sur les rapports des nations. Ils démocratisent les individus, cela est vrai, par l’égalisation de leurs droits qui aboutira inévitablement à diminuer l’inégalité de leurs fortunes ; mais ne peut-on pas dire que \emph{jusqu’ici} ils \emph{aristocratisent} les nations, et en vertu des mêmes causes ? Pendant que, dans chaque nation, la multiplicité des voyages et la vulgarisation des produits, des idées de tout genre, tendent à niveler les classes tout en différenciant les professions, on ne s’aperçoit pas que des aristocraties nouvelles, d’apparence plus impersonnelle, mais non moins redoutables pour cela, surgissent du milieu des aristocraties anciennes qui s’affaissent. Les  \phantomsection
\label{v2p274}noblesses familiales ont été remplacées, en premier lieu, par les noblesses urbaines, par les capitales devenues les patriciats de nos jours ; mais ce n’est pas tout, parmi les nations, il en est quelques-unes qui s’arrogent des droits innés à la suzeraineté cosmique, et par ces \emph{noblesses nationales} d’une envergure d’ambition et de prétention toute récente, on voit apparaître un avatar, sous forme collective, de l’esprit aristocratique prodigieusement agrandi, qui donne une terrible confirmation à la loi d’amplification historique.\par
Or, ces deux agrandissement successifs de l’orgueil nobiliaire, — dans chaque nation l’orgueil nobiliaire d’une cité, et dans l’humanité l’orgueil nobiliaire d’une nation ou d’un petit nombre de nations — ont pour principales causes les inventions locomotrices, ces destructrices de l’orgueil nobiliaire de certaines familles. L’orgueil britannique grandit, par le jingoïsme et l’impérialisme en progrès, à mesure que l’orgueil familial des lords anglais s’adoucit, ou s’efface ; et ces deux effets sont dus à la prospérité de l’Angleterre par les grandes inventions du siècle, machinisme industriel et navigation à vapeur, qui ont travaillé à son profit, qui ont assis, fortifié, développé démesurément sa puissance coloniale, sa langue, ses institutions, son activité productrice. A présent, l’orgueil germanique se soulève aussi pendant que celui des hobereaux allemands se sent atteint dans les moelles par l’extension du suffrage. En d’autres termes, chez les nations triomphantes grâce aux transformations économiques encore plus que militaires du monde, le patriotisme se fait arrogant, insolent, présomptueux, et, chose stupéfiante, trouve une sorte de justification et de consécration de lui-même dans l’abaissement autour de lui des patriotismes déclinants, qui sont en train de faire à cet orgueil national une cour d’humilités nationales, fondées sur l’illusion habituelle du vaincu, de se croire inférieur à son vainqueur.\par
Une aristocratie de nations, donc, de deux ou trois nations, victorieuses économiquement, sinon militairement, et qui  \phantomsection
\label{v2p275}ont tous les caractères distinctifs des aristocraties de tous les temps : très voyageuses, comme l’étaient les patriciens de Rome et nos gentilshommes d’ancien régime, — cosmopolites par suite, mais d’un cosmopolitisme plutôt \emph{supra-national} qu’international, et qui se sent partout chez soi parce qu’il est persuadé que le globe lui appartient, — créancières des autres peuples, comme l’étaient les \emph{patres conscripti} de Rome à l’égard des plébéiens, — copiées en tout par les autres peuples, qui se laissent imposer avec leur idiome et leurs produits leur admiration d’elles-mêmes, et qui, eux, à l’inverse, deviennent de plus en plus sédentaires, dépendants, débiteurs humbles et pauvres : voilà, ce semble, le résultat le plus manifeste, jusqu’à ce jour, de nos progrès économiques. — Mais ce résultat ne doit ni nous surprendre ni nous faire mal augurer de l’avenir. Toute invention nouvelle, quand elle est accaparée par un individu, commence par lui donner une supériorité de richesse plus marquée sur les autres, sauf, ensuite, à rétablir peu à peu l’égalité entre eux, à un niveau plus élevé pour tous, quand l’invention s’est vulgarisée. De l’exploitation unilatérale de certains individus par un seul on passe à leur exploitation réciproque. Cette loi du passage de l’unilatéral au réciproque, régit aussi les rapports internationaux. Tant qu’une nation exploite seule, ou à peu près seule, un faisceau d’inventions nouvelles, elle emploie l’univers d’abord sans réciprocité, puis avec une réciprocité croissante, quand les divers peuples qu’elle inonde de ses produits se sont enrichis à leur tour par l’importation de ses industries. L’Angleterre ne donne-t-elle pas déjà les signes d’une nation aristocrate parvenue à son apogée et qui va décliner ? Après avoir été monopolisées à la fois par quelques nations privilégiées, il est inévitable que les nombreuses inventions de l’industrie moderne, auxquelles les inventions locomotrices ouvrent des débouchés si imprévus, se divisent entre toutes les nations civilisées suivant leurs aptitudes diverses et complémentaires.  \phantomsection
\label{v2p276}Les progrès de la locomotion différencient donc économiquement les peuples tout en les assimulant moralement et intellectuellement. Et cette assimilation, combinée avec cette différenciation, qui les rend nécessaires les uns aux autres, tend finalement à diminuer leur inégalité économique, sinon politique, après une période plus ou moins prolongée d’inégalité grandissante.\par
Les chemins de fer commencent aussi par rendre plus inégale, sur les diverses parties d’un territoire, la répartition de la population qu’elles concentrent dans les villes et raréfient dans les campagnes. Mais on peut prédire à coup sûr que cette période de concentration, qui se prolonge encore, sera suivie d’un mouvement en sens inverse, par lequel la densité des populations urbaines diminuera et celle des populations rurales s’accroîtra. Déjà l’on remarque ce mouvement centrifuge qui se manifeste dans nos capitales et qui s’oppose au mouvement centripète antérieur. A Paris, les quartiers du centre tendent à se dépeupler au profit des quartiers suburbains et de la banlieue ; il en est de même dans toutes les grandes villes, grâce aux tramways électriques et autres moyens rapides de transport. Est-ce que ce ne serait pas là l’indice et l’image en miniature anticipée d’un mouvement centrifuge bien plus grandiose et plus important, qui serait l’inverse de l’émigration des campagnes vers les villes, c’est-à-dire l’exode des villes vers les campagnes ? Qu’on le souhaite ou non, cette réaction est fatale, et, en vertu des mêmes causes, ne peut pas ne pas s’opérer un jour. Déjà, même dans le parti socialiste, on se préoccupe de la dépopulation rurale. « En tant qu’homme de parti politique, dit Bernstein\footnote{ \noindent \emph{Socialisme théorique et social-démocratie pratique}, trad. fr. 1900 (p. 199).
 }, le socialiste constate avec satisfaction le dépeuplement des campagnes et l’immigration des ruraux dans les villes. Ils concentrent les masses ouvrières, sèment la révolte et hâtent l’émancipation politique. Mais,  \phantomsection
\label{v2p277}comme théoricien tant soit peu sérieux, le socialiste sera bien obligé de dire que ce dépeuplement finira à la longue par devenir néfaste... Supposons qu’une grande victoire de la démocratie ouvrière porte au pouvoir le parti socialiste. A en juger d’après les précédents, la première conséquence de cet événement serait de grossir encore le flot des envahisseurs des grandes villes, et il est quelque peu douteux que les \emph{armées industrielles pour l’agriculture} se laisseront alors plus bénévolement diriger vers les campagnes qu’elles ne le firent en 1848. » Bernstein semble croire ici que par la force seule on parviendra à repeupler les champs. Mais ne semble-t-il pas plutôt que, par suite du progrès même des communications, et des transformations de l’agriculture, le séjour des champs, mieux pourvu de ressources intellectuelles, le travail des champs, moins fatigant et plus intéressant, deviendront propres à attirer ou à retenir ceux qu’à présent attirent plutôt les emplois de bureaucrates ou de commis de magasins, objet d’une compétition effrénée ? Pour le moment, il est bien certain que, dans chaque État, il s’opère une désagrégation de la population d’autant plus rapide que les moyens de communication sont plus développés ; et, à partir de la capitale comme centre, on voit vaguement s’y dessiner une décomposition de la population en zones concentriques de moins en moins fines, cultivées, dépravées, riches et denses. Cela rappelle l’\emph{État isolé} de Thünen, avec ses zones concentriques de cultures différentes. Et, sur une échelle plus grande encore, on aperçoit un petit nombre de nations qui servent de centre aux autres dans une immense région, où la densité de population et le degré de culture se mesurent presque à l’éloignement de cette brillante élite. Mais ne voyons-nous pas aussi, par les émigrations coloniales, les États surpeuplés et surcultivés réagir contre cette tendance et nous faire espérer une ère future où, par l’émancipation et la prospérité des colonies, émules de leurs métropoles, une tendance au nivellement international se fera jour ?
 \phantomsection
\label{v2p278}\subsection[{III.3. Les développments de l’imagination économique}]{III.3. Les développments de l’imagination économique}\phantomsection
\label{l3ch3}
\subsubsection[{III.3.a. Quatre points de départ des évolutions économiques, d’après Gumplowicz. Critique de son idée.}]{III.3.a. Quatre points de départ des évolutions économiques, d’après Gumplowicz. Critique de son idée.}
\noindent Après avoir montré les causes internes et externes de l’invention et ses conséquences plus ou moins favorables à l’harmonie sociale, nous avons à compléter et illustrer ces considérations en indiquant quelques-unes des pentes générales suivies par le génie industriel et commercial dans son développement ou plutôt dans ses développements multiples. Si nous nous rappelons que le capital essentiel c’est l’invention, nous reconnaîtrons l’importance de ce sujet qui consiste à étudier les transformations du capital humain.\par
J’accorde volontiers à M. Gumplowicz — si souvent profond en dépit de ses paradoxes — que les nombreuses et dissemblables évolutions des sociétés primitives peuvent se diviser, grosso modo, en un petit nombre de classes différentes d’après leur point de départ, qui a dû être en effet très divers par suite de la diversité de leurs aptitudes natives combinées avec celles de leur habitat, du sol, du climat, de la flore et de la faune ambiante. D’après ce brillant écrivain, les sociétés primitives sont nées adonnées, les unes à la \emph{cueillette}, — d’autres à la \emph{pêche} — d’autres à la \emph{chasse} — d’autres au \emph{vol.} Chacune d’elles a évolué différemment. Les premières auraient passé de la cueillette à l’agriculture, ce qui signifie simplement à nos yeux que la proportion des inventions relatives à la domestication des plantes l’aurait emporté chez elles, dès le début de leur \emph{capitalisation}, sur la proportion des inventions relatives à la domesticité des animaux. — Les  \phantomsection
\label{v2p279}secondes sociétés auraient évolué de la pêche à la navigation, ce qui veut dire que, plus tôt chez elles que chez les précédentes, les inventions relatives à la captation de certaines forces physiques, courants des rivières, vents, l’auraient emporté sur les autres. — Les troisièmes auraient évolué de la chasse à l’art pastoral, ce qui veut dire que les inventions relatives à la domestication des animaux seraient restées non pas uniques mais dominantes chez elles. — Enfin, les dernières auraient évolué du vol à la guerre ; ce qui signifie, à nos yeux, que, chez ces dernières, non les moins progressistes, ni les moins hautes, les inventions qui consistent à combiner des inventions antérieures par l’\emph{association} des hommes en qui elles s’incarnent, autrement dit l’art d’employer pour une même fin (la victoire d’abord, le luxe et le confort plus tard) toutes les ingéniosités humaines déjà connues, — auraient joué un rôle prépondérant.\par
Mais il y aurait lieu, à ce même point de vue, de subdiviser et de diversifier. Suivant la nature de la cueillette, c’est-à-dire des fruits que le sol produit spontanément, baies ou racines ici, dattes ailleurs, l’évolution a dû varier, et l’école de Le Play, tout en exagérant l’importance de ce problème sociologique, a rassemblé d’utiles matériaux pour le résoudre. La pêche aussi, suivant qu’elle est fluviale, lacustre ou maritime, et suivant qu’elle s’exerce dans un port ou dans un golfe méditerranéen, se développe différemment. Il en est de même de la chasse. Enfin, le vol, qui comprend des pillages de troupeaux ou de récoltes, des razzias d’esclaves, des rapts de femmes, etc., prend un sens de développement différent suivant la nature dominante de ses objets habituels.\par
La classification de Gumplowicz porte d’ailleurs la marque du dogmatisme et du radicalisme de son auteur : absolue et tranchante, elle n’admet pas de rapports entre les quatre compartiments où elle enferme le développement de l’humanité, divisé en cloisons étanches. Rien de plus contraire aux  \phantomsection
\label{v2p280}faits observés. En outre, il y a ici une lacune manifeste : où figurent dans ce tableau les sociétés industrielles ? Le développement de la phase industrielle, d’où procède-t-il, dans cette théorie ? Il me semble vraisemblable d’admettre que, dès les plus hauts temps, certaines tribus se sont distinguées — comme, de nos jours encore, certaines tribus sauvages — par leur ingéniosité technique, par leur habileté presque native à utiliser les forces mécaniques et même chimiques, aussi bien que vivantes et humaines, dans un but industriel, le feu pour la poterie et la métallurgie, la pesanteur pour la maçonnerie, le travail de la fileuse et du tisserand pour la fabrication de la toile, etc.\par
Puis, il est clair que ces évolutions multiples sont, comme les rivières différentes d’une même vallée, convergentes vers une même embouchure finale, qui est une grande nation quelconque. Il n’est pas une grande nation dans laquelle ne viennent se confondre ces développements différents, plus ou moins harmonisés dans son sein, y compris le pillage même et la guerre. Toutefois, ces divers développements sont très inégalement répartis entre les grandes nations, et l’on peut encore distinguer celles où domine le caractère \emph{agricole}, ou \emph{maritime}, ou \emph{pastoral}, ou \emph{industriel}, ou \emph{belliqueux} — sans oublier que souvent celle où l’un de ces caractères est le plus accentué à présent le présentait au moindre degré il y a quelques siècles ; par exemple l’Angleterre, si rurale, si peu manufacturière, si timide navigatrice au moyen âge.\par
Cette \emph{multiformité}, de constitution et d’évolution, des sociétés même primitives, a l’avantage de nous faire apercevoir sous un aspect plus harmonieux, ou moins incohérent, qu’il ne semble de prime abord, leurs premiers contacts. Nous sommes surpris de constater, dès les temps préhistoriques, des traces manifestes de commerce et d’échange avec l’étranger, de troc international. C’est que, en réalité, en raison de la diversité même de leurs occupations,  \phantomsection
\label{v2p281}les diverses tribus pouvaient se rendre de grands services et se servir de débouchés les unes aux autres. Les Iroquois s’adonnaient à la culture des champs, les Hurons à la chasse, et, s’étant alliés, ils échangèrent leurs produits. A cela a dû servir l’agriculture naissante : elle a servi de débouché à la chasse ou à la pêche surabondante et aussi à l’élevage des troupeaux. S’il était vrai, conformément à la formule d’évolution unique et rigide qui a eu cours, que la phase agricole eût \emph{succédé} à la phase pastorale et à la chasse ou à la pêche, l’agriculture, en apparaissant, aurait dû engager une lutte acharnée contre les habitudes, enracinées depuis des siècles, de la chasse, de la pêche, de l’art pastoral, et l’on ne comprendrait pas qu’elle fût parvenue à les expulser. Mais elle s’est présentée en alliée encore plus qu’en rivale, et a dû être à ce titre fort bien accueillie, comme l’industrie plus tard.\par
— Est-il nécessaire de dire que j’attache aux soi-disant lois d’évolution formulées par Gumplowicz une importance toute relative ? Elles sont toutes fondées sur de simples hypothèses. De quel droit affirmer que l’agriculture procède de la cueillette, la navigation de la pêche, l’art pastoral de la chasse, la guerre du vol ? Aucun document historique ne nous permet de remonter à l’origine de ces institutions ; dès les temps les plus reculés où nous puissions atteindre, l’agriculture fleurit en Égypte et en Chine ; l’art pastoral en Arabie ; la navigation en Phénicie, la guerre partout. J’ajoute : l’industrie partout ; car, dès l’âge de la pierre éclatée, on trouve des poteries, des racloirs et des aiguilles. Tout ce que nous pouvons dire, c’est qu’il n’y a pas trace d’agriculture avant l’âge de la pierre polie. Mais d’où procède au juste l’agriculture ? Nous n’en savons rien. Rien ne nous prouve que les Égyptiens ont été conduits par la cueillette directement, et non à travers une phase de chasse ou de pêche, à la culture des céréales ; ni que les peuples pasteurs le sont devenus simplement parce qu’ils étaient auparavant des  \phantomsection
\label{v2p282}peuples chasseurs ; ni que tous les peuples marins ont commencé nécessairement par être pêcheurs.\par
Gumplowicz méconnaît ici le rôle essentiel de l’invention individuelle, dont l’apparition est toujours, jusqu’à un certain point, accidentelle, imprévue, impossible à prévoir, et dont les conséquences ont les contre-coups les plus inattendus bien au delà de leur berceau. Ce n’est pas la pêche qui suffit à susciter la navigation, et celle-ci, une fois née, ne se développe pas toute seule. Sa naissance et son développement dépendent de trouvailles heureuses, d’inventions relatives à la métallurgie et au tissage, et, par suite, à l’art des constructions en bois et de la fabrication des voiles ; cela dépend, aussi bien, de découvertes astronomiques faites par des prêtres, non par des pêcheurs, et de la découverte de la boussole, et de la découverte de la vapeur, etc.\par
Ce n’est pas à force de chasser qu’une tribu de chasseurs se transforme en tribu de pasteurs. On n’a vu nulle part s’accomplir ce changement spontané. L’habitude du vol ne suffit pas non plus à donner le génie militaire et la victoire. Et une peuplade a beau, comme dans certaines îles de la Polynésie, avoir vécu pendant des siècles de fruits cueillis sans nulle peine à l’arbre à pain ou à l’arbre à beurre, cela ne lui donne point l’idée capitale et merveilleuse de \emph{semer pour récolter} et d’abord d’épargner une partie des graines cueillies pour servir de semence — prévoyance si extraordinairement répugnante à des primitifs. Cela lui donnera encore moins l’idée, non moins merveilleuse, de labourer. Ne faudra-t-il pas nécessairement que ces deux idées successives éclosent dans quelques cerveaux mieux doués que les autres ou soient importées du dehors ?\par
Ce que l’on a le droit de supposer, c’est que l’habitude de vivre de fruits cueillis a eu pour effet de tourner l’esprit inventif plutôt du côté de la culture des plantes, — l’habitude de la chasse, de le diriger plutôt vers l’apprivoisement des animaux et leur reproduction, — l’habitude de la pêche,  \phantomsection
\label{v2p283}plutôt vers les constructions navales et les expéditions sur mer, — l’habitude du pillage, plutôt vers les armements et la tactique militaire. — C’est possible, c’est même probable. Mais ce qui ôte à cette considération une grande partie de sa portée, et la réduit à fort peu de chose, c’est cette observation que l’on peut faire à chaque pas dans l’histoire d’une branche quelconque de l’activité humaine, — à savoir que les progrès les plus féconds y ont été déterminés par la survenance inopinée d’inventions arrivées là par des chemins détournés et indirects, nées dans d’autres branches du travail, application de vérités théoriques qui ont d’abord servi à satisfaire une passion désintéressée du savoir, une curiosité philosophique d’esprit. Agriculture, navigation, industrie, ont été renouvelées de fond en comble sous nos yeux, par l’invention de la machine à vapeur, application de recherches faites sur l’élasticité des gaz ; et la découverte des explosifs, fruit des veilles d’alchimistes, n’a guère moins servi à l’industrie qu’à la guerre.\par
Nous ne savons si c’est parmi des chasseurs, des voleurs ou des pêcheurs, qu’est née la découverte du bronze, l’observation féconde de la dureté produite par l’alliage en certaines proportions de deux métaux assez mous séparément ; mais nous savons que cette découverte a révolutionné tous les genres de travaux pacifiques ou belliqueux qui s’opéraient auparavant par des outils ou des armes en pierre. Il en est de même, au degré près, de la découverte du fer. Avant les découvertes métallurgiques dont il s’agit, l’invention capitale de la charrue, si merveilleuse, même à ses plus humbles débuts, qu’elle a suscité chez tous les peuples l’apothéose de ses auteurs légendaires, était impossible. C’est cette idée géniale de faire remuer la terre profondément par un animal attelé à un soc, qui a fait franchir à l’agriculture son pas décisif, car, auparavant, elle n’était qu’un superficiel jardinage (abandonné partout aux femmes — d’où peut-être l’une des origines du matriarcat). Notez,  \phantomsection
\label{v2p284}entre parenthèses, que le jardinage, après avoir été l’\emph{alpha} de l’agriculture, semble destiné à devenir son \emph{oméga}, dernier mot de la culture intense...\par
Il est donc nécessaire, quand on cherche à quelles lois obéit l’évolution sociale, en n’importe quel ordre de faits, de ne pas oublier qu’elle est dominée, avant tout, par une suite d’accidents heureux, qui, grâce au rayonnement imitatif et illimité de chacun d’eux, viennent s’insérer les uns sur les autres et font du progrès, non une pente douce et continue, mais une échelle aux échelons superposés et très inégaux. L’évolution sociale, en n’importe quel ordre de faits, je le répète, a lieu de la sorte par voie d’\emph{insertions} successives, et il n’est rien de plus général que cette première constatation. C’est peut-être la seule généralité absolument vraie, sans exception, qui puisse être formulée\footnote{ \noindent Après avoir dû renoncer à la série réglementaire des trois phases chasseresse, pastorale et agricole, jadis érigée en loi invariable d’évolution, nous rabattrons-nous au moins sur cette formule plus humble, que les sociétés ont partout passé d’un état nomade à un état sédentaire ? Ce minimum même d’ordre de succession invariable nous est refusé. Il est démontré que les pasteurs nomades ont été précédés de pasteurs relativement sédentaires, établis dans des huttes. La hutte a précédé la tente, qui est un progrès sur elle. Encore cette observation ne nous autorise-t-elle pas à affirmer que partout l’état sédentaire a précédé l’état nomade. Nous voyons, en fait, ces deux états — dont la distinction, d’ailleurs, est d’une importance majeure en sociologie — alterner plusieurs fois. L’état nomade ne tend-il pas, sous des formes neuves et plus amples qui l’ont transfiguré en le ressuscitant, à se répandre et se généraliser de nouveau dans nos sociétés ultra-civilisées ? Est-ce que l’invention de la navigation à vapeur n’a pas eu pour effet de développer dans des proportions inouïes les pérégrinations à travers les mers de ces nomades nouveaux qu’on appelle les marins ? Est-ce que les Océans conquis de la sorte, mais collectivement appropriés, non individuellement, comme l’est le désert par les Touareg, ne sont pas un autre et plus immense Sahara liquide où circulent des caravanes infiniment plus nombreuses que celles des grands chameliers africains ? Est-ce que la vie des riches cosmopolites qui vont promenant leur oisiveté vagabonde et luxueuse d’un bout du monde à l’autre n’est pas un retour à la vie errante des patriarches hébreux ? — Ainsi, la seule généralisation qui, ici, mérite le nom de loi, c’est l’importance prépondérante des inventions et leur rayonnement imitatif suivant certaines règles formulables.
 }.
\subsubsection[{III.3.b. Généralisations partielles relatives aux voies habituellement suivies par le génie inventif. Sa direction principale vers la captation des forces animales d’abord, puis végétales, enfin physico-chimiques. Exemples.}]{III.3.b. Généralisations partielles relatives aux voies habituellement suivies par le génie inventif. Sa direction principale vers la captation des forces animales d’abord, puis végétales, enfin physico-chimiques. Exemples.}
\noindent — Mais hâtons-nous de dire que, sous le couvert de cette généralisation dominante, beaucoup d’autres, partielles et instructives aussi, se présentent à nous.\par
D’abord, le fait que le génie inventif de l’homme se tourne tantôt d’un côté tantôt d’un autre, aiguillé ou orienté dans  \phantomsection
\label{v2p285}telle ou telle voie par les besoins les plus urgents d’une époque ou d’un pays, a une importance extrême. Le génie de l’invention a lui-même quelque chose d’imitatif et de moutonnier, et suit des courants de mode. Et, si capricieux que soient ses déplacements, il n’est pas impossible d’esquisser un certain ordre, généralement observé, de ses pérégrinations successives.\par
Il est certain, par exemple, que nous avons passé l’âge où l’esprit d’innovation se tournait de préférence vers la domestication des animaux. Mais cet âge a existé, nous n’en saurions douter. Pendant de longs siècles de la préhistoire, il n’est pas douteux que le génie créateur, au lieu de viser surtout, comme en des temps plus reculés encore, au perfectionnement du langage, au progrès des moyens d’expression, ou, comme à notre époque actuelle, au progrès des moyens de locomotion ou de confort, s’est donné pour point de mire la domestication et l’élevage des animaux\footnote{ \noindent Et, auparavant, la chasse. Quand les Peaux-Rouges, dit Robertson « ont entrepris une chasse, ils sortent de cette indolence qui leur est naturelle ; ils développent des facultés d’esprit qui demeuraient presque toujours cachées, et deviennent actifs, constants, infatigables ». Là, ils montrent, une « fécondité d’invention » surprenante.
 }. Et il faut que le déploiement du génie découvreur dans cet ordre particulier de problèmes ait été bien considérable, puisque — M. de Mortillet\footnote{ \noindent \emph{Origines de la chasse, de la pêche}, etc. (1890).
 } en fait la remarque avec surprise, après Isidore-Geoffroy Saint-Hilaire — l’homme, dès ces époques si lointaines « est arrivé à domestiquer toutes les espèces domesticables dans la mesure de leur tempérament et de leur caractère ». En effet, comme le dit Geoffroy Saint-Hilaire, « aux animaux auxiliaires et alimentaires, antérieurement nourris dans nos fermes et dans nos basses-cours, pas un seul n’est venu \emph{s’ajouter \phantomsection
\label{v2p286} depuis trois siècles} », c’est-à-dire depuis la découverte de l’Amérique ; et ceux qui, à cette époque, ont grossi le troupeau de l’ancien continent, tels que le dindon et le cochon d’Inde, étaient déjà, depuis longtemps, domestiqués en Amérique\footnote{ \noindent Cependant Roscher estime que les Peaux-Rouges auraient pu apprivoiser le bison, proche parent de notre bœuf, et qu’ils ne paraissent pas y avoir songé. Mais peut-être des idees superstitieuses sont-elles l’explication de cette exception à la règle. Ajoutons que les Peaux-Rouges n’ont pas su non plus, ou n’ont pas voulu, apprivoiser le renne, qui est resté sauvage sur leur continent.
 }. — C’est après l’âge paléolithique (de la pierre éclatée), dans l’âge néolithique (de la pierre polie) qu’ont eu lieu ces apprivoisements d’animaux, car, au temps du silex taillé, le chien même n’était pas domestiqué. Mortillet conclut de là que « une fois entré dans cette voie, il (l’homme) a dû faire des essais nombreux (de domestication) fréquemment répétés dans les circonstances et les milieux les plus divers » à peu près comme nous avons fait les essais de vélocipèdes et d’automobiles les plus variés avant de nous arrêter aux types définitifs — provisoirement définitifs.\par
Mais cet épuisement complet d’une \emph{mine} d’inventions par le génie inventif acharné à son exploitation, est rare et exceptionnel. Ce phénomène s’explique ici par la simplicité du problème et le facile rassemblement de toutes ses données, fournies par la faune d’une région. Encore faut-il observer que l’art pastoral n’est point réduit pour cela à l’immobilité, et c’est bien à la légère que M. Demôlins le condamne à ne comporter aucun progrès. Il écrit tranquillement : « ce travail simple, traditionnel et improgressif, donne naissance aux populations les plus simples, les plus traditionnelles et les plus improgressives. » Oui, jusqu’à ce que quelqu’un, parmi ces pâtres songe à ne plus seulement nourrir le bétail mais à créer, par le croisement et la sélection volontaire, de nouvelles races domestiques. L’élevage et le perfectionnement des races existantes, la création de nouvelles races de chevaux, de bœufs, de moutons, de volailles, sont aussi de l’art  \phantomsection
\label{v2p287}pastoral, et du plus élevé, à mettre au-dessus de l’agriculture la mieux outillée.\par
Il semble que le génie inventif ne se soit spécialisé tout à fait, pour un temps, dans la domestication des plantes, dans le progrès de l’agriculture, qu’après la période de sa monomanie pastorale. Là aussi l’intensité de l’obsession générale se démontre par le fait que la presque totalité des plantes domesticables ont été domestiquées depuis bien des siècles déjà, du moins dans l’importante famille des graminées. Ce n’est pas tant par la culture de nouvelles espèces végétales que par la variation ingénieuse des anciennes espèces cultivées, et par leur meilleure culture, que s’est opéré le progrès agricole.\par
La caractéristique du monde moderne, au point de vue qui nous occupe, c’est sa préoccupation obsédante de capter les forces physico-chimiques, agents de l’industrie nouvelle, transfigurée par eux. De ce côté s’est donné carrière le génie inventif des trois derniers siècles, surtout du dix-neuvième, où l’hallucination de cette idée fixe a atteint son apogée. Le problème étant des plus difficiles, les données en étant très dispersées, et non rassemblées et à la portée de la main comme l’étaient les espèces animales ou végétales quand il s’agissait de résoudre le problème de la domestication la plus complète possible des animaux et des plantes, il est moins probable que le génie humain parvienne à épuiser cette nouvelle mine, même en ce sens tout relatif où l’on peut dire, avec plus d’apparence du reste que de vérité, qu’il a épuisé les deux autres. Peut-être viendra-t-il un moment où l’on ne découvrira plus aucune nouvelle force physique ou chimique vraiment utilisable, mais il restera, pendant un temps peut-être indéfini, à tirer un meilleur parti des forces connues.\par
Ainsi, bien que l’on rencontre, dès les plus hauts temps, des potiers et des agriculteurs à côté de pasteurs et de pêcheurs, il y a lieu de penser que le grand effort collectif du génie  \phantomsection
\label{v2p288}humain en vue de manier et de dompter les forces animales, a précédé d’abord sa grande conspiration en vue de dominer les forces végétales, et ensuite en vue de s’approprier les forces de la nature inanimée. On peut voir une confirmation de ce lent déplacement de l’esprit inventif dans cette règle générale, déjà indiquée plus haut, — non sans exception d’ailleurs, et peu digne du titre pompeux de \emph{loi} — que la plupart des industries ont traversé des phases successives dont la première a été l’utilisation de substances ou de forces d’origine animale (en particulier, d’origine humaine), la deuxième, de substances ou de forces d’origine végétale ; la troisième, de substances ou de forces de nature inorganique. On s’éclaire avec de la graisse d’abord, puis avec de l’huile végétale, enfin avec le gaz ou l’électricité. On s’abrite d’abord sous des peaux de bêtes cousues ensemble et formant tentes, puis dans des maisons de bois, enfin dans des maisons de pierre. On s’habille de fourrures d’abord, ou de tissus de laine, puis de tissus de chanvre ou de coton. On écrit sur du parchemin d’abord, puis sur des feuilles de végétaux, du papyrus, enfin sur du papier de plus en plus chargé de matières minérales. On se sert, pour écrire, de plumes d’oiseaux d’abord, ou de roseaux, plus tard de plumes de fer. On se pare d’abord de colliers de coquillages ou de plumages, puis de couronnes de fleurs dans les cheveux ou au sein ou à la boutonnière, enfin de bijoux d’or, de diamants ou d’autres pierres précieuses. On se nourrit de viandes animales (ou humaines) d’abord, puis d’aliments végétaux ; et, si nous en croyons M. Berthelot, un jour viendra où l’alimentation principale sera préparée directement par des chimistes avec de l’hydrogène, de l’oxygène, du carbone et de l’azote combinés dans des cornues, etc., etc. — Est-ce qu’on ne pourrait pas voir même dans les métamorphoses de la médecine une application de cette sorte de règle ? Nous voyons à présent la thérapeutique la plus en vogue consister en agents physiques, hydrothérapie,  \phantomsection
\label{v2p289}électrothérapie, et en drogues minérales. Mais, il y a moins d’un siècle encore, les drogues végétales, les vertus des simples, étaient surtout en faveur. Or, en remontant plus haut encore, ne serait-il pas possible de découvrir une époque où les principaux médicaments étaient des vertus merveilleuses attribuées soit à l’absorption ou à l’attouchement de certaines substances animales\footnote{ \noindent Est-ce que certains \emph{totems} n’auraient pas eu pour origine quelque superstition de ce genre ? Est-ce que les \emph{dents} de certains animaux, ou leurs \emph{griffes}, servant d’amulettes, ne seraient pas un vestige survivant de cette primitive et mystique pharmacopée ?
 }, soit à la manducation de certaines parties du corps humain (ce qui serait une des explications de l’antropophagie) ou au regard, au contact, à l’action suggestive d’un homme prestigieux ?\par
Quoi qu’il en soit de cette hypothèse vraisemblable, la règle assez générale dont il s’agit se comprend bien. Il est naturel que l’homme ait commencé par employer les instruments les plus proches de lui, les plus semblables à lui, et qu’il les ait remplacés peu à peu par des instruments de plus en plus difficiles à découvrir, il est vrai, mais de plus en plus maniables, dociles et infatigables\footnote{ \noindent C’est ce qui explique, aussi bien que la règle, les exceptions à la règle, et d’abord les cas, si fréquents, où il manque un terme à la série des trois termes. On a substitué aux moulins à bras les moulins à eau ou à vapeur ; où est ici le terme \emph{végétal ?} On a passé des navires à rames aux navires à voile ou à vapeur ; \emph{idem ?} Dans le dernier exemple même il y a interversion : car, dans le remplacement de la navigation à voile par la navigation à vapeur, on peut voir le contraire du remplacement des forces d’origine végétale par des forces d’origine physique ou chimique. En effet, le vent est une force physique, et le charbon, d’où procède la vapeur qui meut les navires modernes, procède d’antiques plantes.
 }. Remplacés en partie seulement ; car, à vrai dire, les nouveaux agents se sont bien plus ajoutés que substitués aux précédents. Les progrès de la machinofacture ne se sont pas opérés aux dépens de l’agriculture ni de l’élevage des bestiaux. Les machines à vapeur n’ont nullement fait diminuer le nombre des chevaux de course ou de trait, ni des bœufs, ni des sacs de blé. Il n’est même pas exact de dire que l’esclavage ait été aboli dans les pays civilisés, et que les services qu’il rendait aient disparu \phantomsection
\label{v2p290} devant l’industrie nouvelle qui aurait fait des cours d’eau, de la vapeur, de l’électricité, les seuls et vrais esclaves de l’homme moderne. En fait, l’esclavage s’est immensément généralisé, mais en se mutualisant, et il n’est pas d’homme qui ne se serve des hommes qu’il sert, qui n’ait pour instruments de ses desseins les hommes dont il est l’outil vivant.\par
Dans les « Essais sur l’histoire de la civilisation russe », par M. Milioukov, on voit que les divers territoires vierges successivement conquis et envahis par les moscovites, à partir de Moscou, centre et âme de cet immense pays, ont été exploités dans un ordre à peu près invariable, qui s’est répété jusqu’à nos jours et où l’on peut voir la reproduction abréviative de phases traversées par l’Europe occidentale elle-même à des époques préhistoriques. « On commence, nous dit-il, par absorber les richesses zoologiques, la population animale des forêts et des eaux ; on s’en prend ensuite aux richesses botaniques et à celles du sol ; puis, c’est le tour des richesses minérales du sous-sol. Dans l’Europe occidentale, on s’est attaqué à ces ressources naturelles et on les a même épuisées en partie dans la préhistoire. En Russie, ce processus de la dévastation successive, caractérise toute sa vie économique passée, et même elle lui est propre jusqu’à nos jours. » Dans des chartes du {\scshape xiv}\textsuperscript{e} siècle, des villages entiers sont appelés \emph{castoriens} ou \emph{apiculteurs}, parce qu’ils s’occupaient exclusivement de faire la chasse aux castors ou de cueillir le miel des ruches sauvages des forêts. Les villages castoriens sont les plus anciens. Au {\scshape xvii}\textsuperscript{e} siècle, le nombre de ces villages diminue, et ceux de chasseurs ou de pêcheurs se transforment en villages de laboureurs. « En résumé, on usait jadis des produits gratuits de la nature, que l’on épuisa d’abord au centre vers la fin du {\scshape xvi}\textsuperscript{e} siècle ; et, plus on avance vers le sud ou l’ouest, plus ce même mode d’exploitation est récent. Ce n’est qu’après la disparition de ces richesses zoologiques que la population passait définitivement à l’état agricole », commençant \phantomsection
\label{v2p291} par faire de l’agriculture extensive et ne se résignant que par force à cultiver plus tard intensivement.\par
En tout ceci, on voit que ce n’est pas toujours faute de connaître les procédés agricoles qu’une nation ou une peuplade se livre à la chasse ou à l’art pastoral pour tout travail productif ; c’est tout simplement, dans les temps modernes du moins, parce que, en vertu de la loi du moindre effort, elle trouve plus de profits immédiats, dans un pays encore très giboyeux ou très pacager, à chasser ou à paître qu’à labourer. Et nous pouvons supposer qu’il en a été souvent ainsi dans la préhistoire même. — Ainsi, il ne serait pas vrai de dire que l’apparition ou l’importation des inventions successives suffit à expliquer les transformations industrielles. Sans doute, la première condition pour faire de l’agriculture, c’est que la charrue ait été imaginée ; pour faire de l’agriculture plus intensive, que des plantes fourragères aient été importées, etc. Mais la charrue ou les plantes fourragères ont beau être connues, si on croit avoir un avantage à n’en pas faire usage, elles sont comme n’étant pas. Il y a à distinguer ici deux sortes d’évolution, qu’on peut comparer, si l’on veut, l’une à ce que les naturalistes appellent la \emph{phytogénèse}, l’autre à ce qu’ils nomment l’\emph{ontogénèse :} en premier lieu, l’ordre dans lequel se sont, pour la première fois, succédé les divers genres de travaux, fort lentement, par suite de l’apparition des découvertes venues à leur heure (ou venues plus tôt, mais sans succès alors) ; et, en second lieu, l’ordre dans lequel ils se succèdent de nouveau, bien plus rapidement cette fois, et en supprimant bien des détours du premier chemin, en vertu de circonstances qui replacent momentanément les peuples colonisateurs ou conquérants dans les conditions primordiales\footnote{ \noindent C’est ainsi, par exemple, que les colons civilisés reviennent à la phase de l’industrie domestique, en rétrogradant même parfois jusqu’à l’indivision primitive des pâturages et des forêts.
 }.
 \phantomsection
\label{v2p292}\subsubsection[{III.3.c. Agrandissement parallèle des luîtes et des harmonies. Tendance du génie inventif à se tourner vers Jes formes d’association.}]{III.3.c. Agrandissement parallèle des luîtes et des harmonies. Tendance du génie inventif à se tourner vers Jes formes d’association.}
\noindent — En somme, les transformations du capital spirituel ont été, avant tout, des accroissements de ce capital. Indiquons aussi qu’il a eu deux manières différentes de s’accroître : 1\textsuperscript{o} l’apparition de nouvelles inventions ; 2\textsuperscript{o} le rayonnement imitatif des anciennes. Pour chaque pays considéré à part, cette seconde cause équivaut à la première, car l’importation, pour la première fois, d’inventions qui n’y avaient pas encore pénétré, y accroît le capital spirituel tout comme si ces inventions y avaient spontanément apparu. Mais, pour l’humanité civilisée prise dans son ensemble, les deux procédés d’accroissement diffèrent profondément. Les deux sont solidaires et s’influencent réciproquement ; mais le second, qui ne peut commencer à naître qu’après le début du premier, peut continuer à se développer après l’épuisement de celui-ci.\par
De là il résulte une conséquence manifeste, le plus constant, le plus certain et le plus important phénomène que nous présente l’histoire économique de l’humanité : c’est la tendance — entravée souvent, par des catastrophes belliqueuses, des invasions, des épidémies, des obstacles quelconques, mais toujours renaissante, et, à travers ses refoulements momentanés, finalement triomphante, — la tendance à un agrandissement incessant de la reproduction des richesses et du marché où elles se répandent. De là les étapes de l’économie domestique — de l’économie urbaine — de l’économie nationale — de l’économie continentale et mondiale enfin (où nous courons). La diffusion incessamment rayonnante des besoins similaires n’a pas pu ne pas élargir sans cesse, à travers les haies ou les murs des frontières, la clientèle de chaque industrie, et, par suite, la forcer à s’agrandir pour répondre à cet élargissement. De là les phases de la petite, de la grande, de la \emph{plus grande} industrie. A chaque invention qui crée ou développe un nouveau \phantomsection
\label{v2p293} besoin, d’abord exceptionnel, puis de plus en plus répandu et généralisé, ce phénomène se répète, et de ces rayonnements entre-croisés de besoins multiples, suscitant des industries grandissantes, se forme la dilatation totale, majestueuse et irrésistible, du monde économique.\par
A ce phénomène en correspond un autre : la tendance parallèle, d’une part, à l’agrandissement des concurrences, des oppositions économiques, qui ont lieu d’abord entre petites boutiques, puis entre grandes usines, puis entre grands trusts, — d’autre part à l’agrandissement des associations, qui tantôt, passagèrement, s’opposent ainsi les unes aux autres, tantôt s’allient et collaborent à la paix sociale en établissant leur domination souveraine et parfois bienfaisante.\par
Ce dernier aspect de l’évolution économique est le plus digne de notre attention. Le progrès dans l’association est, je crois, le terme final de l’évolution économique, le côté par lequel elle continuera toujours quand les progrès dans la production et l’échange y seront devenus impossibles et que la lutte entre les géants de ses oppositions agrandies démesurément y sera terminée par quelque victoire décisive.\par
Mais, est-ce seulement dans cet avenir éloigné que nous pouvons voir se poser d’urgence le problème dont il s’agit ? En des termes moins absolus, il s’est posé de tout temps, partout où des obstacles quelconques, de nature politique, religieuse, linguistique, ou autre, ont arrêté par une limite momentanée la fécondité de l’esprit inventif et l’expansion d’un groupe d’inventions systématisées, d’un type de civilisation particulière, mûre et achevée, et ont forcé ce type à se replier sur lui-même pour s’harmoniser de mieux en mieux avec lui-même. L’empire romain en était là, depuis deux ou trois siècles, vivant sur un stock industriel et esthétique non renouvelé, avant que le christianisme eût commencé à s’y répandre et y eût apporté un nouveau ferment. \phantomsection
\label{v2p294} Aussi voit-on se multiplier à cette époque les \emph{collegia}, associations — encore bien humbles — d’ouvriers de la même industrie. Au moyen âge, après une courte éruption du génie inventif ou une importation intermittente d’inventions exotiques, d’origine asiatique ou arabe, — en architecture, en verrerie, en draperie, etc. — le progrès industriel s’arrête, mais l’ère des corporations s’ouvre ; de toutes parts, pullulent et fleurissent d’harmonieuses confréries — souvent en lutte les unes avec les autres, il est vrai — dont chacune donne le spectacle d’une harmonie fraternelle. Et, de nos jours, le besoin d’association n’a pas attendu, pour se faire jour, l’épuisement du génie industriel. Il est manifeste cependant que, de tous les besoins nouveaux, il est maintenant le plus vif et le plus prompt à se répandre, et que les moyens de lui donner satisfaction deviennent de plus en plus la préoccupation dominante des esprits novateurs.\par
Cela signifie, à notre point de vue, que, après s’être tourné principalement, jadis vers la domestication des animaux puis des plantes, ensuite vers la captation des forces physiques, le génie inventif et novateur, sans d’ailleurs délaisser aucun de ses anciens domaines, tend à se diriger plus particulièrement vers les nouveaux modes d’association productrice, les meilleurs et les plus vastes.\par
Or, qu’est-ce qu’une nouvelle association de producteurs, un nouveau genre ou une nouvelle variété de collaboration ? C’est une invention aussi, mais une invention qui consiste à en combiner d’autres en ajoutant à leur précédente utilité une utilité nouvelle ou plus grande. Il est vrai qu’une invention quelconque peut être définie aussi une synthèse d’inventions antérieures, et il n’en est pas une qui ne soit de la sorte de l’invention élevée à la deuxième, à la troisième, à la quatrième puissance. Mais, au point de vue de la production des richesses, il importe beaucoup de distinguer le cas où cette synthèse est une fusion apparente de plusieurs  \phantomsection
\label{v2p295}inventions qui semblent se confondre en une invention nouvelle, et le cas où, conservant leur individualité visiblement distincte, elles se fédérèrent sous la forme d’une association entre travailleurs différents. Il importe d’élucider ce point.\par
Remarquons que toute besogne, soit qu’elle s’accomplisse par le fait d’un individu isolé, soit qu’elle s’opère par la collaboration de plusieurs individus indépendants mais solidaires, consiste en une série ou une simultanéité d’opérations intellectuelles, manuelles ou mécaniques, distinctes les unes des autres, et qui ont eu, dans un passé récent ou ancien, parfois dans le haut, très haut passé, des inventeurs distincts, célèbres ou inconnus. Cela est aussi vrai dans le premier cas, celui du travailleur unique, que dans le second cas, celui des collaborateurs multiples. Quand la cuisinière, dans un de nos bourgs arriérés, blute la farine reçue du moulin, la mêle dans l’eau du pétrin, la pétrit, dépose le levain dans la pâte, et porte la pâte levée au four, il y a là une suite de travaux différents qui supposent autant d’inventions géniales : l’idée de séparer le son de la farine, l’idée de fabriquer un blutoir avec son cylindre, son tamis, sa manivelle, l’idée de gâcher la farine dans l’eau, l’idée d’utiliser la levure pour la panification, l’idée de faire cuire la pâte, l’invention du feu, etc.\footnote{ \noindent Autre exemple. Le valet de ferme va prendre à la grange deux bœufs, les attelle au joug, à la charrue, et laboure son champ. Cela implique bien des idées prodigieusement ingénieuses à l’origine : la domestication du taureau, sa castration, l’idée d’ensemencer des céréales, la découverte de la fertilité plus grande des terres remuées, l’invention de la charrue, etc.
 }\par
Ce que la cuisinière ici fait toute seule, plusieurs travailleurs différents le font ensemble quand ce qu’on appelle la division du travail, c’est-à-dire l’union des travaux, a fait des progrès. Cela ne veut pas dire d’ailleurs que la distinction des inventions devienne plus réelle et plus certaine ; elle devient simplement plus visible quand la tâche complexe de la cuisinière d’autrefois se divise et se répartit entre  \phantomsection
\label{v2p296}les divers ouvriers employés dans les minoteries et les boulangeries contemporaines. Citons encore un autre exemple : la fabrication des tissus. Cette opération totale que la femme antique, fileuse et tisserande, accomplissait seule dans son gynécée, se décompose maintenant, se divise et se subdivise entre des travailleurs appartenant à divers métiers indépendants (éleveurs de moutons, cardeurs, fileurs, tisseurs, — producteurs de chanvre, rouisseurs, etc.), et qui sont unis entre eux soit librement soit par leur obéissance commune à une même direction dans un grand atelier. Il est clair que la \emph{division des tâches} entre divers travailleurs n’est que l’image agrandie de la \emph{succession des tâches} accomplies par le travailleur unique d’autrefois. Mais il est non moins manifeste que, au point de vue de la célérité, du bas prix et de l’abondance des produits, cela ne revient pas au même, et que le passage de l’un de ces modes de production à l’autre constitue un progrès immense, dû à l’association.\par
Remarquons que les inventions élémentaires groupées dans une invention complexe, telle que la fabrication du pain ou le tissage de la laine, ont apparu sans lien apparent entre elles : par exemple, pour la panification, l’idée d’ensemencer et de cultiver les céréales, la découverte des effets du levain, la découverte du feu, l’idée de la cuisson des aliments, etc. ; pour le tissage de la laine, l’idée de domestiquer le mouton et de le tondre, l’idée de tresser (d’abord des joncs ou des filaments végétaux), la découverte des colorants, etc. Aussi l’idée de grouper harmonieusement ces inventions élémentaires en une action commune qui les emploie à une même fin par une série logique d’opérations, en dépit de leur ordre chronologique et accidentel d’apparition, mérite-t-elle aussi le nom d’invention. Synthétiser des inventions, c’est une invention encore, au même titre que synthétiser des découvertes de savants en une théorie philosophique est une découverte aussi. Mais, s’il en est  \phantomsection
\label{v2p297}ainsi, si c’est à un inventeur toujours que revient l’honneur de la finalité qui préside à une succession d’opérations accomplies par un même travailleur, à plus forte raison devons-nous assimiler à un inventeur l’homme qui, en organisant un atelier, a le premier fait converger vers un même but des travaux opérés par des travailleurs différents.
 \phantomsection
\label{v2p298}\subsection[{III.4. La propriété}]{III.4. La propriété}\phantomsection
\label{l3ch4}
\subsubsection[{III.4.a. La propriété considérée comme moyen d’adaptation négative, et aussi positive.}]{III.4.a. La propriété considérée comme moyen d’adaptation négative, et aussi positive.}
\noindent Après avoir traité de l’Invention, qui, avec son contraire et son complément, la critique, est la source psychologique de toutes les adaptations économiques, réalisées au dehors par l’échange, la division du travail et l’association sous toutes ses formes, il convient de parler maintenant d’une institution élémentaire, fondamentale, universelle, qui est la condition première et indispensable de l’échange, de la division du travail, de l’association : la propriété.\par
La question de la propriété, à notre point de vue, se présente sous deux aspects : la propriété peut être envisagée, d’abord, comme un moyen \emph{d’adaptation négative} de l’homme à l’homme, ou d’un groupe d’hommes à quelque autre groupe d’hommes, dont les volontés et les jugements cessent, grâce à elle, de s’opposer. Elle peut être considérée, en second lieu, comme un moyen \emph{d’adaptation positive} de l’homme et de la terre, de l’homme et d’un capital, à la production des richesses.\par
Sous le premier rapport, ce qui importe, c’est la nette délimitation, c’est la forte sécurité des propriétés, des champs d’activité et de jouissance réservés aux divers individus ou aux divers groupes, pour qu’ils ne se heurtent point ou se heurtent le moins possible. Ce vœu social sera d’autant mieux rempli que la force publique sera plus complètement au service d’un système de droits mieux définis et plus unanimement admis auxquels se subordonneront  \phantomsection
\label{v2p299}les désirs contraires. Sous le second rapport, ce qui importe, ce n’est point la limite des propriétés, mais la nature des choses appropriées et la convenance de cette nature avec celle des personnes propriétaires. A cet égard, le vœu social ne se réalise que dans la mesure où la terre et les capitaux se trouvent aux mains de ceux qui sont les plus aptes à les faire fructifier dans l’intérêt général. Cette rencontre d’une personne avec l’instrument de travail qui convient le mieux à ses aptitudes peut être due, parfois, assez rarement, à la conquête et à la spoliation du vaincu, bien plus rarement encore au jeu et à la spéculation, le plus fréquemment à la transmission héréditaire des biens ou à leur libre aliénation. Elle peut s’opérer aussi par voie de distribution administrative, de répartition par l’État, comme la nomination des fonctionnaires. Dans un cas comme dans l’autre, il y a \emph{appropriation}, car, dans aucun état social, fût-il le plus collectiviste qui puisse être, il ne saurait être question d’abolir entièrement la propriété individuelle. Mais la question est de savoir si la transmission des propriétés des morts aux vivants, ou des vivants aux vivants, par la voie privée — toujours, d’ailleurs, conformément aux lois, — est plus ou moins propre que leur transmission par la voie publique et officielle, par décret, à obtenir le maximum d’adaptation négative et d’adaptation positive, c’est-à-dire à diminuer les conflits individuels résultant de ces mutations de biens, et à augmenter la production générale des richesses.\par
Encore n’est-ce là que le côté économique de la question de la propriété. Par son côté moral, politique, sentimental, elle est bien plus intéressante. Malgré tout, il faudrait songer à supprimer l’héritage, s’il était prouvé que la transmission héréditaire des biens va à l’encontre de la double adaptation dont il s’agit. Mais en est-il ainsi ? Quand on songe, il est vrai, à tous les procès et à toutes les haines de famille provoqués par les successions et les partages, on se demande  \phantomsection
\label{v2p300}si la paix sociale ne serait pas intéressée à ce que les biens des morts vinssent s’engloutir dans le trésor public comme les fleuves dans la mer, pour être ensuite attribués officiellement à de nouveaux usufruitiers, comme les emplois laissés vacants par le décès des titulaires. Jamais, en effet, la nomination aux fonctions publiques ne donne lieu à des débats judiciaires. Mais, en fait de conflits perturbateurs de la paix publique, il n’y a pas que les procès, et ce ne sont pas les plus graves. Le trouble causé par les litiges domestiques ne sort pas des limites de la famille ; le mécontentement causé par la nomination d’un candidat à un emploi parmi cent autres se répand bien plus loin et met plus de temps encore à se calmer. La plupart des révolutions naissent d’une accumulation de griefs pareils. Que serait-ce si aux compétitions déjà existantes, venait à s’ajouter la masse énorme des convoitises surexcitées dans tout le pays par cet immense butin annuel des biens des morts à distribuer ?\par
Voilà pour l’adaptation négative. Quant à l’adaptation positive, y a-t-il lieu de penser que les fermiers ou usufruitiers nommés par l’État, par le ministre de l’agriculture, après d’incroyables intrigues parlementaires, seraient de meilleurs agriculteurs que les fermiers ou propriétaires actuels, devenus tels par héritage ou par contrat ? Y a-t-il lieu de penser que les gérants d’entreprises industrielles ou commerciales quelconques, nommés par le ministre de l’industrie et du commerce, dirigeraient mieux les affaires que ne le font nos industriels et nos négociants actuels ? Assurément, rien n’est moins probable. En ce qui concerne l’héritage des biens ruraux, il n’est pas douteux qu’il a eu pour effet de former une classe paysanne merveilleusement adaptée jusqu’ici à son genre de travail. C’est là un exemple typique des résultats qu’on peut attendre de l’hérédité professionnelle. La force vitale de l’hérédité est une des grandes forces de la nature que la société doit capter, s’assujettir et faire travailler à son profit. Par des lois intelligentes sur la transmission  \phantomsection
\label{v2p301}des biens, mobiliers ou immobiliers, urbains ou ruraux, la société atteint ce but, quand elle le poursuit avec une conscience claire. Le malheur est que, loin d’être toujours dominée, comme elle le devrait, par cette grande préoccupation d’assurer l’adaptation positive et progressive dont il s’agit, la législation s’inspire souvent de principes contraires, tels que le souci aristocratique d’une hiérarchie sociale à maintenir par des lois de substitution et des majorats, par exemple, ou, à l’inverse, le souci démocratique de l’égalité. En Angleterre, où la législation a toujours visé à perpétuer les effets de l’usurpation primitive du sol par les familles conquérantes qui se le sont divisé, une classe de paysans propriétaires, très prospère, était parvenue cependant à se former au {\scshape xvi}\textsuperscript{e} siècle : elle aurait pu faire envie aux pauvres manants français de la même époque. Mais, depuis le {\scshape xvii}\textsuperscript{e} siècle, par suite de nouvelles mesures législatives, révolutionnaires, des actes de clôture, notamment, au {\scshape xviii}\textsuperscript{e} siècle, la grande propriété s’est peu à peu [{\corr substituée}] à la petite, qu’elle a expulsée. On suit les étapes de cette expulsion et de cette transformation sociale en comparant à diverses dates successives, depuis lors, dans une localité donnée, le nombre des propriétaires, qui a été en diminuant, et celui des fermiers, qui a été en augmentant\footnote{ \noindent Voir dans le livre de M. Métin sur le \emph{Socialisme en Angleterre} (1897) un tableau statistique, p. 149, relatif à la paroisse d’Abbay Quarter (Cumberland).
 }. Il n’est pas surprenant que, dans ces conditions, la question de la nationalisation du sol soulève au-delà de la Manche plus d’agitation peut-être encore que celle de la socialisation du capital. Le livre fameux de Henry George a eu là-bas un retentissement dont il ne nous est venu qu’un faible écho. Il n’y a pas à nier les maux qu’il signale ; mais le remède proposé n’en créerait-il pas directement, et surtout par ses contrecoups indirects, de plus grands encore ? Une transformation profonde du droit de propriété, au point de vue des lois qui  \phantomsection
\label{v2p302}règlent les successions et les testaments, s’impose à nos voisins. C’est tout ce qu’on peut et doit accorder aux partisans de George.\par
La préoccupation égalitaire, dans les lois et dans les mœurs, a eu, en France, des effets beaucoup moins fâcheux : le morcellement excessif du sol entraîne moins d’abus sociaux que le \emph{landlordisme.} Cependant l’égalité des partages, en fait de biens ruraux, conduit à des inconvénients majeurs ; en fait d’usines, d’ateliers, de grandes maisons de commerce, elle est peut-être encore plus désastreuse. Pour y remédier, on a proposé la liberté testamentaire, qui laisserait au père de famille, meilleur juge des aptitudes de ses enfants que ne pourrait l’être l’État, le choix de son successeur. Le malheur est que les mœurs de certains pays, tels que le nôtre, sont devenues aussi opposées que la loi à l’inégalité des parts, entre cohéritiers. La loi nouvelle, dans un pareil état social, devrait donc au moins interdire de morceler les exploitations.\par
Le partage égal n’est inoffensif qu’en fait de capitaux. Quand le propriétaire d’un fonds ne le cultive pas lui-même ou du moins n’en surveille pas l’exploitation de près, la transmission de ce fonds à ses descendants se comprend moins, au point de vue dont nous parlons. Toutefois, même ici, l’hérédité légale n’a-t-elle pas pour effet de renforcer certaines aptitudes ? Les écrivains socialistes, tels que Kautsky, rendent hommage aux progrès agricoles dont la grande propriété héréditaire a été, en divers pays, l’initiatrice. A la condition qu’elle soit l’exception et non la règle, son maintien se justifie par la loi de l’écoulement des exemples de \emph{haut en has.} Sans l’exemple périlleux et fécond, donné par les grands propriétaires français, dans la lutte contre le phylloxera, ce fléau terrible n’aurait pu être refoulé. Si l’État avait dû prendre cette initiative, il est à craindre qu’elle eût été moins heureuse, et infiniment plus coûteuse.\par
C’est pour la succession aux biens mobiliers que la suppression \phantomsection
\label{v2p303} de l’héritage causerait le moins de maux, car il ne faut point d’aptitude spéciale, semble-t-il, pour dépenser un capital ou même pour le placer. Mais, précisément, c’est ici que la suppression de l’héritage présenterait le plus de difficultés pratiques, et qu’il serait le plus facile de l’éluder. En fait, on aurait beau édicter toutes les lois possibles, les parents trouveraient bien moyen de faire passer aux mains de leurs enfants la majeure partie de leurs capitaux ; et ne serait-il pas scandaleux de voir la maison paternelle, le champ paternel, le corps terrestre d’une famille rurale, passer à des étrangers, quand les billets de banque et les titres au porteur d’un millionnaire resteraient la propriété des siens ? Ce serait un nouveau privilège, le plus criant et le plus incompréhensible de tous, conféré à la fortune mobilière. Ajoutons que, alors même qu’on pourrait aisément abolir l’héritage en fait de biens mobiliers, il y aurait des raisons spéciales, tirées de l’utilité générale, de n’y pas toucher. On peut supprimer la succession héréditaire aux biens-fonds sans détruire ou altérer notablement les biens-fonds eux-mêmes. Ils seront plus mal soignés, voilà tout, par le possesseur qui saura ne devoir pas les transmettre à sa famille. Mais, pour les capitaux, il en sera autrement. Sous prétexte de les répartir plus équitablement à la mort d’un homme, on les détruira, ou plutôt on les empêchera de naître de son vivant. Le détenteur d’un capital, s’il sait que ses enfants ne seront pas appelés à en jouir après lui, ne se gênera pas pour le dissiper. Ainsi il sera nécessaire de maintenir l’héritage pour les capitaux mobiliers, si l’on ne veut pas les détruire. Mais comment maintenir celui-ci sans maintenir pareillement l’héritage immobilier, qui, à d’autres égards, est infiniment plus respectable que le précédent ?\par
Supprimez l’héritage, ce ne sera plus le hasard de la naissance, il est vrai, mais ce sera le caprice d’un vote populaire, ou le choix arbitraire d’un élu du peuple, qui conférera le privilège de jouir des biens fonciers ou des richesses  \phantomsection
\label{v2p304}mobilières laissés vacants par la mort ou le déplacement de leur détenteur momentané. Est-il prouvé que ce vote ou ce décret de nomination sera plus équitable dans ses préférences que le hasard dans ses désignations aveugles ? Non, car l’héritage à la longue s’accompagne d’une hérédité naturelle ou acquise des aptitudes, comme cela est manifeste pour la classe des paysans, comme cela n’est pas moins certain pour beaucoup d’autres groupes de professions.\par
Les objections soulevées contre le droit à l’héritage individuel pourraient aussi bien être tournées contre le droit à l’héritage national. N’est-ce pas pour des causes identiques au fond, pour des motifs de paix et de sécurité générales, que nous jugeons le sol du territoire français légitimement dévolu aux générations successives des familles françaises ? Et, si nous décidons que le fait d’être le fils de quelqu’un ne donne aucune préférence pour être investi de la possession de ses biens, est-ce que nous ne sommes pas contraints par la logique de décider aussi que le fait d’être Français ne constitue aucun titre sérieux à posséder indivisément le territoire de France, à l’exclusion des Allemands ou des Anglais ? Si, individuellement, le droit à l’héritage est nié, il ne saurait sans contradiction être affirmé collectivement. Le caractère collectif de la propriété héritée ne change rien à la question.\par
Pendant que, prêtant l’oreille à des dissertations théoriques, intéressantes d’ailleurs et utiles à d’autres égards, le pays en arrive à se demander si la propriété individuelle ou familiale des Français sera respectée, il oublie trop que le grand danger pour un peuple toujours, même quand on parle le plus d’internationalisme — pour un peuple même riche et fort et puissamment armé — est d’être dépossédé de la terre qu’il habite. Et l’on peut se demander, question que je crois des plus importantes, si, en \emph{nationalisant} la terre de France, par hypothèse, on aura amoindri ou accru les risques d’une nouvelle amputation de la patrie, consolidé ou  \phantomsection
\label{v2p305}affaibli le lien de la nation française à son sol. C’est par ce côté qu’il convient de juger le collectivisme ; car, s’il devait avoir pour effet de rassurer le patriotisme, il rencontrerait bien moins de résistance à sa réalisation. Or, il ne me semble pas que cette considération lui soit favorable. En devenant collective, est-ce que la propriété de la terre française ne serait pas plus menacée de quelqu’une de ces expropriations violentes ou perfides, brutales ou dissimulées, à l’usage des États, ces cannibales collectifs, dans leurs rapports mutuels ?\par
La propriété individuelle et héréditaire, en créant entre l’individu et le sol, ou plutôt entre la famille et le sol, un lien d’une vigueur incomparable, me paraît être le meilleur moyen de garantir et de perpétuer la propriété nationale du sol. C’est grâce à ce morcellement du sol qu’on y enfonce et qu’on y multiplie les racines de la race.
\subsubsection[{III.4.b. Problème posé par l’inégalité des propriétés nationales.}]{III.4.b. Problème posé par l’inégalité des propriétés nationales.}
\noindent Je suppose le vœu des collectivistes réalisé. Il ne peut pas l’être partout à la fois, bien entendu ; il aura déjà bien assez de peine à vaincre, dans un État donné, les résistances qui s’opposeront à son triomphe. Cet État deviendra l’objet soit de la réprobation soit de l’imitation de ses voisins. Admettons qu’il sera imité. Chacun des peuples de l’Europe, France, Angleterre, Allemagne, Suisse, Suède et Norvège, etc., aura ainsi nationalisé son sol. Allez-vous me dire que, aussitôt, ces nations vont s’embrasser, se confondre en une seule ? C’est infiniment peu vraisemblable. La suppression de l’inégalité des propriétés individuelles et familiales aura eu, en effet, pour conséquence de démasquer une injustice tout autrement profonde dont on ne souffrait ni ne s’apercevait auparavant : l’inégalité des propriétés nationales. « La propriété, c’est le vol » soit. Mais cela est pour le moins aussi vrai de la  \phantomsection
\label{v2p306}propriété collective que de la propriété privée. Si quelques domaines sont entrés par la violence ou la ruse dans les familles de leurs possesseurs, on peut dire que tous, ou à peu près tous, les territoires nationaux ont été acquis les armes à la main, grâce à des abus de la force, à des usurpations odieuses et, qui plus est, historiquement connues, incontestables, tandis que les usurpations privées sont oubliées ou incertaines. Si donc la justice réclame le redressement des iniquités passées en fait de répartition des domaines privés, il convient, d’abord, de faire une seule masse de tous les territoires nationaux, en Europe et dans le monde entier, et de les répartir équitablement entre les peuples ; puis, le moment serait venu de distribuer avec la même équité aux individus de chaque peuple les parcelles de son domaine national. Ce ne serait pas une petite affaire. Mais la justice exige absolument cela.\par
Et c’est avoir une trop bonne idée de la nature humaine, c’est oublier l’égoïsme collectif et monstrueux inhérent à l’esprit de corps, que de se persuader que les peuples les plus favorisés par la nature et l’étendue de leur territoire renonceraient de leur plein gré à leurs avantages pour y faire participer leurs voisins relativement déshérités. On ne verra jamais cette \emph{nuit du 4 août} des nations. Mais, quand la nation la plus peuplée s’apercevra que, par tête, elle ne possède que tant d’ares tandis que la moins peuplée, près d’elle, en possède le double ou le triple, elle songera à s’annexer celle-ci. Celle dont le territoire est le moins fertile ou le moins salubre ou le moins agréable convoitera les terres plus fertiles, plus salubres, plus riantes, qui l’avoisinent. Les 50 ou 60 millions d’Allemands auront beau devenir propriétaires par indivis du sol germanique, ils n’en souhaiteront pas moins, loin de là, de s’adjoindre la Hollande, et une partie de l’Autriche. A coup sûr, même gouvernée par des collectivistes, l’Angleterre ne cessera pas de convoiter les terres d’autrui, l’impérialisme ne cessera \phantomsection
\label{v2p307} même pas d’y grandir, et, si l’occasion s’offre de faire valoir ses « droits » sur Calais ou la Guyenne, il est à craindre que le plus marxiste des premiers ministres britanniques ne soit le plus prompt à la saisir. Loin d’être amortie par la réalisation du collectivisme, la soif de conquête collective serait peut-être redoublée, puisque chacun espérerait devenir co-propriétaire, pour sa petite part, du territoire à conquérir\footnote{ \noindent Ne semble-t-il pas qu’il y ait un rapport inverse entre le respect de la propriété individuelle et celui de la propriété nationale ? Autrement dit, peut-être y a-t-il à choisir entre la suppression ou l’abaissement des murs de clôture de propriété entre individus, moyennant le relèvement ou le renforcement des murs de clôture de propriété entre peuples, — et à l’inverse, la suppression ou l’abaissement de ceux-ci moyennant le relèvement ou le renforcement de ceux-la. Les socialistes poursuivent le premier but, les économistes le second...
 }.\par
Et, ce qu’il importe aussi de considérer, cette avidité grandissante aurait moins de scrupule à se déchaîner. Telle expropriation, qui, si elle devait avoir pour victimes des propriétaires individuels, révolterait le sens moral des civilisés, leur paraîtra toute naturelle, ou beaucoup moins révoltante assurément, si elle n’atteint que des collectivités. On sait avec quel sans-gêne les États modernes, quand la fantaisie leur en vient, mettent la main sur les biens d’un groupe vivant en communauté. Par là on peut prévoir que, le jour ou un État collectiviste, sous un prétexte quelconque, s’annexerait un autre État collectiviste, il ne se gênerait pas beaucoup, sinon pour exproprier en masse tous les habitants, du moins pour les rançonner d’impôts asservissants qualifiés compensateurs, qui les forceraient à émigrer peu à peu. On tiendrait à la disposition des émigrants, cela va sans dire, de beaux vaisseaux destinés à les transporter en Afrique, dans quelque région malsaine qu’ils seraient chargés d’assainir et de préparer à la colonisation de la métropole. L’indignité, l’iniquité, la cruauté des États les uns à l’égard des autres est incomparablement supérieure à celle dont ils font preuve à l’égard des individus ou dont  \phantomsection
\label{v2p308}les individus font preuve dans leurs relations réciproques.\par
L’attachement au sol natal est un sentiment fondamental de la vie nationale. C’est par lui qu’un peuple civilisé, chez lequel la propriété terrienne est suffisamment divisée, vit content de son territoire, ne convoite pas trop le territoire du voisin, et aussi résiste avec vigueur aux tentatives d’envahissement de celui-ci. Sans la propriété individuelle et familiale, ce sentiment aurait-il pu naître ? Sans elle, pourrait-il se perpétuer ? Est-ce qu’un peuple, devenu collectiviste, opposerait à l’invasion étrangère, à l’expropriation conquérante ou insidieuse, la même énergie résistante qu’une nation de paysans, cultivateurs de leur petit patrimoine ? Et, autre question, implicitement contenue dans la précédente, est-ce qu’un peuple, en devenant collectiviste, ne deviendra pas parfois, du même coup, dégoûté de son sol ? Est-ce qu’il ne s’avisera pas de remarquer, pour la première fois, les inconvénients de son climat, les défauts de sa situation géographique ? On est toujours content de la maison qu’on habite, quand on est propriétaire ; mais il est bien connu qu’un locataire, ou même un usufruitier, au bout d’un temps très court, découvre toujours des imperfections capitales à son appartement, en premier lieu d’être trop étroit, et projette d’en changer. N’en sera-t-il pas de même quand, au lieu d’être propriétaire chacun d’un morceau de notre sol, nous n’en serons plus que fermiers ? Il faut reconnaître que, si une nation se composait en grande majorité de prolétaires sans espoir de sortir du prolétariat, le patriotisme y serait en grand danger d’affaiblissement. Mais la majorité se compose partout de propriétaires, ou de prolétaires qui espèrent le devenir un jour, par succession ou par contrat. Le nombre est infime des déshérités qui sont dépourvus de cette espérance. Et elle suffit pour attacher au sol les candidats à la propriété. Ainsi, établir le collectivisme agricole, ce serait, qu’on le veuille ou non, fomenter la conflagration universelle, la mêlée des avidités  \phantomsection
\label{v2p309}nationales impossibles à assouvir. Alors même qu’on parviendrait à supprimer les nations, on n’aurait point supprimé les groupes humains, et, dans chacun d’eux, on n’empêcherait pas de naître et de croître un âpre désir de vaincre et de spolier un groupe étranger. — Admettons, cependant, que les progrès de la raison soient parvenus à faire évanouir la guerre, il est au moins une conséquence des plus graves qu’entraînerait la nationalisation du sol : c’est que, l’attachement de l’homme à sa terre étant supprimé, la rivalité des hommes en vue de la terre serait redoublée et produirait un \emph{déracinement} général des plus redoutables au point de vue de l’harmonie et de la félicité générales.\par
Voici la terre française nationalisée, tous les propriétaires petits ou grands expropriés. Le sol reste, il est vrai, réparti entre un certain nombre de domaines, comme maintenant, plus intelligemment que maintenant, je le veux bien, mais avec cette différence que, au lieu d’être exploités par des propriétaires, ils le sont par des fonctionnaires, fermiers ou régisseurs de l’État. Or, cette différence sera grande\footnote{ \noindent Inutile de dire que la terre, dans ces conditions, sera infiniment moins travaillée. Aussi ne puis-je comprendre la thèse récente de M. Landry qui, dans son ouvrage ironiquement intitulé l’\emph{Utilité sociale de la propriété individuelle}, préconise la nationalisation du sol au point de vue du rendement brut. Il suffit d’avoir comparé le labeur acharné et amoureux du propriétaire-cultivateur au travail somnolent d’un fonctionnaire pour répondre à cette objection. Epictète nous dit quelque part : « Songe à tous les soins dégoûtants que, tout le long du jour, tu donnes à ton corps, parce qu’il est tien, et suppose qu’il te faille soigner de la sorte le corps d’autrui, de ton meilleur ami même... » Cette idée d’Epictète m’est souvent revenue à la mémoire en regardant un paysan bêcher sa terre ingrate avec amour...
 }, car ces gérants officiels, envoyés de l’Est à l’Ouest, du Sud au Nord, comme le sont les régisseurs du Crédit foncier, ne seront groupés dans les villages ou les hameaux que comme le sont, dans les villes actuelles, les fonctionnaires. Ils formeront des groupes instables et sans lien solide, sans nulle cohésion. Il n’y aura donc bientôt, à la campagne même, aucune population fixe, ferme, stable, attachée au sol ancestral \phantomsection
\label{v2p310} par de vieilles et indéracinables affections. Il n’y aura plus entre voisins ruraux aucun rapport d’amitié héréditaire, aucun échange traditionnel d’obligations et de services ; plus rien d’affectueux, de cordial, de durable. Ce sera la dernière glèbe de la population à tout jamais rompue, et la pulvérisation nationale complète. Alors la tendance de la population à se concentrer dans les villes deviendra torrentielle, irrésistible. Il faudra des lois oppressives pour forcer les gens à habiter la campagne, dépourvue, au point de vue social, de son principal attrait. Encore les lieux pittoresques et charmants trouveront-ils preneurs, mais les lieux tristes, mornes, que l’on ne peut aimer qu’à la condition d’y être né et d’y avoir des racines profondes dans le passé, seront repoussés de tous. On ne songe pas que la propriété individuelle est, avant tout, une digue contre le torrent de l’instabilité, un obstacle à la désertion des champs, à la dissolution de tous les liens du cœur entre les familles, à la destruction du patriotisme local.\par
Regardons en face, au point de vue de notre inter-psychologie, la question de savoir ce que valent ces liens sociaux héréditaires dont je viens de parler, ces rapports de voisinage séculaire, d’attachement commun à un même sol natal et local, qui sont liés intimement à la propriété individuelle. Il est certain que les liens sociaux de ce genre, tenaces et en petit nombre, empêchent de se former une foule d’autres relations sociales à la fois plus étendues et plus fragiles, qui diversifient et divertissent davantage l’esprit. Y a-t-il ou non compensation ? Et est-ce à bon droit que, comme on le voit par l’émigration des campagnes vers les villes, les individus les plus entreprenants donnent de plus en plus la préférence aux relations multiples et instables, diverses et changeantes, sur les liens étroits et plus forts ? — Mais, d’abord, est-il vrai que ces liens et ces relations sont inconciliables, et que le dilemme s’impose ? Est-ce que la facilité plus grande des communications ne  \phantomsection
\label{v2p311}doit pas avoir, peut-être, pour conséquence finale de retenir d’autant mieux l’individu à son coin de terre natal qu’il s’envolera de là plus aisément dans le vaste monde pour y revenir avec plus de joie ? Est-ce que l’homme ultra-civilisé n’est pas destiné, par le concours de la haute culture, qui rend sédentaire, avec le progrès de la locomotion, à devenir en même temps plus voyageur et plus stable, plus curieux de l’Univers entier et plus amoureux de son village ? Voyage et vagabondage font deux. Le voyageur a un domicile, le vagabond n’en a point.\par
Il n’y a pas de créature humaine plus heureuse, ni plus saine, qu’un paysan marié, père de famille, propriétaire d’un champ assez grand pour nourrir lui et les siens et lui permettre quelques économies. Le spectacle de ce bonheur est assez fréquent dans n’importe quelle campagne pour donner à tous les voisins l’espoir de le réaliser à leur tour. Et c’est ce bonheur, rare je le veux mais source unique de cet espoir si général, si propre à stimuler le travail et à colorer la vie et l’âme du cultivateur, que l’on veut détruire par la « nationalisation du sol ». Chimère pour chimère, mieux vaut l’illusion de la propriété, car c’en est une aussi, la bienfaisante erreur de croire que l’on est propriétaire de quelque chose quand on n’est qu’un peu de matière organisée en voie de se dissoudre. De tous les divins mensonges de la vie, il n’en est pas, après l’amour, qui soit plus fécond : par là, il semble à l’individu abusé, mais rassuré, qu’il appuie sur la terre ferme son être fragile et le fait participer à sa solidité, à sa permanence. Un champ, une maison, cela est l’incarnation ou plutôt la pétrification apparente de la famille, cela donne à l’individu la perspective d’un horizon illimité dans le temps. Supprimez cette magie, le mur froid de la mort se montre à nu, à deux pas de nous.\par
Je sais bien que les États européens, du moins les États occidentaux, traversent en ce moment une crise agricole des plus douloureuses. L’ouvrage volumineux de Kautsky  \phantomsection
\label{v2p312}tend à prouver que le métier d’agriculteur petit ou grand, mais surtout du petit, est en train de devenir la pire des professions, qui ne tardera pas à être désertée. Après l’avoir lu, il est curieux de relire le long chapitre où Stuart Mill vante, avec une grande abondance de documents aussi, comme avant lui Sismondi, la félicité du paysan propriétaire. Tous deux sont spécieux, Mill et Kautsky. L’un avait certainement raison à sa date, et il semble que l’autre, Kautsky, n’ait pas tout à fait tort à la sienne. Que s’est-il donc passé dans l’intervalle des deux époques ? Un grand fait, la concurrence des blés et des produits agricoles étrangers, extra-européens. Mais, après tout, ce n’est là qu’un mal passager, et est-ce un mal sans remède ? A la culture extensive des terres neuves, les vieilles terres ne sauraient-elles opposer avec succès la culture intensive, — derrière des barrières douanières, s’il le faut ? Il est certain que les ravages du phylloxéra, qui nous a été importé aussi d’Amérique, ne sont rien comparés au désastre agricole produit par l’importation du blé américain, — bientôt peut-être du blé africain. Le phylloxéra est venu interrompre la plus belle période de prospérité vinicole qui fut jamais. Le blé américain est venu arrêter pareillement, mais pour bien plus longtemps encore, l’ère de la plus grande, de la plus florissante prospérité agricole, dont Mill a tracé un tableau qui n’a rien d’exagéré. Sans cet élargissement prodigieux du marché des blés, on peut dire que jamais l’idée ne serait venue de contester les bons effets de l’appropriation privée du sol. Le rêve du collectivisme agraire est né d’une crise de l’agriculture qui, toute terrible et prolongée qu’elle est, ne doit pas être prise pour une donnée fondamentale du problème à résoudre. Si l’on met en regard les maux, les désastres ruraux prédits par Kautsky avec tant de force, comme un effet nécessaire du libre commerce international des [{\corr céréales}], et les bienfaits moraux, sociaux, inappréciables de la petite propriété, on n’hésitera pas à conclure qu’il  \phantomsection
\label{v2p313}importe de se protéger efficacement, dans une certaine mesure au moins, contre la concurrence étrangère pour maintenir chez soi la classe indispensable des cultivateurs satisfaits.
\subsubsection[{III.4.c. La rente du sol et les profits de l’industrie.}]{III.4.c. La rente du sol et les profits de l’industrie.}
\noindent Tout en reprochant à la propriété rurale sa misère actuelle, qu’ils exagèrent d’ailleurs, les collectivistes lui reprochent aussi d’accaparer indûment la \emph{rente} du sol. C’est là le principal grief contre l’appropriation individuelle de la terre, celui qui a eu le plus de succès parmi les théoriciens, depuis la fameuse théorie de la rente de Ricardo. Karl Marx, il est vrai, avec sa grande sagacité, n’a pas pu ne pas voir l’analogie manifeste qui existe au fond entre les profits d’établissements industriels bénéficiant de certains avantages de situation, et la rente foncière. Mais, d’après lui et son disciple Kautsky, il y a des différences profondes entre ces deux sortes de privilèges, et elles seraient un argument juridique contre la propriété individuelle du sol spécialement, ajouté à tous ceux qui militent contre la propriété individuelle en général. — D’abord, l’avantage dont bénéficie l’industriel qui réalise des profits extraordinaires, est susceptible de s’étendre à d’autres, à des rivaux, à des envieux « qui chercheront à organiser des exploitations dans les mêmes conditions de production », tandis que les avantages dus à la fertilité d’un sol ou à des particularités du climat, ne sont pas extensibles à volonté. En second lieu, le profit extraordinaire de l’industriel est un phénomène essentiellement passager, « tôt ou tard, les conditions de production les plus particulièrement avantageuses seront partout répandues », tandis que la fertilité supérieure d’un terrain est un phénomène essentiellement durable.\par
Aucune de ces différences ne résiste à l’examen. Voyons la première. Quand un établissement industriel est favorisé  \phantomsection
\label{v2p314}par la proximité d’une gare, condition qui lui permet d’abaisser ses prix jusqu’à ce qu’il ait tué tous les établissements rivaux, est-ce qu’il est possible aux concurrents de s’approprier cet avantage, de le conquérir à volonté ? Il en est de même quand le succès d’une usine tient au voisinage d’une belle chute d’eau, ou à l’extrême pureté des eaux qui donne à certaines papeteries de luxe un avantage marqué, ou à toute autre particularité d’ordre géographique ou climatérique\footnote{ \noindent La différence entre les privilèges industriels et agricoles est d’autant plus imaginaire, en effet, que, bien souvent, l’avantage inhérent à certains établissements industriels est lié aux caractères du sol ou du sous-sol. La métallurgie anglaise ne doit-elle pas sa supériorité au rapprochement des mines de charbon et de fer ? et cet avantage n’est-il pas aussi durable que celui des vignobles bordelais ? Appellerons-nous rente foncière ou \emph{sous-foncière} les profits extraordinaires que les métallurgistes anglais ont tirés de là ?
 }. Et même, quand il s’agit de qualités propres à la population d’un pays, d’aptitudes héréditairement transmises de génération en génération et d’où dépend la supériorité d’une fabrication de ce pays, est-ce qu’il est aisé, pour les industriels étrangers, d’importer chez eux ces conditions spéciales ? Alors même que cela serait possible à la rigueur, ne serait-ce pas, le plus souvent, impraticable ? Il n’est pas non plus de terrain si ingrat qu’on ne puisse rendre fertile à volonté moyennant des transports de terre, des drainages, des fumures intenses ; mais la question est de savoir si le jeu en vaut la chandelle. — Ensuite, si l’on se place au point de vue de la brièveté d’une vie humaine, est-ce que l’avantage, pour un établissement industriel, d’être favorisé par la contiguïté d’un chemin de fer ne peut pas être qualifié \emph{durable ?} Et qu’importe que cette voie ferrée doive durer moins, sans doute, que la fertilité d’une terre privilégiée ? C’est toujours une durée immense, pratiquement infinie. Et cette considération s’applique mieux encore au voisinage d’une force physique ou d’une race bien douée. — Au surplus, ne dépend-il pas d’une découverte chimique, d’une invention mécanique, d’une nouvelle plante domestiquée ou  \phantomsection
\label{v2p315}importée, de faire perdre à une terre son rang sur l’échelle de la production agricole ? Telle terre qui se vendait fort cher avant le phylloxéra parce que les cépages français y poussaient merveilleusement ne vaut plus rien maintenant parce que les cépages américains n’y peuvent vivre\footnote{ \noindent Aussi ne puis-je comprendre le passage suivant de Kautsky : « Les moyens de production créés par le travail humain (les capitaux) s’usent physiquement et moralement : \emph{ils sont moralement usés par de nouvelles découvertes.} Ils cessent tôt ou tard d’exister ; il faut sans cesse les renouveler. \emph{Le sol au contraire est indestructible.} » Comme s’il ne dépendait pas d’une \emph{découverte} de faire perdre au sol la seule chose de lui qui intéresse le propriétaire et la société. — sa valeur !\par
 Autre différence non moins étrange. Suivant Kautsky, la rente foncière tend à \emph{monter} tandis que l’intérêt du capital tend à baisser. Allez dire cela aux propriétaires de la Gironde, de la Beauce, d’un peu partout.
 }. Les terres à blé de la Beauce ont perdu les trois quarts de leur valeur depuis que la concurrence des blés étrangers et la hausse des salaires ruraux ont réduit à si peu de chose le bénéfice net du producteur de froment. Que valent les champs propres à cultiver la garance depuis la découverte des substances chimiques colorantes ? Est-ce que l’accroissement de la prospérité d’une ville voisine, ou, inversement, la diminution numérique et l’appauvrissement de sa population, ne font pas hausser ou baisser la valeur d’une propriété rurale ? Et, en Amérique surtout, peut-on appeler \emph{durable}, dans le premier cas, à notre époque, un avantage de situation qui a pour cause des changements si fréquents et, de nos jours, si rapides ?\par
Le cas dont il s’agit, celui où un bien acheté très bon marché est, quelques années après, revendu avec un bénéfice énorme, sans nulle amélioration par le travail, simplement parce que, dans l’intervalle, une ville est née ou s’est développée énormément dans les environs, ce cas singulier et saisissant est une des objections dont on a tiré le plus grand parti contre le droit de propriété. Il est monstrueux, a-t-on dit, que, en se croisant les bras, un propriétaire voie tripler, décupler la valeur de sa terre, tout simplement à raison du peuplement de son pays. On n’a pas réfléchi que, en séparant  \phantomsection
\label{v2p316}ainsi ce cas du cas inverse, qui lui fait contre-poids, on montre le parti-pris le plus injuste. Si je vole la communauté quand je profite seul de la plus-value donnée à ma propriété par la rapide augmentation d’une ville voisine, est-ce que je ne pourrais pas dire aussi que je suis volé par la communauté quand cette population décroît, ce qui arrive fréquemment ? S’il y a la \emph{rente}, il y a aussi l’\emph{anti-rente}, pour ainsi dire, comme, à côté du profit, il y a la perte. Ce que j’accorde volontiers, cependant, c’est que, lorsque la plus-value en question dépasse un certain degré, qu’elle devient, pas sa soudaineté et son importance un phénomène vraiment anormal et exceptionnel, et aussi bien quand une dépréciation brusque et profonde a lieu par suite de causes inverses, il conviendrait peut-être de soumettre à des tarifs spéciaux et compensateurs dans la première hypothèse, et d’indemniser en partie dans la seconde, le propriétaire trop heureux ou trop malheureux. Cela soit dit, sous la réserve d’une marge très large laissée aux chances et aux risques, d’un horizon suffisant ouvert aux espérances et aux entreprises individuelles.\par
En réalité, le bénéficiaire des rapides plus-values dont il vient d’être question a rarement joué à coup sûr, il s’est risqué, il a aventuré son argent ; et c’est son \emph{audace}, chose utile à encourager, qui est récompensée ainsi. Ceux qui, en 1859, ont acheté 250 francs l’action de Suez, qui actuellement vaut plus de 3 550 francs, ont multiplié par 14 leur capital. Dira-t-on, comme on l’a dit parmi les économistes, que c’est là la juste \emph{récompense} de la perspicacité dont ils ont fait preuve en prévoyant le succès de cette entreprise ? La vérité est qu’ils n’ont rien prévu du tout, qu’ils ont joué tout simplement et gagné au jeu. Puis, alors même qu’ils auraient prévu, quel mérite y a-t-il à prévoir quand on est bien renseigné, par un privilège injuste en soi, ou en vertu d’un flair spécial et inné ? La vérité est que, même en prévoyant le succès, on n’y a pas cru très fort, on a été loin d’en être sûr, on s’est risqué, \emph{et c’est parce qu’on s’est risqué}, non  \phantomsection
\label{v2p317}parce qu’on a prévu, qu’on mérite de garder son gain, \emph{puisqu’aussi bien on aurait pu perdre}. — Si, en effet, un acheteur de ces actions avait été \emph{sûr}, absolument \emph{sûr}, que ce qu’il achetait 250 francs en vaudrait bientôt 1 000, 2 000, 3 000, etc., il aurait dû, en bonne justice, faire participer son vendeur à ce bénéfice assuré, — à peu près par la même raison qui a fait décider que l’inventeur d’un trésor partage ce trésor avec le propriétaire du terrain.\par
Observons qu’il n’y a ni plus ni moins de raison de justifier le gain énorme fait par l’acheteur d’actions de Suez à 250 francs que de justifier le bénéfice considérable, pareillement \emph{sans travail}, obtenu par l’acheteur de terres à bon marché qui, après son acquisition, acquièrent une plus-value prodigieuse par suite soit de l’accroissement de la population, soit de l’enrichissement du pays, soit du passage d’un chemin de fer, soit de l’invention d’un procédé économique de culture.\par
On peut bien, pour essayer de démontrer que la propriété individuelle est une source d’injustices intarissable, déclarer injustes les bénéfices ainsi acquis sans peine par d’heureux spéculateurs à la bourse ou aux enchères publiques, qui se sont enrichis par des achats d’actions ou de terrains dans des quartiers neufs des grandes villes. Mais la question est de savoir ce qui arriverait si la propriété individuelle était supprimée. La \emph{plus-value} des terrains disparaîtrait-elle pour cela ? Et, de même, le succès des entreprises industrielles dont les actions ont haussé de la sorte serait-il anéanti ? Non, — ou du moins ce n’est pas, certes, ce que les socialistes veulent dire (car, au fond, il se peut bien que, pour mieux répartir la plus-value, ils la détruisent ou l’amoindrissent ; mais laissons cela par hypothèse). Il y a lieu de penser que, dans un quartier devenu à la mode, où l’État constructeur, par hypothèse, ferait beaucoup bâtir, l’engouement croissant pour ces maisons nouvelles, joint à la population grandissante, ferait croître extrêmement la valeur de  \phantomsection
\label{v2p318}ces habitations, et ferait considérer comme une grande faveur le privilège (car c’en serait un, qu’on le veuille ou non) de les habiter.\par
Je sais bien qu’on éléverait leur loyer, en admettant que, la propriété individuelle des immeubles une fois supprimée, il y eût encore des revenus. Je sais bien aussi que, lorsqu’un domaine rural ou une usine verrait doubler ou tripler sa valeur à la suite d’une invention ou d’une heureuse circonstance quelconque, l’État collectiviste pourrait l’affermer deux ou trois fois plus cher... Et, de la sorte, je vois bien que le groupe entier des collectivistes profiterait de la totalité d’avantages à présent monopolisés par des particuliers.\par
Mais je vois aussi à cela des inconvénients majeurs. D’abord, au point de vue de ma psychologie collective, est-ce que le fractionnement de ces bénéfices totalisés, par portions égales, entre tous les collectivistes, sera senti par chacun d’eux de manière à leur procurer un accroissement de bonheur égal en intensité à l’intensité des joies individuelles auquel il s’est substitué et aussi propre que la perspective de ces joies à activer la production des richesses, à stimuler l’inventivité, dans l’intérêt de tous ?\par
En second lieu, je vois aussi que, si le monopole individuel est détruit, il sera remplacé par un monopole collectif. Chaque collectivité, la commune par exemple, ou le canton, monopolisera à son profit exclusif les avantages naturels ou acquis, fortuits ou voulus, qui auront donné au domaine collectif une plus value. Il y aura ainsi des groupes privilégiés à la place des individus privilégiés. — Que si, pour faire disparaître cette inégalité d’un module plus élevé, on fait fusionner les cantons dans la province, ou les provinces dans l’État, on aboutira toujours à des monopoles, seulement de plus en plus gigantesques, — à moins que, chose impossible, on ne parvienne à faire du globe entier un seul et même domaine de la collectivité humaine totalisée en une seule et même nation.\par
 \phantomsection
\label{v2p319}— Or, en attendant que ce rêve, visiblement chimérique, du \emph{collectivisme mondial} se réalise, il est à remarquer que le \emph{collectivisme municipal} ou aussi bien le \emph{collectivisme provincial} ou aussi bien le \emph{collectivisme national} (ce dernier déjà bien irréalisable) aurait le grave inconvénient de rendre impossible une autre sorte de collectivisme dont nous jouissons, qui se développe tous les jours davantage, et cela grâce à la propriété individuelle même : le \emph{libre collectivisme international.} Car, supposez que, au moment où M. de Lesseps a songé à percer l’isthme de Suez, la propriété française et anglaise eût été divisée en collectivités municipales ; est-ce qu’il eût été possible d’obtenir de ces groupes l’adhésion à son idée, la souscription à ses actions ? On ne voit pas comment pourrait fonctionner à la fois le collectivisme \emph{fondé sur des groupements géographiques} et le collectivisme fondé sur des groupements en \emph{vue d’une idée commune, indépendamment de toute co-habitation...} Une idée d’intérêt international ne pourrait plus se réaliser par des adhésions d’individus détachés de leurs groupes nationaux ou municipaux ; il faudrait qu’elle obtint l’adhésion de ces groupes eux-mêmes, chose bien plus ardue. Ne sait-on pas à quel point les foules, même parlementaires, sont moins intelligentes et moins hardies que les individus d’élite, à moins qu’elles ne soient plus follement extravagantes ?\par
— Un nouvel écrivain socialiste, M. Landry\footnote{ \noindent Dans son livre su l’\emph{utilité sociale de la propriété industrielle} (Alcan, 1901).
 }, s’est efforcé de démontrer qu’il est essentiel à la propriété individuelle de mettre l’intérêt individuel du propriétaire en conflit avec l’intérêt général. S’il en était ainsi, si cette institution, en même temps qu’elle prévient les oppositions entre individus différents, en faisait naître entre chacun d’eux et l’ensemble des autres, il faudrait convenir que son maintien si prolongé ne se conçoit guère. Mais tout ce qu’on nous démontre  \phantomsection
\label{v2p320}c’est qu’entre l’individu et la collectivité il y a de fréquents désaccords de désirs et d’idées ; et la question est de savoir si c’est la division et l’hérédité des biens qu’il convient ici d’accuser. Ce n’est pas seulement à l’occasion de la propriété individuelle, c’est dans l’exercice des droits individuels quelconques, que l’intérêt individuel se trouve parfois, souvent même, en lutte avec l’intérêt collectif. Faut-il conclure de là à la suppression de tous les droits individuels ? Non, mais à la réglementation de ces droits par l’État. Tous les codes civils sont pleins de dispositions qui interdisent telles ou telles manières de contracter obligation, parce qu’elles sont ou paraissent contraires au bien public. M. Landry s’étend beaucoup, après Cournot, sur l’\emph{aménagement des forêts.} C’est son exemple le plus frappant. Mais je ne vois pas pourquoi, à ce sujet, comme au chapitre des servitudes rurales, la loi n’interviendrait pas dans l’intérêt général. — Remarquons aussi que l’intérêt général se présente toujours sous la forme d’une \emph{volonté} générale, c’est-à-dire d’une tendance gouvernementale, avec laquelle on le confond officiellement, quoiqu’il puisse en être fort distinct. Depuis la Révolution française, la volonté générale en France, la législation française, s’oppose à ce que le père de famille fasse passer toute sa propriété foncière ou toute son industrie sur la tête d’un seul de ses enfants, et l’on croit que l’intérêt général réclame l’égalité du partage aussi grande que possible. Cependant, beaucoup de publicistes s’accordent à penser que rien n’est plus désastreux que cette égalité à divers points de vue. Ainsi, c’est, au fond, le conflit entre la volonté de l’individu et la volonté de la masse ou plutôt de la majorité, souvent d’une minorité influente, intrigante, oppressive, que M. Landry signale comme un des plus graves inconvénients attachés à la propriété privée. Et beaucoup penseront que le grand mérite, le mérite éminent de cette institution fondamentale est de permettre cette résistance fréquente de la volonté  \phantomsection
\label{v2p321}individuelle à la volonté collective, d’être une forteresse où la personne humaine se retranche pour repousser les empiètements et les agressions du milieu social qui menace de l’engloutir. Pour qui sait à quel point la collectivité est inférieure en intelligence et en moralité à l’élite, souvent même à la moyenne des individus dont elle est formée, il n’est pas douteux que la suppression des conflits dont il s’agit, par la soumission forcée de l’élite, dépouillée de son dernier bouclier, serait le signal d’un déclin profond de la civilisation. Toute la question est là, dans le problème de la psychologie des foules.\par
L’intérêt général, c’est toujours par des individus en avance sur leur temps, qu’il est d’abord aperçu ; et ils ne peuvent parvenir à faire prévaloir leur conception, qui commence par être unanimement combattue, que moyennant des moyens d’action fournis par la propriété individuelle soit d’eux-mêmes soit d’autrui. Supprimez celle-ci, les hommes supérieurs perdront les trois quarts de leur puissance de résistance et aussi de leur force de suggestion. Ce n’est pas que, toujours, ils soient propriétaires ; mais, quand ils ne le sont pas, il leur suffit de gagner à leur cause quelques grands propriétaires et capitalistes pour propager un mouvement en faveur de leurs idées. S’il leur fallait, sans capitaux propres, ou prêtés par des particuliers, exercer une action décisive sur les corps publics, ils n’y parviendraient presque jamais.\par
En socialisant la terre et les capitaux, prétendez-vous supprimer les conflits entre l’intérêt individuel et l’intérêt général ? Vous les étoufferez violemment, vous les apaiserez à la surface, mais au fond, leur désaccord sera plus douloureux et plus irrémédiable que jamais. L’intérêt de l’individu sera opprimé par celui d’une majorité ou d’une minorité au pouvoir ; à moins que ce ne soit le contraire, toute une nation se courbant sous la loi d’un despote populaire. Les propriétaires fonciers sont en opposition fréquente, je le  \phantomsection
\label{v2p322}veux, avec les vœux ou les intérêts du public ; mais ces conflits existent parce que ces hommes sont égoïstes, non parce qu’ils sont propriétaires. Expropriez-les, ils n’en garderaient pas moins tout leur égoïsme ; mais cet égoïsme s’exprimera autrement, par la gestion partiale et injuste des fonctions publiques dont ils seront chargés, eux ou les prolétaires pareillement. Comme il n’y aura plus moyen de s’enrichir que par l’exercice des fonctions publiques, la corruption s’y déploiera à un degré inouï\footnote{ \noindent On peut relever encore dans le livre de M. Landry l’illusion de croire que l’État collectiviste économiserait et capitaliserait bien plus que ne le font les individus sous le régime de la propriété privée. Quand on voit avec quelle facilité l’État, actuellement, sacrifie les intérêts de l’avenir à ceux du présent, grève son budget, emprunte toujours et n’amortit point, on est peu porté à penser que, devenu collectiviste, il change à ce point de nature.
 }.\par
— En somme, la propriété individuelle soit du sol, soit des capitaux, comme l’initiative individuelle, a fait ses preuves. Nous lui devons, rien qu’en notre siècle, la mise en valeur de tout un continent, le merveilleux essor de l’Amérique. La propriété collective, l’initiative collective, a-t-elle fait les siennes ? Sur la même terre américaine, le médiocre ou lamentable résultat de tant d’expériences collectivistes répond assez clairement. Sous les Incas seulement, nous y avons vu le communisme fleurir ; comme si la propriété indivise n’était adaptée qu’aux conditions de sociétés encore à demi-barbares et était repoussée par les exigences de la civilisation progressive.
\subsubsection[{III.4.d. Nationalisation ou municipalisation de la propriété ?}]{III.4.d. Nationalisation ou municipalisation de la propriété ?}
\noindent — N’y eût-il à choisir qu’entre la propriété individuelle et la propriété nationale, le problème serait déjà passablement ardu. Mais, comme nous venons de le faire entendre, la \emph{nationalisation} du sol et des autres moyens de production n’est pas le seul mode concevable de propriété collective. A la nationalisation peut se juxtaposer ou s’opposer la  \phantomsection
\label{v2p323}\emph{municipalisation}, qui serait bien moins impraticable, et qui a été à bon droit préconisée par Proudhon comme le seul moyen plus ou moins pratique de réaliser le collectivisme\footnote{ \noindent Dans l’ancien Pérou, le sol était municipalisé ; aussi nul n’avait le droit de changer de résidence. Il faut s’attendre, si le sol est de nouveau municipalisé, à ce que des entraves du même genre soient mises à la libre circulation des individus. Car, d’une part, la commune dont on fera partie aura besoin de garder les bras employés par elle ; d’autre part, la commune où l’on voudra aller s’établir aura intérêt à ne pas grossir le nombre des bouches nourries par elle.
 }. Doit-on nationaliser ou seulement municipaliser l’industrie aussi bien que l’agriculture, l’éclairage aussi bien que l’instruction, les services d’omnibus et de tramways aussi bien que les chemins de fer ? On ne voit pas non plus pourquoi certains services seraient municipalisés plutôt que \emph{cantonalisés}, ni pourquoi on n’en \emph{provincialiserait} pas certains autres. Enfin, la famille sera-t-elle, de tous les groupements humains, le seul qui sera oublié par le collectivisme, apparemment parce qu’il est le plus naturel de tous, et ne s’apercevra-t-on pas de la nécessité de \emph{domestiquer}, de \emph{familialiser} pour ainsi dire, beaucoup de productions ? Tout cela complique singulièrement la question du collectivisme.\par
Je sais bien que le plus solide argument, après tout, et avant tout, en faveur du maintien de la propriété privée, c’est qu’elle existe, c’est qu’elle fonctionne depuis des siècles de siècles, c’est que toute notre société est fondée sur elle. Mais cet argument est loin d’être de mince valeur, comme on le suppose en raisonnant en l’air. La force des précédents est souvent telle qu’elle oppose un obstacle insurmontable à des transformations qui, quoique jugées utiles et dépendant de l’homme, ne pourront jamais être réalisées. Par exemple, quand le réseau des chemins de fer sera terminé, si, à ce moment, la fédération de l’Europe s’accomplit sous l’hégémonie de la Russie ou de l’Allemagne — ou, pourquoi pas ? de la France — un ingénieur n’aura pas de peine à tracer un plan de reconstruction des chemins de fer plus rationnel, plus propre à l’\emph{irrigation} égale des marchandises \phantomsection
\label{v2p324} sur tout le continent, que le réseau existant. Cependant jamais, je pense, on ne songera à réaliser le programme de ce Freycinet de l’avenir, venu trop tard après les Freycinet du passé — qui, eux, n’ont été précédés par aucun Freycinet antérieur. — Exemple encore plus frappant : il ne serait pas difficile de concevoir une répartition géographique, sur le territoire de la France, des villes de 100 000 âmes, de 50 000, de 30 000, de 20 000, de 10 000, de 5 000..., plus régulière et plus rationnelle que celle qui existe, plus propre que celle-ci à tirer le meilleur parti possible de toutes les ressources agricoles, industrielles et sociales du pays. Mais, si bizarre et si défectueuse que soit la pittoresque distribution des villes et des bourgs, personne n’a l’idée de la critiquer et de proposer d’y substituer autre chose, — pas plus que de substituer à une langue réelle et vivante un idiome fabriqué de toutes pièces, un volapück ou un esperanto, si admirablement conçu qu’il puisse être.\par
On ne devrait pas oublier cela quand on fait des rêves utopiques. La propriété individuelle, telle qu’elle s’est créée et développée historiquement, est une répartition du sol qui laisse infiniment à désirer, et il n’est nullement impossible — ni même malaisé — d’imaginer quelque chose de mieux. Mais ce serait une folie de faire table rase de cette institution pour la remplacer. Il faut \emph{tabler} sur elle comme ou \emph{table} sur l’existence des villes, c’est-à-dire l’utiliser en la réformant, de même qu’on voit s’agrandir telle ville ou se resserrer et dépérir telle autre, de manière à rapprocher dans une certaine mesure, dans une faible mesure, mais très importante cependant, le \emph{fait} de l’\emph{idéal.}
\subsubsection[{III.4.e. Transformations passées du droit de propriété soit collective, soit individuelle. Causes de ces transformations.}]{III.4.e. Transformations passées du droit de propriété soit collective, soit individuelle. Causes de ces transformations.}
\noindent — Aussi, la suppression de la propriété privée, qui vient d’être discutée dans ce qui précède, n’a-t-elle, à vrai dire, qu’un intérêt théorique ; et tout ce qu’il y a d’éclairé dans  \phantomsection
\label{v2p325}le parti socialiste tend à la reléguer à l’horizon lointain, indéfiniment reculé, des préoccupations du parti. Une question bien plus pratique, et, au fond, bien plus intéressante, est de savoir dans quel sens la transformation du droit de propriété est possible et désirable, si l’on veut que cette antique et universelle institution progresse dans sa double voie d’adaptation positive. A ce point de vue, quelques mots sur ses transformations antérieures ne se seront pas inutiles pour révéler les forces qui sont en jeu dans son évolution actuelle.\par
Remarquons d’abord que le respect mutuel des propriétés, comme le respect mutuel des libertés, a été le fruit d’une longue élaboration du Droit ; il a été précédé du respect unilatéral de la propriété des forts par leurs voisins faibles. Cette première étape de la propriété divisée est caractérisée par l’isolement des propriétés (collectives ou individuelles) qui a été antérieur à leur juxtaposition et à leur bornage. Ce que nous dit Tacite des territoires des tribus germaines est vrai aussi bien, d’après lui-même, du domaine de chacune des familles qui les composaient. « C’est l’honneur des tribus, nous dit-il, d’être environnées d’immenses déserts. » Chaque tribu puissante « regarde comme la meilleure preuve de sa valeur que ses voisins abandonnent leurs terres, et que nul n’ose s’arrêter près d’elle ». De là l’habitude d’interposer des \emph{marches}, terrain neutre et mal délimité, inculte et indivis, entre les territoires, et aussi bien entre les domaines des familles. « Ils (les Germains) ne peuvent souffrir que leurs habitations se touchent ; ils demeurent séparés et à distance... Chacun entoure sa maison d’un espace vide. » Ce même espace vide séparait les primitives maisons romaines. Mêmes usages parmi les sauvages américains. — Mais, à mesure que le sentiment du droit d’autrui se répand, les propriétés, collectives ou individuelles, se rapprochent, se juxtaposent, et apprennent à se respecter réciproquement. Harmonie qui, quoique purement négative, est d’un prix  \phantomsection
\label{v2p326}infini, si l’on songe à toutes les batailles que son acquisition a coûtées.\par
On peut dire qu’à toute époque la propriété présente une tendance à se préciser, à s’étendre, à se fortifier, et, en même temps, d’après la nature des choses appropriées, à s’individualiser ou à se \emph{socialiser} dans la mesure voulue pour la meilleure exploitation des inventions productrices de la richesse, et pour le moins de heurts possible entre les individus composant le groupe social, famille, clan, cité, nation. Le droit de propriété, individuelle ou collective, des choses et des personnes, ne fait jamais que consacrer une jouissance des choses ou des personnes suggérée par une innovation individuelle. L’idée d’épargner la vie du vaincu pour jouir de son travail est à l’origine de l’esclavage, comme l’idée de conserver le gibier vivant au lieu de le tuer est à l’origine de l’art pastoral. A chaque nouvel animal qu’on apprivoise, à chaque nouvelle plante qu’on importe et qu’on cultive, la jouissance pastorale ou agricole de la terre devient plus étendue, plus compliquée, plus précise, et le droit de propriété foncière doit revêtir les mêmes caractères. La proportion de la propriété collective et de la propriété individuelle dépend aussi de la même cause souveraine : aussi longtemps que les forces animales ou les forces physiques ne sont pas captées par une série de découvertes, l’individu ne peut défricher la forêt pour la convertir en prairie ou en terre arable sans le concours de ses concitoyens, de ses co-associés, et il est naturel alors que la propriété des terrains défrichés soit collective. Il en est autrement quand, peu à peu, l’individu, appuyé sur les énergies de la nature, par suite des progrès de l’industrie, n’a plus besoin de la collaboration directe de ses semblables. Alors l’appropriation individuelle se développe. Mais, en même temps, apparaissent certaines formes d’appropriation sociale qui vont se développant aussi à côté de certaines autres qui disparaissent. Pendant que les terrains communaux des villes d’autrefois \phantomsection
\label{v2p327} se morcellent et que le domaine privé, en tout pays qui se civilise, s’agrandit à leurs dépens, le domaine public consacré au fonctionnement des services publics qui se multiplient ne s’agrandit pas moins. Le besoin croissant de communications étend et complique le réseau des routes, des voies terrestres ou aquatiques de tout genre, qui, d’abord propriété privée et assujettie à un péage comme les ponts, deviennent propriété collective de la commune, de la province ou de la nation. Le besoin croissant d’action collective, d’autre part, d’action par l’État, grossit sans cesse les budgets publics et les trésors publics, devenus, par l’impôt, une part plus considérable de la fortune générale. Or, c’est là une propriété nationale mobilière dont les collectivistes primitifs, les hommes des clans, n’avaient nulle idée.\par
Si, spécialement, l’on considère l’importance croissante, parmi les peuples civilisés, à peu près nulle parmi les peuples barbares, de la libre propriété des mers et des océans, indivise entre toutes les nations, on ne sera plus porté à énoncer cet axiome que la propriété collective a été en diminuant depuis les temps primitifs. A la propriété collective des forêts pendant la période chasseresse de l’humanité, des prairies pendant la période pastorale, des terres défrichées en commun pendant les débuts de la période agricole, s’est substituée maintenant et de plus en plus celle de l’Océan. Je dis de plus en plus, car, en réalité, avant les progrès de la navigation, la propriété de l’Océan, collective ou non, n’existait pas. La Méditerranée avant la découverte de la trirème, l’Atlantique avant Colomb, n’appartenaient ni collectivement ni individuellement à personne ; c’étaient d’immenses espaces inappropriables. Et chaque découverte, telle que celle de la boussole, de la marine à voiles, de la marine à vapeur, de l’hélice, ou, aussi bien, celle du Nouveau-Monde, du Cap de Bonne-Espérance, de l’Australie, ou enfin, celle de la télégraphie sous-marine, a eu pour effet d’étendre considérablement cette co-propriété \phantomsection
\label{v2p328} maritime, de la rendre plus efficace et plus profonde. Les câbles sous-marins sont assurément une forme de jouissance de la mer que nos aïeux n’imaginaient pas. Ajoutons que chaque nouvel article d’exportation et de fret, produit à la suite d’une invention industrielle, donne une nouvelle sorte d’utilité aux voies maritimes, suscite de nouveaux bâtiments de transport.\par
Les transformations de la propriété sont dans une dépendance étroite des progrès de la population. C’est, à l’origine, la dissémination clairsemée des groupes humains, l’isolement de chacun d’eux au milieu de l’animalité grondante, des fauves et des reptiles, qui a rendu nécessaire l’association des travaux pour l’appropriation du sol, d’où est résulté le caractère collectif de celle-ci\footnote{ \noindent La bonne harmonie interne du clan primitif est due bien moins sans doute à l’indivision des terres et des femmes qu’à la cause de cette indivision, c’est-à-dire à l’association et à la convergence des efforts de lutte et de conquête collective. On comprend, par exemple, que les femmes étrangères obtenues par rapt collectif, à main armée, soient devenues indivises entre tous ceux qui les ont conquises de force.
 }. Cette appropriation, d’autre part, n’a pu avoir les caractères de limitation stricte et de fixité indéfinie qu’elle a eus beaucoup plus tard. Elle n’a pu être que momentanée, le sol s’épuisant vite à raison de l’imperfection de la culture, et obligeant les co-propriétaires ou plutôt les co-usagers à porter plus loin leurs bras et leurs hoyaux rudimentaires. La possession, vu l’abondance de la « terre libre », était aussi étendue qu’on pouvait le désirer, et, vu l’absence de fumure, était forcément changeante. Elle était aussi instable qu’illimitée\footnote{ \noindent Cette conclusion vient naturellement à la lecture de la belle étude de M. Kovalesky sur l’\emph{Èvolulion de la propriété} (Annales de l’Institut de sociologie, t. 2).
 }. Cette sorte d’appropriation collective par défrichement facultatif des forêts et déplacement indéfini des champs sans bornes, existe encore dans le nord de la Russie, près d’Arkhangel, \emph{là où la population est le plus éparse}, et en diverses autres régions caractérisées par la même dispersion des habitants.  \phantomsection
\label{v2p329}Elle nous explique à merveille le \emph{arva per annos mutant et superest ager} de Tacite, qui a donné lieu à tant de discussions. — Le \emph{mir} et autres formes de propriété analogues, avec limites du sol collectif désormais fixées et répartition périodique de lots précis entre les copropriétaires, n’ont pu venir que longtemps après, quand, la population étant devenue plus dense, les défrichements se sont touchés et heurtés\footnote{ \noindent Dans les lettres de Bakounine à Herzen, il est question du \emph{mir} en termes peu favorables et qui atteignent la propriété collective en général. Bakounine reproche à Herzen ses illusions à cet égard. « Pourquoi, lui demande-t-il, cette commune rurale russe, de laquelle vous espérez tant de prodiges dans l’avenir, n’a-t-elle pu produire, depuis dix siècles de son existence, que l’esclavage le plus abominable et le plus odieux, l’avilissement de la femme et l’inconscience ou plutôt la dénégation absolue de ses droits et de son honneur, l’abomination de la pourriture, l’assujettissement de l’individu au mir, et \emph{le poids écrasant du mir qui tue en germe toute initiative individuelle}, l’absence complète de toute justice dans les décisions du mir ? »
 }.\par
Mais n’oublions pas que, si les progrès de la population apparaissent ici comme la cause immédiate des transformations de la propriété, ils sont eux-mêmes subordonnés à la propagation des inventions agricoles et industrielles, notamment de celles qui ont trait à l’alimentation. Sur un territoire donné ne peuvent vivre qu’un très petit nombre de chasseurs ou de pêcheurs. Quand la domestication de quelques animaux ou de quelques plantes vient diversifier et assurer leur nourriture, ce nombre croît très vite ; il ne cesse de grandir à chaque progrès de l’agriculture. Par l’introduction des plantes fourragères, les têtes de bestiaux se multiplient, et, grâce à des engrais plus abondants et meilleurs, le rendement du froment par hectare s’est, par exemple, accru en France, de 10 hectolitres en 1816 à près de 16 hectolitres en 1895. Ce n’est pas l’augmentation numérique de la population qui a forcé l’agriculteur à substituer au système archaïque des trois assolements une culture plus intensive ; c’est l’importation des plantes fourragères, de la pomme de terre, de la rave (importation équivalente, là où elle a lieu,  \phantomsection
\label{v2p330}à la découverte de ces plantes) qui a accru la productivité du sol et permis aux hommes de pulluler. Par là, en effet, un nombre toujours croissant d’individus parvient à vivre, et à vivre de mieux en mieux, sur un même sol. « Ce n’est qu’à un très haut degré de civilisation, dit Kautsky avec raison, que l’homme parvient à dominer la nature au point de pouvoir choisir sa nourriture conformément à ses besoins. Plus bas est son niveau et plus il doit se contenter de ce qu’il trouve, et, au lieu d’adapter sa nourriture à ses désirs, s’adapte lui-même à la nourriture dont il dispose. » C’est ainsi que, contrairement à ce qu’on voit affirmé partout, beaucoup de nègres africains se nourrissent de viande plus que de légumes, chose contraire à l’hygiène en leur climat tropical, et que beaucoup de tribus des régions polaires mangent plus de légumes que de viande. Mais, en se civilisant, le septentrional devient carnivore et le méridional végétarien. Or, à l’origine de chacune de ces petites révolutions alimentaires, qui font peu de bruit et n’en ont pas moins beaucoup d’influence sur la qualité autant que sur la densité de la population, nous trouvons toujours une \emph{idée heureuse} qui s’est propagée et dont la propagation a eu lieu toujours conformément à la loi générale de la descente des exemples. C’est des grandes propriétés qu’est descendue sur les petites, c’est des villes qu’est descendue sur les campagnes, la transformation contemporaine de l’agriculture, due aux progrès de la chimie, de la biologie, de la mécanique, à l’influence de la bactériologie, de toutes les découvertes de Pasteur, etc.\footnote{ \noindent Noter en passant, que Kautsky, qui parle beaucoup de son compatriote Liebig et lui attribue le rôle prépondérant dans les progrès de l’agriculture contemporaine, ne prononce pas le nom de Pasteur à propos de bactériologie... Cet international est plus nationaliste qu’il ne croit.
 }.\par
Ainsi, de la série des idées civilisatrices découlent les accroissements de la population, qui forcent le droit de propriété à se modifier sous certains rapports dans le sens d’une précision, d’une complication et d’une individualisation  \phantomsection
\label{v2p331}croissantes, mais qui, en même temps, à d’autres égards, augmentent, compliquent, diversifient le domaine public, sans cesse agrandi par des impôts nouveaux ou par des expropriations pour cause d’utilité publique ; le tout afin d’adapter de mieux en mieux la propriété individuelle ou collective à ses fins sociales.\par
Remarquons que ce n’est pas seulement par le moyen de ses effets sur la population que l’invention agit sur les transformations de la propriété. Elle exerce sur elles aussi une action directe. Sous l’empire d’un individu plus imaginatif et plus persuasif que d’autres, les populations primitives se sont toujours formé les idées les plus bizarres et les plus diverses sur leurs rapports avec le sol qu’elles occupent. Ces idées, de nature religieuse surtout, leur font croire que leur prospérité est attachée à la propriété indéfinie, héréditaire, de tel rocher, de telle source, de telle forêt réputée sacrée, de la terre où dorment leurs ancêtres. Ce lieu de la sépulture des morts a dû puissament contribuer, par les idées qu’il éveillait, à la formation précoce de la notion de propriété exclusive, de propriété familiale distincte de la propriété générale de la tribu.\par
Autre remarque. Si répandues qu’aient été les formes d’appropriation collective dont il a été question plus haut, peut-on les considérer comme ayant été universelles ? Non, car rien n’indique que les conditions qui les ont rendues nécessaires aient existé partout dans le passé de tous les peuples. Si haut qu’on remonte en arrière, on voit toujours des lieux privilégiés, des cantons insulaires ou même continentaux, à l’abri des bêtes fauves et des incursions de tribus hostiles, et qui ont permis aux premiers occupants agricoles, dès le début, l’appropriation individuelle. A cet égard, l’importance de l’élément géographique, explication en partie de la diversité des évolutions humaines, est considérable ; et le tort de l’école de Le Play est seulement de l’avoir exagérée. Mais, certainement, les familles ou bandes  \phantomsection
\label{v2p332}primitives, en quête d’abris contre l’invasion des grands carnassiers ou des tribus ennemies, ont dû attacher un prix infini à la possession de la caverne où ils ont souvent résisté avec succès à des coalitions hostiles. Le sentiment de la propriété de cette caverne a dû être singulièrement vif et profond chez ces troglodytes, et, par extension, a dû se répandre sur le sol immédiatement environnant. Il n’a pas dû en être de même dans les pays plats où tous les champs se ressemblent, au point de vue de la sécurité, ou plutôt de l’insécurité.\par
Encore une petite observation. Le grand reproche adressé par M. Loria et d’autres écrivains de la même école à la propriété privée est d’avoir fait disparaître la \emph{terre libre.} La vérité est qu’il n’y a pas encore eu une seule époque où, faute de terre inoccupée et disponible, les ailes du progrès aient été coupées. Ce n’est pas sous l’Empire romain, quand la terre manquait de bras. Ce n’est pas au moyen âge, où la moitié du sol était en friche, surtout après la grande peste et la guerre de Cent ans, et où, de tous côtés, les solitudes à repeupler appelaient les colons. Ce n’est pas de nos jours, où l’immense continent africain, sans parler des autres, s’offre aux Européens avides. On peut prévoir, il est vrai, le jour où une population exubérante et partout également quoique diversement civilisée, couvrant la terre entière après se l’être répartie individuellement, les nouveaux venus ne trouveront plus de place où reposer leur tête. Mais ce temps est loin, et, avant qu’il n’arrive, nous aurons le loisir de réfléchir à la solution des problèmes qui se poseront alors. Puis, il suffira d’une épidémie parmi ces populations si denses pour faire du vide et donner du large aux nouvelles générations. C’est toujours « le fonds qui manque le moins ». Mais l’on voudrait que la \emph{terre libre}, au lieu d’être en Amérique ou en Afrique, fût au cœur des nations civilisées \phantomsection
\label{v2p333} de l’Europe. En cela est l’utopie et la contradiction. Car, précisément parce que ces nations sont civilisées, elles ont dû s’approprier individuellement ou collectivement tout le sol qu’elles couvrent, et dès lors la terre n’y saurait plus être libre, autrement dit \emph{sauvage}.
\subsubsection[{III.4.f. Transformations actuelles, et leur tendance.}]{III.4.f. Transformations actuelles, et leur tendance.}
\noindent — Cela dit sur les transformations de la propriété dans le passé et sur les causes qui les ont produites, demandons-nous maintenant si, après avoir été jusqu’ici s’individualisant et se morcelant de plus en plus, au point de vue de la culture des terres et de l’exploitation industrielle des capitaux, sous l’action des inventions et découvertes civilisatrices, la propriété ne commencerait pas à révéler, par suite d’inventions et de découvertes récentes, d’événements contemporains, une tendance toute nouvelle et contraire. Demandons-nous, en d’autres termes, s’il y a des indices qui nous fassent augurer un mouvement prochain ou futur vers la socialisation de l’agriculture et de l’industrie, et non pas seulement la continuation de ce que nous voyons, l’extension des services publics, guerre, police, justice, instruction, voirie, assistance, etc.\par
Quelques écrivains socialistes comme nous l’avons dit plus haut, dans notre chapitre sur la concurrence, ont mis les statistiques à la torture pour faire rentrer la propriété foncière elle-même sous la loi de la concentration capitaliste qui doit nous conduire inévitablement, d’après Marx, à la nationalisation du sol aussi bien que de tous les instruments de travail quelconque.\par
Voici comment Kautsky parvient à concilier avec le morcellement croissant ou stationnaire du sol, le dogme marxiste de la concentration graduelle de la propriété, ce qui facilitera l’expropriation générale. A ses yeux, quand un agriculteur emprunte sur hypothèque, son créancier hypothécaire devient le véritable propriétaire foncier de sa terre, car ce créancier (p. 129) « est le \emph{propriétaire de la rente foncière},  \phantomsection
\label{v2p334}et, \emph{par suite}, le véritable propriétaire du sol ». L’agriculteur, paysan le plus souvent, n’est plus que « propriétaire nominal ». Si l’on accorde ces prémisses — à la vérité erronées, évidemment fausses — on n’a pas de peine à montrer que, puisque le chiffre des dettes hypothécaires augmente \emph{partout} rapidement, et que, partout aussi, aux prêts hypothécaires des petits capitalistes, se substituent les prêts hypothécaires de grandes associations centralisées, — notamment en Allemagne — ce phénomène équivaut à une concentration de la propriété foncière (p. 132).\par
Mais, alors même qu’il en serait ainsi, qu’est-ce que cela prouverait ? Et en quoi cette transformation serait-elle de nature à faciliter la grande expropriation rêvée ? Est-ce que les Sociétés de crédit qui centralisent le prêt hypothécaire ne sont pas formées d’actionnaires ? Est-ce que le nombre de ces actionnaires ne va pas croissant à mesure que le prêt hypothécaire se développe ? Il s’ensuit donc, si une créance sur hypothèque équivaut à une propriété immobilière, que le nombre des propriétaires va grandissant sous une forme nouvelle et \emph{susceptible, elle, d’une extension indéfinie}, à la différence de la forme directe de propriété immobilière qui ne saurait dépasser certaines limites de morcellement et de multiplication. En tout cas, ce qui est certain, c’est que le nombre de ceux qui sont intéressés à ce que la nationalisation du sol n’ait pas lieu ne cesse de grandir.\par
D’ailleurs, il est clair que l’intérêt payé au créancier hypothécaire ne saurait être, en moyenne, qu’\emph{inférieur} à la rente foncière. Sans cela, verrait-on progresser le nombre des prêts hypothécaires ? On nous apprend que, en Prusse, l’\emph{Institution de crédit de la noblesse de la marche électorale et de la nouvelle marche} a expédié des cédules hypothécaires pour une valeur qui a passé de 38 millions de marks en 1855, à 189 millions en 1895. Peut-on croire que, si les gentilshommes campagnards de ce pays, après les premières expériences faites de ces prêts, avaient compris qu’ils se  \phantomsection
\label{v2p335}ruinaient, ils auraient persisté à emprunter de plus belle ? Il est à noter que cet accroissement si rapide des dettes hypothécaires a coïncidé en Prusse avec une ère de prospérité exceptionnelle de l’agriculture, — car l’amélioration merveilleuse de l’outillage agricole, le nombre rapidement croissant des machines agricoles, en sont la suite — et \emph{il est surtout accentué dans les années de belles récoltes}. Ce développement du crédit rural révèle les progrès de l’utilisation des forces mécaniques, chimiques, végétales, animales, c’est-à-dire de l’adaptation de la terre à l’homme. En même temps on peut y voir le progrès des secours mutuels que se prêtent l’agriculture et l’industrie, celle-ci confiant à celle-là les capitaux dont elle a besoin, celle-là offrant à celle-ci un emploi sûr et avantageux pour les capitaux dont elle cherche le placement.\par
En réalité, il reste très douteux que le nombre des paysans qui vivent uniquement du travail agricole, comme propriétaires-cultivateurs, aille en décroissant. Ce nombre n’a jamais été aussi grand qu’on le pense ; car, jusqu’à nous, les petits métiers de tisserand, de forgeron, de menuisier, etc., donnaient un appoint indispensable aux revenus proprement agricoles des cultivateurs. La transformation de l’industrie a eu pour effet la disparition fréquente de ces petits métiers, et, par suite, la vente des petites propriétés qui ne peuvent plus faire vivre leurs maîtres, devenus émigrants et ouvriers des villes. Notons aussi que, pour cultiver une région donnée, à productivité égale, il faut de moins en moins de cultivateurs, par suite de la vulgarisation des machines. Pour ne parler que des batteuses mécaniques, leur nombre s’est élevé, en France, de 100 000 environ en 1862 à 234 000 en 1892, et, en Allemagne, de 75 000 en 1882 à 299 000 en 1893. Le dépeuplement des campagnes tient à cela surtout, d’après certains agronomes. D’autre part la grande propriété va-t-elle s’étendant ? Rien de moins prouvé ; et, quand même elle s’étendrait un peu,  \phantomsection
\label{v2p336}quelle signification aurait ce mouvement, bientôt peut-être suivi d’un mouvement opposé ? Ce qu’il y a de plus important à considérer ici, c’est que \emph{la grande propriété change de nature}. La propriété jadis féodale, celle des gentilshommes grands chasseurs oisifs, se vend et se morcelle chaque jour ; ou bien elle est achetée par des industriels retirés des affaires qui s’occupent avec une activité intelligente et novatrice de la surveillance de leurs terres et y pratiquent la grande culture. Ainsi, par en bas comme par en haut, la propriété se transforme ; elle devient de plus en plus adaptée à sa fin sociale. Elle est de plus en plus conçue et sentie, par les propriétaires eux-mêmes, comme une fonction sociale remplie par quelques-uns dans l’intérêt de tous, et non comme l’exercice du droit d’user et d’abuser de la chose. Mais se concentre-t-elle ? Les statistiques ne disent rien de net à cet égard. — Aussi Karl Marx, découragé, a-t-il laissé échapper l’aveu suivant : « L’agriculture doit passer indéfiniment de la concentration à l’émiettement et inversement tant que subsistera l’organisation de la société bourgeoise. » L’obsession de l’idée de balancement rythmique se voit ici. Mais que devient l’idée maîtresse du marxisme ?\par
Ce qui n’est pas douteux, c’est qu’une grande fermentation s’opère dans les campagnes, et qu’on doit s’attendre à de grands changements dans la population agricole ; mais dans quel sens ? « Le paysan se prolétarise », dit Kautsky. Cela n’est pas exact, et traduit mal sa propre pensée. Les paysans, il est vrai, depuis que les petits métiers disparaissent des campagnes, écrasés par la grande industrie, s’emploient plus souvent qu’autrefois comme ouvriers industriels, miniers par exemple, quand, dans leur voisinage, une usine, une fabrique est installée, et leur travail agricole devient là peu à peu l’accessoire de leur travail industriel, plus lucratif. Le paysan dans ces conditions, devient ouvrier et \emph{s’urbanise ;} mais, hors de ces conditions, que fait-il ? Là où, pour compléter ses ressources, les grandes propriétés  \phantomsection
\label{v2p337}ne suffisent pas à l’occuper par des journées de travail agricole bien rémunérées, il émigre soit définitivement, aux colonies, pour y continuer sous de nouvelles formes et dans de plus amples dimensions sa vie de paysan, soit passagèrement et avec esprit de retour, comme ouvrier de passage. Devenu nomade et touriste, comme tout le monde, il prend le chemin de fer et va chercher du travail agricole là où il sait, par les journaux, que les bras manquent. Et il \emph{voyage en bande} d’ordinaire comme les oiseaux migrateurs. Par exemple, au temps des vendanges, des paysans du Périgord noir vont en Gironde, et reviennent ensuite. En Bavière, entre les pays de blé et les pays de houblon il y a échange d’ouvriers agricoles, « les pays de houblon envoient leurs ouvriers pour la moisson et réciproquement ».\par
Parfois, il est vrai, ces bandes ne se limitent pas au travail rural et louent leurs services, en passant, à des usiniers mais pour une saison seulement ; et ils reviennent toujours au pays natal, consacrer à la meilleure culture de leur lopin de terre, de plus en plus cher, le fruit de leurs travaux nomades.\par
En se \emph{mobilisant} de la sorte, le paysan ne se \emph{prolétarise} pas le moins du monde, il ne s’\emph{industrialise} même pas, mais il est certain qu’il se \emph{dépaysannise}, car sa psychologie en est toute révolutionnée. Les nouvelles sensations, les nouvelles idées qu’il acquiert lui donnent un besoin de variété intellectuelle dont il ne sentait pas le manque jusque-là ; il sent l’ennui de l’isolement à présent, il lui faut des distractions, il s’efforce de mépriser toutes les superstitions et les préjugés de son village, dont il est encore imbu. Il se prépare enfin, par une transformation lente et dangereuse de son état d’âme, à l’état d’âme nouveau qui sera celui du paysan futur, du paysan éclairé, lettré peut-être, qui comptera des intellectuels et des artistes dans ses rangs.\par
Il est fort possible, donc, que le paysan disparaisse un jour, — l’ouvrier aussi bien. « Une culture scolaire supérieure, dit Kautsky, et le contentement de la vie de paysan  \phantomsection
\label{v2p338}ne sont pas compatibles. » Si cela est vrai, n’est-il pas plus vrai encore qu’une culture scolaire élevée rend l’ouvrier mécontent de son sort — à moins qu’il ne \emph{s’embourgeoise ?} Et est-ce que l’avenir nous réserverait, par hasard, cette surprise d’un embourgeoisement universel des ouvriers et des paysans ? Il est à remarquer que la vie paysanne peut s’améliorer beaucoup, atteindre à un assez grand confort, sans cesser d’être paysanne, sans devenir bourgeoise. Il n’en est pas ainsi, au même degré, de la vie ouvrière. Celle-ci n’a jamais su jusqu’ici s’élever sans s’embourgeoiser.\par
Kautsky n’a pas de peine à montrer la supériorité de la grande culture sur la petite, au point de vue du rendement net (au point de vue du rendement brut, c’est autre chose). Il montre l’économie faite sur les frais généraux, en bâtisses, en clôtures : « pour entourer 50 terrains de 20 ares chacun, il faudra 7 fois autant de palissades et de travail que pour entourer un seul terrain de 10 hectares », en outils aratoires, en semences mêmes, etc. Il insiste surtout sur cette considération que la grande culture permet seule « la division du travail entre les travailleurs nomades et les travailleurs intellectuels » ; c’est là, dit-il, son grand avantage. Car une exploitation agricole ne vaut la peine d’une direction scientifique incarnée dans un « agronome » qu’autant qu’elle dépasse une certaine étendue, — variable d’ailleurs d’après l’espèce de culture. En moyenne, il faut qu’un bien dépasse, dit-il, une centaine d’hectares pour occuper complètement un spécialiste en Allemagne.\par
Mais ce que Kautsky ne dit pas, c’est que, si un minimum d’étendue est exigé pour que la culture donne un maximum de rendement, un \emph{maximum d’étendue ne l’est pas moins.} L’agrandissement des propriétés, poussé au delà d’un certain degré, est aussi contraire que leur morcellement excessif à leur meilleure exploitation. Voilà, sans doute, pourquoi aux États-Unis, l’étendue moyenne des fermes a diminué depuis 1850. De dix ans en dix ans, la diminution est \emph{graduelle :}  \phantomsection
\label{v2p339}de 203 acres en 1830 à 134 en 1880. — Et, remarquons-le, la même considération est applicable à l’industrie. Poussé au delà d’un certain degré, l’agrandissement d’une exploitation industrielle donne lieu à des abus, à des gaspillages, à des négligences qui rappellent les administrations d’un État, et, au lieu du maximum d’effet pour un minimum d’effort, on a un minimum d’effet pour un maximum d’effort, \emph{tout comme dans la toute petite industrie.}\par
Cela veut dire que les collectivistes ne sont nullement autorisés à voir dans le remplacement graduel de la petite industrie par la grande, — et dans la supériorité à certains égards de la grande culture sur la petite culture, — une raison de croire à la nécessité, à l’imminence de la nationalisation de l’industrie et de la nationalisation du sol. Il y a tout lieu de croire, au contraire, que, par cet agrandissement démesuré et gigantesque qui résulterait de la nationalisation, l’industrie et la culture retourneraient peu à peu à l’improductivité relative de leurs débuts.\par
— Il y a, à chaque époque et en chaque région, un degré de grandeur auquel correspond, pour l’agriculture ou pour l’industrie, le maximum de production combiné avec le minimum de travail, — et ce degré, éminemment variable, dépend de la nature des inventions régnantes. Une invention nouvelle modifie toujours ce degré soit en plus, \emph{soit en moins.} Par exemple, le transport de la force à domicile par l’électricité peut avoir pour effet de faire renaître les petits métiers, redevenus plus productifs que les grands ateliers, dans beaucoup de cas, quoique, en général, les inventions nouvelles modifient en plus ce degré. Il y a aussi, à toute époque, un certain degré de grandeur territoriale auquel correspond le maximum de pouvoir politique d’un chef d’État. En deçà et au delà, son pouvoir est affaibli. Et les variations de ce degré dépendent aussi des inventions, surtout de celles qui sont relatives aux communications, au transport de l’action inter-mentale. — Mais ce degré de grandeur territoriale est, en  \phantomsection
\label{v2p340}ce qui [{\corr concerne}] le pouvoir politique, très supérieur à celui auquel correspond le maximum de production industrielle ou agricole. De là l’impossibilité ou le danger de nationaliser l’industrie et surtout l’agriculture.\par
Cependant, comme la moyenne des exploitations agricoles, dans les États civilisés, est bien loin d’atteindre le minimum d’étendue où commence la possibilité du rendement net maximum, on peut prévoir que, malgré la persistance actuelle du morcellement du sol, les petites propriétés, un jour ou l’autre, deviendront moins nombreuses et l’étendue moyenne des propriétés subsistantes se trouvera accru. En cela l’agriculture suivra de loin, avec sa lenteur caractéristique, l’exemple qui lui a été donné par l’industrie, où les petits métiers, après avoir longtemps résisté aux assauts des grandes fabriques, ont fini par céder. — Mais l’erreur, suggérée par ce fait, ou plutôt par cette prévision, est de penser que cette tendance (non encore dessinée) à un certain agrandissement des domaines se poursuivra indéfiniment. En tout cas, répétons que la prévision dont il s’agit est loin, jusqu’ici, d’être confirmée par la statistique. En Allemagne, il n’y a de progrès marqué que pour le nombre des exploitations d’étendue moyenne, de 5 à 20 hectares, non pour celui des grandes. En France, ce sont les grandes et les petites qui ont augmenté numériquement. En Angleterre, comme en Allemagne, il n’y a d’augmentation que pour les domaines moyens.\par
On pourrait donner, dès maintenant, des raisons de penser que l’agrandissement moyen des domaines, si, comme je le crois, il se produit, rencontrera des bornes assez proches. Kautsky n’ignore pas, car Thünen le lui a appris, que les inconvénients attachés à la distance des terres du siège de leur exploitation finissent vite par l’emporter sur les avantages inhérents à leur extension, et que, par suite, celle-ci ne saurait être indéfiniment désirable. Bien mieux, il sait que, plus la culture devient intensive, c’est-à-dire scientifique, \phantomsection
\label{v2p341} et plus s’amoindrit l’étendue maxima qu’une propriété ne peut dépasser sans que les profits du propriétaire en soient diminués. « Une propriété, nous dit-il, doit être d’autant plus petite qu’elle est exploitée d’une manière plus intensive avec un capital donné. » Concluez : plus nous progressons, plus nous tendons à nous éloigner, en somme, de la concentration des propriétés, acheminement prétendu inévitable vers la nationalisation du sol.\par
La persistance de la petite propriété peut s’expliquer encore par bien d’autres considérations, et Kautsky en signale [{\corr une}] que je ne fais aucune difficulté de lui accorder. Il reconnaît, en effet, qu’il ne voit nulle part la moindre trace d’une tendance des \emph{ménages}, des \emph{maisonnées}, à la centralisation : « nulle part, dit-il, le grand nombre des petits ménages ne tend à céder la place à un nombre restreint de grands ». Il eût été plus exact de dire que, au contraire, on constate une tendance manifeste des grands ménages patriarcaux d’autrefois à se disloquer en petits ménages indépendants : constatation peu encourageante, soit dit en passant, pour le rêve phalanstérien. Ainsi, le \emph{ménage} a une tendance « opposée à celle de l’industrie » qui va se centralisant. Pourquoi cela ? Car enfin les avantages économiques de la centralisation sont les mêmes pour le ménage que pour l’industrie. N’est-ce pas parce que le besoin de liberté et d’indépendance grandit sous l’action même des causes qui poussent l’industrie à se développer et à se centraliser ? N’est-ce pas aussi parce que l’individu tient d’autant plus à être indépendant et libre dans son ménage à soi qu’il est plus discipliné et plus assujetti dans son métier ? Quoi qu’il en soit, j’accorde à Kautsky que cette propension du groupe domestique à se resserrer plutôt qu’à s’élargir contribue à maintenir la petite propriété, puisque l’exploitation agricole fait partie intégrante et nécessaire du ménage paysan.\par
Il se peut enfin, — et Kautsky le reconnaît — que le  \phantomsection
\label{v2p342}progrès démocratique des sociétés modernes contribue à retarder la disparition des petits métiers devant les grandes fabriques, et aussi bien le refoulement de la petite propriété et de la petite culture par la grande propriété et la grande culture. Là où règne la souveraineté du nombre, il faut s’attendre à ce que les nombreux artisans menacés par une invention nouvelle se liguent contre elle et l’expulsent même, au préjudice de consommateurs plus nombreux encore, il est vrai, mais, en général, ignorants ou insoucieux de l’intérêt qu’ils auraient à prendre parti pour l’inventeur contre les protecteurs intéressés de la routine. « Plus la lutte des classes passe à l’état aigu, dit Kautsky, plus la démocratie socialiste devient menaçante, et plus les gouvernements sont disposés à faciliter aux petites exploitations, devenues économiquement superflues, une existence plus ou moins parasitaire aux dépens de la société. » Avouons que les gouvernements populaires, contre lesquels cette remarque peut paraître un rude argument, ont lieu d’être fort embarrassés par le problème qui se pose à eux. Entre un agrandissement moyen des ateliers et des exploitations agricoles, qui donneraient une augmentation de richesse, moyennant beaucoup de souffrances, et un rapetissement moyen qui donnerait une augmentation de bonheur ou une diminution de douleur avec moins de richesse, que choisir raisonnablement ? Faute de commune mesure, on ne peut se décider que d’après le vent de l’opinion.\par
— Le livre de Kautsky sur la \emph{question agraire} a pour but de montrer que l’\emph{agriculture doit aller s’industrialisant}. A vrai dire, cela signifie simplement que l’agriculture est sortie enfin, elle aussi, comme l’a fait depuis longtemps l’industrie avant elle, de la phase des marchés locaux et fermés et qu’elle entre à son tour dans l’ère du marché \emph{mondial}. D’ailleurs, c’est plutôt au début de l’évolution économique que l’on voit l’agriculture et l’industrie étroitement unies, mais l’une aussi embryonnaire que l’autre. Alors le  \phantomsection
\label{v2p343}cultivateur est en même temps tisserand, forgeron, menuisier, vannier, cordonnier, etc. ; et, de même, l’artisan des villes anciennes est aussi propriétaire d’un jardin, d’un lopin de terre qu’il cultive dans l’intervalle des travaux de son métier. C’est que ni l’agriculture alors ni l’industrie, pas plus celle-ci que celle-là, n’exigent un grand et coûteux outillage, et ne travaillent pour une clientèle lointaine, imposante, indéfinie. Mais, à présent, les conditions du travail sont changées ; et, comme ce changement s’est produit plus tôt pour l’industrie que pour l’agriculture, on a vu les petits métiers émigrer des champs et le paysan devenir peu à peu un pur agriculteur. Or, dans l’agriculture aussi, commence à se faire sentir, avec le besoin de débouchés extérieurs de plus en plus étendus, la nécessité de machines coûteuses et de méthodes nouvelles\footnote{ \noindent Il y a longtemps que l’agriculture, en ce sens, s’industrialise, c’est-à-dire se spécialise, et, en se spécialisant, s’\emph{adapte} de mieux en mieux au client et au sol, — c’est-à-dire à sa \emph{meilleure productivité...} « Si l’on compare, dit Hitier (\emph{Revue d’écon. polit.}, juin 1901) une carte de la France agricole de la fin du {\scshape xviii}\textsuperscript{e} siècle à une carte de la France actuelle, on constate immédiatement que les zones des plantes cultivées tendent de plus en plus à représenter les zones naturelles du climat et du sol. Si l’on suppose les cultures indiquées par des teintes, ce qui frappe tout d’abord c’est une moindre dispersion des tâches. »
 }. Elle se transforme, c’est clair ; et il est clair aussi que, pour répondre aux exigences de cette évolution, un remaniement de la législation sur la propriété foncière s’imposera sans tarder. Dans quel sens ? Il semble que le succès des syndicats agricoles indique nettement la voie dans laquelle il est permis d’attendre la solution pratique des problèmes qui se dressent devant le travailleur des champs. Il s’agit de savoir comment pourront se combiner avec les avantages sociaux de la petite ou de la moyenne propriété les avantages économiques de la grande culture. L’association agricole commence déjà à répondre à cette demande ; et sa réponse est telle qu’elle dispense d’invoquer des procédés plus radicaux. Il n’en est pas moins vrai que le droit d’expropriation pour cause d’utilité publique est destiné à s’étendre beaucoup encore,  \phantomsection
\label{v2p344}ainsi que la liste des servitudes rurales et urbaines de la propriété bâtie ou non bâtie. On en viendra même, je l’espère, à prendre des mesures légales qui forceront les propriétaires à respecter les beautés pittoresques du sol, à ne pas enlaidir le paysage, propriété visuelle, collective au plus haut degré.\par
La plus grande difficulté peut-être consiste dans la transformation psychologique du paysan, qui est requise par cette transformation économique et qui ne s’accomplira pas en un jour. Le paysan, type formé et consolidé au cours d’une hérédité séculaire, est caractérisé par la rareté à la fois et l’intensité des actions inter-mentales, par le monoïdéisme silencieux et tenace, par une sorte de sobriété cérébrale, pour ainsi dire, qui se contente d’un minimum d’idées profondément ruminées, par l’extrême docilité aux exemples domestiques et ancestraux et la très faible sensibilité aux exemples extérieurs. Est-ce que ces traits de la psychologie paysanne sont destinés à s’effacer ? D’une part, on peut faire observer qu’ils tendent plutôt à s’accentuer chez ceux qui restent aux champs ; car, à mesure que se poursuit l’émigration des garçons les plus intelligents et des filles les plus belles, la densité à la fois et la qualité de la population rurale diminuant, l’action inter-mentale s’y raréfie ; et la disparition des métiers y rend plus profonde encore la monotonie de la vie. En sorte que, si rien ne vient arracher le paysan à son isolement, à son silence, à son misonéisme croissant, il va bientôt devenir un être tout à fait à part, inassimilable, étranger au reste de la nation. Mais d’autre part, quand on songe aux progrès de l’instruction dans les campagnes, à la diffusion des livres, à la fréquence plus grande des voyages, et surtout aux adoucissements graduels du travail agricole qui, suppléé par les machines, devient de moins en moins une simple dépense de forces musculaires, de plus en plus une occupation très intéressante de l’esprit, on ne peut se défendre d’espérer, dans un avenir assez prochain, un mouvement de  \phantomsection
\label{v2p345}décentralisation intellectuelle dont la campagne bénéficiera. Déjà la charrue à vapeur est jugée trop lourde, trop arriérée. La charrue électrique, plus légère, la remplacera. La force perdue des chutes d’eau peut s’utiliser ainsi. Quand les cascades, ou la marée, ou les vents laboureront à la place des bœufs, il faudra des mains plus fines pour tenir la charrue.\par
Ce problème des transformations nécessaires du paysan, qu’il s’agit d’affiner sans le dénaturer, s’imposera bientôt aux socialistes eux-mêmes\footnote{ \noindent Noter ce passage de Kautsky : « La démocratie socialiste, au commencement, se soucia peu du paysan. C’est \emph{qu’elle n’est pas une démocratie au sens bourgeois du mot, une bienfaitrice de tout le monde,} cherchant à donner satisfaction aux intérêts de toutes les classes, si opposés qu’ils puissent être les uns aux autres ; \emph{elle est un parti de lutte de classe.} »
 }. Malgré les services que la campagne rend à la ville et qui sont réciproques, il existe un antagonisme croissant, en somme, entre la ville et la campagne, qui se disputent la population, le pouvoir, le bien-être et le luxe. Comment se résoudra cet antagonisme ? Sera-ce par l’assimilation des citadins aux ruraux ? Non assurément ; ce ne peut être, à l’inverse, que par l’urbanisation des ruraux, dans une certaine mesure au moins. Car il ne faut point oublier que l’agriculture aura beau s’industrialiser, elle restera toujours l’agriculture, c’est-à-dire une production assujettie, par sa collaboration avec la pluie et le beau temps, à des conditions spéciales qui lui imposent une lenteur et un aléa caractéristiques. L’agriculteur ne pourra donc jamais s’urbaniser tout à fait et il serait fâcheux que ce fût possible. L’adaptation de l’homme à la terre et de la terre à l’homme exige, avant tout, une qualité, bien plus développée chez l’homme des champs que chez l’homme des villes, la patience, la résignation, c’est-à-dire l’adaptation de l’homme à son destin. C’est le fond de l’âme paysanne, qu’il importera toujours de conserver.
 \phantomsection
\label{v2p346}\subsection[{III.5. L’échange}]{III.5. L’échange}\phantomsection
\label{l3ch5}
\subsubsection[{III.5.a. Documents inter-psychologiques à puiser : 1o dans les rapports économiques et spontanés des enfants, dans les cours de collèges ; 2o dans ceux des sauvages avec les premiers navigateurs qui les découvraient.}]{III.5.a. Documents inter-psychologiques à puiser : 1\textsuperscript{o} dans les rapports économiques et spontanés des enfants, dans les cours de collèges ; 2\textsuperscript{o} dans ceux des sauvages avec les premiers navigateurs qui les découvraient.}
\noindent A l’origine de toute adaptation économique, nous l’avons montré surabondamment, il y a une invention. Mais, sans la division du travail et l’échange, qui lui servent d’auxiliaires merveilleux, l’invention, réduite à elle-même, ne pourrait pousser bien loin son œuvre d’harmonisation. Elle ne parviendrait à adapter les travaux de l’individu qu’à la satisfaction de ses propres besoins (à moins que ses produits ne fussent volés ou donnés), ce qui, d’une part, ne permettrait qu’à un très petit nombre de besoins individuels d’être par elle satisfaits, à défaut d’une suffisante économie de force et de temps ; d’autre part, elle ne servirait en rien à établir ou à étendre l’accord unilatéral ou réciproque des individus, l’adaptation collective de leurs travaux à leurs besoins.\par
Nous avons déjà parlé de la division du travail, et nous en reparlerons encore en passant, à l’occasion de l’\emph{association} dont elle est, avec l’échange, la première ébauche spontanée... Mais ce sujet, au point de vue qui nous occupe, peut être regardé comme épuisé\footnote{ \noindent Voir, avant tout, Bücher, \emph{Études d’histoire et d’économie politique.}
 } par les écrits des économistes, il est donc inutile d’y consacrer de longs développements. Occupons-nous plutôt de l’échange, et commençons par rechercher ses origines probables, en remontant aux premiers stades de la vie économique, tels qu’il nous est permis, non de les conjecturer, mais de les observer clairement sous nos yeux dans les cours de collège, et, plus manifestement  \phantomsection
\label{v2p347}encore, dans les relations des navigateurs des trois derniers siècles avec les peuplades insulaires qu’ils ont découvertes. Beaucoup d’autres documents peuvent être demandés, et l’ont été par des érudits, à l’archéologie historique ou préhistorique ; mais on peut contester leurs résultats fondés sur beaucoup de conjectures ; tenons-nous-en ici aux deux sources d’informations que je viens d’indiquer.\par
La première ne nous retiendra guère. Il y aurait toute une enquête à faire dans les cours d’écoles primaires ou des basses classes de collèges et de lycées, sur l’initiation spontanée et graduelle des enfants à la vie sociale, sur les manifestations plus ou moins instinctives de sens politique, de sens esthétique, de sens économique, qui apparaissent dans leurs libres jeux. Je ne trouve pas que les psychologues de l’enfance, qui se sont le plus souvent circonscrits dans l’étude de l’enfant isolé ou en rapport avec ses parents et ses maîtres, se soient beaucoup préoccupés de l’intérêt que présentent ses contacts avec ses camarades à ce point de vue. L’\emph{inter-psychologie} infantile est encore un champ où il y a à moissonner.\par
En attendant que cette moisson soit faite, remarquons brièvement, au point de vue économique, que les petits marchés, les petits contrats des enfants entre eux, leurs cadeaux unilatéraux ou réciproques, leurs petits vols eux-mêmes sont un apprentissage de la vie commerciale ; et que leurs petits essais de maçonnerie, de menuiserie, de serrurerie, leurs constructions de petites machines, sont un apprentissage de la vie industrielle. — Y a-t-il là rien qui rappelle les phases que, d’après les \emph{préhistoriens}, l’évolution économique des sociétés aurait traversées à ses débuts ; de telle sorte que la sociologie infantile, à cet égard, pourrait en être considérée comme la répétition abrégée ? Ce serait là le pendant, en science sociale, de ce que la science des êtres vivants nous a appris sur la reproduction en miniature de l’histoire antique de l’espèce par les formes successives  \phantomsection
\label{v2p348}de l’embryon. Mais, pas plus qu’en biologie d’ailleurs, les faits, observés de près, ne confirment pleinement cette hypothèse ; et tout ce qu’on peut dire, c’est que les notions économiques des enfants ne sont pas sans offrir quelques rapprochements, comme nous le verrons, avec celles des sauvages.\par
Ce dont on n’aperçoit jamais la moindre trace, par exemple, dans les cours d’écoliers, c’est la phase de la propriété collective qu’on dit avoir été antérieure à la propriété individuelle. L’enfant le plus jeune fait preuve d’une dose extraordinaire de possessivité. Mais quand l’idée de construire ou de fabriquer quelque chose, par exemple, de bâtir une cabane ou un petit fort, devient commune à un groupe d’enfants, on assiste fréquemment parmi eux à un essai rudimentaire d’organisation du travail. Et leur penchant habituel alors est de se soumettre à l’un d’eux, reconnu unanimement pour le plus industrieux de la bande.\par
Il y a dans toute cour de collège, un enfant, un adolescent, qui se distingue des autres par son ingéniosité à construire de petites machines, — des machines de guerre le plus souvent, balistes ou catapultes des anciens, arbalètes, car c’est de ce côté que l’industrie de l’enfant, comme celle de l’homme primitif, s’est d’abord donné carrière. Cet écolier, né mécanicien, ne doit pas être confondu avec celui qui est né \emph{chef de parti}, qui fomente les révoltes. Ces deux espèces de supériorité s’opposent bien plus qu’elles ne s’allient. Et la supériorité industrielle, ce me semble, si mes souvenirs sont exacts, est beaucoup plus rare. Sur cent élèves, c’est à peine si l’on en trouve un industriellement inventif — et dont les inventions ne sont, au surplus, bien entendu, qu’un tour ingénieux d’imitations précoces et bien combinées.\par
Une chose manifeste, pour qui a observé les enfants, c’est que l’échange n’est nullement un fait primitif dans leurs rapports mutuels. En cela, ils ressemblent parfaitement aux  \phantomsection
\label{v2p349}sauvages. L’enfant que séduit la vue d’un objet en la possession de l’un de ses camarades, cherche d’abord à se le faire donner — ou à le voler par ruse ou à le prendre de force. Il est né voleur, pillard, donateur — surtout donataire — mais il ne naît pas échangiste. Il le devient quand l’expérience lui a montré les inconvénients du don ou du vol et quand il ne se sent ni assez fort pour prendre de force ni assez fin pour voler. C’est conforme à la loi d’après laquelle les relations unilatérales précèdent les relations réciproques. — Alors, le troc étant proposé, le marchandage commence ; et, si l’on doutait de l’importance du degré de désir dans la fixation du prix, on n’aurait qu’à assister à l’une quelconque de ces discussions d’écoliers qui échangent un couteau contre un porte-crayon. — L’échange coexiste avec la vente et l’achat ; parce que, si l’argent manque aux écoliers, ils ne tardent guère à se faire une monnaie à leur usage, des billes par exemple. Mais il n’y a jamais de \emph{prix uniforme} ni de \emph{prix fixe ;} le même écolier vendra le même objet à des prix très inégaux suivant que l’acheteur lui sera plus ou moins sympathique. Le prix est toujours individuel et variable essentiellement. — Quand un objet brillant, rare, — inutile le plus souvent — est importé par un camarade dans sa cour, et qu’on s’attroupe autour de lui pour l’admirer, il faut voir de quels yeux on le regarde ; s’il fait envie à deux, à trois, bientôt à tous, on en offrira un prix fou. L’avivement du désir par sa propagation même est ici un phénomène des plus visibles et des plus habituels.
\subsubsection[{III.5.b. Voyages de Christophe Colomb et d’autres grands marins. Le don et le vol précédant toujours l’échange. Danses et chants, prélude de tout marché.}]{III.5.b. Voyages de Christophe Colomb et d’autres grands marins. Le don et le vol précédant toujours l’échange. Danses et chants, prélude de tout marché.}
\noindent On pourrait faire une infinité d’autres remarques, et bien plus intéressantes, sur ce même sujet. Mais j’ai hâte de passer à une seconde source d’informations. Les récits de navigateurs qui, depuis Christophe Colomb, ont découvert de nouvelles terres, de nouvelles îles, sont précieux à notre  \phantomsection
\label{v2p350}point de vue. Il y avait, entre les navigateurs civilisés et les indiens cuivrés ou noirs qui les voyaient pour la première fois, la plus grande inégalité, la plus profonde dissemblance physiologique ou sociale qui puisse creuser un abîme entre deux groupes humains. Langue, religion, mœurs, traits et couleur du visage, formes corporelles, tout les séparait par un fossé, en apparence, impossible à combler. On aurait pu s’attendre à ce que la sympathie naturelle de l’homme pour l’homme ne fût pas assez forte pour franchir ce fossé et se faire jour en dépit de ces dissemblances qui empêchaient de se comprendre, et de cette inégalité qui semblait devoir empêcher de s’aimer. Mais, en fait, c’est seulement aux voyageurs européens qu’on est en droit de reprocher leur dureté de cœur à l’égard des malheureux indiens qu’ils traitaient en gibier ; ceux-ci, parfois effrayés de la supériorité des nouveaux venus, n’en étaient pas moins disposés, dès leur apparition, à leur témoigner une sympathie naïve, exprimée de la manière la plus touchante. D’où nous devons induire que, à plus forte raison, quand une tribu sauvage rencontrait pour la première fois une autre tribu, supérieure ou non, mais, en tout cas, beaucoup plus rapprochée d’elle que ne le sont les insulaires de nos navigateurs, ce premier contact n’a pas pu être habituellement un choc belliqueux, comme on le suppose sans motif, et qu’il a été le plus souvent une occasion de fêtes et de présents mutuels, un débouché nouveau offert au besoin de curiosité, d’amitié exotique, d’alliance joyeuse. Sur ces rencontres des tribus entre elles, où beaucoup d’érudits voient la véritable origine des phénomènes économiques, nés, disent-ils, de relations internationales, les récits de voyages accomplis dans les temps modernes, jettent un jour clair, pur de tout nuage d’hypothèse.\par
On remarquera, ici comme tout à l’heure, que, comme entrée en relation, on a d’ordinaire le don ou le vol ; l’échange suit. Voici quelles furent les premières relations de Christophe \phantomsection
\label{v2p351} Colomb, le 12 octobre 1492, avec les indigènes d’Amérique. « Désirant leur inspirer de l’amitié pour nous et persuadé qu’ils seraient mieux disposés à embrasser notre sainte foi si nous usions de douceur..., je fis don à plusieurs d’entre eux de bonnets de couleur et de perles de verre qu’ils mirent à leur cou. Ils témoignèrent une véritable joie et ils se montrèrent si reconnaissants que nous en fûmes émerveillés. Quand nous fûmes sur les embarcations, ils vinrent à la nage vers nous pour nous offrir des perroquets, des pelotes de fil de coton, des zagaies et beaucoup d’autres choses : en échange, nous leur donnâmes de petites perles de verre, des grelots et d’autres objets. Ils acceptaient tout ce que nous leurs présentions, de même qu’ils nous donnaient tout ce qu’ils avaient. »\par
Le lendemain, 13 octobre, les indigènes se familiarisent. La cupidité naît de part et d’autre. Colomb remarque qu’ils portent au nez un anneau d’or, et eux, de leur côté, se mettent à dérober ce qui leur plaît dans le navire et à se sauver à la nage « lorsqu’ils n’avaient rien à donner en échange ». Mais « \emph{ils donnaient très volontiers tout ce qu’ils avaient pour nos moindres bagatelles} ». On devine très bien par là l’état d’âme de ces bons sauvages. Incapables, à cause de l’étroitesse de leur esprit, d’avoir deux désirs à la fois, dès lors que le désir vif de posséder des verroteries leur vient, ils perdent instantanément tout désir de retenir les ustensiles les plus utiles, les objets les plus nécessaires qu’ils possèdent. Aussi donnent-ils « très volontiers », tout ce qu’ils ont d’indispensable pour acquérir le superflu qui leur manque. Cette impossibilité d’avoir plusieurs désirs à la fois a dû jouer un grand rôle au début de l’évolution économique et n’a pas cessé d’être un facteur important de la détermination des prix. On n’explique qu’ainsi les prix extravagants où s’élèvent les objets à la mode aussi longtemps qu’ils ont le mérite de la nouveauté.\par
Ce qui est visible, dans tous les échanges des navigateurs  \phantomsection
\label{v2p352}avec les sauvages c’est que \emph{les civilisés gagnent énormément au change.} Il a dû toujours en être ainsi. Par l’appât de la nouveauté, de l’éclat superficiel, les marchands phéniciens exploitaient de même les populations simples des côtes de la Méditerranée. C’est toujours par des cadeaux faits aux indigènes qu’on les amorce, qu’on leur suggère d’apporter quelque chose à leur tour. Ainsi procède Colomb. En approchant de l’île Saint-Ferdinand, il voit s’approcher une pirogue. Il fit servir au sauvage « du pain, du miel, de la boisson ». Le lendemain, « la conduite tenue à l’égard de l’indien avait porté ses fruits. Avant le jour, de grandes pirogues remplies d’habitants vinrent apporter de l’eau et beaucoup d’autres choses ». Colomb fit donner à ces indiens des perles isolées (de verre) ou enfilées par douzaines, de petits tambours de basque, et autres bagatelles.\par
Cependant, Colomb remarque que ces habitants de Saint-Ferdinand « paraissent plus habiles, plus rusés, car ils cherchent à tirer le meilleur parti possible de leurs échanges... » Déjà le marchandage apparaît.\par
Tout cela, remarquons-le, avant qu’on pût parler ensemble, car on ne se faisait comprendre que \emph{par signes.}\par
— Les jours suivants, mêmes faits. On apprivoise les indiens avec de petits cadeaux, ils en apportent en échange, et l’on entre dans la voie des petits marchés — toujours fort avantageux pour les européens.\par
D’autres fois, les indiens viennent d’eux-mêmes, en pirogue, au-devant des navires, pour proposer l’échange avec les objets qu’ils apportent, pelotes de coton, hamacs, fruits.\par
On peut se demander comment les premières tribus qui s’abordaient, ne parlant point la même langue et n’ayant point d’interprètes, pouvaient se comprendre et entrer en relations économiques. Un passage de Magellan peut présenter quelque intérêt à cet égard. Magellan est en Patagonie. « Deux mois s’écoulent avant que nous apercevions aucun des habitants du pays. Un jour que nous nous y attendions le  \phantomsection
\label{v2p353}moins, un homme de figure gigantesque se présenta à nous. Il était sur le sable, presque nu, et chantait et dansait en même temps en se jetant de la poussière sur la tête. Le capitaine envoya à terre un de nos matelots \emph{avec ordre de faire les même gestes, comme une marque d’amitié et de paix, ce qui fut très bien compris.} Et le géant se laissa paisiblement conduire dans une petite île où le capitaine était descendu... » Il semble bien, en effet, que le premier langage des hommes n’a pu être que l’imitation des gestes des uns par les autres. Mais continuons.\par
« On donne à ce géant un petit miroir (qui lui fit peur), des grelots, un peigne, des verroteries. D’autres accoururent, et d’abord — prélude à l’échange, tel que nous le verrons ailleurs, au cap de Bonne-Espérance, par exemple — \emph{ils commencèrent aussitôt leur danse et leur chant}, pendant lesquels ils levaient l’index vers le ciel, pour nous faire entendre qu’ils nous regardaient comme des êtres descendus d’en haut. Ils nous montraient en même temps de la poudre blanche dans des marmites d’argile, et nous les présentèrent. »\par
Les jours suivants, autres échanges des présents... Mais, sous des démonstrations de confiance et d’amour, beaucoup de méfiance, des armes soigneusement cachées. — Et ce n’était pas à tort que ces pauvres sauvages se méfiaient des européens, qui, par ruse, s’emparèrent de deux d’entre eux pour les emmener.\par
On voit souvent, dans les récits des voyageurs, l’impossibilité où sont les sauvages, au moment de leurs premiers rapports avec d’autres peuples, de pratiquer tout d’abord l’échange des marchandises. Ils commencent par offrir des présents, ou bien par voler — ou bien ils volent les voyageurs mêmes à qui ils viennent de faire des cadeaux, et qui viennent de leur en faire. En un mot, ces deux relations unilatérales, le don et le vol, entrent tout naturellement dans leur esprit ; mais cette relation réciproque, l’échange, est trop  \phantomsection
\label{v2p354}complexe pour y pénétrer avant un certain temps, et sans un certain effort. — Pour preuve, entre mille, ce passage du second voyage de l’espagnol Mendana (1595). Dès qu’il est en vue de l’une des îles Marquises, une nuée de pirogues accourt vers ses navires, montées par quatre cents individus tout nus. « Arrivés aux navires, \emph{ils offrirent des cocos}, des espèces de noix, un certain mets ressemblant à de la pâte, de bonnes bananes et de l’eau. On en attira un par la main et on le tira dans le vaisseau. Plus de quarante autres, encouragés par le bon accueil qu’on lui faisait, montent à leur tour ; \emph{ils acceptèrent des présents. Mais bientôt ils se mirent à piller tout ce qui se rencontrait sous leur main}... » Un combat s’ensuivit. Même remarque chez Coock. Aux îles Hawaï, il est fêté, accueilli à bras ouverts, avec des présents de toutes sortes. Mais, ajoute-t-il, « le plaisir que nous causait leur \emph{bienfaisance} et leur douceur fut néanmoins troublé souvent par leur disposition au \emph{vol}, vice commun chez toutes les autres peuplades répandues sur ces mers ».\par
— Jamais on ne débute par l’échange. Bougainville, en arrivant à Taïti, voit accourir des indigènes lui apportant un petit cochon et des bananes. Il accepte, et offre à son tour des bonnets et des mouchoirs « et ces présents furent le gage de notre alliance avec ce peuple ». Après quoi, on échange. « Bientôt, plus de cent pirogues de grandeurs différentes, environnèrent les deux vaisseaux. Elles étaient chargées de cocos, de bananes et d’autres fruits du pays. L’échange de ces fruits, délicieux pour nous, contre toutes sortes de bagatelles, se fit avec bonne foi, mais sans qu’aucun des insulaires voulût monter à bord. (Ce détail montre un reste de méfiance.)\par
La \emph{procédure} de l’échange est à noter... « Il fallait entrer dans les pirogues ou montrer de loin les objets d’échange ; \emph{lorsqu’on était d’accord}, on leur envoyait, au bout d’une corde, un panier ou un filet ; ils y mettaient leurs effets et nous les nôtres, \emph{donnant ou recevant indifféremment avant  \phantomsection
\label{v2p355}que d’avoir reçu ou donné}, avec une bonne foi qui nous fit bien augurer de leur caractère. »\par
Les jours suivants, autres échanges. « \emph{Il s’ouvrit de nouvelles branches de commerce}, les insulaires apportèrent avec eux toutes sortes d’instruments pour la pêche, des herminettes de pierre, des étoffes singulières, des coquilles, etc. Ils demandèrent en échange du \emph{fer} et des \emph{pendants d’oreilles}. » Du \emph{fer} et des \emph{pendants d’oreilles :} à rapprocher du \emph{panem et circenses.}\par
Peu à peu la confiance s’accroît avec les échanges. Les jolies femmes arrivent, les insulaires montent à bord des vaisseaux.\par
— Malgré la bonne foi dont les indigènes faisaient preuve dans les échanges, il ne se faisaient aucun scrupule de voler. « Il n’y a pas, en Europe, de plus adroits filous que les gens de ce pays. » Mais ils ne se volaient pas entre eux. « Il ne semble pas que le vol soit ordinaire entre eux. \emph{Rien ne ferme dans leurs maisons}.\par
« \emph{Au vol près}, tout se passait de la manière la plus amiable... On nous invitait à entrer dans les maisons, on nous y donnait à manger... »\par
Exercer l’hospitalité et voler ses hôtes, \emph{donner} et \emph{voler} alternativement, voilà qui est bien primitif... Dans les villages, même français, des provinces encore arriérées, combien de fois un propriétaire a-t-il occasion de remarquer que ses proches voisins, avec qui il vit en très bonne intelligence, qui lui font des cadeaux à certaines occasions réglées par la coutume, sont précisément ceux qui maraudent dans ses champs, coupent ses bois, font paître leurs troupeaux dans ses prés !\par
— Quand, dès l’arrivée d’un navire, les indigènes proposent l’échange sans commencer par offrir des présents, c’est qu’il s’agit de peuplades déjà habituées à frayer avec des européens ou d’autres peuples sur un grand continent, et qui, depuis longtemps, ont passé le stade des cadeaux réciproques. \phantomsection
\label{v2p356} Chez les peuplades les plus sauvages, si elles habitent un continent et non une île, l’habitude de l’échange se développe vite ; aussi La Pérouse a-t-il été frappé de voir, dans l’Amérique russe, des tribus très primitives animées d’un esprit mercantile. Il signale la chose comme une étrangeté. « Ils nous proposaient, dit-il, en échange de notre fer, du poisson, des peaux de loutre, etc. Ils avaient l’air, \emph{à notre grand étonnement}, d’être accoutumés au trafic, et ils faisaient aussi bien leur marché que les plus habiles acheteurs d’Europe. »\par
A propos d’échange, il faut noter, chez les primitifs, le plaisir qu’ils ont à échanger leur nom avec celui d’une autre personne, comme témoignage et sceau [{\corr d’amitié}]. Quand Mendana aborda à l’île Sainte-Isabelle ({\scshape xvi}\textsuperscript{e} siècle), « un chef vint à nous, dit le narrateur, accompagné d’autres indiens ; il se nommait \emph{Tauriqui Biliban Harra.} Il proposa au général, par amitié, de changer de nom, disant qu’il voulait s’appeler Alvaro de Mendana, et que le général se nommerait Tauriqui Biliban Harra. » Après quoi on fit de la musique, de part et d’autre. — A Santa-Cruz, dans le même voyage, même proposition faite par un autre chef indigène... Parmi des insulaires, « ayant la tête et les narines percées de fleurs rouges », un sauvage, plus distingué que les autres et paraissant leur commander, demande par signes « où était le chef des étrangers. Le général court à lui les bras ouvert. Alors l’indien dit qu’il s’appelait \emph{Malope}. Notre général réplique qu’il s’appelait \emph{Mendana.} Aussitôt l’indien s’efforça de faire entendre qu’il fallait troquer les noms ; qu’il s’appellerait Mendana et que le général s’appellerait Malope... Il parut fort satisfait de cet échange... »\par
Si l’on songe au pouvoir mystérieux que les mots et les noms possèdent aux yeux des primitifs, et qui subsiste quelque peu dans la magie de la poésie chez les civilisés eux-mêmes, on rapprochera cet échange des noms de la fusion et de l’échange des sangs qui a pour effet, dans tant de peuplades, \phantomsection
\label{v2p357} de sceller les alliances et les adoptions. — L’échange des armes a le même sens. Et il est possible que l’échange des armes et des noms ait favorisé celui des marchandises.\par
Il y a lieu de penser que, chez les primitifs, l’étranger qui apprend la langue indigène pénètre par là, — en vertu du pouvoir mystérieux attaché aux mots comme aux noms — dans l’intimité de l’union sociale, bien plus avant que n’y pénètre chez nous un Anglais ou un Allemand qui parle le français. — Un mot du voyageur Pyrard de Laval qui, en 1602, échoua aux Maldives, et fut retenu longtemps en captivité auprès du roi de l’une de ces îles, jette une lumière sur ce point. Il avait appris la langue des insulaires. « J’ai remarqué, dit-il, qu’il n’y a rien qui m’ait tant servi et qui m’ait plus attiré la bienveillance des habitants, des seigneurs et des rois mêmes, que d’avoir appris leur langue, et que c’était l’occasion pour laquelle j’étais préféré à mes compagnons et plus chéri qu’eux... » On le traitait en compatriote, en frère. — Apprendre la langue d’une peuplade a dû être, dans les premières phases de l’humanité, un des procédés les plus sûrs de naturalisation. Il a dû être fréquemment employé quand il y a eu avantage à s’incorporer à une peuplade conquérante, et l’expansion de certaines langues hors des limites de la race originelle s’explique par là.\par
Dans certaines écoles, on a attribué aux primitifs une horreur native et générale des nouveautés, des innovations quelconques. Mais tous les voyages de découvertes donnent un démenti complet à cette affirmation. Jamais, lors des premières explorations et découvertes maritimes, — et avant que les insulaires eussent été avertis, par de terribles expériences, de la rapacité féroce des Européens, — jamais on ne voit se manifester, chez les sauvages, ce \emph{misonéisme} qui serait, nous dit-on, leur caractère le plus frappant. Toujours on voit les indigènes accueillir avec curiosité, avec sympathie, le \emph{nouveau} qui vient à eux sous la forme de l’homme blanc. Non seulement ils ne lui montrent aucune haine,  \phantomsection
\label{v2p358}aucune antipathie, et ne le fuient pas, mais ils accourent vers lui, et ne se lassent pas d’admirer les verroteries, les étoffes, les miroirs, les choses inconnues qui lui sont montrées. Leur passion de ces \emph{nouveautés} est telle qu’elle les pousse à se dépouiller de tout ce qu’ils ont de plus précieux et de plus indispensable, de leurs aliments, de leurs vêtements, de leurs armes même, pour acquérir quelques échantillons insignifiants de ces articles exotiques. Singulier \emph{misonéisme !}\par
On peut dire que le \emph{philonéisme} des primitifs est leur trait le plus frappant et celui qui seul a permis aux Européens d’entrer en relations avec eux — comme il leur a seul permis aussi d’entrer en relations les uns avec les autres.\par
Ce n’est donc pas le désir des marchandises offertes par les navigateurs qu’il est malaisé de faire naître chez les insulaires ; mais c’est bien plutôt la croyance, la confiance, dans la sincérité des offres et la loyauté des offreurs. N’est-ce pas pour faire tomber la méfiance et inspirer confiance — condition \emph{sine quâ non} de tout marché et première grande difficulté à vaincre — qu’a pris naissance l’usage de faire précéder ou d’accompagner de danses, de chants, de réjouissances, toute transaction avec un étranger ? En tout cas, cet usage est des plus répandus et des plus instructifs à notre point de vue.\par
En abordant dans une baie, près du cap de Bonne-Espérance, Vasco de Gama voit accourir des Boschimans ; il leur jette des grelots, et ces sauvages les prennent. « Non seulement ils reçurent ce qu’on leur lançait ainsi, mais ils vinrent prendre les objets des propres mains du capitan-mor, ce qui nous émerveilla fort. » — C’est de la même manière, remarquons-le, que s’y prennent les habitués du Luxembourg ou des Tuileries pour apprivoiser des moineaux. Ils leur lancent des miettes de pain, que les pierrots viennent prendre jusque dans leur main, « ce qui les émerveille fort ». — Mais la différence est que, avec les oiseaux, les choses en restent là, la familiarité de ces petits amis devenant \phantomsection
\label{v2p359} seulement de plus en plus grande, tandis que les sauvages, après avoir reçu des présents, ne tardent pas à en offrir à leur tour afin d’en recevoir d’autres encore. Il y a là un besoin de reconnaissance (plus ou moins intéressée), de \emph{contre-imitation symétrique} pour ainsi dire, qui se fait sentir dans les relations des oiseaux entre eux, des hommes entre eux, mais qui, dans les rapports de l’homme à l’oiseau ou de l’oiseau à l’homme, n’a pas lieu d’apparaître.\par
Quelques jours après, un échange réglé s’établissait entre Vasco et les indigènes. « Le capitan-mor dit à ces gens de se séparer et de venir seulement un ou deux à la fois : le tout s’exécutait par signes. Et à ceux qui venaient, le commandant présentait des grelots, des bonnets écarlates, et eux nous offraient des bracelets d’ivoire qu’ils portaient au bras... »\par
Mais continuons. « Le samedi, arrivèrent environ 200 nègres tant grands que petits ; ils amenaient une douzaine de têtes de bétail, vaches et bœufs, accompagnés de quatre ou cinq moutons ; et, lorsque nous les aperçûmes, nous allâmes à l’instant à terre, et \emph{tout aussitôt ils commencèrent à faire résonner quatre ou cinq flûtes ;} les uns jouaient haut, les autres bas, concertant à merveille pour des nègres. \emph{Ils dansèrent aussi} comme dansent les noirs. Et le capitan-mor ordonna \emph{de sonner des trompettes, et nous dans nos chaloupes nous dansions ; le capitan-mor dansant aussi} après être revenu parmi nous\footnote{ \noindent Mêmes détails dans Cook : « Dès que les habitants (les Hawaïens) s’aperçurent que nous voulions mouiller dans la baie, ils vinrent vers nous ; la foule était immense ; ils témoignèrent leur joie par des chants et des cris, et ils firent toutes sortes de gestes bizarres et extravagants. »
 }. Et, \emph{la fête achevée}, nous fûmes à terre où nous avions déjà débarqué, \emph{et là nous achetâmes un bœuf noir pour trois bracelets}... »\par
On voit là, avec évidence, apparaître la liaison d’idées, qui sera toujours si forte et si générale, entre \emph{fête} et \emph{marché.} Le mot \emph{foire} est la combinaison rurale des deux. Les grands magasins de nos capitales ne sont-ils pas aussi une fête des yeux, une réunion joyeuse de femmes et d’enfants,  \phantomsection
\label{v2p360}autant qu’un lieu d’achats et de ventes ? Préluder à un échange de marchandises par la musique et la danse est une idée tout à fait humaine, et les matelots en sont si peu surpris qu’ils se mettent aussi, spontanément, à sonner de la trompette et à danser.\par
Observons aussi que l’idée de jouer des instruments de musique pour se faire comprendre de gens dont on ne connaît pas la langue a dû venir naturellement à l’esprit.\par
Rapprochons de ces musiques de flûtes nègres annonçant l’intention de vendre et d’acheter, les flûtes de Pan de nos chevriers traversant un village, la corne de nos \emph{pillarots}, les \emph{cris} et les \emph{airs variés}, caractéristiques, des marchands ambulants, des revendeurs dans les rues de Paris (autrefois et même encore), etc.\par
Le lien entre l’idée de marché et l’idée de combat, le peu de distance qu’il y a, pour les primitifs — et même pour les civilisés — entre l’échange des marchandises et l’échange des coups, se montre aussi dans ce qui suit. « Le dimanche, il vint tout autant de monde, et ces gens avaient amené des femmes et des petits enfants ; mais les femmes restaient sur un monticule près de la mer. \emph{Ils amenaient nombre de bœufs et de vaches}. Ils formèrent deux groupes le long de la mer ; \emph{ils jouaient de leurs instruments et ils dansaient}, comme ils avaient fait durant la journée du samedi. \emph{La coutume de ces hommes est que les jeunes gens restent dans le bois avec les armes} (on voit par là qu’ils ont l’habitude de regarder le commerce comme quelque chose de périlleux qui peut facilement dégénérer en guerre), et les plus âgés venaient converser avec nous et portaient de courts bâtons à la main et des queues de renard fixées à une gaule dont ils s’éventaient le visage. Et nous trouvant ainsi en conversation, le tout par signes, nous remarquâmes entre les arbres les jeunes gens accroupis, portant les armes à la main. » Alors Vasco tâte le terrain : il envoie un homme proposer l’échange d’un bœuf contre des bracelets. Les nègres prennent \phantomsection
\label{v2p361} les bracelets, mais, au lieu de lui remettre le bœuf en échange, ils se mettent à se plaindre de ce que les matelots ont pris de l’eau... Pour mettre fin à ce commencement de querelle, le capitan-mor fait une petite démonstration militaire qui met en fuite nègres et bêtes.\par
Ce que j’admire, c’est, malgré tout, à quel point l’homme est naturellement confiant et crédule à l’homme, prompt à cesser de se méfier, alors même qu’on lui donne ou qu’on vient de lui donner les plus légitimes sujets de soupçon. Quand Jacques Cartier aborda au cap Espérance, dans le golfe Saint-Laurent, il commença par recevoir à coups de fusil les indigènes qui accouraient à lui avec un empressement amical dont il eut le tort de se méfier. Mais cela ne les empêcha pas de revenir à la moindre démonstration pacifique. Il leur envoya des couteaux, des objets brillants. « Ce que voyant ils descendirent aussi à terre, portant des peaux, et ils \emph{commencèrent à trafiquer avec nous}, montrant une grande et merveilleuse allégresse d’avoir ces ferrements et autres choses, \emph{dansant toujours} et faisant plusieurs cérémonies. »\par
Ce n’est pas que je veuille nier les cas, assez nombreux, où, sans provocation, les sauvages ont accueilli en ennemis les visiteurs européens. Mais l’accueil cordial est la règle. Remarquons que, pour les sauvages comme pour les enfants, il n’y a guère de milieu entre les témoignages d’amitié expansive et les marques d’hostilité. Dans les récits de voyages, on voit toujours les vaisseaux abordés par des pirogues, soit, d’habitude, pleines de présents offerts par des insulaires chantant et dansant, soit, par exception, remplies de pierres et de cailloux, d’arcs et de flèches. C’est le \emph{milieu} entre l’affection et la guerre, entre le \emph{don} et le \emph{vol}, qui est difficile à tenir pour un primitif, et c’est à s’arrêter dans ce milieu que la civilisation consiste, par le travail et l’échange, comme la culture de l’esprit consiste à s’empêcher d’aller précipitamment de l’affirmation absolue à la négation énergique, \phantomsection
\label{v2p362} ou \emph{vice versa}. Savoir douter, savoir évaluer, calculer, modérer ses élans, c’est le miracle de la civilisation. La volonté est plus une résistance qu’un stimulant.\par
En résumé, il n’est rien dont la psychologie ait plus à s’occuper que de ce qui a trait à la vie économique des peuples sauvages ; et ce que nous apprennent à cet égard les récits des voyageurs s’accorde sur des points importants pour nous avec ce que nous révèle l’observation des enfants groupés et livrés à eux-mêmes dans la liberté de leurs jeux d’écoliers. Ici et là, ce n’est pas de l’égoïsme que nous voyons découler les relations économiques, qui tissent la toile de la solidarité humaine et la prolongent à l’infini ; car l’égoïsme ne saurait qu’isoler l’individu, le blottir en soi, le tenir en garde contre son prochain. Aussi, ce qui nous apparaît clairement, c’est que l’échange provient d’une réciprocité de présents, c’est-à-dire de l’altruisme contagieux et mutuel. Ceci est confirmé par d’autres ordres de recherches. L’hospitalité donnée à l’étranger précède toujours le commerce avec lui, et l’hospitalité réciproque a peut-être été la forme la plus antique de l’échange. A coup sûr, elle a été l’une de ses formes primitives et des plus fécondes. La vie économique, en somme, découle de la vie cordiale, festivale\footnote{ \noindent Peut-être est-ce précisément à ce caractère voluptueux de l’échange primitif qu’il convient d’attribuer le mépris si universel de la profession commerciale parmi les sauvages et les barbares, et longtemps même parmi les civilisés, sauf de rares exceptions. C’est que le commerce extérieur, qui semble avoir précédé, comme profession distincte, le commerce intérieur, a commencé, avant toute division du travail par ne servir qu’à la consommation, et à la consommation d’articles étrangers, propres non à satisfaire des besoins existants (suffisamment satisfaits par les produits indigènes) mais à susciter des besoins nouveaux, jugés superflus, pernicieux, illégitimes. — Le commerce ne s’est élevé en considération qu’à mesure que, par les progrès de la division du travail, il a servi davantage à la production, et à la satisfaction des besoins fondamentaux.
 } et joyeuse, elle est le développement et l’entrelacement des sympathies, de tribu à tribu, d’étranger à étranger, autant, sinon plus, que de celles de parent à parent, de compatriote à compatriote. C’est la mutiler, c’est l’abaisser, c’est méconnaître son origine et sa nature  \phantomsection
\label{v2p363}la plus essentielle, que de la faire consister dans le déchaînement des avidités égoïstes, voire même dans le confortable arrangement, — qui est toujours un équilibre instable — des intérêts bien entendus.
\subsubsection[{III.5.c. Transformations de l’échange à travers les trois formes domestique, urbaine et nationale de l’économie. La vie urbaine et le commerce extérieur.}]{III.5.c. Transformations de l’échange à travers les trois formes domestique, urbaine et nationale de l’économie. La vie urbaine et le commerce extérieur.}
\noindent J’ai pris l’échange à ses plus humbles débuts. Si j’avais la prétention de faire ici un cours d’embryologie économique, j’aurais à montrer maintenant comment, la sympathie humaine agissant toujours, poussant les tribus à s’assimiler toujours davantage, à s’emprunter les unes aux autres leurs besoins et leurs procédés de fabrication et de culture, les échanges de peuplade à peuplade sont devenus assez réguliers pour se localiser dans des endroits déclarés inviolables et privilégiés, propres à nourrir et fortifier la confiance mutuelle ; comment à l’économie domestique des tribus a succédé l’économie déjà politique des cités, puis des États et des fédérations d’États civilisés\footnote{ \noindent Le commerce \emph{intermunicipal} était au moyen âge ce que le commerce international est de nos jours. Il donnait lieu à des traités et à des difficultés analogues.
 } ; comment, enfin, à mesure que le marché s’agrandissait ainsi, les prix devenaient à la fois plus uniformes et moins stables et se conformaient à des lois en même temps plus précises et plus générales. Mais tout cela est assez évident pour que je n’y insiste pas. A ce sujet seulement quelques remarques.\par
Suivant la notion que semble se faire Bücher des transformations économiques\footnote{ \noindent Voir ses \emph{Etudes d’histoire d’économie politique.}
 }, les trois formes successives jusqu’ici connues de l’économie politique, économie \emph{domestique}, économie \emph{urbaine}, économie \emph{nationale} (en attendant l’économie \emph{mondiale}), tout en différant sous bien des rapports, tendraient cependant toutes à constituer un \emph{domaine clos}, un marché de plus en plus étendu, toujours fermé. Mais, à cet égard, il est bon d’observer que le terme moyen ici diffère sensiblement \phantomsection
\label{v2p364} des deux termes extrêmes. Je vois bien que le groupe familial primitif — et aussi bien le groupe féodal du châtelain et de ses vassaux groupés autour de lui — aspire et parvient facilement à \emph{se suffire}, à n’avoir presque que des échanges intérieurs, à ne consommer presque que ce qui est produit dans l’intérieur du groupe. Je vois aussi que cela est encore plus facile à réaliser dans le grand groupe national, à moins que, comme l’Angleterre actuelle, il ne produise pas le blé qu’il lui faut ou toute autre denrée nécessaire. Une nation, en général, est assez vaste et assez ingénieuse pour trouver sur son territoire et dans son génie propre toutes les ressources que réclament les besoins des nationaux. Pour elle l’échange avec le dehors n’est, \emph{normalement}, qu’un objet de luxe, — de luxe nécessaire et essentiellement social, il est vrai.\par
Mais le groupe urbain ne saurait jamais se passer, pour vivre, d’échanges continuels avec les ruraux environnants, dans un rayon très étendu, et dans un rayon qui ne se laisse pas préciser facilement comme les limites de l’enclos familial ou de la ville et aussi bien du territoire national. Et Bücher a bien senti la difficulté qui naît de là. Une ville réduite à ses propres ressources en aliments, en eaux, en matières premières, c’est une ville assiégée qui va être bientôt prise par la faim ou par la soif.\par
Aussi, comparez les proportions des échanges extérieurs et des échanges intérieurs dans une ville et dans une nation quelconque. La première, celle des échanges extérieurs, est toujours beaucoup plus grande dans une ville que dans une nation. Si énorme en apparence que soit notre commerce d’importation et d’exportation, en France, il n’est que peu de chose comparé à l’ensemble, non totalisé, de notre commerce intérieur.\par
Je sais bien que Bücher n’entend par le commerce \emph{extérieur} d’une ville du moyen âge que celui qu’elle fait avec des marchands étrangers à sa banlieue. Mais, d’abord, cette  \phantomsection
\label{v2p365}banlieue est difficile à délimiter, il le reconnaît. Rien ne permettait, à l’entrée en ville, de reconnaître que le vendeur de blé ou de laine ou d’autres denrées et matières premières était domicilié dans le territoire d’approvisionnement habituel de la ville. Puis, en réalité, le paysan de la banlieue même était un forain, un étranger aux yeux de l’artisan. Entre les deux, comme le dit Bücher, un fossé s’était creusé, toujours plus profond. Or, ce rapport d’échange qui s’établissait au moyen âge entre l’artisan, vendeur de ses produits, et le paysan (ou le marchand exotique) vendeur de ses denrées et de ses matières premières, est précisément celui que nous voyons se reproduire, très agrandi, dans l’époque moderne, ère de l’« économie nationale », entre un pays industriel, tel que l’Angleterre, qui vit de l’exportation de ses produits fabriqués, et les pays agricoles, tels que le sud de la Russie ou l’Ouest américain, qui lui fournissent le blé ou la viande. La différence est que la ville a bien plus besoin encore des pays environnants — ou éloignés — pour la nourrir, que l’Angleterre ou la Belgique n’ont besoin des agriculteurs d’outre-mer.\par
Il suit de là que c’est la \emph{vie urbaine} qui a donné le goût, développé le besoin, enraciné l’habitude du commerce extérieur. Par suite, toute nation où la vie urbaine est peu développée a plus de tendance et d’aptitude au protectionnisme qu’une nation très urbanisée. Par le développement de la vie rurale dans un pays, on le rend propre à se claquemurer économiquement dans ses frontières. Par le développement de la vie urbaine on le met dans l’impossibilité de rester longtemps enfermé dans les langes protectionnistes. Donc, favoriser l’urbanisation des peuples en même temps qu’on les pousse à se hérisser de remparts douaniers, c’est une anomalie, un paradoxe social impossible à soutenir jusqu’au bout. Aussi ne faut-il pas s’étonner si les nations où le commerce extérieur a le plus d’importance relative — la Belgique, l’Angleterre, l’Allemagne — sont celles où la proportion de  \phantomsection
\label{v2p366}la population agricole est la moindre et où l’émigration des campagnes vers les villes est le plus rapide. Cet accroissement proportionnel du commerce extérieur et de la production nationale qui l’alimente crée un état d’équilibre des plus instables, mais en même temps établit sur de plus larges fondements la solidarité internationale. Malgré tout, le commerce extérieur, dans les Êtats les plus avancés, reste toujours bien moindre, je le répète, que leur commerce intérieur. « M. Lexis\footnote{ \noindent J’emprunte cette citation à M. Hector Denis.
 } évalue à 20 milliards de marks la production brute annuelle de l’empire d’Allemagne ; l’exportation annuelle est de 3 205 millions de marks ; le rapport entre le commerce intérieur et le commerce extérieur est donc comme 5 est à 1. En France, la production annuelle est de 25 milliards de francs ; l’exportation est de 3 350 millions ; le rapport est comme 6 1/2 à 1. M. Giffero évalue à 1 200 millions de livres sterling la production du Royaume-Uni ; le total des exportations en 1884 était de 298 millions de livres sterling ; le rapport est de 3 à 1. » Ce rapport, en Belgique, descend à celui de 2 1/2 à 1. On voit que, à l’exception de quelques nations condamnées à l’exportation de plus en plus vaste sous peine de mort, la plupart des peuples ont la faculté de se claquemurer dans leurs frontières sans courir le risque de se ruiner ; et c’est une donnée importante du problème de la protection et du libre échange.\par
Mais ce qui s’oppose et s’opposera toujours davantage à ce que les nations usent de la possibilité de s’enfermer chez elles, c’est, outre le développement de la vie urbaine\footnote{ \noindent Elle se développera alors même que, comme nous l’espérons, la population des villes refluera vers les campagnes, qui s’\emph{urbaniseront} de plus en plus, comme déjà elles commencent à le faire.
 } (surtout dans les capitales, si promptes à s’entre-copier et à s’entre-échanger leurs produits de peuple à peuple), le développement de la vie maritime, qui, plus encore que la vie urbaine, est essentiellement assimilatrice des hommes de tout pays. Ce prodigieux entrecroisement de navires sans  \phantomsection
\label{v2p367}nombre, qui est partout l’accompagnement du progrès de la civilisation, qu’est-ce autre chose que la forme civilisée de la vie nomade ? A mesure que diminuent les caravanes à travers des steppes et des déserts, les flottes et les flottilles se multiplient sur les océans et les mers. Non seulement le commerce maritime est l’agent par excellence des assimilations sociales à grandes distances, mais encore il est à noter que les conditions de la vie maritime sont toujours beaucoup plus semblables que les conditions de la vie terrestre. Les distinctions des nationalités tendent, par suite, à s’émousser, sinon à s’effacer, par l’extension de la navigation. Il est certain qu’entre un matelot français et un matelot anglais la différence est bien moindre qu’entre un fermier anglais et un fermier français\footnote{ \noindent Dans ses \emph{Notes sur la Norvège}, M. Hugues Le Roux fait une remarque qui vient à l’appui de ce qui précède. « Si, dit-il, dans la vie des villes, les différences des natures norvégienne et suédoise se sont marquées jusqu’à une accentuation, des deux parts voulue, on peut dire, quand on parle du marin : \emph{le Scandinave}, sans distinguer le montagnard de l’homme de la plaine. »
 }.\par
J’aurais encore une observation à faire pour montrer que, malgré la profondeur des changements sociaux dus à l’élargissement progressif de l’échange, les traits caractéristiques de l’échange primitif, décrits plus haut, persistent longtemps, très longtemps, et ont toujours une tendance à reparaître, parce qu’ils expriment le fond éternel du cœur, la sympathie de l’homme pour l’homme. Ce mélange intime des affaires et des plaisirs qui nous a frappés dans les trafics des sauvages, soit entre eux, soit avec les navigateurs européens, est loin d’être une confusion d’éléments hétérogènes que le progrès aurait pour effet de dissocier entièrement. On le retrouve dans les \emph{foires} du moyen âge et des temps modernes, et jusque dans nos expositions universelles, ces foires immenses où il réapparaît extrêmement agrandi. A la foire Saint-Germain, sous l’ancien régime, les achats servaient d’occasion à toutes sortes de jeux et de divertissements. On y venait pour voir et marchander ; mais « on y  \phantomsection
\label{v2p368}venait autant pour jouer, pour se battre, pour chercher des bonnes fortunes. Les chambres situées au-dessus des loges servaient surtout à cela. Au rez-de-chaussée, les affaires ; au premier, les rendez-vous galants et le tripot\footnote{ \noindent Fagniez, \emph{l’Économie sociale sous Henri IV.}
 }. » Les foires, — toujours si joyeuses que \emph{Kermesse} dérive de là — sont une des manifestations les plus vivantes et les plus populaires de l’adaptation économique, comme les kracks sont le phénomène le plus éclatant de l’opposition économique, ou les batailles celui de l’opposition politique. Elles sont, si l’on veut, des foules, quoiqu’il manque l’unité du but aux individus qui les composent ; mais elles sont plutôt des villes fiévreuses et passagères, des villes féeriques et périodiques. La vie urbaine s’élève là à son plus haut point d’intensité et d’éclat, et s’y éteint brusquement. Si l’on cherche leurs origines, on s’aperçoit qu’elles sont nées, comme la plupart des villes, à l’occasion d’un rassemblement de nature religieuse ou esthétique — pèlerinages, frairies, jeux solennels — encore plus que commerciale\footnote{ \noindent Quelquefois, c’est parce que certains lieux déterminés, au moyen âge, ont été affranchis de tous droits d’achat et de vente, partout ailleurs vexatoires, qu’on y est accouru avec empressement. Mais cette idée d’affranchir de tous droits un lieu spécial, de créer des \emph{lieux d’asile} commerciaux pour ainsi dire, n’a pu venir qu’après que le besoin de ces rassemblements s’était spontanément manifesté depuis des siècles.
 }. Mais, au fond de tous ces rassemblements, ce qui se révèle surtout, c’est l’attrait profond que la foule exerce sur l’individu, et encore plus peut-être sur le citadin constamment coudoyé par ses semblables dans les rues des grandes villes que sur le paysan isolé dans son existence monotone.
\subsubsection[{III.5.d. Le libre échange de nation à nation.}]{III.5.d. Le libre échange de nation à nation.}
\noindent Mais revenons au commerce extérieur, qui nous conduit de nouveau, mais sous un autre point de vue, à la grave question du libre-échange. Jusqu’à une époque assez voisine de nous, quand un propriétaire avait sur son grand  \phantomsection
\label{v2p369}bien à peu près tout ce qu’il lui fallait, on trouvait naturel qu’il se proposât pour but de compléter ce caractère d’indépendance en ajoutant à ses cultures diverses les quelques-unes qui lui manquaient. Cet idéal archaïque, de moins en moins justifié aujourd’hui, était fréquent et raisonnable autrefois. On peut dire que, après avoir été individuel et national à la fois, il tend à devenir simplement national, à se réfugier dans l’État. Pourquoi ? Parce que les raisons qui poussaient raisonnablement les propriétaires féodaux à désirer l’indépendance économique dont il s’agit, existent encore, quoique s’affaiblissant chaque jour, pour les États. Remarquons, en effet, que, pour un État, les échanges intérieurs, de nationaux à nationaux, peuvent constituer et constituent le plus souvent une collaboration à une œuvre commune de puissance et d’influence collectives, en même temps qu’une mutualité d’assistance. C’est donc un système tout autrement cohérent et vigoureux que les échanges extérieurs, les achats et ventes avec l’étranger, lesquels ne sont jamais qu’une assistance réciproque sans nulle collaboration. Donc, en retenant dans son sein, par la protection douanière, des industries qui tendent à s’en échapper, et, à fortiori, en faisant germer dans son sein, grâce au même procédé, des industries qui n’y existent pas encore, par suite en diversifiant et multipliant de la sorte ses échanges intérieurs aux dépens de ses échanges extérieurs simplifiés et raréfiés, un État fait œuvre systématique au plus haut point. Il rend plus forte la cohésion des citoyens et n’affaiblit que leurs liens avec le dehors. Considération qui a son importance.\par
Quand de nouveaux produits sont importés dans un pays qui, jusque-là, les ignorait, ils y font naître un \emph{espoir nouveau} du consommateur, espoir qui éveille et répand le \emph{besoin} correspondant. Maintenant, si après avoir été fabriqués à l’étranger, les produits en question le sont sur le sol national, cette importation d’une nouvelle industrie crée un espoir et un besoin nouveaux de \emph{production}. On  \phantomsection
\label{v2p370}oublie généralement ce second côté de la question, qui n’est pas le moins important. En y ayant égard, on apprécie la reconnaissance que nous devons à Colbert, par exemple, pour avoir introduit l’industrie des soieries à Lyon et celle des glaces à Saint-Gobain.\par
M. Lexis, professeur à Gœttingen, termine ainsi son intéressante étude sur l’\emph{Historique du protectionnisme} (\emph{Revue d’Écon. pol.}, janvier 1896) : « Si l’on embrasse d’un coup d’œil d’ensemble dans le dernier quart de siècle, le développement qu’a pris la politique commerciale en général, on est frappé de voir que le système libre-échangiste ne s’est, à l’exception de l’Angleterre, nulle part établi, malgré de sérieuses tentatives de la part de plusieurs pays, et qu’au contraire les intérêts protectionnistes ont toujours reparu avec plus de force et fini par remporter la victoire. Cela prouve incontestablement que la grande majorité des industriels et des propriétaires fonciers trouvent dans ces pays de plus grands avantages sous le régime protectionniste que sous la liberté commerciale. Car on ne peut admettre que ces classes aient pendant des siècles méconnu leurs propres intérêts. \emph{L’intérêt du capital commercial tend, certes, dans une direction opposée}, mais ce n’est qu’en Angleterre que le commerce est parvenu à une assez grande prospérité pour exercer une action décisive en faveur du libre échange. Il n’y trouve pas de résistance de la part de l’industriel qui n’a guère de concurrence à craindre... Quant à l’aristocratie foncière, elle n’ose plus faire opposition à un régime dont la grande masse des consommateurs sait si bien apprécier les bienfaits. Mais on peut se demander si la classe ouvrière a le même intérêt au libre échange dans les pays dont l’industrie n’est pas encore en état de soutenir la concurrence étrangère, de sorte que la liberté commerciale pourrait y amener la ruine des entreprises existantes et jeter un grand nombre d’ouvriers dans la misère. »\par
Il ne faut pas s’étonner de ce retour à l’exclusivisme économique, \phantomsection
\label{v2p371} qui n’est qu’une des formes du nationalisme aigu dont nous traversons la dernière crise. Mais, pour juger ce phénomène, il convient de le comparer au \emph{protectionnisme provincial} qui, sous l’ancien régime, invoquait aussi des raisons non méprisables. Quand, jadis, les provinces où le blé abondait traitaient d’accapareurs, sinistre injure, les acheteurs en gros qui cherchaient à transporter une partie des grains dans les provinces où régnait la disette, la résistance à ces achats et à ces transports était inspirée non par l’ignorance et la stupidité, mais par un égoïsme collectif ; et, si elle était contraire à l’intérêt général du royaume, elle était conforme à l’intérêt local et momentané des régions favorisées par de bonnes récoltes. Sans le moindre doute, lorsque, à Angoulême, le froment se vendait 17 livres pendant qu’il se vendait 45 livres à Paris — je puise l’exemple dans Turgot — les achats des marchands transporteurs avaient pour effet de faire hausser le prix à Angoulême ; et les consommateurs angoumoisins, — ces consommateurs toujours si chers aux économistes — ne s’abusaient pas en jugeant qu’il était de leur intérêt actuel de s’opposer à cette augmentation de prix. Il est donc infiniment probable que, si le royaume de France était resté décentralisé, morcelé en provinces autonomes, jamais la liberté interprovinciale du commerce des grains, pour le plus grand bien de tous les Français, ne serait parvenue à s’y établir d’une façon durable. Il a fallu arriver à des époques de centralisation et même de pouvoir personnel, sous Napoléon III, pour voir supprimer les dernières entraves aux libres transactions des marchands de blé, des « accapareurs » si honnis, destructeurs de la disette et de la famine.\par
Il se trouve que, dans les provinces riches en céréales, l’intérêt des producteurs de blé, qui voulaient vendre aux provinces en proie à la disette, était d’accord avec l’intérêt général. Au contraire, dans les provinces pauvres en grains, l’intérêt des agriculteurs de ces régions, qui désiraient  \phantomsection
\label{v2p372}vendre aux plus hauts prix, était de repousser les négociants du dehors. Mais, sans se soucier s’il favorisait ou contrariait l’intérêt des producteurs ici ou là, l’intérêt des consommateurs là ou ici, l’État a dû abaisser les barrières de douanes entre les provinces, parce qu’avant tout son but était d’éviter le retour des crises alimentaires, avec le cortège de maux funèbres qui s’ensuivent.\par
A présent, ce n’est plus du libre commerce intérieur des céréales et des autres denrées qu’il s’agit, mais c’est de leur libre commerce extérieur, international. Et, pour des raisons différentes, mais encore plus fortes, l’État doit s’élever au-dessus de la question de savoir s’il favorise ici ou là la classe des consommateurs ou celle des producteurs. Doit-il veiller, pour le salut général de la nation, à ce que la classe des producteurs de blé ne disparaisse pas ou ne s’amoindrisse pas au delà d’une certaine mesure, réclamée soit par la possibilité d’un blocus alimentaire en cas de guerre générale, soit par les exigences du service militaire qui, en tout temps, requiert une forte proportion de population rurale, endurante et robuste ? Voilà toute la question en ce qu’elle a d’essentiel.\par
En ce qui concerne la plupart des autres produits, moins fondamentaux, il n’est pas douteux que, malgré la recrudescence du nationalisme mal entendu, les barrières douanières entre les nations \emph{d’un même continent}, sinon \emph{d’un même type de civilisation}, doivent aller s’abaissant. C’est une grande étroitesse et myopie d’esprit que de protester, au nom du patriotisme, contre toute entreprise, toute institution, toute loi, qui présente un caractère international. C’est ne pas voir que les États civilisés d’Europe sont à présent dans la situation où étaient les États allemands au moment du \emph{Zollverein.} Toutes les formes de l’internationalisme, même les plus détestables, à commencer par les tentatives gigantesques d’accaparement \emph{mondial}, témoignent de ce besoin immense de solidarité qui grandit parmi les  \phantomsection
\label{v2p373}nations européennes. Et, si ce besoin tarde tant à se réaliser, c’est qu’en réalité il y a ici antinomie entre deux intérêts antagonistes, qui vont s’accentuant et se combattant de plus en plus. Autant il est avantageux, en effet, pour un État fort de n’avoir pour voisins que des États faibles, autant il est désavantageux pour une nation riche de n’avoir pour voisines que des nations pauvres. La richesse est intéressée à se propager dans ses alentours, tandis que la puissance militaire ou politique est intéressée à ne pas se propager autour d’elle. Par suite, la prédominance des préoccupations d’ordre économique est favorable à l’extension de l’\emph{altruisme international}, pour ainsi parler, tandis que le point de vue politique, quand il devient obsédant, tend à renforcer l’égoïsme national.\par
Ce n’est pas à dire que le jour soit prochain, si jamais il arrive, où toutes les entraves au commerce de nation à nation seront supprimées. Il y aura toujours des traités de commerce et des jeux de tarif qui, comme les incidences des impôts, produiront souvent des effets inattendus, favorables ou défavorables, sans qu’il y ait lieu, pour cela, de renoncer à prévoir et à calculer en pareille matière, comme si l’absence de règlements quelconques était ici la règle suprême, et l’anarchie le salut. De mieux en mieux, les résultats d’un tarif peuvent être prévus, par suite de l’abondance et de l’exactitude croissantes des informations ; et aussi les résultats de la suppression de tout tarif. Le libre échange porte des fruits différents, en tel lieu ou en tel autre, à telle époque ou à telle autre. Ses bienfaits réels sont dus non pas à la liberté précisément, c’est-à-dire à la non-réglementation, — encore moins à la concurrence et à la lutte, car il peut y avoir libre échange sans libre concurrence, le monopole n’empêchant pas la liberté des achats et des ventes — mais bien à l’organisation spontanée du travail, qui s’opère par lui. Je dis spontanée, et non pas inconsciente ; je dirais plutôt \emph{multi-consciente.} C’est, en effet, consciemment, que  \phantomsection
\label{v2p374}chacun des coéchangistes noue un contrat d’où il attend un avantage qu’il prévoit égal ou supérieur ou inférieur à celui de son co-échangiste. Et l’harmonie totale résultant de tous les échanges qui ont lieu à la fois sur un même marché est composée de toutes ces harmonies partielles, séparément conçues et voulues par les contractants. Mais cette harmonie totale n’est composée que d’harmonies individuelles, d’avantages privés, voilà son défaut. Il s’agit de savoir si cette somme d’avantages particuliers, cette satisfaction donnée à une foule de petits désirs individuels, s’accorde ou non avec le but général du pays, c’est-à-dire avec les volontés patriotiquement convergentes de ces mêmes individus, avec le système national de leurs activités politiques et sociales. Or, fréquemment, entre cette somme d’harmonies individuelles et ce système national, qui est une somme d’unissons individuels pour ainsi dire, il y a désaccord. Quand il y a accord, on s’en aperçoit à la prospérité politique autant qu’économique de la nation qui en bénéficie : exemple, l’Angleterre. Quand il y a désaccord, on en est averti par des symptômes tout contraires. En tout cas, les hommes d’État doivent constamment s’inquiéter de cette question.\par
Ce qui importe, avant tout, ne l’oublions pas, ce n’est pas le libre échange industriel et commercial, c’est le libre échange scientifique, artistique, moral, lequel peut fort bien se concilier avec le protectionnisme économique. Il faut se garder de confondre avec le protectionnisme économique des nations modernes qui se hérissent de récifs douaniers pendant que, par les mille brèches de leurs frontières, circulent librement les trains de voyageurs, les dépêches télégraphiques, les journaux, tous les courants de la pensée, le protectionnisme économique des peuples barbares ou demi-civilisés, essentiellement fondé sur leur protectionnisme moral et social, qui a aussi ses douanes : j’entends par là ce systématique et sincère dénigrement de l’étranger, ce parti pris de ne rien admirer d’exotique, qui est si remarquable \phantomsection
\label{v2p375} chez certains peuples primitifs et qui d’ailleurs n’empêche pas des crises fréquentes d’engouement pour les choses du dehors. Quelle force a dû avoir, chez les Hébreux, par exemple, cette insensibilité voulue aux exemples étrangers, cette cécité systématique pour l’éclat étranger, pendant leur longue captivité en Égypte et à Babylone ! Ne faut-il pas aussi qu’elle soit énergique au plus haut degré chez les Arabes d’Algérie et chez les derniers Peaux-Rouges des États-Unis ! En analysant ce qui se passe dans ces sociétés closes, on voit qu’entre le fait de l’imitation des parents et le jugement porté sur leur supériorité il y a causalité réciproque. L’enfant, le jeune citoyen, a commencé par imiter ses père et mère, ses anciens, parce qu’il a senti leur force supérieure à la sienne, leur savoir supérieur au sien ; puis il a jugé ses parents et ses concitoyens, sa famille et sa cité, supérieurs à tout le reste de la terre, parce qu’il a pris l’habitude de les imiter. Or, quand, pour la première fois, par une fissure au rempart d’airain de l’orgueil barbare, le flot des exemples extérieurs commence à pénétrer, il se produit un rajeunissement subit, une fécondation qui rappelle l’élargissement soudain de l’esprit évadé du préjugé dogmatique ou le déchirement de l’égoïsme par l’amour. C’est ce bénéfice de la civilisation que les peuples modernes ne veulent pas perdre. Aussi n’est-il en rien menacé par les guerres de tarif douanier qu’ils se livrent de temps en temps et qui n’entravent nullement l’assimilation sociale, la communion spirituelle vers laquelle ils tendent. Quand même ils n’échangeraient plus aucun produit, ils ne laisseraient pas d’échanger encore toutes leurs inventions productrices, et, tout en restant même chez eux, de s’entre-visiter idéalement : libre échange social qui, du reste, doit tôt ou tard entraîner à sa suite, finalement, le libre échange économique.
 \phantomsection
\label{v2p376}\subsubsection[{III.5.e. Rôle et place de l’échange parmi les autres formes de la répartition et de l’adaptation des richesses.}]{III.5.e. Rôle et place de l’échange parmi les autres formes de la répartition et de l’adaptation des richesses.}
\noindent Cela dit sur la question du libre échange ou plutôt de l’échange de nation à nation, demandons-nous ce qu’est l’échange en général, quelle est sa vraie place parmi tous les modes de répartition et d’adaptation de la richesse, et s’il y a une formule générale de la vie économique à cet égard, analogue ou non à la formule générale de la vie politique ou de la vie juridique.\par
Nous savons que l’échange suppose la propriété individuelle. N’est échangeable que ce qui est appropriable. Or, les choses appropriées ne peuvent progresser dans la voie de leur adaptation aux désirs humains qu’en changeant de main. Mais l’échange, nous le savons aussi, n’est pas le seul mode de déplacement des propriétés. Les richesses une fois produites, celles des morts se répartissent entre les vivants en vertu des lois de succession ou des testaments, ce qui ne présente aucun des caractères de l’échange. Celles des vivants se répartissent entre eux : 1\textsuperscript{o} par la conquête ou le vol ; 2\textsuperscript{o} par la donation, toujours plus ou moins volontaire, y compris le don obligatoire, l’impôt ; 3\textsuperscript{o} par l’échange, y compris la vente, le louage, le prêt, tous les contrats onéreux. Dans une société en progrès ces trois catégories de distribution des richesses, les deux premières aussi bien que la troisième, se développent ou se transforment, et il n’est pas sûr que la troisième soit toujours ou doive toujours être celle qui est appelée au plus grand développement. En ce qui concerne la première, qu’on songe, non seulement aux vols proprement dits dont le nombre a triplé en France depuis cinquante ans, mais surtout aux trusts, aux monopoles abusifs, aux extorsions des compagnies puissantes ou des associations tyranniques, aux tricheries des jeux de Bourse. La deuxième augmente aussi partout où la population s’accroît, la vie d’un père de famille  \phantomsection
\label{v2p377}n’étant qu’une série continuelle de donations faites à ses enfants.\par
Les économistes, quand ils parlent de la distribution des richesses après avoir traité de leur production, semblent se représenter une masse énorme et unique de richesses qui, venant d’être produite à l’état indivis, demande à être distribuée et répartie entre les consommateurs. Ce point de vue implicite et persistant est évidemment faux ; car la richesse naît distribuée et répartie toujours d’une certaine façon, souvent elle se consomme sur place là où elle a été produite, et, quand il en est autrement, l’opération intermédiaire entre sa production et sa consommation doit s’appeler une \emph{redistribution.} Il en est ainsi même dans le cas où le trésor public répartit entre les fonctionnaires et employés de l’État à tous les degrés les sommes dont il dispose, et qui sont un grand bassin formé par les millions de petites sources coulant de la bourse des contribuables. De même, une compagnie de chemins de fer répartit entre ses employés, ses obligataires et ses actionnaires, les sommes qu’elle recueille aux guichets des gares et qui proviennent de la poche des voyageurs.\par
Dans ces deux cas — qui rentrent dans ce que Courcelle-Seneuil appelle la distribution des richesses par l’autorité — il y a eu d’abord production-distribution des richesses nées sous forme multiple et divisée, puis concentration de ces richesses, enfin \emph{redistribution} de ces richesses.\par
Ici, comme dans l’évolution\footnote{ \noindent Je me permets de renvoyer à mes \emph{Transformations du Pouvoir}.
 } du pouvoir, on passe d’une division de la richesse relativement fortuite et incohérente, à travers une concentration transitoire, à une division relativement harmonieuse et rationnelle de la richesse ; — rationnelle en ce sens que la quantité des appointements est censée se proportionner toujours, et se proportionne souvent, à l’importance des services rendus.\par
La différence avec le cas où l’autorité n’intervient pas —  \phantomsection
\label{v2p378}au moins directement et explicitement — c’est que : ou bien la richesse se consomme là où elle est née, en sorte que sa division native reste sa division finale\footnote{ \noindent Parce que sa division native se trouve être rationnelle (car le paysan qui produit précisément le blé, le chanvre, le bois, nécessaires à son alimentation, à ses vêtements, à sa maison, réalise ainsi, immédiatement, l’adaptation du produit au besoin).
 }, ou bien sa division native se transforme, par l’échange \emph{privé}, par le commerce, sans nulle concentration préalable, en une autre forme de division beaucoup plus rationnelle, puisque, en s’échangeant, les produits se déplacent dans le sens de leur plus grande utilité.\par
Chacun de nous, à vrai dire, — et non pas seulement l’État ou les grandes compagnies — est à la fois centralisateur et distributeur de richesses. Les sources de notre petite bourse sont multiples et variées comme le sont nos dépenses. Et, si l’on compare, dans un budget de père de famille quelconque — ou même de célibataire — la diversité des sources du revenu à la diversité des dépenses, on s’apercevra facilement que la première est bien plus incohérente et fortuite, bien moins justifiable et explicable que la seconde. Il y a une véritable hiérarchie systématique de nos dépenses, correspondante à celle de nos besoins organiques ou sociaux ; de là, dans tous les budgets de famille appartenant à la même classe, la même proportion à peu près des divers articles de dépenses en nombre à peu près égal : on en a pu faire un tableau générique. Au contraire, la proportion des diverses sources de revenu est extrêmement variable et leur nombre très inégal.\par
Comme les sources de la richesse, les sources du pouvoir sont multiples et diverses, et, pas plus pour le pouvoir que pour la richesse, il n’est permis de séparer la production de la répartition comme si celle-là précédait celle-ci. Chacun de nous, père de famille, patron, mari, — et non pas seulement chaque fonctionnaire, chaque homme d’État — a son pouvoir, comme chacun de nous a sa bourse. Et  \phantomsection
\label{v2p379}le pouvoir de chacun de nous comme sa bourse, se forme, se grossit incessamment ou se renouvelle, par des apports variés. Le pouvoir centralisé, par exemple, sur la tête d’un écrivain marié et père de famille, provient de ses enfants, de sa femme, de ses \emph{lecteurs} qui, au fur et à mesure qu’ils se multiplient et qu’ils le prennent plus au sérieux, augmentent son influence spirituelle, etc. Un député possède une source de pouvoir qui n’est que la concentration sur sa tête de tous les votes de ses électeurs, c’est-à-dire de tous les espoirs que ses électeurs fondent sur lui, de tous les vœux qu’ils déposent en lui. Un président de la République \emph{plébiscité}, comme Louis-Napoléon en 1851, concentre dans ses mains tous les pouvoirs de souveraineté politique disséminés, avant son élection, entre ses 7 ou 8 millions d’électeurs ; puis il les dissémine sous une autre forme entre tous les fonctionnaires qu’il emploie à ses fins.\par
Il en est du pouvoir momentané et concentré sur la tête de chaque citoyen, homme privé ou homme public, comme de son budget, domestique ou national. C’est un tuyau d’arrosage où l’eau entre par mille fissures éparses et sans lien et d’où elle sort par mille petits trous régulièrement percés, en pluie féconde et bien répartie.\par
Mais, dans la \emph{redistribution} du pouvoir, qu’est-ce qui est analogue au rôle joué par l’\emph{échange ?} ou n’y a-t-il rien de pareil ?\par
Pour le pouvoir comme pour la richesse, mettons à part les \emph{cas où il est consommé là même où il est produit :} c’est le cas du paysan qui se nourrit de ses produits, c’est le cas du père de famille sauvage qui, tenant son pouvoir de ses enfants (car leur obéissance et leur crédulité spontanées sont les sources productives de son pouvoir) l’exerce, le consomme, en se faisant obéir et croire par ses enfants mêmes. — Mais, pour peu que le pouvoir soit étendu et s’élève au-dessus du pouvoir domestique, il ne peut s’exercer qu’en se déléguant, au moins en partie. Et alors il convient de  \phantomsection
\label{v2p380}distinguer, d’une part, les sources du pouvoir (obéissance spontanée et traditionnelle des millions de sujets au monarque héréditaire, — ou élection des présidents — ou des députés), d’autre part les canaux d’écoulement du pouvoir (c’est-à-dire des délégués, fonctionnaires royaux ou républicains, n’importe). Or, quand il y a délégation du pouvoir, répartition systématique et délibérée du pouvoir, n’y a-t-il pas un véritable échange ? Après réflexion, on s’apercevra qu’il n’en est rien, et que l’échange est un fait étranger à la vie politique.\par
Il est très rare que deux dépositaires de pouvoirs distincts échangent leurs pouvoirs l’un contre l’autre : or, c’est le seul cas où on puisse voir l’équivalent du troc. Et jamais ce pseudo-échange ne s’accomplit que grâce au bon vouloir de l’autorité supérieure, de telle sorte que, même dans ce cas, il y a \emph{délégation} implicite de cette autorité. \emph{Délégation}, qu’est-ce que cela veut dire ? Cela veut dire \emph{donation} et non échange. Sans doute une donation est souvent faite sous condition, mais jamais on n’a eu l’idée de la considérer à cause de cela comme un échange. Un legs conditionnel est-il un échange ? non. D’ailleurs, combien de fois la nomination à un emploi officiel qui donne un grand pouvoir, la nomination d’un général élevé au rang de commandant de corps d’armée, ou d’avocat général élevé au rang de procureur général, est-elle une donation pure et simple, sans nulle renonciation aux pouvoirs que le titulaire possédait auparavant et qui ne font que s’accroître !\par
Il y a échange de \emph{droits} comme de \emph{richesses.} Le contrat de vente ou de fermage, ou toute autre convention, n’est qu’un échange de droits. Tout échange de richesse, au fond, implique un échange de droits. Et c’est par des échanges de droits qu’on exerce ses droits, comme c’est par des échanges de richesses qu’on utilise ses richesses, sauf le cas où on les consomme immédiatement, analogue au cas où l’on use directement de son droit de propriétaire ou de  \phantomsection
\label{v2p381}fermier en labourant sa terre, en taillant sa vigne. Mais ce n’est jamais par des échanges de pouvoir que l’on exerce son pouvoir. Par exemple, quand le capitaine donne un ordre et que le soldat obéit, où est l’échange de pouvoirs ici ?\par
Si l’on cherche en quoi l’exercice du pouvoir diffère de l’exercice du droit ou de la dépense de la richesse, on verra qu’au fond, dans le cas ou le chef militaire ou civil commande et où le subordonné obéit, le subordonné et le chef poursuivent ou sont censés poursuivre le \emph{même but}, patriotique ou social. C’est en vue de ce but \emph{commun aux deux} que le commandement de l’un et l’obéissance de l’autre peuvent être dits co-adaptés. Mais, quand on passe un contrat de vente avec quelqu’un, \emph{chacun des contractants a son but distinct}, et il y a deux adaptations différentes entrelacées dans le même acte complet : l’adaptation de l’achat fait par B à la fin de A vendeur ; et l’adaptation de la vente par A à la fin de B acheteur.\par
En somme, la \emph{vie économique} — et \emph{aussi bien juridique} — est une simple \emph{assistance mutuelle}, tandis que la \emph{vie politique} est une \emph{collaboration}. En cela celle-ci est supérieure aux deux autres (qui n’en font qu’une). Mais la loi qui régit leur fonctionnement est la même, nous l’avons vu. \emph{Systole} et \emph{diastole}, appel et refoulement, concentration et distribution, voilà les deux termes successifs et alternatifs de l’opposition rythmique qui, continuellement, fait la vie et l’adaptation économique, aussi bien que politique. On appelle à soi, de divers côtés, une certaine source de pouvoirs, de droits, de richesses\footnote{ \noindent Remarquons, en passant, le caractère éminemment individualiste de l’idée de \emph{droit}, si on la compare aux deux autres. On conçoit le droit comme existant chez un individu par la conviction même qu’il a de le posséder, alors même que les autres hommes, ses concitoyens, refuseraient de le reconnaître. En tout cas, on ne juge pas que la reconnaissance d’un droit par les autres hommes le constitue \emph{essentiellement}. Au contraire, on juge que le pouvoir d’un homme consiste essentiellement dans la disposition des autres hommes à lui obéir, et que la richesse consiste essentiellement dans la valeur que d’autres hommes attribuent à ses biens.
 }, qu’on centralise momentanément \phantomsection
\label{v2p382} et qui, par la dépense ou l’exercice qu’on en fait, se ramifie harmonieusement.\par
Dans une certaine mesure, par suite du socialisme d’État et du grossissement des budgets d’État, l’on est en train de passer d’une répartition irrégulière et libre des richesses à leur répartition réglée et systématique moyennant leur concentration momentanée dans le trésor public. Cette transformation, louable ou non, tend certainement à s’accentuer encore ; mais il n’est pas nécessaire de supposer, pour qu’elle atteigne son apogée, que l’idéal du collectivisme se réalise. Il suffit d’admettre que, à côté de l’administration publique, grandissent aussi, et même plus vite encore, des administrations privées, telles que nos compagnies de chemins de fer ou d’assurances, nos grandes usines, nos syndicats, en sorte qu’un moment viendrait où il n’y aurait plus que des fonctionnaires, soit officiels, soit pseudo-officiels, les ouvriers eux-mêmes étant investis de fonctions jugées sociales et à appointements fixés par des autorités reconnues de tous. Le travailleur libre, débattant librement ses prix, deviendrait alors un oiseau rare, une curiosité phénoménale, comme le propriétaire d’alleu en pleine féodalité.\par
En terminant, n’oublions pas de remarquer que l’Échange est un simple moyen terme, — à la vérité très prolongé — dans l’évolution économique envisagée de ses phases initiales à ses phases finales et conjecturées. Il tend à se rendre inutile et fait chaque jour un pas vers sa propre destruction, vers un état futur où prédominera la donation, d’où il est éclos. Le progrès de l’adaptation économique, en effet, consiste à faire travailler de plus en plus les \emph{morts} et les \emph{forces physiques} au profit des vivants, pour dispenser finalement ceux-ci de travailler les uns pour les autre. Si les idées des morts et les forces physiques captées grâce à elles, suffisaient à satisfaire tous les besoins des vivants, devenus les directeurs, à peine occupés, de ces forces, de l’emploi de ces idées, tous les produits seraient gratuits,  \phantomsection
\label{v2p383}et il n’y aurait plus d’échange. La proportion des biens gratuits va donc en grandissant à chaque progrès de la civilisation. Mais si la gratuité universalisée — idéal bien lointain — vient jamais à supprimer l’échange, cela ne veut pas dire qu’elle supprimera l’association — dont nous allons parler maintenant. Loin de là, c’est seulement grâce à l’association universalisée elle-même qu’elle pourrait se réaliser.
 \phantomsection
\label{v2p384}\subsection[{III.6. L’association}]{III.6. L’association}\phantomsection
\label{l3ch6}
\subsubsection[{III.6.a. Formes primitives de l’association, antérieures à l’échange et à la division du travail. « Chambres à filer » allemandes. Trois degrés de la division du travail.}]{III.6.a. Formes primitives de l’association, antérieures à l’échange et à la division du travail. « Chambres à filer » allemandes. Trois degrés de la division du travail.}
\noindent Toute solidarité sociale, et spécialement économique, si vaste que soit devenu son domaine à la longue, a son origine première dans un groupement corporel de personnes, dans une foule réunie par un but commun, d’abord individuel. L’association, sous cette forme primitive, est bien plus ancienne que l’échange et la division du travail. On en découvre facilement les débris survivants dans les pays dits arriérés, dans les campagnes, partout où les hommes se rapprochent pour travailler en commun à une besogne très simple, la même pour tous, faucher, faner le foin, moissonner, égrener du maïs, etc. On ne peut pas dire qu’ils \emph{coopèrent} puisque le résultat de l’effort de chacun est, en apparence, indépendant de celui des autres ; mais, en fait, chacun est stimulé par la vue de l’effort d’autrui, et cette mutuelle stimulation, qui rend leur travail rassemblé plus fructueux que leur travail dispersé, est précisément le but qu’ils poursuivent ensemble. Ce but, à présent, semble naître spontanément, par une sorte d’instinct du groupement sympathique, là où ces réunions coutumières ont lieu ; mais, si l’on remonte à leur source probable, on aboutit toujours à un rassemblement imposé ou maintenu par l’autorité d’un homme, \emph{pater familias}, chef de clan, seigneur féodal. Bücher consacre des pages charmantes à décrire les restes de ces antiques communautés de travail. On rencontre partout, dit-il, chez les peuples à l’état de nature, des \emph{maisons de société publique.} « Elles diffèrent suivant les  \phantomsection
\label{v2p385}sexes ; la plus grande partie sert aux hommes non mariés et aux filles. On ne s’y réunit pas seulement pour travailler en commun ; souvent on y dort, et toujours on y danse, on y joue, on y chante, on y rit, on y jase... Les \emph{chambres à filer} de nos jeunes campagnardes nous offrent un petit mode analogue de communauté de travail. Chaque contrée de l’Allemagne avait ses chambres à filer dont les règlements fixes se transmettaient par tradition. Parfois on organisait un concours, mais en tout temps l’émulation était très vive. Dans la principauté de Nassau, on se sert d’un petit morceau de charbon pour dessiner une moustache à la fileuse endormie ; si elle laisse échapper sa quenouille et se dérouler le fil, un gars peut la lui enlever et elle doit la racheter par un baiser\footnote{ \noindent Il ajoute : « La chambre à filer a disparu devant les transformations techniques de l’époque moderne, mais partout encore, dans les campagnes, durant les longues soirées d’hiver, les jeunes filles se réunissent dans une maison amie ».
 }... »\par
Par des degrés insensibles on passe de cette accélération de travaux semblables par leur émulation réciproque à la convergence d’efforts semblables, puis d’efforts différents, vers une même œuvre qu’aucun d’eux ne pourrait accomplir isolément. — D’\emph{efforts semblables}, et, en général, rythmiques, exécutés à l’aide d’un chant ou d’un instrument de musique, quand d’antiques esclaves de l’Égypte se réunissent pour hisser un obélisque, ou des bateliers modernes pour pousser à l’eau un bateau réparé. Les marches et autres opérations militaires sont l’expression la plus complète et la plus grandiose de ce mode de convergence des effets. — D’\emph{efforts différents}, quand, par exemple, le forgeron et celui qui tire le soufflet, le cordier et le tourneur de roue, le rameur et le pilote, etc., collaborent ensemble. Ces groupes de collaborateurs différenciés sont, dans leur ensemble, beaucoup plus étroits que les groupes de coopérateurs similaires. Avec eux nous entrons dans la \emph{division du travail}, qui, utilisant la diversité naturelle des aptitudes,  \phantomsection
\label{v2p386}et la développant, crée une association idéale, sans nul rapprochement corporel, entre les individus des divers métiers, compris d’abord dans les limites du village, puis dans celles du territoire urbain et enfin national. A mesure que la solidarité économique s’élargit ainsi, son cercle excède davantage celui des rassemblements de travailleurs. Dans le village d’autrefois, dans le fief, il y avait des jours où presque tous ceux qui étaient économiquement solidaires étaient rassemblés par une communauté de travail. Maintenant, les plus grandes foules rassemblées dans des ateliers ne sont jamais qu’une fraction très minime des populations serrées dans les liens d’invisibles associations à distance.\par
Il y a trois degrés de la division du travail : 1\textsuperscript{o} ce qu’un seul ouvrier accomplissait se répartit entre plusieurs ouvriers du même atelier, qui exécutent chacun des fragments différents d’une même œuvre ; 2\textsuperscript{o} ce qu’exécutait un seul atelier, domestique ou autre, se répartit entre plusieurs ateliers, les uns, par exemple, accomplissant tous les travaux de filature, les autres tous les travaux de tissage, mais les uns et les autres faisant partie du même État (tribu, cité, nation) ; 3\textsuperscript{o} ce qui s’accomplissait dans une même nation se divise en plusieurs nations, les unes monopolisant l’extraction du fer et de la houille, par exemple, ou la préparation du coton, les autres employant cette matière première. — Ainsi, le domaine de la division du travail va s’élargissant, et, pareillement, celui de l’échange, en attendant qu’ils soient absorbés, dans un avenir infiniment lointain, l’un par le progrès de l’association, l’autre par celui de la gratuité.\par
Pendant que la coopération indirecte des ouvriers distants les uns des autres ne cesse de grandir, la coopération directe des ouvriers rapprochés va en diminuant. C’est de plus en plus avec la machine qu’il surveille ou dirige, c’est de moins en moins avec d’autres travailleurs que le travailleur moderne coopère à une tâche élémentaire. Voyez la jeune femme qui dévide un écheveau de laine. Devant elle  \phantomsection
\label{v2p387}son enfant tient l’écheveau les bras écartés. Si elle a un rouet, le rouet remplace l’enfant. La manœuvre d’une trirème antique exigeait la collaboration de nombreux rameurs ; la manœuvre d’un navire à voiles d’égale dimension ne nécessitait qu’un nombre beaucoup moindre de mousses et de matelots ; enfin, un bateau à vapeur de volume égal est manœuvré par un équipage insignifiant. Pour le service d’une baliste ou d’une catapulte, il fallait apparemment plus de soldats qu’il ne faut de canonniers pour le service du plus gros de nos canons. Le dépiquage du blé par le fléau ou par le rouleau suppose plus de bras réunis que la même opération accomplie par la batteuse mécanique. Au point de vue de notre inter-psychologie, cette transformation offre un grand intérêt. Car la collaboration directe et rapprochée, la solidarité sentie, qui est remplacée par la collaboration indirecte et lointaine, par la solidarité inconsciente ou froidement comprise, constituait un lien d’homme à homme des plus énergiques et des plus répandus jusqu’à nos jours.\par
Mais, si cette \emph{harmonie} de travaux différents qui unissaient si étroitement les petits patrons d’autrefois à leurs apprentis diminue ou disparaît, l’\emph{unisson} des travaux semblables exécutés dans un même lieu grandit toujours. Un grand atelier de nos jours rassemble un grand nombre d’ouvriers qui, collaborant séparément avec une machine, font tous le même travail, dans la même salle du moins, et rappellent ainsi les \emph{chambres à filer} regrettées par Bücher. Il se noue là des liens de camaraderie d’une autre nature, mais dont la force n’est pas douteuse. Par là on peut voir se vérifier l’idée que le vrai fondement de l’union sociale doit être plutôt cherché dans la similitude acquise des individus que dans les services qu’ils se rendent. Mais, à vrai dire, ils ne peuvent pas se rendre de plus grand service que de se rassembler pour se ressembler davantage et de fortifier leur sympathie naturelle en l’exerçant soit par le travail en commun, soit, encore mieux, par le plaisir et la consommation \phantomsection
\label{v2p388} en commun, deux choses connexes dans le groupe villageois de jadis comme dans la grande industrie moderne.\par
— Si grands que soient ses progrès, la division du travail reste toujours une association simplement implicite et imparfaite, et ne peut être regardée que comme une préparation nécessaire aux développements de l’association proprement dite. Celle-ci crée entre ses membres une réciprocité d’aide à laquelle la division du travail n’atteint pas. Sans doute, entre les coopérateurs unis par elle, il y a une réelle solidarité et, par exemple, les extracteurs de matières premières ont besoin des fabricants qui emploient ces matières, comme ceux-ci ont besoin de ceux-là ; mais il y a cette différence que, pour une industrie spéciale qui extrait une matière première, il y a 10, 20, 100 industries spéciales qui l’emploient chacune à sa façon ; aussi toutes ont-elles besoin de la première, qui, elle, peut se passer de chacune d’elles séparément. La réciprocité est donc loin d’être pareille. Et ce n’est pas seulement l’extraction ou la préparation de matières premières qui jouit de ce privilège relatif ; il y a la même chose à dire du \emph{transporteur} ou de l’\emph{informateur}, de l’industrie des transports ou de celle des postes et télégraphes par rapport aux innombrables industries qui les utilisent. Les industries privilégiées dont il s’agit sont des centres dont les autres sont les rayons. — On peut donc voir une application de la loi qui régit le passage de l’unilatéral au réciproque, dans la tendance qui pousse les sociétés contemporaines à passer de la phase de la division du travail dominante à celle où l’association dominera.
\subsubsection[{III.6.b. Avantages de l’association. L’irrigation de l’Égypte. Exemples ’d’une division systématique et autoritaire qui se transforme en une division du travail en apparence spontanée. Puis l’inverse.}]{III.6.b. Avantages de l’association. L’irrigation de l’Égypte. Exemples ’d’une division systématique et autoritaire qui se transforme en une division du travail en apparence spontanée. Puis l’inverse.}
\noindent Quelles que soient les formes de l’association, ses avantages économiques sont évidents.\par
L’irrigation de l’Égypte est un bon exemple de ce que  \phantomsection
\label{v2p389}peut pour le bien de tous l’association des efforts humains. On y voit aussi le lien étroit de l’association et de l’invention, ou, pour mieux dire, que l’association, pour être féconde, doit être une invention réalisée et mise en œuvre. A quel Pharaon, à quel grand prêtre d’Isis, à quel obscur fellah, remonte la première idée de l’irrigation de la plaine égyptienne par le système de digues perpendiculaires et parallèles au cours du Nil, d’où résultait, après la crue, une échelle de bassins superposés où des machines élévatoires puisaient l’eau pour l’arrosage des terres ? Et ces machines elles-mêmes, si ingénieuses en leur préhistorique simplicité, qui les a conçues et fabriquées le premier ? Nous n’en savons rien. Mais nous sommes mieux renseignés sur les auteurs des machines à vapeur qui remplacent maintenant, dans une bonne moitié de l’Égypte, ces antiques appareils ; et nous savons aussi que c’est Méhémet-Ali qui, au système des digues perpendiculaires et parallèles a commencé à substituer celui d’une canalisation perfectionnée. Ce nouveau procédé, repris et complété par les Anglais, a été une révolution économique pour la vallée des Pharaons. Grâce à lui, le cultivateur a, pendant plus longtemps, dans l’intervalle des crues, plus d’eau pour arroser ses terres ; et des terres qui n’étaient jamais arrosées le sont à présent. De la sorte, la culture du coton, notamment, s’est beaucoup développée. De 1888 à 1897, la récolte du coton a \emph{triplé}. La récolte de la canne à sucre, à peu près dans le même laps de temps, a doublé. Ajoutons que la population de l’Égypte croît très vite par suite de cette fertilité croissante de son sol. Elle atteint 10 millions ; elle n’était que de 2 millions environ en 1835\footnote{ \noindent \emph{Revue d’Économie politique} de juillet 1900, article de Jean Brunhes.
 }.\par
Si, au lieu de s’associer pour l’exécution d’un plan d’ensemble, les cultivateurs égyptiens, depuis les Pharaons, s’étaient disputé l’eau du Nil, ou n’avaient songé chacun qu’à sa terre, la récolte de chacun eût été moindre ou nulle,  \phantomsection
\label{v2p390}et certaines récoltes, par exemple celle du coton (qui exige une irrigation permanente) eussent été impossibles. En associant leurs efforts, les hommes les ont adaptés à leur but commun. Et cette adaptation, comme cette association, a grandi à chaque perfectionnement des inventions anciennes. Les hommes s’associent, s’adaptent mutuellement les uns aux autres, d’autant plus qu’ils adaptent davantage la nature à eux, sans réciprocité.\par
Il y a des cas où la convergence des efforts n’empêche pas l’opposition des intérêts de se faire sentir : ce sont les années où la crue du Nil est insuffisante, par exemple en 1900. Alors se pose le problème de savoir quelles parties du territoire seront sacrifiées à telles autres, quelles récoltes (de riz ou de maïs ?) seront sacrifiées à telles autres (de coton ?) qui paraissent plus précieuses (à tort ou à raison). En établissant des barrages qui captent l’eau insuffisante, on la conserve pour les propriétaires d’amont, mais au préjudice des propriétaires d’aval. En prenant des mesures comme en a pris parfois l’administration anglaise, qui permettent de sauver la récolte de coton, mais qui perdent la récolte de riz, on réduit au minimum la perte de richesse, mais on élève au maximum les effets désastreux de la disette. Le sacrifice inverse eût été certainement préférable. Mais ce n’est là qu’une parenthèse.\par
Nous voyons par l’exemple précédent qu’une co-assistance mutuelle, qui a commencé par être la soumission commune à un plan d’ensemble systématiquement imposé par l’autorité d’un Pharaon, s’est continuée par vitesse acquise et a pris un faux air de spontanéité. Cette évolution, — précisément inverse de celle qui serait, d’après tant d’écoles, le développement normal de l’industrie, assujettie, nous dit-on, à passer de la division du travail spontané et libre à la répartition obligatoire des tâches par le pouvoir, — est loin d’être exceptionnelle. Dans la villa romaine, dans le domaine franc, il y avait une distribution autoritaire des  \phantomsection
\label{v2p391}travaux entre les esclaves ou entre les serfs qui collaboraient à l’œuvre commune et qui constituaient ensemble un organisme fermé, se suffisant à lui-même. Et c’est de cette véritable association forcée que dérive, d’après Bücher, la division du travail libre, soi-disant spontanée, telle qu’elle apparaît dans les villes du moyen âge\footnote{ \noindent \emph{Études d’histoire et d’économie politique}, trad. franç. (1901).
 }. « Quelles que soient, dit-il, les objections qu’on puisse faire valoir contre la théorie qui fait dériver la constitution urbaine de la constitution de la cour domaniale, on ne peut cependant bien comprendre et bien expliquer l’organisation économique de la ville (du moyen âge) que si on la considère comme continuant l’organisation de la cour domaniale. Ce qui n’était ici que germe est devenu dans la ville organes et système d’organes ; les éléments qui, dans l’économie domestique, étaient rassemblés en un tout informe, se sont rendus indépendants et autonomes. La division du travail qui, dans la cour domaniale, était imposée, est devenue en ville une répartition libre des branches de production entre paysans et bourgeois, et cette répartition a donné naissance dans la bourgeoisie à quantité de professions distinctes. » Remarquons, dans ce passage intéressant, la singularité apparente — nullement réelle à notre avis — d’un \emph{germe} informe qui, en se formant et s’épanouissant, cesse d’être conscient et volontaire pour devenir quasi-inconscient et quasi instinctif. En réalité, c’est là non l’exception, mais la règle : la volonté tombe dans l’habitude, la conscience dans l’inconscience, par le développement le plus naturel.\par
Puis, l’inverse se produit ; et, après que le souvenir des anciennes coopérations réfléchies et autoritaires d’où procèdent de nouvelles coopérations libres et plus amples a disparu, celles-ci suscitent de nouveaux plans systématiques d’association, tels que les corporations du moyen âge. Celles-ci sont-elles la réapparition et la métamorphose des anciens \emph{collegia} romains ? C’est possible ; mais là n’est pas  \phantomsection
\label{v2p392}la question\footnote{ \noindent Cette question, peut-être la tranche-t-on trop aisément par la négative... « Quand Gallien se rendit triomphalement au Capitole... derrière les sénateurs, les chevaliers, les pontifes, venait le peuple... on voyait briller 500 lances à la hampe dorée et 100 \emph{bannières, appartenant aux diverses corporations, flottaient au vent, au milieu des étendards des temples et des enseignes de toutes les légions.} » (Levasseur.) Rien ne prouve mieux que ce \emph{mélange} l’importance singulière des corporations industrielles sous l’Empire romain, vers la fin surtout. « En Gaule, les habitants d’Autun, voulant dignement recevoir Constantin, décorèrent les rues... et, sur le chemin que devait suivre le prince, \emph{étalèrent les bannières, les ornements des corporations et les statues de tous les dieux.} » Peut-on vraiment croire que des [{\corr corporations}] aussi florissantes, \emph{aussi enracinées} que paraissent l’avoir été ces corporations impériales, aient disparu tout à fait après l’Empire, et qu’il n’y ait \emph{rien de commun} entre elles et les corporations du moyen âge qui ont tant de traits communs avec elles, le même esprit religieux, la même solidarité fraternelle, le même goût de processions avec bannières en tête parmi les statues sinon « des dieux », du moins de la Vierge et des Saints ? Il faudrait, pour croire cela, des raisons tout autrement sérieuses que le silence gardé par les chroniques sur ces corporations pendant plusieurs siècles d’ignorance, silence qui s’explique si bien... En tout cas, le souvenir de ces \emph{collèges} a dû subsister et susciter leur résurrection sous des formes nouvelles. Ashley incline à le penser ; il lui semble possible que « la réglementation romaine (des \emph{collegia)} ait servi de modèle à l’organisation des esclaves artisans sur les terres des monastères et des grands seigneurs », et, dit-il, « c’est probablement, sur le continent, l’origine de quelques-unes des gildes de métier postérieures. » Remarquons que cela suffit pour que, l’exemple de ces gildes exceptionnelles se répandant, tout s’explique. — Blanqui estime que les corporations ont pris naissance dans les monastères. En tout cas, il y a un lien manifeste, signalé avec raison par Claudio Jannet, entre le mouvement social d’où est née l’exubérante floraison des corporations au {\scshape xiii}\textsuperscript{e} siècle et celui d’où étaient nés auparavant les grands ordres religieux du même siècle. « L’unité morale, dit-il, rétablie dans la chrétienté après des troubles profonds, par l’action des confréries dominicaines et franciscaines, favorise singulièrement le développement des corporations. » « \emph{Ab interioribus ad exteriora} » les transformations profondes, d’ordre moral, religieux ou scientifique, précèdent les changements d’ordre économique.
 }. Ressuscitées ou réapparues, ou créées de toutes pièces, peu importe ; elles ont été, au {\scshape xi}\textsuperscript{e} siècle, adaptées à des besoins nouveaux et marquent une nouvelle invasion, en pleine barbarie, de l’esprit de réglementation coordinatrice. Le besoin de réglementer ainsi le commerce, le commerce inter-municipal, équivalent alors de notre commerce international, s’est fait sentir avant le besoin de réglementer de même le travail. Les gildes de commerce datent de la deuxième moitié du {\scshape xi}\textsuperscript{e} siècle ; trois quarts de siècle plus tard apparaissent, en Angleterre, les gildes de métier, leurs filles et leurs rivales (Ashley). En Allemagne,  \phantomsection
\label{v2p393}pareillement, les gildes de commerce ont été en lutte avec les gildes de métier qui, nées à leur ombre, leur disputaient en grandissant le pouvoir municipal. Les premières gildes de métier qui ont pris force et vigueur ont été celles des « arts de la laine », tisserands, foulons. Les articles relatifs aux vêtements sont, en effet, les premiers qui aient pu être, à raison de la facilité de les conserver longtemps et de les transporter au loin, l’objet d’une demande considérable et dont la fabrication a pu, de bonne heure, compter sur un grand débouché. Les articles relatifs à l’ameublement, plus lourds, de plus difficile transport, répondant à des besoins moins rapidement contagieux, ne sont venus qu’après. « Dans la première moitié du {\scshape xiv}\textsuperscript{e} siècle, le système de la gilde a atteint son plus haut degré de puissance. Pendant les deux siècles suivants, cette forme d’organisation continue à être adoptée par toutes les industries. » Après quoi, elle se relâche, se dissout, et la première moitié du {\scshape xix}\textsuperscript{e} siècle à vu s’épanouir une ère de libre concurrence qui, quoique affectant les apparences d’une réaction contre le système réglementateur des corporations, n’en procède pas moins. Enfin, de nos jours, les syndicats inaugurent une nouvelle phase. Et c’est ainsi que la liberté économique, ou plutôt l’anarchie économique, réputée harmonieuse et féconde, se montre à nous toujours comme une transition d’une organisation consciente et voulue à une autre organisation non moins voulue et non moins consciente.
\subsubsection[{III.6.c. Corporations du moyen âge. Tendances nouvelles des confréries du xvc siècle.}]{III.6.c. Corporations du moyen âge. Tendances nouvelles des confréries du xv\textsuperscript{c} siècle.}
\noindent Les règlements des corporations du moyen âge avaient pour but d’adapter la production à la consommation dans des conditions de fixité réputées invariables de l’une et de l’autre, et dans les limites d’un marché bien circonscrit qu’on jugeait devoir rester de même inextensible. On ne  \phantomsection
\label{v2p394}prévoyait pas, on n’avait pas, semblait-il, à prévoir, vu la rareté extrême des inventions industrielles alors, le renouvellement possible des conditions de la production ; et on ne prévoyait pas davantage, vu la domination incontestée de la morale chrétienne, le changement possible des besoins de la consommation. Aussi n’avait-on pas l’idée de remarquer que les règlements de corporation étaient contraires aux progrès de l’industrie, en faisant obstacle à l’initiative inventive des individus ; et cet inconvénient, qui nous paraît majeur, eût passé pour insignifiant si on y avait pris garde. Il s’agissait tout simplement d’empêcher l’excès \emph{en quantité} et le déficit \emph{en qualité} de la production telle qu’elle existait relativement aux exigences actuelles, seules jugées légitimes, de la consommation. De là les longs apprentissages, le chef-d’œuvre, les servitudes minutieuses de la fabrication, les précautions infinies contre la fraude, les amendes infligées aux contrevenants. Tout cela répond parfaitement à la nécessité de l’adaptation qualitative dans ces conditions. Quant à l’adaptation quantitative, il y était pourvu par la limitation du nombre des apprentis, du nombre des maîtres, et de leur productivité. Dans les petites villes closes, dans ces étroits marchés, presque sans horizon extérieur, on redoutait avant tout la surproduction en fait d’objets fabriqués, et, en fait de denrées agricoles, le contraire, le déficit de production. Le monastère alors, économiquement clos et se suffisant à lui-même — comme la villa romaine, son premier modèle peut-être — était l’idéal sur lequel on dirait que cherchaient vaguement à se modeler toutes les villes, tous les bourgs, préoccupés de produire tout ce dont ils avaient besoin et de se passer du reste du monde\footnote{ \noindent De dégradant qu’il était dans l’antiquité classique presque tout entière et aussi dans la société civile du moyen âge, le travail devint \emph{sanctifiant} dans les premiers grands ordres religieux. Il était considéré comme un moyen de mortification et de purification aussi efficace que la prière, à laquelle il s’alliait dans les couvents d’après la règle de Saint-Benoist. La propagation de cette règle dans le haut moyen âge a quelque chose d’analogue à celle des corporations plus tard. On peut la considérer comme un des moyens, les plus réussis qui aient été tentés, d’\emph{organisation du travail} et d’\emph{harmonisation économique.} — Par cette règle, par cette vie monastique, l’accord s’opérait entre le désir mystique du salut et les besoins pratiques qu’il fallait satisfaire : défricher la terre devenait un moyen de conquérir le ciel. Dans les couvents, tout conflit entre co-producteurs est apaisé. Il n’y a pas de grèves, pas de chômage ; le patron c’est-à-dire le \emph{prieur}, ne diffère en rien des ouvriers (des \emph{moines).} D’autre part, les divers monastères d’un même ordre, tous voués aux mêmes métiers, aux mêmes travaux, ne se font point concurrence, ils sont moralement et économiquement associés dans une même œuvre sociale. Par leur exemple, les moines rehaussaient la condition des travailleurs et la considération attachée au travail. Ils répandirent des \emph{jugements} plus adaptés aux \emph{besoins} du temps. — Dans les monastères d’Orient, d’après saint Jérôme, régnait la division du travail : « Les frères d’un même métier se réunissent dans une même maison sous l’autorité d’un préposé ; ceux qui tissent du lin sont ensemble ; ceux qui font des nattes forment un même groupe ; les tailleurs, les charpentiers, les cordonniers sont à part. » etc. — Cette organisation rationnelle du travail fut importée en Occident. Un monastère, du {\scshape v}\textsuperscript{e} au {\scshape x}\textsuperscript{e} siècle, était un phalanstère. Là point de propriété individuelle, même des objets mobiliers. — Jusqu’au {\scshape xi}\textsuperscript{e} siècle les religieux lettrés et les illettrés étaient confondus ; à partir de cette époque, ils se divisèrent en \emph{frères lais} et \emph{religieux} proprement dits.
 }.\par
 \phantomsection
\label{v2p395}Si presque rien n’était libre, presque tout était public dans ces grandes ruches de verre. Pour garantir la bonne qualité des produits, les règlements exigent, en général, que l’atelier soit exposé aux regards des passants, que le serrurier, le forgeron, le menuisier, etc., exécutent leur travail dans une pièce donnant sur la rue. Le travail de nuit, à raison des éclairages si défectueux de ces temps, est interdit pour les fabrications délicates. Le règlement des marchés veut aussi, nous dit Bücher, « que les vendeurs d’un même produit exposent en vente l’un près de l’autre\footnote{ \noindent Voilà pourquoi, entre autres causes, les artisans de même catégorie se concentraient dans une même rue qui portait leur nom.
 }, \emph{se faisant la concurrence au grand jour}, sous la surveillance des maîtres du marché et d’inspecteurs ». Cette concurrence au grand jour sous l’œil d’un inspecteur n’était pas une guerre, c’était, à vrai dire, un \emph{concours}. Les concurrents ne devaient jamais oublier qu’ils avaient juré de vivre en bons confrères. Tous les statuts étaient édictés par le prince, qui réglait aussi le taux des prix et des salaires, car, jusqu’à la vulgarisation des idées de l’économie politique moderne, la fixation du taux des salaires et des prix,  \phantomsection
\label{v2p396}des profits et des bénéfices, faisait essentiellement partie des pouvoirs reconnus par tout le monde à l’État. C’est au gouvernement que s’adressent, dans leurs différends avec les patrons, les ouvriers français et les ouvriers anglais, et aussi bien, dans leurs démêlés avec les ouvriers, les patrons des deux nations. On ne concevait pas alors d’autre moyen d’accorder les intérêts contraires que l’application d’une loi existante ou la promulgation d’une loi nouvelle. Et de fait, là où l’antagonisme des intérêts, des prétentions du moins, est bien réel, où il n’est possible de les concilier, de les adapter mutuellement à l’amiable, qu’aux dépens d’une autre partie de la population — par exemple, les prétentions rivales des patrons et des ouvriers moyennant l’élévation des prix des produits, au préjudice des consommateurs — dans des cas pareils, le seul moyen de mettre fin à l’hostilité des demandes contradictoires, n’est-ce pas de leur imposer d’autorité un bornage légal ? Ainsi, jusqu’à la fin du {\scshape xviii}\textsuperscript{e} siècle, les salaires et la plupart des prix étaient conçus comme des taxes légales, à l’égal de l’impôt, ou comme les taxes municipales qui subsistent encore et en sont les derniers débris. Voilà pourquoi les coalitions d’ouvriers et les grèves étaient sévèrement punies : elles constituaient une rébellion. Dès que cette conception du salaire a disparu, il n’y a plus eu aucune raison valable, sous le régime individualiste du libéralisme économique, d’empêcher les ouvriers de se coaliser ; et, jusqu’à ce que le travail soit de nouveau conçu comme une fonction publique, il en sera ainsi.\par
L’esprit des corporations était dirigé tout autant contre la cessation du travail et l’insuffisance de la fabrication que contre l’excès de production. En cela aussi elles peuvent être comparées, sinon rattachées, aux \emph{collegia} romains. Au deuxième siècle de notre ère, la \emph{grève} des fonctionnaires et des producteurs apparut menaçante. Contre cette menace on n’imagina rien de mieux que « de river les uns et les autres  \phantomsection
\label{v2p397}à leurs fonctions et à leurs métiers\footnote{ \noindent \emph{Documents relatifs à l’histoire de l’industrie et du commerce, en France}. Fagniez (1898).
 } ; et ce fut le système où entrèrent toutes les professions quand, au siècle suivant, Alexandre Sévère eut groupé en collèges tous les artisans et commerçants isolés. Comme le \emph{curiale} devint inséparable de sa curie, le \emph{collegiatus} le devint de son collège. » C’est une conséquence forcée de toute organisation gouvernementale du travail. Si le travail, sur une bien plus grande échelle que jadis, est de nouveau gouvernementalement organisé, il n’est pas douteux que le droit à la grève s’évanouira.\par
— Sur une plus grande échelle : c’est en effet par cette grande inégalité de module que les associations de l’avenir, en germe dans celles du présent, différeront de celles du passé. Le caractère des corporations, et aussi bien des confréries du moyen âge, était de ne pas étendre plus loin leur horizon que les remparts d’une même ville\footnote{ \noindent Ceci ne s’applique pas aux associations du grand commerce, à certaines gildes renommées. Depuis l’Empire romain, dès le plus haut moyen âge, il y a toujours eu de grands commerçants, « héritiers des \emph{negotiatores} romains, se livrant à des opérations étendues, internationales », les \emph{frisons}, par exemple. (Fagniez, \emph{Documents pour servir à l’hist. de l’Indust.).}
 }. Tous les associés étaient concitoyens. Mais dès le {\scshape xv}\textsuperscript{e} siècle, le besoin d’association \emph{inter-municipale}, sinon internationale, se faisait déjà sentir parmi les artisans eux-mêmes. Le \emph{compagnonnage}, la \emph{franc-maçonnerie}, confréries d’un genre tout nouveau, fondées à cette première aube des temps modernes, ont répondu tant bien que mal à ce nouveau besoin. Car « malgré les guerres et la misère, nous dit Levasseur, les hommes du {\scshape xv}\textsuperscript{e} siècle avaient les uns avec les autres des rapports plus fréquents que ceux du {\scshape xii}\textsuperscript{e} ». Fait remarquable, qui atteste l’expansion irrésistible et continue de l’action inter-mentale, du rayonnement réciproque des exemples, si l’on songe à tous les obstacles qui s’étaient accumulés, pendant la guerre de Cent ans, pour refouler ce penchant expansif, pour raréfier les communications entre les hommes :  \phantomsection
\label{v2p398}routes non entretenues, ponts brisés, haines avivées, etc. Mais les mouvements de troupes, les grandes chevauchées, avaient brisé les clôtures de la vie sédentaire, forcé les gens de toute province à se rapprocher, et, quoique ce fût pour se battre, cela suffisait pour élargir la sphère de la sociabilité. Les barrières des classes aussi avaient été abaissées, par les défaites de la noblesse et l’amoindrissement de son prestige. Et le peuple, passant de l’admiration à l’envie à son égard, s’était mis à imiter en tout les seigneurs, à aimer comme eux les beaux habits, les grands cortèges, les fêtes d’apparat ; « il ne lui suffisait plus de vivre libre sous la protection du métier, il voulait des distinctions, des armoiries, des titres. » Louis XI, pour flatter ce prurit d’imitation, ne pouvait rien imaginer de mieux que d’accorder, comme il le fit, aux chefs des corporations le droit de porter la dague et l’habit de guerre, de se mascarader en gentilshommes\footnote{ \noindent A l’exemple de la noblesse peut-être aussi, les classes populaires, pendant la guerre de Cent ans, malgré les horreurs de ce temps, s’étaient prises d’une fureur de fêtes et de plaisirs. « A aucune époque, peut-être, dit Levasseur, depuis le commencement du moyen âge, la classe ouvrière n’avait imaginé plus d’occasions de se distraire. Elle menait probablement plus joyeuse vie qu’au {\scshape xiii}\textsuperscript{e} siècle. » Après la peste noire surtout, les salaires avaient rapidement haussé, pendant que les vivres et les denrées diminuaient de prix par suite de la puissance accrue de l’argent.
 }.
\subsubsection[{III.6.d. Besoins nouveaux auxquels les nouvelles formes d’association doivent répondre. Leur succès ou leur insuccès suivant qu’elles y répondent ou non. Sociétés de publicité, bureaux de placement, bourses du travail. Transformations de l’embauchage. Sociétés d’assurance, mutualités. Essais de production coopérative agricole.}]{III.6.d. Besoins nouveaux auxquels les nouvelles formes d’association doivent répondre. Leur succès ou leur insuccès suivant qu’elles y répondent ou non. Sociétés de publicité, bureaux de placement, bourses du travail. Transformations de l’embauchage. Sociétés d’assurance, mutualités. Essais de production coopérative agricole.}
\noindent Par cette double tendance à l’abaissement des barrières entre les classes d’une part, et, d’autre part entre les villes, les corporations ou confréries du {\scshape xv}\textsuperscript{e} siècle s’orientaient déjà dans la voie où nos associations contemporaines et futures ont ou auront à se lancer. Les nouvelles organisations de métier, dont tout le monde sent le besoin, sont destinées à différer des anciennes non seulement par leurs dimensions mais encore par leur nature. Les anciennes corporations n’étaient adaptées qu’à un régime social immuable et hiérarchisé, \phantomsection
\label{v2p399} divisé en classes multiples, en nombreux rangs superposés ; elles n’étaient adaptées qu’à la vie municipale ; elles n’étaient adaptées qu’à la petite industrie. Leur grand vice, en dépit de leurs vains essais d’élargissement, était, depuis le {\scshape xvi}\textsuperscript{e} siècle, de ne pouvoir s’assouplir aux conditions nouvelles d’un marché en train de s’agrandir très vite comme la production industrielle, et d’une société en voie de transformation non moins rapide. Il s’agit, en les ressuscitant, de les métamorphoser profondément pour les adapter à une société démocratique, à un marché national ou international toujours en train de croître, et à la grande industrie toujours progressiste. Tel est le problème redoutable, prodigieusement ardu, qui se pose aux sociétés modernes ; et toutes leurs créations contemporaines, en fait de syndicats, de sociétés commerciales, industrielles ou financières de tout genre, ne sont que des tentatives de solutions de ce problème, toutes provisoires, toutes plus ou moins approximatives. Il s’agit, avant tout, de ne rien faire qui puisse amortir le libre élan du génie individuel, source de tout progrès ; mais en même temps il faut concilier, avec cette liberté des forts, la justice, la sécurité des faibles. L’évolution historique se passe ainsi toujours à résoudre des problèmes insolubles en toute rigueur, à concilier l’inconciliable, à faire des quadratures de cercle. Et l’on peut se demander si l’évolution de la nature tout entière, même physique, même astronomique, ne consisterait pas, en somme, à mesurer des incommensurables, à donner un faux air de précision mathématique à de simples approximations. De là peut-être cette mutabilité universelle, éternelle, qui semble attester une inquiétude inapaisable du fond des choses, continuellement déçu et insatisfait.\par
Si l’on passe en revue les formes d’association essayées à notre époque contemporaine, on verra qu’elles ont échoué ou réussi suivant qu’elles ont répondu ou non aux exigences en question.\par
 \phantomsection
\label{v2p400}On aurait pu croire que les sociétés pour la construction des maisons d’ouvriers à bon marché étaient destinées au plus bel avenir. Ne semble-t-il pas qu’elles répondent au besoin le plus urgent de l’ouvrier moderne ? Pour celui-ci la grande question n’est pas celle du pain ni même de la boisson ; il gagne toujours bien assez pour s’alimenter ; mais c’est la question du logement. Majeure pour l’ouvrier urbain d’aujourd’hui, elle n’est secondaire que pour l’ouvrier rural. Toutefois ces sociétés, après une vogue passagère, ont été enrayées dans leur marche ; car, de moins en moins, l’ouvrier urbain de nos jours a intérêt à être propriétaire de sa maison. L’instabilité de sa vie, la nécessité de changer souvent de travail et de résidence, lui interdit ce luxe d’une maison à lui, si favorable d’ailleurs à la vie de famille.\par
Les agences de publicité, d’information et de réclame, ont eu un succès inouï et toujours croissant, parce que, servant à faire connaître aux grands industriels comme au public les nouvelles de tout pays, et à faire connaître en tout pays les produits de ces grands industriels, elles sont la condition nécessaire des élargissements du marché. Ces bureaux récepteurs et transmetteurs de dépêches, à chaque instant venues de partout et lancées partout, sont les pourvoyeurs de la Presse, et, par elle, satisfont la curiosité générale et sans cesse élargie du grand public en même temps que la curiosité générale et intéressée, sans cesse croissante aussi, des producteurs à l’égard des consommateurs, et de ceux-ci à l’égard de ceux-là. Par la multiplicité et la précision toujours plus grandes de leurs renseignements, ils remédient à l’inconvénient majeur que présenterait sans eux l’extension d’une clientèle de moins en moins connue, de plus en plus éloignée de ses fournisseurs, à savoir l’incertitude relative au nombre et à la nature des besoins, ou aussi bien des produits. Ils rendent seuls possible, donc, l’ajustement des produits aux besoins malgré cet éloignement.\par
Une branche intéressante de ces sociétés de publicité,  \phantomsection
\label{v2p401}ce sont les bureaux de placement, qui se développent aussi très rapidement et qui aspirent à se centraliser. Les Bourses du travail ont la prétention, plus ou moins justifiée, de répondre à cette aspiration. Quoi qu’il en soit, le moyen âge n’avait nulle idée de ces institutions. Là où tous les producteurs et tous les consommateurs se connaissent personnellement, il n’est nullement besoin de ces moyens compliqués de faire rencontrer soit les produits (agences de publicité), soit les services (bureaux de placement) avec les besoins auxquels ils sont adaptés. Mais, dès que, en vue d’une clientèle de plus en plus vaste et inconnue, il a fallu recruter les ouvriers dans une région plus large, placer les ouvriers est devenu une fonction spéciale que la division du travail devait dégager. Les procédés employés pour produire cette adaptation nécessaire et fondamentale ont beaucoup varié. Il subsiste encore dans nos campagnes et même dans nos villes beaucoup d’embauchages directs, seuls pratiqués jadis. D’une enquête faite par l’\emph{Office du travail} en 1891 sur l’état actuel de l’embauchage, il résulte que « le placement personnel » est encore très prépondérant, et que « toutes les institutions de placement réunies ne donnent que des résultats secondaires auprès de ce mode qui est à la fois le plus simple et le plus usité ». Cet embauchage direct se pratique d’ailleurs d’après des formes réglées : les ouvriers disponibles, dans les villes, se rendent à tels endroits désignés, traditionnels, à tel carrefour, sur telle place (« la place de grève »), chez tel marchand de vins, lieux différents d’après les professions, et ouverts ou fermés suivant les cas. Un progrès notable consiste à se réunir sous des abris couverts et non sur la voie publique. « Dans\footnote{ \noindent Voir \emph{Le placement}, publication de l’\emph{Office du travail}. 1893.
 } tous les chefs-lieux de canton des départements agricoles, les jours de marché, les jours de foire, pendant certains jours de fête, la Saint-Jean, la Toussaint, les domestiques de fermes, des deux sexes, les ouvriers des champs,  \phantomsection
\label{v2p402}les moissonneurs, se tiennent sur la place publique, à la disposition des cultivateurs qui traitent avec eux, de gré à gré. On donne à ces réunions d’ouvriers le nom d’\emph{assemblées} ou de \emph{loues.} Les départements où ces loues sont signalées comme les plus importantes sont : Charente, Eure, Eure-et-Loir, Indre, Vienne, etc. Dans l’Eure, les hommes qui veulent se louer ont, à la main, une branche verte, et les femmes un bouquet. » Toujours, comme chez les sauvages, un air de plaisir et de joie donné aux affaires. Tout simple qu’il est, ce procédé d’embauchage est un grand progrès sur le procédé plus antique et plus simple encore qui consistait, pour l’ouvrier, à aller de porte en porte chercher un patron, ou, pour le patron, à aller d’ami en ami se renseigner sur l’existence d’un ouvrier qui lui convint.\par
Remarquons que le problème du bon placement des ouvriers et des domestiques n’est pas sans analogie avec celui de l’association logique et féconde des idées. Comment une idée parvient-elle à se rencontrer avec une autre idée propre à la féconder ? Comment un ouvrier parvient-il à trouver une place où il travaillera utilement ? Mêmes problèmes, l’un psychologique, l’autre social, parallèlement résolus. Il n’y a en somme, que deux moyens principaux, avec des variantes nombreuses, d’aboutir à ce genre d’adaptation. Quand nous cherchons de nouvelles combinaisons d’idées, tantôt, devant un principe momentanément fixe, admis à titre d’hypothèse, nous faisons défiler, pour le vérifier, des processions de souvenirs — car l’esprit a ses bureaux de placement d’idées qui sont ses mémoires spéciales — tantôt, à l’inverse, un fait précis, certain, nous frappant et nous arrêtant, nous faisons promener devant lui une série d’hypothèses pour l’expliquer. De même, les personnes (ou aussi bien les choses) à placer, tantôt vont à la recherche des employeurs, qui restent chez eux, du public qui reste chez lui et les voit défiler ; tantôt elles se réunissent quelque part où le public défile devant elles et choisit celles qui lui plaisent. Les marchands \phantomsection
\label{v2p403} ambulants, les voyageurs de commerce, les visites de bonnes cherchant des maîtres, les prospectus envoyés à domicile, sont des variétés du premier procédé ; les foires, les \emph{loues}, les grands magasins, les bureaux de placement, sont des variétés du second. On voit que les deux persistent mais en se transformant dans le sens d’une mobilisation plus rapide, plus facile et plus ample. En 1887, fut fondée, à Paris, la première Bourse de travail française. Elle était destinée à être aux bureaux de placement ce que les grands magasins sont aux petites boutiques. Depuis, l’exemple de Paris a été suivi par beaucoup de villes de France.\par
Mais passons à d’autres institutions qui ont eu encore plus de succès. Si les agences de publicité répondent au besoin de certitude, que le progrès de l’incertitude aiguise et développe, les sociétés d’assurances de tout genre, d’assurances-incendies, d’assurances-vie, d’assurances-accidents, les sociétés de secours mutuels, répondent au besoin de sécurité qui grandit avec l’instabilité des fortunes et des situations, créée par les révolutions économique de notre âge. Les mutualités notamment sont le contrepoids nécessaire et le correctif de notre inventivité rénovatrice, qui rend tout instable et précaire, pendant que la prévoyance se propage avec la civilisation. Aussi toutes ces sociétés progressent-elles merveilleusement. Si cette progression s’arrête un jour, c’est que l’assurance par l’État se sera substituée à ces sociétés libres. Reste à savoir si, compensation faite de ce qu’on gagnera et de ce qu’on perdra à cette substitution, il y aura finalement avantage.\par
Les divers offices remplis de nos jours par ces associations distinctes étaient remplis à la fois, mais dans un cadre bien plus restreint, par chacune des corporations de jadis qui se sont, pour ainsi dire, épanouies de la sorte. Nous y trouvons aussi, et surtout, le germe d’associations d’un tout autre genre qui sont non moins florissantes. Nos syndicats, nos \emph{trades-unions}, ont puissamment développé ce côté de la vie  \phantomsection
\label{v2p404}corporative par lequel se faisait sentir souvent aux ouvriers et aux petits patrons d’autrefois la divergence de leurs intérêts, malgré leur lien de confraternité. Là aussi il y a eu division de ce qui était indivis\footnote{ \noindent Nous pouvons remonter plus haut. Dans le \emph{clan} primitif, comme dans la corporation du moyen âge, se trouvaient confondus bien des germes d’associations différentes qui se sont développées à part plus tard. Sur le clan, sujet très obscur, M. Kovalesky a publié une étude singulièrement instructive, à propos de \emph{l’Organisation du clan dans le Daghestan}, dans la \emph{Rivista italiana di sociologia} (mai 1898). Le clan nous apparaît là comme ayant été une société \emph{de secours mutuels}, analogue à nos sociétés de ce nom, à cela près que les liens du sang, réels ou fictifs, y engendrent des liens d’obligation réciproque.
 } : au lieu d’une seule corporation par métier, où maître et ouvriers étaient confondus, nous avons, dans chaque profession, un syndicat ouvrier opposé au syndicat patronal, à l’exception de quelques syndicats industriels mixtes et des syndicats agricoles. Sous la forme du trust, très adaptée aux conditions de la grande industrie, comme les grands magasins le sont aux conditions du grand commerce, les syndicats patronaux sont en train de fleurir. Les syndicats ouvriers, qui les ont suscités, sont, comme eux, des armées levées pour la lutte des classes. Par cette lutte, il est vrai, s’affirme et s’accentue l’inégalité des classes subsistantes, mais s’exprime aussi le besoin de la supprimer ; et, en attendant, la formation même de ces armées sociales, composées souvent de syndicats fédérés lors des \emph{grèves sympathiques}, suppose la disparition des inégalités anciennes qui hiérarchisaient en rangs si différents les divers métiers séparés par des mépris ou des envies héréditaires, et qui formaient autant de classes distinctes et superposées\footnote{ \noindent Le \emph{Compagnonnage} est le débris survivant et archaïque de cet ancien esprit \emph{inégalitaire} des classes populaires.
 }. Le nombre des classes proprement dites, en effet, a été en diminuant pendant que celui des professions s’est augmenté ; comme si la classe, survivance de la caste qui, à l’origine, se justifiait par la nécessité de l’hérédité professionnelle pour le recrutement des métiers et la conservation de leurs procédés, avait perdu de plus en plus sa  \phantomsection
\label{v2p405}raison d’être. L’idéal entrevu est celui d’une société où il n’y aurait plus qu’une seule classe divisée en une multitude de métiers. Que la lutte des classes soit le meilleur chemin à suivre, et le plus sûr, pour aboutir à ce terme, voilà ce qui n’est nullement prouvé. Mais elle y vise, et c’est pourquoi les syndicats ouvriers se répandent, tandis que les syndicats industriels mixtes, qui visent à l’accord des classes et, par suite, reposent sur le maintien de leur distinction, ne font pas de progrès. Il en est autrement en ce qui concerne les syndicats agricoles, qui sont essentiellement mixtes comme les anciennes corporations, et prospèrent pourtant. Mais c’est que l’agriculture a été bien moins transformée que l’industrie par l’invention des machines, et la ferme nouvelle est restée bien plus semblable à l’ancienne ferme que l’usine d’aujourd’hui à l’atelier d’autrefois. Il n’y a pas là, en France du moins et en Allemagne aussi, je ne dis pas en Angleterre, une classe qui détienne la terre et l’autre qui la travaille. L’agriculture s’industrialise, il est vrai, mais, grâce aux syndicats agricoles précisément, il est possible aux petits cultivateurs d’acquérir à bon compte les matières premières, les engrais, et même, en se cotisant, les machines, que requiert cette transformation ; et en cela le voisinage des grands ou des petits propriétaires aide plutôt qu’il n’entrave le paysan français ou allemands. Aussi les syndicats agricoles réussissent-ils d’une façon marquée, comme le remarque M. de Rocquigny, dans les pays de petite culture et dans la zone viticole ; « les contrées de grande culture des céréales et d’élevage leur semblent moins propices. » En Angleterre, ils sont très peu nombreux, et sont des syndicats \emph{d’ouvriers} agricoles\footnote{ \noindent \emph{Les syndicats agricoles et leur œuvre}, par le comte de Rocquigny (1900). Le nombre de ces syndicats s’est élevé, par une progression interrompue, de 5 en 1884 à 2 133 en 1899, avec près de 500 000 membres dans cette dernière année. — Il y a une grande fédération des syndicats agricoles, \emph{l’Union des syndicats agricoles}, annexe à la grande société des agriculteurs de France. — Les syndicats agricoles allemands sont aussi très florissants. Ils peuvent, à la différence des nôtres, s’appuyer sur des fondements autres que strictement économiques. De là leur solidité remarquable. Chez nous « la jurisprudence a condamné les membres d’un syndicat qui avaient cru pouvoir appuyer leur association sur des fondements tirés de la religion et de la morale ». — Notons le caractère complexe des syndicats agricoles, qui ont les buts les plus divers : achat des engrais, emploi des machines agricoles, reconstitution des vignobles, préservation des récoltes, élevage des bestiaux, production et vente du lait, du beurre, etc. Les éleveurs de la Charente obtiennent, nous dit-on, de leur lait, par l’association, le double de ce qu’ils obtenaient par la production individuelle.
 }.\par
 \phantomsection
\label{v2p406}Les sociétés coopératives de consommation ont eu la même puissance de propagation et de développement que les syndicats ouvriers. Leur but est le même : augmenter la force du parti ouvrier en élevant son bien-être. Elles le poursuivent par des moyens différents mais complémentaires : les syndicats, par l’augmentation des salaires ; les coopératives, par la diminution des dépenses. Les coopératives, par l’abaissement du prix des articles, favorisent le développement de la consommation, et, par là, celui de la production. Au contraire, les coopératives de production ont échoué partout jusqu’ici, car elles sont très mal adaptées aux conditions de la grande industrie, comme le montre Bernstein. Un atelier, comme un régiment, exige une discipline d’autant plus rigoureuse qu’il est plus grand ; et l’ouvrier n’obéit pas mieux au chef regardé par lui comme son camarade, que le soldat à l’officier qu’il a élu. Les sociétés coopératives agricoles réussissent fort bien, le succès est même très remarquable en Allemagne\footnote{ \noindent Voir à ce sujet Kautsky, \emph{Question agraire}, p. 176 et s.
 }, mais, quoique plusieurs d’entre elles puissent être qualifiées sociétés de production (laiteries, beurreries, raffineries), elles sont toujours, avant tout, des sociétés de commerce et de crédit, et c’est par ce côté qu’elles se propagent. Quant aux associations agricoles qui ont eu pour objet la culture en commun, leurs rares essais sont assez décourageants. Une tentative de ce genre a été faite en 1842, en Algérie, par le maréchal Bugeaud. Il fonda autour d’Alger trois villages avec des soldats. « Je soumis, dit-il, les colons au travail en commun ; cela était  \phantomsection
\label{v2p407}d’autant plus praticable que, jouissant des vivres et de la solde, ils devaient attacher moins d’importance au produit de leur peine. Ce produit devait former un fonds commun, destiné, au bout de trois ans, à faire les frais de leur mariage et à procurer à tous uniformément le mobilier de la maison et de l’agriculture. Pendant la première année, il y eut assez de zèle. Mais, après, relâchement complet. \emph{Ils comptaient les uns sur les autres pour le travail}, et les plus habituellement laborieux \emph{se mettaient au niveau des paresseux}. Ils prièrent Bugeaud de les \emph{désassocier.} Dans les trois villages, le résultat fut le même. C’est que, comme Kautsky est forcé d’en convenir lui-même, « le paysan est encore plus attaché à son lopin de terre que l’artisan à son échoppe. » D’autres essais pareils ont moins mal tourné. Kautsky s’étend avec complaisance sur l’association fondée en 1831 par sir Vandaleur sur sa propriété de Ralahine (Irlande). Elle réussit, dit-il, admirablement et ne dut sa ruine, au bout de trois ans, qu’à un accident malheureux. Remarquons qu’il s’agissait d’un petit nombre d’associés, une quarantaine d’ouvriers agricoles, et que ces phalanstères, comme le familistère de Guise, avaient été l’exécution d’une idée philanthropique, d’un programme individuel mûrement réfléchi. Il n’est pas douteux que, parmi les programmes de vie plus ou moins phalanstérienne, imaginés par des réformateurs psychologues, il s’en trouve de viables ; et, aussi longtemps que les exécuteurs de ces plans bien conçus, se conformeront docilement à la haute pensée de l’inventeur, on constatera leur prospérité. Rien ne s’oppose non plus à ce que cet exemple soit imité et se répande de proche en proche. C’est ainsi que de la conception d’un saint Bruno, d’un saint Bernard, d’un saint François d’Assise, sont sortis des milliers de monastères où le communisme religieux a fleuri pendant des siècles. Mais, le jour où quelque vivant, initiateur à son tour, se fatigue d’être mené par un mort, et conçoit un autre plan qui se répand dans  \phantomsection
\label{v2p408}son entourage, l’association ébranlée commence à se dissoudre. Ce serait une grande illusion de se fonder sur la réussite plus ou moins durable, ou même prolongée, de ces phalanstères monastiques ou autres dans le passé ou dans le présent, pour en induire la possibilité de l’enrégimentation collectiviste de toute une société. Car, précisément, les monastères ou autres groupes communistes n’ont pu se constituer et vivre que parce qu’ils trouvaient à se recruter dans une vaste société ambiante et non communiste, dont ils attiraient à eux les éléments les plus propres à la vie de communauté, les plus dociles et les plus crédules\footnote{ \noindent En corrigeant les épreuves de ce livre, je lis ces lignes de Guyau (dans \emph{l’Irréligion de l’avenir)}, qui me frappent par leur justesse : « Ce qui a rendu jusqu’ici le socialisme impraticable et utopique, c’est qu’il a voulu s’appliquer à la société tout entière, non à tel ou tel petit groupe social... L’avenir des systèmes socialistes, c’est de s’adresser à de petits groupes, non à des masses confuses, de provoquer des associations très variées et multiples au sein du grand corps social. » Et ceci encore : « Le socialisme, soutenu aujourd’hui par les révoltés, aurait besoin, au contraire, pour sa réalisation (surtout pour sa durée) des gens les plus paisibles du monde, les plus conservateurs, les plus bourgeois ; il ne donnera jamais une satisfaction suffisante à cet amour du risque qui est si fort dans certains cœurs... » C’est un « fonctionnarisme idéal ».
 }. Le jour où ce triage ne serait plus possible, puisqu’on recruterait tout le monde, les révoltes individuelles se multiplieraient. Pour assurer universellement et d’une façon durable le régime collectiviste, il faudrait donc que, à partir d’un moment donné, l’humanité cessât brusquement d’être \emph{inventive} et initiatrice et ne fût plus qu’\emph{imitatrice}, ou qu’elle se divisât, comme chez les Incas, en deux classes superposées : une élite d’initiateurs et une forte masse populaire entièrement dépourvue de forte originalité individuelle.
\subsubsection[{III.6.e. Toute association est sortie du cerveau d’un hotnme.}]{III.6.e. Toute association est sortie du cerveau d’un hotnme.}
\noindent Un point sur lequel il importe d’insister et qui vient d’être indiqué, c’est que toute association est sortie du cerveau d’un homme, et que chacune d’elles vaut ce que vaut l’idée de son fondateur. Si nous demandons comment se  \phantomsection
\label{v2p409}créent les sociétés de secours mutuels, on nous répondra : « Comme toutes les fondations, celles-ci sont l’œuvre d’hommes d’initiative, qui décident quelques camarades... C’est surtout l’exemple des sociétés voisines qui décide les promoteurs, et en même temps ces sociétés fournissent des modèles sur lesquels on se règlera. » (Hubert-Valleroux.) Nous savons que les sociétés coopératives de consommation sont nées en 1844 de la pensée de Robert Owen, comme les sociétés coopératives de production, en 1834, de la pensée de Büchez. En Allemagne ont germé les sociétés de crédit mutuel. Celles-ci, pareillement, se fondent, nous dit-on, « par l’impulsion de dévoués missionnaires qui vont de place en place prêcher l’utilité du crédit populaire, et par l’exemple du voisin... L’exemple d’un propriétaire important, d’un notable du pays, fait beaucoup pour décider les adhésions. » Comment germe, mûrit, se réalise l’idée d’une association nouvelle, nous le voyons clairement par l’exemple de Godin, le fondateur du \emph{familistère de Guise}, qu’on pourrait comparer à Boucicaut, le fondateur du \emph{Bon Marché.} Godin et Boucicaut ont été des inventeurs (V. sur Godin, Paulhan, \emph{Psych. de l’invention}, p. 115 et s.) Dès l’enfance, Godin, fils d’un ouvrier, \emph{se sent une mission}, et rêve de remplir un rôle social. Il était alors à l’école primaire. Après un \emph{tour de France} et une série de petites inventions industrielles, son idée se trouva précisée. Il fit divers essais en vue d’établir \emph{des relations nouvelles et plus justes entre le capital et le travail...} Après vingt années d’expérience, il formule les règles de l’association \emph{familistérienne} qui a prospéré après lui, \emph{d’après lui}, comme le \emph{Bon Marché} après Boucicaut, d’après Boucicaut.\par
Ainsi l’idée d’une association nouvelle à créer, ou même d’une association connue à établir dans un nouveau milieu, a tous les caractères d’une invention\footnote{ \noindent L’idée d’une entreprise industrielle à créer est une petite invention qui a consisté, comme toute invention, dans une \emph{hypothèse}, réalisée par cette \emph{tentative.} Tentative qui, si elle réussit, analogue en cela à une hypothèse qui se vérifie, enrichira son auteur ; mais, dans le cas contraire, le ruinera. Toute entreprise est un \emph{jeu} servi par un \emph{travail.} Quand elle cesse d’être aléatoire, elle commence à pouvoir être régie par l’État et aussi bien à pouvoir être une association coopérative de production. Mais à quelles conditions cesse-t-elle d’être aléatoire ? Là est tout l’avenir de ce genre de société.
 } et, l’âge où nous entrons  \phantomsection
\label{v2p410}sera caractérisé par la direction du génie inventif dans cette voie large et féconde. Quand on voit l’ingéniosité de tant de chercheurs se tourner de ce côté, on ne doit pas désespérer de voir une solution favorable donnée un jour au problème posé par les coopératives de production et mal résolu jusqu’à présent par elles. Cet insuccès relatif, comparé au grand développement des coopératives de consommation, s’explique peut-être par son rapprochement avec d’autres faits. J’ai noté ailleurs que les besoins de consommation de nouveaux articles se propagent dans un pays avant les besoins de production correspondants. Nous pouvons faire remarquer aussi que les besoins de consommation ont commencé à s’organiser avant les besoins de production. Par exemple, c’est pour répondre à des besoins de consommation que le prêt à intérêt a été d’abord imaginé et pratiqué ; et, comme le note M. Gide, à ce caractère habituel du prêt à intérêt au moyen âge tient la défaveur, très justifiée, de l’Église à son égard. Le prêt à intérêt en vue de la production n’est venu que plus tard\footnote{ \noindent « Ceux qui empruntaient, c’étaient, dans l’antiquité, les pauvres plébéiens pour acheter du pain ; au moyen âge les besoigneux pour s’équiper pour la croisade, tous pour des consommations improductives. On empruntait autrefois pour vivre ; à présent, pour faire fortune. » De là une interversion des rôles. « Autrefois on pouvait se préoccuper de protéger les emprunteurs contre la rapacité des prêteurs ; aujourd’hui il faudrait plutôt aviser à protéger les prêteurs contre l’exploitation des emprunteurs (spéculateurs, banquiers, lanceurs d’affaires) dont l’histoire financière de notre temps offre de si scandaleux exemples. » (Gide.)
 }. Peut-être n’est-il pas surprenant, d’après cela, que le problème de la consommation en commun, par mutuelle coopération, ait trouvé sa solution pratique et définitive avant que le problème, tout autrement difficile, de la production en commun, par collaboration volontaire et égalitaire, sans soumission à nulle puissance \phantomsection
\label{v2p411} extérieure, ait rencontré la sienne. Il est dans l’ordre qu’il en soit ainsi. Il est dans l’ordre, aussi bien, que, après avoir eu un caractère de protection ou d’exploitation unilatérale, une association revête, en se développant, un caractère de mutualité. Les sociétés de secours mutuels ont été précédées de sociétés de bienfaisance ; les sociétés d’assurances mutuelles contre les accidents, pour la vie, ont été précédées de sociétés non mutuelles ayant le même objet. La loi du passage de l’unilatéral au réciproque peut porter à penser que les sociétés de collaboration productrice, encore aujourd’hui régies autoritairement, sont destinés à se mutualiser elles-mêmes.\par
Si nulle brutalité révolutionnaire au service d’une utopie ne vient interrompre le grand travail de fermentation qui remue à fond les peuples contemporains, il est permis d’augurer leur régénération complète et assez rapide par le seul développement spontané et prolongé des germes d’association, anciens ou nouveaux, qui s’y répandent, qui s’y disputent paisiblement les hommes, et qui finiront par les enrôler tous sous les bannières multicolores de groupements infiniment variés.
\subsubsection[{III.6.f. Classification des associations en quatre catégories. Vues d’avenir. La fin du monde, au sens nouveau.}]{III.6.f. Classification des associations en quatre catégories. Vues d’avenir. La \emph{fin du monde}, au sens nouveau.}
\noindent Malgré leur extrême diversité, les associations peuvent être divisées en quatre catégories, qui répondent aux quatre branches, distinguées plus haut, de l’opposition économique. En premier lieu, nous avons des sociétés de co-production : ateliers, usines, corporations, syndicats, cartels et trusts, fédérations d’ateliers, coopératives de production, grandes compagnies de chemins de fer et de navigation, etc. En second lieu, des sociétés de co-consommation : ménages, communautés, phalanstères, coopératives de consommation, assurances aussi, etc. En troisième lieu, des sociétés d’échange, ou, plus généralement, d’adaptation de la production à la  \phantomsection
\label{v2p412}consommation : magasins et grands magasins, agences de publicité, bureaux de placement, etc.\footnote{ \noindent Notons les \emph{Unions pour l’exportation} (Exportverein) qui fleurissent en Allemagne.
 } Enfin, des sociétés de crédit, c’est-à-dire d’adaptation de la monnaie à sa fonction d’échange : maisons de change et de banque, monts-de-piété, crédit foncier, crédit industriel et commercial, bourses, etc.\par
Par les corporations, il était remédié aux désaccords et aux luttes internes de la production. Les coopératives de production visent le même but, sans l’atteindre encore. Les syndicats ouvriers ne le poursuivent pas, puisque la plupart sont inspirés par la fameuse « lutte des classes », mais ils l’atteignent sans le vouloir, à mesure qu’ils se développent, car ils rendent les grèves si désastreuses qu’on les évite à tout prix, par l’arbitrage, de même que les progrès du militarisme rendent la guerre impossible ; et, d’autre part, il est d’autant plus facile aux patrons de s’accorder avec les chefs des ouvriers que l’association de ceux-ci est plus nombreuse et mieux disciplinée. Les trades-unions anglaises en sont la preuve. — Quant aux concurrences des patrons entre eux, les sociétés d’accaparement et de monopole, les cartels et trusts, n’y remédient que trop bien.\par
Les désaccords et les conflits internes de la consommation, quoique moins bruyants que ceux de la production, sont de nature plus profonde et de guérison plus malaisée. Quand la lutte à laquelle se livrent nos besoins divers dans notre cœur tient seulement à l’insuffisance de notre budget pour les satisfaire à la fois, les sociétés d’économie mutuelle, pour ainsi dire, le « ménage » d’abord, et aussi les sociétés coopératives de consommation, excellent à l’apaiser. Mais, quand nos besoins se combattent en nous parce qu’ils se contredisent, parce qu’ils impliquent des conceptions opposées de l’ordre universel et de l’ordre social, une morale fondée sur des convictions solides peut seule rétablir  \phantomsection
\label{v2p413}en chacun de nous la paix intérieure, condition de l’équilibre économique dans la société. Et, comme il n’appartient, en général, qu’à une religion ou à une secte librement propagée, de former parmi les hommes une même croyance forte, fondement d’une même moralité durable, on peut dire que les \emph{églises}, les confessions religieuses ou quasi religieuses de tout genre, y compris certaines écoles philosophiques, stoïciennes par exemple, doivent être inscrites en tête des grands procédés d’adaptation économique\footnote{ \noindent Les Sociétés de tempérance travaillent, en refoulant certains besoins désastreux tels que le besoin d’alcool, à supprimer la lutte entre eux et les besoins vrais de la nature humaine. Mais leur action est bien faible, par malheur.
 }. Par leur réglementation et leur hiérarchisation générale des désirs, sous le joug des dogmes, aussi longtemps qu’elles règnent, elles préviennent tous les troubles possibles de la consommation, elles résolvent la question du luxe, et, en imposant de la sorte des bornes et des formes déterminées à la production, elles obligent celle-ci à s’ajuster à celle-là. Les « harmonies économiques » du treizième siècle, très différentes de celles que vantait Bastiat, reposaient, avant tout, sur la hiérarchie catholique des besoins et des activités. Sous des formes nouvelles de moralité, appuyée sur une communion quasi religieuse des esprits en un certain nombre de vérités démontrées et inébranlables, l’humanité aspire à retrouver, élargi, son équilibre d’autrefois. A l’ombre de cette grande et souveraine association libre, ou même d’un certain nombre d’associations pareilles, d’églises philosophiques, paisiblement rivales, il ne serait pas difficile à des sociétés industrielles ou commerciales de répondre à la plupart des problèmes posés par l’opposition économique.\par
Disons un mot des sociétés de crédit, qui ont pour mission de prévenir ou de terminer les luttes de nature monétaire. Par les maisons de change s’adoucissent les chocs des diverses monnaies concurrentes sur un même marché. Il est mis fin à ces luttes par des conventions internationales,  \phantomsection
\label{v2p414}par des associations libres d’un groupe de peuples dans le genre de l’Union latine. La question du bi-métallisme ne saurait être tranchée qu’ainsi. — Par les banques et les Bourses, on remédie soit à l’excès, soit à l’insuffisance de la monnaie en circulation. S’il y a trop d’argent, en apparence, dans un pays, c’est que l’argent cherche des emplois et n’en trouve pas ; les banques, qui sont les bureaux de placement des capitaux, lui procurent des emplois nombreux par le crédit qu’elles ouvrent aux négociants ou par l’achat de valeurs de Bourse. Inversement, les banques d’émission, en lançant de la monnaie de papier dans le torrent de la circulation, suppléent au déficit de la monnaie métallique.\par
Le \emph{Crédit} est un sujet dont j’aurais pu parler au livre de la Répétition économique ; car, comme la monnaie et le capital, dont il est en un sens l’extension, il sert à la reproduction active des richesses. J’aurais pu aussi traiter du crédit à propos de l’Opposition, car, lorsque la confiance dont il est l’expression est démentie par les faits, il y a crise. Mais, quand les faits la confirment, ce qui est le cas habituel et normal, quand les dettes sont régulièrement payées à l’échéance, il n’est pas de cause plus puissante de prospérité et d’harmonie économique, d’adaptation des richesses épargnées à leur fin sociale. Le crédit, c’est, en somme, purement et simplement, le \emph{prêt}, qui, nous l’avons dit, est peut-être plus primitif que l’échange. Dans le prêt, dans la vente à crédit, il entre toujours un aléa, un risque à courir, comme dans le jeu. Et cet élément d’intérêt grandit sans cesse. Il est curieux de voir ainsi, à la fin comme au commencement de l’évolution économique, l’importance du rôle joué par le \emph{plaisir} dans les \emph{affaires}, plaisir des chants et de la danse à l’origine, plaisir du risque et de la spéculation à présent. Il semble que ce soit, malgré ses dangers, une des sources nécessaires de l’harmonie sociale.\par
D’après les rapports des consuls américains publiés il y a  \phantomsection
\label{v2p415}quelques années\footnote{ \noindent Voir dans le \emph{Journal des économistes} de septembre 1886, un article très intéressant sur le crédit, extrait de \emph{The contemporary Review}, par John Raë.
 }, la proportion des affaires à crédit par rapport aux affaires au comptant est remarquablement semblable dans les pays les plus dissemblables ; elle est comprise entre 70 et 90 p. 100 d’un bout à l’autre de l’échelle de la civilisation, Siam, Allemagne, Canada. Il semblerait même, à ne pas décomposer les chiffres, que le crédit, non seulement ne grandit pas avec la civilisation et la richesse publique, mais est presque en raison inverse. En Hollande, la proportion, par exception, descend à 60 p. 100. Mais, ce qui est tout autrement significatif que cette constatation, le crédit, dans les pays pauvres, s’applique davantage aux achats de détail, de consommation, et moins aux achats en gros, en vue de la production. Dans les pays riches, on constate l’inverse. De plus en plus, ici, on paye tout comptant au détail, et même, pour la production, on paye comptant les matières premières, les salaires ; mais, dans la sphère étroite où il se localise, c’est-à-dire dans les achats en gros, en vue de la revente, le crédit prend une grande extension, quoique la durée des délais accordés aille toujours se resserrant. Tel crédit qui était de six mois ou d’un an est tombé à trois mois à mesure que le progrès des communications a diminué le temps moralement nécessaire au détaillant pour débiter sa marchandise. En gros ou en détail les produits agricoles sont payés comptant. — La forme la plus importante que tend à revêtir le crédit dans les nations civilisées, ce sont les emprunts des États modernes. Si les emprunts de cet ordre figuraient dans les calculs précédents, ils en bouleverseraient les résultats ; et la vraie proportion des affaires à crédit apparaîtrait ce qu’elle est, c’est-à-dire énorme et toujours croissante. A la vérité, on ne voit point les États civilisés d’aujourd’hui retarder d’un ou deux ans la paye de leurs officiers et de leurs fonctionnaires, \phantomsection
\label{v2p416} comme cela se pratique dans les États barbares ou demi-civilisés, qui, d’ailleurs, n’ont pas d’autres dettes que celles-là. Mais, si les nations modernes de l’Europe payent fort régulièrement leurs fonctionnaires et leurs soldats, elles ont des créanciers d’autre sorte qu’elles ne rembourseront jamais et dont elles augmentent sans cesse le nombre. L’État est devenu ainsi le grand banquier de la nation, recueillant toutes les épargnes et les plaçant à son gré.\par
Ceci nous conduit à dire que, outre les formes libres de l’association, de l’adaptation économique, énumérées ci-dessus, il y a la forme officielle et obligatoire, l’enrégimentation administrative. Celle-ci doit-elle finalement triompher ? Et est-il bon qu’elle triomphe ? Entièrement, non. Si son succès allait jusqu’à écraser les germes de l’association libre, les initiatives individuelles, source de tout progrès, ce serait la mort du genre humain. Mais on peut admettre que, à certains égards, l’État intervienne là où l’association libre est impuissante. Par exemple, nous espérons qu’un jour, grâce à quelque forme nouvelle d’association libre, les luttes internes de la production, grèves, concurrence, s’évanouiront ; mais les désaccords entre la production et la consommation subsisteront aussi aigus, plus aigus peut-être. Les coopératives de production pourront, tout aussi bien que les trusts, surproduire ou rançonner les consommateurs. Est-ce qu’il ne faudra pas, bon gré, mal gré, recourir à l’intervention de l’État réglementateur, sinon organisateur du travail ? Cette dure extrémité n’est repoussée avec tant d’horreur par les esprits éclairés que parce que, à juste titre, on se méfie de l’État, c’est-à-dire du mauvais recrutement de son personnel gouvernemental. C’est, au fond, le suffrage universel qui épouvante. Mais n’y a-t-il nul remède à cela ?\par
La position de la question montre le lien étroit qui rattache le problème économique au problème politique. Le premier ne pourra être tranché que lorsque le second aura reçu une réponse satisfaisante, c’est-à-dire lorsqu’on aura trouvé autre  \phantomsection
\label{v2p417}chose, politiquement, que le suffrage soi-disant universel sous la direction d’une presse omnipotente et irresponsable. D’autre part, cette considération dit assez que la politique n’est point séparable de la morale. Sans une transformation profonde des sentiments moraux, nulle réorganisation durable des gouvernements n’est possible. Mais, avant tout, la règle de la conduite dérive de la nature des convictions. Le problème moral est intimement lié, nous le savons, au problème intellectuel, c’est-à-dire scientifique et religieux. Avant que le problème moral, et, par suite, politique, et aussi bien économique, comporte une solution définitive, — momentanément définitive — il faut que les hommes se soient de nouveau accordés, spontanément, dans une foi commune sur certains points capitaux. Un accord de convictions fortes et logiques par la science, devenue incontestée à certains égards, et non un équilibre d’opinions faibles et tolérantes par le septicisme ; un accord de passions fortes et concourantes à un idéal commun\footnote{ \noindent Quelle est, dans ce concert passionnel qui sera la moralité future, la passion qui donnera le ton ? L’amour ou l’ambition, le culte du plaisir ou la soif de gloire ? D’après la passion élue reine, tout le système de la conduite devra changer profondément.
 } par la haute morale sociale, et non un équilibre de petits besoins et de petits échanges, par l’industrialisme : voilà l’aspiration de l’évolution humaine.\par
Le mérite du socialisme de notre époque aura été d’avoir une conscience forte et confuse de cette aspiration et de chercher à la réaliser. Mais il a le tort de croire que dans l’état présent des esprits et des cœurs, cette réalisation est possible. Il anticipe sur un avenir encore éloigné. Le moment ne sera venu de résoudre les problèmes qu’il soulève prématurément, de songer sérieusement à une haute réglementation d’ensemble de la production et de la consommation, que lorsque nous toucherons au seuil de la troisième et dernière phase de l’histoire humaine, telle que nous en avons plusieurs fois esquissé le tableau. Au point de  \phantomsection
\label{v2p418}vue économique, notamment, cette histoire se décompose ainsi : l’ère présente des innombrables petits marchés clos, — l’ère des marchés ouverts, de moins en moins nombreux et de plus en plus vastes, — et l’ère du marché unique et total, qualifié \emph{mondial.} Tant que se poursuit la seconde, où nous sommes compris et entraînés, les luttes économiques que soulève l’extension des marchés ont beau grandir avec eux, les concurrences ont beau se dresser effrayantes entre adversaires gigantesques, il suffit d’un nouvel élargissement d’horizon, de débouché, d’un nouveau développement de l’activité productrice, de l’avidité consommatrice, pour rétablir un instant l’harmonie économique, toujours instable, qui va être de nouveau rompue. Et l’on en est venu à croire vaguement que cette solution passagère du problème, sans cesse renaissante par la poursuite d’un mirage de paix sociale toujours fuyante, pourra durer sans fin, c’est-à-dire que la population humaine ne cessera de croître, ni ses besoins de se multiplier, ni la mappemonde de lui offrir d’autres terres à coloniser. Faut-il donc rappeler que la terre n’est pas infinie, et que notre civilisation est bien près de l’avoir envahie tout entière ?\par
La \emph{fin du monde}, cette grande épouvante du moyen âge, est destinée à redevenir bientôt, en un autre sens, une source d’angoisse. Si ce n’est plus dans le temps, c’est dans l’espace que ce monde terrestre se montre à nous comme inextensible ; et le déluge de l’humanité civilisée se heurte déjà à ses limites, à ses nouvelles colonnes d’Hercule, celles-là infranchissables. Qu’allons-nous devenir quand, bientôt, nous ne pourrons plus compter sur des débouchés extérieurs, africains, asiatiques, pour servir de palliatif ou de dérivatif à nos discordes, d’écoulement à nos marchandises, à nos instincts de cruauté, de pillage et de proie, à notre criminalité comme à notre natalité débordante ? Comment ferons-nous pour rétablir, chez nous, une paix relative, qui a toujours eu pour condition, depuis si longtemps, notre projection \phantomsection
\label{v2p419} conquérante hors de nous, loin de nous ? Chercherons-nous, au point de vue économique, à substituer au débouché extérieur un débouché intérieur plus ample en multipliant encore les besoins et les inventions propres à les [{\corr satisfaire}] ? Mais le cœur humain n’est pas inépuisable en désirs toujours nouveaux ; les mines de découvertes ne le sont pas davantage. Là aussi des bornes arrêteront un jour ou l’autre l’élan du progrès économique, tel qu’il est compris depuis la marée montante de notre civilisation européenne. Or, s’il arrive, par hasard, que le génie inventif rencontre ainsi son terme, voie tarir toutes ses sources, en même temps que l’humanité aura achevé, par la fédération ou l’impérialisme, son évolution politique, quel va être le dernier refuge de l’esprit novateur et révolutionnaire ? Une seule issue lui restera, le champ \emph{des expérimentations sociales} en grand, en vue de refondre l’humanité dans des moules tout neufs. Cette troisième période du monde humain sera certainement beaucoup plus intéressante pour la sociologue, sinon pour l’historien, que tout ce qui l’aura précédée. Alors commencera vraiment la prise de possession complète et systématique de la planète par l’homme et de l’homme par lui-même. Le problème de la population devenue stationnaire se présentera sous son aspect qualitatif et non plus numérique. A la culture extensive succédera la culture intensive du genre humain. Il s’agira de procéder à un élevage humain conforme au but général et d’élaborer un plan grandiose de réorganisation sociale et d’exploitation planétaire. Mais laissons aux prophètes le soin de détailler ces grands changements.\par
Quoiqu’il en soit de ces conjectures lointaines, revenons au présent. La grande question qui se pose à nous actuellement est celle de savoir, non si l’esprit d’association ira se développant, ce qui n’est pas douteux, mais bien si l’association ira s’unifiant, s’uniformisant, se centralisant, comme le suppose le socialisme, ou si elle ira se diversifiant et se  \phantomsection
\label{v2p420}compliquant dans une multiplication de sociétés multicolores. C’est cette seconde voie qui me paraît devoir s’offrir au déploiement libre de l’association. Et je n’en puis citer un meilleur indice que le tableau de l’Allemagne actuelle, berceau du socialisme contemporain. Elle est couverte, elle se couvre chaque jour davantage, d’associations, d’\emph{Unions} de tout genre, innombrables, inextricablement mêlées et touffues, qui font sa force expansive dans le monde. Par la discipline de ses armées, elle a conquis politiquement l’Europe ; par l’enrégimentation spontanée, et tout autrement variée, de ses producteurs, elle conquiert industriellement l’Univers\footnote{ \noindent Voir à ce sujet une brochure de M. Pierre Clerget, professeur à l’École de commerce de Loche (Suisse), intitulée \emph{les Méthodes d’expansion commerciale de l’Allemagne}, Lyon, 1901.
 }.\par
De tout temps, et partout, deux grands procédés se sont présentés aux peuples pour la conciliation de leurs intérêts opposés : leur convergence \emph{vers le futur} ou leur convergence \emph{vers l’extérieur}. Le premier s’est déployé : 1\textsuperscript{o} sous forme religieuse, en Égypte, où la préoccupation de la vie d’outre-tombe aplanissait bien des difficultés de la vie présente, et au moyen âge chrétien encore mieux ; 2\textsuperscript{o} sous forme utopique, dans l’Europe moderne, où la perspective d’un avenir merveilleux, dans ce monde même, soutient les courages et empêche les discordes d’éclater dans les rangs des néophytes. Le second procédé s’est épanoui : 1\textsuperscript{o} par la visée des conquêtes militaires, coloniales surtout ; 2\textsuperscript{o} par la visée des débouchés commerciaux. Or, on voit tous ces procédés à la fois, avec mille variantes, coopérer, dans l’Allemagne de nos jours, à l’adaptation économique. Le mouvement socialiste, à ce point de vue, y peut être considéré comme auxiliaire du militarisme qu’il complète en le combattant, et du christianisme aussi, et du capitalisme expansif, ambitieux comme lui\footnote{ \noindent Il y a une classe de producteurs, les agriculteurs, qui, loin de se sentir rapprochés des autres par la préoccupation commune du \emph{débouché extérieur}, se sentent toujours plus opposés aux autres. Il en est de même en Angleterre, aux États-Unis, partout. Les \emph{agrariens} forment un parti toujours plus distinct des \emph{industriels.} Ici se montre la nécessité de l’intervention de l’État pour trancher ce différend, autrement insoluble.
 }.\par
 \phantomsection
\label{v2p421}Le spectacle que nous donne l’Allemagne à cet égard n’est certainement qu’une ébauche de ce qu’est appelé à réaliser le monde civilisé tout entier dans un avenir prochain. On verra ainsi, à force de se multiplier et de se diversifier, l’\emph{association s’individualiser} pour ainsi dire, en ce sens qu’il y aura \emph{pour chaque individu} un certain entre-croisement d’associations différentes qui lui sera particulier, qui ne s’incarnera qu’en lui seul. La religion, qui est, avant tout, une grande association de croyants, ne pourra disparaître que le jour où elle aura été remplacée par une multitude d’associations diverses et entrelacées. Ne rêver qu’une seule grande association, comme le collectivisme, c’est ne souhaiter rien de bien nouveau, car les religions du passé, christianisme, islamisme, bouddhisme, ont réalisé ce rêve infiniment mieux que ne le pourra faire aucune école socialiste. L’avenir n’est pas là, il est à la diversité harmonieuse, à la multiplicité solidaire d’associations, associées en quelque sorte entre elles. Et ce jour-là, sera-ce précisément « l’irréligion » qui prévaudra, suivant la prédiction de Guyau ? On se conforme bien plus à sa véritable pensée en disant que l’association religieuse, alors, de plus en plus diversifiée et morcelée, tendant à accentuer l’\emph{individualisme religieux}, est destinée à prendre rang, finalement, et un rang éminent, parmi les autres formes d’associations qui enserreront l’individu, soutenu et non comprimé par elles.\par
Quand le tissu et l’entrelacement des associations multiples et diverses seront terminés, le faisceau national de ces unions individuelles constituera le système social le plus équilibré, le plus ingénieux à la fois et le plus simple dans sa complexité, qui aura paru depuis la dissolution du régime féodal. Celui-ci consistait en un tissu aussi, mais de petites  \phantomsection
\label{v2p422}sociétés à \emph{deux}, de petites mailles à la fois très solides et très courtes, dont chaque point n’était lié qu’à un autre point, son suzerain, lequel ne l’était lui-même qu’à son propre seigneur, en remontant jusqu’au roi. Dans le \emph{régime associationnistes} de l’avenir, il n’y aura plus d’hommages, plus de lien perpétuel et exclusif d’homme à homme, lien toujours plus unilatéral que réciproque ; il y aura une solidarité très étendue et toujours mutuelle, embrassant un grand nombre d’individus à la fois. Jamais la glèbe humaine n’aura été embrassée et consolidée par tant de racines à la fois. Et les nations alors seront une réalité plus concrète et plus vivante que jamais. Pendant que les associations privées, en se multipliant et s’enchevêtrant, verront par là s’atténuer l’\emph{esprit de corps} propre à chacune d’elles, cette grande association majeure, héréditaire et innée, la nation, qui les comprendra toutes toujours, ou presque toutes, verra s’accentuer son esprit de corps à elle, le patriotisme. L’intensité \emph{relative} de celui-ci s’alimentera de l’affaiblissement des autres, du moins dans les quelques grandes nationalités définitivement survivantes, et destinées à se fédérer.
 \phantomsection
\label{v2p423}\subsection[{III.7. La population. Esquisse d’une étude du problème de la population aux trois points de vue de sa propagation, de ses conflits belliqueux ou économiques, et de son adaptation aux circonstances géographiques ou historiques.}]{III.7. La population. Esquisse d’une étude du problème de la population aux trois points de vue de sa propagation, de ses conflits belliqueux ou économiques, et de son adaptation aux circonstances géographiques ou historiques.}\phantomsection
\label{l3ch7}
\noindent Le sujet de la population occupe en économie politique, comme en politique, comme en droit et en morale aussi bien, une place tout à fait à part. Et, dans chacune de ces sciences, il demande à être traité successivement sous les trois points de vue de la répétition, de l’opposition, de l’adaptation. J’aurais donc pu à la rigueur le rattacher aux deux livres qui précèdent de même qu’au dernier. Mais, à vrai dire, il est préférable de le détacher des trois et de ne lui consacrer qu’un court chapitre, puisque, après tout, abordée par ce côté, la psychologie économique est moins sociale que biologique.\par
La population tend à se multiplier par répétition \emph{héréditaire}, comme chaque forme du travail et de la richesse tend à se multiplier par répétition \emph{imitative}, et il importe de considérer d’abord les rapports que ces deux progressions, d’allure inégale, ont l’une avec l’autre. En second lieu, les rapports de deux populations inégales et inégalement croissantes ou décroissantes, s’imposent à l’attention non seulement de l’homme politique au point de vue des conflits belliqueux, mais encore de l’économiste au point de vue de la concurrence industrielle et commerciale entre les nations. Enfin, le taux numérique de la population n’est pas ce qu’il y a de plus important ; ses qualités distinctives, ses aptitudes plus ou moins marquées à l’exploitation des inventions industrielles régnantes et au succès dans le grand concours international, ont lieu de nous préoccuper encore davantage.\par
 \phantomsection
\label{v2p424}Tel est, dans toute sa complexité, le problème de la population.\par
On en a cependant exagéré ou mal compris l’importance, à notre avis, quand on a vu dans la tendance ascendante de la population « le moteur principal de l’évolution économique » (Kovalesky) et même de l’évolution sociale tout entière (Coste)\footnote{ \noindent Voir la \emph{Sociologie} de ce dernier. Voir aussi Loria, \emph{Problèmes...}
 }. Des deux séries de similitudes vitales et de similitudes sociales qui sont liées l’une à l’autre, des deux progressions ou des deux tendances à la progression qui font se multiplier les exemplaires d’un type vivant, d’une race ou d’un ensemble de races, et les exemplaires d’une \emph{espèce} de richesse, ou aussi bien d’une forme d’expression verbale, d’un mythe ou d’un rite, etc. ; de ces deux répétitions multipliantes, l’une par génération, l’autre par imitation, ce serait donc toujours la première qui pousserait ou entraînerait l’autre. Qu’il en soit ainsi le plus souvent, pas toujours, aux débuts de l’évolution sociale, c’est possible : on peut supposer qu’alors le génie individuel, quand il éclôt, ne se décide à inventer, à inaugurer quelque nouvelle forme, plus productive, de chasse ou de domestication animale, que sous la pression de la nécessité, sous la poussée du flot humain débordant. Encore est-ce méconnaître l’importance de la libre inventivité par jeu, si manifeste dans l’exubérance linguistique des primitifs. Mais, à mesure que le faisceau des inventions se grossit, l’explication des inventions nouvelles doit être demandée bien plutôt aux inventions antérieures dont elles sont la combinaison ingénieuse, qu’aux difficultés croissantes de la vie qui en sont la cause occasionnelle. En outre, à aucune époque, il n’est vrai que le niveau de la civilisation soit subordonné au taux de la population ; l’inverse est manifeste... A toute époque, le taux que la population ne saurait dépasser est déterminé par le niveau de la civilisation ; et la tendance de la population à croître est encouragée ou endiguée, stimulée ou paralysée, par l’état  \phantomsection
\label{v2p425}économique et social, dû à un groupe d’inventions coordonnées. Nous ne reviendrons pas sur cette considération déjà indiquée. Il suffit de rappeler que l’Irlande n’a commencé à voir croître sa population qu’après la découverte de la pomme de terre, et que la population de la Belgique n’est devenue si dense qu’après la découverte de ses mines de charbon et l’importation des inventions étrangères qui lui permettent de les exploiter.\par
Sans doute, en remontant jusqu’aux premiers inventeurs du régime pastoral ou du régime agricole et industriel, on voit que la nécessité de pourvoir aux besoins d’une population tendant à grandir au delà de ce que permettaient les ressources déjà connues a été pour eux l’aiguillon utile de leur génie inventif. Il fallait cette poussée et cette menace de la faim, soit ; mais il fallait aussi et surtout le génie individuel. Sans celui-ci, cette poussée menaçante a beau agir, nul progrès n’éclôt. Elle a seulement, le plus souvent, pour effet, le détachement de bandes guerrières qui, essaimant au loin, vont piller, ravager, coloniser. Et, d’autre part, même au sein de populations clairsemées, où ne se fait sentir nulle urgence de découvrir de nouveaux moyens de subsistance, on voit apparaître des inventions capitales, telles que celle de Watt (l’Angleterre d’alors était un des pays les moins peuplés), inventions prolifiques, s’il en fut, qui donnent un coup de fouet décisif à la population. Papin et Watt ont suscité des milliards d’hommes à l’existence. Mais, plus féconds encore, les premiers domesticateurs d’animaux ou de plantes ont été les grands procréateurs de l’espèce humaine.\par
Le génie inventif tient de la sorte le « génie de l’espèce » sous sa dépendance. Il en fait ce qu’il veut. Car il ne l’aiguillonne pas toujours, il l’arrête aussi parfois ou le force à rétrograder. Quand, dans une population très civilisée, déjà nombreuse, mais qui pourrait facilement s’accroître encore, comme dans la France actuelle, la natalité diminue peu à  \phantomsection
\label{v2p426}peu, cherchez bien, c’est toujours à des idées nouvelles, de plus en plus propagées, qu’est due cette régression. Le principe nouveau de l’égalité démocratique, qui a pour conséquence pratique la copie des besoins d’en haut par les classes d’en bas, l’ambition généralisée de se hisser le plus haut possible sur le mât de cocagne de l’avancement professionnel ou de la hiérarchie sociale, vainement niée de bouche, en s’allégeant le plus possible de tous les \emph{impedimenta} de famille : telle est, au fond, la principale cause de la dépopulation française\footnote{ \noindent Dans sa thèse intéressante sur les \emph{Idées égalitaires} (1899), M. Bouglé attribue en partie pour cause à la diffusion de ces idées la concentration urbaine de la population. Il serait encore plus vrai de dire que cette condensation est due à cette diffusion. L’esprit égalitaire a certainement beaucoup contribué à favoriser l’émigration des campagnes vers les villes — en même temps qu’à entraver ou arrêter le progrès de la natalité. — M. Arsène Dumont, dont tous les travaux sont à lire sur cette question de la Population, dit très bien : « C’est une vérité qu’on ne saurait trop redire, on a la natalité non de la classe sociale à laquelle on appartient mais de la classe à laquelle on voudrait appartenir. » (Essai sur la natalité dans le canton de Condé-sur-Noireau, en Calvados, \emph{Journ. de la Société de statist. de Paris}, 1900.)
 }.\par
Et la marche ascendante de la population est si peu le \emph{moteur principal de l’évolution économique}, que celle-ci n’a jamais marché d’un pas plus rapide, par une accumulation merveilleuse de découvertes et de perfectionnements, que depuis le ralentissement de la natalité, et dans les pays ou dans les provinces précisément où ce phénomène est le plus marqué, dans l’est des États-Unis, dans le midi de l’Angleterre et dans le nord de la France.\par
Cet erreur écartée, avons-nous aussi à combattre celle de Malthus, sur l’accroissement de la population, toujours plus rapide, d’après lui, que celui des subsistances ? M. Nitti, entre autres écrivains\footnote{ \noindent Voir aussi Henry Georges.
 }, en a fait une réfutation décisive. « Quoique dit-il, les subsistances (dans la Grande-Bretagne) se soient accrues autant et plus que la population, l’augmentation de la natalité s’est arrêtée tout à coup et les naissances diminuent sans cesse. En 1878, dans l’Angleterre et le pays de Galles,  \phantomsection
\label{v2p427}les naissances avaient été en proportion de 36,5 par 1000 habitants ; en 1880, elles n’étaient déjà plus que de 34,2 ; et, cinq ans plus tard, en 1885, elles descendaient à 30,5. Le même phénomène s’est produit en Écosse et en Irlande... » « En France aussi la richesse nationale s’est beaucoup plus rapidement accrue que la population\footnote{ \noindent \emph{La population et le système social} (trad. franç.). Voir aussi M. Paul Leroy-Beaulieu à ce sujet, qu’il a beaucoup contribué à élucider.
 }. » En Amérique, de même. La population totale de l’Europe et des États-Unis, de 1873 à 1885, s’est accrue de 14 p. 100, tandis que, dans le même intervalle de temps, la production totale du froment dans le monde s’est accrue de 22 p. 100 (Hector Denis).\par
Ce qu’il importe de remarquer ici, c’est que Malthus a été conduit à cette erreur grossière, et beaucoup d’autres après lui, par l’obsession du point de vue \emph{objectif.} S’il eût pris la peine de remarquer que les cellules cérébrales mènent le corps, y compris les organes de la génération, il eût compris que la clef du problème qui l’inquiétait était, avant tout, psychologique. A ce point de vue, on voit clairement que le désir de paternité est contenu ou refoulé par le désir du luxe ou du confort et que chaque nouveau besoin qui naît empêche de naître un enfant.\par
Courcelle-Seneuil, — qui mérite du reste un rang à part, et très élevé, parmi les économistes — développe une formule propre, d’après lui, à déterminer le lien constant et nécessaire qui existerait entre la production des richesses et le chiffre de la population dans un état social donné. « La somme des revenus annuels d’une société, divisée par la quantité de richesses dont la consommation est indispensable à un individu pour vivre, donne le \emph{maximum possible} de la population de cette société. Et cette population descend d’autant plus au-dessous de ce maximum qu’il y a plus de consommation au delà de ce qui est indispensable à un individu pour vivre. » C’est là « \emph{le chiffre nécessaire de population} ». « Représentant par \emph{p} le chiffre nécessaire de  \phantomsection
\label{v2p428}population, par \emph{r} la somme des revenus, par \emph{i} la somme des inégalités (c’est-à-dire ce qui, pour certains individus, plus ou moins nombreux, excède le minimum de consommation) et par {\scshape c} le minimum de consommation individuelle, la formule économique de la population sera : \emph{p = r-i/c}. » De cette formule se déduisent des conséquences importantes. « La population peut augmenter : 1\textsuperscript{o} par l’accroissement des revenus ; 2\textsuperscript{o} par la diminution des inégalités ; 3\textsuperscript{o} par l’abaissement du minimum de consommation. »\par
A l’inverse, ajouterons-nous, elle peut diminuer, soit, 1\textsuperscript{o} par la diminution du revenu ; soit, 2\textsuperscript{o} par l’accroissement des inégalités ; soit, 3\textsuperscript{o} par l’élévation du minimum de consommation, autrement dit par la complication des besoins.\par
Mais, dans cette formule, l’auteur à omis les influences les plus importantes, et c’est d’autant plus surprenant qu’il les indique ailleurs et qu’il en est souvent préoccupé, mais sans jamais les dégager assez nettement. Il part, ici, d’une \emph{somme donnée} de revenus. Mais pourquoi cette somme, plutôt que telle autre ? Il faudrait répondre. Il part aussi d’une certaine somme d’\emph{inégalités}, c’est-à-dire d’\emph{étalons de vie} très inégaux qui s’imposent aux diverses couches de la population et qui constituent pour chacune d’elles le \emph{minimum} effectif de consommation au-dessous duquel il ne vaut plus la peine de vivre ou du moins d’appeler des enfants à la vie. Mais pourquoi et comment ces conceptions si différentes de la vie et des moyens d’existence indispensables se sont-elles formées et s’entretiennent-elles si différentes, à côté les unes des autres ? Enfin, il part d’un minimum de consommation qui, à vrai dire, n’est que le \emph{minimum} de ces \emph{minima}, le plus bas d’entre eux ; et, à propos de celui-ci comme des autres, on doit se demander : Comment et pourquoi s’impose-t-il et est-il différent d’un pays à l’autre, d’un temps à l’autre ?\par
Courcelle-Seneuil à omis d’éclairer sa lanterne, et cependant il connaît bien l’une des lumières au moins sans lesquelles \phantomsection
\label{v2p429} le problème posé reste obscur : l’idée d’invention. C’est le groupe des inventions industrielles (ou politiques même) connues à un moment donné, qui, à ce moment, nous le savons, détermine le maximum possible de production et de population.\par
Supposons d’abord une nation close, dans une île par exemple, sans commerce avec les nations voisines. A chaque invention nouvelle qui permettra de mieux exploiter les ressources du sol au point de vue de l’alimentation, des vêtements, du chauffage, — quand, notamment, une plante nouvelle telle que la pomme de terre y aura été importée — le maximum de population que comportera à la rigueur cette île s’élèvera. Mais si, comme il arrive d’ordinaire, en même temps que ces inventions répondent à des besoins de première nécessité, des inventions nouvelles apparaissent qui font naître et propagent, par la facilité de les satisfaire, des besoins de seconde nécessité, des besoins voluptueux, des besoins de luxe, ne voit-on pas que ces derniers besoins vont entrer en lutte avec les premiers pour leur disputer une partie des ressources du sol et faire croître du tabac, par exemple, ou des fleurs, dans des terres qui pourraient produire du blé ? La question de savoir si, finalement, la population sera accrue ou sera diminuée par suite des inventions nouvelles, dépend donc du degré d’intensité et de généralité auquel parviendront les besoins de luxe, et de l’obstacle plus ou moins insurmontable qu’ils opposeront dans le cœur de la majorité des habitants au besoin de maternité ou de paternité. C’est une question, donc, de \emph{mesure des désirs antagonistes}.\par
Cette même population insulaire, par suite de principes philosophiques découverts et propagés par quelque Jean-Jacques Rousseau indigène, ira se démocratisant. Est-ce que la diminution des inégalités de consommation jugée indispensable dans les diverses couches de la population aura nécessairement pour effet d’élever le maximum de population \phantomsection
\label{v2p430} ? Non, il se peut, au contraire, — cela découle même des lois de l’imitation — que l’envie imitative des besoins d’en haut élève le minimum de consommation des couches les plus inférieures au niveau du minimum des couches supérieures, et non \emph{vice versà}. Il pourra donc s’ensuivre une tendance à l’abaissement de la natalité.\par
Toutefois, cet abaissement de la natalité, ce déclin de la population, par l’effet des causes indiquées, pourra-t-il se poursuivre indéfiniment ? Non. Une chose que Courcelle-Seneuil n’a pas vue et qui me semble importante, c’est qu’il y a ici à marquer non pas un \emph{maximum} seulement, mais un \emph{minimum} de population, \emph{dans un état donné des inventions et des besoins}, et que, entre les deux, oscillent le flux et le reflux de la natalité. En effet, si, à mesure que les besoins de bien-être, de luxe, d’art, d’amour-propre, se développent, la lutte entre eux et le besoin de génération devient plus vive et finit par tourner à l’avantage des premiers (d’où arrêt de la population parvenue à son maximum, puis décroissance) ; il vient un moment à l’inverse, où, à force de diminuer, la population tombe au-dessous du degré de densité indispensable à l’entretien des industries de luxe et d’art (de locomotion d’abord, de communications mentales rapides), des industries de confort et de bien-être, dont le besoin s’est répandu et enraciné ; et alors, loin d’être combattu par ce besoin, le désir de génération est stimulé par eux. D’où reprise de la natalité.\par
Nous avons supposé un État clos, sans relations extérieures. Mais renversons ces barrières de douanes, qui, dans la réalité des choses, n’existent jamais aussi escarpées. Les États sont en perpétuelles relations de voisinage. Qu’arriverait-il si une nation civilisée dont la population viendrait à décroître ne sentait pas la stimulation du besoin de reproduction par les besoins mêmes de luxe et de bien-être qui lui ont fait obstacle autrefois ? Il arriverait que les populations voisines immigreraient peu à peu en elle, comme on fait déjà chez  \phantomsection
\label{v2p431}nous, pour remplir les places vides de ses ateliers, de ses chantiers et alimenter ses industries. D’une manière ou d’une autre, [{\corr le}] déclin de la population serait arrêté.\par
Il y a donc un \emph{minimum} comme un \emph{maximum} de population inhérent à \emph{chaque état des connaissances, à chaque faisceau d’inventions et de découvertes civilisatrices}.\par
Mais n’y a-t-il pas aussi, entre ce maximum et ce minimum, un point d’élection qui marque l’\emph{optimum} de population, le point de \emph{prospérité maxima} où bat son plein la fortune d’une nation ?\par
La question est plus complexe qu’il ne semble d’après ces aperçus. Il ne s’agit pas seulement de savoir quel est le meilleur parti qu’une \emph{race donnée, sur un territoire donné}, peut tirer des ressources du sol ; il s’agit, en somme, dans les grandes luttes de l’histoire, de savoir \emph{quelle est la race qui tirera le meilleur parti d’un sol donné ;} de là les guerres. Et, en prévision des guerres et des conquêtes, il s’agit pour chaque nation de savoir comment elle défendra son territoire, comment elle se défendra elle-même contre les prétentions d’une nation voisine ou lointaine qui se prétendra orgueilleusement plus apte qu’elle à exploiter les richesses de son sol natal...\par
Or, à \emph{chaque état des inventions militaires}, à chaque progrès de l’art militaire, correspond un \emph{minimum de population} nécessaire à la défense nationale (ce minimum s’est considérablement élevé depuis l’invention de la poudre et des armes à tir rapide et l’accroissement des États centralisés).\par
Remarquons que la défense et la conquête comportent un \emph{minimum} de population, variable d’après l’état de l’art militaire, mais ne \emph{comportent pas de maximum.} Plus les États sont peuplés, plus ils sont forts militairement.\par
Le minimum de population requis au point de vue de la force militaire peut être tantôt au-dessus, tantôt au-dessous du minimum de population exigé par le maintien du  \phantomsection
\label{v2p432}niveau de civilisation. Dans un très grand État, il peut être au-dessous ; dans un très petit État il est au-dessus. Un très grand État, tel que l’Empire romain, est donc exposé à voir sa population descendre plus bas que le minimum requis pour le maintien de sa civilisation, car nulle préoccupation militaire ne lui interdit de s’arrêter avant cette limite dans la voie du dépeuplement. Et, de fait, ce n’est pas faute d’hommes que l’Empire romain a été envahi par les Barbares : sa population eût été bien suffisante si elle eût conservé le goût des armes. — Quoi qu’il en soit, les deux \emph{minima} dont il s’agit ne coïncident pas, ce qui complique étrangement le problème de la population.\par
C’est la nation (ou, dans chaque nation, la couche de population) la plus adaptée à l’exploitation des inventions régnantes, — militaires ou civilisatrices, — qui pullule le plus. De là, peu à peu, l’expansion coloniale ou conquérante de cette nation et la prédominance interne de cette couche, prédominance qui a pour effet de modifier ce qu’on appelle la \emph{race}, amalgame de races différentes. — Peut-on dire, à l’inverse, que les inventions, militaires ou civilisatrices, les plus adaptées à la mise en relief et en vigueur de la nation régnante, ou, dans chaque nation, de la classe dominante, vont se multipliant ? Cela n’est vrai que dans une bien faible mesure ; car les inventions n’obéissent pas au commandement, et \emph{le génie souffle où il veut}, poussé par une logique spéciale plutôt qu’appelé par les gouvernements.\par
Cet examen de la théorie de Courcelle-Seneuil vient de nous forcer à considérer le problème qui nous occupe sous les trois aspects qui nous sont familiers. C’est qu’en effet il est impossible de séparer ici les questions soulevées par la propagation humaine, de celles que soulève la lutte des nations ou l’amélioration et la dégénérescence de leur race. Voici pourquoi. En vertu des considérations qui précèdent, il est clair que, à supposer même les nations closes et séparées les unes des autres, par de grands espaces,  \phantomsection
\label{v2p433}comme les astres dans le firmament, l’accumulation des inventions, par conséquent la complication des industries et des besoins dans chacune d’elles ou dans quelques-unes d’entre elles, ont pour effet inévitable d’élever sans cesse le minimum de population requis pour l’entretien du degré de bien-être accoutumé. Le besoin croissant que les hommes ont les uns des autres, par suite de la mutuelle contagion de leurs désirs, pousse au progrès de la population, c’est-à-dire, d’une part, à sa densité croissante et à l’organisation des villes, d’autre part à la multiplication des centres urbains et à l’extension de leur rayon d’influence. Il s’ensuit que, à force de grandir ainsi, les nations les plus murées finiront par se toucher et se heurter, par se battre ou se fondre ensemble. Aussi ne puis-je que souscrire à ces paroles de Nitti que « il n’est plus possible aujourd’hui (je dirai même jamais) de circonscrire le problème de la population dans les limites étroites d’une question purement nationale », car « étant donné un système d’économie internationale (terme nécessaire, ajouterons-nous, de l’évolution économique), ce problème devient un problème universel ».\par
De là l’extrême difficulté, la difficulté toujours grandissante, de la résoudre. Ni au point de vue du nombre, en effet, ni au point de vue de la qualité, le problème de la population ne se pose par rapport à la répétition et à l’adaptation économique, comme relativement à l’opposition économique et surtout militaire. — Karl Marlo\footnote{ \noindent \emph{L’Œuvre de Karl Marx}, par Edgar Allix, 1898.
 }, qui, comme tous les économistes, se place au premier point de vue seulement, a cherché à formuler les rapports entre la population et la production. Encore n’a-t-il égard qu’à la quantité. Il a cherché quel est le \emph{chiffre normal} de la population. « Normalement peuplés, dit-il, sont les pays dont les habitants rendraient moins bonne leur situation économique, soit en multipliant, soit en restreignant leur nombre. » Allix, précisant ce point, dit que, d’après lui, « la population a atteint  \phantomsection
\label{v2p434}un juste degré lorsque la masse des forces du travail qu’elle contient est en rapport normal avec la masse à peu près invariable des forces naturelles du pays. » — Il ne s’est pas aperçu que, si, parvenu à ce chiffre de population, un pays a intérêt, au point de vue du maximum de produit et du minimum d’effort, à s’y arrêter, il peut au contraire, au point de vue de la lutte militaire et même industrielle avec d’autres pays, avoir un intérêt majeur à le dépasser. Il ne s’est pas demandé non plus quelles sont les \emph{aptitudes} que, par sélection naturelle ou sociale ou par éducation, \emph{la meilleure adaptation de la population à l’emploi des forces du pays} doit développer. — Et il se serait demandé, en étudiant ce problème, si les aptitudes les plus propres à obtenir cette utilisation \emph{optima} ne sont pas toujours celles qui favorisent le mieux le triomphe militaire, \emph{et partant économique}, de la nation dans son conflit possible avec les nations rivales. Rien, donc, de plus complexe ni de plus ambigu que \emph{le} ou plutôt \emph{les} problèmes de la population, et l’impossibilité de les résoudre autrement que par des à peu près et des cotes mal taillées, comme du reste la plupart des questions sociales.\par
Le développement de la population, la concurrence économique, — j’ajoute, militaire, politique, linguistique, religieuse, morale, esthétique, — des populations, et l’adaptation \emph{de la} population à son idéal propre, ou \emph{des} populations à leurs fins réciproques : ce sont là trois grands sujets de méditation inquiétants qu’un sociologue et un homme d’État ne sauraient oublier ni désunir. Nous savons que ces trois termes, répétition, opposition, adaptation, s’impliquent indéfiniment.\par
C’est parce qu’une race, une variété nationale, et historiquement formée, de l’espèce humaine, est adaptée à son genre de vie, à sa civilisation particulière, et a triomphé de races moins adaptées, qu’elle est prolifique et se répand au loin ; et c’est en se répandant qu’elle se heurte ou s’unit à d’autres variétés ethniques, en batailles sanglantes ou en  \phantomsection
\label{v2p435}croisements féconds. De même, c’est parce qu’une forme de civilisation est adaptée aux besoins d’une ou de plusieurs races, et c’est parce qu’elle s’est substituée à des civilisations moins adaptées, qu’elle se propage ; et c’est en se propageant qu’elle se heurte ou s’allie avec des civilisations différentes, ou à la fois se heurte et s’allie, s’allie en se heurtant, avec elles, comme, par exemple, aux troisième et quatrième siècles de notre ère, les idées religieuses de la Judée avec les idées métaphysiques des Grecs à Alexandrie. — Tout ceci est vrai, non seulement des civilisations prises dans leur ensemble, mais de chacun de leurs éléments, langue, religion, droit, industrie, morale et art, et aussi bien de chacune des notions, de chacune des institutions ou des types d’actions, qui constituent chacun de ces éléments.\par
On le voit, la population et le type social sont les deux termes d’une adaptation réciproque, mais la réciprocité est loin d’être parfaite. Plus on remonte haut dans le passé, et plus le destin du type social paraît subordonné à celui de la population à laquelle il s’adapte bien plus qu’elle ne lui est adaptée ; plus, à l’inverse, on descend du passé vers le présent et l’avenir, et plus le destin de la population, son débordement ou son tarissement, dépend du type social auquel il est nécessaire qu’elle s’adapte bien plus qu’il n’a besoin de s’adapter à elle. Né d’une race et pour une race, un type social se propage bien au delà d’elle, et s’impose à d’autres races qui s’accommodent à lui comme elles peuvent ou disparaissent, pendant qu’il ne se plie et ne s’assouplit que très peu à leurs inclinations. — En somme, à l’origine, une race se fait son type social ; à la fin, un type social se fait sa race ou ses races. C’est le triomphe graduel de l’esprit sur la vie.\par
La question de la population n’est donc pas une simple question de subsistance, comme le supposait Malthus, en simplifiant extraordinairement ce problème. Ou plutôt, pour qu’une population, en se répandant, trouve toujours des vivres en abondance, il faut d’abord qu’elle possède ou  \phantomsection
\label{v2p436}qu’elle acquière les aptitudes requises par les nouvelles formes de production des richesses. De là, comme il a été dit plus haut, la multiplication des Anglo-Saxons, plus propres que les nations néo-latines, par leur force et leur fixité d’attention, à la surveillance des machines. En second lieu, une population tant soit peu civilisée ne se propage jamais autant que le lui permettrait à la rigueur la quantité d’aliments dont elle dispose. Il ne lui suffit plus de ne pas mourir de faim. Les exigences de confort et de prévoyance que son type social lui impose, à chaque époque, limitent sa propagation numérique. C’est non seulement à ses besoins présents de plus en plus nombreux et variés, mais à ses besoins futurs ou à ceux de sa postérité de mieux en mieux prévus qu’elle désire pourvoir.\par
Une série de statisticiens avaient, de Quételet à Bertillon, à Bodio, à von Mayr, établi un lien entre les variations du prix des blés et celles des mariages. La hausse des blés s’accompagnait d’une diminution du nombre des mariages et \emph{vice versa\footnote{ \noindent M. Hector Denis a fait aussi des recherches tendant à démontrer que, en Belgique, il y a un certain nombre d’années, le chiffre des mariages avait varié comme le prix du charbon de terre, montant et descendant avec ce prix...
 }}. Mais, voici que, depuis nombre d’années, le prix du blé a baissé plus que jamais, par suite des importations de froment américain, et cependant la \emph{matrimonialité} n’a pas cessé de s’affaiblir aussi (sauf un léger relèvement, en France du moins, dans les deux ou trois dernières années). La raison de cette différence avec le passé, si l’on en croit M. Hector Denis, c’est qu’il s’agissait autrefois du prix du blé \emph{indigène} et à présent du prix du blé \emph{exotique}. L’observation a sa portée, en ce sens que la baisse du prix du blé indigène était un indice assez sûr de bonne récolte générale, de prospérité agricole dans le pays, tandis que, à présent, le bas prix du blé américain ne prouve pas le moins du monde que les agriculteurs de France ou d’Angleterre aient eu « une bonne année ». Mais il n’en est pas moins vrai que  \phantomsection
\label{v2p437}c’est là un échantillon des erreurs auxquelles les constatations de la statistique peuvent nous conduire quand on les généralise sans les avoir analysées dans leurs causes psychologiques.\par
La vraie influence qui agit sur la tendance au mariage, est l’espérance plus ou moins vive en l’avenir, espérance influencée elle-même par toutes sortes d’apparences, trompeuses ou vraies, parmi lesquelles une bonne récolte joue un rôle important, mais de moins en moins important à mesure qu’on s’élève en civilisation. Par le progrès de l’instruction, qui étend sa prévoyance dans l’espace et dans le temps, le dernier paysan même ne restreint plus à une seule année ses perspectives et ne se laisse ni abattre ni rassurer pleinement par une mauvaise ou une bonne récolte. Pour qu’une bonne récolte exalte si fort l’espoir d’un jeune cultivateur qu’il songe aussitôt à se marier, il faut qu’il soit resté bien enfant.\par
Une statistique, bien plus profonde que la précédente, a montré un rapport inverse des plus frappants, et conforme à ce qui vient d’être dit, entre le nombre des livrets de caisse d’épargne ou des polices d’assurance et celui des enfants par ménage, dans les divers départements français. Il s’ensuit que le développement de la prévoyance est contraire à celui de la natalité. Mais ce que la statistique ne dit pas et ce qu’il est nécessaire d’ajouter pour prévenir une erreur assez fréquente, c’est que la prévoyance dont il s’agit n’est pas purement égoïste. Ne confondons pas deux choses différentes que la statistique ici nous présente pêle-même : la prévoyance de l’individu qui ne songe qu’à son bien-être futur, et la prévoyance du père de famille qui songe à assurer le bien-être de ses enfants, à l’assurer de plus en plus ou à le rendre de plus en plus grand. Tout n’est pas blâmable dans les motifs de notre dépopulation : on a, en France, de moins en moins d’enfants, en grande partie parce qu’on aime de plus en plus ses enfants, qu’on  \phantomsection
\label{v2p438}se préoccupe toujours davantage de leur avenir. Une natalité copieuse, en certains pays, révèle simplement l’égoïsme paternel, l’imprévoyance insensible et autoritaire du \emph{pater familias.}\par
Ce n’est pas seulement l’esprit de prévoyance qui est contraire à la natalité, c’est aussi l’esprit d’entreprise, qui semble cependant à première vue être opposé au premier, mais, en fait, se développe en même temps que lui parmi les civilisés. Le progrès de la civilisation, chose étrange, à mesure qu’il déploie l’esprit d’entreprise et d’audace sous sa forme politique, industrielle, commerciale, l’atrophie sous sa forme naturelle et vitale. Il y a là compensation en un sens, mais non au point de vue de la conservation nationale.\par
L’homme d’État patriote blâme fortement ces calculs malthusiens des pères de famille trop tendres. Il a raison, car le salut de la patrie, de la civilisation nationale, à l’heure de la guerre, exige en temps de paix cette témérité d’imprévoyance qui multiplie les hommes en vue de leur mutuelle extermination. S’il y a une \emph{surpopulation}, le problème se résoud ainsi, par les combats, comme le problème de la surproduction industrielle se résout par les crises commerciales. C’est à réaliser son vœu le plus profond, son vœu de durée, qu’une population doit avant tout être adaptée, encore plus qu’à la satisfaction de ses besoins de confort et de luxe, et elle ne l’est qu’à la condition de se multiplier bien au delà de ce que la satisfaction de ses besoins de luxe et de confort lui permet. D’autre part, les aptitudes corporelles et mentales que requiert le triomphe dans la lutte belliqueuse ne sont pas celles qui sont exigées pour le succès dans la concurrence industrielle ou intellectuelle. Quelquefois les unes nuisent aux autres. Il faut donc opter, par suite ; opter entre des avantages et des qualités dissemblables, sans commune mesure. En fait, cette difficulté inextricable est tranchée par quelque cote mal taillée,  \phantomsection
\label{v2p439}par un sacrifice partiel et toujours inégal des aptitudes diverses qui se contrarient ; les chefs de famille en majorité, les chefs de nation, aiguillent tantôt à droite, tantôt à gauche, suivant le vent de l’opinion qui souffle, l’évolution de la race. Rien de plus aléatoire.\par
Si l’on ne considère que l’adaptation d’une population à ses fins économiques dans un État donné et pris à part, il suffit de la \emph{laisser faire} et de la \emph{laisser passer} pour ainsi dire. Par le taux changeant de sa natalité dans ses diverses classes, qui deviennent moins prolifiques là où elles deviennent moins utiles, — et par les migrations de ses diverses fractions à l’intérieur du pays, — elle finit par trouver d’elle-même la meilleure voie pour son meilleur emploi. Les populations s’adaptent en se mouvant, comme les richesses en s’échangeant. Mais cela est surtout vrai des migrations intérieures, comme l’a très bien montré Bücher. Il y explique et y justifie de la sorte l’émigration actuelle des campagnes vers les villes, et, par sa comparaison avec une émigration pareille qui a eu lieu au moyen âge\footnote{ \noindent Sur les mouvements de la population, sur les immigrations et émigrations comparées du présent et du passé, en vue d’un ajustement meilleur de la population à ses conditions d’existence, sur la question de savoir si la population, en se civilisant, devient de plus en plus mobile (ce que nie Bücher), il faut lire le chapitre final de ses \emph{Études d’histoire de l’économie politique.}
 }, en dégage la vraie raison d’être. Au moyen âge se produit, non, comme de nos jours, une poussée générale vers quelques grandes villes, mais, dans la banlieue de chaque petite ville, — il n’y en avait ni ne pouvait y en avoir de grandes alors — un mouvement de la campagne environnante vers les métiers urbains qu’elle envahit ; jusqu’au moment où, au {\scshape xvi}\textsuperscript{e} siècle, « tous les métiers que pouvait faire vivre dans les villes l’étendue restreinte de leur débouché y furent représentés et pourvus du nombre suffisant de maîtres ». De nos jours, une \emph{économie nationale} ayant succédé à cette \emph{économie urbaine}, les courants de la population sont moins nombreux et plus  \phantomsection
\label{v2p440}larges ; mais n’est-il pas certain qu’ils s’arrêteront aussi, quand ils auront achevé l’œuvre d’adaptation économique qu’ils poursuivent ? Toutefois, si rassurante qu’elle soit économiquement, cette perspective ne laisse pas d’être inquiétante politiquement, si l’on songe à la nécessité d’une forte proportion rurale de la population en vue des fatigues de la guerre. Mais il se trouve que le service militaire contribue pour sa forte part à accélérer l’émigration des champs vers les grands centres, qui est contraire aux fins militaires.\par
Si l’on essaie de se placer à un point de vue tout théorique, qui domine l’ensemble des populations du globe, si l’on se demande comment se poserait, pour un sociologue international, le problème de la population, il semble qu’il se simplifie en s’amplifiant de la sorte ou se prête à des solutions plus rationnelles. Il est clair que, à ce point de vue, nos trois termes, répétition, opposition, adaptation, sont loin de pouvoir être mis sur le même rang. Le second se présente comme devant être subordonné aux deux autres. L’idéal serait une humanité divisée en races, en variétés ethniques, qui présenteraient un minimum d’opposition entre elles, et un maximum d’adaptation à leur type de civilisation.\par
Mais, précisément, c’est en essayant de résoudre le problème ainsi posé, qu’on voit se justifier l’existence des nations distinctes et la persistance de leur patriotisme en ce qu’il a de conservateur et de défensif. Si jamais une seule race parvenait à couvrir le globe, déluge nouveau dont nous menace la conquête anglo-saxonne, rien ne serait plus périlleux pour la paix du monde. Ce sont toujours les semblables qui s’opposent. Tant d’ambitions et d’avidités similaires en contact ne tarderaient guère à entrer en conflit. Il importe donc à la pacification même de l’Univers, pour qu’elle soit durable, que le refoulement des races vaincues par une race conquérante ne soit jamais complet, ni même la submersion des civilisations inférieures sous la civilisation triomphante.  \phantomsection
\label{v2p441} Si l’ère des guerres peut se clore, comme il y a lieu de l’espérer, ce sera par la fédération de grandes nationalités, non par l’empire d’une seule. L’impérialisme a pu être, il y a deux mille ans, le seul procédé possible de pacifier les peuples en les broyant. A présent il n’est plus que le masque transparent d’un monstrueux despotisme doublé d’une rapacité collective et gigantesque, pieuvre immense dont les tentacules ne manqueraient pas de se déchirer les unes les autres en se multipliant à leur aise. Le patriotisme, donc, même en ses aberrations, mérite le respect des pacifiques, et ils doivent se garder par-dessus tout d’opposer leur cause à la sienne. Les patriotes qui, avec amour, prennent sous leur protection leurs originalités nationales, éléments nécessaires d’une harmonie internationale dans l’avenir, travaillent à la grande paix de demain et d’après-demain avec autant d’efficacité peut-être, non seulement que des diplomates assis dans un congrès de désarmement, mais même que tous les trains de marchandises et tous les vaisseaux marchands qui tendent à assimiler les besoins des peuples, et que tous les livres ou tous les journaux qui tendent à ensemencer des mêmes idées toutes les nations. Il y a un point où cesse d’agir la vertu pacificatrice de l’assimilation et où, en se prolongeant, elle deviendrait la source de nouveaux combats. Pour résister au torrent, fécond d’abord, dévastateur ensuite, de l’imitation de peuple à peuple, il est bon d’élever des digues qui ne l’empêchent jamais de couler, mais quelquefois de noyer désastreusement telles fleurs infiniment rares, d’un inestimable prix, que l’histoire a mis des siècles à faire éclore, un génie national. Et si la paix future était autre chose qu’une gerbe de ces fleurs, vaudrait-il la peine de la rêver ?\par
Mais, puisqu’il en est ainsi, on voit que nous nous sommes abusés si nous avons cru échapper, en posant le problème de la population sous une forme internationale, aux difficultés dont il se hérisse pour le patriote. Il faut toujours en revenir, fût-on le sociologue le plus cosmopolite, à le formuler \phantomsection
\label{v2p442} en des termes qui mettent aux prises le désir d’avoir une population assez nombreuse et assez aguerrie pour défendre la terre des aïeux, avec le désir d’avoir une population assez industrieuse et paisible et assez peu nombreuse pour jouir des bienfaits de la civilisation qu’elle doit à la fois à ses ancêtres et à ses contemporains. Comment sortir de cet embarras, comment résoudre ce problème autrement que par une préférence irrationnelle et un risque couru ?\par
Même abstraction faite des préoccupations belliqueuses, il n’est pas facile de répondre à cette autre question : qu’est-ce qui \emph{vaut} le mieux (car la notion de \emph{valeur}, ici encore, s’impose à nous) : une population plus nombreuse mais moins heureuse, ou plus heureuse mais moins nombreuse ? Le dilemme est entre la répétition et l’adaptation au fond, car \emph{nombreux} veut dire \emph{répété}, et \emph{heureux} veut dire \emph{adapté} à son sort. — On pourra demander aussi bien : qu’est-ce qui vaut le mieux, en fait de population, le nombre ou l’aptitude au progrès, le nombre ou le génie inventif ?\par
M. Nitti émet ce théorème que « tout pays qui, dans la forme actuelle de sa constitution économique, est capable de faire subsister un nombre donné d’habitants, pourrait en nourrir un nombre beaucoup plus considérable si la forme de sa constitution venait à changer dans le sens d’une plus grande division de la richesse produite\footnote{ \noindent On m’excusera de remarquer, à ce propos, que M. Nitti, dans l’ouvrage d’où j’extrais cette citation (\emph{La population et le système social}, trad. fr., Giard et Brière, 1897), rend fréquemment hommage à la « loi de l’imitation ». J’y trouve souvent cette loi invoquée comme importante, comme irrésistible. — « Il a été démontré, comme nous l’avons déjà vu, — dit-il notamment — qu’une des lois les plus inéluctables de la société, une de celles qui ont le plus contribué au progrès de la société, mais qui peut devenir aussi une cause de décadence, est la \emph{loi de l’imitation.} » Deux pages plus loin, à propos du luxe des classes supérieures, il ajoute : « Naturellement, par l’effet de cette terrible loi de l’imitation qu’aucune société ne parvient à éviter, le peuple ne peut se soustraire à l’influence de la contagion, et, quand la richesse augmente, il sent beaucoup moins le besoin de participer aux biens idéaux de la civilisation et de la culture intellectuelle, que celui d’imiter les classes riches dans leurs dépenses improductives et nuisibles, dans les dépenses qui flattent la vanité et ajoutent aux jouissances sensuelles. » Et ailleurs... « Tout raffinement dans la sphère esthétique, ayant le luxe pour conséquence, \emph{et se propageant de haut en bas et de par la loi de l’imitation dans les sociétés démocratiques}, détermine toujours un plus fort individualisme » (p. 175), etc. — D’après un bref compte rendu, que je viens de lire dans l’\emph{Année sociologique} (1901), d’un ouvrage de M. Schmoller qui a récemment paru (c’est le premier volume de son grand \emph{Traité d’Économie politique}, impatiemment attendu), je crois pouvoir me féliciter d’être en communauté d’idées avec lui aussi sur le point dont il s’agit, et, en général, sur la nécessité de remonter aux sources psychologiques des faits économiques. Je regrette que mon ignorance de l’allemand me prive du profit que j’aurais eu certainement à lire ce livre.
 } ». — Cela semble  \phantomsection
\label{v2p443}probable, en effet, et c’est conforme à la formule de Courcelle-Seneuil ci-dessus discutée. Mais, cela fût-il démontré, s’ensuivrait-il nécessairement que le progrès indéfini dans la division de la richesse, dans le nivellement des fortunes, fût un avantage social ? J’admets que l’égalité parfaite des fortunes marque le point où aurait lieu le maximum numérique de population. Mais celui-ci coïnciderait-il avec l’apogée qualitative ? Et cette médiocrité généralisée, si elle se prolongeait, ne ferait-elle pas perdre au nombre même sa seule raison d’être, à savoir l’élite, la fleur supérieure de la population, sans laquelle on ne voit pas pourquoi la nature se mettrait en frais de tant de millions d’individus platement égaux ?\par
C’est qu’en effet, quand on étudie à fond ce sujet, on s’aperçoit de l’impossibilité de concilier logiquement, de classer hiérarchiquement ces trois points de vue, répétition, opposition, adaptation, si on ne les éclaire ensemble par une autre idée à laquelle les trois précédentes semblent suspendues et subordonnées, l’idée de variation. Il semble qu’il s’agisse, avant tout, pour la grande œuvre qui s’élabore par le fonctionnement des procédés répétiteurs, belliqueux, harmonisants, de faire jaillir ainsi, à chaque instant, des diversités individuelles, des physionomies caractérisées, des singularités uniques, qui ne se répètent jamais, puisqu’elles sont uniques, qui ne s’opposent à rien, puisqu’elles sont dissemblables à tout, qui ne s’adaptent à rien, puisqu’elles sont l’utilité et l’indépendance mêmes, — l’inutilité qui  \phantomsection
\label{v2p444}utilise tout, à peu près comme le Dieu d’Aristote est l’immobile qui meut tout. Ces choses frappantes et fugaces que chaque instant emporte et que le suivant ne rapporte plus, ne sont-elles pas ce qui donne un sens à la marche du temps, à l’évolution sa signification ? Que serait une succession de phénomènes où cet élément ferait défaut, où manquerait cet imprévu nécessaire, cet accidentel essentiel ? — Eh bien, pénétrons-nous de l’importance de cette réalité hors cadres et hors ligne, et nous n’aurons pas de peine à comprendre la vraie solution du problème qui nous occupe. La progression numérique des populations, leurs luttes, leurs alliances même et leur prospérité laborieuse, ne sont un bien que dans la mesure où elles favorisent l’éclosion d’individualités sinon plus tranchées, du moins plus profondes et plus délicates en même temps. On ne saurait donc désirer que la population humaine aille toujours en augmentant, si pendant qu’elle s’accroît elle s’uniformise. Son arrêt, son recul numérique même, sont dans le vœu des choses si elle se différencie en décroissant\footnote{ \noindent Buffon prétendait que la population humaine, prise dans son ensemble, n’augmentait ni ne diminuait d’une époque à une autre. Sur quoi fondait-il cette assertion, à défaut de statistiques ? Sur rien. Mais il est remarquable de voir ici se marquer cette préférence \emph{a priori} pour l’immobilité et le stationnement qui caractérise le {\scshape xvii}\textsuperscript{e} siècle, auquel Buffon appartient encore par sa majestueuse tournure d’esprit. Nous, \emph{a priori}, nous sommes portés à admettre, au contraire, que la population du globe doit aller progressant toujours... Et notre croyance, au fond, n’est pas plus justifiée que la sienne...
 }. Un certain degré de richesse et d’inégalité dans la richesse, étant nécessaire pour mettre en relief les diversités individuelles, il ne faut pas que l’accroissement d’une population marche plus vite que son enrichissement. Mais, un certain chiffre de population étant nécessaire pour l’épanouissement complet des originalités virtuelles d’une race, il ne faut pas non plus que son progrès en richesse et en bien-être la fasse descendre au-dessous de ce taux. Il est bon que la population soit adaptée à son climat et à sa civilisation, afin qu’elle reste féconde ; mais il  \phantomsection
\label{v2p445}n’est pas bon que cet accord soit à ce point parfait d’exclure toute innovation. Il est bon enfin que l’esprit de la population soit adapté à l’Univers extérieur par cette sorte d’adaptation supérieure qui se nomme Vérité ; mais il n’est pas bon que cet ajustement de la pensée aux choses aille jusqu’à tuer toute illusion, là où le mensonge de l’espérance, si c’en est un, est la condition de la paternité.\par
Il s’agit, avant tout, de susciter dans le cœur de l’individu, pour le déterminer à devenir père et à sacrifier son bien-être à son rêve, son bonheur à son espoir, un vif amour pour quelque chose d’idéal. Quel sera ce puissant attrait ? Je crains bien que la pensée de propager la langue française, les institutions françaises, la nationalité française, ne puisse jamais être, dans le cœur de la majorité des Français, un motif assez fort pour les décider à ce sacrifice. La France est un objet trop vaste pour que l’amour de la France, si énergique qu’il soit, suffise à rendre le Français prolifique. Chaque individu se dit que, en échange de ses privations, sa part contributive à la propagation du nom français moyennant une natalité abondante, sera une quantité infinitésimale. Aussi est-ce seulement le culte de la famille, le désir de perpétuer le foyer, qui, dans l’avenir le plus lointain comme dans le passé le plus antique, restera, en somme, le mobile vraiment approprié à l’accomplissement spontané du devoir suprême de procréation. A défaut de ce sentiment, un calcul, qui s’y joint d’ordinaire, peut le suppléer, aussi longtemps que subsiste dans la législation et les mœurs quelque chose de l’antique piété filiale. Grâce à la piété filiale érigée en maxime supérieure et consacrée en loi, chaque père sait qu’il sera payé, et au delà, du mal qu’il se donne, des privations qu’il s’impose pour ses enfants ; et ceux-ci à leur tour, en se dévouant à leur postérité, feront un \emph{placement} pareil. Telle est la principale forme du crédit et aussi bien de la réciprocité des services dans les sociétés primitives ; elle se fonde sur une compensation de sacrifices \phantomsection
\label{v2p446} successifs, et non de services simultanés. La réciprocité des services simultanés ne commence qu’à partir du moment où l’échange et le commerce se développent par la division du travail. Cette forme \emph{économique} de la réciprocité des services est propre aux sociétés avancées, et elle comporte des développements sans fin que l’autre ne comporte pas. Aussi, à mesure que la mutuelle assistance simultanée, l’échange dans l’espace, va grandissant, la mutuelle assistance successive, l’échange dans le temps, le prêt incertain et périlleux d’une génération à l’autre, recule incessamment, et le devoir de piété filiale cesse d’être élevé si haut par les moralistes et les législateurs. Il doit donc venir fatalement un moment, si l’on n’y prend garde, où les hommes, de plus en plus assurés de l’assistance de leurs contemporains, et de moins en moins de l’assistance de leurs enfants, auront de moins en moins recours à cette dernière, qui était jadis la seule offerte à leurs yeux.\par
Alors, quels moyens avisera la société, si elle veut vivre, de remédier à la pénurie des naissances ? Organisera-t-on artificiellement la production des hommes après — ou peut-être même avant — la production des richesses ? La maternité et la paternité, comme tout travail, étant des fonctions publiques dans la cité communiste, il conviendra, sans nul doute, de soumettre l’\emph{élevage humain}, la \emph{viriculture}, au point de vue de la quantité et de la qualité, à des règlements rigoureux. Ce sera là une solution du problème. Peut-être s’apercevra-t-on cependant qu’il y en a une autre de possible : ce serait de renforcer le lien familial, qui tend visiblement à se relâcher avec excès.
 


% at least one empty page at end (for booklet couv)
\ifbooklet
  \pagestyle{empty}
  \clearpage
  % 2 empty pages maybe needed for 4e cover
  \ifnum\modulo{\value{page}}{4}=0 \hbox{}\newpage\hbox{}\newpage\fi
  \ifnum\modulo{\value{page}}{4}=1 \hbox{}\newpage\hbox{}\newpage\fi


  \hbox{}\newpage
  \ifodd\value{page}\hbox{}\newpage\fi
  {\centering\color{rubric}\bfseries\noindent\large
    Hurlus ? Qu’est-ce.\par
    \bigskip
  }
  \noindent Des bouquinistes électroniques, pour du texte libre à participation libre,
  téléchargeable gratuitement sur \href{https://hurlus.fr}{\dotuline{hurlus.fr}}.\par
  \bigskip
  \noindent Cette brochure a été produite par des éditeurs bénévoles.
  Elle n’est pas faîte pour être possédée, mais pour être lue, et puis donnée.
  Que circule le texte !
  En page de garde, on peut ajouter une date, un lieu, un nom ; pour suivre le voyage des idées.
  \par

  Ce texte a été choisi parce qu’une personne l’a aimé,
  ou haï, elle a en tous cas pensé qu’il partipait à la formation de notre présent ;
  sans le souci de plaire, vendre, ou militer pour une cause.
  \par

  L’édition électronique est soigneuse, tant sur la technique
  que sur l’établissement du texte ; mais sans aucune prétention scolaire, au contraire.
  Le but est de s’adresser à tous, sans distinction de science ou de diplôme.
  Au plus direct ! (possible)
  \par

  Cet exemplaire en papier a été tiré sur une imprimante personnelle
   ou une photocopieuse. Tout le monde peut le faire.
  Il suffit de
  télécharger un fichier sur \href{https://hurlus.fr}{\dotuline{hurlus.fr}},
  d’imprimer, et agrafer ; puis de lire et donner.\par

  \bigskip

  \noindent PS : Les hurlus furent aussi des rebelles protestants qui cassaient les statues dans les églises catholiques. En 1566 démarra la révolte des gueux dans le pays de Lille. L’insurrection enflamma la région jusqu’à Anvers où les gueux de mer bloquèrent les bateaux espagnols.
  Ce fut une rare guerre de libération dont naquit un pays toujours libre : les Pays-Bas.
  En plat pays francophone, par contre, restèrent des bandes de huguenots, les hurlus, progressivement réprimés par la très catholique Espagne.
  Cette mémoire d’une défaite est éteinte, rallumons-la. Sortons les livres du culte universitaire, cherchons les idoles de l’époque, pour les briser.
\fi

\ifdev % autotext in dev mode
\fontname\font — \textsc{Les règles du jeu}\par
(\hyperref[utopie]{\underline{Lien}})\par
\noindent \initialiv{A}{lors là}\blindtext\par
\noindent \initialiv{À}{ la bonheur des dames}\blindtext\par
\noindent \initialiv{É}{tonnez-le}\blindtext\par
\noindent \initialiv{Q}{ualitativement}\blindtext\par
\noindent \initialiv{V}{aloriser}\blindtext\par
\Blindtext
\phantomsection
\label{utopie}
\Blinddocument
\fi
\end{document}
