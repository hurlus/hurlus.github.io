%%%%%%%%%%%%%%%%%%%%%%%%%%%%%%%%%
% LaTeX model https://hurlus.fr %
%%%%%%%%%%%%%%%%%%%%%%%%%%%%%%%%%

% Needed before document class
\RequirePackage{pdftexcmds} % needed for tests expressions
\RequirePackage{fix-cm} % correct units

% Define mode
\def\mode{a4}

\newif\ifaiv % a4
\newif\ifav % a5
\newif\ifbooklet % booklet
\newif\ifcover % cover for booklet

\ifnum \strcmp{\mode}{cover}=0
  \covertrue
\else\ifnum \strcmp{\mode}{booklet}=0
  \booklettrue
\else\ifnum \strcmp{\mode}{a5}=0
  \avtrue
\else
  \aivtrue
\fi\fi\fi

\ifbooklet % do not enclose with {}
  \documentclass[french,twoside]{book} % ,notitlepage
  \usepackage[%
    papersize={105mm, 297mm},
    inner=12mm,
    outer=12mm,
    top=20mm,
    bottom=15mm,
    marginparsep=0pt,
  ]{geometry}
  \usepackage[fontsize=9.5pt]{scrextend} % for Roboto
\else\ifav
  \documentclass[french,twoside]{book} % ,notitlepage
  \usepackage[%
    a5paper,
    inner=25mm,
    outer=15mm,
    top=15mm,
    bottom=15mm,
    marginparsep=0pt,
  ]{geometry}
  \usepackage[fontsize=12pt]{scrextend}
\else% A4 2 cols
  \documentclass[twocolumn]{report}
  \usepackage[%
    a4paper,
    inner=15mm,
    outer=10mm,
    top=25mm,
    bottom=18mm,
    marginparsep=0pt,
  ]{geometry}
  \setlength{\columnsep}{20mm}
  \usepackage[fontsize=9.5pt]{scrextend}
\fi\fi

%%%%%%%%%%%%%%
% Alignments %
%%%%%%%%%%%%%%
% before teinte macros

\setlength{\arrayrulewidth}{0.2pt}
\setlength{\columnseprule}{\arrayrulewidth} % twocol
\setlength{\parskip}{0pt} % classical para with no margin
\setlength{\parindent}{1.5em}

%%%%%%%%%%
% Colors %
%%%%%%%%%%
% before Teinte macros

\usepackage[dvipsnames]{xcolor}
\definecolor{rubric}{HTML}{800000} % the tonic 0c71c3
\def\columnseprulecolor{\color{rubric}}
\colorlet{borderline}{rubric!30!} % definecolor need exact code
\definecolor{shadecolor}{gray}{0.95}
\definecolor{bghi}{gray}{0.5}

%%%%%%%%%%%%%%%%%
% Teinte macros %
%%%%%%%%%%%%%%%%%
%%%%%%%%%%%%%%%%%%%%%%%%%%%%%%%%%%%%%%%%%%%%%%%%%%%
% <TEI> generic (LaTeX names generated by Teinte) %
%%%%%%%%%%%%%%%%%%%%%%%%%%%%%%%%%%%%%%%%%%%%%%%%%%%
% This template is inserted in a specific design
% It is XeLaTeX and otf fonts

\makeatletter % <@@@


\usepackage{blindtext} % generate text for testing
\usepackage[strict]{changepage} % for modulo 4
\usepackage{contour} % rounding words
\usepackage[nodayofweek]{datetime}
% \usepackage{DejaVuSans} % seems buggy for sffont font for symbols
\usepackage{enumitem} % <list>
\usepackage{etoolbox} % patch commands
\usepackage{fancyvrb}
\usepackage{fancyhdr}
\usepackage{float}
\usepackage{fontspec} % XeLaTeX mandatory for fonts
\usepackage{footnote} % used to capture notes in minipage (ex: quote)
\usepackage{framed} % bordering correct with footnote hack
\usepackage{graphicx}
\usepackage{lettrine} % drop caps
\usepackage{lipsum} % generate text for testing
\usepackage[framemethod=tikz,]{mdframed} % maybe used for frame with footnotes inside
\usepackage{pdftexcmds} % needed for tests expressions
\usepackage{polyglossia} % non-break space french punct, bug Warning: "Failed to patch part"
\usepackage[%
  indentfirst=false,
  vskip=1em,
  noorphanfirst=true,
  noorphanafter=true,
  leftmargin=\parindent,
  rightmargin=0pt,
]{quoting}
\usepackage{ragged2e}
\usepackage{setspace} % \setstretch for <quote>
\usepackage{tabularx} % <table>
\usepackage[explicit]{titlesec} % wear titles, !NO implicit
\usepackage{tikz} % ornaments
\usepackage{tocloft} % styling tocs
\usepackage[fit]{truncate} % used im runing titles
\usepackage{unicode-math}
\usepackage[normalem]{ulem} % breakable \uline, normalem is absolutely necessary to keep \emph
\usepackage{verse} % <l>
\usepackage{xcolor} % named colors
\usepackage{xparse} % @ifundefined
\XeTeXdefaultencoding "iso-8859-1" % bad encoding of xstring
\usepackage{xstring} % string tests
\XeTeXdefaultencoding "utf-8"
\PassOptionsToPackage{hyphens}{url} % before hyperref, which load url package

% TOTEST
% \usepackage{hypcap} % links in caption ?
% \usepackage{marginnote}
% TESTED
% \usepackage{background} % doesn’t work with xetek
% \usepackage{bookmark} % prefers the hyperref hack \phantomsection
% \usepackage[color, leftbars]{changebar} % 2 cols doc, impossible to keep bar left
% \usepackage[utf8x]{inputenc} % inputenc package ignored with utf8 based engines
% \usepackage[sfdefault,medium]{inter} % no small caps
% \usepackage{firamath} % choose firasans instead, firamath unavailable in Ubuntu 21-04
% \usepackage{flushend} % bad for last notes, supposed flush end of columns
% \usepackage[stable]{footmisc} % BAD for complex notes https://texfaq.org/FAQ-ftnsect
% \usepackage{helvet} % not for XeLaTeX
% \usepackage{multicol} % not compatible with too much packages (longtable, framed, memoir…)
% \usepackage[default,oldstyle,scale=0.95]{opensans} % no small caps
% \usepackage{sectsty} % \chapterfont OBSOLETE
% \usepackage{soul} % \ul for underline, OBSOLETE with XeTeX
% \usepackage[breakable]{tcolorbox} % text styling gone, footnote hack not kept with breakable


% Metadata inserted by a program, from the TEI source, for title page and runing heads
\title{\textbf{ Non, le Nazisme n’est pas le Socialisme }}
\date{1941}
\author{Péri, Gabriel}
\def\elbibl{Péri, Gabriel. 1941. \emph{Non, le Nazisme n’est pas le Socialisme}}
\def\elsource{ \href{https://pandor.u-bourgogne.fr/img-viewer/PAPRIKA/AN/FRAN_IR_050039/Z_4/138/0016/viewer.html?ns=FRAN_0020_10759_L.jpg}{\dotuline{}}\footnote{\href{https://pandor.u-bourgogne.fr/img-viewer/PAPRIKA/AN/FRAN_IR_050039/Z_4/138/0016/viewer.html?ns=FRAN_0020_10759_L.jpg}{\url{https://pandor.u-bourgogne.fr/img-viewer/PAPRIKA/AN/FRAN_IR_050039/Z_4/138/0016/viewer.html?ns=FRAN_0020_10759_L.jpg}}} }

% Default metas
\newcommand{\colorprovide}[2]{\@ifundefinedcolor{#1}{\colorlet{#1}{#2}}{}}
\colorprovide{rubric}{red}
\colorprovide{silver}{lightgray}
\@ifundefined{syms}{\newfontfamily\syms{DejaVu Sans}}{}
\newif\ifdev
\@ifundefined{elbibl}{% No meta defined, maybe dev mode
  \newcommand{\elbibl}{Titre court ?}
  \newcommand{\elbook}{Titre du livre source ?}
  \newcommand{\elabstract}{Résumé\par}
  \newcommand{\elurl}{http://oeuvres.github.io/elbook/2}
  \author{Éric Lœchien}
  \title{Un titre de test assez long pour vérifier le comportement d’une maquette}
  \date{1566}
  \devtrue
}{}
\let\eltitle\@title
\let\elauthor\@author
\let\eldate\@date


\defaultfontfeatures{
  % Mapping=tex-text, % no effect seen
  Scale=MatchLowercase,
  Ligatures={TeX,Common},
}


% generic typo commands
\newcommand{\astermono}{\medskip\centerline{\color{rubric}\large\selectfont{\syms ✻}}\medskip\par}%
\newcommand{\astertri}{\medskip\par\centerline{\color{rubric}\large\selectfont{\syms ✻\,✻\,✻}}\medskip\par}%
\newcommand{\asterism}{\bigskip\par\noindent\parbox{\linewidth}{\centering\color{rubric}\large{\syms ✻}\\{\syms ✻}\hskip 0.75em{\syms ✻}}\bigskip\par}%

% lists
\newlength{\listmod}
\setlength{\listmod}{\parindent}
\setlist{
  itemindent=!,
  listparindent=\listmod,
  labelsep=0.2\listmod,
  parsep=0pt,
  % topsep=0.2em, % default topsep is best
}
\setlist[itemize]{
  label=—,
  leftmargin=0pt,
  labelindent=1.2em,
  labelwidth=0pt,
}
\setlist[enumerate]{
  label={\bf\color{rubric}\arabic*.},
  labelindent=0.8\listmod,
  leftmargin=\listmod,
  labelwidth=0pt,
}
\newlist{listalpha}{enumerate}{1}
\setlist[listalpha]{
  label={\bf\color{rubric}\alph*.},
  leftmargin=0pt,
  labelindent=0.8\listmod,
  labelwidth=0pt,
}
\newcommand{\listhead}[1]{\hspace{-1\listmod}\emph{#1}}

\renewcommand{\hrulefill}{%
  \leavevmode\leaders\hrule height 0.2pt\hfill\kern\z@}

% General typo
\DeclareTextFontCommand{\textlarge}{\large}
\DeclareTextFontCommand{\textsmall}{\small}

% commands, inlines
\newcommand{\anchor}[1]{\Hy@raisedlink{\hypertarget{#1}{}}} % link to top of an anchor (not baseline)
\newcommand\abbr[1]{#1}
\newcommand{\autour}[1]{\tikz[baseline=(X.base)]\node [draw=rubric,thin,rectangle,inner sep=1.5pt, rounded corners=3pt] (X) {\color{rubric}#1};}
\newcommand\corr[1]{#1}
\newcommand{\ed}[1]{ {\color{silver}\sffamily\footnotesize (#1)} } % <milestone ed="1688"/>
\newcommand\expan[1]{#1}
\newcommand\foreign[1]{\emph{#1}}
\newcommand\gap[1]{#1}
\renewcommand{\LettrineFontHook}{\color{rubric}}
\newcommand{\initial}[2]{\lettrine[lines=2, loversize=0.3, lhang=0.3]{#1}{#2}}
\newcommand{\initialiv}[2]{%
  \let\oldLFH\LettrineFontHook
  % \renewcommand{\LettrineFontHook}{\color{rubric}\ttfamily}
  \IfSubStr{QJ’}{#1}{
    \lettrine[lines=4, lhang=0.2, loversize=-0.1, lraise=0.2]{\smash{#1}}{#2}
  }{\IfSubStr{É}{#1}{
    \lettrine[lines=4, lhang=0.2, loversize=-0, lraise=0]{\smash{#1}}{#2}
  }{\IfSubStr{ÀÂ}{#1}{
    \lettrine[lines=4, lhang=0.2, loversize=-0, lraise=0, slope=0.6em]{\smash{#1}}{#2}
  }{\IfSubStr{A}{#1}{
    \lettrine[lines=4, lhang=0.2, loversize=0.2, slope=0.6em]{\smash{#1}}{#2}
  }{\IfSubStr{V}{#1}{
    \lettrine[lines=4, lhang=0.2, loversize=0.2, slope=-0.5em]{\smash{#1}}{#2}
  }{
    \lettrine[lines=4, lhang=0.2, loversize=0.2]{\smash{#1}}{#2}
  }}}}}
  \let\LettrineFontHook\oldLFH
}
\newcommand{\labelchar}[1]{\textbf{\color{rubric} #1}}
\newcommand{\milestone}[1]{\autour{\footnotesize\color{rubric} #1}} % <milestone n="4"/>
\newcommand\name[1]{#1}
\newcommand\orig[1]{#1}
\newcommand\orgName[1]{#1}
\newcommand\persName[1]{#1}
\newcommand\placeName[1]{#1}
\newcommand{\pn}[1]{\IfSubStr{-—–¶}{#1}% <p n="3"/>
  {\noindent{\bfseries\color{rubric}   ¶  }}
  {{\footnotesize\autour{ #1}  }}}
\newcommand\reg{}
% \newcommand\ref{} % already defined
\newcommand\sic[1]{#1}
\newcommand\surname[1]{\textsc{#1}}
\newcommand\term[1]{\textbf{#1}}

\def\mednobreak{\ifdim\lastskip<\medskipamount
  \removelastskip\nopagebreak\medskip\fi}
\def\bignobreak{\ifdim\lastskip<\bigskipamount
  \removelastskip\nopagebreak\bigskip\fi}

% commands, blocks
\newcommand{\byline}[1]{\bigskip{\RaggedLeft{#1}\par}\bigskip}
\newcommand{\bibl}[1]{{\RaggedLeft{#1}\par\bigskip}}
\newcommand{\biblitem}[1]{{\noindent\hangindent=\parindent   #1\par}}
\newcommand{\dateline}[1]{\medskip{\RaggedLeft{#1}\par}\bigskip}
\newcommand{\labelblock}[1]{\medbreak{\noindent\color{rubric}\bfseries #1}\par\mednobreak}
\newcommand{\salute}[1]{\bigbreak{#1}\par\medbreak}
\newcommand{\signed}[1]{\bigbreak\filbreak{\raggedleft #1\par}\medskip}

% environments for blocks (some may become commands)
\newenvironment{borderbox}{}{} % framing content
\newenvironment{citbibl}{\ifvmode\hfill\fi}{\ifvmode\par\fi }
\newenvironment{docAuthor}{\ifvmode\vskip4pt\fontsize{16pt}{18pt}\selectfont\fi\itshape}{\ifvmode\par\fi }
\newenvironment{docDate}{}{\ifvmode\par\fi }
\newenvironment{docImprint}{\vskip6pt}{\ifvmode\par\fi }
\newenvironment{docTitle}{\vskip6pt\bfseries\fontsize{18pt}{22pt}\selectfont}{\par }
\newenvironment{msHead}{\vskip6pt}{\par}
\newenvironment{msItem}{\vskip6pt}{\par}
\newenvironment{titlePart}{}{\par }


% environments for block containers
\newenvironment{argument}{\itshape\parindent0pt}{\vskip1.5em}
\newenvironment{biblfree}{}{\ifvmode\par\fi }
\newenvironment{bibitemlist}[1]{%
  \list{\@biblabel{\@arabic\c@enumiv}}%
  {%
    \settowidth\labelwidth{\@biblabel{#1}}%
    \leftmargin\labelwidth
    \advance\leftmargin\labelsep
    \@openbib@code
    \usecounter{enumiv}%
    \let\p@enumiv\@empty
    \renewcommand\theenumiv{\@arabic\c@enumiv}%
  }
  \sloppy
  \clubpenalty4000
  \@clubpenalty \clubpenalty
  \widowpenalty4000%
  \sfcode`\.\@m
}%
{\def\@noitemerr
  {\@latex@warning{Empty `bibitemlist' environment}}%
\endlist}
\newenvironment{quoteblock}% may be used for ornaments
  {\begin{quoting}}
  {\end{quoting}}

% table () is preceded and finished by custom command
\newcommand{\tableopen}[1]{%
  \ifnum\strcmp{#1}{wide}=0{%
    \begin{center}
  }
  \else\ifnum\strcmp{#1}{long}=0{%
    \begin{center}
  }
  \else{%
    \begin{center}
  }
  \fi\fi
}
\newcommand{\tableclose}[1]{%
  \ifnum\strcmp{#1}{wide}=0{%
    \end{center}
  }
  \else\ifnum\strcmp{#1}{long}=0{%
    \end{center}
  }
  \else{%
    \end{center}
  }
  \fi\fi
}


% text structure
\newcommand\chapteropen{} % before chapter title
\newcommand\chaptercont{} % after title, argument, epigraph…
\newcommand\chapterclose{} % maybe useful for multicol settings
\setcounter{secnumdepth}{-2} % no counters for hierarchy titles
\setcounter{tocdepth}{5} % deep toc
\markright{\@title} % ???
\markboth{\@title}{\@author} % ???
\renewcommand\tableofcontents{\@starttoc{toc}}
% toclof format
% \renewcommand{\@tocrmarg}{0.1em} % Useless command?
% \renewcommand{\@pnumwidth}{0.5em} % {1.75em}
\renewcommand{\@cftmaketoctitle}{}
\setlength{\cftbeforesecskip}{\z@ \@plus.2\p@}
\renewcommand{\cftchapfont}{}
\renewcommand{\cftchapdotsep}{\cftdotsep}
\renewcommand{\cftchapleader}{\normalfont\cftdotfill{\cftchapdotsep}}
\renewcommand{\cftchappagefont}{\bfseries}
\setlength{\cftbeforechapskip}{0em \@plus\p@}
% \renewcommand{\cftsecfont}{\small\relax}
\renewcommand{\cftsecpagefont}{\normalfont}
% \renewcommand{\cftsubsecfont}{\small\relax}
\renewcommand{\cftsecdotsep}{\cftdotsep}
\renewcommand{\cftsecpagefont}{\normalfont}
\renewcommand{\cftsecleader}{\normalfont\cftdotfill{\cftsecdotsep}}
\setlength{\cftsecindent}{1em}
\setlength{\cftsubsecindent}{2em}
\setlength{\cftsubsubsecindent}{3em}
\setlength{\cftchapnumwidth}{1em}
\setlength{\cftsecnumwidth}{1em}
\setlength{\cftsubsecnumwidth}{1em}
\setlength{\cftsubsubsecnumwidth}{1em}

% footnotes
\newif\ifheading
\newcommand*{\fnmarkscale}{\ifheading 0.70 \else 1 \fi}
\renewcommand\footnoterule{\vspace*{0.3cm}\hrule height \arrayrulewidth width 3cm \vspace*{0.3cm}}
\setlength\footnotesep{1.5\footnotesep} % footnote separator
\renewcommand\@makefntext[1]{\parindent 1.5em \noindent \hb@xt@1.8em{\hss{\normalfont\@thefnmark . }}#1} % no superscipt in foot
\patchcmd{\@footnotetext}{\footnotesize}{\footnotesize\sffamily}{}{} % before scrextend, hyperref


%   see https://tex.stackexchange.com/a/34449/5049
\def\truncdiv#1#2{((#1-(#2-1)/2)/#2)}
\def\moduloop#1#2{(#1-\truncdiv{#1}{#2}*#2)}
\def\modulo#1#2{\number\numexpr\moduloop{#1}{#2}\relax}

% orphans and widows
\clubpenalty=9996
\widowpenalty=9999
\brokenpenalty=4991
\predisplaypenalty=10000
\postdisplaypenalty=1549
\displaywidowpenalty=1602
\hyphenpenalty=400
% Copied from Rahtz but not understood
\def\@pnumwidth{1.55em}
\def\@tocrmarg {2.55em}
\def\@dotsep{4.5}
\emergencystretch 3em
\hbadness=4000
\pretolerance=750
\tolerance=2000
\vbadness=4000
\def\Gin@extensions{.pdf,.png,.jpg,.mps,.tif}
% \renewcommand{\@cite}[1]{#1} % biblio

\usepackage{hyperref} % supposed to be the last one, :o) except for the ones to follow
\urlstyle{same} % after hyperref
\hypersetup{
  % pdftex, % no effect
  pdftitle={\elbibl},
  % pdfauthor={Your name here},
  % pdfsubject={Your subject here},
  % pdfkeywords={keyword1, keyword2},
  bookmarksnumbered=true,
  bookmarksopen=true,
  bookmarksopenlevel=1,
  pdfstartview=Fit,
  breaklinks=true, % avoid long links
  pdfpagemode=UseOutlines,    % pdf toc
  hyperfootnotes=true,
  colorlinks=false,
  pdfborder=0 0 0,
  % pdfpagelayout=TwoPageRight,
  % linktocpage=true, % NO, toc, link only on page no
}

\makeatother % /@@@>
%%%%%%%%%%%%%%
% </TEI> end %
%%%%%%%%%%%%%%


%%%%%%%%%%%%%
% footnotes %
%%%%%%%%%%%%%
\renewcommand{\thefootnote}{\bfseries\textcolor{rubric}{\arabic{footnote}}} % color for footnote marks

%%%%%%%%%
% Fonts %
%%%%%%%%%
\usepackage[]{roboto} % SmallCaps, Regular is a bit bold
% \linespread{0.90} % too compact, keep font natural
\newfontfamily\fontrun[]{Roboto Condensed Light} % condensed runing heads
\ifav
  \setmainfont[
    ItalicFont={Roboto Light Italic},
  ]{Roboto}
\else\ifbooklet
  \setmainfont[
    ItalicFont={Roboto Light Italic},
  ]{Roboto}
\else
\setmainfont[
  ItalicFont={Roboto Italic},
]{Roboto Light}
\fi\fi
\renewcommand{\LettrineFontHook}{\bfseries\color{rubric}}
% \renewenvironment{labelblock}{\begin{center}\bfseries\color{rubric}}{\end{center}}

%%%%%%%%
% MISC %
%%%%%%%%

\setdefaultlanguage[frenchpart=false]{french} % bug on part


\newenvironment{quotebar}{%
    \def\FrameCommand{{\color{rubric!10!}\vrule width 0.5em} \hspace{0.9em}}%
    \def\OuterFrameSep{\itemsep} % séparateur vertical
    \MakeFramed {\advance\hsize-\width \FrameRestore}
  }%
  {%
    \endMakeFramed
  }
\renewenvironment{quoteblock}% may be used for ornaments
  {%
    \savenotes
    \setstretch{0.9}
    \normalfont
    \begin{quotebar}
  }
  {%
    \end{quotebar}
    \spewnotes
  }


\renewcommand{\headrulewidth}{\arrayrulewidth}
\renewcommand{\headrule}{{\color{rubric}\hrule}}

% delicate tuning, image has produce line-height problems in title on 2 lines
\titleformat{name=\chapter} % command
  [display] % shape
  {\vspace{1.5em}\centering} % format
  {} % label
  {0pt} % separator between n
  {}
[{\color{rubric}\huge\textbf{#1}}\bigskip] % after code
% \titlespacing{command}{left spacing}{before spacing}{after spacing}[right]
\titlespacing*{\chapter}{0pt}{-2em}{0pt}[0pt]

\titleformat{name=\section}
  [block]{}{}{}{}
  [\vbox{\color{rubric}\large\raggedleft\textbf{#1}}]
\titlespacing{\section}{0pt}{0pt plus 4pt minus 2pt}{\baselineskip}

\titleformat{name=\subsection}
  [block]
  {}
  {} % \thesection
  {} % separator \arrayrulewidth
  {}
[\vbox{\large\textbf{#1}}]
% \titlespacing{\subsection}{0pt}{0pt plus 4pt minus 2pt}{\baselineskip}

\ifaiv
  \fancypagestyle{main}{%
    \fancyhf{}
    \setlength{\headheight}{1.5em}
    \fancyhead{} % reset head
    \fancyfoot{} % reset foot
    \fancyhead[L]{\truncate{0.45\headwidth}{\fontrun\elbibl}} % book ref
    \fancyhead[R]{\truncate{0.45\headwidth}{ \fontrun\nouppercase\leftmark}} % Chapter title
    \fancyhead[C]{\thepage}
  }
  \fancypagestyle{plain}{% apply to chapter
    \fancyhf{}% clear all header and footer fields
    \setlength{\headheight}{1.5em}
    \fancyhead[L]{\truncate{0.9\headwidth}{\fontrun\elbibl}}
    \fancyhead[R]{\thepage}
  }
\else
  \fancypagestyle{main}{%
    \fancyhf{}
    \setlength{\headheight}{1.5em}
    \fancyhead{} % reset head
    \fancyfoot{} % reset foot
    \fancyhead[RE]{\truncate{0.9\headwidth}{\fontrun\elbibl}} % book ref
    \fancyhead[LO]{\truncate{0.9\headwidth}{\fontrun\nouppercase\leftmark}} % Chapter title, \nouppercase needed
    \fancyhead[RO,LE]{\thepage}
  }
  \fancypagestyle{plain}{% apply to chapter
    \fancyhf{}% clear all header and footer fields
    \setlength{\headheight}{1.5em}
    \fancyhead[L]{\truncate{0.9\headwidth}{\fontrun\elbibl}}
    \fancyhead[R]{\thepage}
  }
\fi

\ifav % a5 only
  \titleclass{\section}{top}
\fi

\newcommand\chapo{{%
  \vspace*{-3em}
  \centering % no vskip ()
  {\Large\addfontfeature{LetterSpace=25}\bfseries{\elauthor}}\par
  \smallskip
  {\large\eldate}\par
  \bigskip
  {\Large\selectfont{\eltitle}}\par
  \bigskip
  {\color{rubric}\hline\par}
  \bigskip
  {\Large TEXTE LIBRE À PARTICPATION LIBRE\par}
  \centerline{\small\color{rubric} {hurlus.fr, tiré le \today}}\par
  \bigskip
}}

\newcommand\cover{{%
  \thispagestyle{empty}
  \centering
  {\LARGE\bfseries{\elauthor}}\par
  \bigskip
  {\Large\eldate}\par
  \bigskip
  \bigskip
  {\LARGE\selectfont{\eltitle}}\par
  \vfill\null
  {\color{rubric}\setlength{\arrayrulewidth}{2pt}\hline\par}
  \vfill\null
  {\Large TEXTE LIBRE À PARTICPATION LIBRE\par}
  \centerline{{\href{https://hurlus.fr}{\dotuline{hurlus.fr}}, tiré le \today}}\par
}}

\begin{document}
\pagestyle{empty}
\ifbooklet{
  \cover\newpage
  \thispagestyle{empty}\hbox{}\newpage
  \cover\newpage\noindent Les voyages de la brochure\par
  \bigskip
  \begin{tabularx}{\textwidth}{l|X|X}
    \textbf{Date} & \textbf{Lieu}& \textbf{Nom/pseudo} \\ \hline
    \rule{0pt}{25cm} &  &   \\
  \end{tabularx}
  \newpage
  \addtocounter{page}{-4}
}\fi

\thispagestyle{empty}
\ifaiv
  \twocolumn[\chapo]
\else
  \chapo
\fi
{\it\elabstract}
\bigskip
\makeatletter\@starttoc{toc}\makeatother % toc without new page
\bigskip

\pagestyle{main} % after style

  
\chapteropen
 \chapter[{1 – Au chevet du capitalisme malade}]{1 – Au chevet du capitalisme malade}\renewcommand{\leftmark}{1 – Au chevet du capitalisme malade}


\chaptercont
\noindent « Nous sommes entrés dans l’ère des grands et irrévocables bouleversements. » Cette phrase a été dite et imprimée cent fois depuis huit mois. Il n’est pas très sûr, d’ailleurs, que tous ceux qui l’ont prononcée ou écrite l’aient pensée au préalable. Dans tous les cas, elle exprime chez les uns une résignation attristée ; chez les autres une curiosité inquiète ; chez d’autres un soulagement et une délivrance. Pour certains, elle amorce une manœuvre subtile mais de grand style : crier très fort que l’ère des grands bouleversements a commencé, ce peut être un moyen de dissimuler les efforts souterrains que l’on tente pour conserver sous un nom nouveau le vieil ordre des choses.\par
Autour du malade se poursuit une scène curieuse. Il y a la foule immense des fossoyeurs, ces millions d’hommes rassemblés dans de vastes usines. Ceux-là n’ont qu’un désir : faire neuf, faire jeune, faire socialiste en un mot. Ils ont été les victimes essentielles du régime malade. Ils sont les constructeurs d’un monde meilleur. Ils se sont préparés à cette tâche. Ils ont, pour s’y préparer, bravé les coups et résisté aux illusions de ceux qui leur parlaient, il n’y a pas bien longtemps encore, de « faire l’économie d’un changement essentiel ». Eux n’ont jamais cru à ces sornettes. Ils sont des novateurs, et pas seulement depuis qu’un Maréchal octogénaire règne sur une partie de la France et depuis qu’une armée d’occupation règne sur l’autre. Révolutionnaires, ils l’étaient bien avant, quand ce mot n’était pas prononcé sans horreur par les manieurs d’argent, les douairières et les journalistes corrompus.\par
Seulement, à côté de ces foules impatientes et que l’on tient à l’écart, par exemple en jetant dans les prisons et dans les camps de concentration leurs plus authentiques représentants,   en leur interdisant la parole, en les privant de leurs journaux, d’autres se pressent autour du patient. On ne voit qu’eux, on n’entend qu’eux. Ils se sont installés dans tous les couloirs de la Maison. Ils s’affairent comme des médicastres besogneux. Et que font-ils au juste ? Ils essaient d’inoculer au malade un sérum. Ils veulent prolonger sa vie. Mais, en même temps, ces Diafoirus modernes crient très haut qu’ils rédigent l’acte de décès : « Messieurs et Dames, l’ancien régime se meurt, l’ancien régime est mort. Nous préparons les bandelettes et nous clouons le cercueil. Mais ne nous gênez pas dans cette opération délicate, ne venez pas voir surtout ! Soyez-en certains, nous faisons la Grande Révolution Nationale. Le capitalisme sera enterré par nos soins et, puisque vous semblez y tenir, nous lui substituerons le… socialisme, oui, un socialisme national. Que vous faut-il de plus ? Mais, encore un coup, laissez-nous faire. Allez vous distraire pendant ce temps au Comité de la Révolution Nationale que préside notre ami du Moulin de la Barthète, ou au Rassemblement National que Marcel Déat dirige sur le sentier joyeux de ses destins. »\par
Et tandis qu’ils pérorent ainsi, ils inoculent leur élixir de jouvence. Si d’aventure ils réussissaient leur opération, si les masses populaires écoutaient leur boniment, c’est le moribond qu’on nous présenterait à nouveau dans quelques mois. On lui aurait conservé ses attributs essentiels, mais on l’aurait un peu rafraîchi et, surtout, on lui aurait donné un nom nouveau. On ne l’appellerait plus Régime ploutocratique, on le batiserait : « Communauté nationale socialiste ». Et le tour serait joué ici comme il a été joué ailleurs.
\chapterclose


\chapteropen
\chapter[{2 – Les charlatans de la « Révolution nationale »}]{2 – Les charlatans de la « Révolution nationale »}\renewcommand{\leftmark}{2 – Les charlatans de la « Révolution nationale »}


\chaptercont
\noindent Mais pourquoi toutes ces simagrées ? Ne serait-il pas plus simple d’avouer que l’on veut ragaillardir le patient et lui rendre la vie ? Non ! Car les jeux simples ne sont plus permis à l’heure où nous vivons. Supposez que la bande qui s’est ruée sur la France – sur celle de Vichy et sur celle de Paris – depuis le mois de juin 1940 dise au peuple de France : « le régime actuel est malade, très malade, mais nous allons essayer de le guérir ». Qui parlerait ainsi serait sur le champ châtié. Car le peuple ne veut plus d’un régime qui a accumulé tant de misère et tant de sang, qui a été incapable de faire la paix, qui a plongé dans son immense domaine des millions d’hommes dans le chômage, puis qui les a tirés du chômage pour les condamner à fabriquer des instruments de mort et qui, ces instruments forgés, a précipité le peuple dans le massacre, qui lui a fait subir la plus terrible défaite de son histoire, qui veut le livrer pieds et poings liés à l’esclavage national et qui, s’il subsistait, reprendrait sa marche infernale, en broyant les corps et en empoisonnant l’esprit des hommes. Non, le peuple français ne veut plus de ce régime qu’il identifie à la guerre, à la débâcle, à la trahison, à la servitude. Lui annoncer que l’on va prolonger la vie de ce monstre, c’est lui annoncer que lui, va vivre encore captif et exploité, que ses fils connaîtront la même affreuse existence, le même système d’oppression, de misère et de violence.\par
Qui donc oserait franchement annoncer cela ? Qui donc oserait tenir ce langage et défier ainsi la colère du peuple ? Personne, assurément. Alors, il faut tenir un autre langage. Il faut essayer de sauver le vieil ordre en jurant qu’on le met en terre et qu’on construit un ordre nouveau.\par
\section[{On évoque le socialisme}]{On évoque le socialisme}
\noindent Voilà pourquoi ceux-là même qui ont consacré toute leur vie à la lutte contre le socialisme, qui ont proclamé la pérennité de la ploutocratie, qui ont chanté les louanges au dieu de l’argent ; ceux qui ont assimilé la Révolution à la peste ou au choléra ; ceux qui ont placé leurs intérêts personnels au-dessus des intérêts de la France, ceux-là même font la grosse voix pour excommunier le capitalisme et pour célébrer la vertu de la Révolution. Les grands commis des congrégations économiques, les hobereaux, tous ceux qui tremblaient pour leurs privilèges et criaient « Sus à la Révolution » lorsqu’en 1936 les ouvriers obtenaient une modeste amélioration de leurs conditions d’existence ; ceux qui, pour constituer au patronat une police supplétive, allaient quérir des armes à l’étranger et les expérimentaient un soir, rue de Presbourg ; ceux qui, plus tard, sur l’ordre de Gignoux et de Lambert-Ribot et avec la complicité de Daladier, reprirent aux travailleurs ce que les travailleurs avaient conquis, tous ceux-là, soir et matin, à grands coups de gueule dans les microphones de Radio-Paris et de Radio-Toulouse invoquent avec attendrissement la bienfaisante Révolution et – avec plus de discrétion – le bienfaisant socialisme.\par
Pour que l’illusion soit complète, ils ont embarqué dans leur galère quelques anciens socialistes de contrebande, néosocialistes marrons et syndicalistes mercenaires choisis parmi ceux qui, avant septembre 1939, s’étaient fait une spécialité de condamner comme d’inutiles et encombrants accessoires les enseignements les moins discutables du socialisme et des Révolutions ouvrières.\par
Ils nous affirment, les bons apôtres, qu’il nous faut, sans plus attendre, nous prêter à la salutaire contagion des pays voisins, ces pays étant, vous l’avez deviné, l’Allemagne nazie et l’Italie fasciste. Hitler et Mussolini ont été, paraît-il, des années durant, abominablement calomniés. De grands méchants loups avaient dissimulé au public que ces deux personnages symbolisaient en réalité la lutte contre le capitalisme. Grâce à Dieu et… aux baïonnettes de l’armée d’occupation, Déat, Laval, Doriot sont là, et même un peu là, pour remettre les choses au point : Hitler a réalisé le socialisme. Le nazisme, c’est le socialisme, ou, tout au moins, c’est une forme du socialisme, un « socialisme authentique » comme écrit ce « penseur subtil » qu’est Gabriel Lafaye. Aussi bien, Laval, Déat, Spinasse et Doriot s’en portent-ils garants. Et quiconque ne voit pas dans le Nazisme l’éclosion magnifique du Socialisme, celui-là est un vulgaire conservateur. C’est Laval qui le dit – et avec lui, quelques fameux commis des 200 familles, représentées dans le Rassemblement national. Et ils s’y connaissent, les gaillards !
\section[{On te trompe, peuple de France}]{On te trompe, peuple de France}
\noindent Eh bien nous, les communistes, défenseurs des vrais intérêts de la France, nous prenons la parole pour dénoncer cette imposture grossière, pour dire à notre peuple qu’on lui ment, qu’on prépare sur son dos une monstrueuse escroquerie. On te trompe, peuple de France. Les disciples français du Nazisme sont les agents de la réaction capitaliste. Ils vitupèrent la ploutocratie ; c’est pour mieux sauver ses privilèges. Tu ne les croirais pas, tu refuserais de les entendre s’ils parlaient leur langage et s’ils se présentaient à toi tels qu’ils sont. Alors ils se maquillent et choisissent pour te parler les mots que tu as si souvent employés toi-même, les grands mots de socialisme, de révolution qui expriment tes aspirations et ton effort vers une vie meilleure et un monde plus juste.\par
Pour eux ces mots-là, qu’ils prostituent en s’en servant, sont le « Sésame ouvre-toi » politique, grâce auquel ils prétendent forcer les portes du pouvoir au nom de leurs maîtres ploutocrates français et ploutocrates allemands. Ils sont le passeport truqué, grâce auquel ils s’imaginent pouvoir obtenir l’audience favorable des foules qu’il s’agit de rendre dociles à l’oppression nationale. Cet hommage bien involontaire, ils ont dû faire la grimace avant de le rendre. Ces formules “Révolution”, “Socialisme” leur écorchent encore la bouche, ils les articulent en bredouillant. Mais il fallait en passer par là pour sauver le vieil ordre et les vieux privilèges, pour servir les maîtres nazis. Les vieux et éternels ennemis du peuple, les champions de la contre-révolution et les pourfendeurs de socialisme se sont travestis et ont modifié leur vocabulaire. Ils se sont imaginés que tu t’y laisserais prendre, que tu ne découvrirais pas le bout de leurs grandes oreilles. Comme ils te connaissent mal.\par
En d’autres contrées, en Allemagne singulièrement, d’autres qui étaient autrement effrontés que les spadassins du « Rassemblement national » et les hommes de Vichy, et qui surent pratiquer avec plus d’impudence la démagogie éhontée, réussirent ce tour de passe-passe. Il faut dire qu’ils avaient le bénéfice de l’imprévu. Nos contre-révolutionnaires et nos réacteurs grimés en soldats de la Révolution nationale sont, eux, de pâles imitateurs et de douteux plagiaires. Ils ne peuvent pas vanter l’originalité de leur marchandise, car leurs maîtres directs exigent d’eux qu’ils révèlent constamment son origine. Tout au plus leur permet-on d’écrire et de dire que leur Révolution nationale “adaptera” le nazisme à la France. C’est cette adaptation qu’ils appellent le socialisme.
\section[{Le socialisme à la mode nazie}]{Le socialisme à la mode nazie}
\noindent Aussi bien, Marcel Déat et Coco Fontenoy se sont-ils prudemment gardés de se risquer à une définition de leur   socialisme. N’ont-ils donc point cherché dans la littérature nazie comment, là-bas, on définissait le socialisme ? C’est une investigation instructive. Essayons de nous y livrer à leur place. Hitler a défini sa politique dans un ouvrage copieux, \emph{Mein Kampf}, auquel il se réfère constamment. Il est remarquable qu’aucun développement de cet abondant évangile n’ait été consacré au socialisme dont le grand prêtre se prétendait le champion. Plus tard, en 1937, dans un de ses discours, Hitler devenu Chancelier a posé à ses auditeurs cette question : « \emph{Existe-t-il un socialisme plus magnifique que ce socialisme dont l’organisation permet à chacun parmi des millions de garçons allemands, si la Providence veut se servir de lui, de trouver la voie juste jusqu’à la tête de la Nation} ? »\par
Le socialisme a donc pour tâche d’aider la Providence et la Providence a pour rôle d’utiliser le socialisme ! Drôle de providence, et drôle de socialisme ! D’autres prophètes, plus heureux que le Führer, ont-ils donné une définition plus concrète de ce “socialisme” ? Oh, les définitions ne manquent pas ! Un compilateur des écrits nazis en a découvert une centaine. Mais toutes sont du goût de celles-ci : « \emph{Le Socialisme}, dit Bernard Kohler, \emph{est la défense morale du peuple, de même que le National-socialisme est sa défense physique.} » (!) « \emph{C’est le Socialisme allemand}, écrit von Tschammer, \emph{qui réapparaît dans l’esprit militaire.} » Explication qui s’apparente à celle de robert Ley, chef du Front du Travail, pour qui : « \emph{Le meilleur ordre socialiste est l’ordre militaire.} » Le même Robert Ley dit aussi : « \emph{Socialisme, c’est le sang et la race, la foi sacrée et profondément sérieuse en un Dieu.} » « \emph{Le Socialisme}, dit Göbbels, \emph{c’est la doctrine libératrice du savoir-faire.} » « \emph{La cathédrale de Cologne, voilà le socialisme allemand} », prononce un autre. Et Rosenberg, dans son Mythe du vingtième siècle, va jusqu’à proclamer : « \emph{Le Socialisme, c’est la police.} »\par
Et l’on pourrait poursuivre : le socialisme, c’est tout, et ce n’est rien ; le socialisme, c’est n’importe quoi. C’est ainsi, d’ailleurs, que l’entendent, après les nazis d’Allemagne, les nazillons du « Rassemblement national ». Ils tentent de s’emparer d’un mot justement évocateur pour faire triompher leurs desseins équivoques. Ne leur demandez pas de définir   le mot, d’en expliquer le contenu. Ils ne savent pas. Ou mieux, ils ne peuvent pas. Car leur supercherie éclaterait au grand jour et leur construction de carton pâte s’effondrerait aussitôt, car aux yeux des plus naïfs, il apparaîtrait que le socialisme, c’est exactement le contraire de ce que font et de ce que veulent nazis d’Allemagne et nazillons de France. Ce qu’ils appellent la Révolution nationale est la plus sordide entreprise de réaction et de Terreur blanche ; c’est l’ordre ancien et ses privilèges immoraux et ses scandaleuses injustices qu’ils baptisent « ordre nouveau ». Et ce qu’ils nomment socialisme c’est tout ce contre quoi le socialisme s’inscrit en irréductible ennemi, c’est ce que le socialisme se propose d’abattre.
\chapterclose


\chapteropen
\chapter[{3 – Qu’est-ce que le socialisme ?}]{3 – Qu’est-ce que le socialisme ?}\renewcommand{\leftmark}{3 – Qu’est-ce que le socialisme ?}


\chaptercont
\noindent Qu’est-ce que le socialisme, en effet ? Ce n’est pas tout et rien. Ce n’est pas n’importe quoi. Ce n’est pas une formule magique, un mot sonore, apte à faire naître une mystique confuse et un peu mystérieuse. C’est tout autre chose. Ce mot a un contenu vivant et précis. Des données très précises distinguent le socialisme de ce qui n’est pas le socialisme, le militant socialiste du charlatan maquillé en commis voyageur de la « Révolution nationale ».\par
Le socialisme, c’est la suppression du parasitisme social, c’est la fin de l’exploitation de l’homme par l’homme, c’est la propriété sociale des moyens de production.\par
Être révolutionnaire, être socialiste, ce n’est pas lancer des imprécations d’hystériques sur l’intérêt général, le judéo-maçonnisme et autres fantaisies ; c’est se rendre à cette vérité d’évidence : \emph{il n’est qu’un moyen de résoudre la contradiction qui mine la société, c’est la substitution à l’ancien régime d’un régime où l’état social s’harmonisera avec les forces de production}.\par
Être socialiste, ce n’est pas débiter de pieuses âneries sur l’organisation de la profession et du métier ; c’est lutter pour une société qui mette fin à l’exploitation de l’homme par l’homme.\par
Nous verrons que cette société, les nazis d’Allemagne ne l’ont pas fait naître sur les bords de la Sprée et que les nazillons   français n’ont nullement l’intention de la construire sur les rives de la Seine ou celles de l’Allier.\par
Ils nous affirment, il est vrai, que leur régime substituera à la domination de l’or la primauté du travail. Ils claironnent que leur victoire marquera la fin du règne de l’or. Si vous leur demandiez de préciser leur pensée, ils devraient vous avouer que, dans le système qu’ils conçoivent, l’or continuera à jouer un rôle essentiel dans le commerce intercontinental. Quant à la primauté du travail, que signifie au juste leur formule ? Annonce-t-elle la fin du vol de la plus-value ? En aucune façon. Nous touchons à l’un des plus remarquables exemples de la mystification nazie : l’ère de l’or est finie, proclame-t-on : voilà la révolution ! Oui, mais quand on y regarde d’un peu plus près, on s’aperçoit : 1° que l’or “déchu” continuera à régenter le commerce international ; 2° que le travail “victorieux” continuera à être soumis aux règles de l’exploitation et du profit ; 3° que la richesse produite par ce travail prétendument “victorieux” continuera à être accaparée par une minorité privilégiée.\par
\section[{Charlatans saumâtres}]{Charlatans saumâtres}
\noindent En attendant, des charlatans saumâtres nous expliquent que le fin du fin est de supprimer la lutte des classes. Ils ont tout dit quand ils ont dit ça. Ils veulent supprimer la lutte des classes par décret, en l’interdisant par affiches, comme on interdit certains apéritifs, certains jeux de hasard, la circulation dans Paris minuit passé, et le franchissement de la chaussée en dehors des passages cloutés. Nos charlatans saumâtres, s’ils n’étaient pas de vulgaires coquins, seraient de petits ignorants et mériteraient le bonnet d’âne. Tant qu’il y aura des pauvres et des riches, des exploités et des exploiteurs, ces catégories antagonistes se heurteront. On ne connaissait pas encore les théories du socialisme lorsque les esclaves romains se révoltaient sous la direction de Spartacus ; tout le Moyen Âge a été marqué des Jacqueries, ces révoltes des paysans contre les seigneurs féodaux.\par
Il n’est de société sans classes que la société sans exploitation de l’homme par l’homme. Supprimer l’exploitation de l’homme par l’homme, c’est tarir la source du profit capitaliste.   Les nazillons n’y songent pas un instant. Leurs modèles, les nazis, n’y ont jamais songé.\par
L’appropriation individuelle des instruments de production par les détenteurs de l’argent ne correspond plus au caractère social, collectif du mode de production. Et comme il ne peut être question de revenir à l’artisanat ou à la manufacture, l’harmonie nécessaire entre le mode de production et le mode d’appropriation ne peut être établie que par la remplacement de la propriété individuelle par la propriété sociale.\par
C’est ce qu’exprime ce très beau raccourci de Jaurès :\par

\begin{quoteblock}
 \noindent « Nous voulons l’abolition du salariat et, comme il est impossible, maintenant que le temps du rouet, du fuseau, du marteau à la main, de l’instrument individuel de production, est passé pour faire place au grand mécanisme collectif de la production, comme il est impossible d’assurer à chaque travailleur la propriété de son outil individuel, c’est par la transformation de la propriété capitaliste en propriété collective que nous voulons poursuivre la transformation sociale. »
 \end{quoteblock}

\noindent Oh nous savons bien l’objection des ignorants qui se donnent pour de beaux esprits ! « Vous retardez, nous disent-ils, avec votre B A BA du socialisme. Tout cela est bel et bien dépassé aujourd’hui. Les fondateurs du socialisme vivaient à l’autre siècle. Comment auraient-ils pu prévoir les formes de l’économie moderne et les bouleversements qui ont secoué le monde ? »
\section[{L’époque de l’impérialisme}]{L’époque de l’impérialisme}
\noindent Voici notre réponse :\par
Si le socialisme n’avait pas conservé un si grand prestige, s’il n’était pas la seule formule d’affranchissement humain, ceux qui rêvent pour des desseins suspects de se gagner la sympathie des masses populaires n’arboreraient pas le drapeau du socialisme.\par
Au surplus, s’il est bien vrai que les fondateurs du socialisme scientifique n’ont pu prévoir dans les détails les phénomènes économiques surgis depuis cinquante années, ils en avaient indiqué les prémisses avec une grande lucidité, et leurs continuateurs ont étudié très scrupuleusement ces phénomènes en appliquant à leur étude la très sûre méthode du matérialisme dialectique. Et à quelles conclusions ont abouti ces continuateurs qui s’appellent Lénine et Staline ? De quelle contribution ont-ils enrichi le socialisme ? Ils ont expliqué que nous étions entrés, avec le vingtième siècle, dans l’époque de l’impérialisme.\par
L’époque de l’impérialisme est celle où le heurt entre les forces productives et le régime social atteint sa plus grande acuité. Le régime capitaliste cesse d’être progressif. Ses idéologues se font alors les négateurs du progrès. À l’époque de l’impérialisme, l’effort de la raison, de l’analyse scientifique, ne peut que démontrer le caractère irrationnel de ce système. Les idéologues de l’impérialisme affirment que « \emph{les vrais chefs n’ont nullement besoin de culture et de science} » (Hermann Göring). Au développement de la raison, on oppose les mythes grossiers de la Terre et du Sang. À l’époque de l’impérialisme, enfin, c’est-à-dire à l’époque de la domination d’une oligarchie financière étroite, la couche sociale intéressée au maintien du système capitaliste devient de moins en moins nombreuse.
\section[{Recours au terrorisme}]{Recours au terrorisme}
\noindent Pour assurer leur pouvoir, les oligarchies doivent renoncer au système de gouvernement démocratique. Elles recourent au terrorisme. Elles deviennent anti-démocratiques et réactionnaires. Pour mener à bien leur entreprise, il leur faudra tenter de détourner les masses ouvrières de la voie révolutionnaire, de les maintenir sous la domination de la grande bourgeoisie. C’est dans ce but qu’elles créeront partout où elles pourront des mouvements fascistes ; qu’elles leur conseilleront, pour tromper les masses ouvrières, de se servir de mots d’ordre anti-capitalistes. C’est ainsi qu’à l’époque de l’impérialisme, en combinant une démagogie anti-capitaliste chauvine, anti-sémite avec un terrorisme forcené contre la classe ouvrière, le fascisme a pu, dans un certain nombre de pays, dresser un barrage contre le socialisme, assurer la prolongation du système de domination des oligarchies capitalistes. Telle était la mission des nazis rassemblés par Adolf Hitler. Telle est la mission des nazillons qu’essayent de rassembler Marcel Déat, Deloncle et Jacques Doriot.
\section[{Ce n’est pas du socialisme}]{Ce n’est pas du socialisme}
\noindent Cette entreprise, on peut la baptiser de tous les noms qu’on voudra. Elle n’a rien à voir avec le socialisme ; elle est le contraire du socialisme\footnote{Staline, dans son rapport au dix-septième congrès du Parti bolchévique, caractérise le fascisme « \emph{comme un signe montrant que le grand capital n’est plus en mesure d’exercer son pouvoir au moyen des anciennes méthodes de parlementarisme et de démocratie, ce qui l’oblige à recourir dans sa politique intérieure aux méthodes de domination terroristes} ».}. Le socialisme nous révèle la contradiction qui creuse le tombeau du capitalisme. Le nazisme est né de l’effort du capitalisme décadent pour imposer son pouvoir par la terreur. Le socialisme révèle aux exploités le secret de leur exploitation et le moyen de briser cette exploitation. Le nazisme est la tentative tout à la fois sournoise et violente des couches réactionnaires de la bourgeoisie pour détourner les exploités de la voie de leur affranchissement.\par
Le socialisme repose sur deux assises fondamentales : la propriété sociale des moyens de production, dans le cadre du respect de la petite propriété individuelle, fruit du travail personnel ; la suppression de l’exploitation de l’homme par l’homme.\par
La Société Socialiste, c’est celle où les moyens de production appartiennent soit à la société tout entière, soit à la collectivité des producteurs qui les gère. La Société Socialiste, c’est celle où les moyens de production qui permettaient l’exploitation du travail d’autrui étant devenus propriété sociale, la classe exploiteuse n’existe plus. La Société Socialiste est celle où la classe travailleuse (ouvriers et paysans) a cessé d’être exploitée par les possesseurs des grands moyens de production, pour devenir la propriétaire collective de ces moyens de production, affranchie de toute exploitation.\par
Être socialiste, c’est lutter pour cette société-là ! Est-ce cette société que se proposent de construire les démagogues ivres du “Rassemblement” antinational et impopulaire ? Ce qui est certain, c’est que leurs maîtres de Berlin ne l’ont pas construite. Ils lui ont tourné le dos.
\chapterclose


\chapteropen
\chapter[{4 – Discours anticapitalistes et subvention des magnats}]{4 – Discours anticapitalistes et subvention des magnats}\renewcommand{\leftmark}{4 – Discours anticapitalistes et subvention des magnats}


\chaptercont
\noindent On conçoit aisément que le socialisme tel qu’il est, tel que nous venons de le définir, le socialisme vrai, dont les communistes sont les seuls champions authentiques, ait été depuis toujours le cauchemar des ploutocrates.\par
Le socialisme, parce qu’il se propose de dépouiller les oligarchies de l’instrument d’exploitation, est pour celles-ci l’objet de la plus vive inquiétude. Contre lui, elles ont multiplié les obstacles. Elles ont usé de la puissance que leur confère la richesse pour combattre le mouvement socialiste ; elles ont usé de l’emprise qu’elles exercent sur l’État pour briser les organisations ouvrières, etc.\par
Si le nazisme, dont les hauts-parleurs du RNP prétendent suivre l’exemple et imiter le modèle, avait été le socialisme ; s’il avait été, à tout le moins, une variété, une forme de socialisme, ses progrès auraient provoqué l’indignation des classes possédantes. Contre lui, elles auraient multiplié les entraves. Elles auraient tout essayé pour ruiner le mouvement. Elles auraient, à prix d’or, suscité contre lui des mouvements rivaux, etc. Bref, elles auraient déployé l’effort de corruption que partout et toujours des exploiteurs déploient contre ceux qui menacent leurs privilèges.\par
Or, que s’est-il passé en Allemagne ? Il s’est passé ceci, sur quoi Déat et Doriot observeront, si vous les interrogez, un silence pudique : les possesseurs et les manieurs d’argent, les magnats de l’industrie lourde, les forces les plus actives du capitalisme ont subventionné le mouvement nazi, financé ses journaux, entretenu sa caisse de propagande. Pourquoi ? Parce que le nazisme, loin d’être une forme du socialisme, a été pour la grande bourgeoisie allemande – ce que le RNP tente d’être pour la grande bourgeoisie française – l’instrument de tentative de sauvetage de l’ordre ploutocratique.\par
\section[{Ce jongleur de Mussolini}]{Ce jongleur de Mussolini}
\noindent Sans doute, les dirigeants nazis ont-ils eu soin, toujours, d’entourer de mystère tout ce qui concernait les ressources du Parti. En vertu du “Führerprinzip” (principe du Führer), Hitler s’abstenait de révéler à ses collaborateurs les plus proches l’origine de certaines souscriptions ; mais dans les années qui précédèrent son avènement au pouvoir, chaque fois que s’amorçait un débat public sur les moyens financiers du Parti, Hitler préférait transiger avec ses “diffamateurs” plutôt que de leur intenter un procès. Ce qui ne l’empêchait pas de célébrer la « vertu nazie » et d’accuser d’autres fascistes – les fascistes italiens par exemple – de n’être pas comme lui vêtus de probité candide et de lin blanc ! Le 29 juillet 1922 – treize ou quatorze ans avant l’Axe – le Völkischer Beobachter écrivait que « \emph{Mussolini avait trahi ses camarades parce qu’il avait été corrompu par les représentants de la Société du Gaz de Zurich} ». Le journal ajoutait que le Popolo d’Italia, le journal de Mussolini, avait été fondé avec de l’argent juif et il invitait les « nationaux-socialistes » à « \emph{éviter strictement tout rapport avec ce jongleur de Mussolini} ». Mais les sources où s’abreuvait Hitler n’étaient pas plus pures que celles où Mussolini avait étanché sa soif. L’un et l’autre s’étaient désaltérés aux fontaines du grand capital. L’un et l’autre ont été choisis par les tenants du vieil ordre comme les chiens de garde de leurs privilèges.
\section[{L’Allemagne entre les deux guerres}]{L’Allemagne entre les deux guerres}
\noindent Pour la bonne compréhension de ce qui va suivre, essayons de nous représenter l’Allemagne entre les deux guerres. En 1918, ce pays a vu éclore des Soviets d’ouvriers et de soldats. Mais la Révolution a été écrasée dans le sang par les efforts conjugués de l’Entente et des chefs traîtres de la social-démocratie, lesquels ont fait appel aux junkers et aux officiers de l’ancienne armée. Le courant populaire a été si puissant cependant que les maîtres du pouvoir durent accorder quelques concessions à la classe ouvrière : assurances sociales, contrats collectifs, conseils d’entreprises, droit d’organisation des ouvriers agricoles. Toutefois, les potentats de la grande industrie, aussi bien que les hobereaux, sont parfaitement décidés à reprendre ce qu’ils ont accordé. Pour commencer, ils sabotent l’application des lois sociales ; ils s’emploient à briser les grèves. Ils créent pour cela des corps spéciaux, des ligues de combat, sorte de police prétorienne aux ordres du grand capital. La plus agissante de ces ligues crées et mises au monde par les capitalistes allemands, c’est le Parti National-Socialiste d’Adolf Hitler. Pendant toute une période, ces ligues, que le grand capital a eu l’occasion d’expérimenter lors du putsch de Kapp, sont tenues en réserve. L’heure n’a pas sonné encore de les lancer dans la bagarre. Il attend que les capitaux américains et anglais aient afflué en Allemagne, que l’industrie allemande se soit rééquipée. Seulement, lorsque cet équipement est achevé, lorsque l’Allemagne s’apprête à inonder le monde de ses produits, le monde est secoué par la grande crise économique de 1930. Cette crise touche directement les industriels allemands. Leur grande préoccupation, dès lors, va être de réduire les salaires ouvriers, d’abroger la législation sociale, d’en finir avec le système des contrats collectifs. Ils doivent, pour cela, disposer d’un État docile à leurs vœux. Les corps-francs sont tirés de l’ombre ; le parti national-socialiste est poussé à l’avant-scène. Fritz Thyssen et Kirdorff de la métallurgie de Gelsenkirchen augmentent le chiffre de leurs subventions. Les gouvernements de Brüning, de Von Schleicher, de Papen cèdent à la pression des agitateurs nazis, puis ils leur abandonnent la place.\par
Ne nous y trompons pas, cependant, la volonté et l’argent de quelques magnats n’auraient point suffi à assurer la victoire de cette équipée. Pour qu’elle réussît, ses initiateurs devaient capter la confiance d’une troupe importante, se donner – ce que Déat et Doriot cherchent en vain dans la France de 1941 – une base de masse. Comment s’y prirent-ils ? La question est d’importance. Là-bas, une formidable entreprise de viol des foules a réussi. Réussira-t-elle chez nous ? Ce qui est certain, c’est que l’un des moyens d’empêcher son succès, c’est de révéler les procédés qu’elle employa, de dire comment elle mystifia ceux à qui elle s’adressait, et c’est de montrer que ceux-là furent victimes d’une des plus gigantesques escroqueries de l’histoire.
\section[{Des millions d’êtres dans la misère}]{Des millions d’êtres dans la misère}
\noindent Mais qui étaient-ils au juste, ceux auxquels s’adressait l’agitation nazie ? C’étaient les classes moyennes d’abord, les épargnants, les détenteurs de revenus fixes saignés à blanc par la chute du mark, les petits commerçants pressurés par les banques, écrasés par les magasins cartellisés. C’étaient les paysans endettés, que le fléchissement des prix condamnait à la misère, à partir de 1929 ; les petits paysans s’entend, car pour ce qui est des hobereaux, la politique douanière du gouvernement les avait mis à l’abri des risques les plus graves. À ces catégories sociales ajoutez les anciens combattants et parmi eux la foule des anciens officiers et sous-officiers à qui la fin de la guerre et la défaite ont fait perdre leur emploi ; la jeunesse enfin, la tragique multitude des jeunes chômeurs et des jeunes étudiants sans emploi, tous avides d’action, de mouvement, de renouvellement.\par
Où vont aller ces millions d’hommes et de femmes sur qui la misère s’est abattue comme un manteau de plomb et que déjà la famine guette ? Où vont-ils aller ?\par
Ah certes, si contre le régime qui engendre l’effroyable détresse, la classe ouvrière s’unissait comme tentent de l’unir les communistes, la réponse ne serait pas douteuse. Le prolétariat uni serait pour la petite bourgeoisie une irrésistible force d’attraction. Autour de cette force se grouperaient les victimes de la guerre, les jeunes dont les vingt ans réclament avec tant d’impatience l’action, la rénovation de l’Allemagne. Alors serait mobilisée la formidable armée qui ne s’égarerait pas, ne prendrait pas le mauvais chemin, mais s’attaquerait à la cause du mal : elle désignerait aux multitudes le but à atteindre, le socialisme.
\section[{La division ouvrière favorise le nazisme}]{La division ouvrière favorise le nazisme}
\noindent Oui, mais la classe ouvrière est divisée, la social-démocratie, non contente de repousser les propositions d’action commune des communistes, charge son préfet de police Zorgiebel de faire tirer sur la foule ouvrière de Berlin, le premier mai 1929. Elle s’accroche à la capote de Hindenburg, à la soutane de Brüning, à la jaquette de von Papen.\par
Par sa faute inexpiable, c’est vers d’autres forces que va se diriger la cohorte des victimes. Vers quelles forces ? Vers celles qui ont été mises sur pied par les responsables de ses misères.\par
Ces forces existent ; ce sont le Parti nazi, ses sections de protection, ses sections d’assaut dont le grand capital a alimenté les caisses. Magnats de l’industrie, féodaux de la terre ont réalisé ce tour de force : ils ont groupé les victimes du régime dans des organisations dont la raison d’être et la mission était de sauver ce régime.\par
Nous verrons comment cette supercherie fut possible et quel langage tinrent, à cette foule qui avait faim, les chefs visibles du mouvement. Mais qu’importait le langage. Les autres chefs, invisibles ceux-là, tenaient les cordons de la bourse. Ils étaient les vrais maîtres du mouvement nazi ; ils savaient qu’ils le conduiraient où ils voudraient et ils lui avaient assigné la mission de sauver des privilèges que la tourmente menaçait d’emporter.
\section[{Bailleurs de fonds de Hitler}]{Bailleurs de fonds de Hitler}
\noindent Ces chefs invisibles, ces bailleurs de fonds, quels étaient-ils ? Certains ne vivaient pas sous le ciel allemand. C’est ainsi que, parmi les protecteurs de la première heure du mouvement nazi figurent : le pétrolier anglais Sir Henri Detterding, le potentat américain Ford et quelques financiers suédois liés à la famille de Göring. Parmi les protecteurs allemands, il convient de citer M. Aust, président de l’Union des Industriels bavarois ; M. Kuhlo, avocat-conseil de cette union ; Von Epp qui, en compagnie de Röhm, se procura dans les milieux financiers de Munich les fonds nécessaires à l’achat du journal le Völkischer Beobachter ; le grand industriel Borsig ; le fabricant de dentelle Mutscham ; le chef du trust du charbon de Rhénanie Kirdorff ; l’industriel Thyssen ; l’administrateur du grand trust de la potasse Wintershall de Cassel ; le trust Lahussen-Nordwolle ; l’union minière Bergbanveren d’Essen ; la maison Otto Wolf ; le trust des cigarettes Reemsta ; de grands propriétaires terriens ; enfin les anciens princes allemands : l’ex-Kaiser, le prince Auguste-Guillaume de Hohenzollern, le prince Christian de Schaumburg-Liffe, le duc de Saxe-Cobourg Gotha, les grands-ducs d’Oldenburg, de Mecklemburg, de Hesse, le duc Ernest-Auguste de Brünswick. Et ce fut pour être agréable à ces « nom de Dieu de princes », pour les payer comptant, que les députés nazis, tout en vitupérant le capitalisme, votèrent au Reichstag contre la loi d’expropriation des anciennes familles princières !\par
Voilà quels noms portent les vrais maîtres du mouvement nazi. Ces bailleurs de fonds, dont les journaux de Marcel Déat et de Doriot ne publieront jamais le palmarès, ne gaspillaient pas leurs deniers. Ils ne prêtaient pas à fonds perdus. Ils réalisaient un très bon placement. Ils se ménageaient une assurance. En prélevant un peu sur leurs bénéfices, ils entendaient assurer la protection de leur régime, de leurs privilèges et tirer une traite sur l’avenir.\par
Ils n’ont pas été déçus, mais remboursés avec usure et récompensés au centuple !
\chapterclose


\chapteropen
\chapter[{5 – L’escroquerie du programme : Bons et mauvais capitalistes}]{5 – L’escroquerie du programme : Bons et mauvais capitalistes}\renewcommand{\leftmark}{5 – L’escroquerie du programme : Bons et mauvais capitalistes}


\chaptercont
\noindent Les hauts-parleurs du « Rassemblement national populaire » balbutient et bredouillent. Mettez-vous à leur place. Ils ont mijoté dans les vieilles marmites des 200 familles ; ils ont été ministres de Tardieu, ou de Doumergue, ou de Flandin, ou de Daladier ; ils ont attaché leurs noms à des décrets-lois de déflation qui, pour sauver le superflu des riches, rognaient sur le nécessaire des pauvres. Tel est le cas de Laval, avocat marron qui fut l’agent électoral du financier Octave Homberg ; de Marcel Déat à qui il ne répugnait pas d’être collaborateur de deux journaux, dont l’un : L’Œuvre, était financé par la Banque Lazard, et l’autre : La République, touchait au guichet des Compagnies d’Assurances ; de Rivollet (fâcheusement compromis par ailleurs dans un scandale de Loterie d’anciens combattants), qui fut ministre des Pensions et signa les décrets qui frappaient d’un prélèvement les pensions des anciens combattants ; de Cathala, qui fut ministre de Tardieu ; de Besset, qui fut ministre de Daladier.\par
\section[{Premiers vagissements du RNP}]{Premiers vagissements du RNP}
\noindent Ils ont passé une partie de leur vie à dénoncer la Révolution, à exorciser le socialisme, à frémir chaque fois que le prolétariat entamait un des privilèges sacrés du capitalisme. Aujourd’hui, pour servir les mêmes forces d’argent qu’hier, il leur faut modifier du tout au tout leur vocabulaire, se couvrir d’une vêture nouvelle. Ils bégaient, ils hésitent, ils ont peur. Voilà pourquoi leur déclaration-programme est si mièvre, si pâle, si anémiée et souffreteuse.\par
Nos gaillards désirent « une économie dirigée à base corporative ». Ils suggèrent que la monnaie soit « garantie par le travail ». Ils n’écrivent pas une fois le mot socialisme. Ils veulent bien manipuler des pétards de feu d’artifice ; ils hésitent à faire fonctionner le 420 de la démagogie. Il faudra qu’ils s’y résignent cependant, car leurs formules proprettes et chétives, si elles rassurent pleinement leurs maîtres, n’exerceront aucun attrait sur les masses populaires. Pour essayer de séduire ces masses, il faudra trouver autre chose. Ils chercheront. Ils chercheront dans le Programme de leurs modèles les nazis d’Allemagne.\par
Et que trouveront-ils ? Il faut que nous soyons ici prévenus. Il est possible que dans les semaines à venir, les hommes du « Rassemblement national » mettent en circulation des slogans sonores à allure très radicale et très subversive. Il n’est pas d’autre moyen pour eux d’exécuter la besogne dont l’occupant et les amis de l’occupant les ont chargés. C’est ainsi et pas autrement qu’ils tenteront de mystifier le public, de lui jeter de la poudre aux yeux ; c’est contre cette mystification et cette escroquerie que nous mettons en garde les travailleurs français.
\section[{Panneau-réclame des nazis}]{Panneau-réclame des nazis}
\noindent Supposons que, forçant la note et haussant le ton, les hommes du Rassemblement substituent prochainement à leur Déclaration de principe minable et décolorée un panneau-réclame claironnant et haut en couleurs. Que faudra-t-il en penser ? Pour former à son sujet un jugement sain, le mieux sera de se reporter au panneau-réclame que les agitateurs nazis allemands – les modèles du RNP – ont promené à travers l’Allemagne entre 1929 et 1933. Que disaient-ils alors ? Et quelle était la portée de leur programme ? N’oublions jamais que pour parvenir à leurs fins, les nazis d’Allemagne sont allés dans la voie de la démagogie aussi loin que peut aller un parti mis au monde par les oligarchies, et dont le sauvetage des oligarchies est le but suprême encore que soigneusement dissimulé aux foules.\par
Ils se sont présentés comme les redresseurs du socialisme, comme les amis et les protecteurs des syndicats ouvriers ; ils n’ont pas hésité à proclamer que l’arme de la grève était pour les travailleurs une arme légitime : « \emph{Le national-socialisme}, écrivait le “Völkicher Beobachter ”, \emph{reconnaît sans restriction le droit de grève… C’est mentir honteusement que de dire que les nationaux-socialistes, lorsqu’ils auront pris le pouvoir, enlèveront aux travailleurs leur arme suprême : le droit de grève} ».\par
Nos RNP ne tiennent pas encore ce langage. Ils en sont encore aux vagissements prémonitoires. Mais en 1929, en 1930, non contents de prêcher le “redressement” du socialisme, de déclarer sacré le droit de grève, arme suprême des ouvriers, les nazis d’Allemagne parlaient confusément d’une réforme profonde du droit de propriété. Devant les petits paysans, ils allaient jusqu’à envisager le partage des terres. Ils faisaient la grosse voix contre les agrariens (qui subventionnaient la caisse nazie). Ils disaient : « La grande propriété, dans l’Est, doit disparaître pour la plus grande partie. On ne peut conserver la grande propriété par respect pour les traditions. »
\section[{Ce programme ? Une duperie}]{Ce programme ? Une duperie}
\noindent Arrêtons-nous un moment. Nous pressentons l’objection que l’on nous adresse : que vous le vouliez ou non, nous dit-on, des promesses de ce genre sont tout de même assez précises. Elles ont trait à une transformation assez profonde de la structure capitaliste. Il est possible qu’elles n’aient pas été tenues, nous ne voulons pas le savoir pour l’instant. Elles n’en contenaient pas moins un engagement de transformer le vieil ordre social capitaliste. Le programme, en somme, n’était pas mauvais et si le RNP nous en présente un semblable, pourquoi lui bouderions-nous ?\par
Et voici notre réponse : Non, ce programme, ce n’était pas le programme socialiste et c’était, par conséquent, une duperie.\par
Les nazis d’Allemagne (et vous verrez que les nazillons de France suivront leur exemple) ont pris à partie, avec beaucoup de virulence oratoire il est vrai, les à-côtés du régime ploutocratique. Ils n’ont jamais touché à la racine du mal. Ils n’ont jamais porté une main sacrilège sur le système d’où les oligarchies tirent la source de leur super-profit. Ils ont dénoncé à la colère des masses quelques rouages de la machine, c’est-à-dire que dans cette Allemagne que le socialisme aurait pu sauver de la misère, le nazisme a rempli très exactement la mission que lui avaient confiée ses maîtres : sauver le vieux régime et détourner les multitudes de la lutte pour le socialisme.\par
Soyez sûrs que c’est cette besogne-là qu’accompliraient, si nous les laissions faire, les chefs de bande du RNP.\par
Mais pour réussir ce formidable subterfuge, comment ont procédé les nazis d’Allemagne ?
\section[{Le capitalisme… étranger, ennemi principal}]{Le capitalisme… étranger, ennemi principal}
\noindent Aux masses frémissantes d’Allemagne, ils ont désigné comme ennemi principal le capitalisme… étranger. C’est le capitalisme étranger qui a imposé à l’Allemagne l’injuste traité de Versailles ; c’est lui qui prélève sur le peuple allemand le terrible fardeau des réparations, etc. Donc, sus au capitalisme international ! Discours habile auquel applaudissaient à la fois ceux qui s’étaient endettés à l’étranger, ceux qui rêvaient du réarmement et de la revanche allemande et ceux qui détestaient l’exploitation capitaliste. À cette propagande, la politique insensée des vainqueurs de 1918 apportait un renfort précieux. Des opérations comme l’occupation de la Ruhr en 1923 servirent puissamment l’agitation nazie. Et c’est ainsi que la lutte juste, sans aucun doute, du peuple allemand contre le diktat de Versailles, cette lutte qui avait éveillé la sympathie des prolétaires de tous les pays, cette lutte, monopolisée par le Parti nazi, fut orientée par lui dans le sens du chauvinisme agressif. La colère légitime des masses contre le capitalisme fut muée, par l’alchimie national-socialiste, en une lutte contre Versailles, puis en une excitation véhémente à la guerre pour la suprématie du monde.
\section[{Deuxième stade : l’antisémitisme}]{Deuxième stade : l’antisémitisme}
\noindent Mais le capitalisme étranger était une pâture insuffisante. Les chefs nazis eurent tôt fait de le comprendre. Alors, ils désignèrent un autre ennemi : le capitalisme juif. Et dans leur laboratoire, ils transformèrent l’anticapitalisme des masses en un antisémitisme barbare et grossier. On identifie au juif le banquier, le dirigeant du magasin à pris unique, le créancier anglo-saxon. Nul ne dira combien cet antisémitisme imbécile a servi le système capitaliste allemand dans son ensemble. Non seulement les capitalistes allemands ont été les bénéficiaires des spoliations et des rapines dont furent victimes les capitalistes juifs du Reich et d’Autriche, mais surtout l’antisémitisme a été là-bas, comme partout, le procédé habile pour détourner la colère populaire de la lutte contre le régime d’exploitation des oligarchies. Le juif a été dénoncé comme le mauvais capitaliste. C’est donc qu’il y a de bons capitalistes ? C’est donc que Krupp est un capitaliste très convenable, si Rotschild est un capitaliste détestable ? C’est donc que le régime capitaliste n’est pas condamnable en bloc ; c’est donc qu’il n’est pas question de le condamner ou de le renverser, mais d’assurer sa survie en massacrant les juifs.
\section[{Sus au capitalisme de prêt}]{Sus au capitalisme de prêt}
\noindent De fait, les nazis d’Allemagne ont poussé très loin cette théorie de la division entre le bon et le mauvais capitaliste\footnote{(Note de l’auteur) C’est notamment le thème d’un très mauvais film allemand (un film à boycotter !) donné récemment dans les salles parisiennes : \emph{Les Rapaces}, où s’étale le contraste bébête entre le mauvais capitaliste, un juif cela va de soi, et le bon capitaliste (un de Wendel, un Mercier, par exemple) qui, répondant à l’appel de sa fille, distribue ses générosités sur le monde.}. Le bon capitaliste, c’est l’industriel, le mauvais capitaliste, c’est le prêteur d’argent. Sus au « capitalisme de prêt" ! Les agitateurs nazis s’en vont à travers le pays expliquer que « l’esclavage de l’intérêt » est le mal des maux. « Abolissons l’esclavage de l’intérêt ». Mais les conzerns et les trusts peuvent continuer sans se troubler leur bienfaisante activité.\par
Nous lirons très bientôt sous la plume des pense-petit du RNP des élucubrations de ce genre. Nos hommes ont déjà plagié l’antisémitisme de leurs modèles allemands.\par
De petites feuilles hebdomadaires appellent régulièrement au pogrom. Les équipes de Doriot s’offrent contre argent aux commerçants aryens pour aller saccager les boutiques de leurs concurrents israélites. Les hommes de main du RNP, sous la protection du préfet de police, s’installent dans les immeubles des juifs. Un statut raciste a été promulgué à Vichy. D’autres mesures suivront peut-être. On ne fera croire à personne que l’anticapitalisme a quelque chose de commun avec ces procédés d’apaches. C’est le capitalisme qui gagne à tous les coups. C’est lui, si l’affaire réussissait, qui serait le grand bénéficiaire. Ce truc grossier lui aurait épargné le châtiment et les coups. L’antisémitisme est la planche de salut du capitalisme. Aussi bien, dans ce pays de France, antisémitisme et révolution sont des termes qui s’excluent. L’antisémitisme a toujours été, il demeure l’odieux compagnon de route de la réaction politique et sociale. Le racisme antisémite a ravagé bien des pays de l’Europe ; il a été le déshonneur de la Russie des Tsars, de la Pologne des seigneurs ; il est la honte de l’Allemagne. Il est au monde un pays qui s’est débarrassé de ces vestiges affreux et qui a proclamé la fraternité des races, c’est l’Union Soviétique ; c’est le seul pays où les oligarchies ont été mises hors d’état de nuire.
\section[{Nos nazillons veulent asservir la France}]{Nos nazillons veulent asservir la France}
\noindent Si soucieux qu’ils soient d’imiter leurs modèles d’Allemagne, les nazillons de France devront sur un point, au moins, s’écarter d’eux. Quand les nazis commencèrent leur campagne, ils se posèrent en champions de la libération de leur pays. Ils dirent qu’ils voulaient briser les chaînes d’un injuste traité, affranchir l’Allemagne de la servitude. Sans doute firent-ils de ces nobles mobiles une exploitation impérialiste qui devait accroitre terriblement les dangers de guerre. Du moins appelaient-ils le peuple allemand à se libérer, à regagner son indépendance ; du moins les vit-on lutter pour que le territoire allemand fût délivré de l’occupation étrangère, pour que le peuple allemand ne supportât pas le tribut du vainqueur. Ce qui ne les empêchait pas d’ailleurs de s’offrir parfois au vainqueur comme mercenaires anti-communistes ; tel fut le cas de Hitler. Dans un livre intitulé « Le quai d’Orsay », que son auteur a prudemment retiré de la circulation, M. Paul Allard a établi que le futur chancelier d’Allemagne, à l’époque de l’agitation nazie dans la Ruhr, fut soudoyé par un certain Heinz, autonomiste du Palatinat, homme à tout faire du deuxième Bureau, et qu’entretenaient, au su de Adolf Hitler, les fonds secrets du quai d’Orsay.\par
Les nazillons français, eux, ont été tenus sur les fonds baptismaux par l’armée de l’occupant ; ils se sont placés sous la protection de ses baïonnettes ; C’est pour servir l’occupant qu’ils veulent forcer les portes du gouvernement. Les nazis prétendaient libérer l’Allemagne. En fait ils l’ont précipitée dans la guerre. Nos nazillons, si nous les laissions faire, précipiteraient eux aussi la France dans la guerre. Mais auparavant, ils l’auraient asservie.\par
Les maîtres capitalistes que servaient les nazis voulaient l’abrogation du traité de Versailles. Les maîtres capitalistes que servent les nazillons ont besoin de l’asservissement de la France.
\section[{Un programme attrape-nigaud}]{Un programme attrape-nigaud}
\noindent Jusqu’ici, nous n’avons parlé que du Programme du Parti nazi ; c’est-à-dire de l’ensemble des propositions suggérées par un parti que subventionne une oligarchie capitaliste et qui, pour le compte de cette oligarchie, s’adresse à des masses que la misère accable, que la division ouvrière et la politique criminelle de la social-démocratie détournent de la lutte pour le vrai socialisme. S’adressant à ces masses avec le dessein de les capter et de les séduire, le nazisme, nous l’avons dit, se permet dans son programme certaines audaces.\par
Et pourtant, même dans le Programme, il est gêné aux entournures ; il s’empêtre dans ses contradictions ; il est limité dans sa propension démagogique et il se garde de s’en prendre à l’ensemble du système capitaliste.\par
Même dans le Programme, il se garde de se prononcer pour ce qui caractérise la société socialiste, pour ce sans quoi il n’y a pas de socialisme, pour ce sans quoi il est interdit de se dire socialiste.\par
Même dans le Programme, qui n’est qu’un attrape-nigaud, où la démagogie semble-t-il devrait avoir toute licence, même dans ce Programme plus hardi cependant que l’insignifiant prospectus du RNP, le nazisme se garde de toucher au point névralgique du système ploutocratique.\par
{\itshape Même dans le Programme !}\par
Que sera-ce dans les faits ? Dans les faits, le nazisme se révélera comme un système de renflouement ploutocratique, de régression anti-sociale ; comme l’organisation méthodique du massacre des conquêtes ouvrières.
\chapterclose


\chapteropen
\chapter[{6 – Le Nazisme sans voiles : Une formidable machine de régression sociale}]{6 – Le Nazisme sans voiles : Une formidable machine de régression sociale}\renewcommand{\leftmark}{6 – Le Nazisme sans voiles : Une formidable machine de régression sociale}


\chaptercont
\noindent Ne perdons jamais de vue notre point de départ : Quel est le secret de l’exploitation capitaliste ? Pourquoi et comment le capitalisme exploite-t-il le travailleur salarié ?\par
Dans le régime capitaliste, la force de travail de l’ouvrier est une marchandise. Sa valeur est déterminée par le temps de travail qui y est incorporé. Ce qui caractérise le régime du travail dans le système capitaliste, c’est que l’ouvrier travaille pendant une partie de la journée pour produire, sous forme de salaire, la valeur d’une quantité de subsistance qui lui est nécessaire, et une autre partie pour produire la plus-value, source du profit capitaliste. Survienne un progrès technique qui permette d’augmenter la productivité du travail, cette augmentation accroîtra le profit capitaliste ; elle n’atténuera pas l’exploitation que subit le prolétaire ou l’atténuera fort peu.\par
C’est ainsi que les choses se passent dans le régime capitaliste.\par
Le régime socialiste, au contraire, c’est, répétons-le, le régime de la propriété sociale des moyens de production, c’est le régime qui supprime l’exploitation capitaliste. En d’autres termes, la classe ouvrière, dans une Société Socialiste, n’entre plus en conflit avec des exploiteurs et leur appareil d’État pour exiger une meilleure part dans le produit de son travail : la classe exploiteuse a disparu. La lutte de classes est un souvenir du passé. Ce sont les travailleurs maîtres des moyens de production, maîtres de l’appareil d’État, qui décident combien de temps durera le travail, quelle part du produit du travail sera consommée, quelle part sera consacrée à perfectionner la production socialiste, quelle part sera consacrée à assurer la défense de la patrie socialiste.\par
À quel type d’organisation se rattache l’Allemagne nazie ?\par
Les contorsions de Déat, Deloncle et Doriot n’y changeront rien : l’Allemagne nazie est un pays authentiquement capitaliste, dans lequel un groupe d’hommes, les possédants, tirent leur profit de l’exploitation, du vol de la plus-value.\par
Mais ce capitalisme allemand n’est-il pas d’une nature un peu spéciale ? N’a-t-il pas été « mis au pas », “discipliné” ? N’y a-t-il pas, dans l’organisation allemande, « un peu de socialisme » ?\par
\section[{Dissolution des syndicats. Plus de législation sociale}]{Dissolution des syndicats. Plus de législation sociale}
\noindent Réponse : avant l’avènement du nazisme, les ouvriers allemands possédaient des syndicats, ils disposaient d’une arme redoutable : la grève. Une série de batailles ouvrières avait limité l’absolutisme patronal ; un système de conventions collectives protégeait les salaires. Que sont devenues toutes ces acquisitions sociales ? M. Deloncle s’égosille devant le micro à nous crier : « La Révolution sera sociale ou ne sera pas ! » Soit ! Seulement, le mouvement nazi, dont le Deloncle en question suit les traces, avait trouvé une Allemagne pourvue d’organisations ouvrières et d’une législation sociale qui avaient fourni aux travailleurs des garanties contre l’arbitraire patronal. Qu’a fait le nazisme ? Il a dissous par la violence les organisations ouvrières. Par la force et la ruse, il a brisé la législation sociale.\par
Les procédés qu’employèrent les nazis allemands méritent d’être étudiés de très près. Ils seront peut-être expérimentés par les nazillons français. Il est bon que nous ne soyons pas pris au dépourvu. Avant leur accession au pouvoir, les nazis s’étaient appliqués à constituer des cellules dans les usines. Sans beaucoup de succès d’ailleurs, puisqu’en 1933 encore, les candidats de ces cellules aux Conseils d’Entreprises obtenaient à peine 3 \% des voix. Pour obtenir la première place, les cellules durent attendre… l’incendie du Reichstag, l’occupation des Maisons du Peuple par les Chemises Brunes, la dissolution des Syndicats libres. Les cellules se posèrent alors en héritières des Syndicats, et pour recueillir l’héritage, c’est-à-dire pour gagner la sympathie des syndiqués, elles portèrent la démagogie au diapason le plus élevé.\par
Entendons-nous d’ailleurs ; parmi les membres des cellules nazies qui réclament, à cette époque, le contrôle de l’embauchage, qui interviennent dans la direction des entreprises, qui vont jusqu’à exiger l’arrestation de certains patrons, tous ne sont pas des démagogues forcenés. Nombreux sont les ouvriers trompés par la démagogie des hitlériens et qui croient que la « Révolution Nazie » sera sociale ou ne sera pas. Les magnats capitalistes et les dirigeants du Parti ont vite fait de les mettre à la raison et quelques centaines d’entre eux vont méditer dans des camps de concentration sur les vertus “socialistes” de la doctrine de Hitler.
\section[{Le Front du Travail est créé}]{Le Front du Travail est créé}
\noindent Les “cellules” sont dépossédées sans phrases de l’héritage qu’elles convoitaient. On leur signifie que le patron sera membre de droit de la cellule à qui sera confiée une besogne d’espionnage au sein de l’entreprise. Un nouvel organisme est créé : le Front du Travail qui a pris en charge les biens des organisations dissoutes.\par
Les scribes du RNP ont fait de l’organisation du Front du Travail des descriptions savoureuses. À les en croire, il y avait autrefois en Allemagne des syndicats qui « faisaient de la politique » et qui préconisaient « la lutte de classes ». Le nazisme a, paraît-il, supprimé tout cela et a remplacé ce vieux système par le Front du Travail qui, bien entendu, garantit les droits ouvriers. Que demander de plus ? D’autant que cette organisation sociale repose sur la base solide de la Corporation. Et l’on n’a pas oublié que le programme du RNP prévoit expressément une économie « à base corporative ».\par
Eh bien, essayons de voir quelles sont les attributions du Front du Travail et le rôle de la Corporation dans cet État “communautaire” dont rêvent Deloncle, Doriot et Marcel Déat.
\section[{À genoux devant le patronat}]{À genoux devant le patronat}
\noindent Le Front du Travail ? Ses premiers mois d’existence n’ont pas été de tout repos. Les éléments du parti nazi voyaient s’organiser l’offensive patronale. En fait de Révolution Sociale, ils sentaient se préciser la menace terrible de la régression anti-sociale. Ils s’en prenaient aux dirigeants du Front du Travail, au Docteur Ley en particulier qui, pour ne pas être débordé par ses troupes, lançait quelque anathème contre « l’égoïsme patronal ». Bientôt le conflit s’exaspère. Par qui sera-t-il tranché ? Et dans quel sens ? Par les magnats capitalistes, et dans le sens voulu par le capitalisme. Au nom du patronat allemand, le Dr Schacht, dictateur économique du Reich, déclare qu’il faut en finir avec « les tendances socialisantes du Front du Travail ». Il se rend en personne au Congrès de l’organisation. Il y dicte les ordres de la ploutocratie allemande. Il annonce que, désormais, le Front du Travail n’aura plus le droit d’inspecter les entreprises sans l’assentiment des patrons, que les organes du Front du Travail seront placés sous le contrôle direct des employeurs et que, lorsqu’ils n’auront pas un employeur comme dirigeant, un employeur sera d’office nommé à leur direction. Tel était le diktat du capitalisme allemand. Le Front du Travail s’est soumis.\par
Qu’en pensent les ouvriers de chez nous ? Que pensent-ils de ce genre d’organisation ?\par
Il faut qu’ils le sachent ; c’est ça et pas autre chose que leur réserverait l’avènement du RNP de MM. Froideval, Vigne, Dumoulin et Cie.
\section[{La corporation}]{La corporation}
\noindent Voilà pour le Front du Travail. Et voici pour les Corporations. La Corporation, L’État Corporatif, que de balivernes avons-nous entendues depuis huit mois sur ce sujet. La corporation, voilà la tarte à la crème des “novateurs”. Une organisation englobant les formations patronales et les formations ouvrières, établissant entre elles l’harmonie, la « Communauté d’intérêts », quoi de plus beau ! Quel plus séduisant sujet d’image d’Epinal ! Les corporations n’ont pas vu le jour, la Charte du Travail, objet des cogitations de M. Belin, n’a pas été promulguée\footnote{[note PCF 1942] Une charte du travail qui n’est qu’un carcan de fer pour asservir le prolétariat au capitalisme et qui suscite un mécontentement profond parmi les travailleurs a été promulguée depuis que notre ami a écrit cette brochure} et M. Belin, que les nazillons acclamaient il y a huit mois comme le symbole le plus authentique du nouvel ordre, a failli, en février 1941, être congédié comme un domestique incapable. Il se consolera, le bougre, en évoquant la carrière des ministres du Travail d’Allemagne entre 1933 et 1935. À cette époque, en Allemagne aussi, la discussion sur la Charte Corporative tenait l’avant-scène. Entre les éléments prolétariens du Parti qui voyaient naïvement dans la Corporation le moyen de “discipliner” le patronat et les capitalistes qui, après avoir poussé les nazis au pouvoir, entendait bien se servir de l’État nazi pour imposer leurs volontés, les conflits furent violents parfois. Et comment se terminèrent ces débats ? Comme tous les débats se terminent en régime nazi, par la victoire de la volonté patronale et capitaliste. Pour commencer, Krupp resta président de l’organisation patronale que l’on baptisa « Corporation de l’Industrie allemande ». Pour donner satisfaction à ceux qui continuaient à réclamer l’organisation corporative, le gouvernement prépara la substitution à l’organisation patronale d’un certain nombre de « groupes professionnels ». Mais les patrons renaclent et protestent. Résultats : 1° Le Ministre du Travail qui avait imaginé cette nouvelle organisation pour donner satisfaction à la démagogie nazie est congédié ; 2° Les groupes patronaux professionnels obtiennent le droit de se réunir en un seul groupe, c’est-à-dire, en somme, de reconstituer leur ancienne et puissante organisation ; 3° Dans ces groupes professionnels, les ouvriers ne sont pas représentés ; 4° Des « hommes de confiance » sont désignés dans l’usine pour collaborer avec l’employeur. Mais ils sont désignés par qui ? En fait, par le patron.
\section[{Aucun droit aux ouvriers}]{Aucun droit aux ouvriers}
\noindent Et maintenant, veuillez examiner la structure de l’« État corporatif ». Ses échelons sont : la \emph{Commission de Travail de l’Usine} ; la \emph{Commission économique} et la \emph{Commission de Travail du District} ; le \emph{Conseil économique du Reich} et le \emph{Conseil de Travail du Reich} ; enfin les \emph{Groupes professionnels}. Les représentants patronaux siègent dans les Groupes professionnels ; ils siègent dans les Commissions de Travail de l’Usine, dans les Commissions de Travail et Commissions économiques de District ; ils siègent au Conseil économique du Reich et au Conseil de Travail du Reich. Bref, \emph{ils siègent partout}.\par
Et les représentants ouvriers ? Ils sont exclus des Groupes professionnels ; ils sont exclus des Commissions économiques de District ; ils sont exclus du Conseil économique du Reich. \emph{Le domaine économique est la chasse gardée du patronat}. Défense à la classe ouvrière d’y pénétrer. La classe ouvrière n’est admise par l’escalier de service que dans les Commissions de Travail devant lesquelles sont évoqués les problèmes de salaires (sauf depuis la guerre où cette prérogative elle-même leur a été enlevée).
\section[{Les hommes de confiance}]{Les hommes de confiance}
\noindent Mais afin d’enlever toute efficacité à ces représentations ouvrières dans les Commissions de Travail, il a été entendu que les représentants ouvriers seraient choisis parmi les hommes de confiance. Or, depuis mai 1934, il n’y a pas eu d’élection des hommes de confiance et, pratiquement, les hommes de confiance sont choisis par le patron. Peuvent-ils du moins quand survient un conflit, se faire assister par un délégué du Front du Travail ? Non point. Les fonctionnaires du Front du Travail ne sont pas admis dans les Commissions de travail. Le patron joue sur le velours. Son interlocuteur a été choisi par lui et c’est une créature docile. Mais supposez que cette créature choisie par le patron cesse d’être docile. Qu’adviendrait-il alors ? Alors le patron aurait encore le dernier mot. Le représentant officiel de l’État fasciste, – de cet état dont, nous l’avons vu, toutes les interventions ont servi l’intérêt du patronat contre celui de la classe ouvrière, – se porterait une fois encore au secours du patron et trancherait. Quiconque ne se soumettrait pas à cet arbitrage serait considéré comme un ennemi de l’État.\par
Tel est l’État corporatif. Qu’en pensent les ouvriers français ? Que pensent-ils de cette organisation ? Il faut qu’ils le sachent, dans tous les cas, c’est ça et pas autre chose que leur réserverait l’avènement du RNP de Messieurs Froideval, Vigne, Dumoulin et compagnie.\par
À en croire les mystificateurs, le patronat, dans l’État corporatif fasciste, est “discipliné”, subordonné à la loi, à la morale fasciste ; ce patronat reconnait les organisations ouvrières fascistes et se soumet à la règle de la communauté ! Admirables propos pour les batteurs d’estrades. Mais dans la pratique, comment les choses se traduisent-elles en réalité ? Nous l’avons vu en décrivant la structure de l’État corporatif : dans la pratique, c’est la classe ouvrière et elle seule qui est subordonnée à la volonté et aux diktats du capitalisme patronal par le truchement des syndicats fascistes.\par
Et dans quel sens s’exercent ces diktats ? Nous voulons dire : quel a été le résultat social de la politique de l’État corporatif fasciste qu’on nous donne comme l’exemple à imiter et le modèle à suivre ?
\section[{Suppression du chômage ? Bluff}]{Suppression du chômage ? Bluff}
\noindent Les nazillons répondent : l’État corporatif a donné du travail aux masses ; il a supprimé le chômage.\par
Or ce n’est pas vrai. S’il est exact que le développement de l’industrie de guerre a absorbé dans les années qui précédèrent la guerre une main d’œuvre massive, les statistiques nazies officielles n’en confessaient pas moins en 1938, c’est-à-dire après l’exécution du premier plan quadriennal, l’existence d’une armée de chômeurs de deux millions d’hommes, auxquels il convient d’ajouter les centaines de milliers de sans-travail exclus, pour des raisons politiques, de l’allocation ; le million qui, depuis plusieurs années, sert dans l’armée ; les dizaines de milliers qui peuplent les prisons et les camps de concentration. Le chômage n’a pas été liquidé ; il a été camouflé. Ce camouflage rend un service inapréciable à la dictature fasciste.
\section[{Disqualification du travail ouvrier}]{Disqualification du travail ouvrier}
\noindent Il permet au patronat – qui est censé donner du travail – d’imposer n’importe quelles conditions de travail. Il permet à la Frankfurter Zeitung du 8 février 1941 d’écrire : « \emph{Il est vrai que l’ouvrier n’a plus le droit de faire grève ou d’adhérer à un syndicat de son choix mais, par contre, il n’est plus menacé d’être licencié} ». En réalité, cette menace pèse toujours sur lui, quoiqu’on en dise, mais la théorie énoncée par la Frankfurter Zeitung signifie que l’ouvrier troque contre un droit au travail prétendûment garanti tous ses autres droits, toutes ses autres libertés. Nous écrivons bien « prétendument garanti » car le nazisme n’a pas fourni de travail à tous les Allemands, car le nazisme a abouti à la disqualification du travail ouvrier, car le nazisme a apporté l’appauvrissement des travailleurs. La presse nazie a depuis longtemps avoué cette perte de qualification des ouvriers allemands : « \emph{Une trop grande économie dans les dépenses exclut complètement du processus économique tous ceux qui, si l’occasion leur en avait été donnée, auraient pu, après un court laps de temps, être à la hauteur de la tâche} » (extrait d’un article publié par le propre journal du Docteur Schacht).\par
Qu’est-ce qu’une trop grande économie de dépense ? C’est le nom que les chefs des grands conzerns donnent pudiquement à la « chasse au profit ». La chasse au profit fait que les ouvriers perdent leur qualification, qu’ils se dégradent et que même déchus il leur arrive de ne pas trouver à s’employer. C’est ce régime qu’on appelle en Allemagne “Volksgemeinschaft”, la Communauté du peuple.
\section[{Diminution des salaires}]{Diminution des salaires}
\noindent L’État corporatif ne garantit pas le travail pour tous. Il accélère la disqualification de l’ouvrier. Assure-t-il du moins à celui qui travaille une rémunération plus juste, plus équitable, plus conforme à l’idéal de cette communauté populaire si tapageusement vantée par ses théoriciens et ses imitateurs ?\par
Réponse : depuis 1933, la classe ouvrière allemande n’a cessé de s’apauvrir. Depuis l’avènement du nazisme, la baisse des salaires a atteint 25 à 30 pour 100. La diminution du salaire horaire a été aggravée par la diminution de la rémunération du travail aux pièces. Le salaire réduit a été amputé encore par des prélèvements soi-disant volontaires (20 pour 100 du salaire environ) affectés au bénéfice d’œuvres et organisations nazies. Des salaires spéciaux servis à certaines catégories de travailleurs tendent à déprécier encore le niveau général des ouvriers. Par exemple : deux cent cinquante mille jeunes hommes enrégimentés dans le « Service du travail » reçoivent en tout et pour tout la solde de troupier ; les ouvriers employés dans la construction des autostrades sont moins payés que les terrassiers. Une catégorie d’ouvriers employés à des travaux auxiliaires ne reçoit qu’une allocation de chômage accompagnée de quelques secours en nature.
\section[{Développement de la morbidité}]{Développement de la morbidité}
\noindent La guerre a précipité l’extermination de tous les droits ouvriers. Comme dans les pays “ploutocratiques”, les salaires ont été limités, les majorations pour les heures supplémentaires supprimées. La Commission d’État pour les salaires, pourvue des plus larges prérogatives, a opéré de vastes coupes sombres dans le gain des travailleurs. Dès avant la guerre, le surmenage ouvrier créait en Allemagne un grave et angoissant problème : « \emph{Notre économie} – disait le ministre de l’économie le premier janvier 1939 – \emph{est arrivée aux limites extrêmes de ses possibilités. Le surmenage a abouti à une diminution du rendement.} » « \emph{Il arrive fréquemment} – écrivait de son côté la Frankfurter Zeitung – \emph{que les travailleurs soient sur pied douze, quatorze heures et parfois davantage.} »\par
Voici des chiffres qu’il serait bon de placer sous les yeux des émissaires de Marcel Déat. Ce sont ceux du développement de la morbidité dans la classe ouvrière allemande : le nombre de cas de scarlatine parmi les ouvriers d’usine est passé de soixante-dix-neuf mille en 1933 à cent dix-sept mille cinq cent en 1937 ; la diphtérie suit la même courbe : soixante-dix-sept mille en 1933 ; cent quarante-six mille en 1937 ; on dénombrait un million cinq cent mille tuberculeux en 1937 contre huit cent mille en 1933\footnote{(note de l’auteur) La presse allemande a annoncé au mois de mars 1941 qu’en raison du surmenage auquel les femmes étaient astreintes dans la production de guerre, une journée de repos allait leur être octroyée tous les quinze jours}.
\section[{Réduction des allocations d’assurances et de pensions}]{Réduction des allocations d’assurances et de pensions}
\noindent Toutes les allocations d’assurances et de pensions ont été réduites. Les Caisses ouvrières d’assurances, de Secours mutuel et de prévoyance des Syndicats ont été dissoutes et remplacées par des sociétés d’assurances privées.
\section[{Retraite des vieux ? Une escroquerie}]{Retraite des vieux ? Une escroquerie}
\noindent — Pourtant nous avons lu que la retraite pour les vieux tenait une grande place dans les discours du Chancelier d’Allemagne.\par
— C’est vrai. Depuis que le nazisme est au pouvoir en Allemagne, on promet aux vieux leur retraite, tout comme dans les “ploutodémocraties”. Et comme dans les “ploutodémocraties”, on ne la leur accorde jamais. Ce n’est pas l’une des moindres escroqueries du régime. Dans les premiers mois de la guerre, Hitler avait écrit à Ley pour lui demander de mettre sur pied « un système d’entreprise de prévoyance pour la vieillesse ». La presse salua la grandeur d’âme du Führer. Mais en réalité, de quoi s’agissait-il ? Il s’agissait exclusivement d’un expédient pour financer la guerre. Une institution de prévoyance a été créée sans doute. Un jour viendra où elle distribuera des retraites. Mais en attendant, elle constitue sa caisse en prélevant taxes et contributions diverses sur les travailleurs et consommateurs. L’affaire doit procurer quatre milliards de marks par an. Pour le moment, ces milliards servent à financer la guerre. Plus tard, s’il reste miraculeusement quelques marks au fond des tiroirs, on s’occupera des vieux. Le truc avait été déjà utilisé avant la guerre dans un autre domaine. Les nazis avaient lancé l’idée de la voiture populaire. Chaque ouvrier aurait sa voiture. Un organisme fut créé pour prélever les cotisations. La cotisation devait être payée tout de suite. La voiture populaire ne serait livrée que plus tard. Les cotisations furent donc encaissées pendant trois ans. Elles servirent à accroître le trésor de guerre de l’Allemagne. Après quoi il ne fut plus question de voiture populaire.
\section[{Location aux familles nombreuses}]{Location aux familles nombreuses}
\noindent Autre exemple : il y a quelque temps, lorsque parut à l’Officiel une loi qui faisait obligation aux propriétaires français d’immeubles d’accepter les locations aux familles nombreuses, plusieurs journaux nazifiés de Paris dénoncèrent la carence de l’état qui avait jusqu’ici permis aux propriétaires de jeter l’interdit sur les familles nombreuses. Or l’État nazi a protégé les propriétaires rapaces avec autant de soin et aussi longtemps que l’état “ploutodémocratique” français. C’est en mars 1941 seulement qu’est paru, en effet, dans la presse allemande le décret faisant obligation aux propriétaires allemands de réserver une partie, du reste infime (un dixième) de leurs immeubles pour les familles nombreuses.\par
Et nous ne parlons point, pour ne pas alourdir ces notes, de la hausse des prix, des prix des produits alimentaires notamment, et de leur pénurie. L’affreux régime du rationnement, des ersatz, du marché noir, sous lequel nous vivons depuis huit mois, est celui auquel le nazisme a condamné le peuple allemand depuis cinq ans. Depuis cinq ans on prêche aux prolétaires allemands les douceurs de la « vie spartiate ». C’est cela et pas autre chose que nous préparent les forts en gueule du Rassemblement. L’un d’eux, M. Alphonse de Chateaubriand, n’a-t-il pas reproché déjà aux ouvriers français leur penchant à la paresse, à la vie facile, à la jouissance ?\par
M. de Chateaubriand rêve pour les ouvriers français d’une vie plus difficile, plus dure. Et pour les ploutocrates ? Nous verrons ce qui leur est réservé, à eux, dans la communauté nazie.
\section[{Un bilan anti-social}]{Un bilan anti-social}
\noindent Pour le quart d’heure, dressons le bilan : la classe ouvrière allemande, au prix de luttes sévères, de longs et patients efforts, avait édifié un système de lois sociales. Le nazisme a porté dans cet édifice la hache du démolisseur.\par
La classe ouvrière allemande avait réussi à s’assurer quelques garanties contre l’arbitraire patronal, grâce à ses syndicats et à ses délégués d’usines. Le nazisme a rétabli l’absolutisme patronal. À la place des syndicats dissous, le Front du Travail est placé sous le contrôle de l’employeur ; il n’a pas voix au chapitre dans les conflits qui peuvent naître dans l’entreprise. Aux délégués ouvriers élus, le nazisme a substitué les hommes de confiance désignés par le patron. La classe ouvrière allemande avait réussi à garantir dans une certaine mesure son salaire, par des conventions collectives. Les conventions collectives ont été abolies et les contrats du travail permettent au patron de modifier selon son bon plaisir les taux des salaires ouvriers.\par
Jamais on n’avait assisté à un massacre si radical du progrès social.\par
Mais Marcel Déat, prestidigitateur de cirque pauvre, s’en va répétant : « Les nazis ont aboli la lutte de classes ! Dépéchons-nous de les imiter ! »\par
Ils n’ont pas aboli la lutte de classes, les Chevaliers de la croisade brune. Ils ont arraché à la classe ouvrière ses moyens de défense. Ils l’ont livrée pieds et poings liés au patronat. Ils ont fourni à l’exploiteur les moyens d’exploiter davantage.\par
Mais ça, ce n’est pas la fin de la lutte de classes ; ce n’est point le chant joyeux de l’harmonie sociale. C’est le silence étouffant de l’oppression et de la servitude.
\chapterclose


\chapteropen
\chapter[{7 – L’État nazi, paradis des magnats, des hobereaux, des profiteurs de guerre}]{7 – L’État nazi, paradis des magnats, des hobereaux, des profiteurs de guerre}\renewcommand{\leftmark}{7 – L’État nazi, paradis des magnats, des hobereaux, des profiteurs de guerre}


\chaptercont
\noindent Les bonimenteurs du Rassemblement « national, populaire » expliquent à longueur de journée que la grande affaire est d’en finir avec « l’économie libérale ». Pour découvrir le secret du bonheur ici-bas, il s’agit avant tout de substituer au régime du laissez-faire du capitalisme un système de règlementations strictes et d’orienter, de “diriger” l’économie selon un plan. C’est ce que les petits Cagoulards et les anciens renards de la SFIO ralliés à la bannière de Marcel Déat et Coco Fontenoy expriment lorsqu’ils écrivent dans leur programme d’action : « \emph{Économie dirigée à base corporative, ni étatiste, ni anarchique} ». La formule est équivoque à souhait. On remarquera, dans tous les cas, qu’elle ne prévoit nulle atteinte aux privilèges les plus monstrueux du grand capital. Il y a loin de la coupe aux lèvres et des anathèmes anticapitalistes dont on se gargarise pour essayer d’échauffer les auditoires impavides des meetings à la phrase timide que l’on écrit dans le programme. Faire croire que « l’économie dirigée dans le cadre de la corporation » est une mesure anticapitaliste, une expression de ce socialisme dont Hitler a dit qu’il se sentait “fanatique” est une vulgaire escroquerie. Une escroquerie sans originalité d’ailleurs et assez semblable à celles que commettaient autrefois les amis de Léon Blum lorsqu’ils essayaient de nous représenter les nationalisations comme des mesures spécifiquement socialistes. On connaît la phrase célèbre d’Engels :\par

\begin{quoteblock}
 \noindent « Si l’étatisation du tabac était une mesure socialiste, Napoléon et Metternich compteraient parmi les fondateurs du socialisme ».
 \end{quoteblock}

\section[{Le mot “nationalisation” proscrit au “Rassemblement”}]{Le mot “nationalisation” proscrit au “Rassemblement”}
\noindent Les nationalisations arrachées de haute lutte par un mouvement de masse peuvent – en régime capitaliste – constituer un progrès ; elles peuvent aboutir à « \emph{un certain freinage des représentants les plus cyniques du profit capitaliste} » (Staline). C’est pourquoi, depuis longtemps, les communistes ont tenu ces nationalisations pour infiniment désirables, mais ils ont toujours fait observer qu’il était bouffon et malhonnête de confondre les nationalisations avec l’organisation d’une économie socialiste. Les nationalisations effectuées dans le régime capitaliste ne touchent pas à la propriété privée des moyens de production. Elles ne suppriment ni l’exploitation de l’homme par l’homme ni le profit capitaliste. Elles n’éliminent pas l’anarchie dans la production. En régime capitaliste, la banque nationalisée devient la propriété de l’État capitaliste soumis à la pression des oligarchies ; l’ouvrier de l’entreprise nationalisée en régime capitaliste est exploité selon le mode d’exploitation capitaliste. Or, de ces nationalisations dont nous venons de dire le caractère nécessairement limité et modeste en régime capitaliste, le programme du “Rassemblement” ne souffle mot. Silence bien instructif. Le Rassemblement a un fil à la patte. De sa charte, le vocable “socialisme” était prudemment exclu. Le mot “nationalisation” est lui-même soigneusement proscrit. Les maîtres et patrons du “Rassemblement” n’ont pas voulu qu’on en fît usage. M. Laval a, dans les assurances et autres grands monopoles qui rançonnent le pays, des intérêts trop précis pour qu’il tolère qu’on écrive ce mot malsonnant. Et l’occupant allemand, qui est en train de s’assurer le contrôle de tant d’industries françaises privées et la coparticipation dans les grandes compagnies d’assurances, ne veut pas être gêné en besogne par d’inopportunes nationalisations. Donc, pas de nationalisations pour ne chagriner ni les occupants, ni les conservateurs. Mais « économie dirigée ». Dirigée comment ? Appréciez l’habileté des enfileurs de phrases qui font escorte à Marcel Déat : « Économie dirigée, ni étatiste, ni anarchique ».
\section[{Pur et sot bavardage}]{Pur et sot bavardage}
\noindent Eh bien, tout cela n’est qu’un pur et sot bavardage ! Pourquoi ? Parce que, tant que durera l’exploitation capitaliste – et nous l’avons vu, les gens du “Rassemblement” sont bien résolus à ne pas y porter atteinte ; ils ne veulent même pas en corriger les abus et gardent le silence sur l’urgence des nationalisations – tant que durera l’exploitation capitaliste, aucune économie dirigée, aucun plan si astucieux qu’il paraisse, ne contraindra le capitaliste à continuer la production s’il n’y trouve pas son profit. Dans toute économie capitaliste, la condition de la production, c’est l’appropriation d’un profit. Tant que les moyens de production seront la propriété privée des capitalistes, tant que le profit sera le mobile de la production, la production sera toujours anarchique, le problème des marchés ne sera jamais résolu, et la recette de Marcel Déat sera comme un emplâtre sur une jambe de bois.
\section[{« Économie dirigée » ? Elle existe depuis longtemps}]{« Économie dirigée » ? Elle existe depuis longtemps}
\noindent Mais il y a plus démonstratif encore que les théories et les spéculations ; ce sont les faits. Que nous enseignent les faits ? Ils nous enseignent que c’est une vessie que Marcel Déat veut nous faire prendre pour une lanterne et que ce qu’il tente de faire passer pour une innovation “socialiste” est une pauvre vieillerie capitaliste. Ce que l’on appelle le libéralisme économique est mort bien avant que le directeur de L’Œuvre ne s’avise de prononcer son oraison funèbre. Dès qu’a cessé la période ascendante du capitalisme, c’est-à-dire à partir du moment où les intérêts de classe de la bourgeoisie se sont trouvés en opposition flagrante avec toutes les exigences du mouvement social, l’État est intervenu dans la conduite, dans la direction de l’économie ; il s’est efforcé de diriger l’économie. Presque dans tous les pays, pendant la guerre de 1914 à 1918, l’économie fut dirigée suivant un plan ; depuis plus de dix ans, dans tous les pays capitalistes, l’État est devenu un facteur décisif de l’économie capitaliste.\par
Tantôt il monopolisait le marché intérieur pour les capitalistes nationaux en instaurant des tarifs protecteurs, tantôt il dépréciait les changes pour soulager les capitalistes du poids de leurs dettes, tantôt il renflouait des entreprises capitalistes sur le point de sombrer. La guerre de 1939 a accentué cette tendance. L’État a plus ou moins coordonné les efforts des industries de guerre ; il a présidé à la répartition des matières premières, des marchandises, des moyens de transports, des produits alimentaires. Tous les États capitalistes ont pris des mesures de ce genre. L’Allemagne national-socialiste les a devancés parce que bien avant les autres états capitalistes, elle avait mis sa production industrielle sur un pied de guerre.\par
L’économie dirigée suivant un plan, que les farceurs du “Rassemblement” ont inscrit à l’article 1 de leur programme, c’est la consécration des procédés auxquels le capitalisme de l’époque impérialiste a eu recours – et auxquels il ne pouvait pas ne pas recourir -. Appeler ça la « Révolution Nationale » est une douce espièglerie.\par
Peut-être nous objectera-t-on que l’économie dirigée par l’état “communautaire” s’oriente vers d’autres destins et que cette économie-là, grâce à cette direction, se débarrasse des tares du système impérialiste ?\par
Essayons de répondre. L’état nazi, l’état dont les gens du “Rassemblement” nous recommandent le modèle qu’ils se proposent d’imiter, dont ils prônent les principes, l’État nazi gère-t-il l’économie allemande ? Non point, et ses représentants s’en défendent à chaque occasion. Du moins, l’intervention de l’État nazi dans l’économie, tend-elle à limiter « la domination des trusts », à supprimer « le prolétariat et le profitariat capitaliste », comme disent les auteurs du programme du “Rassemblement” ?\par
Nous allons voir. C’est-à-dire que nous allons énumérer les faits, les faits têtus et obstinés, que n’ébranlent aucune harangue microphonique, aucun prêche du Théâtre des Ambassadeurs.
\section[{Les oligarchies triomphent}]{Les oligarchies triomphent}
\noindent Que s’était-il passé dans les années qui avaient précédé la victoire du nazisme ? Un certain nombre de sociétés capitalistes avaient été renflouées par l’état qui, à la faveur de ce renflouement, était devenu propriétaire d’une partie de leurs actions. Tel était le cas, par exemple, du Trust des Aciéries réunies. Dans le même ordre d’idées, l’état allemand avait acquis le contrôle d’un certain nombre de banques : il possédait 90 \% du capital de la Dresdner Bank et de la Danat Bank (ces deux établissements avaient fusionné) ; 70 \% du capital de la Commerz und Privatbank ; 35 \% du capital de la Deutsche Disconto Bank. Il existait avant 1933 en Allemagne un organisme d’état, le Conseil national de la Potasse. Enfin, un certain noimbre de Régies municipales fonctionnaient à travers l’Allemagne et réalisaient d’appréciables bénéfices.\par
Quelle est l’ambition des oligarchies allemandes ?\par
Elles veulent recouvrer les actions dont l’état est devenu propriétaire ; elles désirent se substituer aux Régies municipales.\par
Et comment va se comporter envers ces ambitions le pouvoir nazi ? Question fort importante et, on en conviendra, tout-à-fait actuelle. Hitler a déclaré, le 24 février 1941, qu’il était un « fanatique du socialisme ». Soit ! Comment s’est exprimé ce fanatisme à l’égard des banques, des oligarchies industrielles qui désiraient se débarrasser du contrôle de l’état ? Répondez, Deloncle, répondez, Marcel Déat !\par
Si vous vous taisez, nous répondrons à votre place.\par
À peine arrivé au pouvoir, Hitler a restitué au Trust des Aciéries réunies les actions dont l’état était devenu détenteur. L’oligarchie de l’industrie lourde pouvait chanter victoire. À peine arrivé au pouvoir, Hitler, « socialiste fanatique », fait adopter des décrets de “reprivatisation” qui restituent aux banques les actions acquises par l’état. L’oligarchie bancaire chante victoire. À peine arrivé au pouvoir, Hitler dissout le Conseil national de la Potasse ; il écrase sous les charges fiscales les Régies municipales, qu’il dissout ensuite au profit des établissements privés. Les magnats de la potasse, les trusts de production et de distribution de l’énergie électrique chantent victoire.
\section[{Exonération de l’impôt aux concessionnaires de l’électricité}]{Exonération de l’impôt aux concessionnaires de l’électricité}
\noindent Du moins, les compagnies concessionnaires d’électricité payaient-elles un impôt aux municipalités. Elles l’ont payé jusqu’au 4 mars 1941. Ce jour-là, un décret les a exonérés de cet impôt dans les villes de moins de trois mille habitants, a réduit cet impôt à 10 \% dans les grandes villes et l’a pratiquement annulé pour les grandes industries. Pour se procurer les ressources qui leur échappent aussi brusquement, les municipalités sont conviées à frapper le consommateur. Belle perspective pour les usagers de Paris et des villes de France à qui Marcel Déat chante ses airs de flûte.
\section[{Dégrèvements fiscaux pour les industriels}]{Dégrèvements fiscaux pour les industriels}
\noindent Poursuivons : avant l’arrivée de Hitler au pouvoir, il existait en Allemagne un système d’impôts sur le revenu. D’autre part, les magnats capitalistes étaient parfois gênés, pour relever artificiellement leur prix de vente, par l’ouverture de nouvelles industries. Qu’exigeaient les oligarchies capitalistes ? Elles exigeaient la fin du système fiscal qui les obligeait à prélever une petite part de leurs bénéfices. Elles exigeaient la liberté de pressurer le consommateur, en relevant leurs prix de vente.\par
Exigences égoïstes d’une ploutocratie avide de bénéfices.\par
En face de ces exigences, comment va se comporter le pouvoir nazi et son chef Adolf Hitler, « fanatique du socialisme" ? Répondez, Deloncle, répondez, Coco Fontenoy, Marcel Déat, répondez !\par
Si vous vous taisez, nous répondrons à votre place : dès que le nazisme fut au pouvoir, les industriels bénéficièrent de dégrèvements fiscaux de toutes sortes ; sous les prétextes les plus divers et les plus inattendus, des exemptions fiscales leur sont octroyées, l’impôt sur le revenu est réduit de moitié, les taxes de succession sont diminuées. L’oligarchie industrielle chante victoire.
\section[{Évincement des petits producteurs}]{Évincement des petits producteurs}
\noindent Elle a d’autres raisons encore de se réjouir ; l’état décide de réunir des entreprises en cartels, conventions ou ententes industrielles. De quoi s’agit-il ? Est-il question d’organiser la production, de diriger l’économie dans l’intérêt de la collectivité ? Non pas : il s’agit beaucoup plus simplement de permettre aux magnats d’évincer les petits producteurs et de relever leurs prix. Pour créer des prix de monopoles, les magnats se constituent en Ententes industrielles. Mais qu’arrivera-t-il si quelques industriels ne veulent pas s’agglutiner au Cartel ? Cette concurrence ne risque-t-elle pas de freiner le relèvement des prix ? Sans aucun doute. Comment venir à bout des dissidents ? En exerçant sur eux une contrainte appropriée. Qui peut exercer cette contrainte ? Qui peut proclamer les Ententes obligatoires ? L’état.\par
Le consommateur paiera, sans doute ; mais la ploutocratie industrielle peut sabler le champagne en l’honneur du « national-socialisme ».\par
Ce sont là de bons et loyaux services, de nature, n’est-il pas vrai, à apaiser la fringale des ploutocrates. Décidément, les subventions de Thyssen, de Krupp et des autres n’ont pas été octroyées en vain ! Décidément, les discours anticapitalistes étaient de fort habiles attrape-nigauds ! Les oligarchies exultent. Le pouvoir nazi est leur pouvoir. Il les a débarrassées des entraves de l’état ; il a restitué à l’industrie privée ce que contrôlait l’état ; grâce à lui, les oligarchies constituées en monopoles, organisées en Conzerns peuvent rançonner le consommateur allemand en relevant les prix de vente. Seulement les potentats de l’industrie lourde allemande exigent davantage encore. On a rétabli l’absolutisme patronal, on a délivré les capitalistes des contrôles qu’ils trouvaient insupportables. Ils veulent maintenant qu’on remplisse leurs caisses. Ils sont à la recherche d’une clientèle, que la crise et la concurrence internationale ont rendue plus rare et dont la capacité d’achat a diminué. Pour accumuler les bénéfices, les rois de l’acier, du ciment, du béton, veulent des clients rentables. Que se passe-t-il alors ? L’État nazi se fait le client des magnats, leur pourvoyeur de bénéfices. Il entreprend la construction de grands travaux, autostrades, voies ferrées, écluses, travaux d’urbanisme. Mais surtout, il passe ses gigantesques commandes à l’industrie de guerre. En 1935, Krupp, Siemens, Klockner accusent des augmentations de bénéfices qui atteignent respectivement 90 \%, 95 \%, 53 \%. L’Allemagne de la “Révolution” nazie devient le paradis rêvé des marchands de canons.
\section[{Et les banques ?}]{Et les banques ?}
\noindent Là-dessus, le Français-qui-lit-Marcel-Déat nous interrompt : « Tout cela est possible, nous dit-il, mais vous admettrez tout de même que la “Révolution” nazie a établi un certain contrôle des banques, qu’elle a empêché les capitalistes d’exporter leurs capitaux à l’étranger. C’est quelque chose, ça ! »\par
Nous allions en parler, précisément. L’État nazi, nous l’avons dit, se fait le client de l’industrie. L’industriel ne travaille pas pour rien. Il entend accumuler des bénéfices. Sans doute l’évangile du National-socialisme affirme-t-il en ses versets que patrons et ouvriers doivent collaborer à la prospérité de la communauté nationale populaire. Mais à cette collaboration l’ouvrier apporte ses sacrifices et de cette collaboration, le magnat de l’industrie retire son profit.\par
Pour que les commandes de l’état procurent aux industriels de substantiels bénéfices, l’état doit payer ses fournisseurs. Seulement, comme ses ressources sont limitées il les paye avec un papier d’État, les bons de création de travail, qui sont réescomptés par les banques. Mais pour honorer sa signature, l’état doit faire en sorte que les épargnants échangent leurs économies contre du papier d’état ; il doit leur interdire d’effectuer aux guichets des banques des retraits massifs ; il doit les empêcher de mettre leur avoir à l’abri à l’étranger. Il ne peut y parvenir qu’en instaurant le contrôle des établissements de crédit et le contrôle des changes. Et c’est ainsi que des mesures préconisées dans tous les pays par les mouvements populaires comme des mesures anti-capitalistes sont utilisées par le pouvoir nazi au profit exclusif du capitalisme, au profit exclusif des magnats qu’elles n’empêchent pas, d’ailleurs, de posséder des avoir à l’étranger, des participations dans l’industrie étrangère. Leur objet est de permettre à l’état d’être pour les potentats industriels un client avantageux.
\section[{Un fonds de dumping}]{Un fonds de dumping}
\noindent C’est dans le même but que l’État nazi est conduit à contrôler tout le commerce extérieur ; que pour forcer les débouchés extérieurs, il abaisse artificiellement les prix de vente des marchandises allemandes à l’étranger ; que pour réaliser ce tour de force, il constitue un fonds de dumping. Sans doute ce fonds est-il alimenté par un prélèvement effectué sur l’ensemble de l’industrie et est-il utilisé pour compenser les pertes subies par les exportateurs qui vendent à bas prix. Mais dans la réalité, ce n’est pas l’ensemble de l’industrie qui subit ce sacrifice, c’est la masse des consommateurs, condamnée à supporter les conséquences du renchérissement de tous les produits destinés à la consommation intérieure.\par
Mais puisque l’état est amené à jouer un rôle si essentiel dans l’économie, – cet État national-socialiste qui s’était flatté de museler la ploutocratie, de faire une révolution sociale – cet état ne va-t-il pas profiter des circonstances pour freiner la course au profit des oligarchies ?
\section[{Un état-major de “compétences” capitalistes}]{Un état-major de “compétences” capitalistes}
\noindent Marcel Déat, dans ses allocutions radiophoniques, vitupère les Comités professionnels qui se sont constitués à Vichy et qui sont dirigés par ceux-là même qui naguère dirigeaient les trusts. M. Déat s’indigne. À l’entendre, il semble bien que les choses se passeraient autrement si les nazillons avaient le pouvoir.\par
Eh bien, voyons comment se passent les choses en Allemagne où les nazis exercent le pouvoir : l’état devenant client de l’industrie, et contraint, pour rétribuer ses fournisseurs, à recourir à un certain nombre de mesures d’administration et de contrôle, doit se créer un état-major bureaucratique très imposant. Où va-t-il recruter cet état-major ? Tout bonnement parmi les “compétences” du monde capitaliste, comme font Belin et Bouthillier. Les magnats cessent d’être les maîtres occultes de la politique économique du pays. Ils en deviennent les dirigeants ouverts et officiels. Le stimulant de l’intérêt individuel (entendez : du profit capitaliste) est et demeure le fondement de toute activité économique ! Qui parle ainsi ? Quel milliardaire ? Quel ploutocrate ? Le Docteur Schacht (30-11-1935), führer de l’économie « national-socialiste ».
\section[{Le rôle du capital-action}]{Le rôle du capital-action}
\noindent « \emph{La base de l’idée communautaire}, précise le Deutscher Volkswirt (19 février 1941), \emph{c’est l’initiative privée, la volonté du risque privée, la propriété privée} ». Mais ici surgit un problème : comment concilier les grandes proclamations anti-capitalistes dont on assortit la propagande, les anathèmes proférés contre les possesseurs de dividendes, les excommunications lancées contre le capital-action, avec cette organisation économique dans laquelle le capital-action est le principal instrument de financement de l’industrie ? Le rôle du capital-action est-il provisoire ? Disparaîtra-t-il après cette victoire allemande dont le Führer affirme qu’elle marquera le triomphe de la « Révolution sociale » ? Point du tout, au contraire. Tous les représentants de la finance, de l’industrie allemands attribuent au capital-action une mission de plus en plus importante dans l’économie. Or, le capital-action est et demeure parasitaire. Il a été condamné comme tel par tous ceux qui se réclament du socialisme. Il est impopulaire et le Deutscher Volkswirt convient de cette impopularité dans l’article que nous venons de citer. Il reconnaît que « \emph{les capitaux sous forme d’actions s’accumulent dans les réserves et deviennent la propriété anonyme d’un cercle de plus en plus étroit} ». Cercle de plus en plus étroit, que l’on appelle ploutocratie et que l’on a dénoncé avec beaucoup de virulence sur les estrades des réunions nazies. En veine de confidences, le journal allemand nous avoue que « \emph{un précipice risque de se creuser entre les moyens d’organisation financière et la compréhension de ces moyens par le public} ». Comment éviter ce divorce entre « la pratique du capital-actions et la sensibilité populaire" ? Il y aurait un moyen très salutaire et très recommandable, un moyen anticapitaliste, révolutionnaire, bref, le moyen qu’on avait promis aux masses d’employer et qui consistait à s’attaquer au parasitisme capitaliste. Ce moyen a été délibérément repoussé par le nazisme.
\section[{« Encourager les industriels à devenir actionnaires »}]{« Encourager les industriels à devenir actionnaires »}
\noindent Pour concilier les principes dont on s’est couvert comme d’une défroque et la soif de gains des potentats, l’État nazi recommande une autre méthode : « \emph{Il convient}, écrit le Deutscher Volkswirt, \emph{d’encourager les industriels qui peuvent acheter des actions à devenir actionnaires.} » C’est tout simple, comme on le voit. Mais qui sont ces industriels qui peuvent se procurer des actions ? Ouvrons la Frankfurter Zeitung : un imposant placard publicitaire recommande les actions de la société Accumulatoren Fabrik. Cette société a émis, en 1933, deux mille cinq cent actions à 500 marks l’action, côtées aux bourses de Frankfort et de Berlin. Huit ans plus tard, la même société émet vingt mille actions nouvelles que lancent la Deutsche Bank, la Berliner Handelsgesellschaft, la Dresdner Bank, etc. Quel est le taux de ces actions ? Mille marks l’action, soit vingt mille francs l’action. Après huit années de régime nazi, le capital-action, symbole du capitalisme parasitaire, est la chose de l’oligarchie des possesseurs de richesses et les théoriciens nazis, s’ils en avaient la franchise, devraient réciter le Credo du capitalisme, proposé par le vieux Lafargue :\par

\begin{quoteblock}
 \noindent « Je crois au Capital qui gouverne la matière et l’esprit ; je crois au Profit, son fils très légitime, et au Crédit, le Saint-Esprit qui procède de lui et est adoré conjointement. »
 \end{quoteblock}

\noindent Un personnage considérable, un « socialiste fanatique » sans doute, le colonel Thomas, chef de la section économique du Ministère allemand de la guerre, a du reste exprimé ainsi les aspirations “révolutionnaires” de l’idéalisme nazi : « \emph{L’économie de guerre allemande ne socialisera pas l’industrie de guerre… L’entrepreneur et le marchand doivent gagner de l’argent ; ils sont là pour ça.} » (sic)\par
Il n’a pas été nécessaire de le leur répéter plusieurs fois, pour que les marchands et les entrepreneurs comprissent.
\section[{Le bénéfice de guerre est tabou}]{Le bénéfice de guerre est tabou}
\noindent Sans doute le Monsieur-qui-lit-Marcel-Déat s’imagine-t-il de fort bonne foi que, dans un pays qui a fait sa « Révolution nationale socialiste », le profit de guerre, cette tare affreuse du régime capitaliste, est rigoureusement interdit et strictement prohibé. Ce Monsieur serait fort étonné si on lui révélait le bréviaire des droits du capitalisme allemand. En Allemagne, comme dans tous les pays capitalistes, il existe des lois qui “règlementent” un profit. Mais en Allemagne comme dans tous les pays capitalistes, des procédés variés sont mis à la disposition des intéressés pour tourner ces règlements. La Frankfurter Zeitung a publié au début du mois de mars 1941, à ce sujet, une explication très instructive des règlements en vigueur au sujet des prix et des bénéfices. Un industriel fabrique et vend un article quelconque. Le prix de vente qu’il pratique lui permet de réaliser un bénéfice de cent mille marks. Survient la guerre. L’industriel reçoit quatre fois plus de commandes ; il vend quatre fois plus d’articles. C’est ici que l’État nazi intervient. Il intervient pour demander à l’industriel de diminuer le prix de vente de chaque article. Cette diminution du prix de vente, l’industriel l’obtiendra en réduisant ses frais de revient. Cette diminution sera pour lui d’autant plus aisée que la concentration et la fabrication d’un plus grand nombre d’articles réduisent les frais de fabrication de chaque article. Résultat ; le bénéfice de cet industriel, s’il n’est pas quatre fois plus élevé qu’avant guerre, pourra l’être trois fois ou trois fois et demi. D’autant que s’il démontre que sur un autre article fabriqué dans son usine, il enregistre une diminution de ses bénéfices, il lui est loisible, avec l’autorisation de l’état, de relever les prix de vente des autres articles. La guerre, en somme, est une bonne affaire pour lui. Aussi bonne sans doute que pour ses honorables collègues et confrères des pays plouto-démocratiques.\par
Le Gauleiter Wagner a exposé, le 5 août 1940, les principes qui inspiraient l’économie national-socialiste en matière de bénéfices de guerre : « \emph{Dans d’autres pays, dit-il, on a instauré un impôt sur les bénéfices de guerre. Chez nous, rien de pareil. Notre ligne politique est que l’économie allemande doit pouvoir faire face à tous ses besoins et qu’après la victoire elle soit en mesure d’affronter de nouvelles tâches. Tout service rendu à l’état doit être récompensé. Tout ce qui pourrait entraver l’esprit d’entreprise est étranger à notre politique qui reconnait expressément qu’un service extraordinaire rendu à l’état peut être récompensé par un bénéfice extraordinaire. Nul ne doit s’inquiéter en supposant que certaines règlementations de l’état de guerre pourraient être la préface de mesures de nivellement.} »\par
Allons, il y a encore de beaux jours sous le régime de la croix gammée pour le profitariat capitaliste !
\section[{Interrogez vos consciences}]{Interrogez vos consciences}
\noindent La Frankfurter Zeitung examine à son tour la question du bénéfice licite, « der erlaubte Gewinn ». L’auteur rassure entièrement les profiteurs de guerre. Le contrôle de l’état, leur dit-il, ne sera pas “mesquin”. L’état se gardera de fixer un taux qu’il serait interdit de dépasser ; tout cela dépendra de la situation générale de l’entreprise. De larges indulgences seront consenties aux industriels « ayant le goût du risque ». Leurs bénéfices pourront être utilisés pour renouveler leurs stocks, développer leurs usines selon le principe de l’autofinancement “autofinanzierung”. L’auteur, remarquez-le, s’exprime au futur. En attendant, les industriels empochent leurs bénéfices et la Frankfurter Zeitung fait appel pour tout ce qui concerne cette délicate question à la “conscience” des profiteurs de guerre !\par
Voilà comment on “discipline” les ploutocrates au pays de la Révolution nationale socialiste choisi comme modèle par Marcel Déat. On leur demande d’interroger leur conscience ! Les réponses des consciences sont très édifiantes. Elles s’étalent tous les jours dans les journaux allemands. Ouvrons au hasard ceux d’une semaine du mois de février 1941. Ils nous révèlent que l’index des sociétés par actions allemandes est passé de 1346 en 1939 à 1788 en 1940. Ils débitent par tranches quotidiennes le palmarès du profit capitaliste allemand. Puisons dans ce précieux mémento autrement éloquent que les prêches de Göbbels, de Rosenberg et les coups de gueule de Marcel Déat : la Deutsche Textil (Berlin) distribue 8 à 10 \% de dividendes ; la Bremer Silberwaren 8 \% ; l’Akumulator Fabrik émet de nouvelles actions et de nouvelles obligations ; l’Aktien Maschinen Fabrik enregistre une progression de ses bénéfices ; la Nuremberg Bund augmente ses dividendes ; la Westland Gummiwerke (caoutchouc), la Burbach Kallich augmentent leur capital. Les grandes brasseries, elles aussi, accumulent les profits : l’Export Bier Bracaurer, la Dortmunder Ritter accusent l’augmentation de leur capital. On affirme, il est vrai, que les sociétés allemandes ne peuvent distribuer plus de 10 \% de dividendes. Cette règlementation n’est-elle pas une mesure anticapitaliste ? N’exprime-t-elle pas la subordination des intérêts privés à la Communauté populaire national socialiste ? À aucun degré. En effet, ce n’est pas l’état, ce n’est pas la collectivité qui profite de cette restriction : c’est encore le capitaliste privé. Le capital qu’il accumule reste sa propriété et une tâche très précise lui est assignée : le développement de l’usine, son approvisionnement en matières premières et, plus encore, la participation de la société à des entreprises étrangères dont elle s’assurera ainsi le contrôle. La capitalisation forcée est un moyen de renforcement de la puissance des grandes entreprises en Allemagne et dans les autres pays d’Europe.
\section[{Le “Conzern” Hermann Göring}]{Le “Conzern” Hermann Göring}
\noindent À la faveur de la guerre et de sa préparation, les grandes entreprises éliminent les autres ou les mettent au pas. L’exemple le plus remarquable de cette concentration capitaliste est fourni par le “Conzern” Hermann Göring, qui englobe les mines, l’électricité, la métallurgie, les produits chimiques, les moyens de transport.\par
Dans cette bénédiction générale de la ploutocratie, les banques ne sont pas oubliées. Moins sollicitées par les grandes firmes privées qui regorgent de profits, elles participent davantage aux emprunts d’état et aggravent ainsi leur pression et leur contrôle sur l’État « national-socialiste ». Elles obtiennent plus aisément l’autorisation de tenir des capitaux en réserve, c’est-à-dire placés sous le contrôle unique de la Direction, à l’exclusion de tout droit de regard des actionnaires. Enfin, elles se ménagent de larges possibilités pour des investissements à l’étranger ou pour des participations aux entreprises étrangères.
\section[{Sauvegarde du profit capitaliste}]{Sauvegarde du profit capitaliste}
\noindent Voilà pour le passé et pour le présent. Et voici maintenant les perspectives d’avenir. Hitler a proclamé dans son plus récent discours, et Déat, Luchaire et quelques autres saltimbanques ont répété après lui comme des phonographes, que la victoire de l’Allemagne nazie serait celle du socialisme. C’est ainsi que s’expriment ceux qui pérorent et ceux qui écrivent, ceux à qui l’on prête le micro ou le stylographe. Mais voyons ce que disent les vrais maîtres, ceux qui ont financé l’aventure et qui en sont les bénéficiaires exclusifs. Dans un discours à Düsseldorf, le vice-président de la Reichsbank, M. Lange, précisant les rapports qui, après la guerre, devront s’établir entre l’état et l’économie allemande, déclare :\par

\begin{quoteblock}
 \noindent « Après la guerre, l’économie allemande se trouvera, comme depuis 1933, devant cette double tâche : agir selon les principes purement commerciaux, en développant l’esprit d’initiative, et d’autre part, avoir conscience de ses devoirs envers l’état. Seulement l’accent sera mis sur la première de ces tâches. L’état indiquera seulement à l’économie d’une façon générale, la direction à suivre. Le marché des capitaux sera plus facilement accessible aux émissions de l’économie privée. Le développement de cette activité dépendra notamment de la capacité des sociétés de financer leurs dépenses sur leurs propres bénéfices. Dans les cas où les entreprises, après la guerre, feront appel plus fréquemment au marché des capitaux pour leur financement, l’action jouera un rôle plus important que maintenant comme moyen de financement. »
 \end{quoteblock}

\noindent Le passage est un peu long, mais il méritait d’être cité.\par
« \emph{Lutte contre la domination des trusts, abolition du Profitariat capitaliste} », salive l’orateur du Théâtre des Ambassadeurs. Et les farouches “révolutionnaires” qui l’écoutent, des dames avec de ridicules petits chapeaux et des voilettes 1900, applaudissent du bout de leurs doigts gantés !\par
Mais derrière le tintamarre des mots, une réalité bien précise celle-là : la défense et la sauvegarde du profit capitaliste.\par
« Le ciel pour les capitalistes, l’enfer pour les prolétaires !", comme disait Lénine, pilote génial de la Révolution socialiste, de la vraie révolution socialiste.
\chapterclose


\chapteropen
\chapter[{8 – Les victimes : ouvriers, paysans, classes moyennes…}]{8 – Les victimes : ouvriers, paysans, classes moyennes…}\renewcommand{\leftmark}{8 – Les victimes : ouvriers, paysans, classes moyennes…}


\chaptercont
\noindent L’enfer pour les prolétaires !\par
Et pas pour eux seulement. Or, la mystification de la propagande nazie, la trahison des chefs social-démocrates, la division ouvrière enfin, avaient permis aux agitateurs nationaux-socialistes d’entraîner bon nombre d’ouvriers dans leur aventure. Ces ouvriers étaient avides d’action, et d’action contre les oligarchies. Que leur a réservé le nazisme ? Que sont-ils devenus dans le parti capitaliste d’Adolf Hitler ? Leur histoire vaut d’être racontée. Si quelque sergent recruteur de Marcel Déat se hasarde auprès des ouvriers de nos usines, il faudra narrer à ceux qui seraient tentés d’écouter les boniments du mauvais berger l’affreuse tragédie de ces prolétaires qui, il y a dix ou quinze ans, se laissaient prendre aux phrases creuses des Déat d’Allemagne, des Déat en chemises brunes et à croix gammée.\par
Ah certes, quand le Reichstag fut incendié, quand les Palais ministériels s’ouvrirent aux dirigeants nationaux-socialistes, ils s’empressèrent de recruter dans le parti, dans les Sections d’Assaut, les détenteurs des fonctions essentielles de l’appareil d’état. Parmi ces nouveaux promus, certains, des ouvriers notamment, qui avaient pris au sérieux les professions de foi anticapitalistes produites si fréquemment dans les réunions monstres, affirmèrent que la révolution ne faisait que commencer et qu’il convenait de la poursuivre.\par
\section[{Les potentats capitalistes ordonnent…}]{Les potentats capitalistes ordonnent…}
\noindent Les potentats capitalistes n’aiment pas beaucoup ce langage. Sur leur désir, le Parti est “épuré”, c’est-à-dire purgé de ses éléments les plus combatifs. Ceux qui insistent un peu trop sur l’aspect “socialiste” du programme que l’on fit miroiter à leurs yeux, les jeunes qui, dans les cellules du Parti, dans les cantonnements des Sections d’Assaut, appellent à la lutte contre la ploutocratie, sont décrétés indésirables. On les chasse du Parti. Les plus turbulents sont dirigés vers les camps de concentration. Cinq mois après la prise du pouvoir, Hitler va prononcer les paroles que les magnats attendent de lui. Il dit :"Je m’opposerai résolument à une seconde vague révolutionnaire. » Krupp et Thyssen applaudissent. L’agitateur frénétique est un commis docile. Il n’oublie pas les services que lui rendirent ses bailleurs de fonds.
\section[{Colère parmi les SA}]{Colère parmi les SA}
\noindent Mais les hommes du rang et de la file, comment vont-ils accueillir ces propos qui ruinent leurs espérances ? Chez eux, la déception est grande, l’amertume est au cœur de la plupart. Bientôt la colère gronde parmi les Sections d’Assaut. Seulement, les potentats veillent. Ils tiennent l’état-major nazi par des fils aussi solides que ceux qu’ils ont attachés, dans la France de 1941, à la patte de Marcel Déat et de Doriot. La direction du Parti envoie en congé les Sections d’Assaut. Cette mesure, loin de calmer l’effervescence des SA, l’exaspère plutôt.
\section[{Krupp et Thyssen l’emportent}]{Krupp et Thyssen l’emportent}
\noindent Les SA continuent à exiger la “Révolution”, la lutte contre les oligarchies, bref ce qu’on leur a promis pendant les années de propagande, ce pourquoi ils sont venus au nazisme. Mais Krupp et Thyssen ont d’autres exigences et c’est eux qui l’emportent. À la faveur d’un répugnant scandale de mœurs, les chefs SA sont abattus comme des chiens. SA et SS sont refoulés de la scène politique.
\section[{Les camps de concentration pour les prolétaires}]{Les camps de concentration pour les prolétaires}
\noindent À la Reichswehr, pilier solide de la réaction capitaliste, est confié le soin de maintenir l’ordre. La Jeunesse hitlérienne perd son autonomie et devient institution d’état. L’homme de confiance du grand capital, Schacht, demeuré quelques semaines dans l’ombre, entre avec armes et bagages dans les conseils du gouvernement. En longs et lamentables cortèges, les prolétaires qui s’étaient fourvoyés dans les rangs du Parti nazi sont conduits vers les camps de concentration. Voilà comment s’achèvent les « Révolutions nationales » du type de celle pour laquelle Marcel Déat et sa troupe battent la grosse caisse devant leur cirque.\par
Mais, à côté des ouvriers, d’autres catégories sociales s’étaient laissées prendre à la démagogie forcenée des agitateurs nazis, celles que l’on désigne sous l’appellation générique de classes moyennes ; les paysans, les petits bourgeois des villes, petits commerçants et artisans. Que sont-elles devenues ? Quel sort le nazisme leur a-t-il réservé ? Ce qui revient à dire : quel sort le “Rassemblement” de M. Déat réserverait-il aux classes moyennes françaises ? Les songe-creux du “Rassemblement” prétendent, on le sait, au rôle de protecteurs de la paysannerie française et des petits-bourgeois des villes auxquels ils prodiguent sourires et amabilités, dont ils veulent faire le noyau de leur mouvement anémique ; ils vantent le « rude bon sens » du paysan français, la pondération, la sagesse, le goût de la mesure, l’esprit d’épargne des petits artisans et commerçants. De ces précieuses qualités, les ouvriers de la grande industrie sont – à les entendre – totalement dépourvus.
\section[{Les paysans sous le nazisme}]{Les paysans sous le nazisme}
\noindent Seulement le rude bon sens de notre paysannerie et de nos classes moyennes risque fort de suggérer aux paysans et aux petits-bourgeois de chez nous une démarche terriblement dangereuse pour M. Déat et ses acolytes, celle par exemple de se renseigner exactement sur le bilan de la politique nazie à l’égard de la paysannerie et des classes moyennes d’Allemagne. Si des paysans de chez nous à qui il ne faut point conter d’histoires à dormir debout s’informent des conditions d’existence de leurs frères d’Allemagne, au lieu de se fier à ce que raconte l’émissaire nazi Cross dans ses conférences à l’Institut de Paris, voici ce qu’ils répondront aux envoyés de Marcel Déat. Ils leur diront : « Quand Hitler recrutait ses adhérents à la campagne, il promettait aux paysans le partage des terres des hobereaux ; il leur affirmait qu’après sa victoire, les domaines des propriétaires fonciers seraient consacrés à la colonisation. Or, qu’est devenue cette promesse ? Le premier soin de Hitler quand il fut maître du pouvoir, fut de désigner comme ministre de l’agriculture l’un des représentants des hobereaux : Hugemberg, lequel, au lieu de partager les terres, renfloua aux frais de l’état des domaines qui n’étaient plus rentables. À Hugemberg succéda un nazi authentique, Walter Darré, qui proclama : « D’accord avec mon Führer, je ne toucherai pas à une propriété, quelle que soit son étendue. » Une loi proclama inaliénables les fermes héréditaires et autorisa l’admission des grands domaines au titre de fermes héréditaires. Avant la victoire du nazisme, il y avait eu dans l’est de l’Allemagne quelques expériences de colonisation intérieure. Neuf mille exploitations avaient été créées ainsi en 1932. Après l’avènement du nazisme, la colonisation accusa une régression : quatre mille exploitations à peine furent créées en 1935. D’où proviennent les terres de ces domaines ? Elles n’ont pas été enlevées aux hobereaux. Dans leur presque totalité, elles proviennent des biens domaniaux ou d’étendues marécageuses ou désertiques.
\section[{La petite propriété sacrifiée}]{La petite propriété sacrifiée}
\noindent Vous nous roucoulez des chansons sur les beautés de la petite propriété et Philippe Pétain radote à merveille sur ce sujet. Mais la politique de vos amis nazis n’a nullement encouragé la petite propriété, tant s’en faut. Les exploitations paysannes qui ont été déclarées inaliénables ont une superficie de 10 hectares au moins. Elles reviennent à un seul héritier, les autres enfants étant réduits à la condition de prolétaires. Bien plus, dans certaines régions, c’est en confisquant une partie des biens des petits propriétaires que l’on a constitué ces exploitations de 10 hectares. Il y a loin de cette réalité aux boniments que racontent vos sergents recruteurs. Mais ce n’est pas tout.
\section[{Conditions de travail féodales pour l’ouvrier agricole}]{Conditions de travail féodales pour l’ouvrier agricole}
\noindent Les ouvriers agricoles, grâce à leur syndicat, avaient obtenu quelques droits. En particulier, ils avaient imposé des conventions collectives de travail. Là-dessus vos amis nazis se sont installés au pouvoir. Qu’ont-ils fait ? Ils ont supprimé les syndicats d’ouvriers agricoles, brisé la plupart des conventions collectives, remis en honneur à la campagne les conditions de travail féodales et barbares. L’ouvrier agricole allemand touche un salaire inférieur de 40 à 50 \% à l’allocation du chômeur industriel. L’état met en outre des chômeurs à la disposition des hobereaux. Ces auxiliaires agricoles, des jeunes gens pour la plupart, sont rétribués selon la fantaisie du maître de l’exploitation. Enfin, pour enchaîner à son village et à sa misère l’ouvrier agricole, le nazisme n’a rien trouvé de mieux que de substituer au salaire en espèces un salaire en nature.
\section[{Les petits propriétaires ploient sous les charges}]{Les petits propriétaires ploient sous les charges}
\noindent Les petits propriétaires ne sont pas logés à meilleure enseigne. Le « national-socialisme » a bien exempté les fermes héréditaires de l’impôt foncier, de l’impôt sur les successions, mais les paysans travailleurs suffoquent sous le poids des impôts, taxes et cotisations de tous ordres. Le national-socialisme a bien institué le moratoire des dettes agricoles ; mais seuls, les grands et les moyens propriétaires ont bénéficié de cette mesure. Les autres sont vendus et ces ventes forcées sont devenues chaque année plus nombreuses depuis 1933. Ce sont les hobereaux qui reçoivent la manne des subventions d’état, des secours de l’est (Osthilfe) ; ce sont eux qui dictent à l’état sa politique douanière de protection des céréales produites surtout par les grands domaines.
\section[{La corporation du ravitaillement}]{La corporation du ravitaillement}
\noindent Afin d’assurer aux hobereaux un prix rémunérateur pour leurs céréales, l’état crée la Corporation du Ravitaillement, qui fixe les prix des céréales. Le seigle, le froment accusent une hausse verticale. Ce système des prix fixes, qui est appliqué aux céréales pour assurer aux hobereaux des cours aussi élevés que possible, est appliqué au lait, aux beurres et aux œufs (produits par les petits paysans) pour enrayer la hausse de ces denrées. Le nazisme a d’ailleurs presque complètement évincé le petit producteur du marché. Le petit paysan ne vend plus son beurre, son lait, son fromage. Il livre aux organisations monopolistes, au trust du lait, les quantités de matière première qu’on lui impose, et cela au prix fixé par ce trust. Voilà la triste réalité, monsieur l’émissaire du “Rassemblement”. Ces faits, voyez-vous, sont beaucoup plus probants que vos méchants discours. C’est pourquoi votre marchandise ne nous dit rien qui vaille. Nous ne voulons pas faire la dure expérience de nos frères d’Allemagne. Veuillez donc déguerpir, et déguerpir très vite, si vous ne tenez pas à ce que nous vous fassions quitter les lieux à coups de fourches !
\section[{Le poteau pour le petit bourgeois protestataire}]{Le poteau pour le petit bourgeois protestataire}
\noindent Remarquez que les petits bourgeois des villes pourraient fort bien – et devraient – tenir aux agitateurs de la Boutique Déat des propos très semblables à ceux de notre paysan. C’est bien la révolte des classes moyennes du Reich qui porta le nazisme au pouvoir. Ce sont les boutiquiers, les artisans que Hitler put embrigader le plus aisément dans les formations du parti nazi et notamment dans l’Union de Combat des Classes moyennes, qu’il excita avec le plus de succès contre le « capitalisme de prêt », contre les grands magasins, contre les juifs, contre la « République des coffres-forts ». Que se passa-t-il ensuite ? L’un des premiers décrets du pouvoir nazi proclamait « indésirable toute action contre les grands magasins ». Indésirable ? Mais les petits bourgeois allemands étaient venus au nazisme précisément parce qu’on leur avait dit qu’une action impitoyable contre les grands magasins était non seulement désirable, mais absolument indispensable et urgente. Les petits bourgeois, désemparés, se frottent les yeux. On ne les laisse pas s’étonner longtemps. Le pouvoir nazi, dès 1933, dissout leur organisation : l’Union de Combat des Classes moyennes. Cette organisation, précisément, avait recommandé à Hitler, pour le poste de ministre de l’Economie nationale, monsieur Wagener. Monsieur Wagener, candidat des classes moyennes, est envoyé… dans un camp de concentration. Les petits bourgeois protestent. L’un d’eux, un pharmacien, Gregor Strasser, crie à la trahison. Il est fusillé le 30 juin 1934. Les coups de fusil contre les petits bourgeois protestataires… et les coups de millions pour renflouer les grandes firmes commerciales Tiez, Karstadt, etc. !
\section[{Le chiffre d’affaires diminue}]{Le chiffre d’affaires diminue}
\noindent Petits commerçants et artisans du Reich vont-ils au moins bénéficier de la “reprise” ? Non. La politique de surarmement qui fait de l’état le client de la grande industrie profite aux magnats. Mais le chiffre d’affaires des artisans dégringole cependant que leurs frais de production augmentent. « \emph{La situation des commerçants détaillants s’est aggravée}, écrit le Deutscher Volkswirt. \emph{Le revenu net des magasins d’alimentation n’est pas supérieur à mille ou mille deux cent marks par an. C’est-à-dire qu’il est inférieur au salaire moyen d’un ouvrier d’usine.} » Le petit commerçant bénéficie-t-il un peu des allègements fiscaux si libéralement accordés par le pouvoir nazi aux grandes firmes ? Non ; il est livré à l’huissier, poursuivi par les ordonnances du préfet de police sur le prix maximum ; traqué par la Gestapo lorsqu’il est soupçonné de se plaindre et de compromettre par ses plaintes le moral du public.
\section[{Misère des intellectuels}]{Misère des intellectuels}
\noindent Le sort des intellectuels n’est pas plus enviable. Depuis 1932, le nombre des étudiants inscrits aux hautes écoles techniques a diminué de 50 \% ; celui des étudiants inscrits à l’université, de 40 \%. On évaluait avant la guerre à plus de deux cent mille le nombre de diplômés d’Université sans travail. Les instituteurs font défaut et pour faciliter l’accession aux postes de l’enseignement, les journaux de mars 1941 annoncent que l’on abaissera le niveau des concours et examens. Séduisant exemple pour les Chevalier, Carcopino et autres destructeurs de l’Enseignement public français.\par
Un compte-rendu d’une réunion tenue à Paris par le “Rassemblement” mentionne que l’auditoire, peu nombreux en vérité, était composé surtout de « représentants des classes moyennes ».\par
Ce sont des auditeurs de ce genre qui venaient écouter la prédication nazie en Allemagne, dans les années qui précédèrent 1933. Quelques milliers sont aujourd’hui parqués dans les camps de concentration ; des dizaines de milliers sont ruinés ; certains ont mis fin à la tragédie en se donnant la mort. Tous ont été les lamentables victimes de l’escroquerie à la « Révolution nationale ».\par
Est-ce que cela vous tente, petits bourgeois de France ?
\chapterclose


\chapteropen
\chapter[{9 – Ordre nouveau ? Non ! Hégémonie impérialiste.}]{9 – Ordre nouveau ? Non ! Hégémonie impérialiste.}\renewcommand{\leftmark}{9 – Ordre nouveau ? Non ! Hégémonie impérialiste.}


\chaptercont
\noindent Essayons de dresser un bilan. Nous avons suivi le National-socialisme à travers les cheminements de son aventure. On nous l’avait représenté comme le bélier qui défoncerait les ploutocraties capitalistes ancestrales. Qu’avons-nous vu ? L’agitation nazie a été subventionnée par les oligarchies capitalistes menacées dans leurs privilèges. Le nazisme s’est-il comporté en mendiant ingrat ? A-t-il faussé compagnie à ses bailleurs de fonds ? Non point. Il avait été favorisé dans son ascension pour qu’il servît les magnats. Il a servi les magnats. Sans doute, dans un monde où le capitalisme avait administré la preuve de son incapacité et de sa faillite, devait-il, pour gagner à lui les multitudes, arborer des mots d’ordre anticapitalistes. Il s’y est prêté d’autant plus que ces mots d’ordre devaient servir de paravent à la politique extérieure du Reich qui allait le mettre en compétition avec les pays du capitalisme traditionnel, la Grande-Bretagne, la France, les États-Unis. Mais ce panneau-réclame lui-même n’était qu’une grossière falsification du socialisme ou, pour mieux dire, n’avait rien de commun avec le socialisme. Il distinguait, comme font toujours les programmes capitalistes, entre “bons” et “mauvais” capitalistes ; il substituait au socialisme des notions que le socialisme a toujours répudiées, qu’il ne peut pas répudier sans cesser d’être le socialisme : nationalisme forcené, antisémitisme barbare. Maître du pouvoir, le nazisme a comblé les espoirs de ceux qui lui avaient ouvert le chemin de la dictature. À l’usine, il a rétabli l’absolutisme patronal et fait des potentats les dirigeants officiels de l’économie allemande. Il a subordonné sa politique douanière aux exigences des hobereaux. Il a délivré les premiers des frêles entraves de l’état et des charges de la fiscalité ; il s’est institué leur client. Il a, sous les formes les plus variées, protégé la puissance des seconds. Grâce à la dictature nazie, la foule des victimes de l’exploitation capitaliste a grandi. Victimes, les ouvriers privés de leurs droits de libre organisation, tenus à l’écart de la marche de l’économie, dépouillés de leurs conventions collectives ; victimes, les paysans écrasés par la machine bureaucratique nazie, construite pour servir les intérêts exclusifs des hobereaux ; victimes, les intellectuels et victimes aussi, les petits artisans et les petits bourgeois des villes, toujours broyés par les trusts et les grands magasins. Ecrivons-le encore et encore, ces catégories de victimes, que le socialisme se propose de défendre, mais que le nazisme victorieux en Allemagne a laissées pantelantes sur le bord du chemin, seraient pareillement sacrifiées en France si elles se laissaient prendre aux attrapes-nigauds du « Rassemblement Déat ».\par
Mais nous avons souligné au passage les différences entre le mouvement national-socialiste et l’aventureuse équipée des nazillons de France.\par
\section[{L’expérience du Front populaire reste vivace}]{L’expérience du Front populaire reste vivace}
\noindent Première différence : le prolétariat français n’a pas négligé les enseignements des évènements d’Allemagne. L’octogénaire vychissois peut bien essayer de tourner en dérision la politique des communistes dans le Front populaire. Les radotages du Maréchal sont impuissants contre ce fait historique : en 1936, grâce à l’union que les communistes avaient cimentée entre le prolétariat et les classes moyennes, les travailleurs français ont connu pendant quelques temps une vie meilleure ; et pour leur arracher leurs acquisitions, le grand capital a dû, au préalable, disloquer le Front populaire. Les masses travailleuses n’ont pas oublié cet épisode récent de l’histoire de France. C’est parce qu’elles ne l’ont pas oublié que les agitateurs du “Rassemblement” ont tant de peine à découvrir pour leur mouvement une base populaire. Leurs modèles nazis bénéficiaient de conditions autrement favorables, dans cette Allemagne de 1930-1931 où ne s’était jamais réalisée la cohésion des victimes du capitalisme sous la direction du prolétariat. Cette différence gêne aux entournures les dirigeants du “Rassemblement”, les contraint à des acrobaties compliquées, à l’usage de formules à double sens susceptibles de rallier à la fois les Cagoulards, les Six-févriéristes, les syndicalistes de contrebande, et les socialistes de contre-marque. Le programme de ces démagogues maladroits ressemble aux résolutions « de synthèse » et aux motions nègre-blanc qui eurent tant de succès jadis dans les congrès de la SFIO.
\section[{Asservissement de la France à l’ennemi}]{Asservissement de la France à l’ennemi}
\noindent Autre différence : Les nazis affirmaient vouloir briser les chaînes de l’oppression étrangère qui pesaient sur l’Allemagne. Programme séduisant aux yeux d’un peuple qui était si cruellement accablé par les injustices du traité de Versailles. Mais la raison d’être des nazillons français est de forger à la France les chaînes de son asservissement à un impérialisme étranger : l’impérialisme allemand. Cet asservissement, sans doute est-il baptisé “collaboration”. Mais tout ce qui a été écrit et avoué depuis six mois, et les faits plus encore que les écrits, ont permis aux Français non atteints de sottise galopante de se faire une idée très précise de la “collaboration”. La définition la plus franche et, n’en doutons pas, la plus orthodoxe de la “collaboration” a été donnée par Marcel Déat : « Je définirais volontiers la collaboration : aider le vainqueur à donner son plein sens à sa victoire ». Mais quel est ce sens ? Ecoutez encore Déat : « \emph{Cela ne s’est peut-être jamais vu parce que jamais les circonstances n’ont été ce qu’elles sont. Parce que c’est la première fois que la conclusion de la guerre civile européenne est la construction de l’Europe.} »
\section[{Une vieillerie capitaliste}]{Une vieillerie capitaliste}
\noindent Nous serions donc en présence d’une nouveauté. Pour la première fois un traité, en consacrant la victoire d’un des belligérants, consacrerait en même temps la paix définitive dans la construction de l’Europe. Voilà qui demande à être examiné de très près. Pour répandre ses vues inespérées, ses enseignements et ses conseils du haut de son pinacle intellectuel, Déat affirme sans sourcillier que l’Allemagne « \emph{a pour mission d’étendre à l’Europe et à l’Afrique les principes nouveaux de sa propre réussite économique et la souveraineté de la production et du travail} ». Mais, nous l’avons montré tout au long de ces notes, ces principes n’ont été qu’un instrument commode de mystification, la « réussite nazie » a été le sauvetage provisoire des privilèges, des oligarchies et la préparation à la guerre, et n’a été que ça. Le régime nazi, c’est celui du profit capitaliste qui persiste et qui s’arrondit ; de l’exploitation de l’homme par l’homme qui continue et s’aggrave. L’ordre nouveau nazi est une vieillerie capitaliste. Etendez cette vieillerie à l’Europe et à l’Afrique, renversez les frontières derrière lesquelles s’est réalisée la concentration du capitalisme allemand, donnez-lui la faculté d’accaparer deux continents, vous ne ferez pas naître une nouveauté. La victoire de l’Allemagne nazie n’a rien à faire avec le triomphe de la souveraineté de la production et du travail. Elle serait la victoire des trusts et des conzerns allemands, des grandes banques allemandes. Qui n’ose pas dire cela (les communistes ont le courage de le dire) est un méprisable pharisien. Les diplomates français, les speakers français de la radio (et parmi eux quelques illustres amis de Marcel Déat), les syndicalistes avariés (aujourd’hui mambres du “Rassemblement”), les renégats putrides (acoquinés présentement à Doriot et au Maréchal) mentaient effrontément lorsque, jusqu’au 12 juin 1940, ils affirmaient que la France assurait la défense du Droit, de la Démocratie. Mais Marcel Déat et ses amis, les syndicalistes avariés et les renégats putrides mentent avec autant d’effronterie lorsque, ayant changé de maîtres, ils proclament, depuis le 20 juin 1940, que la victoire allemande serait le commencement de la construction de l’Europe. Ce n’est pas vrai. Être le complice de l’Allemagne, c’est se prostituer d’une manière très vulgaire, c’est travailler pour le capitalisme, contre le socialisme, contre la Révolution, contre l’Europe.
\section[{Un programme d’expansion impérialiste}]{Un programme d’expansion impérialiste}
\noindent Essayons, du reste, de nous représenter ce que serait la victoire allemande, espoir suprême et suprême pensée des nazillons de France qui rêvent, pour hâter cette victoire, de faire couler le sang français. Essayons de nous le représenter en négligeant la littérature inférieure et brouillonne des journalistes à quatre pattes. Consultons plutôt ceux qui parlent sérieusement, au nom de l’Allemagne. Au début du mois de février 1941, M. Zangen, chef du « Groupe de l’Industrie », parle à Vienne sur les tâches de l’industrie allemande. Il se félicite des profits réalisés par les industriels du Reich. Il fait entrevoir des profits plus considérables encore après la victoire escomptée ; puis il dit : « \emph{Dans la reconstruction économique de l’Europe, le commerce extérieur jouera un rôle important et l’industrie aura alors de grandes tâches constructives à réaliser} ».\par
Ces phrases sybillines peuvent paraître innocentes. Elles expriment le programme – pas nouveau du tout, tant s’en faut ! – de l’impérialisme. La dictature nazie – ne perdons jamais cela de vue – a permis au grand capital de s’assurer d’immenses profits en transformant l’Allemagne en un vaste arsenal de guerre. La guerre finie, le grand capital, pour maintenir ses bénéfices et développer ses investissements, rêve d’accaparer et de contrôler les marchés de l’Europe et de l’Afrique, d’établir à son profit le monopole du commerce extérieur de ces continents ; bref, après avoir sauvegardé ses bénéfices grâce à la dictature nazie, d’étendre ses bénéfices grâce à l’expansion impérialiste\footnote{(Note de l’auteur) Déjà les grandes banques allemandes ont reçu des privilèges importants en Pologne, en Roumanie, en Serbie, en Hollande, en Belgique, en France et… en Italie ! et tous les jours, de nouvelles sociétés s’érigent dans les pays occupés qui sont financées par le grand capital allemand}. Mais que devient la construction européenne nouvelle dans tout cela ? Où est l’ordre nouveau ? Où est la nouveauté ?\par
Nous sommes en face d’un programme d’expansion impérialiste qui, par ses origines et ses buts, ressemble à tous les programmes d’expansion impérialiste, présente toutes leurs tares, porte en lui tous leurs dangers. Ce programme a fait l’objet de commentaires nombreux de la part des porte-parole de l’Allemagne. Quelques semaines après l’armistice, il a été exposé notamment par M. Funk, ministre de l’Economie, et par M. Rosenberg. Que disent ces chefs nazis ? M. Funk explique benoîtement que l’interdépendance des intérêts doit être le grand principe directeur de la vie économique européenne de demain. Dans l’espace européen, les nations devront créer un ordre domestique pour rendre possible la vie commune. Après ce délectable hors-d’œuvre, M. Funk en vient aux précisions. La grande affaire, assure-t-il, est de créer des relations de prix à l’intérieur de la communauté européenne, en prenant le mark comme dénominateur commun. Si l’on établit entre le Reichsmark et les autres devises de telles relations de prix, on crée des conditions particulièrement favorables pour l’échange des marchandises. Toutefois, comme le niveau de vie n’est pas le même dans tous les pays, il conviendra au préalable d’établir les prix en se fondant sur ces différences de niveau de vie. !!! Les prix seront calculés en marks et l’échange des marchandises sera dirigé par un Centre unique, Berlin.\par
Voilà les têtes de chapitre du programme allemand. Retenons ses points essentiels :
\section[{Une seule nation industrielle : l’Allemagne}]{Une seule nation industrielle : l’Allemagne}
\noindent 1° Interdiction est signifiée aux États européens d’entretenir entre eux des relations commerciales particulières. C’est par la Centrale de Berlin que s’organiseront les échanges. C’est la Centrale de Berlin qui aura le monopole du commerce international ;\par
2° Cela suppose que cette Centrale indiquera à chaque membre de la communauté ce qu’il doit produire et ce qu’il doit s’interdire de fabriquer ; ce qu’il doit produire pour la consommation intérieure et ce qu’il doit consacrer à l’exportation. Des indications de ce genre ont été signifiées déjà à la France à qui l’on dit sur tous les tons : Tu dois devenir une nation de cultivateurs et d’artisans. L’industrie n’est point ton affaire. Ferme tes usines, renvoie tes fils à leurs échopes ou à la campagne. L’industrie de l’Europe sera réservée à la Nation-Reine et la Nation-Reine, c’est l’Allemagne. Il est remarquable, d’ailleurs, que dans son schéma, M. Funk, lorsqu’il parle de la direction de l’Europe, néglige de mentionner l’Italie. L’Italie passera sous la toise, comme les autres.
\section[{Le plan nazi}]{Le plan nazi}
\noindent 3° Enfin et surtout l’idée centrale du Plan est celle de la hiérarchie entre les nations. Il prend comme point de départ la différence des niveaux de vie. L’heure de travail n’aura point la même valeur suivant qu’elle sera fournie par un ouvrier français, un ouvrier belge ou un ouvrier allemand. Le retour à la terre, imposé à certaines nations, ne s’inspire ni des possibilités de ces nations (appliquée à la France, cette politique serait une monstrueuse absurdité) ni d’un souci de la répartition du travail dans l’intérêt de la Communauté Européenne, mais du désir d’hégémonie. Dans l’Europe hiérarchisée, le paysan français verra les fruits de ses produits fixés par la Centrale des échanges siégeant à Berlin. Et ces prix seront fixés dans l’intérêt de qui ? Pas des paysans français, bien entendu, mais de l’acheteur allemand. Ainsi s’ouvrira l’âge d’or de l’Economie allemande annoncé par la dernière assemblée des armateurs de Brême.\par
Le socialisme – le vrai, que nous, et nous seuls, représentons, avait souligné que l’égalité juridique des nations sous le régime capitaliste était une fiction. Mais en substituant au désordre capitaliste l’ordre socialiste, le socialisme se propose d’établir l’égalité juridique des nations.\par
Le nazisme s’assigne exactement l’objectif contraire. Pour lui, les nations doivent reconnaître leur inégalité devant les lois internationales, elles doivent reconnaître qu’elles n’ont aucune raison de vivre libres. Et les petites nations sont vouées à la disparition pure et simple. M. Rosenberg le proclame ouvertement. Jugez-en : « \emph{De petits pays d’Europe avaient soulevé le droit de vivre sur un pied d’égalité avec de grandes puissances. Aujourd’hui, ces pays sont obligés de reconnaître le vrai rapport des forces. Nous croyons qu’une petite nation a le devoir de se placer sous la tutelle d’un grand Empire} » (10 juillet 1940).\par
Un quidam s’adressant à de jeunes Français qui, paraît-il, se montraient rebelles à l’esprit de collaboration, leur dit dans L’Œuvre du 5 mars : « \emph{Il n’est pas vrai que chaque nation aura en Europe un rang fixé d’après le traité de paix. Tout commencera au contraire pour chacune d’elles, dans une Europe recommencée} ».\par
Lisez donc Rosenberg, monsieur !
\section[{L’Europe sous la domination allemande}]{L’Europe sous la domination allemande}
\noindent Quand on parle aux Français de “collaboration”, certains pensent à des rapports sur la base de la réciprocité et cette idée les séduit. D’autant que de frivoles plumitifs leur expliquent que Hitler ne fait que suivre la tradition d’Henri quatre, de Napoléon et… d’Aristide Briand qui, eux aussi, rêvaient d’organiser l’Europe. Mais il est très significatif que les formules « poudre aux yeux » si souvent utilisées dans la presse nazie de langue française, comme « États-Unis d’Europe », « Fédération européenne », etc., n’ont presque pas cours dans la presse nazie de langue allemande. Les représentants nazis ne disent pas « États-Unis d’Europe », mais « Europe sous la direction (führung) allemande ».\par
C’est l’Allemagne qui possèdera les machines, fera fonctionner les usines et vendra ses produits industriels ; c’est elle qui dirigera les échanges ; c’est elle qui aura le monopole de la politique étrangère et des forces militaires de l’Europe. C’est son aviation qui exercera le monopole des transports aériens et la Kölnische Zeitung, la Leipziger neuste Nachrichten, d’autres journaux encore nous informent que la Lufthansa se prépare fébrilement à cette tâche.
\section[{Traditions du pangermanisme et de la Hanse}]{Traditions du pangermanisme et de la Hanse}
\noindent Pour que nul ne s’y trompe, à l’heure même où Marcel Déat nous exprime que nous allons assister à l’épanouissement d’une merveilleuse nouveauté, les revues allemandes expliquent qu’il convient de faire revivre les traditions des commis voyageurs allemands qui, avant 1914, furent les fourriers du pangermanisme. Et le Lokal Anzeiger du 20 juillet 1941 veut que l’on remonte plus loin dans l’histoire et que l’on s’inspire de l’esprit d’initiative des hardis commerçants de la Ligue hanséatique\footnote{(Note de l’auteur) ligue des villes allemandes du nord, la Hanse qui dura de 1240 à 1670 groupait 64 villes et avait son siège à Lübeck. Ligue commerciale, elle constituait au temps de sa prospérité un véritable état avec son Conseil de gouvernement et sa flotte}. « \emph{Il faut remarquer}, écrit le journal, \emph{que la Hanse était déjà une communauté du commerce allemand dans le grand espace économique allant de Nijni-Novgorod à Londres, Brest, Nantes et Lisbonne. Il faut faire revivre l’esprit de la Hanse. La vieille Hanse est morte parce qu’il lui manquait l’appui d’un Reich fort. La nouvelle Hanse triomphera parce qu’elle a derrière elle le grand Reich victorieux} ».\par
Les commis voyageurs fourriers du pangermanisme ! La tradition des Hohenzollern et de Von Tirpitz ! L’esprit de la Hanse ! Mais que devient la construction européenne nouvelle dans tout cela ? Où est l’ordre nouveau ? Où est la nouveauté ? Nous sommes en face d’un programme d’expansion impérialiste qui, par ses origines et par ses buts, ressemble à tous les programmes d’expansion impérialiste, présente toutes leurs tares et porte en lui tous leurs dangers.
\section[{Rapine et oppression impérialiste}]{Rapine et oppression impérialiste}
\noindent Depuis l’occupation, nous sommes les témoins, et généralement les victimes, d’opérations de grand style dont nous nous gardons de méconnaître la portée. La France paye à l’occupant une indemnité de quatre cent millions par jour. Les richesses de la France tiennent lieu d’immense butin au vainqueur. La presse allemande dénombre, pour la satisfaction de ses lecteurs, l’importance des stocks sur lesquels l’armée d’occupation a fait main basse. Hitler expose à ses auditoires que le ravitaillement de l’Allemagne sera assuré grâce aux approvisionnements dont le Reich s’est rendu maître dans les pays occupés. Il ajoute dans son discours du mois de mars 1941 que la durée de cette occupation est à ses yeux illimitée (« \emph{Aucune force ne nous fera quitter les territoires que nous occupons à l’heure présente} »). Il est vrai que ces deux passages de la proclamation du Chancelier ont été pudiquement supprimés dans les traductions remises par Son Excellence Monsieur Abetz aux feuilles parisiennes. De même, on a pris la précaution de ne point traduire en Français cette phrase parue dans la Pariser Zeitung in Frankreich qui en dit long sur la mission révolutionnaire de la victoire allemande : « \emph{Les forces armées allemandes en France sont considérées comme un élément d’ordre qui empêche que le peuple se soulève.} » Aucune originalité dans tout cela ! Les vieilles formes de la rapine impérialiste, les vieilles et horribles manifestations de l’oppression impérialiste ; c’est tout.
\section[{Accords commerciaux à sens unique}]{Accords commerciaux à sens unique}
\noindent Poursuivons : Des accords commerciaux ont été passés entre l’Allemagne et les pays qu’elle occupe, des conférences ont réuni les industriels du Reich et ceux des pays occupés, des expositions de produits industriels ont été organisés dans les pays occupés par l’Allemagne. Ces manifestations, nous dit-on, amorcent la collaboration dans l’intérêt des deux pays. C’est à voir : au mois de décembre dernier, un accord de clearing a été signé entre la France et l’Allemagne et un haut fonctionnaire allemand l’a à cette époque commenté devant la presse. Que contient l’accord en question ? Essentiellement ceci : toutes les clauses qui, jusqu’ici, protégeaient l’industrie française contre la concurrence allemande, en particulier les taxes qui empêchaient l’installation en nombre illimité de succursales des firmes allemandes en France, sont abrogées. Mais la contre-partie ? Il n’y a pas de contre-partie. Pourtant, métallurgistes allemands et français se sont rencontrés ! Pourtant, des industriels français ont profité de l’Exposition du Travail allemand du petit Palais pour se faire octroyer des commandes ! Pourtant, les services de M. de Brinon ont annoncé que l’Allemagne s’intéressait à la fourniture de matières premières aux industriels français qui en étaient dépourvus. Essayons de mettre un peu d’ordre parmi tant de démarches apparamment contradictoires, mais qui toutes convergent vers le même but. L’Allemagne utilise l’industrie française pour sa machine de guerre. Elle fournit dans ce but certaines matières premières ; elle aide dans ce but à la réfection de certaines lignes ferroviaires ; dans ce but elle confie à certains industriels la fabrication de certaines pièces dont elle donne le modèle. Ce procédé, qui n’a qu’un caractère temporaire, permet à l’industrie allemande de ne pas se consacrer exclusivement aux fabrications de guerre ; une partie des usines allemandes travaille pour l’exportation. La guerre finie, le passage de l’industrie de guerre à l’industrie de paix perturbera les autres économies, mais grâce à la méthode que nous venons de signaler, il atteindra à un moindre degré l’économie allemande. Il permettra à l’Allemagne de réaliser plus aisément sa politique d’expansion impérialiste.\par
C’est ainsi que l’exposition du petit Palais a été agencée par des spécialistes du ministère de la guerre allemand et des représentants du Parti nazi. « \emph{Il s’agit}, écrit la Frankfurter Zeitung \emph{d’adapter l’économie française aux besoins économiques de l’Europe centrale, et de trouver aux industries françaises un nouveau chemin en les englobant dans le marché allemand.} »\par
Dans un autre ordre d’idées, ce sont les caisses de crédit de la Reichsbank, par le truchement des banques françaises qui lui sont subordonnées, qui financent les travaux de reconstruction entrepris en France, c’est-à-dire qui tiennent à leur merci les industries intéressées à leurs travaux. Ces caisses de crédits sont désignées dans les bilans de la Reichsbank sous le nom de Schnelltruppe, les troupes rapides de la Banque !
\section[{Mainmise sur l’industrie des pays occupés}]{Mainmise sur l’industrie des pays occupés}
\noindent Parallèlement à cet effort, l’Allemagne en poursuit un autre : elle s’assure pour l’avenir le contrôle de l’industrie des pays qu’elle occupe en se faisant attribuer les actions des sociétés industrielles en échange de participations qu’elle accorde, ou plutôt qu’elle impose, dans son industrie à des capitalistes français. C’est une négociation de ce genre qui a abouti quelques semaines après l’armistice à un accord entre de Wendel et le Trust Hermann Göring. Un accord analogue est intervenu également entre les fabricants français et allemands de soie artificielle. Le trust allemand de la soie artificielle a signé un accord de collaboration avec la France-Rayonne. La France-Rayonne comprend dix-neuf sociétés parmi lesquelles la Viscose, le Textile de Givet, la Société lyonnaise de Textile artificiel, la Rayonne d’Avignon, la Viscamine de Pontcharra, les Filets de Calais, la Rayonne de Valenciennes, le Textile de Gauchy, la Cellophane de Bezons, le Textile artificiel de Bezons, la Soie d’Argenteuil. En échange de participations dans l’industrie allemande de la soie artificielle, les dirigeants de France Rayonne ont dû céder 33 \% des actions de la société au trust allemand. Celui-ci suggère une réorganisation de la Rayonne, la suppression de quelques-unes des petites sociétés, le développement de certaines branches. Des machines seront vendues par l’Allemagne. Cette réorganisation nécessitera de nouveaux capitaux ; la société créera de nouvelles actions. Les nouvelles actions seront achetées par le trust allemand qui dirigera l’industrie française de la soie artificielle.\par
Les industriels français sont conviés à ce genre de collaboration. S’ils s’y refusent, la presse nazifiée entreprend contre eux un tir à boulets rouges. Ainsi s’expliquent ces campagnes contre certains trusts brusquement amorcées et soudainement achevées dans Paris-Soir, France au travail et autres feuilles. On amorce la campagne lorsque l’industriel se montre rebelle aux propositions de ses interlocuteurs allemands. On l’arrête lorsque, menacé de voir bloqué son compte en banque, l’industriel a cédé…\par
Manifestation d’une mainmise capitaliste, politique traditionnelle et hardie de l’expansion impérialiste, si l’on veut ; mais innovation ? Nouveauté ? Et surtout nouveauté socialiste ? Non ! Non et non.
\section[{Des hauts fourneaux bombardés en Hollande}]{Des hauts fourneaux bombardés en Hollande}
\noindent Faut-il d’autres exemples pour rendre cette démonstration plus convaincante ? Demandez donc aux Hollandais, à ces Hollandais qui, à la fin du mois de février 1941, se sont insurgés contre l’occupant. Demandez donc aux Norvégiens, dont les meilleurs fils sont condamnés à mort par l’occupant et qui, malgré les baïonnettes allemandes, donnent la chasse aux partisans du Déat norvégien qui s’appelle Quisling ; demandez-leur comment en Hollande et en Norvège se traduit l’ordre nouveau et comment s’exprime ce que le directeur de L’Œuvre appelle la souveraineté du travail. Les Hollandais vous répondront : la principale entreprise métallurgique du pays, la Koningklijke Nederlandsche Hoogovens en Staatfabrieken im Ymuiden a eu ses hauts-fourneaux bombardés par l’aviation l’année dernière. Après l’occupation du pays, les autorités allemandes ont offert à la firme Hoogovens un accord de collaboration avec les Aciéries réunies d’Allemagne. Le conseil municipal d’Amsterdam a été mis dans l’obligation de vendre sa participation dans l’entreprise Hoogovens, soit trente mille actions qui ont été rachetées par les métallurgistes allemands. La participation de l’état hollandais dans l’affaire est, si l’on en croit la Frankfurter Zeitung du 13 février 1941, sur le point d’être vendue dans les mêmes conditions. En échange, Hoogovens obtient une participation dans les Aciéries réunies. Mais la firme hollandaise dépend entièrement de la sidérurgie allemande pour ses matières premières. Si bien que la collaboration se traduit ainsi : la métallurgie allemande s’assure un vrai contrôle sur l’industrie hollandaise tenue à sa merci. Quant à l’industrie hollandaise, sa participation aux industries allemandes ne lui assure aucun contrôle, mais la lie étroitement à l’industrie allemande, fait d’elle l’alliée de l’Allemagne en Hollande.
\section[{Travailler sous la volonté germanique}]{Travailler sous la volonté germanique}
\noindent Les Norvégiens, si vous les interrogez sur le même sujet, pourront vous fournir bien des exemples semblables. Mais peut-être se contenteront-ils de vous renvoyer aux propos affectueux que leur a consacrés Monsieur Rosenberg dans ses déclarations du 10 juillet 1940 à la presse étrangère : « \emph{Le sort a voulu que nous allions en Norvège et au Danemark. Rien ne pourra désormais nous séparer de ces pays. Nous y éprouvons encore des difficultés, mais elles seront surmontées par l’éducation. Après avoir vécu dans le même espace, ces peuples comprendront qu’ils doivent travailler sous la volonté germanique.} »\par
Et Monsieur Rosenberg rappelle que les nations nordiques (dans lesquelles il englobe la Normandie) eurent toujours besoin d’un suzerain : les conquérants danois, les Vikings, enfin Adolf Hitler.\par
Marcel Déat annonce à ses lecteurs : « \emph{Rien de comparable ne s’est jamais produit dans l’histoire.} » Mais Rosenberg rectifie : « \emph{Nous continuons les guerriers danois et les Vikings} ! » Ordre nouveau ! Nouvelle construction de l’Europe !
\section[{Entreprises expropriées}]{Entreprises expropriées}
\noindent En Tchécoslovaquie et en Pologne, l’occupant a pris moins de ménagements. Il a exproprié. Les catalogues allemands font figurer les entreprises tchèques et polonaises parmi les trusts allemands. Dans son entreprise d’expansion, l’Allemagne n’a pas épargné l’allié italien. Elle s’est assurée le contrôle de la péninsule. Au surplus, pour se procurer du charbon en Allemagne, Mussolini a dû consentir à la déportation de trois cent mille métallurgistes italiens dans les usines allemandes.\par
Dans l’Europe “nouvelle”, les nations du sud-est doivent être l’objet d’une attention très particulière de la part de l’Allemagne nazie. De fait, au début du printemps 1941, la guerre étend ses ravages sur ce secteur du continent. La Roumanie est occupée par l’armée allemande depuis le mois d’octobre. Les forces armées allemandes ont franchi le 2 mars la frontière bulgare. La Yougoslavie et la Grèce sont à feu et à sang. De quoi s’agit-il ? Pendant les années qui suivirent le traité de Versailles, les pays du sud-est ont été les vassaux de l’impérialisme anglo-français. La France et l’Angleterre accordaient à ces pays des emprunts et s’assuraient un contrôle de leur diplomatie. Disons, en passant, que ce procédé, souvent flétri par les communistes, ne souleva jamais la moindre critique de la part de Marcel Déat et de ses acolytes. Dans la dette publique des pays du sud-est européen, les emprunts contractés à Londres et à Paris figuraient pour 60 à 70 \%. Toutes les positions clefs de l’économie roumaine et yougoslave (50 \% des pétroles roumains, la majeure partie des exploitations minières yougoslaves) étaient contrôlées par le capital occidental. Ce système caractérisait la politique d’expansion des vieilles ploutocraties françaises. Mais voici qu’intervient l’Allemagne ; elle va, croyez-vous, renverser tout cela. L’ordre nouveau ne peut s’accomoder des vieilles méthodes des oligarchies capitalistes. C’est ce que M. Déat nous explique tous les matins.\par
Mais qu’importe aux maîtres de l’Allemagne les prédications du pêcheur de lunes qui noircit les colonnes de L’Œuvre. Le Dr. Abst, président de la Deutsche Bank, expose l’idée des dirigeants nazis sur la façon de se conduire dans le sud-est de l’Europe et ses vues, telles que les rapporte la Frankfurter Zeitung, méritent la plus grande considération. Que disent les maîtres de l’Allemagne ? Ils constatent que, malgré leurs emprunts et leurs participations aux sociétés balkaniques, la France et l’Angleterre n’ont pas réalisé leurs ambitions. Pendant l’hiver de 1940, des groupes pétroliers roumains, concurrents de ceux que contrôlait le capital anglo-français, ont réussi à faire prévaloir une sorte de monopole d’état sur les pétroles roumains et à frustrer la France et l’Angleterre du bénéfice de leurs efforts. L’Allemagne ne négligera pas cet enseignement. À quoi bon accorder des emprunts, participer aux industries, si on permet à un état d’user de ce qui lui reste de souveraineté pour vous priver des avantages de votre action ? Il convient, par conséquent, de s’assurer en tout état de cause « des relations politiques satisfaisantes » avec l’état en question. Et somme toute, pour que ces relations soient tout à fait satisfaisantes, le mieux n’est-il pas d’occuper militairement le territoire ?
\section[{« Participation indirecte » à l’industrie}]{« Participation indirecte » à l’industrie}
\noindent Mais les pays du sud-est ont besoin de crédits. L’occupation de leur territoire ne leur en procure aucun, au contraire. Comment satisfaire leurs besoins de crédits ? M. Abst a sa recette : « L\emph{es pays du Sud-Est}, dit-il, \emph{seront satisfaits dans leurs besoins de crédits grâce aux soldes actifs qui se produiront à l’avantage de l’Allemagne dans les comptes de clearing.} » Or, ces soldes actifs sont créés artificiellement par l’Allemagne par la dévaluation réalisée des monnaies balkaniques par rapport au mark. Lorsque dans les comptes de clearing, l’Allemagne aura un solde actif, elle n’exigera pas qu’il soit couvert par une contre-livraison. Si le client de l’Allemagne ne peut payer en marchandises, il émettra des obligations ordinaires qui seront placées par les banques allemandes sur le marché des capitaux allemands. C’est ce que le Dr Abst appelle la participation indirecte à l’industrie des pays du sud-est. Cette participation assure, beaucoup plus efficacement que le système pratiqué autrefois par les anglo-français, la mainmise sur l’économie du sud-est.
\section[{Spoliations dans les Balkans}]{Spoliations dans les Balkans}
\noindent Est-ce à dire que l’Allemagne renonce à la méthode des participations directes dans les industries balkaniques ? Non point. À preuve la participation allemande dans trente-six sociétés par actions roumaines avec un capital de 146 millions de leis et le contrôle de toutes les richesses nationales roumaines. À preuve l’acquisition, plusieurs semaines avant l’invasion, par un groupe allemand, de la Compagnie française des Mines de Bor, en Yougoslavie. Le capital de cette compagnie était passé de 15 à 60 millions de francs pour atteindre récemment 120 millions. En 1939, les Mirabau qui dirigeaint la compagnie avaient distribué 70 \% de dividendes. Sur cinq cent cinquante mille tonnes de minerai extraites en 1939, on avait évalué quarante-deux mille tonnes de cuivre brut. Ce sont les Mines de Bor qui font de la Yougoslavie le principal producteur de cuivre européen. La Compagnie exploitait également le minerai des Mines de Loudayana, en Bulgarie, qui sont (ou qui étaient) une propriété française. Les Mirabau allemands remplacent les Mirabau français. Les luttes sanglantes qui se déroulent dans cette partie de l’Europe n’ont pas d’autre objet que la mainmise allemande sur la totalité des richesses balkaniques et la création de bases pour de nouvelles conquêtes et de nouveaux pillages. En dehors de Marcel Déat et de son équipe, nul homme normal ne pourra prétendre que ces spoliations sont un succès de la « révolution sociale ».\par
Dernier article du programme : «\emph{ Lorsque l’Allemagne se sera assuré ces participations}, dit encore M. Abst, \emph{des ingénieurs, des techniciens, des spécialistes allemands seront envoyés sur place.} » Au surplus, ce sont les firmes allemandes associées aux entreprises du sud-est qui devront décider de la création de nouvelles usines, de l’ouverture de nouvelles possibilités. « \emph{L’initiative en ces matières}, écrit la Frankfurter Zeitung, \emph{doit venir des firmes allemandes qui connaissent les besoins du marché allemand} », car il s’agit, n’est-ce pas, de satisfaire le marché allemand et il ne s’agit que de cela !\par
Mais où est l’ordre nouveau ? Où est la nouvelle construction européenne ? Les décisions des maîtres de l’Allemagne nous répondent : La politique d’expansion impérialiste de l’Allemagne dans le Sud-Est sera plus hardie que celle de ses prédécesseurs.\par
Déat, les yeux mouillés, nous dit : « \emph{Regardez-les, les hommes de la croix gammée, ils innovent.} » Les hommes de la croix gammée n’innovent pas. Ils mettent au point la grande machine d’exploitation impérialiste pour broyer les peuples.
\section[{Les nazis colonialistes}]{Les nazis colonialistes}
\noindent La “collaboration”, on le sait, a choisi pour champ d’action l’Europe et l’Afrique, l’Eurafrique comme disent les disciples de Déat quand ils veulent faire croire qu’ils savent la géographie. La France et l’Angleterre se sont appropriées la majeure partie du continent africain, du nord au sud et de l’est à l’ouest. Bref, les aventures coloniales caractérisaient le vieil ordre dont l’Allemagne nazie se prétend l’adversaire. Il semblerait que les représentants d’un ordre nouveau, d’une construction nouvelle dussent rompre avec ces traditions de violence impérialiste. Est-ce cette rupture qu’il faut attendre de la victoire allemande ? Point du tout. Les dirigeants allemands tiennent à l’égard des populations de l’Afrique le langage méprisant que les colonialistes français avaient tenu au début de leur carrière. Ils considèrent les habitants du continent africain comme des hommes inférieurs, aptes à servir de chair à profit, mais dépourvus d’autre intérêt. L’ordre nazi respectera les plus affreuses traditions de la ploutocratie. Mais en quoi consistera la nouveauté ? En ceci essentiellement : la France devra partager avec l’Allemagne ses dépouilles africaines, c’est-à-dire les livrer à l’Allemagne et recevoir quelques reliefs du festin. Une opération de ce genre est en train de s’accomplir en Asie où de larges secteurs de terres soumises à l’administration française passent sous le contrôle japonais.\par
Nous nous refusons, nous qui sommes fidèles à l’esprit et à la lettre du vrai socialisme, à voir dans cette substitution la marque d’un progrès, le signe d’un ordre nouveau.\par
Walter Darré, ministre de l’agriculture du Reich, évoquant un jour les perspectives de la victoire allemande, a déclaré : « \emph{Nous créerons un peuple de seigneurs.} » Le peuple de seigneurs serait le peuple allemand. Les autres seraient les peuples esclaves.\par
Depuis huit ans, on a, sous toutes les formes, répété au peuple allemand la phrase de Walter Darré.
\section[{Mentir, mentir démesurément}]{Mentir, mentir démesurément}
\noindent Ouvrons ici une parenthèse. Si le régime nazi est bien tel que vous l’avez décrit, nous disent certains, comment n’a-t-il pas suscité encore la révolte des victimes ? Comment a-t-il pu les entraîner dans la grande aventure de la guerre ?\par
Deux explications viennent à l’esprit : tout d’abord, il y a un instrument que, sans contredit possible, le nazisme a extraordinairement perfectionné, c’est celui de la Mystification. Le mot revient souvent sous notre plume. C’est qu’il n’en est pas d’autre pour caractériser le phénomène dont l’Allemagne est le théâtre depuis huit ans. « \emph{Il n’y a que les mensonges démesurés qui produisent de l’effet}, a écrit Hitler, \emph{les mensonges tellement exorbitants que personne ne peut croire qu’il s’agit de mensonges… La masse sera plus facilement victime d’un grand mensonge que d’un petit mensonge.} » Et pendant huit ans, dans ce pays où la presse, la radio, les instruments de propagande les plus perfectionnés sont au service des fabricants de mensonges, on a cultivé ce mensonge capital auquel tous les autres sont greffés : le Nazisme réalise le Socialisme.\par
N’accablez pas trop le peuple allemand de s’être laissé abuser, puisque aussi bien hors des frontières de l’Allemagne, dans des pays où les souffrances de l’autre guerre furent moins douloureuses, où le poison nazi n’était pas répandu à la même dose et ne jouissait d’aucun monopole, la propagande allemande a su capter des proies aux mailles de son filet brun.\par
Et n’oubliez pas, en outre, cette explication complémentaire : depuis huit ans, la propagande nazie dit au peuple allemand « tu es aujourd’hui condamné à d’immenses sacrifices, aux restrictions, à la guerre. Si les rivaux de l’Allemagne l’emportaient tu serais condamné à des souffrances plus cruelles encore. Alors que la victoire allemande sera la fin de ton calvaire. Tu seras, comme dit Walter Darré, le peuple des Seigneurs. »
\section[{Caste des seigneurs, peuple esclave}]{Caste des seigneurs, peuple esclave}
\noindent Mais Walter Darré ment au peuple allemand. La victoire nazie créerait sans doute une caste de seigneurs, assurément pas un peuple de seigneurs. De cette victoire, le peuple allemand ne serait pas bénéficiaire ; à brève échéance, il serait victime comme les autres peuples. Les ouvriers d’Allemagne ne seront pas moins misérables lorsque les ouvriers et les paysans d’Europe et d’Afrique travailleront sous le fouet du maître allemand. Rendu plus puissant par les profits de son butin colonial, dont la défense imposera le lourd tribut d’armements suffocants, ce maître sera plus tyrannique et plus féroce envers les ouvriers de la métropole. Ceux-là ne seront pas dans le camp des seigneurs. Ils resteront dans celui des esclaves.\par
Qu’importe, nous répondent les résignés ; qu’importe cette servitude si elle doit nous assurer la paix ? Or, la victoire allemande créera un statut définitif du continent ! Elle éliminera toutes les possibilités de révision, de récrimination. Puisque aucune réclamation ne sera tolérée, la paix sera enfin assurée pour des siècles.\par
Eh bien non, ce statut ne sera pas plus définitif que les autres statuts imposés par la violence. Même cette paix de cimetière sera fragile et précaire. Même avec l’appui de ses Déat, de ses Pétain, de ses Quisling, de ses Degrelle, Hitler n’a pas trouvé le moyen d’imposer que la roue de l’Histoire cesse de tourner après l’heure H du Diktat nazi.\par
Quand fut signé le traité de Versailles, un historien français, Jacques Bainville, écrivit : « \emph{le traité est trop doux pour ce qu’il contient de dur.} » Il voulait dire par là que les Alliés, en même temps qu’ils imposaient des clauses particulièrement sévères à l’Allemagne, avaient laissé encore au vaincu trop de forces et qu’il était vraisemblable qu’un jour viendrait où le vaincu utiliserait ces forces pour briser les chaînes du traité. Il est très certain que Hitler s’inspirera de cette considération. Il ne se contentera pas d’annexer des territoires ; il ne se contentera pas d’une édition nouvelle de Versailles ; il en présentera une édition terriblement aggravée. Il nous en donne déjà la préface. Il ne se contente pas d’un demi-désarmement du vaincu. Au désarmement militaire, il ajoute le désarmement économique. Sur le territoire du vaincu il ne laisse subsister que les industries qui dépendront de l’Allemagne. L’Allemagne monopolise les échanges internationaux. Elle impose au vaincu l’organisation politique la plus susceptible de satisfaire les exigences allemandes. Elle lui dicte une politique de répression et de réaction sociale.
\section[{Il n’y a pas de rapport de forces définitif}]{Il n’y a pas de rapport de forces définitif}
\noindent Mais se trouve-t-il un Français assez naïf pour imaginer que toutes ces mesures de coercition et de violences empêcheront les dynamites de s’accumuler et l’explosion de se produire ? Veuillez considérer que nous sommes en 1941 et non plus au printemps de 1940. Nous voulons dire que nous sommes à une époque où l’incendie de la guerre s’étend au monde entier, où il devient chaque jour plus difficile aux combattants de conserver certaines de leurs forces en réserve, où ils doivent précipiter dans la fournaise tout ce qu’ils possèdent, se battre sur trois ou quatre fronts. Qu’il est plus facile à une machine de guerre puissante d’occuper des territoires que de contraindre l’adversaire à signer la paix ; qu’est-ce à dire, sinon que l’épuisement général rendra bien difficile la construction de ces instruments d’oppression ultra-perfectionnés grâce auxquels le vainqueur pomperait sans compter la substance des vaincus et digèrerait tranquillement sa victoire. Les perspectives de digestions tranquilles s’estompent de plus en plus, à mesure que la guerre se prolonge. Les auteurs des traités de violences s’imaginent toujours qu’ils travaillent pour l’éternité. Mais l’histoire se moque de leurs prétentions. Il n’est pas de carcan de fer si bien ajusté qui ne présente des fissures. Il n’est pas de rapport de forces définitif. Même en supposant que Hitler n’offre comme récompense à la nation allemande victorieuse que la mission de monter la garde contre les peuples asservis, et que la nation allemande se satisfasse de ce présent ; même en supposant que les seigneurs allemands prospèrent dans une Europe où tous leurs clients seraient exangues et affamés ; même en formulant ces suppositions hardies qui feraient de l’Europe de l’Ordre nouveau une Europe plus affreuse encore et moins viable que celle dans laquelle nous avons vécu pendant vingt ans, il est extravagant d’imaginer que dans cette Europe forgée par un vainqueur terriblement saigné lui-même, les rapports de forces demeureront figés pendant des siècles.\par
Aucun artifice, aucune interdiction n’empêcheront le bouillonnement de la chaudière.\par
Le statut de l’ordre nazi ne sera pas un statut définitif. Il portera en lui les germes de sa propre dislocation. Aucun diktat n’empêchera qu’à plus ou moins brève échéance, l’équilibre du traité imposé par la force soit rompu. Et qui nous garantit, au surplus, que la paix ne sera pas mise en cause avant même que cette rupture se produise ? Qui nous garantit contre une croisade guerrière de cette Europe dirigée par l’Allemagne contre le socialisme victorieux en URSS ou contre les États-Unis, rivaux récalcitrants ?\par
Le sauvetage de la dictature ploutocratique, au dedans, et le sacrifice des prolétaires à l’exploitation ploutocratique ; l’hégémonie impérialiste au dehors avec les dangers inéluctables de guerre qu’elle porte en elle et le sacrifice des masses populaires dans les aventures de conquêtes, de spoliation ou de revanche, c’est cela le national-socialisme des nazis d’Allemagne et des nazillons de France !
\chapterclose


\chapteropen
\chapter[{10 – Les communistes contre les falsificateurs du socialisme}]{10 – Les communistes contre les falsificateurs du socialisme}\renewcommand{\leftmark}{10 – Les communistes contre les falsificateurs du socialisme}


\chaptercont
\noindent Parmi tant de services que les nazis d’Allemagne et les nazillons de chez nous ont rendu à la réaction capitaliste, l’un des plus remarquables est probablement leur impudente tentative de falsification du socialisme. Présenter aux masses avides de justice une caricature grossière en leur disant : « Voilà le socialisme ! » Leur offrir les cogitations primaires d’un Rosenberg ou d’un Fergy, les recommandations pédantes d’un Marcel Déat, l’immorale physionomie d’un Laval, les apostrophes antisémites d’un Göbbels ou d’un sous-Göbbels ; restaurer les pratiques médiévales et proclamer : « Voilà le socialisme ! » Leur montrer l’Allemagne de 1941 où subsistent l’exploitation de l’homme par l’homme, le vol de la plus-value, le profitariat capitaliste, l’enrichissement des marchands de canons, la dictature des hobereaux, montrer tout cela et dire aux masses « Voilà le socialisme ! » n’est-ce pas le plus sûr moyen de rendre le socialisme odieux, d’en détourner les hommes et de les river au vieil ordre où quelques oligarchies vivent de la sueur et du sang des multitudes ?\par
Tel est, sans doute, l’un des buts que leurs bailleurs de fonds ont assigné aux gens du “Rassemblement”. Ils l’atteindraient peut-être si nous n’étions pas là pour arracher leur défroque et mettre à jour leur subterfuge. Mais nous sommes résolus à crier la vérité. Et aux travailleurs que l’on veut abuser, nous allons dire : « Méfiez-vous des imposteurs du “Rassemblement” ! La camelote qu’ils vous présentent est truquée, c’est une abominable contrefaçon ; ce n’est pas le socialisme, c’est l’antisocialisme. Pour vous sauver, détournez-vous de ces escrocs, donnez la chasse à ces mystificateurs. Ecoutez-nous : Le salut, c’est le socialisme. Mais l’on n’est socialiste qu’en étant communiste. Car il n’y a dans le monde qu’une expérience socialiste, c’est celle qui se poursuit depuis vingt-trois ans sur le territoire de l’Union Soviétique sous la direction des communistes.\par
\section[{L’œuvre des communistes au pouvoir}]{L’œuvre des communistes au pouvoir}
\noindent Qu’ont fait les communistes lorsque la Révolution – la vraie, pas celle qui se débite par tranches de décrets-lois dans l’Officiel, ou par tranches d’éloquence dans des meetings privés du “Rassemblement” – les eut portés au pouvoir ?\par
Ils ont fait ce que les nazis d’Allemagne n’ont pas fait, ce que les nazillons de France ne songent nullement à faire : Ils ont aboli le parasitisme capitaliste. Ils ont instauré la propriété sociale des moyens de production. La propriété socialiste, ainsi que l’expose la Constitution stalinienne, est soit propriété d’état, c’est-à-dire propriété de l’ensemble du peuple, soit propriété coopérative du kolkhoz (économie collective de paysans fondée sur la propriété collective née du travail collectif). Toutefois, la loi soviétique admet les petites économies privées de paysans individuels et d’artisans fondées sur le travail personnel et excluant le travail d’autrui. La grande affaire du vrai socialisme, c’est la suppression de l’exploitation de l’homme par l’homme, la suppression du vol de la plus-value. Nul ne peut encaisser des revenus provenant du travail d’autrui. Voilà le principe du socialisme. Ce principe n’est appliqué que dans un pays, dans le pays où les communistes ont le pouvoir, en Union Soviétique.\par
Nous avons vu que les nazis sont parfaitement d’accord avec les oligarchies de leur pays et qu’ils les considèrent comme indispensables.\par
Les communistes ont démontré par les faits que l’économie pouvait avoir un autre stimulant que le profit et que l’abolition du mode d’exploitation capitaliste délivrait l’économie, la libérait, élargissait ses horizons dans l’intérêt des masses travailleuses.
\section[{Le stimulant socialiste}]{Le stimulant socialiste}
\noindent Quel est donc le stimulant nouveau ? C’est à la place du profit d’une petite minorité, l’ascension de la société tout entière vers une vie meilleure, grâce à la production de richesses toujours nouvelles. Puisque aucune richesse n’est désormais accaparée sous forme de profit, toutes les richesses peuvent être consacrées à l’amélioration du sort des travailleurs et à l’élargissement continuel de la production. Voilà la vraie révolution : désormais, la production peut se développer sans cesse parce qu’elle ne se heurte plus aux limites du pouvoir d’achat des masses. Il ne s’agit plus d’accumuler un capital pour enrichir un propriétaire, mais bien d’accumuler une fortune nationale, c’est-à-dire des moyens de production qui servent à augmenter la consommation des masses. Il faut croire que le stimulant du socialisme est supérieur au stimulant capitaliste puisque, alors que l’économie du monde capitaliste, au plus fort de son essor, en 1937, n’a jamais atteint plus de 96 \% de son niveau de 1929, l’industrie soviétique a atteint 428 \% de son niveau de 1929 et a septuplé par rapport à l’avant-guerre.
\section[{Répartition des produits du travail}]{Répartition des produits du travail}
\noindent Nous avons vu comment se répartissait le revenu dans l’Allemagne capitaliste. Une partie est distribuée sous forme de dividendes aux actionnaires, une autre partie est capitalisée et demeure propriété privée des maîtres de l’entreprise. La répartition des produits du travail s’effectue d’une manière toute différente dans l’état socialiste : Une première partie est consacrée \emph{à la satisfaction des besoins sociaux et culturels des travailleurs}. Une autre est consacrée au \emph{développement de l’appareil de défense de la société socialiste.} Une autre encore sert à \emph{l’élargissement de la production}. Une quatrième partie est destinée à la \emph{rétribution du travail} et cette rétribution s’effectue selon le principe « \emph{De chacun selon ses capacités, à chacun selon son travail} ».
\section[{Accroissement du bien-être}]{Accroissement du bien-être}
\noindent Dans l’Allemagne nazie, nous l’avons vu, les salaires ont été réduits. En URSS, le salaire annuel moyen est passé en dix ans de 514 roubles à 2882 roubles. Dans l’Allemagne nazie, une machine de régression sociale a été construite. En URSS, de 1932 à 1938, les crédits consacrés aux œuvres sociales sont passés de 6 milliards à 35 milliards de roubles.\par
Dans l’Allemagne nazie, on a prêché la nécessité des restrictions et des privations. Les théoriciens du “Rassemblement” nous recommandent d’en finir avec notre penchant à la jouissance. Au pays du vrai socialisme, la consommation du pain a augmenté en six ans de 28 \%, celle des fruits de 94 \%, celle de la viande de 87 \%, celle des vêtements de 89 \%.\par
Les conséquences de l’instauration du socialisme ne se bornent pas à ces données. Il a fait de la vieille Russie, pays agraire arriéré, l’Union Soviétique, puissance industrielle moderne, où la production libérée des entraves capitalistes se développe sans cesse, où, en même temps que s’installent les machines perfectionnées, tous les travailleurs sont occupés. Le droit au travail est inscrit dans la constitution. Le nombre des ouvriers passe de onze millions quatre cent mille en 1914 à vingt-six millions en 1937 et plus des deux tiers de ces ouvriers suivent les cours des écoles techniques.\par
Ce développement de la production socialiste ne s’effectue pas au petit bonheur, mais selon un plan. L’économie socialiste est une économie planifiée. On répondra peut-être que monsieur Göring aussi a un plan et monsieur Bouthillier de même. Oui, mais il n’y a rien de commun entre ces plans et les plans d’économie socialiste.
\section[{Plan capitaliste et plan socialiste}]{Plan capitaliste et plan socialiste}
\noindent L’économie ne peut être planifiée que lorsque les grands moyens de production dépendent d’une direction unique, c’est-à-dire lorsqu’ils sont devenus la propriété collective de la société tout entière. Il n’y a pas de socialisme sans plan, mais il n’y a pas de plan économique, de planification véritable dans une économie que désagrège la contradiction entre l’état des forces productives et le régime social. Les soi-disant plans de l’Allemagne national-socialiste sont élaborés dans les cabinets ministériels, ils sont l’œuvre des oligarchies financières, des militaires, etc. Il ne peut en être autrement, puisque leur objet est de servir les oligarchies. Le plan socialiste est l’œuvre collective de tous les producteurs qui en sont les bénéficiaires. Ce sont eux qui en discutent l’orientation dans les usines ; ce sont eux qui suggèrent les chiffres qu’il convient d’y inscrire. Le plan adopté, les producteurs rivalisent de zèle pour en dépasser les normes.\par

\begin{quoteblock}
 \noindent « Le plan n’est pas une énumération de chiffres et de tâches. Par essence, il est l’activité vivante, pratique, de millions d’hommes. »\par
 
\bibl{(Staline)}
 \end{quoteblock}

\noindent Ainsi peuvent naître des initiatives grandioses comme celles qui sont à l’origine du mouvement stakhanoviste, mouvement qui n’a rien de comparable dans aucun pays, précisément parce qu’il ne pouvait surgir que là où l’exploitation capitaliste a été abolie, là où l’augmentation de la production accroît le bien-être des masses, là où ne jouent plus les lois du capitalisme.\par
La propriété sociale des moyens de production, la fin de l’exploitation de l’homme par l’homme, la fin du profit capitaliste, une production libérée des entraves où la paralysaient les privilèges d’une oligarchie, une économie qui ne connait plus les crises du vieux monde, des travailleurs qui ne connaissent plus le chômage mais qui s’assurent la maîtrise de la technique et puisent dans les trésors de la science, des millions d’hommes qui, périodiquement, dressent les plans nouveaux de la gigantesque construction. C’est ça le vrai socialisme !\par
Nous sommes loin des mystifications. Nous affrontons des réalités.\par
Nous sommes loin des formules menteuses et des promesses hypocrites ; nous dénombrons des faits, des résultats. Nous avons abandonné les sables mouvants et les marécages ; nous sommes sur la terre ferme, sur le roc.\par
Là est née une société harmonieuse ; là est née une \emph{démocratie sincère}, là est née une \emph{communauté fraternelle de peuples}.
\section[{Une société harmonieuse}]{Une société harmonieuse}
\noindent Vous vous souvenez de la façon dont le Maréchal octogénaire a parlé à Saint-Étienne de la lutte de classes et de la nécessité d’y mettre un terme en supprimant les causes qui l’engendrent. C’est Bergery, paraît-il, qui lui avait soufflé cette formule. Mais comment supprimer les causes qui engendrent la lutte de classes ? Les bons apôtres ne l’ont point révélé. Il est assez facile pourtant de deviner leur pensée. Elle s’apparente à la ruse grossière qu’ont employée pendant des années les chefs réformistes du Parti socialiste et de la CGT qui se mettaient au service de la ploutocratie. L’harmonie sociale, selon eux, c’est la résignation ouvrière et le paternalisme patronal. Que les syndicats soient dissous, que les délégués ouvriers élus soient remplacés par des hommes de confiance du grand patronat et la lutte de classes aura cessé d’exister pour ces Messieurs. Que l’ouvrier renonce à revendiquer et que le patron, tout en continuant à prélever son profit en volant de la plus-value, se donne cependant des airs de philanthrope ; que par dessus le marché, les orateurs du gouvernement parlent de temps à autre de la communauté nationale, de la collaboration et poussent même le luxe jusqu’à adresser quelques reproches aux patrons “égoïstes”, et le secret de l’harmonie sociale sera découvert selon la recette Pétain-Bergery.\par
Grossière supercherie en vérité ! Tant que subsiste l’exploitation de l’homme par l’homme, il y a des exploiteurs et des exploités qui s’opposent les uns aux autres, même lorsque les exploités ne peuvent ouvertement exprimer leurs protestations.\par
Il y a un moyen de réaliser l’harmonie sociale, et il n’y en a qu’un, c’est de s’attaquer à la cause qui l’engendre, mais en ne dissimulant pas cette cause, en ne confondant pas l’essentiel avec l’accessoire, en reconnaissant que pour remplacer les classes antagonistes par des catégories sociales amies, il faut supprimer l’exploitation capitaliste. Le socialisme supprime les classes antagonistes en supprimant l’exploitation de l’homme par l’homme. C’est cette suppression qu’ont réalisée les communistes constructeurs du socialisme en URSS.\par
À la place des classes antagonistes, l’URSS où la seule occupation qui permette de vivre est le travail productif, où la loi interdit et où il est pratiquement impossible de vivre de l’exploitation du travail d’autrui, ne connait que des catégories sociales amies, alliées, associées ; les ouvriers, les paysans, les intellectuels. Le pays du socialisme n’a pas camouflé la lutte de classes, il s’est attaqué à la cause des antagonismes de classe. Il a créé ce que lui seul pouvait créer, l’harmonie sociale.
\section[{Un état d’un type nouveau, qui réalise la démocratie véritable}]{Un état d’un type nouveau, qui réalise la démocratie véritable}
\noindent Les pays capitalistes sont allés de la fausse démocratie à la dictature fasciste.\par
Le pays du socialisme a suivi une autre voie. De la Russie écrasée par l’absolutisme tsariste, il a fait une vraie démocratie. Que le « national-socialisme » ait instauré en Allemagne un pouvoir dictatorial que veulent imiter les petits singes du “Rassemblement”, voilà qui loin de témoigner en faveur de sa “jeunesse” et de sa “nouveauté”, révèle au contraire son caractère rétrograde. Revenir à l’époque du Chef de guerre, du Condottiere, ce n’est pas aller de l’avant, c’est courir à toutes jambes vers le Moyen Âge, et l’on conviendra que c’est un singulier « ordre nouveau » que celui qui se présente avec les atours médiévaux. Mais le recours aux formes du pouvoir réactionnaire, l’abolition de toutes les formes de représentation populaires ont été de tous les temps l’expression de l’offensive des oligarchies parasitaires. Les formes démocratiques de gouvernement ne convenaient plus au maintien de la domination de l’oligarchie. En luttant contre la démocratie, les nazis servaient donc utilement la cause des oligarchies capitalistes. Au contraire, en défendant la démocratie contre le fascisme, les communistes défendaient en somme une forme d’état dont l’oligarchie ne pouvait plus se servir.\par
Le pays du socialisme, lui, a construit la vraie démocratie, « la démocratie des travailleurs », c’est-à-dire la démocratie « pour tous » (Staline). L’état, avant de décider de l’orientation de sa vie politique et culturelle ne se demande pas quelle est la volonté de Krupp ou de Kirdorff, lesquels ont cessé d’exister en URSS, mais quel est l’intérêt du peuple seul et véritable souverain. La constitution stalinienne consacre la victoire de l’ordre socialiste. Elle réalise la démocratie véritable. En proclamant l’universalité du droit à l’instruction, elle assure, grâce à la suppression de la dictature de l’argent, la garantie effective de ces droits.
\section[{Une communauté fraternelle de peuples libres}]{Une communauté fraternelle de peuples libres}
\noindent Parmi les conquêtes du socialisme, celle-là est sans doute la plus remarquable. Le grand Reich allemand n’est pas une communauté de peuples libres, mais plutôt une prison pour les peuples opprimés et exploités. L’Europe sous la führung allemande, l’Europe raciste, l’Europe des pogromes ne serait pas, elle non plus, une communauté de peuples libres. Pourquoi ? Parce que sous le régime de la dictature impérialiste, l’indépendance des petits états est une fiction. Il y a des nations libres et des nations opprimées, comme il y a dans chaque nation des exploiteurs et des exploités. Le socialisme qui a aboli, en Russie, l’exploitation de l’homme par l’homme, a donné en même temps aux peuples de l’ancienne Russie le droit de disposer d’eux-mêmes. Il a proclamé l’égalité de droits pour les nations, non pas seulement l’égalité juridique et formelle que contredisent les entraves économiques, mais l’égalité « dans tous les domaines de la vie économique, publique, culturelle, sociale et politique » selon l’expression même de la Constitution stalinienne qui, fidèle au plus haut idéal du socialisme, à cet idéal outrageusement galvaudé par les “novateurs” moyenâgeux du nazisme, ajoute :\par

\begin{quoteblock}
 \noindent « Toute restriction directe ou indirecte aux droits, ou inversement, l’établissement de privilèges directs ou indirects pour les citoyens selon la race et la nationalité à laquelle ils appartiennent, de même que toute propagande d’exclusivisme ou de haine et de dédain racial ou national sont punis par la loi. »
 \end{quoteblock}

\section[{Socialisme et nazisme}]{Socialisme et nazisme}
\noindent Le socialisme crée une société harmonieuse de catégories sociales amies, parce qu’il abolit l’exploitation d’un groupe d’hommes par un autre. Le nazisme met aux fers la classe ouvrière, dissout les syndicats, rétablit l’absolutisme patronal et proclame : « J’ai aboli la lutte de classes. »\par
Le socialisme, en supprimant les privilèges de la fortune, établit la véritable démocratie. Le nazisme instaure pour le compte de l’oligarchie capitaliste une dictature policière bureaucratique et militaire qui emplit les prisons et livre à la hache du bourreau les représentants du socialisme.\par
Le socialisme proclame l’égalité et la fraternité entre les peuples et les races. Le nazisme prêche la haine des races et professe la hiérarchie des peuples au nom d’une doctrine barbare contre laquelle s’insurgent la science et la raison.\par
Le socialisme c’est donc le contraire de tout ce qui se débite dans le bazar des imprécations sonores et sur le marché aux puces des promesses vides du nazisme.\par
Il n’y a au monde qu’une expérience socialiste : c’est celle que les communistes ont fait triompher en Union Soviétique ; c’est-à-dire qu’il n’y a pas plusieurs socialismes. Il n’y en a qu’un.
\section[{Non, le nazisme n’est pas le socialisme}]{Non, le nazisme n’est pas le socialisme}
\noindent La ronde tragique du chômage, de la misère, de la préparation à la guerre, de la guerre, voilà le nazisme.\par
Tu sais bien, travailleur de France, paysan de France, petit bourgeois de France, que le national-socialisme est l’esclavage qui meurtrit et qui tue.\par
Les agents de l’occupant savent bien que tu ne veux pas de cette machine de régression sociale. Ils savent bien que de toutes les forces de ta raison et de ton cœur, tu te détournes d’elle, que la défaite et l’oppression étrangère ont ajouté encore à tes motifs de haine légitime contre ce régime odieux.
\section[{La voie du socialisme}]{La voie du socialisme}
\noindent Mais, pour que tu te trompes de chemin, ils ont chargé quelques-uns de leurs hommes de se déguiser et de placer au carrefour un écriteau truqué, sur lequel ils ont écrit : « Ici le socialisme, ici la Révolution nationale ». Or, la direction qu’ils indiquent, c’est celle de la pire réaction capitaliste. Ils espèrent que lorsque tu te seras fourvoyé, tu n’oseras pas te ressaisir. Ne te laisse pas prendre à cette ruse grossière. La voie que te désignent les réacteurs de Vichy et ceux de Paris, les généraux déshonorés, les cagoulards, les rénégats, le Maréchal d’outre-tombe, l’amiral politicien et Pierre Laval à la cravate blanche et à la conscience noire, cette route-là conduit tout droit à la sombre caverne où l’on sacrifie au culte du veau d’or.\par
La voie du socialisme est toute droite. Elle n’aboutit pas à une impasse. Elle n’a pas été tracée par des aventuriers. Elle a été arrosée par le sang des martyrs qui au cours d’une longue histoire sont tombés dans l’effort tenace pour la construire. Quelques-uns de ceux qui l’ont ouverte vivent encore dans les prisons ou dans les camps, enchainés par les ennemis du socialisme, par les ennemis du peuple, par les ennemis de la France, par tes ennemis.
\section[{Ils ne représentent pas la France}]{Ils ne représentent pas la France}
\noindent Et c’est pourquoi les hommes de Vichy, ceux du “Rassemblement” et des autres groupements entretenus par l’occupant ne représentent pas le socialisme, ne représentent pas la France. Ils ne conçoivent le gouvernement de la France \emph{que dans l’étouffement des libertés françaises}. Leur formule est celle que flétrissait Lamennais : « \emph{Silence aux pauvres} ! »\par
Ils ne peuvent d’ailleurs pas concevoir autrement le gouvernement de la France. Imaginez-vous qu’un seul des personnages qui pontifient à l’Hôtel du Parc, un seul de ceux qui pérorent à Paris avec la permission de l’occupant, conserverait quelques heures son portefeuille ou sa faconde si l’on donnait la parole au peuple ? Il suffirait que soit entrebaillée la porte des libertés populaires pour qu’une bourrasque purificatrice balaie cette engeance maudite. Les dirigeants apprivoisés des syndicats, les Chevalme, les Dumoulin, les Ehlers savent bien à quoi s’en tenir sur ce sujet.\par
Et n’oubliez pas l’essentiel : les hommes de Vichy, ceux du “Rassemblement”, ceux du PSF n’ont été tirés du néant que par l’occupant. C’est de l’occupant qu’ils tiennent leur pouvoir. C’est l’occupant qui leur a donné l’investiture. Imaginez-vous qu’ils pourraient pratiquer la politique d’asservissement si le peuple avait la parole ? Imaginez-vous qu’un gouvernement français qui serait contrôlé par le peuple aurait pu prendre des décisions comme celles prises par monsieur Darlan et qui assujettissent notre pays tous les jours un peu plus ?\par
Quand un gouvernement supprime les libertés populaires, c’est toujours pour accomplir un mauvais coup. Ainsi procéda Daladier quand il voulut lancer le pays dans la guerre. Les gens de Vichy, eux, ont supprimé les assemblées représentatives, les conseils municipaux, le droit d’élire les municipalités dans les villes de plus de deux mille habitants. Ils ont supprimé avec les libertés de la presse toutes les autres libertés publiques. Ils ont instauré la déification grotesque d’un vieillard déficient que distrait une cour de bouffons. La camarilla qui règne sans contrôle sur le pays est recrutée parmi les militaires dont l’univers se gausse, les inspecteurs des finances, les chefs des trusts, les princes de l’église tombés en enfance et les resquilleurs de quelques maisons closes de l’ancien parlement. Du nouveau, ça ? Et de la Révolution ? Non point ! Le régime du Thermidor, le régime des ordonnances de Charles dix, le régime de Napoléon le petit, le régime de ceux que Victor Hugo appelait : « \emph{Les contemplateurs des gibets de l’Autriche} ».
\section[{Des mots, des phrases}]{Des mots, des phrases}
\noindent Et que te proposent, et que te recommandent les fameux révolutionnaires du “Rassemblement” ? L’économie dirigée ? Mais dans l’époque impérialiste, l’économie dirigée est toujours une bonne affaire pour les oligarchies. Elle est un moyen de renflouement : elle est un instrument de profit. Que te proposent-ils encore les hommes du “Rassemblement” ? Ce qu’ils appellent « la fin du prolétariat et du profitariat capitalistes ». Cette formule n’inquiète nullement les exploiteurs. Ils l’interprètent à juste titre à la lumière de l’expérience nazie, dont les gens du “Rassemblement” sont les imitateurs. Ils se rassurent en lisant le bilan des sociétés allemandes, les bénéfices accumulés par les munitionnaires allemands, les privilèges des hobereaux et des nobles sauvegardés, rétablis ou étendus. Ils applaudissent. Ceux que les exploiteurs applaudissent ne peuvent pas être tes amis, peuple de France.\par
Observe-les, en effet. Ils s’évertuent à te tromper sur les causes de tes souffrances. La grande affaire pour eux est de gaspiller l’énergie des Français en les dispersant vers des objectifs stériles (antisémitisme par exemple), qui n’ont rien à voir avec les raisons vraies de nos malheurs.
\section[{Un programme clair et précis}]{Un programme clair et précis}
\noindent Te défendre, c’est s’engager à extirper la racine du mal. C’est précisément l’engagement que ne prennent pas les escrocs du “Rassemblement” ankylosés par les chaînes qui les tiennent au capitalisme français, au capitalisme allemand. Leur raison d’être est de défendre des maîtres qui sont tes ennemis. La raison d’être d’un gouvernement du peuple est de te libérer de la dictature de ces maîtres en mettant les oligarchies hors d’état de nuire. Par quelles mesures ?\par

\begin{quoteblock}
 \noindent Nationalisation sans indemnité des banques, compagnies d’assurances, mines, chemins de fer et de l’ensemble des sociétés capitalistes aryennes et juives (électricité, produits chimiques, textiles, métallurgie, gaz, etc.). Révision de tous les marchés de guerre, par des commissions composées de délégués des ouvriers, des paysans, des petites gens et confiscation de tous les bénéfices de guerre avec affichage au lieu d’habitation des intéressés des sommes confisquées et des noms des personnes passibles de ces confiscations. Privation des droits civils et politiques pour tous les administrateurs et gros actionnaires des sociétés anonymes, sans distinction de religion ou de race. Recensement des fortunes et des biens des gros capitalistes aryens et juifs en vue de prélèvement et confiscation pour subvenir aux besoins immédiats des masses laborieuses.\par
 
\bibl{(programme du parti communiste).}
 \end{quoteblock}

\noindent Pour donner leur plein sens à ces mesures, le gouvernement du peuple, cela va sans dire, rétablirait le droit syndical dans sa plénitude et il appellerait la classe ouvrière organisée à la direction de l’économie nouvelle.\par
Voilà ce qui est parler clair. Pas de phrases à double sens. Un gouvernement du peuple n’a aucun intérêt impérialiste à ménager. Voilà un langage que ne tiennent pas les saltimbanques du RNP et du PPF !\par
Le “Rassemblement” et avec lui les autres groupements impopulaires célèbrent les vertus des paysans et des classes moyennes des villes. Ils leur promettent une réception triomphale dans la Communauté nationale, et un strapontin à la Réunion du Comité où l’on organisera le marché de la consommation. Mais autant ils sont généreux de phrases creuses, autant ils sont avares de propositions susceptibles d’aider efficacement ceux qu’ils veulent embrigader.
\section[{Mesures à prendre pour les paysans et petits commerçants}]{Mesures à prendre pour les paysans et petits commerçants}
\noindent En effet, le paysan, le petit commerçant, ont besoin de mesures qui concernent le ravitaillement.\par

\begin{quoteblock}
 \noindent Fixation de prix agricoles à la production pour assurer au paysan une juste rémunération de son travail et suppression des intermédiaires fauteurs de vie chère ; suppression des réquisitions ; organisation d’un service de ravitaillement assurant équitablement la répartition des produits et mettant fin aux passe-droits, au règne du bon plaisir, des pots de vin existant actuellement ; contrôle du ravitaillement par les petits et moyens commerçants, par les petits et moyens paysans et par les syndicats ouvriers\par
 
\bibl{(programme du parti communiste). }
 \end{quoteblock}

\noindent Ils ont besoin d’une autre série de mesures qui a trait au dédommagement que les classes moyennes atteintes par les conséquences de la guerre ont le droit d’attendre d’un gouvernement du bien public. Mais il y a d’autres remèdes encore à leurs grands maux dont ne veulent entendre parler les rebouteux du “Rassemblement” :\par

\begin{quoteblock}
 \noindent Confiscation des grandes propriétés foncières privées et répartition de la terre aux ouvriers agricoles, aux petits et moyens paysans, par des commissions composées de représentants des ouvriers agricoles, des petits et moyens paysans ; monopole d’état du commerce extérieur et du commerce de gros avec fixation obligatoire des prix de gros et de détail ; nationalisation sans indemnité des grands magasins et application dans ces grands magasins des prix de détail imposés à l’ensemble des commerçants détaillants.\par
 
\bibl{(programme du parti communiste).}
 \end{quoteblock}

\noindent Les imitateurs du nazisme se gardent bien de proposer de telles mesures et le gouvernement de Vichy de les prendre. Car le gouvernement qui prendrait ces mesures aurait construit les assises d’une véritable démocratie populaire. Il les aurait complétées en liquidant véritablement le passé et en aménageant l’avenir.\par
 Liquider le passé : c’est notamment ouvrir les portes des prisons et des camps où sont enfermés des patriotes français, c’est renoncer aux mesures policières dirigées contre des Français et des Françaises adversaires de l’oppression nationale, c’est déférer devant une Cour suprême, composée d’ouvriers, de paysans, de membres des classes moyennes et de défenseurs des Français persécutés, les responsables de la guerre, les responsables de la défaite, les responsables de l’asservissement.\par
Aménager l’avenir, c’est abolir les ignobles décrets racistes, proclamer l’égalité des droits de tous les citoyens français sans distinction de nationalité ou de race, la liberté de conscience, l’octroi de tous les droits politiques à partir de 18 ans, l’égalité de l’homme et de la femme dans toutes les branches de l’activité.
\section[{Le malheur d’être jeune}]{Le malheur d’être jeune}
\noindent Mais la Révolution nationale n’a point ses regards tournés vers l’avenir, mais vers les époques les plus sombres de l’histoire de l’humanité. C’est pourquoi aussi son attitude scandaleuse envers la jeunesse de France. De cette jeunesse, les fauteurs d’aventure ont voulu faire leur réservoir et leur arsenal. Ils ont eu l’outrecuidance révoltante de se présenter comme les champions de la jeunesse. Les jeunes Français savent à quoi s’en tenir. On les a dégradés, déchus, pervertis dans ces camps qui resteront l’une des hontes les plus ignominieuses du régime Pétain-Laval-Darlan. On voulait faire d’eux la grande armée abrutie qui salue le passage des dictateurs. Toutes les flatteries insidieuses, toutes les flagorneries dont on a accablé les jeunes Français au lendemain de l’armistice ont abouti à cette faillite déshonorante. La Révolution nationale se solde pour des milliers de jeunes par la misère et le chômage, la disqualification, l’ignorance. Ces produits sont les produits très naturels de l’impérialisme résidu de l’uniforme nazi. Car le régime que les nazis ont instauré en Allemagne et que les nazillons voudraient faire prévaloir en France est celui de l’empoisonnement et de l’emprisonnement de la jeunesse, le régime du surmenage à l’usine, de la sous-alimentation, du travail prétendument “volontaire” au service des hobereaux, le régime du camp   et de la caserne. Tout cela baptisé d’un nom trompeur : la communauté populaire.\par
Le socialisme, pour les jeunes, ce n’est pas le camp où l’on déifie un führer ou un maréchal, où ils sont livrés à l’oisiveté déprimante, à l’intoxication et à l’ignorance.\par
Le socialisme veut faire des jeunes des hommes complets et non des recrues des fauteurs de coups de mains ou des fauteurs de guerre.
\section[{Un programme pour les jeunes}]{Un programme pour les jeunes}
\noindent Un gouvernement qui aurait le souci des intérêts de la jeunesse proclamerait : le droit à l’instruction par la réorganisation de l’enseignement public, l’élimination des conceptions obscurantistes admises encore dans les programmes scolaires, l’institution de nombreuses bourses d’études, le droit d’accès de tous au plus haut degré de l’enseignement et la gratuité de l’enseignement à tous les échelons. Il proclamerait le droit à un métier pour chaque citoyen et il créerait un vaste réseau d’écoles d’apprentissage. Il accorderait les droits politiques à chaque citoyen et à chaque citoyenne à partir de 18 ans. Il proclamerait le droit au repos par la limitation de la journée de travail et l’institution de congés payés, le droit à la joie par le développement du sport, le droit au foyer par l’attribution d’une prime aux jeunes ménages, le relèvement des allocations familiales, l’institution d’un congé de trois mois avant, trois mois après l’accouchement.\par
Tel serait le programme d’un gouvernement du peuple. Tel sera le destin de la jeunesse de France quand, avec les patriotes, elle aura chassé les traîtres, les usurpateurs et les faussaires. Ceux-là la flattent pour la mieux abêtir ! Le socialisme lui prêche l’effort pour la mieux exalter. Ils refoulent les intelligences et veulent mécaniser les esprits. Le socialisme fera d’elle une jeunesse active et avide d’apprendre et de savoir. Ils lui proposent le retour aux coutumes barbares, à la haine de race, aux pogroms. Le socialisme développera sa raison, son sens critique. Ils veulent faire de la jeunesse le troupeau ignorant et exténué qui suit un condottiere ou pavoise pour un vieillard médiocre. Le socialisme veut qu’elle soit la cohorte enthousiaste dont l’intelligence et la passion maîtrisent les forces brutales de la nature. Ils l’appellent à rendre hommage à la vassalité. Le socialisme lui apprend qu’il n’y a rien de plus beau que l’indépendance nationale liée à la libération sociale et politique.\par
Il n’y a pas de vraie liberté dans la servitude nationale. Un gouvernement du peuple doit être le gouvernement de la libération sociale et de l’indépendance nationale. Or la France a, depuis sa défaite, son territoire occupé par l’armée allemande, et elle est placée depuis sous la dépendance de plus en plus sévère de l’Allemagne.\par
Quelle est, pour la France, la voie de salut ?
\chapterclose


\chapteropen
\chapter[{11 – Une France indépendante}]{11 – Une France indépendante}\renewcommand{\leftmark}{11 – Une France indépendante}


\chaptercont
\noindent Les hommes du “Rassemblement” et, après une série de contorsions, les gouvernants de Vichy, se sont donnés pour tâche d’aligner le pays sur les volontés du vainqueur. Installés par lui, sortis du néant par lui, ils ne peuvent prendre une autre attitude, et quand ils feignent d’hésiter, leurs oscillations ne sont que d’affreuses grimaces. \emph{Les agitateurs du “Rassemblement” oublient les griefs qu’ils adressent au conservatisme social des Vichyssois toutes les fois que Pétain et ses ministres prennent des décisions favorables à l’occupant.} Les uns et les autres veulent faire de la France l’alliée de l’impérialisme allemand.\par

\begin{quoteblock}
 \noindent « Les vainqueurs sont souvent assez rusés pour confier aux responsables d’une soumission pusillanime la surveillance des esclaves et ces êtres sans caractère exercent la plupart du temps cet office aux dépens de leur propre peuple avec une rigueur plus impitoyable que ne le permet n’importe quelle brute étrangère installée par l’ennemi lui-même dans le pays vaincu. »
 \end{quoteblock}

\noindent C’est Hitler qui a donné, dans Mein Kampf, cette définition anticipée de Pétain, de Laval et de Marcel Déat.\par
\section[{Trahison du socialisme et de notre peuple}]{Trahison du socialisme et de notre peuple}
\noindent \emph{C’est trahir les intérêts ouvriers, c’est trahir la Révolution, c’est trahir le socialisme, c’est trahir notre peuple que de vouloir faire de la France un protectorat de l’Allemagne nazie.} Car il est vain d’ergoter. La collaboration que conçoivent Vichyssois et hommes du “Rassemblement”, ce ne peut être que cela, et l’Allemagne n’en admet d’ailleurs point d’autre. Nous l’avons expliqué et nous avons expliqué pourquoi. Inutile de revenir sur cette démonstration que l’on pourrait, au jour le jour, illustrer et enrichir de témoignages irréfutables extraits de la presse allemande, des discours des hommes d’État allemands, de tous les documents qui expriment la pensée des maîtres du Reich, et plus encore des faits eux-mêmes.
\section[{L’Europe nazie ? Un pénitencier}]{L’Europe nazie ? Un pénitencier}
\noindent De joyeux farceurs nous affirment qu’il est vrai que cette France vassalisée ou, comme ils disent, « intégrée à l’Europe », aurait encore, dans cette Europe, un rôle et même un très grand rôle à jouer : qu’ils relisent dans \emph{Mein Kampf} le « Testament politique allemand » écrit par le Führer :\par

\begin{quoteblock}
 \noindent « Ne permettons jamais que se constituent en Europe deux puissances continentales. Dans toute tentative d’organiser aux frontières de l’Allemagne une deuxième puissance militaire, ne fût-ce que sous la forme d’un état susceptible d’acquérir une telle puissance, voyez une attaque contre l’Allemagne. Considérez que c’est votre droit et votre devoir d’empêcher par tous les moyens et au besoin par les armes la constitution d’un tel état. S’il existe déjà, abattez-le. »
 \end{quoteblock}

\noindent L’Allemagne, c’est-à-dire l’oligarchie capitaliste allemande, d’une part, les vassaux de l’autre. Voilà ce que serait l’Europe de la victoire nazie. Elle ressemblerait à un immense pénitencier, dont les travailleurs allemands seraient condamnés à être les geôliers. Des socialistes véritables se doivent de repousser avec horreur une telle solution. Des Français patriotes se doivent de refuser leur adhésion à cette vassalisation de la France.
\section[{Doctrine ridicule}]{Doctrine ridicule}
\noindent Que les nazillons français ne nous rétorquent pas que cette Europe vassalisée serait, au moins, une Europe unie, que les nations asservies auraient fait un pas vers l’unification, vers la fusion, fin dernière du socialisme : Eh bien, non ! L’unification sous la toise, l’unification des nations sous la domination d’un régime où les libertés ont été massacrées, où le progrès social a été proscrit, ce n’est pas un progrès, c’est une régression. Ce n’est pas une étape vers l’unification des peuples que souhaite le socialisme, c’est une victoire de l’impérialisme que le socialisme abhorre et répudie. Car le socialisme n’a pas prêché aux hommes de se plier à l’oppression impérialiste. Il leur a, au contraire, recommandé de la combattre. Il a proclamé le droit des peuples opprimés de secouer le joug de la nation dominante.\par
Dans une de ses démonstrations capitales, Staline a exposé que deux tendances s’étaient affirmées dans la question nationale, une tendance à la création d’états nationaux indépendants en réaction contre l’oppression impérialiste et une tendance au rapprochement économique entre ces groupes. Ces deux tendances, écrit Staline, sont pour nous deux phases d’un même processus. La fusion économique universelle n’est possible que sur les bases de la confiance mutuelle et en vertu d’un accord librement consenti.\par
Lénine avait écrit déjà :\par

\begin{quoteblock}
 \noindent La marche vers un but unique, l’égalité complète, le rapprochement plus étroit, la fusion de toutes les nations peut emprunter divers chemins. Si, prêchant la fusion des peuples, le socialiste d’un pays oppresseur oublie que Nicolas II, Guillaume II, Poincaré et autres sont aussi pour la fusion avec les petites nations au moyen de l’annexion, il ne sera en théorie qu’un doctrinaire ridicule, et en pratique qu’un agent de l’impérialisme.
 \end{quoteblock}

\noindent Un doctrinaire ridicule, un agent de l’impérialisme. Lénine avait prévu et défini par avance Marcel Déat.\par
La création d’une « union volontaire des peuples », comme dit Staline, doit être précédée de l’acquisition de leur indépendance, de « leur séparation avec le tout impérialiste unique ».
\section[{La véritable entente franco-allemande}]{La véritable entente franco-allemande}
\noindent Rien n’est plus vrai. L’entente franco-allemande serait en effet une chimère si elle était l’oppression d’une nation par une autre ; elle serait un instrument suspect si elle était la coalition de deux forces impérialistes et contre-révolutionnaires. Une entente de ce genre, même consignée sur des parchemins, ne vaudra jamais rien car elle ne recueillera jamais l’adhésion de notre peuple ; il n’y aura d’entente franco-allemande véritable que celle conclue dans l’indépendance des nations, dans le respect de leurs libertés, dans leur désir d’ouvrir à tous le contrat dans l’égalité. Rappelons que c’est le Parti Communiste qui s’est dressé en France contre le Traité de Versailles, contre l’attentat de la Ruhr et qui a répudié l’oppression d’un peuple par un autre. C’est dire combien les communistes sont les partisans sincères d’une entente entre les peuples basée sur le respect de leurs droits nationaux mais aussi combien ils sont les adversaires résolus de l’oppression nationale.\par
Il y a une très belle page de Jaurès, que les nazillons ont eu l’impudence outrageante de citer pour recommander leur camelote, alors qu’elle est la condamnation sans appel de ces doctrinaires ridicules et de ces agents de l’Impérialisme dénoncés par Lénine. Que dit Jaurès ?\par

\begin{quoteblock}
 \noindent « De même qu’on ne réconcilie pas des individus en faisant simplement appel à la fraternité humaine, mais en les associant, s’il est possible, à une œuvre commune et noble où, en s’oubliant eux-mêmes, ils oublient leur inimitié, de même les nations n’abjureront les vieilles jalousies, les vieilles querelles les vieilles prétentions dominatrices, tout ce passé éclatant et triste d’orgueil et de haine, de gloire et de sang, que lorsqu’elles se seront proposé toutes ensemble un objet supérieur à elles, que quand elles auront compris la mission que leur assigne l’histoire…\par
 « Dans l’ivresse et dans la joie de cette grande œuvre accomplie ou même préparée, quand il n’y aura plus de domination politique et économique de l’homme sur l’homme, quand il ne sera plus besoin de gouvernements armés pour maintenir les monopoles des classes accapareuses, quand la diversité des drapeaux égaiera, sans la briser, l’unité des hommes, qui donc alors, je vous le demande, aura intérêt à empêcher un groupe d’hommes de vivre d’une vie plus étroite, plus familière, plus intime, c’est-à-dire d’une vie nationale, avec le groupe historique auquel le rattachent de séculaires amitiés. »
 \end{quoteblock}

\noindent Ce que condamne Jaurès : les vieilles prétentions dominatrices, la tradition de haine et de sang, c’est ce que les communistes (les véritables socialistes) répudient.\par
Le socialisme victorieux en URSS a prouvé par les faits la vérité proclamée par Jaurès, qu’il y avait à l’organisation des hommes, à leur rassemblement, à leur fusion, une autre solution que la solution impérialiste. Le socialisme n’a pas seulement proclamé l’égalité politique et juridique des nations. Il a mis fin aux inégalités de faits résultant de l’organisation capitaliste.\par

\begin{quoteblock}
 \noindent « Par suite d’une certaine inégalité léguée par l’histoire, dit Staline, nous avons hérité du passé d’une nationalité, la Grande-Russie, qui s’est trouvée être plus développée au point de vue politique et industriel que les autres nationalités. De là l’inégalité de fait qui ne saurait être liquidée en une seule année \emph{mais qui doit être liquidée en prêtant une aide économique, politique et culturelle aux nationalités arriérées.} »
 \end{quoteblock}

\noindent Voilà des phrases qu’il faut rapprocher de celles rappelées plus haut, du Dr Funk et de M. Rosenberg ! Ou encore de celles prononcées à Linz par Hitler en mars 1941 contre « les peuples envieux qui jalousent l’Allemagne ». On comprendra mieux alors le sens de notre affirmation : le nazisme, c’est le contraire du socialisme.
\section[{Une solution socialiste et nationale}]{Une solution socialiste et nationale}
\noindent Le socialisme n’a pas dit aux nations de la vieille Russie : « vous vous organiserez sous la domination grand-russe. » Il leur a dit : « vous aurez les mêmes droits que la nation grand-russe. Les lois de l’Union seront faites par le Soviet suprême composé de deux chambres : le Soviet de l’Union et le Soviet des nationalités (élu sur la base nationale) et ces deux chambres auront les mêmes droits. » Plus de chauvinisme, la fraternité entre les nations. Le socialisme n’a pas dit aux nations : « défense à vous de développer une industrie ! Attachez vos fils à la glèbe. L’industrie sera le privilège exclusif de la nation grand-russe, les autres seront ses clients ou ses réservoirs d’hommes. Il y aura une hiérarchie, et c’est moi qui commanderai : un Ukrainien, un Tadjik n’auront pas le droit d’être appelés aux responsabilités de l’état. » Il leur a dit au contraire : « Rattrapez le plus vite possible l’avance que les Grand-russes ont sur vous : couvrez d’usines, de stations électriques, de réseaux de chemin de fer, d’écoles les terres où vivaient malheureuses des populations nomades. Nous vous aiderons de toutes nos forces et de toutes les manières, a dit la Grande-Russie aux autres nationalités, afin que vous puissiez atteindre le plus rapidement possible le même niveau culturel et économique que nous, et nous sommes prêts à consentir des sacrifices pour cela. Quiconque aura mérité la confiance du peuple pourra accéder aux plus hautes responsabilités »\footnote{(Note de l’auteur) On n’a qu’à se rappeler l’origine des dirigeants de l’Union soviétique pour constater que ces affirmations ont été traduites en actes !}. Et la production industrielle de la Géorgie a été, en 1935, dix-huit fois et demie plus grande qu’en 1913, celle de l’Ukraine, dix-sept fois et demie plus grande en 1936 qu’en 1923. Sur des immenses étendues, de nouvelles villes sont nées, les anciennes ont augmenté leur population. Le socialisme n’a pas dit aux nations : « vous collaborerez avec les Grands-Russes en vous soumettant à eux, en travaillant pour eux, en adoptant leur culture, leur littérature, leur musique, leurs films. » Il leur a dit, au contraire : « Vous travaillerez pour augmenter votre bien-être ; vos impôts ne serviront plus à enrichir les seigneurs grand-russes, ils seront utilisés pour votre développement économique et culturel. Vous aurez vos écoles, vos universités, votre littérature ! » Les grands colonisateurs d’occident, les amateurs d’ordre nouveau sous la führung nazie forgent des chaînes. Le socialisme démolit les prisons des peuples.\par
Il n’est donc pas vrai que les hommes soient condamnés soit à vivre derrière des frontières hérissées, soit à se rassembler sous le joug du dominateur ou d’un directeur, selon l’euphémisme dont usent les employés de Göbbels. L’expérience soviétique atteste la possibilité et la supériorité d’une autre solution, d’une solution non impérialiste, d’une solution socialiste et nationale.
\section[{Rapprochement avec l’URSS}]{Rapprochement avec l’URSS}
\noindent Une politique extérieure indépendante et française, voilà la politique qui serait celle d’un gouvernement du peuple. Un gouvernement du peuple consacrerait son effort à la libération du territoire national et des prisonniers de guerre. Mais une tâche pareille, on ne peut l’accomplir lorsqu’on est devenu la risée du monde comme l’est devenue la France de Pétain, lorsqu’on affaiblit le pays, lorsqu’on élargit ses blessures, lorsqu’on prête son aide à l’ennemi pour mieux vous étrangler, lorsqu’en se faisant le fournisseur des bottes de la contre-révolution, on a soulevé contre soi le mépris de tous les peuples.\par
Est-ce que le gouvernement de Vichy, s’il était un gouvernement voulu par le peuple, pourrait pratiquer sa politique d’isolement de la France, d’hostilité à peine cachée envers l’Union Soviétique ? Certainement pas. Un gouvernement du peuple pratiquerait une politique d’intime rapprochement avec l’Union Soviétique. Avec elle, il négocierait un pacte d’amitié et un traité de commerce. La France doit être l’amie de l’URSS parce que la politique de paix ferme et constante du pays des Soviets correspond au désir de paix du peuple français ; parce que la politique de respect de l’indépendance des nations, poursuivie par le pays du socialisme, correspond aux profondes aspirations du peuple français ; parce que le régime de liberté dont jouissent les peuples du pays des Soviets correspond à la volonté de liberté du peuple français ; enfin parce que si un large courant d’échanges était assuré entre la France et le grand pays des Soviets, le redressement de l’économie nationale pourrait être rapidement effectué et de cruelles souffrances et de terribles privations pourraient être épargnées au peuple de France.
\section[{Ils se soucient peu de l’avenir de notre pays}]{Ils se soucient peu de l’avenir de notre pays}
\noindent Et nous vous posons cette question : croyez-vous, ouvriers de France, paysans de France, petits bourgeois de France, les charlatans du “Rassemblement” et ceux de Vichy soucieux de l’avenir de notre pays ? Non, répondez-vous, non et non. Ce sont les rançonneurs, les exploiteurs de prolétaires et les exploiteurs du peuple qui approuvent le programme de “Rassemblement” de Déat et applaudissent au programme de la cour de Vichy. Ils les approuvent, ces programmes, parce qu’ils les savent conformes à leur volonté.\par
Et nous vous disons encore, nous communistes qui depuis vingt ans nous sommes consacrés corps et âme à votre défense, les Déat, Pétain, Doriot, n’ont qu’un souci : vous empêcher d’avoir voix au chapitre. Ils estiment que le seul gouvernement digne de la France sera celui que consacrent les baïonnettes étrangères, mais qui étouffe la voix de la France. Et nous répondons : c’est en baillonnant le peuple qu’on conduit le pays à la catastrophe. Pour assurer le salut de la France, il faut rendre la parole au peuple. Le seul gouvernement digne de la France sera celui qui imposera la volonté de son peuple.\par
Les agents de l’ennemi veulent t’enchaîner au char du conquérant nazi ; ils veulent, peuple de France, que tu sois l’esclave et le soldat de l’Allemagne national-socialiste. Ces lascars qui se prétendent tes amis n’admettent pas d’autre destin pour toi que celui de subordonné du nazi. Les communistes refusent de t’engager dans ce cul-de-sac. Ils ont dit que ton salut ne viendra pas de la victoire de Hitler mais de la libération de la France. Au lieu de faire des vœux pour le triomphe de la croix gammée comme font les “révolutionnaires” des villes thermales, ou de la Gestapo, les communistes luttent pour libérer le sol de la patrie de l’occupation naziste et pour l’amélioration du sort des masses populaires.
\section[{Internationalistes et patriotes}]{Internationalistes et patriotes}
\noindent Les singuliers “Européens” de Vichy et du “Rassemblement” répudient l’internationalisme et ils sacrifient sur les autels pollués du racisme. Les communistes, eux, savent que leur cause est celle de tous les peuples écrasés par la botte allemande. Ils se sentent liés par les fibres les plus profondes aux patriotes qui, dans tous les temps, livrent le même combat contre le même ennemi, avec ceux qui peuplent les prisons, éparpillées à travers l’Europe, de Himmler et des amis de Himmler ; qui, devant les juges dociles aux ordres de leurs assassins, clament leur foi dans l’avenir de leur pays. Les communistes sont des internationalistes et des patriotes. Nous savons que la France sera plus belle, plus libre, plus respectée lorsque les peuples d’Europe auront recouvré leur indépendance et se seront associés librement à la noble tâche de s’entr’aider.\par
Les Vichyssois et les gens du “Rassemblement” ont montré comment on falsifait le socialisme, comment on aggravait l’ordre ancien en le baptisant ordre nouveau. Les communistes ont montré comment on tenait ses promesses, comment on réalisait le vrai socialisme. Nos ennemis ne peuvent vivre que par la mystification, la propagande de l’usine à mensonges, le bourrage de crânes. Ils se réclament d’une escroquerie ; nous nous réclamons d’une victoire du socialisme.
\section[{Politique du “Rassemblement” et politique communiste}]{Politique du “Rassemblement” et politique communiste}
\noindent Le « Rassemblement Déat » n’est pas pressé de faire connaître les noms de ses dirigeants. Ce n’est pas par modestie. C’est par précaution. Ces gaillards connaissent la poigne robuste du peuple de France qui, s’exerçant sur leur échine ou sur leur trogne, ferait œuvre particulièrement salutaire. Au surplus, leurs noms sont des repoussoirs et leurs biographies donnent la nausée. Pour expliquer leur présence et leur turbulence, pour faire excuser leur passé, il leur faut se livrer aux acrobaties les plus comiques et recourir aux considérations les plus tarabiscotées. Les communistes ignorent les épreuves de cette gymnastique. Ils n’ont rien à cacher, de leur vie publique et de leur vie privée ; aucun scandale ne les a éclaboussés. Mieux on les connaît, mieux on connaît leur histoire et plus sûrement on leur fait confiance.\par
À l’époque où Marcel Déat faisait contrition dans le cabinet d’un juge d’instruction et où les amis de Marcel Déat chantaient les louanges de Daladier et de Paul Reynaud, les communistes prêchaient la lutte contre l’impérialisme. Ils affrontaient la prison, ils bravaient la mort pour vous crier la vérité. Plus tard, à l’heure de la défaite, alors que les autres se volatilisaient, les communistes, fidèles à leur poste de combat, disaient au peuple de France qu’une nouvelle lutte commençait pour sa libération, pour son indépendance. Ils avaient clamé la protestation du peuple de France contre les fauteurs de guerre. Ils clamèrent leur protestation contre les fauteurs de servitude. Le parti communiste a été le parti de la clairvoyance et du courage. Dis-moi qui te favorise, je te dirai qui tu es. Les hommes de Vichy et du “Rassemblement” ont quémandé les faveurs de Daladier d’abord, les faveurs de la Gestapo ensuite. Dis-moi qui te persécute et je te dirai qui tu es. Les communistes ont été persécutés par Daladier d’abord, par la Gestapo ensuite et deux cent mille des leurs sont actuellement emprisonnés.\par
{\itshape Les Vichyssois et les hommes du “Rassemblement” ont été les complices des ennemis du socialisme. Les communistes ont été et sont traqués par les ennemis du socialisme.}\par
Les autres exaltent le principe du chef. Mais comment se recrutent les chefs chez eux, par qui sont-ils désignés, quels sont leurs attributs ? Ce sont d’autres chefs qui les désignent et c’est parmi les courtisans du dictateur qu’on les recrute. Après quoi on les regroupe dans un sérail, dans \emph{une maison de formation de Führers} (ainsi qu’on les appelle en Allemagne) où ils apprennent à commander et à être obéis. Ainsi se crée une caste, une dynastie de Führers. Ces Führers et le Führer suprême disposent de la vie et de l’avenir des masses populaires. Nos chefs, chez nous, sont choisis autrement. Ils ne sont point désignés par un état-major. Ils sont choisis par leurs compagnons de lutte. Et choisis d’après quel indice ? D’après leur dévouement à la cause du peuple, leur liaison étroite avec la masse, leur capacité de s’orienter eux-mêmes et de prendre leurs responsabilités, leur sens de la discipline. Le chef communiste, à la différence des “Führers”, ne jouit d’aucun privilège, ou plus exactement son seul privilège est de se voir confier les tâches les plus difficiles, les plus périlleuses. Les chefs communistes ne s’arrogent point le droit de décider du sort du peuple, mais proclament que c’est au peuple seul à décider de son sort. Le parti communiste ne dit point au peuple de France de se fier au génie ou à la puissance de quelques hommes. Il l’appelle à façonner son propre destin. Et c’est pourquoi il demande pour la France un gouvernement du peuple. Il sait qu’un gouvernement ayant l’adhésion du peuple serait un gouvernement qui assurerait le salut de la France. Notre peuple n’est ni abâtardi, ni résigné. Il déteste ses gouvernants d’aujourd’hui et il a raison. Il méprise ceux qui « battent leur coulpe sur sa poitrine » et il a raison. Il a voué une haine implacable à ceux qui se sont mis au service de ses oppresseurs et il a raison. Mais que soient balayés les aventuriers, que la France se donne un gouvernement propre et digne d’elle et les Français dont les ancêtres montaient jadis à l’assaut du ciel retrouveront leur force, leur vigueur : une force, une vigueur qu’ils mettront au service d’une euvre qu’eux seuls peuvent accomplir : le Relèvement de la France.\par
Voilà ce que te disent les communistes, peuple de France. C’est le message des hommes qui, loin des sentiers battus et des chemins qui s’égarent, te convient sur la route solide, à la marche audacieuse vers des horizons clairs !\par

\dateline{Fin avril 1941, Paris.}
\chapterclose

 


% at least one empty page at end (for booklet couv)
\ifbooklet
  \pagestyle{empty}
  \clearpage
  % 2 empty pages maybe needed for 4e cover
  \ifnum\modulo{\value{page}}{4}=0 \hbox{}\newpage\hbox{}\newpage\fi
  \ifnum\modulo{\value{page}}{4}=1 \hbox{}\newpage\hbox{}\newpage\fi


  \hbox{}\newpage
  \ifodd\value{page}\hbox{}\newpage\fi
  {\centering\color{rubric}\bfseries\noindent\large
    Hurlus ? Qu’est-ce.\par
    \bigskip
  }
  \noindent Des bouquinistes électroniques, pour du texte libre à participation libre,
  téléchargeable gratuitement sur \href{https://hurlus.fr}{\dotuline{hurlus.fr}}.\par
  \bigskip
  \noindent Cette brochure a été produite par des éditeurs bénévoles.
  Elle n’est pas faîte pour être possédée, mais pour être lue, et puis donnée.
  Que circule le texte !
  En page de garde, on peut ajouter une date, un lieu, un nom ; pour suivre le voyage des idées.
  \par

  Ce texte a été choisi parce qu’une personne l’a aimé,
  ou haï, elle a en tous cas pensé qu’il partipait à la formation de notre présent ;
  sans le souci de plaire, vendre, ou militer pour une cause.
  \par

  L’édition électronique est soigneuse, tant sur la technique
  que sur l’établissement du texte ; mais sans aucune prétention scolaire, au contraire.
  Le but est de s’adresser à tous, sans distinction de science ou de diplôme.
  Au plus direct ! (possible)
  \par

  Cet exemplaire en papier a été tiré sur une imprimante personnelle
   ou une photocopieuse. Tout le monde peut le faire.
  Il suffit de
  télécharger un fichier sur \href{https://hurlus.fr}{\dotuline{hurlus.fr}},
  d’imprimer, et agrafer ; puis de lire et donner.\par

  \bigskip

  \noindent PS : Les hurlus furent aussi des rebelles protestants qui cassaient les statues dans les églises catholiques. En 1566 démarra la révolte des gueux dans le pays de Lille. L’insurrection enflamma la région jusqu’à Anvers où les gueux de mer bloquèrent les bateaux espagnols.
  Ce fut une rare guerre de libération dont naquit un pays toujours libre : les Pays-Bas.
  En plat pays francophone, par contre, restèrent des bandes de huguenots, les hurlus, progressivement réprimés par la très catholique Espagne.
  Cette mémoire d’une défaite est éteinte, rallumons-la. Sortons les livres du culte universitaire, cherchons les idoles de l’époque, pour les briser.
\fi

\ifdev % autotext in dev mode
\fontname\font — \textsc{Les règles du jeu}\par
(\hyperref[utopie]{\underline{Lien}})\par
\noindent \initialiv{A}{lors là}\blindtext\par
\noindent \initialiv{À}{ la bonheur des dames}\blindtext\par
\noindent \initialiv{É}{tonnez-le}\blindtext\par
\noindent \initialiv{Q}{ualitativement}\blindtext\par
\noindent \initialiv{V}{aloriser}\blindtext\par
\Blindtext
\phantomsection
\label{utopie}
\Blinddocument
\fi
\end{document}
