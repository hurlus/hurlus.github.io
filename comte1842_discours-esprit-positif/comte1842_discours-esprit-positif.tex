%%%%%%%%%%%%%%%%%%%%%%%%%%%%%%%%%
% LaTeX model https://hurlus.fr %
%%%%%%%%%%%%%%%%%%%%%%%%%%%%%%%%%

% Needed before document class
\RequirePackage{pdftexcmds} % needed for tests expressions
\RequirePackage{fix-cm} % correct units

% Define mode
\def\mode{a4}

\newif\ifaiv % a4
\newif\ifav % a5
\newif\ifbooklet % booklet
\newif\ifcover % cover for booklet

\ifnum \strcmp{\mode}{cover}=0
  \covertrue
\else\ifnum \strcmp{\mode}{booklet}=0
  \booklettrue
\else\ifnum \strcmp{\mode}{a5}=0
  \avtrue
\else
  \aivtrue
\fi\fi\fi

\ifbooklet % do not enclose with {}
  \documentclass[french,twoside]{book} % ,notitlepage
  \usepackage[%
    papersize={105mm, 297mm},
    inner=12mm,
    outer=12mm,
    top=20mm,
    bottom=15mm,
    marginparsep=0pt,
  ]{geometry}
  \usepackage[fontsize=9.5pt]{scrextend} % for Roboto
\else\ifav
  \documentclass[french,twoside]{book} % ,notitlepage
  \usepackage[%
    a5paper,
    inner=25mm,
    outer=15mm,
    top=15mm,
    bottom=15mm,
    marginparsep=0pt,
  ]{geometry}
  \usepackage[fontsize=12pt]{scrextend}
\else% A4 2 cols
  \documentclass[twocolumn]{report}
  \usepackage[%
    a4paper,
    inner=15mm,
    outer=10mm,
    top=25mm,
    bottom=18mm,
    marginparsep=0pt,
  ]{geometry}
  \setlength{\columnsep}{20mm}
  \usepackage[fontsize=9.5pt]{scrextend}
\fi\fi

%%%%%%%%%%%%%%
% Alignments %
%%%%%%%%%%%%%%
% before teinte macros

\setlength{\arrayrulewidth}{0.2pt}
\setlength{\columnseprule}{\arrayrulewidth} % twocol
\setlength{\parskip}{0pt} % classical para with no margin
\setlength{\parindent}{1.5em}

%%%%%%%%%%
% Colors %
%%%%%%%%%%
% before Teinte macros

\usepackage[dvipsnames]{xcolor}
\definecolor{rubric}{HTML}{800000} % the tonic 0c71c3
\def\columnseprulecolor{\color{rubric}}
\colorlet{borderline}{rubric!30!} % definecolor need exact code
\definecolor{shadecolor}{gray}{0.95}
\definecolor{bghi}{gray}{0.5}

%%%%%%%%%%%%%%%%%
% Teinte macros %
%%%%%%%%%%%%%%%%%
%%%%%%%%%%%%%%%%%%%%%%%%%%%%%%%%%%%%%%%%%%%%%%%%%%%
% <TEI> generic (LaTeX names generated by Teinte) %
%%%%%%%%%%%%%%%%%%%%%%%%%%%%%%%%%%%%%%%%%%%%%%%%%%%
% This template is inserted in a specific design
% It is XeLaTeX and otf fonts

\makeatletter % <@@@


\usepackage{blindtext} % generate text for testing
\usepackage[strict]{changepage} % for modulo 4
\usepackage{contour} % rounding words
\usepackage[nodayofweek]{datetime}
% \usepackage{DejaVuSans} % seems buggy for sffont font for symbols
\usepackage{enumitem} % <list>
\usepackage{etoolbox} % patch commands
\usepackage{fancyvrb}
\usepackage{fancyhdr}
\usepackage{float}
\usepackage{fontspec} % XeLaTeX mandatory for fonts
\usepackage{footnote} % used to capture notes in minipage (ex: quote)
\usepackage{framed} % bordering correct with footnote hack
\usepackage{graphicx}
\usepackage{lettrine} % drop caps
\usepackage{lipsum} % generate text for testing
\usepackage[framemethod=tikz,]{mdframed} % maybe used for frame with footnotes inside
\usepackage{pdftexcmds} % needed for tests expressions
\usepackage{polyglossia} % non-break space french punct, bug Warning: "Failed to patch part"
\usepackage[%
  indentfirst=false,
  vskip=1em,
  noorphanfirst=true,
  noorphanafter=true,
  leftmargin=\parindent,
  rightmargin=0pt,
]{quoting}
\usepackage{ragged2e}
\usepackage{setspace} % \setstretch for <quote>
\usepackage{tabularx} % <table>
\usepackage[explicit]{titlesec} % wear titles, !NO implicit
\usepackage{tikz} % ornaments
\usepackage{tocloft} % styling tocs
\usepackage[fit]{truncate} % used im runing titles
\usepackage{unicode-math}
\usepackage[normalem]{ulem} % breakable \uline, normalem is absolutely necessary to keep \emph
\usepackage{verse} % <l>
\usepackage{xcolor} % named colors
\usepackage{xparse} % @ifundefined
\XeTeXdefaultencoding "iso-8859-1" % bad encoding of xstring
\usepackage{xstring} % string tests
\XeTeXdefaultencoding "utf-8"
\PassOptionsToPackage{hyphens}{url} % before hyperref, which load url package

% TOTEST
% \usepackage{hypcap} % links in caption ?
% \usepackage{marginnote}
% TESTED
% \usepackage{background} % doesn’t work with xetek
% \usepackage{bookmark} % prefers the hyperref hack \phantomsection
% \usepackage[color, leftbars]{changebar} % 2 cols doc, impossible to keep bar left
% \usepackage[utf8x]{inputenc} % inputenc package ignored with utf8 based engines
% \usepackage[sfdefault,medium]{inter} % no small caps
% \usepackage{firamath} % choose firasans instead, firamath unavailable in Ubuntu 21-04
% \usepackage{flushend} % bad for last notes, supposed flush end of columns
% \usepackage[stable]{footmisc} % BAD for complex notes https://texfaq.org/FAQ-ftnsect
% \usepackage{helvet} % not for XeLaTeX
% \usepackage{multicol} % not compatible with too much packages (longtable, framed, memoir…)
% \usepackage[default,oldstyle,scale=0.95]{opensans} % no small caps
% \usepackage{sectsty} % \chapterfont OBSOLETE
% \usepackage{soul} % \ul for underline, OBSOLETE with XeTeX
% \usepackage[breakable]{tcolorbox} % text styling gone, footnote hack not kept with breakable


% Metadata inserted by a program, from the TEI source, for title page and runing heads
\title{\textbf{ Discours sur l’esprit positif }}
\date{1842}
\author{Auguste Comte}
\def\elbibl{Auguste Comte. 1842. \emph{Discours sur l’esprit positif}}
\def\elsource{Auguste Comte, \emph{{\itshape Discours sur l’esprit positif}}, Paris, Carilian-Goeury et V. Dalmont, 1844. \href{http://gallica.bnf.fr/ark:/12148/bpt6k61282910.r=Discours+sur+l\%27esprit+positif.langFR}{\dotuline{Source Gallica}}\footnote{\href{http://gallica.bnf.fr/ark:/12148/bpt6k61282910.r=Discours+sur+l\%27esprit+positif.langFR}{\url{http://gallica.bnf.fr/ark:/12148/bpt6k61282910.r=Discours+sur+l\%27esprit+positif.langFR}}}.}

% Default metas
\newcommand{\colorprovide}[2]{\@ifundefinedcolor{#1}{\colorlet{#1}{#2}}{}}
\colorprovide{rubric}{red}
\colorprovide{silver}{lightgray}
\@ifundefined{syms}{\newfontfamily\syms{DejaVu Sans}}{}
\newif\ifdev
\@ifundefined{elbibl}{% No meta defined, maybe dev mode
  \newcommand{\elbibl}{Titre court ?}
  \newcommand{\elbook}{Titre du livre source ?}
  \newcommand{\elabstract}{Résumé\par}
  \newcommand{\elurl}{http://oeuvres.github.io/elbook/2}
  \author{Éric Lœchien}
  \title{Un titre de test assez long pour vérifier le comportement d’une maquette}
  \date{1566}
  \devtrue
}{}
\let\eltitle\@title
\let\elauthor\@author
\let\eldate\@date


\defaultfontfeatures{
  % Mapping=tex-text, % no effect seen
  Scale=MatchLowercase,
  Ligatures={TeX,Common},
}


% generic typo commands
\newcommand{\astermono}{\medskip\centerline{\color{rubric}\large\selectfont{\syms ✻}}\medskip\par}%
\newcommand{\astertri}{\medskip\par\centerline{\color{rubric}\large\selectfont{\syms ✻\,✻\,✻}}\medskip\par}%
\newcommand{\asterism}{\bigskip\par\noindent\parbox{\linewidth}{\centering\color{rubric}\large{\syms ✻}\\{\syms ✻}\hskip 0.75em{\syms ✻}}\bigskip\par}%

% lists
\newlength{\listmod}
\setlength{\listmod}{\parindent}
\setlist{
  itemindent=!,
  listparindent=\listmod,
  labelsep=0.2\listmod,
  parsep=0pt,
  % topsep=0.2em, % default topsep is best
}
\setlist[itemize]{
  label=—,
  leftmargin=0pt,
  labelindent=1.2em,
  labelwidth=0pt,
}
\setlist[enumerate]{
  label={\bf\color{rubric}\arabic*.},
  labelindent=0.8\listmod,
  leftmargin=\listmod,
  labelwidth=0pt,
}
\newlist{listalpha}{enumerate}{1}
\setlist[listalpha]{
  label={\bf\color{rubric}\alph*.},
  leftmargin=0pt,
  labelindent=0.8\listmod,
  labelwidth=0pt,
}
\newcommand{\listhead}[1]{\hspace{-1\listmod}\emph{#1}}

\renewcommand{\hrulefill}{%
  \leavevmode\leaders\hrule height 0.2pt\hfill\kern\z@}

% General typo
\DeclareTextFontCommand{\textlarge}{\large}
\DeclareTextFontCommand{\textsmall}{\small}

% commands, inlines
\newcommand{\anchor}[1]{\Hy@raisedlink{\hypertarget{#1}{}}} % link to top of an anchor (not baseline)
\newcommand\abbr[1]{#1}
\newcommand{\autour}[1]{\tikz[baseline=(X.base)]\node [draw=rubric,thin,rectangle,inner sep=1.5pt, rounded corners=3pt] (X) {\color{rubric}#1};}
\newcommand\corr[1]{#1}
\newcommand{\ed}[1]{ {\color{silver}\sffamily\footnotesize (#1)} } % <milestone ed="1688"/>
\newcommand\expan[1]{#1}
\newcommand\foreign[1]{\emph{#1}}
\newcommand\gap[1]{#1}
\renewcommand{\LettrineFontHook}{\color{rubric}}
\newcommand{\initial}[2]{\lettrine[lines=2, loversize=0.3, lhang=0.3]{#1}{#2}}
\newcommand{\initialiv}[2]{%
  \let\oldLFH\LettrineFontHook
  % \renewcommand{\LettrineFontHook}{\color{rubric}\ttfamily}
  \IfSubStr{QJ’}{#1}{
    \lettrine[lines=4, lhang=0.2, loversize=-0.1, lraise=0.2]{\smash{#1}}{#2}
  }{\IfSubStr{É}{#1}{
    \lettrine[lines=4, lhang=0.2, loversize=-0, lraise=0]{\smash{#1}}{#2}
  }{\IfSubStr{ÀÂ}{#1}{
    \lettrine[lines=4, lhang=0.2, loversize=-0, lraise=0, slope=0.6em]{\smash{#1}}{#2}
  }{\IfSubStr{A}{#1}{
    \lettrine[lines=4, lhang=0.2, loversize=0.2, slope=0.6em]{\smash{#1}}{#2}
  }{\IfSubStr{V}{#1}{
    \lettrine[lines=4, lhang=0.2, loversize=0.2, slope=-0.5em]{\smash{#1}}{#2}
  }{
    \lettrine[lines=4, lhang=0.2, loversize=0.2]{\smash{#1}}{#2}
  }}}}}
  \let\LettrineFontHook\oldLFH
}
\newcommand{\labelchar}[1]{\textbf{\color{rubric} #1}}
\newcommand{\milestone}[1]{\autour{\footnotesize\color{rubric} #1}} % <milestone n="4"/>
\newcommand\name[1]{#1}
\newcommand\orig[1]{#1}
\newcommand\orgName[1]{#1}
\newcommand\persName[1]{#1}
\newcommand\placeName[1]{#1}
\newcommand{\pn}[1]{\IfSubStr{-—–¶}{#1}% <p n="3"/>
  {\noindent{\bfseries\color{rubric}   ¶  }}
  {{\footnotesize\autour{ #1}  }}}
\newcommand\reg{}
% \newcommand\ref{} % already defined
\newcommand\sic[1]{#1}
\newcommand\surname[1]{\textsc{#1}}
\newcommand\term[1]{\textbf{#1}}

\def\mednobreak{\ifdim\lastskip<\medskipamount
  \removelastskip\nopagebreak\medskip\fi}
\def\bignobreak{\ifdim\lastskip<\bigskipamount
  \removelastskip\nopagebreak\bigskip\fi}

% commands, blocks
\newcommand{\byline}[1]{\bigskip{\RaggedLeft{#1}\par}\bigskip}
\newcommand{\bibl}[1]{{\RaggedLeft{#1}\par\bigskip}}
\newcommand{\biblitem}[1]{{\noindent\hangindent=\parindent   #1\par}}
\newcommand{\dateline}[1]{\medskip{\RaggedLeft{#1}\par}\bigskip}
\newcommand{\labelblock}[1]{\medbreak{\noindent\color{rubric}\bfseries #1}\par\mednobreak}
\newcommand{\salute}[1]{\bigbreak{#1}\par\medbreak}
\newcommand{\signed}[1]{\bigbreak\filbreak{\raggedleft #1\par}\medskip}

% environments for blocks (some may become commands)
\newenvironment{borderbox}{}{} % framing content
\newenvironment{citbibl}{\ifvmode\hfill\fi}{\ifvmode\par\fi }
\newenvironment{docAuthor}{\ifvmode\vskip4pt\fontsize{16pt}{18pt}\selectfont\fi\itshape}{\ifvmode\par\fi }
\newenvironment{docDate}{}{\ifvmode\par\fi }
\newenvironment{docImprint}{\vskip6pt}{\ifvmode\par\fi }
\newenvironment{docTitle}{\vskip6pt\bfseries\fontsize{18pt}{22pt}\selectfont}{\par }
\newenvironment{msHead}{\vskip6pt}{\par}
\newenvironment{msItem}{\vskip6pt}{\par}
\newenvironment{titlePart}{}{\par }


% environments for block containers
\newenvironment{argument}{\itshape\parindent0pt}{\bigskip}
\newenvironment{biblfree}{}{\ifvmode\par\fi }
\newenvironment{bibitemlist}[1]{%
  \list{\@biblabel{\@arabic\c@enumiv}}%
  {%
    \settowidth\labelwidth{\@biblabel{#1}}%
    \leftmargin\labelwidth
    \advance\leftmargin\labelsep
    \@openbib@code
    \usecounter{enumiv}%
    \let\p@enumiv\@empty
    \renewcommand\theenumiv{\@arabic\c@enumiv}%
  }
  \sloppy
  \clubpenalty4000
  \@clubpenalty \clubpenalty
  \widowpenalty4000%
  \sfcode`\.\@m
}%
{\def\@noitemerr
  {\@latex@warning{Empty `bibitemlist' environment}}%
\endlist}
\newenvironment{quoteblock}% may be used for ornaments
  {\begin{quoting}}
  {\end{quoting}}
\newenvironment{epigraph}{\parindent0pt\raggedleft\it}{\bigskip}

% table () is preceded and finished by custom command
\newcommand{\tableopen}[1]{%
  \ifnum\strcmp{#1}{wide}=0{%
    \begin{center}
  }
  \else\ifnum\strcmp{#1}{long}=0{%
    \begin{center}
  }
  \else{%
    \begin{center}
  }
  \fi\fi
}
\newcommand{\tableclose}[1]{%
  \ifnum\strcmp{#1}{wide}=0{%
    \end{center}
  }
  \else\ifnum\strcmp{#1}{long}=0{%
    \end{center}
  }
  \else{%
    \end{center}
  }
  \fi\fi
}


% text structure
\newcommand\chapteropen{} % before chapter title
\newcommand\chaptercont{} % after title, argument, epigraph…
\newcommand\chapterclose{} % maybe useful for multicol settings
\setcounter{secnumdepth}{-2} % no counters for hierarchy titles
\setcounter{tocdepth}{5} % deep toc
\markright{\@title} % ???
\markboth{\@title}{\@author} % ???
\renewcommand\tableofcontents{\@starttoc{toc}}
% toclof format
% \renewcommand{\@tocrmarg}{0.1em} % Useless command?
% \renewcommand{\@pnumwidth}{0.5em} % {1.75em}
\renewcommand{\@cftmaketoctitle}{}
\setlength{\cftbeforesecskip}{\z@ \@plus.2\p@}
\renewcommand{\cftchapfont}{}
\renewcommand{\cftchapdotsep}{\cftdotsep}
\renewcommand{\cftchapleader}{\normalfont\cftdotfill{\cftchapdotsep}}
\renewcommand{\cftchappagefont}{\bfseries}
\setlength{\cftbeforechapskip}{0em \@plus\p@}
% \renewcommand{\cftsecfont}{\small\relax}
\renewcommand{\cftsecpagefont}{\normalfont}
% \renewcommand{\cftsubsecfont}{\small\relax}
\renewcommand{\cftsecdotsep}{\cftdotsep}
\renewcommand{\cftsecpagefont}{\normalfont}
\renewcommand{\cftsecleader}{\normalfont\cftdotfill{\cftsecdotsep}}
\setlength{\cftsecindent}{1em}
\setlength{\cftsubsecindent}{2em}
\setlength{\cftsubsubsecindent}{3em}
\setlength{\cftchapnumwidth}{1em}
\setlength{\cftsecnumwidth}{1em}
\setlength{\cftsubsecnumwidth}{1em}
\setlength{\cftsubsubsecnumwidth}{1em}

% footnotes
\newif\ifheading
\newcommand*{\fnmarkscale}{\ifheading 0.70 \else 1 \fi}
\renewcommand\footnoterule{\vspace*{0.3cm}\hrule height \arrayrulewidth width 3cm \vspace*{0.3cm}}
\setlength\footnotesep{1.5\footnotesep} % footnote separator
\renewcommand\@makefntext[1]{\parindent 1.5em \noindent \hb@xt@1.8em{\hss{\normalfont\@thefnmark . }}#1} % no superscipt in foot
\patchcmd{\@footnotetext}{\footnotesize}{\footnotesize\sffamily}{}{} % before scrextend, hyperref


%   see https://tex.stackexchange.com/a/34449/5049
\def\truncdiv#1#2{((#1-(#2-1)/2)/#2)}
\def\moduloop#1#2{(#1-\truncdiv{#1}{#2}*#2)}
\def\modulo#1#2{\number\numexpr\moduloop{#1}{#2}\relax}

% orphans and widows
\clubpenalty=9996
\widowpenalty=9999
\brokenpenalty=4991
\predisplaypenalty=10000
\postdisplaypenalty=1549
\displaywidowpenalty=1602
\hyphenpenalty=400
% Copied from Rahtz but not understood
\def\@pnumwidth{1.55em}
\def\@tocrmarg {2.55em}
\def\@dotsep{4.5}
\emergencystretch 3em
\hbadness=4000
\pretolerance=750
\tolerance=2000
\vbadness=4000
\def\Gin@extensions{.pdf,.png,.jpg,.mps,.tif}
% \renewcommand{\@cite}[1]{#1} % biblio

\usepackage{hyperref} % supposed to be the last one, :o) except for the ones to follow
\urlstyle{same} % after hyperref
\hypersetup{
  % pdftex, % no effect
  pdftitle={\elbibl},
  % pdfauthor={Your name here},
  % pdfsubject={Your subject here},
  % pdfkeywords={keyword1, keyword2},
  bookmarksnumbered=true,
  bookmarksopen=true,
  bookmarksopenlevel=1,
  pdfstartview=Fit,
  breaklinks=true, % avoid long links
  pdfpagemode=UseOutlines,    % pdf toc
  hyperfootnotes=true,
  colorlinks=false,
  pdfborder=0 0 0,
  % pdfpagelayout=TwoPageRight,
  % linktocpage=true, % NO, toc, link only on page no
}

\makeatother % /@@@>
%%%%%%%%%%%%%%
% </TEI> end %
%%%%%%%%%%%%%%


%%%%%%%%%%%%%
% footnotes %
%%%%%%%%%%%%%
\renewcommand{\thefootnote}{\bfseries\textcolor{rubric}{\arabic{footnote}}} % color for footnote marks

%%%%%%%%%
% Fonts %
%%%%%%%%%
\usepackage[]{roboto} % SmallCaps, Regular is a bit bold
% \linespread{0.90} % too compact, keep font natural
\newfontfamily\fontrun[]{Roboto Condensed Light} % condensed runing heads
\ifav
  \setmainfont[
    ItalicFont={Roboto Light Italic},
  ]{Roboto}
\else\ifbooklet
  \setmainfont[
    ItalicFont={Roboto Light Italic},
  ]{Roboto}
\else
\setmainfont[
  ItalicFont={Roboto Italic},
]{Roboto Light}
\fi\fi
\renewcommand{\LettrineFontHook}{\bfseries\color{rubric}}
% \renewenvironment{labelblock}{\begin{center}\bfseries\color{rubric}}{\end{center}}

%%%%%%%%
% MISC %
%%%%%%%%

\setdefaultlanguage[frenchpart=false]{french} % bug on part


\newenvironment{quotebar}{%
    \def\FrameCommand{{\color{rubric!10!}\vrule width 0.5em} \hspace{0.9em}}%
    \def\OuterFrameSep{\itemsep} % séparateur vertical
    \MakeFramed {\advance\hsize-\width \FrameRestore}
  }%
  {%
    \endMakeFramed
  }
\renewenvironment{quoteblock}% may be used for ornaments
  {%
    \savenotes
    \setstretch{0.9}
    \normalfont
    \begin{quotebar}
  }
  {%
    \end{quotebar}
    \spewnotes
  }


\renewcommand{\headrulewidth}{\arrayrulewidth}
\renewcommand{\headrule}{{\color{rubric}\hrule}}

% delicate tuning, image has produce line-height problems in title on 2 lines
\titleformat{name=\chapter} % command
  [display] % shape
  {\vspace{1.5em}\centering} % format
  {} % label
  {0pt} % separator between n
  {}
[{\color{rubric}\huge\textbf{#1}}\bigskip] % after code
% \titlespacing{command}{left spacing}{before spacing}{after spacing}[right]
\titlespacing*{\chapter}{0pt}{-2em}{0pt}[0pt]

\titleformat{name=\section}
  [block]{}{}{}{}
  [\vbox{\color{rubric}\large\raggedleft\textbf{#1}}]
\titlespacing{\section}{0pt}{0pt plus 4pt minus 2pt}{\baselineskip}

\titleformat{name=\subsection}
  [block]
  {}
  {} % \thesection
  {} % separator \arrayrulewidth
  {}
[\vbox{\large\textbf{#1}}]
% \titlespacing{\subsection}{0pt}{0pt plus 4pt minus 2pt}{\baselineskip}

\ifaiv
  \fancypagestyle{main}{%
    \fancyhf{}
    \setlength{\headheight}{1.5em}
    \fancyhead{} % reset head
    \fancyfoot{} % reset foot
    \fancyhead[L]{\truncate{0.45\headwidth}{\fontrun\elbibl}} % book ref
    \fancyhead[R]{\truncate{0.45\headwidth}{ \fontrun\nouppercase\leftmark}} % Chapter title
    \fancyhead[C]{\thepage}
  }
  \fancypagestyle{plain}{% apply to chapter
    \fancyhf{}% clear all header and footer fields
    \setlength{\headheight}{1.5em}
    \fancyhead[L]{\truncate{0.9\headwidth}{\fontrun\elbibl}}
    \fancyhead[R]{\thepage}
  }
\else
  \fancypagestyle{main}{%
    \fancyhf{}
    \setlength{\headheight}{1.5em}
    \fancyhead{} % reset head
    \fancyfoot{} % reset foot
    \fancyhead[RE]{\truncate{0.9\headwidth}{\fontrun\elbibl}} % book ref
    \fancyhead[LO]{\truncate{0.9\headwidth}{\fontrun\nouppercase\leftmark}} % Chapter title, \nouppercase needed
    \fancyhead[RO,LE]{\thepage}
  }
  \fancypagestyle{plain}{% apply to chapter
    \fancyhf{}% clear all header and footer fields
    \setlength{\headheight}{1.5em}
    \fancyhead[L]{\truncate{0.9\headwidth}{\fontrun\elbibl}}
    \fancyhead[R]{\thepage}
  }
\fi

\ifav % a5 only
  \titleclass{\section}{top}
\fi

\newcommand\chapo{{%
  \vspace*{-3em}
  \centering % no vskip ()
  {\Large\addfontfeature{LetterSpace=25}\bfseries{\elauthor}}\par
  \smallskip
  {\large\eldate}\par
  \bigskip
  {\Large\selectfont{\eltitle}}\par
  \bigskip
  {\color{rubric}\hline\par}
  \bigskip
  {\Large TEXTE LIBRE À PARTICIPATIONS LIBRES\par}
  \centerline{\small\color{rubric} {hurlus.fr, tiré le \today}}\par
  \bigskip
}}

\newcommand\cover{{%
  \thispagestyle{empty}
  \centering
  {\LARGE\bfseries{\elauthor}}\par
  \bigskip
  {\Large\eldate}\par
  \bigskip
  \bigskip
  {\LARGE\selectfont{\eltitle}}\par
  \vfill\null
  {\color{rubric}\setlength{\arrayrulewidth}{2pt}\hline\par}
  \vfill\null
  {\Large TEXTE LIBRE À PARTICIPATIONS LIBRES\par}
  \centerline{{\href{https://hurlus.fr}{\dotuline{hurlus.fr}}, tiré le \today}}\par
}}

\begin{document}
\pagestyle{empty}
\ifbooklet{
  \cover\newpage
  \thispagestyle{empty}\hbox{}\newpage
  \cover\newpage\noindent Les voyages de la brochure\par
  \bigskip
  \begin{tabularx}{\textwidth}{l|X|X}
    \textbf{Date} & \textbf{Lieu}& \textbf{Nom/pseudo} \\ \hline
    \rule{0pt}{25cm} &  &   \\
  \end{tabularx}
  \newpage
  \addtocounter{page}{-4}
}\fi

\thispagestyle{empty}
\ifaiv
  \twocolumn[\chapo]
\else
  \chapo
\fi
{\it\elabstract}
\bigskip
\makeatletter\@starttoc{toc}\makeatother % toc without new page
\bigskip

\pagestyle{main} % after style

  \section[{I}]{I}\phantomsection
\label{I}\renewcommand{\leftmark}{I}


\begin{argument}
\noindent Considérations fondamentales sur la nature et la destination du véritable esprit Philosophique : appréciation sommaire de l’extrême importance sociale que présente aujourd’hui l’universelle propagation des Principales études positives : application spéciale de ces principes à la science astronomique, d’après sa vraie position encyclopédique.

\end{argument}

\noindent L’ensemble des connaissances astronomiques, trop isolément considéré jusqu’ici, ne doit plus constituer désormais que l’un des éléments indispensables d’un nouveau système indivisible de philosophie générale, graduellement préparé par le concours spontané de tous les grands travaux scientifiques propres aux trois derniers siècles, et finalement parvenu aujourd’hui à sa vraie maturité abstraite. En vertu de cette intime connexité, très peu comprise encore, la nature et la destination de ce {\itshape Traité} ne sauraient être suffisamment appréciées, si ce préambule nécessaire n’était pas surtout consacré à définir convenablement le véritable esprit fondamental de cette philosophie, dont l’installation universelle doit, au fond, devenir le but essentiel d’un tel enseignement. Comme elle se distingue principalement par une continuelle prépondérance, à la fois logique et scientifique, du point de vue historique ou social, je dois d’abord, pour la mieux caractériser, rappeler sommairement la grande loi que j’ai établie, dans mon \emph{Système de philosophie positive}, sur l’entière évolution intellectuelle de l’Humanité, loi à laquelle d’ailleurs nos études astronomiques auront ensuite fréquemment recours.\par
Suivant cette doctrine fondamentale, toutes nos spéculations quelconques sont inévitablement assujetties, soit chez l’individu, soit chez l’espèce, à passer successivement par trois états théoriques différents, que les dénominations habituelles de théologique, métaphysique et positif pourront ici qualifier suffisamment, pour ceux, du moins, qui en auront bien compris le vrai sens général. Quoique d’abord indispensable, à tous égards, le premier état doit désormais être toujours conçu comme purement provisoire et préparatoire ; le second, qui n’en constitue réellement qu’une modification dissolvante, ne comporte jamais qu’une simple destination transitoire, afin de conduire graduellement au troisième ; c’est en celui-ci, seul pleinement normal, que consiste, en tous genres, le régime définitif de la raison humaine.\par
Dans leur premier essor, nécessairement théologique, toutes nos spéculations manifestent spontanément une prédilection caractéristique pour les questions les plus insolubles, sur les sujets les plus radicalement inaccessibles à toute investigation décisive. Par un contraste qui, de nos jours, doit d’abord paraître inexplicable, mais qui, au fond, est alors en pleine harmonie avec la vraie situation initiale de notre intelligence, en un temps où l’esprit humain est encore au-dessous des plus simples problèmes scientifiques, il recherche avidement, et d’une manière presque exclusive, l’origine de toutes choses, les {\itshape causes} essentielles, soit premières, soit finales, des divers phénomènes qui le frappent, et leur mode fondamental de production, en un mot les connaissances absolues. Ce besoin primitif se trouve naturellement satisfait, autant qu’il puisse jamais l’être, par notre tendance à transporter partout le type humain, en assimilant tous les phénomènes quelconques à ceux que nous produisons nous-mêmes, et {\itshape qui}, à ce titre, commencent par nous sembler assez connus, d’après l’intuition immédiate qui les accompagne. Pour bien comprendre l’esprit, purement théologique, résulté du développement, de plus en plus systématique, de cet état primordial, il ne faut pas se borner à le considérer dans sa dernière phase, {\itshape qui} s’achève, sous nos yeux, chez les populations les plus avancées, mais qui n’est point, à beaucoup près, la plus caractéristique : il devient indispensable de jeter un coup d’œil sur l’ensemble de sa marche naturelle, afin d’apprécier son identité fondamentale sous les trois formes principales qui lui sont successivement propres.\par
La plus immédiate et la plus prononcée constitue le {\itshape fétichisme proprement dit}, consistant surtout à attribuer à tous les corps extérieurs une vie essentiellement analogue à la nôtre, mais presque toujours plus énergique, d’après leur action ordinairement plus puissante. L’adoration des astres caractérise le degré le plus élevé de cette première phase théologique, qui, au début, diffère à peine de l’état mental où s’arrêtent les animaux supérieurs. Quoique cette première forme de la philosophie théologique se retrouve avec évidence dans l’histoire intellectuelle de toutes nos sociétés, elle ne domine plus directement aujourd’hui que chez la moins nombreuse des trois grandes races qui composent notre espèce.\par
Sous sa seconde phase essentielle, constituant le vrai {\itshape polythéisme}, trop souvent confondu par les modernes avec l’état précédent, l’esprit théologique représente nettement la libre prépondérance spéculative de l’imagination, tandis que jusqu’alors l’instinct et le sentiment avaient surtout prévalu dans les théories humaines. La philosophie initiale y subit la plus profonde transformation que puisse comporter l’ensemble de sa destinée réelle, en ce que la vie y est enfin retirée aux objets matériels, pour être mystérieusement transportée à divers êtres fictifs, habituellement invisibles, dont l’active intervention continue devient désormais la source directe de tous les phénomènes humains. C’est pendant cette phase caractéristique, mal appréciée aujourd’hui, qu’il faut principalement étudier l’esprit théologique, qui s’y développe avec une plénitude et une homogénéité ultérieurement impossibles : ce temps est, à tous égards, celui de son plus grand ascendant. à la fois mental et social. La majorité de notre espèce n’est point encore sortie d’un tel état qui persiste aujourd’hui chez la plus nombreuse des trois races humaines, outre l’élite de la race noire et la partie la moins avancée de la race blanche.\par
Dans la troisième phase théologique, le {\itshape monothéisme} proprement dit commence l’inévitable déclin de la philosophie initiale, qui, tout en conservant longtemps une grande influence sociale, toutefois plus apparente encore que réelle, subit dès lors un rapide décroissement intellectuel, par une suite spontanée de cette simplification caractéristique, où la raison vient restreindre de plus en plus la domination antérieure de l’imagination en laissant graduellement développer le sentiment universel, jusqu’alors presque insignifiant, de l’assujettissement nécessaire de tous les phénomènes naturels à des lois invariables. Sous des formes très diverses, et même radicalement inconciliables, cet extrême mode du régime préliminaire persiste encore avec une énergie fort inégale, chez l’immense majorité de la race blanche ; mais, quoiqu’il soit ainsi d’une observation plus facile, ces mêmes préoccupations personnelles apportent aujourd’hui un trop fréquent obstacle à sa judicieuse appréciation, faute d’une comparaison assez rationnelle et assez impartiale avec les deux modes précédents.\par
Quelque imparfaite que doive maintenant sembler une telle manière de philosopher, il importe beaucoup de rattacher indissolublement l’état présent de l’esprit humain à l’ensemble de ses états antérieurs, en reconnaissant convenablement qu’elle dût être longtemps aussi indispensable qu’inévitable. En nous bornant ici à la simple appréciation intellectuelle, il serait d’abord superflu d’insister sur la tendance involontaire qui, même aujourd’hui, nous entraîne tous évidemment aux explications essentiellement théologiques, aussitôt que nous voulons pénétrer directement le mystère inaccessible du mode fondamental de production de phénomènes quelconques, et surtout envers ceux dont nous ignorons encore les lois réelles. Les plus éminents penseurs peuvent alors constater leur propre disposition naturelle au plus naïf fétichisme, quand cette ignorance se trouve momentanément combinée avec quelque passion prononcée. Si donc toutes les explications théologiques ont subi, chez les modernes occidentaux, une désuétude croissante et décisive, c’est uniquement parce que les mystérieuses recherches qu’elles avaient en vue ont été de plus en plus écartées comme radicalement inaccessibles à notre intelligence, qui s’est graduellement habituée à y substituer irrévocablement des études plus efficaces, et mieux en harmonie avec nos vrais besoins. Même en un temps où le véritable esprit philosophique avait déjà prévalu envers les plus simples phénomènes et dans un sujet aussi facile que la théorie élémentaire du choc, le mémorable exemple de Malebranche rappellera toujours la nécessité de recourir à l’intervention directe et permanente d’une action surnaturelle, toutes les fois qu’on tente de remonter à la cause première d’un événement quelconque. Or, d’une autre part, de telles tentatives, quelque puériles qu’elles semblent justement aujourd’hui, constituaient certainement le seul moyen primitif de déterminer l’essor continu des spéculations humaines, en dégageant spontanément notre intelligence du cercle profondément vicieux où elle est d’abord nécessairement enveloppée par l’opposition radicale de deux conditions également impérieuses. Car, si les modernes ont dû proclamer l’impossibilité de fonder aucune théorie solide, autrement que sur un suffisant concours d’observations convenables, il n’est pas moins incontestable que l’esprit humain ne pourrait jamais combiner, ni même recueillir, ces indispensables matériaux, sans être toujours dirigé par quelques vues spéculatives préalablement établies. Ainsi, ces conceptions primordiales ne pouvaient, évidemment, résulter que d’une philosophie dispensée, par sa nature, de toute longue préparation, et susceptible en un mot, de surgir spontanément, sous la seule impulsion d’un instinct direct, quelque chimériques que dussent être d’ailleurs des spéculations ainsi dépourvues de tout fondement réel. Tel est l’heureux privilège des principes théologiques, sans lesquels on doit assurer que notre intelligence ne pouvait jamais sortir de sa torpeur initiale, et qui seuls ont pu permettre, en dirigeant son activité spéculative, de préparer graduellement un meilleur régime logique. Cette aptitude fondamentale fut, au reste, puissamment secondée par la prédilection originaire de l’esprit humain pour les questions insolubles que poursuivait surtout cette philosophie primitive. Nous ne pouvions mesurer nos forces mentales, et, par suite, en circonscrire sagement la destination qu’après les avoir suffisamment exercées. Or, cet indispensable exercice ne pouvait d’abord être déterminé, surtout dans les plus faibles facultés de la nature, sans l’énergique stimulation inhérente à de telles études, où tant d’intelligences mal cultivées persistent encore à chercher la plus prompte et la plus complète solution des questions directement usuelles. Il a même longtemps fallu, afin de vaincre suffisamment notre inertie native, recourir aussi aux puissantes illusions que suscitait spontanément une telle philosophie sur le pouvoir presque indéfini de l’homme pour modifier à son gré un monde alors conçu comme essentiellement ordonné à son usage, et qu’aucune grande loi ne pouvait encore soustraire à l’arbitraire suprématie des influences surnaturelles. À peine y a-t-il trois siècles que, chez l’élite de l’Humanité, les espérances astrologiques et alchimiques, dernier vestige scientifique de cet esprit primordial, ont réellement cessé de servir à l’accumulation journalière des observations correspondantes, comme Kepler et Berthollet l’ont respectivement indiqué.\par
Le concours décisif de ces divers motifs intellectuels serait, en outre, puissamment fortifié si la nature de ce Traité me permettait d’y signaler suffisamment l’influence irrésistible des hautes nécessités sociales, que j’ai convenablement appréciées dans l’ouvrage fondamental mentionné au début de ce Discours. On peut d’abord pleinement démontrer ainsi combien l’esprit théologique a dû être longtemps indispensable à la combinaison permanente des idées morales et politiques, encore plus spécialement qu’à celle de toutes les autres, soit en vertu de leur complication supérieure, soit parce que les phénomènes correspondants, primitivement trop peu prononcés, ne pouvaient acquérir un développement caractéristique que d’après un essor très prolongé de la civilisation humaine. C’est une étrange inconséquence, à peine excusable par la tendance aveuglément critique de notre temps, que de reconnaître, pour les anciens, l’impossibilité de philosopher sur les simples sujets autrement que suivant le mode théologique, et de méconnaître néanmoins, surtout chez les polythéistes, l’insurmontable nécessité d’un régime analogue envers les spéculations sociales. Mais il faut sentir, en outre, quoique je ne puisse l’établir ici, que cette philosophie initiale n’a pas été moins indispensable à l’essor préliminaire de notre sociabilité qu’à celui de notre intelligence, soit pour constituer primitivement quelques doctrines communes, sans lesquelles le lien social n’aurait pu acquérir ni étendue ni consistance, soit en suscitant spontanément la seule autorité spirituelle qui pût alors surgir.\par
Quelque sommaires que dussent être ici ces explications générales sur la nature provisoire et la destination préparatoire de la seule philosophie qui convînt réellement à l’enfance de l’Humanité, elles font aisément sentir que ce régime initial diffère trop profondément, à tous égards, de celui que nous allons voir correspondre à la virilité mentale, pour que le passage graduel de l’un à l’autre pût originairement s’opérer, soit dans l’individu, soit dans l’espèce, sans l’assistance croissante d’une sorte de philosophie intermédiaire, essentiellement bornée à cet office transitoire. Telle est la participation spéciale de l’état métaphysique proprement dit à l’évolution fondamentale de notre intelligence, qui antipathique à tout changement brusque, peut ainsi s’élever presque insensiblement de l’état purement théologique à l’état franchement positif, quoique cette situation équivoque se rapproche, au fond, bien davantage du premier que du dernier. Les spéculations dominantes y ont conservé le même caractère essentiel de tendance habituelle aux connaissances absolues : seulement la solution y a subi une transformation notable, propre à mieux faciliter l’essor des conceptions positives. Comme la théologie, en effet, la métaphysique tente surtout d’expliquer la nature intime des êtres, l’origine et la destination de toutes choses, le mode essentiel de production de tous les phénomènes ; mais au lieu d’y employer les agents surnaturels proprement dits, elle les remplace de plus en plus par ces {\itshape entités} ou abstractions personnifiées, dont l’usage, vraiment caractéristique, a souvent permis de la désigner sous le nom d’ontologie. Il n’est que trop facile aujourd’hui d’observer aisément une telle manière de philosopher, qui, encore prépondérante envers les phénomènes les plus compliqués, offre journellement, même dans les théories les plus simples et les moins arriérées, tant de traces appréciables de sa longue domination\footnote{ Presque toutes les explications habituelles relatives aux phénomènes sociaux, la plupart de celles qui concernent l’homme intellectuel et moral, une grande partie de nos théories physiologiques ou médicales, et même aussi plusieurs théories chimiques, etc., rappellent encore directement l’étrange manière de philosopher si plaisamment caractérisée par Molière, sans aucune grave exagération, à l’occasion, par exemple, de la {\itshape vertu dormitive de l’opium}, conformément à l’ébranlement décisif que Descartes venait de faire subir à tout le régime des entités.}. L’efficacité historique de ces entités résulte directement de leur caractère équivoque : car, en chacun de ces êtres métaphysiques, inhérent au corps correspondant sans se confondre avec lui, l’esprit peut, à volonté, selon qu’il est plus près de l’état théologique ou de l’état positif, voir ou une véritable émanation de la puissance surnaturelle, ou une simple domination abstraite du phénomène considéré. Ce n’est plus alors la pure imagination qui domine, et ce n’est pas encore la véritable observation ; mais le raisonnement y acquiert beaucoup d’extension, et se prépare confusément à l’exercice vraiment scientifique. On doit, d’ailleurs, remarquer que sa part spéculative s’y trouve d’abord très exagérée, par suite de cette tendance opiniâtre à argumenter au lieu d’observer, qui, en tous genres, caractérise habituellement l’esprit métaphysique, même chez ses plus éminents organes. Un ordre de conceptions aussi flexible, qui ne comporte aucunement la consistance si longtemps propre au système théologique, doit d’ailleurs parvenir, bien plus rapidement, à l’unité correspondante, par la subordination graduelle des diverses entités particulières à une seule entité générale, la {\itshape nature}, destinée à déterminer le faible équivalent métaphysique de la vague liaison universelle résultée du monothéisme.\par
Pour mieux comprendre, surtout de nos jours, l’efficacité historique d’un tel appareil philosophique, il importe de reconnaître que, par sa nature, il n’est spontanément susceptible que d’une simple activité critique ou dissolvante, même mentale, et à plus forte raison sociale, sans pouvoir jamais rien organiser qui lui soit propre. Radicalement inconséquent, cet esprit équivoque conserve tous les principes fondamentaux du système théologique, mais en leur ôtant de plus en plus cette vigueur et cette fixité indispensable, à leur autorité effective ; et c’est dans une semblable altération que consiste, en effet, à tous égards, sa principale utilité passagère, quand le régime antique, longtemps progressif pour l’ensemble de l’évolution humaine, se trouve inévitablement parvenu à ce degré de prolongation abusive où il tend à perpétuer indéfiniment l’état d’enfance qu’il avait d’abord si heureusement dirigé. La métaphysique n’est donc réellement, au fond, qu’une sorte de théologie graduellement énervée par des simplifications dissolvantes, qui lui ôtent spontanément le pouvoir direct d’empêcher l’essor spécial des conceptions positives, tout en lui conservant néanmoins l’aptitude provisoire à entretenir un certain exercice indispensable de l’esprit de généralisation, jusqu’à ce qu’il puisse enfin recevoir une meilleure alimentation. D’après son caractère contradictoire, le régime métaphysique ou ontologique est toujours placé dans cette inévitable alternative de tendre à une vaine restauration de l’état théologique pour satisfaire aux conditions d’ordre, ou de pousser à une situation purement négative afin d’échapper à l’empire oppressif de la théologie. Cette oscillation nécessaire, qui maintenant ne s’observe plus qu’envers les plus difficiles théories, a pareillement existé jadis à l’égard même des plus simples, tant qu’a duré leur âge métaphysique, en vertu de l’impuissance organique toujours propre à une telle manière de philosopher. Si la raison publique ne l’avait dès longtemps écartée pour certaines notions fondamentales, on ne doit pas craindre d’assurer que les doutes insensés qu’elle suscita, il y a vingt siècles, sur l’existence des corps extérieurs subsisteraient encore essentiellement, car elle ne les a certainement jamais dissipés par aucune argumentation décisive. On peut donc finalement envisager l’état métaphysique comme une sorte de maladie chronique, naturellement inhérente à notre évolution mentale, individuelle ou collective, entre l’enfance et la virilité.\par
Les spéculations historiques ne remontant presque jamais, chez les modernes, au-delà des temps polythéiques, l’esprit métaphysique doit y sembler à peu près aussi ancien que l’esprit théologique lui-même, puisqu’il a nécessairement présidé, quoique d’une manière implicite, à la transformation primitive du fétichisme en polythéisme, afin de suppléer déjà à l’activité purement surnaturelle qui, ainsi directement retirée à chaque corps particulier, y devait spontanément laisser quelque entité correspondante. Toutefois, comme cette première révolution théologique n’a pu alors donner lieu à aucune vraie discussion, l’intervention continue de l’esprit ontologique n’a commencé à devenir pleinement caractéristique que dans la révolution suivante, pour la réduction du polythéisme en monothéisme, dont il a dû être l’organe naturel. Son influence croissante devait d’abord paraître organique, tant qu’il restait subordonné à l’impulsion théologique mais sa nature essentiellement dissolvante a dû ensuite se manifester de plus en plus, quand il a tenté graduellement de pousser la simplification de la théologie au-delà même du monothéisme vulgaire, qui constituait, de toute nécessité, l’extrême phase vraiment possible de la philosophie initiale. C’est ainsi que, pendant les cinq derniers siècles, l’esprit métaphysique a secondé négativement l’essor fondamental de notre civilisation moderne, en décomposant peu à peu le système théologique, devenu finalement rétrograde, depuis que l’efficacité sociale du régime monothéique se trouvait essentiellement épuisée, à la fin du moyen âge. Malheureusement, après avoir accompli, en chaque genre, cet office indispensable mais passager, l’action trop prolongée des conceptions ontologiques a dû toujours tendre à empêcher aussi toute autre organisation réelle du système spéculatif ; en sorte que le plus dangereux obstacle à l’installation finale d’une vraie philosophie résulte, en effet, aujourd’hui de ce même esprit qui souvent s’attribue encore le privilège presque exclusif des méditations philosophiques.\par
Cette longue succession de préambules nécessaires conduit enfin notre intelligence, graduellement émancipée, à son état définitif de positivité rationnelle, qui doit ici être caractérisé d’une manière plus spéciale que les deux états préliminaires. De tels exercices préparatoires ayant spontanément constaté l’inanité radicale des explications vagues et arbitraires propres à la philosophie initiale, soit théologique, soit métaphysique, l’esprit humain renonce désormais aux recherches absolues qui ne convenaient qu’à son enfance, et circonscrit ses efforts dans le domaine, dès lors rapidement progressif, de la véritable observation, seule base possible des connaissances vraiment accessibles, sagement adaptées à des besoins réels. La logique spéculative avait jusqu’alors consisté à raisonner, d’une manière plus ou moins subtile, d’après des principes confus, qui, ne comportant aucune preuve suffisante, suscitaient toujours des débats sans issue. Elle reconnaît désormais, comme règle fondamentale que toute proposition qui n’est pas strictement réductible à la simple énonciation d’un fait, ou particulier ou général, ne peut offrir aucun sens réel et intelligible. Les principes qu’elle emploie ne sont plus eux-mêmes que de véritables faits, seulement plus généraux et plus abstraits que ceux dont ils doivent former le lien. Quel que soit d’ailleurs le mode, rationnel ou expérimental, de procéder à leur découverte, c’est toujours de leur conformité, directe ou indirecte, avec les phénomènes observés que résulte exclusivement leur efficacité scientifique. La pure imagination perd alors irrévocablement son antique suprématie mentale, et se subordonne nécessairement à l’observation, de manière à constituer un état logique pleinement normal, sans cesser néanmoins d’exercer, dans les spéculations positives, un office aussi capital qu’inépuisable, pour créer ou perfectionner les moyens de liaison, soit définitive, soit provisoire. En un mot, la révolution fondamentale qui caractérise la virilité de notre intelligence consiste essentiellement à substituer partout, à l’inaccessible détermination des causes proprement dites, la simple recherche des lois, c’est-à-dire des relations constantes qui existent entre les phénomènes observés. Qu’il s’agisse des moindres ou des plus sublimes effets, de choc et de pesanteur comme de pensée et de moralité, nous n’y pouvons vraiment connaître que les diverses liaisons mutuelles propres à leur accomplissement, sans jamais pénétrer le mystère de, leur production.\par
Non seulement nos recherches positives doivent essentiellement se réduire, en tous genres, à l’appréciation systématique de ce qui est, en renonçant à en découvrir la première origine et la destination finale ; mais il importe, en outre, de sentir que cette étude des phénomènes, au lieu de pouvoir devenir aucunement absolue, doit toujours rester relative à notre organisation et à notre situation. En reconnaissant, sous ce double aspect, l’imperfection nécessaire de nos divers moyens spéculatifs, on voit que, loin de pouvoir étudier complètement aucune existence effective, nous ne saurions garantir nullement la possibilité de constater ainsi, même très superficiellement, toutes les existences réelles, dont la majeure partie peut-être doit nous échapper totalement. Si la perte d’un sens important suffit pour nous cacher radicalement un ordre entier de phénomènes naturels, il y a tout lieu de penser, réciproquement, que l’acquisition d’un sens nouveau nous dévoilerait une classe de faits dont nous n’avons maintenant aucune idée, à moins de croire que la diversité des sens, si différente entre les principaux types d’animalité, se trouve poussée, dans notre organisme, au plus haut degré que puisse exiger l’exploration totale du monde extérieur, supposition évidemment gratuite, et presque ridicule. Aucune science ne peut mieux manifester que l’astronomie cette nature nécessairement relative de toutes nos connaissances réelles, puisque, l’investigation des phénomènes ne pouvant s’y opérer que par un seul sens, il est très facile d’y apprécier les conséquences spéculatives de sa suppression ou de sa simple altération. Il ne saurait exister aucune astronomie chez une espèce aveugle, quelque intelligente qu’on la supposât, ni envers des astres obscurs, qui sont peut-être les plus nombreux, ni même si seulement l’atmosphère à travers laquelle nous observons les corps célestes restait toujours et partout nébuleuse. Tout le cours de ce Traité nous offrira de fréquentes occasions d’apprécier spontanément, de la manière la moins équivoque, cette intime dépendance où l’ensemble de nos conditions propres, tant intérieures qu’extérieures, retient inévitablement chacune de nos études positives.\par
Pour caractériser suffisamment cette nature nécessairement relative de toutes nos connaissances réelles, il importe de sentir, en outre, du point de vue le plus philosophique, que, si nos conceptions quelconques doivent être considérées elles-mêmes comme autant de phénomènes humains, de tels phénomènes ne sont pas simplement individuels, mais aussi et surtout sociaux, puisqu’ils résultent, en effet d’une évolution collective et continue, dont tous les éléments et toutes les phases sont essentiellement connexes. Si donc, sous le premier aspect, on reconnaît que nos spéculations doivent toujours dépendre des diverses conditions essentielles de notre existence individuelle, il faut également admettre, sous le second, qu’elles ne sont pas moins subordonnées à l’ensemble de la progression sociale, de manière à ne pouvoir jamais comporter cette fixité absolue que les métaphysiciens ont supposée. Or, la loi générale du mouvement fondamental de l’Humanité consiste, à cet égard, en ce que nos théories tendent de plus en plus à représenter exactement les sujets extérieurs de nos constantes investigations, sans que néanmoins la vraie constitution de chacun d’eux puisse, en aucun cas, être pleinement appréciée, la perfection scientifique devant se borner à approcher de cette limite idéale autant que l’exigent nos divers besoins réels. Ce second genre de dépendance, propre aux spéculations positives, se manifeste aussi clairement que le premier dans le cours entier des études astronomiques, en considérant, par exemple, la suite des notions de plus en plus satisfaisantes, obtenues depuis l’origine de la géométrie céleste, sur la figure de la Terre, sur la forme des orbites planétaires, etc. Ainsi, quoique, d’une part, les doctrines scientifiques soient nécessairement d’une nature assez mobile pour devoir écarter toute prétention à l’absolu, leurs variations graduelles ne présentent, d’une autre part, aucun caractère arbitraire qui puisse motiver un scepticisme encore plus dangereux ; chaque changement successif conserve d’ailleurs spontanément aux théories correspondantes, une aptitude indéfinie à représenter les phénomènes qui leur ont servi de base, du moins tant qu’on n’y doit pas dépasser le degré primitif de précision effective.\par
Depuis que la subordination constante de l’imagination à l’observation a été unanimement reconnue comme la première condition fondamentale de toute saine spéculation scientifique, une vicieuse interprétation a souvent conduit à abuser beaucoup de ce grand principe logique pour faire dégénérer la science réelle en une sorte de stérile accumulation de faits incohérents, qui ne pourraient offrir d’autre mérite essentiel que celui de l’exactitude partielle. Il importe donc de bien sentir que le véritable esprit positif n’est pas moins éloigné, au fond, de l’empirisme que du mysticisme ; c’est entre ces deux aberrations, également funestes, qu’il doit toujours cheminer : le besoin d’une telle réserve continue, aussi difficile qu’importante, suffirait d’ailleurs pour vérifier, conformément à nos explications initiales, combien la vraie positivité doit être mûrement préparée, de manière à ne pouvoir nullement convenir à l’état naissant de l’Humanité. C’est dans les lois des phénomènes que consiste réellement la science, à laquelle les faits proprement dits, quelque exacts et nombreux qu’ils puissent être, ne fournissent jamais que d’indispensables matériaux. Or, en considérant la destination constante de ces lois, on peut dire, sans aucune exagération, que la véritable science, bien loin d’être formée de simples observations, tend toujours à dispenser, autant que possible, de l’exploration directe, en y substituant cette prévision rationnelle, qui constitue, à tous égards, le principal caractère de l’esprit positif, comme l’ensemble des études astronomiques nous le fera clairement sentir. Une telle prévision, suite nécessaire des relations constantes découvertes entre les phénomènes, ne permettra jamais de confondre la {\itshape science} réelle avec cette vaine érudition qui accumule machinalement des faits sans aspirer à les déduire les uns des autres. Ce grand attribut de toutes nos saines spéculations n’importe pas moins à leur utilité effective qu’à leur propre dignité ; car, l’exploration directe des phénomènes accomplis ne pourrait suffire à nous permettre d’en modifier l’accomplissement, si elle ne nous conduisait pas à le prévoir convenablement. Ainsi, le véritable esprit positif consiste surtout à {\itshape voir pour prévoir}, à étudier ce qui est afin d’en conclure ce qui sera, d’après le dogme général de l’invariabilité des lois naturelles\footnote{ Sur cette appréciation générale de l’esprit et de la marche propres à la méthode positive, on peut étudier, avec beaucoup de fruit, le précieux ouvrage intitulé : \emph{A system of logic, ratiocinative and inductive}, récemment publié à Londres (chez John Parker, West Strand, 1843), par mon éminent ami, M. John Stuart Mill, ainsi pleinement associé désormais à la fondation directe de la nouvelle philosophie. Les sept derniers chapitres du tome premier contiennent une admirable exposition dogmatique, aussi profonde que lumineuse, de la logique inductive, qui ne pourra jamais, j’ose l’assurer, être mieux conçue, ni mieux caractérisée en restant au point de vue où l’auteur s’est placé.}.\par
Ce principe fondamental de toute la philosophie positive, sans être encore, à beaucoup près, suffisamment étendu à l’ensemble des phénomènes, commence heureusement, depuis trois siècles, à devenir tellement familier, que, par suite des habitudes absolues antérieurement enracinées, on a presque toujours méconnu jusqu’ici sa véritable source, en s’efforçant, d’après une vaine et confuse argumentation métaphysique, de représenter comme une sorte de notion innée, ou du moins primitive, ce qui n’a pu certainement résulter que d’une lente induction graduelle, à la fois collective et individuelle. Non seulement aucun motif rationnel, indépendant de toute exploration extérieure, ne nous indique d’abord l’invariabilité des relations physiques ; mais il est incontestable, au contraire, que l’esprit humain éprouve, pendant sa longue enfance, un très vif penchant à la méconnaître, là même où une observation impartiale la lui manifesterait déjà, s’il n’était pas alors entraîné par sa tendance nécessaire à rapporter tous les événements quelconques, et surtout les plus importants, à des volontés arbitraires. Dans chaque ordre de phénomènes, il en existe, sans doute, quelques-uns assez simples et assez familiers pour que leur observation spontanée ait toujours suggéré le sentiment confus et incohérent d’une certaine régularité secondaire ; en sorte que le point de vue purement théologique n’a jamais pu être rigoureusement universel. Mais cette conviction partielle et précaire se borne longtemps aux phénomènes les moins nombreux et les plus subalternes, qu’elle ne peut même nullement préserver alors des fréquentes perturbations attribuées à l’intervention prépondérante des agents surnaturels. Le principe de l’invariabilité des lois naturelles ne commence réellement à acquérir quelque consistance philosophique que lorsque les premiers travaux vraiment scientifiques ont pu en manifester l’exactitude essentielle envers un ordre entier de grands phénomènes ; ce qui ne pouvait suffisamment résulter que de la fondation de l’astronomie mathématique pendant les derniers siècles du polythéisme. D’après cette introduction systématique, ce dogme fondamental a tendu, sans doute, à s’étendre, par analogie, à des phénomènes plus compliqués, avant même que leurs lois propres pussent être aucunement connues. Mais outre sa stérilité effective, cette vague anticipation logique avait alors trop peu d’énergie pour. résister convenablement à l’active suprématie mentale que conservaient encore les illusions théologico-métaphysiques. Une première ébauche spéciale de l’établissement des lois naturelles envers chaque ordre principal des phénomènes a été ensuite indispensable pour procurer à une telle notion cette force inébranlable qu’elle commence à présenter dans les sciences les plus avancées. Cette conviction ne saurait même devenir assez ferme, tant qu’une semblable élaboration n’a pas été vraiment étendue à toutes les spéculations fondamentales, l’incertitude laissée par les plus compliquées devant alors affecter plus ou moins chacune des autres. On ne peut méconnaître cette ténébreuse réaction, même aujourd’hui, où, par suite de l’ignorance encore habituelle envers les lois sociologiques, le principe de l’invariabilité des relations physiques reste quelquefois sujet à de graves altérations, jusque dans les études purement mathématiques, où nous voyons, par exemple, préconiser journellement un prétendu calcul des chances, qui suppose implicitement l’absence de toute loi réelle à l’égard de certains événements, surtout quand l’homme y intervient. Mais lorsque cette universelle extension est enfin suffisamment ébauchée, condition maintenant remplie chez les esprits les plus avancés, ce grand principe philosophique acquiert aussitôt une plénitude décisive, quoique les lois effectives de la plupart des cas particuliers doivent rester longtemps ignorées ; parce qu’une irrésistible analogie applique alors d’avance à tous les phénomènes de chaque ordre ce qui n’a été constaté, que pour quelques-uns d’entre eux, pourvu qu’ils aient une importance convenable.\par
Après avoir considéré l’esprit positif relativement aux objets extérieurs de nos spéculations, il faut achever de le caractériser en appréciant aussi sa destination intérieure, pour la satisfaction continue de nos propres besoins, soit qu’ils concernent la vie contemplative, ou la vie active.\par
Quoique les nécessités purement mentales soient sans doute, les moins énergiques de toutes celles inhérentes à notre nature, leur existence directe et permanente est néanmoins incontestable chez toutes les intelligences : elles y constituent la première stimulation indispensable à nos divers efforts philosophiques, trop souvent attribués surtout aux impulsions pratiques, qui les développent beaucoup, il est vrai, mais ne pourraient les faire naître. Ces exigences intellectuelles, relatives, comme toutes les autres, à l’exercice régulier des fonctions correspondantes, réclament toujours une heureuse combinaison de stabilité et d’activité, d’où résultent les besoins simultanés d’ordre et de progrès, ou de liaison et d’extension. Pendant la longue enfance de l’Humanité, les conceptions théologico-métaphysiques pouvaient seules, suivant nos explications antérieures, satisfaire provisoirement à cette double condition fondamentale, quoique d’une manière extrêmement imparfaite. Mais quand la raison humaine est enfin assez mûrie pour renoncer franchement aux recherches inaccessibles et circonscrire sagement son activité dans le domaine vraiment appréciable à nos facultés, la philosophie positive lui procure certainement une satisfaction beaucoup plus complète, à tous égards, aussi bien que plus réelle, de ces deux besoins élémentaires. Telle est, évidemment, en effet sous ce nouvel aspect, la destination directe des lois qu’elle découvre sur les divers phénomènes, et de la prévision rationnelle qui en est inséparable. Envers chaque ordre d’événements, ces lois doivent, à cet égard, être distinguées en deux sortes, selon qu’elles lient par similitude ceux qui coexistent, ou par filiation ceux qui se succèdent. Cette indispensable distinction correspond essentiellement, pour le monde extérieur, à celle qu’il nous offre toujours spontanément entre les deux états corrélatifs d’existence et de mouvement ; d’où résulte, dans toute science réelle, une indifférence fondamentale entre l’appréciation statique et l’appréciation dynamique d’un sujet quelconque. Les deux genres de relations contribuent également à expliquer les phénomènes, et conduisent pareillement à les prévoir, quoique les lois d’harmonie semblent d’abord destinées surtout à l’explication et les lois de succession à la prévision. Soit qu’il s’agisse, en effet, d’expliquer ou de prévoir, tout se réduit toujours à lier : toute liaison réelle, d’ailleurs statique ou dynamique, découverte, entre deux phénomènes quelconques, permet à la fois de les expliquer et de les prévoir l’un après l’autre ; car la prévision scientifique convient évidemment au présent, et même au passé, aussi bien qu’à l’avenir, consistant sans cesse à connaître un fait indépendamment de son exploration directe, en vertu de ses relations avec d’autres déjà donnés. Ainsi, par exemple, l’assimilation démontrée entre la gravitation céleste et la pesanteur terrestre a conduit, d’après les variations prononcées de la première, à prévoir de faibles variations de la seconde, que l’observation immédiate ne pouvait suffisamment dévoiler, quoiqu’elle les ait ensuite confirmées ; de même, en sens inverse, la correspondance, anciennement observée, entre la période élémentaire des marées et le jour lunaire, s’est trouvée expliquée aussitôt qu’on a reconnu l’élévation des eaux en chaque point comme résultant du passage de la lune au méridien local. Tous nos vrais besoins logiques convergent donc essentiellement vers cette commune destination : consolider, autant que possible, par nos spéculations systématiques, l’unité spontanée de notre entendement, en constituant la continuité et l’homogénéité de nos diverses conceptions, de manière à satisfaire également aux exigences simultanées de l’ordre et du progrès, en nous faisant retrouver la constance au milieu de la variété. Or, il est évident que, sous cet aspect fondamental, la philosophie positive comporte nécessairement, chez les esprits bien préparés, une aptitude très supérieure à celle qu’a pu jamais offrir la philosophie théologico-métaphysique. En considérant même celle-ci aux temps de son plus grand ascendant, à la fois mental et social, c’est-à-dire, à l’état polythéique, l’unité intellectuelle s’y trouvait certainement constituée d’une manière beaucoup moins complète et moins stable que ne le permettra prochainement l’universelle prépondérance de l’esprit positif, quand il sera enfin étendu habituellement aux plus éminentes spéculations. Alors, en effet, régnera partout, sous divers modes, et à différents degrés, cette admirable constitution logique, dont les plus simples études peuvent seules nous donner aujourd’hui une juste idée, où la liaison et l’extension, chacune pleinement garantie, se trouvent, en outre, spontanément solidaires. Ce grand résultat philosophique n’exige d’ailleurs d’autre condition nécessaire que l’obligation permanente de restreindre toutes nos spéculations aux recherches vraiment accessibles, en considérant ces relations réelles, soit de similitude, soit de succession, comme ne pouvant elles-mêmes constituer pour nous que de simples faits généraux, qu’il faut toujours tendre à réduire au moindre nombre possible, sans que le mystère de leur production puisse jamais être aucunement pénétré, conformément au caractère fondamental de l’esprit positif. Mais cette constance effective des liaisons naturelles nous est seule vraiment appréciable, elle seule aussi suffit pleinement à nos véritables besoins, soit de contemplation, soit de direction.\par
Il importe néanmoins de reconnaître, en principe, que, sous le régime positif, l’harmonie de nos, conceptions se trouve nécessairement limitée, à un certain degré, par l’obligation fondamentale de leur réalité, c’est-à-dire d’une insuffisante conformité à des types indépendants de nous. Dans son aveugle instinct de liaison, notre intelligence aspire presque à pouvoir toujours lier entre eux deux phénomènes quelconques, simultanés ou successifs ; mais l’étude du monde extérieur démontre, au contraire, que beaucoup de ces rapprochements seraient purement chimériques, et qu’une foule d’événements s’accomplissent continuellement sans aucune vraie dépendance mutuelle ; en sorte que ce penchant indispensable a autant besoin qu’aucun autre d’être réglé d’après une saine appréciation générale. Longtemps habitué à une sorte d’unité de doctrine, quelque vague et illusoire qu’elle dût être, sous l’empire des fictions théologiques et des entités métaphysiques, l’esprit humain, en passant à l’état positif, a d’abord tenté de réduire tous les divers ordres de phénomènes à une seule loi commune. Mais tous les essais accomplis pendant les deux derniers siècles pour obtenir une explication universelle de la nature n’ont abouti qu’à discréditer radicalement une telle entreprise, désormais abandonnée aux intelligences mal cultivées. Une judicieuse exploration du monde extérieur l’a représenté comme étant beaucoup moins lié que ne le suppose ou ne le désire notre entendement, que sa propre faiblesse dispose davantage à multiplier des relations favorables à sa marche, et surtout à son repos. Non seulement les six catégories fondamentales que nous distinguerons ci-dessous entre les phénomènes naturels ne sauraient certainement être toutes ramenées à une seule loi universelle ; mais il y a tout lieu d’assurer maintenant que l’unité d’explication, encore poursuivie par tant d’esprits sérieux envers chacune d’elles prise à part, nous est finalement interdite, même dans ce domaine beaucoup plus restreint. L’astronomie a fait naître, sous ce rapport, des espérances trop empiriques, qui ne sauraient se réaliser jamais pour les phénomènes plus compliqués, pas seulement quant à la physique proprement dite, dont les cinq branches principales resteront toujours distinctes entre elles, malgré leurs incontestables relations. On est souvent disposé à s’exagérer beaucoup les inconvénients logiques d’une telle dispersion nécessaire, parce qu’on apprécie mal les avantages réels que présente la transformation des inductions en déductions. Néanmoins, il faut franchement reconnaître cette impossibilité directe de tout ramener à une seule loi positive comme une grave imperfection, suite inévitable de la condition humaine, qui nous force d’appliquer une très faible intelligence à un univers très compliqué.\par
Mais, cette incontestable nécessité, qu’il importe de reconnaître, afin d’éviter toute vaine déperdition de forces mentales, n’empêche nullement la science réelle de comporter, sous un autre aspect, une suffisante unité philosophique, équivalente à celles que constituèrent passagèrement la théologie ou la métaphysique, et d’ailleurs très supérieure, aussi bien en stabilité qu’en plénitude. Pour en sentir la possibilité et en apprécier la nature, il faut d’abord recourir à la lumineuse distinction générale ébauchée par Kant entre les deux points de vue {\itshape objectif} et {\itshape subjectif}, propres à une étude quelconque. Considérée sous le premier aspect, c’est-à-dire quant à la destination extérieure de nos théories, comme exacte représentation du monde réel, notre science n’est certainement pas susceptible d’une pleine systématisation, par suite d’une inévitable diversité entre les phénomènes fondamentaux. En ce sens, nous ne devons chercher d’autre unité que celle de la méthode positive envisagée dans son ensemble, sans prétendre à une véritable unité scientifique, en aspirant seulement à l’homogénéité et à la convergence des différentes doctrines. Il en est tout autrement sous l’autre aspect, c’est-à-dire, quant à la source intérieure des théories humaines, envisagées comme des résultats naturels de notre évolution mentale, à la fois individuelle et collective, destinés à la satisfaction normale de nos propres besoins quelconques. Ainsi rapportées, non à l’univers, mais à l’homme, ou plutôt à l’Humanité, nos connaissances réelles tendent, au contraire, avec une évidente spontanéité, vers une entière systématisation, aussi bien scientifique que logique. On ne doit plus alors, concevoir, au fond, qu’une seule science, la science humaine, ou plus exactement sociale, dont notre, existence constitue à la fois le principe et le but, et dans laquelle vient naturellement se fondre l’étude rationnelle du monde extérieur, au double titre d’élément nécessaire et de préambule fondamental, également indispensable quant à la méthode et quant à la doctrine, comme je l’expliquerai ci-dessous. C’est uniquement ainsi que nos connaissances positives peuvent former un véritable système de manière à offrir un caractère pleinement satisfaisant. L’astronomie elle-même, quoique objectivement plus parfaite que les autres branches de la philosophie naturelle, à raison de sa simplicité supérieure, n’est vraiment telle que sous cet aspect humain : car l’ensemble de ce Traité fera nettement sentir qu’elle devrait, au contraire, être jugée très imparfaite si on la rapportait à l’univers et non à l’homme ; puisque toutes nos études réelles y sont nécessairement bornées à notre monde, qui pourtant ne constitue, qu’un minime élément de l’univers, dont l’exploration nous est essentiellement interdite. Telle est donc la disposition générale qui doit finalement prévaloir dans la philosophie vraiment positive, non seulement quant aux théories directement relatives à l’homme et à la société, mais aussi envers celles qui concernent les plus simples phénomènes, les plus éloignés, en apparence, de cette commune appréciation : concevoir toutes nos spéculations comme des produits de notre intelligence, destinés à satisfaire nos divers besoins essentiels, en ne s’écartant jamais de l’homme qu’afin d’y mieux revenir, après avoir étudié les autres phénomènes en tant qu’indispensables à connaître, soit pour développer nos forces, soit pour apprécier notre nature et notre condition. On peut dès lors apercevoir comment la notion prépondérante de l’Humanité doit nécessairement constituer, dans l’état positif, une pleine systématisation mentale, au moins équivalente à celle qu’avait finalement comportée l’âge théologique d’après la grande conception de Dieu, si faiblement remplacée ensuite, à cet égard, pendant la transition métaphysique, par la vague pensée de la Nature.\par
Après avoir ainsi caractérisé l’aptitude spontanée de l’esprit positif à constituer l’unité finale de notre entendement, il devient aisé de compléter cette explication fondamentale en l’étendant de l’individu à l’espèce. Cette indispensable extension était jusqu’ici essentiellement impossible aux philosophes modernes, qui, n’ayant pu suffisamment sortir eux-mêmes de l’état métaphysique, ne se sont jamais installés au point de vue social, seul susceptible néanmoins d’une pleine réalité, soit scientifique, soit logique, puisque l’homme ne se développe point isolément, mais collectivement. En écartant, comme radicalement stérile, ou plutôt profondément nuisible, cette vicieuse abstraction de nos psychologues ou idéologues, la tendance systématique que nous venons d’apprécier dans l’esprit positif acquiert enfin toute son importance, parce qu’elle indique en lui le vrai fondement philosophique de la sociabilité humaine, en tant du moins que celle-ci dépend de l’intelligence, dont l’influence capitale, quoique nullement exclusive, ne saurait y être, contestée. C’est, en effet, le même problème humain, à divers degrés de difficulté, que de constituer l’unité logique de chaque entendement isolé ou d’établir une convergence durable entre des entendements distincts, dont le nombre ne saurait essentiellement influer que sur la rapidité de l’opération. Aussi, en tout temps, celui qui a pu devenir suffisamment conséquent a-t-il acquis, par cela même, la faculté de rallier graduellement les autres, d’après la similitude fondamentale de notre espèce. La philosophie théologique n’a été, pendant l’enfance de l’Humanité, la seule propre à systématiser la société que comme étant alors la source exclusive d’une certaine harmonie mentale. Si donc le privilège de la cohérence logique a désormais irrévocablement passé à l’esprit positif, ce qui ne peut guère être sérieusement contesté, il faut dès lors reconnaître aussi en lui l’unique principe effectif de cette grande communion intellectuelle qui devient la base nécessaire de toute véritable association humaine, quand elle est convenablement liée aux deux autres conditions fondamentales, une suffisante conformité de sentiments, et une certaine convergence d’intérêts. La déplorable situation philosophique de l’élite de l’Humanité suffirait aujourd’hui pour dispenser, à cet égard, de toute discussion, puisqu’on n’y observe plus de vraie communauté d’opinions que sur les sujets déjà ramenés à des théories positives, et qui, malheureusement, ne sont pas, à beaucoup près, les plus importants, Une appréciation directe et spéciale, qui serait ici déplacée, fait d’ailleurs sentir aisément que la philosophie positive peut seule réaliser graduellement ce noble projet d’association universelle que le catholicisme avait, au Moyen Âge, prématurément ébauché, mais qui était, au fond, nécessairement incompatible, comme l’expérience l’a pleinement constaté, avec la nature théologique de sa philosophie, laquelle instituait une trop faible cohérence logique pour comporter une telle efficacité sociale.\par
L’aptitude fondamentale de l’esprit positif étant assez caractérisée désormais par rapport à la vie spéculative, il ne nous reste plus qu’à l’apprécier aussi envers la vie active, qui, sans pouvoir montrer en lui aucune propriété vraiment nouvelle, manifeste, d’une manière beaucoup plus complète et surtout plus décisive, l’ensemble des attributs que nous lui avons reconnus. Quoique les conceptions théologiques aient été, même sous cet aspect, longtemps nécessaires afin d’éveiller et de soutenir l’ardeur de l’homme par l’espoir indirect d’une sorte d’empire illimité, c’est pourtant à cet égard que l’esprit humain a dû témoigner d’abord sa prédilection finale pour les connaissances réelles. C’est surtout, en effet, comme base rationnelle de l’action de l’Humanité sur le monde extérieur que l’étude positive de la nature commence aujourd’hui à être universellement goûtée. Rien n’est plus sage, au fond, que ce jugement vulgaire et spontané ; car, une telle destination, lorsqu’elle est convenablement appréciée, appelle nécessairement, par le plus heureux résumé, tous les grands caractères du véritable esprit philosophique, aussi bien quant à la rationalité que quant à la positivité. L’ordre naturel résulté, en chaque cas pratique, de l’ensemble des lois des phénomènes correspondants, doit évidemment nous être d’abord bien connu pour que nous puissions ou le modifier à notre avantage, ou du moins y adapter notre conduite, si toute intervention humaine y est impossible, comme envers les événements célestes.\par
Une telle application est surtout propre à rendre familièrement appréciable cette prévision rationnelle que nous avons vue constituer, à tous égards, le principal caractère de la vraie science ; car, la pure érudition, où les connaissances, réelles mais incohérentes, consistent en faits et non en lois, ne pourrait, évidemment, suffire à diriger notre activité : il serait superflu d’insister ici sur une explication aussi peu contestable. Il est vrai que l’exorbitante prépondérance maintenant accordée aux intérêts matériels a trop souvent conduit à comprendre cette liaison nécessaire de façon à compromettre gravement l’avenir scientifique, en tendant à restreindre les spéculations positives aux seules recherches d’une utilité immédiate. Mais cette aveugle disposition ne résulte que d’une manière fausse et étroite de concevoir la grande relation de la science, à l’art, faute d’avoir assez profondément apprécié l’une et l’autre. L’étude de l’astronomie est la plus propre de toutes à rectifier une telle tendance, soit parce que sa simplicité supérieure permet d’en mieux saisir l’ensemble, soit en vertu de la spontanéité plus intime des applications correspondantes, qui, depuis vingt siècles, s’y trouvent évidemment liées aux plus sublimes spéculations, comme ce Traité le fera nettement sentir. Mais il importe surtout de bien reconnaître, à cet égard, que la relation fondamentale entre la science et l’art n’a pu jusqu’ici être convenablement conçue, même chez les meilleurs esprits, par une suite nécessaire de l’insuffisante extension de la philosophie naturelle, restée encore étrangère aux recherches les plus importantes et les plus difficiles, celles qui concernent directement la société humaine. En effet, la conception rationnelle de l’action de l’homme sur la nature est ainsi demeurée essentiellement bornée au monde inorganique, d’où résulterait une trop imparfaite excitation scientifique. Quand cette immense lacune aura été suffisamment comblée, comme elle commence à l’être aujourd’hui, on pourra sentir l’importance fondamentale de cette grande destination pratique pour stimuler habituellement, et souvent même pour mieux diriger, les plus éminentes spéculations, sous la seule condition normale d’une constante positivité. Car, l’art ne sera plus alors uniquement géométrique, mécanique ou chimique, etc., mais aussi et surtout politique et moral, la principale action exercée par l’Humanité devant, à tous égards, consister dans l’amélioration continue de sa propre nature individuelle ou collective, entre les limites qu’indique, de même qu’en tout autre cas, l’ensemble des lois réelles. Lorsque cette solidarité spontanée de la science avec l’art aura pu ainsi être convenablement organisée, on ne peut douter que, bien loin de tendre aucunement à restreindre les saines spéculations philosophiques, elle leur assignerait, au contraire, un office final trop supérieur à leur portée effective, si d’avance on n’avait reconnu, en principe général, l’impossibilité de jamais rendre l’art purement rationnel, c’est-à-dire d’élever nos prévisions théoriques au véritable niveau de nos besoins pratiques. Dans les arts même les plus simples et les plus parfaits, un développement direct et spontané reste constamment indispensable, sans que les indications scientifiques puissent, en aucun cas, y suppléer complètement. Quelque satisfaisantes, par exemple, que soient devenues nos prévisions astronomiques, leur précision est encore, et sera probablement toujours, inférieure à nos justes exigences pratiques, comme j’aurai souvent lieu de l’indiquer.\par
Cette tendance spontanée à constituer directement une entière harmonie entre la vie spéculative et la vie active doit être finalement regardée comme le plus heureux privilège de l’esprit positif, dont aucune autre propriété ne peut aussi bien manifester le vrai caractère et faciliter l’ascendant réel. Notre ardeur spéculative se trouve ainsi entretenue, et même dirigée, par une puissante stimulation continue, sans laquelle l’inertie naturelle de notre intelligence la disposerait souvent à satisfaire ses faibles besoins théoriques par des explications faciles, mais insuffisantes, tandis que la pensée de l’action finale rappelle toujours la condition d’une précision convenable. En même temps, cette grande destination pratique complète et circonscrit, en chaque cas, la prescription fondamentale relative à la découverte des lois naturelles, en tendant à déterminer, d’après les exigences de l’application, le degré de précision et d’étendue de notre prévoyance rationnelle, dont la juste mesure ne pourrait, en général, être autrement fixée. Si d’une part, la perfection scientifique ne saurait dépasser une telle limite, au-dessous de laquelle, au contraire, elle se trouvera réellement toujours, elle ne pourrait, d’une autre part, la franchir sans tomber aussitôt dans une appréciation trop minutieuse, non moins chimérique que stérile, et qui même compromettrait finalement tous les fondements de la véritable science, puisque nos lois ne peuvent jamais représenter les phénomènes qu’avec une certaine approximation, au-delà de laquelle il serait aussi dangereux qu’inutile de pousser nos recherches. Quand cette relation fondamentale de la science à l’art sera convenablement systématisée, elle tendra quelquefois, sans doute, à discréditer des tentatives théoriques dont la stérilité radicale serait incontestable ; mais loin d’offrir aucun inconvénient réel, cette inévitable disposition deviendra dès lors très favorable à nos vrais intérêts spéculatifs, en prévenant cette vaine déperdition de nos faibles forces mentales qui résulte trop souvent aujourd’hui d’une aveugle spécialisation. Dans l’évolution préliminaire de l’esprit positif, il a dû s’attacher partout aux questions quelconques qui lui devenaient accessibles, sans trop s’enquérir de leur importance finale, dérivée de leur relation propre à un ensemble qui ne pouvait d’abord être aperçu. Mais cet instinct provisoire, faute duquel la science eût souvent manqué alors d’une convenable alimentation, doit finir par se subordonner habituellement à une juste appréciation systématique, aussitôt que la pleine maturité de l’état positif aura suffisamment permis de saisir toujours les vrais rapports essentiels de chaque partie avec le tout, de manière à offrir. constamment une large destination aux plus éminentes recherches, en évitant, néanmoins toute spéculation puérile.\par
Au sujet de cette intime harmonie entre la science et l’art, il importe enfin de remarquer spécialement l’heureuse tendance qui en résulte pour développer et consolider l’ascendant social de la saine philosophie, par une suite spontanée de la vie industrielle dans notre civilisation moderne. La philosophie théologique ne pouvait réellement convenir qu’à ces temps nécessaires de sociabilité préliminaire, où l’activité humaine doit être essentiellement militaire, afin de préparer graduellement une association normale et complète, qui était d’abord impossible, suivant la théorie historique que j’ai établie ailleurs. Le polythéisme s’adaptait surtout au système de conquête de l’antiquité, et le monothéisme à l’organisation défensive du Moyen Âge. En faisant de plus en plus prévaloir la vie industrielle, la sociabilité moderne doit donc puissamment seconder la grande révolution mentale qui aujourd’hui élève définitivement notre intelligence du régime théologique au régime positif. Non seulement cette active tendance journalière à l’amélioration pratique de la condition humaine est nécessairement peu compatible avec les préoccupations religieuses, toujours relatives, surtout sous le monothéisme, à une tout autre destination. Mais en outre, une telle activité est de nature à susciter finalement une opposition universelle, aussi radicale que spontanée, à toute philosophie théologique. D’une part, en effet, la vie industrielle est, au fond, directement contraire à tout optimisme providentiel, puisqu’elle suppose nécessairement que l’ordre naturel est assez imparfait pour exiger sans cesse l’intervention humaine, tandis que la théologie n’admet logiquement d’autre moyen de la modifier que de solliciter un appui surnaturel. En second lieu, cette opposition, inhérente à l’ensemble de nos conceptions industrielles, se reproduit continuellement, sous formes très variées, dans l’accomplissement spécial de nos opérations, où ‘ nous devons envisager le monde extérieur, non comme dirigé par des volontés quelconques, mais comme soumis à des lois, susceptibles de nous permettre une suffisante prévoyance, sans laquelle notre activité pratique ne comporterait aucune base rationnelle. Ainsi, la même corrélation fondamentale qui rend la vie industrielle si favorable à l’ascendant philosophique de l’esprit positif lui imprime, sous un autre aspect, une tendance anti-théologique, plus ou moins prononcée, mais tôt ou tard inévitable, quels qu’aient pu être les efforts continus de la sagesse sacerdotale pour contenir ou tempérer le caractère anti-industriel de la philosophie initiale, avec laquelle la vie guerrière était seule suffisamment conciliable. Telle est l’intime solidarité qui fait involontairement participer depuis longtemps tous les esprits modernes, même les plus grossiers et les plus rebelles, au remplacement graduel de l’antique philosophie théologique par une philosophie pleinement positive, seule susceptible désormais d’un véritable ascendant social.\par
Nous sommes ainsi conduits à compléter enfin l’appréciation directe du véritable esprit philosophique par une dernière explication qui, quoique étant surtout négative, devient réellement indispensable aujourd’hui pour achever de caractériser suffisamment la nature et les conditions de la grande rénovation mentale maintenant nécessaire à l’élite de l’Humanité, en manifestant directement l’incompatibilité finale des conceptions positives avec toutes les opinions. théologiques quelconques, aussi bien monothéiques que polythéiques ou fétichiques. Les diverses considérations indiquées dans ce Discours ont déjà démontré implicitement l’impossibilité d’aucune conciliation durable entre les deux philosophies, soit quant à la méthode, ou à la doctrine ; en sorte que toute incertitude à ce sujet peut être ici facilement dissipée. Sans doute, la science et la théologie ne sont pas d’abord en opposition ouverte, puisqu’elles ne se proposent point les mêmes questions ; c’est ce qui a longtemps permis l’essor partiel de l’esprit positif malgré l’ascendant général de l’esprit théologique, et même, à beaucoup d’égards, sous sa tutelle préalable. Mais quand la positivité rationnelle, bornée d’abord à d’humbles recherches mathématiques, que la théologie avait dédaigné d’atteindre spécialement, a commencé à s’étendre à l’étude directe de la nature, surtout par les théories astronomiques, la collision est devenue inévitable, quoique latente, en vertu du contraste fondamental, à la fois scientifique et logique, dès lors progressivement développé entre les deux ordres d’idées. Les motifs logiques d’après lesquels la science s’interdit radicalement les mystérieux problèmes dont la théologie s’occupe essentiellement, sont eux-mêmes de nature à discréditer tôt ou tard, chez tous les bons esprits, des spéculations qu’on n’écarte que comme étant, de toute nécessité, inaccessibles à la raison humaine. En outre, la sage réserve avec laquelle l’esprit positif procède graduellement envers des sujets très faciles doit faire indirectement apprécier la folle témérité de l’esprit théologique à l’égard des plus difficiles questions. Toutefois, c’est surtout par les doctrines que l’incompatibilité des deux philosophies doit éclater chez la plupart des intelligences, trop peu touchées d’ordinaire des simples dissidences de méthode, quoique celles-ci soient au fond les plus graves, comme étant la source nécessaire de toutes les autres. Or, sous ce nouvel aspect, on ne peut méconnaître l’opposition radicale des deux ordres de conceptions, où les mêmes phénomènes sont tantôt attribués à des volontés directrices, et tantôt ramenés à des lois invariables. La mobilité irrégulière, naturellement inhérente à toute idée de volonté, ne peut aucunement s’accorder avec la constance des relations réelles. Ainsi à mesure que les lois physiques ont été connues, l’empire des volontés surnaturelles s’est trouvé de plus en plus restreint, étant toujours consacré surtout aux phénomènes dont les lois restaient ignorées. Une telle incompatibilité devient directement évidente quand on oppose la prévision rationnelle, qui constitue le principal caractère de la véritable science, à la divination par révélation spéciale, que la théologie doit représenter comme offrant le seul moyen légitime de connaître l’avenir. Il est vrai que l’esprit positif, parvenu à son entière maturité, tend aussi à subordonner la volonté elle-même à de véritables lois, dont l’existence est, en effet, tacitement supposée par la raison vulgaire, puisque les efforts pratiques pour modifier et prévoir les volontés humaines ne sauraient avoir sans cela aucun fondement raisonnable. Mais une telle notion ne conduit nullement à concilier les deux modes opposés suivant lesquels la science et la théologie conçoivent nécessairement la direction effective des divers phénomènes. Car une semblable prévision et la conduite qui en résulte exigent évidemment une profonde connaissance réelle de l’être au sein duquel les volontés se produisent. Or, ce fondement préalable ne saurait provenir que d’un être au moins égal, jugeant ainsi par similitude ; on ne peut le concevoir de la part d’un inférieur, et la contradiction augmente avec l’inégalité de nature. Aussi la théologie a-t-elle toujours repoussé la prétention de pénétrer aucunement les desseins providentiels, de même qu’il serait absurde de supposer aux derniers animaux la faculté de prévoir les volontés de l’homme ou des autres animaux supérieurs. C’est néanmoins à cette folle hypothèse qu’on se trouverait nécessairement conduit pour concilier finalement l’esprit théologique avec l’esprit positif.\par
Historiquement considérée, leur opposition radicale, applicable à toutes les phases essentielles de la philosophie initiale, est généralement admise depuis longtemps envers celles que les populations les plus avancées ont complètement franchies. Il est même certain que, à leur égard, on exagère beaucoup une telle incompatibilité, par suite de ce dédain absolu qu’inspirent aveuglément nos habitudes monothéiques pour les deux états antérieurs du régime théologique. La saine philosophie, toujours obligée d’apprécier le mode nécessaire suivant lequel chacune des grandes phases successives de l’Humanité a effectivement concouru à notre évolution fondamentale, rectifiera soigneusement ces injustes préjugés, qui empêchent toute véritable théorie historique. Mais, quoique le polythéisme, et même le fétichisme, aient d’abord secondé réellement l’essor spontané de l’esprit d’observation, on doit pourtant reconnaître qu’ils ne pouvaient être vraiment compatibles avec le sentiment graduel de l’invariabilité des relations physiques, aussitôt qu’il a pu acquérir une certaine consistance systématique. Aussi doit-on concevoir cette inévitable opposition comme la principale source secrète des diverses transformations qui ont successivement décomposé la philosophie théologique en la réduisant de plus en plus. C’est ici le lieu de compléter, à ce sujet, l’indispensable explication indiquée au début de ce {\itshape Discours}, où cette dissolution graduelle a été spécialement attribuée à l’état métaphysique proprement dit, qui, au fond, n’en pouvait être que le simple organe, et jamais le véritable agent. Il faut, en effet, remarquer que l’esprit positif, par suite du défaut de généralité qui devait caractériser sa lente évolution partielle, ne pouvait convenablement formuler ses propres tendances philosophiques, à peine devenues directement sensibles pendant nos derniers siècles. De là résultait la nécessité spéciale de l’intervention métaphysique, qui pouvait seule systématiser convenablement l’opposition spontanée de la science naissante à l’antique théologie. Mais, quoiqu’un tel office ait dû faire exagérer beaucoup l’importance effective de cet esprit transitoire, il est cependant facile de reconnaître que le progrès naturel des connaissances réelles donnait seul une sérieuse consistance à sa bruyante activité. Ce progrès continu, qui même avait d’abord déterminé, au fond, la transformation du fétichisme en polythéisme, a surtout constitué ensuite la source essentielle de la réduction du polythéisme au monothéisme. La collision ayant dû s’opérer principalement par les théories astronomiques, ce Traité me fournira l’occasion naturelle de caractériser le degré précis de leur développement auquel il faut attribuer, en réalité, l’irrévocable décadence mentale du régime polythéique, que nous reconnaîtrons alors logiquement incompatible avec la fondation décisive de l’astronomie mathématique par l’école de Thalès.\par
L’étude rationnelle d’une telle opposition démontre clairement qu’elle ne pouvait se borner à la théologie ancienne, et qu’elle a dû s’étendre ensuite au Monothéisme lui-même, quoique son énergie dût décroître avec sa nécessité, à mesure que l’esprit théologique continuait à déchoir par suite du même prodige spontané. Sans doute, cette extrême phase de la philosophie initiale était beaucoup moins contraire que les précédentes à l’essor des {\itshape connaissances} réelles, qui n’y rencontraient plus, à chaque pas, la dangereuse concurrence d’une explication surnaturelle spécialement formulée. Aussi est-ce surtout sous ce régime monothéique qu’a dû s’accomplir l’évolution préliminaire de l’esprit positif. Mais l’incompatibilité, pour être moins explicite et plus tardive, n’en restait pas moins finalement inévitable, même avant le temps où la nouvelle philosophie serait devenue assez générale pour prendre un caractère vraiment organique, en remplaçant irrévocablement la théologie dans son office social aussi bien que dans sa destination mentale. Comme le conflit a dû encore s’opérer surtout par l’astronomie, je démontrerai ici avec précision quelle évolution plus avancée a étendu nécessairement jusqu’au plus simple monothéisme son opposition radicale, auparavant bornée au polythéisme proprement dit : on reconnaîtra alors que cette inévitable influence résulte de la découverte du double mouvement de la Terre bientôt suivie de la fondation de la mécanique céleste. Dans l’état présent de la raison humaine, on peut assurer que le régime monothéique, longtemps favorable à l’essor primitif des connaissances réelles, entrave profondément la marche systématique qu’elles doivent prendre désormais, en empêchant le sentiment fondamental de l’invariabilité des lois physiques d’acquérir enfin son indispensable plénitude philosophique. Car, la pensée continue d’une subite perturbation arbitraire dans l’économie naturelle doit toujours rester inséparable, au moins virtuellement, de toute théologie quelconque, même réduite autant que possible. Sans un tel obstacle, en effet, qui ne peut cesser que par l’entière désuétude de l’esprit théologique, le spectacle journalier de l’ordre réel aurait déjà déterminé une adhésion universelle au principe fondamental de la philosophie positive.\par
Plusieurs siècles avant que l’essor scientifique permît d’apprécier directement cette opposition radicale, la transition métaphysique avait tenté, sous sa secrète impulsion, de restreindre, au sein même du monothéisme, l’ascendant de la théologie, en faisant abstraitement prévaloir, dans la dernière période du Moyen Âge, la célèbre doctrine scolastique qui assujettit l’action effective du moteur suprême à des lois invariables, qu’il aurait primitivement établies en s’interdisant de jamais les changer. Mais cette sorte de transaction {\itshape spontanée} entre le principe théologique et le principe positif ne comportait, évidemment, qu’une existence passagère, propre à faciliter davantage le déclin continu de l’un et le triomphe graduel de l’autre. Son empire était même essentiellement borné aux esprits cultivés ; car, tant que la foi subsista réellement, {\itshape l’instinct} populaire dut toujours repousser avec énergie une conception qui, au fond, tendait à annuler le pouvoir providentiel, en le condamnant à une sublime inertie, qui laissait toute l’activité habituelle à la grande entité métaphysique, la Nature étant ainsi régulièrement associée au gouvernement universel, à titre de ministre obligé et responsable, auquel devaient s’adresser désormais la plupart des plaintes et des vœux. On voit que, sous tous les aspects essentiels, cette {\itshape conception ressemble} beaucoup à celle que la situation moderne a fait de plus en plus prévaloir au sujet de la royauté constitutionnelle ; et cette analogie n’est nullement fortuite, puisque le type théologique a fourni, en effet, la base rationnelle du type politique. Cette doctrine contradictoire, qui ruine l’efficacité sociale du principe théologique, sans consacrer l’ascendant fondamental du principe positif, ne saurait correspondre à aucun état vraiment normal et durable : elle constitue seulement le plus puissant des moyens de transition propres au dernier office nécessaire de l’esprit métaphysique.\par
Enfin, l’incompatibilité nécessaire de la science avec la théologie a dû se manifester aussi sous une autre forme générale, spécialement adaptée à l’état monothéique, en faisant de plus en plus ressortir l’imperfection radicale de l’ordre réel, ainsi opposée à l’inévitable optimisme providentiel. Cet optimisme a dû, sans doute, rester longtemps conciliable avec l’essor spontané des connaissances positives, parce qu’une première analyse de la nature devait alors inspirer partout une naïve admission pour le mode d’accomplissement des principaux phénomènes qui constituent l’ordre effectif. Mais cette disposition initiale tend ensuite à disparaître, non moins nécessairement, à mesure que l’esprit positif, prenant un caractère de plus en plus systématique, substitue peu à peu, au dogme des causes finales, le principe des conditions d’existence, qui en offre, à un plus haut degré, toutes les propriétés logiques, sans présenter aucun de ses graves dangers scientifiques. On cesse alors de s’étonner que la constitution des êtres naturels se trouve, en chaque cas, disposée de manière à permettre l’accomplissement de leurs phénomènes effectifs. En étudiant avec soin cette inévitable harmonie, dans l’unique dessein de la mieux connaître, on finit ensuite par remarquer les profondes imperfections que présente, à tous égards, l’ordre réel, presque toujours inférieur en sagesse à l’économie artificielle qu’établit notre faible intervention humaine dans son domaine borné. Comme ces vices naturels doivent être d’autant plus grands qu’il s’agit de phénomènes plus compliqués, les indications irrécusables que nous offrira, sous cet aspect, l’ensemble de l’astronomie, suffiront ici pour faire pressentir combien une pareille appréciation doit s’étendre, avec une nouvelle énergie philosophique, à toutes les autres parties essentielles de la, science réelle. Mais il importe surtout de comprendre, en général, au sujet d’une telle critique, qu’elle. n’a pas seulement une destination passagère, à titre de moyen anti-théologique. Elle se lie, d’une manière plus intime et plus durable, à l’esprit, fondamental de la philosophie positive, dans la relation générale entre la spéculation et l’action. Si, d’une part, notre active intervention permanente repose, avant tout, sur l’exacte connaissance de l’économie naturelle, dont notre économie artificielle ne doit constituer, à tous égards, que l’amélioration progressive, il n’est, pas moins certain, d’une autre part, que nous supposons ainsi l’imperfection nécessaire de cet ordre spontané, dont la modification graduelle constitue le but journalier de tous nos efforts individuels ou collectifs. Abstraction faite de toute critique passagère, la juste appréciation des divers inconvénients propres à la constitution effective du monde réel, doit être conçue désormais comme inhérente à l’ensemble de la philosophie positive, même envers les cas, inaccessibles à nos faibles moyens de perfectionnement, afin de mieux connaître soit notre condition fondamentale, soit la destination essentielle de notre activité continue.\par
Le concours spontané des diverses considérations générales indiquées dans ce discours suffit maintenant pour caractériser ici, sous tous les aspects principaux, le véritable esprit philosophique, qui, après une lente évolution préliminaire, atteint aujourd’hui son état systématique. Vu l’évidente obligation où nous sommes placés désormais de le qualifier habituellement par une courte dénomination spéciale, j’ai dû préférer celle à laquelle cette universelle préparation a procuré de plus en plus, pendant les trois derniers siècles, la précieuse propriété de résumer le mieux possible l’ensemble de ses attributs fondamentaux. Comme tous les termes vulgaires ainsi élevés graduellement à la dignité philosophique, le mot {\itshape positif} offre, dans nos langues occidentales, plusieurs acceptions distinctes, même en écartant le sens grossier qui d’abord s’y attache chez les esprits mal cultivés. Mais il importe de noter ici que toutes ces diverses significations conviennent également à la nouvelle philosophie générale, dont elles indiquent alternativement différentes propriétés caractéristiques : ainsi, cette apparente ambiguïté n’offrira désormais aucun inconvénient réel. Il y faudra voir, au contraire, l’un des principaux exemples de cette admirable condensation de formules qui, chez les populations avancées, réunit, sous une seule expression usuelle, plusieurs attributs distincts, quand la raison publique est parvenue à reconnaître leur liaison permanente.\par
Considéré d’abord dans son acception la plus ancienne et la plus commune, le mot positif désigne le {\itshape réel}, par opposition au chimérique : sous ce rapport, il convient pleinement au nouvel esprit philosophique, ainsi caractérisé d’après sa constante consécration aux recherches vraiment accessibles à notre intelligence, à l’exclusion permanente des impénétrables mystères dont s’occupait surtout son enfance. En un second sens, très voisin du précédent, mais pourtant distinct, ce terme fondamental indique le contraste de {\itshape l’utile} à l’oiseux : alors il rappelle, en philosophie, la destination nécessaire de toutes nos saines spéculations pour l’amélioration continue de notre vraie condition, individuelle et collective, au lieu de la vaine satisfaction d’une stérile curiosité. Suivant une troisième signification usuelle, cette heureuse expression est fréquemment employée à qualifier l’opposition entre la {\itshape certitude} et l’indécision : elle indique aussi l’aptitude caractéristique d’une telle philosophie à constituer spontanément l’harmonie logique dans l’individu et la communion spirituelle dans l’espèce entière, au lieu de ces doutes indéfinis et de ces débats interminables que devait susciter l’antique régime mental. Une quatrième acception ordinaire, trop souvent confondue avec la précédente, consiste à opposer le {\itshape précis} au vague : ce sens rappelle la tendance constante du véritable esprit philosophique à obtenir partout le degré de précision compatible avec la nature des phénomènes et conforme à l’exigence de nos vrais besoins ; tandis que l’ancienne manière de philosopher conduisait nécessairement à des opinions vagues, ne comportant une indispensable discipline que d’après une compression permanente, appuyée sur une autorité surnaturelle.\par
Il faut enfin remarquer spécialement une cinquième application, moins usitée que les autres, quoique d’ailleurs pareillement universelle, quand on emploie le mot positif comme le contraire de {\itshape négatif.} Sous cet aspect, il indique l’une des plus éminentes propriétés de la vraie philosophie moderne, en la montrant destinée surtout, par sa nature, non à détruire, mais à {\itshape organiser.} Les quatre caractères généraux que nous venons de rappeler la distinguent à la fois de tous les modes possibles, soit théologiques, soit métaphysiques, propres à la philosophie initiale. Cette dernière signification, en indiquant d’ailleurs une tendance continue du nouvel esprit philosophique, offre aujourd’hui une importance spéciale pour caractériser directement l’une de ses principales différences, non plus avec l’esprit théologique, qui fut longtemps organique, mais avec l’esprit métaphysique proprement dit, qui n’a jamais pu être que critique. Quelle qu’ait été, en effet, l’action dissolvante de la science réelle, cette influence fut toujours en elle purement indirecte et secondaire : son défaut même de systématisation empêchait jusqu’ici qu’il. en pût être autrement ; et le grand office organique qui lui est maintenant échu s’opposerait désormais à une telle attribution accessoire, qu’il tend d’ailleurs à rendre superflue. La saine philosophie écarte radicalement, il est vrai, toutes les questions nécessairement insolubles : mais, en motivant leur rejet, elle évite de rien nier à leur égard, ce qui serait contradictoire à cette désuétude systématique, par laquelle seule doivent s’éteindre toutes les opinions vraiment indiscutables. Plus impartiale et plus tolérante envers chacune d’elles, vu sa commune indifférence, que ne peuvent l’être leurs partisans opposés, elle s’attache à apprécier historiquement leur influence respective, les, conditions de leur durée et les motifs de leur décadence, sans prononcer jamais aucune négation absolue, même quand il s’agit, des doctrines les plus antipathiques à l’état présent de la raison humaine chez les populations d’élite. C’est ainsi qu’elle rend une scrupuleuse justice, non seulement aux divers systèmes de monothéisme autres que celui qui expire aujourd’hui parmi nous, mais aussi aux croyances polythéiques, ou même fétichiques, en les rapportant toujours aux phases correspondantes, de l’évolution fondamentale. Sous l’aspect dogmatique, elle professe d’ailleurs que les conceptions quelconques de notre imagination, quand leur nature les rend nécessairement inaccessibles à toute observation, ne sont pas plus susceptibles dès lors de négation que d’affirmation vraiment décisives. Personne, sans doute, n’a jamais démontré logiquement la non existence d’Apollon, de Minerve, etc., ni celle des fées orientales ou des diverses créations poétiques ; ce qui n’a nullement empêché l’esprit humain d’abandonner irrévocablement les dogmes antiques, quand ils ont enfin cessé de convenir à l’ensemble de sa situation.\par
Le seul caractère essentiel du nouvel esprit philosophique qui ne soit pas encore indiqué directement par le mot positif, consiste dans sa tendance nécessaire à substituer partout le {\itshape relatif à l’absolu.} Mais ce grand attribut, à la fois scientifique et logique, est tellement inhérent à la nature fondamentale des connaissances réelles, que sa considération générale ne tardera pas à se lier intimement aux divers aspects que cette formule combine déjà, quand le moderne régime intellectuel, jusqu’ici partiel et empirique, passera communément à l’état systématique. La cinquième acception que nous venons d’apprécier est surtout propre à déterminer cette, condensation du nouveau langage philosophique, dès lors pleinement constitué, d’après l’évidente affinité des deux propriétés. On conçoit, en effet, que la nature absolue des anciennes doctrines, soit théologiques, soit métaphysiques, déterminait nécessairement chacune d’elles à devenir négative envers toutes les autres, sous peine de dégénérer elle-même en un absurde éclectisme. C’est, au contraire, en vertu de son génie relatif que la nouvelle philosophie peut toujours apprécier la valeur propre des théories qui lui sont le plus opposées, sans toutefois aboutir jamais à aucune vaine concession, susceptible d’altérer la netteté de ses vues ou la fermeté de ses décisions. Il y a donc vraiment lieu de présumer, d’après l’ensemble d’une telle appréciation spéciale, que la formule employée ici pour qualifier habituellement cette philosophie définitive rappellera désormais, {\itshape à} tous les bons esprits, l’entière combinaison effective de ses diverses propriétés caractéristiques.\par
Quand on recherche l’origine fondamentale d’une telle manière de philosopher, on ne tarde pas à reconnaître que sa spontanéité élémentaire coïncide réellement avec les premiers exercices pratiques de la raison humaine : car, l’ensemble des explications indiquées dans ce {\itshape Discours} démontre clairement que tous ses attributs principaux, sont, au fond, les mêmes que ceux du bon sens universel. Malgré l’ascendant mental de la plus grossière théologie, la conduite journalière de la vie active a toujours dû susciter, envers chaque ordre de phénomènes, une certaine ébauche des lois naturelles et des prévisions correspondantes, dans quelques cas particuliers, qui seulement semblaient alors secondaires ou exceptionnels : or, tels sont, en effet, les germes nécessaires de la positivité, qui devait longtemps rester empirique avant de pouvoir devenir rationnelle. Il importe beaucoup de sentir que, sous tous les aspects essentiels, le véritable esprit philosophique consiste surtout dans l’extension systématique du simple bon sens à toutes les spéculations vraiment accessibles. Leur domaine est radicalement identique, puisque les plus grandes questions de la saine philosophie se rapportent partout aux phénomènes les plus vulgaires, envers lesquels les cas artificiels ne constituent qu’une préparation plus ou moins indispensable. Ce sont, de part et d’autre, le même point de départ expérimental, le même but de lier et de prévoir, la même préoccupation continue de la réalité, la même intention finale d’utilité. Toute leur différence essentielle consiste dans la généralité systématique de l’un, tenant à son abstraction nécessaire, opposée à l’incohérente spécialité de l’autre, toujours occupé du concret.\par
Envisagée sous l’aspect dogmatique, cette connexité fondamentale représente la science proprement dite comme un simple prolongement méthodique de la sagesse universelle. Aussi, bien loin de jamais remettre en question ce que celle-ci a vraiment décidé, les saines spéculations philosophiques doivent toujours emprunter à la raison commune leurs notions initiales, pour leur faire acquérir, par une élaboration systématique, un degré de généralité et de consistance qu’elles ne pouvaient obtenir spontanément.\par
Pendant tout le cours d’une telle élaboration, le contrôle permanent de cette vulgaire sagesse conserve d’ailleurs une haute importance, afin de prévenir, autant que possible, les diverses aberrations, par négligence ou par illusion, que suscite souvent l’état continu d’abstraction indispensable à l’activité philosophique. Malgré leur affinité nécessaire, le bon sens proprement dit doit surtout rester préoccupé de réalité et d’utilité, tandis que l’esprit spécialement philosophique tend à apprécier davantage la généralité et la liaison, en sorte que leur double réaction journalière devient également favorable à chacun d’eux, en consolidant chez lui les qualités fondamentales qui s’y altéraient naturellement. Une telle relation indique aussitôt combien sont nécessairement creuses et stériles les recherches spéculatives dirigées, en un sujet quelconque, vers les premiers principes, qui, devant toujours émaner de la sagesse vulgaire, n’appartiennent jamais au vrai domaine de la science, dont ils constituent, au contraire, les fondements spontanés et dès lors indiscutables ; ce qui élague radicalement une foule de controverses, oiseuses ou dangereuses, que nous a laissées l’ancien régime mental. On peut également sentir ainsi la profonde inanité finale de toutes les études préalables relatives à la logique abstraite, où il s’agit d’apprécier la vraie méthode philosophique, isolément d’aucune application à un ordre quelconque de phénomènes. En effet, les seuls principes vraiment généraux que l’on puisse établir à cet égard se réduisent nécessairement, comme il est aisé de le vérifier sur les plus célèbres de ces aphorismes, à quelques maximes incontestables mais évidentes, empruntées à la raison commune, et qui n’ajoutent vraiment rien d’essentiel aux indications résultées, chez tous les bons esprits, d’un simple exercice spontané. Quant à la manière d’adapter ces règles universelles aux divers ordres de nos spéculations positives, ce qui constituerait la vraie difficulté et l’utilité de tels préceptes logiques, elle ne saurait comporter de véritables appréciations que d’après une analyse spéciale à la nature propre des phénomènes considérés. La saine philosophie ne sépare donc jamais la logique d’avec la science ; la méthode et la doctrine ne pouvant, en chaque cas, être bien jugées que d’après leurs vraies relations mutuelles : il n’est pas plus possible, au fond, de donner à la logique qu’à la science un caractère universel par des conceptions purement abstraites, indépendantes de tous phénomènes déterminés ; les tentatives de ce genre indiquent encore la secrète influence de l’esprit absolu inhérent au régime théologico-métaphysique.\par
Considérée maintenant sous l’aspect historique, cette intime solidarité naturelle entre le génie propre de la vraie philosophie et le simple bon sens universel, montre l’origine spontanée de l’esprit positif, partout résulté, en effet, d’une réaction spéciale de l’a raison pratique sur la raison théorique, dont le caractère initial a toujours été ainsi modifié de plus en plus. Mais cette transformation graduelle ne pouvait s’opérer à la fois, ni surtout avec une égale vitesse, sur les diverses classes de spéculations abstraites, toutes primitivement théologiques, comme nous l’avons reconnu. Cette constante impulsion concrète n’y pouvait faire pénétrer l’esprit positif que suivant un ordre déterminé, conforme à la complication croissante des phénomènes, et qui sera directement expliqué ci-dessous. La positivité abstraite, nécessairement née dans les plus simples études mathématiques, et propagée ensuite par voie d’affinité spontanée ou d’imitation instinctive, ne pouvait donc offrir d’abord qu’un caractère spécial et même, à beaucoup d’égards, empirique, qui devait longtemps dissimuler, à la plupart de ses promoteurs, soit son incompatibilité inévitable avec la philosophie initiale, soit surtout. sa tendance radicale à fonder un nouveau régime logique. Ses progrès continus, sous l’impulsion croissante de la raison vulgaire, ne pouvaient alors déterminer directement que le triomphe préalable de l’esprit métaphysique, destiné, par sa généralité spontanée, à lui servir d’organe philosophique, pendant les siècles écoulés entre la préparation mentale du monothéisme et sa pleine installation sociale, après laquelle le régime ontologique, ayant obtenu tout l’ascendant que comportait sa nature, est bientôt devenu oppressif pour l’essor scientifique, qu’il avait jusque-là secondé. Aussi l’esprit positif n’a-t-il pu manifester suffisamment sa propre tendance philosophique quand il s’est trouvé enfin conduit, par cette oppression, à lutter spécialement contre l’esprit métaphysique, avec lequel il avait dû longtemps sembler confondu. C’est pourquoi la première fondation systématique de la philosophie positive ne saurait remonter au-delà de la mémorable crise où l’ensemble du régime ontologique a commencé à succomber, dans tout l’occident européen, sous le concours spontané de deux admirables impulsions mentales, l’une, scientifique, émanée de Kepler et Galilée, l’autre, philosophique, due à Bacon et à Descartes. L’imparfaite unité métaphysique constituée à la fin du Moyen Âge a été dès lors irrévocablement dissoute, comme l’ontologie grecque avait déjà détruit à jamais la grande unité théologique, correspondante au polythéisme. Depuis cette crise vraiment décisive, l’esprit positif, grandissant davantage en deux siècles qu’il n’avait pu le faire pendant toute sa longue carrière antérieure, n’a plus laissé possible d’autre unité mentale que celle qui résulterait de son propre ascendant universel, chaque nouveau domaine successivement acquis par lui ne pouvant plus jamais retourner à la théologie ni à la métaphysique, en vertu de la consécration définitive que ses acquisitions croissantes trouvaient de plus en plus dans la raison vulgaire. C’est seulement par une telle systématisation que la sagesse théorique rendra véritablement à la sagesse pratique un digne équivalent, en généralité et en consistance, de l’office fondamental qu’elle en a reçu, en réalité et en efficacité, pendant sa lente initiation graduelle car, les notions positives obtenues dans les deux derniers siècles sont, à vrai dire, bien plus précieuses comme matériaux ultérieurs d’une nouvelle, philosophie générale que par leur valeur directe et spéciale, la plupart d’entre elles n’ayant pu encore acquérir leur caractère définitif, ni scientifique, ni même logique.
\section[{II}]{II}\phantomsection
\label{II}\renewcommand{\leftmark}{II}

\noindent L’ensemble de notre évolution mentale, et surtout le grand mouvement accompli, en Europe occidentale, depuis Descartes et Bacon, ne laissent donc désormais d’autre issue possible que de constituer enfin, après tant de préambules nécessaires, l’état vraiment normal de la raison humaine, en procurant à l’esprit positif la plénitude et la rationalité qui lui manquent encore, de manière à établir, entre le génie philosophique et le bon sens universel, une harmonie qui jusqu’ici n’avait jamais pu exister suffisamment. Or, en étudiant ces deux conditions simultanées, de complément et de systématisation, que doit aujourd’hui remplir la science réelle pour s’élever à la dignité d’une vraie philosophie, on ne tarde pas à reconnaître qu’elles coïncident finalement. D’une part, en effet, la grande crise initiale de la positivité moderne n’a essentiellement laissé en dehors du mouvement scientifique proprement dit que les théories morales et sociales, dès lors restées dans un irrationnel isolement, sous la stérile domination de l’esprit théologico-métaphysique : c’est donc à les amener aussi à l’état positif que devait surtout consister, de nos jours, la dernière épreuve du véritable esprit philosophique, dont l’extension successive à tous les autres phénomènes fondamentaux se trouvait déjà assez ébauchée. Mais, d’une autre part, cette dernière expansion de la philosophie naturelle tendait spontanément à la systématiser aussitôt, en constituant l’unique point de vue, soit scientifique, soit logique, qui puisse dominer l’ensemble de nos spéculations réelles, toujours nécessairement réductibles à l’aspect humain, c’est-à-dire social, seul susceptible d’une active universalité. Tel est le double but philosophique de l’élaboration fondamentale, à la fois spéciale et générale, que j’ai osé entreprendre dans le grand ouvrage indiqué au début de ce Discours : les plus éminents penseurs contemporains la jugent ainsi assez accomplie pour avoir déjà posé les véritables bases directes de l’entière rénovation mentale projetée par Bacon et Descartes, mais dont l’exécution, décisive était réservée à notre siècle.\par
Pour que cette systématisation finale des conceptions humaines soit aujourd’hui assez caractérisée, il ne suffit pas d’apprécier, comme nous venons de le faire, sa destination théorique ; il faut aussi considérer ici, d’une manière distincte quoique sommaire, son aptitude nécessaire à constituer la seule issue intellectuelle que puisse réellement comporter l’immense crise sociale développée, depuis un demi-siècle, dans l’ensemble de l’occident européen et surtout en France.\par
Tandis que s’y accomplissait graduellement, pendant les cinq derniers siècles, l’irrévocable dissolution de la philosophie théologique, le système politique dont elle formait la base mentale subissait de plus en plus une décomposition non moins radicale, pareillement présidée par l’esprit métaphysique. Ce double mouvement négatif avait pour organes essentiels et solidaires, d’une part, les universités, d’abord émanées mais bientôt rivales de la puissance sacerdotale ; d’une autre part, les diverses corporations de légistes, graduellement hostiles aux pouvoirs féodaux : seulement, à mesure que l’action critique se disséminait, ses agents, sans changer de nature, devenaient plus nombreux et plus subalternes ; en sorte que, au dix-huitième siècle, la principale activité révolutionnaire dut passer, dans l’ordre philosophique, des docteurs proprement dits aux simples littérateurs, et ensuite, dans l’ordre politique, des juges aux avocats.\par
La grande crise finale a nécessairement commencé quand cette commune décadence, d’abord spontanée, puis systématique, à laquelle, d’ailleurs, toutes les classes quelconques de la société moderne avaient diversement concouru, est enfin parvenue au point de rendre universellement irrécusable l’impossibilité de conserver le régime ancien et le besoin croissant d’un ordre nouveau. Dès son origine, cette, crise a toujours tendu à transformer en un vaste mouvement organique le mouvement critique des cinq siècles antérieurs, en se présentant comme destinée surtout à opérer directement la régénération sociale, dont tous les préambules négatifs se trouvaient alors suffisamment accomplis. Mais cette transformation décisive, quoique de plus en plus urgente, a dû rester jusqu’ici essentiellement impossible, faute d’une philosophie vraiment propre à lui fournir une base intellectuelle indispensable. Au temps même où le suffisant accomplissement de la décomposition préalable exigeait la désuétude des doctrines purement négatives qui l’avaient dirigée, une fatale illusion, alors inévitable, conduisit, au contraire, à accorder, spontanément à l’esprit métaphysique, seul actif pendant ce long préambule, la présidence générale du mouvement de réorganisation. Quand une expérience pleinement décisive eut à jamais constaté, aux yeux de tous, l’entière impuissance organique d’une telle philosophie, l’absence de toute autre théorie ne permit pas de satisfaire d’abord aux besoins d’ordre, qui déjà prévalaient, autrement que par une sorte de restauration passagère de ce même système, mental et social, dont l’irréparable décadence avait donné lieu à la crise. Enfin, le développement de cette réaction rétrograde dut ensuite déterminer une mémorable manifestation que nos lacunes philosophiques rendaient aussi indispensable qu’inévitable, afin de démontrer irrévocablement que le progrès constitue, tout autant que l’ordre, l’une des deux conditions fondamentales de la civilisation moderne.\par
Le concours naturel de ces deux épreuves irrécusables, dont le renouvellement est maintenant devenu aussi impossible qu’inutile, nous a conduits aujourd’hui à cette étrange situation où rien de vraiment grand ne peut être entrepris, ni pour l’ordre, ni pour le progrès, faute d’une philosophie réellement adaptée à l’ensemble de nos besoins. Tout sérieux effort de réorganisation s’arrête bientôt devant les craintes de rétrogradation qu’il doit naturellement inspirer, en un temps où les idées d’ordre émanent encore essentiellement du type ancien, devenu justement antipathique aux populations actuelles : de même, les tentatives d’accélération directe de la progression politique ne tardent pas à être radicalement entravées par les inquiétudes très légitimes qu’elles doivent susciter sur l’imminence de l’anarchie, tant que les idées de progrès restent surtout négatives. Comme avant la crise, la lutte apparente demeure donc engagée entre l’esprit théologique, reconnu incompatible avec le progrès, qu’il a été conduit à nier dogmatiquement, et l’esprit métaphysique, qui, après avoir abouti, en philosophie, au doute universel, n’a pu tendre, en politique, qu’à constituer le désordre, ou un état équivalent de non gouvernement. Mais, d’après le sentiment unanime de leur commune insuffisance, ni l’un ni l’autre ne peut plus inspirer désormais, chez les gouvernants ou chez les gouvernés, de profondes convictions actives. Leur antagonisme continue pourtant à les alimenter mutuellement, sans qu’aucun d’eux puisse davantage comporter une véritable désuétude qu’un triomphe décisif ; parce que notre situation intellectuelle les rend encore, indispensables pour représenter, d’une manière quelconque, les conditions simultanées, d’une part de l’ordre, d’une autre part du progrès, jusqu’à ce qu’une même philosophie puisse y satisfaire également, de manière à rendre enfin pareillement inutiles l’école rétrograde et l’école négative, dont chacune est surtout destinée aujourd’hui à empêcher l’entière prépondérance de l’autre. Néanmoins, les inquiétudes opposées, relatives à ces deux dominations contraires, devront naturellement persister à la fois, tant que durera cet interrègne mental, par une suite inévitable de cette irrationnelle scission entre les deux faces inséparables du grand problème social. En effet, chacune des deux écoles, en vertu de son exclusive préoccupation, n’est plus même capable désormais de contenir suffisamment les aberrations inverses de son antagoniste.\par
Malgré sa tendance anti-anarchique, l’école théologique s’est montrée, de nos jours, radicalement impuissante à empêcher l’essor des opinions subversives, qui, après s’être développées surtout pendant sa principale restauration, sont souvent propagées par elle, pour de frivoles calculs dynastiques. Semblablement, quel que soit l’instinct anti-rétrograde de l’école métaphysique, elle n’a plus aujourd’hui toute la force logique qu’exigerait son simple office révolutionnaire, parce que son inconséquence caractéristique l’oblige à admettre les principes essentiels de ce même système dont elle attaque sans cesse les vraies conditions d’existence.\par
Cette déplorable oscillation entre deux philosophies, opposées, devenues également vaines, et ne pouvant s’éteindre qu’à la fois, devait susciter le développement d’une sorte d’école intermédiaire, essentiellement stationnaire, destinée surtout à rappeler directement l’ensemble de la question sociale, en proclamant enfin comme pareillement nécessaires les deux conditions fondamentales qu’isolaient les deux opinions actives. Mais, faute d’une philosophie propre à réaliser cette grande combinaison de l’esprit d’ordre avec l’esprit de progrès, cette troisième impulsion reste logiquement encore plus impuissante que les deux autres, parce qu’elle systématise l’inconséquence, en consacrant simultanément les principes rétrogrades et les maximes négatives, afin de pouvoir les neutraliser mutuellement. Loin de tendre à terminer la crise, une telle disposition ne pourrait aboutir qu’à l’éterniser, en s’opposant directement à toute vraie prépondérance d’un système quelconque., si on ne la bornait pas à une simple destination passagère, pour satisfaire empiriquement aux plus graves exigences de notre situation révolutionnaire, jusqu’à l’avènement décisif des seules doctrines qui puissent désormais convenir à l’ensemble de nos besoins. Mais, ainsi conçu, cet expédient provisoire est aujourd’hui devenu aussi indispensable qu’inévitable. Son rapide ascendant pratique, implicitement reconnu par les deux partis actifs, constate de plus en plus, chez les populations actuelles, l’amortissement simultané des convictions et des passions antérieures, soit rétrogrades, soit critiques, graduellement remplacées par un sentiment universel, réel quoique confus, de la nécessité, et même de la possibilité, d’une conciliation permanente entre l’esprit de conservation et l’esprit d’amélioration, également propres à l’état normal de l’humanité. La tendance correspondante des hommes d’État à empêcher aujourd’hui, autant que possible, tout grand mouvement politique, se trouve d’ailleurs spontanément conforme aux exigences fondamentales d’une situation qui ne comportera réellement que des institutions provisoires, tant qu’une vraie philosophie générale n’aura pas suffisamment rallié les intelligences. À l’insu des pouvoirs actuels, cette résistance instinctive concourt à faciliter la véritable solution, en poussant à transformer une stérile agitation politique en une active progression philosophique, de manière à suivre enfin la marche prescrite par la nature propre de la réorganisation finale, qui doit d’abord s’opérer dans les idées, pour passer ensuite aux mœurs, et, en dernier lieu, aux institutions. Une telle transformation, qui déjà tend à prévaloir en France, devra naturellement se développer partout de plus en plus, vu la nécessité croissante où se trouvent maintenant placés nos gouvernements occidentaux, de maintenir à grands frais l’ordre matériel au milieu du désordre intellectuel et moral, nécessité qui doit peu à peu absorber essentiellement leurs efforts journaliers, en les conduisant à renoncer implicitement à toute sérieuse présidence de la réorganisation spirituelle, ainsi livrée désormais à la libre activité des philosophes qui se montreraient dignes de la diriger. Cette disposition naturelle des pouvoirs actuels est en harmonie avec la tendance spontanée des populations à une apparente indifférence politique, motivée sur l’impuissance radicale des diverses doctrines en circulation, et qui doit toujours persister tant que les débats politiques continueront, faute d’une impulsion convenable, à dégénérer en de vaines luttes personnelles, de plus en plus misérables. Telle est l’heureuse efficacité pratique que l’ensemble de notre situation révolutionnaire procure momentanément à une école essentiellement empirique, qui, sous l’aspect théorique, ne peut jamais produire qu’un système radicalement contradictoire, non moins absurde et non moins dangereux, en politique, que l’est, en philosophie, l’éclectisme correspondant, inspiré aussi par une vaine intention de concilier, sans principes propres, des opinions incompatibles.\par
D’après ce sentiment, de plus en plus développé, de l’égale insuffisance sociale qu’offrent désormais l’esprit théologique et l’esprit métaphysique, qui seuls jusqu’ici ont activement disputé l’empire, la raison publique doit se trouver implicitement disposée à accueillir aujourd’hui l’esprit positif comme l’unique base possible d’une vraie résolution de la profonde anarchie intellectuelle et morale qui caractérise surtout la grande crise moderne. Restée encore étrangère à de telles questions, l’école positive s’y est graduellement préparée en constituant, autant que possible, pendant la lutte révolutionnaire des trois derniers siècles, le véritable état normal de toutes les classes plus simples de nos spéculations réelles. Forte de tels antécédents, scientifiques et logiques, pure d’ailleurs des diverses aberrations contemporaines, elle se présente aujourd’hui comme venant enfin d’acquérir l’entière généralité philosophique qui lui manquait jusqu’ici : dès lors, elle ose entreprendre, à son tour, la solution, encore intacte, du grand problème, en transportant convenablement aux études finales la même régénération qu’elle a successivement opérée déjà envers les différentes études préliminaires.\par
On ne peut d’abord méconnaître l’aptitude spontanée d’une telle philosophie à constituer directement la conciliation fondamentale, encore si vainement cherchée, entre les exigences simultanées de l’ordre et du progrès ; puisqu’il lui suffit, à cet effet, d’étendre jusqu’aux phénomènes sociaux une tendance pleinement conforme à sa nature, et qu’elle a maintenant rendue très familière dans tous les autres cas essentiels. En un sujet quelconque, l’esprit positif conduit toujours à établir une exacte harmonie élémentaire entre les idées d’existence et les idées de mouvement, d’où résulte, plus spécialement, envers les corps vivants la corrélation permanente des idées d’organisation aux idées de vie, et ensuite, par une dernière spécialisation propre à l’organisme social, la solidarité continue des idées d’ordre avec les idées de progrès. Pour la nouvelle philosophie, l’ordre constitue sans cesse la condition fondamentale du progrès ; et, réciproquement, le progrès devient le but nécessaire de l’ordre : comme, dans la mécanique animale, l’équilibre et la progression sont mutuellement indispensables, à titre de fondement ou de destination.\par
Spécialement considéré ensuite quant à l’ordre, l’esprit positif lui présente aujourd’hui, dans son extension sociale, de puissantes garanties directes, non seulement scientifiques mais aussi logiques, qui pourront bientôt être jugées très supérieures aux vaines prétentions d’une théologie rétrograde, de plus en plus dégénérée, depuis plusieurs siècles, en, élément actif de discordes, individuelles ou nationales, et désormais incapables de contenir les divagations de ses propres adeptes. Attaquant le désordre actuel à sa véritable source, nécessairement mentale, il constitue, aussi profondément que possible, l’harmonie logique, en régénérant d’abord les méthodes avant les doctrines, par une triple conversion simultanée de la nature des questions dominantes, de la manière de les traiter, et des conditions préalables de leur élaboration. D’une part, en effet, il démontre que les principales difficultés sociales ne sont pas aujourd’hui essentiellement politiques, mais surtout morales, en sorte que leur solution possible dépend réellement des opinions et des mœurs beaucoup plus que des institutions ; ce qui tend à éteindre une activité perturbatrice, en transformant l’agitation politique en mouvement philosophique. Sous le second aspect, il envisage toujours l’état présent comme un résultat nécessaire de l’ensemble de l’évolution antérieure, de manière à faire constamment prévaloir l’appréciation rationnelle du passé pour l’examen actuel des affaires humaines ; ce qui écarte aussitôt les tendances purement critiques, incompatibles avec cette saine conception historique. Enfin, au lieu de laisser la science sociale dans le vague et stérile isolement où la placent encore la théologie et la métaphysique, il la coordonne irrévocablement à toutes les autres sciences fondamentales, qui constituent graduellement, envers cette étude finale, autant de préambules indispensables, où notre intelligence acquiert à la fois les habitudes et les notions sans lesquelles on ne peut utilement aborder les plus éminentes spéculations positives ; ce qui institue déjà une vraie discipline mentale, propre à améliorer radicalement de telles discussions, dès lors rationnellement interdites à une foule d’entendements mal organisés ou mal préparés. Ces grandes garanties logiques sont d’ailleurs ensuite pleinement confirmées et développées par l’appréciation scientifique proprement dite, qui, envers les phénomènes sociaux ainsi que pour tous les autres, représente toujours notre ordre artificiel comme devant surtout consister en un simple prolongement judicieux, d’abord spontané, puis systématique, de l’ordre naturel résulté, en chaque cas, de l’ensemble des lois réelles, dont l’action effective est ordinairement modifiable, par notre sage intervention, entre des limites déterminées, d’autant plus écartées que les phénomènes sont plus élevés. Le sentiment élémentaire de l’ordre est, en un mot, naturellement inséparable de toutes les spéculations positives, constamment dirigées vers la découverte des moyens de liaison entre des observations dont la principale valeur résulte de leur systématisation.\par
Il en est de même, et encore plus évidemment, quant au Progrès, qui, malgré de vaines prétentions ontologiques, trouve aujourd’hui, dans l’ensemble des études scientifiques, sa plus incontestable manifestation. D’après leur nature absolue, et par suite essentiellement immobile, la métaphysique et la théologie ne sauraient comporter, guère plus l’une que l’autre, un véritable progrès, c’est-à-dire une progression continue vers un but déterminé. Leurs transformations historiques consistent surtout, au contraire, en une désuétude croissante, soit mentale, soit sociale, sans que les questions agitées aient jamais pu faire aucun pas réel, à raison même de leur insolubilité radicale. Il est aisé de reconnaître que les discussions ontologiques des écoles grecques se sont essentiellement reproduites sous d’autres formes, chez les scolastiques du Moyen Âge, et nous retrouvons aujourd’hui l’équivalent parmi nos psychologues ou idéologues ; aucune des doctrines controversées n’ayant pu, pendant ces vingt siècles de stériles débats, aboutir à des démonstrations décisives, pas seulement en ce qui concerne l’existence des corps extérieurs, encore aussi problématique pour les argumenteurs modernes que pour leurs plus antiques prédécesseurs. C’est évidemment la marche continue des connaissances positives qui a inspiré, il y a deux siècles, dans la célèbre formule philosophique de Pascal, la première notion rationnelle du progrès humain, nécessairement étrangère à toute l’ancienne philosophie. Étendue ensuite à l’évolution industrielle et même esthétique, mais restée trop confuse envers le mouvement social, elle tend aujourd’hui vaguement vers une systématisation décisive, qui ne peut émaner que de l’esprit positif, enfin convenablement généralisé. Dans ses spéculations journalières, il en reproduit spontanément l’actif sentiment élémentaire, en représentant toujours l’extension et le perfectionnement de nos connaissances réelles comme le but essentiel de nos divers efforts théoriques. Sous l’aspect le plus systématique, la nouvelle philosophie assigne directement, pour destination nécessaire, à toute notre existence, à la fois personnelle et sociale, l’amélioration continue, non seulement de notre condition, mais aussi et surtout de notre nature, autant que le comporte, à tous égards, l’ensemble des lois réelles, extérieures ou intérieures. Érigeant ainsi la notion du progrès en dogme vraiment fondamental de la sagesse humaine, soit pratique, soit théorique, elle lui imprime le caractère le plus noble en même temps que le plus complet, en représentant toujours le second genre de perfectionnement comme supérieur au premier, D’une part, en effet, l’action de l’Humanité sur le monde extérieur dépendant surtout des dispositions de l’agent, leur amélioration doit constituer notre principale ressource : d’autre part, les phénomènes humains, individuels ou collectifs, étant de tous, les plus modifiables, c’est envers eux que notre intervention rationnelle comporte naturellement la plus vaste efficacité. Le dogme du progrès ne peut donc devenir suffisamment philosophique que d’après une exacte appréciation générale de ce qui constitue surtout cette amélioration continue de la progression humaine. Or, à cet égard, l’ensemble de la philosophie positive démontre pleinement, comme on peut le voir dans l’ouvrage indiqué au début de ce Discours, que ce perfectionnement consiste essentiellement, soit pour l’individu, soit pour l’espèce, à faire de plus en plus prévaloir les éminents attributs qui distinguent le plus notre humanité de la simple animalité, c’est-à-dire, d’une part l’intelligence, d’une autre part la sociabilité, facultés naturellement solidaires, qui se servent mutuellement de moyen et de but. Quoique le cours. spontané de l’évolution humaine, personnelle ou sociale, développe toujours leur commune influence, leur ascendant combiné ne saurait pourtant parvenir au point d’empêcher que notre principale activité ne dérive habituellement des penchants intérieurs, que notre constitution réelle rend nécessairement beaucoup plus énergique. Ainsi, cette idéale prépondérance de notre humanité sur notre animalité remplit naturellement les conditions essentielles d’un vrai type philosophique, en caractérisant une limite déterminée, dont tous nos efforts doivent nous rapprocher constamment sans pouvoir toutefois y atteindre jamais.\par
Cette double indication de l’aptitude fondamentale de l’esprit positif à systématiser spontanément les saines notions simultanées de l’ordre et du progrès suffit ici pour signaler sommairement la haute efficacité sociale propre à la nouvelle philosophie générale. Sa valeur, à cet égard, dépend surtout de sa pleine réalité scientifique, c’est-à-dire de l’exacte harmonie qu’elle établit toujours, autant que possible, entre les principes et les faits, aussi bien quant aux phénomènes sociaux qu’envers tous les autres. La réorganisation totale, qui peut seule terminer la grande crise moderne, consiste, en effet, sous l’aspect mental qui doit d’abord prévaloir, à constituer une théorie sociologique propre à expliquer convenablement l’ensemble du passé humain : tel est le mode le plus rationnel de poser la question essentielle, afin d’y mieux écarter toute passion perturbatrice. Or c’est ainsi que la supériorité nécessaire. de l’école positive sur les diverses écoles actuelles peut aussi être le plus nettement appréciée. Car, l’esprit théologique et l’esprit métaphysique sont tous deux conduits, par leur nature absolue, à ne considérer que la portion du passé où chacun d’eux a surtout dominé : ce qui précède et ce qui suit ne leur offre qu’une ténébreuse confusion et un désordre inexplicable, dont la liaison avec cette étroite partie du grand spectacle historique ne peut, à leurs yeux, résulter que d’une miraculeuse intervention. Par exemple, le catholicisme a toujours montré, à l’égard du polythéisme antique, une tendance aussi aveuglément critique que celle qu’il reproche justement aujourd’hui, envers lui-même, à l’esprit révolutionnaire proprement dit. Une véritable explication de l’ensemble du passé, conformément aux lois constantes de notre nature, individuelle ou collective, est donc nécessairement impossible aux diverses écoles absolues qui dominent encore ; aucune d’elles, en effet, n’a suffisamment tenté de l’établir. L’esprit positif, en vertu de sa nature éminemment relative, peut seul représenter convenablement toutes les grandes époques historiques comme autant de phases déterminées d’une même évolution fondamentale, où chacune résulte de la précédente et prépare la suivante selon les lois invariables, qui fixent sa participation spéciale à la commune progression, de manière à toujours permettre, sans plus d’inconséquence que de partialité, de rendre une exacte justice philosophique à toutes les coopérations quelconques. Quoique cet incontestable privilège de la positivité rationnelle doive d’abord sembler purement spéculatif, les vrais penseurs y reconnaîtront bientôt la première source nécessaire de l’actif ascendant social réservé finalement à la nouvelle philosophie.\par
Car, on peut assurer aujourd’hui que la doctrine qui aura suffisamment expliqué l’ensemble du passé obtiendra inévitablement, par suite de cette seule épreuve, la présidence mentale de l’avenir.\par
Une telle indication des hautes propriétés sociales qui caractérisent l’esprit positif ne serait point encore assez décisive si on n’y ajoutait pas une sommaire appréciation de son aptitude spontanée à systématiser enfin la morale humaine, ce qui constituera toujours la principale application de toute vraie théorie de l’Humanité.\par
Dans l’organisme polythéique de l’antiquité, la morale, radicalement subordonnée à la politique, ne pouvait jamais acquérir ni la dignité ni l’universalité convenables à sa nature. Son indépendance fondamentale et même son ascendant normal résultèrent enfin, autant qu’il était alors possible, du régime monothéique propre au Moyen Âge : cet immense service social, dû surtout au catholicisme, formera toujours son principal titre à l’éternelle reconnaissance du genre humain. C’est seulement depuis cette indispensable séparation, sanctionnée et complétée par la division nécessaire des deux puissances, que la morale humaine a pu réellement commencer à prendre un caractère systématique, en établissant, à l’abri des impulsions passagères, des règles vraiment générales pour l’ensemble de notre existence, personnelle, domestique et sociale. Mais les, profondes imperfections de la philosophie monothéique qui présidait alors à cette grande opération ont dû en altérer beaucoup l’efficacité, et même en compromettre gravement la stabilité, en suscitant bientôt un fatal conflit entre l’essor intellectuel et le développement moral. Ainsi liée à une doctrine qui ne pouvait longtemps rester progressive, la morale devait ensuite se trouver de plus en plus affectée par le discrédit croissant qu’allait nécessairement subir une théologie qui, désormais rétrograde, deviendrait enfin radicalement antipathique à la raison moderne. Exposée dès lors à l’action dissolvante de la métaphysique, la morale théorique a reçu, en effet, pendant les cinq derniers siècles, dans chacune de ses trois parties essentielles, des atteintes graduellement dangereuses, que n’ont pu toujours assez réparer, pour la pratique, la rectitude et la moralité naturelles de l’homme, malgré l’heureux développement continu que devait alors procurer le cours spontané de notre civilisation. Si l’ascendant nécessaire de l’esprit positif ne venait mettre un terme à ces anarchiques divagations, elles imprimeraient certainement une mortelle fluctuation à toutes les notions un peu délicates de la morale usuelle, non seulement sociale, mais aussi domestique, et même personnelle, en ne laissant partout subsister que les règles relatives aux cas les plus grossiers, que l’appréciation vulgaire pourrait directement garantir.\par
En une telle situation, il doit sembler étrange que la seule philosophie qui puisse, en effet, consolider aujourd’hui la morale, se trouve, au contraire taxée, à cet égard, d’incompétence radicale par les diverses écoles actuelles, depuis les vrais catholiques jusqu’aux simples déistes, qui, au milieu de leurs vains débats, s’accordent surtout à lui interdire essentiellement l’accès de ces questions fondamentales, d’après cet unique motif que son génie trop partiel s’était borné jusqu’ici à des sujets plus simples. L’esprit métaphysique, qui a souvent tendu à dissoudre activement la morale et l’esprit théologique, qui, dès longtemps, a perdu la force de la préserver, persistent néanmoins à s’en faire une sorte d’apanage éternel et exclusif, sans que la raison publique ait encore convenablement jugé ces empiriques prétentions. On doit, il est vrai, reconnaître, en général, que l’introduction de toute règle morale a dû partout s’opérer d’abord sous les inspirations théologiques, alors profondément incorporées au système entier de nos idées, et aussi seules susceptibles de constituer des opinions suffisamment communes. Mais l’ensemble du passé démontre également que cette solidarité primitive a toujours décru comme l’ascendant même de la théologie ; les préceptes moraux, ainsi que tous les autres, ont été de plus en plus ramenés à une consécration purement rationnelle, à mesure que le vulgaire est devenu plus capable d’apprécier l’influence réelle de chaque conduite sur l’existence humaine, individuelle ou sociale. En séparant irrévocablement la morale de la politique, le catholicisme a dû beaucoup développer cette tendance continue ; puisque l’intervention surnaturelle s’est ainsi trouvée directement réduite à la formation des règles générales, dont l’application particulière était dès lors essentiellement confiée à la sagesse humaine. S’adressant à des populations plus avancées, il a livré à la raison publique une foule de prescriptions spéciales que les anciens sages avaient cru ne pouvoir jamais se passer des injonctions religieuses, comme le pensent encore les docteurs polythéistes de l’Inde, par exemple quant à la plupart des pratiques hygiéniques. Aussi peut-on remarquer, même plus de trois siècles après saint Paul, les sinistres prédictions de plusieurs philosophes ou magistrats païens, sur l’imminente immoralité qu’allait entraîner nécessairement la prochaine révolution théologique. Les déclamations actuelles des diverses écoles monothéiques n’empêcheront pas davantage l’esprit positif d’achever aujourd’hui, sous les conditions convenables, la conquête, pratique et théorique, du domaine moral, déjà spontanément livré de plus en plus à la raison humaine, dont il ne nous reste surtout qu’à systématiser enfin les inspirations particulières, l’Humanité ne saurait, sans doute, demeurer indéfiniment condamnée à ne pouvoir fonder ses règles de conduite que sur des motifs chimériques, de manière à éterniser une désastreuse opposition, jusqu’ici passagère, entre les besoins intellectuels et les besoins moraux.\par
Bien loin que l’assistance théologique soit à jamais indispensable aux préceptes moraux, l’expérience démontre, au contraire, qu’elle leur est devenue, chez les modernes, de plus en plus nuisible, en les faisant inévitablement participer, par suite de cette funeste adhérence, à la décomposition croissante du régime monothéique, surtout pendant les trois derniers siècles. D’abord, cette fatale solidarité devait directement affaiblir, à mesure que la loi s’éteignait, la seule base sur laquelle se trouvaient ainsi reposer des règles qui, souvent exposées à de graves conflits avec des impulsions très énergiques, ont besoin d’être soigneusement préservées de toute hésitation. L’antipathie croissante que l’esprit théologique inspirait justement à la raison moderne, a gravement affecté beaucoup d’importantes notions morales, non seulement relatives aux plus grands rapports sociaux, mais concernant aussi la simple vie domestique, et même l’existence personnelle : une aveugle ardeur d’émancipation mentale n’a que trop entraîné d’ailleurs à ériger quelquefois le dédain passager de ces salutaires maximes en une sorte de folle protestation contre la philosophie rétrograde d’où elles semblaient exclusivement émaner. Jusque chez ceux qui conservaient la foi dogmatique, cette funeste influence se faisait indirectement sentir, parce que l’autorité sacerdotale, après avoir perdu son indépendance politique, voyait aussi décroître de plus en plus l’ascendant social indispensable à son efficacité morale. Outre cette impuissance croissante pour protéger les règles morales, l’esprit théologique leur a souvent nui aussi d’une manière active, par les divagations qu’il a suscitées, depuis qu’il n’est plus suffisamment disciplinable, sous l’inévitable essor du libre examen individuel. Ainsi exercé, il a réellement inspiré ou secondé beaucoup d’aberrations antisociales, que le bon sens, livré à lui-même, eût spontanément évitées ou rejetées. Les utopies subversives que nous voyons s’accréditer aujourd’hui, soit contre la propriété, soit même quant à la famille, etc., ne sont presque jamais émanées ni accueillies des intelligences pleinement émancipées, malgré leurs lacunes fondamentales, mais bien plutôt de celles qui poursuivent activement une sorte de restauration théologique, fondée sur un vague et stérile déisme ou sur un protestantisme équivalent., Enfin, cette antique adhérence à la théologie est aussi devenue nécessairement funeste à la morale, sous un troisième aspect général, en s’opposant à sa solide reconstruction sur des bases purement humaines. Si cet obstacle ne consistait que dans les aveugles déclamations trop souvent émanées des diverses écoles actuelles, théologiques ou métaphysiques, contre le prétendu danger d’une telle opération, les philosophes positifs pourraient se borner à repousser d’odieuses insinuations par l’irrécusable exemple de leur propre vie journalière, personnelle, domestique et sociale. Mais cette opposition est, malheureusement, beaucoup plus radicale ; car, elle résulte de l’incompatibilité nécessaire qui existe évidemment entre ces deux manières de systématiser la morale. Les motifs théologiques devant naturellement offrir, aux yeux du croyant, une intensité très supérieure à celle de tous les autres quelconques, ils ne sauraient jamais devenir les simples auxiliaires des motifs purement humains ils ne peuvent conserver aucune efficacité réelle aussitôt qu’ils ne dominent plus. Il n’existe donc aucune alternative durable, entre fonder enfin la morale sur la connaissance positive de l’Humanité, et la laisser reposer sur l’injonction surnaturelle : les convictions rationnelles ont pu seconder les croyances théologiques, ou plutôt s’y substituer graduellement, à mesure que la foi s’est éteinte ; mais la combinaison inverse ne constitue certainement qu’une utopie contradictoire, où le principal serait subordonné à l’accessoire.\par
Une judicieuse exploration du véritable état de la société moderne représente donc comme de plus en plus démentie, par l’ensemble des faits journaliers, la prétendue impossibilité de se dispenser désormais de toute théologie pour consolider la morale ; puisque cette dangereuse liaison a dû devenir, depuis la fin du Moyen Âge, triplement funeste à la morale, soit en énervant ou discréditant ses bases intellectuelles, soit en y suscitant des perturbations directes, soit en y empêchant une meilleure systématisation. Si, malgré d’actifs principes de désordre, la moralité pratique s’est réellement améliorée, cet heureux résultat ne saurait être attribué à l’esprit théologique, alors dégénéré, au contraire, en un dangereux dissolvant : il est essentiellement dû à l’action croissante de l’esprit positif, déjà efficace sous sa forme spontanée, consistant dans le bon sens universel, dont les sages inspirations ont secondé l’impulsion naturelle de notre civilisation progressive pour combattre utilement les diverses aberrations, surtout celles qui émanaient des divagations religieuses. Lorsque, par exemple, la théologie protestante tendait à altérer gravement l’institution du mariage par la consécration formelle du divorce, la raison publique en neutralisait beaucoup les funestes effets, en imposant presque toujours le respect des mœurs antérieures, seules conformes au vrai caractère de la sociabilité moderne. D’irrécusables expériences ont d’ailleurs prouvé, en même temps, sur une vaste échelle, au sein des masses populaires, que le prétendu privilège exclusif des croyances religieuses pour déterminer de grands sacrifices ou d’actifs dévouements pouvait également appartenir à des opinions directement opposées, et s’attachait, en général, à toute profonde conviction, quelle qu’en puisse être la nature. Ces nombreux adversaires du régime théologique qui, il y a un demi-siècle, garantirent, avec tant d’héroïsme, notre indépendance nationale contre la coalition rétrograde, ne montrèrent pas, sans doute, une moins pleine et moins constante abnégation que les bandes superstitieuses, qui, au sein de la France, secondèrent l’agression extérieure.\par
Pour achever d’apprécier les prétentions actuelles de la philosophie théologico-métaphysique à conserver la systématisation exclusive de la morale usuelle, il suffit d’envisager directement la doctrine dangereuse et contradictoire que l’inévitable progrès de l’émancipation mentale l’a bientôt forcée d’établir à ce sujet, en consacrant partout, sous des formes plus en moins explicites, une sorte d’hypocrisie collective, analogue à celle qu’on suppose très mal à propos avoir été habituelle chez les anciens, quoiqu’elle n’y avait jamais comporté qu’un succès précaire et passager. Ne pouvant empêcher le libre essor de la raison moderne chez les esprits cultivés, on s’est ainsi proposé d’obtenir d’eux, en vue de l’intérêt public, le respect apparent des antiques croyances, afin d’en maintenir, chez le vulgaire, l’autorité jugée indispensable. Cette transaction systématique n’est nullement particulière aux jésuites, quoiqu’elle constitue le fond essentiel de leur tactique ; l’esprit protestant lui a aussi imprimé, à sa manière, une consécration encore plus intime, plus étendue, et surtout plus dogmatique les métaphysiciens proprement dits l’adoptent tout autant que les théologiens eux-mêmes ; le plus grand d’entre eux, quoique sa haute moralité fût vraiment digne de son éminente intelligence, a été entraîné à la sanctionner essentiellement, en établissant, d’une part, que les opinions théologiques quelconques ne comportent aucune véritable démonstration, et, d’une autre part, que la nécessité sociale oblige à maintenir indéfiniment leur empire. Malgré qu’une telle doctrine puisse devenir respectable chez ceux qui n’y rattachent aucune ambition personnelle, elle n’en tend pas moins à vicier toutes les sources de la moralité humaine, en la faisant nécessairement reposer sur un état continu de fausseté, et même de mépris, des supérieurs envers les inférieurs. Tant que ceux qui devaient participer à cette dissimulation systématique sont restés peu nombreux, la pratique en a été possible, quoique fort précaire : mais elle est devenue encore plus ridicule qu’odieuse, quand l’émancipation s’est assez étendue pour que cette sorte de pieux complot dût embrasser, comme il le faudrait aujourd’hui, la plupart des esprits actifs. Enfin, même en supposant réalisée cette chimérique extension, ce prétendu système laisse subsister la difficulté tout entière à l’égard des intelligences affranchies, dont la propre moralité se trouve ainsi abandonnée à leur pure spontanéité, déjà justement reconnue insuffisante chez la classe soumise. S’il faut aussi admettre la nécessité d’une vraie systématisation morale chez ces esprits émancipés, elle ne pourra dès lors reposer que sur des bases positives, qui finalement seront ainsi jugées indispensables, Quant à borner leur destination à la classe éclairée, outre qu’une telle restriction ne saurait changer la nature de cette grande construction philosophique, elle serait évidemment illusoire en un temps où la culture mentale que suppose ce facile affranchissement est déjà devenue très commune, ou plutôt presque universelle, du moins en France. Ainsi, l’empirique expédient suggéré par le vain désir de maintenir, à tout prix, l’antique régime intellectuel, ne peut finalement aboutir qu’à laisser indéfiniment dépourvus de toute doctrine morale la plupart des esprits actifs, comme on le voit trop souvent aujourd’hui.\par
C’est donc surtout au nom de la morale qu’il faut désormais travailler ardemment à constituer enfin l’ascendant universel de l’esprit positif, pour remplacer un système déchu qui, tantôt impuissant, tantôt perturbateur, exigerait de plus en plus la compression mentale en condition permanente de l’ordre moral. La nouvelle philosophie peut seule » établir aujourd’hui, au sujet de nos divers devoirs, des convictions profondes et actives, vraiment susceptibles de soutenir avec énergie le choc des passions. D’après la théorie positive de l’Humanité, d’irrécusables démonstrations, appuyées sur l’immense expérience que possède maintenant notre espèce, détermineront exactement l’influence réelle, directe ou indirecte, privée et publique, propre à chaque acte, à chaque habitude, et à chaque penchant ou sentiment ; d’où résulteront naturellement, comme autant d’inévitables corollaires, les règles de conduite, soit générales, soit spéciales, les plus conformes à l’ordre universel, et qui, par suite, devront se trouver ordinairement les plus favorables au bonheur individuel. Malgré l’extrême difficulté de ce grand sujet, j’ose assurer que, convenablement traité, il comporte des conclusions tout aussi certaines que celles de la géométrie elle-même. On ne peut, sans doute, espérer de jamais rendre suffisamment accessibles à toutes les intelligences ces preuves positives de plusieurs règles morales destinées pourtant à la vie commune : mais il en est déjà ainsi pour diverses prescriptions mathématiques, qui néanmoins sont appliquées sans hésitation dans les plus graves occasions, lorsque, par exemple nos marins risquent journellement leur existence sur la foi de théories astronomiques qu’ils ne comprennent nullement ; pourquoi une égale confiance ne serait-elle pas accordée aussi à des notions plus importantes ? Il est d’ailleurs incontestable que l’efficacité normale d’un tel régime exige, en chaque cas, outre la puissante impulsion résultée naturellement des préjugés publics, l’intervention systématique, tantôt passive, tantôt active, d’une autorité spirituelle, destinée à rappeler avec énergie les maximes fondamentales et à en diriger sagement l’application, comme je l’ai spécialement expliqué dans l’ouvrage ci-dessus indiqué. En accomplissant ainsi le grand office social que le catholicisme n’exerce plus, ce nouveau pouvoir moral utilisera soigneusement l’heureuse aptitude de la philosophie correspondante à s’incorporer spontanément la sagesse de tous les divers régimes antérieurs, suivant la tendance ordinaire de l’esprit positif envers un sujet quelconque. Quand l’astronomie moderne a irrévocablement écarté les principes astrologiques, elle n’en a pas moins précieusement conservé toutes les notions véritables obtenues sous leur domination : il en a été de même pour la chimie, relativement à l’alchimie.\par
Sans pouvoir entreprendre ici l’appréciation morale de la philosophie positive, il y faut pourtant signaler la tendance continue qui résulte directement de sa constitution propre, soit scientifique, soit logique, pour stimuler et consolider le sentiment du devoir en développant toujours l’esprit d’ensemble, qui s’y trouve naturellement lié. Ce nouveau régime mental dissipe spontanément la fatale opposition qui, depuis la fin du Moyen Âge, existe de plus en plus entre les besoins intellectuels et les besoins moraux. Désormais, au contraire, toutes les spéculations réelles, convenablement systématisées, concourront sans cesse à constituer, autant que possible, l’universelle prépondérance de la morale, puisque le point de vue moral y deviendra nécessairement le lien scientifique et le régulateur logique de tous les autres aspects positifs. Il est impossible qu’une telle coordination, en développant familièrement les idées d’ordre et d’harmonie, toujours rattachées à l’Humanité, ne tende point à moraliser profondément, non seulement les esprits d’élite, mais aussi la masse des intelligences, qui toutes devront plus ou moins participer à cette grande initiation, d’après un système convenable d’éducation universelle.\par
Une appréciation plus intime et plus étendue, à la fois pratique et théorique, représente l’esprit positif comme étant, par sa nature, seul susceptible de développer directement le sentiment social, première base nécessaire de toute saine morale. L’antique régime mental ne pouvait le stimuler qu’à l’aide de pénibles artifices indirects, dont le succès réel devait être fort imparfait, vu la tendance essentiellement personnelle d’une telle philosophie, quand la sagesse sacerdotale n’en contenait pas l’influence spontanée. Cette nécessité est maintenant reconnue, du moins empiriquement, quant à l’esprit métaphysique proprement dit, qui n’a jamais pu aboutir, en morale, à aucune autre théorie effective que le désastreux système de l’égoïsme, si usité aujourd’hui, malgré beaucoup de déclamations contraires : même les sectes ontologiques qui ont sérieusement protesté contre une semblable aberration n’y ont finalement substitué que de vagues ou incohérentes notions, incapables d’efficacité pratique. Une tendance aussi déplorable, et néanmoins aussi constante, doit avoir de plus profondes racines qu’on ne le suppose communément. Elle résulte surtout, en effet, de la nature nécessairement personnelle d’une telle philosophie, qui, toujours bornée à la considération de l’individu, n’a jamais pu embrasser réellement l’étude de l’espèce, par une suite inévitable de son vain principe logique, essentiellement réduit à l’intuition proprement dite, qui ne comporte évidemment aucune application collective. Ses formules ordinaires ne font que traduire naïvement son esprit fondamental ; pour chacun de ses adeptes, la pensée dominante est constamment celle du moi : toutes les autres existences quelconques, même humaines, sont confusément enveloppées dans une seule conception négative, et leur vague ensemble constitue le {\itshape non moi} ; la notion du {\itshape nous} n’y saurait trouver aucune place directe et distincte. Mais, en examinant ce sujet encore plus profondément, il faut reconnaître que, à cet égard, comme sous tout autre aspect, la métaphysique dérive, aussi bien dogmatiquement qu’historiquement, de la théologie elle-même, dont elle ne pouvait jamais constituer qu’une modification dissolvante. En effet, ce caractère de personnalité constante appartient surtout, avec une énergie plus directe, à la pensée théologique, toujours préoccupée, chez chaque croyant, d’intérêts essentiellement individuels, dont l’immense prépondérance absorbe nécessairement toute autre considération, sans que le plus sublime dévouement puisse en inspirer l’abnégation véritable, justement regardée alors comme une dangereuse aberration. Seulement l’opposition fréquente de ces intérêts chimériques avec les intérêts réels a fourni à la sagesse sacerdotale un puissant moyen de discipline morale, qui a pu souvent commander, au profit de la société, d’admirables sacrifices, qui pourtant n’étaient tels qu’en apparence, et se réduisaient toujours à une prudente pondération d’intérêts. Les sentiments bienveillants et désintéressés, qui sont propres à la nature humaine, ont dû, sans doute, se manifester à travers un tel régime, et même, à certains égards, sous son impulsion indirecte ; mais, quoique leur essor n’ait pu être ainsi comprimé, leur caractère en a dû recevoir une grave altération, qui probablement ne nous permet pas encore de connaître pleinement leur nature et leur intensité, faute d’un exercice propre et direct. Il y a tout lieu de présumer d’ailleurs que cette habitude continue de calculs personnels envers les plus chers intérêts du croyant a développé, chez l’homme, même à tout autre égard, par voie d’affinité graduelle, un excès de circonspection, de prévoyance, et finalement d’égoïsme, que son organisation fondamentale n’exigeait pas, et qui dès lors pourra diminuer un jour sous un meilleur régime moral. Quoi qu’il en soit de cette conjecture, il demeure incontestable que la pensée théologique est, de sa nature, essentiellement individuelle, et jamais directement collective. Aux yeux de la foi, surtout monothéique, la vie sociale n’existe pas, à défaut d’un but qui lui soit propre ; la société humaine ne peut alors offrir immédiatement qu’une simple agglomération d’individus, dont la réunion est presque aussi fortuite que passagère et qui, occupés chacun de son seul salut, ne conçoivent la participation à celui d’autrui que comme un puissant moyen de mieux mériter le leur en obéissant aux prescriptions suprêmes qui en ont imposé l’obligation. Notre respectueuse admiration sera toujours bien due assurément à la prudence sacerdotale qui, sous l’heureuse impulsion d’un instinct public, a su retirer longtemps une haute utilité pratique d’une si imparfaite philosophie. Mais cette juste reconnaissance ne saurait aller jusqu’à prolonger artificiellement ce régime initial au-delà de sa destination provisoire, quand l’âge est enfin venu d’une économie plus conforme à l’ensemble de notre nature, intellectuelle et affective.\par
L’esprit positif, au contraire, est directement social, autant que possible, et sans aucun effort par suite de sa réalité caractéristique. Pour lui, l’homme proprement dit n’existe pas, il ne peut exister que l’Humanité, puisque tout notre développement est, dû à la société, sous quelque rapport qu’on l’envisage. Si l’idée de société semble encore une abstraction de notre intelligence, c’est surtout en vertu de l’ancien régime philosophique ; car, à vrai dire, c’est à l’idée {\itshape d’individu} qu’appartient un tel caractère, du moins chez notre espèce. L’ensemble de la nouvelle philosophie tendra toujours à faire ressortir, aussi bien dans la vie active que dans la vie spéculative, la liaison de chacun à tous, sous une foule d’aspects divers, de manière à rendre involontairement familier le sentiment intime de la solidarité sociale, convenablement étendue à tous les temps et à tous les lieux. Non seulement l’active recherche du bien public sera sans cesse représentée comme le mode le plus propre à assurer communément le bonheur privé : mais, par une influence à la fois plus directe et plus pure, finalement plus efficace, le plus complet exercice possible des penchants généreux deviendra la principale source de la félicité personnelle, quand même il ne devrait procurer exceptionnellement d’autre récompense qu’une inévitable satisfaction intérieure. Car, si, comme on n’en saurait douter, le bonheur résulte surtout d’une sage activité, il doit donc dépendre principalement des instincts sympathiques, quoique notre organisation ne leur accorde pas ordinairement une énergie prépondérante ; puisque les sentiments bienveillants sont les seuls qui puissent se développer librement dans l’état social, qui naturellement les stimule de plus en plus en leur ouvrant un champ indéfini, tandis qu’il exige, de toute nécessité, une certaine compression permanente des diverses impulsions personnelles, dont l’essor spontané susciterait des conflits continus. Dans cette vaste expansion sociale, chacun retrouvera la satisfaction normale de cette tendance à s’éterniser, qui ne pouvait d’abord être satisfaite qu’à l’aide d’illusions désormais incompatibles avec notre évolution mentale. Ne pouvant plus se prolonger que par l’espèce, l’individu sera ainsi entraîné à s’y incorporer le plus complètement possible, en se liant profondément à toute son existence collective, non seulement actuelle, mais aussi passée, et surtout future, de manière à obtenir toute l’intensité de vie que comporte, en chaque cas, l’ensemble des lois réelles. Cette grande identification pourra devenir d’autant plus intime et mieux sentie que la nouvelle philosophie assigne nécessairement aux deux sortes de vie une même destination fondamentale et une même loi d’évolution, consistant toujours, soit pour l’individu, soit pour l’espèce, dans la progression continue dont le but principal a été ci-dessus caractérisé, c’est-à-dire la tendance à faire, de part et d’autre, prévaloir, autant que possible, l’attribut humain, ou la combinaison de l’intelligence avec la sociabilité, sur l’animalité proprement dite. Nos sentiments quelconques n’étant développables que par un exercice direct et soutenu, d’autant plus indispensable qu’ils sont d’abord moins énergiques, il serait ici superflu d’insister davantage, auprès de quiconque possède, même empiriquement, une vraie connaissance de l’homme, pour démontrer la supériorité nécessaire de l’esprit positif sur l’ancien esprit théologico-métaphysique, quant à l’essor propre et actif de l’instinct social. Cette prééminence est d’une nature tellement sensible que, sans doute, la raison publique la reconnaîtra suffisamment, longtemps avant que les institutions correspondantes aient pu convenablement réaliser ses heureuses propriétés.\par
D’après l’ensemble des indications précédentes, la supériorité spontanée de la nouvelle philosophie sur chacune de celles qui se disputent aujourd’hui l’empire se trouve maintenant aussi caractérisée sous l’aspect social qu’elle l’était déjà du point de vue mental, autant du moins que le comporte ce Discours, et sauf le recours indispensable à l’ouvrage cité. En achevant cette sommaire appréciation, il importe d’y remarquer l’heureuse corrélation qui s’établit naturellement entre un tel esprit philosophique et les dispositions, sages mais empiriques, que l’expérience contemporaine fait désormais prévaloir de plus en plus, aussi bien chez les gouvernés que chez les gouvernants. Substituant directement un immense mouvement mental à une stérile agitation politique, l’école positive explique et sanctionne, d’après un examen systématique, l’indifférence ou la répugnance que la raison publique et la prudence des gouvernements s’accordent à manifester aujourd’hui pour toute sérieuse élaboration directe des institutions proprement dites, en un temps où il n’en peut exister d’efficaces qu’avec un caractère purement provisoire on transitoire, faute d’aucune base rationnelle suffisante, tant que durera l’anarchie intellectuelle, Destinée à dissiper enfin ce désordre fondamental, par les seules voies qui puissent le surmonter, cette nouvelle école a besoin, avant tout, du maintien continu de l’ordre matériel, tant intérieur qu’extérieur, sans lequel aucune grave méditation sociale ne saurait être ni convenablement accueillie ni même suffisamment élaborée. Elle tend donc à justifier et à seconder la préoccupation très légitime qu’inspire aujourd’hui partout le seul grand résultat politique qui soit immédiatement compatible avec la situation actuelle, laquelle d’ailleurs lui procure une valeur spéciale par les graves difficultés qu’elle lui suscite, en posant toujours le problème, insoluble à la longue, de maintenir un certain ordre politique au milieu d’un profond désordre moral. Outre ses travaux d’avenir, l’école positive s’associe immédiatement à cette importante opération par sa tendance directe à discréditer radicalement les diverses écoles actuelles, en remplissant déjà mieux que chacune d’elles les offices opposés qui leur restent encore, et qu’elle seule combine spontanément, de façon à se montrer aussitôt plus organique que l’école théologique et plus progressive que l’école métaphysique, sans pouvoir jamais comporter les dangers de rétrogradation ou d’anarchie qui leur sont respectivement propres. Depuis que les gouvernements ont essentiellement renoncé, quoique d’une manière implicite, à toute sérieuse restauration du passé, et les populations à tout grave bouleversement des institutions, la nouvelle philosophie n’a plus à demander, de part et d’autre, que les dispositions habituelles qu’on est au fond préparé partout à lui accorder (du moins en France, où doit surtout s’accomplir d’abord l’élaboration systématique), c’est-à-dire liberté et attention. Sous ces conditions naturelles, l’école positive tend, d’un côté, à consolider tous les pouvoirs actuels chez leurs possesseurs quelconques, et, de l’autre, à leur imposer des obligations morales de plus en plus conformes aux vrais besoins des peuples.\par
Ces dispositions incontestables semblent d’abord ne devoir aujourd’hui laisser à la nouvelle philosophie d’autres obstacles essentiels que ceux qui résulteront de l’incapacité ou de l’incurie de ses divers promoteurs. Mais une plus mûre appréciation montre, au contraire, qu’elle doit trouver d’énergiques résistances chez presque tous les esprits maintenant actifs, par suite même de la difficile rénovation qu’elle exigerait d’eux pour les associer directement à sa principale élaboration. Si cette inévitable opposition devait se borner aux esprits essentiellement théologiques ou métaphysiques, elle offrirait peu de gravité réelle, parce qu’il resterait un puissant appui chez ceux, dont le nombre et l’influence croissent journellement, qui sont surtout livrés aux études positives. Mais, par une fatalité aisément explicable, c’est de ceux-là mêmes que la nouvelle école doit peut-être attendre le moins d’assistance et le plus d’entraves : une philosophie directement émanée des sciences trouvera probablement ses plus dangereux ennemis chez ceux qui les cultivent aujourd’hui. La principale source de ce déplorable conflit consiste dans la spécialisation aveugle et dispersive qui caractérise profondément l’esprit scientifique actuel, d’après sa formation nécessairement partielle, suivant la complication croissante des phénomènes étudiés, comme je l’indiquerai expressément ci-dessous. Cette marche provisoire, qu’une dangereuse routine académique s’efforce aujourd’hui d’éterniser, surtout parmi les géomètres, développe la vraie positivité, chez chaque intelligence, seulement envers une faible portion du système mental, et laisse tout le reste sous un vague régime théologico-métaphysique, ou l’abandonne à un empirisme encore plus oppressif, en sorte que le véritable esprit positif, qui correspond à l’ensemble des divers travaux scientifiques, se trouve, au fond, ne pouvoir être pleinement compris par aucun de ceux qui l’ont ainsi naturellement préparé. De plus en plus livrés à cette inévitable tendance, les savants proprement dits sont ordinairement conduits, dans notre siècle, à une insurmontable aversion contre toute idée générale, et à l’entière impossibilité d’apprécier réellement aucune conception philosophique. On sentira mieux, au reste, la gravité d’une telle opposition en observant que, née des habitudes mentales, elle a dû s’étendre ensuite jusqu’aux divers intérêts correspondants, que notre régime scientifique rattache profondément, surtout en France, à cette désastreuse spécialité, comme je l’ai soigneusement démontré dans l’ouvrage cité. Ainsi, la nouvelle philosophie, qui exige directement l’esprit d’ensemble, et qui fait à jamais prévaloir, sur toutes les études aujourd’hui constituées, la science naissante du développement social, trouvera nécessairement une intime antipathie, à la fois active et passive, dans les préjugés et les passions de la seule classe qui pût directement lui offrir un point d’appui spéculatif, et chez laquelle elle ne doit longtemps espérer que des adhésions purement individuelles, plus rares là peut-être que partout ailleurs\footnote{ Cette empirique prépondérance de l’esprit de détail chez la plupart des savants actuels, de leur aveugle antipathie envers toute généralisation quelconque, se trouvent beaucoup aggravées, surtout en France, par leur réunion habituelle en académies, où les divers préjuges analytiques se fortifient mutuellement, où d’ailleurs se développent des intérêts trop souvent abusifs, où enfin s’organise spontanément une sorte d’émeute permanente contre le régime synthétique qui doit désormais prévaloir. L’instinct de progrès qui caractérisait, il y a un demi-siècle, le génie révolutionnaire, avait confusément senti ces dangers essentiels, de manière à déterminer la suppression directe de ces compagnies arriérées, qui, ne convenant qu’à l’élaboration préliminaire de l’esprit positif, devenaient de plus en plus hostiles à sa systématisation finale. Quoique cette audacieuse mesure, si mal jugée d’ordinaire, fût alors prématurée, parce que ces graves inconvénients ne pouvaient encore être assez reconnus, il reste néanmoins certain que ces corporations scientifiques, avaient déjà accompli le principal office que comportait leur nature : depuis leur restauration, leur influence réelle a été, au fond, beaucoup plus nuisible qu’utile à la marche actuelle de la grande évolution mentale.}.\par
Pour surmonter convenablement ce concours spontané de résistances diverses que lui présente aujourd’hui la masse spéculative proprement dite, l’école positive ne saurait trouver d’autre ressource générale que d’organiser un appel direct et soutenu au bon sens universel, en s’efforçant désormais de propager systématiquement, dans la masse active, les principales études scientifiques propres à y constituer la base indispensable de sa grande élaboration philosophique. Ces études préliminaires, naturellement dominées jusqu’ici par cet esprit de spécialité empirique qui préside aux sciences correspondantes, sont toujours conçues et dirigées comme si chacune d’elles devait surtout préparer à une certaine profession exclusive ; ce qui interdit évidemment la possibilité, même chez ceux qui auraient le plus de loisir, d’en embrasser jamais plusieurs, ou du moins autant que l’exigerait la formation ultérieure de saines conceptions générales. Mais il n’en peut plus être ainsi quand une telle instruction est directement destinée à l’éducation universelle, qui en change nécessairement le caractère et la direction, malgré toute tendance contraire. Le public, en effet, qui ne veut devenir ni géomètre, ni astronome, ni chimiste, etc., éprouve continuellement le besoin simultané de toutes les sciences fondamentales, réduites chacune à ses notions essentielles : il lui faut, suivant l’expression très remarquable de notre grand Molière, {\itshape des clartés de tout.} Cette simultanéité nécessaire n’existe pas seulement pour lui quand il considère ces études dans leur destination abstraite et générale, comme seule base rationnelle de l’ensemble des conceptions humaines : il la retrouve encore, quoique moins directement, même envers les diverses applications concrètes, dont chacune, au fond, au lieu de se rapporter exclusivement à une certaine branche de la philosophie naturelle, dépend aussi plus ou moins de toutes les autres.. Ainsi, l’universelle propagation des principales études positives n’est pas uniquement destinée aujourd’hui à satisfaire un besoin déjà très prononcé chez le public, qui sent de plus en plus que les sciences ne sont plus exclusivement réservées pour les savants, mais qu’elles existent surtout pour lui-même. Par une heureuse réaction spontanée, une telle destination, quand elle sera convenablement développée, devra radicalement améliorer l’esprit scientifique actuel, en le dépouillant de sa spécialité aveugle et dispersive, de manière à lui faire acquérir peu à peu le vrai caractère philosophique, indispensable à sa principale mission. Cette voie est même la seule qui puisse, de nos jours, constituer graduellement, en dehors de la classe spéculative, proprement dite, un vaste tribunal spontané, aussi impartial qu’irrécusable, formé de la masse des hommes sensés, devant lequel viendront s’éteindre irrévocablement beaucoup de fausses opinions scientifiques, que les vues propres à l’élaboration préliminaire des deux derniers siècles ont dû mêler profondément aux doctrines vraiment positives, qu’elles altéreront nécessairement tant que ces discussions ne seront pas enfin directement soumises au bon sens universel. En un temps où il ne faut attendre d’efficacité immédiate que de mesures toujours provisoires, bien adaptées à notre situation transitoire, l’organisation nécessaire d’un tel point d’appui général pour l’ensemble des travaux philosophiques devient, à mes yeux, le principal résultat social que puisse maintenant produire l’entière vulgarisation des connaissances réelles : le public rendra ainsi à la nouvelle école un plein équivalent des services que cette organisation lui procurera.\par
Ce grand résultat ne pourrait être suffisamment obtenu si cet enseignement continu restait destiné à une seule classe quelconque, même très étendue : on doit, sous peine d’avortement, y avoir toujours en vue l’entière universalité des intelligences. Dans l’état normal que ce mouvement doit préparer, toutes, sans aucune exception ni distinction, éprouveront toujours le même besoin fondamental de cette philosophie première, résultée de l’ensemble des notions réelles, et qui doit alors devenir la base systématique de la sagesse humaine, aussi bien active que spéculative, de manière à remplir plus convenablement l’indispensable office social qui se rattachait jadis à l’universelle instruction chrétienne. Il importe donc beaucoup que, dès son origine, la nouvelle école philosophique développe, autant que possible, ce grand caractère élémentaire d’universalité sociale, qui, finalement relatif à sa principale destination, constituera aujourd’hui sa plus grande force contre les diverses résistances qu’elle doit rencontrer.\par
Afin de mieux marquer cette tendance nécessaire, une intime conviction, d’abord instinctive, puis systématique, m’a déterminé, depuis longtemps, à représenter toujours l’enseignement exposé dans ce Traité comme s’adressant surtout à la classe la plus nombreuse, que notre situation laisse dépourvue de toute instruction régulière, par suite de la désuétude croissante de l’instruction purement théologique, qui, provisoirement remplacée, pour les seuls lettrés, par une certaine instruction métaphysique et littéraire, n’a pu recevoir, surtout en France, aucun pareil équivalent pour la masse populaire. l’importance et la nouveauté d’une telle disposition constante, mon vif désir qu’elle soit convenablement appréciée, et même, si j’ose le dire, imitée, m’obligent à indiquer ici les principaux motifs de ce contact spirituel que doit ainsi spécialement instituer aujourd’hui avec les prolétaires la nouvelle école philosophique, sans toutefois que son enseignement doive jamais exclure aucune classe quelconque. Quelques obstacles que le défaut de zèle ou d’élévation puisse réellement apporter de part et d’autre à un tel rapprochement, il est aisé de reconnaître, en général, que, de toutes les portions de la société actuelle, le peuple proprement dit doit être, au fond, la mieux disposée, par les tendances et les besoins qui résultent de sa situation caractéristique, à accueillir favorablement la nouvelle philosophie, qui finalement doit trouver là son principal appui, aussi bien mental que social.\par
Une première considération, qu’il importe d’approfondir, quoique sa nature soit surtout négative, résulte, à ce sujet, d’une judicieuse appréciation, de ce qui, au premier aspect, pourrait sembler offrir une grave difficulté, c’est-à-dire l’absence actuelle de toute culture spéculative. Sans doute, il est regrettable, par exemple, que cet enseignement populaire de la philosophie astronomique ne trouve pas encore, chez tous ceux auxquels il est surtout destiné, quelques études mathématiques préliminaires, qui le rendraient à la fois plus efficace et plus facile, et que je suis même forcé d’y supposer. Mais la même lacune se rencontrerait aussi chez la plupart des autres classes actuelles, en un temps où l’instruction positive reste bornée, en France, à certaines professions spéciales, qui se rattachent essentiellement à l’École Polytechnique ou aux écoles de médecine. Il n’y a donc rien là qui soit vraiment particulier à nos prolétaires. Quant à leur défaut habituel de cette sorte de culture régulière que reçoivent aujourd’hui les classes lettrées, je ne crains pas de tomber dans une exagération philosophique en affirmant qu’il en résulte, pour les esprits populaires, un notable avantage, au lieu d’un inconvénient réel. Sans revenir ici sur une critique malheureusement trop facile, assez accomplie depuis longtemps, et que l’expérience journalière confirme de plus en plus aux yeux de la plupart des hommes sensés, il serait difficile de concevoir maintenant une préparation plus irrationnelle, et au fond, plus dangereuse, à la conduite ordinaire de la vie réelle, soit active, soit même spéculative, que celle qui résulte de cette vaine instruction, d’abord de mots, puis d’entités, où se perdent encore tant de précieuses années de jeunesse. À la majeure partie de ceux qui, la reçoivent, elle n’inspire guère désormais qu’un dégoût presque insurmontable de tout travail intellectuel pour le cours entier de leur carrière : mais ses dangers deviennent beaucoup plus graves chez ceux qui s’y sont plus spécialement livrés. L’inaptitude à la vie réelle, le dédain des professions vulgaires, l’impuissance d’apprécier convenablement aucune conception positive, et l’antipathie qui en résulte bientôt, les disposent trop souvent aujourd’hui à seconder une stérile agitation métaphysique, que d’inquiètes prétentions personnelles, développées par cette désastreuse éducation, ne tardent pas à rendre politiquement perturbatrice, sous l’influence directe d’une vicieuse érudition historique, qui, en faisant prévaloir une fausse notion du type social propre à l’antiquité, empêche communément de comprendre la sociabilité moderne ; En considérant que presque tous ceux qui, à divers égards, dirigent maintenant les affaires humaines, y ont été ainsi préparés, on ne saurait être surpris de la honteuse ignorance qu’ils manifestent trop souvent sur les moindres sujets, même matériels, ni de leur fréquente disposition à négliger le fond pour la forme, en plaçant au-dessus de tout l’art de bien dire, quelque contradictoire ou pernicieuse qu’en devienne l’application, ni enfin de la tendance spéciale de nos classes lettrées à accueillir avidement toutes les aberrations qui surgissent journellement de notre anarchie mentale. Une telle appréciation dispose, au contraire, à s’étonner que ces divers désastres ne soient pas ordinairement plus étendus ; elle conduit à admirer profondément la rectitude et la sagesse naturelles de l’homme, qui, sous l’heureuse impulsion propre à l’ensemble de notre civilisation, contiennent spontanément, en grande partie, ces dangereuses conséquences d’un absurde système d’éducation générale. Ce système ayant été, depuis la fin du Moyen Âge, comme il l’est encore, le principal point d’appui social de l’esprit métaphysique, soit d’abord contre la théologie, soit ensuite aussi contre la science, on conçoit aisément que les classes qu’il n’a pu développer doivent se trouver, par cela même, beaucoup moins affectées de cette philosophie transitoire, et dès lors mieux disposées à l’état positif. Or, tel est l’important avantage que l’absence d’éducation scolastique procure aujourd’hui à nos prolétaires, et qui les rend, au fond, moins accessibles que la plupart des lettrés aux divers sophismes perturbateurs, conformément à l’expérience journalière, malgré une excitation continue, systématiquement dirigée vers les passions relatives à leur condition sociale. Ils durent être jadis profondément dominés par la théologie, surtout catholique ; mais, pendant leur émancipation mentale, la métaphysique n’a pu que glisser sur eux, faute d’y rencontrer la culture spéciale sur laquelle elle repose : seule, la philosophie positive pourra, de nouveau, les saisir radicalement. Les conditions préalables, tant recommandées par les premiers pères de cette philosophie finale, doivent là se trouver ainsi mieux remplies que partout ailleurs : si la célèbre {\itshape table rase} de Bacon et de Descartes était jamais pleinement réalisable, ce serait assurément chez les prolétaires actuels, qui, principalement en France, sont bien plus rapprochés qu’aucune classe quelconque du type idéal de cette disposition préparatoire à la positivité rationnelle.\par
En examinant, sous un aspect plus intime et plus durable, cette inclination naturelle des intelligences populaires vers la saine philosophie, on reconnaît aisément qu’elle doit toujours résulter de la solidarité fondamentale qui, d’après nos explications antérieures, rattache directement le véritable esprit philosophique au bon sens universel, sa première source nécessaire. Non seulement, en effet, ce bon sens, si justement préconisé par Descartes et Bacon, doit aujourd’hui se trouver plus pur et plus énergique chez les classes inférieures, en vertu même de cet heureux défaut de culture scolastique qui les rend moins accessibles aux habitudes vagues on sophistiques. À cette différence passagère, que dissipera graduellement une meilleure éducation des classes lettrées, il en faut joindre une autre, nécessairement permanente, relative à l’influence mentale des diverses fonctions sociales propres aux deux ordres d’intelligence, d’après le caractère respectif de leurs travaux habituels. Depuis que l’action réelle de l’Humanité sur le monde extérieur a commencé, chez les modernes, à s’organiser spontanément, elle exige la combinaison continue de deux classes distinctes très inégales en nombre, mais également indispensables : d’une part, les entrepreneurs proprement dits, toujours peu nombreux, qui, possédant les divers matériaux convenables, y compris l’argent et le crédit dirigent l’ensemble de chaque opération, en assumant dès lors la principale responsabilité des résultats quelconques ; d’une autre part, les opérateurs directs, vivant d’un salaire périodique et formant l’immense majorité des travailleurs, qui exécutent, dans une sorte d’intention abstraite, chacun des actes élémentaires, sans se préoccuper spécialement de leur concours final. Ces derniers sont seuls immédiatement aux prises avec la nature, tandis que les premiers ont surtout affaire à la société. Par une suite nécessaire de ces diversités fondamentales, l’efficacité spéculative que nous avons reconnue inhérente à la vie industrielle pour développer involontairement l’esprit positif doit ordinairement se faire mieux sentir chez les opérateurs que parmi les entrepreneurs ; car, leurs travaux propres offrent un caractère plus simple, un but plus nettement déterminé, des résultats plus prochains, et des conditions plus impérieuses. L’école positive y devra donc trouver naturellement un accès plus facile pour son enseignement universel, et une plus vive sympathie pour sa rénovation philosophique, quand elle pourra convenablement pénétrer dans ce vaste milieu social. Elle y devra rencontrer, en même temps, des affinités morales non moins précieuses que ces harmonies mentales, d’après cette commune insouciance matérielle qui rapproche spontanément nos prolétaires de la véritable classe contemplative, du moins quand celle-ci aura pris enfin les mœurs correspondantes à sa destination sociale. Cette heureuse disposition, aussi favorable à l’ordre universel qu’à la vraie félicité personnelle, acquerra un jour beaucoup d’importance normale, d’après la systématisation des rapports généraux qui doivent exister entre ces deux éléments extrêmes de la société positive. Mais, dès ce moment ‘ elle peut faciliter essentiellement leur union naissante, en suppléant au peu de loisir que les occupations journalières laissent à nos prolétaires pour leur instruction spéculative. Si, en quelques cas exceptionnels d’extrême surcharge, cet obstacle continu semble, en effet, devoir empêcher tout essor mental, il est ordinairement compensé par ce caractère de sage imprévoyance qui, dans chaque intermittence naturelle des travaux obligés, rend à l’esprit une pleine disponibilité. Le vrai loisir ne doit manquer habituellement que dans la classe qui s’en croit spécialement douée ; car, à raison même de sa fortune et de sa position, elle reste communément préoccupée d’actives inquiétudes, qui ne comportent presque jamais un véritable calme, intellectuel et moral. Cet état doit être facile, au contraire, soit aux penseurs, soit aux opérateurs, d’après leur commun affranchissement spontané des soucis relatifs à l’emploi des capitaux, et indépendamment de la régularité naturelle de leur vie journalière.\par
Quand ces différentes tendances, mentales et morales, auront convenablement agi, c’est donc parmi les prolétaires que devra le mieux se réaliser cette universelle propagation de l’instruction positive, condition indispensable à l’accomplissement graduel de la rénovation philosophique. C’est aussi chez eux que le caractère continu d’une telle étude pourra devenir le plus purement spéculatif, parce qu’elle s’y trouvera mieux exempte de ces vues intéressées qu’apportent, plus ou moins directement, les classes supérieures, presque toujours préoccupées de calculs avides ou ambitieux. Après y avoir d’abord cherché le fondement universel de toute sagesse humaine, ils y viendront puiser ensuite, comme dans les beaux-arts, une douce diversion habituelle à l’ensemble de leurs peines journalières. Leur inévitable condition sociale devant leur rendre beaucoup plus précieuse une telle diversion, soit scientifique, soit esthétique, il serait étrange que les classes dirigeantes voulussent y voir, au contraire, un motif fondamental de les en tenir essentiellement privés en refusant systématiquement la seule satisfaction qui puisse être indéfiniment partagée à ceux-là même qui doivent sagement renoncer aux jouissances les moins communicables. Pour justifier un tel refus, trop souvent dicté par l’égoïsme et l’irréflexion, on a quelquefois objecté, il est vrai, que cette vulgarisation spéculative tendrait à aggraver profondément le désordre actuel, en développant la funeste disposition, déjà trop prononcée, au déclassement universel. Mais cette crainte naturelle, unique objection sérieuse qui, à ce sujet, méritât une vraie discussion, résulte aujourd’hui, dans. la plupart des cas de bonne foi, d’une irrationnelle confusion de l’instruction positive, à la fois esthétique et scientifique, avec l’instruction métaphysique et littéraire, seule maintenant organisée. Celle-ci, en effet, que nous avons déjà reconnue exercer une action sociale très perturbatrice chez les classes lettrées, deviendrait beaucoup plus dangereuse si on l’étendait aux prolétaires, où elle développerait, outre le dégoût des occupations matérielles, d’exorbitantes ambitions. Mais, heureusement, ils sont, en général, encore moins disposés à la demander qu’on ne serait à la leur accorder. Quant aux études positives, sagement conçues et convenablement dirigées, elles ne comportent nullement une telle influence : s’alliant et s’appliquant, par leur nature, à tous les travaux pratiques, elles tendent au contraire à en confirmer ou même inspirer le goût, soit en anoblissant leur caractère habituel, soit en adoucissant leurs pénibles conséquences ; conduisant d’ailleurs à une saine appréciation des diverses positions sociales et des nécessités correspondantes, elles disposent à sentir que le bonheur réel est compatible avec toutes les conditions quelconques, pourvu qu’elles soient honorablement remplies et raisonnablement acceptées. La philosophie générale qui en résulte représente l’homme ou plutôt l’Humanité, comme le Premier des êtres connus, destiné, par l’ensemble des lois réelles, à toujours perfectionner autant que possible, à tous égards, l’ordre naturel, à l’abri de toute inquiétude chimérique ; ce qui tend à relever profondément l’actif sentiment universel de la dignité humaine. En même temps, elle tempère spontanément l’orgueil trop exalté qu’il pourrait susciter, en montrant, sous tous les aspects, et avec une familière évidence, combien nous devons rester sans cesse au-dessous du but et du type ainsi caractérisés, soit dans la vie active, soit même dans la vie spéculative, où l’on sent, presque à chaque pas, que nos plus sublimes efforts ne peuvent jamais surmonter qu’une faible partie des difficultés fondamentales.\par
Malgré la haute importance des divers motifs précédents, des considérations encore plus puissantes détermineront surtout les intelligences populaires à seconder aujourd’hui l’action philosophique de l’école positive par leur ardeur continue pour l’universelle propagation des études réelles elles se rapportent aux principaux besoins collectifs propres à la condition sociale des prolétaires. On peut les résumer en cet aperçu général : il n’a pu exister jusqu’ici une politique spécialement populaire, et la nouvelle philosophie peut seule la constituer.\par
Depuis le commencement de la grande crise moderne, le peuple n’est encore intervenu que comme simple auxiliaire dans les principales luttes politiques, avec l’espoir, sans doute, d’y obtenir quelques améliorations de sa situation générale, mais non d’après des vues et pour un but qui lui fussent réellement propres. Tous les débats habituels sont restés essentiellement concentrés entre les diverses classes supérieures ou moyennes, parce qu’ils se rapportaient surtout à la possession du pouvoir. Or, le peuple ne pouvait longtemps s’intéresser directement à de tels conflits, puisque la nature de notre civilisation empêche évidemment les prolétaires d’espérer, et même de désirer, aucune importante participation à la puissance politique proprement dite. Aussi, après avoir essentiellement réalisé tous les résultats sociaux qu’ils pouvaient attendre de la substitution provisoire des métaphysiciens et des légistes à l’ancienne prépondérance politique des classes sacerdotales et féodales, deviennent-ils aujourd’hui de plus en plus indifférents à la stérile prolongation de ces luttes de plus en plus misérables, désormais réduites presque à de vaines rivalités personnelles. Quels que soient les efforts journaliers de l’agitation métaphysique pour les faire intervenir dans ces frivoles débats, par l’appât de ce qu’on nomme les droits politiques, l’instinct populaire a déjà compris, surtout en France, combien serait illusoire ou puérile la possession d’un tel privilège, qui, même dans son degré actuel de dissémination, n’inspire habituellement aucun intérêt véritable à la plupart de ceux qui en jouissent exclusivement. Le peuple ne peut s’intéresser essentiellement qu’à l’usage effectif du pouvoir, en quelques mains qu’il réside, et non à sa conquête spéciale. Aussitôt que les questions politiques, ou plutôt dès lors sociales, se rapporteront ordinairement à la manière dont le pouvoir doit être exercé pour mieux atteindre sa destination générale, principalement relative, chez les modernes, à la masse prolétaire, on ne tardera pas à reconnaître que le dédain actuel ne tient nullement à une dangereuse indifférence : jusque-là, l’opinion populaire restera étrangère à ces débats, qui, aux yeux des bons esprits, en augmentant l’instabilité de tous les pouvoirs, tendent spécialement à retarder cette indispensable transformation. En un mot, le peuple est naturellement disposé à désirer que la vaine et orageuse discussion des droits se trouve enfin remplacée par une féconde et salutaire appréciation des divers devoirs essentiels, soit généraux, soit spéciaux. Tel est le principe spontané de l’intime connexité qui, tôt ou tard sentie, ralliera nécessairement l’instinct populaire à l’action sociale de la philosophie positive ; car cette grande transformation équivaut évidemment à celle, ci-dessus motivée par les plus hautes considérations spéculatives, du mouvement politique actuel en un simple mouvement philosophique, dont le premier et le principal résultat social consistera, en effet, à constituer solidement une active morale universelle, prescrivant à chaque agent, individuel ou collectif, les règles de conduite les plus conformes à l’harmonie fondamentale. Plus on méditera sur cette relation naturelle, mieux on reconnaîtra que cette mutation décisive, qui ne pouvait émaner que de l’esprit positif, ne peut aujourd’hui trouver un solide appui que chez le peuple proprement dit, seul disposé à la bien comprendre et à s’y intéresser profondément. Les préjugés et les passions propres aux classes supérieures ou moyennes, s’opposent conjointement à ce qu’elle y soit d’abord suffisamment sentie, parce qu’on y doit être ordinairement plus touché des avantages inhérents à la possession du pouvoir que des dangers résultés de son vicieux exercice. Si le peuple est maintenant et doit rester désormais indifférent à la possession directe du pouvoir politique, il ne peut jamais renoncer à son indispensable participation continue au pouvoir moral, qui, seul vraiment accessible à tous, sans aucun danger pour l’ordre universel, et, au contraire, à son grand avantage journalier, autorise chacun, au nom d’une commune doctrine fondamentale, à rappeler convenablement les plus hautes puissances à leurs divers devoirs essentiels. À la vérité, les préjugés inhérents à l’état transitoire ou révolutionnaire ont dû trouver aussi quelque accès parmi nos prolétaires ; ils y entretiennent, en effet, de fâcheuses illusions sur la portée indéfinie des mesures politiques proprement dites ; ils y empêchent d’apprécier combien la juste satisfaction des grands intérêts populaires dépend aujourd’hui davantage des opinions et des mœurs que des institutions elles-mêmes, dont la vraie régénération, actuellement impossible, exige, avant tout, une réorganisation spirituelle. Mais on peut assurer que l’école positive aura beaucoup plus de facilité à faire pénétrer ce salutaire enseignement chez les esprits populaires que partout ailleurs, soit parce que la métaphysique négative n’a pu s’y enraciner autant, soit surtout par l’impulsion constante des besoins sociaux inhérents à leur situation nécessaire. Ces besoins se rapportent essentiellement à deux conditions fondamentales, l’une spirituelle, l’autre temporelle, de nature profondément connexe : il s’agit, en effet, d’assurer convenablement à tous, d’abord l’éducation normale, ensuite le travail régulier ; tel est, au fond, le vrai programme social des prolétaires. Il ne peut plus exister de véritable popularité que pour la politique qui tendra nécessairement vers cette double destination. Or, tel est, évidemment, le caractère spontané de la doctrine sociale propre à la nouvelle école philosophique ; nos explications antérieures doivent ici dispenser, à cet égard, de tout autre éclaircissement, d’ailleurs réservé à l’ouvrage si souvent indiqué dans ce Discours. Il importe seulement d’ajouter, à ce sujet, que la concentration nécessaire de nos pensées et de notre activité sur la vie réelle de l’Humanité, en écartant toute vaine illusion, tendra spécialement à fortifier beaucoup l’adhésion morale et politique du peuple proprement dit à la vraie philosophie moderne. En effet, son judicieux instinct y sentira bientôt un puissant motif nouveau de diriger surtout la pratique sociale vers la sage amélioration continue de sa propre condition générale. Les chimériques espérances inhérentes à l’ancienne philosophie ont trop souvent conduit, au contraire, à négliger avec dédain de tels progrès, ou à les écarter par une sorte d’ajournement continu, d’après la minime importance relative que devait naturellement leur laisser cette éternelle perspective, immense compensation spontanée de toutes les misères quelconques.\par
Cette sommaire appréciation suffit maintenant à signaler, sous les divers aspects essentiels, l’affinité nécessaire des classes inférieures pour la philosophie positive, qui, aussitôt que le contact aura pu pleinement s’établir, trouvera là son principal appui, naturel, à la fois mental et social ; tandis que la philosophie théologique ne convient plus qu’aux classes supérieures, dont elle tend à éterniser la prépondérance politique, comme la philosophie métaphysique s’adresse surtout aux classes moyennes, dont elle seconde l’active ambition. Tout esprit méditatif doit ainsi comprendre enfin l’importance vraiment fondamentale que présente aujourd’hui une sage vulgarisation systématique des études positives, essentiellement destinée aux prolétaires, afin d’y préparer une saine doctrine sociale. Les divers observateurs qui peuvent s’affranchir, même momentanément, du tourbillon journalier s’accordent maintenant à déplorer, et certes avec beaucoup de raison, l’anarchique influence qu’exercent, de nos jours, les sophistes et les rhéteurs. Mais ces justes plaintes resteront inévitablement vaines tant qu’on n’aura pas mieux senti la nécessité de sortir enfin d’une situation mentale, où l’éducation officielle ne peut aboutir, d’ordinaire, qu’à former des rhéteurs et des sophistes, qui tendent ensuite spontanément à propager le même esprit, par le triple enseignement émané des journaux, des romans, et des drames, parmi les classes inférieures, qu’aucune instruction régulière ne garantit de la contagion métaphysique, repoussée seulement par leur raison naturelle. Quoique l’on doive espérer, à ce titre, que les gouvernements actuels sentiront bientôt combien l’universelle propagation des connaissances réelles peut seconder de plus en plus leurs efforts continus pour le difficile maintien d’un ordre indispensable, il ne faut pas encore attendre d’eux, ni même en désirer, une coopération vraiment active à cette grande préparation rationnelle, qui doit longtemps résulter surtout d’un libre zèle privé, inspiré et soutenu par de véritables convictions philosophiques. L’imparfaite conservation d’une grossière harmonie politique sans cesse compromise au milieu de notre désordre mental et moral, absorbe trop justement leur sollicitude journalière, et les tient même placés à un point de vue trop inférieur, pour qu’ils puissent dignement comprendre la nature et les conditions d’un tel travail, dont il faut seulement leur demander d’entrevoir l’importance. Si, par un zèle intempestif, ils tentaient aujourd’hui de le diriger, ils ne pourraient aboutir qu’à l’altérer profondément, de manière à compromettre beaucoup sa principale efficacité, en ne le rattachant pas à une philosophie assez décisive, ce qui le ferait bientôt dégénérer en une incohérente accumulation de spécialités superficielles. Ainsi, l’école positive, résultée d’un actif concours volontaire des esprits vraiment philosophiques, n’aura longtemps à demander à nos gouvernements occidentaux, pour accomplir convenablement son grand office social, qu’une pleine liberté d’exposition et de discussion, équivalente à celle dont jouissent déjà l’école théologique et l’école métaphysique. L’une peut, chaque jour, dans ses mille tribunes sacrées, préconiser, à son gré, l’excellence absolue de son éternelle doctrine, et vouer tous ses adversaires quelconques à une irrévocable damnation ; l’autre, dans les nombreuses chaires que lui entretient la munificence nationale, peut journellement développer, devant d’immenses auditoires, l’universelle efficacité de ses conceptions ontologiques et la prééminence indéfinie de ses études littéraires. Sans prétendre à de tels avantages, que le temps doit seul procurer, l’école positive ne demande essentiellement aujourd’hui qu’un simple droit d’asile régulier dans les localités municipales, pour y faire directement apprécier son aptitude finale à la satisfaction simultanée de tous nos grands besoins sociaux, en propageant avec sagesse la seule instruction systématique qui puisse désormais préparer une véritable réorganisation, d’abord mentale, puis morale, et enfin politique. Pourvu que ce libre accès lui reste toujours ouvert, le zèle volontaire et gratuit de ses rares promoteurs, secondé par le bon sens universel, et sous l’impulsion croissante de la situation fondamentale, ne redoutera jamais de soutenir, même dès ce moment, une active concurrence philosophique envers les nombreux et puissants organes, même réunis, des deux écoles anciennes. Or, il n’est plus à craindre que désormais les hommes d’État s’écartent gravement, à cet égard, de l’impartiale modération de plus en plus inhérente à leur propre indifférence spéculative : l’école positive a même lieu de compter, sous ce rapport, sur la bienveillance habituelle des plus intelligents d’entre eux, non seulement en France, mais aussi dans tout notre Occident. Leur surveillance continue de ce libre enseignement populaire se bornera bientôt à y prescrire seulement la condition permanente d’une vraie positivité, en y écartant, avec une inflexible sévérité, l’introduction, trop imminente encore, des spéculations vagues ou sophistiques. Mais, à ce sujet, les besoins essentiels de l’école positive concourent directement avec les devoirs naturels des gouvernements : car, si ceux-ci doivent repousser un tel abus en vertu de sa tendance anarchique, celle-là, outre ce juste motif, le juge pleinement contraire à la destination fondamentale d’un tel enseignement, comme ranimant ce même esprit métaphysique où elle voit aujourd’hui le principal obstacle à l’avènement social de la nouvelle philosophie. Sous cet aspect, ainsi qu’à tout autre titre, les philosophes positifs se sentiront toujours presque aussi intéressés que les pouvoirs actuels au double maintien continu de l’ordre intérieur et de la paix extérieure, parce qu’ils y voient la condition la plus favorable à une vraie rénovation mentale et morale : seulement, du point de vue qui leur est propre, ils doivent apercevoir de plus loin ce qui pourrait compromettre ou consolider ce grand résultat politique de l’ensemble de notre situation transitoire.
\section[{III}]{III}\phantomsection
\label{III}\renewcommand{\leftmark}{III}

\noindent Nous avons maintenant assez caractérisé, à tous égards, l’importance capitale que présente aujourd’hui l’universelle propagation des études positives, surtout parmi les prolétaires, pour constituer désormais un indispensable point d’appui, à la fois mental et social, à l’élaboration philosophique qui doit déterminer graduellement la réorganisation spirituelle des sociétés modernes. Mais une telle appréciation resterait encore incomplète, et même insuffisante, si la fin de ce Discours n’était pas directement consacrée à établir l’ordre fondamental qui convient à cette série d’études, de manière à fixer la vraie position que doit occuper, dans leur ensemble, celle dont ce Traité s’occupera ensuite exclusivement. Loin que cet arrangement didactique soit presque indifférent, comme notre vicieux régime scientifique le fait trop souvent supposer, on peut assurer, au contraire, que c’est de lui surtout que dépend la principale efficacité, intellectuelle ou sociale, de cette grande préparation. Il existe d’ailleurs une intime solidarité entre la conception encyclopédique d’où il résulte et la loi fondamentale d’évolution qui sert de base à la nouvelle philosophie générale.\par
Un tel ordre doit, par sa nature, remplir deux, conditions essentielles, l’une dogmatique, l’autre historique, dont il faut d’abord reconnaître la convergence nécessaire : la première consiste à ranger les sciences suivant leur dépendance successive, en sorte que chacune repose sur la précédente et prépare la suivante ; la seconde prescrit de les disposer d’après la marche de leur formation effective, en passant toujours des plus anciennes aux plus récentes. Or, l’équivalence spontanée de ces deux voies encyclopédiques tient, en général, à l’identité fondamentale qui existe inévitablement entre l’évolution individuelle et l’évolution collective, lesquelles ayant une pareille origine, une semblable destination, et un même agent, doivent toujours offrir des phases correspondantes, sauf les seules diversités de durée, d’intensité et de vitesse, inhérentes à l’inégalité des deux organismes. Ce concours nécessaire permet donc de concevoir ces deux modes comme deux aspects corrélatifs d’un unique principe encyclopédique, de manière à pouvoir habituellement employer celui qui, en chaque cas, manifestera le mieux les relations considérées, et avec la précieuse faculté de pouvoir constamment vérifier par l’un ce qui sera résulté de l’autre.\par
La loi fondamentale de cet ordre commun, de dépendance dogmatique et de succession historique, a été complètement établie dans le grand ouvrage ci-dessus indiqué, et dont elle détermine le plan général. Elle consiste à classer les différentes sciences, d’après la nature des phénomènes étudiés, selon leur généralité et leur indépendance décroissantes ou leur complication croissante, d’où résultent des spéculations de moins en moins abstraites et de plus en plus difficiles, mais aussi de plus en plus éminentes et complètes, en vertu de leur relation plus intime à l’homme, ou plutôt à l’Humanité, objet final de tout le système théorique. Ce classement tire sa principale valeur philosophique, soit scientifique, soit logique, de l’identité constante et nécessaire qui existe entre tous ces divers modes de comparaison spéculative des phénomènes naturels, et d’où résultent autant de théorèmes encyclopédiques, dont l’explication et l’usage appartiennent à l’ouvrage cité, qui, en outre, sous le rapport actif, y ajoute cette importante relation générale, que les phénomènes deviennent ainsi de plus en plus modifiables, de façon à offrir un domaine de plus en plus vaste à l’intervention humaine. Il suffit ici d’indiquer sommairement l’application. de ce grand principe à la détermination rationnelle de la vraie hiérarchie des études fondamentales, directement conçues désormais comme les différents éléments essentiels d’une science unique, celle de l’Humanité.\par
Cet objet final de toutes nos spéculations réelles exige, évidemment, par sa nature, à la fois scientifique et logique, un double préambule indispensable, relatif, d’une part, à l’homme proprement dit, d’une autre part, au monde extérieur. On ne, saurait, en effet, étudier rationnellement les, phénomènes, statiques ou dynamiques, de la sociabilité, si d’abord, on ne connaît suffisamment l’agent spécial qui les opère, et le milieu général où ils s’accomplissent. De là résulte donc la division nécessaire de la philosophie naturelle, destinée à préparer la, philosophie sociale, en deux, grandes branches, l’une organique, l’autre inorganique. Quant à la disposition, relative de ces deux études également fondamentales, tous les motifs essentiels, soit scientifiques, soit logiques, concourent à prescrire, dans l’éducation individuelle et dans l’évolution collective, de commencer. par la seconde, dont les phénomènes, plus simples et plus indépendants, à raison de leur généralité supérieure, comportent seuls d’abord une appréciation vraiment positive, tandis que leurs lois, directement relatives à l’existence universelle, exercent ensuite une influence nécessaire sur l’existence spéciale des corps vivants. L’astronomie constitue nécessairement, à tous égards, l’élément le plus décisif de cette théorie préalable du monde extérieur, soit comme mieux susceptible d’une pleine positivité, soit en tant que caractérisant le milieu général de tous nos phénomènes quelconques, et manifestant, sans aucune autre complication, la simple existence mathématique, c’est-à-dire géométrique ou mécanique, commune à tous les êtres réels. Mais, même quand on condense le plus possible les vraies conceptions encyclopédiques, on ne saurait réduire la philosophie inorganique à cet élément principal, parce qu’elle resterait alors complètement isolée de la philosophie organique. Leur lien fondamental, scientifique et logique, consiste surtout dans la branche la plus complexe de la première, l’étude des phénomènes de composition et de décomposition, les plus éminents de ceux que comporte l’existence universelle, et les plus rapprochés du mode vital proprement dit. C’est ainsi que la philosophie naturelle, envisagée comme le préambule nécessaire de la philosophie sociale, se décomposant d’abord en deux études extrêmes et une étude intermédiaire, comprend successivement ces trois grandes sciences, l’astronomie, la chimie et la biologie, dont la première touche immédiatement à l’origine spontanée du véritable esprit scientifique, et la dernière à sa destination essentielle. Leur essor initial respectif se rapporte, historiquement, à l’antiquité grecque, au Moyen Âge, et à l’époque moderne.\par
Une telle appréciation encyclopédique ne remplirait pas encore suffisamment les conditions indispensables de continuité et de spontanéité propres à un tel sujet : d’une part, elle laisse une lacune capitale entre l’astronomie et la chimie, dont la liaison ne saurait être directe ; d’une autre part, elle n’indique pas assez la vraie source de ce système spéculatif, comme un simple prolongement abstrait de la raison commune, dont le point de départ scientifique ne pouvait être directement astronomique. Mais, pour compléter la formule fondamentale, il suffit, en premier lieu, de placer, au début de ce vaste ensemble, la science mathématique, seul berceau nécessaire de la positivité rationnelle, aussi bien pour l’individu que pour l’espèce. Si, par une application plus spéciale de notre principe encyclopédique, on décompose, à son tour, cette science initiale dans ses trois grandes branches, le calcul, la géométrie, et la mécanique, on détermine enfin, avec la dernière précision philosophique, la véritable origine de tout le système scientifique, d’abord issu, en effet, des spéculations purement numériques qui étant, de toutes, les plus générales, les plus simples, les plus abstraites, et les plus indépendantes, se confondent presque avec l’élan spontané de l’esprit positif chez les plus vulgaires intelligences, comme le confirme encore, sous nos yeux, l’observation journalière de l’essor individuel.\par
On parvient ainsi graduellement à découvrir l’invariable hiérarchie, à la fois historique et dogmatique, également scientifique et logique, des six sciences fondamentales, la mathématique, l’astronomie, la physique, la chimie, la biologie et la sociologie, dont la première constitue nécessairement le point de départ exclusif et la dernière le seul but essentiel de toute philosophie positive, envisagée désormais comme formant, par sa nature, un système vraiment indivisible, où toute décomposition est radicalement artificielle, sans être d’ailleurs nullement arbitraire, tout s’y rapportant finalement à l’Humanité, unique conception pleinement universelle. L’ensemble de cette formule encyclopédique, exactement conforme aux vraies affinités des études correspondantes et qui d’ailleurs comprend évidemment tous les éléments de nos spéculations réelles, permet enfin à chaque intelligence de renouveler à son gré l’histoire générale de l’esprit positif, en passant, d’une manière presque insensible, des moindres idées mathématiques aux plus hautes pensées sociales. Il est clair, en effet, que chacune des quatre sciences intermédiaires se confond, pour ainsi dire, avec la précédente quant à ses plus simples phénomènes, et avec la suivante quant aux plus éminents. Cette parfaite continuité spontanée deviendra surtout irrécusable à tous ceux qui reconnaîtront, dans l’ouvrage ci-dessus indiqué, que le même principe encyclopédique fournit aussi le classement rationnel des diverses parties constituantes de chaque étude fondamentale, en sorte que les degrés dogmatiques et les phases historiques peuvent se rapprocher autant que l’exige la précision des comparaisons ou la facilité des transitions.\par
Dans l’état présent des intelligences, l’application logique de cette grande formule est encore plus importante que son usage scientifique, la méthode étant, de nos jours, plus essentielle que la doctrine elle-même, et d’ailleurs seule immédiatement susceptible d’une pleine régénération. Sa principale utilité consiste donc aujourd’hui à déterminer rigoureusement la marche invariable de toute éducation vraiment positive, au milieu des préjugés irrationnels et des vicieuses habitudes propres à l’essor préliminaire du système scientifique, ainsi graduellement formé de théories partielles et incohérentes, dont les relations mutuelles devaient jusqu’ici rester inaperçues de leurs fondateurs successifs. Toutes les classes actuelles de savants violent maintenant, avec une égale gravité, quoiqu’à divers titres, cette obligation fondamentale. En se bornant ici à indiquer les deux cas extrêmes, les géomètres, justement fiers d’être placés à, la vraie source de la positivité rationnelle, s’obstinent aveuglément, à retenir l’esprit humain dans, ce degré purement initial du, véritable essor spéculatif, sans jamais considérer son unique but nécessaire ; au contraire, les biologistes préconisant à bon droit, la dignité supérieure de, leur sujet, immédiatement voisin de cette grande destination, persistent à tenir leurs études dans un irrationnel isolement, en s’affranchissant arbitrairement de la difficile préparation qu’exige leur nature. Ces dispositions opposées, mais également, empiriques, conduisent trop souvent aujourd’hui, chez les uns, à une vaine déperdition d’effort intellectuels, désormais consumés, en, majeure partie, en. recherches de plus en plus puériles ; chez les autres, à une instabilité continue des diverses nations essentielles, faute d’une marche vraiment positive. Sous ce, dernier aspect surtout, on doit remarquer, en effet, que les études sociales ne sont pas maintenant les seules restées encore extérieures au système pleinement. positif, sous la stérile domination de l’esprit théologico-métaphysique ; au fond, les études biologiques elles-mêmes, surtout dynamiques, quoiqu’elles soient académiquement constituées, n’ont pas non plus atteint jusqu’ici à une vraie positivité, puisqu’aucune doctrine capitale n’y est aujourd’hui suffisamment {\itshape ébauchée}, en sorte que le champ des illusions et des jongleries y demeure encore presque indéfini. Or, la déplorable prolongation d’une telle situation tient essentiellement, en l’un et l’autre cas, à l’insuffisant accomplissement des grandes conditions logiques déterminées par notre loi encyclopédique : car, personne n’y conteste plus, depuis longtemps, la nécessité d’une marche positive ; mais tous en méconnaissent la nature et les, obligations, que peut seule caractériser la vraie hiérarchie scientifique. Qu’attendre, en effet, soit envers, les phénomènes sociaux, soit même envers l’étude, plus simple, de la vie individuelle, d’une culture qui aborde directement des spéculations aussi complexes, sans s’y être dignement préparée par une saine appréciation des méthodes et des doctrines relatives aux divers phénomènes moins compliqués et plus généraux, de manière à ne pouvoir suffisamment connaître ni la logique inductive, principalement caractérisée, à l’état rudimentaire, par la chimie, la physique, et d’abord l’astronomie, ni même la pure logique déductive, ou l’art élémentaire du raisonnement décisif que l’initiation mathématique peut seule développer convenablement ?\par
Pour faciliter l’usage habituel de notre formule hiérarchique, il convient beaucoup, quand on n’a pas besoin d’une grande précision encyclopédique, d’y grouper les termes deux à deux, de façon à la réduire à trois couples, l’un initial, mathématico-astronomique, l’autre final, biologico-sociologique, séparés et réunis par le couple intermédiaire, physico-chimique. Cette heureuse condensation résulte d’une irrécusable appréciation, puisqu’il existe, en, effet, une plus grande affinité naturelle, soit scientifique, soit logique, entre les deux éléments de chaque couple qu’entre les couples consécutifs eux-mêmes ; comme le confirme souvent la difficulté qu’on éprouve à séparer nettement la mathématique de l’astronomie, et la physique de la chimie, par suite des habitudes vagues qui dominent encore envers toutes les pensées d’ensemble ; la biologie et la sociologie surtout continuent à se confondre presque chez la plupart des penseurs actuels. Sans aller jamais jusqu’à ces vicieuses confusions, qui altéreraient radicalement les transitions encyclopédiques, il sera fréquemment utile de réduire ainsi la hiérarchie élémentaire des spéculations réelles à trois couples essentiels, dont chacun pourra d’ailleurs être brièvement désigné d’après son élément le plus spécial, qui est toujours effectivement le plus caractéristique, et le plus propre à définir les grandes phases de l’évolution positive, individuelle ou collective.\par
Cette sommaire appréciation suffit ici pour indiquer la destination et signaler l’importance d’une telle loi encyclopédique, où réside finalement l’une des deux idées mères dont l’intime combinaison spontanée constitue nécessairement la base systématique de la nouvelle philosophie générale. La terminaison de ce long Discours, où le véritable esprit positif a été caractérisé sous tous les aspects essentiels se rapproche ainsi de son début, puisque cette théorie de classement doit être envisagée, en dernier lieu, comme naturellement inséparable de la théorie d’évolution exposée d’abord ; en sorte que le discours actuel forme lui-même un véritable ensemble, image fidèle, quoique très contractée, d’un vaste système. Il est aisé de comprendre, en effet, que la considération habituelle d’une telle hiérarchie doit devenir indispensable, soit pour appliquer convenablement notre loi initiale des trois états, soit pour dissiper suffisamment les seules objections sérieuses qu’elle puisse comporter ; car, la fréquente simultanéité historique des trois grandes phases mentales envers des spéculations différentes constituerait, de toute autre manière, une inexplicable anomalie, que résout, au contraire, spontanément, notre loi hiérarchique, aussi relative à la succession qu’à la dépendance des diverses études positives. On conçoit pareillement, en sens inverse, que la règle du classement suppose celle de l’évolution, puisque tous les motifs essentiels de l’ordre ainsi établi résultent, au fond, de l’inégale rapidité d’un tel développement chez les différentes sciences fondamentales.\par
La combinaison rationnelle de ces deux idées mères, en constituant l’unité nécessaire du système scientifique, dont toutes les parties concourent de, plus en plus à une même fin, assure aussi, d’une autre part, la juste indépendance des divers éléments principaux, trop souvent altérée encore par de vicieux rapprochements. Dans son essor préliminaire, seul accompli jusqu’ici, l’esprit positif ayant dû ainsi s’étendre graduellement des études inférieures aux études supérieures, celles-ci ont été inévitablement exposées à l’oppressive invasion des premières, contre l’ascendant desquelles leur indispensable originalité : ne trouvait d’abord de garantie que d’après, une prolongation exagérée de la tutelle théologico-métaphysique. Cette déplorable fluctuation, très sensible encore envers la science des corps vivants, caractérise aujourd’hui ce que contiennent de réel, au fond, les longues controverses, d’ailleurs si vaines à tout autre égard, entre le matérialisme et le spiritualisme, représentant d’une manière provisoire, sous des formes également vicieuses, les besoins, également graves, quoique malheureusement opposés jusqu’ici, de la réalité et de la dignité de nos spéculations, quelconques. Parvenu désormais à sa maturité systématique, l’esprit positif dissipe à la fois ces deux ordres d’aberrations en terminant ces stériles conflits, par la satisfaction simultanée de ces deux conditions vicieusement contraires, comme l’indique aussitôt notre hiérarchie scientifique combinée avec notre loi d’évolution, puisque chaque science ne peut parvenir à une vraie positivité qu’autant que l’originalité de son caractère propre est pleinement consolidée.\par
Une application directe de cette théorie encyclopédique, à la fois scientifique et logique, nous conduit enfin à définir exactement la nature et la destination de l’enseignement spécial auquel ce Traité est consacré. Il résulte, en effet, des explications précédentes, que la principale efficacité, d’abord. mentale, puis sociale, que nous devons aujourd’hui chercher, dans une sage propagation universelle des études positives, dépend nécessairement d’une stricte observance didactique de la loi hiérarchique. Pour chaque rapide initiation individuelle, comme pour la lente initiation collective, il restera toujours indispensable que l’esprit positif, développant son régime à mesure qu’il agrandit son domaine, s’élève peu à peu de l’état mathématique initial à l’état sociologique final, en parcourant successivement les quatre degrés intermédiaires, astronomique, physique, chimique et biologique. Aucune supériorité personnelle ne peut vraiment dispenser de cette gradation, fondamentale, au sujet de laquelle on n’a que trop l’occasion de constater aujourd’hui, chez de hautes intelligences, une irréparable lacune, qui a quelquefois neutralisé d’éminents efforts philosophiques. Une telle marche doit donc devenir encore plus indispensable dans l’éducation universelle, où les spécialités ont peu d’importance, et dont la principale utilité, plus logique que scientifique, exige essentiellement une pleine rationalité, surtout quand il s’agit de constituer enfin le vrai régime mental. Ainsi, cet enseignement populaire doit aujourd’hui rapporter principalement au couple scientifique initial, jusqu’à ce qu’il se trouve convenablement vulgarisé. C’est là que tous doivent d’abord puiser les vraies notions élémentaires de sa positivité générale, en acquérant les connaissances qui servent de base à toutes les autres spéculations réelles. Quoique cette stricte obligation conduise nécessairement à placer au début les études purement mathématiques, il faut pourtant considérer qu’il ne s’agit pas encore d’établir une systématisation directe et complète de l’instruction populaire, mais seulement d’imprimer convenablement l’impulsion philosophique qui doit y conduire. Dès lors, on reconnaît aisément qu’un tel mouvement doit surtout dépendre des études astronomiques, qui, par leur nature, offrent nécessairement la pleine manifestation du véritable esprit mathématique, dont elles constituent, au fond, la principale destination. Il y a d’autant moins d’inconvénients actuels à caractériser ainsi le couple initial par la seule astronomie, que les connaissances mathématiques vraiment indispensables à sa judicieuse vulgarisation sont déjà assez répandues ou assez faciles à acquérir pour qu’on puisse aujourd’hui se borner à les supposer résultées d’une préparation spontanée.\par
Cette prépondérance nécessaire de la science astronomique dans la première propagation systématique de l’initiation positive est pleinement conforme à l’influence historique d’une telle étude, principal moteur jusqu’ici des grandes révolutions intellectuelles. Le sentiment fondamental de l’invariabilité des lois naturelles devait, en effet, se développer d’abord envers les phénomènes les plus simples et les plus généraux, dont la régularité et la grandeur supérieures nous manifestent le seul ordre réel qui soit complètement indépendant de toute modification humaine. Avant même de comporter encore aucun caractère vraiment scientifique, cette classe de conceptions a surtout déterminé le passage décisif du fétichisme au polythéisme, partout résulté du culte des astres. Sa première ébauche mathématique, dans les écoles de Thalès et de Pythagore, a constitué ensuite la principale source mentale de la décadence du polythéisme et de l’ascendant du monothéisme. Enfin, l’essor systématique de la positivité moderne, tendant ouvertement à un nouveau régime philosophique, est essentiellement résulté de la grande rénovation astronomique commencée par Copernic, Kepler et Galilée. Il faut donc peu s’étonner que l’universelle initiation positive, sur laquelle doit s’appuyer l’avènement direct de la philosophie définitive, se trouve aussi dépendre d’abord d’une telle étude, d’après la conformité nécessaire de l’éducation individuelle à l’évolution collective. C’est là, sans doute, le dernier office fondamental qui doive lui être propre dans le développement général de la raison humaine, qui, une fois parvenue chez tous à une vraie positivité, devra marcher ensuite sous une nouvelle impulsion philosophique, directement émanée de la science finale, dès lors investie à jamais de sa présence normale. Telle est l’éminente utilité, non moins sociale que mentale, qu’il s’agit ici de retirer enfin d’une judicieuse exposition populaire du système actuel des saines études astronomiques.
 


% at least one empty page at end (for booklet couv)
\ifbooklet
  \pagestyle{empty}
  \clearpage
  % 2 empty pages maybe needed for 4e cover
  \ifnum\modulo{\value{page}}{4}=0 \hbox{}\newpage\hbox{}\newpage\fi
  \ifnum\modulo{\value{page}}{4}=1 \hbox{}\newpage\hbox{}\newpage\fi


  \hbox{}\newpage
  \ifodd\value{page}\hbox{}\newpage\fi
  {\centering\color{rubric}\bfseries\noindent\large
    Hurlus ? Qu’est-ce.\par
    \bigskip
  }
  \noindent Des bouquinistes électroniques, pour du texte libre à participations libres,
  téléchargeable gratuitement sur \href{https://hurlus.fr}{\dotuline{hurlus.fr}}.\par
  \bigskip
  \noindent Cette brochure a été produite par des éditeurs bénévoles.
  Elle n’est pas faite pour être possédée, mais pour être lue, et puis donnée.
  Que circule le texte !
  En page de garde, on peut ajouter une date, un lieu, un nom ;
  comme une fiche de bibliothèque en papier,
  pour suivre le voyage du texte. Qui sait, un jour, vous la retrouverez ?
  \par

  Ce texte a été choisi parce qu’une personne l’a aimé,
  ou haï, elle a pensé qu’il partipait à la formation de notre présent ;
  sans le souci de plaire, vendre, ou militer pour une cause.
  \par

  L’édition électronique est soigneuse, tant sur la technique
  que sur l’établissement du texte ; mais sans aucune prétention scolaire, au contraire.
  Le but est de s’adresser à tous, sans distinction de science ou de diplôme.
  Au plus direct ! (possible)
  \par

  Cet exemplaire en papier a été tiré sur une imprimante personnelle
   ou une photocopieuse. Tout le monde peut le faire.
  Il suffit de
  télécharger un fichier sur \href{https://hurlus.fr}{\dotuline{hurlus.fr}},
  d’imprimer, et agrafer ; puis de lire et donner.\par

  \bigskip

  \noindent PS : Les hurlus furent aussi des rebelles protestants qui cassaient les statues dans les églises catholiques. En 1566 démarra la révolte des gueux dans le pays de Lille. L’insurrection enflamma la région jusqu’à Anvers où les gueux de mer bloquèrent les bateaux espagnols.
  Ce fut une rare guerre de libération dont naquit un pays toujours libre : les Pays-Bas.
  En plat pays francophone, par contre, restèrent des bandes de huguenots, les hurlus, progressivement réprimés par la très catholique Espagne.
  Cette mémoire d’une défaite est éteinte, rallumons-la. Sortons les livres du culte universitaire, débusquons les idoles de l’époque, pour les démonter.
\fi

\end{document}
