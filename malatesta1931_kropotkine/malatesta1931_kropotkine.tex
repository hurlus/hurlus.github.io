%%%%%%%%%%%%%%%%%%%%%%%%%%%%%%%%%
% LaTeX model https://hurlus.fr %
%%%%%%%%%%%%%%%%%%%%%%%%%%%%%%%%%

% Needed before document class
\RequirePackage{pdftexcmds} % needed for tests expressions
\RequirePackage{fix-cm} % correct units

% Define mode
\def\mode{a4}

\newif\ifaiv % a4
\newif\ifav % a5
\newif\ifbooklet % booklet
\newif\ifcover % cover for booklet

\ifnum \strcmp{\mode}{cover}=0
  \covertrue
\else\ifnum \strcmp{\mode}{booklet}=0
  \booklettrue
\else\ifnum \strcmp{\mode}{a5}=0
  \avtrue
\else
  \aivtrue
\fi\fi\fi

\ifbooklet % do not enclose with {}
  \documentclass[french,twoside]{book} % ,notitlepage
  \usepackage[%
    papersize={105mm, 297mm},
    inner=12mm,
    outer=12mm,
    top=20mm,
    bottom=15mm,
    marginparsep=0pt,
  ]{geometry}
  \usepackage[fontsize=9.5pt]{scrextend} % for Roboto
\else\ifav
  \documentclass[french,twoside]{book} % ,notitlepage
  \usepackage[%
    a5paper,
    inner=25mm,
    outer=15mm,
    top=15mm,
    bottom=15mm,
    marginparsep=0pt,
  ]{geometry}
  \usepackage[fontsize=12pt]{scrextend}
\else% A4 2 cols
  \documentclass[twocolumn]{report}
  \usepackage[%
    a4paper,
    inner=15mm,
    outer=10mm,
    top=25mm,
    bottom=18mm,
    marginparsep=0pt,
  ]{geometry}
  \setlength{\columnsep}{20mm}
  \usepackage[fontsize=9.5pt]{scrextend}
\fi\fi

%%%%%%%%%%%%%%
% Alignments %
%%%%%%%%%%%%%%
% before teinte macros

\setlength{\arrayrulewidth}{0.2pt}
\setlength{\columnseprule}{\arrayrulewidth} % twocol
\setlength{\parskip}{0pt} % classical para with no margin
\setlength{\parindent}{1.5em}

%%%%%%%%%%
% Colors %
%%%%%%%%%%
% before Teinte macros

\usepackage[dvipsnames]{xcolor}
\definecolor{rubric}{HTML}{800000} % the tonic 0c71c3
\def\columnseprulecolor{\color{rubric}}
\colorlet{borderline}{rubric!30!} % definecolor need exact code
\definecolor{shadecolor}{gray}{0.95}
\definecolor{bghi}{gray}{0.5}

%%%%%%%%%%%%%%%%%
% Teinte macros %
%%%%%%%%%%%%%%%%%
%%%%%%%%%%%%%%%%%%%%%%%%%%%%%%%%%%%%%%%%%%%%%%%%%%%
% <TEI> generic (LaTeX names generated by Teinte) %
%%%%%%%%%%%%%%%%%%%%%%%%%%%%%%%%%%%%%%%%%%%%%%%%%%%
% This template is inserted in a specific design
% It is XeLaTeX and otf fonts

\makeatletter % <@@@


\usepackage{blindtext} % generate text for testing
\usepackage[strict]{changepage} % for modulo 4
\usepackage{contour} % rounding words
\usepackage[nodayofweek]{datetime}
% \usepackage{DejaVuSans} % seems buggy for sffont font for symbols
\usepackage{enumitem} % <list>
\usepackage{etoolbox} % patch commands
\usepackage{fancyvrb}
\usepackage{fancyhdr}
\usepackage{float}
\usepackage{fontspec} % XeLaTeX mandatory for fonts
\usepackage{footnote} % used to capture notes in minipage (ex: quote)
\usepackage{framed} % bordering correct with footnote hack
\usepackage{graphicx}
\usepackage{lettrine} % drop caps
\usepackage{lipsum} % generate text for testing
\usepackage[framemethod=tikz,]{mdframed} % maybe used for frame with footnotes inside
\usepackage{pdftexcmds} % needed for tests expressions
\usepackage{polyglossia} % non-break space french punct, bug Warning: "Failed to patch part"
\usepackage[%
  indentfirst=false,
  vskip=1em,
  noorphanfirst=true,
  noorphanafter=true,
  leftmargin=\parindent,
  rightmargin=0pt,
]{quoting}
\usepackage{ragged2e}
\usepackage{setspace} % \setstretch for <quote>
\usepackage{tabularx} % <table>
\usepackage[explicit]{titlesec} % wear titles, !NO implicit
\usepackage{tikz} % ornaments
\usepackage{tocloft} % styling tocs
\usepackage[fit]{truncate} % used im runing titles
\usepackage{unicode-math}
\usepackage[normalem]{ulem} % breakable \uline, normalem is absolutely necessary to keep \emph
\usepackage{verse} % <l>
\usepackage{xcolor} % named colors
\usepackage{xparse} % @ifundefined
\XeTeXdefaultencoding "iso-8859-1" % bad encoding of xstring
\usepackage{xstring} % string tests
\XeTeXdefaultencoding "utf-8"
\PassOptionsToPackage{hyphens}{url} % before hyperref, which load url package

% TOTEST
% \usepackage{hypcap} % links in caption ?
% \usepackage{marginnote}
% TESTED
% \usepackage{background} % doesn’t work with xetek
% \usepackage{bookmark} % prefers the hyperref hack \phantomsection
% \usepackage[color, leftbars]{changebar} % 2 cols doc, impossible to keep bar left
% \usepackage[utf8x]{inputenc} % inputenc package ignored with utf8 based engines
% \usepackage[sfdefault,medium]{inter} % no small caps
% \usepackage{firamath} % choose firasans instead, firamath unavailable in Ubuntu 21-04
% \usepackage{flushend} % bad for last notes, supposed flush end of columns
% \usepackage[stable]{footmisc} % BAD for complex notes https://texfaq.org/FAQ-ftnsect
% \usepackage{helvet} % not for XeLaTeX
% \usepackage{multicol} % not compatible with too much packages (longtable, framed, memoir…)
% \usepackage[default,oldstyle,scale=0.95]{opensans} % no small caps
% \usepackage{sectsty} % \chapterfont OBSOLETE
% \usepackage{soul} % \ul for underline, OBSOLETE with XeTeX
% \usepackage[breakable]{tcolorbox} % text styling gone, footnote hack not kept with breakable


% Metadata inserted by a program, from the TEI source, for title page and runing heads
\title{\textbf{ Sur Pierre Kropotkine, souvenirs et critiques d’un de ses vieux amis }}
\date{1931}
\author{Malatesta, Errico}
\def\elbibl{Malatesta, Errico. 1931. \emph{Sur Pierre Kropotkine, souvenirs et critiques d’un de ses vieux amis}}
\def\elsource{\href{http://www.theyliewedie.org/ressources/biblio/fr/Malatesta_Errico_-_Sur_Kropotkine.html}{\dotuline{http://www.theyliewedie.org/ressources/biblio/fr/Malatesta\_Errico\_-\_Sur\_Kropotkine.html}}\footnote{\href{http://www.theyliewedie.org/ressources/biblio/fr/Malatesta_Errico_-_Sur_Kropotkine.html}{\url{http://www.theyliewedie.org/ressources/biblio/fr/Malatesta_Errico_-_Sur_Kropotkine.html}}}\emph{Studi Sociali}, 15 avril 1931 in E. Malatesta, Scritti, vol. III, p. 368-379.}

% Default metas
\newcommand{\colorprovide}[2]{\@ifundefinedcolor{#1}{\colorlet{#1}{#2}}{}}
\colorprovide{rubric}{red}
\colorprovide{silver}{lightgray}
\@ifundefined{syms}{\newfontfamily\syms{DejaVu Sans}}{}
\newif\ifdev
\@ifundefined{elbibl}{% No meta defined, maybe dev mode
  \newcommand{\elbibl}{Titre court ?}
  \newcommand{\elbook}{Titre du livre source ?}
  \newcommand{\elabstract}{Résumé\par}
  \newcommand{\elurl}{http://oeuvres.github.io/elbook/2}
  \author{Éric Lœchien}
  \title{Un titre de test assez long pour vérifier le comportement d’une maquette}
  \date{1566}
  \devtrue
}{}
\let\eltitle\@title
\let\elauthor\@author
\let\eldate\@date


\defaultfontfeatures{
  % Mapping=tex-text, % no effect seen
  Scale=MatchLowercase,
  Ligatures={TeX,Common},
}


% generic typo commands
\newcommand{\astermono}{\medskip\centerline{\color{rubric}\large\selectfont{\syms ✻}}\medskip\par}%
\newcommand{\astertri}{\medskip\par\centerline{\color{rubric}\large\selectfont{\syms ✻\,✻\,✻}}\medskip\par}%
\newcommand{\asterism}{\bigskip\par\noindent\parbox{\linewidth}{\centering\color{rubric}\large{\syms ✻}\\{\syms ✻}\hskip 0.75em{\syms ✻}}\bigskip\par}%

% lists
\newlength{\listmod}
\setlength{\listmod}{\parindent}
\setlist{
  itemindent=!,
  listparindent=\listmod,
  labelsep=0.2\listmod,
  parsep=0pt,
  % topsep=0.2em, % default topsep is best
}
\setlist[itemize]{
  label=—,
  leftmargin=0pt,
  labelindent=1.2em,
  labelwidth=0pt,
}
\setlist[enumerate]{
  label={\bf\color{rubric}\arabic*.},
  labelindent=0.8\listmod,
  leftmargin=\listmod,
  labelwidth=0pt,
}
\newlist{listalpha}{enumerate}{1}
\setlist[listalpha]{
  label={\bf\color{rubric}\alph*.},
  leftmargin=0pt,
  labelindent=0.8\listmod,
  labelwidth=0pt,
}
\newcommand{\listhead}[1]{\hspace{-1\listmod}\emph{#1}}

\renewcommand{\hrulefill}{%
  \leavevmode\leaders\hrule height 0.2pt\hfill\kern\z@}

% General typo
\DeclareTextFontCommand{\textlarge}{\large}
\DeclareTextFontCommand{\textsmall}{\small}

% commands, inlines
\newcommand{\anchor}[1]{\Hy@raisedlink{\hypertarget{#1}{}}} % link to top of an anchor (not baseline)
\newcommand\abbr[1]{#1}
\newcommand{\autour}[1]{\tikz[baseline=(X.base)]\node [draw=rubric,thin,rectangle,inner sep=1.5pt, rounded corners=3pt] (X) {\color{rubric}#1};}
\newcommand\corr[1]{#1}
\newcommand{\ed}[1]{ {\color{silver}\sffamily\footnotesize (#1)} } % <milestone ed="1688"/>
\newcommand\expan[1]{#1}
\newcommand\foreign[1]{\emph{#1}}
\newcommand\gap[1]{#1}
\renewcommand{\LettrineFontHook}{\color{rubric}}
\newcommand{\initial}[2]{\lettrine[lines=2, loversize=0.3, lhang=0.3]{#1}{#2}}
\newcommand{\initialiv}[2]{%
  \let\oldLFH\LettrineFontHook
  % \renewcommand{\LettrineFontHook}{\color{rubric}\ttfamily}
  \IfSubStr{QJ’}{#1}{
    \lettrine[lines=4, lhang=0.2, loversize=-0.1, lraise=0.2]{\smash{#1}}{#2}
  }{\IfSubStr{É}{#1}{
    \lettrine[lines=4, lhang=0.2, loversize=-0, lraise=0]{\smash{#1}}{#2}
  }{\IfSubStr{ÀÂ}{#1}{
    \lettrine[lines=4, lhang=0.2, loversize=-0, lraise=0, slope=0.6em]{\smash{#1}}{#2}
  }{\IfSubStr{A}{#1}{
    \lettrine[lines=4, lhang=0.2, loversize=0.2, slope=0.6em]{\smash{#1}}{#2}
  }{\IfSubStr{V}{#1}{
    \lettrine[lines=4, lhang=0.2, loversize=0.2, slope=-0.5em]{\smash{#1}}{#2}
  }{
    \lettrine[lines=4, lhang=0.2, loversize=0.2]{\smash{#1}}{#2}
  }}}}}
  \let\LettrineFontHook\oldLFH
}
\newcommand{\labelchar}[1]{\textbf{\color{rubric} #1}}
\newcommand{\milestone}[1]{\autour{\footnotesize\color{rubric} #1}} % <milestone n="4"/>
\newcommand\name[1]{#1}
\newcommand\orig[1]{#1}
\newcommand\orgName[1]{#1}
\newcommand\persName[1]{#1}
\newcommand\placeName[1]{#1}
\newcommand{\pn}[1]{\IfSubStr{-—–¶}{#1}% <p n="3"/>
  {\noindent{\bfseries\color{rubric}   ¶  }}
  {{\footnotesize\autour{ #1}  }}}
\newcommand\reg{}
% \newcommand\ref{} % already defined
\newcommand\sic[1]{#1}
\newcommand\surname[1]{\textsc{#1}}
\newcommand\term[1]{\textbf{#1}}

\def\mednobreak{\ifdim\lastskip<\medskipamount
  \removelastskip\nopagebreak\medskip\fi}
\def\bignobreak{\ifdim\lastskip<\bigskipamount
  \removelastskip\nopagebreak\bigskip\fi}

% commands, blocks
\newcommand{\byline}[1]{\bigskip{\RaggedLeft{#1}\par}\bigskip}
\newcommand{\bibl}[1]{{\RaggedLeft{#1}\par\bigskip}}
\newcommand{\biblitem}[1]{{\noindent\hangindent=\parindent   #1\par}}
\newcommand{\dateline}[1]{\medskip{\RaggedLeft{#1}\par}\bigskip}
\newcommand{\labelblock}[1]{\medbreak{\noindent\color{rubric}\bfseries #1}\par\mednobreak}
\newcommand{\salute}[1]{\bigbreak{#1}\par\medbreak}
\newcommand{\signed}[1]{\bigbreak\filbreak{\raggedleft #1\par}\medskip}

% environments for blocks (some may become commands)
\newenvironment{borderbox}{}{} % framing content
\newenvironment{citbibl}{\ifvmode\hfill\fi}{\ifvmode\par\fi }
\newenvironment{docAuthor}{\ifvmode\vskip4pt\fontsize{16pt}{18pt}\selectfont\fi\itshape}{\ifvmode\par\fi }
\newenvironment{docDate}{}{\ifvmode\par\fi }
\newenvironment{docImprint}{\vskip6pt}{\ifvmode\par\fi }
\newenvironment{docTitle}{\vskip6pt\bfseries\fontsize{18pt}{22pt}\selectfont}{\par }
\newenvironment{msHead}{\vskip6pt}{\par}
\newenvironment{msItem}{\vskip6pt}{\par}
\newenvironment{titlePart}{}{\par }


% environments for block containers
\newenvironment{argument}{\itshape\parindent0pt}{\vskip1.5em}
\newenvironment{biblfree}{}{\ifvmode\par\fi }
\newenvironment{bibitemlist}[1]{%
  \list{\@biblabel{\@arabic\c@enumiv}}%
  {%
    \settowidth\labelwidth{\@biblabel{#1}}%
    \leftmargin\labelwidth
    \advance\leftmargin\labelsep
    \@openbib@code
    \usecounter{enumiv}%
    \let\p@enumiv\@empty
    \renewcommand\theenumiv{\@arabic\c@enumiv}%
  }
  \sloppy
  \clubpenalty4000
  \@clubpenalty \clubpenalty
  \widowpenalty4000%
  \sfcode`\.\@m
}%
{\def\@noitemerr
  {\@latex@warning{Empty `bibitemlist' environment}}%
\endlist}
\newenvironment{quoteblock}% may be used for ornaments
  {\begin{quoting}}
  {\end{quoting}}

% table () is preceded and finished by custom command
\newcommand{\tableopen}[1]{%
  \ifnum\strcmp{#1}{wide}=0{%
    \begin{center}
  }
  \else\ifnum\strcmp{#1}{long}=0{%
    \begin{center}
  }
  \else{%
    \begin{center}
  }
  \fi\fi
}
\newcommand{\tableclose}[1]{%
  \ifnum\strcmp{#1}{wide}=0{%
    \end{center}
  }
  \else\ifnum\strcmp{#1}{long}=0{%
    \end{center}
  }
  \else{%
    \end{center}
  }
  \fi\fi
}


% text structure
\newcommand\chapteropen{} % before chapter title
\newcommand\chaptercont{} % after title, argument, epigraph…
\newcommand\chapterclose{} % maybe useful for multicol settings
\setcounter{secnumdepth}{-2} % no counters for hierarchy titles
\setcounter{tocdepth}{5} % deep toc
\markright{\@title} % ???
\markboth{\@title}{\@author} % ???
\renewcommand\tableofcontents{\@starttoc{toc}}
% toclof format
% \renewcommand{\@tocrmarg}{0.1em} % Useless command?
% \renewcommand{\@pnumwidth}{0.5em} % {1.75em}
\renewcommand{\@cftmaketoctitle}{}
\setlength{\cftbeforesecskip}{\z@ \@plus.2\p@}
\renewcommand{\cftchapfont}{}
\renewcommand{\cftchapdotsep}{\cftdotsep}
\renewcommand{\cftchapleader}{\normalfont\cftdotfill{\cftchapdotsep}}
\renewcommand{\cftchappagefont}{\bfseries}
\setlength{\cftbeforechapskip}{0em \@plus\p@}
% \renewcommand{\cftsecfont}{\small\relax}
\renewcommand{\cftsecpagefont}{\normalfont}
% \renewcommand{\cftsubsecfont}{\small\relax}
\renewcommand{\cftsecdotsep}{\cftdotsep}
\renewcommand{\cftsecpagefont}{\normalfont}
\renewcommand{\cftsecleader}{\normalfont\cftdotfill{\cftsecdotsep}}
\setlength{\cftsecindent}{1em}
\setlength{\cftsubsecindent}{2em}
\setlength{\cftsubsubsecindent}{3em}
\setlength{\cftchapnumwidth}{1em}
\setlength{\cftsecnumwidth}{1em}
\setlength{\cftsubsecnumwidth}{1em}
\setlength{\cftsubsubsecnumwidth}{1em}

% footnotes
\newif\ifheading
\newcommand*{\fnmarkscale}{\ifheading 0.70 \else 1 \fi}
\renewcommand\footnoterule{\vspace*{0.3cm}\hrule height \arrayrulewidth width 3cm \vspace*{0.3cm}}
\setlength\footnotesep{1.5\footnotesep} % footnote separator
\renewcommand\@makefntext[1]{\parindent 1.5em \noindent \hb@xt@1.8em{\hss{\normalfont\@thefnmark . }}#1} % no superscipt in foot
\patchcmd{\@footnotetext}{\footnotesize}{\footnotesize\sffamily}{}{} % before scrextend, hyperref


%   see https://tex.stackexchange.com/a/34449/5049
\def\truncdiv#1#2{((#1-(#2-1)/2)/#2)}
\def\moduloop#1#2{(#1-\truncdiv{#1}{#2}*#2)}
\def\modulo#1#2{\number\numexpr\moduloop{#1}{#2}\relax}

% orphans and widows
\clubpenalty=9996
\widowpenalty=9999
\brokenpenalty=4991
\predisplaypenalty=10000
\postdisplaypenalty=1549
\displaywidowpenalty=1602
\hyphenpenalty=400
% Copied from Rahtz but not understood
\def\@pnumwidth{1.55em}
\def\@tocrmarg {2.55em}
\def\@dotsep{4.5}
\emergencystretch 3em
\hbadness=4000
\pretolerance=750
\tolerance=2000
\vbadness=4000
\def\Gin@extensions{.pdf,.png,.jpg,.mps,.tif}
% \renewcommand{\@cite}[1]{#1} % biblio

\usepackage{hyperref} % supposed to be the last one, :o) except for the ones to follow
\urlstyle{same} % after hyperref
\hypersetup{
  % pdftex, % no effect
  pdftitle={\elbibl},
  % pdfauthor={Your name here},
  % pdfsubject={Your subject here},
  % pdfkeywords={keyword1, keyword2},
  bookmarksnumbered=true,
  bookmarksopen=true,
  bookmarksopenlevel=1,
  pdfstartview=Fit,
  breaklinks=true, % avoid long links
  pdfpagemode=UseOutlines,    % pdf toc
  hyperfootnotes=true,
  colorlinks=false,
  pdfborder=0 0 0,
  % pdfpagelayout=TwoPageRight,
  % linktocpage=true, % NO, toc, link only on page no
}

\makeatother % /@@@>
%%%%%%%%%%%%%%
% </TEI> end %
%%%%%%%%%%%%%%


%%%%%%%%%%%%%
% footnotes %
%%%%%%%%%%%%%
\renewcommand{\thefootnote}{\bfseries\textcolor{rubric}{\arabic{footnote}}} % color for footnote marks

%%%%%%%%%
% Fonts %
%%%%%%%%%
\usepackage[]{roboto} % SmallCaps, Regular is a bit bold
% \linespread{0.90} % too compact, keep font natural
\newfontfamily\fontrun[]{Roboto Condensed Light} % condensed runing heads
\ifav
  \setmainfont[
    ItalicFont={Roboto Light Italic},
  ]{Roboto}
\else\ifbooklet
  \setmainfont[
    ItalicFont={Roboto Light Italic},
  ]{Roboto}
\else
\setmainfont[
  ItalicFont={Roboto Italic},
]{Roboto Light}
\fi\fi
\renewcommand{\LettrineFontHook}{\bfseries\color{rubric}}
% \renewenvironment{labelblock}{\begin{center}\bfseries\color{rubric}}{\end{center}}

%%%%%%%%
% MISC %
%%%%%%%%

\setdefaultlanguage[frenchpart=false]{french} % bug on part


\newenvironment{quotebar}{%
    \def\FrameCommand{{\color{rubric!10!}\vrule width 0.5em} \hspace{0.9em}}%
    \def\OuterFrameSep{\itemsep} % séparateur vertical
    \MakeFramed {\advance\hsize-\width \FrameRestore}
  }%
  {%
    \endMakeFramed
  }
\renewenvironment{quoteblock}% may be used for ornaments
  {%
    \savenotes
    \setstretch{0.9}
    \normalfont
    \begin{quotebar}
  }
  {%
    \end{quotebar}
    \spewnotes
  }


\renewcommand{\headrulewidth}{\arrayrulewidth}
\renewcommand{\headrule}{{\color{rubric}\hrule}}

% delicate tuning, image has produce line-height problems in title on 2 lines
\titleformat{name=\chapter} % command
  [display] % shape
  {\vspace{1.5em}\centering} % format
  {} % label
  {0pt} % separator between n
  {}
[{\color{rubric}\huge\textbf{#1}}\bigskip] % after code
% \titlespacing{command}{left spacing}{before spacing}{after spacing}[right]
\titlespacing*{\chapter}{0pt}{-2em}{0pt}[0pt]

\titleformat{name=\section}
  [block]{}{}{}{}
  [\vbox{\color{rubric}\large\raggedleft\textbf{#1}}]
\titlespacing{\section}{0pt}{0pt plus 4pt minus 2pt}{\baselineskip}

\titleformat{name=\subsection}
  [block]
  {}
  {} % \thesection
  {} % separator \arrayrulewidth
  {}
[\vbox{\large\textbf{#1}}]
% \titlespacing{\subsection}{0pt}{0pt plus 4pt minus 2pt}{\baselineskip}

\ifaiv
  \fancypagestyle{main}{%
    \fancyhf{}
    \setlength{\headheight}{1.5em}
    \fancyhead{} % reset head
    \fancyfoot{} % reset foot
    \fancyhead[L]{\truncate{0.45\headwidth}{\fontrun\elbibl}} % book ref
    \fancyhead[R]{\truncate{0.45\headwidth}{ \fontrun\nouppercase\leftmark}} % Chapter title
    \fancyhead[C]{\thepage}
  }
  \fancypagestyle{plain}{% apply to chapter
    \fancyhf{}% clear all header and footer fields
    \setlength{\headheight}{1.5em}
    \fancyhead[L]{\truncate{0.9\headwidth}{\fontrun\elbibl}}
    \fancyhead[R]{\thepage}
  }
\else
  \fancypagestyle{main}{%
    \fancyhf{}
    \setlength{\headheight}{1.5em}
    \fancyhead{} % reset head
    \fancyfoot{} % reset foot
    \fancyhead[RE]{\truncate{0.9\headwidth}{\fontrun\elbibl}} % book ref
    \fancyhead[LO]{\truncate{0.9\headwidth}{\fontrun\nouppercase\leftmark}} % Chapter title, \nouppercase needed
    \fancyhead[RO,LE]{\thepage}
  }
  \fancypagestyle{plain}{% apply to chapter
    \fancyhf{}% clear all header and footer fields
    \setlength{\headheight}{1.5em}
    \fancyhead[L]{\truncate{0.9\headwidth}{\fontrun\elbibl}}
    \fancyhead[R]{\thepage}
  }
\fi

\ifav % a5 only
  \titleclass{\section}{top}
\fi

\newcommand\chapo{{%
  \vspace*{-3em}
  \centering % no vskip ()
  {\Large\addfontfeature{LetterSpace=25}\bfseries{\elauthor}}\par
  \smallskip
  {\large\eldate}\par
  \bigskip
  {\Large\selectfont{\eltitle}}\par
  \bigskip
  {\color{rubric}\hline\par}
  \bigskip
  {\Large TEXTE LIBRE À PARTICPATION LIBRE\par}
  \centerline{\small\color{rubric} {hurlus.fr, tiré le \today}}\par
  \bigskip
}}

\newcommand\cover{{%
  \thispagestyle{empty}
  \centering
  {\LARGE\bfseries{\elauthor}}\par
  \bigskip
  {\Large\eldate}\par
  \bigskip
  \bigskip
  {\LARGE\selectfont{\eltitle}}\par
  \vfill\null
  {\color{rubric}\setlength{\arrayrulewidth}{2pt}\hline\par}
  \vfill\null
  {\Large TEXTE LIBRE À PARTICPATION LIBRE\par}
  \centerline{{\href{https://hurlus.fr}{\dotuline{hurlus.fr}}, tiré le \today}}\par
}}

\begin{document}
\pagestyle{empty}
\ifbooklet{
  \cover\newpage
  \thispagestyle{empty}\hbox{}\newpage
  \cover\newpage\noindent Les voyages de la brochure\par
  \bigskip
  \begin{tabularx}{\textwidth}{l|X|X}
    \textbf{Date} & \textbf{Lieu}& \textbf{Nom/pseudo} \\ \hline
    \rule{0pt}{25cm} &  &   \\
  \end{tabularx}
  \newpage
  \addtocounter{page}{-4}
}\fi

\thispagestyle{empty}
\ifaiv
  \twocolumn[\chapo]
\else
  \chapo
\fi
{\it\elabstract}
\bigskip
\makeatletter\@starttoc{toc}\makeatother % toc without new page
\bigskip

\pagestyle{main} % after style

  \chapter[{Sur Pierre Kropotkine, souvenirs et critiques d’un de ses vieux amis}]{Sur Pierre Kropotkine, \\
souvenirs et critiques d’un de ses vieux amis}
\noindent Pierre Kropotkine est, sans aucun doute, un de ceux qui ont le plus contribué, peut-être plus encore que Bakounine et Élisée Reclus, à l’élaboration et à la propagation de l’idée anarchiste. Il a donc bien mérité l’admiration et la reconnaissance que tous les anarchistes éprouvent pour lui.\par
Mais, par égard pour la vérité et dans l’intérêt supérieur de la cause, il faut reconnaître que l’ensemble de son oeuvre n’a pas été exclusivement bénéfique. Cela n’a pas été de sa faute, au contraire, ce furent ses éminents mérites qui produisirent les maux que j’ai l’intention de souligner.\par
Naturellement, Kropotkine, de même que toute autre personne, ne pouvait pas éviter toute erreur et embrasser toute la vérité. On aurait du donc profiter de sa précieuse contribution et continuer les recherches pour réaliser de nouveaux progrès. Mais ses qualités littéraires, la valeur et la taille de sa production, son infatigable activité, le prestige que lui procurait sa renommée d’homme de science, le fait qu’il avait sacrifié une position hautement privilégiée pour défendre, aux prix de souffrances et de dangers, la cause populaire, et de plus la fascination qu’exerçait sa personne (il enchantait tous ceux qui avaient la chance de l’approcher), tout cela lui donnait une telle notoriété et une telle influence qu’il apparut, et en grande partie c’était vrai, comme le maître reconnu tel par la grande majorité des anarchistes.\par
Ainsi, toute critique fut découragée et cela eut pour cause un arrêt dans le développement de l’idée. Pendant de nombreuses années, malgré l’esprit iconoclaste et progressiste des anarchistes, la plus grande partie de ceux-ci ne fit, pour tout ce qui touchait à la théorie et à la pratique, qu’étudier et répéter Kropotkine. Dire autre chose que lui aurait été pour beaucoup de camarades presque une hérésie.\par
Il apparaît donc nécessaire de soumettre les enseignements de Kropotkine à une critique sévère et sans préjugés pour distinguer ce qu’ils ont de toujours valable et vivant de ce que la théorie et l’expérience postérieures peuvent avoir démontré comme étant erroné. Cela d’ailleurs ne concernerait pas seulement Kropotkine, car les erreurs qu’on peut lui reprocher étaient déjà professées par les anarchistes avant que Kropotkine ait acquis une position éminente dans le mouvement. II ne fit que les confirmer et les entretenir en leur donnant l’appui de son talent et de son prestige, mais nous, les vieux militants, nous avons tous, ou presque tous, notre part de responsabilité.\par
En écrivant maintenant sur Kropotkine, je n’ai pas l’intention d’examiner à fond toute sa doctrine. Je veux seulement enregistrer quelques impressions et quelques souvenirs qui pourront servir, je crois, à faire mieux connaître sa personnalité morale et intellectuelle et aussi mieux comprendre ses mérites et ses défauts.\par
Mais d’abord je veux dire quelques mots qui me viennent du coeur, car je ne peux pas penser à Kropotkine sans être ému par le souvenir de son immense bonté. Je me rappelle ce qu’il a fait à Genève au cours de l’hiver 1875, pour aider un groupe de réfugiés italiens dans un état d’extrême misère et dont je faisais partie ; je me souviens des soins, que j’appellerais maternels, qu’il eut pour moi à Londres, une nuit, alors qu’après avoir été victime d’un accident, j’allais frapper à sa porte ; je me souviens des mille attentions envers tout le monde, je me souviens de l’atmosphère de cordialité qu’on respirait autour de lui. Car il était vraiment bon, d’une bonté presque inconsciente qui a besoin de réconforter toutes les souffrances et de répandre autour de lui le sourire et la joie. On aurait dit qu’il était bon sans le savoir ; en tout cas il ne voulait pas qu’on le dise et il se montra offensé parce que dans un article que j’écrivis à l’occasion de son 70e anniversaire, j’avais dit que la bonté était la première de ses qualités. Lui, il aimait plutôt montrer son énergie et sa fierté, peut-être que ces dernières qualités s’étaient développées en lui dans la lutte et pour la lutte, alors que la bonté était l’expression naturelle de sa nature intime.\par
Moi, j’ai eu l’honneur et la chance d’être pendant de nombreuses années lié à Kropotkine par une amitié des plus fraternelles.\par
Nous nous aimions parce que nous étions animés par la même passion, par la même espérance et par les mêmes... illusions.\par
Tous les deux de tempérament optimiste, (je crois cependant que l’optimisme de Kropotkine dépassait de beaucoup le mien et peut-être aussi avait une source différente) nous voyions les choses en rose, hélas trop en rose ; nous espérions, il y a déjà plus de 50 ans, en une révolution proche, qui aurait dû réaliser notre idéal. Pendant cette longue période il y a eu des moments de doute et de découragement. Je me rappelle, par exemple, qu’une fois Kropotkine me dit: « Mon cher Errico, je crains qu’il n’y ait plus que toi et moi qui croyons dans une proche révolution. » Mais il s’agissait là de moments passagers, bientôt la confiance réapparaissait, nous nous expliquions le scepticisme des camarades et nous continuions à travailler et à espérer.\par
Cependant, il ne faut pas croire que nous ayons les mêmes idées en tout. Au contraire, sur beaucoup de problèmes fondamentaux, nous étions loin d’être en accord. Et il n’y avait pratiquement pas une seule rencontre sans que prennent naissance entre nous des discussions bruyantes et irritées ; mais, comme Kropotkine était toujours sûr d’avoir raison et ne pouvait supporter avec calme qu’on le contredise, et que, par ailleurs, j’avais un respect très grand pour son savoir et beaucoup d’égards pour sa santé chancelante, nous finissions toujours par changer de conversation pour ne pas trop nous énerver.\par
Mais tout cela ne gênait pas l’intimité de nos rapports, parce que nous nous aimions et nous collaborions ensemble pour des raisons sentimentales plutôt qu’intellectuelles. Quelles que fussent les divergences dans les façons d’expliquer les choses et les arguments avec lesquels nous justifiions notre conduite, dans la pratique nous voulions la même chose et étions animés par le même désir intense de liberté, de justice, de bien-être pour tous. Nous pouvions donc nous entendre.\par
En effet, il n’y eut jamais entre nous de désaccord sérieux jusqu’en 1914, quand il fallut résoudre un problème de conduite pratique d’une importance capitale autant pour moi que pour lui : celle de l’attitude que les anarchistes devaient assumer face à la guerre. En cette funeste occasion se réveillèrent et s’exaspérèrent chez Kropotkine les anciennes préférences pour tout ce qui était russe ou français et il se proclama un partisan passionné de \emph{L'Entente.} Il parut oublier qu’il était internationaliste, socialiste et anarchiste, oublia tout ce que lui-même avait dit peu de temps auparavant sur la guerre que les capitalistes étaient en train de préparer et il se mit à admirer les pires hommes d’État et les généraux de \emph{L'Entente} ; il traita de lâches les anarchistes qui refusaient d’entrer dans \emph{l’Union sacrée}, tout en déplorant que l’âge et sa santé l’empêchaient de prendre un fusil et de marcher contre les Allemands. Il n’y avait donc pas de possibilité d’entente. Pour moi il s’agissait d’un vrai cas pathologique. De toute façon ce fut un des moments les plus douloureux, les plus tragiques de ma vie (et j’ose dire aussi de la sienne), celui de notre séparation après une discussion exagérément pénible ; nous nous séparâmes comme des adversaires, presque des ennemis.\par
Ma douleur fut grande à cause de la perte de l’ami et à cause du dommage qu’allait subir l’Idée, conséquence du désarroi que jetterait parmi les camarades une telle défection. Mais, malgré tout, l’amour et l’estime pour l’homme restèrent intactes ; j’avais aussi l’espoir que, passé l’enivrement du moment, et vu les conséquences prévisibles de la guerre, il reconnaîtrait son erreur et redeviendrait parmi nous le Kropotkine de toujours.\par
Kropotkine était en même temps un savant et un réformateur social. Il était possédé par deux passions : le désir de connaître et le désir de faire le bien de l’humanité, deux passions nobles qui peuvent être utiles l’une à l’autre et qu’on voudrait voir en tous les hommes sans qu’elles soient pour autant une seule et même chose. Mais Kropotkine était un tempérament systématique au plus haut degré et voulait tout expliquer par un même principe et tout réduire à l’unité, et il le faisait souvent, à mon avis, aux dépens de la logique. C'est pour cela qu’il appuyait toutes ses aspirations sociales sur la science ; aspirations qui n’étaient, selon lui, que des déductions rigoureusement scientifiques.\par
Je n’ai aucune compétence spéciale pour juger Kropotkine en tant qu’homme de science. Je sais qu’il avait dans sa jeunesse rendu des services remarquables à la géographie et à la géologie, j’apprécie la grande valeur de son livre sur « l’entr'aide » et je suis convaincu qu’il aurait pu, avec sa vaste culture et sa haute intelligence, donner une plus grande contribution aux progrès scientifiques si son attention et son activité n’avaient pas été absorbées par la lutte sociale. Cependant, il me semble qu’il lui manquait quelque chose pour être un vrai homme de science : la capacité d’oublier ses désirs, ses partis pris, pour observer les faits avec une impassible objectivité. Il me paraissait plutôt ce que j’appellerais volontiers un poète de la science. Il aurait pu, par des intuitions géniales, entrevoir de nouvelles vérités qui auraient pu être vérifiées par d’autres ayant moins ou point de génie, mais qui auraient été mieux dotés de ce qu’on appelle l’esprit scientifique. Kropotkine était trop passionné pour être un observateur exact.\par
D'habitude il concevait une hypothèse et cherchait par la suite les faits qui auraient dû la justifier. C'était peut-être une bonne méthode pour découvrir des choses nouvelles mais il lui arrivait de ne pas s’apercevoir des faits qui contredisaient cette hypothèse.\par
Il ne savait pas se décider à admettre un fait et ne savait même pas souvent le prendre en considération, si auparavant il n’était pas arrivé à l’expliquer, c’est à dire à le faire entrer dans son système. [...]\par
Avec cette disposition d’esprit qui le poussait à accommoder les choses à sa guise dans les problèmes scientifiques, dans lesquels il n’y a pas de passions qui puissent influencer l’intellect, on pouvait prévoir ce qui se serait passé pour des questions qui auraient touché de près ses plus grands désirs et espérances.\par
Kropotkine professait la philosophie matérialiste qui prévalait parmi les savants de la 2e moitié du XIX e siècle : Moleschott, Buchner, Vogt, et par conséquent sa conception de l’univers était rigoureusement mécanique.\par
Suivant son système, la volonté (puissance créatrice dont nous ne pouvons pas comprendre la nature et l’origine, comme d’ailleurs nous ne comprenons pas la nature et l’origine de la matière et de tous les autres « principes premiers ») la volonté donc, qui contribue peu ou prou à déterminer la conduite des individus et des sociétés, n’existait pas et n’était qu’une illusion. Tout ce qui fut, qui est et qui sera, des cours des planètes, de la naissance d’une civilisation à sa décadence, du parfum d’une rose au sourire d’une mère, d’un tremblement de terre à la pensée de Newton, de la cruauté d’un tyran à la bonté d’un saint, tout devait, doit et devra se produire par un enchaînement fatal de causes et d’effets de nature mécanique, qui ne laissent aucune possibilité de variation. L'illusion de la volonté ne saurait être elle-même qu’un fait mécanique.\par
Naturellement, si la volonté n’a aucune puissance et si tout est nécessaire et que rien ne peut être autrement, les idées de liberté et de justice n’ont plus aucune signification, ne correspondent à rien de réel.\par
Suivant cette logique, on ne pourrait que contempler ce qui se passe dans le monde avec indifférence, plaisir ou douleur, selon sa propre sensibilité mais sans aucun espoir et sans possibilité de changement.\par
Kropotkine donc, qui se montrait très sévère envers le fatalisme marxiste, tombait ensuite dans un fatalisme mécanique qui paraît bien plus paralysant.\par
Mais la philosophie ne pouvait pas tuer la puissante volonté qui animait Kropotkine. II était trop convaincu de la vérité de son système pour y renoncer ou simplement supporter tranquillement qu’on puisse le mettre en doute, mais il était trop désireux de liberté et de justice pour se laisser arrêter par les difficultés d’une contradiction logique et pour renoncer à la lutte. Il s’en sortait en introduisant l’anarchie dans son système et en en faisant une vérité scientifique.\par
Il se confirmait dans ses convictions en soutenant que toutes les découvertes récentes, dans toutes les sciences, de l’astronomie à la biologie et à la sociologie, permettaient de démontrer de plus en plus que l’anarchie était le mode d’organisation sociale imposé par les lois naturelles.\par
A cela on pouvait répondre que quelles que puissent être les conclusions qu’il pouvait tirer de la science contemporaine, il était certain que si de nouvelles découvertes étaient venues détruire les croyances scientifiques actuelles, il serait resté anarchiste malgré la science ; tout comme il était anarchiste en dépit de la logique. Mais Kropotkine n’aurait pas su admettre un conflit entre la science et ses aspirations sociales et il aurait inventé un moyen, peu importe qu’il eut été logique ou non, pour concilier sa philosophie mécanique avec son anarchisme.\par
Ainsi, après avoir affirmé que « l’anarchisme est une conception de l’univers fondée sur l’interpénétration mécanique des phénomènes embrassant toute la nature y compris la vie en société », (j’avoue n’être jamais arrivé à comprendre ce que cela signifiait) Kropotkine oubliait sa conception mécanique et se lançait dans la lutte avec la verve, l’enthousiasme et la confiance de quelqu’un qui croit en l’efficacité de la volonté et espère pouvoir par son activité contribuer à obtenir ce qu’il désire.\par
En réalité l’anarchisme et le communisme de Kropotkine, avant d’être une question raisonnée, étaient l’effet de sa sensibilité. En lui parlait d’abord le coeur et ensuite la raison pour renforcer et justifier les mouvements du coeur. [...]\par
Parmi les différentes façons de concevoir l’anarchie, il avait choisi et fait sien le programme communiste anarchiste qui, en se fondant sur la solidarité et l’amour, va au-delà de la justice.\par
Mais naturellement, comme il était à prévoir, sa philosophie n’était pas sans influencer sa façon d’envisager l’avenir et la lutte à mener pour y arriver.\par
Puisque, suivant sa philosophie, tout ce qui se produit devait nécessairement se produire, ainsi, même le communisme anarchiste qu’il désirait, devait inéluctablement triompher comme si c’était une loi de la nature.\par
Cette vision lui ôtait tout doute et lui cachait toute difficulté. Le monde bourgeois devait fatalement s’écrouler ; il était déjà en dissolution et l’action révolutionnaire ne devait servir qu’à hâter sa chute.\par
Sa grande influence en tant que propagandiste, ses talents mis à part, lui venait du fait qu’il montrait la chose d’une manière tellement simple, tellement facile, tellement inévitable, que l’enthousiasme se communiquait tout de suite à ceux qui l’écoutaient ou le lisaient.\par
Tout problème « moral » disparaissait puisqu’il attribuait au « peuple », à la masse des travailleurs, toutes les vertus et toutes les capacités. Il exaltait, avec raison, l’influence moralisatrice du travail mais ne voyait pas assez les effets déprimants et la corruption que la misère et la sujétion engendraient. Il pensait qu’il aurait suffi d’abolir les privilèges des capitalistes et le pouvoir des gouvernements pour que tous les hommes commencent à s’aimer comme des frères et se chargent des intérêts des autres comme s’ils étaient les leurs.\par
De même, il ne voyait pas les difficultés matérielles ou, en tout cas, s’en souciait peu. Il avait fait sienne l’idée, commune alors aux anarchistes, que les produits accumulés par le travail de la terre ou de l’industrie étaient tellement importants que l’on n’aurait pas besoin de s’occuper de la production ; il disait toujours que le problème immédiat était celui de la consommation et que pour faire triompher la révolution il fallait satisfaire, tout de suite et largement, les besoins de tous et que la production suivrait la consommation. De là l’idée de la « prise au tas » qu’il mit à la mode et qui est la façon la plus simple de concevoir le communisme et la plus apte à plaire aux foules, mais c’est aussi la plus primitive et la plus réellement utopiste. Et quand on lui faisait observer que cette accumulation de produits ne pouvait pas exister parce que les propriétaires normalement ne font produire que ce qu’ils peuvent vendre avec profit et que peut-être, aux premiers temps de la révolution, il faudrait organiser le rationnement et inciter à la production intensive plutôt qu’inciter les gens à puiser dans un tas qui n’existerait pas, il se mit à étudier directement la question et arriva à la conclusion que, en effet, l’abondance qu’il envisageait n’existait pas et que dans certains pays on était toujours sous la menace de la famine. Mais il se consolait en pensant aux grandes possibilités de l’agriculture aidée par la science. Il prit comme exemple les quelques résultats obtenus par certains agriculteurs et des éminents agronomes sur des étendues limitées et en tira les plus encourageantes conclusions sans se soucier des obstacles qu’aurait posé l’ignorance et l’aversion au changement des paysans, sans se soucier non plus du temps qu’il aurait fallu, de toute façon, pour généraliser les nouveaux moyens de culture et de distribution.\par
Comme toujours, Kropotkine voyait les choses comme nous tous espérions les voir un jour ; il considérait possible ou immédiatement réalisable ce qui doit être conquis par de longs et douloureux efforts.\par
Au fond, Kropotkine concevait la Nature comme une espèce de Providence grâce à laquelle tout devait devenir harmonieux, y compris les sociétés humaines.\par
C'est cela qui fit répéter à beaucoup d’anarchistes cette phrase au goût typiquement kropotkinien : « L'anarchie est l’ordre naturel. »\par
On pourrait se demander, je pense, comment se fait-il que la nature, s’il est vrai que sa loi est l’harmonie, ait attendu que viennent au monde les anarchistes et qu’elle attende encore qu’ils triomphent pour détruire les terribles et meurtrières dissonances dont les hommes ont toujours souffert ?\par
Ne serait-on pas plus proche de la réalité en disant que l’anarchie est la lutte, dans les sociétés humaines, contre les discordances de la Nature ?\par
J'ai insisté sur les deux erreurs dans lesquelles selon moi est tombé Kropotkine : son fatalisme théorique et son optimisme excessif, parce que je crois avoir constaté les mauvais effets qu’ils ont produit dans notre mouvement.\par
Il y a des camarades qui prirent au sérieux la théorie fataliste que par un euphémisme ils appelèrent déterminisme, et ils perdirent toute velléité révolutionnaire. La révolution, disaient-ils, ne se fait pas, elle se produira quand l’heure sera venue, et il est inutile, antiscientifique et parfois ridicule, de vouloir la faire. Et avec ces bonnes raisons, ils s’éloignèrent duÊmouvement et s’occupèrent de leurs affaires. Mais ce serait une erreur de croire qu’il s’agissait là d’une excuse commode pour se retirer de la lutte.ÊJ'ai connu de nombreux camarades au tempérament ardent, prêts à toutesÊles équipées, qui se sont exposés à de grands périls et ont sacrifié leur liberté et même leur propre existence au nom de l’anarchie, tout en étant convaincus de l’inutilité de leur action. Ils l’ont fait par dégoût de la société actuelle, par vengeance, par désespoir, par amour du beau geste, mais sans croire que tout cela puisse rendre service à la cause de la révolution et, par conséquent, sans choisir la cible et sans se soucier de coordonner leurs actions avec celles des autres.\par
D'un autre côté, ceux qui, sans s’occuper de philosophie, ont bien voulu travailler pour hâter et faire la révolution, ont cru que la chose était bien plus facile que ce qu’elle était en réalité ; ils n’ont pas prévu les difficultés et, n’ayant pas fait les préparations nécessaires, ils se sont re trouvés impuissants le jour où il y avait peut-être la possibilité de faire quelque chose de pratique.\par
Puissent les erreurs du passé servir de leçon pour faire mieux à l’avenir.\par
J'ai terminé. Je ne pense pas que mes critiques puissent diminuer la figure de Kropotkine qui reste, malgré tout, une des gloires les plus pures de notre mouvement. Les critiques serviront, si elles sont justes, à montrer qu’aucun homme n’est exempt d’erreurs, même s’il a la haute intelligence et le coeur héroïque d’un Kropotkine. De toute façon les anarchistes trouveront toujours dans ses écrits un trésor d’idées fécondes et dans son existence un exemple et un stimulant dans la lutte pour le bien.
 


% at least one empty page at end (for booklet couv)
\ifbooklet
  \pagestyle{empty}
  \clearpage
  % 2 empty pages maybe needed for 4e cover
  \ifnum\modulo{\value{page}}{4}=0 \hbox{}\newpage\hbox{}\newpage\fi
  \ifnum\modulo{\value{page}}{4}=1 \hbox{}\newpage\hbox{}\newpage\fi


  \hbox{}\newpage
  \ifodd\value{page}\hbox{}\newpage\fi
  {\centering\color{rubric}\bfseries\noindent\large
    Hurlus ? Qu’est-ce.\par
    \bigskip
  }
  \noindent Des bouquinistes électroniques, pour du texte libre à participation libre,
  téléchargeable gratuitement sur \href{https://hurlus.fr}{\dotuline{hurlus.fr}}.\par
  \bigskip
  \noindent Cette brochure a été produite par des éditeurs bénévoles.
  Elle n’est pas faîte pour être possédée, mais pour être lue, et puis donnée.
  Que circule le texte !
  En page de garde, on peut ajouter une date, un lieu, un nom ; pour suivre le voyage des idées.
  \par

  Ce texte a été choisi parce qu’une personne l’a aimé,
  ou haï, elle a en tous cas pensé qu’il partipait à la formation de notre présent ;
  sans le souci de plaire, vendre, ou militer pour une cause.
  \par

  L’édition électronique est soigneuse, tant sur la technique
  que sur l’établissement du texte ; mais sans aucune prétention scolaire, au contraire.
  Le but est de s’adresser à tous, sans distinction de science ou de diplôme.
  Au plus direct ! (possible)
  \par

  Cet exemplaire en papier a été tiré sur une imprimante personnelle
   ou une photocopieuse. Tout le monde peut le faire.
  Il suffit de
  télécharger un fichier sur \href{https://hurlus.fr}{\dotuline{hurlus.fr}},
  d’imprimer, et agrafer ; puis de lire et donner.\par

  \bigskip

  \noindent PS : Les hurlus furent aussi des rebelles protestants qui cassaient les statues dans les églises catholiques. En 1566 démarra la révolte des gueux dans le pays de Lille. L’insurrection enflamma la région jusqu’à Anvers où les gueux de mer bloquèrent les bateaux espagnols.
  Ce fut une rare guerre de libération dont naquit un pays toujours libre : les Pays-Bas.
  En plat pays francophone, par contre, restèrent des bandes de huguenots, les hurlus, progressivement réprimés par la très catholique Espagne.
  Cette mémoire d’une défaite est éteinte, rallumons-la. Sortons les livres du culte universitaire, cherchons les idoles de l’époque, pour les briser.
\fi

\ifdev % autotext in dev mode
\fontname\font — \textsc{Les règles du jeu}\par
(\hyperref[utopie]{\underline{Lien}})\par
\noindent \initialiv{A}{lors là}\blindtext\par
\noindent \initialiv{À}{ la bonheur des dames}\blindtext\par
\noindent \initialiv{É}{tonnez-le}\blindtext\par
\noindent \initialiv{Q}{ualitativement}\blindtext\par
\noindent \initialiv{V}{aloriser}\blindtext\par
\Blindtext
\phantomsection
\label{utopie}
\Blinddocument
\fi
\end{document}
