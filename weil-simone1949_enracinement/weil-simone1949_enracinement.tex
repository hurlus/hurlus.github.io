%%%%%%%%%%%%%%%%%%%%%%%%%%%%%%%%%
% LaTeX model https://hurlus.fr %
%%%%%%%%%%%%%%%%%%%%%%%%%%%%%%%%%

% Needed before document class
\RequirePackage{pdftexcmds} % needed for tests expressions
\RequirePackage{fix-cm} % correct units

% Define mode
\def\mode{a4}

\newif\ifaiv % a4
\newif\ifav % a5
\newif\ifbooklet % booklet
\newif\ifcover % cover for booklet

\ifnum \strcmp{\mode}{cover}=0
  \covertrue
\else\ifnum \strcmp{\mode}{booklet}=0
  \booklettrue
\else\ifnum \strcmp{\mode}{a5}=0
  \avtrue
\else
  \aivtrue
\fi\fi\fi

\ifbooklet % do not enclose with {}
  \documentclass[french,twoside]{book} % ,notitlepage
  \usepackage[%
    papersize={105mm, 297mm},
    inner=12mm,
    outer=12mm,
    top=20mm,
    bottom=15mm,
    marginparsep=0pt,
  ]{geometry}
  \usepackage[fontsize=9.5pt]{scrextend} % for Roboto
\else\ifav
  \documentclass[french,twoside]{book} % ,notitlepage
  \usepackage[%
    a5paper,
    inner=25mm,
    outer=15mm,
    top=15mm,
    bottom=15mm,
    marginparsep=0pt,
  ]{geometry}
  \usepackage[fontsize=12pt]{scrextend}
\else% A4 2 cols
  \documentclass[twocolumn]{report}
  \usepackage[%
    a4paper,
    inner=15mm,
    outer=10mm,
    top=25mm,
    bottom=18mm,
    marginparsep=0pt,
  ]{geometry}
  \setlength{\columnsep}{20mm}
  \usepackage[fontsize=9.5pt]{scrextend}
\fi\fi

%%%%%%%%%%%%%%
% Alignments %
%%%%%%%%%%%%%%
% before teinte macros

\setlength{\arrayrulewidth}{0.2pt}
\setlength{\columnseprule}{\arrayrulewidth} % twocol
\setlength{\parskip}{0pt} % classical para with no margin
\setlength{\parindent}{1.5em}

%%%%%%%%%%
% Colors %
%%%%%%%%%%
% before Teinte macros

\usepackage[dvipsnames]{xcolor}
\definecolor{rubric}{HTML}{800000} % the tonic 0c71c3
\def\columnseprulecolor{\color{rubric}}
\colorlet{borderline}{rubric!30!} % definecolor need exact code
\definecolor{shadecolor}{gray}{0.95}
\definecolor{bghi}{gray}{0.5}

%%%%%%%%%%%%%%%%%
% Teinte macros %
%%%%%%%%%%%%%%%%%
%%%%%%%%%%%%%%%%%%%%%%%%%%%%%%%%%%%%%%%%%%%%%%%%%%%
% <TEI> generic (LaTeX names generated by Teinte) %
%%%%%%%%%%%%%%%%%%%%%%%%%%%%%%%%%%%%%%%%%%%%%%%%%%%
% This template is inserted in a specific design
% It is XeLaTeX and otf fonts

\makeatletter % <@@@


\usepackage{blindtext} % generate text for testing
\usepackage[strict]{changepage} % for modulo 4
\usepackage{contour} % rounding words
\usepackage[nodayofweek]{datetime}
% \usepackage{DejaVuSans} % seems buggy for sffont font for symbols
\usepackage{enumitem} % <list>
\usepackage{etoolbox} % patch commands
\usepackage{fancyvrb}
\usepackage{fancyhdr}
\usepackage{float}
\usepackage{fontspec} % XeLaTeX mandatory for fonts
\usepackage{footnote} % used to capture notes in minipage (ex: quote)
\usepackage{framed} % bordering correct with footnote hack
\usepackage{graphicx}
\usepackage{lettrine} % drop caps
\usepackage{lipsum} % generate text for testing
\usepackage[framemethod=tikz,]{mdframed} % maybe used for frame with footnotes inside
\usepackage{pdftexcmds} % needed for tests expressions
\usepackage{polyglossia} % non-break space french punct, bug Warning: "Failed to patch part"
\usepackage[%
  indentfirst=false,
  vskip=1em,
  noorphanfirst=true,
  noorphanafter=true,
  leftmargin=\parindent,
  rightmargin=0pt,
]{quoting}
\usepackage{ragged2e}
\usepackage{setspace} % \setstretch for <quote>
\usepackage{tabularx} % <table>
\usepackage[explicit]{titlesec} % wear titles, !NO implicit
\usepackage{tikz} % ornaments
\usepackage{tocloft} % styling tocs
\usepackage[fit]{truncate} % used im runing titles
\usepackage{unicode-math}
\usepackage[normalem]{ulem} % breakable \uline, normalem is absolutely necessary to keep \emph
\usepackage{verse} % <l>
\usepackage{xcolor} % named colors
\usepackage{xparse} % @ifundefined
\XeTeXdefaultencoding "iso-8859-1" % bad encoding of xstring
\usepackage{xstring} % string tests
\XeTeXdefaultencoding "utf-8"
\PassOptionsToPackage{hyphens}{url} % before hyperref, which load url package

% TOTEST
% \usepackage{hypcap} % links in caption ?
% \usepackage{marginnote}
% TESTED
% \usepackage{background} % doesn’t work with xetek
% \usepackage{bookmark} % prefers the hyperref hack \phantomsection
% \usepackage[color, leftbars]{changebar} % 2 cols doc, impossible to keep bar left
% \usepackage[utf8x]{inputenc} % inputenc package ignored with utf8 based engines
% \usepackage[sfdefault,medium]{inter} % no small caps
% \usepackage{firamath} % choose firasans instead, firamath unavailable in Ubuntu 21-04
% \usepackage{flushend} % bad for last notes, supposed flush end of columns
% \usepackage[stable]{footmisc} % BAD for complex notes https://texfaq.org/FAQ-ftnsect
% \usepackage{helvet} % not for XeLaTeX
% \usepackage{multicol} % not compatible with too much packages (longtable, framed, memoir…)
% \usepackage[default,oldstyle,scale=0.95]{opensans} % no small caps
% \usepackage{sectsty} % \chapterfont OBSOLETE
% \usepackage{soul} % \ul for underline, OBSOLETE with XeTeX
% \usepackage[breakable]{tcolorbox} % text styling gone, footnote hack not kept with breakable


% Metadata inserted by a program, from the TEI source, for title page and runing heads
\title{\textbf{ L'enracinement }\\ \medskip
\textbf{ Prélude à une déclaration des devoirs envers l’être humain }}
\date{1949}
\author{Simone Weil}
\def\elbibl{Simone Weil. 1949. \emph{L'enracinement}}
\def\elsource{Simone Weil. \emph{L'enracinement. Prélude à une déclaration des devoirs envers l'être humain (1949).}}

% Default metas
\newcommand{\colorprovide}[2]{\@ifundefinedcolor{#1}{\colorlet{#1}{#2}}{}}
\colorprovide{rubric}{red}
\colorprovide{silver}{lightgray}
\@ifundefined{syms}{\newfontfamily\syms{DejaVu Sans}}{}
\newif\ifdev
\@ifundefined{elbibl}{% No meta defined, maybe dev mode
  \newcommand{\elbibl}{Titre court ?}
  \newcommand{\elbook}{Titre du livre source ?}
  \newcommand{\elabstract}{Résumé\par}
  \newcommand{\elurl}{http://oeuvres.github.io/elbook/2}
  \author{Éric Lœchien}
  \title{Un titre de test assez long pour vérifier le comportement d’une maquette}
  \date{1566}
  \devtrue
}{}
\let\eltitle\@title
\let\elauthor\@author
\let\eldate\@date


\defaultfontfeatures{
  % Mapping=tex-text, % no effect seen
  Scale=MatchLowercase,
  Ligatures={TeX,Common},
}


% generic typo commands
\newcommand{\astermono}{\medskip\centerline{\color{rubric}\large\selectfont{\syms ✻}}\medskip\par}%
\newcommand{\astertri}{\medskip\par\centerline{\color{rubric}\large\selectfont{\syms ✻\,✻\,✻}}\medskip\par}%
\newcommand{\asterism}{\bigskip\par\noindent\parbox{\linewidth}{\centering\color{rubric}\large{\syms ✻}\\{\syms ✻}\hskip 0.75em{\syms ✻}}\bigskip\par}%

% lists
\newlength{\listmod}
\setlength{\listmod}{\parindent}
\setlist{
  itemindent=!,
  listparindent=\listmod,
  labelsep=0.2\listmod,
  parsep=0pt,
  % topsep=0.2em, % default topsep is best
}
\setlist[itemize]{
  label=—,
  leftmargin=0pt,
  labelindent=1.2em,
  labelwidth=0pt,
}
\setlist[enumerate]{
  label={\bf\color{rubric}\arabic*.},
  labelindent=0.8\listmod,
  leftmargin=\listmod,
  labelwidth=0pt,
}
\newlist{listalpha}{enumerate}{1}
\setlist[listalpha]{
  label={\bf\color{rubric}\alph*.},
  leftmargin=0pt,
  labelindent=0.8\listmod,
  labelwidth=0pt,
}
\newcommand{\listhead}[1]{\hspace{-1\listmod}\emph{#1}}

\renewcommand{\hrulefill}{%
  \leavevmode\leaders\hrule height 0.2pt\hfill\kern\z@}

% General typo
\DeclareTextFontCommand{\textlarge}{\large}
\DeclareTextFontCommand{\textsmall}{\small}

% commands, inlines
\newcommand{\anchor}[1]{\Hy@raisedlink{\hypertarget{#1}{}}} % link to top of an anchor (not baseline)
\newcommand\abbr[1]{#1}
\newcommand{\autour}[1]{\tikz[baseline=(X.base)]\node [draw=rubric,thin,rectangle,inner sep=1.5pt, rounded corners=3pt] (X) {\color{rubric}#1};}
\newcommand\corr[1]{#1}
\newcommand{\ed}[1]{ {\color{silver}\sffamily\footnotesize (#1)} } % <milestone ed="1688"/>
\newcommand\expan[1]{#1}
\newcommand\foreign[1]{\emph{#1}}
\newcommand\gap[1]{#1}
\renewcommand{\LettrineFontHook}{\color{rubric}}
\newcommand{\initial}[2]{\lettrine[lines=2, loversize=0.3, lhang=0.3]{#1}{#2}}
\newcommand{\initialiv}[2]{%
  \let\oldLFH\LettrineFontHook
  % \renewcommand{\LettrineFontHook}{\color{rubric}\ttfamily}
  \IfSubStr{QJ’}{#1}{
    \lettrine[lines=4, lhang=0.2, loversize=-0.1, lraise=0.2]{\smash{#1}}{#2}
  }{\IfSubStr{É}{#1}{
    \lettrine[lines=4, lhang=0.2, loversize=-0, lraise=0]{\smash{#1}}{#2}
  }{\IfSubStr{ÀÂ}{#1}{
    \lettrine[lines=4, lhang=0.2, loversize=-0, lraise=0, slope=0.6em]{\smash{#1}}{#2}
  }{\IfSubStr{A}{#1}{
    \lettrine[lines=4, lhang=0.2, loversize=0.2, slope=0.6em]{\smash{#1}}{#2}
  }{\IfSubStr{V}{#1}{
    \lettrine[lines=4, lhang=0.2, loversize=0.2, slope=-0.5em]{\smash{#1}}{#2}
  }{
    \lettrine[lines=4, lhang=0.2, loversize=0.2]{\smash{#1}}{#2}
  }}}}}
  \let\LettrineFontHook\oldLFH
}
\newcommand{\labelchar}[1]{\textbf{\color{rubric} #1}}
\newcommand{\milestone}[1]{\autour{\footnotesize\color{rubric} #1}} % <milestone n="4"/>
\newcommand\name[1]{#1}
\newcommand\orig[1]{#1}
\newcommand\orgName[1]{#1}
\newcommand\persName[1]{#1}
\newcommand\placeName[1]{#1}
\newcommand{\pn}[1]{\IfSubStr{-—–¶}{#1}% <p n="3"/>
  {\noindent{\bfseries\color{rubric}   ¶  }}
  {{\footnotesize\autour{ #1}  }}}
\newcommand\reg{}
% \newcommand\ref{} % already defined
\newcommand\sic[1]{#1}
\newcommand\surname[1]{\textsc{#1}}
\newcommand\term[1]{\textbf{#1}}

\def\mednobreak{\ifdim\lastskip<\medskipamount
  \removelastskip\nopagebreak\medskip\fi}
\def\bignobreak{\ifdim\lastskip<\bigskipamount
  \removelastskip\nopagebreak\bigskip\fi}

% commands, blocks
\newcommand{\byline}[1]{\bigskip{\RaggedLeft{#1}\par}\bigskip}
\newcommand{\bibl}[1]{{\RaggedLeft{#1}\par\bigskip}}
\newcommand{\biblitem}[1]{{\noindent\hangindent=\parindent   #1\par}}
\newcommand{\dateline}[1]{\medskip{\RaggedLeft{#1}\par}\bigskip}
\newcommand{\labelblock}[1]{\medbreak{\noindent\color{rubric}\bfseries #1}\par\mednobreak}
\newcommand{\salute}[1]{\bigbreak{#1}\par\medbreak}
\newcommand{\signed}[1]{\bigbreak\filbreak{\raggedleft #1\par}\medskip}

% environments for blocks (some may become commands)
\newenvironment{borderbox}{}{} % framing content
\newenvironment{citbibl}{\ifvmode\hfill\fi}{\ifvmode\par\fi }
\newenvironment{docAuthor}{\ifvmode\vskip4pt\fontsize{16pt}{18pt}\selectfont\fi\itshape}{\ifvmode\par\fi }
\newenvironment{docDate}{}{\ifvmode\par\fi }
\newenvironment{docImprint}{\vskip6pt}{\ifvmode\par\fi }
\newenvironment{docTitle}{\vskip6pt\bfseries\fontsize{18pt}{22pt}\selectfont}{\par }
\newenvironment{msHead}{\vskip6pt}{\par}
\newenvironment{msItem}{\vskip6pt}{\par}
\newenvironment{titlePart}{}{\par }


% environments for block containers
\newenvironment{argument}{\itshape\parindent0pt}{\vskip1.5em}
\newenvironment{biblfree}{}{\ifvmode\par\fi }
\newenvironment{bibitemlist}[1]{%
  \list{\@biblabel{\@arabic\c@enumiv}}%
  {%
    \settowidth\labelwidth{\@biblabel{#1}}%
    \leftmargin\labelwidth
    \advance\leftmargin\labelsep
    \@openbib@code
    \usecounter{enumiv}%
    \let\p@enumiv\@empty
    \renewcommand\theenumiv{\@arabic\c@enumiv}%
  }
  \sloppy
  \clubpenalty4000
  \@clubpenalty \clubpenalty
  \widowpenalty4000%
  \sfcode`\.\@m
}%
{\def\@noitemerr
  {\@latex@warning{Empty `bibitemlist' environment}}%
\endlist}
\newenvironment{quoteblock}% may be used for ornaments
  {\begin{quoting}}
  {\end{quoting}}

% table () is preceded and finished by custom command
\newcommand{\tableopen}[1]{%
  \ifnum\strcmp{#1}{wide}=0{%
    \begin{center}
  }
  \else\ifnum\strcmp{#1}{long}=0{%
    \begin{center}
  }
  \else{%
    \begin{center}
  }
  \fi\fi
}
\newcommand{\tableclose}[1]{%
  \ifnum\strcmp{#1}{wide}=0{%
    \end{center}
  }
  \else\ifnum\strcmp{#1}{long}=0{%
    \end{center}
  }
  \else{%
    \end{center}
  }
  \fi\fi
}


% text structure
\newcommand\chapteropen{} % before chapter title
\newcommand\chaptercont{} % after title, argument, epigraph…
\newcommand\chapterclose{} % maybe useful for multicol settings
\setcounter{secnumdepth}{-2} % no counters for hierarchy titles
\setcounter{tocdepth}{5} % deep toc
\markright{\@title} % ???
\markboth{\@title}{\@author} % ???
\renewcommand\tableofcontents{\@starttoc{toc}}
% toclof format
% \renewcommand{\@tocrmarg}{0.1em} % Useless command?
% \renewcommand{\@pnumwidth}{0.5em} % {1.75em}
\renewcommand{\@cftmaketoctitle}{}
\setlength{\cftbeforesecskip}{\z@ \@plus.2\p@}
\renewcommand{\cftchapfont}{}
\renewcommand{\cftchapdotsep}{\cftdotsep}
\renewcommand{\cftchapleader}{\normalfont\cftdotfill{\cftchapdotsep}}
\renewcommand{\cftchappagefont}{\bfseries}
\setlength{\cftbeforechapskip}{0em \@plus\p@}
% \renewcommand{\cftsecfont}{\small\relax}
\renewcommand{\cftsecpagefont}{\normalfont}
% \renewcommand{\cftsubsecfont}{\small\relax}
\renewcommand{\cftsecdotsep}{\cftdotsep}
\renewcommand{\cftsecpagefont}{\normalfont}
\renewcommand{\cftsecleader}{\normalfont\cftdotfill{\cftsecdotsep}}
\setlength{\cftsecindent}{1em}
\setlength{\cftsubsecindent}{2em}
\setlength{\cftsubsubsecindent}{3em}
\setlength{\cftchapnumwidth}{1em}
\setlength{\cftsecnumwidth}{1em}
\setlength{\cftsubsecnumwidth}{1em}
\setlength{\cftsubsubsecnumwidth}{1em}

% footnotes
\newif\ifheading
\newcommand*{\fnmarkscale}{\ifheading 0.70 \else 1 \fi}
\renewcommand\footnoterule{\vspace*{0.3cm}\hrule height \arrayrulewidth width 3cm \vspace*{0.3cm}}
\setlength\footnotesep{1.5\footnotesep} % footnote separator
\renewcommand\@makefntext[1]{\parindent 1.5em \noindent \hb@xt@1.8em{\hss{\normalfont\@thefnmark . }}#1} % no superscipt in foot
\patchcmd{\@footnotetext}{\footnotesize}{\footnotesize\sffamily}{}{} % before scrextend, hyperref


%   see https://tex.stackexchange.com/a/34449/5049
\def\truncdiv#1#2{((#1-(#2-1)/2)/#2)}
\def\moduloop#1#2{(#1-\truncdiv{#1}{#2}*#2)}
\def\modulo#1#2{\number\numexpr\moduloop{#1}{#2}\relax}

% orphans and widows
\clubpenalty=9996
\widowpenalty=9999
\brokenpenalty=4991
\predisplaypenalty=10000
\postdisplaypenalty=1549
\displaywidowpenalty=1602
\hyphenpenalty=400
% Copied from Rahtz but not understood
\def\@pnumwidth{1.55em}
\def\@tocrmarg {2.55em}
\def\@dotsep{4.5}
\emergencystretch 3em
\hbadness=4000
\pretolerance=750
\tolerance=2000
\vbadness=4000
\def\Gin@extensions{.pdf,.png,.jpg,.mps,.tif}
% \renewcommand{\@cite}[1]{#1} % biblio

\usepackage{hyperref} % supposed to be the last one, :o) except for the ones to follow
\urlstyle{same} % after hyperref
\hypersetup{
  % pdftex, % no effect
  pdftitle={\elbibl},
  % pdfauthor={Your name here},
  % pdfsubject={Your subject here},
  % pdfkeywords={keyword1, keyword2},
  bookmarksnumbered=true,
  bookmarksopen=true,
  bookmarksopenlevel=1,
  pdfstartview=Fit,
  breaklinks=true, % avoid long links
  pdfpagemode=UseOutlines,    % pdf toc
  hyperfootnotes=true,
  colorlinks=false,
  pdfborder=0 0 0,
  % pdfpagelayout=TwoPageRight,
  % linktocpage=true, % NO, toc, link only on page no
}

\makeatother % /@@@>
%%%%%%%%%%%%%%
% </TEI> end %
%%%%%%%%%%%%%%


%%%%%%%%%%%%%
% footnotes %
%%%%%%%%%%%%%
\renewcommand{\thefootnote}{\bfseries\textcolor{rubric}{\arabic{footnote}}} % color for footnote marks

%%%%%%%%%
% Fonts %
%%%%%%%%%
\usepackage[]{roboto} % SmallCaps, Regular is a bit bold
% \linespread{0.90} % too compact, keep font natural
\newfontfamily\fontrun[]{Roboto Condensed Light} % condensed runing heads
\ifav
  \setmainfont[
    ItalicFont={Roboto Light Italic},
  ]{Roboto}
\else\ifbooklet
  \setmainfont[
    ItalicFont={Roboto Light Italic},
  ]{Roboto}
\else
\setmainfont[
  ItalicFont={Roboto Italic},
]{Roboto Light}
\fi\fi
\renewcommand{\LettrineFontHook}{\bfseries\color{rubric}}
% \renewenvironment{labelblock}{\begin{center}\bfseries\color{rubric}}{\end{center}}

%%%%%%%%
% MISC %
%%%%%%%%

\setdefaultlanguage[frenchpart=false]{french} % bug on part


\newenvironment{quotebar}{%
    \def\FrameCommand{{\color{rubric!10!}\vrule width 0.5em} \hspace{0.9em}}%
    \def\OuterFrameSep{\itemsep} % séparateur vertical
    \MakeFramed {\advance\hsize-\width \FrameRestore}
  }%
  {%
    \endMakeFramed
  }
\renewenvironment{quoteblock}% may be used for ornaments
  {%
    \savenotes
    \setstretch{0.9}
    \normalfont
    \begin{quotebar}
  }
  {%
    \end{quotebar}
    \spewnotes
  }


\renewcommand{\headrulewidth}{\arrayrulewidth}
\renewcommand{\headrule}{{\color{rubric}\hrule}}

% delicate tuning, image has produce line-height problems in title on 2 lines
\titleformat{name=\chapter} % command
  [display] % shape
  {\vspace{1.5em}\centering} % format
  {} % label
  {0pt} % separator between n
  {}
[{\color{rubric}\huge\textbf{#1}}\bigskip] % after code
% \titlespacing{command}{left spacing}{before spacing}{after spacing}[right]
\titlespacing*{\chapter}{0pt}{-2em}{0pt}[0pt]

\titleformat{name=\section}
  [block]{}{}{}{}
  [\vbox{\color{rubric}\large\raggedleft\textbf{#1}}]
\titlespacing{\section}{0pt}{0pt plus 4pt minus 2pt}{\baselineskip}

\titleformat{name=\subsection}
  [block]
  {}
  {} % \thesection
  {} % separator \arrayrulewidth
  {}
[\vbox{\large\textbf{#1}}]
% \titlespacing{\subsection}{0pt}{0pt plus 4pt minus 2pt}{\baselineskip}

\ifaiv
  \fancypagestyle{main}{%
    \fancyhf{}
    \setlength{\headheight}{1.5em}
    \fancyhead{} % reset head
    \fancyfoot{} % reset foot
    \fancyhead[L]{\truncate{0.45\headwidth}{\fontrun\elbibl}} % book ref
    \fancyhead[R]{\truncate{0.45\headwidth}{ \fontrun\nouppercase\leftmark}} % Chapter title
    \fancyhead[C]{\thepage}
  }
  \fancypagestyle{plain}{% apply to chapter
    \fancyhf{}% clear all header and footer fields
    \setlength{\headheight}{1.5em}
    \fancyhead[L]{\truncate{0.9\headwidth}{\fontrun\elbibl}}
    \fancyhead[R]{\thepage}
  }
\else
  \fancypagestyle{main}{%
    \fancyhf{}
    \setlength{\headheight}{1.5em}
    \fancyhead{} % reset head
    \fancyfoot{} % reset foot
    \fancyhead[RE]{\truncate{0.9\headwidth}{\fontrun\elbibl}} % book ref
    \fancyhead[LO]{\truncate{0.9\headwidth}{\fontrun\nouppercase\leftmark}} % Chapter title, \nouppercase needed
    \fancyhead[RO,LE]{\thepage}
  }
  \fancypagestyle{plain}{% apply to chapter
    \fancyhf{}% clear all header and footer fields
    \setlength{\headheight}{1.5em}
    \fancyhead[L]{\truncate{0.9\headwidth}{\fontrun\elbibl}}
    \fancyhead[R]{\thepage}
  }
\fi

\ifav % a5 only
  \titleclass{\section}{top}
\fi

\newcommand\chapo{{%
  \vspace*{-3em}
  \centering % no vskip ()
  {\Large\addfontfeature{LetterSpace=25}\bfseries{\elauthor}}\par
  \smallskip
  {\large\eldate}\par
  \bigskip
  {\Large\selectfont{\eltitle}}\par
  \bigskip
  {\color{rubric}\hline\par}
  \bigskip
  {\Large TEXTE LIBRE À PARTICPATION LIBRE\par}
  \centerline{\small\color{rubric} {hurlus.fr, tiré le \today}}\par
  \bigskip
}}

\newcommand\cover{{%
  \thispagestyle{empty}
  \centering
  {\LARGE\bfseries{\elauthor}}\par
  \bigskip
  {\Large\eldate}\par
  \bigskip
  \bigskip
  {\LARGE\selectfont{\eltitle}}\par
  \vfill\null
  {\color{rubric}\setlength{\arrayrulewidth}{2pt}\hline\par}
  \vfill\null
  {\Large TEXTE LIBRE À PARTICPATION LIBRE\par}
  \centerline{{\href{https://hurlus.fr}{\dotuline{hurlus.fr}}, tiré le \today}}\par
}}

\begin{document}
\pagestyle{empty}
\ifbooklet{
  \cover\newpage
  \thispagestyle{empty}\hbox{}\newpage
  \cover\newpage\noindent Les voyages de la brochure\par
  \bigskip
  \begin{tabularx}{\textwidth}{l|X|X}
    \textbf{Date} & \textbf{Lieu}& \textbf{Nom/pseudo} \\ \hline
    \rule{0pt}{25cm} &  &   \\
  \end{tabularx}
  \newpage
  \addtocounter{page}{-4}
}\fi

\thispagestyle{empty}
\ifaiv
  \twocolumn[\chapo]
\else
  \chapo
\fi
{\it\elabstract}
\bigskip
\makeatletter\@starttoc{toc}\makeatother % toc without new page
\bigskip

\pagestyle{main} % after style

  \section[{Première partie. Les besoins de l'âme}]{Première partie. \\
Les besoins de l'âme}\renewcommand{\leftmark}{Première partie. \\
Les besoins de l'âme}

\noindent La notion d'obligation prime celle de droit, qui lui est subordonnée et relative. Un droit n'est pas efficace par lui-même, mais seulement par l'obligation à laquelle il correspond ; l'accomplissement effectif d'un droit provient non pas de celui qui le possède, mais des autres hommes qui se reconnaissent obligés à quelque chose envers lui. L'obligation est efficace dès qu'elle est reconnue. Une obligation ne serait-elle reconnue par personne, elle ne perd rien de la plénitude de son être. Un droit qui n'est reconnu par personne n'est pas grand-chose.\par
Cela n'a pas de sens de dire que les hommes ont, d'une part des droits, d'autre part des devoirs. Ces mots n'expriment que des différences de point de vue. Leur relation est celle de l'objet et du sujet. Un homme, considéré en lui-même, a seulement des devoirs, parmi lesquels se trouvent certains devoirs envers lui-même. Les autres, considérés de son point de vue, ont seulement des droits. Il a des droits à son tour quand il est considéré du point de vue des autres, qui se reconnaissent des obligations envers lui. Un homme qui serait seul dans l'univers n'aurait aucun droit, mais il aurait des obligations.\par
La notion de droit, étant d'ordre objectif, n'est pas séparable de celles d'existence et de réalité. Elle apparaît quand l'obligation descend dans le domaine des faits ; par suite elle enferme toujours dans une certaine mesure la considération des états de fait et des situations particulières. Les droits apparaissent toujours comme liés à certaines conditions. L'obligation seule peut être inconditionnée. Elle se place dans un domaine qui est au-dessus de toutes conditions, parce qu'il est au-dessus de ce monde.\par
Les hommes de 1789 ne reconnaissaient pas la réalité d'un tel domaine. Ils ne reconnaissaient que celle des choses humaines. C'est pourquoi ils ont commencé par la notion de droit. Mais en même temps ils ont voulu poser des principes absolus. Cette contradiction les a fait tomber dans une confusion de langage et d'idées qui est pour beaucoup dans la confusion politique et sociale actuelle. Le domaine de ce qui est éternel, universel, inconditionné, est autre que celui des conditions de fait, et il y habite des notions différentes qui sont liées à la partie la plus secrète de l'âme humaine.\par
L'obligation ne lie que les êtres humains. Il n'y a pas d'obligations pour les collectivités comme telles. Mais il y en a pour tous les êtres humains qui composent, servent, commandent ou représentent une collectivité, dans la partie de leur vie liée à la collectivité comme dans celle qui en est indépendante.\par
Des obligations identiques lient tous les êtres humains, bien qu'elles correspondent à des actes différents selon les situations. Aucun être humain, quel qu'il soit, en aucune circonstance, ne peut s'y soustraire sans crime ; excepté dans les cas où, deux obligations réelles étant en fait incompatibles, un homme est contraint d'abandonner l'une d'elles.\par
L'imperfection d'un ordre social se mesure à la quantité de situations de ce genre qu'il enferme.\par
Mais même en ce cas il y a crime si l'obligation abandonnée n'est pas seulement abandonnée en fait, mais est de plus niée.\par
L'objet de l'obligation, dans le domaine des choses humaines, est toujours l'être humain comme tel. Il y obligation envers tout être humain, du seul fait qu'il est un être humain, sans qu'aucune autre condition ait à intervenir, et quand même lui n'en reconnaîtrait aucune.\par
Cette obligation ne repose sur aucune situation de fait, ni sur les jurisprudences, ni sur les coutumes, ni sur la structure sociale, ni sur les rapports de force, ni sur l'héritage du passé, ni sur l'orientation supposée de l'histoire. Car aucune situation de fait ne peut susciter une obligation.\par
Cette obligation ne repose sur aucune convention. Car toutes les conventions sont modifiables selon la volonté des contractants, au lieu qu'en elle aucun changement dans la volonté des hommes ne peut modifier quoi que ce soit.\par
Cette obligation est éternelle. Elle répond à la destinée éternelle de l'être humain. Seul l'être humain a une destinée éternelle. Les collectivités humaines n'en ont pas. Aussi n'y a-t-il pas à leur égard d'obligations directes qui soient éternelles. Seul est éternel le devoir envers l'être humain comme tel.\par
Cette obligation est inconditionnée. Si elle est fondée sur quelque chose, ce quelque chose n'appartient pas à notre monde. Dans notre monde, elle n'est fondée sur rien. C'est l'unique obligation relative aux choses humaines qui ne soit soumise à aucune condition.\par
Cette obligation a non pas un fondement, mais une vérification dans l'accord de la conscience universelle. Elle est exprimée par certains des plus anciens textes écrits qui nous aient été conservés. Elle est reconnue par tous dans tous les cas particuliers où elle n'est pas combattue par les intérêts ou les passions. C'est relativement à elle qu'on mesure le progrès.\par
La reconnaissance de cette obligation est exprimée d'une manière confuse et imparfaite, mais plus ou moins imparfaite selon les cas, par ce qu'on nomme les droits positifs. Dans la mesure où les droits positifs sont en contradiction avec elle, dans cette mesure exacte ils sont frappés d'illégitimité.\par
Quoique cette obligation éternelle réponde à la destinée éternelle de l'être humain, elle n'a pas cette destinée pour objet direct. La destinée éternelle d'un être humain ne peut être l'objet d'aucune obligation, parce qu'elle n'est pas subordonnée à des actions extérieures.\par
Le fait qu'un être humain possède une destinée éternelle n'impose qu'une seule obligation ; c'est le respect. L'obligation n'est accomplie que si le respect est effectivement exprimé, d'une manière réelle et non fictive ; il ne peut l'être que par l'intermédiaire des besoins terrestres de l'homme.\par
La conscience humaine n'a jamais varié sur ce point. Il y a des milliers d'années, les Égyptiens pensaient qu'une âme ne peut pas être justifiée après la mort si elle ne peut pas dire : « Je n'ai laissé personne souffrir de la faim. » Tous les chrétiens se savent exposés à entendre un jour le Christ lui-même leur dire : « J'ai eu faim et tu ne m’as pas donné à manger. » Tout le monde se représente le progrès comme étant d'abord le passage à un état de la société humaine où les gens ne souffriront pas de la faim. Si on pose la question en termes généraux à n'importe qui, personne ne pense qu'un homme soit innocent si, ayant de la nourriture en abondance et trouvant sur le pas de sa porte quelqu'un aux trois quarts mort de faim, il passe sans rien lui donner.\par
C'est donc une obligation éternelle envers l'être humain que de ne pas le laisser souffrir de la faim quand on a l'occasion de le secourir. Cette obligation étant la plus évidente, elle doit servir de modèle pour dresser la liste des devoirs éternels envers tout être humain. Pour être établie en toute rigueur, cette liste doit procéder de ce premier exemple par voie d'analogie.\par
Par conséquent, la liste des obligations envers l'être humain doit correspondre à la liste de ceux des besoins humains qui sont vitaux, analogues à la faim.\par
Parmi ces besoins, certains sont physiques, comme la faim elle-même. Ils sont assez faciles à énumérer. Ils concernent la protection contre la violence, le logement, les vêtements, la chaleur, l'hygiène, les soins en cas de maladie.\par
D'autres, parmi ces besoins, n'ont pas rapport avec la vie physique, mais avec la vie morale. Comme les premiers cependant ils sont terrestres, et n'ont pas de relation directe qui soit accessible à notre intelligence avec la destinée éternelle de l'homme. Ce sont, comme les besoins physiques, des nécessités de la vie d'ici-bas. C'est-à-dire que s'ils ne sont pas satisfaits, l'homme tombe peu à peu dans un état plus ou moins analogue à la mort, plus ou moins proche d'une vie purement végétative,\par
Ils sont beaucoup plus difficiles à reconnaître et à énumérer que les besoins du corps. Mais tout le monde reconnaît qu'ils existent. Toutes les cruautés qu'un conquérant peut exercer sur des populations soumises, massacres, mutilations, famine organisée, mise en esclavage ou déportations massives, sont généralement considérées comme des mesures de même espèce, quoique la liberté ou le pays natal ne soient pas des nécessités physiques. Tout le monde a conscience qu'il y a des cruautés qui portent atteinte à la vie de l'homme sans porter atteinte à son corps. Ce sont celles qui privent l'homme d'une certaine nourriture nécessaire à la vie de l'âme.\par
Les obligations, inconditionnées ou relatives, éternelles ou changeantes, directes ou indirectes à l'égard des choses humaines dérivent toutes, sans exception, des besoins vitaux de l'être humain. Celles qui ne concernent pas directement tel, tel et tel être humain déterminé ont toutes pour objet des choses qui ont par rapport aux hommes un rôle analogue à la nourriture.\par
On doit le respect à un champ de blé, non pas pour lui-même, mais parce que c'est de la nourriture pour les hommes.\par
D'une manière analogue, on doit du respect à une collectivité, quelle qu'elle soit – patrie, famille, ou toute autre –, non pas pour elle-même, mais comme nourriture d'un certain nombre d'âmes humaines.\par
Cette obligation impose en fait des attitudes, des actes différents selon les différentes situations. Mais considérée en elle-même, elle est absolument identique pour tous.\par
Notamment, elle est absolument identique pour ceux qui sont à l'extérieur.\par
Le degré de respect qui est dû aux collectivités humaines est très élevé, par plusieurs considérations.\par
D'abord, chacune est unique, et, si elle est détruite, n'est pas remplacée. Un sac de blé peut toujours être substitué à un autre sac de blé. La nourriture qu'une collectivité fournit à l'âme de ceux qui en sont membres n'a pas d'équivalent dans l'univers entier.\par
Puis, de par sa durée, la collectivité pénètre déjà dans l'avenir. Elle contient de la nourriture, non seulement pour les âmes des vivants, mais aussi pour celles d'êtres non encore nés qui viendront au monde au cours des siècles prochains.\par
Enfin, de par la même durée, la collectivité a ses racines dans le passé. Elle constitue l'unique organe de conservation pour les trésors spirituels amassés par les morts, l'unique organe de transmission par l'intermédiaire duquel les morts puissent parler aux vivants. Et l'unique chose terrestre qui ait un lien direct avec la destinée éternelle de l'homme, c'est le rayonnement de ceux qui ont su prendre une conscience complète de cette destinée, transmis de génération en génération.\par
À cause de tout cela, il peut arriver que l'obligation à l'égard d'une collectivité en péril aille jusqu'au sacrifice total. Mais, il ne s'ensuit pas que la collectivité soit au-dessus de l'être humain. Il arrive aussi que l'obligation de secourir un être humain en détresse doive aller jusqu'au sacrifice total, sans que cela implique aucune supériorité du côté de celui qui est secouru.\par
Un paysan, dans certaines circonstances, peut devoir s'exposer, pour cultiver son champ, à l'épuisement, à la maladie ou même à la mort. Mais il a toujours présent à l'esprit qu'il s'agit uniquement de pain.\par
D'une manière analogue, même au moment du sacrifice total, il n'est jamais dû à aucune collectivité autre chose qu'un respect analogue à celui qui est dû à la nourriture.\par
Il arrive très souvent que le rôle soit renversé. Certaines collectivités, au lieu de servir de nourriture, tout au contraire mangent les âmes. Il y a en ce cas maladie sociale, et la première obligation est de tenter un traitement ; dans certaines circonstances il peut être nécessaire de s'inspirer des méthodes chirurgicales.\par
Sur ce point aussi, l'obligation est identique pour ceux qui sont à l'intérieur de la collectivité et pour ceux qui sont au-dehors.\par
Il arrive aussi qu'une collectivité fournisse aux âmes de ceux qui en sont membres une nourriture insuffisante. En ce cas il faut l'améliorer.\par
Enfin il y a des collectivités mortes qui, sans dévorer les âmes, ne les nourrissent pas non plus. S'il est tout à fait certain qu'elles sont bien mortes, qu'il ne s'agit pas d'une léthargie passagère, et seulement en ce cas, il faut les anéantir.\par
La première étude à faire est celle des besoins qui sont à la vie de l'âme ce que sont pour la vie du corps les besoins de nourriture, de sommeil et de chaleur. Il faut tenter de les énumérer et de les définir.\par
Il ne faut jamais les confondre avec les désirs, les caprices, les fantaisies, les vices. Il faut aussi discerner l'essentiel et l'accidentel. L'homme a besoin, non de riz ou de pommes de terre, mais de nourriture ; non de bois ou de charbon, mais de chauffage. De même pour les besoins de l'âme, il faut reconnaître les satisfactions différentes, mais équivalentes, répondant aux mêmes besoins. Il faut aussi distinguer des nourritures de l'âme les poisons qui, quelque temps, peuvent donner l'illusion d'en tenir lieu.\par
L'absence d'une telle étude force les gouvernements, quand ils ont de bonnes intentions, à s'agiter au hasard.\par
Voici quelques indications.\par
\textbf{}\par
\subsection[{A. L'ordre}]{A. \\
L'ordre}
\noindent \par
Le premier besoin de l'âme, celui qui est le plus proche de sa destinée éternelle, c'est l'ordre, c'est-à-dire un tissu de relations sociales tel que nul ne soit contraint de violer des obligations rigoureuses pour exécuter d'autres obligations. L'âme ne souffre une violence spirituelle de la part des circonstances extérieures que dans ce cas. Car celui qui est seulement arrêté dans l'exécution d'une obligation par la menace de la mort ou de la souffrance peut passer outre, et ne sera blessé que dans son corps. Mais celui pour qui les circonstances rendent en fait incompatibles les actes ordonnés par plusieurs obligations strictes, celui-là, sans qu'il puisse s'en défendre, est blessé dans son amour du bien.\par
Aujourd'hui, il y a un degré très élevé de désordre et d'incompatibilité entre les obligations.\par
Quiconque agit de manière à augmenter cette incompatibilité est un fauteur de désordre. Quiconque agit de manière à la diminuer est un facteur d'ordre. Quiconque, pour simplifier les problèmes, nie certaines obligations, a conclu en son cœur une alliance avec le crime.\par
On n'a malheureusement pas de méthode pour diminuer cette incompatibilité. On n'a même pas la certitude que l'idée d'un ordre où toutes les obligations seraient compatibles ne soit pas une fiction. Quand le devoir descend au niveau des faits, un si grand nombre de relations indépendantes entrent en jeu que l'incompatibilité semble bien plus probable que la compatibilité.\par
Mais nous avons tous les jours sous les yeux l'exemple de l'univers, où une infinité d'actions mécaniques indépendantes concourent pour constituer un ordre qui, à travers les variations, reste fixe. Aussi aimons-nous la beauté du monde, parce que nous sentons derrière elle la présence de quelque chose d'analogue à la sagesse que nous voudrions posséder pour assouvir notre désir du bien.\par
À un degré moindre, les œuvres d'art vraiment belles offrent l'exemple d'ensembles où des facteurs indépendants concourent, d'une manière impossible à comprendre, pour constituer une beauté unique.\par
Enfin le sentiment des diverses obligations procède toujours d'un désir du bien qui est unique, fixe, identique à lui-même, pour tout homme, du berceau à la tombe. Ce désir perpétuellement agissant au fond de nous empêche que nous puissions jamais nous résigner aux situations où les obligations sont incompatibles. Ou nous avons recours au mensonge pour oublier qu'elles existent, ou nous nous débattons aveuglément pour en sortir.\par
La contemplation des œuvres d'art authentiques, et bien davantage encore celle de la beauté du monde, et bien davantage encore celle du bien inconnu auquel nous aspirons peut nous soutenir dans l'effort de penser continuellement à l'ordre humain qui doit être notre premier objet.\par
Les grands fauteurs de violence se sont encouragés eux-mêmes en considérant comment la force mécanique, aveugle, est souveraine dans tout l'univers.\par
En regardant le monde mieux qu'ils ne font, nous trouverons un encouragement plus grand, si nous considérons comment les forces aveugles innombrables sont limitées, combinées en un équilibre, amenées à concourir à une unité, par quelque chose que nous ne comprenons pas, mais que nous aimons et que nous nommons la beauté.\par
Si nous gardons sans cesse présente à l'esprit la pensée d'un ordre humain véritable, si nous y pensons comme à un objet auquel on doit le sacrifice total quand l'occasion s'en présente, nous serons dans la situation d'un homme qui marche dans la nuit, sans guide, mais en pensant sans cesse à la direction qu'il veut suivre. Pour un tel voyageur, il y a une grande espérance.\par
Cet ordre est le premier des besoins, il est même au-dessus des besoins proprement dits. Pour pouvoir le penser, il faut une connaissance des autres besoins.\par
Le premier caractère qui distingue les besoins des désirs, des fantaisies ou des vices, et les nourritures des gourmandises ou des poisons, c'est que les besoins sont limités, ainsi que les nourritures qui leur correspondent. Un avare n'a jamais assez d'or, mais pour tout homme, si on lui donne du pain à discrétion, il viendra un moment où il en aura assez. La nourriture apporte le rassasiement. Il en est de même des nourritures de l'âme.\par
Le second caractère, lié au premier, c'est que les besoins s'ordonnent par couples de contraires, et doivent se combiner en un équilibre. L'homme a besoin de nourriture, mais aussi d'un intervalle entre les repas ; il a besoin de chaleur et de fraîcheur, de repos et d'exercice. De même pour les besoins de l'âme.\par
Ce qu'on appelle le juste milieu consiste en réalité à ne satisfaire ni l'un ni l'autre des besoins contraires. C'est une caricature du véritable équilibre par lequel les besoins contraires sont satisfaits l'un et l'autre dans leur plénitude.
\subsection[{B. La liberté}]{B. \\
La liberté}
\noindent \par
Une nourriture indispensable à l'âme humaine est la liberté. La liberté, au sens concret du mot, consiste dans une possibilité de choix. Il s'agit, bien entendu, d'une possibilité réelle. Partout où il y a vie commune, il est inévitable que des règles, imposées par l'utilité commune, limitent le choix.\par
Mais la liberté n'est pas plus ou moins grande selon que les limites sont plus étroites ou plus larges. Elle a sa plénitude à des conditions moins facilement mesurables.\par
Il faut que les règles soient assez raisonnables et assez simples pour que quiconque le désire et dispose d'une faculté moyenne d'attention puisse comprendre, d'une part l'utilité à laquelle elles correspondent, d'autre part les nécessités de fait qui les ont imposées. Il faut qu'elles émanent d'une autorité qui ne soit pas regardée comme étrangère ou ennemie, qui soit aimée comme appartenant à ceux qu'elle dirige. Il faut qu'elles soient assez stables, assez peu nombreuses, assez générales, pour que la pensée puisse se les assimiler une fois pour toutes, et non pas se heurter contre elles toutes les fois qu'il y a une décision à prendre.\par
À ces conditions, la liberté des hommes de bonne volonté, quoique limitée dans les faits, est totale dans la conscience. Car les règles s'étant incorporées à leur être même, les possibilités interdites ne se présentent pas à leur pensée et n'ont pas à être repoussées. De même l'habitude, imprimée par l'éducation, de ne pas manger les choses repoussantes ou dangereuses n'est pas ressentie par un homme normal comme une limite à la liberté dans le domaine de l'alimentation. Seul l'enfant sent la limite.\par
Ceux qui manquent de bonne volonté ou restent puérils ne sont jamais libres dans aucun état de la société.\par
Quand les possibilités de choix sont larges au point de nuire à l'utilité commune, les hommes n'ont pas la jouissance de la liberté. Car il leur faut, soit avoir recours au refuge de l'irresponsabilité, de la puérilité, de l'indifférence, refuge où ils ne peuvent trouver que l'ennui, soit se sentir accablés de responsabilité en toute circonstance par la crainte de nuire à autrui. En pareil cas les hommes, croyant à tort qu'ils possèdent la liberté et sentant qu'ils n'en jouissent pas, en arrivent à penser que la liberté n'est pas un bien.
\subsection[{C. L’obéissance}]{C. \\
L’obéissance}
\noindent \par
L'obéissance est un besoin vital de l'âme humaine. Elle est de deux espèces : obéissance à des règles établies et obéissance à des êtres humains regardés comme des chefs. Elle suppose le consentement, non pas à l'égard de chacun des ordres reçus, mais un consentement accordé une fois pour toutes, sous la seule réserve, le cas échéant, des exigences de la conscience. Il est nécessaire qu'il soit généralement reconnu, et avant tout par les chefs, que le consentement et non pas la crainte du châtiment ou l'appât de la récompense constitue en fait le ressort principal de l'obéissance, de manière que la soumission ne soit jamais suspecte de servilité. Il faut qu'il soit connu aussi que ceux qui commandent obéissent de leur côté ; et il faut que toute la hiérarchie soit orientée vers un but dont la valeur et même la grandeur soit sentie par tous, du plus haut au plus bas.\par
L'obéissance étant une nourriture nécessaire à l'âme, quiconque en est définitivement privé est malade. Ainsi toute collectivité régie par un chef souverain qui n'est comptable à personne se trouve entre les mains d'un malade.\par
C'est pourquoi, là où un homme est placé pour la vie à la tête de l'organisation sociale, il faut qu'il soit un symbole et non un chef, comme c'est le cas pour le roi d'Angleterre ; il faut aussi que les convenances limitent sa liberté plus étroitement que celle d'aucun homme du peuple. De cette manière, les chefs effectifs, quoique chefs, ont quelqu'un au-dessus d'eux ; d'autre part ils peuvent, sans que la continuité soit rompue, se remplacer, et par suite recevoir chacun sa part indispensable d'obéissance.\par
Ceux qui soumettent des masses humaines par la contrainte et la cruauté les privent à la fois de deux nourritures vitales, liberté et obéissance ; car il n'est plus au pouvoir de ces masses d'accorder leur consentement intérieur à l'autorité qu'elles subissent. Ceux qui favorisent un état de choses où l'appât du gain soit le principal mobile enlèvent aux hommes l'obéissance, car le consentement qui en est le principe n'est pas une chose qui puisse se vendre.\par
Mille signes montrent que les hommes de notre époque étaient depuis longtemps affamés d'obéissance. Mais on en a profité pour leur donner l'esclavage.
\subsection[{D. La responsabilité}]{D. \\
La responsabilité}
\noindent \par
L'initiative et la responsabilité, le sentiment d'être utile et même indispensable, sont des besoins vitaux de l'âme humaine.\par
La privation complète à cet égard est le cas du chômeur, même s'il est secouru de manière à pouvoir manger, s'habiller et se loger. Il n'est rien dans la vie économique, et le bulletin de vote qui constitue sa part dans la vie politique n'a pas de sens pour lui.\par
Le manœuvre est dans une situation à peine meilleure.\par
La satisfaction de ce besoin exige qu'un homme ait à prendre souvent des décisions dans des problèmes, grands ou petits, affectant des intérêts étrangers aux siens propres, mais envers lesquels il se sent engagé. Il faut aussi qu'il ait à fournir continuellement des efforts. Il faut enfin qu'il puisse s'approprier par la pensée l'œuvre tout entière de la collectivité dont il est membre, y compris les domaines où il n'a jamais ni décision à prendre ni avis à donner. Pour cela, il faut qu'on la lui fasse connaître, qu'on lui demande d'y porter intérêt, qu'on lui en rende sensible la valeur, l'utilité, et s'il y a lieu la grandeur, et qu'on lui fasse clairement saisir la part qu'il y prend.\par
Toute collectivité, de quelque espèce qu'elle soit, qui ne fournit pas ces satisfactions à ses membres, est tarée et doit être transformée.\par
Chez toute personnalité un peu forte, le besoin d'initiative va jusqu'au besoin de commandement. Une vie locale et régionale intense, une multitude d'œuvres éducatives et de mouvements de jeunesse, doivent donner à quiconque n'en est pas incapable, l'occasion de commander pendant certaines périodes de sa vie.\par

\subsection[{E. L'égalité}]{E. \\
L'égalité}
\noindent \par
L'égalité est un besoin vital de l'âme humaine. Elle consiste dans la reconnaissance publique, générale, effective, exprimée réellement par les institutions et les mœurs, que la même quantité de respect et d'égards est due à tout être humain, parce que le respect est dû à l'être humain comme tel et n'a pas de degrés.\par
Par suite, les différences inévitables parmi les hommes ne doivent jamais porter la signification d'une différence dans le degré de respect. Pour qu'elles ne soient pas ressenties comme ayant cette signification, il faut un certain équilibre entre l'égalité et l'inégalité.\par
Une certaine combinaison de l'égalité et de l'inégalité est constituée par l'égalité des possibilités. Si n'importe qui peut arriver au rang social correspondant à la fonction qu'il est capable de remplir, et si l'éducation est assez répandue pour que nul ne soit privé d'aucune capacité du seul fait de sa naissance, l'espérance est la même pour tous les enfants. Ainsi chaque homme est égal en espérance à chaque autre, pour son propre compte quand il est jeune, pour le compte de ses enfants plus tard.\par
Mais cette combinaison, quand elle joue seule et non pas comme un facteur parmi d'autres, ne constitue pas un équilibre et enferme de grands dangers.\par
D'abord, pour un homme qui est dans une situation inférieure et qui en souffre, savoir que sa situation est causée par son incapacité, et savoir que tout le monde le sait, n'est pas une consolation, mais un redoublement d'amertume ; selon les caractères, certains peuvent en être accablés, certains autres menés au crime.\par
Puis il se crée ainsi inévitablement dans la vie sociale comme une pompe aspirante vers le haut. Il en résulte une maladie sociale si un mouvement descendant ne vient pas faire équilibre au mouvement ascendant. Dans la mesure où il est réellement possible qu'un enfant, fils de valet de ferme, soit un jour ministre, dans cette mesure il doit être réellement possible qu'un enfant, fils de ministre, soit un jour valet de ferme. Le degré de cette seconde possibilité ne peut être considérable sans un degré très dangereux de contrainte sociale.\par
Cette espèce d'égalité, si elle joue seule et sans limites, donne à la vie sociale un degré de fluidité qui la décompose.\par
Il y a des méthodes moins grossières pour combiner l'égalité et la différence. La première est la proportion. La proportion se définit comme la combinaison de l'égalité et de l'inégalité, et partout dans l'univers elle est l'unique facteur de l'équilibre.\par
Appliquée à l'équilibre social, elle imposerait à chaque homme des charges correspondantes à la puissance, au bien-être qu'il possède, et des risques correspondants en cas d'incapacité ou de faute. Par exemple, il faudrait qu'un patron incapable ou coupable d'une faute envers ses ouvriers ait beaucoup plus à souffrir, dans son âme et dans sa chair, qu'un manœuvre incapable, ou coupable d'une faute envers son patron. De plus, il faudrait que tous les manœuvres sachent qu'il en est ainsi. Cela implique, d'une part, une certaine organisation des risques, d'autre part, en droit pénal, une conception du châtiment où le rang social, comme circonstance aggravante, joue toujours dans une large mesure pour la détermination de la peine. À plus forte raison l'exercice des hautes fonctions publiques doit comporter de graves risques personnels.\par
Une autre manière de rendre l'égalité compatible avec la différence est d'ôter autant qu'on peut aux différences tout caractère quantitatif. Là où il y a seulement différence de nature, non de degré, il n'y a aucune inégalité.\par
En faisant de l'argent le mobile unique ou presque de tous les actes, la mesure unique ou presque de toutes choses, on a mis le poison de l'inégalité partout. Il est vrai que cette inégalité est mobile ; elle n'est pas attachée aux personnes, car l'argent se gagne et se perd ; elle n'en est pas moins réelle.\par
Il y a deux espèces d'inégalités, auxquelles correspondent deux stimulants différents. L'inégalité à peu près stable, comme celle de l'ancienne France, suscite l'idolâtrie des supérieurs – non sans un mélange de haine refoulée – et la soumission à leurs ordres. L'inégalité mobile, fluide, suscite le désir de s'élever. Elle n'est pas plus proche de l'égalité que l'inégalité stable, et elle est tout aussi malsaine. La Révolution de 1789, en mettant en avant l'égalité, n'a fait en réalité que consacrer la substitution d'une forme d'inégalité à l'autre.\par
Plus il y a égalité dans une société, moindre est l'action des deux stimulants liés aux deux formes d'inégalité, et par suite il en faut d'autres.\par
L'égalité est d'autant plus grande que les différentes conditions humaines sont regardées comme étant, non pas plus ou moins l'une que l'autre, mais simplement autres. Que la profession de mineur et celle de ministre soient simplement deux vocations différentes, comme celles de poète et de mathématicien. Que les duretés matérielles attachées à la condition de mineur soient comptées à l'honneur de ceux qui les souffrent.\par
En temps de guerre, si une armée a l'esprit qui convient, un soldat est heureux et fier d'être sous le feu et non au quartier général ; un général est heureux et fier que le sort de la bataille repose sur sa pensée ; et en même temps le soldat admire le général et le général admire le soldat. Un tel équilibre constitue une égalité. Il y aurait égalité dans les conditions sociales s'il s'y trouvait cet équilibre.\par
Cela implique pour chaque condition des marques de considération qui lui soient propres, et qui ne soient pas des mensonges.
\subsection[{F. La hiérarchie}]{F. \\
La hiérarchie}
\noindent \par
La hiérarchie est un besoin vital de l'âme humaine. Elle est constituée par une certaine vénération, un certain dévouement à l'égard des supérieurs, considérés non pas dans leurs personnes ni dans le pouvoir qu'ils exercent, mais comme des symboles. Ce dont ils sont les symboles, c'est ce domaine qui se trouve au-dessus de tout homme et dont l'expression en ce monde est constituée par les obligations de chaque homme envers ses semblables. Une véritable hiérarchie suppose que les supérieurs aient conscience de cette fonction de symbole et sachent qu'elle est l'unique objet légitime du dévouement de leurs subordonnés. La vraie hiérarchie a pour effet d'amener chacun à s'installer moralement dans la place qu'il occupe.
\subsection[{G. L’honneur}]{G. \\
L’honneur}
\noindent \par
L'honneur est un besoin vital de l'âme humaine. Le respect dû à chaque être humain comme tel, même s'il est effectivement accordé, ne suffit pas à satisfaire ce besoin ; car il est identique pour tous et immuable ; au lieu que l'honneur a rapport à un être humain considéré, non pas simplement comme tel, mais dans son entourage social. Ce besoin est pleinement satisfait, si chacune des collectivités dont un être humain est membre lui offre une part à une tradition de grandeur enfermée dans son passé et publiquement reconnue au-dehors.\par
Par exemple, pour que le besoin d'honneur soit satisfait dans la vie professionnelle, il faut qu'à chaque profession corresponde quelque collectivité réellement capable de conserver vivant le souvenir des trésors de grandeur, d'héroïsme, de probité, de générosité, de génie, dépensés dans l'exercice de la profession.\par
Toute oppression crée une famine à l'égard du besoin d'honneur, car les traditions de grandeur possédées par les opprimés ne sont pas reconnues, faute de prestige social.\par
C'est toujours là l'effet de la conquête. Vercingétorix n'était pas un héros pour les Romains. Si les Anglais avaient conquis la France au XV\textsuperscript{e} siècle, Jeanne d'Arc serait bien oubliée, même dans une large mesure par nous. Actuellement, nous parlons d'elle aux Annamites, aux Arabes ; mais ils savent que chez nous on n'entend pas parler de leurs héros et de leurs saints ; ainsi l'état où nous les maintenons est une atteinte à l'honneur.\par
L'oppression sociale a les mêmes effets. Guynemer, Mermoz sont passés dans la conscience publique à la faveur du prestige social de l'aviation ; l'héroïsme parfois incroyable dépensé par des mineurs ou des pêcheurs a à peine une résonance dans les milieux de mineurs ou de pêcheurs.\par
Le degré extrême de la privation d'honneur est la privation totale de considération infligée à des catégories d'êtres humains. Tels sont en France, avec des modalités diverses, les prostituées, les repris de justice, les policiers, le sous-prolétariat d'immigrés et d'indigènes coloniaux... De. telles catégories ne doivent pas exister.\par
Le crime seul doit placer l'être qui l'a commis hors de la considération sociale, et le châtiment doit l'y réintégrer.
\subsection[{H. Le châtiment}]{H. \\
Le châtiment}
\noindent \par
Le châtiment est un besoin vital de l'âme humaine. Il est de deux espèces, disciplinaire et pénal. Ceux de la première espèce offrent une sécurité contre les défaillances, à l'égard desquelles la lutte serait trop épuisante s'il n'y avait un appui extérieur. Mais le châtiment le plus indispensable à l'âme est celui du crime. Par le crime un homme se met lui-même hors du réseau d'obligations éternelles qui lie chaque être humain à tous les autres. Il ne peut y être réintégré que par le châtiment, pleinement s'il y a consentement de sa part, sinon imparfaitement. De même que la seule manière de témoigner du respect à celui qui souffre de la faim est de lui donner à manger, de même le seul moyen de témoigner du respect à celui qui s'est mis hors la loi est de le réintégrer dans la loi en le soumettant au châtiment qu'elle prescrit.\par
Le besoin de châtiment n'est pas satisfait là où, comme c'est généralement le cas, le code pénal est seulement un procédé de contrainte par la terreur.\par
La satisfaction de ce besoin exige d'abord que tout ce qui touche au droit pénal ait un caractère solennel et sacré ; que la majesté de la loi se communique au tribunal, à la police, à l'accusé, au condamné, et cela même dans les affaires peu importantes, si seulement elles peuvent entraîner la privation de la liberté. Il faut que le châtiment soit un honneur, que non seulement il efface la honte du crime, mais qu'il soit regardé comme une éducation supplémentaire qui oblige à un plus grand degré de dévouement au bien public. Il faut aussi que la dureté des peines réponde au caractère des obligations violées et non aux intérêts de la sécurité sociale.\par
La déconsidération de la police, la légèreté des magistrats, le régime des prisons, le déclassement définitif des repris de justice, l'échelle des peines qui prévoit une punition bien plus cruelle pour dix menus vols que pour un viol ou pour certains meurtres, et qui même prévoit des punitions pour le simple malheur, tout cela empêche qu'il existe parmi nous quoi que ce soit qui mérite le nom de châtiment.\par
Pour les fautes comme pour les crimes, le degré d'impunité doit augmenter non pas quand on monte, mais quand on descend l'échelle sociale. Autrement les souffrances infligées sont ressenties comme des contraintes ou même des abus de pouvoir, et ne constituent pas des châtiments. Il n'y a châtiment que si la souffrance s'accompagne à quelque moment, fût-ce après coup, dans le souvenir, d'un sentiment de justice. Comme le musicien éveille le sentiment du beau par les sons, de même le système pénal doit savoir éveiller le sentiment de la justice chez le criminel par la douleur, ou même, le cas échéant, par la mort. Comme on dit de l'apprenti qui s'est blessé que le métier lui entre dans le corps, de même le châtiment est une méthode pour faire entrer la justice dans l'âme du criminel par la souffrance de la chair.\par
La question du meilleur procédé pour empêcher qu'il s'établisse en haut une conspiration en vue d'obtenir l'impunité est un des problèmes politiques les plus difficiles à résoudre. Il ne peut être résolu que si un ou plusieurs hommes ont la charge d'empêcher une telle conspiration, et se trouvent dans une situation telle qu'ils ne soient pas tentés d'y entrer eux-mêmes.
\subsection[{I. La liberté d’opinion}]{I. \\
La liberté d’opinion}
\noindent \par
La liberté d'opinion et la liberté d'association sont généralement mentionnées ensemble. C'est une erreur. Sauf le cas des groupements naturels, l'association n'est pas un besoin, mais un expédient de la vie pratique.\par
Au contraire, la liberté d'expression totale, illimitée, pour toute opinion quelle qu'elle soit, sans aucune restriction ni réserve, est un besoin absolu pour l'intelligence. Par suite c'est un besoin de l'âme, car quand l'intelligence est mal à l'aise, l'âme entière est malade. La nature et les limites de la satisfaction correspondant à ce besoin sont inscrites dans la structure même des différentes facultés de l'âme. Car une même chose peut être limitée et illimitée, comme on peut prolonger indéfiniment la longueur d'un rectangle sans qu'il cesse d'être limité dans sa largeur.\par
Chez un être humain, l'intelligence peut s'exercer de trois manières. Elle peut travailler sur des problèmes techniques, c'est-à-dire chercher des moyens pour un but déjà posé. Elle peut fournir de la lumière lorsque s'accomplit la délibération de la volonté dans le choix d'une orientation. Elle peut enfin jouer seule, séparée des autres facultés, dans une spéculation purement théorique d'où a été provisoirement écarté tout souci d'action.\par
Dans une âme saine, elle s'exerce tour à tour des trois manières, avec des degrés différents de liberté. Dans la première fonction, elle est une servante. Dans la seconde fonction, elle est destructrice et doit être réduite au silence dès qu'elle commence à fournir des arguments à la partie de l'âme qui, chez quiconque n'est pas dans l'état de perfection, se met toujours du côté du mal. Mais quand elle joue seule et séparée, il faut qu'elle dispose d'une liberté souveraine. Autrement il manque à l'être humain quelque chose d'essentiel.\par
Il en est de même dans une société saine. C'est pourquoi il serait désirable de constituer, dans le domaine de la publication, une réserve de liberté absolue, mais de manière qu'il soit entendu que les ouvrages qui s'y trouvent publiés n'engagent à aucun degré les auteurs et ne contiennent aucun conseil pour les lecteurs. Là pourraient se trouver étalés dans toute leur force tous les arguments en faveur des causes mauvaises. Il est bon et salutaire qu'ils soient étalés. N'importe qui pourrait y faire l'éloge de ce qu'il réprouve le plus. Il serait de notoriété publique que de tels ouvrages auraient pour objet, non pas de définir la position des auteurs en face des problèmes de la vie, mais de contribuer, par des recherches préliminaires, à l'énumération complète et correcte des données relatives à chaque problème. La loi empêcherait que leur publication implique pour l'auteur aucun risque d'aucune espèce.\par
Au contraire, les publications destinées à influer sur ce qu'on nomme l'opinion, c'est-à-dire en fait sur la conduite de la vie, constituent des actes et doivent être soumises aux mêmes restrictions que tous les actes. Autrement dit, elles ne doivent porter aucun préjudice illégitime à aucun être humain, et surtout elles ne doivent jamais contenir aucune négation, explicite ou implicite, des obligations éternelles envers l'être humain, une fois que ces obligations ont été solennellement reconnues par la loi.\par
La distinction des deux domaines, celui qui est hors de l'action et celui qui en fait partie, est impossible à formuler sur le papier en langage juridique. Mais cela n'empêche pas qu'elle soit parfaitement claire. La séparation de ces domaines est facile à établir en fait, si seulement la volonté d'y parvenir est assez forte.\par
Il est clair, par exemple, que la presse quotidienne et hebdomadaire tout entière se trouve dans le second domaine. Les revues également, car elles constituent toutes un foyer de rayonnement pour une certaine manière de penser ; seules celles qui renonceraient à cette fonction pourraient prétendre à la liberté totale.\par
De même pour la littérature. Ce serait une solution pour le débat qui s'est élevé récemment au sujet de la morale et de la littérature, et qui a été obscurci par le fait que tous les gens de talent, par solidarité professionnelle, se trouvaient d'un côté, et seulement des imbéciles et des lâches de l'autre.\par
Mais la position des imbéciles et des lâches n'en était pas moins dans une large mesure conforme à la raison. Les écrivains ont une manière inadmissible de jouer sur les deux tableaux. Jamais autant qu'à notre époque ils n'ont prétendu au rôle de directeurs de conscience et ne l'ont exercé. En fait, au cours des années qui ont précédé la guerre, personne ne le leur a disputé excepté les savants. La place autrefois occupée par des prêtres dans la vie morale du pays était tenue par des physiciens et des romanciers, ce qui suffit à mesurer la valeur de notre progrès. Mais si quelqu'un demandait des comptes aux écrivains sur l'orientation de leur influence, ils se réfugiaient avec indignation derrière le privilège sacré de l'art pour l'art.\par
Sans aucun doute, par exemple, Gide a toujours su que des livres comme {\itshape Les Nourritures terrestres} ou {\itshape Les Caves du Vatican} ont eu une influence sur la conduite pratique de la vie chez des centaines de jeunes gens, et il en a été fier. Il n'y a dès lors aucun motif de mettre de tels livres derrière la barrière intouchable de l'art pour l'art, et d'emprisonner un garçon qui jette quelqu'un hors d'un train en marche. On pourrait tout aussi bien réclamer les privilèges de l'art pour l'art en faveur du crime. Autrefois les surréalistes n'en étaient pas loin. Tout ce que tant d'imbéciles ont répété à satiété sur la responsabilité des écrivains dans notre défaite est, par malheur, certainement vrai.\par
Si un écrivain, à la faveur de la liberté totale accordée à l'intelligence pure, publie des écrits contraires aux principes de morale reconnus par la loi, et si plus tard il devient de notoriété publique un foyer d'influence, il est facile de lui demander s'il est prêt à faire connaître publiquement que ces écrits n'expriment pas sa position. Dans le cas contraire, il est facile de le punir. S'il ment, il est facile de le déshonorer. De plus, il doit être admis qu'à partir du moment où un écrivain tient une place parmi les influences qui dirigent l'opinion publique, il ne peut pas prétendre à une liberté illimitée. Là aussi, une définition juridique est impossible, mais les faits ne sont pas réellement difficiles à discerner. Il n'y a aucune raison de limiter la souveraineté de la loi au domaine des choses exprimables en formules juridiques, puisque cette souveraineté s'exerce aussi bien par des jugements d'équité.\par
De plus, le besoin même de liberté, si essentiel à l'intelligence, exige une protection contre la suggestion, la propagande, l'influence par obsession. Ce sont là des modes de contrainte, une contrainte particulière, que n'accompagnent pas la peur ou la douleur physique, mais qui n'en est pas moins une violence. La technique moderne lui fournit des instruments extrêmement efficaces. Cette contrainte, par sa nature, est collective, et les âmes humaines en sont victimes.\par
L'État, bien entendu, se rend criminel s'il en use lui-même, sauf le cas d'une nécessité criante de salut public. Mais il doit de plus en empêcher l'usage. La publicité, par exemple, doit être rigoureusement limitée par la loi ; la masse doit en être très considérablement réduite ; il doit lui être strictement interdit de jamais toucher à des thèmes qui appartiennent au domaine de la pensée.\par
De même, il peut y avoir répression contre la presse, les émissions radiophoniques, et toute autre chose semblable, non seulement pour atteinte aux principes de moralité publiquement reconnus, mais pour la bassesse du ton et de la pensée, le mauvais goût, la vulgarité, pour une atmosphère morale sournoisement corruptrice. Une telle répression peut s'exercer sans toucher si peu que ce soit à la liberté d'opinion. Par exemple, un journal peut être supprimé sans que les membres de la rédaction perdent le droit de publier où bon leur semble, ou même, dans les cas les moins graves, de rester groupés pour continuer le même journal sous un autre nom. Seulement, il aura été publiquement marqué d'infamie et risquera de l'être encore. La liberté d'opinion est due uniquement, et sous réserves, au journaliste, non au journal ; car le journaliste seul possède la capacité de former une opinion.\par
D'une manière générale, tous les problèmes concernant la liberté d'expression s'éclaircissent si l'on pose que cette liberté est un besoin de l'intelligence, et que l'intelligence réside uniquement dans l'être humain considéré seul. Il n'y a pas d'exercice collectif de l'intelligence. Par suite nul groupement ne peut légitimement prétendre à la liberté d'expression, parce que nul groupement n'en a le moins du monde besoin.\par
Bien au contraire, la protection de la liberté de penser exige qu'il soit interdit par la loi à un groupement d'exprimer une opinion. Car lorsqu'un groupe se met à avoir des opinions, il tend inévitablement à les imposer à ses membres. Tôt ou tard les individus se trouvent empêchés, avec un degré de rigueur plus ou moins grand, sur un nombre de problèmes plus ou moins considérables, d'exprimer des opinions opposées à celles du groupe, à moins d'en sortir. Mais la rupture avec un groupe dont on est membre entraîne toujours des souffrances, tout au moins une souffrance sentimentale. Et autant le risque, la possibilité de souffrance, sont des éléments sains et nécessaires de l'action, autant ce sont choses malsaines dans l'exercice de l'intelligence. Une crainte, même légère, provoque toujours soit du fléchissement, soit du raidissement, selon le degré de courage, et il n'en faut pas plus pour fausser l'instrument de précision extrêmement délicat et fragile que constitue l'intelligence. Même l'amitié à cet égard est un grand danger. L'intelligence est vaincue dès que l'expression des pensées est précédée, explicitement ou implicitement, du petit mot « nous ». Et quand la lumière de l'intelligence s'obscurcit, au bout d'un temps assez court l'amour du bien s'égare.\par
La solution pratique immédiate, c'est l'abolition des partis politiques. La lutte des partis, telle qu'elle existait dans la Troisième République, est intolérable ; le parti unique, qui en est d'ailleurs inévitablement l'aboutissement, est le degré extrême du mal ; il ne reste d'autre possibilité qu'une vie publique sans partis. Aujourd'hui, pareille idée sonne comme quelque chose de nouveau et d'audacieux. Tant mieux, puisqu'il faut du nouveau. Mais en fait c'est simplement la tradition de 1789. Aux yeux des gens de 1789, il n'y avait même pas d'autre possibilité ; une vie publique telle que la nôtre au cours du dernier demi-siècle leur aurait paru un hideux cauchemar ; ils n'auraient jamais cru possible qu'un représentant du peuple pût abdiquer sa dignité au point de devenir le membre discipliné d'un parti.\par
Rousseau d'ailleurs avait montré clairement que la lutte des partis tue automatiquement la République. Il en avait prédit les effets. Il serait bon d'encourager en ce moment la lecture du {\itshape Contrat Social}. En fait, à présent, partout où il y avait des partis politiques, la démocratie est morte. Chacun sait que les partis anglais ont des traditions, un esprit, une fonction tels qu'ils ne sont comparables à rien d'autre. Chacun sait aussi que les équipes concurrentes des États-Unis ne sont pas des partis politiques. Une démocratie où la vie publique est constituée par la lutte des partis politiques est incapable d'empêcher la formation d'un parti qui ait pour but avoué de la détruire. Si elle fait des lois d'exception, elle s'asphyxie elle-même. Si elle n'en fait pas, elle est aussi en sécurité qu'un oiseau devant un serpent.\par
Il faudrait distinguer deux espèces de groupements, les groupements d'intérêts, auxquels l'organisation et la discipline seraient autorisées dans une certaine mesure, et les groupements d'idées, auxquels elles seraient rigoureusement interdites. Dans la situation actuelle, il est bon de permettre aux gens de se grouper pour défendre leurs intérêts, c'est-à-dire les gros sous et les choses similaires, et de laisser ces groupements agir dans des limites très étroites et sous la surveillance perpétuelle des pouvoirs publics. Mais il ne faut pas les laisser toucher aux idées. Les groupements où s'agitent des pensées doivent être moins des groupements que des milieux plus ou moins fluides. Quand une action s'y dessine, il n'y a pas de raison qu'elle soit exécutée par d'autres que par ceux qui l'approuvent.\par
Dans le mouvement ouvrier par exemple, une telle distinction mettrait fin à une confusion inextricable. Dans la période qui a précédé la guerre, trois orientations sollicitaient et tiraillaient perpétuellement tous les ouvriers. D'abord la lutte pour les gros sous ; puis les restes, de plus en plus faibles, mais toujours un peu vivants, du vieil esprit syndicaliste de jadis, idéaliste et plus ou moins libertaire ; enfin les partis politiques. Fréquemment, au cours d'une grève, les ouvriers qui souffraient et luttaient auraient été bien incapables de se rendre compte s'il s'agissait de salaires, ou d'une poussée du vieil esprit syndical, ou d'une opération politique menée par un parti ; et personne non plus ne pouvait s'en rendre compte du dehors.\par
Une telle situation est impossible. Quand la guerre a éclaté, les syndicats en France étaient morts ou presque, malgré les millions d'adhérents ou à cause d'eux. Ils ont repris un embryon de vie, après une longue léthargie, à l'occasion de la résistance contre l'envahisseur. Cela ne prouve pas qu'ils soient viables. Il est tout à fait clair qu'ils avaient été tués ou presque par deux poisons dont chacun séparément était mortel.\par
Des syndicats ne peuvent pas vivre si les ouvriers y sont obsédés par les sous au même degré que dans l'usine, au cours du travail aux pièces. D'abord parce qu'il en résulte l'espèce de mort morale toujours causée par l'obsession de l'argent. Puis parce que, dans les conditions sociales présentes, le syndicat, étant alors un facteur perpétuellement agissant dans la vie économique du pays, finit inévitablement par être transformé en organisation professionnelle unique, obligatoire, mise au pas dans la vie officielle. Il est alors passé à l'état de cadavre.\par
D'autre part, il est non moins clair que le syndicat ne peut pas vivre à côté des partis politiques. Il y a là une impossibilité qui est de l'ordre des lois mécaniques. Pour une raison analogue, d'ailleurs, le parti socialiste ne peut pas vivre à côté du parti communiste, parce que le second possède la qualité de parti, si l'on peut dire, à un degré beaucoup plus élevé.\par
D'ailleurs l'obsession des salaires renforce l'influence communiste, parce que les questions d’argent, si vivement qu'elles touchent presque tous les hommes, dégagent en même temps pour tous les hommes un ennui si mortel que la perspective apocalyptique de la révolution, selon la version communiste, est indispensable pour compenser. Si les bourgeois n'ont pas le même besoin d'apocalypse, c'est que les chiffres élevés ont une poésie, un prestige qui tempère un peu l'ennui lié à l'argent, au lieu que quand l'argent se compte en sous, l'ennui est à l'état pur. D'ailleurs le goût des bourgeois grands et petits pour le fascisme montre que, malgré tout, eux aussi s'ennuient.\par
Le gouvernement de Vichy a créé en France pour les ouvriers des organisations professionnelles uniques et obligatoires. Il est regrettable qu'il leur ait donné, selon la mode moderne, le nom de corporation, qui désigne en réalité quelque chose de tellement différent et de si beau. Mais il est heureux que ces organisations mortes soient là pour assumer la partie morte de l'activité syndicale. Il serait dangereux de les supprimer. Il vaut bien mieux les charger de l'action quotidienne pour les gros sous et les revendications dites immédiates. Quant aux partis politiques, s'ils étaient tous rigoureusement interdits dans un climat général de liberté, il faut espérer que leur existence clandestine serait au moins difficile.\par
En ce cas, les syndicats ouvriers, s'il y reste encore une étincelle de vie véritable, pourraient redevenir peu à peu l'expression de la pensée ouvrière, l'organe de l'honneur ouvrier. Selon la tradition du mouvement ouvrier français, qui s'est toujours regardé comme responsable de tout l'univers, ils s'intéresseraient à tout ce qui touche à la justice – y compris, le cas échéant, les questions de gros sous, mais de loin en loin et pour sauver des êtres humains de la misère.\par
\par
Bien entendu, ils devraient pouvoir exercer une influence sur les organisations professionnelles selon des modalités définies par la loi.\par
Il n'y aurait peut-être que des avantages à interdire aux organisations professionnelles de déclencher une grève, et à le permettre aux syndicats, avec des réserves, en faisant correspondre des risques à cette responsabilité, en interdisant toute contrainte, et en protégeant la continuité de la vie économique.\par
Quant au lock-out, il n'y a pas de motif de ne pas l'interdire tout à fait.\par
L'autorisation des groupements d'idées pourrait être soumise à deux conditions. L'une, que l'excommunication n'y existe pas. Le recrutement se ferait librement par voie d'affinité, sans toutefois que personne puisse être invité à adhérer à un ensemble d'affirmations cristallisées en formules écrites ; mais un membre une fois admis ne pourrait être exclu que pour faute contre l'honneur ou délit de noyautage ; délit qui impliquerait d'ailleurs une organisation illégale et par suite exposerait à un châtiment plus grave.\par
Il y aurait là véritablement une mesure de salut public, l'expérience ayant montré que les États totalitaires sont établis par les partis totalitaires, et que les partis totalitaires se forgent à coups d'exclusions pour délit d'opinion.\par
L'autre condition pourrait être qu'il y ait réellement circulation d'idées, et témoignage tangible de cette circulation, sous forme de brochures, revues ou bulletins dactylographiés dans lesquels soient étudiés des problèmes d'ordre général. Une trop grande uniformité d'opinions rendrait un groupement suspect.\par
Au reste, tous les groupements d'idées seraient autorisés à agir comme bon leur semblerait, à condition de ne pas violer la loi et de ne contraindre leurs membres par aucune discipline.\par
Quant aux groupements d'intérêts, leur surveillance devrait impliquer d'abord une distinction ; c'est que le mot intérêt exprime quelquefois le besoin et quelquefois tout autre chose. S'il s'agit d'un ouvrier pauvre, l'intérêt, cela veut dire la nourriture, le logement, le chauffage. Pour un patron, cela veut dire autre chose. Quand le mot est pris au premier sens, l'action des pouvoirs publics devrait consister principalement à stimuler, soutenir, protéger la défense des intérêts. Au cas contraire, l'activité des groupements d'intérêts doit être continuellement contrôlée, limitée, et toutes les fois qu'il y a lieu réprimée par les pouvoirs publics. Il va de soi que les limites les plus étroites et les châtiments les plus douloureux conviennent à celles qui par nature sont les plus puissantes.\par
Ce qu'on a appelé la liberté d'association a été en fait jusqu'ici la liberté des associations. Or les associations n'ont pas à être libres ; elles sont des instruments, elles doivent être asservies. La liberté ne convient qu'à l'être humain.\par
Quant à la liberté de pensée, on dit vrai dans une large mesure quand on dit que sans elle il n'y a pas de pensée. Mais il est plus vrai encore de dire que quand la pensée n'existe pas, elle n'est pas non plus libre. Il y avait eu beaucoup de liberté de pensée au cours des dernières années, mais il n'y avait pas de pensée. C'est à peu près la situation de l'enfant qui, n'ayant pas de viande, demande du sel pour la saler.
\subsection[{J. La sécurité}]{J. \\
La sécurité}
\noindent \par
La sécurité est un besoin essentiel de l'âme. La sécurité signifie que l'âme n'est pas sous le poids de la peur ou de la terreur, excepté par l'effet d'un concours de circonstances accidentelles et pour des moments rares et courts. La peur ou la terreur, comme états d'âme durables, sont des poisons presque mortels, que la cause en soit la possibilité du chômage, ou la répression policière, ou la présence d'un conquérant étranger, ou l'attente d'une invasion probable, ou tout autre malheur qui semble surpasser les forces humaines.\par
Les maîtres romains exposaient un fouet dans le vestibule à la vue des esclaves, sachant que ce spectacle mettait les âmes dans l'état de demi-mort indispensable à l'esclavage. D'un autre côté, d'après les Égyptiens, le juste doit pouvoir dire après la mort : « Je n'ai causé de peur à personne. »\par
Même si la peur permanente constitue seulement un état latent, de manière à n'être que rarement ressentie comme une souffrance, elle est toujours une maladie. C'est une demi-paralysie de l'âme.
\subsection[{K. Le risque}]{K. \\
Le risque}
\noindent \par
Le risque est un besoin essentiel de l'âme. L'absence de risque suscite une espèce d'ennui qui paralyse autrement que la peur, mais presque autant. D'ailleurs il y a des situations qui, impliquant une angoisse diffuse sans risques précis, communiquent les deux maladies à la fois.\par
Le risque est un danger qui provoque une réaction réfléchie ; c'est-à-dire qu'il ne dépasse pas les ressources de l'âme au point de l'écraser sous la peur. Dans certains cas, il enferme une part de jeu ; dans d'autres cas, quand une obligation précise pousse l'homme à y faire face, il constitue le plus haut stimulant possible.\par
La protection des hommes contre la peur et la terreur n'implique pas la suppression du risque ; elle implique au contraire la présence permanente d'une certaine quantité de risque dans tous les aspects de la vie sociale ; car l'absence de risque affaiblit le courage au point de laisser l'âme, le cas échéant, sans la moindre protection intérieure contre la peur. Il faut seulement que le risque se présente dans des conditions telles qu'il ne se transforme pas en sentiment de fatalité.
\subsection[{L. La propriété privée}]{L. \\
La propriété privée}
\noindent \par
La propriété privée est un besoin vital de l'âme. L'âme est isolée, perdue, si elle n'est pas dans un entourage d'objets qui soient pour elle comme un prolongement des membres du corps. Tout homme est invinciblement porté à s'approprier par la pensée tout ce dont il a fait longtemps et continuellement usage pour le travail, le plaisir ou les nécessités de la vie. Ainsi un jardinier, au bout d'un certain temps, sent que le jardin est à lui. Mais là où le sentiment d'appropriation ne coïncide pas avec la propriété juridique, l'homme est continuellement menacé d'arrachements très douloureux.\par
Si la propriété privée est reconnue comme un besoin, cela implique pour tous la possibilité de posséder autre chose que les objets de consommation courante. Les modalités de ce besoin varient beaucoup selon les circonstances ; mais il est désirable que la plupart des gens soient propriétaires de leur logement et d'un peu de terre autour, et, quand il n'y a pas impossibilité technique, de leurs instruments de travail. La terre et le cheptel sont au nombre des instruments du travail paysan.\par
Le principe de la propriété privée est violé dans le cas d'une terre travaillée par des ouvriers agricoles et des domestiques de ferme aux ordres d'un régisseur, et possédée par des citadins qui en touchent les revenus. Car de tous ceux qui ont une relation avec cette terre, il n'y a personne qui, d'une manière ou d'une autre, n'y soit étranger. Elle est gaspillée, non du point de vue du blé, mais du point de vue de la satisfaction qu'elle pourrait fournir au besoin de propriété.\par
Entre ce cas extrême et l'autre cas limite du paysan qui cultive avec sa famille la terre qu'il possède, il y a beaucoup d'intermédiaires où le besoin d'appropriation des hommes est plus ou moins méconnu.
\subsection[{M. La propriété collective}]{M. \\
La propriété collective}
\noindent \par
La participation aux biens collectifs, participation consistant non pas en jouissance matérielle, mais en un sentiment de propriété, est un besoin non moins important. Il s'agit d'un état d'esprit plutôt que d'une disposition juridique. Là où il y a véritablement une vie civique, chacun se sent personnellement propriétaire des monuments publics, des jardins, de la magnificence déployée dans les cérémonies, et le luxe que presque tous les êtres humains désirent est ainsi accordé même aux plus pauvres. Mais ce n'est pas seulement l'État qui doit fournir cette satisfaction, c'est toute espèce de collectivité.\par
Une grande usine moderne constitue un gaspillage en ce qui concerne le besoin de propriété. Ni les ouvriers, ni le directeur qui est aux gages d'un conseil d'administration, ni les membres du conseil qui ne la voient jamais, ni les actionnaires qui en ignorent l'existence, ne peuvent trouver en elle la moindre satisfaction à ce besoin.\par
Quand les modalités d'échange et d'acquisition entraînent le gaspillage des nourritures matérielles et morales, elles sont à transformer.\par
Il n'y a aucune liaison de nature entre la propriété et l'argent. La liaison établie aujourd'hui est seulement le fait d'un système qui a concentré sur l'argent la force de tous les mobiles possibles. Ce système étant malsain, il faut opérer la dissociation inverse.\par
Le vrai critérium, pour la propriété, est qu'elle est légitime pour autant qu'elle est réelle. Ou plus exactement, les lois concernant la propriété sont d'autant meilleures qu'elles tirent mieux parti des possibilités enfermées dans les biens de ce monde pour la satisfaction du besoin de propriété commun à tous les hommes.\par
Par conséquent, les modalités actuelles d'acquisition et de possession doivent être transformées au nom du principe de propriété. Toute espèce de possession qui ne satisfait chez personne le besoin de propriété privée ou collective peut raisonnablement être regardée comme nulle.\par
Cela ne signifie pas qu'il faille la transférer à l'État, mais plutôt essayer d'en faire une propriété véritable.
\subsection[{N. La vérité}]{N. \\
La vérité}
\noindent \par
Le besoin de vérité est plus sacré qu'aucun autre. Il n'en est pourtant jamais fait mention. On a peur de lire quand on s'est une fois rendu compte de la quantité et de l'énormité des faussetés matérielles étalées sans honte, même dans les livres des auteurs les plus réputés. On lit alors comme on boirait l'eau d'un puits douteux.\par
Il y a des hommes qui travaillent huit heures par jour et font le grand effort de lire le soir pour s'instruire. Ils ne peuvent pas se livrer à des vérifications dans les grandes bibliothèques. Ils croient le livre sur parole. On n'a pas le droit de leur donner à manger du faux. Quel sens cela a-t-il d'alléguer que les auteurs sont de bonne foi ? Eux ne travaillent pas physiquement huit heures par jour. La société les nourrit pour qu'ils aient le loisir et se donnent la peine d'éviter l'erreur. Un aiguilleur cause d'un déraillement serait mal accueilli en alléguant qu'il est de bonne foi.\par
À plus forte raison est-il honteux de tolérer l'existence de journaux dont tout le monde sait qu'aucun collaborateur ne pourrait y demeurer s'il ne consentait parfois à altérer sciemment la vérité.\par
Le public se défie des journaux, mais sa défiance ne le protège pas. Sachant en gros qu'un journal contient des vérités et des mensonges, il répartit les nouvelles annoncées entre ces deux rubriques, mais au hasard, au gré de ses préférences. Il est ainsi livré à l'erreur.\par
Tout le monde sait que, lorsque le journalisme se confond avec l'organisation du mensonge, il constitue un crime. Mais on croit que c'est un crime impunissable. Qu'est-ce qui peut bien empêcher de punir une activité une fois qu'elle a été reconnue comme criminelle ? D'où peut bien venir cette étrange conception de crimes non punissables ? C'est une des plus monstrueuses déformations de l'esprit juridique.\par
Ne serait-il pas temps de proclamer que tout crime discernable est punissable, et qu'on est résolu, si on a en l'occasion, à punir tous les crimes ?\par
Quelques mesures faciles de salubrité publique protégeraient la population contre les atteintes à la vérité.\par
La première serait l'institution, pour cette protection, de tribunaux spéciaux, hautement honorés, composés de magistrats spécialement choisis et formés. Ils seraient tenus de punir de réprobation publique toute erreur évitable, et pourraient infliger la prison et le bagne en cas de récidive fréquente, aggravée par une mauvaise foi démontrée.\par
Par exemple un amant de la Grèce antique, lisant dans le dernier livre de Maritain : « les plus grands penseurs de l'antiquité n'avaient pas songé à condamner l'esclavage », traduirait Maritain devant un de ces tribunaux. Il y apporterait le seul texte important qui nous soit parvenu sur l'esclavage, celui d'Aristote. Il y ferait lire aux magistrats la phrase : « quelques-uns affirment que l'esclavage est absolument contraire à la nature et à la raison ». Il ferait observer que rien ne permet de supposer que ces quelques-uns n'aient pas été au nombre des plus grands penseurs de l'antiquité. Le tribunal blâmerait Maritain pour avoir imprimé, alors qu'il lui était si facile d'éviter l'erreur, une affirmation fausse et constituant, bien qu'involontairement, une calomnie atroce contre une civilisation tout entière. Tous les journaux quotidiens, hebdomadaires et autres, toutes les revues et la radio seraient dans l'obligation de porter à la connaissance du public le blâme du tribunal, et, le cas échéant, la réponse de Maritain. Dans ce cas précis, il pourrait difficilement y en avoir une.\par
Le jour où {\itshape Gringoire} publia {\itshape in extenso} un discours attribué à un anarchiste espagnol qui avait été annoncé comme orateur dans une réunion parisienne, mais qui en fait, au dernier moment, n'avait pu quitter l'Espagne, un pareil tribunal n'aurait pas été superflu. La mauvaise foi étant dans un tel cas plus évidente que deux et deux font quatre, la prison ou le bagne n'auraient peut-être pas été trop sévères.\par
Dans ce système, il serait permis à n'importe qui, ayant reconnu une erreur évitable dans un texte imprimé ou dans une émission de la radio, de porter une accusation devant ces tribunaux.\par
La deuxième mesure serait d'interdire absolument toute propagande de toute espèce par la radio ou par la presse quotidienne. On ne permettrait à ces deux instruments de servir qu'à l'information non tendancieuse.\par
Les tribunaux dont il vient d'être question veilleraient à ce que l'information ne soit pas tendancieuse.\par
\par
Pour les organes d'information ils pourraient avoir à juger, non seulement les affirmations erronées, mais encore les omissions volontaires et tendancieuses.\par
Les milieux où circulent des idées et qui désirent les faire connaître auraient droit seulement à des organes hebdomadaires, bi-mensuels ou mensuels. Il n'est nullement besoin d'une fréquence plus grande si l'on veut faire penser et non abrutir.\par
La correction des moyens de persuasion serait assurée par la surveillance des mêmes tribunaux, qui pourraient supprimer un organe en cas d'altération trop fréquente de la vérité. Mais ses rédacteurs pourraient le faire reparaître sous un autre nom.\par
Dans tout cela il n'y aurait pas la moindre atteinte aux libertés publiques. Il y aurait satisfaction du besoin le plus sacré de l'âme humaine, le besoin de protection contre la suggestion et l'erreur.\par
Mais qui garantit l'impartialité des juges ? objectera-t-on. La seule garantie, en dehors de leur indépendance totale, c'est qu'ils soient issus de milieux sociaux très différents, qu'ils soient naturellement doués d'une intelligence étendue, claire et précise, et qu'ils soient formés dans une école où ils reçoivent une éducation non pas juridique, mais avant tout spirituelle, et intellectuelle en second lieu. Il faut qu'ils s'y accoutument à aimer la vérité.\par
Il n'y a aucune possibilité de satisfaire chez un peuple le besoin de vérité si l'on ne peut trouver à cet effet des hommes qui aiment la vérité.
\section[{Deuxième partie. Le déracinement}]{Deuxième partie. \\
Le déracinement}\renewcommand{\leftmark}{Deuxième partie. \\
Le déracinement}

\noindent \par
L'enracinement est peut-être le besoin le plus important et le plus méconnu de l'âme humaine. C'est un des plus difficiles à définir. Un être humain a une racine par sa participation réelle, active et naturelle à l'existence d'une collectivité qui conserve vivants certains trésors du passé et certains pressentiments d'avenir. Participation naturelle, c'est-à-dire amenée automatiquement par le lieu, la naissance, la profession, l'entourage. Chaque être humain a besoin d'avoir de multiples racines. Il a besoin de recevoir la presque totalité de sa vie morale, intellectuelle, spirituelle, par l'intermédiaire des milieux dont il fait naturellement partie.\par
Les échanges d'influences entre milieux très différents ne sont pas moins indispensables que l'enracinement dans l'entourage naturel. Mais un milieu déterminé doit recevoir une influence extérieure non pas comme un apport, mais comme un stimulant qui rende sa vie propre plus intense. Il ne doit se nourrir des apports extérieurs qu'après les avoir digérés, et les individus qui le composent ne doivent les recevoir qu'à travers lui. Quand un peintre de réelle valeur va dans un musée, son originalité en est confirmée. Il doit en être de même pour les diverses populations du globe terrestre et les différents milieux sociaux.\par
Il y a déracinement toutes les fois qu'il y a conquête militaire, et en ce sens la conquête est presque toujours un mal. Le déracinement est au minimum quand les conquérants sont des migrateurs qui s'installent dans le pays conquis, se mélangent à la population et prennent racine eux-mêmes. Tel fut le cas des Hellènes en Grèce, des Celtes en Gaule, des Maures en Espagne. Mais quand le conquérant reste étranger au territoire dont il est devenu possesseur, le déracinement est une maladie presque mortelle pour les populations soumises. Il atteint le degré le plus aigu quand il y a déportations massives, comme dans l'Europe occupée par l'Allemagne ou dans la boucle du Niger, ou quand il y a suppression brutale de toutes les traditions locales, comme dans les possessions françaises d'Océanie (s'il faut croire Gauguin et Alain Gerbault).\par
Même sans conquête militaire, le pouvoir de l'argent et la domination économique peuvent imposer une influence étrangère au point de provoquer la maladie du déracinement.\par
Enfin les relations sociales à l'intérieur d'un même pays peuvent être des facteurs très dangereux de déracinement. Dans nos contrées, de nos jours, la conquête mise à part, il y a deux poisons qui propagent cette maladie. L'un est l'argent. L'argent détruit les racines partout où il pénètre, en remplaçant tous les mobiles par le désir de gagner. Il l'emporte sans peine sur les autres mobiles parce qu'il demande un effort d'attention tellement moins grand. Rien n'est si clair et si simple qu'un chiffre.\par
\subsection[{A. Déracinement ouvrier}]{A. \\
Déracinement ouvrier}
\noindent \par
Il est une condition sociale entièrement et perpétuellement suspendue à l'argent, c'est salariat, surtout depuis que le salaire aux pièces oblige chaque ouvrier à avoir l'attention toujours fixée sur le compte des sous. C'est dans cette condition sociale que la maladie du déracinement est la plus aiguë. Bernanos a écrit que nos ouvriers ne sont quand même pas des immigrés comme ceux de M. Ford. La principale difficulté sociale de notre époque vient du fait qu'en un sens ils le sont. Quoique demeurés sur place géographiquement, ils ont été moralement déracinés, exilés et admis de nouveau, comme par tolérance, à titre de chair à travail. Le chômage est, bien entendu, un déracinement à la deuxième puissance. Ils ne sont chez eux ni dans les usines, ni dans leurs logements, ni dans les partis et syndicats soi-disant faits pour eux, ni dans les lieux de plaisir, ni dans la culture intellectuelle s'ils essayent de l'assimiler.\par
Car le second facteur de déracinement est l’instruction telle qu'elle est conçue aujourd'hui. La Renaissance a partout provoqué une coupure entre les gens cultivés et la masse ; mais en séparant la culture de la tradition nationale, elle la plongeait du moins dans la tradition grecque. Depuis, les liens avec les traditions nationales n'ont pas été renoués, mais la Grèce a été oubliée. Il en est résulté une culture qui s'est développée dans un milieu très restreint, séparé du monde, dans une atmosphère confinée, une culture considérablement orientée vers la technique et influencée par elle, très teintée de pragmatisme, extrêmement fragmentée par la spécialisation, tout à fait dénuée à la fois de contact avec cet univers-ci et d'ouverture vers l'autre monde.\par
De nos jours, un homme peut appartenir aux milieux dits cultivés, d'une part sans avoir aucune conception concernant la destinée humaine, d'autre part sans savoir, par exemple, que toutes les constellations ne sont pas visibles en toutes saisons. On croit couramment qu’un petit paysan d'aujourd'hui, élève de l'école primaire, en sait plus que Pythagore, parce qu'il répète docilement que la terre tourne autour du soleil. Mais en fait il ne regarde plus les étoiles. Ce soleil dont on lui parle en classe n'a pour lui aucun rapport avec celui qu'il voit. On l'arrache à l'univers qui l'entoure, comme on arrache les petits Polynésiens à leur passé en les forçant à répéter : « Nos ancêtres les Gaulois avaient les cheveux blonds. »\par
Ce qu'on appelle aujourd'hui instruire les masses, c'est prendre cette culture moderne, élaborée dans un milieu tellement fermé, tellement taré, tellement indifférent à la vérité, en ôter tout ce qu'elle peut encore contenir d'or pur, opération qu'on nomme vulgarisation, et enfourner le résidu tel quel dans la mémoire des malheureux qui désirent apprendre, comme on donne la becquée à des oiseaux.\par
D'ailleurs le désir d'apprendre pour apprendre, le désir de vérité est devenu très rare. Le prestige de la culture est devenu presque exclusivement social, aussi bien chez le paysan qui rêve d'avoir un fils instituteur ou l'instituteur qui rêve d'avoir un fils normalien, que chez les gens du monde qui flagornent les savants et les écrivains réputés.\par
Les examens exercent sur la jeunesse des écoles le même pouvoir d'obsession que les sous sur les ouvriers qui travaillent aux pièces. Un système social est profondément malade quand un paysan travaille la terre avec la pensée que, s'il est paysan, c'est parce qu'il n'était pas assez intelligent pour devenir instituteur.\par
Le mélange d'idées confuses et plus ou moins fausses connu sous le nom de marxisme, mélange auquel depuis Marx il n'y a guère eu que des intellectuels bourgeois médiocres qui aient eu part, est aussi pour les ouvriers un apport complètement étranger, inassimilable, et d'ailleurs en soi dénué de valeur nutritive, car on l'a vidé de presque toute la vérité contenue dans les écrits de Marx. On y ajoute parfois une vulgarisation scientifique de qualité encore inférieure. Le tout ne peut que porter le déracinement des ouvriers à son comble.\par
Le déracinement est de loin la plus dangereuse maladie des sociétés humaines, car il se multiplie lui-même. Des êtres vraiment déracinés n'ont guère que deux comportements possibles : ou ils tombent dans une inertie de l'âme presque équivalente à la mort, comme la plupart des esclaves au temps de l'Empire romain, ou ils se jettent dans une activité tendant toujours à déraciner, souvent par les méthodes les plus violentes, ceux qui ne le sont pas encore ou ne le sont qu'en partie.\par
Les Romains étaient une poignée de fugitifs qui se sont agglomérés artificiellement en une cité ; et ils ont privé les populations méditerranéennes de leur vie propre, de leur patrie, de leur tradition, de leur passé, à un degré tel que la postérité les a pris, sur leur propre parole, pour les fondateurs de la civilisation sur ces territoires, Les Hébreux étaient des esclaves évadés, et ils ont exterminé ou réduit en servitude toutes les populations de Palestine. Les Allemands, au moment où Hitler s'est emparé d'eux, étaient vraiment, comme il le répétait sans cesse, une nation de prolétaires, c'est-à-dire de déracinés ; l'humiliation de 1918, l'inflation, l'industrialisation à outrance et surtout l'extrême gravité de la crise de chômage avaient porté chez eux la maladie morale au degré d'acuité qui entraîne l'irresponsabilité. Les Espagnols et les Anglais qui, à partir du XVI\textsuperscript{e} siècle, ont massacré ou asservi des populations de couleur étaient des aventuriers presque sans contact avec la vie profonde de leur pays. Il en est de même pour une partie de l'Empire français, qui d'ailleurs a été constitué dans une période où la tradition française avait une vitalité affaiblie. Qui est déraciné déracine. Qui est enracine ne déracine pas.\par
Sous le même nom de révolution, et souvent sous des mots d'ordre et des thèmes de propagande identiques, sont dissimulées deux conceptions absolument opposées. L'une consiste à transformer la société de manière que les ouvriers puissent y avoir des racines ; l'autre consiste à étendre à toute la société la maladie du déracinement qui a été infligée aux ouvriers. Il ne faut pas dire ou penser que la seconde opération puisse jamais être un prélude de la première ; cela est faux. Ce sont deux directions opposées, qui ne se rejoignent pas.\par
La seconde conception est aujourd'hui beaucoup plus fréquente que la première, à la fois parmi les militants et dans la masse des ouvriers. Il va de soi qu'elle tend à l'emporter de plus en plus, à mesure que le déracinement se prolonge et augmente ses ravages. Il est facile de comprendre que, d'un jour à l'autre, le mal peut devenir irréparable.\par
Du côté des conservateurs, il y a une équivoque analogue. Un petit nombre désire réellement le réenracinement des ouvriers ; simplement leur désir s'accompagne d'images dont la plupart, au lieu d'être relatives à l'avenir, sont empruntées à un passé d'ailleurs en partie fictif. Les autres désirent purement et simplement maintenir ou aggraver la condition de matière humaine à laquelle le prolétariat est réduit.\par
Ainsi ceux qui désirent réellement le bien, déjà peu nombreux, s'affaiblissent encore en se répartissant entre deux camps hostiles avec lesquels ils n'ont rien de commun.\par
L'effondrement subit de la France, qui a surpris tout le monde partout, a simplement montré à quel point le pays était déraciné. Un arbre dont les racines sont presque entièrement rongées tombe au premier choc. Si la France a présenté un spectacle plus pénible qu'aucun autre pays d'Europe, c'est que la civilisation moderne avec ses poisons y était installée plus avant qu'ailleurs, à l'exception de l'Allemagne. Mais en Allemagne le déracinement avait pris la forme agressive, et en France il a pris celui de la léthargie et de la stupeur. La différence tient à des causes plus ou moins cachées, mais dont on pourrait trouver quelques-unes, sans doute, si l'on cherchait. Inversement, le pays qui devant la première vague de terreur allemande s'est de loin le mieux tenu est celui où la tradition est la plus vivante et la mieux préservée, c'est-à-dire l'Angleterre.\par
En France, le déracinement de la condition prolétarienne avait réduit une grande partie des ouvriers à un état de stupeur inerte et jeté une autre partie dans une attitude de guerre à l'égard de la société. Le même argent qui avait brutalement coupé les racines dans les milieux ouvriers les avait rongées dans les milieux bourgeois, car la richesse est cosmopolite ; le faible attachement au pays qui pouvait y demeurer intact était de bien loin dépassé, surtout depuis 1936, par la peur et la haine à l'égard des ouvriers. Les paysans étaient, eux aussi, presque déracinés depuis la guerre de 1914, démoralisés par le rôle de chair à canon qu'ils y avaient joué, par l'argent qui prenait dans leur vie une part toujours croissante, et par des contacts beaucoup trop fréquents avec la corruption des villes. Quant à l'intelligence, elle était presque éteinte.\par
Cette maladie générale du pays a pris la forme d'une espèce de sommeil qui seul a empêché la guerre civile. La France a haï la guerre qui menaçait de l'empêcher de dormir. À moitié assommée par le coup terrible de mai et juin 1940, elle s'est jetée dans les bras de Pétain pour pouvoir continuer à dormir dans un semblant de sécurité. Depuis lors l'oppression ennemie a transformé ce sommeil en un cauchemar tellement douloureux qu'elle s'agite et attend anxieusement les secours extérieurs qui l'éveilleront.\par
Sous l'effet de la guerre, la maladie du déracinement a pris dans toute l'Europe une acuité telle qu'on peut légitimement en être épouvanté. La seule indication qui donne quelque espoir, c'est que la souffrance a rendu un certain degré de vie à des souvenirs naguère presque morts, comme en France ceux de 1789.\par
Quant aux pays d'Orient, où depuis quelques siècles, mais surtout depuis cinquante ans, les Blancs ont porté la maladie du déracinement dont ils souffrent, le Japon montre suffisamment quelle acuité prend chez eux la forme active de la maladie. L'Indochine est un exemple de la forme passive. L'Inde, où existe encore une tradition vivante, est assez contaminée pour que ceux mêmes qui parlent publiquement au nom de cette tradition rêvent d'établir sur leur territoire une nation du type occidental et moderne. La Chine est très mystérieuse. La Russie, qui est toujours mi-européenne, mi-orientale, l'est bien autant ; car on ne peut savoir si l'énergie qui la couvre de gloire procède, comme pour les Allemands, d'un déracinement du genre actif, ce que l'histoire des vingt-cinq dernières années porterait d'abord à croire, ou s'il s'agit surtout de la vie profonde du peuple issue du fond des âges et demeurée souterrainement presque intacte.\par
Quant au continent américain, comme son peuplement, depuis plusieurs siècles, est fondé avant tout sur l'immigration, l'influence dominante qu'il va probablement exercer aggrave beaucoup le péril.\par
Dans cette situation presque désespérée, on ne peut trouver ici-bas de secours que dans les îlots de passé demeurés vivants sur la surface de la terre. Ce n'est pas qu'il faille approuver le tapage fait par Mussolini autour de l'Empire romain, et essayer d'utiliser de la même manière Louis XIV. Les conquêtes ne sont pas de la vie, elles sont de la mort au moment même où elles se produisent. Ce sont les gouttes de passé vivant qui sont à préserver jalousement, partout, à Paris ou à Tahiti indistinctement, car il n'y en a pas trop sur le globe entier.\par
Il serait vain de se détourner du passé pour ne penser qu'à l'avenir. C'est une illusion dangereuse de croire qu'il y ait même là une possibilité. L'opposition entre l'avenir et le passé est absurde. L'avenir ne nous apporte rien, ne nous donne rien ; c'est nous qui pour le construire devons tout lui donner, lui donner notre vie elle-même. Mais pour donner il faut posséder, et nous ne possédons d'autre vie, d'autre sève, que les trésors hérités du passé et digérés, assimilés, recréés par nous. De tous les besoins de l'âme humaine, il n'y en a pas de plus vital que le passé.\par
L'amour du passé n'a rien à voir avec une orientation politique réactionnaire. Comme toutes les activités humaines, la révolution puise toute sa sève dans une tradition. Marx l'a si bien senti qu'il a tenu à faire remonter cette tradition aux âges les plus lointains en faisant de la lutte des classes l'unique principe d'explication historique. Au début de ce siècle encore, peu de choses en Europe étaient plus près du Moyen Âge que le syndicalisme français, unique reflet chez nous de l'esprit des corporations. Les faibles restes de ce syndicalisme sont au nombre des étincelles sur lesquelles il est le plus urgent de souffler.\par
Depuis plusieurs siècles, les hommes de race blanche ont détruit du passé partout, stupidement, aveuglément, chez eux et hors de chez eux. Si à certains égards il y a eu néanmoins progrès véritable au cours de cette période, ce n'est pas à cause de cette rage, mais malgré elle, sous l'impulsion du peu de passé demeuré vivant.\par
Le passé détruit ne revient jamais plus. La destruction du passé est peut-être le plus grand crime. Aujourd'hui, la conservation du peu qui reste devrait devenir presque une idée fixe. Il faut arrêter le déracinement terrible que produisent toujours les méthodes coloniales des Européens, même sous leurs formes les moins cruelles. Il faut s'abstenir, après la victoire, de punir l'ennemi vaincu en le déracinant encore davantage ; dès lors qu'il n'est ni possible ni désirable de l'exterminer, aggraver sa folie serait être plus fou que lui. Il faut aussi avoir en vue avant tout, dans toute innovation politique, juridique ou technique susceptible de répercussions sociales, un arrangement permettant aux êtres humains de reprendre des racines.\par
Cela ne signifie pas les confiner. Jamais au contraire l'aération n'a été plus indispensable. L'enracinement et la multiplication des contacts sont complémentaires. Par exemple, si, partout où la technique le permet – et au prix d'un léger effort dans cette direction elle le permettrait largement –, les ouvriers étaient dispersés et propriétaires chacun d'une maison, d'un coin de terre et d'une machine ; et si en revanche on ressuscitait pour les jeunes le Tour de France d'autrefois, au besoin à l'échelle internationale ; si les ouvriers avaient fréquemment l'occasion de faire des stages à l'atelier de montage où les pièces qu'ils fabriquent se combinent avec toutes les autres, ou d'aller aider à former des apprentis ; avec en plus une protection efficace des salaires, le malheur de la condition prolétarienne disparaîtrait.\par
On ne détruira pas la condition prolétarienne avec des mesures juridiques, qu'il s'agisse de la nationalisation des industries-clefs, ou de la suppression de la propriété privée, ou de pouvoirs accordés aux syndicats pour la conclusion de conventions collectives, ou de délégués d'usines, ou du contrôle de l'embauche. Toutes les mesures qu'on propose, qu'elles aient l'étiquette révolutionnaire ou réformiste, sont purement juridiques, et ce n'est pas sur le plan juridique que se situent le malheur des ouvriers et le remède à ce malheur. Marx l'aurait parfaitement compris s'il avait eu de la probité à l'égard de sa propre pensée, car c'est une évidence qui éclate dans les meilleures pages du {\itshape Capital.}\par
On ne peut pas chercher dans les revendications des ouvriers le remède à leur malheur. Plongés dans le malheur corps et âme, y compris l'imagination, comment imagineraient-ils quelque chose qui n'en porte pas la marque ? S'ils font un violent effort pour s'en dégager, ils tombent dans des rêveries apocalyptiques, ou cherchent une compensation dans un impérialisme ouvrier qui n'est pas plus à encourager que l'impérialisme national.\par
Ce qu'on peut chercher dans leurs revendications, c'est le signe de leurs souffrances. Or les revendications expriment toutes ou presque la souffrance du déracinement. S'ils veulent le contrôle de l'embauche et la nationalisation, c'est qu'ils sont obsédés par la peur du déracinement total, du chômage. S'ils veulent abolir la propriété privée, c'est qu'ils en ont assez d'être admis sur le lieu du travail comme des immigrés qu'on laisse entrer par grâce. C'est aussi là le ressort psychologique des occupations d'usines en juin 1936. Pendant quelques jours, ils ont éprouvé une joie pure, sans mélange, à être chez eux dans ces mêmes lieux ; une joie d'enfant qui ne veut pas penser au lendemain. Personne ne pouvait raisonnablement croire que le lendemain serait bon.\par
Le mouvement ouvrier français issu de la Révolution a été essentiellement un cri, moins de révolte que de protestation, devant la dureté impitoyable du sort à l'égard de tous les opprimés. Relativement à ce que l'on peut attendre d'un mouvement collectif, il y avait en celui-là beaucoup de pureté. Il a pris fin en 1914 ; depuis, il n'en est resté que des échos ; les poisons de la société environnante ont corrompu même le sens du malheur. Il faut tenter d'en retrouver la tradition ; mais on ne saurait souhaiter le ressusciter. Si belle que puisse être l'intonation d'un cri de douleur, on ne peut souhaiter l'entendre encore ; il est plus humain de souhaiter guérir la douleur.\par
La liste concrète des douleurs des ouvriers fournit celle des choses à modifier. Il faut supprimer d'abord le choc que subit le petit gars qui à douze ou treize ans sort de l'école et entre à l'usine. Certains ouvriers seraient tout à fait heureux si ce choc n'avait laissé une blessure toujours douloureuse ; mais ils ne savent pas eux-mêmes que leur souffrance vient du passé. L'enfant à l'école, bon ou mauvais élève, était un être dont l'existence était reconnue, qu'on cherchait à développer, chez qui on faisait appel aux meilleurs sentiments. Du jour au lendemain il devient un supplément à la machine, un peu moins qu'une chose, et on ne se soucie nullement qu'il obéisse sous l'impulsion des mobiles les plus bas, pourvu qu'il obéisse. La plupart des ouvriers ont subi au moins à ce moment de leur vie cette impression de ne plus exister, accompagnée d'une sorte de vertige intérieur, que les intellectuels ou les bourgeois, même dans les plus grandes souffrances, ont très rarement l'occasion de connaître. Ce premier choc, reçu si tôt, imprime souvent une marque ineffaçable. Il peut rendre l'amour du travail définitivement impossible.\par
Il faut changer le régime de l'attention au cours des heures de travail, la nature des stimulants qui poussent à vaincre la paresse ou l'épuisement – stimulants qui aujourd'hui ne sont que la peur et les sous –, la nature de l'obéissance, la quantité trop faible d'initiative, d'habileté et de réflexion demandée aux ouvriers, l'impossibilité où ils sont de prendre part par la pensée et le sentiment à l'ensemble du travail de l'entreprise, l'ignorance parfois complète de la valeur, de l'utilité sociale, de la destination des choses qu'ils fabriquent, la séparation complète de la vie du travail et de la vie familiale. On pourrait allonger la liste.\par
Le désir de réformes mis à part, trois espèces de facteurs jouent dans le régime de la production : techniques, économiques, militaires. Aujourd'hui, l'importance des facteurs militaires dans la production correspond à celle de la production dans la conduite de la guerre ; autrement dit, elle est très considérable.\par
Du point de vue militaire, l'entassement de milliers d'ouvriers dans d'immenses bagnes industriels où les ouvriers vraiment qualifiés sont en tout petit nombre, est une double absurdité. Les conditions militaires actuelles exigent, d'une part que la production industrielle soit dispersée, d'autre part que le plus grand nombre des ouvriers du temps de paix consiste en professionnels instruits, sous les ordres de qui on puisse mettre immédiatement, en cas de crise internationale ou de guerre, une multitude de femmes, de jeunes garçons, d'hommes d'âge mûr, pour augmenter immédiatement le volume de la production. Rien n'a contribué davantage à paralyser si longtemps la production de guerre anglaise que le manque d'ouvriers qualifiés.\par
Mais comme on ne peut pas faire exécuter par des professionnels hautement qualifiés la fonction de manœuvres sur machines, il faut supprimer cette fonction, sauf pour le cas de guerre.\par
Il est tellement rare que les nécessités militaires soient en accord et non en contradiction avec les meilleures aspirations humaines qu'il faut en profiter.\par
Du point de vue technique, la facilité relative du transport de l'énergie sous forme d'électricité permet certainement un large degré de décentralisation.\par
Quant aux machines, elles ne sont pas au point pour une transformation du régime de la production; mais les indications qu'on trouve dans les machines automatiques réglables actuellement en usage permettraient sans doute d'aboutir au prix d'un effort, si l'on faisait un effort.\par
D'une manière générale, une réforme d'importance sociale infiniment plus grande que toutes les mesures rangées sous l'étiquette de socialisme serait une transformation dans la conception même des recherches techniques. Jusqu'ici on n'a jamais imaginé qu'un ingénieur occupé à des recherches techniques concernant de nouveaux types de machines puisse avoir autre chose en vue qu'un double objectif : d'une part augmenter les bénéfices de l'entreprise qui lui a commandé ces recherches, d'autre part servir les intérêts des consommateurs. Car en pareil cas, quand on parle des intérêts de la production, il s'agit de produire plus et moins cher ; c'est-à-dire que ces intérêts sont en réalité ceux de la consommation. On emploie sans cesse ces deux mots l'un pour l'autre.\par
Quant aux ouvriers qui donneront leurs forces à cette machine, personne n'y songe. Personne ne songe même qu'il soit possible d'y songer. Tout au plus prévoit-on de temps à autre de vagues appareils de sécurité, bien qu'en fait les doigts coupés et les escaliers d'usines quotidiennement mouillés de sang frais soient si fréquents.\par
Mais cette faible marque d'attention est la seule. Non seulement on ne pense pas au bien-être moral des ouvriers, ce qui exigerait un trop grand effort d'imagination ; mais on ne pense même pas à ne pas meurtrir leur chair. Autrement on aurait peut-être trouvé autre chose pour les mines que cet affreux marteau-piqueur à air comprimé, qui agite de secousses ininterrompues, pendant huit heures, l'homme qui y est accroché.\par
On ne pense pas non plus à se demander si la nouvelle machine, en augmentant l'immobilisation du capital et la rigidité de la production, ne va pas aggraver le danger général de chômage.\par
À quoi sert-il aux ouvriers d'obtenir à force de lutte une augmentation des salaires et un adoucissement de la discipline, si pendant ce temps les ingénieurs de quelques bureaux d'études inventent, sans aucune mauvaise intention, des machines qui épuisent leur corps et leur âme ou aggravent les difficultés économiques ? À quoi leur servirait la nationalisation partielle ou totale de l'économie, si l'esprit de ces bureaux d'études n'a pas changé ? Et jusqu'ici, autant qu'on sache, il n'a pas changé là où il y a eu nationalisation. Même la propagande soviétique n'a jamais prétendu que la Russie ait trouvé un type radicalement nouveau de machine, digne d'être employé par un prolétariat dictateur.\par
Pourtant, s'il y a une certitude qui apparaisse avec une force irrésistible dans les études de Marx, c'est qu'un changement dans le rapport des classes doit demeurer une pure illusion s'il n'est pas accompagné d'une transformation de la technique, transformation cristallisée dans des machines nouvelles.\par
Du point de vue ouvrier, une machine a besoin de posséder trois qualités. D'abord elle doit pouvoir être maniée sans épuiser ni les muscles, ni les nerfs, ni aucun organe – et aussi sans couper ou déchirer la chair, sinon d'une manière très exceptionnelle.\par
En second lieu, relativement au danger général de chômage, l'appareil de production dans son ensemble doit être aussi souple que possible, pour pouvoir suivre les variations de la demande. Par suite une même machine doit être à usages multiples, très variés si possible et même dans une certaine mesure indéterminée. C'est aussi une nécessité militaire, pour la plus grande aisance du passage de l'état de paix à l'état de guerre. Enfin c'est un facteur favorable pour la joie au travail, car on peut ainsi éviter cette monotonie si redoutée des ouvriers pour l'ennui et le dégoût qu'elle engendre.\par
En troisième lieu, elle doit normalement correspondre à un travail de professionnel qualifié. C'est là aussi une nécessité militaire, et de plus c'est indispensable à la dignité, au bien-être moral des ouvriers. Une classe ouvrière formée presque entièrement de bons professionnels n'est pas un prolétariat.\par
Un très grand développement de la machine automatique, réglable, à usages multiples, satisferait dans une large mesure à ces besoins. Les premières réalisations dans ce domaine existent, et il est certain qu'il y a dans cette direction de très grandes possibilités. De telles machines suppriment l'état de manœuvre sur machine. Dans une immense entreprise telle que Renault, peu d'ouvriers ont l'air heureux en travaillant ; parmi ces quelques privilégiés se trouvent ceux qui s'occupent des tours automatiques réglables par des cames.\par
Mais l'essentiel est l'idée même de poser en termes techniques les problèmes concernant les répercussions des machines sur le bien-être moral des ouvriers. Une fois posés, les techniciens n'ont qu'à les résoudre. Ils en ont résolu bien d'autres. Il faut seulement qu'ils le veuillent. Pour cela, il faut que les lieux où on élabore des machines nouvelles ne soient plus plongés entièrement dans le réseau des intérêts capitalistes. Il est naturel que l'État ait prise sur eux par des subventions. Et pourquoi pas les organisations ouvrières par des primes ? Sans compter les autres moyens d'influence et de pression. Si les syndicats ouvriers pouvaient devenir vraiment vivants, il devrait y avoir des contacts perpétuels entre eux et les bureaux d'études où s'ébauchent des techniques nouvelles. On pourrait préparer de tels contacts en établissant une atmosphère favorable aux ouvriers dans les écoles d'ingénieurs.\par
Jusqu'ici les techniciens n'ont jamais eu autre chose en vue que les besoins de la fabrication. S'ils se mettaient à avoir toujours présents à l'esprit les besoins de ceux qui fabriquent, la technique entière de la production devrait être peu à peu transformée.\par
Cela devrait devenir une matière d'enseignement dans les écoles d'ingénieurs et toutes les écoles techniques – mais d'un enseignement qui ait une réelle substance.\par
Il n'y aurait peut-être que des avantages à mettre en train dès à présent des études sur cet ordre de problèmes.\par
Le thème de ces études serait facile à définir. Un pape, a dit : « La matière sort ennoblie de la fabrique, les travailleurs en sortent avilis. » Marx a exprimé exactement la même pensée en termes encore plus vigoureux. Il s'agit que tous ceux qui cherchent à accomplir des progrès techniques aient continuellement fixée dans la pensée la certitude que, parmi toutes les carences de toutes natures qu'il est possible de remarquer dans l'état actuel de la fabrication, celle à laquelle il est de très loin le plus impérieusement urgent de remédier est celle-là ; qu'il ne faut jamais rien faire qui l'aggrave ; qu'il faut tout faire pour la diminuer. Cette pensée devrait désormais faire partie du sentiment de l'obligation professionnelle, du sentiment de l'honneur professionnel, chez quiconque a des responsabilités dans l'industrie. Ce serait une des tâches essentielles des syndicats ouvriers, s'ils étaient capables de s'en acquitter, que de faire pénétrer cette pensée dans la conscience universelle.\par
Si la plus grande partie des ouvriers étaient des professionnels hautement qualifiés, ayant à faire preuve assez souvent d'ingéniosité et d'initiative, responsables de leur production et de leur machine, la discipline actuelle du travail n'aurait plus aucune raison d'être. Certains ouvriers pourraient travailler chez eux, d'autres, dans de petits ateliers qui pourraient souvent être organisés sur le mode coopératif. De nos jours, l'autorité s'exerce dans les petites usines d'une manière plus intolérable encore que dans les grandes, mais c'est qu'elles copient les grandes. De tels ateliers ne seraient pas de petites usines, ce seraient des organismes industriels d'une espèce nouvelle, où pourrait souffler un esprit nouveau ; quoique petits, ils auraient entre eux des liens organiques assez forts pour qu'ils forment ensemble une grande entreprise. Il y a dans la grande entreprise, malgré toutes ses tares, une poésie d'une espèce particulière dont les ouvriers ont aujourd'hui le goût.\par
Le paiement aux pièces n'aurait plus d'inconvénient, une fois aboli l'encasernement des travailleurs. Il n'impliquerait plus l'obsession de la vitesse à tout prix. Il serait le mode normal de rémunération pour un travail librement accompli. L'obéissance ne serait plus une soumission de chaque seconde. Un ouvrier ou un groupe d'ouvriers pourrait avoir un certain nombre de commandes à effectuer dans un délai donné, et disposer d'un libre choix dans l'aménagement du travail. Ce serait autre chose que de savoir qu'on doit répéter indéfiniment le même geste, imposé par un ordre, jusqu'à la seconde précise où un nouveau commandement viendra imposer un nouveau geste pour une durée qu'on ignore, Il y a une certaine relation avec le temps qui convient aux choses inertes, et une autre qui convient aux créatures pensantes. On a tort de les confondre.\par
Coopératifs ou non, ces petits ateliers ne seraient pas des casernes. Un ouvrier pourrait parfois montrer à sa femme le lieu où il travaille, sa machine, comme ils ont été si heureux de le faire en juin 1936, à la faveur de l'occupation. Les enfants viendraient après la classe y retrouver leur père et apprendre à travailler, à l'âge où le travail est de bien loin le plus passionnant des jeux. Plus tard, au moment d'entrer en apprentissage, ils seraient déjà presque en possession d'un métier, et pourraient à leur choix se perfectionner dans celui-là ou en acquérir un second. Le travail serait éclairé de poésie pour toute la vie par ces émerveillements enfantins, au lieu d'être pour toute la vie couleur de cauchemar à cause du choc des premières expériences.\par
Si, même au milieu de la démoralisation actuelle, les paysans ont bien moins besoin que les ouvriers d'être continuellement aiguillonnés par des stimulants, cela tient peut-être à cette différence. Un enfant peut être déjà malheureux aux champs à neuf ou dix ans, mais presque toujours il y a eu un moment où le travail était pour lui un jeu merveilleux, réservé aux grandes personnes.\par
Si les ouvriers devenaient pour la plupart à peu près heureux, plusieurs problèmes en apparence essentiels et angoissants seraient non pas résolus, mais abolis. Sans qu'ils aient été résolus, on oublierait qu'ils se sont jamais posés. Le malheur est un bouillon de culture pour faux problèmes. Il suscite des obsessions. Le moyen de les apaiser n'est pas de fournir ce qu'elles réclament, mais de faire disparaître le malheur. Si un homme a soif à cause d'une blessure au ventre, il ne faut pas le faire boire, mais guérir la blessure.\par
Malheureusement, il n'y a guère que le destin des jeunes qui soit modifiable. Il faudra fournir un grand effort pour la formation de la jeunesse ouvrière, et d'abord pour l'apprentissage. L'État sera obligé d'en prendre la responsabilité, parce qu'aucun autre élément de la société n'en est capable.\par
Rien ne montre mieux la carence essentielle de la classe capitaliste que la négligence des patrons à l'égard de l'apprentissage. Elle est de l'espèce qu'on nomme en Russie négligence criminelle. On ne saurait trop insister là-dessus, trop répandre dans le public cette vérité simple, facile à saisir, incontestable. Les patrons, depuis vingt ou trente ans, ont oublié de songer à la formation de bons professionnels. Le manque d'ouvriers qualifiés a contribué autant que tout autre facteur à la perte du pays. Même en 1934 et 1935, au point le plus aigu de la crise de chômage, quand la production était au point mort, des usines de mécanique et d'aviation cherchaient de bons professionnels et n'en trouvaient pas. Les ouvriers se plaignaient que les essais étaient trop difficiles ; mais c'étaient eux qui n'avaient pas été formés de manière à pouvoir exécuter les essais. Comment, dans ces conditions, aurions-nous pu avoir un armement suffisant ? Mais d'ailleurs, même sans la guerre, le manque de professionnels, en s'aggravant avec les années, devait finir par rendre impossible la vie économique elle-même.\par
Il faut faire savoir une fois pour toutes à tout le pays et aux intéressés eux-mêmes que les patrons se sont montrés en fait incapables de soutenir les responsabilités que le système capitaliste fait peser sur eux. Ils ont une fonction à remplir, mais non celle-là, parce que l'expérience fait voir que celle-là. est trop lourde et trop vaste pour eux. Une fois cela bien entendu, on n'aura plus peur d'eux, et eux de leur côté cesseront de s'opposer aux réformes nécessaires ; ils resteront dans les limites modestes de leur fonction naturelle. C'est leur seule chance de salut ; c'est parce qu'on a peur d'eux qu'on pense si souvent à se débarrasser d'eux.\par
Ils accusaient d'imprévoyance un ouvrier qui prenait un apéritif, mais leur sagesse à eux n'allait pas jusqu'à prévoir que, si l'on ne forme pas d'apprentis, au bout de vingt ans on n'a plus d'ouvriers, du moins qui méritent ce nom. Apparemment ils sont incapables de penser plus de deux ou trois ans à l'avance. Sans doute aussi une secrète inclination leur faisait préférer avoir dans leurs usines un bétail de malheureux, d'êtres déracinés et sans aucun titre à aucune considération. Ils ne savaient pas que, si la soumission des esclaves est plus grande que celle des hommes libres, leur révolte est aussi bien plus terrible. Ils en ont fait l'expérience, mais sans la comprendre.\par
La carence des syndicats ouvriers à l'égard du problème de l'apprentissage est tout aussi scandaleuse d'un autre point de vue. Eux n'avaient pas à se préoccuper de l'avenir de la production ; mais, ayant pour unique raison d'être la défense de la justice, ils auraient dû être touchés par la détresse morale des petits gars. En fait, la partie vraiment misérable de la population des usines, les adolescents, les femmes, les ouvriers immigrés, étrangers ou coloniaux, était abandonnée. La somme entière de leur douleur comptait beaucoup moins dans la vie syndicale que le problème d'une augmentation de salaire pour des catégories déjà largement payées.\par
Rien ne montre mieux combien il est difficile qu'un mouvement collectif soit réellement orienté vers la justice, et que les malheureux soient réellement défendus. Ils ne peuvent pas se défendre eux-mêmes, parce que le malheur les en empêche ; et on ne les défend pas de l'extérieur, parce que le penchant de la nature humaine est de ne pas faire attention aux malheureux.\par
La J.O.C. seule s'est occupée du malheur de l'adolescence ouvrière ; l'existence d'une telle organisation est peut-être le seul signe certain que le christianisme n'est pas mort parmi nous.\par
Comme les capitalistes ont trahi leur vocation en négligeant criminellement non seulement les intérêts du peuple, non seulement ceux de la nation, mais même le leur propre, de même les syndicats ouvriers ont trahi la leur en négligeant la protection des misérables dans les rangs ouvriers pour se tourner vers la défense des intérêts. Cela aussi est bon à faire connaître, en vue du jour où ils pourraient avoir la responsabilité et la tentation de commettre des abus de pouvoir. La mise au pas des syndicats transformés en organisations uniques et obligatoires était l'aboutissement naturel, inévitable de ce changement d'esprit. Au fond, l'action du gouvernement de Vichy à cet égard a été presque nulle. La C. G. T. n'a pas été victime d'un viol de sa part. Il y a longtemps qu'elle n'était plus en état de l'être.\par
L'État n'est pas particulièrement qualifié pour prendre la défense des malheureux. Il en est même à peu près incapable, s'il n'y est pas contraint par une nécessité de salut public urgente, évidente, et par une poussée de l'opinion.\par
En ce qui concerne la formation de la jeunesse ouvrière, la nécessité de salut public est aussi urgente et évidente que possible. Quant à la poussée de l'opinion, il faut la susciter, et commencer dès maintenant, en se servant des embryons d'organismes syndicaux authentiques, de la J.O.C., des groupes d'études et des mouvements de jeunesse, même officiels.\par
Les bolcheviks russes ont passionné leur peuple en lui proposant la construction d'une grande industrie. Ne pourrions-nous passionner le nôtre en lui proposant la construction d'une population ouvrière d'un type nouveau ? Un tel objet serait en accord avec le génie de la France.\par
La formation d'une jeunesse ouvrière doit dépasser la formation purement professionnelle. Elle doit, bien entendu, comporter une éducation, comme la formation de toute jeunesse ; et pour cela il est désirable que l'apprentissage ne se fasse pas dans les écoles, où il se fait toujours mal, mais soit baigné tout de suite dans la production elle-même. Pourtant on ne peut pas non plus le confier aux usines. Il y a là des efforts d'invention à faire. Il faudrait quelque chose qui combine les avantages de l'école professionnelle, ceux de l'apprentissage en usine, ceux du chantier de compagnons du type actuel, et beaucoup d'autres en plus.\par
Mais la formation d'une jeunesse ouvrière, surtout dans un pays comme la France, implique aussi une instruction, une participation à une culture intellectuelle. Il faut qu'ils se sentent chez eux dans le monde de la pensée.\par
\par
Quelle participation, et quelle culture ? C'est un débat qui dure depuis longtemps. Dans certains milieux, autrefois, on parlait beaucoup de culture ouvrière. D'autres disaient qu'il n'y a pas de culture ouvrière ou non ouvrière, mais la culture tout court. Cette observation a eu pour effet, somme toute, de faire accorder aux ouvriers les plus intelligents et les plus avides d'apprendre le traitement qu'on accorde aux lycéens à demi idiots. Les choses ont pu parfois se passer un peu mieux, mais globalement c'est bien là le principe de la vulgarisation, telle qu'on la comprend à notre époque. Le mot est aussi affreux que la chose. Quand on aura quelque chose d'à peu près satisfaisant à désigner, il faudra trouver un autre mot.\par
Certes, la vérité est une, mais l'erreur est multiple ; et dans toute culture, sauf le cas de perfection, qui pour l'homme n'est qu'un cas limite, il y a mélange de vérité et d'erreur. Si notre culture était proche de la perfection, elle serait située au-dessus des classes sociales. Mais comme elle est médiocre, elle est dans une large mesure une culture d'intellectuels bourgeois, et plus particulièrement, depuis quelque temps, une culture d'intellectuels fonctionnaires.\par
Si l'on voulait pousser l'analyse dans ce sens, on trouverait qu'il y a dans certaines idées de Marx beaucoup plus de vérité qu'il n'apparaît d'abord ; mais ce ne sont pas les marxistes qui feront jamais une telle analyse ; car il leur faudrait d'abord se regarder dans un miroir, et c'est là une opération trop pénible, pour laquelle les vertus spécifiquement chrétiennes fournissent seules un courage suffisant.\par
Ce qui rend notre culture si difficile à communiquer au peuple, ce n'est pas qu'elle soit trop haute, c'est qu'elle est trop basse. On prend un singulier remède en l'abaissant encore davantage avant de la lui débiter par morceaux.\par
Il y a deux obstacles qui rendent difficile l'accès du peuple à la culture. L'un est le manque de temps et de forces. Le peuple a peu de loisir à consacrer à un effort intellectuel ; et la fatigue met une limite à l'intensité de l'effort.\par
Cet obstacle-là n'a aucune importance. Du moins il n'en aurait aucune, si l'on ne commettait pas l'erreur de lui en attribuer. La vérité illumine l'âme à proportion de sa pureté et non pas d'aucune espèce de quantité. Ce n'est pas la quantité du métal qui importe, mais le degré de l'alliage. En ce domaine, un peu d'or pur vaut beaucoup d'or pur. Un peu de vérité pure vaut autant que beaucoup de vérité pure. De même, une statue grecque parfaite contient autant de beauté que deux statues grecques parfaites.\par
Le péché de Niobé a consisté à ignorer que la quantité est sans rapport avec le bien, et elle en a été punie par la mort de ses enfants. Nous commettons le même péché tous les jours, et nous en sommes punis de la même manière.\par
Si un ouvrier, en une année d'efforts avides et persévérants, apprend quelques théorèmes de géométrie, il lui sera entré dans l'âme autant de vérité qu'à un étudiant qui, pendant le même temps, aura mis la même ferveur à assimiler une partie de la mathématique supérieure.\par
Il est vrai que cela n'est guère croyable, et ne serait peut-être pas facile à démontrer. Du moins ce devrait être un article de foi pour les chrétiens, s'ils se souvenaient que la vérité est au nombre des biens purs que l'Évangile compare à du pain, et que qui demande du pain ne reçoit pas des pierres.\par
Les obstacles matériels – manque de loisir, fatigue, manque de talent naturel, maladie, douleur physique – gênent pour l'acquisition des éléments inférieurs ou moyens de la culture, non pour celle des biens les plus précieux qu'elle enferme.\par
Le second obstacle à la culture ouvrière est qu'à la condition ouvrière, comme à toute autre, correspond une disposition particulière de la sensibilité. Par suite, il y a quelque chose d'étranger dans ce qui a été élaboré par d'autres et pour d'autres.\par
Le remède à cela, c'est un effort de traduction. Non pas de vulgarisation, mais de traduction, ce qui est bien différent.\par
Non pas prendre les vérités, déjà bien trop pauvres, contenues dans la culture des intellectuels, pour les dégrader, les mutiler, les vider de leur saveur ; mais simplement les exprimer, dans leur plénitude, au moyen d'un langage qui, selon le mot de Pascal, les rende sensibles au cœur, pour des gens dont la sensibilité se trouve modelée par la condition ouvrière.\par
L'art de transposer les vérités est un des plus essentiels et des moins connus. Ce qui le rend difficile, c'est que, pour le pratiquer, il faut s'être placé au centre d'une vérité, l'avoir possédée dans sa nudité, derrière la forme particulière sous laquelle elle se trouve par hasard exposée.\par
Au reste, la transposition est un critérium pour une vérité. Ce qui ne peut pas être transposé n'est pas une vérité ; de même que ce qui ne change pas d'apparence selon le point de vue n'est pas un objet solide, mais un trompe-l'œil. Dans la pensée aussi il y a un espace à trois dimensions.\par
La recherche des modes de transposition convenables pour transmettre la culture au peuple serait bien plus salutaire encore pour la culture que pour le peuple. Ce serait pour elle un stimulant infiniment précieux. Elle sortirait ainsi de l'atmosphère irrespirablement confinée où elle est enfermée. Elle cesserait d'être une chose de spécialistes. Car elle est actuellement une chose de spécialistes, du haut en bas, seulement dégradée à mesure qu'on va vers le bas. De même qu'on traite les ouvriers comme s'il s'agissait de lycéens un peu idiots, on traite les lycéens comme si c'étaient des étudiants considérablement fatigués, et les étudiants comme des professeurs qui auraient souffert d'amnésie et auraient besoin d'être rééduqués. La culture est un instrument manié par des professeurs pour fabriquer des professeurs qui à leur tour fabriqueront des professeurs.\par
Parmi toutes les formes actuelles de la maladie du déracinement, le déracinement de la culture n'est pas le moins alarmant. La première conséquence de cette maladie est généralement, dans tous les domaines, que les relations étant coupées chaque chose est regardée comme un but en soi. Le déracinement engendre l'idolâtrie.\par
Pour prendre un seul exemple de la déformation de notre culture, le souci, absolument légitime, de conserver aux raisonnements géométriques leur caractère de nécessité, fait qu'on présente la géométrie aux lycéens comme une chose absolument sans relation avec le monde. Ils ne peuvent guère s'y intéresser que comme à un jeu, ou bien pour avoir de bonnes notes. Comment y verraient-ils de la vérité ?\par
La plupart ignoreront toujours que presque toutes nos actions, simples ou savamment combinées, sont des applications de notions géométriques, que l'univers où nous vivons est un tissu de relations géométriques, et que la nécessité géométrique est celle même à laquelle nous sommes soumis en fait, comme créatures enfermées dans l'espace et le temps. On présente la nécessité géométrique de telle manière qu'elle paraît arbitraire. Quoi de plus absurde qu'une nécessité arbitraire ? Par définition, une nécessité s'impose.\par
D'un autre côté, quand on veut vulgariser la géométrie et l'approcher de l'expérience, on omet les démonstrations. Il ne reste plus alors que quelques recettes tout à fait sans intérêt. La géométrie a perdu sa saveur, son essence. Son essence est d'être une étude qui a pour objet la nécessité, cette même nécessité qui, en fait, est souveraine ici-bas.\par
L'une et l'autre de ces déformations seraient faciles à éviter. Il n'y a pas à choisir entre la démonstration et l'expérience. On démontre avec du bois ou du fer aussi facilement qu'avec de la craie.\par
Il y aurait une manière simple d'introduire la nécessité géométrique dans une école professionnelle, en associant l'étude et l'atelier. On dirait aux enfants : « Voici un certain nombre de tâches à exécuter (fabriquer des objets satisfaisant à telles, telles et telles conditions). Les unes sont possibles, les autres, impossibles. Exécutez celles qui sont possibles, et celles que vous n'exécutez pas, forcez-moi à admettre qu'elles sont impossibles. » Par cette fente, toute la géométrie peut s'introduire dans le travail. L'exécution est une preuve empirique suffisante de la possibilité, mais pour l'impossibilité il n'y a pas de preuve empirique ; il y faut une démonstration. L'impossibilité est la forme concrète de la nécessité.\par
Quant au reste de la science, tout ce qui appartient à la science classique – et on ne peut pas intégrer dans la culture ouvrière Einstein et les quanta – procède principalement d'une méthode analogique, consistant à transporter dans la nature les relations qui dominent le travail humain. Par conséquent, cela appartient aux travailleurs, si on sait le leur présenter, bien plus naturellement qu'aux lycéens.\par
À plus forte raison en est-il ainsi de la partie de la culture rangée sous la rubrique des « Lettres ». Car l'objet en est toujours la condition humaine, et c'est le peuple qui a l'expérience la plus réelle, la plus directe de la condition humaine.\par
Dans l'ensemble, sauf exceptions, les œuvres de deuxième ordre et au-dessous conviennent mieux à l'élite, et les œuvres de tout premier ordre conviennent mieux au peuple.\par
Par exemple, quelle intensité de compréhension pourrait naître d'un contact entre le peuple et la poésie grecque, qui a pour objet presque unique le malheur ! Seulement il faudrait savoir la traduire et la présenter. Par exemple, un ouvrier, qui a l'angoisse du chômage enfoncée jusque dans la moelle des os, comprendrait l'état de Philoctète quand on lui enlève son arc, et le désespoir avec lequel il regarde ses mains impuissantes. Il comprendrait aussi qu'Électre a faim, ce qu'un bourgeois, excepté dans la période présente, est absolument incapable de comprendre – y compris les éditeurs de l'édition Budé.\par
Il y a un troisième obstacle à la culture ouvrière c'est l’esclavage. La pensée est par essence libre et souveraine, quand elle s'exerce réellement. Être libre et souverain, en qualité d'être pensant, pendant une heure ou deux, et esclave le reste du jour, est un écartèlement tellement déchirant qu'il est presque impossible de ne pas renoncer, pour s'y soustraire, aux formes les plus hautes de la pensée.\par
Si des réformes efficaces étaient accomplies, cet obstacle disparaîtrait peu à peu. Bien plus, le souvenir de l'esclavage récent et les restes d'esclavage en train de disparaître seraient un stimulant puissant pour la pensée pendant le cours de la libération.\par
Une culture ouvrière a pour condition un mélange de ceux qu'on nomme les intellectuels – nom affreux, mais aujourd'hui ils n'en méritent pas un plus beau – avec les travailleurs. Il est difficile qu'un tel mélange soit réel. Mais la situation actuelle y est favorable. Quantité de jeunes intellectuels ont été précipités dans l'esclavage, dans les usines et les champs d'Allemagne. D'autres se sont mélangés aux jeunes ouvriers dans les camps de compagnons. Mais les premiers surtout ont eu une expérience qui compte. Beaucoup auront été détruits par elle, ou du moins trop affaiblis d'âme et de corps. Mais quelques-uns peut-être auront été vraiment instruits.\par
Cette expérience si précieuse risque de se perdre à cause de la tentation presque irrésistible d'oublier l'humiliation et le malheur dès qu'on en sort. Dès maintenant, il faudrait approcher ceux de ces prisonniers qui sont revenus, les engager à continuer les contacts avec les travailleurs qu'ils avaient commencés sous la contrainte, à repenser pour eux leur expérience récente, en vue d'un rapprochement de la culture et du peuple, en vue d'une orientation nouvelle de la culture.\par
Les organisations syndicales de résistance pourraient en ce moment être l'occasion de tels rapprochements. Mais d'une manière générale, s'il doit y avoir une vie de la pensée dans les syndicats ouvriers, ils devront avoir avec les intellectuels d'autres contacts que ceux consistant à les grouper dans la C. G. T. en organisations professionnelles pour la défense de leurs propres gros sous. C'était de la dernière absurdité.\par
La relation naturelle serait qu'un syndicat admette comme membres d'honneur, mais avec défense d'intervenir dans des délibérations sur l'action, des intellectuels qui se mettraient gratuitement à son service pour l'organisation de cours et de bibliothèques.\par
Il serait hautement désirable que dans la génération qui, par sa jeunesse, a échappé au mélange avec les travailleurs dans la contrainte de la captivité, il surgisse un courant analogue à celui qui a agité les étudiants russes il y a cinquante ans, mais avec des pensées plus claires, et que des étudiants aillent faire des stages volontaires et prolongés, comme ouvriers anonymement mélangés à la masse, dans les champs et les usines.\par
En résumé, la suppression de la condition prolétarienne, qui est définie avant tout par le déracinement, se ramène à la tâche de constituer une production industrielle et une culture de l'esprit où les ouvriers soient et se sentent chez eux.\par
Bien entendu, les ouvriers eux-mêmes auraient une large part dans une telle construction. Mais par la nature des choses cette part irait en croissant à mesure que s'accomplirait leur libération réelle. Elle est inévitablement au minimum tant que les ouvriers sont dans l'emprise du malheur.\par
Ce problème de la construction d'une condition ouvrière réellement nouvelle est urgent et doit être examiné sans retard. Une orientation doit être décidée dès maintenant. Car aussitôt la guerre finie, on construira, au sens littéral du mot. On construira des maisons et des bâtiments. Ce qu'on aura construit ne sera plus démoli, à moins d'une nouvelle guerre, et la vie s'y adaptera. Il serait paradoxal qu'on laissât s'assembler au hasard les pierres qui doivent décider, peut-être pour beaucoup de générations, de toute la vie sociale. Par conséquent, il faudra avoir d'avance une pensée nette concernant l'organisation des entreprises industrielles dans l'avenir prochain.\par
Si par hasard on se dérobait à cette nécessité par crainte des divisions possibles, cela signifierait simplement que nous ne sommes pas qualifiés pour intervenir dans les destinées de la France.\par
Il est donc urgent d'examiner un plan de réenracinement ouvrier, dont voici, en résumé, une esquisse possible.\par
Les grandes usines seraient abolies. Une grande entreprise serait constituée par un atelier de montage relié à un grand nombre de petits ateliers, d'un ou de quelques ouvriers chacun, dispersés à travers la campagne. Ce seraient ces ouvriers, et non des spécialistes, qui iraient à tour de rôle, par périodes, travailler à l'atelier central de montage, et ces périodes devraient constituer des fêtes. Le travail n'y serait que d'une demi-journée, le reste devant être consacré aux liens de camaraderie, à l'épanouissement d'un patriotisme d'entreprise, à des conférences techniques pour faire saisir à chaque ouvrier la fonction exacte des pièces qu'il produit et les difficultés surmontées par le travail des autres, à des conférences géographiques pour leur apprendre où vont les produits qu'ils aident à fabriquer, quels êtres humains en font usage, dans quelle espèce de milieu, de vie quotidienne, d'atmosphère humaine ces produits tiennent une place, et quelle place. À cela s'ajouterait de la culture générale. Une université ouvrière serait voisine de chaque atelier central de montage. Elle aurait des liens étroits avec la direction de l'entreprise, mais n'en serait pas la propriété.\par
Les machines n'appartiendraient pas à l'entreprise. Elles appartiendraient aux minuscules ateliers dispersés partout, et ceux-ci à leur tour seraient soit individuellement, soit collectivement la propriété des ouvriers. Chaque ouvrier posséderait en plus une maison et un peu de terre.\par
Cette triple propriété – machine, maison et terre – lui serait conférée par un don de l'État, au moment du mariage, et à la condition qu'il ait accompli avec succès un essai technique difficile, accompagné d'une épreuve pour contrôler l'intelligence et la culture générale.\par
Le choix de la machine devrait répondre, d'une part aux goûts et connaissances de l'ouvrier, d'autre part aux besoins très généraux de la production. Ce doit être évidemment, le plus possible, une machine automatique réglable et à usages multiples.\par
Cette triple propriété ne pourrait être ni transmise par héritage, ni vendue, ni aliénée d'aucune manière. (La machine, seule pourrait dans certains cas être échangée.) Celui qui en jouit n'aurait que la faculté d'y renoncer purement et simplement. En ce cas, il devrait lui être rendu non pas impossible, mais difficile d'en recevoir plus tard ailleurs l'équivalent.\par
Quand un ouvrier meurt, cette propriété retourne à l'État, qui, bien entendu, le cas échéant, doit assurer un bien-être égal à la femme et aux enfants. Si la femme est capable d'exécuter le travail, elle conserve la propriété.\par
Tous ces dons sont financés par des impôts, soit directs, sur les profits des entreprises, soit indirects, sur la vente des produits. Ils sont gérés par une administration où se trouvent des fonctionnaires, des patrons d'entreprises, des syndicalistes, des députés.\par
Ce droit de propriété peut être retiré pour incapacité professionnelle par la sentence d'un tribunal. Ceci, bien entendu, suppose que des mesures pénales analogues soient prévues pour punir, le cas échéant, l'incapacité professionnelle d'un patron d'entreprise.\par
Un ouvrier qui désirerait devenir patron d'un petit atelier devrait en obtenir l'autorisation d'un organisme professionnel chargé de l'accorder avec discernement, et aurait alors des facilités pour l'achat de deux ou trois autres machines ; non davantage.\par
Un ouvrier incapable de passer l'essai resterait dans la condition de salarié. Mais il pourrait toute sa vie, sans limite d'âge, faire de nouvelles tentatives. Il pourrait à tout âge et à plusieurs reprises demander à faire un stage gratuit de quelques mois dans une école professionnelle.\par
Ces salaries par incapacité travailleraient soit dans les petits ateliers non coopératifs, soit comme aides d'un ouvrier travaillant chez lui, soit comme manœuvres dans les ateliers de montage. Mais ils ne devraient être tolérés dans l'industrie qu'en petit nombre. La plupart devraient être poussés dans les besognes de manœuvres ou de gratte-papier indispensables aux services publics et au commerce.\par
Jusqu'à l'âge de se marier et de s'établir chez lui pour la vie – c'est-à-dire, selon les caractères, jusqu'à vingt-deux, vingt-cinq, trente ans –, un jeune ouvrier serait regardé comme étant toujours en apprentissage.\par
Dans l'enfance, l'école devrait laisser aux enfants assez de loisir pour qu'ils puissent passer des heures et des heures à bricoler autour du travail de leur père. La demi-scolarité – quelques heures d'études, quelques heures de travail – devrait ensuite se prolonger longtemps. Ensuite il faudrait un mode de vie très varié – voyages du mode « Tour de France », séjour et travail tantôt chez des ouvriers travaillant individuellement, tantôt dans de petits ateliers, tantôt dans des ateliers de montage de différentes entreprises, tantôt dans des groupements de jeunesse du genre « Chantiers » ou « Compagnons » ; séjours qui, selon les goûts et les capacités, pourraient se répéter à plusieurs reprises et se prolonger durant des périodes variant de quelques semaines à deux ans, dans des collèges ouvriers. Ces séjours devraient d'ailleurs, à certaines conditions, être possibles à tout âge. Ils devraient être entièrement gratuits, et n'entraîner aucune espèce d'avantage social.\par
Quand le jeune ouvrier, rassasié et gorgé de variété, songerait à se fixer, il serait mûr pour l'enracinement. Une femme, des enfants, une maison, un jardin lui fournissant une grande partie de sa nourriture, un travail le liant à une entreprise qu'il aimerait, dont il serait fier, et qui serait pour lui une fenêtre ouverte sur le monde, c'est assez pour le bonheur terrestre d'un être humain.\par
Bien entendu, une telle conception de la jeunesse ouvrière implique une refonte totale de la vie de caserne.\par
Pour les salaires, il faudrait surtout éviter, d'abord, bien entendu, qu'ils soient bas au point de jeter dans la misère – mais ce ne serait guère à craindre dans de pareilles conditions – puis qu'ils occupent l'esprit et empêchent l'attachement de l'ouvrier à l'entreprise.\par
Les organismes corporatifs, d'arbitrage, etc., devraient être conçus uniquement à cet effet – fonctionner de manière que chaque ouvrier pense rarement aux questions d'argent.\par
La profession de chef d'entreprise devrait, comme celle de médecin, être au nombre de celles que l'État, dans l'intérêt public, autorise à exercer seulement sous la condition de certaines garanties. Les garanties devraient avoir rapport non seulement à la capacité, mais à l'élévation morale.\par
Les capitaux engagés seraient bien plus réduits que maintenant. Un système de crédit pourrait facilement permettre à un jeune homme pauvre qui a la capacité et la vocation d'être chef d'entreprise de le devenir.\par
L'entreprise pourrait ainsi redevenir individuelle. Quant aux sociétés anonymes, il n'y aurait peut-être pas d'inconvénient, en ménageant un système de transition, à les abolir et à les déclarer interdites.\par
Bien entendu, la variété des entreprises exigerait l'étude de modalités très variées. Le plan esquissé ici ne peut apparaître que comme terme de longs efforts au nombre desquels des efforts d'invention technique seraient indispensables.\par
En tout cas, un tel mode de vie sociale ne serait ni capitaliste ni socialiste.\par
Il abolirait la condition prolétarienne, au lieu que ce qu'on nomme socialisme a tendance, en fait, à y précipiter tous les hommes.\par
Il aurait pour orientation, non pas, selon la formule qui tend aujourd'hui à devenir à la mode, l'intérêt du consommateur – cet intérêt ne peut être que grossièrement matériel – mais la dignité de l'homme dans le travail, ce qui est une valeur spirituelle.\par
L'inconvénient d'une telle conception sociale, c'est qu'elle n'a aucune chance de sortir du domaine des mots sans un certain nombre d'hommes libres qui auraient au fond du cœur une volonté brûlante et inébranlable de l'en faire sortir. Il n'est pas sûr qu'ils puissent être trouvés ou suscités.\par
Pourtant, hors de là, il semble bien qu'il n'y a de choix qu'entre des formes différentes et presque également atroces de malheur.\par
Bien qu'une telle conception soit d'une réalisation longue, la reconstruction d'après-guerre devrait avoir immédiatement pour règle la dispersion du travail industriel.
\subsection[{B. Déracinement paysan}]{B. \\
Déracinement paysan}
\noindent \par
Le problème du déracinement paysan n'est pas moins grave que celui du déracinement ouvrier. Quoique la maladie soit moins avancée, elle a quelque chose d'encore plus scandaleux ; car il est contre nature que la terre soit cultivée par des êtres déracinés. Il faut accorder la même attention aux deux problèmes.\par
Au reste il ne faut jamais donner une marque publique d'attention aux ouvriers sans en donner une autre symétrique aux paysans. Car ils sont très ombrageux, très sensibles, et toujours tourmentés par la pensée qu'on les oublie. Il est certain que parmi les souffrances actuelles ils trouvent un réconfort dans l'assurance qu'on pense à eux. Il faut avouer qu'on pense beaucoup plus à eux quand on a faim que quand on mange à discrétion ; et cela même parmi les gens qui avaient cru placer leur pensée sur un plan très au-dessus de tous les besoins physiques.\par
Les ouvriers ont une tendance qu'il ne faut pas encourager à croire, quand on parle du peuple, qu'il doit s'agir uniquement d'eux. Il n'y a absolument aucun motif légitime pour cela ; à moins de compter comme tel le fait qu'ils font plus de bruit que les paysans. Ils sont arrivés à persuader sur ce point les intellectuels qui ont une inclination vers le peuple. Il en est résulté, chez les paysans, une sorte de haine pour ce qu'on nomme en politique la gauche – excepté là où ils sont tombés sous l'influence communiste, et là où l'anticléricalisme est la passion principale ; et sans doute encore dans quelques autres cas.\par
La division entre paysans et ouvriers, en France, date de loin. Il y a une complainte de la fin du XIV\textsuperscript{e} siècle où les paysans énumèrent, avec un accent déchirant, les cruautés que leur font subir toutes les classes de la société, y compris les artisans.\par
Dans l'histoire des mouvements populaires en France, il n'est guère arrivé, sauf erreur, que paysans et ouvriers se soient trouvés ensemble. Même en 1789, il s'agissait peut-être davantage d'une coïncidence que d'autre chose.\par
Au XIV\textsuperscript{e} siècle, les paysans étaient de très loin les plus malheureux. Mais même quand ils sont matériellement plus heureux – et quand c'est le cas, ils ne s'en rendent guère compte, parce que les ouvriers qui viennent passer au village quelques jours de vacances succombent à la tentation des vantardises – ils sont toujours tourmentés par le sentiment que tout se passe dans les villes, et qu'ils sont « out of it ».\par
Bien entendu, cet état d'esprit est aggravé par l'installation dans les villages de T. S. F., de cinémas, et par la circulation de journaux tels que {\itshape Confidences} et {\itshape Marie-Claire}, auprès desquels la cocaïne est un produit sans danger.\par
La situation étant telle, il faut d'abord inventer et mettre en application quelque chose qui donne désormais aux paysans le sentiment qu'ils sont « in it ».\par
Il est peut-être regrettable que dans les textes émanant officiellement de Londres il ait toujours été fait mention des ouvriers beaucoup plus que d'eux. Il est vrai qu'ils ont une part beaucoup, beaucoup moindre à la résistance. Mais c'est peut-être une raison de plus pour donner des preuves répétées qu'on sait qu'ils existent.\par
Il faut avoir présent à l'esprit qu'on ne peut pas dire que le peuple français soit pour un mouvement quand ce n'est pas vrai de la majorité des paysans.\par
On devrait se faire une règle de ne jamais promettre du nouveau et du mieux aux ouvriers sans en promettre autant aux paysans. La grande habileté du parti nazi avant 1933 a été de se présenter aux ouvriers comme un parti spécifiquement ouvrier, aux paysans comme un parti spécifiquement paysan, aux petits bourgeois comme un parti spécifiquement petit bourgeois, etc. Cela lui était facile, car il mentait à tout le monde. Il faudrait en faire autant, mais sans mentir à personne. C'est difficile, mais ce n'est pas impossible.\par
Le déracinement paysan a été, au cours des dernières années, un danger aussi mortel pour le pays que le déracinement ouvrier. Un des symptômes les plus graves a été, il y a sept ou huit ans, le dépeuplement des campagnes se poursuivant en pleine crise de chômage.\par
\par
Il est évident que le dépeuplement des campagnes, à la limite, aboutit à la mort sociale. On peut dire qu'il n'ira pas jusque-là. Mais on n'en sait rien. Jusqu'ici, on n'aperçoit rien qui soit susceptible de l'arrêter.\par
Au sujet de ce phénomène, deux choses sont à remarquer.\par
L'une est que les Blancs le transportent partout où ils vont. La maladie a gagné même l'Afrique noire, qui pourtant était sans doute depuis des milliers d'années un continent fait de villages. Ces gens-là au moins, quand on ne venait pas les massacrer, les torturer ou les réduire en esclavage, savaient vivre heureux sur leur terre. Notre contact est en train de leur faire perdre cette capacité. Cela pourrait faire douter si même les Noirs d'Afrique, quoique les plus primitifs parmi les colonisés, n'avaient pas somme toute plus à nous apprendre qu'à apprendre de nous. Nos bienfaits envers eux ressemblent à celui du financier envers le savetier. Rien au monde ne compense la perte de la joie au travail.\par
L'autre remarque à faire, c'est que les ressources, illimitées en apparence, de l'État totalitaire, sont impuissantes contre ce mal. Il y a eu à ce sujet, en Allemagne, des aveux officiels, formels, maintes fois répétés. En un sens, tant mieux, puisque cela donne une possibilité de faire mieux qu'eux.\par
La destruction des stocks de blé pendant la crise a beaucoup frappé l'opinion publique, avec raison ; mais si l'on y pense, la désertion des campagnes en période de crise industrielle a quelque chose, si possible, d'encore plus scandaleux. Il est évident qu'il n'y a aucun espoir de résoudre le problème ouvrier à part de celui-là. Il n'y a aucun moyen d'empêcher que la population ouvrière ne soit un prolétariat si elle s'augmente constamment d'un afflux de paysans en état de rupture avec leur vie passée.\par
La guerre a montré quel est le degré de gravité de la maladie chez les paysans. Car les soldats étaient de jeunes paysans. En septembre 1939, on entendait des paysans dire : « Il vaut mieux vivre Allemand que mourir Français. » Que leur avait-on fait pour qu'ils aient cru n'avoir rien à perdre ?\par
Il faut bien prendre conscience d'une des plus grandes difficultés de la politique. Si les ouvriers souffrent cruellement de se sentir en exil dans cette société, les paysans, eux, ont l'impression que dans cette société, au contraire, les ouvriers seuls sont chez eux. Aux yeux des paysans, les intellectuels défenseurs des ouvriers n'apparaissent pas comme des défenseurs d'opprimés, mais comme des défenseurs de privilégiés. Les intellectuels ne soupçonnent pas cet état d'esprit.\par
Le complexe d'infériorité dans les campagnes est tel qu'on voit des paysans millionnaires trouver naturel d'être traités par des petits bourgeois retraités avec une hauteur de coloniaux envers des indigènes. Il faut qu'un complexe d'infériorité soit très fort pour ne pas être effacé par l'argent.\par
Par suite, plus on se propose de fournir de satisfactions morales aux ouvriers, plus il faut se préparer à en procurer aux paysans. Autrement le déséquilibre produit serait dangereux pour la société et par répercussion pour les ouvriers eux-mêmes.\par
Le besoin d'enracinement, chez les paysans, a d'abord la forme de la soif de propriété. C'est vraiment une soif chez eux, et une soif saine et naturelle. On est sûr de les toucher en leur offrant des espérances dans ce sens ; et il n'y a aucune raison de ne pas le faire dès lors qu'on regarde comme sacré le besoin de propriété, et non pas les titres juridiques qui déterminent les modalités de la propriété. Il y a quantité de dispositions légales possibles pour faire passer peu à peu aux mains des paysans les terres qui n'y sont pas. Rien ne peut légitimer un droit de propriété d'un citadin sur une terre. La grande propriété agricole n'est justifiable que dans certains cas, pour des raisons techniques ; et dans ces cas eux-mêmes on peut concevoir des paysans cultivant intensément en légumes et produits de ce genre chacun son bout de terrain, et en même temps appliquant les méthodes de culture extensive, avec un outillage moderne, sur de vastes espaces possédés par eux en commun, sous forme coopérative.\par
Une mesure qui toucherait le cœur des paysans serait celle par laquelle on déciderait de regarder la terre comme un moyen de travail, et non comme une richesse dans la répartition des héritages. Ainsi on ne verrait plus le spectacle scandaleux d'un paysan endetté tout au cours de sa vie envers un frère fonctionnaire qui travaille moins et gagne davantage.\par
Des retraites, même minimes, pour les vieillards auraient peut-être une grande portée. Le mot de retraite est malheureusement un mot magique qui tire les jeunes paysans vers la ville. L'humiliation des vieux est souvent grande à la campagne, et un peu d'argent, accordé avec des formes honorables, leur donnerait du prestige.\par
Par un effet de contraste, une trop grande stabilité produit chez les paysans un effet de déracinement. Un petit paysan commence à labourer seul vers quatorze ans ; le travail est alors une poésie, une ivresse, quoique ses forces y suffisent à peine. Quelques années plus tard, cet enthousiasme enfantin est épuisé, le métier est connu, les forces physiques sont débordantes et dépassent de loin le travail à fournir ; et il n'y a rien d'autre à faire que ce qui a été fait tous les jours pendant plusieurs années. Il se met alors à passer la semaine à rêver de ce qu'il fera le dimanche. Dès ce moment il est perdu.\par
Il faudrait que ce premier contact complet du petit paysan avec le travail, à l'âge de quatorze ans, que cette première ivresse soit consacrée par une fête solennelle qui la fasse pénétrer pour toujours au fond de l'âme. Dans les villages les plus chrétiens, une telle fête devrait avoir un caractère religieux.\par
Mais aussi, trois ou quatre ans plus tard, il faudrait fournir un aliment à la soif de nouveau qui le saisit. Pour un jeune paysan, il n'y en a qu'un, le voyage. Il faudrait donner à tous les jeunes paysans la possibilité de voyager sans dépenses d'argent, en France et même à l'étranger, non pas dans les villes, mais dans les campagnes. Cela impliquerait l'organisation pour les paysans de quelque chose d'analogue au Tour de France. On pourrait y joindre des œuvres d'éducation et d'instruction, Car souvent les meilleurs des jeunes paysans, après avoir mis une sorte de violence, à treize ans, à se détourner de l'école pour se jeter dans le travail, sentent de nouveau vers dix-huit ou vingt ans le goût de s'instruire. Cela arrive d'ailleurs aussi aux jeunes ouvriers. Des systèmes d'échange pourraient permettre de partir même aux jeunes gens indispensables à leurs familles. Il va de soi que ces voyages seraient entièrement volontaires. Mais les parents n'auraient pas le droit de les empêcher.\par
On n'imagine pas la puissance de l'idée de voyage chez les paysans, et l'importance morale qu'une telle réforme pourrait prendre, même avant d'être réalisée, à l'état de promesse, et bien plus une fois la chose entrée dans les mœurs. Le jeune garçon, ayant roulé par le monde plusieurs années sans jamais cesser d'être un paysan, rentrerait chez lui, ses inquiétudes apaisées, et fonderait un foyer.\par
Il faudrait peut-être quelque chose d'analogue pour les jeunes filles ; il leur faut bien quelque chose pour remplacer {\itshape Marie-Claire}, et on ne peut pas leur laisser {\itshape Marie-Claire}.\par
La caserne a été un terrible facteur de déracinement pour les jeunes paysans. C'est à ce point que l'instruction militaire a eu finalement un effet contraire à son but ; les jeunes gens avaient appris l'exercice, mais étaient moins préparés à se battre qu'avant de l'apprendre, car quiconque sortait de la caserne en sortait antimilitariste. C'est la preuve expérimentale qu'on ne peut pas, dans l'intérêt même de la machine militaire, laisser les militaires disposer souverainement de deux années de chaque vie, ou fût-ce même d'une année. Comme on ne peut pas laisser le capitalisme maître de la formation professionnelle de la jeunesse, on ne peut pas laisser l'Armée maîtresse de sa formation militaire. Les autorités civiles doivent y prendre part, et cela de manière à faire qu'elle constitue une éducation et non une corruption.\par
Le contact entre jeunes paysans et jeunes ouvriers au service militaire n'est pas du tout désirable. Les seconds cherchent à épater les premiers et cela fait du mal aux uns et aux autres. De tels contacts ne suscitent pas de vrais rapprochements. Seule l'action commune rapproche ; et par définition, il n'y a pas d'action commune à la caserne, puisqu'on s'y prépare à la guerre en temps de paix.\par
Il n'y a aucune raison d'installer les casernes dans les villes. À l'usage des jeunes paysans, on pourrait très bien établir des casernes loin de toute ville.\par
Il est vrai que les patrons des maisons de tolérance y perdraient. Mais il est inutile de songer à aucune espèce de réforme si l'on n'est pas absolument décidé à mettre fin à la collusion des pouvoirs publics avec ces gens-là, et à abolir une institution qui est une des hontes de la France.\par
Soit dit en passant, nous avons payé cher cette honte. La prostitution établie comme une institution officielle, selon le régime propre à la France, a largement contribué à pourrir l'Armée, et a complètement pourri la police, ce qui devait entraîner la ruine de la démocratie. Car il est impossible qu'une démocratie subsiste, quand la police, qui représente la loi aux yeux des citoyens, est ouvertement l'objet du mépris public. Les Anglais ne peuvent pas comprendre qu'il puisse y avoir une démocratie où la police n'est pas l'objet d'un tendre respect. Mais leur police ne dispose pas d'un bétail de prostituées pour sa distraction.\par
Si l'on pouvait supputer exactement les facteurs de notre désastre, on trouverait peut-être que toutes nos hontes – comme celle-là, et celle des appétits coloniaux, et celle des traitements infligés aux étrangers – ont eu leur contrecoup effectif pour notre perte. On peut dire beaucoup de choses de notre malheur, mais non qu'il soit immérité.\par
La prostitution est un exemple typique de cette propriété de propagation à la deuxième puissance que possède le déracinement. La situation de prostituée professionnelle constitue le degré extrême du déracinement ; et pour cette maladie du déracinement, une poignée de prostituées possède un vaste pouvoir de contamination. Il est évident qu'on n'aura pas une paysannerie saine tant que l'État s'obstinera à opérer lui-même le rapprochement des jeunes paysans et des prostituées. Tant que la paysannerie n'est pas saine, la classe ouvrière ne peut pas l'être non plus, ni le reste du pays.\par
Au reste, rien ne serait plus populaire auprès des paysans que le projet de réformer le régime du service militaire avec la préoccupation de leur bien-être moral.\par
Le problème de la culture de l'esprit se pose pour les paysans comme pour les ouvriers. À eux aussi il faut une traduction qui leur soit propre ; elle ne doit pas être celle des ouvriers.\par
Pour tout ce qui concerne les choses de l'esprit, les paysans ont été brutalement déracinés par le monde moderne. Ils avaient auparavant tout ce dont un être humain a besoin comme art et comme pensée, sous une forme qui leur était propre, et de la meilleure qualité. Quand on lit tout ce qu'a écrit Restif de la Bretonne sur son enfance, on doit conclure que les plus malheureux des paysans d'alors avaient un sort infiniment préférable à celui des plus heureux des paysans d'aujourd'hui. Mais on ne peut pas retrouver ce passé, quoique si proche. Il faut inventer des méthodes pour empêcher que les paysans restent étrangers à la culture d'esprit qui leur est offerte.\par
La science doit être présentée aux paysans et aux ouvriers de manières très différentes. Pour les ouvriers, il est naturel que tout soit dominé par la mécanique. Pour les paysans, tout devrait avoir pour centre le merveilleux circuit par lequel l'énergie solaire, descendue dans les plantes, fixée par la chlorophylle, concentrée dans les graines et les fruits, entre dans l'homme qui mange ou boit, passe dans ses muscles et se dépense pour l'aménagement de la terre. Tout ce qui se rapporte à la science peut être disposé autour de ce circuit, car la notion d'énergie est au centre de tout. La pensée de ce circuit, si elle pénétrait dans l'esprit des paysans, envelopperait le travail de poésie.\par
D'une manière générale, toute instruction, dans les villages, devrait avoir pour objet essentiel d'augmenter la sensibilité à la beauté du monde, à la beauté de la nature. Les touristes, il est vrai, ont découvert que les paysans ne s'intéressent pas aux paysages. Mais quand on partage avec des paysans des journées de travail épuisantes, ce qui est le seul procédé pour causer à cœur ouvert avec eux, on entend certains regretter que leur travail soit trop dur pour les laisser jouir des beautés de la nature.\par
Bien entendu, augmenter la sensibilité à la beauté ne s'accomplit pas en disant : « Regardez comme c'est beau ! » C'est moins facile.\par
Le mouvement qui s'est produit récemment dans les milieux cultivés vers le folklore devrait aider à restituer aux paysans le sentiment qu'ils sont chez eux dans la pensée humaine. Le système actuel consiste à leur présenter tout ce qui a rapport à la pensée comme une propriété exclusive des villes, dont on veut bien leur accorder une petite part, très petite, parce qu'ils n'ont pas la capacité d'en concevoir une grande.\par
C'est la mentalité coloniale, à un degré seulement moins aigu. Et comme il arrive qu'un indigène des colonies, frotté d'un peu d'instruction européenne, méprise son peuple plus que ne ferait un Européen cultivé, il en est souvent de même pour un instituteur fils de paysans.\par
La première condition d'un réenracinement moral de la paysannerie dans le pays, c'est que le métier d'instituteur rural soit quelque chose de distinct, de spécifique, dont la formation soit non seulement partiellement, mais totalement autre que celle d'un instituteur des villes. Il est absurde au plus haut point de fabriquer dans un même moule des instituteurs pour Belleville ou pour un petit village. C'est une des nombreuses absurdités d'une époque dont le caractère dominant est la bêtise.\par
La deuxième condition est que les instituteurs ruraux connaissent les paysans et ne les méprisent pas, ce qu'on n'obtiendra pas simplement en les recrutant dans la paysannerie. Il faudrait donner une très large part, dans l'enseignement qu'on leur fournit, au folklore de tous les pays, présenté non comme un objet de curiosité, mais comme une grande chose ; leur parler de la part qu'ont eue les bergers dans les premières spéculations de la pensée humaine, celles sur les astres, et aussi, comme le montrent les comparaisons qui reviennent partout dans les textes antiques, celles sur le bien et le mal ; leur faire lire la littérature paysanne, Hésiode, {\itshape Pier the Ploughman}, les complaintes du Moyen Âge, les quelques ouvrages contemporains qui sont authentiquement paysans; tout cela, bien entendu, sans préjudice de la culture générale. Après une telle préparation, on pourrait les envoyer servir un an comme domestiques de ferme, anonymement, dans un autre département ; puis les rassembler de nouveau dans les écoles normales pour les aider à y voir clair dans leur propre expérience. De même pour les instituteurs des quartiers ouvriers et les usines. Seulement de telles expériences doivent être moralement préparées ; autrement elles suscitent le mépris ou la répulsion au lieu de la compassion et de l'amour.\par
Il serait bien avantageux aussi que les Églises fassent de la condition de curé ou pasteur de village quelque chose de spécifique. C'est un scandale de voir combien, dans un village français entièrement catholique, la religion peut être absente de la vie quotidienne, réservée à quelques heures du dimanche, quand on songe avec quelle prédilection le Christ a emprunté le thème de ses paraboles à la vie des champs. Mais un grand nombre de ces paraboles ne figurent pas dans la liturgie, et celles qui y figurent ne provoquent aucune attention. De même que les étoiles et le soleil dont parle l'instituteur habitent dans les cahiers et les livres et n'ont aucun rapport avec le ciel, de même la vigne, le blé, les brebis dont il est fait mention le dimanche dans l'église n'ont rien de commun avec la vigne, le blé, les brebis qui se trouvent dans les champs et auxquels on donne tous les jours un peu de sa vie. Les paysans chrétiens sont déracinés aussi dans leur vie religieuse. L'idée de représenter un village sans église dans l’Exposition de 1937 n'était pas aussi absurde que beaucoup l'ont dit.\par
Comme les petits Jocistes s'exaltent à la pensée du Christ ouvrier, les paysans devraient puiser la même fierté dans la part qu'accordent les paraboles de l'Évangile à la vie des champs et dans la fonction sacrée du pain et du vin, et en tirer le sentiment que le christianisme est une chose à eux.\par
Les polémiques autour de la laïcité ont été une des principales sources d'empoisonnement de la vie paysanne en France. Malheureusement elles ne sont pas près de finir. Il est impossible d'éviter de prendre position sur ce problème, et il semble d'abord presque impossible d'en trouver une qui ne soit pas très mauvaise.\par
Il est certain que la neutralité est un mensonge. Le système laïque n'est pas neutre, il communique aux enfants une philosophie qui est d'une part très supérieure à la religion genre Saint-Sulpice, d'autre part très inférieure au christianisme authentique. Mais celui-ci, aujourd'hui, est très rare. Beaucoup d'instituteurs portent à cette philosophie un attachement d'une ferveur religieuse.\par
La liberté de l'enseignement n'est pas une solution. Le mot est vide de sens. La formation spirituelle d'un enfant n'appartient à personne ; ni à l'enfant, puisqu'il n'est pas en mesure d'en disposer ; ni aux parents ; ni à l'État. Le droit des familles invoqué si souvent n'est qu'une machine de guerre. Un prêtre qui, ayant une occasion naturelle de le faire, s'abstiendrait de parler du Christ à un enfant de famille non chrétienne serait un prêtre qui n'aurait guère de foi. Maintenir l'école laïque, telle quelle et permettre ou même favoriser, à côté, la concurrence de l'école confessionnelle est une absurdité du point de vue théorique et du point de vue pratique. Les écoles privées, confessionnelles ou non, doivent être autorisées, non pas en vertu d'un principe de liberté, mais pour un motif d'utilité publique dans chaque cas particulier où l'école est bonne, et sous réserve d'un contrôle.\par
Accorder au clergé une part dans l'enseignement public n'est pas une solution. Si c'était possible, ce ne serait pas désirable, et ce n'est pas possible en France sans guerre civile.\par
Donner ordre aux instituteurs de parler de Dieu aux enfants, comme l'a fait quelques mois le gouvernement de Vichy, sur l'initiative de M. Chevalier, est une plaisanterie de très mauvais goût.\par
Conserver à la philosophie laïque son statut officiel serait une mesure arbitraire, injuste en ce qu'elle ne répond pas à l'échelle des valeurs, et qui nous précipiterait tout droit dans le totalitarisme. Car, bien que la laïcité ait excité un certain degré de ferveur presque religieuse, c'est par la nature des choses un degré faible ; et nous vivons dans une époque d'enthousiasmes chauffés à blanc. Le courant idolâtre du totalitarisme ne peut trouver d'obstacle que dans une vie spirituelle authentique. Si l'on habitue les enfants à ne pas penser à Dieu, ils deviendront fascistes ou communistes par besoin de se donner à quelque chose.\par
On voit plus clairement ce que la justice exige en ce domaine quand on a remplacé la notion de droit par celle d'obligation liée au besoin. Une âme jeune qui s'éveille à la pensée a besoin du trésor amassé par l'espèce humaine au cours des siècles. On fait tort à un enfant quand on l'élève dans un christianisme étroit qui l'empêche de jamais devenir capable de s'apercevoir qu'il y a des trésors d'or pur dans les civilisations non chrétiennes. L'éducation laïque fait aux enfants un tort plus grand. Elle dissimule ces trésors, et ceux du christianisme en plus.\par
La seule attitude à la fois légitime et pratiquement possible que puisse avoir, en France, l'enseignement public à l'égard du christianisme consiste à le regarder comme un trésor de la pensée humaine parmi tant d'autres. Il est absurde au plus haut point qu'un bachelier français ait pris connaissance de poèmes du Moyen Âge, de {\itshape Polyeucte}, d'{\itshape Athalie}, de {\itshape Phèdre}, de Pascal, de Lamartine, de doctrines philosophiques imprégnées de christianisme comme celles de Descartes et de Kant, de la {\itshape Divine Comédie} ou du {\itshape Paradise Lost}, et qu'il n'ait jamais ouvert la Bible.\par
Il n'y aurait qu'à dire aux futurs instituteurs et aux futurs professeurs : la religion a eu de tout temps et en tout pays, sauf tout récemment en quelques endroits de l'Europe, un rôle dominant dans le développement de la culture, de la pensée, de la civilisation humaine. Une instruction dans laquelle il n'est jamais question de religion est une absurdité. D'autre part, de même qu'en histoire on parle beaucoup de la France aux petits Français, il est naturel qu'étant en Europe, si l'on parle de religion, il s'agisse avant tout du christianisme.\par
\par
En conséquence, il faudrait inclure dans l'enseignement de tous les degrés, pour les enfants déjà un peu grands, des cours qu'on pourrait étiqueter, par exemple, histoire religieuse. On ferait lire aux enfants des passages de l'Écriture, et pardessus tout l'Évangile. On commenterait dans l'esprit même du texte, comme il faut toujours faire.\par
On parlerait du dogme comme d'une chose qui a joué un rôle de première importance dans nos pays, et à laquelle des hommes de toute première valeur ont cru de toute leur âme ; on n'aurait pas non plus à dissimuler que quantité de cruautés y ont trouvé un prétexte ; mais surtout on essaierait de rendre sensible aux enfants la beauté qui y est contenue. S'ils demandent : « Est-ce vrai? » il faut répondre : « Cela est si beau que cela contient certainement beaucoup de vérité. Quant à savoir si c'est ou non absolument vrai, tâchez de devenir capables de vous en rendre compte quand vous serez grands. » Il serait rigoureusement interdit de rien inclure dans les commentaires qui implique la négation du dogme ; rien non plus qui implique une affirmation. Tout instituteur ou professeur qui le désirerait et qui aurait les connaissances et le talent pédagogique nécessaires serait libre de parler aux enfants non seulement du christianisme, mais aussi, quoiqu'en insistant beaucoup moins, de n'importe quel autre courant de pensée religieuse authentique. Une pensée religieuse est authentique quand elle est universelle par son orientation. (Ce n'est pas le cas du judaïsme, qui est lié à une notion de race.)\par
Si une telle solution était appliquée, la religion cesserait peu à peu, il faut l'espérer, d'être une chose pour ou contre laquelle on prend parti de la même manière qu'on prend parti en politique. Ainsi s'aboliraient les deux camps, le camp de l'instituteur et le camp du curé, qui mettent une sorte de guerre civile latente dans tant de villages français. Le contact avec la beauté chrétienne, présentée simplement comme une beauté à savourer, imprégnerait insensiblement de spiritualité la masse du pays, si toutefois le pays en est capable, bien plus efficacement qu'aucun enseignement dogmatique des croyances religieuses.\par
Le mot de beauté n'implique nullement qu'il faille considérer les choses religieuses à la manière des esthètes. Le point de vue des esthètes est sacrilège, non seulement en matière de religion, mais même en matière d'art. Il consiste à s'amuser avec la beauté en la manipulant et en la regardant. La beauté est quelque chose qui se mange ; c'est une nourriture. Si l'on offrait au peuple la beauté chrétienne simplement à titre de beauté, ce devrait être comme une beauté qui nourrit.\par
Dans l'école rurale, la lecture attentive, souvent répétée, souvent commentée, -toujours reprise, des textes du Nouveau Testament où il est question de la vie rurale, pourrait faire beaucoup pour rendre à la vie des champs la poésie perdue. Si d'une part toute la vie spirituelle de l'âme, d'autre part toutes les connaissances scientifiques concernant l'univers matériel, sont orientées vers l'acte du travail, le travail tient sa juste place dans la pensée d'un homme. Au lieu d'être une espèce de prison, il est un contact avec ce monde et l'autre.\par
Pourquoi, par exemple, un paysan en train de semer n'aurait-il pas présentes au fond de sa pensée, sans paroles même intérieures, d'une part quelques comparaisons du Christ : « Si le grain ne meurt... », « La semence est la parole de Dieu... », « Le grain de sénevé est la plus petite des graines... », d'autre part le double mécanisme de la croissance : celui par lequel la graine, en se consommant elle-même et avec l'aide des bactéries, arrive à la surface du sol ; celui par lequel l'énergie solaire descend dans la lumière, et, captée par le vert de la tige, remonte dans un mouvement ascendant irrésistible. L'analogie qui fait des mécanismes d'ici-bas un miroir des mécanismes surnaturels, si l'on peut employer cette expression, devient alors éclatante, et la fatigue du travail, selon le mot populaire, la fait entrer dans le corps. La peine toujours plus ou moins liée à l'effort du travail devient la douleur qui fait pénétrer au centre même de l'être humain la beauté du monde.\par
Une méthode analogue peut charger d'une signification analogue le travail ouvrier. Elle est tout aussi facile à concevoir.\par
Ainsi seulement la dignité du travail serait pleinement fondée. Car, en allant au fond des choses, il n'y a pas de véritable dignité qui n'ait une racine spirituelle et par suite d'ordre surnaturel.\par
La tâche de l'école populaire est de donner au travail davantage de dignité en y infusant de la pensée, et non pas de faire du travailleur une chose à compartiments qui tantôt travaille et tantôt pense. Bien entendu, un paysan qui sème doit être attentif à répandre le grain comme il faut, et non à se souvenir de leçons apprises à l'école. Mais l'objet de l'attention n'est pas tout le contenu de la pensée. Une jeune femme heureuse, enceinte pour la première fois, qui coud une layette, pense à coudre comme il faut. Mais elle n'oublie pas un instant l'enfant qu'elle porte en elle. Au même moment, quelque part dans un atelier de prison, une condamnée coud en pensant aussi à coudre comme il faut, car elle craint d'être punie. On pourrait imaginer que les deux femmes font au même instant le même ouvrage, et ont l'attention occupée par la même difficulté technique. Il n'y en a pas moins un abîme de différence entre l'un et l'autre travail. Tout le problème social consiste à faire passer les travailleurs de l'une à l'autre de ces deux situations.\par
Ce qu'il faudrait, c'est que ce monde et l'autre, dans leur double beauté, soient présents et associés à l'acte du travail, comme l'enfant qui va naître à la fabrication de la layette. Cette association peut s'opérer par une manière de présenter les pensées qui les mette en rapport direct avec les gestes et les opérations particulières de chaque travail, par une assimilation assez profonde pour qu'elles pénètrent dans la substance même de l'être, et par une habitude imprimée dans la mémoire et liant ces pensées aux mouvements du travail.\par
Nous ne sommes pas aujourd'hui, ni intellectuellement ni spirituellement, capables d'une telle transformation. Ce serait beaucoup si nous étions capables de commencer à la préparer. Bien entendu, l'école n'y suffirait pas. Il faudrait que tous les milieux où subsiste quelque chose qui ressemble à de la pensée y participent – les Églises, les syndicats, les milieux littéraires et scientifiques. On ose à peine mentionner dans cette catégorie les milieux politiques.\par
Notre époque a pour mission propre, pour vocation, la constitution d'une civilisation fondée sur la spiritualité du travail. Les pensées qui se rapportent au pressentiment de cette vocation, et qui sont éparses chez Rousseau, George Sand, Tolstoï, Proudhon, Marx, dans les encycliques des papes, et ailleurs, sont les seules pensées originales de notre temps, les seules que nous n'ayons pas empruntées aux Grecs. C'est parce que nous n'avons pas été à la hauteur de cette grande chose qui était en train d'être enfantée en nous que nous nous sommes jetés dans l'abîme des systèmes totalitaires. Mais si l'Allemagne est vaincue, peut-être que notre faillite n'est pas définitive. Peut-être avons-nous encore une occasion. On ne peut pas y penser sans angoisse ; si nous l'avons, médiocres comme nous sommes, comment ferons-nous pour ne pas la manquer ?\par
Cette vocation est la seule chose assez grande pour la proposer aux peuples au lieu de l'idole totalitaire. Si on ne la propose pas de manière à en faire sentir la grandeur, ils resteront sous l'emprise de l'idole ; elle sera seulement peinte en rouge au lieu de brun. Si on donne à choisir aux hommes entre le beurre et les canons, bien qu'ils préfèrent, et de très loin, le beurre, une fatalité mystérieuse les contraint malgré eux à choisir les canons. Le beurre manque par trop de poésie – du moins lorsqu'on en a, car il prend une espèce de poésie quand on n'en a pas. La préférence qu’on a pour lui est inavouable.\par
Actuellement, les Nations Unies, surtout l'Amérique, passent leur temps à dire aux populations affamées d'Europe : Avec nos canons, nous allons vous procurer du beurre. Ceci ne provoque qu'une seule réaction, la pensée qu'ils ne se hâtent guère. Quand on donnera ce beurre, les gens se jetteront dessus ; et aussitôt après ils se tourneront vers quiconque leur fera voir de jolis canons, décemment enveloppés de n'importe quelle idéologie. Qu'on n'imagine pas qu'étant épuisés ils ne demanderont que le bien-être. L'épuisement nerveux causé par un malheur récent empêche de s'installer dans le bien-être. Il contraint à chercher l'oubli, soit dans une soûlerie de jouissances exaspérées – comme ce fut le cas après 1918 – soit dans quelque sombre fanatisme. Le malheur qui a mordu trop profondément suscite une disposition au malheur qui contraint à y précipiter soi-même et autrui. L'Allemagne en est un exemple.\par
Les populations malheureuses du continent européen ont besoin de grandeur encore plus que de pain, et il n'y a que deux espèces de grandeur, la grandeur authentique, qui est d'ordre spirituel, et le vieux mensonge de la conquête du monde. La conquête est l'ersatz de la grandeur.\par
\par
La forme contemporaine de la grandeur authentique, c'est une civilisation constituée par la spiritualité du travail. C'est une pensée qu'on peut lancer en avant sans risquer aucune désunion. Le mot de spiritualité n'implique aucune affiliation particulière. Les communistes eux-mêmes, dans l'atmosphère actuelle, ne le repousseraient sans doute pas. Il serait facile d'ailleurs de trouver dans Marx des citations qui se ramènent toutes au reproche de manque de spiritualité adressé à la société capitaliste ; ce qui implique qu'il doit y en avoir dans la société nouvelle. Les conservateurs n'oseraient pas repousser cette formule. Les milieux radicaux, laïques, francs-maçons, non plus. Les chrétiens s'en empareraient avec joie. Elle pourrait susciter l'unanimité.\par
Mais on ne peut toucher à une telle formule qu'en tremblant. Comment y toucher sans la souiller, sans en faire un mensonge ? Notre époque est tellement empoisonnée de mensonge qu'elle change en mensonge tout ce qu'elle touche. Et nous sommes de notre époque ; nous n'avons aucune raison de nous croire meilleurs qu'elle.\par
Discréditer de tels mots en les lançant dans le domaine public sans des précautions infinies serait faire un mal irréparable ; ce serait tuer tout reste d'espoir que la chose correspondante puisse apparaître. Ils ne doivent pas être liés à une cause, à un mouvement, ni même à un régime, ni non plus à une nation. Il ne faut pas leur faire le mal que Pétain a fait aux mots « Travail, Famille, Patrie », ni non plus le mal que la III\textsuperscript{e} République a fait aux mots Liberté, Égalité, Fraternité. Ils ne doivent pas être un mot d'ordre.\par
Si on les propose publiquement, ce doit être seulement comme l'expression d'une pensée qui dépasse de très loin les hommes et les collectivités d'aujourd'hui, et qu'on s'engage en toute humilité à garder présente à l'esprit comme guide en toutes choses. Si cette modestie est moins puissante pour entraîner les masses que des attitudes plus vulgaires, peu importe. Il vaut mieux échouer que réussir à faire du mal.\par
Mais cette pensée n'aurait pas besoin d'être lancée avec fracas pour imprégner peu à peu les esprits, parce qu'elle répond aux inquiétudes de tous dans le moment présent. Tout le monde répète, avec des termes légèrement différents, que nous souffrons d'un déséquilibre dû à un développement purement matériel de la technique. Le déséquilibre ne peut être réparé que par un développement spirituel dans le même domaine, c'est-à-dire dans le domaine du travail.\par
La seule difficulté, c'est la méfiance douloureuse et malheureusement trop légitime des masses, qui regardent toute formule un peu élevée comme un piège dressé pour les duper.\par
Une civilisation constituée par une spiritualité du travail serait le plus haut degré d'enracinement de l'homme dans l'univers, par suite l'opposé de l'état où nous sommes, qui consiste en un déracinement presque total. Elle est ainsi par nature l'aspiration qui correspond à notre souffrance.
\subsection[{C. Déracinement et nation}]{C. \\
Déracinement et nation}
\noindent \par
Une autre espèce de déracinement encore doit être étudiée pour une connaissance sommaire de notre principale maladie. C'est le déracinement qu'on pourrait nommer géographique, c'est-à-dire par rapport aux collectivités qui correspondent à des territoires. Le sens même de ces collectivités a presque disparu, excepté pour une seule, pour la nation. Mais il y en a, il y en a eu beaucoup d'autres. Certaines plus petites, toutes petites parfois : ville ou ensemble de villages, province, région certaines englobant plusieurs nations ; certaines englobant plusieurs morceaux de nations.\par
La nation seule s'est substituée à tout cela. La nation, c'est-à-dire l'État ; car on ne peut pas trouver d'autre définition au mot nation que l'ensemble des territoires reconnaissant l'autorité d'un même État. On peut dire qu'à notre époque l'argent et l'État avaient remplacé tous les autres attachements.\par
La nation seule, depuis déjà longtemps, joue le rôle qui constitue par excellence la mission de la collectivité à l'égard de l'être humain, à savoir assurer à travers le présent une liaison entre le passé et l'avenir. En ce sens, on peut dire que c'est la seule collectivité qui existe dans l'univers actuel. La famille n'existe pas. Ce qu'on appelle aujourd'hui de ce nom, c'est un groupe minuscule d'êtres humains autour de chacun ; père et mère, mari ou femme, enfants ; frères et sœurs déjà un peu loin. Ces derniers temps, au milieu de la détresse générale, ce petit groupe est devenu une force d'attraction presque irrésistible, au point de faire oublier parfois toute espèce de devoir ; mais c'est que là seulement se trouvait un peu de chaleur vivante, parmi le froid glacé qui s’était abattu tout d'un coup. C'était une réaction presque animale.\par
Mais personne aujourd'hui ne pense à ceux de ses aïeux qui sont morts cinquante ans, ou fût-ce vingt ou dix ans, avant sa naissance, ni à ceux de ses descendants qui naîtront cinquante ans, ou fût-ce vingt ou dix ans après sa mort. Par suite, du point de vue de la collectivité et de sa fonction propre, la famille ne compte pas.\par
La profession, de ce point de vue, ne compte pas non plus. La corporation était un lien entre les morts, les vivants et les hommes non encore nés, dans le cadre d'un certain travail. Il n'y a rien aujourd'hui qui soit si peu que ce soit orienté vers une telle fonction. Le syndicalisme français vers 1900 a peut-être eu quelques velléités en ce sens, vite effacées.\par
Enfin le village, la ville, la contrée, la province, la région, toutes les unités géographiques plus petites que la nation, ont presque cessé de compter. Celles qui englobent plusieurs nations ou plusieurs morceaux de nations aussi. Quand on disait, par exemple, il y a quelques siècles, « la chrétienté », cela avait une tout autre résonance affective qu'aujourd'hui l'Europe.\par
En somme, le bien le plus précieux de l'homme dans l'ordre temporel, c'est-à-dire la continuité dans le temps, par delà les limites de l'existence humaine, dans les deux sens, ce bien a été entièrement remis en dépôt à l'État.\par
Et pourtant c'est précisément dans cette période où la nation subsiste seule que nous avons assisté à la décomposition instantanée, vertigineuse de la nation. Cela nous a laissés étourdis, au point qu'il est extrêmement difficile de réfléchir là-dessus.\par
Le peuple français, en juin et juillet 1940, n'a pas été un peuple à qui des escrocs, cachés dans l'ombre, ont soudain par surprise volé sa patrie. C'est un peuple qui a ouvert la main et laissé la patrie tomber par terre. Plus tard – mais après un long intervalle – il s'est consumé en efforts de plus en plus désespérés pour la ramasser, mais quelqu'un avait mis le pied dessus.\par
Maintenant le sens national est revenu. Les mots « mourir pour la France » ont repris un accent qu'ils n'avaient pas eu depuis 1918. Mais dans le mouvement de refus qui a soulevé le peuple français, la faim, le froid, la présence toujours odieuse de soldats étrangers possédant tout pouvoir pour commander, la séparation des familles, pour certains l'exil, la captivité, toutes ces souffrances ont eu pour le moins une très large part, probablement décisive. La meilleure preuve est la différence d'esprit qui séparait la zone occupée et l'autre. Il n'y a pas par nature une plus grande quantité de grâce patriotique au nord qu'au sud de la Loire. La différence des situations a produit des états d'esprit différents. L'exemple de la résistance anglaise, l'espoir de la défaite allemande ont été aussi des facteurs importants.\par
La France aujourd'hui n'a d'autre réalité que le souvenir et l'espérance. La République n'a jamais été aussi belle que sous l'Empire ; la patrie n'est jamais si belle que sous l'oppression d'un conquérant, si l'on a l'espoir de la revoir intacte. C'est pourquoi on ne doit pas juger, par l'intensité actuelle du sentiment national, de l'efficacité réelle qu'il possédera, après la libération, pour la stabilité de la vie publique.\par
L'effritement instantané de ce sentiment en juin 1940 est un souvenir chargé de tant de honte qu'on aime mieux ne pas y penser, le mettre hors de compte, ne penser qu'au redressement ultérieur. Dans la vie privée aussi, chacun est toujours tenté de mettre ses propres défaillances, en quelque sorte, entre parenthèses, de les ranger dans quelque lieu de débarras, de trouver un mode de calcul en vertu duquel elles ne comptent pas. Céder à cette tentation, c'est ruiner l'âme ; c'est la tentation à vaincre par excellence.\par
Nous avons tous succombé à cette tentation, pour cette honte publique qui a été si profonde qu'elle a blessé chacun dans le sentiment intime de son propre honneur. Sans cette tentation, les réflexions autour d'un fait tellement extraordinaire auraient déjà abouti à une nouvelle doctrine, à une nouvelle conception de la patrie.\par
Du point de vue social notamment, on n'évitera pas la nécessité de penser la notion de patrie. Non pas la penser à nouveau ; la penser pour la première fois ; car, sauf erreur, elle n'a jamais été pensée. N'est-ce pas singulier, pour une notion qui a joué et qui joue un tel rôle ? Cela fait voir quelle place la pensée tient en réalité parmi nous.\par
La notion de patrie avait perdu tout crédit parmi les ouvriers français au cours du dernier quart de siècle. Les communistes l'ont remise en circulation après 1934, avec grand accompagnement de drapeaux tricolores et de chants de la « Marseillaise ». Mais ils n'ont pas eu la moindre difficulté à la mettre de nouveau en sommeil peu avant la guerre. Ce n'est pas en son nom qu'ils ont commencé l'action de résistance. Ils ne l'ont adoptée de nouveau que trois quarts d'année environ après la défaite. Peu à peu ils l'ont adoptée intégralement. Mais il serait par trop naïf de voir là une réconciliation véritable entre la classe ouvrière et la patrie. Les ouvriers meurent pour la patrie, ce n'est que trop vrai. Mais nous vivons dans un temps tellement perdu de mensonges que même la vertu du sang volontairement sacrifié ne suffit pas à remettre dans la vérité.\par
Pendant des années, on a enseigné aux ouvriers que l'internationalisme est le plus sacré des devoirs, et le patriotisme, le plus honteux des préjugés bourgeois. On a passé d'autres années à leur enseigner que le patriotisme est un devoir sacré, et ce qui n'est pas patriotisme, une trahison. Comment, en fin de compte, seraient-ils dirigés autrement que par des réactions élémentaires et par de la propagande ?\par
Il n'y aura pas de mouvement ouvrier sain s'il ne trouve à sa disposition une doctrine assignant une place à la notion de patrie, et une place déterminée, c'est-à-dire limitée. D'ailleurs, ce besoin n'est davantage évident pour les milieux ouvriers que parce que le problème de la patrie y a été beaucoup discuté depuis longtemps. Mais c'est un besoin commun à tout le pays. Il est inadmissible que le mot qui aujourd'hui revient presque continuellement accouplé à celui de devoir, n'ait presque jamais fait l'objet d'aucune étude. En général, on ne trouve à citer à son sujet qu'une page médiocre de Renan.\par
La nation est un fait récent. Au Moyen Âge la fidélité allait au seigneur, ou à la cité, ou aux deux, et par delà à des milieux territoriaux qui n'étaient pas très distincts. Le sentiment que nous nommons patriotisme existait bien, à un degré parfois très intense ; c'est l'objet qui n'en était pas territorialement défini. Le sentiment couvrait selon les circonstances des surfaces de terre variables.\par
À vrai dire le patriotisme a toujours existé, aussi haut que remonte l'histoire. Vercingétorix est vraiment mort pour la Gaule ; les tribus espagnoles qui ont résisté à la conquête romaine parfois jusqu'à l'extermination, mouraient pour l'Espagne, et le savaient, et le disaient ; les morts de Marathon et de Salamine sont morts pour la Grèce ; au temps où la Grèce, non encore réduite en province, était par rapport à Rome dans le même état que la France de Vichy par rapport à l'Allemagne, les enfants des villes grecques jetaient des pierres, dans la rue, aux collaborateurs, et les appelaient traîtres, avec la même indignation qui est la nôtre aujourd'hui.\par
Ce qui n'avait jamais existé jusqu'à une époque récente, c'est un objet cristallisé, offert d'une manière permanente au sentiment patriotique. Le patriotisme était diffus, errant, et s'élargissait ou se resserrait selon les affinités et les périls. Il était mélangé à des loyautés différentes, celles envers des hommes, seigneurs ou rois, celles envers des cités. Le tout formait quelque chose de très confus, mais aussi de très humain. Pour exprimer le sentiment d'obligation que chacun éprouve envers son pays, on disait le plus souvent « le public », « le bien public », mot qui peut à volonté désigner un village, une ville, une province, la France, la chrétienté, le genre humain.\par
On parlait aussi du royaume de France. Dans ce terme était mélangé le sentiment de l'obligation envers le pays et celui de la fidélité envers le roi. Mais deux obstacles ont empêché que ce double sentiment ait jamais pu être pur, non pas même au temps de Jeanne d'Arc. Il ne faut pas oublier que la population de Paris était contre Jeanne d'Arc.\par
Un premier obstacle était qu'après Charles V la France, si on veut employer le vocabulaire de Montesquieu, a cessé d'être une monarchie pour tomber dans l'état de despotisme, dont elle n'est sortie qu'au XVIII\textsuperscript{e} siècle. Nous trouvons aujourd'hui tellement naturel de payer des impôts à l'État que nous n'imaginons pas au milieu de quel bouleversement moral cette coutume s'est établie. Au XIV\textsuperscript{e} siècle le paiement des impôts, excepté les contributions exceptionnelles consenties pour la guerre, était regardé comme un déshonneur, une honte réservée aux pays conquis, le signe visible de l'esclavage. On trouve le même sentiment exprimé dans le Romancero espagnol, et aussi dans Shakespeare – « Cette terre... a fait une honteuse conquête d'elle-même. »\par
Charles VI enfant, aidé de ses oncles, par l'usage de la corruption et d'une atroce cruauté, a brutalement contraint le peuple de France à accepter un impôt absolument arbitraire, renouvelable à volonté, qui affamait littéralement les pauvres et était gaspillé par les seigneurs. C'est pourquoi les Anglais de Henri V furent d'abord accueillis comme des libérateurs, à un moment où les Armagnacs étaient le parti des riches et les Bourguignons celui des pauvres.\par
Le peuple français, courbé brutalement et d'un coup, n'eut plus ensuite, jusqu'au XVIII\textsuperscript{e} siècle, que des secousses d'indépendance. Pendant toute cette période, il fut regardé par les autres Européens comme le peuple esclave par excellence, le peuple qui était à la merci de son souverain comme un bétail.\par
Mais en même, temps s'installa au plus profond du cœur de ce peuple une haine refoulée et d'autant plus amère à l'égard du roi, haine dont la tradition ne s'éteignit jamais. On la sent déjà dans une déchirante complainte de paysans du temps de Charles VI. Elle dut avoir une part dans la mystérieuse popularité de la Ligue à Paris. Après l'assassinat de Henri IV, un enfant de douze ans fut mis à mort pour avoir dit publiquement qu'il en ferait autant au petit Louis XIII. Richelieu commença sa carrière par un discours où il demandait au clergé de proclamer la damnation de tous les régicides ; il donnait comme motif que ceux qui nourrissaient ce dessein étaient animés d'un enthousiasme bien trop fanatique pour être retenus par aucune peine temporelle.\par
Cette haine atteignit son degré d'exaspération le plus intense à la fin du règne de Louis XIV. Ayant été comprimée par une terreur d'intensité égale, elle explosa, selon la coutume déconcertante de l'histoire, avec quatre-vingts années de retard ; ce fut ce pauvre Louis XVI qui reçut le coup. Cette même haine empêcha qu'il pût y avoir vraiment une restauration de la royauté en 1815. Aujourd'hui encore, elle empêche absolument que le Comte de Paris puisse être librement accepté par le peuple de France, malgré l'adhésion d'un homme comme Bernanos. À certains égards c'est dommage ; beaucoup de problèmes pourraient être ainsi résolue ; mais c'est ainsi.\par
Une autre source de poison dans l'amour des Français pour le royaume de France est le fait qu'à chaque époque, parmi les territoires placés sous l'obéissance du roi de France, certains se sentaient des pays conquis et étaient traités comme tels. Il faut reconnaître que les quarante rois qui en mille ans firent la France ont souvent mis à cette besogne une brutalité digne de notre époque. S'il y a correspondance naturelle entre l'arbre et les fruits, on ne doit pas s'étonner que le fruit soit en fait loin de la perfection.\par
Par exemple, on peut trouver dans l'histoire des faits d'une atrocité aussi grande, mais non plus grande, sauf peut-être quelques rares exceptions, que la conquête par les Français des territoires situés au sud de la Loire, au début du XIII\textsuperscript{e} siècle. Ces territoires où existait un niveau élevé de culture, de tolérance, de liberté, de vie spirituelle, étaient animés d'un patriotisme intense pour ce qu'ils nommaient leur « langage » ; mot par lequel ils désignaient la patrie. Les Français étaient pour eux des étrangers et des barbares, comme pour nous les Allemands. Pour imprimer immédiatement la terreur, les Français commencèrent par exterminer la ville entière de Béziers, et ils obtinrent l'effet cherché. Une fois le pays conquis, ils y installèrent l'Inquisition. Un trouble sourd continua à couver parmi ces populations, et les poussa plus tard à embrasser avec ardeur le protestantisme, dont d'Aubigné dit, malgré les différences si considérables de doctrine, qu'il procède directement des Albigeois. On peut voir combien était forte dans ces pays la haine du pouvoir central, par la ferveur religieuse témoignée à Toulouse aux restes du duc de Montmorency, décapité pour rébellion contre Richelieu. La même protestation latente les jeta avec enthousiasme dans la Révolution française. Plus tard, ils devinrent radicaux-socialistes, laïques, anticléricaux ; sous la III\textsuperscript{e} République ils ne haïssaient plus le pouvoir central, ils s'en étaient dans une large mesure emparés et l'exploitaient.\par
On peut remarquer qu'à chaque fois leur protestation a pris un caractère de déracinement plus intense et un niveau de spiritualité et de pensée plus bas. On peut remarquer aussi que depuis qu'ils ont été conquis, ces pays ont apporté à la culture française une contribution assez faible, alors qu'auparavant ils étaient tellement brillants. La pensée française doit davantage aux Albigeois et aux troubadours du XII\textsuperscript{e} siècle, qui n'étaient pas français, qu'à tout ce que ces territoires ont produit au cours des siècles suivants.\par
Le comté de Bourgogne était le siège d'une culture originale et extrêmement brillante qui ne lui survécut pas. Les villes de Flandre avaient, à la fin du XIV\textsuperscript{e} siècle, des relations fraternelles et clandestines avec Paris et Rouen ; mais des Flamands blessés aimaient mieux mourir que d'être soignés par les soldats de Charles VI. Ces soldats firent une expédition de pillage du côté de la Hollande, et en ramenèrent de riches bourgeois qu'on décida de tuer ; un mouvement de pitié amena à leur offrir la vie s'ils voulaient être sujets du roi de France ; ils répondirent qu'une fois morts leurs os refuseraient, s'ils pouvaient, d'être soumis à l'autorité du roi de France. Un historien catalan de la même époque, racontant l'histoire des Vêpres siciliennes, dit – « Les Français, qui, partout où ils dominent, sont aussi cruels qu'il est possible de l'être ... »\par
Les Bretons furent désespérés quand leur souveraine Anne fut contrainte d'épouser le roi de France. Si ces hommes revenaient aujourd'hui, ou plutôt il y a quelques années, auraient-ils de très fortes raisons pour penser qu'ils s'étaient trompés ? Si discrédité que soit l'autonomisme breton par la personne de ceux qui le manœuvrent et les fins inavouables qu'ils poursuivent, il est certain que cette propagande répond à quelque chose de réel à la fois dans les faits et dans les sentiments de ces populations. Il y a des trésors latents, dans ce peuple, qui n'ont pas pu sortir. La culture française ne lui convient pas ; la sienne ne peut pas germer ; dès lors il est maintenu tout entier dans les bas-fonds des catégories sociales inférieures. Les Bretons fournissent une large part des soldats illettrés ; les Bretonnes, dit-on, une large part des prostituées de Paris. L'autonomie ne serait pas un remède, mais cela ne signifie pas que la maladie n'existe pas.\par
La Franche-Comté, libre et heureuse sous la suzeraineté très lointaine des Espagnols, se battit au XVII\textsuperscript{e} siècle pour ne pas devenir française. Les gens de Strasbourg se mirent à pleurer quand ils virent les troupes de Louis XIV entrer dans leur ville en pleine paix, sans aucune déclaration préalable, par une violation de la parole donnée digne d'Hitler.\par
Paoli, le dernier héros corse, dépensa son héroïsme pour empêcher son pays de tomber aux mains de la France. Il y a un monument en son honneur dans une église de Florence ; en France on ne parle guère de lui. La Corse est un exemple du danger de contagion impliqué par le déracinement. Après avoir conquis, colonisé, corrompu et pourri les gens de cette île, nous les avons subis sous forme de préfets de police, policiers, adjudants, pions et autres fonctions de cette espèce, à la faveur desquelles ils traitaient à leur tour les Français comme une population plus ou moins conquise. Ils ont aussi contribué à donner à la France auprès de beaucoup d'indigènes des colonies, une réputation de brutalité et de cruauté.\par
Quand on loue les rois de France d'avoir assimilé les pays conquis, la vérité est surtout qu'ils les ont dans une large mesure déracinés. C'est un procédé d'assimilation facile, à la portée de chacun. Des gens à qui on enlève leur culture ou bien restent sans culture ou bien reçoivent des bribes de celle qu'on veut bien leur communiquer. Dans les deux cas, ils ne font pas des taches de couleur différente, ils semblent assimilés. La vraie merveille est d'assimiler des populations qui conservent leur culture vivante, bien que modifiée. C'est une merveille rarement réalisée.\par
Certainement, sous l'Ancien Régime, il y a eu une grande intensité de conscience française à tous les moments de grand éclat de la France ; au XIII\textsuperscript{e} siècle, quand l'Europe accourait à l'Université de Paris ; au XVI\textsuperscript{e} siècle, quand la Renaissance, déjà éteinte ou non encore allumée ailleurs, avait son siège en France ; dans les premières années de Louis XIV, quand le prestige des lettres s'unissait à celui des armes. Il n'en est pas moins vrai que ce ne sont pas les rois qui ont soudé ces territoires disparates. C'est uniquement la Révolution.\par
Déjà au cours du XVIII\textsuperscript{e} siècle il y avait en France, dans des milieux très différents, à côté d'une corruption effroyable, une flamme brûlante et pure de patriotisme. Témoin ce jeune paysan, frère de Restif de la Bretonne, brillamment doué, qui devint soldat presque enfant encore par pur amour du bien public, et fut tué à dix-sept ans. Mais c'était déjà la Révolution qui produisait cela. On l'a pressentie, attendue, désirée, tout le long du siècle.\par
La Révolution a fondu les populations soumises à la couronne de France en une masse unique, et cela par l'ivresse de la souveraineté nationale. Ceux qui avaient été Français de force le devinrent par libre consentement ; beaucoup de ceux qui ne l'étaient pas souhaitaient le devenir. Car être Français, dès ce moment, c'était être la nation souveraine. Si tous les peuples étaient devenus souverains partout, comme on l'espérait, la France ne pouvait perdre la gloire d'avoir commencé. D'ailleurs les frontières n'avaient plus d'importance. Les étrangers étaient seulement ceux qui demeuraient esclaves des tyrans. Les étrangers d'âme vraiment républicaine étaient volontiers admis comme Français à titre honorifique.\par
Ainsi il y a eu en France ce paradoxe d'un patriotisme fondé, non sur l'amour du passé, mais sur la rupture la plus violente avec le passé du pays. Et pourtant la Révolution avait un passé dans la partie plus ou moins souterraine de l'histoire de France ; tout ce qui avait rapport à l'émancipation des serfs, aux libertés des villes, aux luttes sociales ; les révoltes du XIV\textsuperscript{e} siècle, le début du mouvement des Bourguignons, la Fronde, des écrivains comme d'Aubigné, Théophile de Viau, Retz. Sous François Ier un projet de milice populaire fut écarté, parce que les seigneurs objectèrent que si on le réalisait les petits-fils des miliciens seraient seigneurs et leurs propres petits-fils seraient serfs. Si grande était la force ascendante qui soulevait souterrainement ce peuple.\par
Mais l'influence des Encyclopédistes, tous intellectuels déracinés, tous obsédés par l'idée de progrès, empêcha qu'on fit aucun effort pour évoquer une tradition révolutionnaire. D'ailleurs la longue terreur du règne de Louis XIV faisait un espace vide, difficile à franchir. C'est à cause d'elle que, malgré les efforts de Montesquieu en sens contraire, le courant de libération du XVIII\textsuperscript{e} siècle se trouva sans racines historiques. 1789 fut vraiment une rupture.\par
Le sentiment qu'on nommait alors patriotisme avait pour objet uniquement le présent et l'avenir. C'était l'amour de la nation souveraine, fondé dans une large mesure sur la fierté d'en faire partie. La qualité de Français semblait être non pas un fait, mais un choix de la volonté, comme aujourd'hui l'affiliation à un parti ou à une Église.\par
Quant à ceux qui étaient attachés au passé de la France, leur attachement prit la forme de fidélité personnelle et dynastique au roi. Ils n'éprouvèrent aucune gêne à chercher un secours dans les armes des rois étrangers. Ce n'étaient pas des traîtres. Ils demeuraient fidèles à ce à quoi ils croyaient devoir de la fidélité, exactement comme les hommes qui firent mourir Louis XVI.\par
Les seuls à cette époque qui furent patriotes au sens que le mot a pris plus lard, ce sont ceux qui sont apparus aux yeux des contemporains et de la postérité comme les archi-traîtres, les gens comme Talleyrand, qui ont servi, non pas, comme on l'a dit, tous les régimes, mais la France derrière tous les régimes. Mais pour eux la France n'était ni la nation souveraine, ni le roi ; c'était l'État français. La suite des événements leur a donné raison.\par
Car, quand l'illusion de la souveraineté nationale apparut manifestement comme une illusion, elle ne put plus servir d'objet au patriotisme ; d'autre part, la royauté était comme ces plantes coupées qu'on ne replante plus ; le patriotisme devait changer de signification et s'orienter vers l'État. Mais dès lors il cessait d'être populaire. Car l'État n'était pas une création de 1789, il datait du début du XVII\textsuperscript{e} siècle et avait part à la haine vouée par le peuple à la royauté. C'est ainsi, que par un paradoxe historique à première vue surprenant, le patriotisme changea de classe sociale et de camp politique ; il avait été à gauche, il passa à droite.\par
Le changement s'opéra complètement à la suite de la Commune et des débuts de la III\textsuperscript{e} République. Le massacre de mai 1871 a été un coup dont, moralement, les ouvriers français ne se sont peut-être pas relevés. Ce n'est pas tellement loin. Un ouvrier âgé aujourd'hui de cinquante ans peut en avoir recueilli les souvenirs terrifiés de la bouche de son père alors enfant. L'armée du XIX\textsuperscript{e} siècle était une création spécifique de la Révolution française. Même les soldats aux ordres des Bourbons, de Louis-Philippe ou de Napoléon III devaient se faire une extrême violence pour tirer sur le peuple. En 1871, pour la première fois depuis la Révolution, si l'on excepte le court intermède de 1848, la France possédait une armée républicaine. Cette armée, composée de braves garçons des campagnes françaises, se mit à massacrer les ouvriers avec un débordement inouï de joie sadique. Il y avait de quoi produire un choc.\par
La cause principale en était sans doute le besoin de compensation à la honte de la défaite, ce même besoin qui nous mena un peu plus tard conquérir les malheureux Annamites. Les faits montrent que, sauf opération surnaturelle de la grâce, il n'y a pas de cruauté ni de bassesse dont les braves gens ne soient capables, dès qu'entrent en jeu les mécanismes psychologiques correspondants.\par
La III\textsuperscript{e} République fut un second choc. On peut croire à la souveraineté nationale tant que de méchants rois ou empereurs la bâillonnent ; on pense : s'ils n'étaient pas là !... Mais quand ils ne sont plus là, quand la démocratie est installée et que néanmoins le peuple n'est manifestement pas souverain, le désarroi est inévitable.\par
1871 fut la dernière année de ce patriotisme français particulier né en 1789. Le prince impérial allemand Frédéric – plus tard Frédéric III – homme humain, raisonnable et intelligent, a été vivement surpris par l'intensité de ce patriotisme, rencontré partout au cours de la campagne. Il ne pouvait comprendre les Alsaciens qui, ignorant presque le français, parlant un dialecte tout proche de l'allemand, brutalement conquis à une date en somme récente, ne voulaient pas entendre parler de l'Allemagne. Il constatait qu'ils avaient pour mobile la fierté d'appartenir au pays de la Révolution française, à la nation souveraine. L'annexion, en les séparant de la France, leur fit peut-être conserver partiellement cet état d'esprit jusqu'en 1918.\par
La Commune de Paris avait été au début, non un mouvement social, mais une explosion de patriotisme et même de chauvinisme aigu. Tout au long du XIX\textsuperscript{e} siècle d'ailleurs, la tournure agressive du patriotisme français avait inquiété l'Europe ; la guerre de 1870 en avait été le résultat direct ; car la France n'avait pas préparé cette guerre, mais elle ne l'avait pas moins déclarée sans aucun motif raisonnable. Les rêves de conquête impériale étaient restés vivants dans le peuple tout le long du siècle. En même temps on buvait à l'indépendance du monde. Conquérir le monde et libérer le monde sont deux formes de gloire incompatibles en fait, mais qui se concilient très bien dans la rêverie.\par
Tout ce bouillonnement de sentiment populaire est tombé après 1871. Deux causes ont pourtant maintenu une apparence de continuité dans le patriotisme. D'abord le ressentiment de la défaite. Il n'y avait alors vraiment pas encore de motif raisonnable d'en vouloir aux Allemands ; ils n'avaient pas commis d'agression ; ils s'étaient à peu près abstenus d'atrocités ; et nous avons eu mauvaise grâce à leur reprocher la violation des droits des peuples au sujet de l'Alsace-Lorraine, population en grande partie germanique, à partir de nos premières expéditions en Annam. Mais nous leur en voulions de nous avoir vaincus, comme s'ils avaient violé un droit divin, éternel, imprescriptible de la France à la victoire.\par
Dans nos haines actuelles, auxquelles il y a par malheur tant de causes trop légitimes, ce sentiment singulier entre aussi pour une part. Il a été également un des mobiles de certains collaborateurs de la première heure ; si la France était dans le camp de la défaite, pensaient-ils, ce ne pouvait être que parce qu'il y avait eu maldonne, erreur, malentendu ; sa place naturelle est dans le camp de la victoire ; le procédé le plus facile, le moins pénible, le moins douloureux, pour opérer la rectification indispensable, est de changer de camp. Cet état d'esprit dominait certains milieux de Vichy en juillet 1940.\par
Mais surtout ce qui empêcha le patriotisme français de disparaître au cours de la III\textsuperscript{e} République, après qu'il eut perdu presque toute sa substance vivante, c'est qu'il n'y avait pas autre chose. Les Français n'avaient pas autre chose que la France à quoi être fidèles ; et quand ils l'abandonnèrent pour un moment, en juin 1940, on vit combien peut être hideux et pitoyable le spectacle d'un peuple qui n'est lié à rien par aucune fidélité. C'est pourquoi, plus tard, ils se sont de nouveau accrochés exclusivement à la France. Mais si le peuple français retrouve ce qu'on appelle aujourd'hui du nom de souveraineté, la même difficulté qu'avant 1940 reparaîtra ; c'est que la réalité désignée par le mot France sera avant tout un État.\par
L'État est une chose froide qui ne peut pas être aimée mais il tue et abolit tout ce qui pourrait l'être ; ainsi on est forcé de l'aimer, parce qu'il n'y a que lui. Tel est le supplice moral de nos contemporains.\par
C'est peut-être la vraie cause de ce phénomène du chef qui a surgi partout et surprend tant de gens. Actuellement, dans tous les pays, dans toutes les causes, il y a un homme vers qui vont les fidélités à titre personnel. La nécessité d'embrasser le froid métallique de l'État a rendu les gens, par contraste, affamés d'aimer quelque chose qui soit fait de chair et de sang. Ce phénomène n'est pas près de prendre fin, et, si désastreuses qu'en aient été jusqu'ici les conséquences, il peut nous réserver encore des surprises très pénibles ; car l'art, bien connu à Hollywood, de fabriquer des vedettes avec n'importe quel matériel humain permet à n'importe qui de s'offrir à l'adoration des masses.\par
Sauf erreur, la notion d'État comme objet de fidélité est apparue, pour la première fois en France et en Europe, avec Richelieu. Avant lui on pouvait parler, sur un ton d'attachement religieux, du bien public, du pays, du roi, du seigneur. Lui, le premier, adopta le principe que quiconque exerce une fonction publique doit sa fidélité tout entière, dans l'exercice de cette fonction, non pas au public, non pas au roi, mais à l'État et à rien d'autre. Il serait difficile de définir l'État d'une manière rigoureuse. Mais il n'est malheureusement pas possible de douter que ce mot ne désigne une réalité.\par
Richelieu, qui avait la clarté d'intelligence si fréquente à cette époque, a défini en termes lumineux cette différence entre morale et politique autour de laquelle on a semé depuis tant de confusion. Il a dit à peu près : On doit se garder d'appliquer les mêmes règles au salut de l'État qu'à celui de l'âme ; car le salut des âmes s'opère dans l'autre monde, au lieu que celui des États ne s'opère que dans celui-ci.\par
Cela est cruellement vrai. Un chrétien ne devrait pouvoir en tirer qu'une seule conclusion : c'est qu'au lieu qu'on doit au salut de l'âme, c'est-à-dire à Dieu, une fidélité totale, absolue, inconditionnée, la cause du salut de l'État est de celles auxquelles on doit une fidélité limitée et conditionnelle.\par
Mais bien que Richelieu crût être chrétien, et sans doute sincèrement, sa conclusion était tout autre. Elle était que l'homme responsable du salut de l'État, et ses subordonnés, doivent employer à cette fin tous les moyens efficaces, sans aucune exception, et en y sacrifiant au besoin leurs propres personnes, leur souverain, le peuple, les pays étrangers, et toute espèce d'obligation.\par
C'est, avec beaucoup plus de grandeur, la doctrine de Maurras : « Politique d'abord. » Mais Maurras, très logiquement, est athée. Ce cardinal, en posant comme un absolu une chose dont toute la réalité réside ici-bas, commettait le crime d'idolâtrie. D'ailleurs le métal, la pierre et le bois ne sont pas vraiment dangereux. L'objet du véritable crime d'idolâtrie est toujours quelque chose d'analogue à l'État. C'est ce crime que le diable a proposé au Christ en lui offrant les royaumes de ce monde. Le Christ a refusé. Richelieu a accepté. Il a eu sa récompense. Mais il a toujours cru n'agir que par dévouement, et, en un sens c'était vrai.\par
Son dévouement à l'État a déraciné la France. Sa politique était de tuer systématiquement toute vie spontanée dans le pays, pour empêcher que quoi que ce soit pût s’opposer à l'État. Si son action en ce sens semble avoir eu des limites, c'est qu'il commençait et qu'il était assez habile pour procéder graduellement. Il suffit de lire les dédicaces de Corneille pour sentir à quel degré de servilité ignoble il avait su abaisser les esprits. Depuis, pour préserver de la honte nos gloires nationales, on a imaginé de dire que c'était simplement le langage de politesse de l'époque. Mais c'est un mensonge. Pour s'en convaincre, il n'y a qu'à lire les écrits de Théophile de Viau. Seulement Théophile est mort prématurément des conséquences d'un emprisonnement arbitraire au lieu que Corneille a vécu très vieux.\par
La littérature n'a d'intérêt que comme signe, mais elle est un signe qui ne trompe pas. Le langage servile de Corneille montre que Richelieu voulait asservir les esprits eux-mêmes. Non pas à sa personne, car dans son abnégation de soi-même il était probablement sincère, mais à l'État représenté par lui. Sa conception de l'État était déjà totalitaire. Il l'a appliquée autant qu'il pouvait en soumettant le pays, dans toute la mesure où le permettaient les moyens de son temps, à un régime policier. Il a ainsi détruit une grande partie de la vie morale du pays. Si la France s'est soumise à cet étouffement, c'est que les nobles l'avaient tellement désolée de guerres civiles absurdes et atrocement cruelles qu'elle a accepté d'acheter la paix civile à ce prix.\par
Après l'explosion de la Fronde, qui en ses débuts, par bien des points, annonçait 1789, Louis XIV s'installa au pouvoir dans un esprit de dictateur bien plutôt que de souverain légitime. C'est ce qu'exprime sa phrase : « L'État c'est moi. »» Ce n'est pas là une pensée de roi. Montesquieu a très bien expliqué cela, à mots couverts. Mais ce qu'il ne pouvait encore apercevoir à son époque, c'est qu'il y a eu deux étapes dans la déchéance de la monarchie française. La monarchie après Charles V a dégénéré en despotisme personnel. Mais à partir de Richelieu, elle a été remplacée par une machine d'État à tendances totalitaires, qui, comme le dit Marx, non seulement a subsisté à travers tous les changements, mais a été perfectionnée et accrue par chaque changement de régime.\par
Pendant la Fronde et sous Mazarin, la France, malgré la détresse publique, a respiré moralement. Louis XIV l'a trouvée pleine de génies brillants qu'il a reconnus et encouragés. Mais en même temps il a continué, avec un degré d'intensité bien plus élevé, la politique de Richelieu. Il a ainsi réduit la France, en très peu de temps, à un état moralement désertique, sans parler d'une atroce misère matérielle.\par
Si on lit Saint-Simon, non pas à titre de curiosité littéraire et historique, mais comme un document sur la vie que des êtres humains ont réellement vécue, on est pris d'horreur et de dégoût devant une telle intensité de mortel ennui, une bassesse si générale d'âme, de cœur et d'intelligence. La Bruyère, les lettres de Liselotte, tous les documents de l'époque, lus dans le même esprit, donnent la même impression. En remontant même un peu plus haut, on devrait bien penser, par exemple, que Molière n'a pas écrit le Misanthrope pour s'amuser.\par
Le régime de Louis XIV était vraiment déjà totalitaire. La terreur, les dénonciations ravageaient le pays. L'idolâtrie de l'État, représenté par le souverain, était organisée avec une impudence qui était un défi à toutes les consciences chrétiennes. L'art de la propagande était déjà très bien connu, comme le montre l'aveu naïf du chef de la police à Liselotte concernant l'ordre de ne laisser paraître aucun livre sur aucun sujet, qui ne contînt l'éloge outré du roi.\par
Sous ce régime, le déracinement des provinces françaises, la destruction de la vie locale, atteignit un degré bien plus élevé. Le XVIII\textsuperscript{e} siècle fut une accalmie. L'opération par laquelle la Révolution substitua au roi la souveraineté nationale n'avait qu'un inconvénient, c'est que la souveraineté nationale n'existait pas. Comme pour la jument de Roland, c'était là son seul défaut. Il n'existait en fait aucun procédé connu pour susciter quelque chose de réel correspondant à ces mots. Dès lors il ne restait que l'État, au bénéfice de qui tournait naturellement la ferveur pour l'unité – « unité ou la mort » – surgie autour de la croyance à la souveraineté nationale. D'où nouvelles destructions dans le domaine de la vie locale. La guerre aidant – la guerre est dès le début le ressort de toute cette histoire – l'État, sous la Convention et l'Empire, devint de plus en plus totalitaire.\par
Louis XIV avait dégradé l'Église française en l'associant au culte de sa personne et en lui imposant l'obéissance même en matière de religion. Cette servilité de l'Église envers le souverain fut pour beaucoup dans l'anticléricalisme du siècle suivant.\par
Mais quand l'Église commit l'erreur irréparable d'associer son sort à celui des institutions monarchiques, elle se coupa de la vie publique. Rien ne pouvait mieux servir les aspirations totalitaires de l'État. Il devait en résulter le système laïque, prélude à l'adoration avouée de l'État comme tel en faveur aujourd'hui.\par
Les chrétiens sont sans défense contre l'esprit laïque. Car ou ils se donnent entièrement à une action politique, une action de parti, pour remettre le pouvoir temporel aux mains d'un clergé, ou de l'entourage d'un clergé ; ou bien ils se résignent à être eux-mêmes irréligieux dans toute la partie profane de leur propre vie, ce qui est généralement le cas aujourd'hui, à un degré bien plus élevé que les intéressés eux-mêmes n'en ont conscience. Dans les deux cas est abandonnée la fonction propre de la religion, qui consiste à imprégner de lumière toute la vie profane, publique et privée, sans jamais aucunement la dominer.\par
Pendant le XIX\textsuperscript{e} siècle, les chemins de fer firent d'affreux ravages dans le sens du déracinement. George Sand voyait encore dans le Berry des coutumes peut-être vieilles de beaucoup de milliers d'années, dont le souvenir même aurait disparu sans les notes sommaires qu'elle a prises.\par
La perte du passé, collective ou individuelle, est la grande tragédie humaine, et nous avons jeté le nôtre comme un enfant déchire une rose. C'est avant tout pour éviter cette perte que les peuples résistent désespérément à la conquête.\par
Mais le phénomène totalitaire de l'État est constitué par une conquête que les pouvoirs publics exécutent sur les peuples dont ils ont la charge, sans pouvoir leur éviter les malheurs dont toute conquête est accompagnée, afin d'avoir un meilleur instrument pour la conquête extérieure. C’est ainsi que se sont passées les choses jadis en France et plus récemment en Allemagne, sans compter la Russie.\par
Mais le développement de l'État épuise le pays. L'État mange la substance morale du pays, en vit, s'en engraisse, jusqu'à ce que la nourriture vienne à s'épuiser, ce qui le réduit à la langueur par la famine. La France en était arrivée là. En Allemagne au contraire, la centralisation étatique est toute récente, de sorte que l'État y possède toute l'agressivité que donne une surabondance de nourriture de haute qualité énergétique. Quant à la Russie, la vie populaire y a un tel degré d'intensité qu'on se demande si, en fin de compte, ce ne sera pas le peuple qui mangera l'État, ou plutôt le résorbera.\par
La III\textsuperscript{e} République, en France, était une chose bien singulière ; un de ses traits les plus singuliers est que toute sa structure, hors le jeu même de la vie parlementaire, provenait de l'Empire. Le goût des Français pour la logique abstraite les rend très susceptibles d’être dupés par des étiquettes. Les Anglais ont un royaume à contenu républicain ; nous avions une République à contenu impérial. Encore l'Empire lui-même se rattache-t-il, par-dessus la Révolution, par des liens sans discontinuité, à la monarchie ; non pas l'antique monarchie française, mais la monarchie totalitaire, policière du XVII\textsuperscript{e} siècle.\par
Le personnage de Fouché est un symbole de cette continuité. L'appareil de répression de l'État français a mené à travers tous les changements une vie sans trouble ni interruption, avec une capacité d'action toujours accrue.\par
De ce fait, l'État en France était resté l'objet des rancunes, des haines, de la répulsion, excitées jadis par une royauté tournée en tyrannie. Nous avons vécu ce paradoxe, d'une étrangeté telle qu'on ne pouvait même pas en prendre conscience : une démocratie où toutes les institutions publiques, ainsi que tout ce qui s'y rapporte, étaient ouvertement haïes et méprisées par toute la population.\par
Aucun Français n'avait le moindre scrupule à voler ou escroquer l'État en matière de douanes, d'impôt, de subventions, ou en toute autre matière. Il faut excepter certains milieux de fonctionnaires ; mais eux faisaient partie de la machine publique. Si les bourgeois allaient beaucoup plus loin que le reste du pays dans les opérations de ce genre, c'est uniquement parce qu'ils avaient beaucoup plus d'occasions. La police est en France l'objet d'un mépris tellement profond que pour beaucoup de Français ce sentiment fait partie de la structure morale éternelle de l'honnête homme. Guignol est du folklore français authentique, qui remonte à l'Ancien Régime et n'a pas vieilli. L'adjectif policier constitue en français une des injures les plus sanglantes, dont il serait curieux de savoir s'il y a des équivalents dans d'autres langues. Or la police n'est pas autre chose que l'organe d'action des pouvoirs publics. Les sentiments du peuple français à l'égard de cet organe sont restés les mêmes qu'au temps où les paysans étaient obligés, comme le constate Rousseau, de cacher qu'ils possédaient un peu de jambon.\par
De même tout le jeu des institutions politiques était un objet de répulsion, de dérision et de mépris. Le mot même de politique s'était chargé d'une intensité de signification péjorative incroyable dans une démocratie. « C'est un politicien », « tout cela, c'est de la politique » ; ces phrases exprimaient des condamnations sans appel. Aux yeux d'une partie des Français, la profession même de parlementaire – car c'était une profession – avait quelque chose d'infamant. Certains Français étaient fiers de s'abstenir de tout contact avec ce qu'ils nommaient « la politique », excepté le jour des élections, ou y compris ce jour ; d'autres regardaient leur député comme une espèce de domestique, un être créé et mis au monde pour servir leur intérêt particulier. Le seul sentiment qui tempérât le mépris des affaires publiques était l'esprit de parti, chez ceux du moins que cette maladie avait contaminés.\par
On chercherait vainement un aspect de la vie publique, qui ait excité chez les Français le plus léger sentiment de loyauté, de gratitude ou d'affection. Aux beaux temps de l'enthousiasme laïque, il y avait eu l'enseignement ; mais depuis longtemps l'enseignement n'est plus, aux yeux des parents comme des enfants, qu'une machine à procurer des diplômes, c'est-à-dire des situations. Quant aux lois sociales, jamais le peuple français, dans la mesure où il en était satisfait, ne les a regardées comme autre chose que comme des concessions arrachées à la mauvaise volonté des pouvoirs publics par une pression violente.\par
Aucun autre intérêt ne tenait lieu de celui qui manquait aux affaires publiques. Chacun des régimes successifs ayant détruit à un rythme plus rapide la vie locale et régionale, elle avait finalement disparu. La France était comme ces malades dont les membres sont déjà froids et dont le cœur seul palpite encore. Presque nulle part il n'y avait de pulsation de vie, excepté à Paris ; dès la banlieue qui entourait la ville, la mort morale commençait à peser.\par
À cette époque extérieurement paisible d'avant la guerre, l'ennui des petites villes de province françaises constituait peut-être une cruauté aussi réelle que des atrocités plus visibles. Des êtres humains condamnés à passer ces années uniques, irremplaçables, entre le berceau et la tombe, dans un morne ennui, n'est-ce pas aussi atroce que la faim ou les massacres ? C'est Richelieu qui a commencé à jeter cette brume d'ennui sur la France, et elle n'a pas cessé depuis de devenir de plus en plus irrespirable. Au moment de la guerre, cela avait atteint le degré de l'asphyxie.\par
Si l'État a tué moralement tout ce qui était, territorialement parlant, plus petit que lui, il a aussi transformé les frontières territoriales en murs de prison pour enfermer les pensées. Dès qu'on regarde l'histoire d'un peu près, et hors des manuels, on est stupéfait de voir combien certaines époques presque dépourvues de moyens matériels de communication dépassaient la nôtre pour la richesse, la variété, la fécondité, l'intensité de vie dans les échanges de pensées à travers les plus vastes territoires. C'est le cas du Moyen Âge, de l’Antiquité pré-romaine, de la période immédiatement antérieure aux temps historiques. De nos jours, avec la T. S. F., l'aviation, le développement des transports de toute espèce, l'imprimerie, la presse, le phénomène moderne de la nation enferme en petits compartiments séparés même une chose aussi naturellement universelle que la science. Les frontières, bien entendu, ne sont pas infranchissables ; mais de même que pour voyager il faut en passer par une infinité de formalités ennuyeuses et pénibles, de même tout contact avec une pensée étrangère, dans n'importe quel domaine, demande un effort mental pour passer la frontière. C'est un effort considérable, et beaucoup de gens ne consentent pas à le fournir. Même chez ceux qui le fournissent, le fait qu'un effort est indispensable empêche que des liens organiques puissent être noués par-dessus les frontières.\par
Il est vrai qu'il existe des Églises et des partis internationaux. Mais quant aux Églises, elles présentent le scandale intolérable de prêtres et de fidèles demandant à Dieu en même temps, avec les mêmes rites, les mêmes paroles, et, il faut le supposer, un degré égal de foi et de pureté de cœur, la victoire militaire pour l'un ou l'autre de deux camps ennemis. Ce scandale date de loin ; mais dans notre siècle la vie religieuse est subordonnée à celle de la nation plus qu'elle ne l'a jamais été. Quant aux partis, ou ils ne sont internationaux que par fiction, ou l'internationalisme y a la forme de la subordination totale à une certaine nation.\par
Enfin l'État a également supprimé tous les liens qui pouvaient, en dehors de la vie publique, donner une orientation à la fidélité. Autant la Révolution française, en supprimant les corporations, a favorisé le progrès technique, autant moralement elle a fait de mal, ou du moins elle a consacré, achevé, un mal déjà partiellement accompli. On ne saurait trop répéter qu'aujourd’hui, quand on emploie ce mot, dans quelque milieu que ce soit, ce dont il s'agit n'a rien de commun avec les corporations.\par
Une fois les corporations disparues, le travail est devenu, dans la vie individuelle des hommes, un moyen ayant pour fin correspondante l'argent. Il y a quelque part, dans les textes constitutifs de la Société des Nations, une phrase affirmant que le travail désormais ne serait plus une marchandise. C'était une plaisanterie du dernier mauvais goût. Nous vivons dans un siècle où quantité de braves gens, qui pensent être très loin de ce que Lévy-Bruhl nommait la mentalité pré-logique, ont cru à l'efficacité magique de la parole bien plus qu'aucun sauvage du fond de l'Australie. Quand on retire de la circulation commerciale un produit indispensable, on prévoit pour lui un autre mode de distribution. Rien de tel n'a été prévu pour le travail qui, bien entendu, est demeuré une marchandise.\par
Dès lors la conscience professionnelle est simplement une modalité de la probité commerciale. Dans une société fondée sur les échanges, le plus grand poids de réprobation sociale tombe sur le vol et l'escroquerie, et notamment sur l'escroquerie du marchand qui vend de la marchandise avariée en garantissant qu'elle est bonne. De même, quand on vend son travail, la probité exige que l'on fournisse une marchandise d'une qualité qui réponde au prix. Mais la probité n'est pas la fidélité. Une très grande distance sépare ces deux vertus.\par
Il y a un fort élément de fidélité dans la camaraderie ouvrière qui a été longtemps le mobile dominant de la vie syndicale. Mais plusieurs obstacles ont empêché cette fidélité de constituer un support solide de la vie morale. D'un côté, le mercantilisme de la vie sociale s'est étendu aussi au mouvement ouvrier, en mettant les questions de sous au premier plan ; or plus les soucis d'argent dominent, plus l'esprit de fidélité disparaît. D'autre part, dans la mesure où le mouvement ouvrier est révolutionnaire, il échappe à cet inconvénient, mais contracte les faiblesses inhérentes à toute rébellion.\par
Richelieu, dont certaines observations sont si prodigieusement lucides, dit avoir reconnu par expérience que, toutes choses égales d'ailleurs, les rebelles sont toujours moitié moins forts que les défenseurs du pouvoir officiel. Même si l'on pense soutenir une bonne cause, le sentiment d'être en rébellion affaiblit. Sans un mécanisme psychologique de ce genre, il ne pourrait y avoir aucune stabilité dans les sociétés humaines. Ce mécanisme explique l'emprise du parti communiste. Les ouvriers révolutionnaires sont trop heureux d'avoir derrière eux un État – un État qui donne à leur action ce caractère officiel, cette légitimité, cette réalité, que l'État seul confère, et qui en même temps est situé trop loin d'eux, géographiquement, pour pouvoir les dégoûter. De la même manière les Encyclopédistes, profondément mal à l'aise d'être en conflit avec leur propre souverain, avaient soif de la faveur des souverains de Prusse ou de Russie. On peut aussi comprendre par cette analogie que des militants ouvriers plus ou moins révolutionnaires qui avaient résisté au prestige de la Russie n'aient pas pu s’empêcher de succomber à celui de l'Allemagne.\par
Hors ceux qui se sont donnés tout entiers au parti communiste, les ouvriers ne peuvent pas trouver à la fidélité envers leur classe un objet assez précis, assez nettement délimité, pour en recevoir la stabilité intérieure. Peu de notions sont aussi indéterminées que celle de classe sociale. Marx, qui fait reposer sur elle tout son système, n'a jamais cherché à la définir, ni même simplement à l'étudier. Le seul renseignement qu'on puisse tirer de ses ouvrages concernant les classes sociales, c'est que ce sont des choses qui luttent. Ce n'est pas suffisant. Ce n'est pas non plus une de ces notions qui, sans pouvoir être définies en paroles, sont claires pour la pensée. Il est encore plus difficile de la concevoir ou de la sentir sans définition que de la définir.\par
La fidélité impliquée par une affiliation religieuse compte également assez peu, si singulier que ce soit, dans la vie moderne. Malgré des différences évidentes et considérables, un effet en un sens analogue est produit par le système anglais de l'Église nationale et par le système français de la séparation des Églises et de l'État. Seulement le second semble plus destructeur.\par
La religion a été proclamée une affaire privée. Selon les habitudes d'esprit actuelles, cela ne veut pas dire qu'elle réside dans le secret de l'âme, dans ce lieu profondément caché où même la conscience de chacun ne pénètre pas. Cela veut dire qu'elle est affaire de choix, d'opinion, de goût, presque de fantaisie quelque chose comme le choix d'un parti politique ou même comme le choix d'une cravate ; ou encore qu'elle est affaire de famille, d'éducation, d'entourage. Étant devenue une chose privée, elle perd le caractère obligatoire réservé aux choses publiques, et par suite n’a plus de titre incontesté à la fidélité.\par
Quantité de paroles révélatrices montrent qu'il en est ainsi. Combien de fois, par exemple, n'entend-on pas répéter ce lieu commun : « Catholiques, protestants, juifs ou libres penseurs, nous sommes tous Français », exactement comme s'il s'agissait de petites fractions territoriales du pays, comme on dirait : « Marseillais, Lyonnais ou Parisiens, nous sommes tous Français. » Dans des textes émanés du pape, on peut lire : « Non seulement du point de vue chrétien, mais plus généralement du point de vue humain... » ; comme si le point de vue chrétien, qui ou bien n'a aucun sens, ou bien prétend envelopper toutes choses dans ce monde et dans l'autre, avait un degré de généralité moindre que le point de vue humain. On ne peut concevoir un aveu de faillite plus terrible. Voilà comment se paient les « anathema sit ». En fin de compte, la religion, dégradée au rang d'affaire privée, se réduit au choix d'un lieu où aller passer une heure ou deux, le dimanche matin.\par
Ce qu'il y a là de comique, c'est que la religion, c'est-à-dire la relation de l'homme avec Dieu, n’est pas regardée aujourd'hui comme une chose trop sacrée pour l'intervention d'aucune autorité extérieure, mais est mise au nombre des choses que l'État laisse à la fantaisie de chacun, comme, étant de peu d'importance au regard des affaires publiques. Du moins il en a été ainsi dans un passé récent. C'est là la signification actuelle du mot de « tolérance ».\par
Ainsi il n'y a rien, hors l'État, où la fidélité puisse s'accrocher. C'est pourquoi jusqu’à 1940 elle ne lui avait pas été refusée. Car l'homme sent qu'une vie humaine sans fidélité est quelque chose de hideux. Parmi la dégradation générale de tous les mots du vocabulaire français qui ont rapport à des notions morales, les mots de traître et de trahison n'ont rien perdu de leur force. L'homme sent aussi qu'il est né pour le sacrifice ; et il ne restait plus dans l'imagination publique d'autre forme de sacrifice que le sacrifice militaire, c'est-à-dire offert à l'État.\par
Il s'agissait bien uniquement de l'État. L'illusion de la Nation, au sens où les hommes de 1789, de 1792, prenaient ce mot, qui faisait alors couler des larmes de joie, c'était là du passé complètement aboli. Le mot même de nation avait changé de sens. En notre siècle, il ne désigne plus le peuple souverain, mais l'ensemble des populations reconnaissant l'autorité d'un même État ; c'est l'architecture formée par un État et le pays dominé par lui. Quand on parle de souveraineté de la nation, aujourd'hui, cela veut dire uniquement souveraineté de l’État. Un dialogue entre un de nos contemporains et un homme de 1792 mènerait à des malentendus bien comiques. Or non seulement l'État en question n'est pas le peuple souverain, mais il est identiquement ce même État inhumain, brutal, bureaucratique, policier, légué par Richelieu à Louis XIV, par Louis XIV à la Convention, par la Convention à l'Empire, par l'Empire à la III\textsuperscript{e} République. Qui plus est, il est instinctivement connu et haï comme tel.\par
Ainsi on a vu cette chose étrange, un État, objet de haine, de répulsion, de dérision, de mépris et de crainte, qui, sous le nom de patrie, a réclamé la fidélité absolue, le don total, le sacrifice suprême, et les a obtenus, de 1914 à 1918, à un point qui a dépassé toute attente. Il se posait comme un absolu ici-bas, c'est-à-dire comme un objet d'idolâtrie ; et il a été accepté et servi comme tel, honoré d'une quantité effroyable de sacrifices humains. Une idolâtrie sans amour, quoi de plus monstrueux et de plus triste ?\par
Quand quelqu'un va dans le dévouement beaucoup plus loin que son cœur ne le pousse, il se produit inévitablement par la suite une réaction violente, une sorte de révulsion dans les sentiments. Cela se voit souvent dans les familles, quand un malade a besoin de soins qui dépassent l'affection qu'il inspire. Il est l'objet d'une rancune refoulée parce qu'inavouable, mais toujours présente comme un poison secret.\par
La même chose s'est produite entre les Français et la France, après 1918. Ils lui avaient trop donné. Ils lui avaient donné davantage qu'ils n'avaient dans le cœur pour elle.\par
Tout le courant d'idées antipatriotiques, pacifistes, internationalistes d'après 1918 s'est réclamé des morts de la guerre et des anciens combattants ; et quant à ceux-ci, il émanait réellement dans une large mesure de leurs milieux. Il y avait aussi, il est vrai, des associations d'anciens combattants intensément patriotiques. Mais l'expression de leur patriotisme sonnait creux et manquait tout à fait de force persuasive. Elle ressemblait au langage de gens qui, ayant trop souffert, éprouvent continuellement le besoin de se rappeler qu'ils n'ont pas souffert pour rien. Car des souffrances trop grandes par rapport aux impulsions du cœur peuvent pousser à l'une ou l'autre attitude ; ou on repousse violemment ce à quoi on a trop donné, ou on s'y accroche avec une sorte de désespoir.\par
Rien n'a fait plus de mal au patriotisme que l'évocation, répétée à satiété, du rôle joué par la police derrière les champs de bataille. Rien ne pouvait blesser davantage les Français, en les forçant à constater, derrière la patrie, la présence de cet État policier, objet traditionnel de leur haine. En même temps les extraits de la presse extravagante d'avant 1918, relus après coup dans le sang-froid et avec dégoût, rapprochés de ce rôle de la police, leur donnait l'impression d'avoir été roulés. Il n'est rien qu'un Français soit moins capable de pardonner. Les mots mêmes qui exprimaient le sentiment patriotique ayant été discrédités, il passait en un sens dans la catégorie des sentiments inavouables. Il y a eu un temps, et il n'est pas loin, où l'expression d'un sentiment patriotique dans les milieux ouvriers, du moins dans certains d'entre eux, aurait fait l'effet d'un manquement aux convenances.\par
\par
Des témoignages concordants affirment que les plus courageux, en 1940, ont été les anciens combattants de l'autre guerre. Il faut seulement en conclure que leurs réactions d'après 1918 ont eu une influence plus profonde sur l'âme des enfants qui les entouraient que sur la leur propre. C'est un phénomène très fréquent et facile à comprendre. Ceux qui avaient dix-huit ans en 1914 ont eu leur caractère formé au cours des années antérieures.\par
On a dit que l'école du début du siècle avait forgé une jeunesse pour la victoire, et que celle d'après 1918 a fabriqué une génération de vaincus. Il y a certainement là beaucoup de vrai. Mais les maîtres d'école d'après 1918 étaient des anciens combattants. Beaucoup des enfants qui ont eu dix ans entre 1920 et 1930 ont eu des instituteurs qui avaient fait la guerre.\par
Si la France a subi l'effet de cette réaction plus que d'autres pays, cela est dû à un déracinement beaucoup plus aigu, correspondant à une centralisation étatique bien plus ancienne et plus intense, à l'effet démoralisant de la victoire, et à la licence accordée à toutes les propagandes.\par
Il y a eu aussi rupture d'équilibre, et compensation par rupture en sens inverse, autour de la notion de patrie, dans le domaine de la pure pensée. Du fait que l'État était demeuré, au milieu d'un vide total, la seule chose qualifiée pour demander à l'homme la fidélité et le sacrifice, la notion de patrie se posait comme un absolu dans la pensée. La patrie était hors du bien et du mal. C'est ce qu'exprime le proverbe anglais « right or wrong, my country ». Mais souvent on va plus loin. On n’admet pas que la patrie puisse avoir tort.\par
Si peu enclins que soient les hommes de tous les milieux à l'effort de l'examen critique, une absurdité éclatante, même s'ils ne la reconnaissent pas, les met dans un état de malaise qui affaiblit l'âme. Il n'y a au fond rien de plus mélangé à la vie humaine commune et quotidienne que la philosophie, mais une philosophie implicite.\par
Poser la patrie comme un absolu que le mal ne peut souiller est une absurdité éclatante. La patrie est un autre nom de la nation ; et la nation est un ensemble de territoires et de populations assemblés par des événements historiques où le hasard a une grande part, autant que l'intelligence humaine peut en juger, et où se mélangent toujours le bien et le mal. La nation est un fait, et un fait n'est pas un absolu. Elle est un fait parmi d'autres analogues. Il y a plus d'une nation sur la surface de la terre. La nôtre est certes unique. Mais chacune des autres, considérée en elle-même et avec amour, est unique au même degré.\par
Il était de mode avant 1940 de parler de la « France éternelle ». Ces mots sont une espèce de blasphème. On est obligé d'en dire autant de pages si touchantes écrites par de grands écrivains catholiques français sur la vocation de la France, le salut éternel de la France, et autres thèmes semblables. Richelieu voyait bien plus juste quand il disait que le salut des États ne s'opère qu’ici-bas. La France est une chose temporelle, terrestre. Sauf erreur, il n'a jamais été dit que le Christ soit mort pour sauver des nations. L'idée d'une nation appelée par Dieu en tant que nation n'appartient qu'à l'ancienne loi.\par
L'Antiquité dite païenne n'aurait jamais commis une confusion si grossière. Les Romains se croyaient élus, mais uniquement pour une domination terrestre. L'autre monde ne les intéressait pas. Nulle part il n'apparaît qu'aucune cité, aucun peuple, se soit cru élu pour une destinée surnaturelle. Les Mystères, qui constituaient en quelque sorte la méthode officielle du salut, comme aujourd'hui les Églises, étaient des institutions locales, mais on reconnaissait qu'ils étaient équivalents entre eux. Platon décrit comment l'homme secouru par la grâce sort de la caverne de ce monde ; mais il ne dit pas qu'une cité puisse en sortir. Au contraire, il représente la collectivité comme quelque chose d'animal qui empêche le salut de l'âme.\par
On accuse souvent l'Antiquité de n'avoir su reconnaître que les valeurs collectives. En réalité, cette erreur n'a été commise que par les Romains, qui étaient athées, et par les Hébreux ; et par ceux-ci, seulement jusqu'à l'exil à Babylone. Mais si nous avons tort d'attribuer cette erreur à l'Antiquité pré-chrétienne, nous avons tort aussi de ne pas reconnaître que nous la commettons continuellement, corrompus que nous sommes par la double tradition romaine et hébraïque, qui l'emporte trop souvent en nous sur l'inspiration chrétienne pure.\par
Les chrétiens aujourd'hui sont gênés pour reconnaître que, si l'on donne au mot de patrie le sens le plus fort possible, un sens complet, un chrétien n'a qu'une seule patrie qui est située hors de ce monde. Car il n'a qu'un père, qui habite hors de ce monde. « Constituez-vous des trésors dans le ciel... car où est le trésor d'un homme, là sera aussi son cœur. » Il est donc interdit d'avoir son cœur sur terre.\par
Les chrétiens aujourd'hui n'aiment pas poser la question des droits respectifs, sur leur cœur, de Dieu et de leur pays. Les évêques allemands ont terminé une de leurs protestations les plus courageuses en disant qu'ils se refusaient à avoir jamais à choisir entre Dieu et l'Allemagne. Et pourquoi s'y refusent-ils ? Il peut toujours se produire des circonstances impliquant un choix à faire entre Dieu et n'importe quoi de terrestre, et le choix ne doit jamais être douteux. Mais les évêques français auraient tenu le même langage. La popularité de Jeanne d'Arc au cours du dernier quart de siècle n'était pas quelque chose d'entièrement sain ; c'était une ressource commode pour oublier qu'il y a une différence entre la France et Dieu. Pourtant cette lâcheté intérieure devant le prestige de l'idée de patrie n'a pas rendu le patriotisme plus énergique. La statue de Jeanne d'Arc se trouvait placée de manière à attirer les regards, dans toutes les églises du pays, pendant ces jours affreux où les Français ont abandonné la France.\par
« Si quelqu'un vient vers moi et ne hait pas son père et sa mère et sa femme et ses enfants et ses frères et ses sœurs, et de plus sa propre âme, il ne peut pas être mon disciple. » S'il est prescrit de haïr tout cela, en un certain sens du mot hair, il est certainement interdit d'aimer son pays, en un certain sens du mot aimer. Car l'objet propre de l'amour, c'est le bien, et « Dieu seul est bon ».\par
Ce sont là des évidences, mais, par quelque sortilège, complètement méconnues dans notre siècle. Autrement il aurait été impossible qu'un homme comme le Père de Foucauld, qui avait choisi d'être par la charité le témoin du Christ au milieu de populations non chrétiennes, se crût en même temps le droit de fournir des renseignements au 2\textsuperscript{e} Bureau au sujet de ces mêmes populations.\par
Il serait sain pour nous de méditer les terribles paroles du diable au Christ, montrant tous les royaumes de ce monde et disant à leur sujet : « Toute puissance m'a été abandonnée. » Aucun d'eux n'est excepté.\par
Ce qui n'a pas choqué les chrétiens a choqué les ouvriers. Une tradition encore assez récente pour n'être pas tout à fait morte fait de l'amour de la justice l'inspiration centrale du mouvement ouvrier français. Dans la première moitié du XIX\textsuperscript{e} siècle, c'était un amour brûlant, qui prenait fait et cause pour les opprimés du monde entier.\par
Tant que la patrie était le peuple constitué en nation souveraine, aucun problème ne se posait sur ses rapports avec la justice. Car on admettait – tout à fait arbitrairement, et par une interprétation très superficielle du {\itshape Contrat Social} – qu'une nation souveraine ne commet pas d'injustice envers ses membres ni envers ses voisins ; on supposait que les causes qui produisent l'injustice sont toutes liées à la non-souveraineté de la nation.\par
Mais dès lors que derrière la patrie il y a le vieil État, la justice est loin. Dans l'expression du patriotisme moderne, il n'est pas beaucoup question de la justice, et surtout il n'est rien dit qui puisse permettre de penser les relations entre la patrie et la justice. On n'ose pas affirmer qu'il y ait équivalence entre les deux notions ; on n'oserait pas, notamment, l'affirmer aux ouvriers, qui, à travers l’oppression sociale, sentent le froid métallique de l'État, et se rendent compte confusément que le même froid doit exister dans les relations internationales. Quand on parle beaucoup de la patrie, on parle peu de la justice ; et le sentiment de la justice est si puissant chez les ouvriers, fussent-ils matérialistes, du fait qu'ils ont toujours l'impression d'être privés d'elle, qu'une forme d'éducation morale où la justice ne figure presque pas ne peut pas avoir prise sur eux. Quand ils meurent pour la France, ils ont toujours besoin de sentir qu'ils meurent en même temps pour quelque chose de beaucoup plus grand, qu'ils ont part à la lutte universelle contre l'injustice. Pour eux, selon une parole devenue célèbre, la patrie ne suffit pas.\par
Il en est de même partout où brûle une flamme, une étincelle, fût-elle imperceptible, de vie spirituelle véritable. À ce feu, la patrie ne suffit pas. Et pour ceux chez qui elle est absente, le patriotisme, dans ses suprêmes exigences, est bien trop élevé ; il ne peut alors constituer un stimulant assez fort que sous la forme du plus aveugle fanatisme national.\par
Il est vrai que les hommes sont capables de diviser leur âme en compartiments, dans chacun desquels une idée a une espèce de vie sans relation avec les autres. Ils n'aiment ni l'effort critique ni l'effort de synthèse, et ne se les imposent pas sans violence.\par
Mais dans la peur, l'angoisse, quand la chair recule devant la mort, devant la trop grande souffrance, devant l'excès du danger, il apparaît dans l'âme de tout homme, fût-il tout à fait inculte, un fabricateur de raisonnements qui élabore des preuves pour établir qu'il est légitime et bon de se soustraire à cette mort, à cette souffrance, à ce danger. Ces preuves peuvent, selon les cas, être bonnes ou mauvaises. De toutes manières, sur le moment, le désarroi de la chair et du sang leur imprime une intensité de force persuasive qu'aucun orateur n'a jamais obtenue.\par
Il y a des gens chez qui les choses ne se passent pas ainsi. C'est ou bien que leur nature les soustrait à la peur, que leur chair, leur sang et leurs entrailles sont insensibles à la présence de la mort ou de la douleur ; ou bien qu'il y a un tel degré d'unité dans leur âme que ce fabricateur de raisonnements n'a pas la possibilité d'y travailler. Chez d'autres encore il travaille, il fait sentir sa persuasion, mais elle est pourtant méprisée. Cela même suppose soit un degré déjà élevé d'unité intérieure soit des stimulants extérieurs puissants.\par
L'observation géniale d'Hitler sur la propagande, à savoir que la force brutale ne peut pas l’emporter sur des idées si elle est seule, mais qu’elle y parvient aisément, en s'adjoignant quelques idées d'aussi basse qualité qu'on voudra, cette observation fournit aussi la clef de la vie intérieure. Les tumultes de la chair, si violents soient-ils, ne peuvent pas l'emporter dans l'âme sur une pensée, s’ils sont seuls. Mais leur victoire est facile s'ils communiquent leur puissance persuasive à une autre pensée, si mauvaise soit-elle. C'est ce point qui est important. Aucune pensée n'est de qualité trop médiocre pour cette fonction d'alliée de la chair. Mais il faut à la chair de la pensée pour alliée.\par
C'est pourquoi, alors qu'en temps ordinaire les gens, même cultivés, vivent, sans aucun malaise, avec les plus énormes contradictions intérieures, dans les moments de crise suprême, la moindre faille dans le système intérieur acquiert la même importance que si le philosophe le plus lucide se tenait quelque part, malicieusement prêt à en profiter ; et il en est ainsi chez tout homme, si ignorant soit-il.\par
Dans les moments suprêmes, qui ne sont pas nécessairement ceux du plus grand danger, mais ceux où l'homme se trouve, devant le tumulte des entrailles, du sang et de la chair, seul et sans stimulants extérieurs, ceux dont la vie intérieure procède tout entière d'une même idée sont les seuls qui résistent. C'est pourquoi les systèmes totalitaires forment des hommes à toute épreuve.\par
\par
La patrie ne peut être cette idée unique que dans un régime du genre hitlérien. Cela pourrait facilement être prouvé, jusque dans les détails, mais c'est inutile tant l'évidence est grande. Si la patrie n'est pas cette idée, et si néanmoins elle tient une place, alors ou bien il y a incohérence intérieure, et une faiblesse cachée dans l'âme, ou bien il faut qu'il y ait quelque autre idée, dominant tout le reste, et relativement à laquelle la patrie tienne une place bien clairement reconnue, place limitée et subordonnée.\par
Ce n'était pas le cas dans notre III\textsuperscript{e} République. Ce n'était le cas dans aucun milieu. Ce qui se trouvait partout, c'était l'incohérence morale. Aussi le fabricateur intérieur de raisonnements fut-il actif dans les âmes entre 1914 et 1918. La plupart résistèrent en un raidissement suprême, par cette réaction qui pousse souvent les hommes à se jeter aveuglément, par crainte de se déshonorer, du côté opposé à celui où pousse la peur. Mais l'âme, quand elle s'expose à la douleur et au danger sous l'effet de cette impulsion seulement, s'use très vite. Ces raisonnements nourris d'angoisse, qui n'ont pas pu influer sur la manière d'agir, mordent d'autant plus sur les profondeurs mêmes de l'âme, et leur influence s'exerce après coup. C'est ce qui s'est passé après 1918. Et ceux qui n'avaient rien donné et en avaient honte ont été prompts, pour d'autres motifs, à saisir la contagion. Cette atmosphère entourait les enfants à qui un peu plus tard on allait demander de mourir.\par
Combien loin est allée la désagrégation intérieure chez les Français, on peut s'en rendre compte si l'on songe qu'aujourd'hui encore l'idée de la collaboration avec l'ennemi n'a pas perdu tout prestige. D'un autre côté, si l'on cherche un réconfort dans le spectacle de la résistance, si l'on se dit que les résistants n'ont aucune difficulté à trouver leur inspiration à la fois dans le patriotisme et dans une foule d'autres mobiles, il faut en même temps se dire et se redire que la France en tant que nation se trouve en ce moment aux côtés de la justice, du bonheur général et des choses de ce genre, c'est-à-dire dans la catégorie des belles choses qui n'existent pas. La victoire alliée la sortira de cette catégorie, la rétablira dans le domaine des faits ; beaucoup de difficultés qui semblaient écartées reparaîtront. En un sens, le malheur simplifie tout. Le fait que la France est entrée dans la voie de la résistance plus lentement, plus tard que la plupart des pays occupés montre qu'on aurait tort d'être sans inquiétude pour l'avenir.\par
On peut voir clairement jusqu'où allait l'incohérence morale de notre régime si l'on songe à l'école. La morale y fait partie du programme, et mêmes les instituteurs qui n'aimaient pas en faire l'objet d'un enseignement dogmatique l'enseignaient inévitablement d'une manière diffuse. La notion centrale de cette morale, c'est la justice et les obligations qu'elle impose envers le prochain.\par
Mais quand il est question d'histoire, la morale n'intervient plus. Il n'est jamais question des obligations de la France à l'extérieur. Quelquefois on la nomme juste et généreuse, comme si c'était là un surcroît, une plume au chapeau, un couronnement à la gloire. Les conquêtes qu'elle a faites et perdues peuvent à la rigueur être l'objet d'un léger doute, comme celles de Napoléon ; jamais celles qu'elle a conservées. Le passé n'est que l'histoire de la croissance de la France, et il est admis que cette croissance est toujours un bien à tous égards. Jamais on ne se demande si en s'accroissant elle n'a pas détruit. Examiner s'il ne lui est pas peut-être arrivé de détruire des choses qui la valaient semblerait le plus affreux blasphème. Bernanos dit que les gens d'Action Française regardent la France comme un marmot à qui on ne demande que de grandir, de prendre de la chair. Mais il n'y a pas qu'eux. C'est la pensée générale qui, sans jamais être exprimée, est toujours implicite dans la manière dont on regarde le passé du pays. Et la comparaison avec un marmot est encore trop honorable. Les êtres auxquels on ne demande que de prendre de la chair, ce sont les lapins, les porcs, les poulets. Platon a le mot le plus juste en comparant la collectivité à un animal. Et ceux que son prestige aveugle, c'est-à-dire tous les hommes, hors des prédestinés, « appellent justes et belles les choses nécessaires, étant incapables de discerner et d'enseigner quelle distance il y a entre l'essence du nécessaire et celle du bien ».\par
On fait tout pour que les enfants sentent, et ils le sentent d'ailleurs naturellement, que les choses relatives à la patrie, à la nation, à l'accroissement de la nation, ont un degré d'importance qui les met à part des autres. Et c'est précisément au sujet de ces choses que la justice, les égards dus à autrui, les obligations rigoureuses assignant des limites aux ambitions et aux appétits, toute cette morale à laquelle on s'efforce de soumettre la vie des petits garçons, n'est jamais évoquée.\par
Que conclure de là, sinon qu'elle est au nombre des choses d'importance moindre, que, comme la religion, le métier, le choix d'un médecin ou d'un fournisseur, elle a sa place dans le domaine inférieur de la vie privée ?\par
Mais si la morale proprement dite est ainsi abaissée, il ne s'y substitue pas un système différent. Car le prestige supérieur de la nation est lié à l'évocation de la guerre. Il ne fournit pas de mobiles, pour le temps de paix, excepté dans un régime qui constitue une préparation permanente à la guerre, comme le régime nazi. Excepté dans un tel régime, il serait dangereux de trop rappeler que cette patrie qui demande à ses enfants leur vie a pour autre face l'État, avec ses impôts, ses douanes, sa police. On s'en abstient soigneusement ; et ainsi il ne vient à l'idée de personne que ce puisse être manquer de patriotisme que de haïr la police et de frauder en matière de douanes et d'impôt. Un pays comme l'Angleterre fait dans une certaine mesure exception, à cause d'une tradition millénaire de liberté garantie par les pouvoirs publics. Ainsi la dualité de la morale, en temps de paix, affaiblit le pouvoir de la morale éternelle sans rien mettre à sa place.\par
Cette dualité est présente d'une manière permanente, toujours, partout, et non pas seulement à l'école. Car il arrive presque journellement en temps normal à tout Français, quand il lit le journal, quand il discute en famille ou au bistrot, de penser pour la France, au nom de la France. Dès cet instant, et jusqu'à ce qu'il revienne dans son personnage privé, il perd jusqu'au souvenir des vertus dont il admet, d'une manière plus ou moins vague et abstraite, l'obligation pour lui-même. Quand il s'agit de soi-même, et même de sa famille, il est plus ou moins admis qu'il ne faut pas trop se vanter soi-même, qu'il faut se défier de ses jugements lorsqu'on est à la fois juge et partie, qu'il faut se demander si les autres n'ont pas au moins partiellement raison contre soi-même, qu'il ne faut pas trop se mettre en avant, qu'il ne faut pas penser uniquement à soi-même ; bref qu'il faut mettre des bornes à l'égoïsme et à l'orgueil. Mais en matière d'égoïsme national, d'orgueil national, non seulement il y a une licence illimitée, mais le plus haut degré possible est imposé par quelque chose qui ressemble à une obligation. Les égards envers autrui, l'aveu des torts propres, la modestie, la limitation volontaire des désirs, deviennent dans ce domaine des crimes, des sacrilèges. Parmi plusieurs paroles sublimes que le {\itshape Livre des Morts} égyptien met dans la bouche du juste après la mort, la plus touchante peut-être est celle-ci : « Je ne me suis jamais rendu sourd à des paroles justes et vraies. » Mais sur le plan international, chacun regarde comme un devoir sacré de se rendre sourd à des paroles justes et vraies, si elles sont contraires à l'intérêt de la France. Ou bien admet-on que des paroles contraires à l'intérêt de la France ne peuvent jamais être justes et vraies ? Cela reviendrait exactement au même.\par
Il y a des fautes de goût que la bonne éducation, à défaut de la morale, empêche de commettre dans la vie privée, et qui semblent absolument naturelles sur le plan national. Même les plus odieuses des dames patronnesses hésiteraient à rassembler leurs protégés pour leur exposer dans un discours la grandeur des bienfaits accordés et de la reconnaissance due en échange. Mais un gouverneur français d'Indochine n'hésite pas, au nom de la France, à tenir ce langage, même immédiatement après les actes de répression les plus atroces ou les famines les plus scandaleuses ; et il attend, il impose des réponses qui lui fassent écho.\par
C'est une coutume héritée des Romains. Ils ne commettaient jamais de cruautés, ils n'accordaient jamais de faveur, sans vanter dans les deux cas leur générosité et leur clémence. On n'était jamais reçu à leur demander quoi que ce fût, même un simple allégement à la plus horrible oppression, sans débuter par les mêmes éloges. Ils ont ainsi déshonoré la supplication, qui était honorable avant eux, en lui imposant le mensonge et la flatterie. Dans l'{\itshape Iliade}, jamais un Troyen agenouillé devant un Grec et implorant la vie ne met la plus légère nuance de flatterie dans son langage.\par
Notre patriotisme vient tout droit des Romains. C'est pourquoi les petits Français sont encouragés à en chercher l'inspiration dans Corneille. C'est une vertu païenne, si les deux mots sont compatibles. Le mot de païen, quand il est appliqué à Rome, a vraiment à titre légitime la signification chargée d'horreur que lui donnaient les premiers polémistes chrétiens. C'était vraiment un peuple athée et idolâtre ; non pas idolâtre de statues faites en pierre ou en bronze, mais idolâtre de lui-même. C'est cette idolâtrie de soi qu'il nous a léguée sous le nom de patriotisme.\par
Aussi la dualité dans la morale est-elle un scandale bien plus éclatant si, au lieu de la morale laïque, on songe à la vertu chrétienne dont la morale laïque est d'ailleurs simplement une édition pour grand public, une solution diluée. La vertu chrétienne a pour centre, pour essence, pour saveur spécifique l'humilité, le mouvement librement consenti vers le bas. C'est par là que les saints ressemblent au Christ. « Étant dans la condition de Dieu, il n'a pas regardé l'égalité avec Dieu comme butin... Il s'est vidé... Bien qu'il soit le Fils, ce qu'il a souffert lui a enseigné l'obéissance. »\par
Mais quand un Français pense à la France, l'orgueil est pour lui un devoir, selon la conception actuelle ; l'humilité serait une trahison. Cette trahison est celle peut-être qu'on reproche le plus amèrement au gouvernement de Vichy. On a raison, car son humilité est de mauvais aloi, elle est celle de l'esclave qui flatte et ment pour éviter les coups. Mais dans ce domaine une humilité qui serait de bon aloi est parmi nous chose inconnue. Nous n'en concevons même pas la possibilité. Pour parvenir seulement à en concevoir la possibilité, il nous faudrait déjà un effort d'invention.\par
Dans une âme chrétienne, la présence de la vertu païenne du patriotisme est un dissolvant. Elle est passée de Rome entre nos mains sans avoir été baptisée. Chose étrange, les barbares, ou ceux qu'on nommait ainsi, ont été baptisés presque sans difficulté lors des invasions ; mais l'héritage de la Rome antique ne l'a jamais été, sans doute parce qu'il ne pouvait pas l'être, et cela bien que l'Empire romain ait fait du christianisme une religion d'État.\par
Il serait difficile d'ailleurs d'imaginer une plus cruelle injure. Quant aux barbares, il n'est pas étonnant que les Goths soient entrés facilement dans le christianisme, si, comme le croyaient les contemporains, ils étaient du sang de ces Gêtes, les plus justes des Thraces, qu'Hérodote nommait les immortaliseurs à cause de l'intensité de leur foi dans la vie éternelle. L'héritage des barbares s'est mélangé à l'esprit chrétien pour former ce produit unique, inimitable, parfaitement homogène qu'on a nommé la chevalerie. Mais entre l'esprit de Rome et celui du Christ il n'y a jamais eu fusion. Si la fusion avait été possible, l'Apocalypse aurait menti en représentant Rome comme la femme assise sur la bête, la femme pleine des noms du blasphème.\par
La Renaissance a été une résurrection d'abord de l'esprit grec, puis de l'esprit romain. C'est dans cette seconde étape seulement qu'elle a agi comme un dissolvant du christianisme. C'est au cours de cette seconde étape qu'est née la forme moderne de la nationalité, la forme moderne du patriotisme. Corneille a eu raison de dédier son {\itshape Horace} à Richelieu, et de le faire en termes dont la bassesse est un pendant à l'orgueil presque délirant qui inspire la tragédie. Cette bassesse et cet orgueil sont inséparables ; on le voit bien aujourd'hui en Allemagne. Corneille lui-même est un excellent exemple de l'espèce d'asphyxie qui saisit la vertu chrétienne au contact de l'esprit romain. Son {\itshape Polyeucte} nous paraîtrait comique si nous n'étions pas aveuglés par l'habitude. Polyeucte, sous sa plume, est un homme qui tout d'un coup a compris qu'il y a un territoire beaucoup plus glorieux à conquérir que les royaumes terrestres, et une technique particulière pour y parvenir ; aussitôt il se met en devoir de partir pour cette conquête, sans aucun égard pour quoi que ce soit d'autre, et dans le même état d'esprit que lorsque auparavant il faisait la guerre au service de l'empereur. Alexandre pleurait, dit-on, de n'avoir à conquérir que le globe terrestre. Corneille croyait apparemment que le Christ était descendu sur terre pour combler cette lacune.\par
Si le patriotisme agit invisiblement comme un dissolvant pour la vertu soit chrétienne, soit laïque, en temps de paix, le contraire se produit en temps de guerre ; et c'est tout à fait naturel. Quand il y a dualité morale, c'est toujours la vertu exigée par les circonstances qui en subit le préjudice. La pente à la facilité donne naturellement l'avantage à l'espèce de vertu qu'en fait il n'y a pas lieu 'exercer ; à la moralité de guerre en temps de paix, à la moralité de paix en temps de guerre.\par
En temps de paix, la justice et la vérité, à cause de la cloison étanche qui les sépare du patriotisme, sont dégradées au rang des vertus purement privées, telles que par exemple la politesse ; mais quand la patrie demande le sacrifice suprême, cette même séparation prive le patriotisme de la légitimité totale qui peut seule provoquer l'effort total.\par
Quand on a pris l'habitude de considérer comme un bien absolu et clair de toute ombre cette croissance au cours de laquelle la France a dévoré et digéré tant de territoires, comment une propagande inspirée exactement de la même pensée, et mettant seulement le nom de l'Europe à la place de celui de la France, ne s'infiltrera-t-elle pas dans un coin de l'âme ? Le patriotisme actuel consiste en une équation entre le bien absolu et une collectivité correspondant à un espace territorial, à savoir la France ; quiconque change dans sa pensée le terme territorial de l'équation, et met à la place un terme plus petit, comme la Bretagne, ou plus grand, comme l'Europe, est regardé comme un traître. Pourquoi cela ? C'est tout à fait arbitraire. L'habitude nous empêche de nous rendre compte à quel point c'est arbitraire. Mais au moment suprême, cet arbitraire donne prise au fabricant intérieur de sophismes.\par
Les collaborateurs actuels \footnote{Écrit en 1943.} ont à l'égard de l'Europe nouvelle que forgerait une victoire allemande l'attitude qu'on demande aux Provençaux, aux Bretons, aux Alsaciens, aux Francs-Comtois d'avoir, quant au passé, à l'égard de la conquête de leur pays par le roi de France. Pourquoi la différence des époques changerait-elle le bien et le mal ? On entendait couramment dire entre 1918 et 1919, par les braves gens qui espéraient la paix : « Autrefois il y avait la guerre entre provinces, puis elles se sont unies en formant des nations. De la même manière les nations vont s'unir dans chaque continent, puis dans le monde entier, et ce sera la fin de toute guerre. » C'était un lieu commun très répandu ; il procédait de ce raisonnement par extrapolation qui a eu tant de puissance au XIX\textsuperscript{e} siècle et encore au XX\textsuperscript{e}. Les braves gens qui parlaient ainsi connaissaient en gros l'histoire de France, mais ils ne réfléchissaient pas, au moment où ils parlaient, que l'unité nationale s'était accomplie presque exclusivement par les conquêtes les plus brutales. Mais s'ils s'en sont souvenus en 1939, ils se sont souvenus aussi que ces conquêtes leur étaient toujours apparues comme un bien. Quoi d'étonnant si une partie au moins de leur âme s'est mise à penser : « Pour le progrès, pour l'accomplissement de l'Histoire, il faut peut-être en passer par là ? » Ils ont pu se dire : « La France a eu la victoire en 1918 ; elle n'a pu accomplir l'unité de l'Europe ; maintenant l'Allemagne essaie de l'accomplir ; ne la gênons pas. » Les cruautés du système allemand, il est vrai, auraient dû les arrêter. Mais ils pouvaient soit n'en avoir pas entendu parler, soit supposer qu'elles étaient inventées par une propagande mensongère, soit les juger de peu d'importance, comme étant infligées à des populations inférieures. Est-il plus difficile d'ignorer les cruautés des Allemands envers les Juifs ou les Tchèques que celles des Français envers les Annamites ?\par
Péguy disait heureux ceux qui sont morts dans une juste guerre. Il doit s'ensuivre que ceux qui les tuent injustement sont des malheureux. Si les soldats français de 1914 sont morts dans une juste guerre, alors c'est certainement aussi le cas, au moins au même degré, pour Vercingétorix. Si l’un pense ainsi, quels sentiments peut-on avoir envers l'homme qui l'a tenu pendant six ans enchaîné dans un cachot complètement noir, puis l'a exposé en spectacle aux Romains, puis l'a fait égorger ? Mais Péguy était un fervent admirateur de l'Empire romain. Si l'on admire l'Empire romain, pourquoi en vouloir à l'Allemagne qui essaie de le reconstituer, sur un territoire plus vaste, avec des méthodes presque identiques ? Cette contradiction n'a pas empêché Péguy de mourir en 1914. Mais c'est elle, quoique non formulée, non reconnue, qui a empêché beaucoup de jeunes en 1940 d'aller au feu dans le même état d'esprit que Péguy.\par
Ou la conquête est toujours un mal ; ou elle est toujours un bien ; ou elle est tantôt un bien, tantôt un mal. Dans ce dernier cas, il faut un critérium pour la discrimination. Donner comme critérium que la conquête est un bien lorsqu'elle accroît la nation dont on est membre par le hasard de la naissance, un mal lorsqu'elle la diminue, cela est tellement contraire à la raison que c'est seulement acceptable pour des gens qui, de parti pris et une fois pour toutes, ont chassé la raison, comme c'est le cas en Allemagne. Mais l'Allemagne peut le faire, parce qu'elle vit d'une tradition romantique. La France ne le peut pas, car l'attachement à la raison fait partie de son patrimoine national. Une partie des Français peut se dire hostile au christianisme ; mais avant comme après 1789, tous les mouvements de pensée qui ont eu lieu en France se sont réclamés de la raison. La France ne peut pas écarter la raison au nom de la patrie.\par
C'est pourquoi la France se sent mal à l'aise dans son patriotisme, et cela bien qu'elle-même, au XVIII\textsuperscript{e} siècle, ait inventé le patriotisme moderne. Il ne faut pas croire que ce qu'on a nommé la vocation universelle de la France rende la conciliation entre le patriotisme et les valeurs universelles plus facile aux Français qu'à d'autres. C'est le contraire qui est vrai. La difficulté est plus grande pour les Français, parce qu'ils ne peuvent pas complètement réussir, ni à supprimer le second terme de la contradiction, ni à séparer les deux termes par une cloison étanche. Ils trouvent la contradiction à l'intérieur de leur patriotisme même. Mais de ce fait ils sont comme obligés d'inventer un patriotisme nouveau. S'ils le font, ils rempliront ce qui a été jusqu'à un certain point, dans le passé, la fonction de la France, à savoir de penser ce dont le monde a besoin. Le monde a besoin en ce moment d'un patriotisme nouveau. Et c'est maintenant que cet effort d'invention doit être accompli, alors que le patriotisme est quelque chose qui fait couler le sang. Il ne faut pas attendre qu'il soit redevenu une chose dont on parle dans les salons, les Académies et aux terrasses des cafés.\par
Il est facile de dire, comme Lamartine : « Ma patrie est partout où rayonne la France... La vérité, c'est mon pays. » Malheureusement, cela n'aurait un sens que si France et vérité étaient des mots équivalents. Il est arrivé, il arrive, il arrivera que la France mente et soit injuste ; car la France n'est pas Dieu, il s'en faut de beaucoup. Le Christ seul a pu dire : « Je suis la vérité. » Cela n'est permis à rien d'autre sur terre, ni hommes, ni collectivités, mais bien moins encore aux collectivités. Car il est possible qu'un homme parvienne à un degré de sainteté tel que ce ne soit plus lui, mais le Christ qui vive en lui. Au lieu qu'il n'y a pas de nation sainte.\par
Il y a eu une nation jadis qui s'est crue sainte, et cela lui a très mal réussi ; et à ce sujet il est bien étrange de penser que les Pharisiens étaient les résistants, dans cette nation, et les publicains les collaborateurs, et de se rappeler quels étaient les rapports du Christ avec les uns et les autres.\par
Cela semble obliger à penser que notre résistance serait une position spirituellement dangereuse, même spirituellement mauvaise, si parmi les mobiles qui l'animent nous ne savons pas restreindre le mobile patriotique dans de justes limites. C'est ce danger même qu'expriment, dans le langage extrêmement vulgaire de notre époque, ceux qui, sincèrement ou non, disent craindre que ce mouvement ne tourne au fascisme ; car le fascisme est toujours lié à une certaine variété du sentiment patriotique.\par
La vocation universelle de la France ne peut pas, à moins de mensonge, être évoquée avec une fierté sans mélange. Si l'on ment, on la trahit dans les mots mêmes par lesquels on l'évoque ; si l'on se souvient de la vérité, la honte doit toujours se mêler à la fierté, car il y a eu quelque chose de gênant dans tous les exemples historiques qu'on peut en fournir. Au XIII\textsuperscript{e} siècle, la France a été un foyer pour toute la chrétienté. Mais c'est au début même de ce siècle qu'elle avait détruit pour toujours, au sud de la Loire, une civilisation naissante qui brillait déjà d'un grand éclat ; et c'est au cours de cette opération militaire, en liaison avec elle, qu'a été établie pour la première fois l'Inquisition. C'est là une souillure qui compte. Le XIII\textsuperscript{e} siècle est celui où le gothique s'est substitué au roman, la musique polyphonique au chant grégorien, et, en théologie, les constructions tirées d'Aristote à l'inspiration platonicienne ; dès lors on peut douter que l'influence française en ce siècle ait correspondu à un progrès. Au XVII\textsuperscript{e} siècle, la France a de nouveau rayonné sur l'Europe. Mais le prestige militaire lié à ce rayonnement a été obtenu par des méthodes inavouables, du moins si l'on aime la justice ; au reste, autant la conception classique française a produit des œuvres merveilleuses en langue française, autant elle a exercé une influence destructrice à l'étranger. En 1789, la France est devenue l'espoir des peuples. Mais trois années plus tard elle est partie en guerre, et dès les premières victoires elle a substitué aux expéditions de délivrance des expéditions de conquête. Sans l'Angleterre, la Russie et l'Espagne, elle aurait imposé à l'Europe une unité peut-être à peine moins étouffante que celle qui est aujourd'hui promise par l'Allemagne. Dans la deuxième partie du siècle dernier, quand on s'est aperçu que l'Europe n'est pas le monde, et qu'il y a plusieurs continents sur cette planète, la France a été reprise d'aspirations à un rôle universel. Mais elle n'a abouti qu'à fabriquer un Empire colonial imité de celui des Anglais, et dans le cœur d'un certain nombre d'hommes de couleur, son nom est maintenant lié à des sentiments auxquels il est intolérable de penser.\par
Ainsi la contradiction inhérente au patriotisme français se retrouve aussi le long de l'histoire de France. Il ne faut pas en conclure que la France, ayant vécu si longtemps avec cette contradiction, peut continuer. D'abord, si l'on reconnaît une contradiction, il est honteux de la supporter. Puis en fait la France a failli mourir d'une crise du patriotisme français. Tout porte à croire qu'elle serait morte si le patriotisme anglais n'était de qualité heureusement plus solide. Mais on ne peut pas le transporter chez nous. C’est le nôtre qu'il faut refaire. Il est encore à refaire. Il donne de nouveau des signes de vitalité parce que les soldats allemands sont chez nous des agents de propagande incomparables pour le patriotisme français ; mais ils n'y seront pas toujours.\par
Il y a là une responsabilité terrible. Car il s'agit de ce qu'on appelle refaire une âme au pays : et il y a une si forte tentation de la refaire à coups de mensonges ou de vérités partielles qu'il faut plus que de l'héroïsme pour s'attacher à la vérité.\par
La crise du patriotisme a été double. En se servant du vocabulaire politique, on peut dire qu'il y a eu une crise à gauche et une crise à droite.\par
À droite, dans la jeunesse bourgeoise, la coupure entre le patriotisme et la morale, jointe à d'autres causes, avait complètement discrédité toute moralité ; mais le patriotisme avait à peine davantage de prestige. L'esprit exprimé par les mots : « Politique d'abord » s'était, étendu beaucoup plus loin, que l'influence même de Maurras. Or ces mots expriment une absurdité, car la politique n'est qu'une technique, un recueil de procédés. C'est comme si l'on disait : « mécanique d'abord », La question qui se pose immédiatement est : Politique en vue de quoi ? Richelieu répondrait : Pour la grandeur de l'État. Et pourquoi pour ce but et non pour un autre ? À cette question, il n'y a aucune réponse.\par
C'est la question qu'il ne faut pas poser. La politique dite réaliste, transmise de Richelieu à Maurras, non sans avoir été endommagée en chemin, n'a de sens que si cette question n'est pas posée. Il y a une condition simple, pour qu'elle ne le soit pas. Quand le mendiant disait à Talleyrand : « Monseigneur, il faut bien que je vive », Talleyrand répondait : « Je n'en vois pas la nécessité. » Mais le mendiant, lui, en voyait très bien la nécessité. De même Louis XIV voyait très bien la nécessité que l'État fût servi avec un dévouement total, parce que l'État, c'était lui. Richelieu ne pensait en être que le premier serviteur ; néanmoins, en un sens, il le possédait, et pour cette raison s'identifiait avec lui. La conception politique de Richelieu n'a de sens que pour ceux qui, à titre soit individuel soit collectif, se sentent ou bien les maîtres de leur pays ou bien capables de le devenir.\par
La jeunesse bourgeoise française ne pouvait plus, depuis 1924, avoir le sentiment que la France était son domaine. Les ouvriers faisaient bien trop de bruit. D'autre part elle souffrait de cet épuisement mystérieux qui s'est abattu sur la France après 1918, et dont les causes sont sans doute en grande partie physiques. Qu'il faille incriminer l'alcoolisme, l'état nerveux des parents quand ils ont mis au monde et élevé cette jeunesse, ou autre chose, la jeunesse française donne depuis longtemps des signes certains de fatigue. La jeunesse allemande, même en 1932, alors que les pouvoirs publics ne s'occupaient pas d'elle, était d'une vitalité incomparablement plus grande, malgré les privations très dures et très longues qu'elle avait souffertes.\par
Cette fatigue empêchait que la jeunesse bourgeoise de France se sentit en état de devenir maîtresse du pays. Dès lors, à la question « Politique en vue de quoi ? » la réponse qui s'imposait était : « en vue d'être installés par d'autres au pouvoir dans ce pays ». Par d'autres, c'est-à-dire par l'étranger. Rien dans le système moral de ces jeunes gens ne pouvait empêcher ce désir. Le choc de 1936 le fit pénétrer en eux à une profondeur irréparable. On ne leur avait fait aucun mal ; mais ils avaient eu peur ; ils avaient été humiliés, et, crime impardonnable à leurs yeux, humiliés par ceux qu'ils regardaient comme leurs inférieurs. En 1937, la presse italienne citait un article, paru dans une revue française d'étudiants, où une jeune Française souhaitait que Mussolini trouvât, parmi ses nombreux soucis, le loisir de venir remettre de l'ordre en France.\par
Si peu sympathiques que soient ces milieux, si criminelle qu'ait été par la suite leur attitude, ce sont des êtres humains, et des êtres humains malheureux. Le problème à leur égard se pose en ces termes : Comment les réconcilier avec la France sans la livrer entre leurs mains ?\par
À gauche, c'est-à-dire surtout chez les ouvriers et chez les intellectuels qui penchent de leur côté, il y a deux courants tout à fait distincts, quoique parfois, mais non pas toujours, les deux courants coexistent dans le même être. L'un est le courant issu de la tradition ouvrière française, qui remonte visiblement au XVIII\textsuperscript{e} siècle, quand tant d'ouvriers lisaient Jean-Jacques, mais qui peut-être remonte souterrainement jusqu'aux premiers mouvements d'affranchissement des communes. Ceux que ce courant seul entraîne se vouent entièrement à la pensée de la justice. Malheureusement, aujourd'hui, le cas est assez rare parmi les ouvriers et extrêmement rare parmi les intellectuels.\par
Il y a des gens de cette espèce dans tous les milieux dits de gauche, chrétiens, syndicalistes, anarchistes, socialistes ; et notamment il y en a parmi les ouvriers communistes, car la propagande communiste parle beaucoup de la justice. En cela elle suit les enseignements de Lénine et de Marx, si étrange que cela puisse paraître à ceux qui n'ont pas pénétré les replis de la doctrine.\par
Ces hommes sont tous profondément internationalistes en temps de paix, parce qu'ils savent que la justice n'a pas de nationalité. Ils le sont souvent au cours d'une guerre tant qu'il n'y a pas de défaite. Mais l'écrasement de la patrie fait aussitôt surgir au plus profond de leur cœur un patriotisme parfaitement solide et pur. Ceux-là seront réconciliés d'une manière permanente avec la patrie si on leur propose la conception d'un patriotisme subordonné à la justice.\par
L'autre courant est une réplique à l'attitude bourgeoise. Le marxisme, en offrant aux ouvriers la certitude prétendue scientifique qu'ils seront bientôt les maîtres souverains du globe terrestre, a suscité un impérialisme ouvrier très semblable aux impérialismes nationaux. La Russie a apporté une apparence de vérification expérimentale, et de plus on compte sur elle pour se charger de la partie la plus difficile de l'action qui doit aboutir au renversement du pouvoir.\par
Pour des êtres moralement exilés et immigrés, en contact surtout avec le côté répressif de l'État, qui par une tradition séculaire sont aux confins des catégories sociales constituant le gibier de la police, et sont eux-mêmes traités comme tels toutes les fois que l'État penche vers la réaction, il y a là une tentation irrésistible. Un État souverain, grand, puissant, commandant un territoire bien plus vaste que leur pays, leur dit : « Je vous appartiens, je suis votre bien, votre propriété. Je n'existe que pour vous aider, et un jour prochain je ferai de vous les maîtres absolus dans votre propre pays. »\par
De leur part, repousser cette amitié serait à peu près aussi facile que repousser de l'eau quand on n'a pas bu depuis deux jours. Quelques-uns, qui ont accompli un grand effort sur eux-mêmes pour y parvenir, se sont tellement épuisés dans cet effort qu'ils ont succombé sans combat aux premières pressions de l'Allemagne. Beaucoup d'autres ne résistent qu'en apparence, et en réalité se tiennent simplement à l'écart, par peur des risques qu'entraîne l'action à laquelle on est engagé une fois qu'on a adhéré. Ceux-là, nombreux ou non, ne sont jamais une force.\par
L'U.R.S.S., hors de la Russie, est vraiment la patrie des ouvriers. Pour le sentir, il n'y avait qu'à voir les yeux des ouvriers français quand ils regardaient, autour des kiosques à journaux, les titres annonçant les premières grandes défaites russes. Ce n'était pas la pensée des répercussions de ces défaites sur les relations franco-allemandes qui mettait le désespoir dans leurs yeux, car les défaites anglaises ne les ont jamais touchés ainsi. Ils se sentaient menacés de perdre plus que la France. Ils étaient un peu dans l'état d'esprit où auraient été les premiers chrétiens si on leur avait apporté des preuves matérielles établissant que la résurrection du Christ était une fiction. D'une manière générale, il y a sans doute une assez grande ressemblance entre l'état d'esprit des premiers chrétiens et celui de beaucoup d'ouvriers communistes. Eux aussi attendent une catastrophe prochaine, terrestre, établissant d'un coup pour toujours ici-bas le bien absolu et en même temps leur propre gloire. Le martyre était plus facile aux premiers chrétiens qu'à ceux des siècles suivants, et infiniment plus facile qu'à l'entourage du Christ, pour qui, au moment suprême, il avait été impossible. De même aujourd'hui le sacrifice est plus facile pour un communiste que pour un chrétien.\par
L'U.R.S.S. étant un État, le patriotisme envers elle enferme les mêmes contradictions que tout autre. Mais il n'en résulte pas le même affaiblissement. Au contraire. La présence d'une contradiction, quand elle est sentie, même sourdement, ronge le sentiment ; quand elle n'est pas sentie du tout, le sentiment en est rendu plus intense, puisqu'il bénéficie à la fois de mobiles incompatibles. Ainsi l'U.R.S.S. a tout le prestige d'un État, et de la froide brutalité qui imprègne la politique d'un État, surtout totalitaire ; et en même temps elle a tout le prestige de la justice. Si la contradiction n'est pas sentie, c'est d'une part à cause de l'éloignement, d'autre part parce qu'elle promet à ceux qui l'aiment toute la puissance. Un tel espoir ne diminue pas le besoin de justice, mais le rend aveugle. Comme chacun se croit suffisamment capable de justice, chacun croit aussi qu'un système où il serait puissant serait assez juste. C'est la tentation que le diable a fait subir au Christ. Les hommes y succombent continuellement.\par
Bien que ces ouvriers, animés d'impérialisme ouvrier, soient très différents des jeunes bourgeois fascistes, et constituent une variété humaine plus belle, il se pose à leur égard un problème analogue. Comment leur faire suffisamment aimer leur pays sans le leur livrer ? Car on ne peut pas le leur livrer, ni même leur y faire une position privilégiée ; ce serait une injustice criante à l'égard du reste de la population, et notamment des paysans.\par
L'attitude actuelle de ces ouvriers envers l'Allemagne ne doit pas aveugler sur la gravité du problème. Il se trouve que l'Allemagne est l’ennemie de l’U. R. S. S. Avant qu'elle ne le fût, il y avait déjà de l'agitation parmi eux ; mais c'est une nécessité vitale pour le parti communiste de toujours entretenir l'agitation. Et cette agitation était « contre le fascisme allemand et l'impérialisme anglais ». La France, il n'en était pas question. D'autre part, pendant une année qui fut décisive, de l’été 1939 à l'été 1940, l'influence communiste en France s'est exercée entièrement contre le pays. Il ne sera pas facile d'obtenir que ces ouvriers tournent leur cœur vers leur pays.\par
Dans le reste de la population, la crise du patriotisme n'a pas été aussi aiguë ; elle n'a pas été jusqu'au reniement, en faveur d'autre chose ; il y a eu seulement une espèce d'extinction. Chez les paysans, c'était dû sans doute à ce qu'ils avaient le sentiment de ne pas compter dans le pays, sinon comme chair à canon pour des intérêts étrangers aux leurs ; chez les petits bourgeois, cela devait être dû surtout à l'ennui.\par
À toutes les causes particulières de désaffection s'en est ajoutée une très générale qui est comme le rebours de l'idolâtrie. L'État avait cessé d'être, sous le nom de nation ou de patrie, un bien infini, dans le sens d'un bien à servir par le dévouement. En revanche il était devenu aux yeux de tous un bien illimité à consommer. L'absolu lié à l'idolâtrie lui est resté attaché, une fois l'idolâtrie effacée, et a pris cette forme nouvelle. L'État a paru être une corne d'abondance inépuisable qui distribuait les trésors proportionnellement aux pressions qu'il subissait. Ainsi on lui en voulait toujours de ne pas accorder davantage. Il semblait qu'il refusât tout ce qu'il ne fournissait pas. Quand il demandait, c'était une exigence qui paraissait paradoxale. Quand il imposait, c'était une contrainte intolérable. L'attitude des gens envers l'État était celle des enfants non pas envers leurs parents, mais envers des adultes qu'ils n'aiment ni ne craignent ; ils demandent sans cesse et ne veulent pas obéir.\par
Comment passer tout d'un coup de cette attitude au dévouement sans bornes exigé par la guerre ? Mais même pendant la guerre les Français ont cru que l'État avait la victoire quelque part dans ses coffres, à côté des autres trésors qu'il ne voulait pas se donner la peine de sortir. On a tout fait pour encourager cette opinion, comme en témoigne le slogan : « Nous vaincrons parce que nous sommes les plus forts. »\par
La victoire va libérer un pays où tous auront été presque exclusivement occupés à désobéir, pour des motifs bas ou élevés. On a écouté la radio de Londres, lu et distribué des papiers interdits, voyagé en fraude, caché du blé, travaillé le plus mal possible, fait du marché noir, on s'est vanté de tout cela entre amis et en famille. Comment fera-t-on comprendre aux gens que c'est fini, que désormais il faut obéir ?\par
On aura aussi passé ces années à rêver de rassasiement. Ce sont des rêveries de mendiants, en ce sens qu'on ne pense qu'à recevoir de bonnes choses sans aucune contrepartie. En fait, les pouvoirs publics assureront la distribution ; comment éviter alors que cette attitude de mendiant insolent, qui déjà avant guerre était celle des citoyens envers l'État, ne devienne infiniment plus accentuée ? Et si elle prend pour objet un pays étranger, par exemple l'Amérique, le danger est encore bien plus grave.\par
Un second rêve très répandu est celui de tuer. Tuer au nom des plus beaux motifs, mais bassement et sans risques. Soit que l'État succombe à la contagion de ce terrorisme diffus, comme il est à craindre, soit qu'il essaie de le limiter, dans les deux cas l'aspect répressif et policier de l'État, qui par tradition est tellement haï et méprisé en France, sera au premier plan.\par
Le gouvernement qui surgira en France après la libération du territoire sera devant le triple danger causé par ce goût du sang, ce complexe de mendicité, cette incapacité d'obéir.\par
De remède, il n'y en a qu'un. Donner aux Français quelque chose à aimer. Et leur donner d'abord à aimer la France. Concevoir la réalité correspondant au nom de France de telle manière que telle qu'elle est, dans sa vérité, elle puisse être aimée avec toute l'âme.\par
Le centre de la contradiction inhérente au patriotisme, c'est que la patrie est une chose limitée dont l'exigence est illimitée. Au moment du péril extrême, elle demande tout. Pourquoi accorderait-on tout à une chose limitée ? D'un autre côté, ne pas être résolu à lui donner tout en cas de besoin, c'est l'abandonner tout à fait, car sa conservation ne peut être assurée à un moindre prix. Ainsi on semble toujours être ou en deçà ou au-delà de ce qu'on lui doit, et si l'on va au-delà, par réaction on revient plus tard d'autant plus en deçà.\par
La contradiction n'est qu'apparente. Ou plus exactement elle est réelle, mais vue dans sa vérité elle se ramène à une de ces contradictions fondamentales de la situation humaine, qu'il faut reconnaître, accepter, et utiliser comme marchepied pour monter au-dessus de ce qui est humain. Jamais dans cet univers il n'y a égalité de dimensions entre une obligation et son objet. L'obligation est un infini, l'objet ne l'est pas. Cette contradiction pèse sur la vie quotidienne de tous les hommes, sans exception, y compris ceux qui seraient tout à fait incapables de la formuler même confusément. Tous les procédés que les hommes ont cru trouver pour en sortir sont des mensonges.\par
L'un d'eux consiste à ne se reconnaître d'obligations qu'envers ce qui n'est pas de ce monde. Une variété de ce procédé constitue la fausse mystique, la fausse contemplation. Une autre est la pratique des bonnes œuvres accomplie dans un certain esprit, « pour l'amour de Dieu », comme on dit, les malheureux secourus n'étant que la matière de l'action, une occasion anonyme de témoigner de la bienveillance à Dieu. Dans les deux cas il y a mensonge, car « celui qui n'aime pas son frère qu'il voit, comment aimerait-il Dieu qu'il ne voit pas ? ». C'est seulement à travers les choses et les êtres d'ici-bas que l'amour humain peut percer jusqu'à ce qui habite derrière.\par
Un autre procédé consiste à admettre qu'il y a ici-bas un ou plusieurs objets enfermant cet absolu, cet infini, cette perfection qui sont essentiellement liés à l'obligation comme telle. C'est le mensonge de l'idolâtrie.\par
Le troisième procédé consiste à nier toute obligation. On ne peut pas prouver par une démonstration de l’espèce géométrique que c'est une erreur, car l'obligation est d'un ordre de certitude bien supérieur à celui où habitent les preuves. En fait, cette négation est impossible. Elle constitue un suicide spirituel. Et l'homme est ainsi fait qu'en lui la mort spirituelle s'accompagne de maladies psychologiques elles-mêmes mortelles. En fait, l'instinct de conservation empêche que l'âme fasse davantage que s'approcher d'un tel état ; et même ainsi elle est saisie d'un ennui qui la transforme en désert. Presque toujours, ou plutôt presque certainement toujours, celui qui nie toute obligation ment aux autres et à lui-même ; en fait il le reconnaît. Il n'est pas d'homme qui ne porte parfois des jugements sur le bien et le mal, ne fût-ce que pour blâmer autrui.\par
Il faut accepter la situation qui nous est faite et qui nous soumet à des obligations absolues envers des choses relatives, limitées et imparfaites. Pour discriminer quelles sont ces choses et comment peuvent se composer leurs exigences envers nous, il faut seulement voir clairement en quoi consiste leur relation avec le bien.\par
Pour la patrie, les notions d'enracinement, de milieu vital, suffisent à cet effet. Elles n'ont pas besoin d'être établies par des preuves, car depuis quelques années elles sont vérifiées expérimentalement. Comme il y a des milieux de culture pour certains animaux microscopiques, des terrains indispensables pour certaines plantes, de même il y a une certaine partie de l'âme en chacun et certaines manières de penser et d'agir circulant des uns aux autres qui ne peuvent exister que dans le milieu national et disparaissent quand un pays est détruit.\par
Aujourd'hui, tous les Français savent ce qui leur a manqué dès que la France a sombré. Ils le savent comme ils savent ce qui manque quand on ne mange pas. Ils savent qu'une partie de leur âme colle tellement à la France que lorsque la France leur est ôtée elle y reste collée, comme la peau à un objet brûlant, et est ainsi arrachée. Il existe donc une chose à laquelle est collée une partie de l'âme de chaque Français, la même pour tous, unique, réelle quoique impalpable, et réelle à la manière des choses qu'on peut toucher. Dès lors, ce qui menace la France de destruction – et dans certaines circonstances une invasion est une menace de destruction – équivaut à la menace d'une mutilation physique de tous les Français, et de leurs enfants et petits-enfants, et de leurs descendants à perte de vue. Car il y a des populations qui ne se sont jamais guéries d'avoir été une fois conquises.\par
Cela suffit pour que l'obligation envers la patrie s'impose comme une évidence. Elle coexiste avec d'autres ; elle ne contraint pas à donner tout toujours ; elle contraint à donner tout quelquefois. De même un mineur doit quelquefois donner tout, lorsqu'il y a accident dans la mine et des camarades en péril de mort. Cela est admis, reconnu. L'obligation envers la patrie est tout aussi évidente, dès lors que la patrie est éprouvée concrètement comme une réalité. Elle l'est aujourd'hui. La réalité de la France est devenue sensible à tous les Français par l'absence.\par
Jamais on n'a osé nier l'obligation envers la patrie autrement qu'en niant la réalité de la patrie. Le pacifisme extrême selon la doctrine de Gandhi n'est pas une négation de cette obligation, mais une méthode particulière pour l'accomplir. Cette méthode n'a jamais été appliquée, que l'on sache ; notamment elle ne l'a pas été par Gandhi, qui est bien trop réaliste. Si elle avait été appliquée en France, les Français n'auraient opposé aucune arme à l'envahisseur ; mais ils n'auraient jamais consenti à rien faire, dans aucun domaine, qui pût aider l'armée occupante, ils auraient tout fait pour la gêner, et ils auraient persisté indéfiniment, inflexiblement dans cette attitude. Il est clair qu'ils auraient péri en bien plus grand nombre et bien plus douloureusement. C'est l'imitation de la passion du Christ portée à l'échelle nationale.\par
Si une nation dans son ensemble était assez proche de la perfection pour qu'on pût lui proposer d'imiter la passion du Christ, certainement cela vaudrait la peine de le faire. Elle disparaîtrait, mais cette disparition vaudrait infiniment mieux que la survie la plus glorieuse. Mais il n'en est pas ainsi. Très probablement, presque certainement, il ne peut pas en être ainsi. C'est seulement l'âme, dans le plus secret de sa solitude, à qui il peut être donné de s'orienter vers une telle perfection.\par
Cependant, s'il y a des hommes qui aient comme vocation de témoigner pour cette perfection impossible, les pouvoirs publics sont obligés de les y autoriser, bien plus, de leur en donner les moyens. L'Angleterre reconnaît l'objection de conscience.\par
Mais ce n'est pas assez. Pour ceux-là, il faudrait se donner la peine d'inventer quelque chose qui, sans être une participation ni directe ni indirecte aux opérations stratégiques, soit une présence à la guerre proprement dite, et une présence beaucoup plus pénible et plus dangereuse que celle des soldats eux-mêmes.\par
Ce serait là l'unique remède aux inconvénients de la propagande pacifiste. Car cela permettrait, sans injustice, de déshonorer ceux qui, faisant profession de pacifisme intégral ou presque intégral, se refuseraient à un témoignage de cette nature. Le pacifisme n'est susceptible de faire du mal que par la confusion entre deux répugnances, la répugnance à tuer et la répugnance à mourir. La première est honorable, mais très faible ; la seconde, presque inavouable, mais très forte ; leur mélange forme un mobile d'une grande énergie, qui n'est pas inhibé par la honte, et où la seconde répugnance est seule agissante. Les pacifistes français des dernières années répugnaient à mourir, nullement à tuer, sans quoi ils n'auraient pas couru si précipitamment, en juillet 1940, à la collaboration avec l'Allemagne. Le petit nombre qui se trouvait dans ces milieux par une véritable répugnance au meurtre a été tristement dupe.\par
En séparant ces deux répugnances, on supprime tout danger. L'influence de la répugnance à tuer n'est pas dangereuse ; d'abord elle est bonne, car elle procède du bien ; puis elle est faible, et il n'y a malheureusement aucune chance qu'elle cesse de l'être. Quant à ceux qui sont faibles devant la peur de la mort, il convient qu'ils soient des objets de compassion, car tout être humain, s'il n'est pas fanatisé, est au moins par moments susceptible de cette faiblesse ; mais s'ils font de leur faiblesse une opinion à propager, ils deviennent criminels, et il est alors nécessaire et facile de les déshonorer.\par
En définissant la patrie comme un certain milieu vital, on évite les contradictions et les mensonges qui rongent le patriotisme. Il est un certain milieu vital ; mais il y en a d'autres. Il a été produit par un enchevêtrement de causes où se sont mélangés le bien et le mal, le juste et l'injuste, et de ce fait il n'est pas le meilleur possible. Il s'est peut-être constitué aux dépens d'une autre combinaison plus riche en effluves vitaux, et au cas où il en serait ainsi les regrets seraient légitimes ; mais les événements passés sont accomplis ; ce milieu existe, et tel qu'il est doit être préservé comme un trésor à cause du bien qu'il contient.\par
Les populations conquises par les soldats du roi de France ont dans beaucoup de cas souffert un mal. Mais tant de liens organiques ont poussé au cours des siècles qu'un remède chirurgical ne ferait qu'ajouter à ce mal un mal nouveau. Le passé n'est que partiellement réparable, et il ne peut l'être que par une vie locale et régionale autorisée, encouragée sans réserves par les pouvoirs publics dans le cadre de la nation française. D'autre part, la disparition de la nation française, loin de réparer si peu que ce soit le mal de la conquête passée, le renouvelle avec une gravité considérablement accrue ; si des populations ont subi, il y a quelques siècles, une perte de vitalité du fait des armes françaises, elles seront moralement tuées par une nouvelle blessure infligée par les armes allemandes. En ce sens seulement est vrai le lieu commun, selon lequel il n'y a pas incompatibilité entre l'amour de la petite patrie et celui de la grande. Car de cette manière un homme de Toulouse peut regretter passionnément que sa ville soit jadis devenue française ; que tant de merveilleuses églises romanes aient été détruites pour faire place à un médiocre gothique d'importation ; que l'Inquisition ait arrêté l'épanouissement spirituel ; et il peut plus passionnément encore se promettre de ne jamais accepter que cette même ville devienne allemande.\par
De même pour l'extérieur. Si la patrie est considérée comme un milieu vital, elle n'a besoin d'être soustraite aux influences extérieures que dans la mesure nécessaire pour le demeurer, et non pas absolument. L'État cesse d'être de droit divin le maître absolu des territoires dont il a la charge ; une autorité raisonnable et limitée sur ces territoires, émanant d'organismes internationaux et ayant pour objet des problèmes essentiels dont les données sont internationales, cesserait d'apparaître comme un crime de lèse-majesté. Il pourrait aussi s'établir des milieux pour la circulation des pensées, plus vastes que la France et l'englobant, ou liant certains territoires français à des territoires non français. Ne serait-il pas naturel, par exemple, que dans un certain domaine la Bretagne, le pays de Galles, la Cornouaille, l'Irlande, se sentent des parties d'un même milieu ?\par
Mais de nouveau, plus on est attaché à ces milieux non nationaux, plus on veut conserver la liberté nationale, car de telles relations par-dessus les frontières n'ont pas lieu pour les populations asservies. C'est ainsi que les échanges de culture entre pays méditerranéens ont été incomparablement plus intenses et plus vivants avant qu'après la conquête romaine, alors que tous ces pays, réduits au malheureux état de provinces, sont tombés dans une morne uniformité. Il n'y a échange que si chacun conserve son génie propre, et cela n'est pas possible sans liberté.\par
D'une manière générale, si l'on reconnaît l'existence d'un grand nombre de milieux porteurs de vie, la patrie ne constituant que l'un d'entre eux, néanmoins, quand elle est en danger de disparaître, toutes les obligations impliquées par la fidélité à tous ces milieux s'unissent dans l'obligation unique de secourir la patrie. Car les membres d'une population asservie à un État étranger sont privés de tous ces milieux à la fois, et non pas seulement du milieu national. Ainsi quand une nation se trouve à ce degré de péril, l'obligation militaire devient l'expression unique de toutes les fidélités d'ici-bas. Cela est vrai même pour les objecteurs de conscience, si l'on prend la peine de leur trouver un équivalent à l'acte de guerre.\par
Cela une fois reconnu, il devrait en résulter certaines modifications dans la manière de considérer la guerre, en cas de péril pour la nation. D'abord la distinction entre militaires et civils, que la pression des faits a déjà presque effacée, doit être entièrement abolie. C'est elle en grande partie qui avait provoqué la réaction d'après 1918. Chaque individu dans la population doit au pays la totalité de ses forces, de ses ressources, et sa vie même, jusqu'à ce que le danger soit écarté. Il est désirable que les souffrances et les périls soient répartis à travers toutes les catégories de la population, jeunes et vieux, hommes et femmes, bien portants et mal portants, dans toute la mesure des possibilités techniques, et même un peu au-delà. Enfin l'honneur est tellement lié à l'accomplissement de cette obligation, et la contrainte extérieure est tellement contraire à l'honneur, qu'on devrait bien autoriser ceux qui le désirent à se soustraire à cette obligation ; on leur infligerait la perte de la nationalité, et de plus soit l'expulsion avec interdiction de revenir jamais dans le pays, soit des humiliations permanentes comme marque publique qu'ils sont sans honneur.\par
Il est choquant que le manquement à l'honneur soit puni de la même manière que le vol et l'assassinat. Ceux qui ne veulent pas défendre leur patrie doivent perdre, non pas la vie ni la liberté, mais purement et simplement la patrie.\par
Si l'état du pays est tel que ce soit là pour un grand nombre un châtiment insignifiant, alors le code militaire aussi se trouve sans efficacité. Nous ne pouvons pas l'ignorer.\par
Si l'obligation militaire enferme à certains moments toutes les fidélités terrestres, parallèlement l'État a le devoir, en tout temps, de préserver tout milieu, au-dedans ou au-dehors du territoire, où une partie petite ou grande de la population puise de la vie pour l'âme.\par
Le devoir le plus évident de l'État, c'est de veiller efficacement en tout temps à la sécurité du territoire national. La sécurité ne signifie pas l'absence de danger, car dans ce monde le danger est toujours là, mais une chance raisonnable de se tirer d'affaire en cas de crise. Mais ce n'est là que le devoir le plus élémentaire de l'État. S'il ne fait que cela, il ne fait rien, car s'il ne fait que cela il ne peut pas même y réussir.\par
Il a le devoir de faire de la patrie, au degré le plus élevé possible, une réalité. Elle n'était pas une réalité pour beaucoup de Français en 1939. Elle l'est redevenue par la privation. Il faut qu'elle le demeure dans la possession, et pour cela il faut qu'elle soit réellement, en fait, fournisseuse de vie, qu'elle soit réellement un terrain d'enracinement. Il faut aussi qu'elle soit un cadre favorable pour la participation et l'attachement fidèle à toute espèce de milieux autres qu'elle-même.\par
Aujourd'hui, en même temps que les Français ont retrouvé le sentiment que la France est une réalité, ils sont devenus bien plus conscients que jadis des différences locales. La séparation de la France en morceaux, la censure de la correspondance qui enferme les échanges de pensée dans un petit territoire, y est pour quelque chose, et, chose paradoxale, le brassage forcé de la population y a aussi beaucoup contribué. On a aujourd'hui d'une manière beaucoup plus continuelle et plus aiguë qu'auparavant le sentiment qu'on est Breton, Lorrain, Provençal, Parisien. Il y a dans ce sentiment une nuance d'hostilité qu'il faut essayer d'effacer ; d'ailleurs il est urgent aussi d'effacer la xénophobie. Mais ce sentiment en lui-même ne doit pas être découragé, au contraire. Il serait désastreux de le déclarer contraire au patriotisme. Dans la détresse, le désarroi, la solitude, le déracinement où se trouvent les Français, toutes les fidélités, tous les attachements sont à conserver comme des trésors trop rares et infiniment précieux, à arroser comme des plantes malades.\par
Peu importe que le gouvernement de Vichy ait mis en avant une doctrine régionaliste. Son seul tort en la matière est de ne pas l'avoir appliquée. Loin de prendre en toutes choses le contre-pied de ses mots d'ordre, nous devons conserver beaucoup des pensées lancées par la propagande de la Révolution Nationale, mais en faire des vérités.\par
De même, les Français dans leur isolement même ont acquis le sentiment que la France est petite, que enfermé à l'intérieur on étouffe, et qu'il faut davantage. L'idée de l'Europe, de l'unité européenne, a fait beaucoup pour le succès de la propagande collaborationniste dans les premiers temps. Ce sentiment aussi, on ne saurait trop l'encourager, l'alimenter. Il serait désastreux de l'opposer à la patrie.\par
Enfin on ne saurait trop encourager l'existence de milieux d'idées ne constituant pas des rouages de la vie publique ; car à cette seule condition ils ne sont pas des cadavres. C'est le cas des syndicats, s'ils ne sont pas chargés de responsabilités quotidiennes dans l'organisation économique. C'est le cas des milieux chrétiens, protestants ou catholiques, et plus particulièrement d'organisations comme la J. O. C. ; mais un État qui succomberait le moins du monde à des velléités cléricales les tuerait à coup sûr. C'est le cas de collectivités surgies après la défaite, les unes officiellement, Chantiers de Jeunesse, Compagnons, les autres clandestinement, à savoir les groupes de résistance. Les unes ont un peu de vie malgré leur caractère officiel, par un concours exceptionnel de circonstances ; mais si on leur conservait ce caractère elles mourraient. Les autres sont nées de la lutte contre l'État, et si l'on succombait à la tentation de leur donner une existence officielle dans la vie publique, cela les ravagerait moralement à un degré terrible.\par
D'un autre côté, si des milieux de cette espèce sont à l'écart de la vie publique, ils cessent d'exister. Il faut donc qu'ils n'en fassent pas partie et ne soient pas non plus à l'écart. Un procédé à cet effet pourrait être, par exemple, que l'État désigne fréquemment des hommes choisis dans ces milieux pour des missions spéciales, à titre temporaire. Mais il faudrait d'une part que l'État même fasse le choix des personnes, d'autre part que tous leurs camarades y trouvent un motif de fierté. Une telle méthode pourrait passer à l'état d'institution.\par
Là encore, il faut, tout en essayant d'empêcher les haines, encourager les différences. Jamais le bouillonnement des idées ne peut faire du mal à un pays comme le nôtre. C'est l'inertie mentale qui est mortelle pour lui.\par
Le devoir qui incombe à l'État d'assurer au peuple quelque chose qui soit réellement une patrie ne saurait être une condition pour l'obligation militaire qui incombe à la population en cas de péril national. Car si l'État manque à sa charge, si la patrie dépérit, néanmoins, tant que l'indépendance nationale subsiste, il y a espoir de résurrection ; si on regarde de près, on constate dans le passé de tous les pays, à des dates parfois rapprochées, des abaissements et des relèvements très surprenants. Mais si le pays est subjugué par des armes étrangères, il n'y a plus rien à espérer, sauf le cas de libération rapide. L'espérance seule, quand même il n'y aurait rien d'autre, vaut la peine qu'on meure pour la préserver.\par
Ainsi, bien que la patrie soit un fait et comme telle soumise à des conditions extérieures, à des hasards, l'obligation de la secourir en cas de danger mortel n'en est pas moins inconditionnée. Mais il est évident qu'en fait la population sera d'autant plus ardente que la réalité de la patrie lui aura été rendue plus sensible.\par
La notion de patrie ainsi définie est incompatible avec la conception actuelle de l'histoire du pays avec la conception actuelle de la grandeur nationale, et par-dessus tout avec la manière dont on parle actuellement de l'Empire.\par
La France a un Empire, et par suite, quelle que soit la position de principe adoptée, il en découle des problèmes de fait qui sont très complexes et très différents selon les localités. Mais il ne faut pas tout mélanger. Il se pose d'abord une question de principe ; et même quelque chose de moins précis encore, une question de sentiment. Dans l'ensemble, un Français a-t-il lieu d'être heureux que la France ait un Empire, et d'y penser, d'en parler avec joie, avec fierté, et sur le ton d'un propriétaire légitime ?\par
Oui, si ce Français est patriote à la manière de Richelieu, de Louis XIV ou de Maurras. Non, si l'inspiration chrétienne, si la pensée de 1789 sont indissolublement mélangées à la substance même de son patriotisme. Toute autre nation avait à la rigueur le droit de se tailler un Empire, mais non pas la France ; pour la même raison qui a fait de la souveraineté temporelle du pape un scandale aux yeux de la chrétienté. Quand on assume, comme a fait la France en 1789, la fonction de penser pour l'univers, de définir pour lui la justice, on ne devient pas propriétaire de chair humaine. Même s'il est vrai qu'à défaut de nous d'autres se seraient emparés de ces malheureux et les auraient traités plus mal encore, ce n'était pas un motif légitime ; tout compte fait, le mal total aurait été moindre. Les motifs de ce genre sont la plupart du temps mauvais. Un prêtre ne devient pas patron d'une maison close dans la pensée qu'un marlou traiterait ces femmes plus mal. La France n'avait pas à manquer au respect d'elle-même par compassion. Et d'ailleurs elle ne l'a pas fait. Personne n'oserait soutenir sérieusement qu'elle est allée conquérir ces populations pour empêcher que d'autres nations ne les maltraitent. D'autant plus que, dans une large mesure, c'est elle-même, au XIX\textsuperscript{e} siècle, qui a pris l'initiative de remettre à la mode les aventures coloniales.\par
Parmi ceux qu'elle a soumis, certains sentent très vivement combien il est scandaleux que ce soit elle qui ait fait cela ; leur rancune contre nous en est aggravée par une espèce d'amertume terriblement douloureuse et par une sorte de stupéfaction.\par
Il est possible qu'aujourd'hui la France ait à choisir entre l'attachement à son Empire et le besoin d'avoir de nouveau une âme. Plus généralement, elle doit choisir entre une âme et la conception romaine, cornélienne de la grandeur.\par
Si elle choisit mal, si nous-mêmes la poussons à choisir mal, ce qui n'est que trop probable, elle n'aura ni l'un ni l'autre, mais seulement le plus affreux malheur, qu'elle subira avec étonnement sans que personne puisse en discerner la cause. Et tous ceux qui sont en état de parler, de tenir une plume, auront éternellement la responsabilité d'un crime.\par
Bernanos a compris et dit que l'hitlérisme, c'est la Rome païenne qui revient. Mais a-t-il oublié, avons-nous oublié quelle part a eue son influence dans notre histoire, dans notre culture, et aujourd’hui encore dans nos pensées ? Si nous avons pris, par horreur d'une certaine forme du mal, la détermination terrible de faire la guerre, avec toutes les atrocités qu'elle implique, pouvons-nous être excusés si nous faisons une guerre moins impitoyable à cette même forme du mal dans notre propre âme ? Si la grandeur de l'espèce cornélienne nous séduit par le prestige de l'héroïsme, l'Allemagne peut bien nous séduire aussi, car les soldats allemands sont certainement des « héros ». Dans la confusion actuelle des pensées et des sentiments autour de l'idée de patrie, avons-nous aucune garantie que le sacrifice d'un soldat français en Afrique est plus pur par l'inspiration que celui d'un soldat allemand en Russie ? Actuellement nous n'en avons pas. Si nous ne sentons pas quelle terrible responsabilité il en résulte, nous ne pouvons pas être innocents au milieu de ce déchaînement de crimes à travers le monde.\par
S'il y a un point sur lequel il faille tout mépriser et tout braver par amour de la vérité, c'est celui-là. Nous sommes tous rassemblés au nom de la patrie. Que sommes-nous, quel mépris ne mériterons-nous pas, si dans la pensée de la patrie se trouve mêlée la moindre trace de mensonge ?\par
Mais si les sentiments du genre cornélien n'animent pas notre patriotisme, on peut demander quel mobile les remplacera.\par
Il y en a un, non moins énergique, absolument pur, et répondant complètement aux circonstances actuelles. C'est la compassion pour la patrie. Il y a un répondant glorieux. Jeanne d'Arc disait qu'elle avait pitié du royaume de France.\par
Mais on peut alléguer une autorité infiniment plus haute. Dans l'Évangile, on ne peut pas trouver de marque que le Christ ait éprouvé à l'égard de Jérusalem et de la Judée rien qui ressemble à de l'amour, sinon seulement l'amour enfermé dans la compassion. Il n'a jamais témoigné à son pays aucun attachement d'une autre espèce. Mais la compassion, il l'a exprimée plus d'une fois. Il a pleuré sur la ville, en prévoyant, comme il était facile de le faire à cette époque, la destruction qui s'abattrait prochainement sur elle. Il lui a parlé comme à une personne. « Jérusalem, Jérusalem, combien de fois j'ai voulu... » Même portant sa croix, il lui a encore témoigné sa pitié.\par
Qu'on ne pense pas que la compassion pour la patrie n'enferme pas d'énergie guerrière. Elle a animé les Carthaginois à un des exploits les plus prodigieux de l'histoire. Vaincus et réduits à peu de chose par Scipion l'Africain, ils subirent ensuite pendant cinquante ans un processus de démoralisation auprès duquel la capitulation de la France à Munich est peu de chose. Ils furent exposés sans aucun recours à toutes les injures des Numides, et, ayant renoncé par traité à la liberté de faire la guerre, ils imploraient vainement de Rome la permission de se défendre. Quand ils le firent enfin sans autorisation, leur armée fut exterminée. Il fallut alors implorer le pardon des Romains. Ils consentirent à livrer trois cents enfants nobles et toutes leurs armes. Puis leurs délégués reçurent l'ordre d'évacuer entièrement et définitivement la ville afin qu'elle pût être rasée. Ils éclatèrent en cris d'indignation, puis en larmes. « Ils appelaient leur patrie par son nom, et, lui parlant comme à une personne, ils lui disaient les choses les plus déchirantes. » Puis ils supplièrent les Romains, s'ils voulaient leur faire du mal, d'épargner cette cité, ces pierres, ces monuments, ces temples, à qui on ne pouvait rien reprocher, et d'exterminer plutôt la population tout entière ; ils dirent que ce parti serait moins honteux pour les Romains et bien préférable pour le peuple de Carthage. Les Romains restant inflexibles, la ville se souleva, bien que sans ressources, et Scipion l'Africain, à la tête d'une armée nombreuse, mit trois années entières pour s'en emparer et la détruire.\par
Ce sentiment de tendresse poignante pour une chose belle, précieuse, fragile et périssable, est autrement chaleureux que celui de la grandeur nationale. L'énergie dont il est chargé est parfaitement pure. Elle est très intense. Un homme n'est-il pas facilement capable d'héroïsme pour protéger ses enfants, ou ses vieux parents, auxquels ne s'attache pourtant aucun prestige de grandeur ? Un amour parfaitement pur de la patrie a une affinité avec les sentiments qu'inspirent à un homme ses jeunes enfants, ses vieux parents, une femme aimée. La pensée de la faiblesse peut enflammer l'amour comme celle de la force, mais c'est d'une flamme bien autrement pure. La compassion pour la fragilité est toujours liée à l'amour pour la véritable beauté, parce que nous sentons vivement que les choses vraiment belles devraient être assurées d'une existence éternelle et ne le sont pas.\par
On peut aimer la France pour la gloire qui semble lui assurer une existence étendue au loin dans le temps et l'espace. Ou bien on peut l'aimer comme une chose qui, étant terrestre, peut-être détruite, et dont le prix est d'autant plus sensible.\par
Ce sont deux amours distincts ; peut-être, probablement, incompatibles, quoique le langage les mélange. Ceux dont le cœur est fait pour éprouver le second peuvent, par la force de l'habitude, employer le langage qui ne convient qu'au premier.\par
Le second seul est légitime pour un chrétien, car seul il a la couleur de l'humilité chrétienne. Il appartient seul à l'espèce d'amour qui peut recevoir le nom de charité. Qu'on ne croie pas que cet amour puisse seulement avoir pour objet un pays malheureux.\par
Le bonheur est un objet pour la compassion au même titre que le malheur, parce qu'il est terrestre, c'est-à-dire incomplet, fragile et passager. Au reste il y a malheureusement toujours dans la vie d'un pays un certain degré de malheur.\par
Qu'on ne croie pas non plus qu'un tel amour risquerait d'ignorer ou de négliger ce qu'il y a de grandeur authentique et pure dans le passe, le présent et les aspirations de la France. Bien au contraire. La compassion est d'autant plus tendre, d'autant plus poignante, qu'on discerne davantage de bien dans l'être qui en est l'objet, et elle dispose à discerner le bien. Quand un chrétien se représente le Christ en croix, la compassion en lui n'est pas diminuée par la pensée de la perfection, ni inversement. Mais d'un autre côté, un tel amour peut avoir les yeux ouverts sur les injustices, les cruautés, les erreurs, les mensonges, les crimes, les hontes, contenus dans le passé, le présent et les appétits du pays, sans dissimulation ni réticence, et sans en être diminué; il en est rendu seulement plus douloureux. Pour la compassion, le crime lui-même est une raison, non pas de s'éloigner, mais de s'approcher, pour partager, non pas la culpabilité, mais la honte. Les crimes des hommes n'ont pas diminué la compassion du Christ. Ainsi la compassion a les yeux ouverts sur le bien et le mal et trouve dans l'un et l'autre des raisons d'aimer. C'est le seul amour ici-bas qui soit vrai et juste.\par
C'est en ce moment le seul amour qui convienne aux Français. Si les événements que nous venons de traverser ne suffisent pas à nous avertir d'avoir à changer notre manière d'aimer la patrie, quelle leçon peut nous instruire ? Que peut-il y avoir de plus pour éveiller l'attention qu'un coup de massue sur la tête ?\par
La compassion pour la patrie est le seul sentiment qui ne sonne pas faux en ce moment, qui convienne à l'état où se trouvent les âmes et la chair des Français, qui ait à la fois l'humilité et la dignité l'une et l'autre convenables dans le malheur ; et aussi la simplicité que le malheur exige par-dessus tout. Évoquer en ce moment la grandeur historique de la France, ses gloires passées et futures, l'éclat dont son existence a été entourée, cela n'est pas possible sans une espèce de raidissement intérieur qui donne au ton quelque chose de forcé. Rien qui ressemble à de l'orgueil ne peut convenir aux malheureux.\par
Pour les Français qui souffrent, une telle évocation entre dans la catégorie des compensations. La recherche des compensations dans le malheur est un mal. Si cette évocation est trop souvent répétée, si elle est fournie comme unique source de réconfort, elle peut faire un mal illimité. Les Français sont affamés de grandeur. Mais aux malheureux ce n'est pas de la grandeur romaine qu'il faut ; ou cela leur semble de la dérision, ou cela leur empoisonne l'âme, comme ce fut le cas en Allemagne.\par
La compassion pour la France n'est pas une compensation, mais une spiritualisation des souffrances subies ; elle peut transfigurer même les souffrances les plus charnelles, le froid, la faim. Celui qui a froid et faim, et qui est tenté d'avoir pitié de soi-même, peut, au lieu de cela, à travers sa propre chair contractée, diriger sa pitié vers la France ; le froid et la faim mêmes font alors entrer l'amour de la France par la chair jusqu'au fond de l'âme. Et cette compassion peut sans obstacles franchir les frontières, s'étendre à tous les pays malheureux, à tous les pays sans exception ; car toutes les populations humaines sont soumises aux misères de notre condition. Alors que l'orgueil de la grandeur nationale est par nature exclusif, non transposable, la compassion est universelle par nature ; elle est seulement plus virtuelle pour les choses lointaines et étrangères, plus réelle, plus charnelle, plus chargée de sang, de larmes et d'énergie efficace pour les choses proches.\par
L'orgueil national est loin de la vie quotidienne. En France, il ne peut trouver d'expression que dans la résistance ; mais beaucoup, ou bien n'ont pas l'occasion de prendre effectivement part à la résistance, ou n'y consacrent pas tout leur temps. La compassion pour la France est un mobile au moins aussi énergique pour l'action de résistance ; mais de plus elle peut trouver une expression quotidienne, ininterrompue, en toutes espèces d'occasions, même les plus ordinaires, par un accent de fraternité dans les relations entre Français. La fraternité germe aisément dans la compassion pour un malheur qui, tout en infligeant à chacun sa part de souffrance, met en péril quelque chose de bien plus précieux que le bien-être de chacun. L'orgueil national, soit dans la prospérité, soit dans le malheur, est incapable de susciter une fraternité réelle, chaleureuse. Il n'y en avait pas chez les Romains. Ils ignoraient les sentiments vraiment tendres.\par
Un patriotisme inspiré par la compassion donne à la partie la plus pauvre du peuple une place morale privilégiée. La grandeur nationale n'est un excitant parmi les couches sociales d'en bas que dans les moments où chacun peut espérer, en même temps que la gloire du pays, une part personnelle à cette gloire aussi large qu'il peut désirer. Ce fut le cas au début du règne de Napoléon. N'importe quel petit gars de France, né dans n'importe quel faubourg, avait le droit de porter en son cœur n'importe quel rêve d'avenir ; aucune ambition ne pouvait être grande au point d'être absurde. On savait que toutes les ambitions ne seraient pas accomplies, mais chacune en particulier avait des chances de l'être, et beaucoup pouvaient l'être partiellement. Un document singulier de l'époque affirme que la popularité de Napoléon était due, moins au dévouement des Français pour sa personne, qu'aux possibilités d'avancement, aux chances de faire carrière qu'il leur offrait. C'est exactement le sentiment qui apparaît dans Le Rouge et le Noir. Les romantiques furent des enfants qui s'ennuyaient parce qu'il n'y avait plus devant eux la perspective d'une ascension sociale illimitée. Ils cherchèrent la gloire littéraire comme produit de remplacement.\par
Mais cet excitant n'existe que dans les moments troublés. On ne peut dire qu'il s'adresse jamais au peuple comme tel ; tout homme du peuple, qui le subit rêve de sortir du peuple, de sortir de l'anonymat qui définit la condition populaire. Cette ambition, quand elle est largement répandue, est l'effet d'un état social troublé et la cause de troubles aggravés ; car la stabilité sociale est pour elle un obstacle. Bien que ce soit un stimulant, on ne peut dire que ce soit quelque chose de sain ni pour l'âme ni pour le pays. Il est possible que ce stimulant ait une large place dans le mouvement actuel de résistance ; car quant à l'avenir de la France, l'illusion est facilement accueillie, et quant à l'avenir personnel, n'importe qui, s'il a su faire ses preuves au milieu du danger, peut s'attendre à n'importe quoi dans l'état de révolution latente où se trouve le pays. Mais s'il en est ainsi, c'est un danger terrible pour la période de reconstruction, et il est urgent de trouver un autre stimulant.\par
Dans une période de stabilité sociale, où sauf exception ceux qui se trouvent dans l'anonymat y demeurent plus ou moins, où ils ne songent même pas à en sortir, le peuple ne peut pas se sentir chez lui dans un patriotisme fondé sur l'orgueil et l'éclat de la gloire. Il y est aussi étranger que dans les salons de Versailles, qui en sont une expression. La gloire est le contraire de l'anonymat. Si aux gloires militaires on ajoute les gloires littéraires, scientifiques et autres, il continuera à se sentir étranger. Savoir que certains de ces Français couverts de gloire sont sortis du peuple ne lui apportera, en période stable, aucun réconfort ; car s'ils en sont sortis, ils ont cessé d'en être.\par
Au contraire, si la patrie lui est présentée comme une chose belle et précieuse, mais d'une part imparfaite, d'autre part très fragile, exposée au malheur, qu'il faut chérir et préserver, il s'en sentira avec raison plus proche que les autres classes sociales. Car le peuple a le monopole d'une connaissance, la plus importante de toutes peut-être, celle de la réalité du malheur ; et par là même il sent bien plus vivement combien sont précieuses les choses qui méritent d'y être soustraites, combien chacun est obligé de les chérir, de les protéger. Le mélodrame reflète cet état de la sensibilité populaire. Pourquoi c'est une forme littéraire tellement mauvaise, c'est une question qui vaudrait la peine d'être étudiée. Mais loin que ce soit un genre faux, il est très près, en un sens, de la réalité.\par
Si une telle relation s'établissait entre le peuple et la patrie, il ne ressentirait plus ses propres souffrances comme des crimes de la patrie envers lui, mais comme des maux soufferts par la patrie en lui. La différence est immense. En un autre sens, elle est légère, et il suffirait de peu de chose pour la franchir. Mais peu de chose qui vienne d'un autre monde. Cela suppose une dissociation entre la patrie et l’État. Cela est possible si la grandeur du genre cornélien est abolie. Mais cela impliquerait l'anarchie si, en compensation l'État ne trouve pas moyen d'acquérir par lui-même un surcroît de considération.\par
Pour cela, il doit certainement ne pas revenir aux anciennes modalités de la vie parlementaire et de la lutte des partis. Mais le plus important peut-être est la refonte totale de la police. Les circonstances y seraient favorables. La police anglaise serait intéressante à étudier. En tout cas, la libération du territoire entraînera, il faut l'espérer, la liquidation du personnel de la police, hors ceux qui ont personnellement agi contre l'ennemi. Il faut mettre à la place des hommes qui aient la considération publique, et, comme aujourd'hui malheureusement l'argent et les diplômes en sont la source principale, il faut exiger même à partir des agents et des inspecteurs un degré d'instruction assez élevé, plus haut des diplômes très sérieux, et rétribuer largement. Même, si la mode des grandes Écoles continue en France – ce qui peut-être n'est pas désirable –, il en faudrait une pour la police, recrutée par concours. Ce sont des méthodes grossières, mais quelque chose de ce genre est indispensable. De plus, ce qui est encore beaucoup plus important, il ne faut plus de catégories sociales comme celles des prostituées et des repris de justice, qui aient une existence officielle comme bétail livré au bon plaisir de la police et lui fournissant à la fois des victimes et des complices ; car une double contamination est alors inévitable, le contact déshonore des deux côtés. Il faut abolir en droit l'une et l'autre de ces catégories.\par
Il faut aussi que le crime d'improbité envers l'État chez les hommes publics soit effectivement puni plus sévèrement que le vol à main armée.\par
L’État dans sa fonction administrative doit apparaître comme l'intendant des biens de la patrie ; un intendant plus ou moins bon, et dont il faut raisonnablement s'attendre qu'il soit en général plutôt mauvais que bon, parce que sa tâche est difficile et accomplie dans des conditions moralement défavorables. L'obéissance n'en est pas moins obligatoire, non pas à cause d'un droit que possèderait l'État à commander, mais parce qu'elle est indispensable à la conservation et au repos de la patrie. Il faut obéir à l'État, quel qu'il soit, à peu près comme des enfants affectueux, que les parents en voyage ont confiés à une gouvernante médiocre, lui obéissent néanmoins pour l'amour des parents. Si l'État n'est pas médiocre, tant mieux ; il faut d'ailleurs toujours que la pression de l'opinion publique s'exerce comme stimulant pour le pousser à sortir de la médiocrité ; mais qu'il soit médiocre ou non, l'obligation d'obéissance est identique.\par
Elle n'est certes pas illimitée, mais elle ne peut avoir d'autre limite que la révolte de la conscience. Aucun critérium ne peut être fourni pour cette limite ; il est même impossible à chacun de s'en fixer un à son propre usage une fois pour toutes ; quand on sent qu'on ne peut plus obéir, on désobéit. Mais en tout cas une condition nécessaire, quoique non suffisante, pour pouvoir désobéir sans crime, c'est d'être poussé par une obligation si impérieuse qu'elle contraigne à mépriser tous les risques sans exception. Si l'on incline à désobéir, mais qu'on soit arrêté par l'excès du danger, on est impardonnable, selon les cas, ou bien d'avoir songé à désobéir, ou bien de ne l'avoir pas fait. Au reste, toutes les fois qu'on n'est pas rigoureusement obligé de désobéir, on est rigoureusement obligé d'obéir. Un pays ne peut pas posséder la liberté s'il n'est pas reconnu que la désobéissance envers les autorités publiques, toutes les fois qu'elle ne procède pas d'un sentiment impérieux de devoir, déshonore plus que le vol. C'est-à-dire que l'ordre public doit être tenu pour plus sacré que la propriété privée. Les pouvoirs publics peuvent répandre cette manière de voir par l'enseignement et par des mesures appropriées qu'il s'agirait d'inventer.\par
Mais seule la compassion pour la patrie, la préoccupation anxieuse et tendre de lui éviter le malheur, peut donner à la paix, et notamment à la paix civile, ce que la guerre civile ou étrangère possède malheureusement d'elle-même ; quelque chose d'exaltant, de touchant, de poétique, de sacré. Cette compassion seule peut nous faire retrouver le sentiment, depuis si longtemps perdu, d'ailleurs si rarement éprouvé au cours de l'histoire, que Théophile exprimait dans le beau vers : « La sainte majesté des lois. »\par
Le moment où Théophile écrivait ce vers est peut-être le dernier moment où ce sentiment ait été profondément ressenti en France. Ensuite est venu Richelieu, puis la Fronde, puis Louis XIV, puis le reste. Montesquieu a vainement essayé de le faire de nouveau pénétrer dans le public au moyen d'un livre. Les hommes de 1789 s'en réclamaient, mais ils ne l'avaient pas au fond du cœur, sans quoi le pays n'aurait pas glissé si facilement dans la guerre à la fois civile et étrangère.\par
Depuis, notre langage même est devenu impropre à l'exprimer. C'est là pourtant le sentiment qu'on essaie d'évoquer, ou sa réplique plus pâle, quand on parle de légitimité. Mais nommer un sentiment n'est pas un procédé suffisant pour le susciter. C'est là une vérité fondamentale et que nous oublions trop.\par
Pourquoi se mentir à soi-même ? En 1939, avant la guerre, sous le régime des décrets-lois, il n'y avait déjà plus de légitimité républicaine. Elle était partie comme la jeunesse de Villon « qui son partement m'a celé », sans bruit, sans prévenir qu'elle partait, et sans que personne ait fait un geste, dit un mot pour la retenir. Quant au sentiment de légitimité, il était tout à fait mort. Qu'il reparaisse maintenant dans les pensées des exilés, qu'il occupe une certaine place, à côté d'autres sentiments en fait incompatibles avec lui, dans les rêves de guérison d'un peuple malade, cela ne signifie rien ou peu de chose. S'il était nul en 1939, comment serait-il efficace, immédiatement après des années de désobéissance systématique ?\par
D'autre part, la Constitution de 1875 ne peut plus être un fondement de légitimité après avoir sombré en 1940 dans l'indifférence ou même le mépris général, après avoir été abandonnée par le peuple de France. Car le peuple de France l'a abandonnée. Ni les groupes de résistance, ni les Français de Londres n'y peuvent rien. Si une ombre de regret a été exprimée, ce ne fut pas par une portion du peuple, mais par des parlementaires, chez qui la profession maintenait vivant un intérêt pour les institutions républicaines mort partout ailleurs. Encore une fois, peu importe que longtemps après il ait quelque peu reparu. Actuellement la faim communique à la III\textsuperscript{e} République toute la poésie d'une époque où il y avait du pain. C'est une poésie fugitive. D'ailleurs en même temps le dégoût ressenti plusieurs années et qui a atteint son degré extrême en 1940 persiste. (La III\textsuperscript{e} République a d'ailleurs été condamnée dans un texte émanant officiellement de Londres ; dès lors elle peut difficilement être prise comme un fondement de légitimité.)\par
Il est néanmoins certain que dans la mesure où les choses de Vichy disparaîtront, dans la mesure où des institutions révolutionnaires, peut-être communistes, ne surgiront pas, il y aura un retour des structures de la III\textsuperscript{e} République. Mais cela simplement parce qu'il y aura un vide et qu'il faudra quelque chose. C'est là de la nécessité, non de la légitimité. Il y correspond dans le peuple, non pas de la fidélité, mais une morne résignation. La date de 1789 éveille, elle, un écho bien autrement profond ; mais il n'y correspond qu'une inspiration, non des institutions.\par
Étant donné qu'en fait il y a eu rupture de continuité dans notre histoire récente, la légitimité ne peut plus avoir un caractère historique ; elle doit procéder de la source éternelle de toute légitimité. Il faut que les hommes qui se proposeront au pays pour le gouverner reconnaissent publiquement certaines obligations répondant aux aspirations essentielles du peuple, éternellement inscrites au fond des âmes ; il faut que le peuple ait confiance dans leur parole et dans leur capacité et reçoive le moyen de le témoigner ; et il faut que le peuple sente qu'en les acceptant il s'engage à leur obéir.\par
L'obéissance du peuple envers les pouvoirs publics, étant un besoin de la patrie, est de ce fait une obligation sacrée, et qui confère aux pouvoirs publics eux-mêmes, parce qu'ils en sont l'objet, le même caractère sacré. Ce n'est pas là l'idolâtrie envers l'État liée au patriotisme à la romaine. C'en est l'opposé. L'État est sacré, non pas à la manière d'une idole, mais comme les objets du culte, ou les pierres de l'autel, ou l'eau du baptême, ou toute autre chose semblable. Tout le monde sait que c'est seulement de la matière. Mais des morceaux de matière sont regardés comme sacrés parce qu'ils servent à un objet sacré. C'est l'espèce de majesté qui convient à l'État.\par
Si on ne sait pas insuffler au peuple de France une semblable inspiration, il aura le choix seulement entre le désordre et l'idolâtrie. L'idolâtrie peut prendre la forme communiste. C'est ce qui se produirait probablement. Elle peut aussi prendre la forme nationale. Elle aurait alors vraisemblablement pour objet le couple, si caractéristique de notre époque, constitué par un homme acclamé comme chef et la machine d'acier de l'État. Or d'une part, la publicité peut fabriquer des chefs ; d'autre part, si les circonstances amènent un homme de valeur véritable à une telle fonction, il devient rapidement prisonnier de son rôle d'idole. Autrement dit, en langage moderne, l'absence d'une inspiration pure ne laisserait au peuple français d'autres possibilités que le désordre, le communisme ou le fascisme.\par
Il y a des gens, par exemple en Amérique, qui se demandent si les Français de Londres n'inclineraient pas au fascisme. C'est très mal poser la question. Les intentions par elles-mêmes n'ont que très peu d'importance, excepté quand elles vont tout droit vers le mal, car pour le mal il y a toujours des ressources à portée de la main. Mais les bonnes intentions ne comptent que jointes aux ressources correspondantes. Saint Pierre n'avait nullement l'intention de renier le Christ ; mais il l'a fait parce qu'il ne possédait pas en lui-même la grâce qui lui aurait permis de s'en abstenir. Et même l'énergie, le ton catégorique dont il avait usé pour affirmer l'intention contraire avaient contribué à le priver de cette grâce. C'est un exemple qui vaut qu'on y pense dans toutes les épreuves que propose la vie.\par
Le problème est de savoir si les Français de Londres possèdent les moyens nécessaires pour empêcher le peuple de France de glisser dans le fascisme, et le retenir en même temps de tomber, soit dans le communisme, soit dans le désordre. Fascisme, communisme et désordre n'étant que les expressions à peine distinctes, équivalentes, d'un mal unique, il s'agit de savoir s'ils possèdent un remède à ce mal.\par
S'ils ne le possèdent pas, leur raison d'être, qui est le maintien de la France dans la guerre, se trouve entièrement épuisée par la victoire, qui doit en ce cas les replonger dans la foule de leurs compatriotes. S'ils le possèdent, ils doivent avoir déjà commencé à l'appliquer, dans une large quantité, et efficacement, dès avant la victoire. Car un tel traitement ne peut pas être commencé au milieu des désordres nerveux qui accompagneront, en chaque individu et dans les foules, la libération du pays. Il peut encore moins être commencé une fois les nerfs apaisés, si toutefois l'apaisement survient un jour ; il serait beaucoup trop tard, il ne pourrait plus même être question d'aucun traitement.\par
L'important n'est donc pas qu'ils affirment devant l'étranger leur droit à gouverner la France ; de même que pour un médecin l'important n'est pas d'affirmer son droit à soigner un malade. L'essentiel est d'avoir établi un diagnostic, conçu une thérapeutique, choisi des médicaments, vérifié qu'ils sont à la disposition du malade. Quand un médecin sait faire tout cela, non sans risque d'erreur, mais avec des chances raisonnables d'avoir vu juste, alors, si on veut l'empêcher d'exercer sa fonction et mettre à sa place un charlatan, il a le devoir de s'y opposer de toutes ses forces. Mais si, dans un endroit sans médecins, plusieurs ignorants s'agitent autour d'un malade dont l'état demande les soins les plus précis, les plus éclairés, qu'importe aux mains duquel d'entre eux il se trouve pour mourir ou pour être sauvé seulement par le hasard ? Sans doute, il vaut mieux de toute manière qu'il soit aux mains de ceux qui l'aiment. Mais ceux qui l'aiment ne lui infligeront pas la souffrance d'une bataille faisant rage à son chevet, à moins de se savoir en possession d'une méthode susceptible de le sauver.
\section[{Troisième partie. L’enracinement}]{Troisième partie. \\
L’enracinement}\renewcommand{\leftmark}{Troisième partie. \\
L’enracinement}

\noindent \par
Le problème d'une méthode pour insuffler une inspiration à un peuple est tout neuf. Platon y fait des allusions dans le {\itshape Politique} et ailleurs ; sans doute il y avait des enseignements à ce sujet dans le savoir secret de l'Antiquité pré-romaine, qui a entièrement disparu. Peut-être s'entretenait-on encore de ce problème et d'autres semblables dans les milieux des Templiers et des premiers francs-maçons. Montesquieu, sauf erreur, l'a ignoré. Rousseau, qui était un esprit puissant, en a très clairement reconnu l'existence, mais n'est pas allé plus loin. Les hommes de 1789 ne semblent pas l'avoir soupçonné. En 1793, sans s'être donné la peine de le poser, moins encore de l'étudier, on a improvisé des solutions hâtives : fêtes de l'Être suprême, fêtes de la Déesse Raison. Elles ont été ridicules et odieuses. Au XIX\textsuperscript{e} siècle, le niveau des intelligences était descendu bien au-dessous du domaine où se posent de telles questions.\par
De nos jours, on a étudié et pénétré le problème de la propagande. Hitler notamment a apporté sur ce point une contribution durable au patrimoine de la pensée humaine. Mais c'est un problème tout autre. La propagande ne vise pas à susciter une inspiration ; elle ferme, elle condamne tous les orifices par où une inspiration pourrait passer ; elle gonfle l'âme tout entière avec du fanatisme. Ses procédés ne peuvent convenir pour l'objet contraire. Il ne s'agit pas non plus d'adopter des procédés opposés ; la relation de causalité n'est pas si simple.\par
Il ne faut pas penser non plus que l'inspiration d'un peuple est un mystère réservé à Dieu seul, et qui par suite échappe à toute méthode. Le degré suprême et parfait de la contemplation mystique est chose infiniment plus mystérieuse encore, et pourtant saint Jean de la Croix a écrit sur la manière d'y parvenir des traités qui, par la précision scientifique, l'emportent de loin sur tout ce qu'ont écrit les psychologues ou pédagogues de notre époque. S'il a cru devoir le faire, il avait raison sans doute, car il était compétent ; la beauté de son œuvre est une marque suffisamment évidente d'authenticité. À vrai dire, depuis une antiquité indéterminée, bien antérieure au christianisme, jusqu'à la deuxième moitié de la Renaissance, il a toujours été universellement reconnu qu'il y a une méthode dans les choses spirituelles et dans tout ce qui a rapport au bien de l'âme. L'emprise de plus en plus méthodique que les hommes exercent sur la matière depuis le XVI\textsuperscript{e} siècle leur a fait croire, par contraste, que les choses de l'âme sont ou bien arbitraires, ou bien livrées à une magie, à l'efficacité immédiate des intentions et des mots.\par
Il n'en est pas ainsi. Tout dans la création est soumis à la méthode, y compris les points d'intersection entre ce monde et l'autre. C'est ce qu'indique le mot {\itshape Logos}, qui veut dire relation plus encore que parole. La méthode est seulement autre quand le domaine est autre. À mesure qu'on s'élève, elle s'accroît en rigueur et en précision. Il serait bien étrange que l'ordre des choses matérielles reflétât davantage de sagesse divine que l'ordre des choses de l'âme. Le contraire est vrai.\par
Il est fâcheux pour nous que ce problème, sur lequel, sauf erreur, il n'y a rien qui puisse nous guider, soit précisément le problème que nous avons aujourd'hui à résoudre de toute urgence, sous peine non pas tant de disparaître que de n'avoir jamais existé.\par
De plus, si Platon par exemple en avait formulé une solution générale, il ne nous suffirait pas de l'étudier pour nous tirer d'affaire ; car nous sommes devant une situation à l'égard de laquelle l'histoire nous est d'un faible secours. Elle ne nous parle d'aucun pays qui ait été dans une situation ressemblant même de loin à celle où la France sera susceptible de se trouver en cas de défaite allemande. D'ailleurs nous ignorons même ce que sera cette situation. Nous savons seulement qu'elle sera sans précédent. Ainsi, même si nous savions comment on insuffle une inspiration à un pays, nous ne saurions pas encore comment procéder pour la France.\par
D'un autre côté, puisqu'il s'agit d'un problème pratique, la connaissance d'une solution générale n'est pas indispensable pour un cas particulier. Quand une machine s'arrête, un ouvrier, un contremaître, un ingénieur, peuvent apercevoir un procédé pour la remettre en marche, sans posséder une connaissance générale de la réparation des machines. La première chose qu'on fasse en pareil cas, c'est de regarder la machine. Pourtant, pour la regarder utilement, il faut avoir dans l'esprit la notion même des relations mécaniques.\par
De la même manière, en regardant au jour le jour la situation changeante de la France, il faut avoir dans l'esprit la notion de l'action publique comme mode d'éducation du pays.\par
Il ne suffit pas d'avoir aperçu cette notion, d'y avoir fait attention, de l'avoir comprise, il faut l'installer en permanence dans l'âme, de manière qu'elle soit présente même quand l'attention se porte vers autre chose.\par
Il y faut un effort d'autant plus grand que parmi nous c'est une pensée entièrement nouvelle. Depuis la Renaissance, l'activité publique n'a jamais été conçue sous cet aspect, mais seulement comme moyen pour établir une forme de pouvoir regardée comme désirable à tel ou tel égard.\par
L'éducation – qu'elle ait pour objet des enfants ou des adultes, des individus ou un peuple, ou encore soi-même – consiste à susciter des mobiles. Indiquer ce qui est avantageux, ce qui est obligatoire, ce qui est bien, incombe à l'enseignement. L'éducation s'occupe des mobiles pour l'exécution effective. Car jamais aucune action n'est exécutée en l'absence de mobiles capables de fournir pour elle la somme indispensable d'énergie. Vouloir conduire des créatures humaines – autrui ou soi-même – vers le bien en indiquant seulement la direction, sans avoir veillé à assurer la présence des mobiles correspondants, c'est comme si l'on voulait, en appuyant sur l'accélérateur, faire avancer une auto vide d'essence.\par
Ou encore c'est comme si l'on voulait faire brûler une lampe à huile sans y avoir mis d'huile. Cette erreur a été dénoncée dans un texte assez célèbre, assez lu, relu et cité depuis vingt siècles. Néanmoins on la commet toujours.\par
On peut assez facilement classer les moyens d'éducation enfermés dans l'action publique.\par
D'abord la crainte et l'espérance, provoquées par les menaces et les promesses.\par
La suggestion.\par
L'expression, soit officielle, soit approuvée par une autorité officielle, d'une partie des pensées qui, dès avant d'avoir été exprimées, se trouvaient réellement au cœur des foules, ou au cœur de certains éléments actifs de la nation.\par
L'exemple.\par
Les modalités mêmes de l'action et des organisations forgées pour elle.\par
Le premier moyen est le plus grossier, et il est toujours employé. Le second l'est par tous aujourd'hui ; c'est celui dont le maniement a été génialement étudié par Hitler.\par
\par
Les trois autres sont ignorés.\par
Il faut essayer de les concevoir relativement aux trois formes successives que notre action publique est susceptible d'avoir ; la forme actuelle ; l'acte de la prise du pouvoir au moment de la libération du territoire ; l'exercice du pouvoir à titre provisoire au cours des mois suivants.\par
Actuellement nous ne disposons que de deux intermédiaires, la radio et le mouvement clandestin. Pour les foules françaises, la radio compte presque seule.\par
Le troisième des cinq procédés énumérés ne doit nullement être confondu avec le second. La suggestion est, comme l'a vu Hitler, une emprise. Elle constitue une contrainte. La répétition d'une part, d'autre part la force dont le groupe d'où elle émane dispose ou qu'il se propose de conquérir, lui donnent une grande part de son efficacité.\par
L'efficacité du troisième procédé est d'une tout autre espèce. Elle a son fondement dans la structure cachée de la nature humaine.\par
Il arrive qu'une pensée, parfois intérieurement formulée, parfois non formulée, travaille sourdement l'âme et pourtant n'agit sur elle que faiblement.\par
Si l'on entend formuler cette pensée hors de soi-même, par autrui et par quelqu'un aux paroles de qui on attache de l'attention, elle en reçoit une force centuplée et peut parfois produire une transformation intérieure.\par
Il arrive aussi qu'on ait besoin, soit qu'on s'en rende compte ou non, d'entendre certaines paroles, qui, si elles sont effectivement prononcées et viennent d'un lieu d'où l'on attende naturellement du bien, injectent du réconfort, de l'énergie et quelque chose comme une nourriture.\par
Ces deux fonctions de la parole, ce sont, dans la vie privée, des amis ou des guides naturels qui les remplissent ; d'ailleurs, en fait, très rarement.\par
Mais il est des circonstances où le drame public l'emporte tellement, dans la vie personnelle de chacun, sur les situations particulières, que beaucoup de pensées sourdes et de besoins sourds de cette espèce se trouvent être les mêmes chez presque tous les êtres humains qui composent un peuple.\par
Cela fournit la possibilité d'une action qui, tout en ayant pour objet tout un peuple, reste par essence une action, non pas collective, mais personnelle. Ainsi, loin d'étouffer les ressources profondes situées au secret de chaque âme, ce que fait inévitablement, par la nature des choses, toute action collective, quelle que soit l'élévation des buts poursuivis, cette espèce d'action les réveille, les excite et les fait croître.\par
Mais qui peut exercer une telle action ?\par
Dans les circonstances habituelles, il n'est peut-être aucun lieu d'où elle puisse être exercée. Des obstacles très forts empêchent qu'elle puisse l'être, sinon partiellement et à un faible degré, par un gouvernement. D'autres obstacles apportent un empêchement semblable à ce qu'elle soit exercée d'un lieu autre que l'État.\par
Mais à cet égard les circonstances où se trouve actuellement la France sont merveilleusement, providentiellement favorables.\par
À beaucoup d'autres égards il a été désastreux que la France n'ait pas eu à Londres, comme d'autres pays, un gouvernement régulier. Mais à cet égard-là c'est exceptionnellement heureux ; et de même il est heureux à cet égard que l'affaire d'Afrique du Nord n'ait pas abouti à la transformation du Comité National en gouvernement régulier.\par
La haine de l'État, qui existe d'une manière latente, sourde et très profonde en France depuis Charles VI, empêche que des paroles émanant directement d'un gouvernement puissent être accueillies par chaque Français comme la voix d'un ami.\par
D'un autre côté, dans une action de cette espèce, les paroles doivent avoir un caractère officiel pour être vraiment efficaces.\par
Les chefs de la France combattante constituent quelque chose d'analogue à un gouvernement dans la mesure exacte qui est indispensable pour que leurs paroles aient un caractère officiel.\par
Le mouvement garde assez sa nature originelle, celle d'une révolte jaillie du fond de quelques âmes fidèles et complètement isolées, pour que les paroles qui en émanent puissent avoir à l'oreille de chaque Français l'accent proche, intime, chaleureux, tendre d'une voix d'ami.\par
Et par-dessus le reste le général de Gaulle, entouré de ceux qui l'ont suivi, est un symbole. Le symbole de la fidélité de la France à elle-même, concentrée un moment en lui presque seul ; et surtout le symbole de tout ce qui dans l'homme refuse la basse adoration de la force.\par
Tout ce qui est dit en son nom a en France 1'autorité attachée à un symbole. Par suite, quiconque parle en son nom peut à son gré et selon ce qui paraît préférable à tel ou tel moment puiser l'inspiration au niveau des sentiments et des pensées qui fermentent en fait dans l'esprit des Français, ou à un niveau plus élevé, et en ce cas aussi élevé qu'il veut ; rien n'empêche certains jours de la puiser dans la région située au-dessus du ciel. Autant ce serait inconvenant pour des paroles émanant d'un gouvernement, souillé par nécessité de toutes les bassesses liées à l'exercice d'un pouvoir, autant c'est convenable pour des paroles émanant d'un symbole qui représente ce qui aux yeux de chacun est le plus haut.\par
Un gouvernement qui emploie des paroles, des pensées trop élevées pour lui, loin d'en recevoir un éclat quelconque, les discrédite et se ridiculise. C'est ce qui s'est produit pour les principes de 1789 et la formule « Liberté, Égalité, Fraternité » au cours de la III\textsuperscript{e} République. C'est ce qui s'est produit pour les mots, souvent par eux-mêmes d'un niveau très élevé, mis en avant par la prétendue Révolution Nationale. Dans de ce second cas, il est vrai, la honte de la trahison a amené le discrédit avec une rapidité foudroyante. Mais presque certainement il serait venu même autrement, quoique beaucoup moins vite.\par
Le mouvement français de Londres a actuellement, pour peu de temps peut-être, ce privilège extraordinaire qu'étant dans une large mesure symbolique il lui est permis de faire rayonner les inspirations les plus élevées sans discrédit pour elles ni inconvenance de sa part.\par
Ainsi de l'irréalité même dont il est atteint dès l'origine – à cause de l'isolement primitif de ceux qui l'ont lancé – il peut tirer, s'il sait en faire usage, une bien plus grande plénitude de réalité.\par
« L'efficacité est rendue parfaite dans la faiblesse », dit saint Paul.\par
C'est un singulier aveuglement qui a causé, dans une situation pleine de possibilités aussi merveilleuses, le désir de descendre à la situation banale, vulgaire, d'un gouvernement d'émigrés. Il est providentiel que ce désir n'ait pas été satisfait.\par
À l'égard de l'étranger, d'ailleurs, les avantages de la situation sont analogues.\par
Depuis 1789, la France a en fait parmi les nations une position unique. C'est quelque chose de récent ; 1789 n'est pas loin. De la fin du XIV\textsuperscript{e} siècle, époque des répressions féroces accomplies dans les villes flamandes et françaises par Charles VI enfant, jusqu'en 1789, la France n'avait guère représenté aux yeux de l'étranger, du point de vue politique, que la tyrannie de l'absolutisme et la servilité des sujets. Quand du Bellay écrivait : « France, mère des arts, des armes et des lois », le dernier mot était de trop ; comme Montesquieu l'a très bien montré, comme Retz avant lui l'avait expliqué avec une lucidité géniale, il n'y avait pas du tout de lois en France depuis la mort de Charles VI. De 1715 à 1789, la France s'est mise à l'école de l'Angleterre avec une ferveur pleine d'humilité. Les Anglais semblaient alors être seuls dignes du nom de citoyens au milieu de populations esclaves. Mais après 1792, quand la France, après avoir remué le cœur de tous les opprimés, se trouva engagée dans une guerre où elle avait l'Angleterre pour ennemie, tout le prestige des idées de justice et de liberté fut concentré sur elle. Il en est résulté pour le peuple français au cours du siècle suivant une espèce d'exaltation que les autres peuples n'ont pas connue, et dont ils ont reçu de lui le rayonnement.\par
La Révolution française a correspondu, malheureusement d'ailleurs, à un si violent arrachement du passé sur tout le continent européen qu'une tradition qui remonte à 1789 est en pratique l'équivalent d'une tradition antique.\par
La guerre de 1870 a montré ce qu'était la France aux yeux du monde. Dans cette guerre, les Français étaient les agresseurs, malgré la ruse de la dépêche d'Ems ; cette ruse même est la preuve que l'agression est venue du côté français. Les Allemands, désunis entre eux, frémissants encore du souvenir de Napoléon, s'attendaient à être envahis. Ils furent très surpris d'entrer dans la France comme dans du beurre. Mais ils furent bien plus surpris encore de se trouver un objet d'horreur aux yeux de l'Europe, alors que leur seule faute était de s'être défendus victorieusement. Mais la vaincue était la France ; et, malgré Napoléon, à cause de 1789, c'était assez pour que les vainqueurs fissent horreur.\par
On voit dans le journal intime du prince impérial Frédéric quelle surprise douloureuse causa aux meilleurs des Allemands cette réprobation pour eux incompréhensible.\par
De là peut-être date chez les Allemands le complexe d'infériorité, le mélange en apparence contradictoire d'une mauvaise conscience et du sentiment qu'on leur fait une injustice, et la réaction de férocité. En tout cas, à partir de ce moment, dans la conscience européenne, le Prussien se substitua à ce qui jusqu'alors avait semblé le type de l'Allemand, c'est-à-dire le musicien rêveur aux yeux bleus, « gutmütig », fumeur de pipe et buveur de bière, totalement inoffensif, qu'on trouve encore dans Balzac. Et l'Allemagne ne cessa de devenir de plus en plus semblable à sa nouvelle image.\par
La France subit un préjudice moral à peine moins grand. On admire son relèvement après 1871. Mais on ne voit pas à quel prix il a été acheté. La France était devenue réaliste. Elle avait cessé de croire en elle-même. Le massacre de la Commune, tellement surprenant par la quantité et par la férocité, mit d'une manière permanente chez les ouvriers le sentiment d'être des parias exclus de la nation, et chez les bourgeois, par l'effet d'une mauvaise conscience, une espèce de peur physique des ouvriers. On s'en est aperçu encore en juin 1936 ; et l'effondrement de juin 1940 est en un sens un effet direct de cette guerre civile si brève et si sanglante de mai 1871 qui a persisté sourdement pendant presque trois quarts de siècle. Dès lors l'amitié entre la jeunesse des Écoles et le peuple, amitié dans laquelle toute la pensée française du XIX\textsuperscript{e} siècle avait puisé une sorte de nourriture, devenait un simple souvenir. D'un autre côté, l'humiliation de la défaite orientait la pensée de la jeunesse bourgeoise, par réaction, vers la conception la plus médiocre de la grandeur nationale. Obsédée par la conquête qu'elle avait subie et qui l'avait diminuée, la France ne se sentait plus capable d'une vocation plus haute que celle de conquérir.\par
\par
Ainsi la France devint une nation comme les autres, ne songeant plus qu'à se tailler dans le monde sa part de chair jaune et noire, et à se procurer en Europe l'hégémonie.\par
Après une vie de si intense exaltation, la chute à un niveau si bas ne pouvait s'opérer sans un profond malaise. Le point extrême de ce malaise a été juin 1940.\par
Il faut bien le dire, parce que c'est vrai, après le désastre la première réaction de la France a été de vomir son propre passé, son passé proche. Ce ne fut pas un effet de la propagande de Vichy. Au contraire, ce fut la cause qui procura d'abord à la Révolution Nationale une apparence de succès. Et ce fut une réaction légitime et saine. L'unique aspect du désastre qui pût être regardé comme un bien, c'était la possibilité de vomir un passé dont il avait été l'aboutissement. Un passé où la France n'avait pas fait autre chose que de réclamer les privilèges d'une mission qu'elle avait reniée parce qu'elle n'y croyait plus.\par
À l'étranger, l'écroulement de la France n'a causé d'émotion que là où l'esprit de 1789 avait apporté quelque chose.\par
L'anéantissement momentané de la France en tant que nation peut lui permettre de redevenir parmi les nations ce qu'elle a été et ce qu'on attendait depuis longtemps qu'elle redevînt, une inspiration. Et pour que la France retrouve une grandeur dans le monde – grandeur indispensable à la santé même de sa vie intérieure – il faut qu'elle devienne une inspiration avant d'être redevenue, par la défaite des ennemis, une nation. Après, ce serait probablement impossible pour plusieurs raisons.\par
Là aussi, le mouvement français de Londres est dans la meilleure situation qu'on puisse rêver, s'il sait l'utiliser. Il est exactement aussi officiel qu'il est nécessaire de l'être pour parler au nom d'un pays. N'ayant pas sur les Français d'autorité gouvernementale même nominale, même fictive, tirant tout du libre consentement, il a quelque chose d'un pouvoir spirituel. La fidélité incorruptible aux heures les plus sombres, le sang versé tous les jours librement en son nom, lui donnent droit à user librement des plus beaux mots du langage. Il est situé exactement comme il doit l'être pour faire entendre au monde le langage de la France. Un langage qui tire son autorité, non pas d'une puissance, qui a été anéantie par la défaite, ni d'une gloire, qui a été effacée par la honte, mais d'abord d'une élévation de pensée qui soit à la mesure de la tragédie présente, ensuite d'une tradition spirituelle gravée au cœur des peuples.\par
La double mission de ce mouvement est facile à définir. Aider la France à trouver au fond de son malheur une inspiration conforme à son génie et aux besoins actuels des hommes en détresse. Répandre cette inspiration, une fois retrouvée ou du moins entrevue, à travers le monde.\par
Si l'on s'attache à cette double mission, beaucoup de choses d'un ordre moins élevé seront accordées par surcroît. Si l'on s'attache d'abord à ces choses-là, celles-là même nous seront refusées.\par
Bien entendu, il ne s'agit pas d'une inspiration verbale. Toute inspiration réelle passe dans les muscles et sort en actions ; et aujourd'hui les actions des Français ne peuvent être que celles qui contribuent à chasser l'ennemi.\par
Pourtant il ne serait pas juste de penser que le mouvement français de Londres a pour mission seulement d'élever au plus haut degré d'intensité possible l'énergie des Français dans la lutte contre l'ennemi.\par
Sa mission est d'aider la France à retrouver une inspiration authentique, et qui, par son authenticité même, s'épanche naturellement en dépense d'effort et d'héroïsme pour la libération du pays.\par
Cela ne revient pas au même.\par
C'est parce qu'il est nécessaire d'accomplir une mission d'un ordre si élevé que les moyens grossiers et efficaces des menaces, des promesses et de la suggestion ne sauraient suffire.\par
Au contraire, l'usage de paroles répondant à des pensées sourdes et à des besoins sourds des êtres humains qui composent le peuple français, c'est là un procédé merveilleusement bien adapté à la tâche qu'il s'agit d'accomplir, à condition qu'il soit mis en œuvre comme il faut.\par
Pour cela, il faut d'abord en France un organisme récepteur. C'est-à-dire des gens dont la première tâche, la première préoccupation, soit de discerner ces pensées sourdes, ces besoins sourds, et de les communiquer à Londres.\par
Ce qui est indispensable pour cette tâche, c'est un intérêt passionné pour les êtres humains, quels qu'ils soient, et pour leur âme, une capacité de se mettre à leur place et de faire attention aux signes des pensées non exprimées, un certain sens intuitif de l'histoire en cours d'accomplissement, et la faculté d'exprimer par écrit des nuances délicates et des relations complexes.\par
Étant donné l'étendue et la complexité de la chose à observer, il devrait y avoir un grand nombre de tels observateurs ; mais en fait c'est impossible. Du moins est-il urgent d'utiliser ainsi quiconque est utilisable ainsi, sans exception.\par
En supposant qu'il y a en France un organe récepteur, insuffisant – il ne peut pas ne pas l'être – mais réel, la seconde opération, la plus importante de très loin, a lieu à Londres. C'est celle du choix. C'est celle qui est susceptible de modeler l'âme du pays.\par
La connaissance des paroles susceptibles d'avoir un écho au cœur des Français, comme répondant à quelque chose qui est déjà dans leur cœur, cette connaissance est uniquement une connaissance de fait. Elle ne contient aucune indication d'un bien, et la politique, comme toute activité humaine, est une activité dirigée vers un bien.\par
L'état du cœur des Français, ce n'est pas autre chose qu'un fait. En principe cela ne constitue ni un bien ni un mal ; en fait cela est composé d'un mélange de bien et de mal, selon des proportions qui peuvent beaucoup varier.\par
C'est là une vérité évidente, mais qu'il est bon de se répéter, parce que la sentimentalité naturellement attachée à l'exil pourrait la faire plus ou moins oublier.\par
Parmi toutes les paroles susceptibles d'éveiller un écho dans le cœur des Français, il faut choisir celles dont il est bon qu'un écho soit éveillé ; dire et redire celles-là ; taire les autres, afin de provoquer l'extinction de ce qu'il est avantageux de faire disparaître.\par
Quels seront les critères du choix ?\par
On peut en concevoir deux. L'un, le bien, au sens spirituel du mot. L'autre, l'utilité. C'est-à-dire, bien entendu, l'utilité relativement à la guerre et aux intérêts nationaux de la France.\par
Au sujet du premier critère, il y a tout d'abord un postulat à examiner. Il faut le peser très attentivement, très longuement, en son âme et conscience, puis l'adopter ou le rejeter une fois pour toutes.\par
Un chrétien ne peut que l'adopter.\par
C'est le postulat que ce qui est spirituellement bien est bien à tous égards, sous tous les rapports, en tout temps, en tout lieu, en toutes circonstances.\par
C'est ce qu'exprime le Christ par les paroles : « Est-ce qu'on récolte dans les épines des grappes mûres, ou dans les chardons des figues ? Ainsi tout arbre bon fait de beaux fruits ; l'arbre pourri fait des fruits mauvais. Un arbre bon ne peut pas porter de mauvais fruits, ni un arbre pourri porter de beaux fruits. »\par
Voici le sens de ces mots. Au-dessus du domaine terrestre, charnel, où se meuvent d'ordinaire nos pensées, et qui est partout un mélange inextricable de bien et de mal, il s'en trouve un autre, le domaine spirituel, où le bien n'est que bien et, même dans le domaine inférieur, ne produit que du bien ; où le mal n'est que mal et ne produit que du mal.\par
C'est une conséquence directe de la foi en Dieu. Le bien absolu n'est pas seulement le meilleur de tous les biens – ce serait alors un bien relatif – mais le bien unique, total, qui enferme en lui à un degré éminent tous les biens, y compris ceux que recherchent les hommes qui se détournent de lui.\par
Tout bien pur issu directement de lui a une propriété analogue.\par
Ainsi parmi la liste des échos susceptibles d'être excités de Londres dans le cœur des Français, il faut d'abord choisir tout ce qui est purement et authentiquement bien, sans aucune considération d'opportunité, sans aucun autre examen que celui de l'authenticité ; et il faut leur renvoyer tout cela, souvent, inlassablement, par l'intermédiaire de paroles aussi simples et nues que possible.\par
Bien entendu, tout ce qui est seulement du mal, de la haine, de la bassesse doit être de même rejeté, sans considération d'opportunité.\par
Restent les mobiles moyens, qui sont inférieurs au bien spirituel sans être par eux-mêmes nécessairement mauvais, et pour lesquels la question d'opportunité se pose.\par
Pour chacun de ceux-là, il faut examiner, complètement si possible, en faisant vraiment le tour, tous les effets qu'il est susceptible de produire, à tel, tel, ou tel égard, dans tel, tel et tel ensemble possible de circonstances.\par
Faute de ce soin, on peut par erreur provoquer ce dont on ne veut pas au lieu de ce qu'on veut.\par
Par exemple les pacifistes, après 1918, ont cru devoir faire appel au goût de la sécurité, du confort, pour être plus facilement écoutés. Ils espéraient ainsi conquérir assez d'influence pour diriger la politique extérieure du pays. Ils comptaient bien dans ce cas la diriger de manière à assurer la paix.\par
Ils ne se sont pas demandé quels effets auraient les mobiles excités, encouragés par eux, au cas où l'influence conquise, tout en étant grande, ne la serait pas assez pour procurer la direction de la politique étrangère.\par
Si seulement ils s'étaient posé la question, la réponse serait apparue tout de suite, et clairement. En pareil cas, les mobiles ainsi excités ne pouvaient ni empêcher ni retarder la guerre, mais seulement la faire gagner au camp le plus agressif, le plus belliqueux, et déshonorer ainsi pour longtemps l'amour même de la paix.\par
Soit dit en passant, le jeu même des institutions démocratiques, tel que nous le comprenons, est une invitation perpétuelle à cette espèce de négligence criminelle et fatale.\par
Pour éviter de la commettre, il faut pour chaque mobile se dire : ce mobile peut produire des effets dans tel, tel et tel milieu ; et dans quel autre encore ? Il peut produire des effets dans tel, tel et tel domaine ; et dans quel autre encore ? Telle, telle, telle situation peut se produire ; quelle autre encore ? Dans chacune, quels effets serait-il susceptible de produire dans chaque milieu, dans chaque domaine, immédiatement, plus tard, encore plus tard ? À quels égards chacun de ces effets possibles serait-il avantageux, à quels égards nuisible ? Quel semble être le degré de probabilité de chaque possibilité ?\par
Il faut considérer attentivement chacun de ces points et tous ces points ensemble ; suspendre quelques moments toute inclination vers un choix ; puis décider ; et courir, comme dans toute décision humaine, le risque d'erreur.\par
Le choix fait, il faut le mettre à l'épreuve de l'application et, bien entendu, l'appareil enregistreur placé en France s'efforce de discerner progressivement les résultats.\par
Mais l'expression n'est qu'un commencement. L'action est un outil plus puissant de modelage des âmes.\par
Elle a une double propriété à l'égard des mobiles. D'abord un mobile n'est vraiment réel dans l'âme que lorsqu'il a provoqué une action exécutée par le corps.\par
Il ne suffit pas d'encourager tels, tels ou tels mobiles présents ou embryonnaires au cœur des Français, en comptant sur ceux-ci pour réaliser eux-mêmes leurs propres mobiles en actions. Il faut de plus, de Londres, dans la plus grande mesure possible, le plus continuellement possible, avec le plus de détails possible, et par tous les moyens appropriés, radio ou autres, indiquer des actions.\par
Un soldat disait un jour, racontant son propre, comportement pendant une campagne : « J'ai obéi à tous les ordres, mais je sentais qu'il aurait été impossible pour moi, infiniment au-dessus de mon courage, d'aller au-devant d'un danger volontairement et sans ordres. »\par
Cette observation enferme une vérité très profonde. Un ordre est un stimulant d'une efficacité incroyable. Il enferme en lui-même, dans certaines circonstances, l'énergie indispensable à l'action qu'il indique.\par
Soit dit en passant, étudier en quoi consistent ces circonstances, qu'est-ce qui les définit, quelles en sont les variétés, en faire la liste complète, ce serait acquérir une clef pour la solution des problèmes les plus essentiels et les plus urgents de la guerre et de la politique.\par
La responsabilité clairement reconnue, imposant, des obligations précises et tout à fait strictes, pousse vers le danger de la même manière qu'un ordre. Elle ne se présente qu'une fois engagée dans l'action et par l'effet de telles ou telles circonstances particulières de l'action. L'aptitude à la reconnaître est d'autant plus grande que l'intelligence est plus claire ; elle dépend plus encore de la probité intellectuelle, vertu infiniment précieuse qui empêche de se mentir pour éviter l'inconfort.\par
Ceux qui peuvent s'exposer au danger sans la pression d'un ordre ou d'une responsabilité précise sont de trois espèces. Il y a ceux qui ont beaucoup de courage naturel, un tempérament dans une large mesure étranger à la peur, une imagination peu tournée au cauchemar ; ceux-là vont souvent au danger avec légèreté, dans un esprit aventureux, sans dépenser beaucoup d'attention pour choisir le danger. Il y a ceux pour qui le courage est difficile, mais qui en puisent l'énergie dans des mobiles impurs. Le désir d'une décoration, la vengeance, la haine, sont des exemples de ce genre de mobiles ; il y en a un très grand nombre, très différents selon les caractères et les circonstances. Il y a ceux qui obéissent à un ordre direct et particulier venu de Dieu.\par
Ce dernier cas est moins rare qu'on ne croit ; car là où il existe il est souvent secret, souvent même secret pour l'intéressé lui-même ; car ceux dont c'est le cas sont quelquefois au nombre de ceux qui croient qu'ils ne croient pas en Dieu. Pourtant, quoique moins rare qu'on ne croit, il n'est malheureusement pas fréquent.\par
Aux deux autres catégories correspond un courage qui, bien que souvent très spectaculaire et honoré du nom d'héroïsme, est très inférieur en qualité humaine à celui du soldat qui obéit aux ordres de ses chefs.\par
Le mouvement français de Londres a précisément le degré qui convient de caractère officiel pour que les directives envoyées par lui contiennent le stimulant attaché à des ordres, sans pourtant ternir l'espèce d'ivresse lucide et pure qui accompagne le libre consentement au sacrifice.\par
Il en résulte pour lui des possibilités et des responsabilités immenses.\par
Plus il y aura en France d'actions accomplies par ses ordres, de gens agissant sous ses ordres, plus la France aura de chances de retrouver une âme qui lui permette une rentrée triomphale dans la guerre – triomphale non pas seulement militairement, mais aussi spirituellement – et une reconstruction de la patrie dans la paix.\par
En plus de la quantité, le problème du choix des actions est capital.\par
Il est capital à plusieurs égards, dont certains sont d'un tel niveau d'élévation et d'importance qu'il faut regarder comme désastreuse la compartimentation qui met ce domaine entièrement aux mains de techniciens de la conspiration.\par
D'une manière tout à fait générale, en toute espèce de domaine, il est inévitable que le mal domine partout où la technique se trouve soit entièrement soit presque entièrement souveraine.\par
Les techniciens tendent toujours à se rendre souverains, parce qu'ils sentent qu'ils connaissent leur affaire ; et c'est tout à fait légitime de leur part. La responsabilité du mal qui, lorsqu'ils y parviennent, en est l'effet inévitable incombe exclusivement à ceux qui les ont laissé faire. Quand on les laisse faire, c'est toujours uniquement faute d'avoir toujours présente dans l'esprit la conception claire et tout à fait précise des fins particulières auxquelles telle, telle et telle technique doit être subordonnée.\par
La direction imprimée de Londres à l'action menée en France doit répondre à plusieurs fins.\par
La plus évidente est la fin militaire immédiate, en ce qui concerne les renseignements et les sabotages.\par
À cet égard, les Français de Londres ne peuvent être que des intermédiaires entre les besoins de l'Angleterre et la bonne volonté des Français de France.\par
L'importance extrême de ces choses est évidente, si l'on se rend compte qu'il est de plus en plus clair que les communications bien plus que les batailles décideront de la guerre. Le couple locomotives-sabotage est symétrique du couple bateau-sous-marin. La destruction des locomotives vaut celle des sous-marins. La relation de ces deux espèces de destruction est celle de l'offensive à la défensive.\par
La désorganisation de la production n'est pas moins essentielle.\par
Le volume, la quantité de notre influence sur l'action menée en France dépend principalement des moyens matériels mis à notre disposition par les Anglais. Notre influence sur la France, celle que nous avons et plus encore celle que nous pouvons acquérir, peut être pour les Anglais d'un usage très précieux. Il y a donc besoin mutuel ; mais le nôtre est beaucoup plus grand ; du moins dans l'immédiat, qui trop souvent est seul considéré.\par
Dans cette situation, s'il n'y a pas entre eux et nous des relations non seulement bonnes, mais chaleureuses, vraiment amicales et en quelque sorte intimes, c'est quelque chose d'intolérable et qui doit cesser. Partout où des relations humaines ne sont pas ce qu'elles doivent être, il y a généralement faute des deux côtés. Mais il est toujours bien plus utile de songer à ses propres fautes, pour y mettre fin, qu'à celles de l'autre. De plus le besoin est beaucoup plus grand de notre côté, au moins le besoin immédiat. Puis nous sommes des émigrés accueillis par eux, et il existe une dette de gratitude. Enfin il est notoire que les Anglais n'ont pas l'aptitude à sortir d'eux-mêmes et à se mettre à la place d'autrui ; leurs meilleures qualités, leur fonction propre sur cette planète, sont presque incompatibles avec elle. Cette aptitude est en fait, par malheur, presque aussi rare chez nous ; mais elle appartient par la nature des choses à ce qu'on appelle la vocation de la France. Pour tous ces motifs, c'est à nous à faire effort pour porter les relations au degré de chaleur convenable ; il faut que de notre part un sincère désir de compréhension, pur, bien entendu, de toute nuance de servilité, perce à travers leur réserve jusqu'à la réelle capacité d'amitié qu'elle dissimule.\par
Les sentiments personnels jouent dans les grands événements du monde un rôle qu'on ne discerne jamais dans toute son étendue. Le fait qu'il y a ou qu'il n'y a pas amitié entre deux hommes, entre deux milieux humains, peut dans certains cas être décisif pour la destinée du genre humain.\par
C'est tout à fait compréhensible. Une vérité n'apparaît jamais que dans l'esprit d'un être humain particulier. Comment la communiquera-t-il ? S'il essaie de l'exposer, il ne sera pas écouté ; car les autres, ne connaissant pas cette vérité, ne la reconnaîtront pas pour telle ; ils ne sauront pas que ce qu'il est en train de dire est vrai ; ils n'y porteront pas une attention suffisante pour s'en apercevoir ; car ils n'auront aucun motif d'accomplir cet effort d'attention.\par
Mais l'amitié, l'admiration, la sympathie, ou tout autre sentiment bienveillant les disposerait naturellement à un certain degré d'attention. Un homme qui a quelque chose de nouveau à dire – car pour les lieux communs nulle attention n'est nécessaire – ne peut être d'abord écouté que de ceux qui l'aiment.\par
Ainsi la circulation des vérités parmi les hommes dépend entièrement de l'état des sentiments ; et il en est ainsi pour toutes les espèces de vérités.\par
Chez des exilés qui n'oublient pas leur pays – et ceux qui l'oublient sont perdus – le cœur est si irrésistiblement tourné vers la patrie malheureuse qu'il y a peu de ressources affectives pour l'amitié à l'égard du pays qu'on habite. Cette amitié ne peut pas vraiment germer et, pousser dans leur cœur s'ils ne se font pas une sorte de violence. Mais cette violence est une obligation.\par
Les Français qui sont à Londres n'ont pas de plus impérieuse obligation envers le peuple français, qui vit les yeux tournés vers eux, que de faire en sorte qu'il y ait entre eux-mêmes et l'élite des Anglais une amitié réelle, vivante, chaleureuse, intime, efficace.\par
En dehors de l'utilité stratégique, d'autres considérations encore doivent avoir part au choix des actions. Elles ont bien plus d'importance encore, mais viennent en second lieu, parce que l'utilité stratégique est une condition pour que l'action soit réelle ; là où elle est absente, il y a agitation, non action, et la vertu indirecte de l'action, qui en fait le prix principal, est absente du même coup.\par
Cette vertu indirecte, encore une fois, est double.\par
L'action confère la plénitude de la réalité aux mobiles qui la produisent. L'expression de ces mobiles, entendue du dehors, ne leur confère encore qu'une demi-réalité. L'action a une tout autre vertu.\par
\par
Beaucoup de sentiments peuvent coexister dans le cœur. Le choix de ceux qu'il faut, après les avoir discernés dans le cœur des Français, porter au degré d'existence que confère l'expression officielle, ce choix est déjà limité par des nécessités matérielles. Si par exemple on parle chaque soir un quart d'heure aux Français, si l'on est obligé de se répéter souvent parce que le brouillage empêche d'être sûr qu'on a été entendu, et que de toutes manières la répétition est une nécessité pédagogique, on ne peut dire qu'un nombre de choses limité.\par
Dès qu'on passe au domaine de l'action, les limites sont encore plus étroites. Il faut opérer un nouveau choix, d'après les critères déjà esquissés.\par
La manière dont un mobile se transforme en acte est une chose à étudier. Un même acte peut être produit par tel mobile, ou tel autre, ou encore tel autre ; ou par un mélange ; au contraire tel autre mobile peut n'être pas susceptible de le produire.\par
Pour amener les gens non seulement à accomplir telle action, mais encore à l'accomplir par l'impulsion de tel mobile, le meilleur procédé, peut-être le seul, semble consister dans l'association établie au moyen de la parole. C'est-à-dire que, toutes les fois qu'une action est conseillée par radio, ce conseil doit être accompagné de l'expression d'un ou de quelques mobiles ; toutes les fois que le conseil est répété, le mobile doit être de nouveau exprimé.\par
Il est vrai que les instructions précises sont communiquées par une voie autre que la radio. Mais elles devraient toutes être doublées par des encouragements transmis par radio, portant sur le même objet, désigné seulement autant que le permet la prudence, avec les précisions en moins et l'expression des mobiles en plus.\par
L'action a une seconde vertu dans le domaine des mobiles. Elle ne confère pas seulement la réalité à des mobiles qui, auparavant, existaient dans un état semi-fantomatique. Elle fait aussi surgir dans l'âme des mobiles et des sentiments qui auparavant n'existaient pas du tout.\par
Cela se produit toutes les fois que soit l'entraînement soit la contrainte des circonstances fait pousser l'action au-delà de la somme d'énergie enfermée dans le mobile qui a produit l'action.\par
Ce mécanisme – dont la connaissance est essentielle aussi bien pour la conduite de sa propre vie que pour l'action sur les hommes – est également susceptible de susciter du mal ou du bien.\par
Par exemple, il arrive souvent qu'un malade chronique, dans une famille, soigné tendrement par l'effet d'une sincère affection, finisse par faire naître chez les siens une hostilité sourde, inavouée, parce qu'ils ont été obligés de lui donner plus d'énergie que leur affection n'en contenait.\par
\par
Dans le peuple, où de telles obligations, ajoutées aux fatigues habituelles, sont tellement lourdes, il en résulte parfois une apparence d'insensibilité, ou même de cruauté, incompréhensible du dehors. C'est pour cela que, comme le remarquait un jour charitablement Gringoire, les cas d'enfants martyrs se rencontrent dans le peuple plus qu'ailleurs.\par
Les ressources de ce mécanisme pour produire le bien sont illustrées par une merveilleuse histoire bouddhiste.\par
Une tradition bouddhiste dit que le Bouddha a fait vœu de faire monter au ciel, à ses côtés, quiconque dirait son nom avec le désir d'être sauvé par lui. Sur cette tradition repose la pratique nommée : « La récitation du nom du Seigneur. » Elle consiste à répéter un certain nombre de fois quelques syllabes sanscrites, chinoises ou japonaises, qui veulent dire : « Gloire au Seigneur de Lumière. »\par
Un jeune moine bouddhiste était inquiet pour le salut éternel de son père, vieil avare qui ne pensait qu'à l'argent. Le prieur du couvent se fit amener le vieux et lui promit un sou chaque fois qu'il pratiquerait la récitation du nom du Seigneur ; s'il venait le soir dire combien de sous on lui devait, on les lui paierait. Le vieux, ravi, consacra à cette pratique tous ses moments disponibles. Il venait se faire payer au couvent chaque soir. Soudain on ne le vit plus. Après une semaine, le prieur envoya le jeune moine chercher des nouvelles de son père. On apprit ainsi que le vieux était maintenant absorbé par la récitation du nom du Seigneur au point qu'il ne pouvait plus compter combien de fois il la pratiquait ; c'est ce qui l'empêchait de venir chercher son argent. Le prieur dit au jeune moine de ne plus rien faire et d'attendre. Quelque temps après, le vieux arriva au couvent avec des yeux rayonnants, et raconta qu'il avait eu une illumination.\par
C'est à des phénomènes de ce genre que fait allusion le précepte du Christ : « Thésaurisez pour vous des trésors dans le ciel... car là où est ton trésor, là sera aussi ton cœur. »\par
Cela signifie qu'il y a des actions qui ont la vertu de transporter de la terre dans le ciel une partie de l'amour qui se trouve dans le cœur d'un homme.\par
Un avare n'est pas un avare quand il commence à amasser. Il est stimulé d'abord, sans doute, par la pensée des jouissances qu'on se procure avec de l'argent. Mais les efforts et les privations qu'il s'impose chaque jour produisent un entraînement. Quand le sacrifice dépasse de loin l'impulsion initiale, le trésor, objet du sacrifice, devient pour lui une fin en soi, et il y subordonne sa propre personne. La manie du collectionneur repose sur un mécanisme analogue. On pourrait citer quantité d'autres exemples.\par
Ainsi quand les sacrifices faits à un objet dépassent de loin l'impulsion qui les a causés, il en résulte, à l'égard de cet objet, ou un mouvement de répulsion, ou un attachement d'une espèce nouvelle et plus intense, étranger à l’impulsion première.\par
Dans le second cas, il y a bien ou mal selon la nature de l'objet.\par
Si dans le cas du malade il y a souvent répulsion, c'est que ce genre d'effort est privé d'avenir ; rien d'extérieur n'y répond à l'accumulation intérieure de la fatigue. L'avare, lui, voit croître son trésor.\par
Il est d'ailleurs aussi des situations, des combinaisons de caractères, telles qu'un malade dans une famille inspire au contraire un attachement fanatique. En étudiant suffisamment tout cela, on pourrait sans doute discerner les lois.\par
Mais même une connaissance sommaire de ces phénomènes peut nous fournir des règles pratiques.\par
Pour éviter l'effet de répulsion, il faut prévoir l'épuisement possible des mobiles ; il faut de période en période donner l'autorité de l'expression officielle à des mobiles nouveaux pour les mêmes actions, mobiles répondant à ce qui aura pu germer spontanément au secret des cœurs.\par
Il faut surtout veiller à ce que le mécanisme de transfert qui attache l'avare au trésor joue de manière à produire du bien et non du mal ; éviter ou en tout cas réduire au strict minimum tout le mal qui pourrait être ainsi suscité.\par
Il est facile de comprendre comment.\par
Le mécanisme en question consiste en ceci, qu'une action, après avoir été menée avec effort pour des motifs extérieurs à elle-même, devient par elle-même objet d'attachement. Il en résulte du bien ou du mal selon que l'action est par elle-même bonne ou mauvaise.\par
Si l'on tue des soldats allemands pour servir la France et qu'au bout d'un certain temps assassiner des êtres humains devienne un goût, il est clair que c'est un mal.\par
Si l'on aide des ouvriers qui fuient l'envoi en Allemagne pour servir la France, et qu'au bout d'un certain temps le secours aux malheureux devienne un goût, il est clair que c'est un bien.\par
Tous les cas ne sont pas aussi clairs, mais tous peuvent être examinés de cette manière. Toutes choses égales d'ailleurs, il faut toujours choisir les modes d'action qui contiennent en eux-mêmes un entraînement vers le bien. Il le faut même très souvent quand toutes choses ne sont pas égales d'ailleurs. Il le faut non seulement pour le bien, ce qui suffirait, mais aussi, par surcroît, pour l'utilité.\par
\par
Le mal est beaucoup plus facilement que le bien un mobile agissant, mais une fois que du bien pur est devenu un mobile agissant dans une âme, il y est la source d'une impulsion inépuisable et invariable, ce qui n'est jamais le cas du mal.\par
On peut très bien devenir un agent double par patriotisme, pour mieux servir son pays en trompant l'ennemi. Mais si les efforts qu'on accomplit dans cette activité dépassent l'énergie du mobile patriotique, et si par suite on prend goût à l'activité elle-même, il vient presque inévitablement un moment où l'on ne sait plus soi-même qui l'on sert et qui l'on trompe, où l'on est prêt à servir ou tromper n'importe qui.\par
Au contraire, si par patriotisme on est poussé à des actions qui font germer et croître l'amour d'un bien supérieur à la patrie, l'âme acquiert cette trempe qui fait les martyrs et la patrie en profite.\par
La foi est plus réaliste que la politique réaliste. Qui n'en a pas la certitude n'a pas la foi.\par
Il faut donc examiner et peser d'extrêmement près, en faisant chaque fois le tour du problème, chacun des modes d'action qui constituent la résistance illégale en France.\par
Une observation attentive sur place, accomplie uniquement de ce point de vue, est indispensable à cet effet.\par
Il n'est pas non plus exclu qu'il puisse y avoir lieu d'inventer des formes d'action nouvelles, en tenant compte à la fois de ces considérations et des buts immédiats.\par
(Par exemple, nouer tout de suite une vaste conspiration pour la {\itshape destruction des documents officiels} relatifs au contrôle des individus par l’État, destruction qui peut être opérée par des procédés très variés, incendies, etc. ; cela aurait des avantages immédiats et lointains immenses.)\par
Un degré de réalité supérieur encore à l'action est constitué par l'organisation qui coordonne les actions ; quand une telle organisation n'a pas été fabriquée artificiellement, mais a poussé comme une plante au milieu des nécessités quotidiennes, et en même temps a été modelée par une vigilance patiente d'après la vue claire d'un bien, c'est là peut-être le degré de réalité le plus haut possible.\par
Il y a des organisations en France. Mais il y a aussi, ce qui est d'un intérêt plus grand encore, des embryons, des germes, des ébauches d'organisations en voie de croissance.\par
Il faut les étudier, les contempler sur place, et user de l'autorité qui réside à Londres comme d'un outil pour les modeler discrètement et patiemment, comme un sculpteur qui devine la forme contenue dans le bloc de marbre pour l'en extraire.\par
Ce modelage doit être guidé à la fois par des considérations immédiates et non immédiates.\par
Tout ce qui a été dit précédemment à propos de la parole et de l'action s'applique encore ici.\par
Une organisation qui cristallise et capte les paroles lancées officiellement, qui en traduise l'inspiration en paroles différentes et bien à elle, qui les réalise en actions coordonnées pour lesquelles elle constitue une garantie d'efficacité toujours croissante, qui soit un milieu vivant, chaleureux, plein d'intimité, de fraternité et de tendresse – voilà la terre végétale où les malheureux Français, déracinés par le désastre, peuvent vivre et trouver le salut pour la guerre et pour la paix.\par
Cela doit se faire maintenant. Après la victoire, dans le déchaînement irrésistible des appétits individuels de bien-être ou de pouvoir, il sera absolument impossible de rien commencer.\par
Cela doit se faire immédiatement. C'est indescriptiblement urgent. Manquer le moment serait encourir une responsabilité presque équivalente peut-être à un crime.\par
L'unique source de salut et de grandeur pour la France, c'est de reprendre contact avec son génie au fond de son malheur. Cela doit se faire maintenant, tout de suite ; alors que le malheur est encore écrasant ; alors que la France a devant elle, dans l'avenir, la possibilité de rendre réelle la première lueur de conscience de son génie retrouvé, en l'exprimant à travers une action guerrière.\par
Après la victoire, cette possibilité serait passée, et la paix n'en présenterait pas d'équivalente. Car il est infiniment plus difficile d'imaginer, de concevoir une action de paix qu'une action de guerre ; pour passer à travers une action de paix, une inspiration doit avoir déjà un degré élevé de conscience, de lumière, de réalité. Ce ne sera le cas pour la France, au moment de la paix, que si la dernière période de la guerre a produit cet effet. Il faut que la guerre soit l'institutrice qui développe et nourrisse l'inspiration ; pour cela il faut qu'une inspiration profonde, authentique, une vraie lumière, surgisse en pleine guerre.\par
Il faut que la France soit de nouveau pleinement présente à la guerre, participe au prix de son sang à la victoire ; mais cela ne saurait suffire. Cela pourrait se produire dans les ténèbres, et le vrai profit alors serait faible.\par
Il faut de plus que l'aliment de son énergie guerrière ne soit pas autre chose que son véritable génie, retrouvé dans les profondeurs du malheur, bien qu'avec un degré de conscience inévitablement faible d'abord après une pareille nuit.\par
La guerre même peut alors en faire une flamme.\par
La vraie mission du mouvement français de Londres est, en raison même des circonstances politiques et militaires, une mission spirituelle avant d'être une mission politique et militaire.\par
Elle pourrait être définie comme étant la direction de conscience à l'échelle d'un pays.\par
Le mode d'action politique esquissé ici exige que chaque choix soit précédé par la contemplation simultanée de plusieurs considérations d'espèce très différente. Cela implique un degré d'attention élevé, à peu près du même ordre que celui qui est exigé par le travail créateur dans l'art et la science.\par
Mais pourquoi la politique, qui décide du destin des peuples et a pour objet la justice, exigerait-elle une attention moindre que l'art et la science, qui ont pour objet le beau et le vrai ?\par
La politique a une affinité très étroite avec l'art ; avec des arts tels que la poésie, la musique, l'architecture.\par
La composition simultanée sur plusieurs plans est la loi de la création artistique et en fait la difficulté.\par
Un poète, dans l'arrangement des mots et le choix de chaque mot, doit tenir compte simultanément de cinq ou six plans de composition au moins. Les règles de la versification – nombre de syllabes et rimes – dans la forme de poème qu'il a adoptée ; la coordination grammaticale des mots ; leur coordination logique à l'égard du développement de la pensée ; la suite purement musicale des sons contenus dans les syllabes ; le rythme pour ainsi dire matériel constitué par les coupes, les arrêts, la durée de chaque syllabe et de chaque groupe de syllabes ; l'atmosphère que mettent autour de chaque mot les possibilités de suggestion qu'il enferme, et le passage d'une atmosphère à une autre à mesure que les mots se succèdent ; le rythme psychologique constitué par la durée des mots correspondant à telle atmosphère ou à tel mouvement de la pensée ; les effets de la répétition et de la nouveauté; sans doute d'autres choses encore ; et une intuition unique de beauté donnant une unité à tout cela.\par
L'inspiration est une tension des facultés de l'âme qui rend possible le degré d'attention indispensable à la composition sur plans multiples.\par
Celui qui n'est pas capable d'une telle attention en recevra un jour la capacité, s'il s'obstine avec humilité, persévérance et patience, et s'il est poussé par un désir inaltérable et violent.\par
S'il n'est pas la proie d'un tel désir, il n'est pas indispensable qu'il fasse des vers.\par
La politique, elle aussi, est un art gouverné par la composition sur plans multiples. Quiconque se trouve avoir des responsabilités politiques, s'il a en lui la faim et la soif de la justice, doit désirer recevoir cette faculté de composition sur plans multiples, et par suite doit infailliblement la recevoir avec le temps.\par
Seulement, aujourd'hui, le temps presse. Les besoins sont urgents.\par
La méthode d'action politique esquissée ici dépasse les possibilités de l'intelligence humaine, du moins autant que ces possibilités sont connues. Mais c'est là précisément ce qui en fait le prix. Il ne faut pas se demander si l'on est ou non capable de l'appliquer. La réponse serait toujours non. Il faut la concevoir d'une manière parfaitement claire ; la contempler longtemps et souvent ; l'enfoncer pour toujours au lieu de l'âme où les pensées prennent leurs racines ; et qu'elle soit présente à toutes les décisions. Il y a peut-être alors une probabilité pour que les décisions, bien qu'imparfaites, soient bonnes.\par
Celui qui compose des vers avec le désir d'en réussir d'aussi beaux que ceux de Racine ne fera jamais un beau vers. Encore bien moins s'il n'a même pas cette espérance.\par
Pour produire des vers où réside quelque beauté, il faut avoir désiré égaler par l'arrangement des mots la beauté pure et divine dont Platon dit qu'elle habite de l'autre côté du ciel.\par
Une des vérités fondamentales du christianisme, c'est qu'un progrès vers une moindre imperfection n'est pas produit par le désir d'une moindre imperfection. Seul le désir de la perfection a la vertu de détruire dans l'âme une partie du mal qui la souille. De là le commandement du Christ : « Soyez parfaits comme votre Père céleste est parfait. »\par
Autant le langage humain est loin de la beauté divine, autant les facultés sensibles et intellectuelles des hommes sont loin de la vérité, autant les nécessités de la vie sociale sont loin de la justice. Par suite, il n'est pas possible que la politique n'ait pas besoin d'efforts d'invention créatrice autant que l'art et la science.\par
C'est pourquoi la presque totalité des opinions politiques et des discussions où elles s'opposent est aussi étrangère à la politique que le choc des opinions esthétiques dans les brasseries de Montparnasse est étranger à l'art. L'homme politique dans un cas comme l'artiste dans l'autre ne peuvent trouver là qu'un certain stimulant, qui doit être pris à très faible dose.\par
On ne regarde presque jamais la politique comme un art d'espèce tellement élevée. Mais c'est qu'on est accoutumé depuis des siècles à la regarder seulement, ou en tout cas principalement, comme la technique de l'acquisition et de la conservation du pouvoir.\par
Or le pouvoir n'est pas une fin. Par nature, par essence, par définition, il constitue exclusivement un moyen. Il est à la politique ce qu'est un piano à la composition musicale. Un compositeur qui a besoin d'un piano pour l'invention des mélodies se trouvera embarrassé s'il est dans un village où il n'y en ait pas. Mais si on lui en procure un, il s'agit alors qu'il compose.\par
Malheureux que nous sommes, nous avions confondu la fabrication d'un piano avec la composition d'une sonate.\par
Une méthode d'éducation n'est pas grand-chose si elle n'a pas pour inspiration la conception d'une certaine perfection humaine. Quand il s'agit de l'éducation d'un peuple, cette conception doit être celle d'une civilisation. Il ne faut pas la chercher dans le passé, qui ne contient que de l'imparfait. Bien moins encore dans nos rêves d'avenir, qui sont par nécessité aussi médiocres que nous-mêmes, et par suite de très loin inférieurs au passé. Il faut chercher l'inspiration d'une telle éducation, comme la méthode elle-même, parmi les vérités éternellement inscrites dans la nature des choses.\par
Voici, à ce sujet, quelques indications.\par
Quatre obstacles surtout nous séparent d'une forme de civilisation susceptible de valoir quelque chose. Notre conception fausse de la grandeur ; la dégradation du sentiment de la justice ; notre idolâtrie de l'argent ; et l'absence en nous d'inspiration religieuse. On peut s'exprimer à la première personne du pluriel sans aucune hésitation, car il est douteux qu'à l'instant présent un seul être humain sur la surface du globe terrestre échappe à cette quadruple tare, et plus douteux encore qu'il y en ait un seul dans la race blanche. Mais s'il y en a quelques-uns, comme il faut malgré tout l'espérer, ils sont cachés.\par
Notre conception de la grandeur est la tare la plus grave et celle dont nous avons le moins conscience comme d'une tare. Du moins comme d'une tare en nous ; chez nos ennemis elle nous choque, mais, malgré l'avertissement contenu dans la parole du Christ sur la paille et la poutre, nous ne songeons pas à la reconnaître comme nôtre.\par
Notre conception de la grandeur est celle même qui a inspiré la vie tout entière d'Hitler. Quand nous la dénonçons sans la moindre trace de retour sur nous-mêmes, les anges doivent pleurer ou rire, s'il y a des anges qui s'intéressent à notre propagande.\par
Il paraît qu'aussitôt la Tripolitaine occupée, on y a arrêté l'enseignement fasciste de l'histoire. C'est fort bien. Mais il serait intéressant de savoir en quoi, pour l'Antiquité, l'enseignement fasciste de l'histoire différait de celui de la République française. La différence devait être faible, car la grande autorité de la France républicaine en matière d'histoire ancienne, M. Carcopino, prononçait à Rome des conférences sur la Rome antique et la Gaule qui étaient tout à fait propres à être prononcées en ce lieu et y étaient très bien accueillies.\par
Aujourd'hui, les Français de Londres ont quelques reproches à faire à M. Carcopino, mais ce n'est pas sur ses conceptions historiques. Un autre historien de la Sorbonne disait en janvier 1940 à quelqu'un qui avait écrit quelque chose d'assez dur sur les Romains : « Si l'Italie se met contre nous, vous aurez eu raison. » Comme critère de jugement historique, c'est insuffisant.\par
Les vaincus bénéficient souvent d'une sentimentalité parfois même injuste, mais seulement les vaincus provisoires. Le malheur est un immense prestige quand celui de la force s'y joint. Le malheur des faibles n'est même pas un objet d'attention ; si toutefois il n'est pas un objet de répulsion. Quand les chrétiens eurent acquis la conviction solide que le Christ, quoique ayant été crucifié, était ensuite ressuscité et devait prochainement revenir dans la gloire pour récompenser les siens et punir tous les autres, aucun supplice ne les effraya plus. Mais auparavant, quand le Christ était seulement un être absolument pur, dès que le malheur le toucha il fut abandonné. Ceux qui l'aimaient le plus ne purent trouver dans leur cœur la force de courir des risques pour lui. Les supplices sont au-dessus du courage quand il n'y a pas pour les affronter le stimulant d'une revanche. La revanche n'a pas besoin d'être personnelle ; un Jésuite martyrisé en Chine est soutenu par la grandeur temporelle de l'Église, bien qu'il ne puisse en espérer lui-même aucun secours. Il n'y a pas ici-bas d'autre force que la force. Cela pourrait servir d'axiome. Quant à la force qui n'est pas d'ici-bas, le contact avec elle ne peut pas être acheté à un prix moindre que le passage à travers une sorte de mort.\par
Il n'y a pas ici-bas d'autre force que la force, et c'est elle qui communique de la force aux sentiments, y compris la compassion. On pourrait en citer cent exemples. Pourquoi les pacifistes d'après 1918 se sont-ils tellement plus attendris sur l'Allemagne que sur l'Autriche ? Pourquoi la nécessité des congés payés a-t-elle paru à tant de gens un axiome d'une évidence géométrique en 1936 et non en 1935 ? Pourquoi y a-t-il tellement plus de gens pour s'intéresser aux ouvriers d'usine qu'aux ouvriers agricoles ? Et ainsi de suite.\par
De même dans l'histoire. On admire la résistance héroïque des vaincus quand la suite des temps apporte une certaine revanche ; non autrement. On n'a pas de compassion pour les choses totalement détruites. Qui en accorde à Jéricho, à Gaza, Tyr, Sidon, à Carthage, à Numance, à la Sicile grecque, au Pérou précolombien ?\par
Mais, objectera-t-on, comment pleurer la disparition de choses dont on ne sait pour ainsi dire rien ? On ne sait rien d'elles parce qu'elles ont disparu. Ceux qui les ont détruites n'ont pas cru devoir se faire les conservateurs de leur culture.\par
D'une manière générale, les erreurs les plus graves, celles qui faussent toute la pensée, qui perdent l'âme, qui la mettent hors du vrai et du bien, sont indiscernables. Car elles ont pour cause le fait que certaines choses échappent à l'attention. Si elles échappent à l'attention, comment y ferait-on attention, quelque effort que l'on fasse ? C'est pourquoi, par essence, la vérité est un bien surnaturel.\par
Il en est ainsi pour l'histoire. Les vaincus y échappent à l'attention. Elle est le siège d'un processus darwinien plus impitoyable encore que celui qui gouverne la vie animale et végétale. Les vaincus disparaissent. Ils sont néant.\par
Les Romains ont, dit-on, civilisé la Gaule. Elle n'avait pas d'art avant l'art gallo-romain ; pas de pensée avant que les Gaulois n'eussent le privilège de lire les écrits philosophiques de Cicéron ; et ainsi de suite.\par
Nous ne savons pour ainsi dire rien sur la Gaule, mais les indications presque nulles que nous possédons prouvent assez que tout cela est du mensonge.\par
L'art gaulois ne risque pas d'être l'objet de mémoires de la part de nos archéologues, parce que la matière en était le bois. Mais la ville de Bourges était une si pure merveille de beauté que les Gaulois perdirent leur dernière campagne faute d'avoir le courage de la détruire eux-mêmes. Bien entendu, César la détruisit, et massacra du même coup la totalité des quarante mille êtres humains qui s'y trouvaient.\par
On sait par César que les études des Druides duraient vingt ans et consistaient à apprendre par cœur des poèmes concernant la divinité et l'univers. La poésie gauloise contenait donc en tout cas une quantité de poèmes religieux et métaphysiques telle qu'elle constituait la matière de vingt ans d'études. À coté de l'incroyable richesse suggérée par cette seule indication, la poésie latine, malgré Lucrèce, est quelque chose de misérable.\par
Diogène Laërce dit qu'une tradition attribuait à la sagesse grecque plusieurs origines étrangères, parmi lesquelles les Druides de Gaule. D'autres textes indiquent que la pensée des Druides s'apparentait à celle des Pythagoriciens.\par
Ainsi il y avait dans ce peuple une mer de poésie sacrée dont les œuvres de Platon peuvent seules nous permettre de nous représenter l'inspiration.\par
Tout cela disparut quand les Romains exterminèrent, pour crime de patriotisme, la totalité des Druides.\par
Il est vrai que les Romains ont mis fin aux sacrifices humains pratiqués, disaient-ils, en Gaule. Nous ne savons rien sur ce qu'ils étaient, sur la manière et l'esprit dans lesquels ils étaient pratiqués, si c'était un mode d'exécution des criminels ou une mise à mort d'innocents et, en ce dernier cas, si c'était avec consentement ou non. Le témoignage des Romains est très vague et ne saurait être admis sans méfiance. Mais ce que nous savons avec certitude, c'est que les Romains ont institué eux-mêmes en Gaule et partout la mise à mort de milliers d'innocents, non pas pour honorer les dieux, mais pour amuser les foules. C'était l'institution romaine par excellence, celle qu'ils transportaient partout ; eux que nous osons regarder comme des civilisateurs.\par
Néanmoins si l'on disait publiquement que la Gaule d'avant la conquête était beaucoup plus civilisée que Rome, cela sonnerait comme une absurdité.\par
C'est là simplement un exemple caractéristique. Bien qu'à la Gaule ait succédé sur le même sol une nation qui est la nôtre, bien que le patriotisme ait chez nous comme ailleurs une forte tendance à s'étendre dans le passé, bien que le peu de documents conservés constitue un témoignage irrécusable, la défaite des armes gauloises est un obstacle insurmontable à ce que nous reconnaissions la haute qualité spirituelle de cette civilisation détruite.\par
Encore y a-t-il eu en sa faveur des tentatives comme celle de Camille Jullian. Mais le territoire de Troie n'ayant plus jamais été le siège d'une nation, qui a pris la peine de discerner la vérité qui éclate de la manière la plus évidente dans l'{\itshape Iliade}, dans Hérodote, dans l'{\itshape Agamemnon} d'Eschyle ; à savoir que Troie était d'un niveau de civilisation, de culture, de spiritualité bien plus haut que ceux qui l'ont attaquée injustement et détruite ; et que sa disparition a été un désastre dans l'histoire de l'humanité ?\par
Avant juin 1940, on pouvait lire dans la presse française, à titre d'encouragement patriotique, des articles comparant le conflit franco-allemand à la guerre de Troie ; on y expliquait que cette guerre était déjà une lutte de la civilisation contre la barbarie, les barbares étant les Troyens. Or il n'y a pas à cette erreur une ombre de motif sinon la défaite de Troie.\par
Si l'on ne peut s'empêcher de tomber dans cette erreur au sujet des Grecs, qui ont été hantés par le remords du crime commis et ont témoigné eux-mêmes en faveur de leurs victimes, combien davantage au sujet des autres nations, dont la pratique invariable est de calomnier ceux qu'elles ont tués ?\par
L'histoire est fondée sur les documents. Un historien s'interdit par profession les hypothèses qui ne reposent sur rien. En apparence c'est très raisonnable ; mais en réalité il s'en faut de beaucoup. Car, comme il y a des trous dans les documents, l'équilibre de la pensée exige que des hypothèses sans fondement soient présentes à l'esprit, à condition que ce soit à ce titre et qu'autour de chaque point il y en ait plusieurs.\par
À plus forte raison faut-il dans les documents lire entre les lignes, se transporter tout entier, avec un oubli total de soi, dans les événements évoqués, attarder très longtemps l'attention sur les petites choses significatives et en discerner toute la signification.\par
Mais le respect du document et l'esprit professionnel de l'historien ne disposent pas la pensée à ce genre d'exercice. L'esprit dit historique ne perce pas le papier pour trouver de la chair et du sang ; il consiste en une subordination de la pensée au document.\par
Or par la nature des choses, les documents émanent des puissants, des vainqueurs. Ainsi l'histoire n'est pas autre chose qu'une compilation des dépositions faites par les assassins relativement à leurs victimes et à eux-mêmes.\par
Ce qu'on nomme le tribunal de l'histoire, informé de la sorte, ne saurait juger d'une autre manière que celui des {\itshape Animaux malades de la peste.}\par
Sur les Romains, on ne possède absolument rien d'autre que les écrits des Romains eux-mêmes et de leurs esclaves grecs. Ceux-ci, les malheureux, parmi leurs réticences serviles, en ont dit assez, si l'on prenait la peine de les lire avec une véritable attention. Mais pourquoi en prendrait-on la peine ? Il n'y a pas de mobile pour cet effort. Ce ne sont pas les Carthaginois qui disposent des prix de l'Académie ni des chaires en Sorbonne.\par
Pourquoi, de même, prendrait-on la peine de mettre en doute les renseignements donnés par les Hébreux sur les populations de Canaan qu'ils ont exterminées ou asservies ? Ce ne sont pas les gens de Jéricho qui font des nominations à l'Institut catholique.\par
On sait par une des biographies d'Hitler qu'un des livres qui ont exercé la plus profonde influence sur sa jeunesse était un ouvrage de dixième ordre sur Sylla. Qu'importe que l'ouvrage ait été de dixième ordre ? Il reflétait l'attitude de ceux qu'on nomme l'élite. Qui écrirait sur Sylla avec mépris ? Si Hitler a désiré l'espèce de grandeur qu'il voyait glorifiée dans ce livre et partout, il n'y a pas eu erreur de sa part. C'est bien cette grandeur-là qu'il a atteinte, celle même devant laquelle nous nous inclinons tous bassement dès que nous tournons les yeux vers le passé.\par
Nous nous en tenons à la basse soumission d'esprit à son égard, nous n'avons pas, comme Hitler, tenté de la saisir dans nos mains. Mais en cela il vaut mieux que nous. Si l'on reconnaît quelque chose comme un bien, il faut vouloir le saisir. S'en abstenir est une lâcheté.\par
\par
Qu'on imagine cet adolescent misérable, déraciné, errant dans les rues de Vienne, affamé de grandeur. Il était bien de sa part d'être affamé de grandeur. À qui la faute s'il n'a pas discerné d'autre mode de grandeur que le crime ? Depuis que le peuple sait lire et n'a plus de traditions orales, ce sont les gens capables de manier une plume qui fournissent au public des conceptions de la grandeur et des exemples susceptibles de les illustrer.\par
L'auteur de ce livre médiocre sur Sylla, tous ceux qui en écrivant sur Sylla ou sur Rome avaient rendu possible l'atmosphère où ce livre a été écrit, plus généralement tous ceux qui, ayant autorité pour manier la parole ou la plume, ont contribué à l'atmosphère de pensée où Hitler adolescent a grandi, tous ceux-là sont peut-être plus coupables qu'Hitler des crimes qu'il commet. La plupart sont morts ; mais ceux d'aujourd'hui sont pareils à leur aînés, et ne peuvent être rendus plus innocents par le hasard d'une date de naissance.\par
On parle de châtier Hitler. Mais on ne peut pas le châtier. Il désirait une seule chose et il l’a : c'est d'être dans l'histoire. Qu'on le tue, qu'on le torture, qu'on l'enferme, qu'on l'humilie, l'histoire sera toujours là pour protéger son âme contre toute atteinte de la souffrance et de la mort. Ce qu'on lui infligera, ce sera inévitablement de la mort historique, de la souffrance historique ; de l'histoire. Comme, pour celui qui est parvenu à l'amour parfait de Dieu, tout événement est un bien comme provenant de Dieu, ainsi pour cet idolâtre de l'histoire, tout ce qui est de l'histoire est du bien. Encore a-t-il de loin l'avantage ; car l'amour pur de Dieu habite le centre de l'âme ; il laisse la sensibilité exposée aux coups ; il ne constitue pas une armure. L'idolâtrie est une armure ; elle empêche la douleur d'entrer dans l'âme. Quoi qu'on inflige à Hitler, cela ne l'empêchera pas de se sentir un être grandiose. Surtout cela n'empêchera pas, dans vingt, cinquante, cent ou deux cents ans, un petit garçon rêveur et solitaire, allemand ou non, de penser qu'Hitler a été un être grandiose, a eu de bout en bout un destin grandiose, et de désirer de toute son âme un destin semblable. En ce cas, malheur à ses contemporains.\par
Le seul châtiment capable de punir Hitler et de détourner de son exemple les petits garçons assoiffés de grandeur des siècles à venir, c'est une transformation si totale du sens de la grandeur qu'il en soit exclu.\par
C'est une chimère, due à l'aveuglement des haines nationales, que de croire qu'on puisse exclure Hitler de la grandeur sans une transformation totale, parmi les hommes d'aujourd'hui, de la conception et du sens de la grandeur. Et pour contribuer à cette transformation, il faut l'avoir accomplie en soi-même. Chacun peut en cet instant même commencer le châtiment d'Hitler dans l'intérieur de sa propre âme, en modifiant la distribution du sentiment de la grandeur. C'est loin d'être facile, car une pression sociale aussi lourde et enveloppante que celle de l'atmosphère s'y oppose. Il faut, pour y parvenir, s'exclure spirituellement de la société. C'est pourquoi Platon disait que la capacité de discerner le bien n'existe que chez les âmes prédestinées qui ont reçu une éducation directe de la part de Dieu.\par
Cela n'a pas de sens de chercher jusqu'où vont les ressemblances et les différences entre Hitler et Napoléon. Le seul problème qui ait un intérêt est de savoir si l'on peut légitimement exclure l'un de la grandeur sans en exclure l'autre ; si leurs titres à l'admiration sont analogues ou essentiellement différents. Et si, après avoir posé la question clairement et l'avoir regardée longuement en face, on se laisse glisser dans le mensonge, on est perdu. Marc-Aurèle disait à peu près, à propos d'Alexandre et de César : s'ils n'ont pas été justes, rien ne me force à les imiter. De même, rien ne nous force à les admirer.\par
Rien ne nous y force, excepté l'influence souveraine de la force.\par
Peut-on admirer sans aimer ? Et si l'admiration est un amour, comment ose-t-on aimer autre chose que le bien ?\par
Il serait simple de faire avec soi-même le pacte de n'admirer dans l'histoire que les actions et les vies au travers desquelles rayonne l'esprit de vérité, de justice et d'amour ; et, loin au-dessous, celles à l'intérieur desquelles on peut discerner à l'œuvre un pressentiment réel de cet esprit.\par
Cela exclut, par exemple, saint Louis lui-même, à cause du fâcheux conseil donné à ses amis, de plonger leur épée au ventre de quiconque tiendrait en leur présence des propos entachés d'hérésie ou d'incrédulité.\par
On dira, il est vrai, pour l'excuser, que c'était l'esprit de son temps, lequel, étant situé sept siècles avant le nôtre, était obnubilé en proportion. C'est un mensonge. Peu avant saint Louis, les catholiques de Béziers, loin de plonger leur épée dans le corps des hérétiques de leur ville, sont tous morts plutôt que de consentir à les livrer. L'Église a oublié de les mettre au rang des martyrs, rang qu'elle accorde à des inquisiteurs punis de mort par leurs victimes. Les amateurs de la tolérance, des lumières et de la laïcité, au cours des trois derniers siècles, n'ont guère commémoré ce souvenir non plus ; une forme aussi héroïque de la vertu qu'ils nomment platement tolérance aurait été gênante pour eux.\par
Mais quand même ce serait vrai, quand même la cruauté du fanatisme aurait dominé toutes les âmes du Moyen Âge, l'unique conclusion à en tirer serait qu'il n'y a rien à admirer ni à aimer dans cette époque. Cela ne mettrait pas saint Louis un millimètre plus près du bien. L'esprit de vérité, de justice et d'amour n'a absolument rien à voir avec un millésime ; il est éternel ; le mal est la distance qui sépare de lui les actions et les pensées ; une cruauté du X\textsuperscript{e} siècle est exactement aussi cruelle, ni plus ni moins, qu'une cruauté du XIX\textsuperscript{e}.\par
Pour discerner une cruauté, il faut tenir compte des circonstances, des significations variables attachées aux actes et aux paroles, du langage symbolique propre à chaque milieu ; mais une fois qu'une action a été indubitablement reconnue comme une cruauté, quels qu'en soient le lieu et la date, elle est horrible.\par
On le sentirait irrésistiblement si l'on aimait comme soi-même tous les malheureux qui, il y a deux ou trois mille ans, ont souffert de la cruauté de leurs semblables.\par
On ne pourrait pas alors écrire, comme M. Carcopino, que l'esclavage était devenu doux à Rome sous l'Empire, vu qu'il comportait rarement un châtiment plus rigoureux que les verges.\par
La superstition moderne du progrès est un sous-produit du mensonge par lequel on a fait du christianisme la religion romaine officielle ; elle est liée à la destruction des trésors spirituels des pays conquis par Rome, à la dissimulation de la parfaite continuité entre ces trésors et le christianisme, à une conception historique de la Rédemption, qui en fait une opération temporelle et non éternelle. La pensée du progrès a été plus tard laïcisée ; elle est maintenant le poison de notre époque. En posant que l'inhumanité était au XIV\textsuperscript{e} siècle une grande et bonne chose, mais une horreur au XIX\textsuperscript{e}, pouvait-on empêcher un petit gars du XX\textsuperscript{e} siècle, amateur de lectures historiques, de se dire : « Je sens en moi-même que maintenant l'époque où l'humanité était une vertu est finie et que l'époque de l'inhumanité revient » ? Qui interdit d'imaginer une succession cyclique au lieu d'une ligne continue ? Le dogme du progrès déshonore le bien en en faisant une affaire de mode.\par
C'est d'ailleurs seulement parce que l'esprit historique consiste à croire les meurtriers sur parole que ce dogme semble si bien répondre aux faits. Quand par moments l'horreur arrive à percer l'insensibilité épaisse d'un lecteur de Tite-Live, il se dit : « C'étaient les mœurs de l'époque. » Or on sent à l'évidence dans les historiens grecs que la brutalité des Romains a horrifié et paralysé leurs contemporains exactement comme fait aujourd'hui celle des Allemands.\par
Sauf erreur, parmi tous les faits relatifs à des Romains qu'on trouve dans l'histoire ancienne, il n'y a qu'un exemple de bien parfaitement pur. Sous le triumvirat, pendant les proscriptions, les personnages consulaires, les consuls, les préteurs dont les noms étaient sur la liste embrassaient les genoux de leurs propres esclaves et imploraient leur secours en les nommant leurs maîtres et leurs sauveurs ; car la fierté romaine ne résistait pas au malheur. Les esclaves, avec raison, les repoussaient. Il y eut très peu d'exceptions. Mais un Romain, sans avoir eu à s'abaisser, fut caché par ses esclaves dans sa propre maison. Des soldats, qui l'avaient vu entrer, mirent les esclaves à la torture pour les forcer à livrer leur maître. Les esclaves souffrirent tout sans plier. Mais le maître, de sa cachette, voyait la torture. Il ne put en supporter le spectacle, vint se livrer aux soldats et fut immédiatement tué.\par
Quiconque a le cœur bien placé, s'il avait à choisir entre plusieurs destinées, choisirait d'être indifféremment ce maître ou l'un de ces esclaves, plutôt que l'un des Scipions, ou César, ou Cicéron, ou Auguste, ou Virgile, ou même l'un des Gracques.\par
Voilà un exemple de ce qu'il est légitime d'admirer. Il y a dans l'histoire peu de choses parfaitement pures. La plupart concernent des êtres dont le nom a disparu, comme ce Romain, comme les habitants de Béziers au début du XIII\textsuperscript{e} siècle. Si l'on cherche des noms qui évoquent de la pureté, on en trouverait peu. Dans l'histoire grecque, on ne pourrait peut-être nommer qu'Aristide, Dion, l'ami de Platon, et Agis, le petit roi socialiste de Sparte, tué à vingt ans. Dans l'histoire de France, trouverait-on un autre nom que Jeanne d'Arc ? Ce n'est pas sûr.\par
Mais peu importe. Qui oblige à admirer beaucoup de choses ? L'essentiel est de n'admirer que ce qu'on peut admirer de toute son âme. Qui peut admirer Alexandre de toute son âme, s'il n'a l'âme basse ?\par
Il y a des gens qui proposent de supprimer l'enseignement de l'histoire. Il est vrai qu'il faudrait supprimer la coutume absurde d'apprendre des leçons d'histoire, hors un squelette aussi réduit que possible de dates et de points de repère, et appliquer à l'histoire la même espèce d'attention qu'à la littérature. Mais quant à supprimer l'étude de l'histoire, ce serait désastreux. Il n'y a pas de patrie sans histoire. On voit trop bien aux États-Unis ce que c'est qu'un peuple privé de la dimension du temps.\par
D'autres proposent d'enseigner l'histoire en mettant les guerres au dernier plan. Ce serait mentir. Nous ne sentons que trop aujourd'hui, il est également évident pour le passé, que rien n'est plus important pour les peuples que la guerre. Il faut parler de la guerre autant ou plus qu'on ne fait ; mais il faut en parler autrement.\par
Il n'y a pas d'autre procédé pour la connaissance du cœur humain que l'étude de l'histoire jointe à l'expérience de la vie, de telle manière qu'elles s'éclairent mutuellement. On a l'obligation de fournir cette nourriture aux esprits des adolescents et des hommes. Mais il faut que ce soit une nourriture de vérité. Il faut non seulement que les faits soient exacts autant qu'on peut les contrôler, mais qu'ils soient montrés dans leur perspective vraie relativement au bien et au mal.\par
L'histoire est un tissu de bassesses et de cruautés où quelques gouttes de pureté brillent de loin en loin. S'il en est ainsi, c'est d'abord qu'il y a peu de pureté parmi les hommes ; puis que la plus grande partie de ce peu est et demeure cachée. Il faut en chercher si l'on peut des témoignages indirects. Les églises romanes, le chant grégorien n'ont pu surgir que parmi des populations où il y avait beaucoup plus de pureté qu'il n'y en a eu aux siècles suivants.\par
Pour aimer la France, il faut sentir qu'elle a un passé, mais il ne faut pas aimer l'enveloppe historique de ce passé. Il faut en aimer la partie muette, anonyme, disparue.\par
\par
Il est absolument faux qu'un mécanisme providentiel transmette à la mémoire de la postérité ce qu'une époque possède de meilleur. Par la nature des choses, c'est la fausse grandeur qui est transmise. Il y a bien un mécanisme providentiel, mais il opère seulement de manière à mêler un peu de grandeur authentique à beaucoup de fausse grandeur ; à nous de les discerner. Sans lui nous serions perdus.\par
La transmission de la fausse grandeur à travers les siècles n'est pas particulière à l'histoire. C'est une loi générale. Elle gouverne aussi par exemple les lettres et les arts. Il y a une certaine domination du talent littéraire sur les siècles qui répond à la domination du talent politique dans l'espace ; ce sont des dominations de même nature, également temporelles, appartenant également au domaine de la matière et de la force, également basses. Aussi peuvent-elles être un objet de marché et d'échange.\par
L'Arioste n'a pas rougi de dire à son maître le duc d'Este, au cours de son poème, quelque chose qui revient à ceci : Je suis en votre pouvoir pendant ma vie, et il dépend de vous que je sois riche ou pauvre. Mais votre nom est en mon pouvoir dans l'avenir, et il dépend de moi que dans trois cents ans on dise de vous du bien, du mal, ou rien. Nous avons intérêt à nous entendre. Donnez-moi la faveur et la richesse et je ferai votre éloge.\par
Virgile avait bien trop le sens des convenances pour exposer publiquement un marché de cette nature. Mais en fait, c'est exactement le marché qui a eu lieu entre Auguste et lui. Ses vers sont souvent délicieux à lire, mais malgré cela, pour lui et ses pareils, il faudrait trouver un autre nom que celui de poète. La poésie ne se vend pas. Dieu serait injuste si l'{\itshape Énéid}e, ayant été composée dans ces conditions, valait l'{\itshape Iliade}. Mais Dieu est juste, et l'{\itshape Énéide} est infiniment loin de cette égalité.\par
Ce n'est pas seulement dans l'étude de l'histoire, c'est dans toutes les études proposées aux enfants que le bien est méprisé, et une fois hommes, ils ne trouvent dans les nourritures offertes à leur esprit que des motifs de s'endurcir dans ce mépris.\par
Il est évident, c'est une vérité passée à l'état de lieu commun parmi les enfants et les hommes, que le talent n'a rien à voir avec la moralité. Or on ne propose à l'admiration des enfants et des hommes que le talent dans tous les domaines. Dans toutes les manifestations du talent, quelles qu'elles soient, ils voient s'étaler avec impudence l'absence des vertus qu'on leur recommande de pratiquer. Que peut-on en conclure, sinon que la vertu est le propre de la médiocrité ? Cette persuasion a pénétré si avant que le mot même de vertu est maintenant ridicule, lui qui était autrefois si plein de sens, comme aussi ceux d'honnêteté et de bonté. Les Anglais sont plus proches du passé que les autres pays ; aussi n'y a-t-il aujourd'hui aucun mot dans la langue française pour traduire « good » et « wicked ». – Comment un enfant qui voit glorifier dans les leçons d'histoire la cruauté et l'ambition ; dans celles de littérature l'égoïsme, l'orgueil, la vanité, la soif de faire du bruit ; dans celles de science toutes les découvertes qui ont bouleversé la vie des hommes, sans qu'aucun compte soit tenu ni de la méthode de la découverte ni de l'effet du bouleversement ; comment apprendrait-il à admirer le bien ? Tout ce qui essaie d'aller contre ce courant si général, par exemple les éloges de Pasteur, sonne faux. Dans l'atmosphère de la fausse grandeur, il est vain de vouloir retrouver la véritable. Il faut mépriser la fausse grandeur.\par
Il est vrai que le talent n'a pas de lien avec la moralité ; mais c'est qu'il n'y a pas de grandeur dans le talent. Il est faux qu'il n'y ait pas de liens entre la parfaite beauté, la parfaite vérité, la parfaite justice ; il y a plus que des liens, il y a une unité mystérieuse, car le bien est un.\par
Il y a un point de grandeur où le génie créateur de beauté, le génie révélateur de vérité, l'héroïsme et la sainteté sont indiscernables. Déjà, à l'approche de ce point, on voit les grandeurs tendre à se confondre. On ne peut pas discerner chez Giotto le génie du peintre et l'esprit franciscain ; ni dans les tableaux et les poèmes de la secte Zen en Chine le génie du peintre ou du poète et l'état d'illumination mystique ; ni, quand Velasquez met sur la toile des rois et des mendiants, le génie du peintre et l'amour brûlant et impartial qui transperce le fond des âmes. L'{\itshape Iliade}, les tragédies d'Eschyle et celles de Sophocle portent la marque évidente que les poètes qui ont fait cela étaient dans l'état de sainteté. Du point de vue purement poétique, sans tenir compte de rien d'autre, il est infiniment préférable d'avoir composé le Cantique de saint François d'Assise, ce joyau de beauté parfaite, plutôt que toute l'œuvre de Victor Hugo. Racine a écrit la seule œuvre de toute la littérature française qui puisse presque être mise à côté des grands chefs-d'œuvre grecs au moment où son âme était travaillée par la conversion. Il était loin de la sainteté quand il a écrit ses autres pièces, mais aussi on n'y trouve pas cette beauté déchirante. Une tragédie comme {\itshape King Lear} est le fruit direct du pur esprit d'amour. La sainteté rayonne dans les églises romanes et le chant grégorien. Monteverdi, Bach, Mozart furent des êtres purs dans leur vie comme dans leur œuvre.\par
S'il y a des génies chez qui le génie est pur au point d'être manifestement tout proche de la grandeur propre aux plus parfaits des saints, pourquoi perdre son temps à en admirer d'autres ? On peut user des autres, puiser chez eux des connaissances et des jouissances ; mais pourquoi les aimer ? Pourquoi accorder son cœur à autre chose qu'au bien ?\par
Il y a dans la littérature française un courant discernable de pureté. Dans la poésie, il faut commencer par Villon, le premier, le plus grand. Nous ne savons rien de ses fautes, ni même s'il y a eu faute de sa part ; mais la pureté de l'âme est manifeste à travers l'expression déchirante du malheur. Le dernier ou presque est Racine, à cause de Phèdre et des Cantiques spirituels ; entre les deux on peut nommer Maurice Scève, d'Aubigné, Théophile de Viau, qui furent trois grands poètes et trois êtres d'une rare élévation. Au XIX\textsuperscript{e} siècle, tous les poètes furent plus ou moins gens de lettres, ce qui souille honteusement la poésie ; du moins Lamartine et Vigny ont réellement aspiré à quelque chose de pur et d'authentique. Il y a un peu de vraie poésie dans Gérard de Nerval. À la fin du siècle, Mallarmé a été admiré autant comme une espèce de saint que comme un poète, et c'étaient en lui deux grandeurs indiscernables l'une de l'autre. Mallarmé est un vrai poète.\par
Dans la prose, il y a peut-être une pureté mystérieuse dans Rabelais, où d'ailleurs tout est mystérieux. Il y en a certainement dans Montaigne, malgré ses nombreuses carences, parce qu'il était toujours habité par la présence d'un être pur et sans lequel il serait sans doute demeuré dans la médiocrité, c'est-à-dire La Boétie. Au XVII\textsuperscript{e} siècle, on peut penser à Descartes, à Retz, à Port-Royal, surtout à Molière. Au XVIII\textsuperscript{e}, il y a Montesquieu et Rousseau. C'est peut-être tout.\par
En supposant quelque exactitude dans cette énumération, cela ne signifierait pas qu'il ne faille pas lire le reste, mais qu'il faut le lire sans croire y trouver le génie de la France. Le génie de la France ne réside que dans ce qui est pur.\par
On a absolument raison de dire que c'est un génie chrétien et hellénique. C'est pourquoi il serait légitime de donner une part bien moindre dans l'éducation et la culture des Français aux choses spécifiquement françaises qu'à l'art roman, au chant grégorien, à la poésie liturgique et à l'art, à la poésie, à la prose des Grecs de la bonne époque. Là on peut boire à flots de la beauté absolument pure à tous égards.\par
Il est malheureux que le grec soit regardé comme une matière d'érudition pour spécialistes. Si l'on cessait de subordonner l'étude du grec à celle du latin, et si l'on cherchait seulement à rendre un enfant capable de lire facilement et avec plaisir un texte grec facile avec une traduction à côté, on pourrait diffuser une légère connaissance du grec très largement, même en dehors du secondaire. Tout enfant un peu doué pourrait entrer en contact direct avec la civilisation où nous avons puisé les notions mêmes de beauté, de vérité et de justice.\par
Jamais l'amour du bien ne s'allumera dans les cœurs à travers toute la population, comme il est nécessaire au salut du pays, tant qu'on croira que dans n'importe quel domaine la grandeur peut être l'effet d'autre chose que du bien.\par
C'est pourquoi le Christ a dit : « Un bon arbre produit de beaux fruits, un mauvais arbre produit des fruits pourris. » Ou une œuvre d'art parfaitement belle est un fruit pourri, ou l'inspiration qui la produit est proche de la sainteté.\par
Si le bien pur n'était jamais capable de produire ici-bas de la grandeur réelle dans l'art, dans la science, dans la spéculation théorique, dans l'action publique, si dans tous ces domaines il n'y avait que de la fausse grandeur, si dans tous ces domaines tout était méprisable, et par suite condamnable, il n'y aurait aucune espérance pour la vie profane. Il n'y aurait pas d'illumination possible de ce monde par l'autre.\par
Il n'en est pas ainsi, et c'est pourquoi il est indispensable de discerner la grandeur réelle de la fausse, et de proposer à l'amour seulement la première. La grandeur réelle est le beau fruit qui pousse sur le bon arbre, et le bon arbre est une disposition de l'âme proche de la sainteté. Les autres prétendues grandeurs doivent être examinées froidement, comme on examine des curiosités naturelles. Si, en fait, la répartition sous les deux rubriques peut comporter des erreurs, il n'en est pas moins essentiel d'enfoncer au plus profond du cœur le principe même de la répartition.\par
La conception moderne de la science est responsable, comme celle de l'histoire et celle de l'art des monstruosités actuelles, et doit être, elle aussi transformée avant qu'on puisse espérer voir poindre une civilisation meilleure.\par
Cela est d'autant plus capital que, bien que la science soit rigoureusement une affaire de spécialistes, le prestige de la science et des savants sur tous les esprits est immense, et dans les pays non totalitaires dépasse de loin tout autre. En France, quand la guerre a éclaté, c'était peut-être même l'unique qui subsistât ; rien d'autre n'était plus objet de respect. Dans l'atmosphère du Palais de la Découverte, en 1937, il y avait quelque chose à la fois de publicitaire et de presque religieux, en prenant ce mot dans son sens le plus grossier. La science, avec la technique qui n'en est que l'application, est notre seul titre à être fiers d'être des Occidentaux, des gens de race blanche, des modernes.\par
Un missionnaire qui persuade un Polynésien d'abandonner ses traditions ancestrales, si poétiques et si belles, sur la création du monde, pour celles de la Genèse, imprégnées d'une poésie très semblable, ce missionnaire puise sa force persuasive dans la conscience qu'il a de sa supériorité d'homme blanc, conscience fondée sur la science. Il est pourtant personnellement étranger à la science autant que le Polynésien, car quiconque n'est pas spécialiste y est tout à fait étranger. La Genèse y est bien plus étrangère encore. Un instituteur de village qui se moque du curé, et dont l'attitude détourne les enfants d'aller à la messe puise sa force persuasive dans la conscience qu'il a de sa supériorité d'homme moderne sur un dogme moyenâgeux, conscience fondée sur la science. Pourtant, relativement à ses possibilités de contrôle, la théorie d'Einstein est pour le moins aussi peu fondée et aussi contraire au bon sens que la tradition chrétienne concernant la conception et là naissance du Christ.\par
On doute de tout en France, on ne respecte rien il a des gens qui méprisent la religion, la patrie, l’État, les tribunaux, la propriété, l'art, enfin toutes choses ; mais leur mépris s'arrête devant la science. Le scientisme le plus grossier n'a pas d'adeptes plus fervents que les anarchistes. Le Dantec est leur grand homme. Les « bandits tragiques » de Bonnot y puisaient leur inspiration, et celui d'entre eux qui était plus que les autres un héros aux yeux de ses camarades était surnommé « Raymond la Science ». À l'autre pôle, on rencontre des prêtres ou des religieux pris par la vie religieuse au point de mépriser toutes les valeurs profanes, mais leur mépris s'arrête devant la science. Dans toutes les polémiques où la religion et la science semblent être en conflit, il y a du côté de l'Église une infériorité intellectuelle presque comique, car elle est due, non à la force des arguments adverses, généralement très médiocres, mais uniquement à un complexe d'infériorité.\par
\par
Par rapport au prestige de la science il n'y a pas aujourd'hui d'incroyants. Cela confère aux savants, et aussi aux philosophes et écrivains en tant qu'ils écrivent sur la science, une responsabilité égale à celle qu'avaient les prêtres du XIII\textsuperscript{e} siècle. Les uns et les autres sont des êtres humains que la société nourrit pour qu'ils aient le loisir, de chercher, de trouver et de communiquer ce que c'est que la vérité. Au XX\textsuperscript{e} siècle comme au XIII\textsuperscript{e}, le pain dépensé à cet effet est probablement, par malheur, du pain gaspillé, ou peut-être pire.\par
L'Église du XIII\textsuperscript{e} siècle avait le Christ ; mais elle avait l'Inquisition. La science du XX\textsuperscript{e} siècle n'a pas d'Inquisition ; mais elle n'a pas non plus le Christ, ni rien d'équivalent.\par
La charge assumée aujourd'hui par les savants et par tous ceux qui écrivent autour de la science est d'un poids tel qu'eux aussi, comme les historiens et même davantage, sont peut-être plus coupables des crimes d'Hitler qu'Hitler lui-même.\par
C'est ce qui apparaît dans un passage de Mein Kampf : « L'homme ne doit jamais tomber dans l'erreur de croire qu'il est seigneur et maître de la nature... Il sentira dès lors que dans un monde où les planètes et les soleils suivent des trajectoires circulaires, où des lunes tournent autour des planètes, où la force règne partout et seule en maîtresse de la faiblesse, qu'elle contraint à la servir docilement ou qu'elle brise, l'homme ne peut pas relever de lois spéciales. »\par
Ces lignes expriment d'une manière irréprochable la seule conclusion qu'on puisse raisonnablement tirer de la conception du monde enfermée dans notre science. La vie entière d'Hitler n'est que la mise en œuvre de cette conclusion. Qui peut lui reprocher d'avoir mis en œuvre ce qu'il a cru reconnaître pour vrai ? Ceux qui, portant en eux les fondements de la même croyance, n'en ont pas pris conscience et ne l'ont pas traduite en actes, n'ont échappé au crime que faute de posséder une certaine espèce de courage qui est en lui.\par
Encore une fois, ce n'est pas l'adolescent abandonné, misérable vagabond, à l'âme affamée, qu'il est juste d'accuser, mais ceux qui lui ont donné à manger du mensonge. Et ceux qui lui ont donné à manger du mensonge, c'étaient nos aînés, à qui nous sommes semblables.\par
Dans la catastrophe de notre temps, les bourreaux et les victimes sont, les uns et les autres, avant tout les porteurs involontaires d'un témoignage sur l'atroce misère au fond de laquelle nous gisons.\par
Pour avoir le droit de punir les coupables, il faudrait d'abord nous purifier de leur crime, contenu sous toutes sortes de déguisements dans notre propre âme. Mais si nous réussissons cette opération, une fois qu'elle sera accomplie nous n'aurons plus aucun désir de punir, et si nous croyons être obligés de le faire, nous le ferons le moins possible et avec une extrême douleur.\par
Hitler a très bien vu l'absurdité de la conception du XVIII\textsuperscript{e} siècle encore en faveur aujourd'hui, et qui d'ailleurs a déjà sa racine dans Descartes. Depuis deux ou trois siècles on croit à la fois que la force est maîtresse unique de tous les phénomènes de la nature, et que les hommes peuvent et doivent fonder sur la justice, reconnue au moyen de la raison, leurs relations mutuelles. C'est une absurdité criante. Il n'est pas concevable que tout dans l'univers soit absolument soumis à l'empire de la force et que l'homme puisse y être soustrait, alors qu'il est fait de chair et de sang et que sa pensée vagabonde au gré des impressions sensibles.\par
Il n'y a qu'un choix à faire. Ou il faut apercevoir à l'œuvre dans l'univers, à côté de la force, un principe autre qu'elle, ou il faut reconnaître la force comme maîtresse unique et souveraine des relations humaines aussi.\par
Dans le premier cas, on se met en opposition radicale avec la science moderne telle qu'elle a été fondée par Galilée, Descartes et plusieurs autres, poursuivie au XVIII\textsuperscript{e} siècle, notamment par Newton, au XIX\textsuperscript{e}, au XX\textsuperscript{e}. Dans le second, on se met en opposition radicale avec l'humanisme qui a surgi à la Renaissance, qui a triomphé en 1789, qui, sous une forme considérablement dégradée, a servi d'inspiration à toute la III\textsuperscript{e} République.\par
La philosophie qui a inspiré l'esprit laïque et la politique radicale est fondée à la fois sur cette science et sur cet humanisme, qui sont, on le voit, manifestement incompatibles. On ne peut donc pas dire que la victoire d'Hitler sur la France de 1940 ait été la victoire d'un mensonge sur une vérité. Un mensonge incohérent a été vaincu par un mensonge cohérent. C'est pourquoi, en même temps que les armes, les esprits ont fléchi.\par
Au cours des derniers siècles, on a confusément senti la contradiction entre la science et l'humanisme, quoiqu'on n'ait jamais eu le courage intellectuel de la regarder en face. Sans l'avoir d'abord exposée aux regards, on a tenté de la résoudre. Cette improbité d'intelligence est toujours punie d'erreur.\par
L'utilitarisme a été le fruit d'une de ces tentatives. C'est la supposition d'un merveilleux petit mécanisme au moyen duquel la force, en entrant dans la sphère des relations humaines, devient productrice automatique de justice.\par
Le libéralisme économique des bourgeois du XIX\textsuperscript{e} siècle repose entièrement sur la croyance en un tel mécanisme. La seule restriction était que, pour avoir la propriété d'être productrice automatique de justice, la force doit avoir la forme de l'argent, à l'exclusion de tout usage soit des armes soit du pouvoir politique.\par
Le marxisme n'est que la croyance en un mécanisme de ce genre. Là, la force est baptisée histoire ; elle a pour forme la lutte des classes ; la justice est rejetée dans un avenir qui doit être précédé d'une espèce de catastrophe apocalyptique.\par
Et Hitler aussi, après son moment de courage intellectuel et de clairvoyance, est tombé dans la croyance en ce petit mécanisme. Mais il lui fallait un modèle de machine inédit. Seulement il n'a pas le goût ni la capacité de l'invention intellectuelle, en dehors de quelques éclairs d'intuition géniale. Aussi a-t-il emprunté son modèle de machine aux gens qui l'obsédaient continuellement par la répulsion qu'ils lui inspiraient. Il a simplement choisi pour machine la notion de la race élue, la race destinée à tout faire plier, et ensuite à établir parmi ses esclaves l'espèce de justice qui convient à l'esclavage.\par
À toutes ces conceptions en apparence diverses et au fond si semblables, il n'y a qu'un seul inconvénient, le même pour toutes. C'est que ce sont des mensonges.\par
La force n'est pas une machine à créer automatiquement de la justice. C'est un mécanisme aveugle dont sortent au hasard, indifféremment, les effets justes ou injustes, mais, par le jeu des probabilités, presque toujours injustes. Le cours du temps n'y fait rien ; il n'augmente pas dans le fonctionnement de ce mécanisme la proportion infime des effets par hasard conformes à la justice.\par
Si la force est absolument souveraine, la justice est absolument irréelle. Mais elle ne l'est pas. Nous le savons expérimentalement. Elle est réelle au fond du cœur des hommes. La structure d'un cœur humain est une réalité parmi les réalités de cet univers, au même titre que la trajectoire d'un astre.\par
Il n'est pas au pouvoir d'un homme d'exclure absolument toute espèce de justice des fins qu'il assigne à ses actions. Les nazis eux-mêmes ne l'ont pas pu. Si c'était possible à des hommes, eux sans doute l'auraient pu.\par
(Soit dit en passant, leur conception de l'ordre juste qui doit en fin de compte résulter de leurs victoires repose sur la pensée que, pour tous ceux qui sont esclaves par nature, la servitude est la condition à la fois la plus juste et la plus heureuse. Or c'est là la pensée même d'Aristote, son grand argument pour l'apologie de l'esclavage. Saint Thomas, bien qu'il n'approuvât pas l'esclavage, regardait Aristote comme la plus grande autorité pour tous les sujets d'étude accessibles à la raison humaine, au nombre desquels la justice. Par suite, l'existence dans le christianisme contemporain d'un courant thomiste constitue un lien de complicité – parmi beaucoup d'autres, malheureusement – entre le camp nazi et le camp adverse. Car, bien que nous repoussions cette pensée d'Aristote, nous sommes forcément amenés dans notre ignorance à en accueillir d'autres qui ont été en lui la racine de celle-là. Un homme qui prend la peine d'élaborer une apologie de l'esclavage n'aime pas la justice. Le siècle où il vit n'y fait rien. Accepter comme ayant autorité la pensée d'un homme qui n'aime pas la justice, cela constitue une offense à la justice, inévitablement punie par la diminution du discernement. Si saint Thomas a commis cette offense, rien ne nous contraint à la répéter.)\par
Si la justice est ineffaçable au cœur de l'homme, elle a une réalité en ce monde. C'est la science alors qui a tort.\par
Non pas la science, s'il faut parler exactement, mais la science moderne. Les Grecs possédaient une science qui est le fondement de la nôtre. Elle comprenait l'arithmétique, la géométrie, l'algèbre sous une forme qui leur était propre, l'astronomie, la mécanique, la physique, la biologie. La quantité des connaissances accumulées était naturellement beaucoup moindre. Mais par le caractère scientifique, dans la signification que ce mot a pour nous, d'après les critères valables à nos yeux, cette science égalait et dépassait la nôtre. Elle était plus exacte, plus précise, plus rigoureuse. L'usage de la démonstration et celui de la méthode expérimentale étaient conçus l'un et l'autre dans une clarté parfaite.\par
Si cela n'est pas généralement reconnu, c'est uniquement parce que le sujet lui-même est peu connu. Peu de gens, s'ils n'y sont poussés par une vocation particulière, auront l'idée de se plonger dans l'atmosphère de la science grecque comme dans une chose actuelle et vivante. Ceux qui l'ont fait n'ont pas eu de peine à reconnaître la vérité.\par
La génération de mathématiciens qui approche aujourd'hui de la quarantaine a reconnu qu'après un long fléchissement de l'esprit scientifique dans le développement de la mathématique, le retour à la rigueur indispensable à des savants est en train de s'opérer par l'usage de méthodes presque identiques aux méthodes des géomètres grecs.\par
Quant aux applications techniques, si la science grecque n'en a pas beaucoup produit, ce n'est pas qu'elle n'en fût pas susceptible, c'est que les savants grecs ne le voulaient pas. Ces gens, visiblement très arriérés relativement à nous, comme il convient à des hommes d'il y a vingt-cinq siècles, redoutaient l'effet d'inventions techniques susceptibles d'être mises en usage par les tyrans et les conquérants. Ainsi, au lieu de livrer au public le plus grand nombre possible de découvertes techniques et de les vendre au plus offrant, ils conservaient rigoureusement secrètes celles qu'il leur arrivait de faire pour s'amuser ; et vraisemblablement ils restaient pauvres. Mais Archimède mit une fois en œuvre son savoir technique pour défendre sa patrie. Il le mit en œuvre lui-même, sans révéler aucun secret à personne. Le récit des merveilles qu'il sut accomplir est encore aujourd'hui en grande partie incompréhensible pour nous. Il réussit si bien que les Romains n'entrèrent dans Syracuse qu'au prix d'une demi-trahison.\par
Or cette science, aussi scientifique que la nôtre ou davantage, n'était absolument pas matérialiste. Bien plus, ce n'était pas une étude profane. Les Grecs la regardaient comme une étude religieuse.\par
\par
Les Romains tuèrent Archimède. Peu après ils tuèrent la Grèce, comme les Allemands, sans l'Angleterre, auraient tué la France. La science grecque disparut complètement. Dans la civilisation romaine il n'en subsista rien. Si le souvenir en fut transmis au Moyen Âge, ce fut avec la pensée dite gnostique, dans des milieux initiatiques. Même en ce cas, il semble bien qu'il y ait eu seulement conservation et non continuation créatrice ; excepté peut-être en ce qui concerne l'alchimie, dont on sait si peu de choses.\par
Quoi qu'il en soit, dans le domaine public, la science grecque ne ressuscita qu'au début du XVI\textsuperscript{e} siècle (sauf erreur de date), en Italie et en France. Elle prit très vite un essor prodigieux et envahit la vie entière de l'Europe. Aujourd'hui, la presque totalité de nos pensées, de nos coutumes, de nos réactions, de notre comportement à tous porte une marque imprimée soit par son esprit soit par ses applications.\par
Cela est vrai plus particulièrement des intellectuels, même s'ils ne sont pas ce qu'on nomme des « scientifiques », et bien plus vrai encore des ouvriers, qui passent toute leur vie dans un univers artificiel constitué par les applications de la science.\par
Mais, comme dans certains contes, cette science réveillée après presque deux millénaires de léthargie n'était plus la même. On l'avait changée. C'en était une autre, absolument incompatible avec tout esprit religieux.\par
C'est pour cela qu'aujourd'hui la religion est une chose du dimanche matin. Le reste de la semaine est dominé par l'esprit de la science.\par
Les incroyants, qui y soumettent toute leur semaine, ont un sentiment triomphant d'unité intérieure. Mais ils ont tort, car leur morale n'est pas moins en contradiction avec la science que la religion des autres. Hitler l'a clairement vu. Il le fait voir d'ailleurs à beaucoup de gens, partout où est sensible la présence ou la menace des S. S., et même plus loin. Aujourd'hui il n'y a guère que l'adhésion sans réserves à un système totalitaire brun, rouge ou autre, qui puisse donner, pour ainsi dire, une illusion solide d'unité intérieure. C'est pourquoi elle constitue une tentation si forte pour tant d'âmes en désarroi.\par
Chez les chrétiens, l'incompatibilité absolue entre l'esprit de la religion et l'esprit de la science, qui ont l'un et l'autre leur adhésion, loge dans l'âme en permanence un malaise sourd et inavoué. Il peut être presque insensible ; il est selon les cas plus ou moins sensible ; il est, bien entendu, à peu près toujours inavoué. Il empêche la cohésion intérieure. Il s'oppose à ce que la lumière chrétienne imprègne toutes les pensées. Par un effet indirect de sa présence continuelle, les chrétiens les plus fervents portent à chaque heure de leur vie des jugements, des opinions, où se trouvent appliqués à leur insu des critères contraires à l'esprit du christianisme. Mais la conséquence la plus funeste de ce malaise est de rendre impossible que s'exerce dans sa plénitude la vertu de probité intellectuelle.\par
\par
Le phénomène moderne de l'irréligiosité du peuple s'explique presque entièrement par l'incompatibilité entre la science et la religion. Il s'est développé quand on a commencé à installer le peuple des villes dans un univers artificiel, cristallisation de la science. En Russie, la transformation a été hâtée par une propagande qui, pour déraciner la foi, s'appuyait presque entièrement sur l'esprit de la science et de la technique. Partout, après que le peuple des villes fut devenu irréligieux, le peuple des campagnes, rendu influençable par son complexe d'infériorité à l'égard des villes, a suivi, bien qu'à un degré moindre.\par
Du fait même de la désertion des églises par le peuple, la religion fut automatiquement située à droite, devint une chose bourgeoise, une chose de bien-pensants. Car en fait une religion instituée est bien obligée de s'appuyer sur ceux qui vont à l'église. Elle ne peut s'appuyer sur ceux qui restent dehors. Il est vrai, que dès avant cette désertion, la servilité du clergé envers les pouvoirs temporels lui a fait faire des fautes graves. Mais elles auraient été réparables sans cette désertion. Si elles ont provoqué cette désertion pour une part, ce fut pour une part très petite. C'est presque uniquement la science qui a vidé les églises.\par
Si une partie de la bourgeoisie a été moins gênée dans sa piété par la science que ne l'a été la classe ouvrière, c'est d'abord parce qu'elle avait un contact moins permanent et moins charnel avec les applications de la science. Mais c'est surtout parce qu'elle n'avait pas la foi. Qui n'a pas la foi ne peut pas la perdre. Sauf quelques exceptions, la pratique de la religion était pour elle une convenance. La conception scientifique du monde n'empêche pas d'observer les convenances.\par
Ainsi le christianisme est en fait, à l'exception de quelques foyers de lumière, une convenance relative aux intérêts de ceux qui exploitent le peuple.\par
Il n'est donc pas étonnant qu'il ait une part somme toute si médiocre, en ce moment, dans la lutte contre la forme actuelle du mal.\par
D'autant plus que, même dans les milieux, dans les cœurs où la vit religieuse est sincère et intense, elle a trop souvent au centre même un principe d'impureté par une insuffisance de l'esprit de vérité. L'existence de la science donne mauvaise conscience aux chrétiens. Peu d'entre eux osent être certains que, s'ils partaient de zéro et s'ils considéraient tous les problèmes en abolissant toute préférence, dans un esprit d'examen absolument impartial, le dogme chrétien leur apparaîtrait comme étant manifestement et totalement la vérité.\par
Cette incertitude devrait relâcher leurs liens avec la religion ; il n'en est pas ainsi, et ce qui empêche qu'il en soit ainsi, c'est que la vie religieuse leur fournit quelque chose dont ils ont besoin. Ils sentent plus ou moins confusément eux-mêmes qu'ils sont attachés à la religion par un besoin. Or le besoin n'est pas un lien légitime de l'homme à Dieu. Comme dit Platon, il y a une grande distance entre la nature de la nécessité et celle du bien. Dieu se donne à l'homme gratuitement et par surcroît, mais l'homme ne doit pas désirer recevoir. Il doit se donner totalement, inconditionnellement, et pour le seul motif qu'après avoir erré d'illusion en illusion dans la recherche ininterrompue du bien, il est certain d'avoir discerné la vérité en se tournant vers Dieu.\par
Dostoïevski a commis le plus affreux blasphème quand il a dit : « Si le Christ n'est pas la vérité, je préfère être hors de la vérité avec le Christ. » Le Christ a dit : « Je suis la vérité. » Il a dit aussi qu'il était du pain, de la boisson ; mais il a dit : « Je suis le pain vrai, la boisson vraie », c'est-à-dire le pain qui est seulement de la vérité, la boisson qui est seulement de la vérité. Il faut le désirer d'abord comme vérité, ensuite seulement comme nourriture.\par
Il faut bien qu'on ait complètement oublié ces choses, puisqu'on a pu prendre Bergson pour un chrétien ; lui qui croyait voir dans l'énergie des mystiques la forme achevée de cet élan vital dont il s'est fait une idole. Alors que la merveille, dans le cas des mystiques et des saints, n'est pas qu'ils aient plus de vie, une vie plus intense que les autres mais qu'en eux la vérité soit devenue de la vie. Dans ce monde-ci la vie, l'élan vital cher à Bergson, n'est que du mensonge, et la mort seule est vraie. Car la vie contraint à croire ce qu'on a besoin de croire pour vivre ; cette servitude a été érigée en doctrine sous le nom de pragmatisme ; et la philosophie de Bergson est une forme du pragmatisme. Mais les êtres qui malgré la chair et le sang ont franchi intérieurement une limite équivalente à la mort reçoivent par-delà une autre vie, qui n'est pas en premier lieu de la vie, qui est en premier lieu de la vérité. De la vérité devenue vivante. Vraie comme la mort et vivante comme la vie. Une vie, comme disent les contes de Grimm, blanche comme la neige et rouge comme le sang. C'est elle qui est le souffle de vérité, l'Esprit divin.\par
Pascal déjà avait commis le crime du manque de probité dans la recherche de Dieu. Ayant eu l'intelligence formée par la pratique de la science, il n'a pas osé espérer qu'en laissant à cette intelligence son libre jeu elle reconnaîtrait dans le dogme chrétien une certitude. Et il n'a pas osé non plus courir le risque d'avoir à se passer du christianisme. Il a entrepris une recherche intellectuelle en décidant à l'avance où elle devait le mener. Pour éviter tout risque d'aboutir ailleurs, il s'est soumis à une suggestion consciente et voulue. Après quoi il a cherché des preuves. Dans le domaine des probabilités, des indications, il a aperçu des choses très fortes. Mais quant aux preuves proprement dites, il n'en a mis en avant que de misérables, l'argument du pari, les prophéties, les miracles. Ce qui est plus grave pour lui, c'est qu'il n'a jamais atteint la certitude. Il n'a jamais reçu la foi, et cela parce qu'il avait cherché à se la procurer.\par
La plupart de ceux qui vont au christianisme, ou qui, y étant nés et ne l'ayant jamais quitté, s'y attachent d'un mouvement vraiment sincère et fervent, sont poussés et ensuite maintenus par un besoin du cœur. Ils ne pourraient pas se passer de la religion. Du moins ils ne pourraient pas s'en passer sans qu'il en résulte en eux une espèce de dégradation. Or pour que le sentiment religieux procède de l'esprit de vérité, il faut être totalement prêt à abandonner sa religion, dût-on perdre ainsi toute raison de vivre, au cas où elle serait autre chose que la vérité. Dans cette disposition d'esprit seulement on peut discerner s'il y a en elle ou non de la vérité. Autrement on n'ose pas même poser le problème dans sa rigueur.\par
Dieu ne doit pas être pour un cœur humain une raison de vivre comme est le trésor pour l'avare. Harpagon et Grandet aimaient leur trésor ; ils se seraient fait tuer pour lui ; ils seraient morts de chagrin à cause de lui ; ils auraient accompli des merveilles de courage et d'énergie pour lui. On peut aimer Dieu ainsi. Mais on ne le doit pas. Ou plutôt c'est seulement à une certaine partie de l'âme que cette espèce d'amour est permise, parce qu'elle n'est susceptible d'aucun autre, mais elle doit rester soumise et abandonnée à la partie de l'âme qui vaut davantage.\par
On peut affirmer sans crainte d'exagération qu'aujourd'hui l'esprit de vérité est presque absent de la vie religieuse.\par
Cela se constate entre autres dans la nature des arguments apportés en faveur du christianisme. Plusieurs sont de l'espèce publicité pour pilules Pink. C'est le cas pour Bergson et tout ce qui s'en inspire. Dans Bergson la foi apparaît comme une pilule Pink de l'espèce supérieure, qui communique un degré, prodigieux de vitalité. Il en est de même pour l'argumentation historique. Elle consiste à dire : « Voyez comme les hommes étaient médiocres avant le Christ. Le Christ est venu, et voyez comme les hommes, malgré les défaillances, ont été ensuite, dans l'ensemble, quelque chose de bien ! » Cela est absolument contraire à la vérité. Mais même si c'était vrai, c'est ramener, l'apologétique au niveau des réclames pour spécialités pharmaceutiques, qui décrivent le malade avant et après. C'est mesurer l'efficacité de la Passion du Christ, qui, si elle n'est pas fictive, est nécessairement infinie, par un effet historique, temporel, humain, qui, fût-il même réel, ce qui n’est pas, serait nécessairement fini.\par
Le pragmatisme a envahi et souillé la conception même de la foi.\par
Si l'esprit de vérité est presque absent de la vie religieuse, il serait singulier qu'il fût présent dans la vie profane. Ce serait le renversement d'une hiérarchie éternelle. Mais il n'en est pas ainsi.\par
Les savants exigent du public qu'il accorde à la science ce respect religieux qui est dû à la vérité, et le public les croit. Mais on le trompe. La science n'est pas un fruit de l'Esprit de vérité, et cela est évident dès qu'on fait attention.\par
Car l'effort de la recherche scientifique, telle qu'elle a été comprise depuis le XVI\textsuperscript{e} siècle jusqu'à nos jours, ne peut pas avoir pour mobile l'amour de la vérité.\par
Il y a là un critère dont l'application est universelle et sûre ; il consiste, pour apprécier une chose quelconque, à tenter de discerner la proportion de bien contenue, non dans la chose elle-même, mais dans les mobiles de l'effort qui l'a produite. Car autant il y a de bien dans le mobile, autant il y en a dans la chose elle-même, et non davantage. La parole du Christ sur les arbres et les fruits le garantit.\par
Dieu seul, il est vrai, discerne les mobiles dans le secret des cœurs. Mais la conception qui domine une activité, conception qui généralement n'est pas secrète, est compatible avec certains mobiles et non avec d'autres ; il en est qu'elle exclut par nécessité, par la nature des choses.\par
Il s'agit donc d'une analyse qui mène à apprécier le produit d'une activité humaine particulière par l'examen des mobiles compatibles avec la conception qui y préside.\par
De cette analyse découle une méthode pour améliorer les hommes – peuples et individus, et soi-même pour commencer – en modifiant les conceptions de manière à faire jouer les mobiles les plus purs.\par
La certitude que toute conception incompatible avec des mobiles vraiment purs est elle-même entachée d'erreur est le premier des articles de foi. La foi est avant tout la certitude que le bien est un. Croire qu'il y a plusieurs biens distincts et mutuellement indépendants, comme vérité, beauté, moralité, c'est cela qui constitue le péché de polythéisme, et non pas laisser l'imagination jouer avec Apollon et Diane.\par
En appliquant cette méthode à l'analyse de la science des trois ou quatre derniers siècles, on doit reconnaître que le beau nom de vérité est infiniment au-dessus d'elle. Les savants, dans l'effort qu'ils fournissent jour après jour tout le long de leur vie, ne peuvent pas être poussés par le désir de posséder de la vérité. Car ce qu'ils acquièrent, ce sont simplement des connaissances, et les connaissances ne sont pas par elles-mêmes un objet de désir.\par
Un enfant apprend une leçon de géographie pour avoir une bonne note, ou par obéissance aux ordres reçus, ou pour faire plaisir à ses parents, ou parce qu'il sent une poésie dans les pays lointains et dans leurs noms. Si aucun de ces mobiles n'existe, il n'apprend pas sa leçon.\par
Si à un certain moment il ignore quelle est la capitale du Brésil, et si au moment suivant il l'apprend, il a une connaissance de plus. Mais il n'est aucunement plus proche de la vérité qu'auparavant. L'acquisition d'une connaissance fait dans certains cas approcher de la vérité, mais dans d'autres cas n'en approche pas. Comment discerner les cas ?\par
Si un homme surprend la femme qu'il aime et à qui il avait donné toute sa confiance en flagrant délit d'infidélité, il entre en contact brutal avec de la vérité. S'il apprend qu'une femme qu'il ne connaît pas, dont il entend pour la première fois le nom, dans une ville qu'il ne connaît pas davantage, a trompé son mari, cela ne change aucunement sa relation avec la vérité.\par
\par
Cet exemple fournit la clef. L'acquisition des connaissances fait approcher de la vérité quand il s'agit de la connaissance de ce qu'on aime, et en aucun autre cas.\par
Amour de la vérité est une expression impropre. La vérité n'est pas un objet d'amour. Elle n'est pas un objet. Ce qu'on aime, c'est quelque chose qui existe, que l'on pense, et qui par là peut être occasion de vérité ou d'erreur. Une vérité est toujours la vérité de quelque chose. La vérité est l'éclat de la réalité. L'objet de l'amour n'est pas la vérité, mais la réalité. Désirer la vérité, c'est désirer un contact direct avec de la réalité. Désirer un contact avec une réalité, c'est l'aimer. On ne désire la vérité que pour aimer dans la vérité. On désire connaître la vérité de ce qu'on aime. Au lieu de parler d'amour de la vérité, il vaut mieux parler d'un esprit de vérité dans l'amour.\par
L'amour réel et pur désire toujours avant tout demeurer tout entier dans la vérité, quelle qu'elle puisse être, inconditionnellement. Toute autre espèce d'amour désire avant tout des satisfactions, et de ce fait est principe d'erreur et de mensonge. L'amour réel et pur est par lui-même esprit de vérité. C'est le Saint-Esprit. Le mot grec qu'on traduit par esprit signifie littéralement souffle igné, souffle mélangé à du feu, et il désignait, dans l'Antiquité, la notion que la science désigne aujourd'hui par le mot d'énergie. Ce que nous traduisons « esprit de vérité » signifie l'énergie de la vérité, la vérité comme force agissante. L'amour pur est cette force agissante, l'amour qui ne veut à aucun prix, en aucun cas, ni du mensonge ni de l'erreur.\par
Pour que cet amour fût le mobile du savant dans son effort épuisant de recherche, il faudrait qu'il eût quelque chose à aimer. Il faudrait que la conception qu'il se fait de l'objet de son étude enfermât un bien. Or le contraire a lieu. Depuis la Renaissance – plus exactement, depuis la deuxième moitié de la Renaissance – la conception même de la science est celle d'une étude dont l'objet est placé hors du bien et du mal, surtout hors du bien, considéré sans aucune relation ni au bien ni au mal, plus particulièrement sans aucune relation au bien. La science n'étudie que les faits comme tels, et les mathématiciens eux-mêmes regardent les relations mathématiques comme des faits de l’esprit. Les faits, la force, la matière, isolés, considérés en eux-mêmes, sans relation avec rien d'autre, il n'y a rien là qu'une pensée humaine puisse aimer.\par
Dès lors l'acquisition de connaissances nouvelles n'est pas un stimulant suffisant à l'effort des savants. Il en faut d'autres.\par
Ils ont d'abord le stimulant contenu dans la chasse, dans le sport, dans le jeu. On entend souvent des mathématiciens comparer leur spécialité au jeu d'échecs. Quelques-uns la comparent aux activités où il faut du flair, de l'intuition psychologique, parce qu'ils disent qu'il faut deviner d'avance quelles conceptions mathématiques seront, si on s'y attache, stériles ou fécondes. C'est encore du jeu, et presque du jeu de hasard. Très peu de savants pénètrent assez profondément dans la science pour avoir le cœur pris par de la beauté. Il y a un mathématicien qui compare volontiers la mathématique à une sculpture dans une pierre particulièrement dure. Des gens qui se donnent au public comme des prêtres de la vérité dégradent singulièrement le rôle qu'ils assument en se comparant à des joueurs d'échecs ; la comparaison avec un sculpteur est plus honorable. Mais si l'on a la vocation d'être sculpteur, il vaut mieux être sculpteur que mathématicien. En l'examinant de près, cette comparaison, dans la conception actuelle de la science, n'a pas de sens. Elle est un pressentiment très confus d'une autre conception.\par
La technique est pour une si grande part dans le prestige de la science qu'on inclinerait à supposer que la pensée des applications est un stimulant puissant pour les savants. En fait, ce qui est un stimulant, ce n'est pas la pensée des applications, c'est le prestige même que les applications donnent à la science. Comme les hommes politiques qui sont enivrés de faire de l'histoire, les savants sont enivrés de se sentir dans une grande chose. Grande au sens de la fausse grandeur ; une grandeur indépendante de toute considération du bien.\par
En même temps certains d'entre eux, ceux dont les recherches sont surtout théoriques, tout en goûtant cette ivresse, sont fiers de se dire indifférents aux applications techniques. Ils jouissent ainsi de deux avantages en réalité incompatibles, mais compatibles dans l'illusion ; ce qui est toujours une situation extrêmement agréable. Ils sont au nombre de ceux qui font le destin des hommes, et dès lors leur indifférence à ce destin réduit l'humanité aux proportions d'une race de fourmis ; c'est une situation de dieux. Ils ne se rendent pas compte que dans la conception actuelle de la science, si l'on retranche les applications techniques, il ne reste plus rien qui soit susceptible d'être regardé comme un bien. L'habileté à un jeu analogue aux échecs est une chose de valeur nulle. Sans la technique, personne aujourd'hui dans le public ne s'intéresserait à la science ; et si le public ne s'intéressait pas à la science, ceux qui suivent une carrière scientifique en auraient choisi une autre. Ils n'ont pas droit à l'attitude de détachement qu'ils assument. Mais, quoiqu'elle ne soit pas légitime, elle est un stimulant.\par
Pour d'autres, la pensée des applications au contraire sert de stimulant. Mais ils ne sont sensibles qu'à l'importance, non au bien et au mal. Un savant qui se sent sur le point de faire une découverte susceptible de bouleverser la vie humaine tend toutes ses forces pour y parvenir. Il n'arrive guère ou jamais, semble-t-il, qu'il s'arrête pour supputer les effets probables du bouleversement en bien et en mal, et renonce à ses recherches si le mal paraît plus probable. Un tel héroïsme semble même impossible ; il devrait pourtant aller de soi. Mais là comme ailleurs la fausse grandeur domine, celle qui se définit par la quantité et non par le bien.\par
Enfin les savants sont perpétuellement piqués par des mobiles sociaux qui sont presque inavouables tant ils sont mesquins, et ne jouent pas un grand rôle apparent, mais qui sont extrêmement forts. Qui a vu les Français, en juin 1940, abandonner si facilement la patrie, et quelques mois plus tard, avant d'être réellement mordus par la faim, faire des prodiges d'endurance, braver la fatigue et le froid pendant des heures, pour se procurer un oeuf, celui-là ne peut pas ignorer l'incroyable énergie des mobiles mesquins.\par
Le premier mobile social des savants, c'est purement et simplement le devoir professionnel. Les savants sont des gens qu'on paie pour fabriquer de la science ; on attend d'eux qu'ils en fabriquent ; ils se sentent obligés d'en fabriquer. Mais c'est insuffisant comme excitant. L'avancement, les chaires, les récompenses de toute espèce, honneurs et argent, les réceptions à l'étranger, l'estime ou l'admiration des collègues, la réputation, la célébrité, les titres, tout cela compte pour beaucoup.\par
Les mœurs des savants en sont la meilleure preuve. Aux XV\textsuperscript{e} et XVII\textsuperscript{e} siècles, les savants se lançaient des défis. Quand ils publiaient leurs découvertes, ils omettaient exprès des chaînons dans l'enchaînement des preuves, ou bien ils en bouleversaient l'ordre, pour empêcher leurs collègues de comprendre tout à fait ; ils se garantissaient ainsi du danger qu'un rival pût prétendre avoir fait la même découverte avant eux. Descartes lui-même avoue avoir fait cela dans sa {\itshape Géométrie}. Cela prouve qu'il n'était pas un philosophe au sens qu'avait le mot pour Pythagore et Platon, un amant de la Sagesse divine ; depuis la disparition de la Grèce il n'y a pas eu de philosophe.\par
Aujourd'hui, dès qu'un savant a trouvé quelque chose, avant même d'en avoir mûri et éprouvé la valeur, il se précipite pour envoyer ce qu'on appelle une « note au compte rendu » afin de s'assurer la priorité. Un cas comme celui de Gauss est peut-être unique dans notre science ; il oubliait dans des fonds de tiroirs des manuscrits enfermant les découvertes les plus merveilleuses, puis, quand quelqu'un mettait au jour quelque chose de sensationnel, il faisait remarquer négligemment : « Tout cela est exact, je l'avais trouvé il y a quinze ans ; mais on peut aller beaucoup plus loin dans ce sens et poser encore tel, tel et tel théorème. » Mais aussi c'était un génie de tout premier ordre. Peut-être y en a-t-il eu quelques-uns comme cela, une minuscule poignée, au cours des trois ou quatre derniers siècles ; ce qu'a signifié la science pour eux est resté leur secret. Les stimulants inférieurs tiennent une très grande place dans l'effort quotidien de tous les autres.\par
Aujourd'hui la facilité des communications à travers le monde en temps de paix et une spécialisation poussée à l'extrême ont pour effet que les savants de chaque spécialité, qui constituent les uns pour les autres l'unique public, forment l'équivalent d'un village. Les cancans y circulent continuellement ; chacun connaît chaque autre, a pour chaque autre de la sympathie ou de l'antipathie. Les générations et les nationalités s'y heurtent ; la vie privée, la politique, les rivalités de carrière y tiennent une grande place. Dès lors l'opinion collective de ce village est viciée par nécessité ; or elle constitue l'unique contrôle du savant, car ni les profanes, ni les savants des autres spécialités ne prennent aucune connaissance de ses travaux. La force des stimulants sociaux soumet la pensée du savant à cette opinion collective ; il cherche à lui plaire. Ce qu'elle consent à admettre est admis dans la science ; ce qu'elle n'admet pas en est exclu. Il n'existe aucun juge désintéressé, puisque chaque spécialiste, du fait même qu'il est spécialiste, est un juge intéressé.\par
\par
On dira que la fécondité d'une théorie est un critère objectif. Mais ce critère joue seulement parmi celles qui sont admises. Une théorie refusée par l'opinion collective du village des savants est forcément stérile, parce qu'on ne cherche pas à en tirer des développements. C'est surtout le cas pour la physique, où les moyens mêmes de recherche et de contrôle sont un monopole aux mains d'un milieu très fermé. Si les gens ne s'étaient pas engoués pour la théorie des quanta quand Planck la lança pour la première fois, et cela quoiqu'elle fût absurde – ou peut-être parce qu'elle l'était, car on était fatigué de la raison –, on n'aurait jamais su qu'elle était féconde. Au moment où l'on s'est engoué d'elle, on ne possédait aucune donnée permettant de prévoir qu'elle le serait. Ainsi il a y un processus darwinien dans la science. Les théories poussent comme au hasard, et il y a survivance des plus aptes. Une telle science peut être une forme de l'élan vital, mais non pas une forme de la recherche de la vérité.\par
Le grand public même ne peut pas ignorer, et n’ignore pas, que la science, comme tout produit d'une opinion collective, est soumise à la mode. Les savants lui parlent assez souvent de théories démodées. Ce serait un scandale, si nous n'étions pas trop abrutis pour être sensibles à aucun scandale. Comment peut-on porter un respect religieux à une chose soumise à la mode ? Les nègres fétichistes nous sont bien supérieurs ; ils sont infiniment moins idolâtres que nous. Ils portent un respect religieux à un morceau de bois sculpté qui est beau, et auquel la beauté confère un caractère d'éternité.\par
Nous souffrons réellement de la maladie d'idolâtrie ; elle est si profonde qu'elle ôte aux chrétiens la faculté du témoignage pour la vérité. Aucun dialogue de sourds ne peut approcher en force comique le débat de l'esprit moderne et de l'Église. Les incroyants choisissent pour en faire des arguments contre la foi chrétienne, au nom de l'esprit scientifique, des vérités qui constituent indirectement ou même directement des preuves manifestes de la foi. Les chrétiens ne s'en aperçoivent jamais, et ils s'efforcent faiblement, avec une mauvaise conscience, avec un manque affligeant de probité intellectuelle, de nier ces vérités. Leur aveuglement est le châtiment du crime d'idolâtrie.\par
Non moins comique est l'embarras des adorateurs de l'idole quand ils souhaitent exprimer leur enthousiasme. Ils cherchent quoi louer, et ne trouvent pas. Il est facile de louer les applications ; seulement les applications, c'est la technique, ce n'est pas la science. Que louer dans la science elle-même ? Et plus précisément, étant donné que la science réside dans des hommes, que louer dans les savants ? Ce n'est pas facile à discerner. Quand on veut proposer un savant à l'admiration du public, on choisit toujours Pasteur, du moins en France. Il sert de couverture à l'idolâtrie de la science comme Jeanne d'Arc à l'idolâtrie nationaliste.\par
On le choisit parce qu'il a fait beaucoup pour soulager les maux physiques des hommes. Mais si l'intention d'y réussir n'a pas été le mobile dominant de ses efforts, il faut regarder le fait qu'il y a réussi comme une simple coïncidence. Si ce fut là le mobile dominant, l'admiration qu'on lui doit n'a pas de relation avec la grandeur de la science ; il s'agit d'une vertu pratique ; Pasteur serait à ranger en ce cas dans la même catégorie qu'une infirmière dévouée jusqu'à l'héroïsme, et ne différerait d'elle que par l'étendue des résultats.\par
L'esprit de vérité, étant absent des mobiles de la science, ne peut pas être présent dans la science. Si l'on comptait le trouver en revanche à un degré élevé dans la philosophie et les lettres, on serait déçu.\par
Y a-t-il beaucoup de livres ou d'articles qui donnent l'impression que l'auteur, d'abord avant de commencer à écrire, puis avant de livrer la copie à l'impression, s'est demandé avec une réelle anxiété : « Est-ce que je suis dans la vérité ? » Y a-t-il beaucoup de lecteurs qui, avant d'ouvrir un livre, se demandent avec une réelle anxiété : « Est-ce que je vais trouver là de la vérité ? » Si l'on proposait à tous ceux qui ont pour profession de penser, prêtres, pasteurs, philosophes, écrivains, savants, professeurs de toute espèce, le choix, à partir de l'instant présent, entre deux destinées : ou sombrer immédiatement et définitivement dans l'idiotie, au sens littéral, avec toutes les humiliations qu'un tel effondrement entraîne, et en gardant seulement assez de lucidité pour en éprouver toute l'amertume ; ou un développement soudain et prodigieux des facultés intellectuelles, qui leur assure une célébrité mondiale immédiate et la gloire après leur mort pendant des millénaires, mais avec cet inconvénient que leur pensée séjournerait toujours un peu en dehors de la vérité ; peut-on croire que beaucoup d'entre eux éprouveraient pour un tel choix même une légère hésitation ?\par
L'esprit de vérité est aujourd'hui presque absent et de la religion et de la science et de toute la pensée. Les maux atroces au milieu desquels nous nous débattons, sans parvenir même à en éprouver tout le tragique, viennent entièrement de là. « Cet esprit de mensonge et d'erreur, – De la chute des rois funeste avant-coureur », dont parlait Racine, n'est plus aujourd'hui le monopole des souverains. Il s'étend à toutes les classes de la population ; il saisit des nations entières et les met dans la frénésie.\par
Le remède est de faire redescendre l'esprit de vérité parmi nous ; et d'abord dans la religion et la science ; ce qui implique qu'elles se réconcilient.\par
L'esprit de vérité peut résider dans la science à la condition que le mobile du savant soit l'amour de l'objet qui est la matière de son étude. Cet objet, c'est l'univers dans lequel nous vivons. Que peut-on aimer en lui, sinon sa beauté ? La vraie définition de la science, c'est qu'elle est l'étude de la beauté du monde.\par
Dès qu'on y pense, c'est évident. La matière, la force aveugle ne sont pas l'objet de la science. La pensée ne peut les atteindre ; elles fuient devant elle. La pensée du savant n'atteint jamais que des relations qui saisissent matière et force dans un réseau invisible, impalpable et inaltérable d'ordre et d'harmonie. « Le filet du ciel est vaste, dit Lao-Tseu ; ses mailles sont larges ; pourtant rien ne passe au travers. »\par
\par
Comment la pensée humaine aurait-elle pour objet autre chose que de la pensée ? C'est là une difficulté tellement connue dans la théorie de la connaissance qu'on renonce à la considérer, on la laisse de côté comme un lieu commun. Mais il y a une réponse. C'est que l'objet de la pensée humaine est, lui aussi, de la pensée. Le savant a pour fin l'union de son propre esprit avec la sagesse mystérieuse éternellement inscrite dans l'univers. Dès lors comment y aurait-il opposition ou même séparation entre l'esprit de la science et celui de la religion ? L'investigation scientifique n'est qu'une forme de la contemplation religieuse.\par
C'était bien le cas en Grèce. Que s'est-il donc passé de-puis ? Comment se fait-il que cette science, qui, quand l'épée romaine la fit tomber en défaillance, avait l'esprit religieux pour essence, se soit éveillée matérialiste au sortir de sa longue léthargie ? Quel événement était survenu dans l'intervalle ?\par
Il s'était produit une transformation dans la religion. Il ne s'agit pas de l'avènement du christianisme. Le christianisme originel, tel qu'il se, trouve encore présent pour nous dans le Nouveau Testament, et surtout dans les Évangiles, était, comme la religion antique des Mystères, parfaitement apte à être l'inspiration centrale d'une science parfaitement rigoureuse. Mais le christianisme a subi une transformation, probablement liée à son passage au rang de religion romaine officielle.\par
Après cette transformation, la pensée chrétienne, excepté quelques rares mystiques toujours exposés au danger d'être condamnés, n'admit plus d'autre notion de la Providence divine que celle d'une Providence personnelle.\par
Cette notion se trouve dans l'Évangile, car Dieu y est nommé le Père. Mais la notion dune Providence impersonnelle, et en un sens presque analogue à un mécanisme, s'y trouve aussi. « Devenez les fils de votre Père, celui des cieux ; car il fait lever le soleil sur les méchants et les bons, et fait tomber la pluie sur les justes et les injustes... Soyez donc parfaits comme votre Père céleste est parfait. » (Matth., 5, 45.)\par
Ainsi c'est l'impartialité aveugle de la matière inerte, c'est cette régularité impitoyable de l'ordre du monde, absolument indifférente à la qualité des hommes, et de ce fait si souvent accusée d'injustice – c'est cela qui est proposé comme modèle de perfection à l'âme humaine. C'est une pensée d'une profondeur telle que nous ne sommes pas même aujourd'hui capables de la saisir ; le christianisme contemporain l'a tout à fait perdue.\par
Toutes les paraboles sur la semence répondent à la notion d'une providence impersonnelle. La grâce tombe de chez Dieu dans tous les êtres – ce qu'elle y devient dépend de ce qu'ils sont ; là où elle pénètre réellement, les fruits qu'elle porte sont l'effet d'un processus analogue à un mécanisme, et qui, comme un mécanisme, a lieu dans la durée. La vertu de patience, ou pour traduire plus exactement le mot grec, d'attente immobile, est relative à cette nécessité de la durée.\par
La non-intervention de Dieu dans l'opération de la grâce est exprimée aussi clairement que possible : « Le royaume de Dieu, c'est comme si un homme jette du grain dans la terre, puis dort et veille la nuit et le jour, et le grain germe et pousse sans qu'il sache comment. Automatiquement la terre porte le fruit ; d'abord la tige, puis l'épi, puis la plénitude du grain dans l'épi. » (Marc, 4, 26.)\par
Tout ce qui concerne la demande évoque aussi quelque chose d'analogue à un mécanisme. Tout désir réel d'un bien pur, à partir d'un certain degré d'intensité, fait descendre le bien correspondant. Si l'effet ne se produit pas, le désir n'est pas réel ou il est trop faible, ou le bien désiré est imparfait, ou il est mélangé de mal. Quand les conditions sont remplies, Dieu ne refuse jamais. Comme la germination de la grâce, c'est un processus qui s'accomplit dans la durée. C'est pourquoi le Christ nous prescrit d'être importuns. Les comparaisons dont il use sur ce point évoquent, elles aussi, un mécanisme. C'est un mécanisme psychologique qui contraint le juge à satisfaire la veuve : « Je ferai justice à cette veuve parce qu'elle ne fait que me fatiguer » (Luc, 18, 5), et l'homme endormi à ouvrir à son ami : « S'il ne se lève pas par amitié pour lui, il se lèvera à cause de son impudence. » (Luc, 11, 8.) Si nous exerçons une espèce de contrainte sur Dieu, il ne peut s'agir que d'un mécanisme institué par Dieu. Les mécanismes surnaturels sont au moins aussi rigoureux que la loi de la chute des corps ; mais les mécanismes naturels sont les conditions de la production des événements comme tels, sans égard à aucune considération de valeur ; et les mécanismes surnaturels sont les conditions de la production du bien pur comme tel.\par
C'est ce qui est confirmé par l'expérience pratique des saints. Ils ont constaté, dit-on, qu'ils pouvaient parfois, à force de désir, faire descendre sur une âme plus de bien qu'elle-même n'en désirait. Cela confirme que le bien descend du ciel sur la terre seulement dans la proportion où certaines conditions sont en fait réalisées sur terre.\par
L'œuvre entière de saint Jean de la Croix n'est qu'une étude rigoureusement scientifique des mécanismes surnaturels. La philosophie de Platon aussi n'est pas autre chose.\par
Même le jugement, dans l'Évangile, apparaît comme quelque chose d'impersonnel : « Celui qui croit en lui n'est pas jugé ; celui qui ne croit pas est déjà jugé. Ceci est le jugement : ... quiconque fait des choses médiocres hait la lumière ; ... celui qui fait la vérité vient vers la lumière. » (Jean, 3, 18.) « Comme j'entends, je juge et mon jugement est juste. » (Jean, 5, 30.) « Si quelqu'un entend mes paroles et ne les garde pas, je ne le juge pas, car je ne suis pas venu pour juger le monde, mais pour sauver le monde. Celui qui me refuse et ne reçoit pas mes paroles, il a un juge ; la parole que j'ai parlée, c'est elle qui le jugera au dernier jour. »\par
Dans l'histoire des ouvriers de la onzième heure, il semble y avoir caprice de la part du maître de la vigne. Mais si l'on fait un peu attention, c'est le contraire. Il ne paie qu'un seul salaire parce qu'il ne possède qu'un seul salaire. Il n'a pas de monnaie. Saint Paul définit le salaire : « Je connaîtrai comme je suis connu. » Cela ne comporte pas de degrés. De même il n'y a pas de degrés dans l'acte qui fait mériter le salaire. On est appelé ; on accourt ou on n'accourt pas. Il n'est au pouvoir de personne de devancer l'appel, même d'une seconde. Le moment ne compte pas ; il n'est pas tenu compte non plus de la quantité ni de la qualité du travail dans la vigne. On passe ou non du temps dans l'éternité selon qu'on a consenti ou refusé.\par
« Quiconque s'élève soi-même sera abaissé, quiconque s'abaisse soi-même sera élevé » ; cela évoque une balance, comme si la partie terrestre de l'âme était dans un plateau, et la partie divine dans l'autre. Un hymne du Vendredi Saint compare aussi la Croix à une balance. « ... Ceux-là ont reçu leur récompense. » Dieu n'a donc le pouvoir de récompenser que les efforts qui sont sans récompense ici-bas, les efforts accomplis à vide ; le vide attire la grâce. Les efforts à vide constituent l'opération que le Christ appelle « amasser des trésors dans le ciel ».\par
On pourrait trouver dans les Évangiles, quoiqu'ils ne nous aient transmis qu'une faible partie des enseignements du Christ, ce qu'on pourrait nommer une physique surnaturelle de l'âme humaine. Comme toute doctrine scientifique, elle ne contient que des choses clairement intelligibles et expérimentalement vérifiables. Seulement la vérification est constituée par la marche vers la perfection, et par suite il faut croire sur parole ceux qui l'ont accomplie. Mais nous croyons bien sur parole et sans contrôle ce que nous disent les savants de ce qui se passe dans leurs laboratoires, bien que nous ignorions s'ils aiment la vérité. Il serait plus juste de croire sur parole les saints, du moins ceux qui sont authentiques, car il est certain qu'ils aiment parfaitement la vérité.\par
Le problème des miracles ne fait difficulté entre la religion et la science que parce qu'il est mal posé. Il faudrait pour bien le poser définir le miracle. En disant que c'est un fait contraire aux lois de la nature on dit une chose absolument dénuée de signification. Nous ne connaissons pas les lois de la nature. Nous ne pouvons faire à leur sujet que des suppositions. Si celles que nous supposons sont contredites par des faits, c'est que notre supposition était au moins partiellement erronée. Dire qu'un miracle est l'effet d'un vouloir particulier de Dieu n'est pas moins absurde. Parmi les événements qui se produisent, nous n'avons aucune raison d'affirmer que certains plus que d'autres procèdent du vouloir de Dieu. Nous savons seulement, d'une manière générale, que tout ce qui se produit, sans aucune exception, est conforme à la volonté de Dieu en tant que Créateur ; et que tout ce qui enferme au moins une parcelle de bien pur procède de l'inspiration surnaturelle de Dieu en tant que bien absolu. Mais quand un saint fait un miracle, ce qui est bien, c'est la sainteté, non le miracle.\par
Un miracle est un phénomène physique parmi les conditions préalables duquel se trouve un abandon total de l'âme soit au bien, soit au mal.\par
\par
Il faut dire soit au bien, soit au mal, car il y a des miracles diaboliques. « Il surgira des pseudo-christs et des pseudo-prophètes qui feront des signes et des prodiges capables de faire errer, si c'était possible, même les élus. » (Marc, 13, 22.) « Beaucoup me diront en ce jour : Seigneur, Seigneur, est-ce qu'en ton nom nous n'avons pas prophétisé, et en ton nom chassé les démons, et en ton nom fait beaucoup de miracles ? Et alors je leur déclarerai : Je ne vous ai jamais connus ; allez-vous-en loin de moi, vous qui avez opéré des choses illégitimes. » (Matth., 3, 22.)\par
Il n'est nullement contraire aux lois de la nature qu'à un abandon total de l'âme au bien ou au mal correspondent des phénomènes physiques qui ne se produisent que dans ce cas. Il serait contraire aux lois de la nature qu'il en fût autrement. Car à chaque manière d'être de l'âme humaine correspond quelque chose de physique. À la tristesse correspond de l'eau salée dans les yeux ; pourquoi pas à certains états d'extase mystique, comme on raconte, un certain soulèvement du corps au-dessus du sol ? Le fait est exact ou non ; peu importe. Ce qui est certain, c'est que, si l'extase mystique est quelque chose de réel dans l'âme, il doit y correspondre dans le corps des phénomènes qui n'apparaissent pas quand l'âme est dans un autre état. La liaison entre l'extase mystique et ces phénomènes est constituée par un mécanisme analogue à celui qui lie la tristesse et les larmes. Nous ne savons rien du premier mécanisme. Mais nous ne savons pas davantage du second.\par
L'unique fait surnaturel ici-bas, c'est la sainteté elle-même et ce qui en approche ; c'est le fait que les commandements divins deviennent chez ceux qui aiment Dieu un mobile, une force agissante, une énergie motrice, au sens littéral, comme l'essence dans une automobile. Si trois pas sont accomplis sans aucun autre mobile que le désir d'obéir à Dieu, ces trois pas sont miraculeux ; ils le sont également, qu'ils soient exécutés sur le sol ou sur l'eau. Seulement s'ils sont exécutés sur le sol rien d'extraordinaire n'apparaît.\par
On dit que les histoires de marche sur les eaux et de résurrection de morts sont fréquentes en Inde au point que personne, sauf les badauds, ne se dérangerait pour voir un fait de ce genre. Il est certain en tout cas que les récits sur ces thèmes y sont très fréquents. Ils étaient très fréquents aussi dans la Grèce de la basse époque, comme on peut voir dans Lucien. Cela diminue singulièrement la valeur apologétique des miracles pour le christianisme.\par
Une anecdote hindoue raconte qu'un ascète, après quatorze ans de solitude, revint voir sa famille. Son frère lui demanda ce qu'il avait acquis. Il l'emmena jusqu'à un fleuve et le traversa à pied sous ses yeux. Le frère héla le passeur, traversa en barque, paya un sou, et dit à l'ascète : « Cela vaut-il la peine d'avoir fait quatorze ans d'efforts pour acquérir ce que je peux me procurer pour un sou ? » C'est l'attitude du bon sens.\par
Sur l'exactitude des faits extraordinaires racontés dans l'Évangile, on ne peut rien affirmer ou nier qu'au hasard, et le problème est sans intérêt. Il est certain que le Christ possédait certains pouvoirs particuliers ; comment en douterions-nous, puisque nous pouvons vérifier que des saints hindous ou thibétains en possèdent ? Savoir quel est le degré d'exactitude de chaque anecdote particulière ne nous serait pas utile.\par
Les pouvoirs exercés par le Christ constituaient, non une preuve, mais un chaînon dans l'enchaînement d'une démonstration. Ils étaient le signe certain que le Christ était situé hors de l'humanité ordinaire, parmi ceux qui se sont donnés ou au mal ou au bien. Ils n'indiquaient pas lequel des deux. Mais la discrimination était facile à faire par la perfection manifeste du Christ, la pureté de sa vie, la parfaite beauté de ses paroles, et le fait qu'il exerçait ses pouvoirs seulement pour des actes de compassion. Il résultait de là seulement qu'il était un saint. Mais ceux qui étaient certains qu'il était un saint, quand ils l'entendaient affirmer qu'il était le fils de Dieu, pouvaient hésiter sur le sens de ces paroles, mais étaient tenus de croire qu'elles enfermaient une vérité. Car un saint, quand il dit de telles choses, ne peut ni mentir ni se tromper. Nous de même, nous sommes tenus de croire tout ce qu'a dit le Christ, sauf là où nous pouvons supposer une mauvaise transcription ; et ce qui fait la force de la preuve, c'est la beauté. Quand ce qui est en question est le bien, la beauté est une preuve rigoureuse et certaine ; et même il ne peut y en avoir aucune autre. Il est absolument impossible qu'il y en ait aucune autre.\par
Le Christ a dit : « Si je n'avais pas fait parmi eux des actes que nul autre n'a faits, ils n'auraient pas de faute », mais il a dit aussi: « Si je n'étais pas venu leur parler, ils n'auraient pas de faute. » Ailleurs il parle de ses « belles actions ». Les actes et les paroles sont mis ensemble. Le caractère exceptionnel des actes n'avait pour fin que d'attirer l'attention. Une fois l'attention attirée, il ne peut y avoir d'autre preuve que la beauté, la pureté, la perfection.\par
La parole adressée à Thomas : « Heureux ceux qui croient sans avoir vu » ne peut pas porter sur ceux qui, sans l'avoir vu, croient le fait de la résurrection. Ce serait un éloge de la crédulité, non de la foi. Il y a partout des vieilles femmes qui ne demandent qu'à croire indifféremment toutes les histoires de morts ressuscités. Sûrement ceux qui sont dits heureux sont ceux qui n'ont pas besoin de la résurrection pour croire, et pour qui la perfection et la Croix sont des preuves.\par
Ainsi du point de vue religieux les miracles sont chose secondaire, et du point de vue scientifique ils entrent naturellement dans la conception scientifique du monde. Quant à l'idée de prouver Dieu par la violation des lois de la nature, elle aurait sans doute paru monstrueuse, aux premiers chrétiens. Elle ne pouvait surgir que dans nos esprits malades, qui croient que la fixité de l'ordre du monde peut fournir des arguments légitimes aux athées.\par
La succession des événements du monde apparaît, elle aussi, dans l'Évangile, comme réglée par une Providence en un sens au moins impersonnelle et analogue à un mécanisme. Le Christ dit à ses disciples : « Voyez les oiseaux du ciel qui ne sèment ni ne moissonnent ni n'amassent dans les greniers, et votre Père céleste les nourrit... Voyez les lis des champs, comme ils croissent ; ils ne travaillent ni ne filent, et je vous dis que Salomon dans tout son éclat n'a pas été vêtu comme l'un d'eux... Est-ce que deux moineaux ne se vendent pas un sou ? Et il n'en tombe pas un sur la terre sans votre Père. » Cela signifie que la sollicitude dont les saints sont l'objet de la part de Dieu est de la même espèce que celle qui enveloppe les oiseaux et les lis. Les lois de la nature règlent la manière dont la sève monte dans les plantes et s'épanouit en fleurs, dont les oiseaux trouvent la nourriture ; et elles sont disposées de telle sorte qu'il se produit de la beauté. Les lois de la nature sont aussi disposées providentiellement de telle sorte que, parmi les créatures humaines, la résolution de rechercher premièrement le royaume et la justice du Père céleste n'entraîne pas automatiquement la mort.\par
On peut aussi dire, si l'on veut, que Dieu veille sur chaque oiseau, chaque fleur et chaque saint ; cela revient au même. La relation du tout aux parties est propre à l'intelligence humaine. Sur le plan des événements en tant que tels, soit que l'on considère l'univers comme un tout, ou l'une quelconque de ses parties, découpée comme on voudra dans l'espace, dans le temps, dans n'importe quelle classification ; ou une autre partie, ou une autre, ou une collection de parties ; bref qu'on use des notions de tout et de partie comme on voudra, la conformité à la volonté de Dieu reste invariable. Il y a autant de conformité à la volonté de Dieu dans une feuille qui tombe sans être vue que dans le déluge. Sur le plan des événements, la notion de conformité à la volonté de Dieu est identique à la notion de réalité.\par
Sur le plan du bien et du mal, il peut y avoir conformité ou non-conformité à la volonté de Dieu selon la relation au bien et au mal. La foi dans la Providence consiste à être certain que l'univers dans sa totalité est conforme à la volonté de Dieu non seulement au premier sens, mais aussi au second ; c'est-à-dire que dans cet univers le bien l'emporte sur le mal. Il ne peut s'agir là que de l'univers dans sa totalité, car dans les choses particulières nous ne pouvons malheureusement pas douter qu'il y ait du mal. Ainsi l'objet de cette certitude est une disposition éternelle et universelle constituant le fondement de l'ordre invariable du monde. La Providence divine n'apparaît jamais autrement, sauf erreur, ni dans les textes sacrés de la Chine, de l'Inde et de la Grèce, ni dans les Évangiles.\par
Mais quand la religion chrétienne fut officiellement adoptée par l'Empire romain, on mit dans l'ombre l'aspect impersonnel de Dieu et de la Providence divine. On fit de Dieu une doublure de l'Empereur. L'opération fut rendue facile par le courant judaïque dont le christianisme, du fait de son origine historique, n'avait pu se purifier. Jéhovah, dans les textes antérieurs à l'exil, a avec les Hébreux la relation juridique d'un maître avec des esclaves. Ils étaient esclaves du Pharaon ; Jéhovah, les ayant tirés des mains du Pharaon, a succédé à ses droits. Ils sont sa propriété, et il les domine comme n'importe quel homme domine ses esclaves, sauf qu'il dispose d'un choix plus large de récompenses et de châtiments. Il leur commande indifféremment le bien ou le mal, mais beaucoup plus souvent le mal, et dans les deux cas ils n'ont qu'à obéir. Il importe peu qu'ils soient maintenus dans l'obéissance par les mobiles les plus vils, pourvu que les ordres soient exécutés.\par
Une telle conception était précisément à la hauteur du cœur et de l'intelligence des Romains. Chez eux l'esclavage avait pénétré et dégradé toutes les relations humaines. Ils ont avili les plus belles choses. Ils ont déshonoré les suppliants en les forçant à mentir. Ils ont déshonoré la gratitude en la regardant comme un esclavage atténué ; dans leur conception, en recevant un bienfait, on aliénait en échange une partie de sa liberté. Si le bienfait était important, les mœurs courantes contraignaient à dire au bienfaiteur qu'on était son esclave. Ils ont déshonoré l'amour ; être amoureux, pour eux, c'était ou bien acquérir la personne aimée comme propriété, ou bien, si on ne le pouvait pas, se soumettre servilement à elle pour en obtenir des plaisirs charnels, dût-on accepter le partage avec dix autres. Ils ont déshonoré la patrie en concevant le patriotisme comme la volonté de réduire en esclavage tous les hommes qui ne sont pas des compatriotes. Mais il serait plus court d'énumérer ce qu'ils n'ont pas déshonoré. On ne trouverait probablement rien.\par
Entre autres choses ils ont déshonoré la souveraineté. La notion antique de souveraineté légitime, autant qu'on peut la deviner, semble avoir été extrêmement belle. On ne peut que la deviner, car elle n'existait pas chez les Grecs. Mais c'est probablement elle qui a survécu en Espagne jusqu'au XVII\textsuperscript{e} siècle, et, à un degré beaucoup plus faible, en Angleterre jusqu'à nos jours.\par
Le Cid, après un exil brutal et injuste, après avoir conquis tout seul des terres plus étendues que le royaume où il est né, obtient la faveur d'une entrevue avec le roi ; et du plus loin qu'il l'aperçoit il descend de cheval, se jette au sol sur les mains et les genoux et baise la terre. Dans l'{\itshape Étoile de Séville} de Lope de Vega, le roi veut empêcher la condamnation à mort d'un assassin, parce que le meurtre avait été commandé en secret par lui-même ; il convoque séparément chacun des trois juges pour lui signifier ses ordres ; chacun, agenouillé, proteste de sa soumission totale. Après quoi, aussitôt réunis en tribunal, ils condamnent à mort à l'unanimité. Au roi qui exige des explications, ils répondent : « Comme sujets nous te sommes soumis en toutes choses, mais comme juges nous n'obéissons qu'à notre conscience. »\par
Cette conception est celle d'une soumission inconditionnée, totale, mais accordée uniquement à la légitimité, sans aucun égard ni à la puissance ni aux possibilités de prospérité ou de malheur, de récompense ou de châtiment. C'est exactement la même conception que celle de l'obéissance au supérieur dans les ordres monastiques. Un roi obéi ainsi était réellement une image de Dieu pour ses sujets, comme un prieur de couvent pour ses religieux, non par une illusion qui l'aurait fait paraître divin, mais uniquement par l'effet d'une convention qu'on croyait divinement ratifiée. C'était un respect religieux absolument pur d'idolâtrie. La même conception de seigneurie légitime était transposée, au-dessous du roi, du haut en bas de l'échelle sociale. La vie publique entière se trouvait ainsi imprégnée de la vertu religieuse d'obéissance, comme celle d'un couvent bénédictin de la bonne époque.\par
\par
Dans les époques connues de nous, on trouve cette conception parmi les Arabes, où T. E. Lawrence l'a encore observée ; en Espagne jusqu'au moment où ce malheureux pays dut subir le petit-fils de Louis XIV et perdit ainsi son âme ; dans les pays du sud de la Loire jusqu'à leur conquête par la France et même après, car cette inspiration est encore sensible chez Théophile de Viau. La royauté française a longtemps hésité entre cette conception et la conception romaine, mais elle a choisi la conception romaine, et c'est pour cela qu'il ne peut être question de la restaurer en France. Nous serions trop heureux s'il y avait pour nous une possibilité quelconque de royauté vraiment légitime.\par
Un certain nombre d'indications amènent à conclure que la conception espagnole de la royauté légitime était celle des monarchies orientales de l'Antiquité. Mais elle ne fut que trop souvent blessée. Les Assyriens lui firent beaucoup de mal. Alexandre aussi – ce produit de la pédagogie d'Aristote, et qui ne fut jamais désavoué par son maître. Les Hébreux, ces esclaves fugitifs, l'ont toujours ignorée. Les Romains sans doute aussi, poignée d'aventuriers réunis par le besoin.\par
Ce qui en tenait lieu à Rome, c'était la relation de maître à esclave. Cicéron déjà avouait avec honte qu'il se regardait comme étant à moitié l'esclave de César. À partir d'Auguste, l'Empereur fut regardé comme étant le maître de tous les habitants de l'Empire romain au sens d'un propriétaire d'esclaves.\par
Les hommes n'imaginent pas qu'on puisse leur infliger les malheurs qu'ils trouvent tout naturel d'infliger à autrui. Mais quand cela se produit en fait, à leur propre horreur, ils trouvent cela naturel ; ils ne trouvent au fond de leur cœur aucune ressource pour l'indignation et la résistance contre un traitement que leur cœur n'a jamais répugné à infliger. C'est ainsi du moins quand les circonstances sont telles que, même pour l'imagination, rien ne puisse servir de soutien extérieur, quand il ne peut y avoir aucune ressource que dans le secret du cœur. Si les crimes passés ont détruit ces ressources, la faiblesse est totale et on accepte n'importe quel degré de honte. C'est sur ce mécanisme du cœur humain que repose la loi de réciprocité exprimée dans l'Apocalypse par la formule : « Si quelqu'un traîne dans l'esclavage, il sera traîné dans l'esclavage. »\par
C'est ainsi que beaucoup de Français, ayant trouvé tout naturel de parler de collaboration aux indigènes opprimés des colonies françaises, ont continué à prononcer ce mot sans aucune peine en parlant à leurs maîtres allemands.\par
De même les Romains, regardant l'esclavage comme l'institution de base de la société, ne trouvaient rien dans leur cœur qui pût dire non à un homme qui affirmait avoir sur eux le droit d'un propriétaire, et avait victorieusement soutenu cette affirmation par les armes. Rien non plus qui pût dire non à ses héritiers, dont ils étaient la propriété par droit d'héritage. De là toutes les lâchetés dont l'énumération écœurait Tacite, d'autant plus qu'il y avait eu part. Ils se suicidaient dès qu'ils en recevaient l'ordre, non autrement ; un esclave ne se suicide pas, ce serait voler le maître. Caligula avait derrière lui, quand il mangeait, des sénateurs debout en tunique, ce qui était à Rome pour les esclaves la marque caractéristique de la dégradation. Aux banquets il s'absentait un quart d'heure pour emmener une femme noble dans son cabinet particulier, puis la ramenait rouge et décoiffée parmi les convives, au nombre desquels était son mari. Mais ces gens avaient toujours trouvé tout naturel de traiter ainsi non seulement leurs esclaves, mais les populations colonisées des provinces.\par
Ainsi dans le culte de l'Empereur, ce qui était divinisé, c'était l'institution de l'esclavage. Des millions d'esclaves rendaient un culte idolâtre à leur propriétaire.\par
C'est là ce qui a déterminé l'attitude des Romains en matière religieuse. On a dit qu'ils étaient tolérants. Ils toléraient en effet toutes les pratiques religieuses vides de contenu spirituel.\par
Il est probable qu'Hitler, s'il en avait la fantaisie, pourrait tolérer la théosophie sans danger. Les Romains pouvaient facilement tolérer le culte de Mithra, orientalisme truqué pour snobs et femmes oisives.\par
Il y avait deux exceptions à leur tolérance. D'abord ils ne pouvaient naturellement souffrir que qui que ce fût prétendît à un droit de propriété sur leurs esclaves. De là leur hostilité contre Jéhovah. Les Juifs étaient leur propriété et ne pouvaient pas avoir un autre propriétaire, humain ou divin. Il s'agissait simplement d'une contestation entre esclavagistes. Finalement les Romains, par souci de prestige et pour démontrer expérimentalement qu'ils étaient les maîtres, tuèrent presque dans sa totalité le bétail humain dont la propriété se trouvait contestée.\par
L'autre exception était relative à la vie spirituelle. Les Romains ne pouvaient rien tolérer qui fût riche en contenu spirituel. L'amour de Dieu est un feu dangereux dont le contact pouvait être funeste à leur misérable divinisation de l'esclavage. Aussi ont-ils impitoyablement détruit la vie spirituelle sous toutes ses formes. Ils ont très cruellement persécuté les Pythagoriciens et tous les philosophes affiliés à des traditions authentiques. Soit dit en passant, il est extrêmement mystérieux qu'une éclaircie ait permis une fois à un stoïcien véritable, d'inspiration grecque et non romaine, de monter sur le trône ; et le mystère est redoublé par le fait qu'il a maltraité les chrétiens. Ils ont exterminé tous les Druides de Gaule; anéanti les cultes égyptiens ; noyé dans le sang et déshonoré par d'ingénieuses calomnies l'adoration de Dionysos On sait ce qu'ils ont fait des chrétiens au début.\par
Pourtant ils se sentaient mal à l'aise dans leur idolâtrie trop grossière. Comme Hitler, ils connaissaient le prix d'une enveloppe illusoire de spiritualité. Ils auraient voulu prendre l'écorce extérieure d'une tradition religieuse authentique pour l'appliquer sur leur athéisme trop visible. Hitler aussi aimerait bien trouver ou fonder une religion.\par
Auguste fit une tentative auprès du clergé d'Éleusis. L'institution des mystères d'Éleusis se trouvait déjà dégradée presque jusqu'au néant, on ignore à la suite de quoi, au temps des successeurs d'Alexandre. Les massacres de Sylla, qui firent monter le sang dans les rues d'Athènes comme l'eau monte dans une inondation, ne durent leur faire aucun bien. Il est très douteux qu'au temps de l'Empire il subsistât aucune trace de la tradition authentique. Néanmoins les gens d'Éleusis se refusèrent à l'opération.\par
Les chrétiens y consentirent quand ils furent par trop las d'être massacrés, par trop malheureux de ne pas voir arriver la fin triomphale du monde. C'est ainsi que le Père du Christ, accommodé à la mode romaine, devint un maître et un propriétaire d'esclaves. Jéhovah fournissait la transition. Il n'y avait plus aucun inconvénient à l'accueillir. Il n'y avait plus contestation de propriété entre l'empereur romain et lui, depuis la destruction de Jérusalem.\par
L'Évangile, il est vrai, est plein de comparaisons tirées de l'esclavage. Mais dans la bouche du Christ ce mot est une ruse de l'amour. Les esclaves, ce sont les hommes qui ont voulu de tout leur cœur se donner à Dieu comme esclaves. Et, quoique ce soit là un don accompli en un instant et une fois pour toutes, dans la suite ces esclaves ne cessent pas une seconde de supplier Dieu de consentir à les maintenir dans l'esclavage.\par
Cela est incompatible avec la conception romaine. Si nous étions la propriété de Dieu, comment pourrions-nous nous donner à lui comme esclaves ? Il nous a affranchis du fait qu'il nous a créés. Nous sommes hors de son royaume. Notre consentement seul peut, avec le temps, accomplir l'opération inverse, et faire de nous-mêmes quelque chose d'inerte, quelque chose d'analogue au néant, où Dieu soit maître absolu.\par
L'inspiration vraiment chrétienne a été heureusement conservée par la mystique. Mais en dehors de la mystique pure, l'idolâtrie romaine a tout souillé. Idolâtrie, car c'est le mode de l'adoration, non pas le nom attribué à l'objet, qui sépare l'idolâtrie de la religion. Si un chrétien adore Dieu avec un cœur disposé comme le cœur d'un païen de Rome dans l'hommage rendu à l'empereur, ce chrétien aussi est idolâtre.\par
La conception romaine de Dieu subsiste encore aujourd'hui, jusque dans des esprits tels que Maritain.\par
Il a écrit : « La notion de droit est même plus profonde que celle d'obligation morale, car Dieu a un droit souverain sur les créatures et il n'a pas d'obligation morale envers elles (encore qu'il se doive à lui-même de leur donner ce qui est requis par leur nature). »\par
Ni la notion d'obligation ni celle de droit ne sauraient convenir à Dieu, mais celle de droit infiniment moins. Car la notion de droit est infiniment plus éloignée du bien pur. Elle est mélangée de bien et de mal ; car la possession d'un droit implique la possibilité d'en faire soit un bon, soit un mauvais usage. Au contraire l'accomplissement d'une obligation est toujours, inconditionnellement, un bien à tous égards. C'est pourquoi les gens de 1789 ont commis une erreur si désastreuse en choisissant comme principe de leur inspiration la notion de droit.\par
Un droit souverain, c'est le droit de propriété selon la conception romaine ou toute autre qui lui soit essentiellement identique. Attribuer à Dieu un droit souverain sans obligation, c'est en faire l'équivalent infini d'un propriétaire d'esclaves à Rome. Cela ne permet qu'un dévouement servile. Le dévouement d'un esclave pour l'homme qui le regarde comme sa propriété est chose basse. L'amour qui pousse un homme libre à abandonner son corps et son âme en servitude à ce qui constitue le bien parfait, c'est le contraire d'un amour servile.\par
Dans la tradition mystique de l'Église catholique, l'un des objets principaux des purifications à travers lesquelles l'âme doit passer est l'abolition totale de la conception romaine de Dieu. Tant qu'il en reste une trace, l'union d'amour est impossible.\par
Mais le rayonnement des mystiques a été impuissant à anéantir cette conception dans l'Église comme elle était anéantie dans leur âme, parce que l'Église avait besoin d'elle comme l'Empire en avait eu besoin. Elle en avait besoin pour sa domination temporelle. De sorte que la division du pouvoir en pouvoir spirituel et temporel, dont on parle si fréquemment à propos du Moyen Âge, est chose plus complexe qu'on ne pense. L'obéissance au roi selon la conception espagnole classique est une chose infiniment plus religieuse et plus pure que l'obéissance à une Église armée de l'Inquisition et proposant une conception esclavagiste de Dieu, comme ce fut dans une large mesure le cas au XIII\textsuperscript{e} siècle. Il se pourrait bien que, par exemple au XIII\textsuperscript{e} siècle en Aragon, le roi ait été détenteur d'une autorité réellement spirituelle, et l'Église d'une autorité réellement temporelle. Quoi qu'il en soit, l'esprit romain d'impérialisme et de domination n'a jamais suffisamment abandonné l'Église pour qu'elle abolît la conception romaine de Dieu.\par
Par contrecoup, la conception de la Providence est devenue méconnaissable. Elle est d'une absurdité criante au point d'étourdir la pensée. Les mystères authentiques de la foi sont, eux aussi, absurdes, mais d'une absurdité qui illumine la pensée et lui fait produire en abondance des vérités évidentes à l'intelligence. Les autres absurdités sont peut-être des mystères diaboliques. Les uns et les autres se trouvent mélangés dans la pensée chrétienne courante comme le blé et l'ivraie.\par
La conception de la Providence qui répond au Dieu du type romain, c'est une intervention personnelle de Dieu dans l'univers pour ajuster certains moyens en vue de fins particulières. On admet que l'ordre du monde, laissé à lui-même et sans intervention particulière de Dieu à tel lieu, en tel instant, pour telle fin, pourrait produire des effets non conformes au vouloir de Dieu. On admet que Dieu pratique les interventions particulières. Mais on admet que ces interventions, destinées à corriger le jeu de la causalité, sont elles-mêmes soumises à la causalité. Dieu viole l'ordre du monde pour y faire surgir, non ce qu'il veut produire, mais des causes qui amèneront ce qu'il veut produire à titre d'effet.\par
Si l'on y réfléchit, ces suppositions correspondent exactement à la situation de l'homme devant la matière. L'homme a des fins particulières qui l'obligent à des interventions particulières, lesquelles sont soumises à la loi de causalité. Qu'on imagine un grand propriétaire romain qui a de vastes domaines et de nombreux esclaves ; puis qu'on élargisse le domaine aux dimensions mêmes de l'univers. Telle est la conception de Dieu qui domine en fait une partie du christianisme, et dont la souillure contamine peut-être même plus ou moins le christianisme entier, excepté les mystiques.\par
Si l'on suppose un tel propriétaire vivant seul, sans jamais rencontrer d'égaux, sans aucune relation sinon avec ses esclaves, on se demande comment une fin particulière peut surgir dans sa pensée. Il n'a pas lui-même de besoins insatisfaits. Cherchera-t-il le bien de ses esclaves ? En ce cas il s'y prendrait bien mal, car en fait les esclaves sont en proie au crime et au malheur. Si l'on cherche à leur inspirer de bons sentiments en énumérant tout ce qu'il y a d'heureux dans leur sort – comme faisaient sans doute jadis les prédicateurs esclavagistes en Amérique – on ne rend que plus manifeste combien cette part de bien est limitée, combien il y a disproportion entre la puissance attribuée au maître et la part respective du bien et du mal. Comme on ne peut le dissimuler, on dira aux esclaves que s'ils sont malheureux, c'est par leur faute. Mais cette affirmation, si on l'accepte, n'apporte aucun éclaircissement au problème de savoir ce que peuvent être les volontés du propriétaire. Il est impossible de se représenter ces volontés autrement que comme des caprices dont certains sont bienveillants. En fait, on se les représente ainsi.\par
Toutes les tentatives pour déceler dans la structure de l'univers les marques de la bienveillance du propriétaire sont sans aucune exception du même niveau que la phrase de Bernardin de Saint-Pierre sur les melons et les repas en famille. Il y a dans ces tentatives la même absurdité centrale que dans les considérations historiques sur les effets de l'Incarnation. Le bien qu'il est donné à l'homme d'observer dans l'univers est fini, limité. Essayer d'y trouver une marque de l'action divine, c'est faire de Dieu lui-même un bien fini, limité. C'est un blasphème.\par
Les tentatives du même genre dans l'analyse de l'histoire peuvent être illustrées par une pensée ingénieuse exprimée dans une revue catholique de New York, lors du dernier anniversaire de la découverte de l'Amérique. Elle disait que Dieu avait envoyé Christophe Colomb en Amérique afin qu'il y eût quelques siècles plus tard un pays capable de vaincre Hitler. Cela est encore bien au-dessous de Bernardin de Saint-Pierre ; cela est atroce. Dieu apparemment méprise, lui aussi, les races de couleur ; l'extermination des populations d'Amérique au XVI\textsuperscript{e} siècle lui paraissait peu de chose au prix du salut des Européens du XX\textsuperscript{e} siècle ; et il ne pouvait pas leur amener le salut par des moyens moins sanglants. On croirait qu'au lieu d'envoyer Christophe Colomb en Amérique plus de quatre siècles à l'avance, il aurait été plus simple d'envoyer quelqu'un assassiner Hitler aux environs de 1923.\par
On aurait tort de penser que c'est là un degré exceptionnel de bêtise. Toute interprétation providentielle de l'histoire est par nécessité située exactement à ce niveau. C'est le cas pour la conception historique de Bossuet. Elle est à la fois atroce et stupide, également révoltante pour l'intelligence et pour le cœur. Il faut être bien sensible à la sonorité des mots pour regarder ce prélat courtisan comme un grand esprit.\par
Quand la notion de Providence est introduite dans la vie privée, le résultat n'est pas moins comique. Quand la foudre tombe à un centimètre de quelqu'un sans le toucher, il croit souvent avoir été préservé par la Providence. Ceux qui sont à un kilomètre de là ne pensent pas devoir la vie à une intervention de Dieu. Apparemment, quand le mécanisme de l'univers est sur le point de tuer un être humain, Dieu se demande s'il lui plaît ou non de lui sauver la vie, et s'il décide de le faire, il donne un coup de pouce presque imperceptible au mécanisme. Il peut bien déplacer la foudre d'un centimètre pour sauver une vie, mais non pas d'un kilomètre, encore moins l'empêcher purement et simplement de tomber. Il faut croire qu'on pense ainsi. Autrement on se dirait que la Providence intervient pour nous empêcher d'être tués par la foudre à tous les instants de notre vie, au même degré qu'à l'instant où la foudre tombe à un centimètre de nous. L'unique instant où elle n'intervienne pas pour empêcher que tel être humain soit tué par la foudre, c'est l'instant même où la foudre le tue, si du moins cela se produit. Tout ce qui n'arrive pas est empêché par Dieu au même degré. Tout ce qui arrive est permis par Dieu au même degré.\par
La conception absurde de la Providence comme intervention personnelle et particulière de Dieu à des fins particulières est incompatible avec la vraie foi. Mais ce n'est pas une incompatibilité évidente. Elle est incompatible avec la conception scientifique du monde ; et là l'incompatibilité est évidente. Les chrétiens qui, sous l'influence de l'éducation et du milieu, ont en eux cette conception de la Providence ont aussi la conception scientifique du monde, et cela sépare leur esprit en deux compartiments entre lesquels se trouve une cloison étanche ; l'un pour la conception scientifique du monde, l'autre pour la conception du monde comme domaine où agit la Providence personnelle de Dieu. De ce fait ils ne peuvent penser ni l'une ni l'autre. La seconde d'ailleurs n'est pas pensable. Les incroyants, n'étant arrêtés par aucun respect, discernent facilement que cette Providence personnelle et particulière est ridicule, et la foi elle-même est de ce fait, à leurs yeux, frappée de ridicule.\par
Les desseins particuliers qu'on attribue à Dieu sont des découpages pratiqués par nous dans la complexité plus qu'infinie des connexions de causalité. Nous les pratiquons en joignant à travers la durée certains événements à certains de leurs effets choisis parmi des milliers d'autres. En disant de ces découpages qu'ils sont conformes au vouloir de Dieu, nous avons raison. Mais cela est vrai au même degré et sans aucune exception de tous les découpages qui pourraient être pratiqués par toute espèce d'esprit humain ou non humain, à n'importe quelle échelle de grandeur, à travers l'espace et le temps, dans la complexité de l'univers.\par
On ne peut pas découper dans la continuité de l'espace et du temps un événement qui serait comme un atome ; mais l'infirmité du langage humain oblige à parler comme si on le pouvait.\par
Tous les événements qui composent l'univers dans la totalité du cours des temps, chacun de ces événements, chaque assemblage possible de plusieurs événements, chaque relation entre deux événements ou davantage, entre deux assemblages d'événements ou davantage, entre un événement et un assemblage d'événements – tout cela, au même degré, a été permis par le vouloir de Dieu. Tout cela, ce sont les intentions particulières de Dieu. La somme des intentions particulières de Dieu, c'est l'univers lui-même. Seul ce qui est mal est excepté, et cela même doit être excepté non pas tout entier, sous tous les rapports, mais uniquement pour autant que cela est mal. Sous tous les autres rapports, cela est conforme au vouloir de Dieu.\par
Un soldat frappé d'une blessure très douloureuse, et empêché par elle de prendre part à une bataille où tout son régiment est massacré, pourra croire que Dieu a voulu, non lui causer de la douleur, mais lui sauver la vie. C'est là une extrême naïveté et un piège de l'amour-propre. Dieu a voulu et lui causer de la douleur et lui sauver la vie et produire tous les effets qui en fait se sont produits, mais non pas l'un d'entre eux davantage qu'un autre.\par
Il n'y a qu'un cas où il soit légitime de parler de vouloir particulier de Dieu ; c'est quand surgit dans une âme une impulsion particulière qui porte la marque reconnaissable des commandements de Dieu. Mais il s'agit alors de Dieu en tant que source d'inspiration.\par
La conception actuelle de la Providence ressemble à l'exercice scolaire qu'on nomme explication française, quand il est exécuté par un mauvais professeur sur un texte poétique parfaitement beau. Le professeur dira : « Le poète a mis tel mot pour obtenir tel effet. » Cela ne peut être vrai que pour la poésie de deuxième, dixième ou cinquantième ordre. Dans un fragment poétique parfaitement beau, tous les effets, toutes les résonances, toutes les évocations susceptibles d'être amenés par la présence de tel mot à telle place, répondent au même degré, c'est-à-dire parfaitement à l'inspiration du poète. Il en est de même pour tous les arts. C'est ainsi que le poète imite Dieu. L'inspiration poétique à son point de suprême perfection est une des choses humaines qui peuvent par analogie donner une notion du vouloir de Dieu. Le poète est une personne ; pourtant dans les moments où il touche à la perfection poétique, il est traversé par une inspiration impersonnelle. C'est dans les moments médiocres que son inspiration est personnelle ; et ce n'est pas alors vraiment de l'inspiration. En se servant de l'inspiration poétique comme d'une image pour concevoir par analogie le vouloir de Dieu, il ne faut pas en prendre la forme médiocre, mais la forme parfaite.\par
La Providence divine n'est pas un trouble, une anomalie dans l'ordre du monde. C'est l'ordre du monde lui-même. Ou plutôt c'est le principe ordonnateur de cet univers. C'est la Sagesse éternelle, unique, étendue à travers tout l'univers en un réseau souverain de relations.\par
C'est ainsi que l'a conçue toute l'Antiquité pré-romaine. Toutes les parties de l'Ancien Testament où a pénétré l'inspiration universelle du monde antique nous en apportent la conception enveloppée d'une splendeur verbale incomparable. Mais nous sommes aveugles. Nous lisons sans comprendre.\par
La force brute n'est pas souveraine ici-bas. Elle est par nature aveugle et indéterminée. Ce qui est souverain ici-bas, c'est la détermination, la limite. La Sagesse éternelle emprisonne cet univers dans un réseau, dans un filet de déterminations. L'univers ne s'y débat pas. La force brute de la matière, qui nous paraît souveraineté, n'est pas autre chose en réalité que parfaite obéissance.\par
C'est là la garantie accordée à l'homme, l'arche d'alliance, le pacte, la promesse visible et palpable ici-bas, l'appui certain de l'espérance. C'est là la vérité qui nous mord le cœur chaque fois que nous sommes sensibles à la beauté du monde. C'est la vérité qui éclate avec d'incomparables accents d'allégresse dans les parties belles et pures de l'Ancien Testament, en Grèce chez les Pythagoriciens et tous les sages, en Chine chez Lao-Tseu, dans les écritures sacrées hindoues, dans les fragments égyptiens. Elle est peut-être cachée dans d'innombrables mythes et contes. Elle apparaîtra devant nous, sous nos yeux, dans notre propre science, si un jour, comme à Agar, Dieu nous dessille les yeux.\par
On la discerne à travers les paroles mêmes dans lesquelles Hitler affirme l'erreur contraire: « ... dans un monde où les planètes et les soleils suivent des trajectoires circulaires, où des lunes tournent autour des planètes, où la force règne partout et seule en maîtresse de la faiblesse qu'elle contraint à la servir docilement ou qu'elle brise... » Comment la force aveugle susciterait-elle des cercles ? Ce n'est pas la faiblesse qui sert docilement la force. C'est la force qui est docile à la Sagesse éternelle.\par
Hitler et sa jeunesse fanatique n'ont jamais senti cela en regardant les astres la nuit. Mais qui a jamais essayé de le leur enseigner ? La civilisation dont nous sommes si fiers a tout fait pour le dissimuler ; et tant que quelque chose dans notre âme est capable d'être fier d'elle, nous ne sommes innocents d'aucun des crimes d'Hitler.\par
En Inde, un mot dont le sens originel est « équilibre » signifie à la fois l'ordre du monde et la justice. Voici un texte sacré à ce sujet, qui, sous une forme symbolique, est relatif à la fois à la création du monde et à la société humaine.\par
\par
« Dieu, en vérité, existait à l'origine, absolument seul. Étant seul, il ne manifestait pas. Il produisit une forme supérieure, la souveraineté... C'est pourquoi il n'est rien au-dessus de la souveraineté. C'est pourquoi dans les cérémonies le prêtre est assis au-dessus du souverain...\par
« Dieu ne se manifestait pas encore. Il produisit la classe des paysans, artisans et marchands.\par
« Il ne se manifestait pas encore. Il produisit la classe des serviteurs.\par
« Il ne se manifestait pas encore. Il produisit une forme supérieure, la justice. La justice est la souveraineté de la souveraineté. C'est pourquoi il n'est rien au-dessus de la justice. Celui qui est sans puissance peut égaler celui qui est très puissant au moyen de la justice, comme au moyen d'une autorité royale.\par
« Ce qui est justice, cela est vérité. C'est pourquoi, quand quelqu'un dit la vérité, on dit : « Il est juste. » Et quand quelqu'un dit la justice, on dit : « Il est vrai. » C'est que réellement la justice et la vérité sont la même chose. »\par
Une stance hindoue très antique, dit :\par


\begin{verse}
Cela d'où le soleil se lève,\\
Cela dans quoi il se couche,\\
Cela, les dieux l'ont fait justice,\\
La même aujourd'hui, la même demain\\
\end{verse}

\noindent Anaximandre a écrit :\par
« C'est à partir de l'indéterminé qu'a lieu la naissance pour les choses ; et la destruction est un retour à l'indéterminé, qui s'accomplit en vertu de la nécessité. Car les choses subissent un châtiment et une expiation les unes de la part des autres, à cause de leur injustice, selon l'ordre du temps. »\par
C'est cela, la vérité, et non pas la conception monstrueuse puisée par Hitler dans la vulgarisation de la science moderne. Toute force visible et palpable est sujette à une limite invisible qu'elle ne franchira jamais. Dans la mer, une vague monte, monte et monte ; mais un point, où il n'y a pourtant que du vide, l'arrête et la fait redescendre. De la même manière le flot allemand s'est arrêté, sans que personne ait su pourquoi, au bord de la Manche.\par
Les Pythagoriciens disaient que l'univers est constitué à partir de l'indéterminé et du principe qui détermine, qui limite, qui arrête. C'est lui toujours qui domine.\par
\par
La tradition relative à l'arc-en-ciel, sûrement empruntée par Moïse aux Égyptiens, exprime de la manière la plus touchante l'espérance que l'ordre du monde doit donner aux hommes :\par
« Dieu dit : ... À l'avenir, lorsque j'amoncellerai des nuages sur la terre et que l'arc apparaîtra dans la nue, je me souviendrai de mon alliance avec vous et tous les êtres animés, et les eaux ne deviendront plus un déluge. »\par
Le beau demi-cercle de l'arc-en-ciel est le témoignage que les phénomènes d’ici-bas, si terrifiants soient-ils, sont tous soumis à une limite. La splendide poésie de ce texte veut qu'il fasse souvenir Dieu d'exercer sa fonction de principe qui limite.\par
« Tu as fixé aux eaux des barrières infranchissables pour les empêcher de submerger de nouveau la terre. » (Psaume 104.)\par
Et comme les oscillations des vagues, toutes les successions d'événements ici-bas, étant toutes des ruptures d'équilibre mutuellement compensées, des naissances et des destructions, des accroissements et des amoindrissements, rendent toutes sensible la présence invisible d'un réseau de limites sans substance et plus dures qu'aucun diamant. C'est pourquoi les vicissitudes des choses sont belles, quoiqu'elles laissent apercevoir une nécessité impitoyable. Impitoyable, mais qui n'est pas la force, qui est maîtresse souveraine de toute force.\par
Mais la pensée qui a véritablement enivré les anciens, c'est que ce qui a fait obéir la force aveugle de la matière n'est pas une autre force, plus forte. C'est l'amour. Ils pensaient que la matière est docile à la Sagesse éternelle par la vertu de l'amour qui la fait consentir à l'obéissance.\par
Platon dit dans le Timée que la Providence divine domine la nécessité en exerçant sur elle une sage persuasion. Un poème stoïcien du III\textsuperscript{e} siècle avant l'ère chrétienne, mais dont il est prouvé que l'inspiration est bien plus ancienne, dit à Dieu :\par


\begin{verse}
« À toi tout ce monde qui roule autour de la terre\\
obéit où que tu le mènes et il consent à ta domination.\\
Telle est la vertu du serviteur que tu tiens sous tes invincibles mains,\\
à double tranchant, en feu, éternellement vivant, la foudre. »\\
\end{verse}

\noindent La foudre, le trait de feu vertical qui jaillit du ciel à la terre, c'est l'échange d'amour entre Dieu et sa création, et c'est pourquoi « lanceur de la foudre » est par excellence l'épithète de Zeus.\par
C'est de là que vient la conception stoïcienne de l'{\itshape amor fati}, l'amour de l'ordre du monde, mis par eux au centre de toute vertu. L'ordre du monde doit être aimé parce qu'il est pure obéissance à Dieu. Quoi que cet univers nous accorde ou nous inflige, il le fait exclusivement par obéissance. Quand un ami depuis longtemps absent et anxieusement attendu nous serre la main, il n'importe pas que la pression en elle-même soit agréable ou pénible ; s'il serre trop fort et fait mal, on ne le remarque même pas. Quand il parle, on ne se demande pas si le son de la voix est par lui-même agréable. La pression de la main, la voix, tout est seulement pour nous le signe d'une présence, et à ce titre infiniment cher. De même tout ce qui nous arrive au cours de notre vie, étant amené par l'obéissance totale de cet univers à Dieu, nous met au contact du bien absolu que constitue le vouloir divin ; à ce titre, tout sans exception, joies et douleurs indistinctement, doit être accueilli dans la même attitude intérieure d'amour et de gratitude.\par
Les hommes qui ignorent le vrai bien désobéissent à Dieu en ce sens qu'ils ne lui obéissent pas comme il convient à une créature pensante, par un consentement de la pensée. Mais leur corps et leurs âmes sont absolument soumis aux lois des mécanismes qui régissent souverainement la matière physique et psychique. La matière physique et psychique en eux obéit parfaitement ; ils sont parfaitement obéissants en tant qu'ils sont matière, et ils ne sont pas autre chose, s'ils ne possèdent ni ne désirent la lumière surnaturelle qui seule élève l'homme au-dessus de la matière. C'est pourquoi le mal qu'ils nous font doit être accueilli comme le mal que nous fait la matière inerte. Outre la compassion qu'il convient d'accorder à une pensée humaine égarée et souffrante, ils doivent être aimés comme doit être aimée la matière inerte, en tant que parties de l'ordre parfaitement beau de l'univers.\par
Bien entendu, quand les Romains crurent devoir déshonorer le stoïcisme en l'adoptant, ils remplacèrent l'amour par une insensibilité à base d'orgueil. De là le préjugé, commun encore aujourd'hui, d'une opposition entre le stoïcisme et le christianisme, alors que ce sont deux pensées jumelles. Les noms mêmes des personnes de la Trinité, Logos, Pneuma, sont empruntés au vocabulaire stoïcien. La connaissance de certaines théories stoïciennes jette une vive lumière sur plusieurs passages énigmatiques du Nouveau Testament. Il y avait échange entre les deux pensées à cause de leurs affinités. Au centre de l'une et de l'autre se trouvent l'humilité, l'obéissance et l'amour.\par
Mais plusieurs textes indiquent que la pensée stoïcienne fut aussi celle du monde antique tout entier, jusqu'à l'Extrême-Orient. Toute l'humanité jadis à vécu dans l'éblouissement de la pensée que l'univers où nous nous trouvons n'est pas autre chose que de la parfaite obéissance.\par
Les Grecs furent enivrés d'en trouver dans la science une confirmation éclatante, et ce fut le mobile de leur enthousiasme pour elle.\par
L'opération de l'intelligence dans l'étude scientifique fait apparaître à la pensée la nécessité souveraine sur la matière comme un réseau de relations immatérielles et sans force. La nécessité n'est parfaitement conçue qu'au moment où les relations apparaissent comme parfaitement immatérielles. Elles ne sont alors présentes à la pensée que par l'effet d'une attention élevée et pure, qui part d'un point de l'âme non soumis à la force. Ce qui dans l'âme humaine est soumis à la force, c'est ce qui se trouve sous l'empire des besoins. Il faut oublier tout besoin pour concevoir les relations dans leur pureté immatérielle. Si l'on y parvient, on se rend compte du jeu des forces par lesquelles la satisfaction est accordée ou refusée aux besoins.\par
Les forces d'ici-bas sont souverainement déterminées par la nécessité ; la nécessité est constituée par des relations qui sont des pensées ; par suite la force qui est souveraine ici-bas est souverainement dominée par la pensée. L'homme est une créature pensante ; il est du côté de ce qui commande à la force. Il n'est certes pas seigneur et maître de la nature, et Hitler avait raison de dire qu'en croyant l'être il se trompe ; mais il est le fils du maître, l'enfant de la maison. La science en est la preuve. Un enfant tout jeune dans une riche maison est en bien des choses soumis aux domestiques ; mais quand il est sur les genoux de son père et s'identifie à lui par l'amour, il a part à l'autorité.\par
Tant que l'homme tolère d'avoir l'âme emplie de ses propres pensées, de ses pensées personnelles, il est entièrement soumis jusqu'au plus intime de ses pensées à la contrainte des besoins et au jeu mécanique de la force. S'il croit qu'il en est autrement, il est dans l'erreur. Mais tout change quand, par la vertu d'une véritable attention, il vide son âme pour y laisser pénétrer les pensées de la sagesse éternelle. Il porte alors en lui les pensées mêmes auxquelles la force est soumise.\par
La nature de la relation, et de l'attention indispensable pour la concevoir, était aux yeux des Grecs une preuve que la nécessité est réellement obéissance à Dieu. Ils en avaient une autre. C'étaient les symboles inscrits dans les relations elles-mêmes, comme la signature du peintre est écrite dans un tableau.\par
La symbolique grecque explique le fait que Pythagore ait offert un sacrifice dans sa joie d'avoir trouvé l'inscription du triangle rectangle dans le demi-cercle.\par
Le cercle aux yeux des Grecs, était l'image de Dieu. Car un cercle qui tourne sur lui-même, c'est un mouvement où rien ne change et parfaitement bouclé sur soi-même. Le symbole du mouvement circulaire exprimait chez eux la même vérité qui est exprimée dans le dogme chrétien par la conception de l'acte éternel d'où procèdent les relations entre les Personnes de la Trinité.\par
La moyenne proportionnelle était à leurs yeux l'image de la médiation divine entre Dieu et les créatures. Les travaux mathématiques des Pythagoriciens avaient pour objet la recherche de moyennes proportionnelles entre nombres qui ne font pas partie d'une même progression géométrique, par exemple entre l'unité et un nombre non carré. Le triangle rectangle leur a fourni la solution. Le triangle rectangle est le réservoir de toutes les moyennes proportionnelles. Mais dès lors qu'il est inscriptible dans le demi-cercle, le cercle se substitue à lui pour cette fonction. Ainsi le cercle, image géométrique de Dieu, est la source de l'image géométrique de la Médiation divine. Une rencontre aussi merveilleuse valait un sacrifice.\par
La géométrie est ainsi un double langage, qui en même temps donne des renseignements sur les forces qui sont en action dans la matière et parle des relations surnaturelles entre Dieu et les créatures. Elle est comme ces lettres chiffrées qui paraissent également cohérentes avant et après le déchiffrage.\par
Le souci du symbole a complètement disparu de notre science. Néanmoins il suffirait de s'en donner la peine pour lire facilement, dans certaines parties au moins de la mathématique moderne, comme la théorie des ensembles ou le calcul intégral, des symboles aussi clairs, aussi beaux, aussi pleins de signification spirituelle que celui du cercle et de la médiation.\par
De la pensée moderne à la sagesse antique le chemin serait court et direct, si l'on voulait le prendre.\par
Dans la philosophie moderne sont apparues un peu partout, sous différentes formes, des analyses susceptibles de préparer une théorie complète de la perception sensible. La vérité fondamentale que révélerait une telle théorie, c'est que la réalité des objets perçus par les sens ne réside pas dans les impressions sensibles, mais uniquement dans les nécessités dont les impressions constituent les signes.\par
Cet univers, sensible où nous sommes n'a pas d'autre réalité que la nécessité ; et la nécessité est une combinaison de relations qui s'évanouissent dès qu'elles ne sont pas soutenues par une attention élevée et pure. Cet univers autour de nous est de la pensée matériellement présente à notre chair.\par
La science, dans ses différentes branches, saisit à travers tous les phénomènes des relations mathématiques ou analogues aux relations mathématiques. La mathématique éternelle, ce langage à deux fins, est l'étoffe dont est tissé l'ordre du monde.\par
Tout phénomène est une modification dans la distribution de l'énergie, et par suite est déterminé par les lois de l'énergie. Mais il y a plusieurs espèces d'énergie, et elles sont disposées dans un ordre hiérarchique. La force mécanique, pesanteur, ou gravitation au sens de Newton, qui nous fait continuellement sentir sa contrainte, n'est pas l'espèce la plus élevée. La lumière impalpable et sans poids est une énergie qui fait monter malgré la pesanteur les arbres, et les tiges des blés. Nous la mangeons dans le blé et les fruits, et sa présence en nous nous donne la force de nous tenir debout et de travailler.\par
Un infiniment petit, dans certaines conditions, opère d'une manière décisive. Il n'est pas de masse si lourde qu'un point ne lui soit égal ; car une masse ne tombe pas si l'on en soutient un seul point, à condition que ce point soit le centre de gravité. Certaines transformations chimiques ont pour condition l'opération de bactéries presque invisibles. Les catalyseurs sont d'imperceptibles fragments de matière dont la présence est indispensable à d'autres transformations chimiques. D'autres fragments minuscules, de composition presque identique, ont par leur présence une vertu non moins décisive d'inhibition ; sur ce mécanisme est fondée la plus puissante des médications récemment découvertes.\par
Ainsi ce n'est pas seulement la mathématique, c'est la science entière qui, sans que nous songions à le remarquer, est un miroir symbolique des vérités surnaturelles.\par
La psychologie moderne voudrait faire de l'étude de l'âme une science. Un peu plus de précision suffirait pour y parvenir. Il faudrait mettre à la base la notion de matière psychique, liée à l'axiome de Lavoisier, valable pour toute matière, « rien ne se perd, rien ne se crée » ; autrement dit les changements sont ou bien des modifications de forme, sous lesquelles quelque chose persiste, ou bien des déplacements, mais jamais simplement des apparitions et disparitions. Il faudrait y introduire la notion de limite, et poser en principe que dans la partie terrestre de l'âme tout est fini, limité, susceptible d'épuisement. Enfin il faudrait y introduire la notion d'énergie, en posant que les phénomènes psychiques, comme les phénomènes physiques, sont des modifications dans la répartition et la qualité de l'énergie et sont déterminés par les lois de l'énergétique.\par
Les tentatives contemporaines pour fonder une science sociale aboutiraient, elles aussi, au prix d'un peu plus de précision. Il faudrait mettre à la base la notion platonicienne du gros animal ou la notion apocalyptique de la Bête. La science sociale est l'étude du gros animal et doit en décrire minutieusement l'anatomie, la physiologie, les réflexes naturels et conditionnels, les facilités de dressage.\par
La science de l'âme et la science sociale sont l'une et l'autre tout à fait impossibles si la notion de surnaturel n'est pas rigoureusement définie et introduite dans la science, à titre de notion scientifique, pour y être maniée avec une extrême précision.\par
Si les sciences de l'homme étaient ainsi fondées par des méthodes d'une rigueur mathématique, et maintenues en même temps en liaison avec la foi ; si dans les sciences de la nature et la mathématique l'interprétation symbolique reprenait la place qu'elle avait jadis ; l'unité de l'ordre établi dans cet univers apparaîtrait dans sa souveraine clarté.\par
L'ordre du monde, c'est la beauté du monde. Seul diffère le régime de l'attention, selon qu'on essaie de concevoir les relations nécessaires qui le composent ou d'en contempler l'éclat.\par
C'est une seule et même chose qui relativement à Dieu est Sagesse éternelle, relativement à l'univers parfaite obéissance, relativement à notre amour beauté, relativement à notre intelligence équilibre de relations nécessaires, relativement à notre chair force brutale.\par
\par
Aujourd'hui, la science, l'histoire, la politique, l'organisation du travail, la religion même pour autant qu'elle est marquée de la souillure romaine, n'offrent à la pensée des hommes que la force brutale. Telle est notre civilisation. Cet arbre porte les fruits qu'il mérite.\par
Le retour à la vérité ferait apparaître entre autres choses la vérité du travail physique.\par
Le travail physique consenti est, après la mort consentie, la forme la plus parfaite de la vertu d'obéissance.\par
Le caractère pénal du travail, indiqué par le récit de la Genèse, a été mal compris faute d'une notion juste du châtiment. On lit à tort dans ce texte une nuance de dédain pour le travail. Il est plus probable qu'il a été transmis par une civilisation très antique où le travail physique était honoré par-dessus toute autre activité.\par
Plusieurs signes indiquent qu'il y a eu une telle civilisation, qu'il y a très longtemps le travail physique était par excellence une activité religieuse et par suite une chose sacrée. Les Mystères, religion de toute l'Antiquité pré-romaine, étaient entièrement fondés sur des expressions symboliques du salut de l'âme tirées de l'agriculture. Le même symbolisme se retrouve dans les paraboles de l'Évangile. Le rôle d'Héphaïstos dans le {\itshape Prométhée} d'Eschyle semble évoquer une religion de forgerons. Prométhée est exactement la projection intemporelle du Christ, un Dieu crucifié et rédempteur qui est venu jeter un feu sur la terre ; dans le symbolisme grec comme dans l'Évangile, le feu est l'image du Saint-Esprit. Eschyle, qui ne dit jamais rien au hasard, dit que le feu donné par Prométhée aux hommes était la propriété personnelle d'Héphaïstos, ce qui semble indiquer qu'Héphaïstos en est la personnification. Héphaïstos est un dieu forgeron. On imagine une religion de forgerons voyant dans le feu qui rend le fer docile l'image de l'opération du Saint-Esprit sur la nature humaine.\par
Il y a peut-être eu un temps où une vérité identique était traduite en différents systèmes de symboles, et où chaque système était adapté à un certain travail physique de manière à en faire l'expression directe de la foi.\par
En tout cas toutes les traditions religieuses de l'Antiquité, y compris l'Ancien Testament, font remonter les métiers à un enseignement direct de Dieu. La plupart affirment que Dieu s'est incarné pour cette mission pédagogique. Les Égyptiens, par exemple, pensaient que l'incarnation d'Osiris avait eu pour objet à la fois cet enseignement pratique et la Rédemption par la Passion.\par
Quelle que soit la vérité enfermée dans ces récits extrêmement mystérieux, la croyance en un enseignement direct des métiers par Dieu implique le souvenir d'un temps où l'exercice des métiers était par excellence une activité sacrée.\par
Il n'en reste aucune trace dans Homère, ni Hésiode, ni la Grèce classique, ni dans le peu que nous savons des autres civilisations de l'Antiquité. En Grèce le travail était chose servile. Nous ne pouvons savoir s'il l'était déjà avant l'invasion hellène, au temps des Pélasges, ni si les Mystères conservaient explicitement dans leur enseignement secret le souvenir d'un temps où il était honoré. Tout au début de la Grèce classique nous voyons finir une forme de civilisation où, sauf le travail physique, toutes les activités humaines étaient sacrées ; où art, poésie, philosophie, science et politique ne se distinguaient pour ainsi dire pas de la religion. Un siècle ou deux plus tard, par un mécanisme que nous discernons mal, mais où en tout cas l'argent a joué un grand rôle, toutes ces activités étaient devenues exclusivement profanes et coupées de toute inspiration spirituelle. Le peu de religion qui demeurait était relégué dans les lieux affectés au culte. Platon, à son époque, était une survivance d'un passé déjà lointain. Les Stoïciens grecs furent une flamme jaillie d'une étincelle encore vivante du même passé.\par
Les Romains, nation athée et matérialiste, anéantirent les restes de vie spirituelle sur les territoires occupés par eux au moyen de l'extermination ; ils n'adoptèrent le christianisme qu'en le vidant de son contenu spirituel. Sous leur domination toute activité humaine sans distinction fut chose servile ; et ils finirent par ôter toute réalité à l'institution de l'esclavage, ce qui en prépara la disparition, en abaissant à l'état d'esclavage tous les êtres humains.\par
Les prétendus Barbares, dont beaucoup sans doute étaient originaires de Thrace et par suite nourris de la spiritualité des Mystères, prirent le christianisme au sérieux. Le résultat fut qu'il faillit y avoir une civilisation chrétienne. Nous en voyons apparaître les promesses aux XI\textsuperscript{e} et XII\textsuperscript{e} siècles. Les pays du sud de la Loire, qui en étaient le principal foyer de rayonnement, étaient imprégnés à la fois de spiritualité chrétienne et de spiritualité antique ; s'il est vrai du moins que les Albigeois sont des Manichéens, et par suite procèdent non seulement de la pensée perse, mais aussi de la pensée gnostique, stoïcienne, pythagoricienne, égyptienne. La civilisation alors en germe aurait été pure de toute souillure d'esclavage. Les métiers auraient été au centre.\par
Le tableau que fait Machiavel de la Florence du XII\textsuperscript{e} siècle est un modèle de ce que le jargon moderne appellerait une démocratie syndicale. À Toulouse les chevaliers et les ouvriers se battaient côte à côte contre Simon de Montfort pour défendre le même trésor spirituel qui leur était commun. Les corporations, instituées au cours de cette période de gestation, étaient des institutions religieuses. Il suffit de regarder une église romane, d'entendre une mélodie grégorienne, de lire un des quelques poèmes parfaits des troubadours, ou, mieux encore, les textes liturgiques, pour reconnaître que l'art était indiscernable de la foi autant que dans la Grèce à son meilleur moment.\par
Mais une civilisation chrétienne, où la lumière chrétienne aurait illuminé la vie entière, n'aurait été possible que si la conception romaine de l’asservissement des esprits par l'Église avait été éliminée. La lutte acharnée et victorieuse de saint Bernard contre Abélard montre qu'il s'en fallait de beaucoup. Au début du XIII\textsuperscript{e} siècle la civilisation encore à venir fut détruite par l'anéantissement de son principal foyer, c'est-à-dire les pays du sud de la Loire, par l'établissement de l'Inquisition, et par l'étouffement de la pensée religieuse sous la notion d'orthodoxie.\par
La notion d'orthodoxie, en séparant rigoureusement le domaine relatif au bien des âmes, qui est celui d'une soumission inconditionnée de la pensée à une autorité extérieure, et le domaine relatif aux choses dites profanes, dans lequel l'intelligence est libre, rend impossible cette pénétration mutuelle du religieux et du profane qui serait l'essence d'une civilisation chrétienne. C'est vainement que tous les jours, à la messe, un peu d'eau est mélangée au vin.\par
Le XIII\textsuperscript{e} siècle, le XIV\textsuperscript{e}, le début du XV\textsuperscript{e}, sont la période de décadence du Moyen Âge. Dégradation progressive et mort d'une civilisation qui n'avait pas eu le temps de naître, dessèchement progressif d'un simple germe.\par
Vers le XV\textsuperscript{e} siècle eut lieu la première Renaissance, qui fut comme un faible pressentiment de résurrection de la civilisation pré-romaine et de l'esprit du XII\textsuperscript{e} siècle. La Grèce authentique, Pythagore, Platon, furent alors l'objet d'un respect religieux qui s'unissait en parfaite harmonie avec la foi chrétienne. Mais cette attitude d'esprit fut très brève.\par
Bientôt vint la seconde Renaissance dont l'orientation était opposée. C'est elle qui a produit ce que nous nommons notre civilisation moderne.\par
Nous en sommes très fiers, mais nous n'ignorons pas qu'elle est malade. Et tout le monde est d'accord sur le diagnostic de la maladie. Elle est malade de ne pas savoir au juste quelle place accorder au travail physique et à ceux qui l'exécutent.\par
Beaucoup d'intelligences s'épuisent sur ce problème en se débattant à tâtons. On ne sait pas par où commencer, d'où partir, sur quoi se guider ; ainsi les efforts sont stériles.\par
Le mieux est de méditer le vieux récit de la Genèse, en le situant dans le milieu qui est le sien, celui de la pensée antique.\par
Quand un être humain s'est mis par un crime hors du bien, le vrai châtiment constitue sa réintégration dans la plénitude du bien par le moyen de la douleur. Rien n'est merveilleux comme un châtiment.\par
L'homme s'est mis hors de l'obéissance. Dieu a choisi comme châtiments le travail et la mort. Par conséquent le travail et la mort, si l'homme les subit en consentant à les subir, constituent un transport dans le bien suprême de l'obéissance à Dieu.\par
Cela est d'une évidence lumineuse si, comme l'Antiquité, on regarde la passivité de la matière inerte comme la perfection de l'obéissance à Dieu, et la beauté du monde comme l'éclat de la parfaite obéissance.\par
Quelle que soit dans le ciel la signification mystérieuse de la mort, elle est ici-bas la transformation d'un être fait de chair frémissante et de pensée, d'un être qui désire et hait, espère et craint, veut et ne veut pas, en un petit tas de matière inerte.\par
Le consentement à cette transformation est pour l'homme l'acte suprême de totale obéissance. C'est pourquoi saint Paul dit du Christ lui-même, au sujet de la Passion, « ce qu'il a souffert lui a enseigné l'obéissance et l'a rendu partait ».\par
Mais le consentement à la mort ne peut être pleinement réel que quand la mort est là. Il ne peut être proche de la plénitude que quand la mort est proche. Quand la possibilité de la mort est abstraite et lointaine, il est abstrait.\par
Le travail physique est une mort quotidienne.\par
Travailler, c'est mettre son propre être, âme et chair, dans le circuit de la matière inerte, en faire un intermédiaire entre un état et un autre état d'un fragment de matière, en faire un instrument. Le travailleur fait de son corps et de son âme un appendice de l'outil qu'il manie. Les mouvements du corps et l'attention de l'esprit sont fonction des exigences de l'outil, qui lui-même est adapté à la matière du travail.\par
La mort et le travail sont choses de nécessité et non de choix. L'univers ne se donne à l'homme dans la nourriture et la chaleur que si l'homme se donne à l'univers dans le travail. Mais la mort et le travail peuvent être subis avec révolte ou consentement. Ils peuvent être subis dans leur vérité nue ou enrobés de mensonge.\par
Le travail fait violence à la nature humaine. Tantôt il y a surabondance de forces juvéniles qui veulent se dépenser et n'y trouvent pas leur emploi ; tantôt il y a épuisement, et la volonté doit sans cesse suppléer, au prix d'une tension très douloureuse, à l'insuffisance de l'énergie physique ; il y a mille préoccupations, soucis, angoisses, mille désirs, mille curiosités qui entraînent la pensée ailleurs ; la monotonie cause du dégoût ; et le temps pèse d'un poids presque intolérable.\par
La pensée humaine domine le temps et parcourt sans cesse rapidement le passé et l'avenir en franchissant n'importe quel intervalle ; mais celui qui travaille est soumis au temps à la manière de la matière inerte qui franchit un instant après l'autre. C'est par là surtout que le travail fait violence à la nature humaine. C'est pourquoi les travailleurs expriment la souffrance du travail par l'expression « trouver le temps long ».\par
Le consentement à la mort, quand la mort est présente et vue dans sa nudité, est un arrachement suprême, instantané, à ce que chacun appelle moi. Le consentement au travail est moins violent. Mais là où il est complet, il se renouvelle chaque matin tout au long d'une existence humaine, jour après jour, et chaque jour il dure jusqu'au soir, et cela recommence le lendemain, et cela se prolonge souvent jusqu'à la mort. Chaque matin le travailleur consent au travail pour ce jour-là et pour la vie tout entière. Il y consent qu'il soit triste ou gai, soucieux ou avide d'amusement, fatigué ou débordant d'énergie.\par
Immédiatement après le consentement à la mort, le consentement à la loi qui rend le travail indispensable à la conservation de la vie est l'acte le plus parfait d'obéissance qu'il soit donné à l'homme d'accomplir.\par
Dès lors les autres activités humaines, commandement des hommes, élaboration de plans techniques, art, science, philosophie, et ainsi de suite, sont toutes inférieures au travail physique en signification spirituelle.\par
Il est facile de définir la place que doit occuper le travail physique dans une vie sociale bien ordonnée. Il doit en être le centre spirituel.\par
Fin.
 


% at least one empty page at end (for booklet couv)
\ifbooklet
  \pagestyle{empty}
  \clearpage
  % 2 empty pages maybe needed for 4e cover
  \ifnum\modulo{\value{page}}{4}=0 \hbox{}\newpage\hbox{}\newpage\fi
  \ifnum\modulo{\value{page}}{4}=1 \hbox{}\newpage\hbox{}\newpage\fi


  \hbox{}\newpage
  \ifodd\value{page}\hbox{}\newpage\fi
  {\centering\color{rubric}\bfseries\noindent\large
    Hurlus ? Qu’est-ce.\par
    \bigskip
  }
  \noindent Des bouquinistes électroniques, pour du texte libre à participation libre,
  téléchargeable gratuitement sur \href{https://hurlus.fr}{\dotuline{hurlus.fr}}.\par
  \bigskip
  \noindent Cette brochure a été produite par des éditeurs bénévoles.
  Elle n’est pas faîte pour être possédée, mais pour être lue, et puis donnée.
  Que circule le texte !
  En page de garde, on peut ajouter une date, un lieu, un nom ; pour suivre le voyage des idées.
  \par

  Ce texte a été choisi parce qu’une personne l’a aimé,
  ou haï, elle a en tous cas pensé qu’il partipait à la formation de notre présent ;
  sans le souci de plaire, vendre, ou militer pour une cause.
  \par

  L’édition électronique est soigneuse, tant sur la technique
  que sur l’établissement du texte ; mais sans aucune prétention scolaire, au contraire.
  Le but est de s’adresser à tous, sans distinction de science ou de diplôme.
  Au plus direct ! (possible)
  \par

  Cet exemplaire en papier a été tiré sur une imprimante personnelle
   ou une photocopieuse. Tout le monde peut le faire.
  Il suffit de
  télécharger un fichier sur \href{https://hurlus.fr}{\dotuline{hurlus.fr}},
  d’imprimer, et agrafer ; puis de lire et donner.\par

  \bigskip

  \noindent PS : Les hurlus furent aussi des rebelles protestants qui cassaient les statues dans les églises catholiques. En 1566 démarra la révolte des gueux dans le pays de Lille. L’insurrection enflamma la région jusqu’à Anvers où les gueux de mer bloquèrent les bateaux espagnols.
  Ce fut une rare guerre de libération dont naquit un pays toujours libre : les Pays-Bas.
  En plat pays francophone, par contre, restèrent des bandes de huguenots, les hurlus, progressivement réprimés par la très catholique Espagne.
  Cette mémoire d’une défaite est éteinte, rallumons-la. Sortons les livres du culte universitaire, cherchons les idoles de l’époque, pour les briser.
\fi

\ifdev % autotext in dev mode
\fontname\font — \textsc{Les règles du jeu}\par
(\hyperref[utopie]{\underline{Lien}})\par
\noindent \initialiv{A}{lors là}\blindtext\par
\noindent \initialiv{À}{ la bonheur des dames}\blindtext\par
\noindent \initialiv{É}{tonnez-le}\blindtext\par
\noindent \initialiv{Q}{ualitativement}\blindtext\par
\noindent \initialiv{V}{aloriser}\blindtext\par
\Blindtext
\phantomsection
\label{utopie}
\Blinddocument
\fi
\end{document}
